

\documentclass[oldfontcommands,twoside,a4paper,12pt]{article} 
\usepackage{fontspec}
\usepackage{natbib}
\usepackage{booktabs}
\usepackage{fancyhdr}
\usepackage{xltxtra} 
\usepackage[text={16.3cm,25.5cm}, top=1.7cm, left=3cm]{geometry}
\usepackage{polyglossia} 
%\usepackage[table]{xcolor}
\usepackage{gb4e} 
\usepackage{multicol}
\usepackage{graphicx}
\usepackage{float}
\usepackage{textcomp}
\usepackage{hyperref} 
\hypersetup{bookmarks=false,bookmarksnumbered,bookmarksopenlevel=5,bookmarksdepth=5,xetex}
\usepackage[all]{hypcap}
\usepackage{memhfixc}
\usepackage{lscape}
 

%\setmainfont[Mapping=tex-text,Numbers=OldStyle,Ligatures=Common]{Charis SIL} 
 \setmainfont[Mapping=tex-text,Numbers=OldStyle,Ligatures=Common]{Palatino Linotype} 
\newfontfamily\phon[Mapping=tex-text,Ligatures=Common,Scale=MatchLowercase,FakeSlant=0.3]{Charis SIL} 
\newcommand{\ipa}[1]{{\phon \mbox{#1}}} %API tjs en italique
 
\newcommand{\grise}[1]{\cellcolor{lightgray}\textbf{#1}}
\newfontfamily\cn[Mapping=tex-text,Ligatures=Common,Scale=MatchUppercase]{MingLiU}%pour le chinois
\newcommand{\zh}[1]{{\cn #1}}



\usepackage{titlesec}
\titleformat{\section}{\large\bfseries}{}{0pt}{\thesection.{ }}
\titleformat{\subsection}{\itshape\large\bfseries}{}{0pt}{\thesubsection.{ }}


\newcommand{\acc}{\textsc{acc}}
 \newcommand{\acaus}{\textsc{acaus}}
 \newcommand{\advers}{\textsc{advers}}
\newcommand{\apass}{\textsc{apass}}
\newcommand{\appl}{\textsc{appl}}
\newcommand{\allat}{\textsc{all}}
\newcommand{\aor}{\textsc{pfv}}
\newcommand{\assert}{\textsc{assert}}
\newcommand{\auto}{\textsc{autoben}}
\newcommand{\caus}{\textsc{caus}}
\newcommand{\cl}{\textsc{cl}}
\newcommand{\cisl}{\textsc{cisl}}
\newcommand{\classif}{\textsc{class}}
\newcommand{\concsv}{\textsc{concsv}}
\newcommand{\comit}{\textsc{comit}}
\newcommand{\compl}{\textsc{compl}} %complementizer
\newcommand{\comptv}{\textsc{comptv}} %comparative
\newcommand{\cond}{\textsc{cond}}
\newcommand{\conj}{\textsc{conj}}
\newcommand{\const}{\textsc{const}}
\newcommand{\conv}{\textsc{conv}}
\newcommand{\cop}{\textsc{cop}}
\newcommand{\dat}{\textsc{dat}}
\newcommand{\dem}{\textsc{dem}}
\newcommand{\degr}{\textsc{degr}}
\newcommand{\deexp}{\textsc{deexp}}
\newcommand{\distal}{\textsc{dist}}
\newcommand{\du}{\textsc{du}}
\newcommand{\duposs}{\textsc{du.poss}}
\newcommand{\dur}{\textsc{dur}}
\newcommand{\erg}{\textsc{erg}}
\newcommand{\emphat}{\textsc{emph}}
\newcommand{\evd}{\textsc{evd}}
\newcommand{\fut}{\textsc{fut}}
\newcommand{\gen}{\textsc{gen}}
\newcommand{\genr}{\textsc{genr}}
\newcommand{\hort}{\textsc{hort}}
\newcommand{\hypot}{\textsc{hyp}}
\newcommand{\ideo}{\textsc{ideo}}
\newcommand{\imp}{\textsc{imp}}
\newcommand{\indef}{\textsc{indef}}
\newcommand{\inftv}{\textsc{inf}}
\newcommand{\instr}{\textsc{instr}}
\newcommand{\intens}{\textsc{intens}}
\newcommand{\intrg}{\textsc{intrg}}
\newcommand{\inv}{\textsc{inv}}
\newcommand{\ipf}{\textsc{ipfv}}
\newcommand{\irr}{\textsc{irr}}
\newcommand{\loc}{\textsc{loc}}
\newcommand{\med}{\textsc{med}}
\newcommand{\mir}{\textsc{mir}}
\newcommand{\negat}{\textsc{neg}}
\newcommand{\neu}{\textsc{indef.poss}}
\newcommand{\nmlz}{\textsc{nmlz}}
\newcommand{\npst}{\textsc{n.pst}}
\newcommand{\pfv}{\textsc{pfv}}
\newcommand{\pl}{\textsc{pl}}
\newcommand{\plposs}{\textsc{pl.poss}}
\newcommand{\pass}{\textsc{pass}}
\newcommand{\poss}{\textsc{poss}}
\newcommand{\pot}{\textsc{pot}}
\newcommand{\pres}{\textsc{pres}}
\newcommand{\prohib}{\textsc{prohib}}
\newcommand{\prox}{\textsc{prox}}
\newcommand{\pst}{\textsc{pst}}
\newcommand{\qu}{\textsc{qu}}
\newcommand{\recip}{\textsc{recip}}
\newcommand{\redp}{\textsc{redp}}
\newcommand{\refl}{\textsc{refl}}
\newcommand{\sg}{\textsc{sg}}
\newcommand{\sgposs}{\textsc{sg.poss}}
\newcommand{\stat}{\textsc{stat}}
\newcommand{\topic}{\textsc{top}}
\newcommand{\volit}{\textsc{vol}}
\newcommand{\transloc}{\textsc{transloc}}
\newcommand{\cisloc}{\textsc{cisl}}
\newcommand{\quind}{\textsc{qu.ind}} %revoir glose
\newcommand{\trop}{\textsc{trop}} 
 \newcommand{\abil}{\textsc{abil}}  
 \newcommand{\facil}{\textsc{facil}}  
  

 

\XeTeXlinebreaklocale "zh" %使用中文换行
\XeTeXlinebreakskip = 0pt plus 1pt %
 %CIRCG
 
 \pagestyle{fancyplain}
\fancyhf{}

\begin{document} 
\begin{flushright}

\end{flushright}
\renewcommand{\headrulewidth}{0pt}
\renewcommand{\refname}{\uppercase{references}}
\large
\bibpunct[: ]{(}{)}{,}{a}{}{:}

\begin{center}
 \textbf{\uppercase{Applicative and tropative derivations in Japhug Rgyalrong}}\footnote{I wish to thank Anton Antonov, Nathan Hill and two anonymous reviewers for useful comments on this article.  This research was funded by the \textit{HimalCo} project (ANR-12-CORP-0006) and  is related to the research strand LR-4.11  Automatic paradigm generation and language description of the Labex EFL (funded by the ANR/CGI). 
The glosses follow the Leipzig glossing rules. Other abbreviations used here are: \textsc{acaus} anticausative, \textsc{appl} applicative, \textsc{cisloc} cislocativ, \textsc{const} constative,   \textsc{dem} demonstrative, \textsc{dist} distal, \textsc{emph} emphatic, \textsc{indef} indefinite, \textsc{inv} inverse, \textsc{lnk} linker, \textsc{poss} possessor, \textsc{transloc} translocative. %\textsc{trop} tropative, 
In all examples in this paper taken from traditional narratives, we indicate the name of the story followed by the line number. These stories will eventually be made available on the Pangloss website (http://lacito.vjf.cnrs.fr/archivage/). Elicited examples or examples from conversations are from Chenzhen and Dpalcan, both born in 1950.
} 
 \end{center}

\begin{center}
 \textbf{Guillaume Jacques}
 \end{center}

\normalsize
 \sloppy
Abstract:  This paper presents the morphosyntactic properties of the applicative  derivation  in  Japhug Rgyalrong, and shows that an additional   valency-increasing derivation, the tropative, is also found in this language. Although the applicative and tropative prefixes are superficially similar, their morphosyntactic properties are distinct.


Keywords: Japhug, Rgyalrong, Applicative, Tropative, Derivational morphology, Valency
\large
\section{\uppercase{introduction}}



Unlike most Sino-Tibetan languages, Japhug and the other Rgyalrong languages have a relatively complex verbal derivation morphology (see  \citealt{jackson06paisheng}, \citealt{jacques08}, \citealt{jacques12demotion} in particular). All productive derivational affixes are prefixes, and most of these prefixes influence the valency of the verb  in derivations such as causative, anticausative, antipassive and applicative. 
 
The present paper deals with the morphosyntactic functions of two valency-increasing derivational prefixes in Japhug Rgyalrong, the applicative  \ipa{nɯ-} and the tropative \ipa{nɤ-}. 

  The term `tropative' designates a derivation from an adjective or a stative  verb into a transitive verb meaning `to consider to be ...'. This term  is borrowed from Arabic linguistics, where it is applied to a particular verbal pattern  in examples such as \textit{ħasuna} `be good' $\rightarrow$  \textit{istaħsana} `deem to be good'     (see for instance \citealt{larcher96}; the term `estimative' is also found).

   Apart from  Arabic, other examples of tropative are found for instance in Turkish (the suffix  \ipa{-(I)msA} or \ipa{-sA}, \citealt[56]{goksel05grammar}) and in Lakhota (the suffix \ipa{-la} or \ipa{-lakA}, \citealt[317]{ullrich08}) as in the following examples:\footnote{The Turkish tropative can also be   applied to a pronoun in the example \textit{ben} `I' $\rightarrow$  \textit{ben-imse-} `adopt, embrace (= consider to be one's own)', a property not found in Arabic or Japhug. 
   }
   
        \begin{exe}
\ex 
 \glt   \ipa{büyük} `big' $\rightarrow$  \ipa{büyük-se-} `to overestimate'
 \glt \ipa{kötü } `bad' $\rightarrow$  \ipa{kötü-mse-} `to think ill of'
 \ex \label{ex:wakhaN}
    \glt   \ipa{wakȟáŋ} `sacred' $\rightarrow$  \ipa{wakȟáŋ-la} `to consider sacred'
 \glt \ipa{wašté} `good' $\rightarrow$  \ipa{wašté-lakA} `to like'
   \end{exe}
   
%   Defining the headers here so that they don't appear on the first page
   \lhead[]{\thepage}
\rhead[\thepage]{}
\lfoot[]{}
\cfoot[]{}
\rfoot[]{}
   \chead[\textit{Guillaume Jacques}]{\textit{Applicative and tropative derivations in Japhug Rgyalrong}}
   Crosslinguistically, few languages   have a special derivation  restricted to the tropative meaning like Turkish, Lakhota or Japhug. Common ways of expressing the same meaning include a construction with verbs such as `think' or `consider' and a complement clause, or use a causative  derivation with a tropative meaning. These two alternative strategies are also found in Japhug, and are described in section \ref{sub:other}.
    

\section{\uppercase{applicative}} \label{sub:applicative}
The applicative is a valency-increasing derivation by means of which an oblique argument or adjunct is promoted to the O function, while the S of the original  verb becomes the A of the applicative verb.\footnote{On the status of S, A and O see \citet{haspelmath11SAPTR}. } In Japhug, only intransitive verbs are subject to the applicative derivation. 

In Rgyalrong languages, the existence of an applicative derivation has been mentioned in previous work (in particular  \citealt{jackson06paisheng} and  \citealt{jackson13morpho} concerning   Tshobdun   and \citealt{jacques08} concerning Japhug), but was not clearly distinguished from the tropative.

The applicative in Japhug is marked by the prefix \ipa{nɯ-} / \ipa{nɯɣ-} / \ipa{nɤ-} (the distribution of the three allomorphs is discussed in \ref{sec:appl.morphophon}). It is only moderately productive; Table \ref{tab:applicative} contains all known examples of applicative in Japhug.

\begin{table}[h]
\caption{Examples of the \ipa{nɯ}- applicative prefix}\label{tab:applicative} \centering
\begin{tabular}{lllllllll} \toprule
basic verb  & &derived  verb &\\
\midrule
 \ipa{aʑɯʑu}  & wrestle	& \ipa{nɤʑɯʑu}  & wrestle with\\
\ipa{akhu}  &	shout, call&\ipa{nɤkhu}  & shout at \\
\ipa{akhɤzŋga}&	shout, call&\ipa{nɤkhɤzŋga}  & shout at \\
\ipa{andzɯt}  &	bark&\ipa{nɤndzɯt}  & bark at \\
\ipa{amdzɯ}  &sit & \ipa{nɤmdzɯ}  &look after\\
\ipa{aɣro}&play&\ipa{nɤɣro} & play with \\
    \ipa{stu}  &believe (vi)	& \ipa{nɤstu}  & believe (vt)\\
\midrule
\ipa{mbɣom}  &	be hurried & \ipa{nɯmbɣom}  & look  forward to, miss s.o.\\
\ipa{ŋke}  &go on foot	& \ipa{nɯŋke}  & look for \\
\ipa{rga}  &	like, be glad (vi) & \ipa{nɯrga}  &like (vt) \\
\ipa{sŋom}  &	envy (vi) & \ipa{nɯsŋom}  &envy (vt) \\
\midrule
  \ipa{bɯɣ}  &miss (vi)	& \ipa{nɯɣbɯɣ}  & miss (vt)\\
  \ipa{mu} & be afraid & \ipa{nɯɣmu} & be afraid of \\
\bottomrule
\end{tabular}
\end{table}
Following Peterson's \citeyear{peterson07appl} classification, the examples of applicative found in Japhug mainly belong to the \textit{ stimulus} applicative subtype, though we also find two examples of \textit{comitative} applicative (`wrestle with', `play with').
  
\subsection{Morphophonology} \label{sec:appl.morphophon}
The applicative  has three distinct allomorphs: \ipa{nɯ-}, \ipa{nɤ-} and \ipa{nɯɣ-}. Of these, \ipa{nɯ-} is homonymous with many other derivational prefixes (denominal and autobenefactive-spontaneous) and even flexional prefixes (aorist/imperative directional `towards east', 2/3 plural possessive). \ipa{nɤ-} is identical to the tropative prefix, or to one allomorph of the \ipa{nɯ-} denominal prefix. Even for the last allomorph \ipa{nɯɣ-}, the facilitative of transitive verbs \ipa{nɯɣɯ-} has an irregular homophonous allomorph \ipa{nɯɣ-} with at least one verb. 

\ipa{nɯ-} is obviously the most basic allomorph, and the only one to appear with a Tibetan loanword (\ipa{rga} `like' from \ipa{dga}). 

\ipa{nɤ-} results from the fusion of the applicative \ipa{nɯ-} with the \ipa{a-} determiner of many intransitive verbs. This rule is the same as that according to which the causative \ipa{sɯ-} is realised as \ipa{sɤ-} with verbs of this type. \ipa{nɤkhu} `shout at', \ipa{nɤmdzɯ} `look after' and \ipa{nɤʑɯʑu} `wrestle with' could therefore be rewritten as \ipa{nɯ-ɤkhu}, \ipa{nɯ-ɤmdzɯ} and \ipa{nɯ-ɤʑɯʑu}.

\ipa{nɯɣ-} only appears with two examples that have a labial onset without cluster. The distribution of \ipa{nɯ-} and \ipa{nɯɣ-} might have been at an earlier stage like that of \ipa{sɯ-} and \ipa{sɯɣ-}, the latter occurring with intransitive verbs whose initial does not contain a cluster or a velar/uvular. 

For most verbs, the applicative is formally difficult to distinguish from the autobenefactive-spontaneous \ipa{nɯ-}; only the meaning and the fact that the former adds an argument, while the latter does not change the valency of  the verb, permits the two grammatical categories to be distinguished. The two prefixes however are distinct in the case of verbs prefixed with the determiner \ipa{a-}: while the applicative (being located in slot 10 of the verbal template)\footnote{See \citet{jacques12incorp} and \citet{jacques13harmonization} for a general account of the Japhug verbal template.} appears \ipa{before} the \ipa{a-}, the spontaneous-autobenefactive appears after it. The difference between \ipa{nɤkhu} `shout at' and \ipa{anɯkhu} `shout' can be analysed as follows:


\begin{tabular}{llllllll}
Position & 10 & 11 & 12 & 15 \\

& & \ipa{a-} & &\ipa{khu} \\
applicative& \ipa{nɯ-}& \ipa{a-} & &\ipa{khu} \\
autobenefactive& & \ipa{a-} & \ipa{nɯ-}&\ipa{khu} \\
\end{tabular}

One counterexample however is found: the autobenefactive-spontanenous of \ipa{atɯɣ} `to meet' is  \ipa{nɯ-ɤtɯɣ} (instead of *\ipa{anɯtɯɣ}), with this prefix in position 10.

\subsection{Compatibilities} \label{subsub:appl.compat}

 The applicative \ipa{nɯ-} is compatible with the causative \ipa{sɯ-}, with which it regularly combines as \ipa{z-nɯ-}:
    \begin{exe}
   \ex 
\gll \ipa{nɯ} 	\ipa{kɯ-fse} 	\ipa{ci} 	\ipa{jɯm} 	\ipa{ko-z-nɯ-ŋke-j} 	\ipa{tɕe,} 	\ipa{nɯ-me} 	\ipa{nɯ} 	\ipa{ɯ-rca} 	\ipa{tɤ-ɣe-j}    \\
   \dem{} \nmlz{}:S/A-be.like.that \indef{} wife \evd{}:east-\caus{}-\appl{}-walk-1\pl{} \textsc{lnk} 2\pl{}.\poss{}-girl \topic{} 3\sg{}.\poss{}-following \aor{}:up-come[II]-1\pl{}    \\
 \glt So (the king) sent us to look for a wife (for his son), and we followed your daughter here. (The prince, 70)
   \end{exe} 

The applicative can also  combine with both the deexperiencer \ipa{sɤ-} and the antipassive \ipa{sɤ-}, producing homophonous forms, as \ipa{sɤ-nɯ-rga}, which can either be interpreted as `to be likeable' (deexperiencer) or `to like people' (antipassive).
 
Additionally, the  applicative is  compatible with the reciprocal in examples such as \ipa{anɯɣbɯɣbɯɣ} `to miss each other' or \ipa{anɯrgɯrga} `to like each other' % (see \ref{subsub:recip.compat}), 
and with the reflexive as in \ipa{ʑɣɤ-nɤstu} `to believe in oneself'. %(see \ref{subsub:reflexive.compat}).
 
We failed to find other combinations of prefixes with the applicative, but this might reflect the rarity of the applicative rather than a structural principle of the language.  
 

 
\subsection{Syntactic constructions} \label{subsub:appl.syntax}
The applicative makes an intransitive (in some cases even stative) verb transitive. The original S becomes the A, and a new argument is promoted to O status. There are three types of added arguments.

First, in the case of the verbs \ipa{akhu} and \ipa{akhɤzŋga} `to shout', the  argument promoted to O status of the applicative form \ipa{nɤkhu} is a   dative adjunct, as in the following example:

\begin{exe}
   \ex 
\gll   \ipa{aʑo} \ipa{a-ɕki} \ipa{ɲɯ-ɤkhu} \\
\textsc{1sg}     \textsc{1sg-dat} \textsc{const}-call\\
 \glt  He is calling me. (elicitation, Dpalcan)
   \end{exe} 

Second, the case of \ipa{rga} `like, be glad' is quite peculiar. Although this verb has intransitive morphology, and only agrees with the experiencer (the S), it can be used with an overt stimulus, which receives no special marking, as \ipa{qrorni} `red ant' in the example (\ref{ex:qrorni}).

\begin{exe}
   \ex \label{ex:qrorni}
\gll \ipa{qrorni}  	\ipa{nɯnɯ,}  	\ipa{pri}  	\ipa{kɯ}  	\ipa{ɣɯ-tu-ndze}  	\ipa{wuma}  	\ipa{ʑo}  	\ipa{rga}  	 \\
red.ant \textsc{dem} bear \textsc{erg} \textsc{cisloc-ipfv}-eat[III] really \textsc{emph} \textsc{n.pst}:like \\
 \glt  The red ants, the bear likes to come to eat them very much. (the ants, 41)
   \end{exe} 
   
 Overt stimuli of the verb \ipa{rga} are syntactically adjuncts, despite having no locative or dative marker, and our corpus only includes examples with left dislocation of the noun phrase corresponding to the stimulus, as in example (\ref{ex:qrorni}).
 
%tɤlɤxtɕur nɯ kɯnɤ kɯ-rga tu, mɤ-kɯ-rga tu. (The sheep, 169)

With the applicative derivation, the stimulus is promoted to O status, and is indexed on the verb, as in example (\ref{ex:nWrga}), where the verb has stem 3 vowel alternation.\footnote{In Rgyalrong languages, stem 3 is used in non-past transitive direct third person O and singular A forms, see \citet{jackson00puxi} and  \citet{jacques12incorp}.}
 
 \begin{exe}
   \ex \label{ex:nWrga}
\gll \ipa{iɕqha} 	\ipa{tɕheme} 	\ipa{nɯ}  	\ipa{ɲɯ-nɯ-rge-a}   \\
the.aforementioned woman \topic{}  \textsc{ipfv-appl}-like[III]-1\textsc{sg}     \\
 \glt  I like this woman. (elicited, Dpalcan)
   \end{exe} 

The promoted argument can be relativized  as any O argument with the \ipa{kɤ-} prefix:
     \begin{exe}
   \ex 
\gll  \ipa{thaχtsa} 	\ipa{nɯ} 	\ipa{iʑo} 	\ipa{kɯrɯ} 	\ipa{tɕheme} 	\ipa{ra} 	\ipa{nɯnɯ} 	\ipa{mɤlɤn} 	\ipa{ʑo} 	\ipa{pjɯ-tu} 	\ipa{kɯ-ra} 	\ipa{tɕe}, \ipa{stu} 	\ipa{ji-kɤ-nɯ-rga} 	\ipa{ɕti,}   \\
     coloured.belt \topic{} we Tibetan woman \pl{} \dem{} absolutely \emphat{} \ipf{}-be.there \nmlz{}:S/A-have.to \textsc{lnk} most 1\pl{}.\poss{}-\nmlz{}:O-\appl{}-like \npst{}:be.\emphat{}\\
 \glt  Coloured belts are something we Tibetan women absolutely need to have, it is what we like most. (Coloured belts, 93)
   \end{exe} 
   
  Third, in the case of the other applicative verbs, the added O does not correspond to an adjunct used with its intransitive counterpart.
   \subsection{Complements}
Applicative verbs can also be used with infinitive complements:
     \begin{exe}
   \ex 
\gll  \ipa{tɯ-ŋga} 	\ipa{kɤ-χtɯ} 	\ipa{ɕ-pɯ-nɯ-ŋke-t-a}  \\
  \neu{}-clothes \inftv{}-buy \transloc{}-\aor{}:down-\appl{}-walk-\pst{}-1\sg{}\\
 \glt   I took a walk to buy clothes. (elicited, Dpalcan)
   \end{exe} 
Note that motion verbs normally appear with complements using the S/A (\ipa{kɯ-}) participle form instead, but here the applicative of `to walk' \ipa{nɯ-ŋke} `to look for', a transitive verb (unlike other motion verbs such as \ipa{ɕe} `to go', \ipa{ɣi} `to come', \ipa{rɟɯɣ} `to run' etc which are intransitive), appears with a \ipa{kɤ-} infinitive complement. In this construction, The S/A of the complement verb must be coreferent with that of the A of the applicative verb.
 
 When the S/A of the complement clause is not coreferent with the A of the applicative verb, a finite form is necessary:
      \begin{exe}
   \ex 
\gll   \ipa{ɯʑo} \ipa{ju-nɯɣi} \ipa{ɲɯ-nɯ-mbɣom-a}  \\
 he \ipf{}-come.back \const{}-\appl{}-be.in.a.hurry-1\sg{} \\
 \glt   I am looking forward to  his coming back. (elicited, Chen Zhen)
   \end{exe} 
With the applicative verb \ipa{nɯ-mbɣom} `to look forward to', finite complements are always in the imperfective form, even when the verb is in the aorist:

      \begin{exe}
   \ex 
\gll   \ipa{jɯfɕɯr} \ipa{a-ʑɯβ} \ipa{mɯ-pɯ-ɣe} \ipa{tɕe,} \ipa{lu-fsoʁ} \ipa{tɤ-nɯ-mbɣom-a}  \\
yesterday 1\sg{}.\poss{}-sleep \negat{}-\aor{}-come[II] \textsc{lnk}  \ipf{}-be.clear
 \aor{}-\appl{}-be.in.a.hurry-1\sg{} \\
 \glt   Yesterday I could not sleep, I looked forward to the daybreak. (elicited, Chen Zhen)
   \end{exe} 
 
 
 
\subsection{Semantics} \label{subsub:appl.sem}
The applicative verbs in Japhug may be divided into three groups depending on the semantics of the original verb: experiencer verbs, action verbs and reciprocal action verbs.

For the experiencer verbs (\ipa{rga} `to like', \ipa{bɯɣ} `to miss', \ipa{mu} `to be afraid' etc), the applicative adds the stimulus of the feeling:
   \begin{exe}
   \ex 
\gll \ipa{ɲɯ-ta-nɯɣ-bɯɣ-nɯ}   \\
\ipf{}-1$\rightarrow$2-\appl{}-miss-\pl{}   \\
 \glt  I miss {you_p}. (elicited, Chen Zhen)
   \end{exe} 
   
      \begin{exe}
   \ex 
\gll  \ipa{nɤʑo} 	\ipa{tɕhi} 	\ipa{tɯ-nɯɣ-me?}    \\
   you what 2-\appl{}-\npst{}:be.afraid[III]\\
 \glt   What are you afraid of? (Gesar, 378, \citealt[68]{jacques10gesar})
   \end{exe} 
 
 


 Unlike most transitive verbs, applicative verbs derived from experiencer verbs can be used with non-periphrastic past imperfective \ipa{pɯ-} or with the non-periphrastic evidential imperfective \ipa{pjɤ-}; this property is shared with tropative verbs.\footnote{ \citet[64]{lin11direction} asserted that the  non-periphrastic past imperfective  in the Datshang dialect of Japhug was restricted to stative verbs. One could argue that these applicative and tropative are indeed transitive stative verbs, a category that is attested in some active-stative languages, for instance Lakhota (see \citealt[77]{deloria41} and \citealt[707]{ullrich08}). However, in Lakhota the tropative suffix \ipa{-la} derives active (=dynamic) verbs, not transitive stative ones, out of stative verbs (as in  example \ref{ex:wakhaN}).   
 
In any case, the   non-periphrastic past imperfective does appear  with   dynamic verbs in conditionals (in the apodosis of counterfactuals), and with transitive verbs that have the progressive prefix \ipa{asɯ-}.   }
 
   \begin{exe}
   \ex 
\gll \ipa{lɯlu} 	\ipa{nɯ} 	\ipa{wuma} 	\ipa{pjɤ-nɯɣ-mu-ndʑi} 	\ipa{ɕti}   \\
cat \topic{} really \evd{}.\ipf{}-\appl{}-be.afraid-\du{} \npst{}:be.\emphat{} \\
 \glt   {They_d} were very afraid of the cat. (The mouse and the sparrow, 15)
   \end{exe} 
 
  
 With the action verbs \ipa{ŋke} `walk', \ipa{andzɯt}  `bark' and \ipa{akhu} `shout, call', the added argument is the goal towards which the action is directed:
 
   \begin{exe}
   \ex 
\gll \ipa{ɯ-rkɤrkɯ} \ipa{jilco} 	\ipa{nɯ} 	\ipa{ra} 	\ipa{tɯ-sŋi} 	\ipa{ɲo-z-nɤʁaʁ,} 	\ipa{ɲo-nɯ-ɤkhu}  \\
 3\sg{}.\poss{}-around neighbour \topic{} \pl{} one-day \evd{}-\caus{}-have.a.good.time \evd{}-\appl{}-call \\
 \glt    One day, she invited  neighbours from all places around, she invited them. (The raven 98)
   \end{exe} 
   
     \begin{exe}
   \ex 
\gll   \ipa{tɯrme}  	\ipa{mɤ-kɤ-nɯfse}  	\ipa{jɤ-ɣe}  	\ipa{tɕe}  	\ipa{tu-nɯ-ɤndzɯt}  \\
person \textsc{neg-nmzl:O}-know \textsc{aor}-come[II] \textsc{lnk} \textsc{ipfv-appl}-bark \\
  \glt  When a person that it does not know comes, it barks at him.  (The dogs, 9)
   \end{exe}  
   \begin{exe}
   \ex 
\gll \ipa{a-ɣe} 	\ipa{rɟɤlpu} 	\ipa{ɕɯ-nɯ-ɤkhu-tɕi} \\
  1\sg{}.\poss{}-grandson king \transloc{}-\appl{}-\npst{}:call-1\du{} \\
 \glt    Grandson, let us go to invite the king. (Kunbzang 342)
   \end{exe} 
  
  In the case of  \ipa{amdzɯ} `sit', whose applicative \ipa{nɤmdzɯ} means `take care for, look after', the semantic derivation is less transparent, though it reminds one of idiomatic expressions such as `baby-sitting':
   \begin{exe}
   \ex 
\gll \ipa{ki} 	\ipa{tɤpɤtso} 	\ipa{kɤ-nɯ-ɤmdzi} 	\ipa{a-mɤ-pɯ-ndʐaβ}  \\
\dem{}.\prox{} child \imp{}-\appl{}-sit[III] \irr{}-\negat-\pfv{}-\acaus{}:make.fall \\
 \glt    Look after this child, do not let him fall. (elicited, Dpalcan)
   \end{exe} 


Finally, with the intrinsically reciprocal \ipa{aʑɯʑu} `wrestle', the applicative is used as an `anti-reciprocal':

   \begin{exe}
   \ex 
\gll \ipa{tɤ́-wɣ-nɯ-ɤʑɯʑu-a} 	   \\
\aor{}-\inv{}-\appl{}-wrestle-1\sg{}  \\
 \glt    He wrestled with me. (elicited, Dpalcan)
   \end{exe} 
The applicative however cannot be combined with reciprocal derivation in any regular way. The verb \ipa{aʑɯʑu} itself is historically the reciprocal of a non-attested transitive verb *\ipa{ʑu} `wrestle', but is not analysable as such synchronically since the base verb is lost. One reason why the combination of applicative with reciprocal is not more common is that   it would be homophonous with  atelic derivation (\ipa{ŋke} `to walk' $\rightarrow$ \ipa{nɤŋkɯŋke} `to walk in all directions'). The reciprocal is formed by combining the reduplicated verb stem with the \ipa{a-} prefix. Adding the applicative \ipa{nɯ-} yields \ipa{nɯ-ɤ-} $\rightarrow$  \ipa{nɤ-} with reduplication, which is exactly the same as the atelic form of the verb. This combination would also derive a transitive verb out of a transitive one (through a stage as an intransitive reflexive), exactly as in atelic derivation.

We notice that neither place nor instrument can be promoted to O using the applicative in Japhug; its range of uses is quite limited.


\section{\uppercase{tropative}} \label{sec:tropative}
 

The tropative \ipa{nɤ-}  is a very productive derivation that can be applied to most stative verbs, having the meaning `to consider to be X'.  This derivation was briefly reported as an example of applicative  in Tshobdun (\citealt[5-6]{jackson06paisheng}).  \citet{jacques12demotion, jacques12incorp}   mentioned the existence of this derivation in Japhug, but did not provide any detail on its actual use.


In the tropative derivation,  the S of the original verb becomes the O of the derived transitive verb, while the added argument (the experiencer) becomes the A of the derived verb. This resembles  causative derivation (though the semantics is different), but differs from the applicative.

For instance, the stative verb  \ipa{mpɕɤr} `be beautiful' has the derived transitive verb \ipa{nɤ-mpɕɤr} `consider to be beautiful':


 \begin{exe}
\ex
\gll  \ipa{ɯ-mdoʁ} 	\ipa{maka} 	\ipa{mɯ́j-nɤsci} 	\ipa{tɕe,} 	\ipa{nɯ} 	\ipa{ni} 	\ipa{stu} 	\ipa{nɯ-kɤ-nɤ-mpɕɤr} 	\ipa{ɲɯ-ŋu} 
\\
3\sg{}.\poss{}-colour at.all \negat{}:\const{}-change \textsc{lnk} \distal{}.\dem{} \du{} most 3\pl{}-\nmlz{}:O-\textsc{tropative}-beautiful \ipf{}-be \\
 \glt  Its colour does not change, and these two are the ones that they consider the most beautiful. (Coloured belts, 85)
\end{exe} 

It also derives perception verbs like \ipa{nɤ-mnɤm} `to smell' and \ipa{nɤ-mŋɤm} `to feel pain' derived from \ipa{mnɤm} `to smell (it)' and \ipa{mŋɤm} `to ache (it)'.  

 
 

  \begin{exe}
\ex \label{ex:namnam}
\gll \ipa{tɯɣ}  	\ipa{kɯ-fse}  	\ipa{kɯ-tu}  	\ipa{nɯra}  	\ipa{tu-nɤmnɤm}  	\ipa{tɕe}  	\ipa{ɯ-kɯ-sɯχsɤl,}  	\ipa{nɯnɯ}  	\ipa{ɯ-kɯ-sɯχpjɤt}  	\ipa{ɲɯ-ŋu}   \\
poison  \textsc{nmlz:S/A}-be.like  \textsc{nmlz:S/A}-exist \textsc{dem:pl} \textsc{ipf}-smell \textsc{lnk} 3sg-\textsc{nmlz:S/A}-recognize \textsc{dem} \textsc{nmlz:S/A}-perceive \textsc{const}-be \\
\glt The poisonous things, it is able to recognize them when it smells them. (the buzzard, 34)
\end{exe}


 \begin{exe}
\ex \label{ex:namNam2}
\gll  \ipa{ɯ-xtu} \ipa{ɲɯ-nɤ-mŋɤm}  \\
\textsc{3.sg.poss}-belly \textsc{const}-\textsc{tropative}-ache \\
\glt  He feels pain in his belly (elicited, Chenzhen)
\end{exe}

The verb \ipa{nɤ-mnɤm} `to smell' is only used for volitional perception. For non-volitional perception,  the general  perception verb \ipa{mtshɤm} `to hear, to smell' is used instead. The verb \ipa{mtshɤm}  can be employed to refer to all types of non-visual non-volitional perception, including audition, smell, taste, vibration of an earthquake etc, as in example (\ref{ex:mtsham3}).
 \begin{exe}
\ex \label{ex:mtsham3}
\gll \ipa{tɤ-di}   	\ipa{ci}   	\ipa{kɯ-mɯmɯm}   	\ipa{ʑo}   	\ipa{pɯ-mtsham-a}   	 \\
\textsc{indef.poss}-smell \textsc{indef} \textsc{nmlz:S}-tasty \textsc{emph} \textsc{aor-}hear-\textsc{1sg} \\
 \glt I smelled a nice smell. (The lotus, 2)
\end{exe} 


As such, the tropative \ipa{nɤ-} is clearly  distinct from the applicative \ipa{nɯ-} / \ipa{nɯɣ-}, which presents a different redistribution of syntactic roles: the S of the base verb becomes the A, and an adjunct is promoted to become O, as in \ipa{bɯɣ} `to miss home (it)' vs. \ipa{nɯɣ-bɯɣ} `to miss someone (vt)'. The difference between applicative, tropative and causative can be represented as follows in (\ref{ex:three.derivations}).
  \begin{exe}
  \ex \label{ex:three.derivations}
%\glt \textsc{applicative}: S$\rightarrow$A + O
%\glt \textsc{tropative}: S$\rightarrow$O + A (experiencer)
%\glt \textsc{causative}: S$\rightarrow$O + A (causer)
% \end{exe} 
\glt \begin{tabular}{llll}
 \textsc{applicative}: &S$\rightarrow$A + O\\
  \textsc{tropative}: &S$\rightarrow$O + A (experiencer)\\
 \textsc{causative}: &S$\rightarrow$O + A (causer)\\
\end{tabular}
\end{exe} 
Table \ref{tab:tropative}  shows additional examples of tropative verbs in Japhug; this list is far from exhaustive, as the tropative is a fully productive derivation which can be applied to most stative verbs.

\begin{table}[h]
\caption{Examples of the \ipa{nɤ}- tropative prefix in Japhug}\label{tab:tropative} \centering
\begin{tabular}{lllllllll} \toprule 
basic verb  & &derived  verb &\\
\midrule
 \ipa{wxti} & be big & \ipa{nɤ-wxti} & consider to be  big \\
 \ipa{zri} & be long & \ipa{nɤ-zri} & consider to be  long \\
       \midrule
  \ipa{chi} &be sweet & \ipa{nɤx-chi}  &consider to be  sweet \\
    %\ipa{mnɤm} & have an odour & \ipa{nɤ-mnɤm} & smell (tr.) \\
  \ipa{maʁ} & not be & \ipa{nɤɣ-maʁ} & consider to not to be right \\
  \ipa{mbat} & be easy & \ipa{nɤɣ-mbat} & finish easily \\
\bottomrule
\end{tabular}
\end{table}
The examples show that the morphophonology and semantic derivation of this prefix is not entirely straightforward.

Aside from the regular \ipa{nɤ-} allomorph, one also finds a \ipa{nɤɣ-} / \ipa{nɤx-} allomorph on a few verbs. A similar allomorphy is found with the causative prefix (which has \ipa{sɯɣ-} and \ipa{ɕɯɣ-} allomorphs alongside \textit{sɯ-}) and the applicative \ipa{nɯ-} (which has the allomorph \ipa{nɯɣ-} in a few examples). The  \ipa{sɯ-} vs. \ipa{sɯɣ-} allomorphy is still productive: the latter allomorph is found when the original verb is intransitive, without an initial consonant cluster and without initial velar or uvular. It is possible that a similar distribution used to exist  at a former stage for the \ipa{nɤ-} / \ipa{nɤɣ-} allomorphs, but the data at hand do not permit a firm conclusion.

Moreover, the semantics of the derived verb is not always simply `to consider to be X'. In the case of stative verbs whose meaning is neutral (not explicitly positive like `beautiful'), the tropative often has the additional meaning `to consider to be too X'. Second, in the case of \ipa{nɤɣ-maʁ} `consider to not to be right', it seems that of of the original meanings of   \ipa{maʁ}  was `not to be right', as the nominalized form \ipa{kɯ-maʁ} can mean `something which is not right'. Here the tropative preserved the original meaning of the verb, and the base verb underwent an independent semantic change.

Like some of the applicative verbs,  tropative verbs can be used with non-periphrastic past imperfective, as in example (\ref{sec:pWwGnAmWm}).

 \begin{exe}
\ex \label{sec:pWwGnAmWm}
\gll
 \ipa{ɯ-tɯ-tɕur}  	\ipa{mɤ-tɕhom}  	\ipa{tɕe,}  	\ipa{pɯ́-wɣ-nɤ-mɯm}  	\ipa{ɕti.}  \\
\textsc{3sg-nmlz:degree}-sour \textsc{neg-n.pst}:be.exceedingly \textsc{lnk} \textsc{pst.ipfv-inv-tropative}-be.tasty \textsc{n.pst}:be.\textsc{affirm} \\
 \glt It is not too sour, and we used to find it tasty. (ɴɢojom, 52)
\end{exe}  

The use of the evidential or the perfective with tropative verbs indicates a change of state. For instance, \ipa{to-nɤ-mɯm} (\textsc{evd-tropative}-be.tasty) means `(he used not to find it tasty, but now) he finds it tasty'.

\subsection{The tropative and other derivations}

The tropative is compatible with other derivation prefixes, in particular the deexperiencer \ipa{sɤ-} (see \citealt{jacques12demotion}). For instance, we observe the derivation chain:

 \begin{exe}
\ex
 \glt  \ipa{scit} `happy' (of a person)
 \glt Deexperiencer: \ipa{sɤ-scit} `funny, nice'. The literal meaning of this stative verb is in fact `to be such that people are happy'. 
 \glt Tropative: \ipa{nɤ-sɤ-scit} `to consider to be nice'
\end{exe} 
The doubly derived verb can be illustrated by the following example:
 \begin{exe}
\ex
\gll   \ipa{nɯ-sɤ-rma} 	\ipa{pjɯ-ɣɤrcoʁ} 	\ipa{pjɤ-ra} 	\ipa{ma} 	\ipa{kɯ-zbaʁ} 	\ipa{nɯ} 	\ipa{mɯ́j-nɤ-sɤ-scit-nɯ} 	\ipa{ɲɯ-ŋgrɤl.} \\
 3\textsc{pl.poss}-\textsc{nmlz:oblique}-live \textsc{ipf}-be.muddy \textsc{evd.ipf}-have.to because \textsc{nmlz:S/A}-dry \textsc{top} \textsc{neg:const}-\textsc{tropative}-\textsc{deexperiencer}-happy-\textsc{pl} \textsc{ipf}-be.usually.the.case \\
 \glt   (The frogs)  had to live in a muddy place, because they did not like dry (places).  (Aesop adaptation, the frogs)
\end{exe} 
  
 It is however impossible to combine the tropative with the reflexive \ipa{ʑɣɤ-}. To express the meaning `to consider oneself X', one uses a different construction, with the complex prefix \ipa{znɤ-} and reduplication of the verb stem:
 
 
   \begin{exe}
\ex
 \glt   \ipa{ɕqraʁ} `be intelligent' $\rightarrow$  \ipa{znɤ-ɕqraʁ-ɕqraʁ} `consider oneself intelligent'.
\end{exe} 
 
   \begin{exe}
\ex
\gll \ipa{tɯ-znɤɕqraʁɕqraʁ} 	\ipa{nɯɣe,} 	\ipa{kɯmtɕhɯ} 	\ipa{ra} 	\ipa{mɯ́j-ra} 	\ipa{ɣe}  \\
 2-\npst{}:\trop{}:\refl{}:intelligent isn't.it toy \pl{} \negat{}:\const{}-have.to isn't.it \\
  \glt   You think you are so smart, you don't need toys, don't you? (Conversation 2003, 68)
   \end{exe}
 
  
 \subsection{Other tropative constructions} \label{sub:other}
Apart from the tropative prefix \ipa{nɤ-}, we find two other  constructions which can express a tropative meaning in Japhug.
 
 
 First, the causative  prefix \ipa{sɯ-} / \ipa{sɯɣ-} / \ipa{z-} appears to have a tropative semantics in two verbs: \ipa{znɤja} `consider to be a shame' and \ipa{znɤkɤro} `consider to be acceptable'.

The intransitive verb \ipa{nɤja} means `to be a shame, to be a pity'.
  \begin{exe}
\ex
\gll \ipa{iɕqha} 	\ipa{laχtɕha} 	\ipa{pjɤ-ɴɢrɯ,} 	\ipa{pɯ-nɤja} \\
the.aforementioned thing \evd{}-\acaus{}:break \aor{}-be.a.shame \\
  \glt  That thing broke, what a shame! (elicited, Chenzhen)
   \end{exe}

   The transitive \ipa{z-nɤja}, rather than meaning `to cause to be a shame' as expected regularly, rather means `to regret, be reluctant', in other words `to consider something to be a pity':
   
     \begin{exe}
\ex 
\gll \ipa{wuma} 	\ipa{ʑo} 	\ipa{pɯ-znɤja-t-a}  \\
really \emphat{} \aor{}-regret-\pst{}-1\sg{} \\
  \glt  I regretted it very much (about a lost cellphone cover, Dpalcan, conversation, 2010)
   \end{exe}
   
 
   
  The other way to express the same meaning is to use the estimative verb  
  verb \ipa{sɯpa} `to consider, to regard as' and combine it with a nominalized stative verb.

     \begin{exe}
\ex 
\gll \ipa{tɤkhe-pɣɤtɕɯ} 	\ipa{nɯ} 	\ipa{ɯʑo} 	\ipa{pɣɤtɕɯ} 	\ipa{nɯ} 	\ipa{kɯ-khe} 	\ipa{tu-sɯpa-nɯ} \\
stupid-bird \topic{} he bird \topic{} \nmlz{}:S/A-stupid \ipf{}-consider-\pl{} \\
 \glt The \ipa{tɤkhe-pɣɤtɕɯ} is considered to be a stupid bird (the buzzard, 13)
   \end{exe}
 
 This construction with  \ipa{sɯpa} expresses the same meaning as the tropative derivation but with a complement clause, in particular in the case of verbs for which this derivation is not appropriate.
  
 \section{\uppercase{conclusion}}
 This paper has shown that apart from the causative prefixes, Japhug has two other valency-increasing derivations, the applicative and the tropative. While superficially similar, these two derivations are morphosyntactically and semantically quite distinct. Their similar shape (\ipa{nɯ-} for the applicative and \ipa{nɤ-} for the tropative) suggests a   common origin. It is possible that they derive historically from the transitive denominal derivations of action nouns (see \citealt{jacques14antipassive}).
 
Rgyalrong languages appear to be the only languages in the Sino-Tibetan family to present these prefixes, and it could be one of the many defining common innovations of the Rgyalrong branch. In the closely related Wobzi Lavrung language, \citet[165]{lai13affixale} proposed to interpret the \ipa{n-} prefix in the verb \ipa{n-lələ̀m} `to smell' as a potential trace of the tropative derivation in this language, but more research is needed to confirm this idea.
 
 \titleformat{\section}{\large\bfseries\center}{}{0pt}{}

\bibliographystyle{linquiry2}
\bibliography{bibliogj}
\end{document}
