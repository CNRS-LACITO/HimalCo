\documentclass[twoside,a4paper,11pt]{article} 
\usepackage{polyglossia}
\usepackage{natbib}
\usepackage{booktabs}
\usepackage{xltxtra} 
 \usepackage{geometry}
\usepackage[usenames,dvipsnames,svgnames,table]{xcolor}
\usepackage{multirow,slashbox}
%\usepackage{gb4e} 
\usepackage{multicol}
\usepackage{graphicx}
\usepackage{float}
\usepackage{varioref,hyperref} 
\hypersetup{colorlinks=true,linkcolor=blue,citecolor=blue}
\usepackage{memhfixc}
\usepackage{lscape}
\usepackage[footnotesize,bf]{caption}


%%%%%%%%%%%%%%%%%%%%%%%%%%%%%%%
\setmainfont[Mapping=tex-text,Numbers=OldStyle,Ligatures=Common]{Charis SIL} 
%\setsansfont[Mapping=tex-text,Ligatures=Common,Mapping=tex-text,Ligatures=Common,Scale=MatchLowercase]{Ubuntu} 
\newfontfamily\phon[Mapping=tex-text,Ligatures=Common,Scale=MatchLowercase]{Charis SIL} 
%\newfontfamily\phondroit[Mapping=tex-text,Ligatures=Common,Scale=MatchLowercase]{Doulos SIL} 
%\newfontfamily\greek[Mapping=tex-text,Scale=MatchLowercase]{Galatia SIL} 
\newcommand{\ipa}[1]{{\phon\textit{#1}}} 
\newcommand{\ipab}[1]{{\phon #1}}
\newcommand{\ipapl}[1]{{\phondroit #1}}
\newcommand{\captionft}[1]{{\captionfont #1}} 
%\newfontfamily\cn[Mapping=tex-text,Scale=MatchUppercase]{IPAGothic}%pour le chinois
%\newcommand{\zh}[1]{{\cn #1}}
\newcommand{\tgf}[1]{\mo{#1}}
%\newfontfamily\mleccha[Mapping=tex-text,Ligatures=Common,Scale=MatchLowercase]{Galatia SIL}%pour le grec

\newcommand{\sg}{\textsc{sg}}
\newcommand{\pl}{\textsc{pl}}
\newcommand{\grise}[1]{\cellcolor{lightgray}\textbf{#1}} 
\newcommand{\Σ}{\greek{Σ}}
\newcommand{\ro}{$\Sigma$}
\newcommand{\ra}{$\Sigma_1$} 
\newcommand{\rc}{$\Sigma_3$}  
\newfontfamily\cn[Mapping=tex-text,Ligatures=Common,Scale=MatchUppercase]{SimSun}%pour le chinois
\newcommand{\zh}[1]{{\cn #1}}



\begin{document}

\title{La phonologie historique de l'arapaho et la linguistique panchronique }

\author{Guillaume JACQUES }
%\date{}
\maketitle

\section{Introduction}
 	Parmi les langues du nouveau monde, une des seules familles, sinon la seule, pour laquelle on dispose d'un système de reconstruction d'un rigueur comparable à celle de l'indo-européen est l'algonquien. Les travaux sur cette famille ont commencé dès l'arrivée des colons en Amérique du Nord, car c'était la famille de langue la plus répandue sur la côte du nord-est. de l'Amérique du nord.
 	
	On dispose de documents riches dès le dix-septième siècle, en particulier la bible de John Eliot en wampanoag, datant de 1663. La linguistique historique de l'algonquien a une tradition très ancienne pour une famille non-européenne: la première correspondance phonétique portant sur les langues algonquienne a été proposée en 1643 par Roger Williams.
	
	
	La reconstruction du proto-algonquien de Bloomfield qui fait toujours autorité, pourtant, a été basée au moins dans ses débuts exclusivement sur l'étude de langues modernes sur lesquelles l'auteur avait fait du terrain, et qui contrairement aux langues anciennes, déjà disparues à son époque, ne posaient pas de problèmes d'interprétation.
	
	Dans cet article, on s'intéressera dans un premier temps à l'histoire de la reconstruction du proto-algonquien, puis à son application à une des langues les plus divergentes du point de vue phonologique: l'arapaho. On évaluera également la contribution des langues algonquiennes à la phonologie panchronique; on verra en particulier qu'un certain nombre de changements observé dans diverses langues algonquiennes n'ont pas d'équivalent dans les langues indo-européennes au même dans toute l'Eurasie (voir en particulier \citealt{kuemmel07wandel} pour une synthèse des changements consonantiques en indo-européen,  en sémitique et en ouralique).
	
\section{Reconstruction du proto-algonquien}	
La reconstruction rigoureuse du proto-algonquien commence avec l'article de \citet{bloomfield25central}, basé sur quatre langues: le fox, le cree des plaines et le menominee, sur lesquelles il avait effectué un travail de première main, et l'ojibwe, en utilisant les données de \citet{jones17ojibwe}. Dans ce travail, Bloomfield se propose de reconstruire ``l'algonquien central'', et néglige sciemment les langues de l'ouest (le blackfoot, l'arapho et le cheyenne) dont la divergence est bien connue, et les langues algonquiennes de l'est (delaware, abenaki, micmac etc) sur lesquelles il n'avait pas d'expérience de première main.

Le système reconstruit est très proche du fox (qui préserve les voyelles du proto-algonquien quasiment inchangée); le système vocalique comprend seulement *a *e *i *o et leurs variantes longues; dans son article de 1925, Bloomfield distingue aussi une voyelle additionnelle *\ipa{ɪ} que l'on reconstruit maintenant comme une combinaison *yi.

Le système consonantique n'admet que des groupes initiaux ayant une semi-voyelle *w ou *y comme second élément, et les consonnes possibles sont indiqués dans le tableau \ref{tab:c.simple};\footnote{Les voisées de l'ojibwe et les sourdes sont en distribution complémentaires, au moins dans le dialecte utilisé par Bloomfield.} les reconstructions pour ces correspondances sont toujours admises (mis à part que *l est parfois reconstruit *r, et que *θ avait peut-être pour valeur phonétique plutôt *[ɬ]).

\begin{table}[H]
\caption{Correspondances des groupes en --k-- dans les langues algonquiennes centrales.} \centering  \label{tab:c.simple}
\begin{tabular}{lllllll}
\toprule
Proto-algonquien & fox & ojibwe & cree des plaines & menomini \\
\midrule
\ipa{*p} & 	\ipa{p} & 	\ipa{p/b} & 	\ipa{p} & 	\ipa{p} & 	\\
\ipa{*t} & 	\ipa{t} & 	\ipa{t/d} & 	\ipa{t} & 	\ipa{t} & 	\\
\ipa{*č} & 	\ipa{č} & 	\ipa{č/dž} & 	\ipa{ts} & 	\ipa{ts} & 	\\
\ipa{*k} & 	\ipa{k} & 	\ipa{k/g} & 	\ipa{k} & 	\ipa{k} & 	\\
\ipa{*m} & 	\ipa{m} & 	\ipa{m} & 	\ipa{m} & 	\ipa{m} & 	\\
\ipa{*n} & 	\ipa{n} & 	\ipa{n} & 	\ipa{n} & 	\ipa{n} & 	\\
\ipa{*s} & 	\ipa{s} & 	\ipa{s/z} & 	\ipa{s} & 	\ipa{s} & 	\\
\ipa{*š} & 	\ipa{š} & 	\ipa{š/ž} & 	\ipa{s} & 	\ipa{s} & 	\\
\ipa{*θ} & 	\ipa{n} & 	\ipa{n} & 	\ipa{t} & 	\ipa{n} & 	\\
\ipa{*l} & 	\ipa{n} & 	\ipa{n} & 	\ipa{y} & 	\ipa{n} & 	\\
\ipa{*h} & 	\ipa{h} & 	\ipa{h} & 	\ipa{h} & 	\ipa{h} & 	\\
\bottomrule
\end{tabular}
\end{table}

Si les correspondances des consonnes simples sont relativement triviales en arapaho, il n'en va pas de même des groupes de consonnes, dont la reconstruction est moins certaine. Le tableau \ref{tab:clusters.k} représente les groupes ayant *k pour deuxième élément reconstruits par Bloomfield en 1925 (la transcription a été légèrement modifiée). On observe en particulier que le contraste entre les  groupes  \ipa{*çk}, \ipa{*xk} et \ipa{*šk} n'existe tel quel dans aucune des langues utilisées dans cet article; certaines confondent \ipa{*çk}  et \ipa{*šk} (ojibwe, fox) tandis que d'autres confondent \ipa{*çk}  et \ipa{*hk} (cree, menominee).
 

\begin{table}[H]
\caption{Correspondances des groupes en --k-- dans les langues algonquiennes centrales.} \centering  \label{tab:clusters.k}
\begin{tabular}{lllll}
\toprule
Proto-algonquien & fox & ojibwe & cree des plaines & menomini \\
\midrule
\ipa{*čk} & \ipa{hk} & \ipa{šk} & \ipa{sk} & \ipa{tsk} \\
\ipa{*šk} & \ipa{šk} & \ipa{šk} & \ipa{sk} & \ipa{sk} \\
\ipa{*xk} & \ipa{hk} & \ipa{hk} & \ipa{sk} & \ipa{hk} \\
\ipa{*hk} & \ipa{hk} & \ipa{kk} & \ipa{hk} & \ipa{hk} \\
\ipa{*çk} & \ipa{šk} & \ipa{šk} (en fait \ipa{sk}) & \ipa{hk} & \ipa{hk} \\
\ipa{*nk} & \ipa{g} & \ipa{ng} & \ipa{hk} & \ipa{hk} \\
\bottomrule
\end{tabular}
\end{table}

Toutefois, Bloomfield découvre quelques années plus tard un dialecte du cree (le swampy cree) dans lequel le groupe \ipa{*--çk--} apapraît comme \ipa{--htk--} et est donc distinct de \ipa{*çk}  et \ipa{*hk} (\citealt{bloomfield28thk}). Cette découverte montre que même dans les langues à tradition orales, le postulat de la régularité des changements phonétique est valide, et qu'il est justifier de reconstruire des unités phonologiques distinctes (phonème unique ou groupes de phonèmes) dans la proto-langue pour rendre compte de correspondances entre langues attestées, même si ces unités phonologiques ne sont distinctes dans aucune des langue modernes. 

\citet{bloomfield46proto} découvre ensuite que l'ojibwe faisait en fait la distinction entre  \ipa{*--çk--} et \ipa{*--šk--}, qui apparaissent comme \ipa{--sk--} et \ipa{--šk--} respectivement, mais William Jones, sur les données duquel il s'était basé, locuteur natif du fox, avait omis cette distinction qui n'existait pas dans sa langue maternelle en transcrivant l'ojibwe, et avait ainsi induit Bloomfield en erreur.

Aux groupes reconstruits dans le tableau \ref{tab:clusters.k}, il faut ajouter la distinction entre \ipa{*--xk--} et \ipa{*--θk--} découverte par \citet{siebert41clusters} sur la base des données de langues algonquiennes de l'est. 

Les seules autres modifications apportés au système de Bloomfield sont la découverte du groupe  \ipa{*--hm--} par \citet{goddard79comparative} et la distinction entre *\ipa{kw} et \ipa{kʷ} proposée par \citet{pentland79phd}. Mis à part ces quelques groupes mineurs, le système   de Bloomfield permet de rendre compte non seulement des langues algonquiennes centrales, mais aussi de l'ensemble de la famille à l'exception du blackfoot. La notation du proto-algonquien a été légèrement modifiée (voir \citealt{goddard98arapaho}). On note à présent \ipa{*r} ce que Bloomfield transcrivait \ipa{*l}, et \ipa{*--rk--},  \ipa{*--sk--} ce qu'il écrivait \ipa{*--çk--},  \ipa{*--xk--}.


Le succès de la reconstruction de Bloomfield  est dû à une combinaison de facteurs: l'existence du fox, une langue extrêmement conservatrice, le degré modéré d'emprunts entre les langues étudiées, la proximité entre ces langues,  et le fait que l'auteur avait une connaissance de première main (sauf sur l'ojibwe) et qu'il n'avait pas eu recours aux données anciennes difficile à interpréter.

\section{Application à l'arapaho}	
L'arapaho et les langues prochement apparentées que sont le gros-ventre et le nawathinehena présentent des changements phonétiques inhabituels qui rendent méconnaissables les cognats avec les langues algonquiennes plus conservatrices. Si \citealt{kroeber16arapaho} et \citealt{michelson35shifts} ont mis en évidences les premières correspondances phonétiques enter arapaho et le reste de l'algonquien, c'est seulement  \citet{goddard74arapaho} qui est parvenu à rendre compte de la phonologie historique de l'arapaho, sur la base des données de \citet{salzmann56phono}, \citet{salzmann63arapaho} et \citet{taylor67atsina}, plus fiables que celles de Kroeber.
 
 Les correspondances de base des voyelles sont relativement simples: \ipa{*a}, \ipa{*e} et \ipa{*o/we/i} correspondent respectivement à \ipa{o}, \ipa{e} et \ipa{i}; les voyelles finales, ainsi que d'autres voyelles dans des conditions plus spécifiques  chutent sans laisser de traces. Par ailleurs, des phénomènes d'harmonie vocaliques changent \ipa{e} en \ipa{o} , \ipa{i} en \ipa{u} [ʉ~ɯ] (une voyelle encore quasiment en distribution complémentaire avec \ipa{i}), et \ipa{e} en \ipa{o} dans des conditions complexes et encore imparfaitement comprises.
 
Les correspondances générales des consonnes simples du proto-algonquien en arapaho sont présentées en \ref{tab:c.simple.arapaho} sous forme simplifiée, en début de mot et entre deux voyelles.\footnote{Les réflexes des groupes de consonnes du proto-algonquien en arapaho ne sont pas traités ici.} Certaines consonnes finales chutent avec la voyelle finale (en particulier les nasales).
 \begin{table}[H]
\caption{Correspondances du proto-algonquien à l'arapaho.} \centering  \label{tab:c.simple.arapaho}
\begin{tabular}{lllllll}
\toprule
Proto-algonquien & début de mot  & autre \\
\midrule
\ipa{*p} & 	\ipa{k/c} & 	\ipa{k/c} &   	\\
\ipa{*t} & 	\ipa{t} & 	\ipa{t} & 	  	\\
\ipa{*č} & 	\ipa{θ} & 	\ipa{θ}   	\\
\ipa{*k} & 	$\emptyset$ & 	$\emptyset$   	\\
\ipa{*m} & 	\ipa{w/b} & 	\ipa{b/w}  & 	  	\\
\ipa{*n} & 	\ipa{n} & 	\ipa{n} & 	  	\\
\ipa{*s} & 	\ipa{n} & 	\ipa{h} & 	  	\\
\ipa{*š} & 	\ipa{x/s} & 	\ipa{x/s}  & 	  & 	\\
\ipa{*θ} & 	\ipa{θ} & 	\ipa{θ} & 	  	\\
\ipa{*l} & 	\ipa{n} & 	\ipa{n} & 	  	\\
\ipa{*h} & 	\ipa{h} & 	\ipa{'} & 	  	\\
\bottomrule
\end{tabular}
\end{table}

Les consonnes \ipa{*p}, \ipa{*m} et \ipa{*š} du proto-algonquien ont deux formes en arapaho selon que les voyelles qui suivent ou qui précèdent sont antérieures ou postérieures: les formes \ipa{k}, \ipa{w} et \ipa{x} apparaissent devant \ipa{o} et \ipa{u}, et \ipa{c}, \ipa{b} et \ipa{s} devant \ipa{e} et \ipa{i}. La distribution complémentaire n'est toutefois pas parfaite, car les groupes \ipa{*py/w} et \ipa{*my/w} du proto-algonquien deviennent \ipa{c} et \ipa{b} même devant voyelle postérieure. Ainsi \ipa{cóóθo'} `ennemis' provient de \ipa{*pwaaθaki} (Ojibwe \ipa{bwaanag} 'sioux') et \ipa{bóoó} `chemin' de \ipa{*myeehkani} (Ojibwe \ipa{miikan}).

Dans la suite de cette contribution, on s'intéresse à deux des changements qui ont eu lieu entre le proto-algonquien et l'arapaho: \ipa{*p} $\rightarrow$ \ipa{k/c} d'une part et \ipa{*s} $\rightarrow$ \ipa{n} d'autre part.

\subsection{\ipa{*p} $\rightarrow$ \ipa{k/c} }
Le passage de \ipa{*p} à \ipa{k}, déjà connu de \citet{kroeber16arapaho} et \citet{michelson35shifts}, participe à un changement en chaine:

\begin{itemize}
\item \ipa{*k} $\rightarrow \emptyset$
\item \ipa{*p} $\rightarrow$ \ipa{*k} $\rightarrow$  \ipa{k/c}
\end{itemize}

Le  \ipa{*k} du proto-algonquien disparaît dans tous les contextes, y compris dans les groupes, où il laisse parfois une trace indirecte (ainsi le proto-algonquien \ipa{*--θk--} devient \ipa{x} ou \ipa{s} et non \ipa{θ} en arapaho, car un changement \ipa{*--θk--} $\rightarrow $  \ipa{*--šk--} a lieu avant la chute de \ipa{*k}).

Si des cas de changement de labiale à vélaire sont attestés dans des contextes très spécifiques en indo-européen (par exemple dans le cas de premier élément de groupes de consonnes comme *\ipa{-pt-} $\rightarrow$ *\ipa{-xt-} en celtique), l'arapaho est la seule langue au monde à présenter le changement \textit{hors contexte} de  \ipa{*p} à \ipa{k}. 

 \begin{table}[H]
\caption{Exemples du changement \ipa{*p} $\rightarrow$ \ipa{*k} $\rightarrow$  \ipa{k/c}} \centering  \label{tab:p.k}
\begin{tabular}{lllllll}
\toprule
Proto-algonquien & Arapaho & Autre langue (Ojibwe)\\
\midrule
 \ipa{*kespak-yaa--} `` être épais" & \textit{hookoyóó}--    &  \textit{gipagaa}-- \\

\bottomrule
\end{tabular}
\end{table}
 

\subsection{\ipa{*s} $\rightarrow$ \ipa{n} }

 \begin{table}[H]
\caption{Exemples du changement \ipa{*p} $\rightarrow$ \ipa{*k} $\rightarrow$  \ipa{k/c}} \centering  \label{tab:p.k}
\begin{tabular}{lllllll}
\toprule
Proto-algonquien & Arapaho & Autre langue (Ojibwe)\\
\midrule
 \ipa{*siipiiwi} `` fleuve" &  \ipa{níícíí}       &  \ipa{ziibi} \\
\ipa{*sakimeewa} ``moustique" & \ipa{nóúbee}  &  \ipa{zagime} \\
\ipa{*sarak-yaa--} ``être faux" & \textit{nónoyoo}--  & \textit{zanagizi} ``avoir des difficultés" (*sanak-esi--) \\
\bottomrule
\end{tabular}
\end{table}

 \ipa{*siipiiwi} ``river" >  Ar \ipa{níícíí}   compare Ojibwe \ipa{ziibi}

\ipa{*sakimeewa} ``mosquito" > Arapaho \ipa{nóúbee}  compare Ojibwe \ipa{zagime}

 
Arapaho \textit{niiθóun}-- \textsc{vta} ``to milk" originates from PA *sii-θakw-en-.   \citet{hewson93proto} compared  Ojibwe \textit{ziinin}-- \textsc{vta} ``to milk" with Menomini \textit{seenenen-ɛɛw} ``he squeezes him in his hand" and Cree \textsc{vti} \textit{siin-eew} ``he wrings", and proposed a reconstruction *siin-en-(eew), with initial *siin-- and \textsc{vta} final *-en-- ``by hand". 

\citealt{jacques13arapaho}
\citealt{pentland98}

\begin{table}[H]
\caption{Development of some Arapaho  consonants in initial position, hypothesis I}  \centering
\begin{tabular}{lllllllllllll}
&PA & 1 & 2 & 3 & 4 & 5 & 6 & \\
&*n   &&&&& > n\\
&*w   & >*y  \grise{} & \grise{} & \grise{}  &>*r	  \grise{} &> n \grise{}\\
&*s  &  &>*z &>*r	 \grise{}& 	 \grise{} &> n \grise{}\\
&*r	   & 	&  & 	& 	  &> n \grise{}\\
&*θ >  & &&&&&θ \\
&*t > & &&&&&t \\
\end{tabular}
\end{table}


\begin{table}[H]
\caption{Development of some Arapaho   consonants in initial position, hypothesis II}   \centering
\begin{tabular}{lllllllllllll}
&PA & 1 & 2 & 3 & 4 & 5 & 6   \\
&*n  &&&&&& n\\
&*w   && >*y  \grise{}   & \grise{}& \grise{}   &>*l \grise{} &> n \grise{}\\
&*s  && &  >*ɬ  &>*l \grise{}&  \grise{} & > n \grise{}\\
&*r > *l	 &  &  &    &  &    &> n \grise{}\\
&*θ  &  >θ\\
\end{tabular}
\end{table}


\section{Conclusion}	


 \bibliographystyle{linquiry2}
 \bibliography{bibliogj}

\end{document}