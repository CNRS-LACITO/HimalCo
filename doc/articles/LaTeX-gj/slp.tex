\documentclass[twoside,a4paper,11pt]{article} 
\usepackage{polyglossia}
\usepackage{natbib}
\usepackage{booktabs}
\usepackage{xltxtra} 
 \usepackage{geometry}
\usepackage[usenames,dvipsnames,svgnames,table]{xcolor}
\usepackage{multirow}
\usepackage{gb4e} 
\usepackage{multicol}
\usepackage{graphicx}
\usepackage{float}
\usepackage{varioref,hyperref} 
\hypersetup{colorlinks=true,linkcolor=blue,citecolor=blue}
\usepackage{memhfixc}
\usepackage{lscape}
\usepackage[footnotesize,bf]{caption}


%%%%%%%%%%%%%%%%%%%%%%%%%%%%%%%
\setmainfont[Mapping=tex-text,Numbers=OldStyle,Ligatures=Common]{Charis SIL} 
%\setsansfont[Mapping=tex-text,Ligatures=Common,Mapping=tex-text,Ligatures=Common,Scale=MatchLowercase]{Ubuntu} 
\newfontfamily\phon[Mapping=tex-text,Ligatures=Common,Scale=MatchLowercase]{Charis SIL} 
%\newfontfamily\phondroit[Mapping=tex-text,Ligatures=Common,Scale=MatchLowercase]{Doulos SIL} 
%\newfontfamily\greek[Mapping=tex-text,Scale=MatchLowercase]{Galatia SIL} 
\newcommand{\ipa}[1]{{\phon\textit{#1}}} 
\newcommand{\ipab}[1]{{\phon #1}}
\newcommand{\ipapl}[1]{{\phondroit #1}}
\newcommand{\captionft}[1]{{\captionfont #1}} 
%\newfontfamily\cn[Mapping=tex-text,Scale=MatchUppercase]{IPAGothic}%pour le chinois
%\newcommand{\zh}[1]{{\cn #1}}
\newcommand{\tgf}[1]{\mo{#1}}
%\newfontfamily\mleccha[Mapping=tex-text,Ligatures=Common,Scale=MatchLowercase]{Galatia SIL}%pour le grec

\newcommand{\sg}{\textsc{sg}}
\newcommand{\pl}{\textsc{pl}}
\newcommand{\grise}[1]{\cellcolor{lightgray}\textbf{#1}} 
\newcommand{\Σ}{\greek{Σ}}
\newcommand{\ro}{$\Sigma$}
\newcommand{\ra}{$\Sigma_1$} 
\newcommand{\rc}{$\Sigma_3$}  
\newfontfamily\cn[Mapping=tex-text,Ligatures=Common,Scale=MatchUppercase]{SimSun}%pour le chinois
\newcommand{\zh}[1]{{\cn #1}}



\begin{document}

\title{La phonologie historique de l'arapaho et la linguistique panchronique }

\author{Guillaume JACQUES }
%\date{}
\maketitle

\section{Introduction}
 	Parmi les langues du nouveau monde, une des seules familles, sinon la seule, pour laquelle on dispose d'un système de reconstruction d'un rigueur comparable à celle de l'indo-européen est l'algonquien. Les travaux sur cette famille ont commencé dès l'arrivée des colons en Amérique du Nord, car c'était la famille de langue la plus répandue sur la côte du nord-est. 
 	
	On dispose de documents riches dès le dix-septième siècle, en particulier la bible de John Eliot en wampanoag, datant de 1663. La linguistique historique de l'algonquien a une tradition très ancienne pour une famille non-européenne: la première correspondance phonétique portant sur les langues algonquienne a été proposée en 1643 par Roger Williams.
	
	
	La reconstruction du proto-algonquien de Bloomfield qui fait toujours autorité, pourtant, a été basée au moins dans ses débuts exclusivement sur l'étude de langues modernes sur lesquelles l'auteur avait fait du terrain, et qui contrairement aux langues anciennes, déjà disparues à son époque, ne posaient pas de problèmes d'interprétation.
	
	Dans cet article, on s'intéressera dans un premier temps à l'histoire de la reconstruction du proto-algonquien, puis à son application à une des langues les plus divergentes du point de vue phonologique: l'arapaho. On évaluera également la contribution des langues algonquiennes à la phonologie panchronique; on verra en particulier qu'un certain nombre de changements observé dans diverses langues algonquiennes n'ont pas d'équivalent dans les langues indo-européennes au même dans toute l'Eurasie (voir en particulier \citet{kuemmel07wandel} pour une synthèse des changements consonantiques en indo-européen,  en sémitique et en ouralique).
	
\section{Reconstruction du proto-algonquien}	
La reconstruction rigoureuse du proto-algonquien commence avec l'article de \citet{bloomfield25central}, basé sur quatre langues: le fox, le cree des plaines et le menominee, sur lesquelles il avait effectué un travail de première main, et l'ojibwe, en utilisant les données de \citet{jones17ojibwe}. Dans ce travail, Bloomfield se propose de reconstruire ``l'algonquien central'', et néglige sciemment les langues de l'ouest (le blackfoot, l'arapho et le cheyenne) dont la divergence est bien connue, et les langues algonquiennes de l'est (delaware, abenaki, micmac etc) sur lesquelles il n'avait pas d'expérience de première main.

Le système reconstruit est très proche du fox (qui préserve les voyelles du proto-algonquien quasiment inchangées); le système vocalique comprend seulement *a *e *i *o et leurs variantes longues; dans son article de 1925, Bloomfield distingue aussi une voyelle additionnelle *\ipa{ɪ} que l'on reconstruit maintenant comme une combinaison *yi.

Le système consonantique n'admet que des groupes initiaux ayant une semi-voyelle *w ou *y comme second élément, et les consonnes possibles sont indiqués dans le tableau \ref{tab:c.simple};\footnote{Les voisées de l'ojibwe et les sourdes sont en distribution complémentaires, au moins dans le dialecte utilisé par Bloomfield.} les reconstructions pour ces correspondances sont toujours admises.

\begin{table}[h]
\caption{Correspondances des groupes en --k-- dans les langues algonquiennes centrales.} \centering  \label{tab:c.simple}
\begin{tabular}{lllllll}
\toprule
Proto-algonquien & fox & ojibwe & cree des plaines & menomini \\
\midrule
\ipa{*p} & 	\ipa{p} & 	\ipa{p/b} & 	\ipa{p} & 	\ipa{p} & 	\\
\ipa{*t} & 	\ipa{t} & 	\ipa{t/d} & 	\ipa{t} & 	\ipa{t} & 	\\
\ipa{*č} & 	\ipa{č} & 	\ipa{č/dž} & 	\ipa{ts} & 	\ipa{ts} & 	\\
\ipa{*k} & 	\ipa{k} & 	\ipa{k/g} & 	\ipa{k} & 	\ipa{k} & 	\\
\ipa{*m} & 	\ipa{m} & 	\ipa{m} & 	\ipa{m} & 	\ipa{m} & 	\\
\ipa{*n} & 	\ipa{n} & 	\ipa{n} & 	\ipa{n} & 	\ipa{n} & 	\\
\ipa{*s} & 	\ipa{s} & 	\ipa{s/z} & 	\ipa{s} & 	\ipa{s} & 	\\
\ipa{*š} & 	\ipa{š} & 	\ipa{š/ž} & 	\ipa{s} & 	\ipa{s} & 	\\
\ipa{*θ} & 	\ipa{n} & 	\ipa{n} & 	\ipa{t} & 	\ipa{n} & 	\\
\ipa{*l} & 	\ipa{n} & 	\ipa{n} & 	\ipa{y} & 	\ipa{n} & 	\\
\ipa{*h} & 	\ipa{h} & 	\ipa{h} & 	\ipa{h} & 	\ipa{h} & 	\\
\bottomrule
\end{tabular}
\end{table}

Si les correspondances des consonnes simples sont relativement triviales en arapaho, il n'en va pas de même des groupes de consonnes, dont la reconstruction est moins certaine. Le tableau \ref{tab:clusters.k} représente les groupes ayant *k pour deuxième élément reconstruits par Bloomfield en 1925 (la transcription a été légèrement modifiée). On observe en particulier que le contraste entre les  groupes  \ipa{*çk}, \ipa{*xk} et \ipa{*šk} n'existe tel quel dans aucune des langues utilisées dans cet article; certaines confondent \ipa{*çk}  et \ipa{*šk} (ojibwe, fox) tandis que d'autres confondent \ipa{*çk}  et \ipa{*hk} (cree, menominee).
 

\begin{table}[h]
\caption{Correspondances des groupes en --k-- dans les langues algonquiennes centrales.} \centering  \label{tab:clusters.k}
\begin{tabular}{lllll}
\toprule
Proto-algonquien & fox & ojibwe & cree des plaines & menomini \\
\midrule
\ipa{*čk} & \ipa{hk} & \ipa{šk} & \ipa{sk} & \ipa{tsk} \\
\ipa{*šk} & \ipa{šk} & \ipa{šk} & \ipa{sk} & \ipa{sk} \\
\ipa{*xk} & \ipa{hk} & \ipa{hk} & \ipa{sk} & \ipa{hk} \\
\ipa{*hk} & \ipa{hk} & \ipa{kk} & \ipa{hk} & \ipa{hk} \\
\ipa{*çk} & \ipa{šk} & \ipa{šk} (en fait \ipa{sk}) & \ipa{hk} & \ipa{hk} \\
\ipa{*nk} & \ipa{g} & \ipa{ng} & \ipa{hk} & \ipa{hk} \\
\bottomrule
\end{tabular}
\end{table}

Toutefois, Bloomfield découvre quelques années plus tard un dialecte du cree (le swampy cree) dans lequel le groupe \ipa{*--çk--} apparaît comme \ipa{--htk--} et est donc distinct de \ipa{*çk}  et \ipa{*hk} (\citealt{bloomfield28thk}). Cette découverte montre que même dans les langues à tradition orales, le postulat de la régularité des changements phonétique est valide, et qu'il est justifié de reconstruire des unités phonologiques distinctes (phonème unique ou groupes de phonèmes) dans la proto-langue pour rendre compte de correspondances entre langues attestées, même si ces unités phonologiques ne sont distinctes dans aucune des langue modernes. 

\citet{bloomfield46proto} découvre ensuite que l'ojibwe faisait en fait la distinction entre  \ipa{*--çk--} et \ipa{*--šk--}, qui apparaissent comme \ipa{--sk--} et \ipa{--šk--} respectivement, mais William Jones, sur les données duquel il s'était basé, locuteur natif du fox, avait omis cette distinction qui n'existait pas dans sa langue maternelle en transcrivant l'ojibwe, et avait ainsi induit Bloomfield en erreur.

Aux groupes reconstruits dans le tableau \ref{tab:clusters.k}, il faut ajouter la distinction entre \ipa{*--xk--} et \ipa{*--θk--} découverte par \citet{siebert41clusters} sur la base des données de langues algonquiennes de l'est. 

Les seules autres modifications apportés au système de Bloomfield sont la découverte du groupe  \ipa{*--hm--} par \citet{goddard79comparative} et la distinction entre *\ipa{kw} et \ipa{kʷ} proposée par \citet{pentland79phd}. Mis à part ces quelques groupes mineurs, le système   de Bloomfield permet de rendre compte non seulement des langues algonquiennes centrales, mais aussi de l'ensemble de la famille à l'exception du blackfoot. La notation du proto-algonquien a été légèrement modifiée (voir \citealt{goddard98arapaho}). On note à présent \ipa{*r} ce que Bloomfield transcrivait \ipa{*l}, et \ipa{*--rk--},  \ipa{*--sk--} ce qu'il écrivait \ipa{*--çk--},  \ipa{*--xk--}.


Le succès de la reconstruction de Bloomfield  est dû à une combinaison de facteurs: l'existence du fox, une langue extrêmement conservatrice, le degré modéré d'emprunts entre les langues étudiées, la proximité entre ces langues,  et le fait que l'auteur avait une connaissance de première main (sauf sur l'ojibwe) et qu'il n'avait pas eu recours aux données anciennes difficile à interpréter.

\section{Application à l'arapaho}	
L'arapaho et les langues prochement apparentées que sont le gros-ventre et le nawathinehena présentent des changements phonétiques inhabituels qui rendent méconnaissables les cognats avec les langues algonquiennes plus conservatrices. Si \citet{kroeber16arapaho} et \citet{michelson35shifts} ont mis en évidence  les premières correspondances phonétiques enter arapaho et le reste de l'algonquien, c'est seulement  \citet{goddard74arapaho} qui est parvenu à rendre compte de la phonologie historique de l'arapaho, sur la base des données de \citet{salzmann56phono}, \citet{salzmann63arapaho} et \citet{taylor67atsina}, plus fiables que celles de Kroeber.
 
 Les correspondances de base des voyelles sont relativement simples: \ipa{*a}, \ipa{*e} et \ipa{*o/we/i} correspondent respectivement à \ipa{o}, \ipa{e} et \ipa{i}; les voyelles finales, ainsi que d'autres voyelles dans des conditions plus spécifiques  chutent sans laisser de traces. Par ailleurs, l'harmonie vocalique change  \ipa{e} en \ipa{o} , \ipa{i} en \ipa{u} [ʉ~ɯ] (une voyelle encore quasiment en distribution complémentaire avec \ipa{i}), et \ipa{e} en \ipa{o} dans des conditions complexes et encore imparfaitement comprises (voir \citealt[15-18, 20-22]{cowell06arapaho}).
 
Les correspondances générales des consonnes simples du proto-algonquien en arapaho sont présentées en \ref{tab:c.simple.arapaho} sous forme simplifiée, en début de mot et entre deux voyelles.\footnote{Les réflexes des groupes de consonnes du proto-algonquien en arapaho ne sont pas traités ici.} Certaines consonnes finales chutent avec la voyelle finale (en particulier les nasales).
 \begin{table}[h]
\caption{Correspondances du proto-algonquien à l'arapaho pour les consonnes simples.} \centering  \label{tab:c.simple.arapaho}
\begin{tabular}{lllllll}
\toprule
Proto-algonquien & début de mot  & autre \\
\midrule
\ipa{*p} & 	\ipa{k/c} & 	&   	\\
\ipa{*t} & 	\ipa{t} & 	 & 	  	\\
\ipa{*č} & 	\ipa{θ} & 	   	\\
\ipa{*k} & 	$\emptyset$ & 	   	\\
\ipa{*m} & 	\ipa{w/b} & 	 & 	  	\\
\ipa{*n} & 	\ipa{n} & 	 & 	  	\\
\ipa{*s} & 	\ipa{n} & 	\ipa{h} & 	  	\\
\ipa{*š} & 	\ipa{x/s} & 	 & 	  & 	\\
\ipa{*θ} & 	\ipa{θ} &  & 	  	\\
\ipa{*l} & 	\ipa{n} &  & 	  	\\
\ipa{*h} & 	\ipa{h} & 	\ipa{'} & 	  	\\
\bottomrule
\end{tabular}
\end{table}

Les consonnes \ipa{*p}, \ipa{*m} et \ipa{*š} du proto-algonquien ont deux formes en arapaho selon que les voyelles qui suivent ou qui précèdent sont antérieures ou postérieures: les formes \ipa{k}, \ipa{w} et \ipa{x} apparaissent devant \ipa{o} et \ipa{u}, et \ipa{c}, \ipa{b} et \ipa{s} devant \ipa{e} et \ipa{i}. La distribution complémentaire n'est toutefois pas parfaite, car les groupes \ipa{*py/w} et \ipa{*my/w} du proto-algonquien deviennent \ipa{c} et \ipa{b} même devant voyelle postérieure. Ainsi \ipa{cóóθo'} `ennemis' provient de \ipa{*pwaaθaki} (Ojibwe \ipa{bwaanag} 'sioux') et \ipa{bóoó} `chemin' de \ipa{*myeehkani} (Ojibwe \ipa{miikan}).

Dans la suite de cette contribution, on s'intéresse à deux des changements qui ont eu lieu entre le proto-algonquien et l'arapaho: \ipa{*p} $\rightarrow$ \ipa{k/c} d'une part et \ipa{*s} $\rightarrow$ \ipa{n} d'autre part.

\subsection{\ipa{*p} $\rightarrow$ \ipa{k/c} }
Le passage de \ipa{*p} à \ipa{k}, déjà connu de \citet{kroeber16arapaho} et \citet{michelson35shifts}, participe à un changement en chaine:

\begin{itemize}
\item \ipa{*k} $\rightarrow \emptyset$
\item \ipa{*p} $\rightarrow$ \ipa{*k} $\rightarrow$  \ipa{k/c}
\end{itemize}

Le  \ipa{*k} du proto-algonquien disparaît dans tous les contextes, y compris dans les groupes, où il laisse parfois une trace indirecte. Ainsi le proto-algonquien \ipa{*--θk--} devient \ipa{x} ou \ipa{s} et non \ipa{θ} en arapaho, car le changement \ipa{*--θk--} $\rightarrow $  \ipa{*--šk--} a lieu avant la chute de \ipa{*k}, et *\ipa{kw--} donne \ipa{y} au lieu de \ipa{n} comme le *\ipa{w} simple.

Si des cas de changement de labiale à vélaire sont attestés dans des contextes très spécifiques en indo-européen (par exemple dans le cas de premier élément de groupes de consonnes comme *\ipa{-pt-} $\rightarrow$ *\ipa{-xt-} en celtique), l'arapaho est la seule langue au monde à présenter le changement \textit{hors contexte} de  \ipa{*p} à \ipa{k}. Le tableau \ref{tab:p.k} illustre plusieurs exemples de ce changement (les quatre premiers sont tirés de  \citealt{goddard74arapaho}), ainsi que de la disparition du *k (il convient de noter qu'en arapaho, le segment \ipa{h--} est ajouté automatiquement aux mots commençant en voyelle). L'arapaho est comparé à l'ojibwe, langue présentant moins d'innovations phonologique et relativement plus proche du proto-algonquien.

 \begin{table}[h]
\caption{Exemples du changement \ipa{*p} $\rightarrow$ \ipa{*k} $\rightarrow$  \ipa{k/c}} \centering  \label{tab:p.k}
\begin{tabular}{lllllll}
\toprule
Proto-algonquien & Arapaho & Autre langue (Ojibwe)\\
\midrule
  \ipa{*nepyi} ``eau" & \textit{néc}--    &  \textit{nibi}-- \\
 \ipa{*kespak-yaa--} `` être épais" & \textit{hooko-yóó}--    &  \textit{gipag-aa}-- \\
 \ipa{*waaposwa} ``lapin" & \textit{nóóku}--    &  \textit{waabooz}-- \\
  \ipa{*čiipay-(a/i)} ``cadavre" & \textit{θiik}--    &  \textit{jiibay}-- \\
  \ipa{*paaʔt-ee--} ``être enfumé" & \textit{koot-éé}--    &  \textit{baat-e}-- ``être sec"\\
    \ipa{*pahkwee-n--} ``enlever (un morceau de)" & \textit{koyei-n}--    &  \textit{bakwe-n}-- \\
        \ipa{*pa[n/r]ankw-} ``lâche, pas assez serré" & \textit{kono'-(óé)}--    &  \textit{banangw-(ad)}-- \\ 
   \ipa{*paʔs-askamikw-i} ``ravin" & \textit{koho'ówu'}--    &  \textit{basakamig(aa)}-- ``être un ravin"\\ 
\bottomrule
\end{tabular}
\end{table}

L'arapaho offre un parallèle intéressant au changement en chaîne bien connu \ipa{*k} $\rightarrow$ \ipa{ʔ},  \ipa{*t} $\rightarrow$ \ipa{k}  attesté dans de nombreuses langues océaniques (voir \citealt{blust04tk}). Ces deux types de changements ont en commun la débuccalisation des occlusives vélaires, suivi du remplissage de la ``case vide" causé par ce changement par une occlusive d'un autre lieu d'articulatoire (dental ou labial). 


Il est significatif que ce type de changement n'a lieu que dans des langues n'ayant qu'une seule série d'occlusive et avec un inventaire consonantique réduit (moins de 15 phonèmes consonantiques pour la proto-langue). Dans un système avec opposition de voisement ou d'aspiration, la débuccalisation n'est jamais attesté pour toutes les occlusives d'un même lieu d'articulation. Par exemple, en Kiranti (\citealt{michailovsky94stops}), le yamphu a $\emptyset$ pour le proto-kiranti *p, *t , et le belhare pour  *p, mais leurs équivalents voisés *b et *d et les groupes *Xp et *Xt correspondent respectivement à \ipa{p}, \ipa{t} et \ipa{pʰ}, \ipa{tʰ} dans les deux langues, et ces changements résultent en des systèmes équilibrés.

Dans les systèmes consonantiques comprenant plusieurs séries d'occlusives, les changements de lieu d'articulation attestés sont le résultat d'assimilations, et sont d'une nature complètement différents des changements en chaîne observés en arapaho et en océanique. Par exemple, le passage de \ipa{tʰ} à \ipa{kʰ} en Dene Suline (\citealt{haas68chipewyan}) et en Kansa (\citealt{csd2006}), entraînant la perte de l'opposition entre dentale et vélaire aspirées, est dû à l'uvularisation de l'aspiration /\ipa{tʰ}/ [\ipa{tχ}] $\rightarrow$ /\ipa{kʰ}/ [\ipa{kχ}]. Dans ces deux langues, les consonnes non aspirées et éjectives n'ont pas été affectées par un tel changement, et le trou dans le système phonologique reste marginal.
 
On remarque aussi qu'aussi en arapaho qu'en océanique, c'est l'occlusive vélaire qui est affectée par la débuccalisation. Il est toutefois trop tôt pour affirmer que les changements combinant débuccalisation et changement de lieu d'articulation sont unidirectionels et ne peuvent commencer que par l'occlusive vélaire.


\subsection{\ipa{*s--} $\rightarrow$ \ipa{n--} }
Contrairement aux changements   précédents, qui sont des atomiques et ne comprennent pas d'étapes intermédiaires possibles, la correspondance en \ipa{*s--} à l'initiale de mot en proto-algonquien et \ipa{n--} en arapaho ne peur s'analyser que comme résultant de l'accumulation de plusieurs changements phonétiques, et n'est pas elle-même un changement phonétique à proprement parler.

Cette correspondance est surprenante surtout si l'on considère que le *s du proto-algonquien devient \ipa{h} en arapaho dans tous les autres contextes. Depuis sa découverte par \citet{goddard74arapaho}, d'autres auteurs (\citealt{pentland98} et \citealt{jacques13arapaho}) ont confirmé sa validité; elle est attestée dans environ 16 exemples  et ne connaît pas d'exception. Comme le montre le tableau \ref{tab:s.n}, cette correspondance ne dépend pas de la voyelle qui suit et ne peut pas s'interpréter comme un nasalisation, puisqu'elle est présente dans des mots ne contenant aucun segment nasal par ailleurs. 

 \begin{table}[H]
\caption{Exemples du changement \ipa{*s} $\rightarrow$ \ipa{*n} en début de mots} \centering  \label{tab:s.n}
\begin{tabular}{lllllll}
\toprule
Proto-algonquien & Arapaho & Autre langue (Ojibwe)\\
\midrule
 \ipa{*siipiiwi} &  \ipa{níícíí}    `` fleuve"    &  \ipa{ziibi} \\
\ipa{*sakimeewa}  & \ipa{nóúbee} ``moustique" &  \ipa{zagime} \\
\ipa{*sarak-yaa--}  & \textit{nónoyoo}-- ``être faux" & \textit{zanagizi} ``avoir des difficultés" (*sanak-esi--) \\
\ipa{*sii-θakw-en-}  & \textit{niiθóun}--  ``traire"& \textit{ziinin} ``traire" (*siin-en--) \\
\bottomrule
\end{tabular}
\end{table}

Dans son article de 1974, Goddard ne discute pas en détail des étapes intermédiaires.  \citet{picard94sn} propose l'évolution suivante:

 \begin{exe}
\ex
\glt *s-- $\rightarrow$ *h-- $\rightarrow$ *ç-- $\rightarrow$ *y-- $\rightarrow$ *l-- $\rightarrow$ n-
\end{exe}

Le problème fondamental de cette hypothèse est la supposition d'une rebuccalisation du *h en *ç opérant hors contexte, quelle que soit la voyelle qui suit. Si un changement du type /h/  $\rightarrow$ /ç/ est parfaitement attesté au contact de voyelles antérieures (voir par exemple en naxi \citealt[27-33]{michaud06neutralisation}), il est impossible de supposer une telle évolution en arapaho, où \ipa{*s} passe à \ipa{n} même lorsqu'il est suivi de \ipa{*a} (qui devient ensuite \ipa{o} mais ne devient jamais une voyelle antérieure).

 \citet[76]{goddard01plains} propose une hypothèse différente:

\begin{quote}
``One possibility [to explain the correspondences PA *s-- to Arapaho/Atsina n-- and Nawathinehena *t--] is that *s shifted to *z in Nawathinehena, and word-initially in Arapaho-Gros Ventre; then *z shifted to *r; and then *r underwent its regular shifts to Nawathinehena [t] and Arapaho-Gros-Ventre \textit{n}."
\end{quote}  
 Goddard suppose un rhotacisme du *s exclusivement en position initiale, suivie d'une confusion avec le *r puis du *r avec le *n. Le tableau suivant présente cette hypothèse en explicitant chaque étape intermédiaire.
\begin{table}[H]
\caption{Développement de plusieurs consonnes en arapaho à l'initiale de mot, hypothèse n°1}  \centering
\begin{tabular}{lllllllllllll}
&PA & 1 & 2 & 3 & 4 & 5 & 6 & \\
&*n   &&&&& > n\\
&*w   & >*y  \grise{} & \grise{} & \grise{}  &>*r	  \grise{} &> n \grise{}\\
&*s  &  &>*z &>*r	 \grise{}& 	 \grise{} &> n \grise{}\\
&*r	   & 	&  & 	& 	  &> n \grise{}\\
&*θ >  & &&&&&θ \\
&*t > & &&&&&t \\
\end{tabular}
\end{table}
Les cas de rhotacisme à l'initiale de mot sont inconnus en indo-européen (\citealt[80-81]{kuemmel07wandel}); Néanmoins, ce type de changement est bien attesté en vietnamien (\citet{ferlus82spirantisation}), langue où le *-s final devient d'ailleurs un *\ipa{h}, puis un ton, sans que le *s initial ne soit affecté par ce changement. L'hypothèse de Goddard est donc tout à fait vraisemblable.


Toutefois, une autre hypothèse peu être envisagée (\citealt{jacques13arapaho}): le changement de *r à *l, suivi de la latéralisation du *s à l'initiale en *ɬ (changement purement phonétique), puis confusion de *ɬ et *l en *l, selon le modèle présenté dans le tableau suivant:
\begin{table}[H]
\caption{Développement de plusieurs consonnes en arapaho à l'initiale de mot, hypothèse n°2}   \centering
\begin{tabular}{lllllllllllll}
&PA & 1 & 2 & 3 & 4 & 5 & 6   \\
&*n  &&&&&& n\\
&*w   && >*y  \grise{}   & \grise{}& \grise{}   &>*l \grise{} &> n \grise{}\\
&*s  && &  >*ɬ  &>*l \grise{}&  \grise{} & > n \grise{}\\
&*r > *l	 &  &  &    &  &    &> n \grise{}\\
&*θ  &  \\
\end{tabular}
\end{table}

Ces deux hypothèses sont équivalentes en complexité, et impliquent toutes les deux deux changements phonétiques et deux confusions. C'est l'accumulation de changements phonétiques multiples qui a permis une évolution phonétique aussi radicale entre le proto-algonquien et l'arapaho. Si un jour on découvre un autre cas de correspondance similaire entre *s et \ipa{n} dans une autre famille de langue, le modèle proposé ici suggère que l'on doivent également observer soit un passage de *r et *n ou de *l à *n dans les mêmes contextes où *s devient \ipa{n}.

\section{Conclusion}	
Dans cet article, nous avons présenté une brève introduction à la phonologie comparée des langues algonquiennes, et avons discuté plus en détail de deux changements phonétiques entre le proto-algonquien et l'arapaho qui n'ont pas de parallèles dans d'autres familles de langue: le passage de *p à \ipa{k/c} et celui de *s à \ipa{n}. 

La découverte de ces changements rares n'a pu être possible que grâce à l'accumulation de travaux descriptifs et diachroniques de la plus haute qualité en algonquien, et l'étude typologique des changements phonétiques dans une perspective panchronique et dépendante de telles études. L'immense majorité des groupes de langues des Amériques n'ont pas fait l'objet d'études diachroniques rigoureuses, et il est probable que l'existence de changements tout aussi inattendus puisse être démontrée lorsque notre connaissance de la phonologie historique d'autres familles aura progressé.

 
 \bibliographystyle{linquiry2}
 \bibliography{bibliogj}

\end{document}