\documentclass[oneside,a4paper,11pt]{article} 
\usepackage{fontspec}
\usepackage{natbib}
\usepackage{booktabs}
\usepackage{xltxtra} 
\usepackage{polyglossia} 
\usepackage[table]{xcolor}
\usepackage{gb4e} 
\usepackage{multicol}
\usepackage{graphicx}
\usepackage{float}
\usepackage{lineno}
\usepackage{textcomp}
\usepackage{hyperref} 
\hypersetup{bookmarks=false,bookmarksnumbered,bookmarksopenlevel=5,bookmarksdepth=5,xetex,colorlinks=true,linkcolor=blue,citecolor=blue}
\usepackage[all]{hypcap}
\usepackage{memhfixc}
\usepackage{lscape}
 \usepackage{setspace}

%\setmainfont[Mapping=tex-text,Numbers=OldStyle,Ligatures=Common]{Charis SIL} 
\newfontfamily\phon[Mapping=tex-text,Ligatures=Common,Scale=MatchLowercase,FakeSlant=0.3]{Charis SIL} 
\newcommand{\ipa}[1]{{\phon#1}} %API tjs en italique
 
\newcommand{\grise}[1]{\cellcolor{lightgray}\textbf{#1}} 
\linenumbers

\begin{document} 

\title{The genesis on infixation in Siouan languages\footnote{Glosses follow the Leipzig Glossing rules. The underscore \_ in a verb form indicates where the personal prefixes are inserted, in the case of verbs with discontinuous stems. In Lakota, capital \ipa{A} indicates an ablauting final vowel, which is realized as \ipa{a}, \ipa{e} or \ipa{iŋ} depending on the present of particular enclitics. Acknowledgements will be added after editorial decision.}} 
 \author{Guillaume Jacques}
\maketitle
 
 
\ipa{matĥó},  \ipa{ma<ní>tĥo} 


\ipa{rãx-'ũ}

\ipa{naĥ'úŋ},  \ipa{na<wá>ĥ'uŋ} `hear'

\ipa{nakpá} ear



An interesting example which is reconstructible to proto-Siouan is *\ipa{yó•\_kʔį} "to roast", which become Lakota \ipa{čho\_k'íŋ} "to roast" (1sg \ipa{čho<wá>k'iŋ}). This verb has cognates in Hidatsa, Chiwere-Winnebago and Biloxi, and its root is discontinous also in Winnebago \ipa{roo\_kʔį́} ‘roast, bake, cook in oven’ KM-2639. Although the first element *\ipa{yó•-} can be etymologized as the root for "meat, flesh" (Lakota čho-), the second element *\ipa{-kʔį} is totally opaque. Although this must have been a case of noun incorporation historically, this is not the case anymore in any Siouan language.

\ipa{núŋǧe} ear

*\ipa{rãxu}

\citet{jacques12bear}

\citet{csd2006}

\citet{ullrich08}


 
\bibliographystyle{unified}
\bibliography{bibliogj}

 \tableofcontents
\end{document}