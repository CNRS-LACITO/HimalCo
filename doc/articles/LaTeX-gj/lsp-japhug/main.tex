%%%%%%%%%%%%%%%%%%%%%%%%%%%%%%%%%%%%%%%%%%%%%%%%%%%%
%%%                                              %%%
%%%     Language Science Press Master File       %%%
%%%         follow the instructions below        %%%
%%%                                              %%%
%%%%%%%%%%%%%%%%%%%%%%%%%%%%%%%%%%%%%%%%%%%%%%%%%%%%
 
% Everything following a % is ignored
% Some lines start w--ith %. Remove the % to include them

\documentclass[output=book,
  nonflat,
  modfonts,
  colorlinks,
showindex
		  ]{langsci/langscibook}    
  
%%%%%%%%%%%%%%%%%%%%%%%%%%%%%%%%%%%%%%%%%%%%%%%%%%%%
%%%                                              %%%
%%%          additional packages                 %%%
%%%                                              %%%
%%%%%%%%%%%%%%%%%%%%%%%%%%%%%%%%%%%%%%%%%%%%%%%%%%%%

% put all additional commands you need in the 
% following files. {I}f you do not know what this might 
% mean, you can safely ignore this section

\title{A grammar of Japhug}  %look no further, you can change those things right here.
% \subtitle{Change subtitle}
\BackTitle{Change your backtitle in localmetadata.tex} % Change if BackTitle != Title
\BackBody{Change your blurb in localmetadata.tex}
%\dedication{Change dedication in localmetadata.tex}
%\typesetter{Change typesetter in localmetadata.tex}
%\proofreader{Change proofreaders in localmetadata.tex}
\author{Guillaume Jacques}
% \BookDOI{}%ask coordinator for DOI
\renewcommand{\lsISBNdigital}{000-0-000000-00-0}
\renewcommand{\lsISBNhardcover}{000-0-000000-00-0}
\renewcommand{\lsISBNsoftcover}{000-0-000000-00-0}
\renewcommand{\lsISBNsoftcoverus}{000-0-000000-00-0}
\renewcommand{\lsSeries}{sidl} % use lowercase acronym, e.g. sidl, eotms, tgdi
\renewcommand{\lsSeriesNumber}{99} %will be assigned when the book enters the proofreading stage
% \renewcommand{\lsURL}{http://langsci-press.org/catalog/book/000} % contact the coordinator for the right number

 
 
 
 
  

% add all extra packages you need to load to this file  
\usepackage{tabularx} 

%%%%%%%%%%%%%%%%%%%%%%%%%%%%%%%%%%%%%%%%%%%%%%%%%%%%
%%%                                              %%%
%%%           Examples                           %%%
%%%                                              %%%
%%%%%%%%%%%%%%%%%%%%%%%%%%%%%%%%%%%%%%%%%%%%%%%%%%%% 
%% to add additional information to the right of examples, uncomment the following line
% \usepackage{jambox}
%% if you want the source line of examples to be in italics, uncomment the following line
% \renewcommand{\exfont}{\itshape}
\usepackage{./langsci/styles/langsci-optional}
\usepackage{./langsci/styles/langsci-gb4e}
\usepackage{./langsci/styles/langsci-lgr}
\usepackage{./langsci/styles/langsci-glyphs}

\usepackage[english]{babel}
 
% \usepackage[hang,flushmargin]{footmisc}
% \setlength\footnotemargin{10pt}

\usepackage{multicol}
\usepackage{calculator}
\usepackage{lineno}
\usepackage{amsmath}
%\usepackage{tikz}
%\usepackage{tikz-qtree}
\usepackage{qtree}
\input{localhyphenation.tex}
\bibliography{bibliogj} 

%%%%%%%%%%%%%%%%%%%%%%%%%%%%%%%%%%%%%%%%%%%%%%%%%%%%
%%%                                              %%%
%%%             Frontmatter                      %%%
%%%                                              %%%
%%%%%%%%%%%%%%%%%%%%%%%%%%%%%%%%%%%%%%%%%%%%%%%%%%%% 
%\usepackage{lipsum}
\begin{document}     
%add all your local new commands to this file

\newcommand{\smiley}{:)}

\renewbibmacro*{index:name}[5]{%
  \usebibmacro{index:entry}{#1}
    {\iffieldundef{usera}{}{\thefield{usera}\actualoperator}\mkbibindexname{#2}{#3}{#4}{#5}}}

% \newcommand{\noop}[1]{}

\makeatletter
\def\blx@maxline{77}
\makeatother

\newcommand{\appref}[1]{Appendix \ref{#1}}
\newcommand{\fnref}[1]{Appendix \ref{#1}}
\newcommand{\regel}[1]{#1}
\newcommand{\vernacular}[1]{\emph{#1}}
\newcommand{\gloss}[1]{#1}

\newfontfamily\phon[Mapping=tex-text,Ligatures=Common,Scale=MatchLowercase]{Charis SIL} 
\newcommand{\ipa}[1]{\mbox{\phon/#1/}}  
\newcommand{\phonet}[1]{\mbox{\phon[#1]}}  
\newcommand{\ipab}[1]{{\phon#1}}  
\newcommand{\japhug}[2]{\textit{\phon#1} `#2'}  
\newcommand{\deux}[1]{\ipa{#1}\addtocounter{2clusters}{1}}
\newcommand{\trois}[1]{\ipa{#1}\addtocounter{3clusters}{1}}
 
\newcommand{\grise}[1]{\cellcolor{lightgray}\textbf{#1}}
\newfontfamily\cn[Mapping=tex-text,Ligatures=Common,Scale=MatchUppercase]{SimSun}%pour le chinois
\newcommand{\zh}[1]{{\cn #1}}


\newcommand{\tib}[1]{\cellcolor{lightgray}\textbf{#1}}
\newcommand{\idph}[1]{\cellcolor{gray}\textbf{#1}}
\newcommand{\tld}{\textasciitilde{}}

\XeTeXlinebreaklocale "zh" %使用中文换行
\XeTeXlinebreakskip = 0pt plus 1pt %
 %CIRCG
 \newcommand{\resetcounters}[2]{
\newcounter{#1}
\newcounter{#2}
\setcounter{#1}{\value{2clusters}}
\setcounter{#2}{\value{3clusters}}
 \setcounter{2clusters}{0}
  \setcounter{3clusters}{0}
}
 \newcommand{\addition}[2]{\ADD{\value{#1}}{\value{#2}}{\solution}\solution}
 
 

\maketitle                
\frontmatter
% %% uncomment if you have preface and/or acknowledgements

\currentpdfbookmark{Contents}{name} % adds a PDF bookmark
\tableofcontents
% \include{chapters/preface}
% \include{chapters/acknowledgments}
% \include{chapters/abbreviations} 
\mainmatter         
  

%%%%%%%%%%%%%%%%%%%%%%%%%%%%%%%%%%%%%%%%%%%%%%%%%%%%
%%%                                              %%%
%%%             Chapters                         %%%
%%%                                              %%%
%%%%%%%%%%%%%%%%%%%%%%%%%%%%%%%%%%%%%%%%%%%%%%%%%%%%
%\chapter{The Japhug language}
This paper focuses on the Japhug language (local name \ipa{kɯrɯ skɤt}) of Kamnyu village (\ipa{kɤmɲɯ}, Chinese \textit{Ganmuniao} \zh{干木鸟}) in Gdong-brgyad area (\ipa{ʁdɯrɟɤt}, Chinese  \textit{Longerjia} \zh{龙尔甲}), Mbarkhams county (Chinese \textit{Maerkang} \zh{马尔康}), Rngaba prefecture, Sichuan province, China.
 
 Japhug belongs to the Sino-Tibetan family, and is one of the four Rgyalrong languages, alongside Tshobdun, Zbu and Situ.\footnote{See  \citet{jackson00sidaba} for an overview of the Rgyalrong group, whose closest relatives include Khroskyabs (\citealt{lai15person}) and Horpa (\citealt{jackson07shangzhai}). A text collection of Japhug with sound files is included in the Pangloss archive (\citealt{michailovsky14pangloss}). A short grammar (\citealt{jacques08}), a series of articles on morphosyntax (see for instance  \citealt{jacques13harmonization} and
 \citealt{jacques14antipassive}) and a dictionary (\citealt{jacques15japhug}) are available but little has been published specifically on its phonology. } 
 
The description is based on the author’s fieldwork, and the word list and the short story in the appendix have been provided by Tshendzin (Chenzhen \zh{陈珍}, female, born 1950), a retired schoolteacher (a native speaker of Japhug, bilingual in Sichuan Mandarin since childhood).  %The Japhug language
%\chapter{Corpus/02.tex}
  %The Japhug Corpus
\chapter{Phonology} \label{sec:phonology}

  

 Japhug has a highly  developed system of ideophones (\citealt{japhug14ideophones}), which present unusual phonological features, in particular rare clusters. In the following discussion, phonemes or clusters found exclusively on  ideophones will be treated separately. In addition, about a quarter of the Japhug vocabulary is borrowed from Tibetan, and these loanwords, like the ideophones, fill some gaps in the phonotactic distribution of vowels and consonants (on gap-filling by loanwords see \citealt[63-64]{martinet05economie}).  These cases are carefully distinguished from the native vocabulary in the analyses that follow, in order to bring out the phonotactics of inherited Japhug vocabulary.

%\begin{exe} 
% \ex 
%\gll   nɯ kɯ pʰɤn, ɯ-pʰɤntʰoʁ tu kɤ-ti ɲɯ-ŋu.\\ 
% \textsc{dem} \textsc{erg} be.efficient 3\textsc{sg}.\textsc{poss}-advantage    exist \textsc{inf}-say \textsc{testim}-be\\ 
% \glt  it is more efficient, more advantageous, it is said.
%\end{exe} 

\section{Consonants}

\subsection{Consonant clusters}
%\&[ \t]*\\ipa\{([^}]*)\} *\&[\t ]*([^\&]*)
%\& \\japhug{\1}{\2} 
%regex for converting tables

\newcounter{2clusters}
\newcounter{3clusters}

 \begin{table}
 \caption{List of consonant clusters with \ipa{w} as a first element (15+8)}  \centering \label{prein.w}
\begin{tabular}{l|lll|lll|lll|lllllll}
  \lsptoprule
%\ipa{p}  & 	  & 	  & 	  & 	 \\
%\ipa{pʰ}  & 	  & 	  & 	  & 	 \\
%\ipa{b}  & 	  & 	  & 	  & 	 \\
%\ipa{mb}  & 	  & 	  & 	  & 	 \\
%\ipa{m}  & 	  & 	  & 	  & 	 \\
\ipa{t}  & 	 \deux{wt}  &  	 \japhug{ɯ-ftaʁ}{sign} \\
%\ipa{tʰ}  & 	  & 	  & 	  & 	 \\
\ipa{d}  & 	 \deux{wd}  & 	\japhug{βdɯt}{demon} \\
%%\ipa{nd}  & 	  & 	  & 	  & 	 \\
%%\ipa{n}  & 	  & 	  & 	  & 	 \\
\ipa{ts}  & 	 \deux{wts}  & 	\japhug{ftsoʁ}{female hybrid yak} \\
\ipa{tsʰ}  & 	 \deux{wtsʰ}  & 	\japhug{ftsʰi}{it is not serious (disease)} \\
%%\ipa{dz}  & 	  & 	  & 	  & 	 \\
%%\ipa{ndz}  & 	  & 	  & 	  & 	 \\
\ipa{s}  & 	 \deux{ws}  & 	\japhug{fsaŋ}{fumigation} \\
\ipa{z}  & 	 \deux{wz} \tib{}  & 	\japhug{βzaŋsa}{friend} \\
%\ipa{ɬ}  & 	  & 	  & 	  & 	 \\
\ipa{tɕ}  & 	 \deux{wtɕ}  & 	\japhug{ftɕar}{summer} \\
\ipa{tɕʰ}  & 	 \deux{wtɕʰ}  & 	\japhug{ftɕʰur}{he pours it down} \\
%\ipa{dʑ}  & 	  & 	  & 	  & 	 \\
%\ipa{ndʑ}  & 	  & 	  & 	  & 	 \\
\ipa{ɕ}  & 	 \deux{wɕ}  & 	\japhug{fɕaʁ}{he repents for it} \\
\ipa{ʑ}  & 	 \deux{wʑ}  & 	\japhug{βʑar}{buzzard} \\
\ipa{tʂ}  & 	 \deux{wtʂ}  & 	\japhug{ftʂi}{he melts it} \\
%\ipa{tʂʰ}  & 	  & 	  & 	  & 	 \\
%\ipa{dʐ}  & 	  & 	  & 	  & 	 \\
%\ipa{ndʐ}  & 	  & 	  & 	  & 	 \\
%\ipa{ʂ}  & 	  & 	  & 	  & 	 \\
\ipa{c}  & 	 \deux{wc}  & 	\japhug{tɯ-fcaʁ}{dorsal mat} \\
%\ipa{cʰ}  & 	  & 	  & 	  & 	 \\
\ipa{ɟ}  & 	 \deux{wɟ}  & 	\japhug{βɟi}{he runs after it} \\
%\ipa{ɲɟ}  & 	  & 	  & 	  & 	 \\
%\ipa{ɲ}  & 	  & 	  & 	  & 	 \\
\ipa{k}  & 	 \deux{wk}  & 	\japhug{fka}{order} \\
%\ipa{kʰ}  & 	  & 	  & 	  & 	 \\
\ipa{g}  & 	 \deux{wg} \tib{} & 	\japhug{βgoz}{he prepares it} \\
\midrule
&	\trois{wxt}  &	\japhug{wxti}{it is big} \\
&	\trois{wst} \tib{} &	\japhug{fstɯn}{he serves him} \\
&	\trois{wrt}  \tib{} &	\japhug{frtɤn}{he is trustworthy} \\
&	\trois{wsk}  \tib{} &	\japhug{fskɤr}{he goes around it} \\
&	\trois{wzg}  \tib{} &	\japhug{βzgɤr}{he delays it} \\
&	\trois{wzd}  \tib{} &	\japhug{βzdɯ-nɯ}{they collect it} \\
&	\trois{wzɟ}  \tib{} &	\japhug{βzɟɯr}{he transforms it} \\
&	\trois{wrɟ}  \tib{} &	\japhug{βrɟaŋ}{he stretches it (skin)} \\					
\lspbottomrule
\end{tabular} 
\end{table}
\resetcounters{2wC}{3wC}

 \begin{table}
 \caption{List of consonant clusters with \ipa{s} or \ipa{z} as a first element (23+0)}  \centering \label{prein.sz}
\begin{tabular}{l|ll}
  \lsptoprule
\ipa{p}  & 	 \deux{sp}  & \japhug{spoz}{incense} 	  \\
%\ipa{pʰ}  & 	  & 	  & 	  & 	  & 	  & 	  \\
\ipa{b}  & 	 	 \deux{zb}  & \japhug{zbaʁ}{dry} \\
\ipa{mb}  & 	 \deux{zmb}  & \japhug{tɤzmbɯr}{silt}  \\
\ipa{m}  & 	 \deux{sm}  & \japhug{smar}{river} \\
& 	 \deux{zm}  & \japhug{zmɤrɤβ}{he eat it with} \\
\ipa{t}  & 	 \deux{st}  & \japhug{staχpɯ}{pea} 	  \\
\ipa{tʰ}  & 	 \deux{stʰ}  & \japhug{stʰaβ}{he touches it} 	  \\
\ipa{d}  & 	   \deux{zd}  & \japhug{zdɯm}{cloud} \\
\ipa{nd}  & 	 	 \deux{znd}  & \japhug{znde}{wall} \\
\ipa{n}  & 	 \deux{sn}  & \japhug{sna}{it is worth} \\
&	 \deux{zn}  & \japhug{znɤje}{he feels sorry} \\
%\ipa{ts}  & 	  & 	  & 	  & 	  & 	  & 	  \\
%\ipa{tsʰ}  & 	  & 	  & 	  & 	  & 	  & 	  \\
%\ipa{dz}  & 	  & 	  & 	  & 	  & 	  & 	  \\
%\ipa{ndz}  & 	  & 	  & 	  & 	  & 	  & 	  \\
%\ipa{s}  & 	  & 	  & 	  & 	  & 	  & 	  \\
%\ipa{z}  & 	  & 	  & 	  & 	  & 	  & 	  \\
%\ipa{ɬ}  & 	  & 	  & 	  & 	  & 	  & 	  \\
%\ipa{tɕ}  & 	  & 	  & 	  & 	  & 	  & 	  \\
%\ipa{tɕʰ}  & 	  & 	  & 	  & 	  & 	  & 	  \\
%\ipa{dʑ}  & 	  & 	  & 	  & 	  & 	  & 	  \\
%\ipa{ndʑ}  & 	  & 	  & 	  & 	  & 	  & 	  \\
%\ipa{ɕ}  & 	  & 	  & 	  & 	  & 	  & 	  \\
%\ipa{ʑ}  & 	  & 	  & 	  & 	  & 	  & 	  \\
%\ipa{tʂ}  & 	  & 	  & 	  & 	  & 	  & 	  \\
%\ipa{tʂʰ}  & 	  & 	  & 	  & 	  & 	  & 	  \\
%\ipa{dʐ}  & 	  & 	  & 	  & 	  & 	  & 	  \\
%\ipa{ndʐ}  & 	  & 	  & 	  & 	  & 	  & 	  \\
%\ipa{ʂ}  & 	  & 	  & 	  & 	  & 	  & 	  \\
\ipa{c}  & 	 \deux{sc}  & \japhug{scoʁ}{scoop} 	  \\
\ipa{cʰ}  & 	 \deux{scʰ}  & \japhug{scʰɤt}{it recedes (water)} 	  \\
\ipa{ɟ}  & 	   	 \deux{zɟ}  & \japhug{kɯ-nɯzɟɯ}{suffering losses} \\
\ipa{ɲɟ}  & 		 \deux{zɲɟ}  & \japhug{zɲɟa}{plant sp.} \\
\ipa{ɲ}  & 	 \deux{sɲ}  & \japhug{sɲaŋne}{fasting} 	  \\
%\midrule  \\
\ipa{k}  & 	 \deux{sk}  & \japhug{skɤm}{ox}  	  \\
\ipa{kʰ}  & 	 \deux{skʰ}  & \japhug{rɟɤskʰi}{pan}   \\
\ipa{g}  & 		 \deux{zg}  & \japhug{zga}{sauce} \\
\ipa{ŋg}  & 		  	 \deux{zŋg}  & \japhug{kɤ-ɤkʰɤzŋga}{to call} \\
\ipa{ŋ}  & 	 \deux{sŋ}  & \japhug{sŋaʁ}{he curses him} 	  \\
%\ipa{x}  & 	  & 	  & 	  & 	  & 	  & 	  \\
\ipa{q}  & 	 \deux{sq}  & \japhug{sqamnɯz}{twelve} 	  \\
\ipa{qʰ}  & 	 \deux{sqʰ}  & \japhug{sqʰi}{tripod} 	  \\
%\ipa{ɴɢ}  & 	  & 	  & 	  & 	  & 	  & 	  \\
%\ipa{χ}  & 	  & 	  & 	  & 	  & 	  & 	  \\
\lspbottomrule
\end{tabular} 
\end{table}
\resetcounters{2szC}{3szC}


\begin{table}
 \caption{List of consonant clusters with \ipa{l}  as a first element (17+1)} \label{prein.l}  \centering
\begin{tabular}{l|lll}
\lsptoprule
\ipa{p}   & 	 	 \deux{lp}   & \japhug{tɯ-lpɤɣ}{one piece}  \\ 
%\ipa{pʰ}   & 	 	   & 	 	   & 	 	   \\ 
%\ipa{b}   & 	 	   & 	 	   & 	 	   \\ 
%\ipa{mb}   & 	 	   & 	 	   & 	 	   \\ 
\ipa{m}   & 	 	 \deux{lm}   & \japhug{tɤlmɯz}{straw covering the balcony}  \\ 
\ipa{t}   & 	 	 \deux{lt}   & \japhug{ltɤβ}{he  folds it}  \\ 
\ipa{tʰ}   & 	 	 \deux{ltʰ} \idph{}   & \japhug{ltʰɯmɯmi}{coming slowly (sleep)}  \\ 
\ipa{d}   & 	 	 \deux{ld}   & \japhug{ldɯɣi}{bharal  }  \\ 
%\ipa{nd}   & 	 	   & 	 	   & 	 	   \\ 
\ipa{n}   & 	 	 \deux{ln}   & \japhug{lni}{it withers   }  \\ 
\ipa{ts}   & 	 	 \deux{lts}   & \japhug{ɕɤltsaʁ}{leather coat}  \\ 
\ipa{tsʰ}   & 	 	 \deux{ltsʰ} \idph{}   & \japhug{ltshɤltshɤt}{small and weak}  \\ 
%\ipa{dz}   & 	 	   & 	 	   & 	 	   \\ 
%\ipa{ndz}   & 	 	   & 	 	   & 	 	   \\ 
%\ipa{s}   & 	 	   & 	 	   & 	 	   \\ 
%\ipa{z}   & 	 	   & 	 	   & 	 	   \\ 
%\ipa{ɬ}   & 	 	   & 	 	   & 	 	   \\ 
\ipa{tɕ}   & 	 	 \deux{ltɕ}   & \japhug{rtɤltɕaʁ}{horse whip   }  \\ 
\ipa{tɕʰ}   & 	 	 \deux{ltɕʰ} \idph{}   & \japhug{ltɕʰɤltɕʰɤt}{hanging (of fluffy objects)   }  \\ 
\ipa{dʑ}   & 	 	 \deux{ldʑ} \tib{}   & \japhug{ldʑaŋkɯ}{green  }  \\ 
%\ipa{ndʑ}   & 	 	   & 	 	   & 	 	   \\ 
%\ipa{ɕ}   & 	 	   & 	 	   & 	 	   \\ 
%\ipa{ʑ}   & 	 	   & 	 	   & 	 	   \\ 
%\ipa{tʂ}   & 	 	   & 	 	   & 	 	   \\ 
%\ipa{tʂʰ}   & 	 	   & 	 	   & 	 	   \\ 
\ipa{dʐ}   & 	 	 \deux{ldʐ} \idph{}   & \japhug{ldʐaŋldʐaŋ}{hanging (big object)}  \\ 
%\ipa{ndʐ}   & 	 	   & 	 	   & 	 	   \\ 
%\ipa{ʂ}   & 	 	   & 	 	   & 	 	   \\ 
\ipa{c}   & 	 	 \deux{lc} \idph{}  & \japhug{lcɯɣlcɯɣ}{drenching}  \\ 
\ipa{cʰ}   & 	 	 \deux{lcʰ}   & \japhug{tɯ-lcʰɯɣ}{section (of a bag)}  \\ 
%\ipa{ɟ}   & 	 	   & 	 	   & 	 	   \\ 
%\ipa{ɲɟ}   & 	 	   & 	 	   & 	 	   \\ 
%\ipa{ɲ}   & 	 	   & 	 	   & 	 	   \\ 
%\ipa{k}   & 	 	   & 	 	   & 	 	   \\ 
%\ipa{kʰ}   & 	 	   & 	 	   & 	 	   \\ 
%\ipa{g}   & 	 	   & 	 	   & 	 	   \\ 
%\ipa{ŋg}   & 	 	   & 	 	   & 	 	   \\ 
\ipa{ŋ}   & 	 	 \deux{lŋ} \idph{}   & \japhug{lŋɤlŋɤt}{hanging (fruit)}  \\ 
\ipa{x}   & 	 	 \deux{lx} \idph{}   & \japhug{lxɤβlxɤβ}{thick (clothes)}  \\ 
\ipa{q}   & 	 	\deux{lq}   & 	 \japhug{lqɤnɤlqɤt}{toddling}	   \\ 
%\ipa{qʰ}   & 	 	   & 	 	   & 	 	   \\ 
%\ipa{ɴɢ}   & 	 	   & 	 	   & 	 	   \\ 
%\ipa{χ}   & 	 	   & 	 	   & 	 	   \\ 
% & 	 & 	 & 	 \\ 
\midrule
&\trois{lpɕ}	&\japhug{qalpɕa}{it opens (fern leaf)} \\
\lspbottomrule
\end{tabular}
\end{table}
\resetcounters{2lC}{3lC} %deux

  \begin{table}
 \caption{List of consonant clusters with \ipa{r} and \ipa{ʂ}  as a first element (35+0)} \label{prein.r}  \centering
\begin{tabular}{l|lll|lll|lll|l}
\lsptoprule
\ipa{p}   & \deux{ʂp}   & \japhug{tɯ-rpa}{axe}  \\ 
\ipa{pʰ}   & \deux{ʂpʰ} \idph{}   & \japhug{rpʰɤβrpʰɤβ}{flapping wings}  \\ 
%\ipa{b}   &   \\ 
\ipa{mb}   & \deux{rmb}   & \japhug{armbat}{near}  \\ 
\ipa{m}   & \deux{rm}   & \japhug{rmɤβja}{peacock}  \\ 
\ipa{t}   & \deux{ʂt}   & \japhug{rtalu}{horse year}  \\ 
\ipa{tʰ}   & \deux{ʂtʰ}   & \japhug{ɯ-pɤrtʰɤβ}{middle}  \\ 
\ipa{d}   & \deux{rd}   & \japhug{rdɤstaʁ}{stone}  \\ 
\ipa{nd}   & \deux{rnd}   & \japhug{rnde}{he finds it}  \\ 
\ipa{n}   & \deux{rn}   & \japhug{rnaʁ}{it is deep}  \\ 
\ipa{ts}   & \deux{ʂts}   & \japhug{rtsot}{vengeance}  \\ 
\ipa{tsʰ}   & \deux{ʂtsʰ}   & \japhug{rtsʰom}{it has a crack (bucket)}  \\ 
\ipa{dz}   & \deux{rdz} \idph{}   & \japhug{rdzardza}{insolent}  \\ 
\ipa{ndz}   & \deux{rndz}   & \japhug{rndzɤkɤŋe}{shade of the mountain}  \\ 
\ipa{s}   & \deux{ʂs} \idph{}   & \japhug{rsɯβrsɯβ}{hairy}  \\ 
\ipa{z}   & \deux{rz}   & \japhug{tɯ-rzɯɣ}{one section}  \\ 
%\ipa{ɬ}   &  &  &  \\ 
\ipa{tɕ}   & \deux{ʂtɕ}   & \japhug{nɯrtɕe}{he teases him}  \\ 
\ipa{tɕʰ}   & \deux{ʂtɕʰ}   & \japhug{rtɕʰɯʁjɯ}{caterpillar}  \\ 
%\ipa{dʑ}   &  &  &  \\ 
\ipa{ndʑ}   & \deux{rndʑ}   & \japhug{cɯrndʑi}{sand}  \\ 
\ipa{ɕ}   & \deux{ʂɕ}   & \japhug{rɕɯβrɕɯβ}{rough}  \\ 
\ipa{ʑ}   & \deux{rʑ}   & \japhug{tɤ-rʑaβ}{wife}  \\ 
%\ipa{tʂ}   &  &  &  \\ 
%\ipa{tʂʰ}   &  &  &  \\ 
%\ipa{dʐ}   &  &  &  \\ 
%\ipa{ndʐ}   &  &  &  \\ 
%\ipa{ʂ}   &  &  &  \\ 
\ipa{c}   & \deux{ʂc}   & \japhug{tɤ-rcoʁ}{mud}  \\ 
\ipa{cʰ}   & \deux{ʂcʰ}   & \japhug{ɯ-rcʰarcʰɤβ}{interstice}  \\ 
\ipa{ɟ}   & \deux{rɟ}   & \japhug{rɟaʁ}{he dances}  \\ 
\ipa{ɲɟ}   & \deux{rɲɟ}   & \japhug{rɲɟaʁlo}{bolt}  \\ 
\ipa{ɲ}   & \deux{ʂɲ} \idph{}   & \japhug{ʂɲoʁʂɲoʁ}{long and thin}  \\ 
    & \deux{rɲ} \tib{}   & \japhug{rɲaŋ}{it is ancient}  \\ 
\ipa{k}   & \deux{ʂk}   & \japhug{rko}{it is hard}  \\ 
\ipa{kʰ}   & \deux{ʂkʰ}   & \japhug{tɤ-rkʰom}{feather rachis}  \\ 
\ipa{g}   & \deux{rg}   & \japhug{rga}{he likes it}  \\ 
\ipa{ŋg}   & \deux{rŋg}   & \japhug{rŋgɤm}{hard piece}  \\ 
\ipa{ŋ}   & \deux{rŋ}   & \japhug{tɯ-rŋa}{face}  \\ 
%\ipa{x}   &  &  &  \\ 
\ipa{q}   & \deux{ʂq}   & \japhug{rqoʁ}{he hugs him}  \\ 
\ipa{qʰ}   & \deux{ʂqʰ}   & \japhug{tɤ-rqʰu}{bark, skin}  \\ 
\ipa{ɴɢ}   & \deux{rɴɢ}   & \japhug{ɕɯrɴɢo}{Anisodus tanguticus}  \\ 
\ipa{χ}   & \deux{ʂχ}   & \japhug{ʂχɯʂχi}{with big nostrils}  \\ 
\lspbottomrule
\end{tabular}
\end{table}
\resetcounters{2rC}{3rC} %deux 

   \begin{table}
 \caption{List of consonant clusters with \ipa{ɕ} and \ipa{ʑ}  as a first element (18+0)} \label{prein.C.Z}  \centering
\begin{tabular}{l|ll}
\lsptoprule
\ipa{p} & \deux{ɕp} & \japhug{ɕpaʁ}{he is thirsty} \\ 
\ipa{pʰ} & \deux{ɕpʰ} & \japhug{ɕpʰɤt}{he patches it} \\ 
%\ipa{b} & & & \\ 
\ipa{mb} & \deux{ʑmb} & \japhug{ʑmbɤr}{ulcer} \\ 
\ipa{m} & \deux{ɕm} & \japhug{ɕmi}{he mixes it} \\ 
\ipa{t} & \deux{ɕt} & \japhug{ɕte}{he contaminates him} \\ 
\ipa{tʰ} & \deux{ɕtʰ} & \japhug{ɕtʰɯz}{he turns towards} \\ 
\ipa{d} & \deux{ʑd} \idph{} & \japhug{ʑdɯɣʑdɯɣ}{strong, tough} \\ 
%\ipa{nd} & & & \\ 
\ipa{n} & \deux{ɕn} & \japhug{ɕnat}{weaving implement} \\ 
  & \deux{ʑn} & \japhug{ʑ-nɯ-ɕar}{go and look for it} \\ 
%\ipa{ts} & & & \\ 
%\ipa{tsʰ} & & & \\ 
%\ipa{dz} & & & \\ 
%\ipa{ndz} & & & \\ 
%\ipa{s} & & & \\ 
%\ipa{z} & & & \\ 
%\ipa{ɬ} & & & \\ 
%\ipa{tɕ} & & & \\ 
%\ipa{tɕʰ} & & & \\ 
%\ipa{dʑ} & & & \\ 
%\ipa{ndʑ} & & & \\ 
%\ipa{ɕ} & & & \\ 
%\ipa{ʑ} & & & \\ 
\ipa{tʂ} & \deux{ɕtʂ} \idph{} & \japhug{ɕtʂaŋlaŋ}{hanging and swinging} \\ 
%\ipa{tʂʰ} & & & \\ 
%\ipa{dʐ} & & & \\ 
%\ipa{ndʐ} & & & \\ 
%\ipa{ʂ} & & & \\ 
%\ipa{c} & & & \\ 
%\ipa{cʰ} & & & \\ 
%\ipa{ɟ} & & & \\ 
%\ipa{ɲɟ} & & & \\ 
%\ipa{ɲ} & & & \\ 
\ipa{k} &   \deux{ɕk} & \japhug{ɕkom}{muntjac} \\ 
\ipa{kʰ} &   \deux{ɕkʰ} & \japhug{ɕkʰo-nɯ}{they spread it} \\ 
\ipa{g} &   \deux{ʑg} & \japhug{ʑgaʁ}{exactly} \\ 
\ipa{ŋg} &   \deux{ʑŋg} & \japhug{ʑŋgu}{he crosses river on boat} \\ 
\ipa{ŋ} &   \deux{ɕŋ} \idph{} & \japhug{ɕŋaʁɕŋaʁ}{bright yellow} \\ 
%\ipa{x} & 	 & & \\ 
\ipa{q} &   \deux{ɕq} & \japhug{ɕqɤjɤr}{cross-eyed} \\ 
\ipa{qʰ} &   \deux{ɕqʰ} & \japhug{ɕqʰaloʁ}{latch} \\ 
\ipa{ɴɢ} &   \deux{ʑɴɢ} & \japhug{ʑɴɢɯloʁ}{walnut} \\ 
%\ipa{χ} & 	 & & \\ 
\lspbottomrule
\end{tabular}
\end{table}
\resetcounters{2CZC}{3CZC} %deux 


   \begin{table}
 \caption{List of consonant clusters with \ipa{j}  as a first element (12+1)} \label{prein.j}  \centering
\begin{tabular}{l|lll|lll|l}
\lsptoprule
\ipa{p}   & 	 	 \deux{jp}   & \japhug{jpum}{it is thick}  \\ 
%\ipa{pʰ}   & 	 	   & 	 	   & 	 	   \\ 
%\ipa{b}   & 	 	   & 	 	   & 	 	   \\ 
%\ipa{mb}   & 	 	   & 	 	   & 	 	   \\ 
\ipa{m}   & 	 	 \deux{jm}   & \japhug{jmɯt}{he forgets it}  \\ 
\ipa{t}   & 	 	 \deux{jt}   & \japhug{ajtɯ}{it accumulates}  \\ 
%\ipa{tʰ}   & 	 	   & 	 	   & 	 	   \\ 
%\ipa{d}   & 	 	   & 	 	   & 	 	   \\ 
%\ipa{nd}   & 	 	   & 	 	   & 	 	   \\ 
\ipa{n}   & 	 	 \deux{jn}   & \japhug{jnom}{it is flexible}  \\ 
\ipa{ts}   & 	 	 \deux{jts}   & \japhug{tɤ-jtsi}{pillar}  \\ 
\ipa{tsʰ}   & 	 	 \deux{jtsʰ}   & \japhug{jtsʰi}{he gives him to drink}  \\ 
%\ipa{dz}   & 	 	   & 	 	   & 	 	   \\ 
%\ipa{ndz}   & 	 	   & 	 	   & 	 	   \\ 
%\ipa{s}   & 	 	   & 	 	   & 	 	   \\ 
%\ipa{z}   & 	 	   & 	 	   & 	 	   \\ 
%\ipa{ɬ}   & 	 	   & 	 	   & 	 	   \\ 
%\ipa{tɕ}   & 	 	   & 	 	   & 	 	   \\ 
%\ipa{tɕʰ}   & 	 	   & 	 	   & 	 	   \\ 
%\ipa{dʑ}   & 	 	   & 	 	   & 	 	   \\ 
%\ipa{ndʑ}   & 	 	   & 	 	   & 	 	   \\ 
%\ipa{ɕ}   & 	 	   & 	 	   & 	 	   \\ 
%\ipa{ʑ}   & 	 	   & 	 	   & 	 	   \\ 
%\ipa{tʂ}   & 	 	   & 	 	   & 	 	   \\ 
\ipa{tʂʰ}   & 	 	 \deux{jtʂʰ}   & \japhug{qajtʂʰa}{vulture}  \\ 
%\ipa{dʐ}   & 	 	   & 	 	   & 	 	   \\ 
\ipa{ndʐ}   & 	 	 \deux{jndʐ}   & \japhug{jndʐɤz}{it is thick (powder)}  \\ 
%\ipa{ʂ}   & 	 	   & 	 	   & 	 	   \\ 
%\ipa{c}   & 	 	   & 	 	   & 	 	   \\ 
%\ipa{cʰ}   & 	 	   & 	 	   & 	 	   \\ 
%\ipa{ɟ}   & 	 	   & 	 	   & 	 	   \\ 
%\ipa{ɲɟ}   & 	 	   & 	 	   & 	 	   \\ 
%\ipa{ɲ}   & 	 	   & 	 	   & 	 	   \\ 
\ipa{k}   & 		 \deux{jk}   & \japhug{tɤ-jkɯz}{secret}  \\ 
%\ipa{kʰ}   & 		   & 		   & 		   \\ 
%\ipa{g}   & 		   & 		   & 		   \\ 
%\ipa{ŋg}   & 		   & 		   & 		   \\ 
\ipa{ŋ}   & 		 \deux{jŋ}   & \japhug{tɤ-jŋoʁ}{hook}  \\ 
%\ipa{x}   & 		   & 		   & 		   \\ 
\ipa{q}   & 		 \deux{jq}   & \japhug{jqe}{he is able to lift it}  \\ 
%\ipa{qʰ}   & 		   & 		   & 		   \\ 
%\ipa{ɴɢ}   & 		   & 		   & 		   \\ 
\ipa{χ}   & 		 \deux{jχ}   & \japhug{ajχoʁ}{it is flat (belly)}  \\ 
&\trois{jmŋ} & \japhug{tɯ-jmŋo}{dream (n)} \\  
\lspbottomrule
\end{tabular}
\end{table}
   \resetcounters{2jC}{3jC} 

 \begin{table}
 \caption{List of consonant clusters with  \ipa{x} and \ipa{ɣ} as a first element (23+0)} \label{prein.x}  \centering
\begin{tabular}{l|lll}
\lsptoprule
\ipa{p}	 & 	 	 \deux{xp}	 & \japhug{tɯ-xpa}{one year} \\ 
%\ipa{pʰ}	 & 		 & 		 & 		 \\ 
%\ipa{b}	 & 		 & 		 & 		 \\ 
\ipa{mb}	 & 	 	 \deux{ɣmb}	 & \japhug{tɯ-ɣmba}{cheek}  \\ 
\ipa{m}	 & 	 	 \deux{ɣm}	 & \japhug{tɯ-ɣmaz}{wound}  \\ 
\ipa{t}	 & 	 	 \deux{xt}	 & \japhug{xtɯt}{wild cat}  \\ 
\ipa{tʰ}	 & 	 	 \deux{xtʰ}	 & \japhug{xtʰom}{he puts it horizontally}  \\ 
\ipa{d}	 & 	 	 \deux{ɣd}	 & \japhug{ɣdɤso}{species of grub}  \\ 
\ipa{nd}	 & 	 	 \deux{ɣnd}	 & \japhug{ɣnde}{he hits with a hammer}  \\ 
\ipa{n}	 & 	 	 \deux{ɣn}	 & \japhug{ɣnɤsqi}{twenty}  \\ 
\ipa{ts}	 & 	 	 \deux{xts}	 & \japhug{xtsɤɕna}{tip of boot}  \\ 
\ipa{tsʰ}	 & 	 	 \deux{xtsʰ}	 & \japhug{xtsʰɯm}{it is thin}  \\ 
%\ipa{dz} 	 & 		 & 		 & 		 \\ 
%\ipa{ndz}	 & 		 & 		 & 		 \\ 
\ipa{s}	 & 	 	 \deux{xs}	 & \japhug{xsar}{goral}  \\ 
\ipa{z}	 & 	 	 \deux{ɣz}	 & \japhug{ɣzɯ}{monkey}  \\ 
%\ipa{ɬ} 	 & 		 & 		 & 		 \\ 
\ipa{tɕ}	 & 	 	 \deux{xtɕ}	 & \japhug{xtɕi}{it is small}  \\ 
\ipa{tɕʰ}	 & 	 	 \deux{xtɕʰ}	 & \japhug{xtɕʰɯt}{it can hold}  \\ 
%\ipa{dʑ} 	 & 		 & 		 & 		 \\ 
\ipa{ndʑ}	 & 	 	 \deux{ɣndʑ}	 & \japhug{ɣndʑɤβ}{fire}  \\ 
\ipa{ɕ}	 & 	 	 \deux{xɕ}	 & \japhug{xɕaj}{grass}  \\ 
\ipa{ʑ}	 & 	 	 \deux{ɣʑ}	 & \japhug{ɣʑo}{bee}  \\ 
\ipa{tʂ}	 & 	 	 \deux{xtʂ}	 & \japhug{nɤxtʂi}{he will bring it with him}  \\ 
%\ipa{tʂʰ}	 & 		 & 		 & 		 \\ 
%\ipa{dʐ}	 & 		 & 		 & 		 \\ 
%\ipa{ndʐ}	 & 		 & 		 & 		 \\ 
\ipa{ʂ}	 & 	 	 \deux{xʂ} \idph{}	 & \japhug{xʂɤxʂɤt}{long and thin}  \\ 
\ipa{c}	 & 	 	 \deux{xc}	 & \japhug{xcat}{many}  \\ 
\ipa{cʰ}	 & 	 	 \deux{xcʰ}	 & \japhug{tɤlɤxcʰi}{curdled milk}  \\ 
\ipa{ɟ}	 & 	 	 \deux{ɣɟ}	 & \japhug{ɣɟaβ}{he will churn (milk)}  \\ 
%\ipa{ɲɟ}	 &	 & 	 	 	 & 	 	 	 \\ 
\ipa{ɲ}	 & 	 	 \deux{ɣɲ}	 & \japhug{ɯ-ɣɲaʁ}{disaster}  \\ 
\lspbottomrule
\end{tabular}
\end{table}
\resetcounters{2xGC}{3xGC} %deux 


 \begin{table}
 \caption{List of consonant clusters with  \ipa{χ} and \ipa{ʁ} as a first element (25+0)} \label{prein.X.R}  \centering
\begin{tabular}{l|llllllll}
\lsptoprule
\ipa{p}	 &	   \deux{χp} \tib{}	 & \japhug{χpi}{story}  &	   		 \\
\ipa{pʰ}	 &	 	 \deux{χpʰ}	 & \japhug{taχpʰe}{slap}  &	   	 \\
\ipa{b}	 &	\deux{ʁb}  \idph{}	 & \japhug{ʁbɤʁbɤβ}{thick and big}  \\
\ipa{mb}	 &	 	  \deux{ʁmb}	 & \japhug{aʁmbɯm}{concave}  \\
\ipa{m}	 &	 	 \deux{ʁm}	 & \japhug{ʁmaʁ}{army}  \\
\ipa{t}	 &	 	 \deux{χt}	 & \japhug{χtɤrma}{offerings}  &	   	 \\
\ipa{tʰ}	 &	 	 \deux{χtʰ}	 & \japhug{naχthɤβ}{he seizes the opportunity}  &	  	 \\
\ipa{d}	 &	 	 \deux{ʁd} \tib{} 	 & \japhug{ʁdɯɣ}{it is serious}  \\
\ipa{nd}	 &	 	 \deux{ʁnd}	 & \japhug{ʁndɤr}{it scatters}  \\
\ipa{n}	 &	 \deux{ʁn}	 & \japhug{ʁnaʁna}{both}  \\
\ipa{ts}	 &	 	 \deux{χts}	 & \japhug{χtso}{it is clean}  &	   	 \\
\ipa{tsʰ}	 &	 	 \deux{χtsʰ} \idph{}	 & \japhug{χtsʰɤχtsʰɤt}{small and active}  &	  	 \\
%\ipa{dz} 	 &	 	    \\
\ipa{ndz}	 &		 \deux{ʁndz}	 & \japhug{ʁndzɤr}{he cuts it (with scissors)}  \\
\ipa{s}	 &	 	 \deux{χs}	 & \japhug{χsɤr}{gold}  &	  	 \\
\ipa{z}	 &	 \deux{ʁz}  \tib{}	 &	   \japhug{ʁzɤβ}{he is careful in}   		 \\
%\ipa{ɬ} 	 &	 	    \\
\ipa{tɕ}	 &	 	 \deux{χtɕ}  \tib{} 	 & \japhug{χtɕoŋ}{rheumatism}  &	   	 \\
%\ipa{tɕʰ}	 &	   		 \\
%\ipa{dʑ} 	 &	 	    \\
%\ipa{ndʑ}	 &		   	 \\
\ipa{ɕ}	 &	 	 \deux{χɕ}	 & \japhug{χɕu}{it is strong}  &	  	 \\
\ipa{ʑ}	 &	\ipa{ʁʑ}  &\japhug{ʁʑɯnɯ}{young man}   		 \\
\ipa{tʂ}	 &	 	 \deux{χtʂ}  \tib{}	 & \japhug{χtʂɯɣdʑa}{butter tea}  &	  	 \\
%\ipa{tʂʰ}	 &		   	 \\
%\ipa{dʐ}	 &		   	 \\
%\ipa{ndʐ}	 &		   	 \\
\ipa{ʂ}	 &	 	 \deux{χʂ} \idph{}	 & \japhug{χʂɤχʂɤt}{light (clothes)}  &	   	 \\
\ipa{c}	 &	 	 \deux{χc} \tib{}	 & \japhug{χcoŋkroŋ}{cross-legged (sitting)}  &	  	 \\
\ipa{cʰ}	 &	 	 \deux{χcʰ}	 & \japhug{χcʰa}{right}  &	  	 \\
\ipa{ɟ}	 &		 \deux{ʁɟ}	 & \japhug{ʁɟa}{completely}  \\
\ipa{ɲɟ}	 &		 \deux{ʁɲɟ}	 & \japhug{ʁɲɟiʁɲɟi}{enormous}  \\
\ipa{ɲ}	 &	 	\deux{χɲ} \idph{}	 & \japhug{χɲɤχɲɤr}{without energy} \\
& \deux{ʁɲ}\tib{}	 & \japhug{ʁɲɤrpa}{steward (monastery)}  \\
\lspbottomrule
\end{tabular}
\end{table}
\resetcounters{2XRC}{3XRC} %deux 

 \begin{table} %ɴqiaβ
 \caption{List of consonant clusters with a homorganic nasal as  first element (14+1)} \label{prein.nasal}  \centering
\begin{tabular}{l|lll}
\lsptoprule
\ipa{p} 	 &	 \deux{mp} 	 & \japhug{mpɯ}{it is soft} \\	
\ipa{pʰ} 	 &	 \deux{mpʰ} 	 & \japhug{mpʰɯl}{it reproduces} \\	
%\ipa{b} 	 &	 	 &	 	 &	 	 \\	
%\ipa{mb} 	 &	 	 &	 	 &	 	 \\	
%\ipa{m} 	 &	 	 &	 	 &	 	 \\	
\ipa{t} 	 &	 \deux{nt} 	 & \japhug{ntaw}{it is stable} \\	
\ipa{tʰ} 	 &	 \deux{ntʰ} 	 & \japhug{ntʰɤβ}{it is caught between} \\	
%\ipa{d} 	 &	 	 &	 	 &	 	 \\	
%\ipa{nd} 	 &	 	 &	 	 &	 	 \\	
%\ipa{n} 	 &	 	 &	 	 &	 	 \\	
\ipa{ts} 	 &	 \deux{nts} 	 & \japhug{ntsɯ}{always} \\	
\ipa{tsʰ} 	 &	 \deux{ntsʰ} 	 & \japhug{ntsʰɤr}{it neighs} \\	
%\ipa{dz} 	 &	 	 &	 	 &	 	 \\	
%\ipa{ndz} 	 &	 	 &	 	 &	 	 \\	
%\ipa{s} 	 &	 	 &	 	 &	 	 \\	
%\ipa{z} 	 &	 	 &	 	 &	 	 \\	
%\ipa{ɬ} 	 &	 	 &	 	 &	 	 \\	
%\ipa{tɕ} 	 &	 	 &	 	 &	 	 \\	
\ipa{tɕʰ} 	 &	 \deux{ntɕʰ} 	 & \japhug{ntɕʰoz}{he uses it} \\	
%\ipa{dʑ} 	 &	 	 &	 	 &	 	 \\	
%\ipa{ndʑ} 	 &	 	 &	 	 &	 	 \\	
%\ipa{ɕ} 	 &	 	 &	 	 &	 	 \\	
%\ipa{ʑ} 	 &	 	 &	 	 &	 	 \\	
\ipa{tʂ} 	 &	 \deux{ntʂ} 	 & \japhug{ntʂu-nɯ}{they weed} \\	
%\ipa{tʂʰ} 	 &	 	 &	 	 &	 	 \\	
%\ipa{dʐ} 	 &	 	 &	 	 &	 	 \\	
%\ipa{ndʐ} 	 &	 	 &	 	 &	 	 \\	
%\ipa{ʂ} 	 &	 	 &	 	 &	 	 \\	
\ipa{c} 	 &	 \deux{ɲc} 	 & \japhug{ɲcɤr}{he presses on} \\	
\ipa{cʰ} 	 &	 \deux{ɲcʰ} 	 & \japhug{ɲcʰoʁ}{it shrinks} \\	
%\ipa{ɟ} 	 &	 	 &	 	 &	 	 \\	
%\ipa{ɲɟ} 	 &	 	 &	 	 &	 	 \\	
%\ipa{ɲ} 	 &	 	 &	 	 &	 	 \\	
\ipa{k} 	 &	 \deux{ŋk} 	 & \japhug{ŋke}{he walks} \\	
\ipa{kʰ} 	 &	 \deux{ŋkʰ} 	 & \japhug{ŋkʰor}{he arrives} \\	
%\ipa{g} 	 &	 	 &	 	 &	 	 \\	
%\ipa{ŋg} 	 &	 	 &	 	 &	 	 \\	
%\ipa{ŋ} 	 &	 	 &	 	 &	 	 \\	
%\ipa{x} 	 &	 	 &	 	 &	 	 \\	
\ipa{q} 	 &	   \deux{ɴq} 	 & \japhug{ɴqa}{it is difficult} \\	
\ipa{qʰ}   	 &	   \deux{ɴqʰ} 	 & \japhug{ɴqʰi}{it is dirty} \\	
%\ipa{ɴɢ}  	 &	 	 &		 &		 \\	
\midrule
&\trois{mpɕ} &\japhug{mpɕɤr}{it is beautiful} \\
\lspbottomrule
\end{tabular}
\end{table}		
\resetcounters{2NC}{3NC}

 \begin{table} 
 \caption{List of consonant clusters with a non-homorganic nasal as  first element (24+0)} \label{prein.nh.nasal}  \centering
\begin{tabular}{l|lll}
\lsptoprule
%\ipa{p} & & & \\
%\ipa{pʰ} & & & \\
%\ipa{b} &\\
\ipa{mb} &  \deux{nb} 	& \japhug{anbaʁ}{he hides}  \\
%\ipa{m} & & & \\
\ipa{t} & \deux{mt} & \japhug{tɤ-mtɯ}{knot} \\
\ipa{tʰ} & \deux{mtʰ}\tib{} & \japhug{mtʰɯ}{spell} \\
%\ipa{d} & & & \\
\ipa{nd} & \deux{md} & \japhug{mda}{it reaches} \\
\ipa{n} & \deux{mn} & \japhug{mna}{it heals} \\
\ipa{ts} & \deux{mts} & \japhug{tɤ-mtsɯ}{button} \\
\ipa{tsʰ} & \deux{mtsʰ} & \japhug{mtsʰɤm}{he hears} \\
%\ipa{dz} & & & \\
\ipa{ndz} & \deux{mdz} & \japhug{mdzadi}{flea} \\
%\ipa{s} & & & \\
%\ipa{z} & & & \\
%\ipa{ɬ} & & & \\
\ipa{tɕ} & \deux{mtɕ} & \japhug{mtɕoʁ}{it is sharp} \\
\ipa{tɕʰ} & \deux{mtɕʰ} & \japhug{tɤ-mtɕʰo}{wedge} \\
%\ipa{dʑ} & & & \\
\ipa{ndʑ} & \deux{mdʑ} & \japhug{tɯ-mdʑu}{tongue} \\
%\ipa{ɕ} & & & \\
%\ipa{ʑ} & & & \\
\ipa{tʂ} & \deux{mtʂ} & \japhug{kɯ-ɤrɤmtʂɯmtʂaj}{sticky} \\
%\ipa{tʂʰ} & & & \\
%\ipa{dʐ} & & & \\
\ipa{ndʐ} & \deux{mdʐ} & \japhug{mdʐɯɕɯɣ}{bedbug} \\
%\ipa{ʂ} & & & \\
\ipa{c} & \deux{mc} & \japhug{tɤmcar}{tongs} \\
\ipa{cʰ} & \deux{mcʰ} & \japhug{tɯ-mcʰi}{gall} \\
%\ipa{ɟ} & & & \\
\ipa{ɲɟ} & \deux{mɟ} & \japhug{tɯ-mɟa}{jaw} \\
\ipa{ɲ} & \deux{mɲ} & \japhug{mɲɤm}{species of tree} \\
\ipa{k} & \deux{mk} & \japhug{tɯ-mke}{neck} \\
\ipa{kʰ} & \deux{mkʰ} & \japhug{mkʰɤz}{he is expert} \\
%\ipa{g} & \\
\ipa{ŋg} & \deux{mg} & \japhug{tɯ-mga}{advantage} \\
&\deux{ng} 	& \japhug{ngɯt}{it is strong}\\\
\ipa{ŋ} & \deux{mŋ} & \japhug{mŋɤm}{it hurts} \\
& \deux{nŋ} & \japhug{nŋo-nɯ}{they will lose} \\
%\ipa{x} & & & \\
%\ipa{q} & & & \\
%\ipa{qʰ} & & & \\
\ipa{ɴɢ} & \deux{mɢ} & \japhug{tamɢom}{clamp} \\
\lspbottomrule
\end{tabular}
\end{table}		
\resetcounters{2mnC}{3mnC}

 \begin{table}
 \caption{List of consonant clusters ending in   \ipa{w} (10+0)} \label{med.w}  \centering
 \begin{tabular}{l|lll}
\lsptoprule
%\ipa{p}   &    &    &   \\
%\ipa{pʰ}   &    &    &   \\
%\ipa{b}   &    &    &   \\
%\ipa{mb}   &    &    &   \\
%\ipa{m}   &    &    &   \\
%\ipa{w}   &    &    &   \\
%\ipa{t}   &    &    &   \\
%\ipa{tʰ}   &    &    &   \\
\ipa{d}   & \deux{dw}\idph{}   & \japhug{dwaŋdwaŋ}{out of his head} \\
%\ipa{nd}   &    &    &   \\
%\ipa{n}   &    &    &   \\
%\ipa{ts}   &    &    &   \\
%\ipa{tsʰ}   &    &    &   \\
%\ipa{dz}   &    &    &   \\
%\ipa{ndz}   &    &    &   \\
%\ipa{s}   &    &    &   \\
\ipa{z}   & \deux{zw}   & \japhug{zwɤr}{mugwort} \\
\ipa{l}   & \deux{lw}   & \japhug{lwɤz}{he will be sick again} \\
%\ipa{ɬ}   &    &    &   \\
%\ipa{tɕ}   &    &    &   \\
%\ipa{tɕʰ}   &    &    &   \\
%\ipa{dʑ}   &    &    &   \\
%\ipa{ndʑ}   &    &    &   \\
%\ipa{ɕ}   &    &    &   \\
%\ipa{ʑ}   &    &    &   \\
%\ipa{tʂ}   &    &    &   \\
%\ipa{tʂʰ}   &    &    &   \\
%\ipa{dʐ}   &    &    &   \\
%\ipa{ndʐ}   &    &    &   \\
\ipa{r}   & \deux{rw}\tib{}   & \japhug{rwa}{yak felt tent} \\
\ipa{ʂ}   & \deux{ʂw} \tib{}  & \japhug{aɣɯʂwaŋ}{it comes in pairs} \\
%\ipa{c}   &    &    &   \\
%\ipa{cʰ}   &    &    &   \\
%\ipa{ɟ}   &    &    &   \\
%\ipa{ɲɟ}   &    &    &   \\
%\ipa{ɲ}   &    &    &   \\
\ipa{j}   & \deux{jw}   & \japhug{jwajwa}{very thin} \\
\ipa{k}   & \deux{kw}\tib{}   & \japhug{kwitsɯt}{cupboard} \\
%\ipa{kʰ}   &     &    &   \\
%\ipa{g}   &    &    &   \\
%\ipa{ŋg}   &    &    &   \\
%\ipa{ŋ}   &    &    &   \\
\ipa{x}   & \deux{xw}\idph{}   & \japhug{xwɤrnɤxwɤr}{rotating quickly} \\
%\ipa{ɣ}   &    &    &   \\
%\ipa{q}   &    &    &   \\
%\ipa{qʰ}   &    &    &   \\
%\ipa{ɴɢ}   &    &    &   \\
\ipa{χ}   & \deux{χw} \tib{}  & \japhug{χwɤr}{Hor (name)} \\
%\ipa{ʁ}   &    &    &   \\
\ipa{h}   & \deux{hw} \idph{}  & \japhug{hwɤrhwɤr}{wide-mouthed} \\
\lspbottomrule
\end{tabular}
\end{table}		
\resetcounters{2Cw}{3Cw}

  \begin{table}
 \caption{List of consonant clusters ending in  \ipa{j} (20+18)} \label{med.j}  \centering
 \begin{tabular}{l|lll}
\lsptoprule
\ipa{p}   &    \deux{pj}   & \japhug{pjalu}{year of the cock} \\  
%\ipa{pʰ}   &       &    & \\  
\ipa{b}   &    \deux{bj}\idph{}   & \japhug{bjɯbjɯɣ}{hanging in great number} \\  
\ipa{mb}   &    \deux{mbj}   & \japhug{mbjom}{it is fast} \\  
%\ipa{m}   &       &    & \\  
\ipa{w}   &    \deux{wj}   & \japhug{tɕʰɯβja}{duck} \\  
%\ipa{t}   &       &    & \\  
%\ipa{tʰ}   &       &    & \\  
\ipa{d}   &    \deux{dj} \idph{}  & \japhug{dioʁdioʁ}{evenly mixed} \\  
\ipa{nd}   &    \deux{ndj} \idph{}  & \japhug{ndiɤndiɤt}{gracious} \\  
%\ipa{n}   &       &    & \\  
\ipa{ts}   &     \deux{tsj}   & \japhug{tsiaŋnɤtsiaŋ}{very tall, moving} \\  
%\ipa{tsʰ}   &       &    & \\  
%\ipa{dz}   &       &    & \\  
\ipa{ndz}   &    \deux{ndzj}   & \japhug{ndziaʁ}{it is tight (knot)} \\  
\ipa{s}   &    \deux{sj} \idph{}  & \japhug{sjaŋnɤsjaŋ}{shaking one's head} \\  
\ipa{z}   &    \deux{zj} \idph{}  & \japhug{zjaŋzjaŋ}{big} \\  
\ipa{l}   &    \deux{lj}   & \japhug{qaliaʁ}{eagle} \\  
%\ipa{ɬ}   &       &    & \\  
%\ipa{tɕ}   &       &    & \\  
%\ipa{tɕʰ}   &       &    & \\  
%\ipa{dʑ}   &       &    & \\  
%\ipa{ndʑ}   &       &    & \\  
%\ipa{ɕ}   &       &    & \\  
%\ipa{ʑ}   &       &    & \\  
%\ipa{tʂ}   &       &    & \\  
%\ipa{tʂʰ}   &       &    & \\  
%\ipa{dʐ}   &       &    & \\  
%\ipa{ndʐ}   &       &    & \\  
\ipa{r}   &    \deux{rj}   & \japhug{tɯ-rju}{word} \\  
%\ipa{ʂ}   &       &    & \\  
%\ipa{c}   &       &    & \\  
%\ipa{cʰ}   &       &    & \\  
%\ipa{ɟ}   &       &    & \\  
%\ipa{ɲɟ}   &       &    & \\  
%\ipa{ɲ}   &       &    & \\  
%\ipa{j}   &       &    & \\  
\ipa{k}   &    \deux{kj}   & \japhug{pa-kio}{he caused it to slip} \\  
\ipa{kʰ}   &       \deux{kʰj} \idph{} & \japhug{kʰiɤt}{gliding} \\  
%\ipa{g}   &       &    & \\  
\ipa{ŋg}   &    \deux{ŋgj}   & \japhug{ŋgio}{he slips} \\  
%\ipa{ŋ}   &       &    & \\  
%\ipa{x}   &       &    & \\  
\ipa{ɣ}   &    \deux{ɣj}   & \japhug{tɯ-ɣjɤn}{one time} \\  
\ipa{q}   &    \deux{qj}   & \japhug{qiaβ}{it is bitter} \\  
\ipa{qʰ}   &    \deux{qʰj} \idph{}  & \japhug{qʰiɯqʰiɯ}{blunt (colour)} \\  
\ipa{ɴɢ}   &    \deux{ɴɢj}   & \japhug{ɴɢia}{it will come loose} \\  
%\ipa{χ}   &       &    & \\  
\ipa{ʁ}   &    \deux{ʁj}   & \japhug{ʁjit}{he thinks about him} \\ 
%\ipa{h}   &       &    & \\  
\hline
&    \trois{wsj}    & \japhug{tɤ-fsjit}{whistle} \\ 
&    \trois{wzj}  \tib{}   & \japhug{βzjoz}{he learns it} \\ 
\hline
&    \trois{spj}    & \japhug{spjaŋkɯ}{wolf} \\ 
&    \trois{spʰj}    & \japhug{spʰjar}{he dries it} \\ 
&    \trois{stj}  \idph{}   & \japhug{stiaŋnɤstiaŋ}{jumping} \\ 
&    \trois{sqʰj}    & \japhug{sqʰiar}{he stretches it} \\ 
\hline
&    \trois{ltʰj}  \idph{}   & \japhug{ltʰiɤltʰiɤt}{well-ironed (clothes)} \\ 
&    \trois{lbj} \idph{}   & \japhug{lbjɯlbjɯɣ}{hanging} \\ 
\hline
&    \trois{ʂpj}    & \japhug{rpjɯ}{it is spoiled (milk)} \\ 
&    \trois{rmbj}    & \japhug{tɤ-rmbja}{flash of lightning} \\ 
&    \trois{ʂtsj}    & \japhug{rtsiaʁ}{it is steep (road)} \\ 
&    \trois{ʂqʰj}    & \japhug{ɯ-rqʰioʁ}{its notch} \\ 
&    \trois{rɴɢj}    & \japhug{arɤrɴɢioʁ}{having a notch} \\ 
\hline
&    \trois{χtsj}    & \japhug{χtsiɯ}{pint} \\ 
&    \trois{χpj} \tib{}   & \japhug{χpjɤt}{he  observes it} \\ 
&    \trois{χsj}    & \japhug{ɯ-χsjɯβ}{its slough} \\ 
\hline
&    \trois{mpj}    & \japhug{mpja}{it is warm} \\ 
&    \trois{mtsj}    & \japhug{ɯ-mtsioʁ}{its beak} \\ 
\lspbottomrule
\end{tabular}
\end{table}		
\resetcounters{2Cj}{3Cj}


  \begin{table}
 \caption{List of consonant clusters ending in  \ipa{l} (18+12)} \label{med.l}  \centering
 \begin{tabular}{l|lll}
\lsptoprule
\ipa{p}  &  \deux{pl}  & \japhug{plɯt}{he destroys it} \\ 
%\ipa{pʰ}  &    &    &    \\ 
%\ipa{b}  &    &    &    \\ 
\ipa{mb}  &  \deux{mbl}  & \japhug{mblɯt}{it is destroyed} \\ 
%\ipa{m}  &    &    &    \\ 
\ipa{w}  &  \deux{wl}  & \japhug{βli}{he plants it} \\ 
%\ipa{t}  &    &    &    \\ 
%\ipa{tʰ}  &    &    &    \\ 
%\ipa{d}  &    &    &    \\ 
%\ipa{nd}  &    &    &    \\ 
%\ipa{n}  &    &    &    \\ 
\ipa{ts}   &  \deux{tsl}\idph{}  & \japhug{tslɯɣtslɯɣ}{completely wrapped up} \\ 
%\ipa{tsʰ}  &    &    &    \\ 
%\ipa{dz}  &    &    &    \\ 
%\ipa{ndz}  &    &    &    \\ 
\ipa{s}  &  \deux{sl}  & \japhug{sloʁ}{it digs (with its snout)} \\ 
\ipa{z}  &  \deux{zl} \tib{}  & \japhug{tɯ-zloʁ}{one time} \\ 
%\ipa{l}  &    &    &    \\ 
%\ipa{ɬ}  &    &    &    \\ 
%\ipa{tɕ}  &    &    &    \\ 
%\ipa{tɕʰ}  &    &    &    \\ 
%\ipa{dʑ}  &    &    &    \\ 
%\ipa{ndʑ}  &    &    &    \\ 
\ipa{ɕ}  &  \deux{ɕl}  & \japhug{ɕlu}{he ploughs} \\ 
%\ipa{ʑ}  &    &     &    \\ 
%\ipa{tʂ}  &    &    &    \\ 
%\ipa{tʂʰ}  &    &    &    \\ 
%\ipa{dʐ}  &    &    &    \\ 
%\ipa{ndʐ}  &    &    &    \\ 
\ipa{r}  &  \deux{rl}  & \japhug{rlaʁ}{it disappears} \\ 
%\ipa{ʂ}  &    &    &    \\ 
\ipa{c}  &  \deux{cl} \idph{} & \japhug{claŋclaŋ}{round and smooth} \\ 
%\ipa{cʰ}  &    &    &    \\ 
%\ipa{ɟ}  &    &    &    \\ 
%\ipa{ɲɟ}  &    &    &    \\ 
%\ipa{ɲ}  &    &    &    \\ 
\ipa{j}  &  \deux{jl}  & \japhug{jla}{hybrid yak} \\ 
\ipa{k}  &  \deux{kl}  & \japhug{klɯklɯɣ}{stiff} \\ 
%\ipa{kʰ}  &    &    &    \\ 
\ipa{g}  &  \deux{gl} \idph{} & \japhug{glɤɣglɤɣ}{pressed} \\ 
\ipa{ŋg}  &  \deux{ŋgl}  & \japhug{cɯŋglɯɣ}{pestle} \\ 
%\ipa{ŋ}  &    &    &    \\ 
%\ipa{x}  &    &    &    \\ 
\ipa{ɣ}  &  \deux{ɣl}  & \japhug{ɣle}{he rubs it} \\ 
\ipa{q}  &  \deux{ql}  & \japhug{qlɯt}{he breaks it} \\ 
\ipa{qʰ}  &  \deux{qʰl} \tib{} & \japhug{qʰlɯ}{naga} \\ 
\ipa{ɴɢ}  &  \deux{ɴɢl}  & \japhug{ɴɢlɯt}{it breaks} \\ 
%\ipa{χ}  &    &    &    \\ 
\ipa{ʁ}  &  \deux{ʁl}  & \japhug{tɯ-ʁla}{forearm} \\ 
\hline 
 & \trois{scl}  \idph{} & \japhug{sclaŋsclaŋ}{bald} \\ 
 & \trois{sql}   & \japhug{sqlɯm}{it will sink in} \\ 
 & \trois{sqʰl}   & \japhug{asqʰlu}{it is concave} \\ 
\hline 
 & \trois{ɕpl} & \japhug{ɕploʁɕploʁ}{round and smooth} \\ 
 & \trois{ɕkl} & \japhug{ɕkliɕkli}{round and stiff} \\ 
 & \trois{ɕql}  \idph{} & \japhug{ɕqlɯwnɤɕqlɯw}{walking in the water} \\ 
 & \trois{ɕqʰl}   & \japhug{ɕqʰlɤt}{it disappears} \\ 
\hline 
 &  \trois{rɴɢl}   & \japhug{arɴɢlɯm}{it is concave} \\ 
\hline 
 & \trois{χpl} \idph{} & \japhug{χploʁχploʁ}{round like a ball} \\ 
 & \trois{ʁɲɟl}  \idph{} & \japhug{ʁɲɟliʁɲɟli}{big and tall} \\ 
\hline 
 & \trois{mql}   & \japhug{mqlaʁ}{he swallows it} \\ 
 & \trois{mɢl}   & \japhug{tɯ-mɢla}{one step} \\ 
\lspbottomrule
\end{tabular}
\end{table}		
\resetcounters{2Cl}{3Cl}

  \begin{table}
 \caption{List of consonant clusters with two elements ending in  \ipa{r} (26)} \label{med.r2}  \centering
 \begin{tabular}{l|lll}
\lsptoprule
\ipa{p} &  \deux{pr} & \japhug{pri}{bear} \\ 
\ipa{pʰ} &  \deux{pʰr} & \japhug{kʰɤpʰrɯ}{spraying water with the mouth} \\ 
\ipa{b} &  \deux{br} \idph{} & \japhug{brɯbrɯz}{having pimples} \\ 
\ipa{mb} &  \deux{mbr} & \japhug{mbrɤt}{it breaks} \\ 
%\ipa{m} &  & & \\ 
\ipa{w} &  \deux{wr} & \japhug{βraʁ}{he attaches it} \\ 
%\ipa{t} &  & & \\ 
%\ipa{tʰ} &  & & \\ 
\ipa{d} &  \deux{dr} \idph{} & \japhug{droŋdroŋ}{big and dirty} \\ 
\ipa{nd} &  \deux{ndr} & \japhug{qɯmndroŋ}{wild goose} \\ 
%\ipa{n} &  & & \\ 
\ipa{ts} &  \deux{tsr} & \japhug{tsri}{it is salty} \\ 
%\ipa{tsʰ} &  & & \\ 
%\ipa{dz} &  & & \\ 
\ipa{ndz} &  \deux{ndzr} & \japhug{ndzri}{he wrings it} \\ 
\ipa{s} &  \deux{sr} & \japhug{srɯn}{cotton} \\ 
\ipa{z} &  \deux{zr} & \japhug{zrɯ}{sunny side of the mountain} \\ 
%\ipa{l} &  & & \\ 
%\ipa{ɬ} &  & & \\ 
\ipa{tɕ} &  \deux{tɕr} \idph{} & \japhug{tɕrɯɣnɤtɕrɯɣ}{crunching} \\ 
%\ipa{tɕʰ} &  & & \\ 
%\ipa{dʑ} &  & & \\ 
%\ipa{ndʑ} &  & & \\ 
\ipa{ɕ} &  \deux{ɕr} & \japhug{ɕri}{it leaks} \\ 
\ipa{ʑ} &  \deux{ʑr} & \japhug{ʑru}{it is strong} \\ 
%\ipa{tʂ} &  & & \\ 
%\ipa{tʂʰ} &  & & \\ 
%\ipa{dʐ} &  & & \\ 
%\ipa{ndʐ} &  & & \\ 
%\ipa{r} &  & & \\ 
%\ipa{ʂ} &  & & \\ 
\ipa{c} &  \deux{cr} \idph{} & \japhug{crɯɣcrɯɣ}{in a mess} \\ 
\ipa{cʰ} &  \deux{cʰr}\idph{} & \japhug{cʰrɤβcʰrɤβ}{messy and dirty} \\ 
\ipa{ɟ} &  \deux{ɟr} \idph{} & \japhug{ɟrɯɣɟrɯɣ}{gurgling} \\ 
%\ipa{ɲɟ} &  & & \\ 
%\ipa{ɲ} &  & & \\ 
\ipa{j} &  \deux{jr} & \japhug{ɯ-jroʁ}{its furrow} \\ 
\ipa{k} &  \deux{kr} & \japhug{krɤɣ}{he cuts/mows it} \\ 
\ipa{kʰ} &  \deux{kʰr} & \japhug{kʰro}{much} \\ 
\ipa{g} &  \deux{gr} & \japhug{grɯβgrɯβ}{matsutake} \\ 
\ipa{ŋg} &  \deux{ŋgr} & \japhug{ŋgrɤl}{it is usually the case} \\ 
%\ipa{ŋ} &  & & \\ 
%\ipa{x} &  & & \\ 
\ipa{ɣ} &  \deux{ɣr} & \japhug{ɣro}{he suffocates} \\ 
\ipa{q} &  \deux{qr} & \japhug{qro}{pigeon} \\ 
%\ipa{qʰ} &  & & \\ 
\ipa{ɴɢ} &  \deux{ɴɢr} & \japhug{ɴɢraʁ}{it is torn} \\ 
%\ipa{χ} &  & & \\ 
\ipa{ʁ} &  \deux{ʁr} & \japhug{ʁrɯlu}{without horns} \\ 
\lspbottomrule
\end{tabular}
\end{table}

  \begin{table}
 \caption{List of consonant clusters with three elements ending in \ipa{r} (41)} \label{med.r3}  \centering
 \begin{tabular}{l|lll}
\lsptoprule
 & \trois{wkr} \tib{} & \japhug{fkrɯz}{he is greedy} \\ 
 & \trois{wɣr} & \japhug{wɣrum}{it is white} \\ 
 & \trois{wsr} & \japhug{fsraŋ}{he protects it} \\ 
\hline
 & \trois{spr} & \japhug{sprɯskɯ}{reincarnated} \\ 
 & \trois{zbr} \tib{} & \japhug{zbrilu}{year of the snake} \\ 
 & \trois{zmbr} & \japhug{sɤzmbri}{he makes him angry} \\ 
 & \trois{stʰr} \idph{} & \japhug{stʰrɯβ}{dangling (of snot)} \\ 
 & \trois{scr} \idph{} & \japhug{scraʁscraʁ}{very small} \\
 & \trois{zɟr} \idph{} & \japhug{zɟraŋzɟraŋ}{soft and bloated} \\ 
 & \trois{skr} & \japhug{skraskra}{impolite} \\ 
 & \trois{skʰr} & \japhug{tɯ-skʰrɯ}{body} \\ 
 & \trois{zgr} \tib{} & \japhug{zgrawa}{leather sack} \\ 
 & \trois{sqr} & \japhug{sɤsqra}{limit} \\ 
\hline
 & \trois{ɕpr} & \japhug{aɕprɯm}{it is badly sewed} \\ 
 & \trois{ʑmbr} & \japhug{ʑmbri}{willow} \\ 
 & \trois{ɕtr} \idph{} & \japhug{ɕtraŋɕtraŋ}{long and soft} \\ 
 & \trois{ʑdr} \idph{} & \japhug{ʑdraŋʑdraŋ}{long and soft} \\ 
 & \trois{ɕkr} & \japhug{ɕkrɤz}{oak} \\ 
 & \trois{ʑgr} & \japhug{ʑgrɯɣ}{certainly} \\ 
 & \trois{ʑŋgr} & \japhug{ʑŋgri}{star} \\ 
 & \trois{ɕqr} & \japhug{ɕqraʁ}{he is intelligent} \\ 
 &  \trois{ʑɴɢr} & \japhug{ʑɴɢro}{Jew's harp} \\ 
\hline
 & \trois{jkr} & \japhug{jkrɯt}{it will solidify} \\ 
 & \trois{jtsr} & \japhug{jtsraβ}{he delays his departure} \\ 
\hline 
&  \trois{xpr} & \japhug{ta-ɣɤxpra}{he sent him} \\ 
\hline
 & \trois{χpr} & \japhug{tɕʰɯχpri}{newt} \\ 
 & \trois{ʁmbr} & \japhug{taʁmbra}{crying and shouting} \\ 
 & \trois{χsr} & \japhug{ɣɤχsrɯ}{handsome} \\ 
 & \trois{ʁzr} & \japhug{ʁzraŋʁzraŋ}{dishevelled} \\ 
 & \trois{χcr} \idph{} & \japhug{χcɯχcri}{thin, diluted} \\ 
 & \trois{ʁɟr} \idph{} & \japhug{ʁɟɯʁɟri}{fat and soft} \\ 
 & \trois{ʁgr} \tib{} & \japhug{ʁgra}{enemy} \\ 
\hline
 & \trois{ɲcr} \idph{} & \japhug{ɲcɯɲcri}{thin, diluted} \\ 
 & \trois{ŋkʰr} & \japhug{ŋkʰrɯli}{screw} \\ 
 & \trois{ngr} & \japhug{ngrɯβ}{accomplish} \\
 & \trois{ɴqr} & \japhug{ɯ-ɴqra}{shabby} \\ 
\hline
 & \trois{mtsr} & \japhug{mɯmtsrɯɣ}{he drinks it with a straw} \\ 
 & \trois{mpʰr} & \japhug{mpʰrɯmɯ}{divination} \\ 
 & \trois{mkʰr} & \japhug{mkʰroŋ}{he will be reincarnated} \\ 
 & \trois{mgr} & \japhug{mgrɯn}{he receives him} \\ 
\hline
 &\trois{nbr} & \japhug{nbraʁ}{he hoes it} \\ 
\lspbottomrule
\end{tabular}
\end{table}		
\resetcounters{2Cr}{3Cr}




\begin{table}
 \caption{Count of consonant clusters} \label{tab:clusters.tot}  \centering
\begin{tabular}{lrrrr}
  \lsptoprule	
type &CC& CCC& total\\		
\midrule
\ipab{wC}  & 	\arabic{2wC}  & \arabic{3wC}  &   \addition{2wC}{3wC}  & 	\\	
\ipab{s/zC}  & 	\arabic{2szC}  & \arabic{3szC}  &   \addition{2szC}{3szC}  & 	\\	
\ipab{lC}  & 	\arabic{2lC}  & \arabic{3lC}  &   \addition{2lC}{3lC}  & 	\\	
\ipab{ʂ/rC}  & 	\arabic{2rC}  & \arabic{3rC}  &   \addition{2rC}{3rC}  & 	\\	
\ipab{jC}  & 	\arabic{2jC}  & \arabic{3jC}  &   \addition{2jC}{3jC}  & 	\\	
\ipab{ɕ/ʑC}  & 	\arabic{2CZC}  & \arabic{3CZC}  &   \addition{2CZC}{3CZC}  & 	\\	
\ipab{x/ɣC}  & 	\arabic{2xGC}  & \arabic{3xGC}  &   \addition{2xGC}{3xGC}  & 	\\	
\ipab{χ/ʁC}  & 	\arabic{2XRC}  & \arabic{3XRC}  &   \addition{2XRC}{3XRC}  & 	\\	
\ipab{NC}  & \arabic{2NC}  & \arabic{3NC}  &   \addition{2NC}{3NC}  & 	\\	
\ipab{m/nC}  & \arabic{2mnC}  & \arabic{3mnC}  &   \addition{2mnC}{3mnC}  & 	\\	
\midrule
\ipab{Cɕ}  & 	2  & 	  & 	  2& 	\\	
\midrule
\ipab{Cw}  & 	 \arabic{2Cw}  & \arabic{3Cw}  &   \addition{2Cw}{3Cw}  & 	\\
\ipab{Cj}  & 	 \arabic{2Cj}  & \arabic{3Cj}  &   \addition{2Cj}{3Cj}  & 	\\
\ipab{Cl}  & 	 \arabic{2Cl}  & \arabic{3Cl}  &   \addition{2Cl}{3Cl}  & 	\\
\ipab{Cr}  & 	 \arabic{2Cr}  & \arabic{3Cr}  &   \addition{2Cr}{3Cr}  & 	\\
%\ipab{Cɣ; Cʁ} & \arabic{2Cg}  & \arabic{3Cg}  &   \addition{2Cg}{3Cg}  & 	\\
%\midrule
%total & \totdeux & \tottrois & \ADD{\totdeux}{\tottrois}{\total}\total \\
\lspbottomrule
\end{tabular}
\end{table}

\subsection{Syllabic contraints} 
\subsubsection{Uvular harmony} \label{sec:uvular.harmony}
Velars and uvulars do not coexist well within the same syllable in Japhug. There are no syllables of the type $\dagger$\forme{QMVɣ} or $\dagger$\forme{KMVʁ} (where K and Q  represent any velar and uvular initial consonants respectively): the initial and the coda have to be both velars, or both uvulars. Thus, syllables such as \japhug{qraʁ}{ploughshare} and \japhug{krɤɣ}{shear, mow} are possible, but not $\dagger$\forme{kraʁ} or $\dagger$\forme{qrɤɣ}. 

With preinitials and medial consonants, the constraint depends on the context. Several cases have to be distinguished.

First, the uvular coda \ipa{-ʁ} is compatible with the velar medial \ipa{-ɣ-}, as shown by examples such as \japhug{pɣaʁ}{turn over}; the opposite case is not attested, but given the relative rarity of medial \ipa{-ʁ-}, this may be fortuitous.

Second, the uvular preinitial  \ipa{ʁ-} does occur with velar initials in Tibetan loanwords, as in \japhug{ʁgra}{enemy} (from Tibetan \tibet{དགྲ་}{dgra}{enemy}).

Third, a uvular preinitial with the velar medial \ipa{-ɣ-} is only found in the dialectal word \japhug{tɯ-χpɣi}{thigh}.

These constraints do not apply across syllables, as shown by words such as \japhug{koʁmɯz}{just as}, in which the \ipa{ʁ} is the preinitial of the second syllable.

The discrimination of uvulars and velars is due to a recent sound change that occurred in Japhug and affected both native words and Tibetan loanwords, the uvularization of velar initial consonants in syllables with uvular \ipa{-ʁ}. This sound change explains for instance why Japhug words such as \japhug{tɯ-qʰoχpa}{organs, state of mind} (phonologically \ipa{qʰoʁ.pa} with internal sandhi) from Tibetan \tibet{ཁོག་པ་}{kʰog.pa}{insides}\footnote{See section \ref{sec:body.part} for an account of the prefix \forme{tɯ-}.} has a uvular \ipa{qʰ-} corresponding to a velar \ipa{kʰ-} in Tibetan: dorsal codas (transcribed as \forme{-g}) are realized as uvulars after \forme{a} and \forme{o} in most Tibetan varieties (\citealt{gong16amdo}), so that a correspondence of Tibetan \forme{-ag} and \forme{-og} to Japhug \ipa{-aʁ} and \ipa{-oʁ} is expected. At an earlier stage, \forme{tɯ-qʰoχpa} has probably been borrowed as *\forme{tɯ-kʰoʁpa} and the sound law *\textsc{velar} \fl{} \textsc{uvular} /\_V\forme{ʁ} applied to it as to the rest of the vocabulary.
 %Phonology
%\chapter{Nominal morphology} \label{chap:nominal.morphology}
This chapter does not treat of grammatical categories expressed by independent words or clitics, such as number and grammatical relations (discussed in chapter \ref{chap:noun.phrase}), and focuses on possessive prefixes, compounding and  noun derivations. Nominalization (including lexicalized deverbal nouns) and denominal verbalization are treated in chapters XXX and XXX respectively. 

The morphology of counted nouns (quantifiers, time nominals) is discussed in § \ref{sec:counted.nouns}, and that of nouns of location in chapter XXX.

\section{Possessive prefixes}  \label{sec:possessive.prefixes}
 Nouns in Japhug can be divided into four main subclasses, inalienably possessed nouns (IPN), alienably possessed nouns (APN), unpossessible nouns (UN) and counted nouns (CN), depending on the type of prefixes they can take.  The present section focuses on the first two, the ones that are compatible with possessive prefixes. UNs are treated in § \ref{sec:unpossessible.nouns}, and CNs in § \ref{sec:counted.nouns}.

\subsection{Possessive paradigm} \label{sec:possessive.paradigm}
The paradigm of possessive prefixes in Japhug is indicated in Table \ref{tab:possessive.prefixes}. It presents obvious commonalities with the personal pronouns (section \ref{sec:pers.pronouns}) and the indexation suffixes (section XXX), a question studied in more detail in XXX. 

\begin{table}[h] \centering
\caption{Possessive prefixes }\label{tab:possessive.prefixes}
\begin{tabular}{lllllllll} \lsptoprule
 Prefix & Person \\
\midrule
\forme{a-}  &		1\sg{} \\
\forme{nɤ-}  &			2\sg{} \\
\forme{ɯ-}  &			3\sg{} \\
\midrule
\forme{tɕi-}  &			1\du{} \\
\forme{ndʑi-}  &		2/3\du{} \\	
\midrule
\forme{i-}  &			1\pl{} \\
\forme{nɯ-}  &			2/3\pl{} \\
\midrule
\forme{tɯ-/tɤ-/ta-}  &			indefinite \\
\forme{tɯ-}  &			generic \\
\lspbottomrule
\end{tabular}
\end{table}

In the possessive paradigm, the contrast between second and third person is neutralized in the dual and plural, while it is preserved in pronouns and person indexation.

Unlike languages like Situ which have two series of possessive pronouns with the same initial consonant but distinct vocalism (\citealt[168-169]{linxr93jiarongen}),\footnote{\citet[118-119]{prins16kyomkyo} analyzes the vowel as part of the nominal root.} Japhug preserves the vowel contrast \ipa{ɯ} vs \ipa{ɤ} only with the indefinite possessor form of inalienably possessed nouns; with definite possessors, the contrast is neutralized.

Stacking of possessive prefixes is not allowed in Japhug, with the exception of the combination of a definite possessor prefix with an indefinite possessor prefix \forme{tɯ-} or \forme{tɤ-} to turn an inalienably possessed noun into an alienably possessed one (see § \ref{sec:alienabilization}).

Unlike verbs (see § XXX), nouns whose stem begins in \forme{a-} are extremely few in Japhug. Nevertheless, as is the case with verbs, the vowel \forme{a-} merges with any prefixed element, so that nouns of this type do not have regular possessive forms. The only noun in \forme{a-} to commonly receive possessive prefixes is \japhug{araʁ}{vodka}; its possessive forms are \textsc{1sg} \forme{aʑɤ-raʁ}, \textsc{2sg} \forme{nɤʑɤ-nɤ-raʁ} and \textsc{2pl} \forme{nɯʑɤ-nɯ-raʁ} (as in \ref{ex:nWZAnWraR}), combining the pronoun in \textit{status constructus} followed by the possessive prefix, which takes over the initial \forme{a-}.

\begin{exe}
\ex \label{ex:nWZAnWraR}
\gll nɯʑɤ-nɯ-raʁ ɯ́-ra \\
\textsc{2pl}-\textsc{2pl.poss}-vodka \textsc{qu}-have.to:\textsc{fact} \\
\glt `Do you need vodka?' (elicited)
\end{exe}

\subsubsection{The expression of possession} \label{ex:prefix.expression.of.possession}
Possession cannot be expressed without a possessive prefix on the possessee. Possessive prefixes can be used on nearly any noun (except the unpossessed nouns, see § \ref{sec:unpossessible.nouns}), including recent borrowings from Chinese (or quasi-code switching), as \zh{老家} \forme{lǎojiā} `place of origin; old house' in (\ref{ex:aZo.GW.alaojia}). They also occur on several non-finite verbal forms, including participles (see § XXX and § XXX), bare infinitives (§ XXX) and degree nominal (§ XXX).

\begin{exe}
\ex \label{ex:aZo.GW.alaojia}
\gll
aʑo ɣɯ a-<laojia> ɣɯ ɯ-lɤcu nɯre ri ku-rɤʑi-nɯ ŋu \\
\textsc{1sg} \textsc{gen} \textsc{1sg.poss}-old.house \textsc{gen} \textsc{3sg.poss}-upstream there \textsc{loc} \textsc{ipfv}-stay-\textsc{pl} be:\textsc{fact} \\
\glt `They live in a place upstream from my old house.' (14-tApitaRi, 238)
\end{exe}

In the case of first or second person possessors, it is possible to have simply a possessive prefix on the noun, (\japhug{a-ɣɲi}{my friend}, \japhug{a-mbro}{my horse} and \japhug{a-ʁgra}{my enemy} in \ref{ex:ambro}), a personal pronoun and a possessive prefix (same person and number, as in \ref{ex:aZo.ambro}) or even a pronoun, the genitive clitic \forme{ɣɯ} and a possessive prefix as in (\ref{ex:aZo.GW.alaojia}) (§ \ref{sec:gen.possession}).

 \begin{exe}
\ex \label{ex:ambro} 
\gll a-ɣɲi ci tɯ\redp{}tɯ-ŋu nɤ, a-mbro ɯ-lwa ɯ-taʁ kɤ-zo, a-ʁgra ci tɯ\redp{}tɯ-ŋu nɤ, a-mbro ɯ-jme ɯ-taʁ kɤ-zo \\
\textsc{1sg.poss}-friend \textsc{indef} \textsc{cond}\redp{}2-be:\textsc{fact} \textsc{lnk} \textsc{1sg.poss}-horse \textsc{3sg.poss}-mane \textsc{3sg}-on \textsc{imp}-land \textsc{1sg.poss}-enemy \textsc{indef} \textsc{cond}\redp{}2-be:\textsc{fact} \textsc{lnk} \textsc{1sg.poss}-horse \textsc{3sg.poss}-tail  \textsc{3sg}-on \textsc{imp}-land  \\
\glt `If you are my friend, land on my horse's mane, if you are my enemy, land on my horse's tail.' (2002qaCpa, 196)
\end{exe}

\begin{exe}
\ex \label{ex:aZo.ambro}
\gll aʑo a-mbro nɤrwɯrɯnbotɕʰi ŋu, tɯ-sŋi χpaχtsʰɤt ci ɲɯ́-wɣ-tsɯm-a cʰa \\
\textsc{1sg} \textsc{1sg.poss}-horse p.n. be:\textsc{fact} one-day yojana \textsc{indef} \textsc{ipfv:west}-\textsc{inv}-take.away-\textsc{1sg} can:\textsc{fact} \\
\glt `My horse is Norbu Rinpoche, he can make me cross one yojana per day.' (2003smanmi2, 54)
\end{exe}

It is possible to have a first singular possessive preceded by a first plural pronoun, as in (\ref{ex:iZo.amu}) (see § XXX for other examples of person mismatch involving \textsc{1pl} pronouns).

\begin{exe}
\ex \label{ex:iZo.amu}
\gll iʑo a-mu nɯ tʰamtʰam kɯrcɤsqaptɯɣ tʰɯ-azɣɯt ŋu. \\
\textsc{1pl} \textsc{1sg.poss}-mother \textsc{dem} now 81 \textsc{pfv}-reach  be:\textsc{fact} \\
\glt `My mother is now 81.' (2010-histoire09-2, 15)
\end{exe}

\subsubsection{Definiteness and obviation}
Nouns with a definite possessor in Japhug can be indefinite, unlike in most languages of Europe. They can occur with an indefinite determiner (example \ref{ex:ambro}  above). With a quantifier such as \japhug{tɯ-rdoʁ}{one piece} as in (\ref{ex:Wzda.tWrdoR}), a noun with a definite possessor is interpreted as referring to a certain number of persons out of a group (`one of his X').

 \begin{exe}
\ex \label{ex:Wzda.tWrdoR}
\gll tɤ-tɕɯ nɯ kɯ ɯ-zda tɯ-rdoʁ ɯ-pʰe to-ti, tɯ-rdoʁ nɯ kɯ li ci ɯ-pʰe tɕe ɲɤ-k-ɤ-sɯ-ɤmɯ-mtsʰɯ\redp{}mtsʰɤm-nɯ \\
\textsc{indef.poss}-son \textsc{dem} \textsc{erg} \textsc{3sg.poss}-companion one-\textsc{cl} \textsc{3sg-dat} \textsc{ifr}-say one-\textsc{cl}  \textsc{dem} \textsc{erg} again \textsc{indef} \textsc{3sg-dat} \textsc{lnk}   \textsc{ifr}-\textsc{evd}-\textsc{pass}-\textsc{caus}-\textsc{recip}-hear-\textsc{pl} \\
\glt `The boy told one of his companion, and that one another one, and they informed each other.' (x1-sloXpWn, 82)
\end{exe}

Unlike Algonquian languages, but like Mapudungun (\citealt{haude16symmetrical}), possessed nouns are not automatically obviative, and inverse marking on the verb is not required if a possessed noun is subject, and its possessor object of a transitive verb, as shown by example (\ref{ex:prox.naBde}) where the direct form \forme{na-βde} appears (see \citealt{jacques10inverse} for other examples, and § XXX on inverse marking). The inverse \forme{nɯ́-wɣ-βde} is also possible in exactly the same context -- example  (\ref{ex:obv.nWwGBde}) comes from the same text and refers to the same event.

\begin{exe}
\ex \label{ex:prox.naBde}
\gll ɯ-rʑaβ nɯ kɯ na-βde \\
\textsc{3sg.poss}-wife \textsc{dem} \textsc{erg} \textsc{pfv}:3\fl3'-throw.away \\
\glt `His wife left him.' (14-tApitaRi, 289)
\end{exe}

\begin{exe}
\ex \label{ex:obv.nWwGBde}
\gll
ɯ-rʑaβ cʰo ɯ-tɕɯ nɯ ʁnaʁna kɯ nɯ́-wɣ-βde qʰe \\
\textsc{3sg.poss}-wife \textsc{comit} \textsc{3sg.poss}-son \textsc{dem} both \textsc{erg} \textsc{pfv}-\textsc{inv}-throw.away \textsc{lnk} \\
\glt `His wife and his son left him.' (14-tApitaRi, 294)
\end{exe}

\subsubsection{Other uses of possessive prefixes} \label{sec:other.uses.poss.prefixes}
Possessive prefixes are also used to express beneficiaries, recipients and other oblique arguments, such as the `person needing' in the construction with the verb \japhug{ra}{need, have to},  as in (\ref{ex:ambro.tARndo.kWtso}).\footnote{See § \ref{sec:gen.beneficiary} for a more detailed account of the expression of beneficiaries in Japhug.}

 \begin{exe}
\ex \label{ex:ambro.tARndo.kWtso}
\gll a-mbro taʁndo kɯ-tso ci tɕi ra \\
\textsc{1sg.poss}-horse speech \textsc{nmls}:S/A-understand one also have.to:\textsc{fact} \\
\glt `I also need a horse who understands speech.' (2003kAndzwsqhaj2, 52)
\end{exe}

In the case of beneficiaries and recipients, if a genitive pronoun or genitive phrase is present, the presence of a possessive prefix is possible (\ref{ex:aZWG.akWra}) but not obligatory (\ref{ex:aZWG.kWra}), in particular in the case of possessed nouns that already have a definite possessor (\ref{ex:aZWG.Wlu}).

 \begin{exe}
\ex \label{ex:aZWG.akWra}
\gll aʑɯɣ a-kɯ-ra ci tu tɕe nɯ `ɣa' tɤ-ti ra \\
\textsc{1sg:gen} \textsc{1sg.poss}-\textsc{nmlz}:S/A-have.to \textsc{indef} exist:\textsc{fact} \textsc{lnk} \textsc{dem} yes \textsc{imp}-say have.to:\textsc{fact} \\
\glt `There is one thing I need, and you have to say `yes' to it.' (140429 qingwa wangzi, 47)
\end{exe}

 \begin{exe}
\ex \label{ex:aZWG.kWra}
\gll  aʑɯɣ kɯ-ra me \\
\textsc{1sg:gen} \textsc{nmlz}:S/A-have.to not.exist:\textsc{fact} \\
\glt `I don't need anything.' (Norbzang, 275)
\end{exe}

 \begin{exe}
\ex \label{ex:aZWG.Wlu}
\gll aʑɯɣ ɯ-lu ra \\
\textsc{1sg:gen} \textsc{3sg.poss}-milk have.to:\textsc{fact} \\
\glt `I want its milk.' (02-deluge2012, 12)
\end{exe}

\subsubsection{The form of the \textsc{3sg} possessive prefix}
Japhug differs from other Gyalrong languages (Table \ref{tab:3sg.inv}, data from \citealt{jackson02rentongdengdi}, \citealt{gongxun14agreement}) in that the third person prefix is \textit{not} homophonous with the inverse prefix. 

\begin{table}
\caption{The form of the \textsc{3sg} possessive prefix in Gyalrong languages} \label{tab:3sg.inv} 
\begin{tabular}{lllll}
\toprule
& \textsc{3sg.poss} & inverse \\
\midrule
Japhug &  \forme{ɯ-} & \forme{ɣɯ-}/\forme{-wɣ-} \\
Tshobdun &  \forme{o-} & \forme{o-}  \\
Zbu &   \forme{wə-} & \forme{wə-} \\
Situ &    \forme{wə-} & \forme{wə-} \\
\bottomrule
\end{tabular}
\end{table}

Independently of the question of whether these two prefixes could be historically related, it is probable that Japhug is innovative here, as there is a trace of an allomorph \forme{-w-} similar to the inverse (see § XXX on the allomorphs of the inverse prefix) . The linker \japhug{núndʐa}{for this reason} originates from a phrase combining the demonstrative \japhug{nɯ}{this} (on which see § \ref{sec:anaphoric.demonstrative.pro} and § XXX) with the \textsc{3sg} possessed form of the noun \japhug{ɯ-ndʐa}{reason}; in this frozen compound, the third person possessive survives as colouring of the vowel of the demonstrative.

There is a possible trace of the expected allomorph \forme{ɣɯ-} (from proto-Gyalrong \forme{*wə-}) in the noun \japhug{ɣɯfsu}{friend}, etymologically `his equal'; the IPN \japhug{ɯ-fsu}{equal in size to}, which shares the same root, has a regular possessive prefix that is coreferent with the standard of comparison (as in \ref{ex:nWfsu} with the \textsc{3pl}; see § XXX on this construction). The APN \japhug{ɣɯfsu}{friend} is thus possibly a lexicalized equivalent of \japhug{ɯ-fsu}{equal in size to}, whose \textsc{3sg} prefix was frozen before the change from \forme{*ɣɯ-} to \forme{ɯ-} occurred.

\begin{exe}
\ex \label{ex:nWfsu}
\gll tɯrme kɯ-mbro ra nɯ-fsu jamar tu-zɣɯt ma mɤ-cha \\
person \textsc{nmlz}:S/A-be.high \textsc{pl}  \textsc{3pl.poss}-equal about \textsc{ipfv}-reach apart.from \textsc{neg}-can:\textsc{fact} \\
\glt `It can only grow about as high as a tall human.' (15-babW, 4)
\end{exe}


It remains unclear why the regular allomorph of the third person possessive is \forme{ɯ-} rather than expected \forme{ɣɯ-}. A possible explanation could be false segmentation, due to reanalysis with the genitive marker \forme{ɣɯ}, since the genitive can optionally occur between the possessor and the possessee, as in (\ref{ex:qaCpa.GW.WpW}). A pre-Japhug form such as \forme{*qaɕpa ɣɯ-pɯ} could have been misanalyzed as \forme{qaɕpa ɣɯ ɯ-pɯ} due to vowel fusion sandhi (section XXX), and a new allomorph \forme{ɯ-} extracted from such constructions.\footnote{The weakness of this hypothesis is that some Japhug dialects have \forme{kɯ} rather than \forme{ɣɯ} as their genitive marker.}

\begin{exe}
\ex \label{ex:qaCpa.GW.WpW}
 \gll nɯnɯ qaɕpa ɣɯ ɯ-pɯ ŋu tɕe, \\
 \textsc{dem} frog \textsc{gen} \textsc{3sg.poss}-young be:\textsc{fact} \textsc{lnk} \\
 \glt `It (the tadpole) is the young of the frog.' (hist-28-kWpAz, 220)
\end{exe} 


\subsection{Inalienably possessed nouns} \label{sec:inalienably.possessed}
IPNs differ from APNs in that they require the presence of a possessive prefix.  When the possessive prefix is definite, IPNs are not formally distinguishable from APNs; for instance, \japhug{a-pi}{my elder sibling} and \japhug{a-mbro}{my horse} both take the 1\sg{} \forme{a-} prefix and no direct clue indicates that the first noun is IPN and that the second one is APN.

The citation form however differs between IPN and APN: the former must take an indefinite possessor prefix (or in some cases a 3\sg{} \forme{ɯ-}), while the latter can occur without possessive prefix, as for instance \japhug{tɤ-pi}{elder sibling} (with the indefinite \forme{tɤ-}; the bare root $\dagger$\forme{pi} is not a correct form) vs \japhug{mbro}{horse} (without prefix).

IPN are divided into four classes depending on their citation form. There are three distinct forms for the indefinite possessor prefix (\forme{tɯ-}, \forme{tɤ-}, \forme{ta-}) whose distribution is not completely predictable on the basis of phonology or semantics (though some generalizations are provided below). In addition, some IPN always only take definite possessive prefixes. The contrast between these four classes is neutralized when the noun takes a definite possessor prefix (unlike in Situ, see \citealt[168-169]{linxr93jiarongen} and \citealt[118-119]{prins16kyomkyo}).

The most common allomorph of the indefinite possessor prefix is \forme{tɯ-}. IPN which select this allomorph, such as \japhug{tɯ-jaʁ}{hand}, have identical indefinite and generic possessor forms (see § \ref{sec:indef.genr.poss}).

The allomorph \forme{tɤ-} is also very common, in particular with kinship terms and some body parts (see \ref{sec:body.part} and \ref{sec:kinship}). The form \forme{ta-} is mainly a phonological variant of \forme{tɤ-}, occurring mainly with nouns whose stem begins with a uvular such as \japhug{ta-ʁrɯ}{horn} or \japhug{ta-ʁi}{younger sibling}. The contrast between \ipa{ɤ} and \ipa{a} is very difficult to perceive before uvulars with some speakers (see § XXX), and the transcription adopted in this grammar (and the online corpus and dictionary) is based on the slow syllable-by-syllable pronunciation of these words by Tshendzin. Two IPNs, however, \japhug{ta-ma}{work} and \japhug{ta-mar}{butter}, have the \forme{ta-} allomorph with an initial \forme{m-}, probably originally due to vowel harmony (see § XXX).

The minimal pair between \japhug{tɤ-ma}{mother} and \japhug{ta-ma}{work} shows that this vowel contrast, however marginal, is distinctive, and that even if the two allomorphs \forme{tɤ-} and \forme{ta-} were originally phonologically conditioned, it is not the case any more in Kamnyu Japhug.

Some IPN never occur with indefinite possessor prefix, for instance \japhug{ɯ-tʰoʁ}{ground} is only attested with the 3\sg{} \forme{ɯ-} prefix (see \ref{sec:earth.IPN}). For some IPNs, the indefinite possessor form is difficult to elicit and in case of doubt the third singular form is given in the dictionary \citet{jacques16japhug} (for instance \japhug{ɯ-mdoʁ}{colour}). Future investigations may reveal an indefinite possessor form for some of these nouns.

When denominal verbs are derived from IPNs, the vocalism of the denominal prefix tends to be the same as that of the  indefinite possessor prefix (for instance \japhug{tɤ-βɟu}{mattress} \fl{} \japhug{nɤβɟu}{use as a mattress}, not $\dagger$\forme{nɯβɟu}), though there are exceptions (\japhug{tɯ-rpaʁ}{shoulder} \fl{} \japhug{mɤrpaʁ}{carry on the shoulder}), as discussed in § XXX.

By analogy with several non-finite verb forms, in particular the subject participle of transitive verbs and the bare infinitive, which index one argument (the object) by a possessive prefix (section XXX), the possessors of IPNs are considered to be \textit{core arguments}, while those of APNs are treated as \textit{adjuncts}. In other words, IPNs have a valency of 1 like intransitive verbs, while APNs have a valency of 0. The indefinite possessive prefix can be viewed as a valency-decreasing device, the nominal equivalent of passive and antipassive derivations, especially given its use in the alienabilization of IPNs (see \ref{sec:alienabilization}). Wider implications of the assumption that possessors of IPNs are core arguments are explored in the chapters on complementation (§  XXX) and relativization (§  XXX).

\subsubsection{Conversion from non-IPN to IPN} \label{sec:apn.to.ipn}
Derivation from APNs to IPNs is not common in Japhug. A interesting case is that of \japhug{ɯ-ʁle}{reputation}, which originates from the APN \japhug{qale}{wind} with a reduced form \forme{ʁ-} of the class prefix \forme{qa-}, as some second members of compounds (see \ref{sec:second.member.alternation} and \ref{sec:class.prefixes}).

Conversion of CN (counted nouns, § XXX) to IPNs is a regular process, studied in more detail in § XXX.

\subsubsection{Body parts} \label{sec:body.part}
The great majority of body parts are IPN with the indefinite possessor \forme{tɯ-}. These include native words, but also borrowings from Tibetan such as \japhug{tɯ-qʰoχpa}{organs, state of mind} from Tibetan \tibet{ཁོག་པ་}{kʰog.pa}{insides} (see \ref{sec:uvular.harmony} on the phonology of this word).

Body parts IPN selecting the prefix \forme{tɤ-} are mainly liquids from the body such as \japhug{tɤ-se}{blood}, \japhug{tɤ-spɯ}{pus} and \japhug{tɤ-lu}{milk} (though some liquids also take the prefix \forme{tɯ-}, for instance \japhug{tɯ-ɕtʂi}{sweat}), hair (\japhug{tɤ-rme}{hair, fur}, \japhug{tɤ-kɤrme}{hair (head)}) and some body parts of animals (\japhug{tɤ-jme}{tail}, \japhug{tɤ-ŋkɯ}{pig skin}, \japhug{tɤ-rkʰom}{feather rachis}).

Parts of plants on the other hand mainly have the prefix \forme{tɤ-}, as \japhug{tɤ-jwaʁ}{leaf}, \japhug{tɤ-tsrɯ}{sprout}, \japhug{tɤ-zrɤm}{root} etc.

APNs are rare among body parts. Some nouns with the \forme{qa-} class prefix (see § \ref{sec:class.prefixes}) such as \japhug{qame}{mole} and \japhug{qambɣo}{earwax} referring to physical defects or excretions from the body are APNs. A similar situation is observed in Koyukon Athabaskan, where nouns `denoting certain temporary or abnormal parts of the body' are also APN (\citealt[660]{thompson96koyukon}), though in Koyukon this subclass is considerably larger than in Japhug.

The compound \japhug{tɯcɯste}{amniotic sac} from \japhug{tɯ-ci}{water} and \japhug{tɤ-ste}{bladder} has a \forme{tɯ-} which is originally an indefinite possessive prefix (see \ref{sec:earth.IPN}), but which has become frozen after being integrated into a compound (\ref{sec:frozen.indef}), as can be shown by (\ref{ex:WtWciste}). 

\begin{exe}
\ex \label{ex:WtWciste}
\gll ɯ-tɯcɯste cʰɤ-ndʑɣaʁ \\
\textsc{3sg.poss}-amniotic.sac \textsc{ifr}-\textsc{anticaus}:squeeze.out \\
\glt `Her waters have broken.' (elicited)
\end{exe}


\subsubsection{Kinship terms} \label{sec:kinship}
The great majority of kinship terms select the indefinite possessor prefix \forme{tɤ-} or \forme{ta-} (see § XXX for a description of the kinship system). The only kinship terms in \forme{tɯ-} are \japhug{tɯ-me}{daughter} (but this form is not attested in the text corpus) and \japhug{tɯlɤt}{second sibling}; however, the \forme{tɯ-} prefix in the latter word has become non-analyzable and this word has become an UN (see \ref{sec:unpossessible.nouns}). 

There are other UNs among kinship terms, including \japhug{woɬaʁ}{(bad) stepmother}, which derives from \japhug{tɤ-ɬaʁ}{mother's sister} by replacing the possessive prefix with an unidentified element \forme{wo-}, and the social relation collectives (\ref{sec:social.collective}). Being a UN, \japhug{woɬaʁ}{(bad) stepmother} cannot take possessive prefixes, and the forms of \japhug{tɤ-ɬaʁ}{mother's sister} are used instead (\forme{a-ɬaʁ} can mean `my (bad) stepmother').

Kinship terms do not commonly occur with the indefinite possessive prefix. For those denoting spouses, forms with the indefinite prefix are found in the expression `look for a wife/husband', as in (\ref{ex:tArZaB.WkWCar}).
 
\begin{exe}
\ex \label{ex:tArZaB.WkWCar}
 \gll `ŋoj tɯ-ɕe?' to-ti, `aʑo tɤ-rʑaβ ɯ-kɯ-ɕar ɕe-a' to-ti. tɕe `ndʑiʑo ŋoj tɯ-ɕe-ndʑi?' to-ti ri, `tɕiʑo tɤ-nmaʁ ɯ-kɯ-ɕar ɕe-tɕi' to-ti. \\
 where 2-go:\textsc{fact} \textsc{ifr}-say \textsc{1sg} \textsc{indef}.\textsc{poss}-wife \textsc{3sg}.\textsc{poss}-\textsc{nmlz}:S/A-search go:\textsc{fact}-\textsc{1sg} \textsc{ifr}-say \textsc{lnk} \textsc{2du} where 2-go:\textsc{fact}-\textsc{du} \textsc{ifr}-say \textsc{lnk} \textsc{1du} \textsc{indef}.\textsc{poss}-husband  \textsc{3sg}.\textsc{poss}-\textsc{nmlz}:S/A-search go:\textsc{fact}-\textsc{1du} \textsc{ifr}-say  \\
 \glt `She said: `Where are you going?'; He said: `I am looking for a wife. Where are you going?'; She said `We are looking for a husband.'' (2003-kWBRa, 42-45)
\end{exe}

Kinship terms also occur with the indefinite possessive prefix to discuss about kinship in abstract terms, as in (\ref{ex:tArpW.tAftsa}) (see also \ref{ex:tArpW} below). Note that in this example the verb is in generic A form, with the inverse prefix (see § XXX) implying that the nouns \japhug{tɤ-rpɯ}{mother's brother} and \japhug{tɤ-ftsa}{sister's child} are in generic use (see § XXX; for the absence of ergative marker in this sentence, see § XXX). The possessive prefix cannot be the generic possessor prefix \forme{tɯ-} (\japhug{tɯ-rpɯ}{one's mother's brother} and \japhug{tɯ-ftsa}{one's sister's child}), since only one argument in a given sentence can be generic (§ XXX) and even if this were possible, the meaning would be completely different (`One's uncle cannot marry one's nephew').

\begin{exe}
\ex \label{ex:tArpW.tAftsa}
\gll tɤ-rpɯ cʰo tɤ-ftsa ni ci kú-wɣ-pa mɤ-kɯ-kʰɯ ɲɯ-ŋu. \\
\textsc{indef.poss}-MB \textsc{comit} \textsc{indef.poss}-ZC \textsc{du} one \textsc{ipfv}-\textsc{inv}-make \textsc{neg}-\textsc{inf}:\textsc{stat}-be.possible  \textsc{sens}-be \\
\glt `Maternal uncles and sister's children cannot marry each other.  (140427 kWmdza stWnmW, 14)
\end{exe}

Some kinship terms have an extended meaning when they take the indefinite possessor prefix: they can alternatively be used to denote a class of humans based on gender and age. The noun \japhug{tɤ-tɕɯ}{son} also commonly means `boy' or even `male human' (regardless of age). The nouns \japhug{tɤ-wa}{father} and \japhug{tɤ-mu}{mother} can denote older people without reference to their children; translations such as `old man' and `old lady' are more appropriate in these cases, for instance in (\ref{ex:tAmu.ci}). The same applies to  \japhug{tɤ-wɯ}{grandfather} and \japhug{tɤ-wi}{grandmother}.


\begin{exe}
\ex \label{ex:tAmu.ci}
\gll praʁkʰaŋ  zɯ tɤ-mu ci ɯ-ku tɤ-kɯ-wɣrum ci zɯŋzɯŋ pjɤ-rɤʑi tɕe, \\
cave \textsc{loc} \textsc{indef.poss}-mother \textsc{indef} \textsc{3sg}.\textsc{poss}-head \textsc{pfv}-\textsc{nmlz}:S/A-be.white \textsc{idph}:II:white \textsc{ifr}.\textsc{ipfv}-stay \textsc{lnk} \\
\glt `In the cave, there was an old woman whose hair was completely white.' (2003sras, 69)
\end{exe}

\subsubsection{Property nouns} \label{sec:property.nouns}
Property nouns are a subclass of IPN that designate a entity that possesses a particular (mainly derogative) characteristic. They generally follow another noun as in (\ref{ex:penzi.WpW}) and (\ref{ex:kha.WNqra}), but not exclusively (\ref{ex:Wxso.tWrme}). In the /noun+property noun/ phrase, the latter is the syntactic head but semantically modifies the former (see § XXX on the various property modifiers in Japhug).  

\begin{exe}
\ex \label{ex:penzi.WpW}
 \gll <penzi> ɯ-pɯ, sɤlaŋpʰɤn ɯ-pɯ jamar ɲɯ-wxti cʰa  \\
 basin \textsc{3sg.poss}-little.one   basin \textsc{3sg.poss}-little.one  about \textsc{ipfv}-be.big can:\textsc{fact} \\
 \glt `It can grow about as big as a little basin.' (18-NGolo, 48)
\end{exe}

\begin{exe}
\ex \label{ex:kha.WNqra}
 \gll
kʰa ɯ-ɴqra tɕe znde ɯ-mbe ma tʰam kɯ-tu me. \\
house \textsc{3sg.poss}-broken.one \textsc{lnk} wall \textsc{3sg.poss}-old.one apart.from now \textsc{nmlz}:S/A-exist not.exist:\textsc{fact} \\ 
\glt `Now there is nothing (there), apart from a ruin and old walls. (140522 tshupa, 58)
\end{exe}

These phrases can be turned into compounds made of the first noun and a quasi-suffix corresponding to the property noun, as all diminutive and derogative suffixes described in § \ref{sec:diminutive} and \ref{sec:derogative} (Table \ref{tab:property.nouns}) have corresponding property nouns. In the case of \forme{sɤlaŋpʰɤn ɯ-pɯ}  from example (\ref{ex:penzi.WpW}) for instance, it is possible to say \japhug{sɤlaŋpʰɤn-pɯ}{little basin} as one word. In some cases the corresponding noun has \textit{status constructus} on the first element, as in \japhug{kʰɤɴqra}{ruin} from \japhug{kʰa}{house} and \japhug{ɯ-ɴqra}{broken one}, a form which occurs in (\ref{ex:khANqra}), in the same text as  (\ref{ex:kha.WNqra}) (referring to the same house). The opposite however is not always possible; for instance, lexicalized diminutives like \japhug{staχpɯ}{pea} from \japhug{stoʁ}{broad bean} cannot be turned into a phrase with \japhug{ɯ-pɯ}{little one} as second element.

\begin{exe}
\ex \label{ex:khANqra}
 \gll tɕe nɯ tɤtsoʁsta nɯnɯ kʰɤɴqra ɕti tʰam tɕe kɯ-rɤʑi me \\
\textsc{lnk} \textsc{dem} place.name \textsc{dem} ruin be:\textsc{affirm}:fact  now \textsc{lnk} \textsc{nmlz}:S/A-stay not.exist:\textsc{fact} \\
\glt `Now Tatsogsta (`the place of silverweed') is a ruin, nobody lives there.' (140522 tshupa, 56)
\end{exe}

\begin{table}
\caption{Property nouns and corresponding quasi-suffixes} \label{tab:property.nouns}
\begin{tabular}{l|ll}
\lsptoprule
Property Noun & Suffix& \\
\midrule
\japhug{ɯ-pɯ}{little one} & \forme{-pɯ} &diminutive \\
\japhug{ɯ-ɴqra}{broken one} &  \forme{-ɴqra} &derogative \\
\japhug{ɯ-do}{old one} &  \forme{-do} & \\
\japhug{tɤ-mbe}{old thing} &  \forme{-mbe} & \\
\japhug{ɯ-rqɯ}{cold thing} &  \forme{-rqɯ} & other \\
\japhug{ɯ-xso}{empty, normal} & \\
\japhug{ɯ-maŋ}{in big groups} & \\
\lspbottomrule
\end{tabular}
\end{table}

The property nouns \japhug{ɯ-do}{old one}  and \japhug{tɤ-mbe}{old thing} differ in that the former one is used for living things (including animals and plants), while the second occurs with inanimate objects. The quasi-suffix \forme{-rqɯ} is mainly used in \japhug{tɯ-cirqɯ}{cold water}.



Property nouns are not commonly used with an indefinite possessor prefix; in attested examples, it is always \forme{tɤ-}. They origins are diverse: \japhug{ɯ-pɯ}{little one} derives from \japhug{tɤ-pɯ}{offspring, young} (see \ref{sec:diminutive}), while \japhug{tɤ-mbe}{old thing} and \japhug{ɯ-do}{old thing}  originate  from \japhug{mbe}{be old} and \japhug{do}{be old (of plants)} by deverbal derivation (section XXX).  The property  noun \japhug{ɯ-maŋ}{in big groups} derives from \japhug{maŋ}{be many}, itself from Tibetan \tibet{མང་}{maŋ}{many}. Some \forme{tɤ-} prefixed nouns of verbal origin like \japhug{tɤkʰe}{idiot, fool} (from \japhug{kʰe}{be stupid}) may come from former property nouns.
 
The property noun \japhug{ɯ-xso}{empty, normal} is related to the verb \japhug{so}{be empty}; it originally comes from its subject participle (the regular form \japhug{kɯ-so}{empty} is still attested) with loss of vowel and fricativization of the velar participle prefix (see § XXX). It had no corresponding quasi-suffix, but does appear as second element in some compounds  (see for instance \ref{sec:collective}).

The most common meaning of \forme{ɯ-xso} is `normal, usual, common', a meaning already very different from the base verb. It occurs both before and after the noun with which it is linked (compare \ref{ex:Wxso.tWrme} and \ref{ex:tWrme.Wxso}). It is also used adverbially meaning `usually' (\ref{ex:Wxso.kurAZi}).

\begin{exe}
\ex \label{ex:Wxso.tWrme}
\gll ɯ-pa ɲɯ-kɯ-ɕe nɯ tɕe kɯmaʁ tɯrme, ɯ-xso tɯrme ra nɯ-tɕʰaʁra pjɤ-ŋu \\
\textsc{3sg.poss}-down \textsc{ipfv}:\textsc{west}-\textsc{nmlz}:S/A-go \textsc{dem} \textsc{lnk} other people \textsc{3sg.poss}-normal people \textsc{pl} \textsc{3pl.poss}-toilet \textsc{ifr.ipfv}-be \\
\glt `The toilets for other people, for normal people (not lamas), were on the (balcony) oriented towards west under it.' (08-kWqhi, 10)
\end{exe} 

\begin{exe}
\ex \label{ex:tWrme.Wxso}
\gll   nɤʑo tɯrme ɯ-xso tɯ-maʁ \\
2sg people  \textsc{3sg.poss}-normal  2-not.be:\textsc{fact} \\
\glt `You are not a normal human.' (150829 taishan zhi zhu-zh, 40)
\end{exe} 

\begin{exe}
\ex \label{ex:Wxso.kurAZi}
\gll  ɯ-xso ku-rɤʑi tɕe,  ɯ-βri nɯnɯ scoʁ-pɯ pɯ-kɤ-βʁum ʑo fse  \\
\textsc{3sg.poss}-normal \textsc{ipfv}-stay \textsc{lnk} \textsc{3sg.poss}-body \textsc{dem} ladle-\textsc{dim} \textsc{pfv}:\textsc{down}-\textsc{nmlz}:P-cover \textsc{emph} be.like:\textsc{fact} \\
\glt `(The ladybug) usually stays (in one place), its body looks like a little laddle put upside down.' (26-kWlAGpopo, 3)
\end{exe} 

The meaning `empty' is however also attested; in (\ref{ex:nWxso.chAnWlhoRnW}) it is used adverbially, and note that the possessive prefix is coreferent with the plural intransitive subject.

\begin{exe}
\ex \label{ex:nWxso.chAnWlhoRnW}
\gll toʁde tɕe tɕendɤre, nɯ-xso chɤ-nɯ-ɬoʁ-nɯ. \\
a.moment \textsc{lnk} \textsc{lnk} \textsc{3pl.poss}-empty \textsc{ifr}:\textsc{downstream}-\textsc{auto}-come.out-\textsc{pl} \\
\glt `A moment later, they came out empty-handed.' (140512 alibaba-zh, 34)
\end{exe} 

\subsubsection{Alienabilization of IPN} \label{sec:alienabilization}
 It is possible to turn an IPN  into an APN one by adding a definite possessive prefix before the indefinite one; this is the only case of possessive prefix stacking in Japhug. This process is very productive, and better illustrated by minimal pairs; the following examples involve the IPNs \japhug{tɯ-ci}{water}, \japhug{tɤ-lu}{milk} and \japhug{tɤ-muj}{feather}.
 
The noun \japhug{tɯ-ci}{water} with a definite possessor (\japhug{ɯ-ci}{its juice/water}) refers either to the juice of a plant, or to water in which a plant has been soaked as in (\ref{ex:Wci})

  \begin{exe}
\ex \label{ex:Wci}
 \gll  ɯʑo tɯ-ci kɯ-sɤ-ɕke ɯ-ŋgɯ pjɯ́-wɣ-ɣɤ-la, tɕe nɯ ɣɯ ɯ-ci ɯ-ŋgɯ nɯtɕu tɯ-mi pjɯ́-wɣ-ɣɤ-la tɕe nɯnɯ, χtɕoŋ nɯ ɲɯ-pʰɤn ɲɯ-ti-nɯ ri, \\
\textsc{3sg}  \textsc{indef.poss}-water \textsc{nmlz}:S/A-\textsc{deexp}-burn \textsc{3sg}-in \textsc{ipfv}-\textsc{inv}-\textsc{caus}-soak \textsc{lnk} \textsc{dem} \textsc{gen} \textsc{3sg.poss}-water \textsc{3sg}-in \textsc{dem:loc}  \textsc{genr.poss}-foot  \textsc{ipfv}-\textsc{inv}-\textsc{caus}-soak \textsc{lnk} \textsc{dem}, rheumatism \textsc{dem} \textsc{sens}-be.efficient \textsc{sens}-say-\textsc{pl} \textsc{lnk}  \\
 \glt `One puts it in hot water, and then one puts one's feet in that water, and it efficient against rheumatism, they say.' (20-sWrna, 144)
 \end{exe}

The alienabilized form \japhug{ɯ-tɯ-ci}{its water}, as in (\ref{ex:WtWci}), is used to talk about water given to an animal to drink, or water (artificially irrigated or not) absorbed by a plant.

 \begin{exe}
\ex \label{ex:WtWci}
 \gll  tɕeri ɯ-tɯ-ci wuma ʑo na-ʁzi tɕe, ɯ-tɯ-ci nɯ mɯ-pjɯ-mbrɤt ɲɯ-ra. \\
 but \textsc{3sg.poss}-\textsc{indef.poss}-water really \textsc{emph} \textsc{trop}-need:\textsc{fact} \textsc{lnk} \textsc{3sg.poss}-\textsc{indef.poss}-water dem \textsc{neg-ipfv-anticaus}:break \textsc{sens}-have.to \\
 \glt  `But it needs water a lot, it needs to have water continuously.'  (07-Zmbri, 11)
 \end{exe}
 
 When a definite possessor is present on the noun \japhug{tɤ-lu}{milk} in a form such as \japhug{ɯ-lu}{her milk}, that prefix refers to the animal producing the milk, as in (\ref{ex:Wlu}).
 
 \begin{exe}
\ex \label{ex:Wlu}
 \gll 
tɤ-pi kɯ-wxti nɯ kɯ nɯŋa ɣɯ ɯ-lu nɯ cʰondɤre  ɯ-ɕa nɯ to-nɯ-ndo. \\
\textsc{indef.poss}-elder.sibling \textsc{nmlz}:S/A-be.big \textsc{dem} \textsc{erg} cow \textsc{gen} \textsc{3sg.poss}-milk \textsc{dem} \textsc{comit} \textsc{3sg.poss}-meat \textsc{dem} \textsc{ifr}-\textsc{auto}-take \\
\glt `The elder brother took the cow's milk and meat.' (02-deluge2012, 19)
 \end{exe}

The form \japhug{ɯ-tɤ-lu}{his/its milk} with alienabilization is used on the other hand when indicating the person or animal drinking the milk, as in (\ref{ex:WtAlu}).

  \begin{exe}
\ex \label{ex:WtAlu}
 \gll tɕe ɯ-tɤ-lu pjɯ́-wɣ-rku tɕe nɯnɯ pjɯ-tsʰi qʰe, tɯ-sŋi tɕe tɯ-kʰɯtsa jamar tɯ-rdoʁ kɯ pjɯ-tsʰi ɲɯ-cʰa. \\
\textsc{lnk} \textsc{3sg.poss}-\textsc{indef.poss}-milk \textsc{ipfv}:\textsc{down}-\textsc{inv}-put.in  \textsc{lnk} \textsc{dem} \textsc{ipfv}-drink \textsc{lnk} one-day \textsc{lnk} one-bowl about one-\textsc{cl} \textsc{erg} \textsc{ipfv}-drink \textsc{sens}-can \\
\glt `People pour drink for it (the cat) to drink, and it drinks it, one (cat) can drink about a bowl of milk per day.' (21-lWLU, 45)
  \end{exe}

The IPN \japhug{tɤ-muj}{feather} takes as its possessor a bird (or a bird body part such as `wings'), as in (\ref{ex:Wmuj}).

    \begin{exe}
\ex \label{ex:Wmuj}
 \gll   jinde tɕe ɯ-kɯ-sat koŋla maŋe tɕe, nɯ qarma ɯ-muj kɯnɤ tɯ-jaʁ mɯ́j-ɣi wo \\
 nowadays \textsc{lnk} \textsc{3sg.poss}-\textsc{nmlz}:S/A-kill completely not.exist:C \textsc{lnk} \textsc{dem} crossoptilon \textsc{3sg.poss}-feather also \textsc{genr.poss}-hand \textsc{neg}:\textsc{sens}-come \textsc{sfp} \\
 \glt `Nowadays, nobody kills them, and one cannot get crossoptilon feathers.' (23-qapGAmtWmtW, 173)
  \end{exe}

Its alienabilized form, such as \japhug{ɯ-tɤ-muj}{his feather} in (\ref{ex:WtAmuj}), is used when the feather is detached from the body of the bird, and belongs to a human.
  
\begin{exe}
\ex \label{ex:WtAmuj}
 \gll tɤtɕɯpɯ kɯ-xtɕi nɯ ɣɯ ɯ-tɤ-muj nɯ li ɯ-tʰoʁ nɯtɕu pjɤ-nɯ-jɣɤt  \\
 boy:\textsc{dim} \textsc{nmlz}:S/A-be.small dem gen \textsc{3sg.poss}-\textsc{indef.poss}-feather \textsc{dem} again \textsc{3sg.poss}-ground \textsc{dem:loc} \textsc{ifr}:\textsc{down}-\textsc{auto}-go.back  \\
 \glt `The younger boy's feather fell back on the ground again.' (140510 sanpian yumao, 68)
\end{exe}
 
As the examples above show, alienabilized IPNs occur to refer to disconnected or severed body parts, for instance body parts removed from an animal that are used or owned by a human or another animal on which they do not grow. They are also used for bodily fluids that have left the body, or also clothes that are not worn but held in the hand. Similar phenomena are observed in other Gyalrong languages (see \citealt[140]{jackson98morphology} on Tshobdun).

The referent marked by the possessive prefix can be beneficiary as in (\ref{ex:WtAlu}) or possessor as in (\ref{ex:WtAmuj}).
 
Alienabilization is also observed with prenominal modifiers (§ \ref{sec:possessive.prefixes.prenominal}), in compounding, when the indefinite possessive prefix of an IPN is preserved in the final member of the compound (see § \ref{sec:possessive.prefix.second.compounds}), in comitative adverbs derived from IPNs (\ref{sec:comitative.adverb}) and in conversion from IPN to counted noun (§ \ref{sec:CN.IPN}). A related phenomenon is also the optional neutralization of possessive prefixes in relative clauses (§ XXX).
 
\subsubsection{Frozen indefinite possessors} \label{sec:frozen.indef}
APNs with a disyllabic root whose first element is \forme{tɯ-} or \forme{tɤ-}, with the exception of loanwords such as \japhug{tɯrsa}{grave} (from \tibet{དུར་ས་}{dur.sa}{grave}) or counted nouns (see § XXX), are mainly ancient IPNs whose indefinite possessive prefix \forme{tɯ-} has become frozen and reanalyzed as part of the root. Comparison with other Gyalrong languages can demonstrate that such reanalysis took place in Japhug.

For instance, the noun \japhug{tɯrme}{man} is APN in Japhug, but in Situ the 3\sg{} form of \forme{tə-rmî} `man' is \forme{wǝ-rmî}  (\citealt[183;197]{lin09phd}), showing that \forme{tə-} is the indefinite possessive prefix, cognate of Japhug \forme{tɯ-}. This shift may be due to the fact that the 3\sg{} form is used in Situ in constructions where the non-possessed form is preferred in Japhug, such as in prenominal relatives (\citealt[190]{lin09phd}), and was therefore less prone to lexicalization. The stem \forme{-rme} of  \japhug{tɯrme}{man} is still attested as second element of compounds like \japhug{tɯ-pɤrme}{one year of life} (with \japhug{tɯ-xpa}{one year} as first element).  

The noun \japhug{tɤjmɤɣ}{mushroom} is APN, but the stem \forme{jmɤɣ-} appears as first element of compounds such as \japhug{jmɤɣni}{russula}, suggesting that it is a former IPN occurring without its indefinite possessor prefix in this compound (see \ref{sec:loss.possessive.prefix.compounds}). The status of \japhug{tɤjmɤɣ}{mushroom} as a former IPN is less surprising if one takes into account the likely etymological relationship with Chinese \zh{帽} \forme{mawH} `hat' (from \forme{*mˤuk-s}; etymology suggested by L. Sagart; see also the Tibetan cognate \tibet{རྨོག་}{rmog}{helmet}).  If the noun for `mushroom' in Japhug and other Gyalrongic languages comes from `hat' (cf Breton \forme{tog touseg} `toad hat' for `mushroom'), it is expected that it would become an IPN (like \japhug{tɤ-rte}{hat}), and for the indefinite possessor \forme{tɤ-} to become frozen after the noun ceases to be a term for head covers.
 
\subsubsection{Unusual IPNs in Japhug} \label{sec:earth.IPN}
While IPN membership of nouns such as body parts of kinship terms is expected from a crosslinguistic point of view, Japhug has IPNs for nouns denoting natural entities such as \japhug{tɯ-ci}{water}, \japhug{tɯ-mɯ}{sky} or \japhug{ɯ-tʰoʁ}{ground}, a highly unusual fact. There is no grand insight about Gyalrong Weltanschauung to be gained from this observation however; explanations should be sought in the etymology of these words, and solved on an item per item basis.  

The noun \japhug{tɯ-ci}{water} also means `juice' or `water in which X has been plunged into' with a definite possessor, as was seen in § \ref{sec:alienabilization}. Cognates are found in Core Gyalrong languages, but not in West Gyalrongic (Stau \forme{ɣrə} and Wobzi Khroskyabs \forme{jdə̂}, \citealt[610]{jacques17stau}) or elsewhere in the family, and it is therefore a good candidate for a Core Gyalrong lexical innovation. 

Japhug has a transitive verb \japhug{ci}{pour completely (of grains or liquids)}, from which a IPN \forme{*ɯ-ci} `(liquid/grain) that has been poured out' could have been regularly derived (see § XXX; similar to \japhug{ɯ-ndzɯ}{instruction, advice} from \japhug{ndzɯ}{educate} in example \ref{ex:WmW.mbWt} below). The meaning `water' would then be trivial narrowing of the meaning of this noun `water poured out', then replacing the older term for `water' still preserved in West Rgyalrongic. In this view, the status of \japhug{tɯ-ci}{water}  as an IPN is a consequence of its morphological derivation from a transitive verb (as there are no unprefixed deverbal nouns in Japhug, see § XXX). The stative verb \japhug{aci}{be wet} is then derived, after the semantic change, from the noun  \japhug{tɯ-ci}{water} by denominal derivation (section XXX) -- despite superficially looking like a passive of \japhug{ci}{pour completely (of grains or liquids)} (section XXX), it is only indirectly deriving from it. In addition to this verb, many derived forms have been created from the noun  \japhug{tɯ-ci}{water}, including the body part \japhug{tɯ-mci}{saliva} (\ref{ex:body.part.prefix}): these words show that the morphological formation by which they were constructed was still productive after the breakup of proto-Gyalrongic.


Concerning \japhug{tɯ-mɯ}{sky}, it superficially resembles a noun with non-analyzable \forme{tɯ-} prefixal element, but the status of this element as an indefinite possessor prefix can be ascertained with rare examples such as (\ref{ex:WmW.mbWt}).

\begin{exe}
\ex  \label{ex:WmW.mbWt}
\gll  ɯ-ndzɯ mɤ-kɯ-sɤŋo ɯ-mɯ mbɯt  \\
\textsc{3sg.poss}-instruction \textsc{neg}-\textsc{nmlz}-S/A-listen \textsc{3sg.poss}-sky \textsc{anticaus}:take.off:\textsc{fact} \\
\glt `Those who do not listen to advice from other people do not end well (their sky falls).' (elicited)
\end{exe}
There is no clear explanation of how this noun come have become IPN, but I propose here a tentative etymology. Cognates of \japhug{tɯ-mɯ}{sky} are attested elsewhere in Trans-Himalayan language, but mainly in languages that poorly preserve presyllables (for instance Yongning Na \forme{mv̩˥°}, \citealt[132]{michaud17yongning}). Yet, in Rawang, among the conservative languages that preserve presyllables, has a word \forme{dvmø̀}  `celestial being'  (\citealt[13]{lapolla01rawang}), with the same vowel correspondence to Japhug \ipa{-ɯ} as \forme{sharø}  `bone' with \japhug{ɕɤrɯ}{bone}. If  \forme{dvmø̀} is indeed cognate of Japhug \japhug{tɯ-mɯ}{sky},\footnote{Another potential cognate could be Rawang  \forme{muq} `sky, thunder', but the final glottal stop transcribed \forme{-q} is from a former \forme{*-k}, and this word is better compared to Situ \forme{ta-rmōk} `thunder' (\citealt[73]{zhang16bragdbar})} the word may originally have been disyllabic, and its first syllable reinterpreted as indefinite possessive; the form \japhug{ɯ-mɯ}{his sky} in (\ref{ex:WmW.mbWt}) would then be a backformation, an idea compatible with its very marginal character.

 
The Japhug noun \japhug{ɯ-tʰoʁ}{ground} cannot take any possessive prefix other than 3\sg{} \forme{ɯ-}, not even the indefinite possessor prefix. It has no known cognates in other Gyalrongic languages, but is a perfect match for a Tibetan word with the shape \forme{tʰog} (compare the other borrowed noun \japhug{tʰoʁ}{thunder} from Tibetan \tibet{ཐོག་}{tʰog}{thunder}). The etymology of this word requires a four steps scenario.

First, Japhug borrowed the Tibetan relator noun  \tibet{ཐོག་ཏུ་}{tʰog(tu)}{on} as \forme{ɯ-tʰoʁ} *`on' (not attested), adding a third person possessive prefix like all relator nouns (see § XXX). This relator noun was in competition with the existing native equivalent \japhug{ɯ-taʁ}{on}.\footnote{It is not surprising in Japhug to have several competing relator noun for the same functional slot; the same is true of the dative \ipa{ɯ-ɕki} and \ipa{ɯ-pʰe}, see section XXX. }
  
  Second, it  became restricted to the collocation \forme{*sɤtɕʰa ɯ-tʰoʁ zɯ} `on the ground' (not attested), with the native locative \forme{zɯ} and the  noun of Tibetan origin \japhug{sɤtɕʰa}{earth}.
  
    Third, the collocation  \forme{*sɤtɕʰa ɯ-tʰoʁ zɯ}`on the ground', becoming tautological, was reduced to \japhug{ɯ-tʰoʁ zɯ}{on the ground} (attested).
 
 Fourth, the noun \japhug{ɯ-tʰoʁ}{ground} was created by backformation from the locative phrase \forme{ɯ-tʰoʁ zɯ} `on the ground'. The fact that the locative postposition \ipa{zɯ} is always optional (section XXX) undoubtedly made this step easier.  Thus, Japhug attests an example of degrammation (see \citealt[135]{norde09degrammaticalization}) from a relator noun meaning `on' (with or without motion) to a common noun meaning `ground'. 

The three etymologies discussed above show that natural phenomena IPNs in Japhug were created by unrelated pathways.


\subsubsection{Biactantial IPNs}
A few IPNs, such as words designating speech or presents, select more than one argument, and can be considered to be the nominal equivalent of ditransitive verbs (if APNs and IPNs are compared to intransitive and transitive verbs respectively, see § \ref{sec:inalienably.possessed}). Since only one argument however is marked by a possessive prefix on the noun (possessive prefix stacking is not possible except for alienabilization, see \ref{sec:alienabilization}) a choice has to be made as which of the two arguments, the speaker/giver or the addressee/recipient, is marked on the noun.

The possessive prefix of IPN \japhug{tɤ-pɤro}{present} always marks the giver; the recipient of the present (which is optional) receives genitive case. 

For instance, in (\ref{ex:apAro}) the recipient is marked by the genitive pronoun \forme{nɤʑɯɣ}, and the possessive prefix on the noun is \textsc{1sg}, the subject of the main verb. In (\ref{ex:nApAro}), the recipient is not overt, and the \textsc{2sg} prefix on the noun again corresponds to the transitive subject of \japhug{ɣɯt}{bring}.

\begin{exe}
\ex \label{ex:apAro}
\gll nɤʑɯɣ a-pɤro tɕʰi ju-ɣɯt-a ra? \\
\textsc{2sg:gen} \textsc{1sg}.\textsc{poss}-present what \textsc{ipfv}-bring-\textsc{1sg} have.to:\textsc{fact} \\
\glt `What present should I bring for you?' (140504 huiguniang-zh, 28)
\end{exe}

\begin{exe}
\ex \label{ex:nApAro}
\gll nɤʑo dɯxpa pɯ-tɯ-tu ma li nɤ-pɤro jɤ-tɯ-ɣɯt! \\
\textsc{2sg} hardship \textsc{pst}.\textsc{ipfv}-2-exist \textsc{lnk} again \textsc{2sg}.\textsc{poss}-present \textsc{pfv}-2-bring \\
\glt `Thank you, you brought another present (for me).' (elicited)
\end{exe}

Other biactantial nouns use possessive prefixes to indicate the recipient rather than the agent. From instance, the IPN \japhug{tɤ-rkuz}{parting present} always marks the recipient, as in (\ref{ex:nArkuz}), never the agent -- the form \japhug{nɤ-rkuz}{your parting present} with \textsc{2sg} possessive prefix can only mean `a parting present for you', not `the parting present you give to me/him'.

\begin{exe}
\ex \label{ex:nArkuz}
\gll  kɯki nɤ-rkuz ŋu \\
\textsc{dem}:\textsc{prox} \textsc{2sg}.\textsc{poss}-parting.present be:\textsc{fact} \\
\glt `This is a parting present for you.' (28-smAnmi, 266)
\end{exe}

In other cases, the alignment of possessive prefixes on an IPN depends on the  particular construction where it appears. For instance, the ICN \japhug{tɯ-tɕʰa}{information, news} marks the recipient when used with the verbs \japhug{kʰo}{give} or \japhug{tu}{exist}, but has neutral alignment in other contexts. In (\ref{ex:atCha.mWnakho}), the indirective verb \japhug{kʰo}{give} (§ XXX) does not index the recipient, whose only mark is the possessive prefix on \japhug{a-tɕʰa}{news for me}. Changing the prefix to the third singular \forme{ɯ-} to refer to the subject here would be agrammatical (see however § XXX on hybrid indirect speech).


\begin{exe}
\ex \label{ex:atCha.mWnakho}
\gll  a-tɕɯ kɯ a-tɕʰa mɯ-na-kʰo. \\
\textsc{1sg}.\textsc{poss}-son erg \textsc{1sg}.\textsc{poss}-news \textsc{neg}-\textsc{pfv}:3\fl{}3'-give \\
\glt `I have not heard from my son (My son did not give me any response).' 
\end{exe}

However, in other constructions, for instance with the verb \japhug{ɣɯt}{bring}, there are no such constraints on the use of possessive prefixes on \japhug{tɯ-tɕʰa}{information, news}: for instance, in (\ref{ex:WtCha.pjWGWta}), although the recipient is second person dual, \forme{ɯ-tɕʰa} takes the \textsc{3sg} prefix, coreferent with the preceding complement clause (using second person singular \forme{nɤ-tɕʰa} here would be agrammatical).

\begin{exe}
\ex \label{ex:WtCha.pjWGWta}
\gll `ma-nɯ-tɯ-ɣɤwu-ndʑi tɕe aʑo tu-ɕe-a tɕe atu ɕ-tu-tʰe-a tɕe,
ndʑi-pa ndʑi-ma ni ɯ-ɲɯ-nɯjʁo-ndʑi kɯ' ɯ-tɕʰa pjɯ-ɣɯt-a \\
\textsc{neg}-\textsc{imp}-2-cry-\textsc{du} \textsc{lnk} \textsc{1sg} \textsc{ipfv}:\textsc{up}-go-\textsc{1sg} \textsc{lnk} up \textsc{transloc}-\textsc{ipfv}-ask[III]-\textsc{1sg} \textsc{lnk} \textsc{2du}.\textsc{poss}-father \textsc{2du}.\textsc{poss}-mother \textsc{du} \textsc{qu}-\textsc{ipfv}-scold-\textsc{du} \textsc{qu} \textsc{3sg}.\textsc{poss}-information \textsc{ipfv}:\textsc{down}-bring-\textsc{1sg} \\
\glt `Don't cry, I will go up there, ask whether your parents will scold you' and come back to tell you.' (2003-kWBRa, 15)
\end{exe}

Alignment effects are also found with APN. For instance, the APN \japhug{skɯrma}{present} can also optionally take a possessive prefix, which is always coreferent with the recipient, not with the agent, as shown by (\ref{ex:askWrma}), where \forme{nɤ-mu ɣɯ ɯ-skɯrma} means `a present (sent) to your mother' and \forme{a-skɯrma} can only mean `a present sent to me' (from a text explaining the meaning difference between \forme{skɯrma}, \forme{tɤ-pɤro} and \forme{tɤ-rkuz}).\footnote{Japhug has three words that can be translated as `present': \forme{tɤ-pɤro} is used for presents one hands in to the recipient in person, \forme{tɤ-rkuz} is a parting present one gives before a person leaves a place, and \forme{skɯrma} is a present that one transmits to the recipient through the help of another person.}

\begin{exe}
\ex \label{ex:askWrma}
\gll nɤ-mu ɣɯ tʰɯci tu-rke-a tɕe ju-tɯ-tsɯm tɕe nɯnɯ nɤ-mu ɣɯ ɯ-skɯrma ju-sɯ-ɣɯt-a ŋu nɤ-mu kɯ a-tɤ-rke tɕe, a-jɤ-tɯ-ɣɯt tɕe, tɕe nɯnɯ li a-skɯrma jɤ-kɤ-sɯ-ɣɯt ŋu \\
\textsc{2sg}.\textsc{poss}-mother \textsc{gen} something \textsc{ipfv}-put.in[III]-\textsc{1sg} \textsc{lnk} \textsc{ipfv}-2-take.away \textsc{lnk} \textsc{dem} \textsc{2sg}.\textsc{poss}-mother \textsc{gen} \textsc{3sg}.\textsc{poss}-present  \textsc{ipfv}-\textsc{caus}-bring-\textsc{1sg} be:\textsc{fact} \textsc{2sg}.\textsc{poss}-mother  \textsc{erg} \textsc{irr}-\textsc{pfv}-put.in[III] \textsc{lnk} \textsc{irr}-\textsc{pfv}-2-bring \textsc{lnk} \textsc{lnk} \textsc{dem} again \textsc{1sg}.\textsc{poss}-present \textsc{pfv}-\textsc{nmlz}:P-\textsc{caus}-bring be:\textsc{fact} \\
\glt `When I prepare something for your mother and you take it to her, (I can say) `I sent a present to you mother', if your mother prepares something and you bring it (to me), it is `a present sent to me.'' (def-skWrma, 18-19)
\end{exe}

\subsection{Indefinite vs generic possessor} \label{sec:indef.genr.poss}
The generic possessive prefix \forme{tɯ-} is formally identical to the indefinite possessor prefix of some IPNs, but must be strictly distinguished from it. Four criteria can be used to determine if a \forme{tɯ-} prefix is generic, rather than indefinite.

First, the generic possessor prefix appears on APNs, as in example (\ref{ex:tWlaXtCha}) with \japhug{tɯ-kʰa}{one's house} and \japhug{tɯ-laχtɕʰa}{one's things}, the generic forms of \japhug{kʰa}{house} and \japhug{laχtɕʰa}{thing}.

\begin{exe}
\ex \label{ex:tWlaXtCha}
\gll tɕe  	aʁɤndɯndɤt  	ʑo  	ku-zo  	qhe  	ɯ-qe  	ku-lɤt  	qʰe	wuma  ʑo  	tɯ-kʰa  	cʰo  	tɯ-laχtɕʰa  	ra  	sɯ-ɴqʰi.  \\
\textsc{lnk} everywhere \textsc{emph} \textsc{ipfv}-land \textsc{lnk} \textsc{3sg.poss}-feces \textsc{ipfv}-throw \textsc{lnk} really \textsc{emph} \textsc{genr.poss}-house \textsc{comit} \textsc{genr.poss}-thing \textsc{pl} \textsc{caus}-be.dirty:\textsc{fact} \\
\glt `(Flies) land everywhere, shit on it and make one's houses and things dirty.' (25 akWzgumba, 59)
\end{exe}

Second, the generic \forme{tɯ-} occurs on IPNs that normally select the \forme{tɤ-} indefinite possessive prefix, such as \japhug{tɯ-rɟit}{one's child} and \japhug{tɯ-rpɯ}{one's maternal uncle} in examples (\ref{ex:tWrJit}) and (\ref{ex:tWrpW}), by contrast with the citation forms  \japhug{tɤ-rɟit}{child} and \japhug{tɤ-rpɯ}{maternal uncle}.

\begin{exe}
\ex \label{ex:tWrJit}
\gll nɯ 	kɯ-fse 	tɕe 	tɯʑo 	tɯ-rɟit 	kɯnɤ 	ʑa 	mɤ-sci 	tu-ti-nɯ \\
\textsc{dem} \textsc{nmlz}:S-be.like \textsc{lnk} \textsc{genr} \textsc{genr.poss}-child also early \textsc{neg-fact}:be.born \textsc{ipfv}-say-\textsc{pl} \\
\glt `People say that in this way, one's child will be born late.' (27 qartshaz, 111)
\end{exe}

\begin{exe}
\ex \label{ex:tWrpW}
\gll  tɯ-rpɯ 	ɯ-rɟit 	ɯ-ɕki 	tɕe 	tɕe 	``a-rpɯ a-ɬaʁ" 	tu-kɯ-ti 	ŋu. \\
\textsc{genr.poss}-uncle \textsc{3sg.poss}-offspring \textsc{3sg-dat} \textsc{lnk} \textsc{lnk} \textsc{1sg.poss}-uncle \textsc{1sg.poss}-aunt \textsc{ipfv-genr}-say  be:\textsc{fact} \\
\glt `One has to say `my maternal uncle, my maternal aunt to one's maternal uncle's sons and daughters.' (140425 kWmdza01, 69)
\end{exe}

The used of the generic possessive \japhug{tɯ-rpɯ}{one's maternal uncle} in (\ref{ex:tWrpW}) can be contrasted with the indefinite possessed form with \forme{tɤ-} in example (\ref{ex:tArpW}).

\begin{exe}
\ex  \label{ex:tArpW}
\gll
nɤʑo 	tɤ-rpɯ 	ɯ-rɟit 	a-pɯ-tɯ-ŋu, 	tɕe 	tɕe 	aʑo 	kɯ 	`a-rpɯ' 	tu-ti-a 	kɯ-ra.  \\
\textsc{2sg} \textsc{indef.poss}-uncle \textsc{3sg.poss}-offspring \textsc{irr-ipfv}-2-be \textsc{lnk} \textsc{lnk} \textsc{1sg} \textsc{erg}  \textsc{1sg.poss}-uncle \textsc{ipfv}-say-\textsc{1sg} \textsc{nmlz:S/A}-have.to  \\
\glt `If you are the maternal uncle's son, (and I am the nephew) I have to say `my uncle' (to you).'  (hist140425 kWmdza, 114)
\end{exe}

Third, in the case of IPNs whose indefinite possessive is \forme{tɯ-}, such as \japhug{tɯ-mtɕʰi}{mouth}, the indefinite and generic forms are homophonous, but are nevertheless distinguishable. In the case of a generic form the generic pronoun \japhug{tɯʑo}{one} (\ref{sec:genr.pro}) can always be added as in (\ref{ex:tWZo.tWmtChi}). 

\begin{exe}
\ex  \label{ex:tWZo.tWmtChi}
\gll tɯʑo sɤz kɯ-mna, kɯ-ɤʑɯχtso ra a-pɯ-ŋu nɤ,  tɯʑo tɯ-mtɕʰi maŋtaʁ nɯtɕu ɲɯ-ɬoʁ ŋu. \\
\textsc{genr} \textsc{comp} \textsc{nmlz}:S/A-be.better \textsc{nmlz}:S/A-be.clean \textsc{pl} \textsc{irr}-\textsc{ipfv}-be \textsc{lnk}  \textsc{genr} \textsc{genr.poss}-mouth above \textsc{dem:pl} \textsc{ipfv}-come.out be:\textsc{fact} \\
\glt `If (one uses the bowl of) someone who is cleaner than oneself, the (pimple) will appear above one's mouth.' (25-khArWm, 11)
\end{exe}

Additionally, a generic noun such as \japhug{tɯrme}{person} can occur as possessor of a noun with a generic possessive prefix as in (\ref{ex:tWrme.tWCa}). This usage is similar to that found in other generic constructions (see § XXX).

\begin{exe}
\ex  \label{ex:tWrme.tWCa}
\gll tɯrme ɣɯ tɯ-ɕa ɯ-mdoʁ tsa asɯ-ndo kɯ-fse \\
people \textsc{gen} \textsc{genr.poss}-flesh \textsc{3sg.poss}-colour a.little \textsc{prog}-take:\textsc{fact} \textsc{nmlz}:S/A-be.like \\
\glt `It has a little the colour of human flesh.' (14-sWNgWJu, 97)
\end{exe}

Even when the generic pronoun or a generic noun is not present, it is possible to identify generic possessors, as they are coreferent with the generic argument indexed on the verb (by \forme{kɯ-} for S and P, and \forme{wɣɯ-} for A, see § XXX). For instance, in (\ref{ex:genr.tWmtChi}), we know that  the \forme{tɯ-} prefixes in \japhug{tɯ-mtɕʰi}{one's mouth} and  \japhug{tɯ-ɕɣa}{one's teeth} are generic and not indefinite possessor because they refer to the same generic human as the transitive subject of the verbs \japhug{pʰɯt}{take out} and \japhug{ndza}{eat} in the previous clause, marked by the inverse prefix.

\begin{exe}
\ex  \label{ex:genr.tWmtChi}
\gll
tɕe ɲɯ́-wɣ-pʰɯt tɕe tú-wɣ-ndza ŋgrɤl ri, wuma ʑo tɯ-mtɕʰi cʰo tɯ-ɕɣa ra ɲɯ-sɯɣ-ɲaʁ ŋu. \\
\textsc{lnk} \textsc{ipfv}-\textsc{inv}-take.out \textsc{lnk} \textsc{ipfv}-\textsc{inv}-eat be.usually.the.case:\textsc{fact} but really \textsc{emph}  \textsc{genr.poss}-mouth \textsc{comit} \textsc{genr.poss}-tooth \textsc{pl}  \textsc{ipfv}-\textsc{caus}-be.black be:\textsc{fact} \\
\glt `One can pluck it and eat it, but it causes one's mouth and teeth to become black.' (11-qarGW, 70) 
\end{exe}

Fourth, possessed case markers such as the dative \forme{-ɕki} (section XXX) do not have indefinite possessive forms, and therefore if prefixed in \forme{tɯ-}, it will always mark a generic possessor, as in (\ref{ex:tWCki}) -- such forms are often preceded by the generic pronoun \japhug{tɯʑo}{one} anyway.

\begin{exe}
\ex  \label{ex:tWCki}
\gll ma tɯ-ɕki wuma ʑo ʑɣɤ-sɯ-ɤrmbat tɕe núndʐa kʰe tu-ti-nɯ ɲɯ-ŋu. \\
\textsc{lnk} \textsc{genr-dat} really \textsc{emph} \textsc{refl}-\textsc{caus}-be.near:\textsc{fact} \textsc{lnk} for.this.reason stupid:\textsc{fact} \textsc{ipfv}-say-\textsc{pl} \textsc{sens}-be \\
\glt `It easily comes near oneself, so people call it `stupid'.' (23-scuz, 62) 
\end{exe}

\subsubsection{Comparative perspectives} \label{sec:indef.t.comparative}
Indefinite and generic possessive dental stop prefixes are found in all Gyalrong languages (\citealt{jackson98morphology}), but only indirect traces exist in Khroskyabs  (\citealt[155]{lai17khroskyabs}). 

Outside of Gyalrongic, potential cognates of these prefixes include the `relational prefix' \forme{tə-} in Ao (\citealt[84-85]{coupe07mongsen}, as first noticed by \citealt[141-2]{wolfenden29outlines}) and some \forme{d-} or \forme{g-} prefixes in body parts in Tibetan (see \citealt{jacques14snom}).

\subsection{Prenominal modifiers} \label{sec:possessive.prefixes.prenominal}
In noun clauses with prenominal modifiers, possessive prefixes on the head noun are generally \textsc{3sg}, and express a possessive relation between the nouns, as in (\ref{ex:XsAr.WloR}). This type of noun complement is studied in more detail in § XXX.

\begin{exe}
\ex \label{ex:XsAr.WloR}
\gll kɯstʰɯci ʑo pɣɤtɕɯ kɯ-mpɕɤr nɯ χsɤr ɯ-loʁ ɯ-ŋgɯ tu-rke-a ndɤre, tɕʰi nɯ mɤ-nɯ-fse? \\
such \textsc{emph} bird \textsc{nmlz}:S/A-be.beautiful \textsc{dem} gold \textsc{3sg}.\textsc{poss}-nest \textsc{3sg}.\textsc{poss}-inside \textsc{ipfv}-put.in[III]-\textsc{1sg} \textsc{lnk} what \textsc{dem} \textsc{neg}-\textsc{auto}-be.like:\textsc{fact} \\
\glt `Such a beautiful bird, what wrong could there be to put it in the golden nest?' (2012-qachGa, 41)
\end{exe}

If the head noun is an IPN, it can also undergo alienabilization and the possessive prefix is changed to the indefinite possessor (§ \ref{sec:alienabilization}). Thus in (\ref{ex:XsAr.tAsno}) we find \forme{χsɤr tɤ-sno} `golden saddle' with the indefinite possessor prefix \forme{tɤ-}; \forme{χsɤr ɯ-sno} with the \textsc{3sg} possessive prefix is also attested in the same text.

\begin{exe}
\ex \label{ex:XsAr.tAsno}
\gll si tɤ-sno nɯ tʰa-sɤndu nɤ, χsɤr tɤ-sno tʰa-nɯ-ta ɲɯ-ŋu, \\
wood \textsc{indef}.\textsc{poss}-saddle \textsc{pfv}:3\fl{}3'-exchange \textsc{lnk} gold  \textsc{indef}.\textsc{poss}-saddle \textsc{pfv}:3\fl{}3'-put \textsc{sens}-be \\
\glt `He exchanged the wooden saddle for the golden one.' (2003qachga,111)
\end{exe}

When the whole modifier+head noun complex is possessed however, the possessor is rarely marked by a possessive prefix on the head noun; rather, the prefix occurs on the leftmost noun of the phrase, as in (\ref{ex:aXsAr.tArte}), where the \textsc{1sg} prefix \forme{a-} occurs on the modifier \japhug{χsɤr}{gold}, and alienabilization of the head noun \japhug{tɤ-rte}{hat} (compare with the form \japhug{a-rte}{my hat} when no prenominal modifier is present). %A noun phrase such as $\dagger$\forme{χsɤr a-rte} is incorrect.

\begin{exe}
\ex \label{ex:aXsAr.tArte}
\gll a-rte, a-χsɤr tɤ-rte ra kɯnɤ nɤʑɯɣ ɲɯ-kʰam-a jɤɣ \\
\textsc{1sg}.\textsc{poss}-hat \textsc{1sg}.\textsc{poss}-gold \textsc{indef}.\textsc{poss}-hat \textsc{pl} also \textsc{2sg:gen} \textsc{ipfv}-give[III]-\textsc{1sg} be.possible:\textsc{fact} \\
\glt `I will even give you my hat, my golden hat.' (140429 qingwa wangzi-zh, 54)
\end{exe}

When the head noun is an APN, it is not usual either to strand the modifier and the following noun by putting a possessive prefix on the latter. The possessor is normally indicated by a possessive prefix on the leftmost word. For instance in (\ref{ex:nAXsAr.khWtsa}) the \textsc{2sg} prefix \forme{nɤ-} occurs on the modifier \japhug{χsɤr}{gold}.

\begin{exe}
\ex \label{ex:nAXsAr.khWtsa}
\gll nɤ-χsɤr kʰɯtsa nɯra ku-kɯ-sɯ-ntɕʰoz-a \\
\textsc{2sg}.\textsc{poss}-gold bowl \textsc{dem}:\textsc{pl} \textsc{ipfv}-2\fl{}1-\textsc{caus}-use-\textsc{1sg} \\
\glt `Let me use your golden bowl.' (140429 qingwa wangzi-zh, 135)
\end{exe}

However, we do find cases with a stranded NP modifier when the possessor on the head noun is first or second person, as in (\ref{ex:XsAr.akWmtChW}), where the prenominal modifier \japhug{χsɤr}{gold} does appear before the possessive prefix \forme{a-}.\footnote{The alternative form \forme{a-χsɤr kɯmtɕʰɯ} is possible to express the same meaning, and does occur in the same text. } This construction, though rarer, is not considered to be incorrect by native speakers. Other examples are found with UPN modifiers (§ \ref{sec:place.names}).

\begin{exe}
\ex \label{ex:XsAr.akWmtChW}
\gll  nɯnɯ a-kɯmtɕʰɯ, χsɤr a-kɯmtɕʰɯ nɯnɯ kɤ-ɣɯt a-pɯ-tɯ-cʰa qʰe,  \\
\textsc{dem} \textsc{1sg}.\textsc{poss}-toy gold \textsc{1sg}.\textsc{poss}-toy \textsc{dem} \textsc{inf}-bring \textsc{irr}-\textsc{ipfv}-2-can \textsc{lnk} \\
\glt `If you can bring my toy, my golden toy back...' (140429 qingwa wangzi-zh, 52)
\end{exe}

Note that unlike the IPN and APN modifiers discussed above, \textit{pronominal} prenominal modifiers (§ XXX) do not take possessive prefixes that have scope on the head noun. For instance, with the modifier \japhug{kɯmaʁ}{other}, the possessive prefix must appear on the following noun, as second person \forme{nɤ-} on \japhug{slama}{student} in (\ref{ex:kWmaR.nAslama}).

\begin{exe}
\ex \label{ex:kWmaR.nAslama}
 \gll  kɯmaʁ nɤ-slama ci tu-tɯ-ndɤm ju-tɯ-ɣɯt ɯ́-ŋu \\
 other \textsc{2sg}.\textsc{poss}-student \textsc{indef} \textsc{ipfv}-2-take[III] \textsc{ipfv}-2-bring \textsc{qu}-be:\textsc{fact} \\
 \glt `So you are bringing other students of yours?' (conversation, 150418)
\end{exe}

\section{Unpossessible nouns} \label{sec:unpossessible.nouns}
In addition to IPNs and APNs seen in the previous sections, Japhug also has a category of unpossessible nouns (UN), which includes names of places and ethnic groups (as in Koyukon Athabaskan, \citealt[651]{thompson96koyukon}), colour terms of Tibetan origin and some derived nouns like the social relation collectives (\ref{sec:social.collective}). With the exception of colours terms (§ \ref{sec:tibetan.colours}), these nouns can be used as modifiers of other nouns and are one of the three classes of `property words' (corresponding to the adjectives of Standard Average European),  alongside adjectival stative verbs (see § XXX) and property nouns (\ref{sec:property.nouns}).

 
%schluecker17proper

\subsection{Place names}  \label{sec:place.names}
Place names (\japhug{mbarkʰom}{Mbarkham}, \japhug{kɤmɲɯ}{Kamnyu} etc) and names of ethnic groups (such as \japhug{kɯrɯ}{Tibetan, Gyalrong} or \japhug{kupa}{Chinese}), like personal names, cannot take possessive prefixes when used independently. They can only be used with independent pronouns as in (\ref{ex:jiphe.kAmYW}) or (\ref{ex:iZo.KamnYW}) below.

\begin{exe}
\ex \label{ex:jiphe.kAmYW}
\gll   iʑora ji-pʰe kɤmɲɯ nɯtɕu <xiaoxue> <yinianji> <ernianji> pɯ-ndɯn-a. \\
\textsc{1pl} \textsc{1pl}.\textsc{poss}-\textsc{dat} pl.n. \textsc{dem}:\textsc{loc} primary school first.grade second.grade \textsc{pfv}-read-\textsc{1sg} \\
\glt `I studied the first and second grade of primary school at our place in Kamnyu.' (140501 tshering skyid, 12)
\end{exe}

These types of nouns can serve as strictly prenominal modifiers (as in \japhug{kɯrɯ sɤtɕʰa}{Tibetan areas}) and commonly occur as first member of nominal compounds (as in \japhug{kɯrɯɕɤmɯɣdɯ}{traditional gun}, with \japhug{ɕɤmɯɣdɯ}{gun} as second element, see \ref{sec:karmadharaya.n.n}). Although these nouns are  unpossessible by themselves, when used as first member of a compound, or even as prenominal modifier (on which see § \ref{sec:possessive.prefixes.prenominal}), they can take a possessive prefix which has scope on the head noun, as in examples (\ref{ex:jikWrWlAsAr}) and (\ref{ex:jikAmYWskAt}), where the \textsc{1pl} possessive prefix \forme{ji-} occurs prefixed on the name \japhug{kɯrɯ}{Tibetan} and on the place name  \japhug{kɤmɲɯ}{Kamnyu}.

\begin{exe}
\ex \label{ex:jikWrWlAsAr}
 \gll nɯtɕu tɕe iʑora ji-kɯrɯ-lɤsɤr ŋu \\
 \textsc{dem}:\textsc{loc} \textsc{lnk} \textsc{1pl} \textsc{1pl}.\textsc{poss}-Tibetan-new.year be:\textsc{fact} \\
 \glt `At that time, it is our Tibetan new year.' (conversation, 150102)
\end{exe}

\begin{exe}
\ex \label{ex:jikAmYWskAt}
 \gll nɤʑo ji-kɤmɲɯ-skɤt nɯnɯ <quanshijie> ʑo ju-tɯ-sɯ-ɤzɣɯt ŋu \\
 \textsc{2sg} \textsc{1pl}.\textsc{poss}-pl.n.-language \textsc{dem} whole.world \textsc{emph} \textsc{ipfv}-2-\textsc{caus}-reach be:\textsc{fact} \\
  \glt `You are spreading our Kamnyu language to the whole world.' (conversation, 150618)
\end{exe}

The pair of examples in (\ref{ex:nArmi}) illustrates the different behaviour of UPNs as first elements of compounds on the one hand, and as noun modifier on the other hand. In (\ref{ex:kWrW.nArmi}), \japhug{kɯrɯ}{Tibetan} and \japhug{tɤ-rmi}{name} constitute a single compound noun. As in the examples above, the possessive prefix occurs before \forme{kɯrɯ}. Stranding the modifier in a form such as $\dagger$\forme{kɯrɯ a-rmi} is considered to be agrammatical by native speakers. In (\ref{ex:faguo.nArmi}) however, the modifier \forme{faguo} `France, French' (from Chinese) cannot be compounded with  \japhug{tɤ-rmi}{name} and cannot take possessive prefixes. This is a rare example where a noun modifier can be stranded from the stem of the head noun by a definite possessive prefix (§ \ref{sec:possessive.prefixes.prenominal}).

\begin{exe}
\ex \label{ex:nArmi}
\begin{xlist}
\ex  \label{ex:kWrW.nArmi}
\gll nɤ-kɯrɯ-rmi   \\
\textsc{2sg}:\textsc{poss}-Tibetan-name \\
\glt `Your Tibetan name.' 
\ex  \label{ex:faguo.nArmi}
\gll nɤʑɯɣ <faguo> nɤ-rmi   \\
\textsc{2sg}:\textsc{gen} France \textsc{2sg}.\textsc{poss}-name \\
\glt `Your French name.' 
\end{xlist}
\end{exe}
 
 

Place names followed by the plural \forme{ra} designate the people living in the place (\ref{ex:iZo.KamnYW}), even without \forme{-pɯ} suffixation (§ \ref{ex:inhabitant.pW}). Example (\ref{ex:iZo.KamnYW}) also shows that in this usage, it is possible to use a personal pronoun in apposition as in \forme{iʑo kɤmɲɯ ra}  `we Kamnyu people'.

\begin{exe}
\ex \label{ex:iZo.KamnYW}
 \gll iʑo kɤmɲɯ ra kɯ tɕʰɯχpri tu-ti-j ŋu. rcaqo ra cho mɤŋi ra kɯ tɕʰɯχpɯχpri tu-ti-nɯ ŋu \\
 \textsc{1pl} pl.n. \textsc{pl} \textsc{erg} salamander \textsc{ipfv}-say-\textsc{1pl} be:\textsc{fact} pl.n. \textsc{pl} \textsc{comit} pl.n. \textsc{pl} \textsc{erg}  salamander \textsc{ipfv}-say-\textsc{pl} be:\textsc{fact} \\
 \glt `We Kamnyu people call it \forme{tɕʰɯχpri}, and people from Rqakyo and Mangi call it \forme{tɕʰɯχpɯχpri}. ' (25-tChWXpri, 20)
\end{exe}

Place names can take some prenominal modifiers such as \japhug{pʰa}{whole} as in (\ref{ex:pha.RdWrJAt}), but no example of bare place names with prenominal demonstrative have been found. %XXXXX à revérifier

\begin{exe}
\ex \label{ex:pha.RdWrJAt}
 \gll pʰa ʁdɯrɟɤt nɯ ɯ-ŋgɯ tɕe rqaco cʰo katɕa nɯ stu ɣɤndʐo \\
 whole pl.n. \textsc{dem} \textsc{3sg}.\textsc{poss}-inside \textsc{lnk} pl.n. \textsc{comit} pl.n. \textsc{dem} most cold:\textsc{fact} \\
 \glt  `In the whole of Gdongbrgyad, Rqakyo and Kacha are the coldest.' (140522 RdWrJAt, 104)
\end{exe}

Place names and ethnic names can be used as core arguments, or nominal predicates with a copula, as in (\ref{ex:kupa.Nu}) and (\ref{ex:taRdo.Nu}).

\begin{exe}
\ex \label{ex:kupa.Nu}
\gll  a-wa nɯnɯ kupa ŋu \\
\textsc{1sg}.\textsc{poss}-father \textsc{dem} Chinese be:\textsc{fact}  \\
\glt `My father is Chinese.' 140501 tshering skyid, 4)
\end{exe}

\begin{exe}
\ex \label{ex:taRdo.Nu}
\gll  tɕe ʁnɯ-tɯpɯ nɯnɯ taʁrdo ŋu  \\
\textsc{lnk} two-household \textsc{dem} pl.n. be:\textsc{fact} \\
\glt `These two household are Taqrdo.' 
\end{exe}

Like other locative nouns (§ XXX), bare place names can be used without postposition to express motion (\ref{ex:mbarkhOm.thWwGGWta}) or static location (\ref{ex:kAmYW.GJW}), but are also found with locative postpositions, most often \forme{ri} as in (\ref{ex:tshuBdWn.ri}) but also \forme{tɕu} or \forme{zɯ} (as in \ref{ex:jiphe.kAmYW} above).

\begin{exe}
\ex \label{ex:mbarkhOm.thWwGGWta}
 \gll a-pi kɯ tɤ́-wɣ-ndo-a tɕe tɕe, mbarkʰom tʰɯ́-wɣ-ɣɯt-a, \\
 \textsc{1sg}.\textsc{poss}-elder.sibling  \textsc{erg} \textsc{pfv}-\textsc{inv}-take-\textsc{1sg} \textsc{lnk} \textsc{lnk} pl.n.   \textsc{pfv:downstream}-\textsc{inv}-bring-\textsc{1sg}  \\
\glt `My elder brother took me and brought me to Mbarkham.' (140501 tshering skyid, 28)
\end{exe}

\begin{exe}
\ex \label{ex:kAmYW.GJW}
 \gll  kɯɕɯŋgɯ tɕe kɤmɲɯ ɣɟɯ kɯɕnɯz pjɤ-tu. \\
 former.days \textsc{lnk} pl.n. watchtower  seven \textsc{ifr}.\textsc{ipfv}-exist \\
 \glt `In former times, there were seven watchtowers in Kamnyu.' (140522 GJW, 1)
\end{exe}

\begin{exe}
\ex \label{ex:tshuBdWn.ri}
 \gll  tɕe alo tsʰuβdɯn ri pɯ-rɤʑi-j tɕe \\
 \textsc{lnk} upstream pl.n. \textsc{loc} \textsc{pst}.\textsc{ipfv}-stay-\textsc{1sg} \textsc{lnk} \\
 \glt `We were living up there in Tshobdun.' (28-kWpAz, 178)
\end{exe}

\subsection{Colour nouns} \label{sec:tibetan.colours}
Colour names of Tibetan origin, such as \japhug{ldʑaŋkɯ}{blue/green} from \tibet{ལྗང་གུ་}{ldʑaŋ.gu}{green} or \japhug{ʁmɤrsmɯɣ}{dark red} from \tibet{དམར་སྨུག་}{dmar.smug}{dark red} designate objects or animals with a particular colour, occurring in the same context as free S-participles of adjectival stative verbs of colour (\ref{ex:ldZaNkW}). Although such nouns could be expected to occur as noun modifiers, good examples are not found in the corpus. The adjectival stative verbs in \forme{arɯ-} derived from them (for instance \japhug{arɯldʑaŋkɯ}{be green}, see § XXX) are as common as the colour nouns.
 
\begin{exe}
\ex \label{ex:ldZaNkW}
 \gll qambalɯla rcanɯ ɯ-mdoʁ ʑakastaka ʑo kɯ-ŋu tu. .... kɯ-qarŋe tu, ldʑaŋkɯ tu, kɯ-ɤrŋi tu, kɯ-ɲaʁ tu. \\
 butterfly unexpectedly \textsc{3sg}.\textsc{poss}-colour each \textsc{emph} \textsc{nmlz}:S/A-be exist:\textsc{fact} .... \textsc{nmlz}:S/A-be.yellow exist:\textsc{fact} blue/green exist:\textsc{fact}  \textsc{nmlz}:S/A-be.green exist:\textsc{fact} \textsc{nmlz}:S/A-be.black exist:\textsc{fact}\\
 \glt `There are butterflies with all kinds of colours, yellow, green, blue/green, black. (26-qambalWla, 6)
\end{exe}

\subsection{Other UPN}   \label{sec:other.upn}
UPN other than proper names and colour terms include nouns occurring as postnominal modifiers include kinship UNs, for instance \japhug{tɯlɤt}{second sibling}\footnote{In \japhug{tɯlɤt}{second sibling} the \forme{tɯ-} element in this word is originally an indefinite possessive prefix, but has become lexicalized.} as in (\ref{ex:tWlAt}), terms of gender (such as \japhug{mu}{female}, see also \ref{sec:gender}) and  privative nouns in \forme{-lu} described in (\ref{sec:privative}). 

\begin{exe}
\ex \label{ex:tWlAt}
\gll  nɯ-me tɯlɤt nɯ ɲɤ-mbi-nɯ \\
\textsc{3pl.poss}-daughter second.sibling \textsc{dem} \textsc{ifr}-give-\textsc{pl} \\
\glt `They gave him their second daughter.' (2002qaCpa, 40)
\end{exe} 

Numerals under 99 are also unable to take possessive prefixes and serve as postnominal modifiers (§ \ref{sec:one.to.ten}), and can be considered to be a subclass of UPN.

\section{Personal names}  \label{sec:personal.names}
This section focuses on three topics: the absence of vocative forms, the Tibetan origin of personal names, and their use with pronouns and possessive prefixes. The question of the status of personal names in the empathy hierarchy is treated in § XXX in the section on inverse marking.

\subsection{Vocative} \label{sec:vocative}
Gyalrong languages other than Japhug have specific vocative forms for personal names and kinship terms. In Tshobdun, \citet[133]{jackson98morphology} and \citealt[53]{jackson05yingao} reports that personal names in the vocative have stress retraction. In Situ, IPN nouns have their possessive prefixes replaced by \forme{a-} in vocative forms (\citealt[471]{nagano03cogtse}, \citealt[177]{prins16kyomkyo}). 

In Japhug, due to the almost entire loss of contrastive stress (§ XXX) and the fact that the \textsc{1sg} possessive prefix has the form \forme{a-} (§ \ref{sec:possessive.paradigm}) unlike in Tshobdun and Situ (where it is \forme{ŋa-/ŋə-}), there is no distinct vocative form for either personal names or kinship terms. 

A prefix \forme{a-} does occur in the familiar form of personal names (reminding of Lin's \citeyear[162]{linxr93jiarongen} description of this prefix as a \zh{爱称} `pet name' marker), but not exclusively in vocative use as in example (\ref{ex:kWlAGacAB.nW.kW}) where we see the name \forme{acɤβ} as transitive subject, familiar form of a Tibetan name with \forme{scɤβ} as second element (see section \ref{sec:names.tibet} below).

 \begin{exe}
\ex \label{ex:kWlAGacAB.nW.kW}
\gll kɯ-lɤɣ acɤβ nɯ kɯ, ɯ-pʰɯŋgɯ nɯtɕu qapɯtɯm ci na-rku ɲɯ-ŋu, \\
\textsc{nmlz}:S/A-graze Askyabs \textsc{dem} \textsc{erg} \textsc{3sg}.\textsc{poss}-fold.of.clothes \textsc{dem:loc} pebble.from.flint \textsc{indef} \textsc{pfv}:3\fl3'-put.in \textsc{sens}-be \\
\glt `The shepherd Askyabs put a pebble in the folds of his clothes (to avoid forgetting what he had to told the king).'  (Kunbzang 332)
\end{exe}

Since similar \forme{a-} prefixes exist in Tibetan and Chinese, and since personal names are exclusively from either of these languages (there are no clear remnants of native personal names in Japhug), it is likely that the familiar form of the names was also borrowed.

\subsection{Tibetan names} \label{sec:names.tibet}
Speakers of Japhug generally have Tibetan names (given by a lama), and in addition a Chinese official name which may or may not be related to the Tibetan one. In one traditional story, we find an example of person names based on Japhug words as in (\ref{ex:zrAntCW}), but it looks so strange that the narrator felt it necessary to specify that these are people's names.

\begin{exe}
\ex  \label{ex:zrAntCW}
 \gll zrɤntɕɯ tɯrme ci pjɤ-tu, tɯpɕi kɯ-rmi ci pjɤ-tu, tɯrme nɯ-rmi ɲɯ-ŋu nɤ \\
mung.bean person \textsc{indef} \textsc{ifr}.\textsc{ipfv}-exist flax \textsc{nmlz}:S/A-call  \textsc{indef} \textsc{ifr}.\textsc{ipfv}-exist people  \textsc{3pl}.\textsc{poss}-name \textsc{sens}-be \textsc{sfp} \\
\glt `There was (a lady) was was called `Mung bean', and (another one) called `flax', these are names of people.' (zrAntCW, 1)
\end{exe}

Names used by Japhug speakers are not markedly different from those found in other Tibetan areas. Lady names often include the suffixes \forme{ltɕɤm}, \forme{rcit} or \forme{mtsʰu}, (from \tibet{ལྕམ་}{ltɕam}{lady, sister}, \tibet{སྐྱིད་}{skʲid}{happy} and \tibet{མཚོ་}{mtsʰo}{lake}), and there are also non-gender specific suffixes like \forme{scɤβ} (from \tibet{སྐྱབས་}{skʲabs}{protector}, for instance \forme{tsʰɯraŋ scɤβ} from \tibet{ཚེ་རིང་སྐྱབས་}{tsʰe.riŋ.skʲabs}{p.n.}).). 

Many Tibetan names have alternative readings reflecting distinct reading traditions belonging to more than two distinct layers (see § XXX on the layers of Tibetan borrowings in Japhug). For instance, some people with the Tibetan name \tibet{འཕྲིན་ལས་}{ⁿpʰrin.las}{Karma} are called \forme{mpʰrɯlɤz} (with preservation of the coda), other \forme{mpʰrɯli} (with Amdo-type change to \forme{-i}). The names however tend to have non-Amdo phonological features even for people of younger generation. For instance, the name \tibet{ཀུན་དགའ་}{kun.dga}{Ânanda} is pronounced \forme{kɯnga} without assimilation of the dental nasal to a velar nasal, and  \tibet{ཀུན་བཟང་}{kun.bzaŋ}{Sarvabhadra} is \forme{kɯnɯβzaŋ} with an anaptyctic vowel (see § XXX for examples of this type in the borrowed vocabulary).  

%\tibet{བཀྲ་ཤིས་}{bkra.ɕis}{good fortune} appears as \forme{krɤɕiz} with preservation of the coda and part of the initial cluster, \forme{krɤɕi} with Amdo-like loss of final \forme{-s} and \forme{tʂaɕi} with cluster simplification. 

\subsection{APN or UPN?} \label{sec:personal.name.APN}
Personal names superficially look like UPN, as they do not usually occur with possessive prefixes, even when taking noun modifiers in genitival relation. (XXXX)

Personal names commonly occur preceded by kinship terms which, being IPNs (§ \ref{sec:kinship}), have a possessive prefix as in (\ref{ex:anmaR.dpalcan}). It is considered impolite to address someone from an older generation than oneself without adding a kinship term -- for instance, the author of this book, being much younger, has to call him \forme{a-βɣo χpɤltɕin} with the \textsc{1sg} form of \japhug{tɤ-βɣo}{father's brother}.

\begin{exe}
\ex \label{ex:anmaR.dpalcan}
\gll a-nmaʁ χpɤltɕin \\
\textsc{1sg}.\textsc{poss}-husband p.n. \\
\glt `My husband Dpalcan.' (heard in context)
\end{exe}

Although personal names rarely occur with possessive prefixes, there is no grammatical constraint against it. There is one such example in the whole corpus, in a conversation where a clarification was needed. Tshendzin asks about Dpalcan, younger brother of Tshering Sgrolma, but she does not understand at once, because Tshendzin's husband is also called Dpalcan; thus Tshendzin says (\ref{ex:nWχpAltCin}) with the possessed form \japhug{nɯ-χpɤltɕin}{your Dpalcan} to disambiguate between the two. 

\begin{exe}
\ex   
\begin{xlist}
\ex 
\gll χpɤltɕin kɯmaʁ kɯ-nɯhɯɲi mɯ-jo-ɕe ɯ́-ŋu. \\
p.n. other \textsc{nmlz}:S/A-do.work \textsc{neg}-\textsc{ifr}-go \textsc{qu}-be:\textsc{fact} \\
\\
\glt (Tshendzin): `Dpalcan did not go for another job, did he?'
\ex 
\gll ka? \\
\textsc{sfp} \\
\glt  (Tshering Sgrolma): `What?'
\ex  \label{ex:nWχpAltCin}
\gll χpɤltɕin, nɯʑo nɯ-χpɤltɕin nɯ \\
p.n. \textsc{2pl} \textsc{2pl}.\textsc{poss}-p.n. \textsc{dem} \\
\\
\glt (Tshendzin): `Dpalcan, your Dpalcan.'  
\end{xlist}
\glt  (140510 tshering)
\end{exe}

Given the existence of such forms, personal names are treated as a subclass of APN rather than as UPNs. Note that only plural forms (\forme{ji-χpɤltɕin} `our Dpalcan', \forme{ʑara nɯ-χpɤltɕin} `Their Dpalcan' etc) are possible; singular forms such as $\dagger$\forme{a-χpɤltɕin} are not grammatical.
 
\subsection{Personal names and modifiers} \label{sec:personal.names.modifiers}
Personal names are more often than not used without demonstratives and determiners (see § \ref{sec:indefinite.markers}), but examples are not difficult to find either (see \ref{ex:Yimawodzer.NW}).

\begin{exe}
\ex \label{ex:Yimawodzer.NW}
 \gll  ɲimawozɤr nɯ kɯ, srɯnmɯ nɯ pjɤ-ftɯl, \\
 p.n. \textsc{dem} \textsc{erg} râkshasî \textsc{dem} \textsc{ifr}-subdue \\
 \glt `Nyima 'Odzer subdued the râkshasî.' (2011-4-smanmi, 258)
\end{exe}

It is possible for personal names to take dual or plural markers, even without an associative plural meaning, to designate a group of people with the same name, as in (\ref{ex:Dpalcan.XsWm}).

\begin{exe}
\ex \label{ex:Dpalcan.XsWm}
 \gll a-pɯ-ŋu tɕe, χpɤltɕin ʁnɯz, nɯ maʁ nɤ χsɯm kɯ-fse kɯ-naχtɕɯɣ tɯtɯrca a-pɯ-rɤʑi-nɯ tɕe, `χpɤltɕin ni, χpɤltɕin ra" nɯra tu-kɯ-ti kʰɯ. \\
 \textsc{irr}-\textsc{ipfv}-be \textsc{lnk} p.n. two \textsc{dem} not.be:\textsc{fact} \textsc{lnk} three \textsc{nmzl}:S/A-be.like \textsc{nmzl}:S/A-be.identical together \textsc{irr}-\textsc{ipfv}-stay-\textsc{pl} \textsc{lnk} p.n. \textsc{du} p.n. \textsc{pl} \textsc{dem}:\textsc{pl} \textsc{ipfv}-\textsc{genr}-say be.possible:\textsc{fact} \\ 
 \glt `For instance, if two or three (people called) Dpalcan live together, one can say `the two Dpalcans', `the Dpalcans'. (elicited)
\end{exe}

To distinguish between persons with the same name (a common occurrence among speakers of Japhug, given the relatively limited inventory of Tibetan names available), house names (\forme{kʰa ɯ-rmi}) are generally added as prenominal modifiers, as in (\ref{ex:taRrdo.dpalcan}).

\begin{exe}
\ex \label{ex:taRrdo.dpalcan}
\gll χpɤltɕin ɯ-kʰa nɯ taʁrdo rmi tɕe taʁrdo χpɤltɕin tu-kɯ-ti. \\
p.n. \textsc{3sg}.\textsc{poss}-house \textsc{dem} pl.n. be.called:\textsc{fact} \textsc{lnk} pl.n. p.n. \textsc{ipfv}-\textsc{genr}-say \\
\glt `Dpalcan's house is called Taqrdo, so one (can) call him `Taqrdo Dpalcan'.' (elicited)
\end{exe}

If  two persons from the same household have the same name, locational modifiers (§ XXX) can be used instead, as illustrated in (\ref{ex:maNlo.dpalcan}).

\begin{exe}
\ex \label{ex:maNlo.dpalcan}
\gll  nɯ maʁ nɤ, ndʑi-kʰa ɯ-rmi kɯnɤ a-pɯ-naχtɕɯɣ tɕe, kʰa kundi, lotʰi kɯ-fse nɯra tɕe,
maŋlo χpɤltɕin, maŋtʰi χpɤltɕin, maŋkɯ χpɤltɕin, maŋndi χpɤltɕin, nɯra tu-kɯ-ti ŋgrɤl. \\
\textsc{dem} not.be:\textsc{fact} \textsc{lnk} 3du.poss-house 3sg.poss-name also \textsc{irr}-\textsc{ipfv}-be.identical \textsc{lnk} house east.west up.down.stream \textsc{dem}:\textsc{pl} \textsc{lnk} upstream p.n. downstream p.n. east p.n. west p.n. \textsc{dem}:\textsc{pl} \textsc{ipfv}-\textsc{genr}-say be.usually.the.case:\textsc{fact} \\
\glt `If their house name is also the same, using the east-west or the upstream-downstream dimensions, one can say `Dpalcan from upstream, downstream, east or west.' (elicited)
\end{exe}

Like other nouns, personal names can also occur as head of non-restrictive relatives, as in  (\ref{ex:mWtAkWrZaR}) and (\ref{ex:WnmaR.pWkWNu}), though such uses are rather uncommon. No examples of personal names as heads of head-internal relatives have been found.

  \begin{exe}
\ex \label{ex:mWtAkWrZaR}
\gll  tɕendɤre 	iɕqʰa 	ʁlaŋsaŋtɕʰin 	χsɯm 	ma 	mɯ-tɤ-kɯ-rʑaʁ 	nɯ, \\
\textsc{lnk} the.aforementioned Gesar three apart.from \textsc{neg}-\textsc{pfv}-\textsc{nmlz}:S/A-pass.days \textsc{dem} \\
\glt `Gesar, who was only three days old,'  (Gesar 81)
\end{exe}

\begin{exe}
\ex \label{ex:WnmaR.pWkWNu}
\gll nɯ ɕɯŋgɯ ɯ-nmaʁ pɯ-kɯ-ŋu tsʰɯraŋ nɯ pjɤ-mto  \\
\textsc{dem} before \textsc{3sg}.\textsc{poss}-husband \textsc{pst}-\textsc{nmlz}:S/A-be p.n. \textsc{dem} \textsc{ifr}-see \\
\glt `See saw Tshering, who had been her husband before.' (2002qajdoskAt, 
\end{exe}

\section{Status constructus} \label{sec:status.constructus}
The term \textit{status constructus} is used in Gyalrongic linguistics (\citealt{jacques12incorp}, \citealt[163-4]{lai17khroskyabs}) to refer to the non-autonomous form of (mainly nominal, but also verbal and adverbial) roots occurring as non-final element of compounds. The use of this term, adopted from Semitic linguistics,  differs from works such as \citet{creissels06hongrois} or \citet{creissels17construct} in which `construct form' refers to a specific form used that is obligatory on the head noun in specific noun-modifier constructions (including with a possessive marker). Given the fact in that Japhug and other Gyalrongic languages, nominal compounds generally follow modifier-head order (the opposite of Semitic), the form undergoing \textit{status constructus} alternation in Japhug is often the modifier noun,\footnote{In addition, in Japhug the possessed form of nouns do not present morphological alternations (\ref{sec:possessive.paradigm}) with only one exception (\ref{sec:apn.to.ipn}).} except in Noun-Verb compounds where the second element is an adjectival stative verb.

In this section, the various types of alternations attested for first or other non-final members of compounds are described, in particular vowel alternation (the most common type). Additionally, exceptional changes to the final members of compounds are discussed in \ref{sec:final.compounds}.

\subsection{Vowel alternations in non-final members of compounds} \label{sec:vowel.alternations.compounds}
Regular \textit{status constructus} is Japhug applies to open syllables, and involves shift of all vowels to either \ipa{ɯ} or \ipa{ɤ} back unrounded vowels, following the correspondences in Table \ref{tab:sc.regular}.

\begin{table}
\caption{Regular \textit{status constructus} in Japhug} \label{tab:sc.regular}
\begin{tabular}{lllll}
\lsptoprule
Base & SC & Example \\
\ipa{-a} &\ipa{-ɤ} & \japhug{βɣɤsni}{mill axle} from  \japhug{βɣa}{mill} + \japhug{tɯ-sni}{heart} \\
\ipa{-e} &\ipa{-ɤ} & \japhug{tɕʰemɤpɯ}{little girl} from  \japhug{tɕʰeme}{girl} + \japhug{ɯ-pɯ}{little one} \\
\ipa{-i} &\ipa{-ɯ} & \japhug{smɯɣot}{light of the fire} from  \japhug{smi}{fire}+ \japhug{ɣot}{light}  \\
\ipa{-o} &\ipa{-ɤ} &  \japhug{mbrɤsno}{horse saddle} from  \japhug{mbro}{horse} + \japhug{tɤ-sno}{saddle}\\
\ipa{-u} &\ipa{-ɤ} & \japhug{tɤ-kɤrme}{head hair} from  \japhug{tɯ-ku}{head} + \japhug{tɤ-rme}{hair} \\
\lspbottomrule
\end{tabular}
\end{table}

Alternatively, the vowel \ipa{i} alternates with \ipa{ɤ} in \textit{status constructus}, as in \japhug{qaprɤftsa}{centipede} from \japhug{qapri}{snake} and \japhug{tɤ-ftsa}{nephew} or 
\japhug{tɯ-mɤmɲaʁ}{astragalus} from \japhug{tɯ-mi}{leg, foot} and \japhug{tɯ-mɲaʁ}{eye}. %\japhug{ɯ-χtɯrca}{with the others} &&from  \japhug{tɯ-χti}{companion} + \japhug{tɤ-rca}{together with}

In very few case, \ipa{u} can also alternate with \ipa{ɯ}, as in \forme{ŋɤtɕɯ-} which occurs in the expression \japhug{ŋɤtɕɯkɤti,kʰɯ}{obey to everything} (\ref{sec:interrogative.indef}), the status constructus of \japhug{ŋotɕu}{where}.

Nouns ending in \ipa{-ɯ} never have a \textit{status constructus} form distinct from the base form, as for instance \japhug{tɯmɯpaʁ}{slug} from \japhug{tɯ-mɯ}{sky} and \japhug{paʁ}{pig}.

Vowel alternation in closed syllables is very rare, and affects only a few stems with \ipa{o} as the main vowel (Table \ref{tab:sc.irregular}). The \textit{status constructus}  \forme{ɕɤm-} of \japhug{ɕom}{iron} occurs in a few other nouns, but the form \forme{staʁ-} (with internal sandhi to \forme{staχ-}, cf \ref{sec:internal.sandhi.compounds}) from \japhug{stoʁ}{broad bean} is unique.

\begin{table}
\caption{Irregular \textit{status constructus} in closed syllable stems} \label{tab:sc.irregular}
\begin{tabular}{lllll}
\lsptoprule
Base & SC & Example \\
\ipa{-oʁ} &\ipa{-aʁ} & \japhug{staχpɯ}{pea} from  \japhug{stoʁ}{broad bean} + \japhug{ɯ-pɯ}{little one} \\
\ipa{-om} &\ipa{-ɤm} & \japhug{ɕɤmtsʰoʁ}{iron nail} from  \japhug{ɕom}{iron} + \japhug{tɤtsʰoʁ}{nail} \\
\lspbottomrule
\end{tabular}
\end{table}


\subsection{Other alternations} \label{sec.compounds.first.other.alternations}
Apart from the regular vowel changes described above, four types of alternations are observed in non-final member of compounds: internal sandhi, loss of coda, reduced forms and loss of possessive prefix.

\subsubsection{Internal sandhi in compounds} \label{sec:internal.sandhi.compounds}
First, the first element of a cluster undergoes internal sandhi (section XXX), with voicing and nasal assimilation as in Table \ref{tab:sandhi.compounds}. 

\begin{table}
\caption{Internal sandhi in compounds} \label{tab:sandhi.compounds} 
\begin{tabular}{lllll}
\lsptoprule
Type & Example \\
Nasal assimilation & \ipa{t} \fl{} \ipa{n} /\_[+nasal] & \japhug{tsʰɤnmu}{ewe} \\
&&from  \japhug{tsʰɤt}{goat} + \japhug{mu}{female} \\
Voicing assimilation & \ipa{ɣ} \fl{} \ipa{x} /\_[-voiced] & \japhug{zrɯxpɯ}{little louse} \\
&&from  \japhug{zrɯɣ}{louse} + \japhug{ɯ-pɯ}{little one} \\
  & \ipa{ʁ} \fl{} \ipa{χ} /\_[-voiced] & \japhug{tɯ-jaχpa}{palm} \\
&&from  \japhug{tɯ-jaʁ}{arm, hand} + \japhug{pa}{down} \\
  & \ipa{z} \fl{} \ipa{s} /\_[-voiced] & \japhug{mbrɤstshi}{rice soup} \\
&&from  \japhug{mbrɤz}{rice} + \japhug{tɯtsʰi}{rice soup} \\
\lspbottomrule
\end{tabular} 
\end{table}

There are cases of irregular internal sandhi only attested in lexicalized compounds. For instance \japhug{jaŋntsɤrpa}{one-handed axe} from \japhug{tɯ-jaʁ}{arm, hand}, \japhug{ɯ-ntsi}{one of a pair} and \japhug{tɯ-rpa}{axe}, showing a nasal assimilation rule \hbox{\ipa{ʁ} \fl{} \ipa{ŋ} /\_[+nasal]} which is not productive in the language (as shown by words such \japhug{tɯ-jaʁndzu}{finger}, also with \japhug{tɯ-jaʁ}{arm, hand} as first element).

\subsubsection{Loss of codas in compounds} \label{sec:loss.codas.compounds}
Loss of coda is not a regular process in first elements of compounds. The following list collects some of the most representative examples. Many examples are found in numerals (see § \ref{sec:approx.numerals} and § \ref{sec:numeral.prefixes}).

\begin{itemize}
\item Loss of \ipa{-β}: 

\japhug{ɴqiaβ}{dark side of the mountain}  + \japhug{zwɤr}{mugwort} \fl{}  \japhug{ɴqiazwɤr}{Artemisia sp.}  
\item Loss of  \ipa{-t}: 

\japhug{xtɯt}{be short} + \japhug{rɲɟi}{be long} \fl{} \japhug{xtɯrɲɟi}{length (n)}  

 \japhug{tsʰɤt}{goat} + \japhug{ta-ʁrɯ}{horn} \fl{} \japhug{tsʰɤʁrɯ}{goat horn}  
\item Loss of \ipa{-z}: 

\japhug{qartsʰaz}{deer}  + \japhug{tɯ-ndʐi}{skin} \fl{}  \japhug{qartsʰɤndʐi}{deer hide}  
\item Loss of \ipa{-r}:

 \japhug{zwɤr}{mugwort} + \japhug{wɣrum}{be white} \fl{} \japhug{zwɤɣrum}{Artemisia sp.}  

\japhug{ɕɤr}{night} + \japhug{ɯ-χcɤl}{middle} \fl{}  \japhug{ɕɤχcɤl}{middle of the night}  
\item Loss of \ipa{-ɣ}:

 \japhug{tɤjmɤɣ}{mushroom}  + \japhug{tɯ-sti}{alone}  \fl{}  \japhug{jmɤtɤsti}{species of mushroom}  
 
\japhug{tɯ-mtʰɤɣ}{waist}  + \japhug{rŋgɤβ}{attach} \fl{}  \japhug{tɯ-mtʰɤrɴɢɤβ}{part of the trouser where one can tuck things in} 
\item Loss of \ipa{-ʁ}: 

\japhug{ɕoʁ}{buckwheat} + \japhug{wɣrum}{be white} \fl{}  \japhug{ɕɤɣrum}{buckwheat sp}  

\japhug{paʁ}{pig} + \japhug{tɯ-qa}{root, paw, bottom} \fl{}  \japhug{pɤqa}{stuffed pig feet}  
\end{itemize}

With the exception of the loss of \forme{-t}, which is relatively common, the other cases are rare and cannot be predicted by any rule based on phonology (the presence of a cluster in the following element is irrelevant, for instance). Some of them occur with other alternations in the second syllable (cf \ref{sec:second.member.alternation}).

\subsubsection{Reduced forms} \label{sec:reduced.forms.compounds}   
A handful of nouns have reduced \textit{status constructus} forms when occurring as first member of compounds; the nouns \japhug{nɯŋa}{cow} and \japhug{kʰɯna}{dog} are treated below. 

The noun \japhug{nɯŋa}{cow} corresponds to the syllable \forme{ŋɤ-} in the compounds \japhug{ŋɤnɯ}{udder} (with \japhug{tɯ-nɯ}{teat} as second element), \japhug{ŋɤqe}{cow dung} (with \japhug{tɯ-qe}{shit, dung}) and \japhug{ŋɤlitɕaʁmbɯm}{dung beetle} (on whcih see \ref{sec:second.member.alternation}), which would be the regular \textit{status constructus} from a stem \forme{ŋa-}. The apparent `loss' of a \forme{nɯ-} element is due to the fact that the noun \japhug{nɯŋa}{cow} is itself an ancient compound comprising \japhug{tɯ-nɯ}{teat} as first element (`bovid with udders').

In the case of \japhug{kʰɯna}{dog}, we find the \textit{status constructus} \forme{kʰɯ-} in the compounds \japhug{kʰɯndʐi}{dog skin} (with \japhug{tɯ-ndʐi}{skin} as second element), \japhug{kʰɯdo}{old dog} (see \ref{sec:derogative}), \japhug{kʰɯtsʰoʁ}{hunting with dog} (probably a noun-verb compound with \japhug{tsʰoʁ}{attach}, see also the related incorporating verb in § XXX) and a few plant names such as \japhug{kʰɯlu}{Euphorbia helioscopia} (a \textit{bahuvrīhi} meaning `(the plant) having dog milk' -- referring to its toxic juice, see \ref{sec:bahuvrihi.n.n}) and \japhug{kʰɯrtsʰɤz}{unindentified plant} (`dog lung'; the second element is \japhug{tɯ-rtsʰɤz}{lung}). Unlike  \japhug{nɯŋa}{cow}, whose reduced \textit{status constructus} corresponds to the second syllable, the syllable \forme{kʰɯ-} corresponds to the first syllable of \japhug{kʰɯna}{dog}, which must also be an obscured compound. The etymology of  the element \forme{-na} is unclear.

\subsubsection{Loss of possessive prefix} \label{sec:loss.possessive.prefix.compounds}
Some IPNs or alienabilized former IPNs lose their possessive prefix (or frozen indefinite possessive \forme{tɯ-/tɤ-}, see \ref{sec:frozen.indef}), as for instance the noun  \japhug{jmɤrtaʁ}{silverfish}, which comes from \japhug{tɤ-jme}{tail} and \japhug{artaʁ}{be forked} (`forked tail').\footnote{The verb \japhug{artaʁ}{be forked} itself is denominal from \japhug{tɤ-jwaʁ}{branch}.} Its first element \japhug{tɤ-jme}{tail} loses the prefix \forme{tɤ-} and undergoes regular vowel alternation.

Similar examples are particularly common with \japhug{tɯ-xtsa}{shoe}, as mainly parts of the shoes are referred to by APN compounds with \forme{xtsɤ-} as first element (\japhug{xtsɤɕna}{tip of the shoe}, \japhug{xtsɤrkɯ}{sides of the shoe} etc).

In some derivations that originate from compounds, such as the privative (\ref{sec:privative}) or the derogative  (\ref{sec:derogative}), the indefinite possessor prefix is also removed.

\subsection{Final member of compounds} \label{sec:final.compounds}
Morphological changes affecting the last members of compounds are less common that those on the first members. The only productive morphological alternation in this context is the loss of possessive prefix when the last member is an IPN.

\subsubsection{Loss of possessive prefix} \label{sec:possessive.prefix.second.compounds}
In compounds with an IPN as final element, the indefinite possessive prefix is lost as a rule, as in for example in the plant name \japhug{kʰɯnajme}{Setaria viridis} from \japhug{kʰɯna}{dog} and \japhug{tɤ-jme}{tail}.

Exceptions are very few. They include compounds whose second element is itself a compound, such as \japhug{lɤndʐitɤlɤtsʰaʁ}{Delphinium sp.} from \japhug{lɤndʐi}{ghost} and \japhug{tɤlɤtsʰaʁ}{milk filter}; the second element is from \japhug{tɤ-lu}{milk} in \textit{status constructus} and \japhug{tsʰaʁ}{sieve}. In \japhug{lɤndʐitɤlɤtsʰaʁ}{Delphinium sp.}, the indefinite possessive prefix \forme{tɤ-} has become frozen when the compound \japhug{tɤlɤtsʰaʁ}{milk filter} was formed, and is therefore not subject to deletion.

Another exceptional example is \japhug{ɯ-qataʁrɯ}{hoof} from \japhug{tɯ-qa}{root, paw, bottom} and \japhug{ta-ʁrɯ}{horn}, perhaps because the second element was perceived as being alienabilized, meaning `the horn-like thing on the foot'; in alienabilized possessive forms, definite possessive prefixes are stacked onto the indefinite possessive instead of replacing it,  see \ref{sec:alienabilization}).

\subsubsection{Alternations} \label{sec:second.member.alternation} 
Morphophonological alternations affecting last members of compounds are very rare in Japhug. 

Internal sandhi influencing the second member of a compound rather than the first occur when a root ending in \ipa{-ʁ} is followed by a cluster with a velar fricative as first element. Thus, the incorporating verb \japhug{amɲaχtsʰɯm}{be petty} is the denominal of a lost compound \forme{*mɲaχtsʰɯm} comprising \japhug{tɯ-mɲaʁ}{eye} as first element and \japhug{xtsʰɯm}{be thin}: the combination of \forme{-ʁ+xtsʰ-} yields \forme{-χtsʰ-}.

Several cases of alternations in the last member are found with animal nouns with the uvular class prefix \forme{qa-}, which has a variant \ipa{χ-/ʁ-} in this context in some compounds (see \ref{sec:uvular.animal} and \ref{sec:uvular.other}). 

Other alternations are restricted to specific lexical items, which are discussed below one by one (\japhug{rŋgɤβ}{attach}, \japhug{ɣɯrni}{be red}, \japhug{tʂu}{path} and \japhug{tɯ-ɣli}{excrement, dung}).

The IPN \japhug{tɯ-mtʰɤrɴɢɤβ}{part of the trouser where one can tuck things in} (a noun whose meaning is better explained by an example sentence like \ref{ex:WmthArNGAB}) is a compound of the noun \japhug{tɯ-mtʰɤɣ}{waist} with the transitive verb \japhug{rŋgɤβ}{attach}, which appears as a uvularized allomorph \forme{-rɴɢɤβ} not attested otherwise; it is unclear why uvularization took place in this word (dissimilation with the coda \ipa{-ɣ} of the previous root is unlikely).

\begin{exe}
\ex \label{ex:WmthArNGAB}
\gll tsʰi tɤ-mda tɕe nɯ ɯʑo ɯ-cʰɤmdɤru nɯ pjɯ-nɯ-rʁe tɕe pjɯ-nɯ-tsʰi, mɯ-na-tsʰi tɕe tɕe li tu-nɯ-χɕoʁ tɕe ɯ-mtʰɤrɴɢɤβ cʰɯ-nɯ-rʁe \\
drink:\textsc{fact} \textsc{pfv}-be.the.moment \textsc{lnk} \textsc{dem} \textsc{3sg} \textsc{3sg.poss}-drinking.straw \textsc{dem} \textsc{ipfv}:\textsc{down}-\textsc{auto}-insert \textsc{lnk} \textsc{ipfv}:\textsc{down}-\textsc{auto}-drink \textsc{neg}-\textsc{pfv}:3\fl3'-drink \textsc{lnk} \textsc{lnk} again \textsc{ipfv}:\textsc{up}-\textsc{auto}-take.out \textsc{lnk} \textsc{3sg.poss}-tuck \textsc{ipfv}:\textsc{downstream}-\textsc{auto}-insert \\
\glt `When it is time to drink, he inserts his straw (into the jar) and drinks from it, and when he does not drink any more, he takes it out and tucks it back into his trousers.' (30-tChorzi, 45)
\end{exe}

The noun \japhug{ftɕɤru}{path in the middle of the fields} is a compound of \japhug{ftɕar}{summer} and \japhug{tʂu}{path} (such paths are made during summer to allow workers to work in the field without damaging the crops). The first element \forme{ftɕɤ-} is the \textit{status constructus} of \forme{ftɕar} (with loss of final consonant) and the form \forme{-ru} for the second member of the compound is a clue that \forme{tʂu} comes from earlier \forme{*t-ro} with a dental stop+\ipa{r} cluster changing to a retroflex affricate (see § XXX and § \ref{sec:teens}) -- the \forme{*t-} element being prefixal (perhaps a fossilized indefinite possessor prefix).

The noun \japhug{jmɤɣni}{russula} clearly derives from \japhug{tɤjmɤɣ}{mushroom} and \japhug{ɣɯrni}{be red}, but while the loss of the \forme{tɤ-} prefix can be explained (see \ref{sec:frozen.indef}), the form of the second element (without \forme{r-} preinitial) is a mystery. The form \forme{-rni} (without \forme{ɣɯ-}, a prefix possibly of denominal origin, see § XXX) is found in \japhug{qrorni}{red ant} with \japhug{qro}{ant} as first element (a late innovation specific to the Kamnyu dialect, § XXX).
 
 The compound \japhug{ŋɤlitɕaʁmbɯm}{dung beetle}, with the irregular \textit{status constructus} \forme{ŋɤ-} (see \ref{sec:reduced.forms.compounds}) of the noun \japhug{nɯŋa}{cow}, contains a syllable  \forme{-li} clearly derived from the IPN \japhug{tɯ-ɣli}{excrement, dung}.\footnote{The second part of the noun \forme{-tɕaʁmbɯm} contains \japhug{aʁmbɯm}{be concave}. }  
 
The examples above show that most of the forms with irregular second member also present some irregularity in the first member of the compound. 

\section{Compound nouns}
Nominal compounds in Japhug can be build by compounding nouns, but also verbs and adverbs. In this section, compounds are first classified by the part of speech of their elements, and then by the semantic relationship between these elements.

\subsection{Noun-Noun compounds} \label{sec.n.n.compounds}
Noun-Noun compounds can be divided in three classes: \textit{tatpurusha} (determinative),   \textit{karmadhāraya} (attributive), appositive, \textit{bahuvrīhi} (possessive)  and \textit{dvandva}.

\subsubsection{Tatpurusha} \label{sec:tatpurusha.n.n}
\textit{Tatpurusha} or determinative compounds (corresponding to a genitive phrase followed by its head noun) are the most common type of compounds in Japhug. Almost all compounds of this type follow Modifier-Head order.

While genitive phrases are followed by a noun with a third person possessive prefix (see § XXX), in the corresponding compounds the possessive prefix is deleted (except the indefinite possessor prefix in exceptional examples, see \ref{sec:possessive.prefix.second.compounds}).

In this type of compounds, the first element is most commonly in \textit{status constructus} if from a word ending in open syllable, both for highly lexicalized compounds \japhug{qaɕpɤrnoʁ}{wild strawberry} (`frog's brain', from \japhug{qaɕpa}{frog} and \japhug{tɯ-rnoʁ}{brain}) and more transparent ones (\japhug{jlɤndʐi}{hybrid yak hide} from \japhug{jla}{hybrid yak}  and \japhug{tɯ-ndʐi}{skin}). 

Yet, \textit{tatpurusha}-s without \textit{status constructus} are also attested even among lexicalized terms such as plant names, for instance \japhug{qaprimdʑu}{Sagittaria trifolia} (from \japhug{qapri}{snake} and \japhug{tɯ-mdʑu}{tongue}). At least some of these compounds can be turned back into a noun phrase with a possessive prefix on the head noun, for instance \forme{qapri ɯ-mdʑu} (snake \textsc{3sg.poss}-tongue) `a snake's tongue'.

Among \textit{tatpurusha}-s are compounds with a participle as second element, including both S/A-participles in \forme{kɯ-} and oblique participles in \forme{sɤ-}. Common examples include \japhug{tʂɤsɤɴɢɤt}{crossroad}  from the \textit{status constructus} of \japhug{tʂu}{path} and the oblique participle \japhug{ɯ-sɤ-ɴɢɤt}{place where X part ways} from \japhug{ɴɢɤt}{part ways, part company}.\footnote{This intransitive verb itself is the anticausative of \japhug{qɤt}{separate}, § XXX. }

Compounds with oblique participles as first, rather than second, element, are also attested, for instance the obsolete noun \japhug{sɤqrɤcʰa}{alcohol to treat the guests}, from the oblique participle \forme{sɤ-qru} of the verb \japhug{qru}{greet, welcome, receive} and the noun \japhug{cʰa}{alcohol}.

In the case of subject participles, the compound does not derive from a genitival construction, though it is superficially similar to it. For instance \japhug{qalekɯtsʰi}{species of kite} comes from \japhug{qale}{wind} and the participle \japhug{ɯ-kɯ-tsʰi}{blocking (it)} of the transitive verb \japhug{tsʰi}{block}; the phrase \forme{qale ɯ-kɯ-tsʰi} (wind \textsc{3sg.poss}-\textsc{nmlz}:S/A-block) `blocking the wind' is more properly a headless participial relative (§ XXX), and is more similar to Object-Verb compounds (\ref{sec:object.verb.compounds}).

An example of Head-Modifier \textit{tatpurusha} in Japhug is provided by the ICN \japhug{tɯ-pɤrme}{one year of life}, which comes from \japhug{tɯ-xpa}{one year} and   \japhug{tɯrme}{man} (see § \ref{sec:frozen.indef} on the \forme{tɯ-} prefix). In this word, which originally means `man's year (of life)', the word `man' (originally a modifier) appears second element.

\subsubsection{Karmadhāraya}  \label{sec:karmadharaya.n.n}

Karmadhāraya are compounds built from nouns and their pre- or post nominal modifiers (§ XXX). These modifiers are distinct from genitival modifiers (§ XXX) and include in particular nouns referring to places, species, or people (\ref{sec:unpossessible.nouns}).  Both Modifier-Head and Head-Modifier orders are attested.


The ethnic names \japhug{kɯrɯ}{Tibetan} and \japhug{kupa}{Chinese} commonly occur as first-element modifiers in compounds such as \japhug{kupaŋga}{Chinese-style clothes} (with \japhug{tɯ-ŋga}{clothes}) or \japhug{kupastaχpɯ}{soja}(with \japhug{staχpɯ}{pea}, on which see \ref{sec:vowel.alternations.compounds}). 

Nouns denoting locations and places as first element of compounds include \japhug{sɯŋgɯ}{forest} in \japhug{sɯŋgɯrmɤβja}{lophophoprus} (with \japhug{rmɤβja}{peacock}) and \japhug{sɯŋgɯpɤjka}{type wild squash} (with \japhug{pɤjka}{squash}) -- although \forme{sɯŋgɯ} means `forest' (this word itself is a compound from \japhug{si}{tree} and \japhug{ɯ-ŋgɯ}{inside}, and literally means `among the trees), it is better translated as `wild' when occurring as prenominal modifier or first member of compounds. Compounds also exist with specific placenames such as \japhug{tɕʰɯtɕɯn}{Jinchuan}, for instance in \japhug{tɕʰɯtɕɯnpaχɕi}{pear} (with  \japhug{paχɕi}{apple} as second element), a noun which can undergo denominal derivation to  \japhug{nɯtɕʰɯtɕɯnpaχɕi}{pick pears} (§ XXX), showing that the place name modifier has been integrated.

The noun \japhug{qajɯ}{bug} occurs as first element of many compounds, such as \japhug{qajɯsmɤnba}{leech} (with \japhug{smɤnba}{doctor}). In this case meaning alone makes it clear that the  first element of the compound does not derive from a genitive phrase, as this compound means `bug acting as a doctor' or `doctor who is a bug', not `doctor treating bugs'.

Head-Modifier \textit{karmadhāraya}-s are the lexicalized versions of nouns followed by post-nominal modifiers (\ref{sec:unpossessible.nouns}). A good example of such compounds is provided by \japhug{ʑmbrɯkɤlu}{willow that does not grow high} from \japhug{ʑmbri}{willow} and the privative form \japhug{kɤlu}{headless} (\ref{sec:privative}) of \japhug{tɯ-ku}{head}), a name explained in (\ref{ex:kAlu}). 

\begin{exe}
\ex \label{ex:kAlu}  
\gll ɯ-taʁ ɯ-mnɯ kɯnɤ kɯ-zri tu-ɬoʁ mɯ́j-cʰa tɕe, nɯ-kɤ-ʁndzɤr ʑo ɲɯ-fse tɕe nɯ ʑmbrɯkɤlu tu-kɯ-ti ŋu. tɕe nɯ ɯ-ku kɯ-me kɤ-ti ɲɯ-ŋu.  kɤlu nɯ ɯ-ku kɯ-me kɤ-ti ɲɯ-ŋu.\\
\textsc{3sg}-on \textsc{3sg.poss}-new.twig also \textsc{nmlz}:S/A-be.long \textsc{ipfv}:\textsc{up}-come.out \textsc{neg:sens}-can \textsc{lnk} \textsc{pfv}-\textsc{nmlz}:P-cut \textsc{emph} \textsc{sens}-be.like \textsc{lnk} \textsc{dem} plant.name \textsc{ipfv}-\textsc{genr}-say be:\textsc{fact} \textsc{lnk}  \textsc{dem} \textsc{3sg.poss}-head \textsc{nmlz}:S/A-not.exist \textsc{inf}-say \textsc{sens}-be headless \textsc{dem} \textsc{3sg.poss}-head \textsc{nmlz}:S/A-not.exist \textsc{inf}-say \textsc{sens}-be\\
\glt `Its new twigs cannot grow very long, and look like they have been sawed short, therefore it is called `headless willow'. `Headless' means `without head'.'(07-Zmbri, 34-36)
\end{exe}

\subsubsection{Appositive} \label{sec:appositive.n.n}
Appositive compounds, traditionally analyzed as a subclass of \textit{karmadhāraya}, are rare in Japhug. We find one example both of whose members are participles: \japhug{kɯrŋukɯɣndʑɯr}{harvestman}, from the \forme{kɯ-} participles of the transitive verbs \japhug{rŋu}{parch} and  \japhug{ɣndʑɯr}{grind}. The two elements of the compound refers to the actions supposedly performed by that type of chelicerate (`the parcher-grinder') like the participial form of a bipartite verb (section XXX).


\subsubsection{Bahuvrīhi} \label{sec:bahuvrihi.n.n}
\textit{Bahuvrīhi}-s are considerably less common than \textit{tatpurusha}-s, and tend to be synchronically obscure. Head-Modifier order is found for instance in the \textit{bahuvrīhi} plant name  \japhug{kʰɯlu}{Euphorbia helioscopia} combining the reduced \textit{status constructus} of \japhug{kʰɯna}{dog} (\ref{sec:reduced.forms.compounds}) with \japhug{tɤ-lu}{milk}. It presumably means `having dog milk', a reference to a whitish toxic liquid that comes from it (\ref{ex:khWlu}).

\begin{exe}
\ex \label{ex:khWlu}  
\gll tɕe nɯ kʰɯlu nɯnɯ sɤndɤɣ. ɯ-lu tu tɕe, tɤ-lu kɯ-fse kɯ-wɣrɯ\redp{}wɣrum ŋu. koŋla ʑo, pjɯ́-wɣ-qlɯt tɕe, nɯre ɯ-lu tu. \\
\textsc{lnk} \textsc{dem} Euphorbia.helioscopia \textsc{dem} poisonous:\textsc{fact} \textsc{3sg.poss}-milk exist:\textsc{fact} \textsc{lnk}  \textsc{indef.poss}-milk \textsc{nmlz}:S/A-be.like \textsc{nmlz}:S/A-\textsc{emph}\redp{}be.white be:\textsc{fact} completely \textsc{emph} \textsc{ipfv}-\textsc{inv}-break there \textsc{3sg.poss}-milk exist:\textsc{fact} \\
\glt `The Euphorbia is toxic, it has a juice white like milk, when it is broken, there is milk in (the stalk). (19-khWlu, 20-22)
\end{exe}
Among \textit{bahuvrīhi}-s are found compounds containing a numeral (\textit{dvigu}). The numeral \japhug{kɯngɯt}{nine} in \japhug{kɯngɯttɤrtsɤɣ}{Leonurus} (with \japhug{tɤ-rtsɤɣ}{stairs}, `(plant) having nine stairs') is prenominal, unlike normal order (see § XXX) but similar to quantifiers (see § XXX). The same Numeral-Noun order is found in the more complex compound \japhug{kɯngɯttɤrqʰɤɴɢaʁ}{unidentified plant} discussed in \ref{sec.n.v.compounds}.

\subsubsection{Dvandva} \label{sec:dvandva.n.n}
\textit{Dvandva} compounds are less common, and formally indistinguishable from the previous classes, but semantically intrinsically collectives (see § \ref{sec:collective} for collective derivations, some of which also combine several nominal roots). Examples include \japhug{cʰɤmtʰɯm}{food and drinks} from \japhug{cʰa}{alcohol} and \japhug{tɤ-mtʰɯm}{meat}, \japhug{sŋiɕɤr}{night and day} from \japhug{tɯ-sŋi}{one day} and \japhug{ɕɤr}{night} and \japhug{χcʰoʁe}{right and left} from \japhug{χcʰa}{right} and \japhug{ʁe}{left}, the latter two being mainly used as adverbs.


\subsection{Verb-Verb compounds} \label{sec.v.v.compounds}
There are two types of Verb-Verb nominal compounds in Japhug, action nominals (involving transitive action verbs) and degree nouns (with adjectival stative verbs).

\subsubsection{Action nominals} \label{sec.v.v.compounds.action}
Action nominals made of two verb roots are rare in Japhug. The clearest example is \japhug{joʁβzɯr}{tidying up} from   \japhug{joʁ}{raise} and \japhug{βzɯr}{move}, which occurs in a light verb construction as in (\ref{ex:joRBzWr}). The denominal compound verb  \japhug{rɤjoʁβzɯr}{tidy up} has a meaning verb close to this construction (see § XXX).

\begin{exe}
\ex \label{ex:joRBzWr}
 \gll joʁβzɯr tɤ-βzu-t-a \\
 tidying.up \textsc{pfv}-do-\textsc{pst:tr-1sg} \\
 \glt `I did some tidying up.' (elicited)
\end{exe}

At an earlier stage, such compound action nominals may have been common, as is suggested by the existence of denominal compound verbs without corresponding noun, such as \japhug{raχtɯtsɣe}{do commerce} (from \japhug{χtɯ}{buy} and \japhug{ntsɣe}{sell}, § XXX on the \forme{-n-} element).

\subsubsection{Degree nouns} \label{sec.v.v.compounds.degree}
The productive way of building degree nouns in Japhug is by adding the prefix \forme{tɯ-} to an adjectival stative verb (§ XXX), but an alternative formation involves the compounding of two antonymic verbs, such as \japhug{jpumxtsʰɯm}{thickness} from \japhug{jpum}{be thick} and \japhug{xtsʰɯm}{be thin}. All known examples are collected in Table \ref{tab:degree.comp}; note that the first member of these compounds is in \textit{status constructus} if possible.

\begin{table}
\caption{Degree compound nouns} \label{tab:degree.comp}
\begin{tabular}{llll}
 \lsptoprule 
 Compound & First verb & Second verb \\
 \midrule
\japhug{jpumxtsʰɯm}{thickness} (diameter) &\japhug{jpum}{be thick} &\japhug{xtsʰɯm}{be thin} \\
\japhug{jaʁmba}{thickness} (of a sheet)&\japhug{jaʁ}{be thick} &\japhug{mba}{be thin} \\
\japhug{xtɯrɲɟi}{length} &\japhug{xtɯt}{be short} &\japhug{rɲɟi}{be long} \\
\japhug{xtɕɯxte}{size} &\japhug{xtɕi}{be small} &\japhug{wxti}{be be} \\
 \lspbottomrule
\end{tabular}
\end{table}

In the case of \japhug{xtɕɯxte}{size}, the second element \forme{xte} is a variant also found in the derived verb \japhug{mɯxte}{be the majority}, probably a remnant of a former \forme{*i/e} alternation still observed in the verb  \japhug{ɣi}{come} (§ XXX).

These nouns can further derive denominal verbs in \forme{a-} meaning `of uneven X', for instance   \japhug{ajpomxtsʰɯm}{having uneven thickness}  (with unexplained \ipa{a} / \ipa{o} alternation, see § XXX).

\subsection{Adverb-Verb compounds} \label{sec.adv.v.compounds}
Adverb-Verb compounds are relative marginal. Compounds with \japhug{kɯzɣa}{a long time, many times} in \textit{status constructus} \forme{kɯzɣɤ-} followed by a verb are however attested, as \japhug{kɯzɣɤ-ɕar}{searching a long time} from the verb \japhug{ɕar}{search} in (\ref{ex:kWZGACar}). These compounds only occur as objects of the light verb \japhug{βzu}{make}. This construction is studied in more detail in section XXX (see also \citealt[252]{jacques16complementation}).

\begin{exe}
\ex \label{ex:kWZGACar}
\gll kɯ-xtɕi nɯ ɣɯ pjɤ-me tɕe, tɕendɤre rca kɯzɣɤ-ɕar ʑo ɲɤ-βzu-nɯ ri pjɤ-me.\\
\textsc{nmlz}:S/A-be.small \textsc{dem} \textsc{gen} \textsc{ifr}.\textsc{ipfv}-not.exist \textsc{lnk} \textsc{lnk} unexpectedly long.time-search \textsc{emph} \textsc{ifr}-do-\textsc{pl}  \textsc{lnk} \textsc{ifr}.\textsc{ipfv}-not.exist \\
\glt `The (pigeon skin) of the youngest girl was not there, there looked for it a long time but it was not there.' (the flood 2002, 55)
\end{exe}

\subsection{Noun-Verb compounds} \label{sec.n.v.compounds}
Noun-Verb compounds include three main types, Subject-Verb, Object-Verb and Adjunct-Verb compounds. Participles or other nominalized verbs forms are treated in sections \ref{sec.n.n.compounds}, but criteria to discriminate between ambiguous forms in cases of homophony between noun and verb are provided in \ref{sec:object.verb.compounds}.

\subsubsection{Subject-Verb compounds} \label{sec:subject.verb.compounds}
Subject-Verb compounds exclusively occur with intransitive verbs, mainly adjectival stative verbs. The noun is generally in \textit{status constructus} (see Table \ref{tab:subj.v.compounds} below). Three types are found depending on their meaning: \textit{karmadhāraya}, \textit{bahuvrīhi} or action nominals.

 Compounds of the \textit{karmadhāraya} type are equivalent to a participial relative with the intransitive subject as the relativized element. If the nominal and verbal elements are represented as N and V respectively, a \textit{karmadhāraya} NV compound means  `N which Vs'. They  are common with stative verbs of colour such as \japhug{ɲaʁ}{be black} or \japhug{wɣrum}{be white} as in Table \ref{tab:subj.v.compounds}. 
 
\begin{table}
\caption{Examples of Subject-Verb  \textit{karmadhāraya} compound nouns} \label{tab:subj.v.compounds}
\begin{tabular}{llllll}
\lsptoprule
 Compound& Base Noun & Verb\\
 \midrule
\japhug{tɤɕɤɲaʁ}{black barley} & \japhug{tɤɕi}{barley} & \japhug{ɲaʁ}{be black} \\
\japhug{tɤɕɤɣrum}{white barley} & & \japhug{wɣrum}{be white}  \\
\japhug{mtsʰalɤɲaʁ}{black nettle} & \japhug{mtsʰalu}{nettle} & \japhug{ɲaʁ}{be black} \\
\japhug{mtsʰalɤɣrum}{white nettle} & & \japhug{wɣrum}{be white}  \\
\japhug{qartsɯɲaʁ}{cold winter} & \japhug{qartsɯ}{winter} & \japhug{ɲaʁ}{be black} \\
\japhug{pɣɤɲaʁ}{Pucrasia macrolopha} & \japhug{pɣa}{bird} &   \\
\japhug{tɤmtɯɲaʁ}{deadlock} & \japhug{tɤ-mtɯ}{knot} &   \\
\lspbottomrule
\end{tabular}
\end{table}

The compounds in Table \ref{tab:subj.v.compounds} are highly lexicalized; in the case for instance of \japhug{pɣɤɲaʁ}{Pucrasia macrolopha}, this bird is not even black as the speakers themselves point out (\ref{ex:pGAYaR}).

\begin{exe}
\ex \label{ex:pGAYaR}
 \gll pɣɤɲaʁ kɤ-ti ci tu tɕe, nɯnɯ ʁo lɯski li nɯ pɣa ŋu, tɕeri mɤ-ɲaʁ ma ɲɯ-mpɕɤr. ɯ-muj nɯra wuma ʑo ɲɯ-mpɕɤr qhe kɯ-tu ra ɲɯ-nɤmbju ʑo \\
 Pucrasia.macrolopha \textsc{nmlz}:P-say \textsc{indef} exist:\textsc{fact} \textsc{lnk} \textsc{dem} \textsc{advers} of.course again \textsc{dem} bird be:\textsc{fact} but \textsc{neg}-be.black:\textsc{fact} \textsc{lnk} \textsc{sens}-be.beautiful \textsc{3sg.poss}-feather \textsc{dem:pl} really \textsc{emph}   \textsc{sens}-be.beautiful \textsc{lnk} \textsc{nmlz}:S/A-exist \textsc{pl} \textsc{sens}-be.brilliant \textsc{emph} \\
 \glt `The Pucrasia macrolopha is of course also a bird (like the previous ones we talked about), but it is not black, it is beautiful, its feathers are very beautiful and those that are there (visible) are iridescent.'
\end{exe}

More complex NV compounds of this type are found, such as \japhug{tɯ-jaʁndzumɤpaχcɤl}{middle finger} from \japhug{tɯ-jaʁndzu}{finger} and \japhug{mɤpaχcɤl}{be in the middle} (itself a denominal verb from \japhug{ɯ-χcɤl}{middle}).

In such compounds, some stative verbs occur with a \forme{-x-} element in individual forms. This is the case of \japhug{tɤlɤxcʰi}{fresh milk} from \japhug{tɤ-lu}{milk} and \japhug{cʰi}{be sweet}. It is possible that this velar fricative represents the remnant of a participle prefix \forme{kɯ-}; note however its presence in some derivations, in particular causative and tropative (\japhug{nɤxcʰi}{to find sweet}, § XXX).
 
\textit{Bahuvrīhi} compounds are equivalent to a participial relative with the possessor of the subject as the relativized element (section XXX); In other words, a \textit{bahuvrīhi} NV means `(person/animal/entity) whose N Vs'. Examples are considerably fewer than \textit{karmadhāraya}-s, but include \japhug{ɕnɤsti}{person with a stuffy nose} (from \japhug{tɯ-ɕna}{nose} and \japhug{asti}{be blocked}, see the discussion in \ref{sec:object.verb.compounds}) and  \japhug{ɕnaβndʑɣi}{snotty-nosed kid} (from \japhug{tɯ-ɕnaβ}{snot} and a verbal root  \forme{-ndʑɣi} attested in \japhug{nɤndʑɣi}{have (snot)}). 
 
A particularly interesting Noun-Verb \textit{bahuvrīhi} is the plant name \japhug{kɯngɯttɤrqʰɤɴɢaʁ}{unidentified plant}, which comprises three elements: the numeral \japhug{kɯngɯt}{nine}, the IPN \japhug{tɤ-rqʰu}{skin, hull} and the intransitive verb \japhug{ɴɢaʁ}{peel, shed skin} (anticausative of \japhug{qaʁ}{peel}, see § XXX). This compound is to be parsed [\forme{kɯngɯt-tɤrqʰɤ-}][\forme{ɴɢaʁ}] from a morphological point of view, as its meaning is `(plant) whose nine skins shed off' as is explained in the text excerpt (\ref{ex:kWngWttArqhANGaR}): the first element \forme{kɯngɯt-tɤrqʰɤ-}\footnote{Note that this compound has Numeral-Noun order as in other examples (see \ref{sec:bahuvrihi.n.n}).} corresponds to the intransitive subject of \japhug{ɴɢaʁ}{peel, shed skin}. 

\begin{exe}
\ex \label{ex:kWngWttArqhANGaR}
\gll kɯngɯttɤrqʰɤɴɢaʁ ɯ-rmi kɯra nɯnɯ tɕendɤre, ɯ-rqʰu kɯ-dɯ\redp{}dɤn ʑo pjɯ-ɴɢaʁ ɲɯ-ŋu. nɯnɯ tɯ-mpɕar nɤ tɯ-mpɕar, tɯ-mpɕar nɤ tɯ-mpɕar, pɯ-ɴɢaʁ qʰe ɯ-ŋgɯ li mɤʑɯ ɲɯ-βze qʰe, \\
plant.name \textsc{3sg.poss}-name \textsc{dem:prox:pl} \textsc{dem} \textsc{lnk} \textsc{3sg.poss}-skin \textsc{nmlz}:S/A-\textsc{emph}\redp{}be.many \textsc{emph} \textsc{ipfv}-\textsc{anticaus}:peel \textsc{sens}-be \textsc{dem} one-leaf \textsc{lnk}  one-leaf  one-leaf \textsc{lnk}  one-leaf \textsc{pfv}-\textsc{anticaus}:peel \textsc{lnk} \textsc{3sg.poss}-inside again yet \textsc{ipfv}-grow \textsc{lnk}  \\
\glt `As for the name of the \forme{kɯngɯttɤrqʰɤɴɢaʁ}, (it is because) it has a lot of skins that shed off, one after the other, and after one has shed off, another one grows again inside.' (14-sWNgWJu, 72-5)
\end{exe}

Yet, from a phonological point of view, the form should rather be parsed as [\forme{kɯngɯt-}][\forme{tɤrqʰɤ-ɴɢaʁ}], as the phonological integration between \forme{tɤrqʰɤ-} in \textit{status constructus} and the following verb root is stronger than that between the numeral \japhug{kɯngɯt}{nine} and the rest, as shown by the preservation of the final \forme{-t} (§ XXX on internal sandhi).

Action nominals NV compounds are rare with intransitive verbs. Examples include \japhug{pɣɤmbri}{bird song} from \japhug{pɣa}{bird} and the intransitive \japhug{mbri}{cry, sing} or \japhug{snɯɲaʁ}{harming people} from  \japhug{tɯ-sni}{heart} and \japhug{ɲaʁ}{be black}, a compound serving as the base of many denominal verbs (§ XXX).

 
\subsubsection{Object-Verb compounds} \label{sec:object.verb.compounds}
Object-Verb nominal compounds in Japhug are very productive, and can be classified into two main types: actor OV compounds, and action OV compounds.

Actor OV compounds are common in names of trades, animals and even plants, such as \japhug{rŋɯlfɕi}{silversmith}, \japhug{zrɯɣndza}{praying mantis} and \japhug{tɤtɕɯβraʁ}{burdock}. The first of these examples, from the Tibetan loanword \japhug{rŋɯl}{silver} and the labile verb \japhug{fɕi}{forge}, requires little explanation, but some compounds of this type do not make much sense without some cultural background; as an illustration of how the Japhug corpus can be used to better understand the origin of these compounds, I discuss below the latter two nouns.

The compound \japhug{zrɯɣndza}{praying mantis} derives from \japhug{zrɯɣ}{louse} and \japhug{ndza}{eat}, and literally means `louse eater', a descriptive term based on the feeding habits of that insect, as described in (\ref{ex:zrWGndza}).

\begin{exe}
\ex \label{ex:zrWGndza}
\gll nɯ ɯ-taʁ ri zrɯɣndza kɤ-ndo-tɕi tɕe, .... tɕendɤre zrɯɣ rcanɯ lɤŋɤtʂɤɣ jamar ʑo ɯ-ɕki kɤ-ta-tɕi. tɕe kɯ-mɤku nɯra tɕe, tɕe zrɯɣ nɯ lonba ʑo cʰɯ-mqlaʁ tɕe tu-ndze ɲɯ-ŋu. tɕendɤre kɯ-maqʰu tɕe nɤki ɲɯ-ŋu,  tɕendɤre ku-nɯni kɯ-fse qhe, ɯ-ŋgɯ nɯnɯ,  ɯ-se nɯ lu-nɯ-tɕɤt qʰe cʰɯ-mqlaʁ ɲɯ-ŋu. \\
\textsc{dem} \textsc{3sg}-on \textsc{loc} praying.mantis \textsc{pfv}-take-\textsc{1du} \textsc{lnk} .... \textsc{lnk} louse unexpectedly five.or.six about \textsc{emph} \textsc{3sg}-\textsc{dat} \textsc{pfv}:\textsc{east}-put-\textsc{1du} \textsc{lnk} \textsc{nmlz}:S/A-be.first \textsc{dem:pl} \textsc{lnk} \textsc{lnk} louse \textsc{dem} \textsc{all} \textsc{emph} \textsc{ipfv}-swallow \textsc{lnk} \textsc{ipfv}-eat[III] \textsc{sens}-be \textsc{lnk} \textsc{nmlz}:S/A-be.after \textsc{lnk} \textsc{dem}:\textsc{cataph} sens-be \textsc{lnk} \textsc{ipfv}-suck[III] \textsc{nmlz}:S/A-be.like \textsc{lnk} \textsc{3sg.poss}-inside \textsc{dem} \textsc{3sg.poss}-blood \textsc{dem} \textsc{ipfv}:\textsc{upstream}-\textsc{auto}-take.out \textsc{lnk}  \textsc{ipfv}-swallow \textsc{sens}-be \\
\glt `(When we were little, one of my classmate had a lot of lice, and) we took a praying mantis (and put it on his clothes), then put five or six lice near it; the first ones, it swallowed them whole, and the following ones,  it did the following, it would kind of suck them, drink the blood inside them, and then swallow it (and then throw them away).' (26-zrWGndza, 25-35)
\end{exe}

The nouns \japhug{tɤtɕɯβraʁ}{burdock} from (\japhug{tɤ-tɕɯ}{son, boy} and \japhug{βraʁ}{attach}) and  \japhug{tɕʰemeβraʁ}{little burdock} (with \japhug{tɕʰeme}{girl} as first element) literally mean `attaching boys/girls'; an explanation for these names from local folklore is provided in (\ref{ex:tAtCWBraR}).

\begin{exe}
\ex \label{ex:tAtCWBraR}
\gll tɤtɕɯβraʁ tɕe, ɕɯ kɯ pa-mto nɯnɯ tɕe tɕe nɯ ɣɯ ɯʑɤɣ maʁ nɤ, ɯ-kʰa ɣɯ maʁ nɤ,  ɯ-kɯmdza kɯ-fse ra ɣɯ, nɯ-tɕɯ maʁ nɤ nɯ-me tu tu-ti-nɯ ɲɯ-ŋu. tɕe nɯ nɯ-kɯmdza kɯ-fse kɯ-ɤrɕɤt ra, nɯ-skʰrɯ mɤ-kɯ-βdi a-pɯ-tu tɕe, ``wo ... ɯ-rɟit tɤ-tɕɯ sci ma tɤtɕɯβraʁ pɯ-mto-t-a" \\
burdock \textsc{lnk} who \textsc{erg}  pfv:3\fl{}3'-see \textsc{dem} \textsc{lnk} \textsc{lnk} \textsc{dem} \textsc{gen} \textsc{3sg:gen} not.be:\textsc{fact} \textsc{lnk} \textsc{3sg.poss}-house \textsc{gen} not.be:\textsc{fact} \textsc{lnk}  \textsc{3sg.poss}-relative \textsc{nmlz}:S/A-be.like \textsc{pl} \textsc{gen} \textsc{3pl.poss}-son not.be:\textsc{fact} \textsc{lnk} \textsc{3pl.poss}-daughter exist:\textsc{fact} \textsc{ipfv}-say-\textsc{pl} \textsc{sens}-be \textsc{lnk} \textsc{dem} \textsc{3pl.poss}-relative  \textsc{nmlz}:S/A-be.like  \textsc{nmlz}:S/A-be.related \textsc{pl}  \textsc{3pl.poss}-body  \textsc{neg}-\textsc{nmlz}:S/A-be.well \textsc{irr}-\textsc{ipfv}-exist \textsc{lnk} \textsc{interj} .... \textsc{3sg.poss}-child \textsc{indef.poss}-son \textsc{lnk} burdock \textsc{pfv}-see-\textsc{tr:pst}-\textsc{1sg} \\
\glt `The burdock, whoever saw it will have a boy or a girl, him or someone from his house or among his relatives. If someone among his relatives is pregnant, he will say `her child will be a boy, as I saw a burdock.'' (26-NalitCaRmbWm, 109+)
\end{exe}

Action OV compounds generally occur in constructions with light verbs such as \japhug{lɤt}{throw, release} or \japhug{βzu}{make}. Examples include \japhug{cʰɤtsʰi}{alcohol drinking} from \japhug{cʰa}{alcohol} and \japhug{tsʰi}{drink} (compare with the denominal verb  \japhug{ɣɯcʰɤtsʰi}{drink too much alcohol}, section XXX) or \japhug{ʁrɯrpu}{hitting with horns} (not goring) from \japhug{ta-ʁrɯ}{horn} and \japhug{rpu}{bump into}  (see also \japhug{nɯʁrɯrpu}{hit with horns}).\footnote{The object of \japhug{rpu}{bump into} is the body part bumping into something, § XXX.} A complete list of such compounds is provided in section XXX, together with corresponding denominal quasi-incorporating verbs, which are more common in texts. 

These nouns mainly occur with light verbs such as \japhug{lɤt}{throw, release} as in example (\ref{ex:RrWrpu}) (§ XXX), but are also found in other constructions as in (\ref{ex:chAtshi.koGAtChom}), where a free object \japhug{cʰa}{alcohol} with the bare infinitive \forme{ɯ-tsʰi} could also be used (see § XXX).

\begin{exe}
\ex \label{ex:RrWrpu}
 \gll jla kɯ a-taʁ ʁrɯrpu ta-lɤt \\
 hybrid.yak \textsc{erg} \textsc{1sg}-on hitting.with.horns \textsc{pfv}:3\fl{}3'-throw \\
 \glt `The hybrid yak hit me with his horn.' (elicited)
\end{exe}

\begin{exe}
\ex \label{ex:chAtshi.koGAtChom}
 \gll cʰɤtsʰi ko-ɣɤ-tɕʰom tɕe  \\
 alochol.drinking \textsc{ifr}-\textsc{caus}-be.too.much \textsc{lnk} \\
\glt `He had drunk too much alcohol.' (150829 jidian-zh, 16)
\end{exe}
Not all compounds whose second element originates from a transitive verb are Object-Verb (or Adjunct-Verb) compounds. Two potentially ambiguous cases must be pointed out. 

First, there are Noun-Noun compounds whose second element is an IPN deriving from a transitive verb (see § XXX), but which loses its possessive prefix as is regular in compounding (\ref{sec:possessive.prefix.second.compounds}). In such cases the resulting Noun-Noun compound is not formally distinguishable from an Noun-Verb compound, and only the meaning can be used to differentiate between the two classes. For instance, the plant name \japhug{tʂɤɕpʰɤt}{plantain} has the \textit{status constructus} of \japhug{tʂu}{path} as a first element, while its second part \forme{-ɕpʰɤt} can be interpreted as either directly from the verb \japhug{ɕpʰɤt}{patch} (`road patcher') or from the derived noun \japhug{tɤ-ɕpʰɤt}{patch (n)} (a piece of fabric used to patch worn clothes) (`road patch'). In this particular case, the second interpretation is more likely, and hence \japhug{tʂɤɕpʰɤt}{plantain} is better analyzed as a Noun-Noun compound.

Second, when the second element of a Noun-Verb compound is a \forme{a-} passive verb (see § XXX), the \forme{a-/ɤ-} prefix is absorbed by the first element of the compound and becomes invisible. In the resulting form, the second element superficially looks similar to the transitive verb. For instance, the noun \japhug{ɕnɤsti}{person with a stuffy nose} appears to derive from \japhug{tɯ-ɕna}{nose} and the transitive verb  \japhug{sti}{block}. However, semantics rules out such a derivation: the compound is a \textit{bahuvrīhi} literally meaning `whose nose is blocked' (see \ref{sec:subject.verb.compounds}), and cannot be interpreted as `(person) blocking noses', the expected meaning of an Object-Verb compound. Since the passive \japhug{asti}{be blocked} of  \japhug{sti}{block} is well-attested, as shown by (\ref{ex:pjAkAstici}), it is better to analyze \japhug{ɕnɤsti}{person with a stuffy nose} as a Subject-Verb \textit{bahuvrīhi} compound (\ref{sec:subject.verb.compounds}) derived from that passive form.

\begin{exe}
\ex \label{ex:pjAkAstici}
\gll maka ɲɯ́-wɣ-ɕɯɣ-mu mɯ-pjɤ-cʰa ma mɯ-pjɤ-mtsʰɤm matɕi ɯ-rna pjɤ-k-ɤ-sti-ci. \\
at.all \textsc{ipfv}-\textsc{inv}-\textsc{caus}-be.afraid \textsc{neg}-\textsc{ifr.ipfv}-can \textsc{lnk} \textsc{neg}-\textsc{ifr}-hear because \textsc{3sg.poss}-ear \textsc{ifr.ipfv}-\textsc{evd}-\textsc{pass}-block-\textsc{evd} \\
\glt `The noise could not frighten him, as he did not hear it, because his ears were blocked.' (140514 huishuohua de niao, 203)
\end{exe}

\subsubsection{Adjunct-Verb compounds} \label{sec:adjunct.verb.compounds}
Adjunct-Verb compounds are all action nouns, the nominal element being either locative or instrument. Like other action nominal compounds, they can undergo denominal derivation to become quasi-incorporating verb (§ XXX).  

Adjunct-Verb compounds can be made from transitive verbs, as \japhug{zgrɯtɕʰɯ}{nudge} and \japhug{kɤtɕʰɯ}{headbutt} with the verb \japhug{tɕʰɯ}{gore} as second element and the body parts  \japhug{tɯ-zgrɯ}{elbow} and  \japhug{tɯ-ku}{head} as first element. Here the body parts are instrumental adjuncts, as the object of \japhug{tɕʰɯ}{gore} is the person being gored/hit, not the part of the body one uses. These nouns occur with the light verb \japhug{lɤt}{throw, release} as in (\ref{ex:zgrWtChW}). 

\begin{exe}
\ex \label{ex:zgrWtChW}
\gll zgrɯtɕʰɯ tɤ-lat-a \\
nudge \textsc{pfv}-throw-\textsc{1sg} \\
\glt `I nudged (him).' (elicited)
\end{exe}

Examples from intransitive verbs are also attested, as \japhug{qʰaru}{look back} from the locative noun \japhug{ɯ-qʰu}{back} and the semi-transitive \japhug{ru}{look at}. It is also used with  \japhug{lɤt}{throw, release} as in (\ref{ex:qharu}).

\begin{exe}
\ex \label{ex:qharu}
\gll ɯʑo nɯ  tatpa ta-ta ma qʰaru mucin ʑo mɯ-pa-lɤt nɤ tɤ-ari ɲɯ-ŋu. \\
\textsc{3sg} \textsc{dem} faith \textsc{pfv}:3\fl{}3'-put \textsc{lnk} look.back at.all emph \textsc{neg}-\textsc{pfv}:3\fl{}3'-throw \textsc{lnk} \textsc{pfv}:\textsc{up}-go[II] \textsc{sens}-be \\
\glt `He had faith, did not look back at all and (succeeded in) going up to (the abode of the goads). (Norbzang, 129)
\end{exe}

The quasi-incorporating verbs \japhug{sɯzgrɯtɕʰɯ}{nudge}, \japhug{nɤkɤtɕʰɯ}{nudge}, and \japhug{nɤqʰaru}{look back} are considerably more common that light verb constructions with compound action nouns such as (\ref{ex:zgrWtChW}) and (\ref{ex:qharu}).

There is a very productive Noun-Verb compound formation with the noun \japhug{kʰramba}{lie} as first element, meaning `pretending to do  X', compatible with both transitive and intransitive verbs. It occurs in a light verb construction with \japhug{βzu}{make} as in (\ref{ex:khrambatshi}). This construction is studied in more detail in section XXX (see also \citealt[252]{jacques16complementation}).

\begin{exe}
\ex \label{ex:khrambatshi}
\gll ʑara kɯ cʰa nɯ kʰramba-tsʰi ka-βzu-nɯ,  \\
\textsc{3pl} \textsc{erg} alcohol \textsc{dem} lie-drink \textsc{pfv}:3\fl{}3'-make-\textsc{pl} \\
\glt `They pretended to drink alcohol.' (Norbzang, 100)
\end{exe}

The compound \japhug{mɲaʁmtsaʁ}{grasshoper} from \japhug{tɯ-mɲaʁ}{eye} and \japhug{mtsaʁ}{jump} is obscure, but unlikely to be a \textit{bahuvrīhi} `whose eyes jump', and should rather be analyzed as an adjunct compound (maybe `jumping with (big) eyes', as if from a comitative adverb \ref{sec:comitative.adverb}). 

\subsection{Verb-Noun compounds} \label{sec.v.n.compounds}
Verb-Noun compounds are extremely rare in Japhug, as they are in general in Trans-Himalayan languages other than Chinese.  

Adjectival stative verbs nearly always occur as second element in compounds with a noun (\ref{sec:subject.verb.compounds}), but the opposite order is attested in \japhug{sɤŋɤβdi}{disagreeable smell} from \japhug{tɤ-di}{smell} and \japhug{sɤŋɤβ}{be disagreeable},\footnote{The meaning of this verb is difficult to render exactly in English; the best approximation would be `which does not tempt one' (French `qui ne donne pas envie'); it can be used with an infinite complement (§ XXX). It is of denominal origin (§ XXX), from the same nominal root as the transitive \japhug{nɤŋɤβ}{be embarrassed to} (Chinese \zh{觉得不好意思}). }  a noun which can occur with the intransitive verb \japhug{mnɤm}{smell} as in (\ref{ex:sANABdi}).

\begin{exe}
\ex \label{ex:sANABdi}
\gll sɤŋɤβ-di ʑo ɲɯ-mnɤm \\
disagreeable-smell \textsc{emph} \textsc{sens}-smell \\
\glt `There is a disagreeable smell.' (elicited)
\end{exe}
A possible example of Verb-Noun compound with a transitive verb is \japhug{ndzɤpri}{brown bear}, comprising \japhug{pri}{bear} and \japhug{ndza}{eat} -- as shown by (\ref{ex:ndzApri}) from a text about bears, it is considered by some native speakers of Japhug as a man eater, though this explanation could be folk-etymology. Note that this compound is also anomalous in that when transitive verbs are used in compounds with a noun, that noun is either an object (\ref{sec:object.verb.compounds}) or adjunct (\ref{sec:adjunct.verb.compounds}), never the subject.

\begin{exe}
\ex \label{ex:ndzApri}
\gll tɕe ndzɤpri kɤ-ti nɯ tɕe tɯrme tu-kɯ-ndza ɲɯ-ŋgrɤl \\
\textsc{lnk} brown.bear \textsc{inf}-say \textsc{dem} \textsc{lnk} people \textsc{ipfv}-\textsc{genr}:S/P-eat \textsc{sens}-be.usually.the.case \\
\glt `It eats people, so is it called \forme{ndzɤpri}.' (21-pri, 94)
\end{exe} 

We find several examples of nominal compounds whose structure is \forme{tɤ-}+Verb+Noun, where the verb is an adjectival stative verb. This category includes \japhug{tɤqiaβjmɤɣ}{lactarius sp.}, literally `bitter mushroom', from the noun \japhug{tɤjmɤɣ}{mushroom} see \ref{sec:frozen.indef} concerning the lost of \forme{tɤ-}) and the verb \japhug{qiaβ}{be bitter}, or \japhug{tɤmbextsa}{type of shoes} from \japhug{tɯ-xtsa}{shoe} and \japhug{mbe}{be old}. These should not be analyzed as Verb-Noun compounds however, as the first element originates from a nominalized form of the verb (such as \japhug{tɤ-mbe}{old thing}, see § \ref{sec:property.nouns} and § XXX on this derivation): they rather are a subtype of Noun-Noun compounds.

The same applies to compounds whose first element comes from a participle, such as \japhug{kɤrŋijmɤɣ}{type of mushroom} from \japhug{tɤjmɤɣ}{mushroom} with the S-participle \japhug{kɯ-ɤrŋi}{green one} \ipa{kɤrŋi} from the verb \japhug{arŋi}{be green}. Note that the compounding order is unexpected, as participles of adjectival stative verbs generally following the noun (§ XXX). Note that compounds with participles of transitive verbs as first element are also found, as for instance \japhug{kɯqurʑŋgri}{evening star}, literally `the star of the helper', for reasons explained in the following text (\ref{ex:kWqur.ZNgri}).

\begin{exe}
\ex \label{ex:kWqur.ZNgri}
\gll ɯnɯnɯ kɯɕɯŋgɯ tɕe kɯ-qur ju-kɯ-ɕe tɕe nɯnɯ, mɯ-nɯ-ɬoʁ mɤɕtʂa nɯ tu-kɯ-nɯna mɯ-pjɤ-jɤɣ ɲɯ-ŋu tɕe,  tɕe núndʐa kɯqurʑŋgri tu-sɤrmi-nɯ \\
\textsc{dem} before \textsc{lnk} \textsc{nmlz}:S/A-help \textsc{ipfv}-\textsc{genr}:S/P-go \textsc{lnk} \textsc{dem} \textsc{neg}-\textsc{pfv}:\textsc{west}-come.out until \textsc{dem} \textsc{ipfv}-\textsc{genr}:S/P-rest \textsc{neg}-\textsc{ifr}.\textsc{ipfv}-be.possible \textsc{sens}-be \textsc{lnk} \textsc{lnk} for.this.reason evening.star \textsc{ipfv}-call-\textsc{pl} \\
\glt `Long ago, when one would go helping, one was not supposed to rest until it comes out, and for this reason it was called `star of the helper'.' (29-mWBZi, 62)
\end{exe}

 

\section{Noun class prefixes} \label{sec:class.prefixes}
Noun class prefixes are prefixal elements that occur in some nouns, whose root cannot occur on its own, except for a few rare exceptions (such as \japhug{qapɣɤmtɯmtɯ}{hoopoe} discussed in \ref{sec:uvular.animal}). Uvular \forme{qa-/χ-/ʁ-} and velar \forme{kɯ-/x-/ɣ-} prefixes are attested, and occur on animal names, plant names and nouns referring to traditional objects. Additional body part class prefixes, in particular \forme{m-} are also present in Japhug.

Dental prefixal elements such as \forme{tɤ-} or \forme{tɯ-} are very common, but  are  better interpreted as frozen indefinite possessor prefixes (see \ref{sec:frozen.indef}), rather as noun class prefixes.

\subsection{Uvular animal name prefix} \label{sec:uvular.animal}
The uvular animal prefix has a plene form \forme{qa-} (Table \ref{tab:animal.qa}) and a reduced allomorph \forme{χ-/ʁ-}, attested in a few names like \japhug{ʁmbroŋ}{wild yak}, \japhug{rtɕʰɯrjɯ}{caterpillar} and \japhug{tɕʰɯχpri}{salamander}.

Note that \japhug{ʁmbroŋ}{wild yak} is a borrowing from Tibetan \tibet{འབྲོང་}{ⁿbroŋ}{wild yak}, a fact that possibly suggests that the \forme{χ-/ʁ-} prefix has some degree of productivity (see \citealt{jacques14snom}). 

The noun \japhug{qapɣɤmtɯmtɯ}{hoopoe} is clearly a compound containing the \textit{status constructus} of \japhug{pɣa}{bird} and the reduplicated form of the noun \japhug{ɯ-mtɯ}{crest}, to which the class prefix \forme{qa-} has been added. 

The allomorph \forme{qa-} is reduced to its non-syllabic variants \forme{χ-/ʁ-} when the prefixed noun occurs as second member of compound. The nouns \japhug{tɕʰɯχpri}{salamander} and \japhug{rtɕʰɯrjɯ}{caterpillar} are examples of this reduction. The former is a compound of \forme{tɕʰɯ-} (a syllable borrowed  from Tibetan \tibet{ཆུ་}{tɕʰu}{water}) and \forme{-χpri}, a variant of \japhug{qapri}{snake}. The latter comprises the syllable \forme{rtɕʰɯ-}, \textit{status constructus} of the unprefixed root of \japhug{tɯrtɕʰi}{type of vegetable (\zh{酸酸菜})}, and the second \forme{-ʁjɯ} is the reduced variant of \japhug{qajɯ}{worm}.

\begin{table}
\caption{Animal name \forme{qa-} prefix} \label{tab:animal.qa}
\begin{tabular}{l|l}
 \lsptoprule 
\japhug{qacʰɣa}{fox} &	\japhug{qandʐe}{earthworm} \\
\japhug{qaɕɣi}{big fly} &	\japhug{qandʐi}{anadromous fish} \\
\japhug{qaɕpa}{frog} &	\japhug{qandʑɣi}{fox} \\
\japhug{qajdo}{crow} &	\japhug{qaɲi}{mole} \\
\japhug{qajtʂʰa}{aegyptius monachus} &	\japhug{qapar}{dhole} \\
\japhug{qajɯ}{worm} &	\japhug{qapɣɤmtɯmtɯ}{hoopoe} \\
\japhug{qaɟy}{fish} &	\japhug{qapri}{fox} \\
\japhug{qala}{rabbit} &	\japhug{qarma}{crossoptilon} \\
\japhug{qaliaʁ}{eagle} &	\japhug{qartsʰaz}{deer} \\
\japhug{qambalɯla}{butterfly} &	\japhug{qartsʰi}{deer} \\
\japhug{qambrɯ}{male yak} &	\japhug{qaʑo}{sheep} \\
\japhug{qamtɕɯr}{shrew} &	\\
%\japhug{qacʰɣa}{fox} \\
%\japhug{qaɕɣi}{big fly} \\
%\japhug{qaɕpa}{frog} \\
%\japhug{qajdo}{crow} \\
%\japhug{qajtʂʰa}{aegyptius monachus} \\
%\japhug{qajɯ}{worm} \\
%\japhug{qaɟy}{fish} \\
%\japhug{qala}{rabbit} \\
%\japhug{qaliaʁ}{eagle} \\
%\japhug{qambalɯla}{butterfly} \\
%\japhug{qambrɯ}{male yak} \\
%\japhug{qamtɕɯr}{shrew} \\
%\japhug{qandʐe}{earthworm} \\
%\japhug{qandʐi}{anadromous fish} \\
%\japhug{qandʑɣi}{fox} \\
%\japhug{qaɲi}{mole} \\
%\japhug{qapar}{dhole} \\
% \japhug{qapɣɤmtɯmtɯ}{hoopoe} \\
%\japhug{qapri}{fox} \\
%\japhug{qarma}{crossoptilon} \\
%\japhug{qartsʰaz}{deer} \\
%\japhug{qartsʰi}{deer} \\
%\japhug{qaʑo}{sheep} \\
 \lspbottomrule
\end{tabular}
\end{table}

\subsection{Velar animal name prefix} 
While most nouns beginning in \forme{kɯ-} are frozen participles (see § XXX), there is a residue of forms which cannot be accounted as deverbal nouns. Table \ref{tab:animal.kW} presents animal names that are not derivable from any verb root, and appear to bear a \forme{kɯ-} class prefix, distinct from the uvular one.
 
\begin{table}
\caption{Animal name \forme{kɯ-} prefix} \label{tab:animal.kW}
\begin{tabular}{ll}
 \lsptoprule 
\japhug{kɯɕpaz}{marmot} \\
\japhug{kɯjka}{pyrrhocorax} \\
\japhug{kɯmu}{tetraogallus tibetanus} \\
\japhug{kɯpɤz}{type of bug} \\
\japhug{kɯrtsɤɣ}{snow leopard} \\
\japhug{kɯrŋi}{beast} \\
\japhug{kɯrnɯ}{mite} \\
 \lspbottomrule
\end{tabular}
\end{table} 

There is a handful of nouns with reduced allomorphs \forme{ɣ-}, \forme{x-} or even metathesized as \forme{βɣ-} in some words, corresponding to \forme{kə-} in Situ (see the phonological discussion in \citealt[6]{jacques14antipassive}), including \japhug{xɕiri}{weasel}, \japhug{xtɯt}{wild cat},  \japhug{ɣzɯ}{monkey}, \japhug{ɣni}{flying squirrel}, \japhug{βɣɯz}{badger} and \japhug{βɣɤza}{fly}.

\subsection{Uvular plant name prefix} \label{sec:uvular.plant}
Quite a number of plant names have a uvular class prefix \forme{qa-}, including both cultivated and wild plants (and even plant parts), such as \japhug{qaɕti}{peach}, \japhug{qaɟɤɣi}{oat}, \japhug{qampʰoʁ}{oak leaves},  \japhug{qandzi}{type of fir}, \japhug{qaʑmbri}{vine}, \japhug{qawɯz}{edelweiss} and many others.
 
\subsection{Other uses of the uvular class prefix} \label{sec:uvular.other}
In addition to animal, plant and body part names, the class prefix \forme{qa-} appears on some tools (\japhug{qajo}{earthen pot}, \japhug{qase}{leather rope}, \japhug{qarɤt}{rake}, 
\japhug{qapi}{flint stone}), names of periods of the year (\japhug{qartsɯ}{winter}, \japhug{qartsɤβ}{harvest}), materials (\japhug{qandʑi}{tin}, \japhug{qambɯt}{sand}) or natural forces like \japhug{qale}{wind}.

The reduced form \forme{ʁ-} of the class prefix occurs with the noun \japhug{qale}{wind} in some compounds such as \japhug{akɯcʰoʁle}{north/east wind} and the abstract IPN \japhug{ɯ-ʁle}{reputation} (and the verbs derived from it, such as \japhug{raʁle}{be polite}).
  
\subsection{Body part noun prefixes}  \label{ex:body.part.prefix}
The identification of class prefixes in body parts mainly rests on comparative evidence. Other Trans-Himalayan languages that preserve clusters such as Tibetan have in some names for body parts cluster that do not match those found in Japhug, for instance \tibet{མཁྲིས་པ་}{mkʰris.pa}{bile} and \tibet{སྐེ་}{ske}{neck} corresponding to the Japhug IPNs \japhug{tɯ-ɕkrɯt}{bile} and \japhug{tɯ-mke}{neck} respectively (see § \ref{sec:body.part}), suggesting that body part class prefixes such as \forme{ɕ-} and \forme{m-} have been added to these words in Gyalrongic and Tibetan independently.

Apart from the \forme{m-} and \forme{ɕ-/ʑ-} class prefixes, some APN body parts such as \japhug{qambɣo}{earwax} have a \forme{qa-}  prefix (\ref{sec:body.part}).

The only evidence for a derivational use of these class prefixes in Japhug is the noun \japhug{tɯ-mci}{saliva}, that may be derived from \japhug{tɯ-ci}{water} by addition of the \forme{m-} class prefix.\footnote{The noun \japhug{tɯ-mgɯr}{back} could be another example, but the verb root from which it could be derivable, \japhug{fkur}{carry on the back}, is likely to be a Tibetan loanword and has a different vocalism. } Given the fact that \japhug{tɯ-ci}{water} is a lexical innovation that is not even shared by Stau and Khroskyabs (see \ref{sec:earth.IPN}), this suggests that the class prefix \forme{m-} may have remained productive even relatively recently.

\section{Nominal derivations}
Nominal derivations pale in comparison of the rich verbal (§ XXX) and even ideophonic (§ XXX) derivations in Japhug. There is little derivational prefixation in nouns (aside from the collective \forme{kɤndʑɯ-} prefix and derivational uses of class prefixes, as seen in \ref{sec:class.prefixes} above), and nearly all of the suffixes or quasi-suffixes involved in these derivations are traceable to inalienably possessed nouns that are still attested in the language, and have thus nearly no antiquity. 

\subsection{Privative} \label{sec:privative}
The suffix \forme{-lu} can be combined with the \textit{status constructus} form of body part nouns, without possessive prefix, to derive a noun meaning `...less', `without ...' that can be used as a modifier (\ref{sec:unpossessible.nouns}). Examples attested in the  corpus are indicated in Table \ref{tab:privative.lu}, but this derivation appears to be productive.

\begin{table}
\caption{Privative \forme{-lu} suffix} \label{tab:privative.lu}
\begin{tabular}{l|l}
 \lsptoprule 
\japhug{ta-ʁrɯ}{horn} &\japhug{ʁrɯlu}{hornless} \\
\japhug{tɤ-jme}{tail} &\japhug{jmɤlu}{without tail}  \\
\japhug{tɯ-jaʁ}{hand} &\japhug{jaʁlu}{missing a hand} \\
\japhug{tɯ-ku}{head} &\japhug{kɤlu}{headless} \\
 \lspbottomrule
\end{tabular}
\end{table}

These privative forms can be used as modifiers of other nouns, and are placed after the nouns and before determiners such as demonstratives or numerals, as in (\ref{ex:RrWlu}) and (\ref{ex:jmAlu}).

\begin{exe}
\ex \label{ex:RrWlu}
\gll ʑɤni ɣɯ ftsoʁ ʁrɯlu ci ta-rku-nɯ ɲɯ-ŋu \\
\textsc{3du} \textsc{gen} female.hybrid.yak hornless \textsc{indef} \textsc{pfv}:3\fl3'-put.in-\textsc{pl} \textsc{sens}-be \\
\glt `They gave them a hornless female yak (to take with them back to the husband's home.' (2005-stod, 243)
\end{exe}

Privative nouns are systematically glossed in Japhug with possessor participial relatives in \japhug{kɯ-me}{not having} (see § XXX), as in (\ref{ex:jmAlu}) (see also example \ref{ex:kAlu} from section \ref{sec:karmadharaya.n.n})
 

\begin{exe}
\ex \label{ex:jmAlu}
\gll tɕe kɯju jmɤlu nɯnɯ tɯrme ɲɯ-ŋu, ɯ-jme kɯ-me nɯ tɕe, tɕe kɯju jmɤlu nɯnɯ ɲɯ-sɲu ɕti tɕe nɯ  nɯ-sɲu tɕe tɕe iɕqʰa tɯ-rɣi cʰɯ-kɯ-χtɤr nɯ nɯ-kɯ-sɲu tu-sɤrmi-nɯ. \\
\textsc{lnk} animal tailless \textsc{dem} man \textsc{sens}-be \textsc{3sg.poss}-tail \textsc{nmlz}:S/A-not.exist \textsc{dem} \textsc{lnk} \textsc{lnk} animal tailless  \textsc{dem} \textsc{ipfv}-be.crazy  be:\textsc{affirm}:\textsc{fact} \textsc{lnk} \textsc{dem} \textsc{pfv}-be.crazy \textsc{lnk} \textsc{lnk} the.aforementioned \textsc{indef.poss}-seed \textsc{ipfv}-\textsc{nmlz}:S/A-spread \textsc{dem} \textsc{pfv}-\textsc{nmlz}:S/A-be.crazy \textsc{ipfv}-call-\textsc{pl} \\
\glt `The `tailless animal' is the man, and `he becomes crazy', (when the crow say) that (people) became crazy, it means that they are sowing seeds.' (22-qajdo, 47-9)
\end{exe}

\subsection{Diminutive} \label{sec:diminutive}
There are four diminutive formations in Japhug, with the quasi-suffixes \forme{-pɯ}, \forme{-tsa}, \forme{-tɕɯ} and \forme{-li}.

The most productive is the \forme{-pɯ} suffixation. This transparent suffix comes from the noun \japhug{tɤ-pɯ}{offspring, young} (from Tibetan \tibet{བུ་}{bu}{son}). A diminutive formation based on the same noun also exists in Tibetan (\citealt{uray52diminutive},  \citealt[627]{hill14derivational}); whether the diminutive formation was independently innovated, or was borrowed from Tibetan is a question that deserves further investigation. It is also attested in Situ (\citealt{zhang16bragdbar}, \citealt[151]{lai17khroskyabs}).

Earlier diminutives are formed with the\textit{ status constructus} of the noun, for instance \japhug{tɕʰemɤpɯ}{young girl} from \japhug{tɕʰeme}{girl}, \japhug{staχpɯ}{pea} from \japhug{stoʁ}{broad bean}, or \japhug{kʰɯzɤpɯ}{puppy} from a non-attested form \forme{*kʰɯza}, propably itself the \forme{-tsa} diminutive of \japhug{kʰɯna}{dog}, borrowed from a Situ dialect.

More recent diminutives are directly formed with the base form, such as \japhug{qapripɯ}{little serpent}. This formation is extremely productive, and applies to plants, animals and even objects as in (\ref{ex:srWnloR}).

\begin{exe}
\ex \label{ex:srWnloR}
\gll tɕe srɯnloʁ-pɯ ci ɲɤ-kʰo tɕe \\
lnk ring-\textsc{dim} \textsc{indef} \textsc{ifr}-give \textsc{lnk} \\
\glt `He handed him a little ring.' (2011-4-smanmi, 120)
\end{exe}

Suffixation with \forme{-pɯ} is the fused variant of the property noun construction with \japhug{ɯ-pɯ}{little one} described in section \ref{sec:property.nouns}.

A diminutive that is common to all Gyalrongic languages is the suffix \forme{-tsa}/\forme{-za} (Situ \forme{-tsa} or \forme{-za} (\citealt[163]{linxr93jiarongen}), Khroskyabs \forme{-ze} / \forme{-zə} / \forme{-zɑ}, \forme{-tsi} (\citealt[158]{lai17khroskyabs}), Stau \forme{-zə}), found in fossilized forms in nouns such as \japhug{kʰɯtsa}{bowl} and \japhug{βɣɤza}{fly},\footnote{The noun \japhug{βɣɤza}{fly} is cognate to Brag-dbar \forme{kəvɐ̂s}, Khroskyabs \forme{jvɑzɑ́} (\citealt{zhang16bragdbar}, \citealt[156]{lai17khroskyabs}) and originates from proto-Gyalrong \forme{*kpɔs-tsa} (\citealt[53]{jacques08zh}). } but still visible in diminutive forms like \japhug{paʁtsa}{piglet} (from \japhug{paʁ}{pig}). It originates from the noun `son' that is lost in Japhug but still attested in Situ and Khroskyabs (Wobzi \forme{zî} `young man'). 

In Japhug the \forme{-tsa} diminutive is not very productive; it applies to some nouns that already have a \forme{-pɯ} diminutive such as \japhug{stoʁtsa}{name of plant} from \japhug{stoʁ}{broad bean} (besides \japhug{staχpɯ}{pea}).

The third diminutive suffix \forme{-tɕɯ}, like the two preceding ones, originates from a noun meaning `offspring', \japhug{tɤ-tɕɯ}{son}, and requires \textit{status constructus}.

It is  used for animals (\japhug{kumpɣɤtɕɯ}{sparrow} from \japhug{kumpɣa}{fowl}) or inanimate objects (\japhug{kʰɤtɕɯ}{little house} from \japhug{kʰa}{house} or \japhug{lʁɤtɕɯ}{little gunny bag} from \japhug{lʁa}{gunny bag}). It occurs in some lexicalized forms such as \japhug{mbrɯtɕɯ}{knife}.\footnote{The root of this noun is metathesized from \forme{*mbɯr}; its cognates have a \forme{-tsa} diminutive in Situ (Brag-dbar \forme{mbərtsiɛ̄}, \citealt[228]{zhang16bragdbar}) and Khroskyabs (Wobzi \forme{(bərzé}, \citealt[115]{lai17khroskyabs}).}

The suffix \forme{-li} is the least productive of all diminutive formations, and the only that cannot be traced to an existing noun. It appears is \japhug{tɕʰemɤli}{little girl} (a synonym of \japhug{tɕʰemɤpɯ}{little girl}) and in \japhug{rgali}{young cow}.

\subsection{Augmentative} \label{sec:augmentative}
A handful of nouns, some of Tibetan origin, have an augmentation form in \forme{-te}, originally from a property noun \forme{*ɯ-te} `big' (related to the verb \japhug{wxti}{be big}).

Augmentatives include \japhug{tɕɣomte}{cultivated xanthoxylum} (from \japhug{tɕɣom}{xanthoxylum}), \japhug{tɯjite}{big field} (from \japhug{tɯ-ji}{field}, name of several fields in Kamnyu), \japhug{tɕʰɯte}{big river} (from \tibet{ཆུ་}{tɕʰu}{water, river}) and the \textit{bahuvrīhi} \japhug{ŋgute}{person with a big head} (from Tibet \tibet{འགོ་}{ⁿgo}{head, top}, not attested independently).

\subsection{Derogative} \label{sec:derogative}
There are three derogative quasi-suffixes in Japhug, deriving designations of old or broken things: \forme{-do} and \forme{-mbe} `old X' and \forme{-ɴqra} `broken X'. These suffixes are the fused variants of the property nouns \japhug{ɯ-ɴqra}{broken one}, \japhug{ɯ-do}{old one} and \japhug{tɤ-mbe}{old thing}  (see \ref{sec:property.nouns}). 

The suffixes \forme{-do} and \forme{-mbe}, like their corresponding property nouns, differ in that the former occurs with animals and plants (\japhug{nɯŋa-do}{old cow}, \japhug{rɟɤlpu-do}{old king}), while the latter is used for inanimate objects.

 In a few cases, the suffixed noun is in status constructus (as \japhug{kʰɤɴqra}{ruin} from \japhug{kʰa}{house} and \forme{-ɴqra}, or \japhug{kʰɯdo}{old dog} (from \japhug{kʰɯna}{dog} and \forme{-do}, see \ref{sec:reduced.forms.compounds}). When the suffixed noun is an IPN, addition of a derogative suffix does not turn it into a APN, as in \japhug{tɯ-rcɤmbe}{old jacket} from \japhug{tɯ-rcu}{jacket} and \forme{-mbe} (unlike other types of compounds, § XXX).

\subsection{Inhabitant} \label{ex:inhabitant.pW}
The inhabitant suffix \forme{-pɯ} derives from the same noun \japhug{tɤ-pɯ}{offspring, young} from which the diminutive \forme{-pɯ} ultimately originates (see \ref{sec:diminutive}). It is used to derive nouns referring to inhabitants of a certain place, and occurs without \textit{status constructus}. For instance, from the village names of \forme{kɤmɲɯ} (the village whose speech is described in this grammar) and \forme{snarndi} (a village in Tshobdun), one derives \japhug{kɤmɲɯpɯ}{person from Kamnyu} and \japhug{snarndipɯ}{person from Snarndi} (see the text 26-tshubdWnpW in the corpus). Given the high productivity of this derivation, these nouns are not indicated in the dictionary, as it would unnecessarily inflate the number of entries.

Note that alternatively, adding the plural \forme{ra}  to a placename suffices to refer to the inhabitants of that place (§ \ref{sec:place.names}).

\subsection{Gender} \label{sec:gender}
There is no morphological expression of gender in Japhug. For animals, the nouns \japhug{pʰu}{male} and \japhug{mu}{female} (from Tibetan \tibet{ཕོ་}{pʰo}{male} and \tibet{མོ་}{mo}{female}) can be used on their own (as in \ref{ex:phu.mu}) or occur as second member of compounds, as \japhug{kumpɣapʰu}{rooster} and \japhug{kumpɣamu}{hen} from \japhug{kumpɣa}{fowl}, or \japhug{lɯlɤmu}{female cat} from \japhug{lɯlu}{cat}, with \textit{status constructus} of the first noun.

\begin{exe}
\ex \label{ex:phu.mu}
\gll tɤkʰe pɣɤtɕɯ ndɤre pʰu mu saχsɤl \\
stupid bird:\textsc{dim} on.the.other.hand male female be.clear:\textsc{fact} \\
\glt `The male and the female of the `stupid bird', as opposed (to the birds previously discussed), are easy to distinguish.' (23-scuz, 45)
\end{exe}

The suffixes \forme{-pa} and \forme{-mɯ} (from Tibetan \forme{-pa} and \forme{-mo} respectively) also occur for a handful of nouns, some of Tibetan origin (\japhug{srɯnmɯ}{râkshasî} from \tibet{སྲིན་མོ་}{srin.mo}{râkshasî}) but also some local names such as \japhug{ɴɢarpa}{male one quarter yak hybrid}  vs \japhug{ɴɢarmɯ}{female one quarter yak hybrid}.

The noun \japhug{paʁɟu}{boar} from \japhug{paʁ}{pig} has a suffix \forme{-ɟu} that is not found in any other word.

For some domestic animals, a lexical distinction is made between male and female animals (see Table \ref{tab:lexical.gender}).

\begin{table}
\caption{Lexical distinction of male and female animals} \label{tab:lexical.gender}
\begin{tabular}{l|l}
 \lsptoprule 
 Male & Female \\
 \midrule
\japhug{qambrɯ}{male yak} & \japhug{qra}{female yak} \\
\japhug{jla}{male hybrid yak} & \japhug{ftsoʁ}{female hybrid yak} \\
\japhug{mbala}{bull} & \japhug{nɯŋa}{cow}  \\
\japhug{zrɤβ}{he-goat} & (\japhug{tsʰɤnmu}{ewe})  \\
 \lspbottomrule
\end{tabular}
\end{table}

\subsection{Collective} \label{sec:collective}
While Japhug lacks number inflection, there are several four collective derivations: the social relation collective, three reduplicated collectives and the \textit{dvandva} collective.

\subsubsection{Social relation collective}  \label{sec:social.collective}
The first type of collective is a noun prefixed in \forme{kɤndʑɯ-} and built either from kinship or social relation terms (which can be either IPNs or APNs), designating a group of people linked to one another by a specific relation.  

Two distinct types of social relation collectives should be distinguished: reciprocal and non-reciprocal collectives.

Reciprocal collectives (Table \ref{tab:reciprocal.collectives}) are from nouns designating an relationship in which all members of the group call each other by the same term; it can be non-kinship terms like `companion' or `friend' or kinship terms like \japhug{tɤ-sqʰaj}{sister (of a girl)} (on the use of this term see § XXX). 

\begin{table}
\caption{Reciprocal social relation collectives} \label{tab:reciprocal.collectives}
\begin{tabular}{lllllll}
 \lsptoprule 
 Collective & Base noun \\
\midrule
\japhug{kɤndʑɯɣɯfsu}{friends} & \japhug{ɣɯfsu}{friend} \\
\japhug{kɤndʑɯβzaŋsa}{friends} & \japhug{βzaŋsa}{friend} \\
\japhug{kɤndʑɯɕaχpu}{friends} & \japhug{ɕaχpu}{friend} \\
\japhug{kɤndʑɯkɯmdza}{relatives} & \japhug{kɯmdza}{relative} \\
\japhug{kɤndʑɯrɣa}{neighbours} & \japhug{tɤ-rɣa}{neighbour} \\
\japhug{kɤndʑɯslamaχti}{classmates} & \japhug{slamaχti}{classmate} \\
\japhug{kɤndʑɯsqʰaj}{sisters} & \japhug{tɤ-sqʰaj}{sister (of a girl)} \\
\japhug{kɤndʑɯmɤtsa}{mother's sister's children} & \japhug{tɤ-mɤtsa}{mother's sister's child} \\
\japhug{kɤndʑɯtɤtɕɯχti}{friends (between boys)} & \japhug{tɤtɕɯχti}{friend (between boys)} \\
\japhug{kɤndʑɯtɕʰemɤχti}{friends (between girls)} & \japhug{tɕʰemɤχti}{friend (between girls)} \\
\japhug{kɤndʑɯxtɤɣ}{brothers} & \japhug{tɤ-xtɤɣ}{brother (of a boy)} \\
\japhug{kɤndʑɯχti}{companions} & \japhug{tɯ-χti}{companion} \\
\japhug{kɤndʑɯzda}{companions} & \japhug{tɯ-zda}{companion} \\
 \lspbottomrule
\end{tabular}
\end{table}

Non-reciprocal collectives (Table \ref{tab:non.reciprocal.collectives}) are based on nouns designating unequal relationships, in which the members designate each other by different terms, in particular kinship terms involving relatives from different generations or different gender.  Aside from kinship terms, groups comprising domestic animals and their owners can also be formed by the same process from the name of the animal, as \japhug{kɤndʑɯmbro}{horseman and his horse} and \japhug{kɤndʑɯftsoʁ}{female hybrid yak  and its owners} (\ref{ex:kAndZWftsWftsoR} below).

Non-reciprocal collectives are either formed from one of the two nouns, which can be either the one from the lower (\japhug{kɤndʑɯɣe}{grandparents and grandchildren} ) or the higher generation (\japhug{kɤndʑɯɲi}{paternal aunt and her nephews}), or by a combination of two kinship terms, the first of which undergoes in some cases changes to the point of being barely recognizable (\japhug{kɤndʑɯpɤmdɯ}{paternal uncle and his nephews}).\footnote{In the case of \japhug{kɤndʑɯwɤɬaʁ}{maternal aunt and her nephews}, the origin of the element \forme{-wɤ-} is not identifiable.}

\begin{table}
\caption{Non-reciprocal social relation collectives} \label{tab:non.reciprocal.collectives}
\begin{tabular}{lllllll}
 \lsptoprule 
  Collective & Base noun \\
\midrule
\japhug{kɤndʑɯɣe}{grandparents and grandchildren} & \japhug{tɤ-ɣe}{grandchild} \\
\japhug{kɤndʑɯʁi}{siblings} & \japhug{ta-ʁi}{younger sibling} \\
\japhug{kɤndʑɯme}{parents and daughter} & \japhug{ɯ-me}{daughter} \\
\japhug{kɤndʑɯɲi}{paternal aunt and her nephews} & \japhug{tɤ-ɲi}{father's sister} \\
\midrule
\japhug{kɤndʑɯmbro}{horseman and his horse} & \japhug{mbro}{horse} \\
\japhug{kɤndʑɯjla}{male hybrid yak and its owners} & \japhug{jla}{male hybrid yak} \\
\japhug{kɤndʑɯftsoʁ}{female hybrid yak  and its owners} & \japhug{ftsoʁ}{female hybrid yak}  \\
\japhug{kɤndʑɯpaʁ}{pig and its owners} & \japhug{paʁ}{pig} \\
\japhug{kɤndʑɯqaʑo}{sheep and its owners} & \japhug{qaʑo}{sheep} \\
\japhug{kɤndʑɯtsʰɤt}{goat and its owners} & \japhug{tsʰɤt}{goat} \\
\midrule
\japhug{kɤndʑɯrpɯftsa}{maternal uncle and his nephews} & \japhug{tɤ-rpɯ}{mother's uncle} \\
& \japhug{tɤ-ftsa}{sister's son} \\
\japhug{kɤndʑɯwɤɬaʁ}{maternal aunt and her nephews} & \japhug{tɤ-ɬaʁ}{mother's sister} \\
\japhug{kɤndʑɯpɤmdɯ}{paternal uncle and his nephews} & \japhug{tɤ-mdɯ}{brother's child} \\
& \japhug{tɤ-βɣo}{father's brother} \\
\japhug{kɤndʑɯwɤmɯsnom}{brother and sisters} & \japhug{tɤ-wɤmɯ}{brother (of a girl)} \\
& \japhug{tɤ-snom}{sister (of a boy)} \\
 \lspbottomrule
\end{tabular}
\end{table}

The collective nouns can be used as normal nouns and take case marking, numerals and other modifiers, as in   (\ref{ex:kAndZWxtAG.XsWm}).

\begin{exe}
\ex \label{ex:kAndZWxtAG.XsWm}
\gll  kɤndʑɯ-xtɤɣ χsɯm pjɤ-tu-nɯ \\
\textsc{coll}-brother three \textsc{ifr}.\textsc{ipfv}-exist-\textsc{pl} \\
\glt `There were three brothers.' (07-deluge, 1)
\end{exe}

Social relationship collectives are also found in Situ and Tshobdun (\citealt[107]{jackson98morphology}), where they have optional reduplication; in Japhug, reduplication is used by some speakers, as \japhug{kɤndʑɯftsɯftsoʁ}{female hybrid yak  and its owners} in example (\ref{ex:kAndZWftsWftsoR}), from a story by Kunbzang Mtsho.

\begin{exe}
\ex \label{ex:kAndZWftsWftsoR}
 \gll kɤndʑɯ-ftsɯ\redp{}ftsoʁ χsɯm nɯ, tsʰɯntsʰɯn kɯ-pa kɤ-nɯ-ɬoʁ-nɯ ɲɯ-ŋu, \\
 \textsc{coll}-female.yak.hybrid three \textsc{dem} \textsc{idph}:II:in.order \textsc{inf}:\textsc{stat}-\textsc{aux} \textsc{pfv}:\textsc{east}-\textsc{auto}-come.out-\textsc{pl} \textsc{sens}-be \\
\glt `(The girl, her husband) and their female hybrid yak crossed (the large river) without damage.' (kunbzang2003, 186)
\end{exe}

The lists in Tables \ref{tab:reciprocal.collectives} and \ref{tab:non.reciprocal.collectives} comprise all most common social relation collectives, but is by no means a complete list. For instance, next to \japhug{kɤndʑɯrpɯftsa}{maternal uncle and his nephews} from \japhug{tɤ-rpɯ}{mother's uncle} and \japhug{tɤ-ftsa}{sister's son}, the terms \japhug{kɤndʑɯrpɯ}{maternal uncle and his nephews} and \japhug{kɤndʑɯftsa}{nephew with his maternal uncles and aunts} are also possible though less common. However, some combinations are considered incorrect. For instance, Tshendzin considers that $\dagger$\forme{kɤndʑɯrʑaβ} (from \japhug{tɤ-rʑaβ}{wife}) is only found in children's language (\forme{nɯ tɤ-pɤtso ra kɯ tu-ti-nɯ ŋgrɤl} `children talk like that'), as the correct term  is \japhug{ʁzɤmi}{husband and wife} from Tibetan \tibet{བཟའ་མི་}{bza.mi}{husband and wife}.

There is in addition an irregular collective \japhug{kɤtsa}{parents and children}, with the same element \forme{-tsa} found in some diminutives (see \ref{sec:diminutive}), from an earlier word for `child'.

It can be used without any preceding noun as in (\ref{ex:kAtsa.ra}), but more commonly with a preceding noun, as in (\ref{ex:tAmu.kAtsa}) (note also \forme{tɤ-tɕɯ kɤtsa} `father and son' and \forme{tɕʰeme kɤtsa} `mother and daughter' from \japhug{tɤ-tɕɯ}{son, boy} and \japhug{tɕʰeme}{girl}).

\begin{exe}
\ex \label{ex:kAtsa.ra}
\gll tɕe tɤ-mu nɯ kɯ ɯ-pɯ nɯnɯ ju-ɕpʰɣɤm tɕe, ʑara kɤtsa ra stɯsti ʁɟa ʑo ɕe-nɯ ɲɯ-ra. \\
\textsc{lnk} \textsc{indef}.\textsc{poss}-mother \textsc{dem} \textsc{erg} \textsc{3sg}.\textsc{poss}-young dem \textsc{ipfv}-flee.with[III] \textsc{lnk} \textsc{3pl} parents.and.children \textsc{pl} alone completely \textsc{emph} go:\textsc{fact}-\textsc{pl} \textsc{sens}-have.to \\
\glt `And the mother (lioness) flees with her youngs, and they (mother and children) have to go alone (without the father).' (20-sWNgi, 75)
\end{exe}

\begin{exe}
\ex \label{ex:tAmu.kAtsa}
\gll
tɤ-mu kɤtsa ci pjɤ-tu-ndʑi tɕe \\
\textsc{indef}.\textsc{poss}-mother  parents.and.children \textsc{indef} \textsc{ifr}.\textsc{ipfv}-exist-\textsc{du} \textsc{lnk} \\
\glt `There was a mother and her son.' (2003tamukatsa, 1)
\end{exe}

Like comitative adverbs (§ \ref{sec:comitative.adverb}), it is clear that social relation collectives originate from participles of denominal verbs. The only example of the verbal denominal \forme{andʑɯ-} derivation from which they originate is \japhug{andʑɯrɣa}{be together as neighbours} from the IPN \japhug{tɤ-rga}{neighbourg}, as in (\ref{ex:YAndZWrGandZi}) (see § XXX on the denominal derivation). The social relation collective \japhug{kɤndʑɯrɣa}{neighbours} can thus be analyzed as the participle of this verb \forme{kɯ-ɤndʑɯrɣa}.  

\begin{exe}
\ex \label{ex:YAndZWrGandZi}
\gll ɲɯ-ɤndʑɯrɣa-ndʑi \\
\textsc{sens}-be.neighbours-\textsc{du} \\
\glt `They are one next to the other.' (elicitation)
\end{exe}

The \forme{andʑɯ-} denominal derivation being however restricted to this single example, from a synchronic point of view this collective formation is a strictly nominal derivation.

\subsubsection{Reduplicated collectives} \label{sec:redp.coll}
The reduplicated collectives are built using partial reduplication, and three distinct patterns exist.


First, some nouns allow standard partial reduplication with \forme{ɯ} in the reduplicated syllable (§ XXX) expressing a vague collective. This reduplication can apply to loanwords from Tibetan, such as \japhug{χsɯ\redp{}χsɤr}{things in gold} and \japhug{rŋɯ\redp{}rŋɯl}{things in silver} from \japhug{χsɤr}{gold} and \japhug{rŋɯl}{silver} (Tibetan \tibet{གསེར་}{gser}{gold} and \tibet{དངུལ་}{dŋul}{silver}).

\begin{exe}
\ex 
\gll a-χsɯ\redp{}χsɤr ra, a-rŋɯ\redp{}rŋɯl ra mɤ-ra kɯ ɕom rɟɤskɤt ɯ-taʁ tu-ɕe-a ŋu \\
\textsc{1sg}.\textsc{poss}-\textsc{coll}\redp{}gold \textsc{pl} \textsc{1sg}.\textsc{poss}-\textsc{coll}\redp{}silver \textsc{pl} \textsc{neg}-have.to:\textsc{fact} \textsc{erg} iron  stairs \textsc{3sg}.\textsc{poss}-on \textsc{ipfv}:\textsc{up}-go-\textsc{1sg} be:\textsc{fact} \\
\glt `I don't need things in gold or silver, I will go up the iron stairs.' (not the golden or silver stairs, 2005-Kunbzang, 215)
\end{exe}

Some nouns only appearing in reduplicated form are presumably ancient collectives, like \japhug{kʰrambaχtɯχtɤm}{lies} from a possible non-reduplicated form *\forme{kʰrambaχtɤm} (from  \tibet{ཁྲམ་པ་གཏམ་}{kʰram.pa.gtam}{deceiving words}).

%tɕendɤre nɯnɯ ʑɯ-ʑɯmkhɤm ʑo nɯ, nɯ-rɟɤlpu ɣɯ ɯ-me ci ʑo staʁlu pjɤ-tu nɯ ma pjɤ-k-ɤrɕo-ci, 
%nyimaowdzer2002, 86

Second, we find the reduplicated collectives with the vowel \ipa{a}, not \ipa{ɯ}, in the replicated syllable. Examples are few, as shown in Table \ref{tab:coll.n}, but several of them are borrowings from Tibetan. In one case, \japhug{fɕafɕɤt}{words}, the base word is a transitive verb (\japhug{fɕɤt}{tell}).

\begin{table}
\caption{Collective noun derivation} \label{tab:coll.n}
\begin{tabular}{l|lll}
 \lsptoprule 
 Base form & Collective & Tibetan \\
 \midrule
\japhug{rdɯl}{dust, dirt} & \japhug{rdardɯl}{dust, dirt} & \tibet{རྡུལ་}{rdul}{dust} \\
\japhug{tɯ-ntɕʰɯr}{fragment}  & \japhug{ɯ-ntɕʰantɕʰɯr}{fragments} & \\
\japhug{ɯ-zɯr}{side}  & \japhug{ɯ-zarzɯr}{sides} & \tibet{ཟུར་}{zur}{side, corner} \\
\japhug{ɯ-rkɯ}{side} & \japhug{ɯ-rkarkɯ}{sides} & \\
\japhug{fɕɤt}{tell}  & \japhug{fɕafɕɤt}{words} &  \tibet{བཤད་}{bɕad}{explain, tell} \\
 \lspbottomrule
\end{tabular}
\end{table}

Reduplicated collective nouns in \forme{a-} can be used without number clitic, as in (\ref{ex:WntChantChWr}), but they often appear with the \japhug{ra}{plural} as in (\ref{ex:rdardWl}).

\begin{exe}
\ex \label{ex:WntChantChWr}
\gll znɤrɣama nɯ mtʰa ɯ-kɤcu ŋu. tɕe nɯnɯtɕu tɯ-ji ɯ-ntɕhantɕhɯr pɯ-dɤn, jinde kʰro ɲɤ-s-qapɯ-nɯ,\\
p.n. \textsc{dem} p.n. \textsc{3sg.poss}-east be:\textsc{fact} \textsc{lnk} \textsc{dem:pl} \textsc{indef}.\textsc{poss}-field \textsc{3sg.poss}-fragment:\textsc{coll} \textsc{pst}.\textsc{ipfv}-be:many now much \textsc{ifr}-\textsc{caus}-be.fallow-\textsc{pl}\\
\glt `Znargama ('The place where one calls the rain') is on the east of Mtha, there used to be many little fragments of fields, but now people have left them become fallow.' (150903 kAmYW tWji3, 19)
\end{exe}

\begin{exe}
\ex \label{ex:rdardWl}
\gll tɕe tɤɕi nɯ tú-wɣ-χtɕi tɕʰɣaʁtɕʰɣaʁ ʑo tɕe, rdardɯl nɯra ɲɯ́-wɣ-ɣɤ-me tɕe \\
\textsc{lnk} barley \textsc{dem} \textsc{ipfv}-\textsc{inv}-wash \textsc{idph}:II:completely.clean \textsc{emph} \textsc{lnk} dush:\textsc{coll} \textsc{dem:pl} \textsc{ipfv}-\textsc{inv}-\textsc{caus}-not.exist \textsc{lnk} \\
\glt `Then one washes the barley very thoroughly, one removes all the dirt.' (2002tWsqar, 118)
\end{exe}
 
The noun \japhug{rgargɯn}{old person} has the form of a collective noun as those of Table \ref{tab:coll.n}, but it is commonly used with singular or dual referents (as in \ref{ex:rgargWn}). It could be originally the collective form of a borrowing from Tibetan \tibet{རྒན་པོ་}{rgan.po}{old person}, though the expected form would have been $\dagger$\forme{rga-rgɤn}. 
 
\begin{exe}
\ex \label{ex:rgargWn}
\gll  rgargɯn ni kɤ-fstɯn pɯ-ra \\
old.person \textsc{du} \textsc{inf}-serve \textsc{pst.ipfv}-have.to \\
\glt `She had to take care of two old people.' (14-tApitaRi, 34)
\end{exe}

A third reduplicated collective derivation is only attested by one example, the form \japhug{qajɯqaja}{all kinds of worms} (see \ref{ex:qajWqaja}) which derives from \japhug{qajɯ}{worm}  by reduplicating the whole word and changing the last rime to \ipa{-a}, a reduplication template reminiscent of that found in Khroskyabs (see \citealt{lai13fuyin}, \citealt[22-24]{lai17khroskyabs}).

\begin{exe}
\ex \label{ex:qajWqaja}
\gll
tɯ-ci ɯ-ŋgɯ qajɯqaja tʰamtɕɤt, sɯŋgɯ ɣɯ ɯ-rɯdaʁ kɯ-xtɕi kɯ-wxti, mɤʑɯ pɣa nɯnɯra lonba ʑo kɤ-fsraŋ kɯ-ra ɲɯ-ɕti ma \\
\textsc{indef.poss}-water \textsc{3sg}-inside worm:\textsc{coll} all forest \textsc{gen} animal \textsc{nmlz}:S/A-be.small \textsc{nmlz}:S/A-be.big yet bird \textsc{dem:pl} all \textsc{emph} \textsc{inf}-protect \textsc{inf:stat}-have.to \textsc{sens}-be:\textsc{affirm} \textsc{lnk} \\
\glt `All the creatures in the water, the small and big animals of the forest, and also the birds have to be protected.' (160703 jingyu, 43)
\end{exe}


\subsubsection{Dvandva collective} \label{sec:dvandva.coll}
The \textit{dvandva} collective is derived from two nouns, the first one in \textit{status constructus} form followed by the element \forme{-lɤ-}  and then by the second noun stem without possessive prefix. Known forms are listed in Table \ref{tab:dvandva.coll.n}.  

 \begin{table}
\caption{Dvandva collectives} \label{tab:dvandva.coll.n}
\begin{tabular}{l|lll}
 \lsptoprule 
Collective & First noun & Second Noun \\
 \midrule
 \japhug{tɯ-kɤlɤmɲaʁ}{facial features} & \japhug{tɯ-ku}{head} & \japhug{tɯ-mɲaʁ}{eye} \\
\japhug{tɯ-mɤlɤjaʁ}{the four limbs} & \japhug{tɯ-mi}{leg, foot} & \japhug{tɯ-jaʁ}{arm, hand} \\
 \japhug{ɯ-kɤlɤjme}{head upside down} & \japhug{tɯ-ku}{head} & \japhug{tɤ-jme}{tail} \\
  \japhug{kɯmɤlɤxso}{in vain} & \japhug{kɯ-me}{not existing} & \japhug{ɯ-xso}{empty, normal} \\
 \lspbottomrule
\end{tabular}
\end{table}

Among the examples in Table \ref{tab:dvandva.coll.n},  \japhug{ɯ-kɤlɤjme}{head upside down}  and   \japhug{kɯmɤlɤxso}{in vain} are mainly used adverbially. The first one is mainly used with verbs such as \japhug{ɕtʰɯz}{turn towards} and \japhug{ru}{look} at, as in (\ref{ex:WkAlAjme}).

\begin{exe}
\ex \label{ex:WkAlAjme}
 \gll tɕe nɯ ɯ-sta nɯ lɤtɕʰom nɯ ɲɯ́-wɣ-ʁɟo ʑo kʰrɯŋkʰrɯŋ ʑo qʰe tɕe ɯ-kɤlɤjme pjɯ́-wɣ-ɕtʰɯz qʰe, ɯ-mŋu nɯ pa pjɯ́-wɣ-ɕtʰɯz \\
 \textsc{lnk} \textsc{dem} \textsc{3sg.poss}-place \textsc{dem} churning.bucket \textsc{dem} \textsc{ipfv}-\textsc{inv}-rinse \textsc{emph}  \textsc{idpf}:II:completely.clean \textsc{emph} \textsc{lnk} \textsc{lnk} \textsc{3sg.poss}-head.upside.down \textsc{ipfv}:\textsc{down}-\textsc{inv}-turn.towards \textsc{lnk} \textsc{3sg.poss}-opening \textsc{dem} down   \textsc{ipfv}:\textsc{down}-\textsc{inv}-turn.towards \\
 \glt `One rinses the churning bucket very clean, and put it upside down at its place, the opening down.'(30-macha, 66)
\end{exe}

The adverb \japhug{kɯmɤlɤxso}{in vain} combines the subject participle of \japhug{me}{not exist} with the property noun \japhug{ɯ-xso}{empty, normal} (a lexicalized participle, whose uses and etymology are described in \ref{sec:property.nouns}). It is originally probably a noun used adverbially (§ XXX).


The  \forme{-lɤ-/-la-} element found in collective \textit{dvandva}-s is also attested in approximate numerals (see § XXX) and in adverbs such \japhug{tɯxpalɤskɤr}{during the whole year} (from \japhug{tɯ-xpa}{one year} and \japhug{fskɤr}{turn around}) and  \japhug{rtsɯɕaŋlaŋmtɕɤt}{all the plants} (from \japhug{rtsɯɕaŋ}{plant} and \japhug{tʰamtɕɤt}{all}, respectively from Tibetan \tibet{རྩི་ཤིང༌}{rtsi.ɕiŋ}{plant} and \tibet{ཐམས་ཅད་}{tʰams.tɕad}{all}).
 
\subsection{Superlative}
While there is no adjectival superlative derivation in Japhug (superlative constructions are synthetic, see § XXX), we find nevertheless a derivation applied to locative nouns, expressing the most extreme location. As shown in Table \ref{tab:superlative.n}, it is built by adding an element \forme{-ɕɯ-} followed by a complete copy of the root of the noun without \textit{status constructus} alternation or partial replication; the resulting noun is still an inalienably possessed locative noun. Example (\ref{ex:WqaCWqa}) illustrates the use of one of these forms.

\begin{table}
\caption{Superlative noun derivation} \label{tab:superlative.n}
\begin{tabular}{l|lll}
 \lsptoprule
\japhug{tɯ-ku}{head, top} & \japhug{ɯ-kuɕɯku}{the highest place} \\
\japhug{tɯ-qa}{root, paw, bottom} & \japhug{ɯ-qaɕɯqa}{the deepest place} \\
\japhug{ɯ-rkɯ}{side} & \japhug{ɯ-rkɯɕɯrkɯ}{the place most on the side} \\
\japhug{ɯ-zɯr}{side} & \japhug{ɯ-zɯrɕɯzɯr}{the place most on the side} \\
 \lspbottomrule
\end{tabular}
\end{table}

\begin{exe}
\ex \label{ex:WqaCWqa}
\gll rɟɤmtsʰu ɯ-qaɕɯqa pjɯ-ɕe tɕe, nɯnɯ ɯ-kɤ-nɤ-mɯm nɯra ɕ-tu-nɯ-tɕɤt ɲɯ-ŋu. \\
ocean \textsc{3sg.poss}-bottom:\textsc{super} \textsc{ipfv}:\textsc{down}-go  \textsc{lnk} \textsc{dem} \textsc{3sg.poss}-\textsc{nmlz}:P-\textsc{trop}-be.tasty \textsc{dem:pl} \textsc{transloc-ipfv}-\textsc{auto}-take.out \textsc{sens}-be \\
\glt `(The sperm whale) goes to the lowest depths of the ocean and catches the things it likes to eat.' (160703 jingyu, 24)
\end{exe}

\section{Denominal adverbs}

\subsection{Comitative adverbs} \label{sec:comitative.adverb}
The highly productive comitative adverb formation derives from nouns adverbs meaning `having X' , `together with X', `including X' or in the case of clothes or covers `wearing X'.

Comitative adverbs are built by partially reduplicating the last syllable of the noun stem (following the morphophonological rules in XXX) and prefixing either \forme{kɤ́-} or \forme{kɤɣɯ-}. This derivation applies to native words and loanwords from Tibetan. From instance, \japhug{χɕɤlmɯɣ}{glasses} (from Tibetan \tibet{ཤེལ་མིག་}{ɕel.mig}{glasses}) yields \forme{kɤ́-χɕɤlmɯ\tld{}lmɯɣ} or \forme{kɤɣɯ-χɕɤlmɯ\tld{}lmɯɣ} `together with glasses'.\footnote{Note that reduplication disregards morpheme boundaries, as the coda of \japhug{χɕɤl}{glass} (from \tibet{ཤེལ་}{ɕel}{glass}) is reduplicated with the following syllable. } 

No semantic difference between the comitative adverbs in \forme{kɤ́-} and those in \forme{kɤɣɯ-} has been detected; both are fully productive and can be built from the same nouns. As argued in \citet{jacques17comitative}, the \forme{kɤɣɯ-} form is inherited (from proto-Gyalrong \forme{*kɐwə-}), while \forme{kɤ́-} is borrowed from Tshobdun \forme{ko-}, the exact cognate of \forme{kɤɣɯ-}  (\citealt[107]{jackson98morphology}). The prefix \forme{kɤɣɯ-} and its Tshobdun cognate \forme{ko-} both originate from the S-participle \forme{kɯ-} (see § XXX) of the proprietive \forme{aɣɯ-} denominal derivation (see § XXX), attesting a \textsc{proprietive} $\Rightarrow$ \textsc{comitative} grammaticalization pathway  (\citealt{jacques17comitative}). 

When the base noun is an IPN, it is possible to build a comitative adverb with the indefinite possessor prefix or with the bare stem. For instance, from \japhug{tɤ-rte}{hat} one can derive both \forme{kɤ́-rtɯ\tld{}rte} / \forme{kɤɣɯ-rtɯ\tld{}rte} `with his/her hat' and \forme{kɤ́-tɤ-rtɯ\tld{}rte} /  \forme{kɤɣɯ-tɤ-rtɯ\tld{}rte} `with a hat' with the indefinite possessor prefix \forme{tɤ-}. These two sets of forms have different meanings: the former \forme{kɤ́-rtɯ\tld{}rte} / \forme{kɤɣɯ-rtɯ\tld{}rte} mean `wearing one's hat' (example \ref{ex:kAGWrtWrte}), while the latter \forme{kɤ́-tɤ-rtɯ\tld{}rte} /  \forme{kɤɣɯ-tɤ-rtɯ\tld{}rte} imply that the subject is not wearing the hat (\ref{ex:kAGWtArtWrte}); preserving the indefinite possessor in the derived form alienabilizes the IPN (see § \ref{sec:alienabilization}).

\begin{exe}
\ex \label{ex:kAGWrtWrte}
\gll kɤɣɯ-rtɯ\tld{}rte 	ʑo 	kʰa	ɯ-ŋgɯ	lɤ-tɯ-ɣe	\\
\textsc{comit}-hat \textsc{emph} house \textsc{3sg}-inside \textsc{pfv}-2-come[II] \\
\glt `You came inside the house wearing your hat.' (You were expected to take it off before coming in, elicited)
\end{exe}

\begin{exe}
\ex \label{ex:kAGWtArtWrte}
\gll  laχtɕʰa	kɤɣɯ-tɤ-rtɯ\redp{}rte	ʑo	ta-ndo \\
thing \textsc{comit-indef.poss}-hat \textsc{emph} \textsc{pfv}:3$\rightarrow$3'-take \\
\glt `He took the things together with the hat.' (Not wearing it, elicited)
\end{exe}

Example (\ref{ex:tsxha.kAtAlWlu}) illustrates the use of the alienabilized comitative adverb \japhug{kɤ́tɤlɯlu}{with milk} (from \japhug{tɤ-lu}{milk}) for the expression `milk tea' -- the inalienably form \japhug{kɤ́lɯlu}{with its milk} is only compatible with the animal producing the milk.

\begin{exe}
\ex \label{ex:tsxha.kAtAlWlu}
\gll   ʁja ku-te ɲɯ-ŋu, tʂʰa kɤ́-tɤlɯ\redp{}lu tú-wɣ-sɯ-rku kɯnɤ \\
verdigris \textsc{ipfv}-put[III] \textsc{sens}-be tea \textsc{comit}-milk \textsc{ipfv}-\textsc{inv}-\textsc{caus}-put.in also \\
\glt  `It gets verdigris, even when uses it to pour milk tea.'  (30-Com, 93-4)
\end{exe}

Comitative adverbs can be used as sentential adverbs, with scope over the whole sentence (\ref{ex:kAGWrtWrte}, \ref{ex:kAGWtArtWrte}, \ref{ex:kAsnWsno}). 

\begin{exe}
\ex \label{ex:kAsnWsno}
\gll kɤ́-snɯ\tld{}sno 	ʑo 	kɤ-rŋgɯ \\
\textsc{comit}-saddle \textsc{emph} \textsc{pfv}-lie.down \\
\glt `(The horse) slept with its saddle.' (elicited)
\end{exe}

Alternatively the comitative adverb is used as noun modifier (even in fixed expressions, as in \ref{ex:tsxha.kAtAlWlu}), and either follow (\ref{ex:tsxha.kAtAlWlu}, \ref{ex:kAjWjaR}) or precede (\ref{ex:kArnWrna}, \ref{ex:kAthAlwWlwa}) the noun which it modifies.

The noun in question can either correspond to the object (\ref{ex:tsxha.kAtAlWlu}, \ref{ex:kAjWjaR}, \ref{ex:kAthAlwWlwa}), the intransitive subject (\ref{ex:kArnWrna}, \ref{ex:kAsnWsno}) or even the transitive subject (\ref{ex:kArJWrJit.kW}). This last option is not attested in the text corpus, but speakers have no trouble producing sentences of this type.


\begin{exe}
\ex \label{ex:kAjWjaR}
\gll tɤ-sno 	kɤ́-jɯ\redp{}jaʁ 	nɯ 	lu-ta-nɯ \\
\textsc{indef.poss}-saddle \textsc{comit}-hand \textsc{dem} \textsc{ipfv}-put-\textsc{pl} \\
\glt `(Then), they put the saddle with its handles.' (30-tAsno, 77)
\end{exe}
 

\begin{exe}
\ex \label{ex:kArnWrna}
\gll pɣɤkʰɯ 	nɯ 	ɯ-ku 	nɯnɯ 	lɯlu 	tsa 	ɲɯ-fse, 	ɯ-mtsioʁ 	ɣɤʑu 	ma kɤ́-rnɯ\redp{}rna 	lɯlu 	ɯ-tɯ-fse 	ɲɯ-sɤre 	ʑo. \\
owl \textsc{dem} \textsc{3sg.poss}-head \textsc{dem} cat a.little \textsc{sens}-be.like \textsc{3sg.poss}-beak exist:\textsc{sens} a.part.from \textsc{comit}-ear cat \textsc{3sg-nmlz:degree}-be.like \textsc{sens}-be.extremely/be.funny \textsc{emph} \\
\glt `The owl's head looks a little like that of a cat, apart from the fact that it has a beak, it looks very much like a cat with its ears.' (22-pGAkhW, 7)
\end{exe}


\begin{exe}
\ex \label{ex:kAthAlwWlwa}
\gll kɤ́-tʰɤlwɯ\tld{}lwa 	ɯ-zrɤm 	ra 	kɯnɤ 	cʰɯ́-wɣ-ɣɯt 	pjɯ́-wɣ-ji 	ri 	maka 	tu-ɬoʁ 	mɯ́j-cʰa \\  
\textsc{comit}-earth \textsc{3sg.poss}-root \textsc{pl} also \textsc{ipfv-inv}-bring \textsc{ipfv-inv}-plant but at.all \textsc{ipfv}-come.out \textsc{neg:sens}-can \\
\glt `Even if one takes its root with earth (around it) and plant it, it cannot grow.' (15-babW, 121)
\end{exe}


\begin{exe}
\ex \label{ex:kArJWrJit.kW}
\gll lɯlu	kɤ́-rɟɯ\tld{}rɟit	ra	kɯ	ʑo	βʑɯ	 to-ndza-nɯ. \\
cat \textsc{comit}-offspring \textsc{pl}  \textsc{erg} \textsc{emph} mouse \textsc{ifr}-eat-\textsc{pl} \\
\glt `The cat and its young ate the mouse.' (elicited)
\end{exe}

The comitative adverbs have additional meanings in certain contexts. With the verb \japhug{fse}{be like},  comitative adverbs from body parts occurring with names of animals, as in (\ref{ex:kArnWrna}) and (\ref{ex:BZW.kAmtChWmtChi}), mean `to have a body part that looks like that of the other animal'.

\begin{exe}
\ex \label{ex:BZW.kAmtChWmtChi}
\gll
li βʑɯ kɤ́-mtɕʰɯ\redp{}mtɕʰi ci nɯ ɲɯ-fse  \\
again mouse \textsc{comit}-mouth \textsc{indef} \textsc{dem} \textsc{sens}-be.like \\
\glt `(The bat's) mouth is like that of a mouse.' (literally `It looks like a mouse with its mouth.' 25-qarmWrwa, 12)
\end{exe}

Nouns incorporated into comitative adverbs lose their nominal status and cannot be determined by relative clauses (including attributive adjectives), numerals or demonstratives. In a sentence such as \ref{ex:kAGWNkhWNkhor} for instance, the attributive participial relative [\forme{kɯ\tld{}kɯ-ŋɤn}] `all the ones who are evil' does not determine \forme{kɤɣɯ-ŋkhɯ\tld{}ŋkhor} `with his subjects', a syntactic structure which would correspond to the translation `with all his evil subjects'. Rather, it determines the head noun together with the comitative adverb  \forme{rɟɤlpu} \forme{kɤɣɯ-ŋkhɯ\tld{}ŋkhor} `the king with his subjects', which implies the translation given below.

\begin{exe}
\ex \label{ex:kAGWNkhWNkhor}
\gll rɟɤlpu 	kɤɣɯ-ŋkhɯ\tld{}ŋkhor 	kɯ\tld{}kɯ-ŋɤn  	ʑo 	to-ndo 	tɕe, 	tɕendɤre 	kɯ-mɤku 	nɯ 	sɤtɕʰa 	kɯ\tld{}kɯ-sɤ-scit 	ʑo 	jo-tsɯm 	ɲɯ-ŋu 	ri 	kɯ-maqʰu 	tɕe, 	kɯ\tld{}kɯ-sɤɣ-mu 	ʑo 	jo-tsɯm 	tɕe \\
king \textsc{comit}-subjects \textsc{total}\tld{}\textsc{nmlz}:S/A-be.bad \textsc{emph} \textsc{ifr}-take \textsc{lnk}  \textsc{lnk} \textsc{nmlz}:S/A-be.before \textsc{dem} place \textsc{total}\tld{}\textsc{nmlz}:S/A-\textsc{deexp}-be.happy \textsc{emph} \textsc{ifr}-take.away \textsc{sens}-be \textsc{lnk} \textsc{nmlz}:S/A-be.after \textsc{lnk} \textsc{total}\tld{}\textsc{nmlz}:S/A-\textsc{deexp}-fear \textsc{emph} \textsc{ifr}-take.away \textsc{lnk} \\
\glt `She took the king and his subjects, all the evil ones, in the beginning she took them to nice places, but later she took them to fearful places.' (Norbzang2012, 390)
\end{exe}

Other denominal adverb formations are also attested in Japhug, but are studied in the sections on time nominals (§ XXX) and locational nouns (§  XXX) in other chapters.


\subsection{Other denominal adverbs} \label{sec:other.denominal.adverbs}
Partial reduplication of nouns, in addition to the reduplicated collectives (§ \ref{sec:redp.coll}), can also derive vague location adverbs such as \japhug{tʂɯtʂu}{on the road} from \japhug{tʂu}{road}, as in (\ref{ex:tsxWtsxu}).

\begin{exe}
\ex \label{ex:tsxWtsxu}
\gll cʰa ra tʂɯ\redp{}tʂu kú-wɣ-nɯ-tsʰi tɕe \\
alcohol \textsc{pl} path\redp{}\textsc{location}  \textsc{ipfv}-\textsc{inv}-\textsc{auto}-drink \textsc{lnk} \\
\glt `One drinks alcohol on the way (back home).' (2010-histoire10)
\end{exe} %Nominal morphology
\chapter{Pronouns and Demonstratives}
\section{Personal pronouns} \label{sec:pers.pronouns}

%\\ipa\{([\w-]*)\}
%\1

%(\d)\\(\w\w)\{\}
%\\textsc{\1\2}

The pronominal system of Japhug distinguishes singular, dual and plural. Alongside the free pronouns, a system of pronominal prefixes is used not only to express possession on noun (see § XXX for an account of the possessive constructions), but also appears in various constructions in the verbal system. These prefixes do not distinguish the second and the third person in the dual and plural forms; their use is described in section \ref{sec:possessive.paradigm}.

\begin{table}[h] \centering
\caption{Pronouns and possessive prefixes }\label{tab:pronoun}
\begin{tabular}{lllllllll} \lsptoprule
 Free pronoun & Prefix & \\
\midrule
 \forme{aʑo},    \forme{aj} &	\forme{a-}  &		1\sg{} \\
\forme{nɤʑo},  \forme{nɤj} &	\forme{nɤ-}  &			2\sg{} \\
\forme{ɯʑo}  &	\forme{ɯ-}  &			3\sg{} \\
\midrule
\forme{tɕiʑo}  &	\forme{tɕi-}  &			1\du{} \\
\forme{ndʑiʑo}  &	\forme{ndʑi-}  &		2\du{} \\	
\forme{ʑɤni}  &	\forme{ndʑi-}  &		3\du{} \\	
\midrule
\forme{iʑo}, \forme{iʑora},   \forme{iʑɤra}   &	\forme{i-}  &			1\pl{} \\
\forme{nɯʑo}, \forme{nɯʑora},   \forme{nɯʑɤra}  &	\forme{nɯ-}  &			2\pl{} \\
\forme{ʑara}  &	\forme{nɯ-}  &			3\pl{} \\
\lspbottomrule
\end{tabular}
\end{table}

Free pronouns and possessive prefixes are remarkably similar in Kamnyu Japhug. In the eastern Japhug dialects, a different \textsc{1sg} pronoun distinct from the possessive prefix  is used: \forme{ŋa} (possibly borrowed from Situ \forme{ŋā}). In the table above, we observe that all the pronouns except the third person dual and plural are formed by adding the root \forme{-ʑo} to the corresponding possessive prefix. 

The first and second person singular pronouns  \forme{aʑo} and \forme{nɤʑo} also have the shorter monosyllabic forms \forme{aj} and \forme{nɤj} respectively. These short forms are considerably less common in stories (in the reported speech of the characters), but appear frequently in free conversations.

Japhug lacks any inclusive / exclusive distinction, unlike other Gyalrongic languages such as Tshobdun, Situ or Khroskyabs (see \citealt{jackson98morphology}, \citealt[177]{linxr93jiarongen}, \citealt[92]{prins16kyomkyo}, \citealt[170]{lai17khroskyabs}). Example  (\ref{ex:tCiZo.CetCi}) shows the dual pronoun \japhug{tɕiʑo}{we (dual)} in inclusive use (it is clear from the context that the son tells his mother to come with him), and (\ref{ex:tCiZo.tCitAYi}) illustrates the same pronoun in exclusive use. Similar pairs of examples can be found with the first plural pronoun \japhug{iʑo}{we (plural} and its variants.

\begin{exe}
\ex \label{ex:tCiZo.CetCi}
\gll a-mu tɕetʰa tɕiʑo kɯnɤ ɕe-tɕi \\
\textsc{1sg.poss}-mother later \textsc{1du} also go:\textsc{fact}-\textsc{1du} \\
\glt `Mother, you and I will go too.' (2003tWxtsa, 138)
\end{exe}

\begin{exe}
\ex \label{ex:tCiZo.tCitAYi}
\gll nɯʑora ɣɯ nɯ-ɕɤmɯɣdɯ cʰo kɯ-fse nɯ ɯ-tsʰɤt nɯ, tɕiʑo ɣɯ tɕi-tɤɲi tɯ-ldʑa pɯ-tu tɕe, nɯ kɤ-nɯ-tʰɯ-tɕi ɕti wo \\
\textsc{2pl} \textsc{gen} \textsc{2pl.poss}-gun \textsc{comit} \textsc{nmlz}:S/A-be.like \textsc{dem} \textsc{3sg}-instead \textsc{dem} \textsc{1du} \textsc{gen} \textsc{1du.poss}-staff one-long.object \textsc{pst.ipfv}-exist \textsc{lnk} \textsc{dem} \textsc{pfv}-\textsc{auto}-spread-\textsc{1du} be:\textsc{affirm}:\textsc{fact} \textsc{sfp} \\
\glt `Instead of guns and other things like you, we only had a staff, and we used it as a bridge (to cross the river).' (2003kunbzang, 164)
\end{exe}

Third person pronouns can be used with inanimate referents, as the third person dual \forme{ʑɤni} in example (\ref{ex:rNgW}).

\begin{exe}
\ex \label{ex:rNgW}
\gll tɕe rŋgɯ nɯ  to-k-ɤmɯrpu-ndʑi-ci tɕe, tɕendɤre ʑɤni pjɤ-nɯ-ɴɢrɯ-ndʑi   \\
\textsc{lnk} boulder \textsc{dem} \textsc{ifr-evd}-bump.into:\textsc{recip}-\textsc{du-evd} \textsc{lnk} \textsc{lnk} \textsc{3du} \textsc{ifr-auto}-crush-\textsc{du} \\
\glt `The boulders bumped into each other and they were pulverized.' (smanmi4.82-83)
\end{exe}

In some contexts, demonstrative pronouns rather than person pronouns are used to refer to a third person, even human (see § \ref{sec:demonstrative.pronouns}).

Personal pronouns are not used as head of relative clauses (as in Chinese \zh{……的你} `you who are ...'), though there are case of relativization of first or second person possessor, as in (\ref{ex:amu.kWme}) (see § XXX).

\begin{exe}
\ex \label{ex:amu.kWme}
\gll aʑo nɯ a-mu kɯ-me ŋu-a tɕe tɕe \\
1sg \textsc{dem} \textsc{1sg.poss}-mother \textsc{nmlz}:S/A-exist be:\textsc{fact-1sg} \textsc{lnk} \textsc{lnk} \\
\glt `I am someone who does not have a mother.' (2003Nyimawodzer2, 12)
\end{exe}

Personal pronouns can take determiners, in particular the demonstrative \forme{nɯ} as in (\ref{ex:amu.kWme}), numerals (§ \ref{sec:uses.numerals}) and can also precede a noun in apposition, in expressions such as \forme{iʑo kɯrɯ} `we, Tibetans' (\ref{ex:iZo.kWrW}) or \forme{nɤʑo qaɕpa} `you frog' in (\ref{ex:nAZo.qaCpa}).

\begin{exe}
\ex \label{ex:iZo.kWrW}
\gll
iʑo kɯrɯ tɕe pɤjka tu-nɯ-ti-j ŋu tɕe, \\
\textsc{1pl} Tibetan \textsc{lnk} species.of.squash \textsc{ipfv}-\textsc{auto}-say-\textsc{1pl} \textsc{lnk} \\
\glt `We Tibetans call it \forme{pɤjka}.' (16-CWrNgo, 71)
\end{exe}

\begin{exe}
\ex  \label{ex:nAZo.qaCpa}
\gll  nɯ-nɯ-nɤre ma nɤʑo qaɕpa nɤ-rʑaβ nɤ-kɯ-mbi kɯ-tu me   \\
\textsc{imp-auto}-laugh \textsc{lnk} \textsc{2sg} frog \textsc{2sg.poss}-wife \textsc{2sg.poss}-\textsc{nmlz}:S/A-give \textsc{nmlz}:S/A-exist not.exist:\textsc{fact} \\
\glt `Laugh as you wish, nobody will give you a wife, you frog.'   (2002 qaCpa, 176)
\end{exe} 

Personal pronouns occur as member of compounds only in two constructions: with the root \forme{-sɯso} `as X wish' (from the verb \japhug{sɯso}{think}), as in example (\ref{ex:ʑara.sWso}) and with \forme{-sti} `alone'. These constructions is discussed in more detail in § XXX and § \ref{sec:stWsti}.

\begin{exe}
\ex \label{ex:ʑara.sWso}
\gll a-zda ra ʑara-sɯso tu-nɯ-nɤŋkɯŋke-nɯ ɲɯ-kʰɯ \\
\textsc{1sg.poss}-companion \textsc{pl} \textsc{3pl}-as.wish \textsc{ipfv}-\textsc{auto}-go.here.and.there-\textsc{pl} \textsc{sens}-be.possible \\
\glt `The other (snakes) can go here and there as they wish.' (The divination, 43)
\end{exe}

As in most languages with polypersonal indexation, pronouns (especially first and second person pronouns) are never obligatory, and a finite verb form without overt argument NPs is a perfectly well-formed sentence (see § XXX). 

\section{Generic pronoun}  \label{sec:genr.pro}
The generic pronoun \japhug{tɯʑo}{one} has the same morphological structure as personal pronouns as seen in the previous section, combining the generic possessive prefix \ipa{tɯ-} with the pronominal root \forme{-ʑo}. Note that this generic possessive has to be strictly distinguished from the homophonous indefinite possessive prefix \forme{tɯ-} (see § \ref{sec:indef.genr.poss}).

In Japhug, sentences have at most one generic human referent(§ XXX). If this referent is core argument, the verb has generic indexation (\forme{kɯ-} for S/P, \forme{wɣ-} for A, as in the following examples; see also section XXX). The generic argument can be realized as the generic pronoun \forme{tɯʑo} as in (\ref{ex:pjWkWZGAGANgi}) or by a generic noun (such as \japhug{tɯrme}{person}, see § XXX).

\begin{exe}
\ex \label{ex:pjWkWZGAGANgi}
\gll tɯ-zda pjɯ́-wɣ-z-ɣɤtɕa, \textbf{tɯʑo}  ntsɯ  pjɯ-kɯ-ʑɣɤ-ɣɤŋgi   	tɕe,  pɯ-kɯ-nɯ-ɣɤtɕa 	kɯ́nɤ   	pjɯ-kɯ-ʑɣɤ-ɣɤŋgi   	tɕe,    ɯ-mbrɤzɯ   	kɯ-tu   	me  	tu-kɯ-ti   	ɲɯ-ŋu.   \\
\textsc{genr.poss}-companion \textsc{ipfv-inv-caus}-be.wrong oneself always \textsc{ipfv-genr:S/P-refl}-be.right \textsc{lnk} \textsc{pfv-genr:S/P-auto}-be.wrong also \textsc{ipfv-genr:S/P-refl}-be.right lnk \textsc{3sg.poss}-result \textsc{nmlz:S/A}-have  not.exist:\textsc{fact} \textsc{ipfv-genr}-say \textsc{sens}-be \\
\glt  `If one considers that one's companion is wrong, and always considers himself to be right even if one is wrong, there is can be no good result.' (Mouse and sparrow, 80-82)
\end{exe} 

The generic pronoun can occur before a noun with the generic possessive as in \forme{tɯʑo tɯ-skɤt}  `one's language' in example (\ref{ex:tWZo.tWskAt}); this contributes to disambiguating between the indefinite possessive and the generic possessive in the case of inalienably possessed nouns (thus on its own \forme{tɯ-skɤt} can mean either `a language' or `one's language').

\begin{exe}
\ex \label{ex:tWZo.tWskAt}
\gll tɕendɤre tɯʑo tɯ-skɤt ʑara ɣɯ-sɯxɕɤt ɲɯ-ra, ʑara nɯ-skɤt tɯʑo kɯ-sɯxɕɤt ɲɯ-ra \\
\textsc{lnk} \textsc{genr} \textsc{genr.poss}-language \textsc{3pl} \textsc{inv}-teach \textsc{sens}-have.to \textsc{3pl} \textsc{3pl.poss}-language \textsc{genr} \textsc{genr}:S/P-teach \textsc{sens}-have.to \\
\glt `One has to teach them one's language, and they have to teach you their language.'  (150901 tshuBdWnskAt, 29)
\end{exe} 

When occurring in A function, the generic pronoun \forme{tɯʑo} obligatorily receives the ergative \forme{kɯ} as in (\ref{ex:tWZo.kW}) (note that in example \ref{ex:tWZo.tWskAt}, although the generic referent is A in the first clause, \forme{tɯʑo} does not take ergative because it is a determiner of \forme{tɯ-skɤt}). 

\begin{exe}
\ex \label{ex:tWZo.kW}
\gll tɯʑo kɯ tɯ-χti ɲɯ́-wɣ-nɯ-ɕar kɯ-maʁ kɯ,  tɯ-pʰama ra kɯ tɯ-χti ɲɯ-ɕar-nɯ tɕe tɯ-sɯm pɯ-a<nɯ>ri nɤ ju-kɯ-ɕe,
mɯ-pɯ-a<nɯ>ri nɤ ju-kɯ-ɕe pɯ-ra \\
\textsc{genr} \textsc{erg} \textsc{genr.poss}-spouse \textsc{ipfv-inv-auto}-search \textsc{inf:stat}-not.be \textsc{erg} \textsc{genr.poss}-parent \textsc{pl} \textsc{erg} \textsc{genr.poss}-spouse \textsc{ipfv}-search-\textsc{pl} \textsc{lnk} \textsc{genr.poss}-mind \textsc{pst.ipfv}-<\textsc{auto}>go[II] \textsc{lnk} \textsc{ipfv}-\textsc{genr}:S/P-go, \textsc{neg-pst.ipfv}-<\textsc{auto}>go[II] \textsc{lnk} \textsc{ipfv-genr}:S/P-go \textsc{pst.ipfv}-have.to \\
\glt `One could not choose one's spouse, one's parents chose one's spouse, and one had to go whether one liked it or not.' (14-tApitaRi, 212-215)
\end{exe} 

Other cases like dative are treated like inalienably possessed nouns (see § XXX); for instance, when the generic argument is in the dative, the forms \forme{tɯ-ɕki} or \forme{tɯ-pʰe} `to one' occur, with no indexation on the verb selecting this dative argument, as in example (\ref{ex:tWZo.tWCki}).

\begin{exe}
\ex \label{ex:tWZo.tWCki}
\gll
tɯ-ɲi ɣɯ ɯ-rɟit nɯra kɯ tɯʑo tɯ-ɕki ``a-rpɯ", tɤ-tɕɯ pɯ-kɯ-ŋu nɤ ``a-rpɯ" tu-ti-nɯ, tɕʰeme pɯ-kɯ-ŋu nɤ ``a-ɬaʁ" tu-ti-nɯ kɯ-ra ŋu \\
\textsc{genr.poss}-FZ \textsc{gen} \textsc{3sg.poss}-child \textsc{dem:pl} \textsc{erg} \textsc{genr} \textsc{genr}-dat \textsc{1sg.poss}-MB \textsc{indef.poss}-son \textsc{pst.ipfv-genr}:S/P-be if \textsc{1sg.poss}-MB \textsc{ipfv}-say-\textsc{pl} girl \textsc{pst.ipfv-genr}:S/P-be  if \textsc{1sg.poss}-MZ \textsc{ipfv}-say-\textsc{pl} \textsc{inf.stat}-have.to be:\textsc{fact} \\
\glt `One's father's sister's children have to call oneself `my mather uncle' if one is a boy, `my mather aunt' if one is a girl.'  (see § XXX about Omaha kinship, 140425kWmdza03, 1)
\end{exe} 

As examples (\ref{ex:pjWkWZGAGANgi}) to (\ref{ex:tWZo.tWCki}) illustrate, generic agreement between pronoun, possessive prefix and verb indexation is very systematic, and suffers no exception.

Due to the constraint against more than one generic argument per clause (§ XXX), the only case that the generic pronoun can appear two times in the same clause occurs in reflexive constructions, as in (\ref{ex:tWZo.kW.tWZo}).

\begin{exe}
\ex \label{ex:tWZo.kW.tWZo}
\gll tɯʑo kɯ tɯʑo tu-kɯ-nɯ-ʑɣɤ-βri ra kɤ-ti ɲɯ-ŋu \\
\textsc{genr} \textsc{erg} \textsc{genr} \textsc{ipfv}-\textsc{genr}:S/P-\textsc{auto}-\textsc{refl}-protect have.to:\textsc{fact} \textsc{inf}-say \textsc{sens}-be \\
\glt `One has to protect oneself.' (04-qala1, 25)
\end{exe} 

\section{Genitive forms} \label{sec:pronouns.gen}
The form of pronouns and personal prefixes undergoes few morphophonological changes in combination with postpositions and relational nouns. However, in combination with the genitive postposition \forme{ɣɯ} (cf \ref{sec:genitive}), some  personal pronouns have special forms indicated in Table  \ref{tab:pronoun.gen}.

\begin{table}[h] \centering
\caption{Pronouns and possessive prefixes }\label{tab:pronoun.gen}
\begin{tabular}{lllllllll} \lsptoprule
 Free pronoun & Genitive & \\
\midrule
 \forme{aʑo}  &	\forme{aʑɯɣ}  &		\textsc{1sg} \\ 
\forme{nɤʑo}  &	\forme{nɤʑɯɣ}  &			\textsc{2sg} \\ 
\forme{ɯʑo}  &	\forme{ɯʑɤɣ}  &			\textsc{3sg} \\ 
\forme{tɕiʑo}  &	\forme{tɕiʑɤɣ}  &			\textsc{1du} \\ 
\forme{ndʑiʑo}  &	\forme{ndʑiʑɤɣ}  &		\textsc{2du} \\	 
\forme{ʑɤni}  &	\forme{ʑɤniɣɯ}  &		\textsc{3du} \\	 
\forme{iʑo}  &	\forme{iʑɤɣ}, 	\forme{iʑɤra ɣɯ}   &			\textsc{1pl} \\ 
\forme{nɯʑo}  &	\forme{nɯʑɤɣ}, 	\forme{nɯʑɤra ɣɯ}  &			\textsc{2pl} \\ 
\forme{ʑara}  &	\forme{ʑaraɣ},   \forme{ʑara ɣɯ}&			\textsc{3pl}  \\  
\lspbottomrule
\end{tabular}
\end{table}

While some degree of variation exists with dual and plural pronouns (for instance the regular \forme{iʑo ɣɯ} is found alongside \forme{iʑɤɣ} and \forme{iʑɤra ɣɯ}), for the singular pronouns only one form is attested.

\begin{exe}
\ex
\gll aʑɯɣ 	ndʐa 	ŋu 	ɕi, 	nɤʑɯɣ 	ndʐa 	ŋu, 	aj 	mɯ́j-tso-a   \\
\textsc{1sg:gen} reason be:\textsc{fact} \textsc{qu} \textsc{2sg:gen} reason be:\textsc{fact} \textsc{1sg} \textsc{neg:sens}-understand-\textsc{1sg} \\
\glt  `I don't know if it is because of me, or because of you.' (that the phone line is not working well) (phone conversation, 2011) %\wav{8_ndzxa})
\end{exe} 

In the genitive forms of the pronouns, the vowel of the genitive marker is generally dropped, and the pronominal root \forme{-ʑo} undergoes vowel change to \forme{-ʑɯɣ} (in the case of first and second person) and \forme{-ʑɤɣ} (in other forms). Note that \forme{ʑaraɣ} is the only case of the rhyme \ipa{aɣ} in Japhug.

When genitive pronouns occur as determiners of nouns (including in the possessive existential construction, see § XXX), these nouns almost always take a possessive prefix coreferent with the genitive pronoun, as in (\ref{ex:tɕithAfkAlAGi}).

\begin{exe}
\ex \label{ex:tɕithAfkAlAGi}
\gll 
tɕiʑɤɣ tɕi-tʰɤfkɤlɤɣi tɯ-ɕkat pɯ-tu tɕe, nɯ kɤ-nɯ-χtɤr-tɕi ɕti wo \\
\textsc{1du:gen} 2du-plant.ash one-load \textsc{pst.ipfv}-exist \textsc{lnk} \textsc{dem} \textsc{pfv-auto}-spread-\textsc{1du} be:\textsc{affirm}:\textsc{fact} \textsc{sfp} \\
\glt `We had one load of plant ash, and spill it there.' (2003kunbzang, 171)
\end{exe} 

The genitive pronouns can be used as possessive pronouns (`mine', `my own' etc) and take the determiner \forme{nɯ} and the plural \forme{ra}, as in (\ref{ex:aZWG.nW}) and (\ref{ex:WZAG.nWra}).

\begin{exe}
\ex \label{ex:aZWG.nW}
\gll ``tɕe ɣnɤsqaptɯ-rʑaʁ tu-tsu tɕe ɲɯ-ʁaʁ ŋu" ɲɯ-ti-nɯ ri, aʑɯɣ nɯ ɣnɤsqamnɯz tɤ-rʑaʁ mɤɕtʂa mɯ-nɯ-ʁaʁ. \\
\textsc{lnk} eleven-night \textsc{ipfv}-pass \textsc{lnk} \textsc{ipfv}-hatch be:\textsc{fact} \textsc{sens}-say-\textsc{pl} \textsc{lnk} \textsc{1sg:gen} \textsc{dem} twelve one-night  until \textsc{neg-pfv}-hatch \\
\glt `People say that (chicken eggs) hatch after eleven days, mine took twelve days to hatch.' (150819 kumpGa)
\end{exe} 

\begin{exe}
\ex \label{ex:WZAG.nWra}
\gll ɯʑɤɣ nɯra tu-nɯ-ɣɤ-βdi tɕe, ɕɯ-sɤ-sqɤr mɤ-ra \\
\textsc{3sg:gen} \textsc{dem:pl} \textsc{ipfv-auto-caus}-be.good \textsc{lnk} \textsc{transloc-antipass}-hire \textsc{neg}-have.to:\textsc{fact} \\
\glt `He repairs his own (machines) himself, he does not need to ask other people.' (14-tApitaRi, 168)
\end{exe} 

\section{The emphatic use of pronouns} \label{sec:pronouns.emph}
In addition to their referential and anaphoric functions, pronouns in Japhug can be used in an emphatic way in combination with the particle \forme{ʑo}, as in  (\ref{ex:WZo.Zo}).

\begin{exe}
\ex \label{ex:WZo.Zo}
\gll aʑo ɯʑo ʑo kɤ-mto mɯ-pɯ-rɲo-t-a. \\
\textsc{1sg} \textsc{3sg} \textsc{emph} \textsc{inf}-see \textsc{neg-pfv}-experience-\textsc{pst:tr-1sg} \\
\glt `I never saw it itself.' (24-kWmu, 7)
\end{exe} 

In combination with the autobenefactive \forme{nɯ-} on the verb, pronouns express the meaning `do X on one's own'. In the case of transitive verbs, the pronoun in this use does not take the ergative even if the referent is the transitive subject (example \ref{ex:pjWnWtCAtnW}, where \japhug{tɕɤt}{take out} is transitive).

\begin{exe}
\ex
\gll tɕe ɲɯ-tɯ-nɤm qhe, tɕe ʑara ku-nɯ-nɯɣi-nɯ ŋu ɕi? \\
\textsc{lnk} \textsc{ipfv:east}-2-chase[III] \textsc{lnk} \textsc{lnk} \textsc{3pl} \textsc{ipfv:west}-auto-come.back-\textsc{pl} be:\textsc{fact} \textsc{qu} \\
\glt `Do you chase them, or do they come back home on their own?' (taRrdo conversation, 29)
\end{exe} 

\begin{exe}
\ex \label{ex:pjWnWtCAtnW}
\gll tɕe lu-nɯ-rɤji-nɯ tɕe, nɯ-kɤ-ndza nɯra ʑara pjɯ-nɯ-tɕɤt-nɯ pjɤ-ŋu tɕe \\
\textsc{lnk} \textsc{ipfv-auto}-plant.crops-\textsc{pl} \textsc{lnk} \textsc{3pl.poss-nmlz:P}-eat \textsc{dem:pl} \textsc{3pl} \textsc{ipfv-auto}-take.out-pl \textsc{ifr.ipfv}-be \textsc{lnk} \\
\glt `They planted crops, and earned their food on their own.' (about lepers, who were settled in the special place by the government, 25-khArWm, 70)
\end{exe}

This construction is also attested with first of second person pronouns, as in (\ref{ex:aZo.Zo}).

\begin{exe}
\ex \label{ex:aZo.Zo}
\gll aʑo ʑo nɯnɯ ɕ-pjɯ-sat-a ra \\
\textsc{1sg} \textsc{emph} \textsc{dem} \textsc{transloc-ipfv}-kill-\textsc{1sg} have.to:\textsc{fact} \\
\glt `I have to kill her myself.' (140504 baixuegongzhu, 117)
\end{exe}

The emphatic pronoun \japhug{raŋ}{oneself} borrowed from Tibetan \tibet{རང་}{raŋ}{oneself}, can also be used with any person, though this usage is not very common. It can occur with the autobenefactive (\ref{ex:aZo.raN}) or without it (\ref{ex:aZo.raN.Zo}).

\begin{exe}
\ex \label{ex:aZo.raN}
\gll
nɤʑo tu-tɯ-ti mɤ-ra ma aʑo raŋ tu-nɯ-ti-a jɤɣ \\
\textsc{2sg} \textsc{ipfv}-2-say \textsc{neg}-have.too:\textsc{fact} \textsc{lnk} \textsc{1sg} oneself \textsc{ipfv}-\textsc{auto}-say-\textsc{1sg} be.possible:\textsc{fact} \\
\glt `You don't need to say it, I can say it myself.' (elicited)
\end{exe}

\begin{exe}
\ex \label{ex:aZo.raN.Zo}
\gll aʑo raŋ ʑo ju-ɕe-a ra \\
\textsc{1sg} oneself \textsc{emph} \textsc{ipfv}-go-\textsc{1sg} have.to:\textsc{fact} \\
\glt `I have to go there myself.' (150830 afanti-zh, 96)
\end{exe}
\section{Interrogative pronouns}
The interrogative pronouns in Japhug are indicated in Table \ref{tab:interrog.pronoun}. These pronouns are used in independent interrogative clauses (\ref{ex:tChi.pWNu}), in subordinate clauses (\ref{ex:tChi.kWNu}), and also in correlatives (\ref{ex:NotCu.WsAzrAZi}), and also occur to express non-specific referents (these uses are described in section  \ref{sec:interrogative.indef}, after the indefinite pronouns).

\begin{exe}
\ex \label{ex:tChi.pWNu}
\gll
tɕe mɤʑɯ tɕʰi pɯ-ŋu? \\
\textsc{lnk} yet what \textsc{pst.ipfv}-be \\
\glt `What was there (after this one)?' (12-ndZiNgri, 100)
\end{exe}  

\begin{exe}
\ex \label{ex:tChi.kWNu}
\gll ɯʑo tɕʰi kɯ-ŋu nɯ ko-tso-nɯ tɕe tɕe cʰɤ́-wɣ-tɕɤt \\
\textsc{3sg} what \textsc{nmlz}:S/A-be \textsc{dem} \textsc{ifr}-understand-\textsc{pl} \textsc{lnk} \textsc{lnk} \textsc{ifr:downstream-inv}-take.out \\
\glt `They understood what he was, and expelled him (from their group).' (140427 hanya yu gezi-zh, 19)
\end{exe}  

\begin{exe}
\ex \label{ex:NotCu.WsAzrAZi}
\gll 
ɯ-pʰoŋbu tɕʰi kɯ-fse nɯ, ŋotɕu ɯ-sɤz-rɤʑi nɯnɯ ɣɯ kɯ-nɯtsa kɯ-fse ɲɯ-ɕti tɕe \\
3sg.poss-body what \textsc{nmlz}:S/A-be.like \textsc{dem} where \textsc{3sg-nmlz:oblique}-remain \textsc{dem} \textsc{gen}  nmlz:S/A-fit \textsc{nmlz}:S/A-be.like \textsc{sens}-be:\textsc{affirm} \textsc{lnk} \\
\glt `The way its body is like is well-fitted to the place where it lives.' (19-rNamoN, 24)
\end{exe}  

\begin{table}[h] \centering
\caption{Interrogative pronouns }\label{tab:interrog.pronoun}
\begin{tabular}{lllllllll} \lsptoprule
\japhug{tɕʰi}{what} \\
\japhug{ɕɯ}{who} \\
\japhug{tʰɤstɯɣ}{how many} \\
\japhug{tʰɤjtɕu}{when} \\
\japhug{ŋotɕu}{where}, \japhug{ŋoj}{where} \\
\japhug{tɕʰindʐa}{why} \\
\lspbottomrule
\end{tabular}
\end{table}

In addition to these pronouns, some indefinite pronouns are also marginally used in questions, see for instance (\ref{ex:thWthAci.totia}) in § \ref{sec:thWci}.

\subsection{\japhug{tɕʰi}{what}} \label{sec:tChi}
The interrogative pronoun  `what' considerably varies across Japhug dialects. In Kamnyu we find \forme{tɕʰi}, apparently borrowed from Tibetan \forme{tɕʰi}. Neighbouring dialects of Gdongbrgyad area have either \forme{tsʰi} (in Mangi) or \forme{tʰi} (in Rqaco), which represents the original Rgyalrongic root for this interrogative pronoun (cognate with Tibetan \tibet{ཆི་}{tɕʰi}{what} and Limbu \forme{the}). Even in the Kamnyu dialect, the form \forme{tsʰi-} is directly attested in the indefinite \japhug{tsʰitsuku}{some} (\ref{sec:tshitsuku}). Mangi Japhug shares with Kamnyu the sound change \forme{*tʰi} \fl{} \forme{tsʰi} which also affects the verb \japhug{tsʰi}{drink} (this sound change occurred after the pronoun  \forme{*tʰi} underwent \textit{status constructus} alternation to \forme{tʰɯ-} and was used to build the indefinite pronoun \japhug{tʰɯci}{something}, see \ref{sec:thWci}). Note that Kamnyu Japhug \japhug{tɕʰi}{what} is homophonous with the noun \japhug{tɕʰi}{tree-trunk stairs} attested for instance in example (\ref{ex:tChi.tukWndW}) -- Japhug texts readers have to be aware of potential ambiguities.

\begin{exe}
\ex \label{ex:tChi.tukWndW}
\gll
ɕom ɣɟɯ kɯ-mbɯ\redp{}mbro ʑo,  ɯ-ɣmbaj zɯ tɕʰi tu-kɯ-ndɯ ci pɯ-tu ɲɯ-ŋu \\
iron tower \textsc{nmlz}:S/A-\textsc{emph}\redp{}be.high \textsc{emph} \textsc{3sg}.\textsc{poss}-side \textsc{loc} treetrunk.stairs \textsc{ipfv}:\textsc{up}-\textsc{nmlz}:S/A-\textsc{acaus}:spread \textsc{indef} \textsc{pst}.\textsc{ipfv}-be \textsc{sens}-be \\
\glt `There was a huge iron toward, with a tree-trunk stairs on its side.' (Norbzang05, 65)
\end{exe}  

The Eastern dialects of Gsardzong and Datshang have \forme{xto} instead, a word of unknown etymology.

In the Kamnyu dialect, \japhug{tɕʰi}{what} is by far the most common interrogative pronoun in the corpus. In interrogative clauses, it can be used to ask about objects, non-human animals (\ref{ex:nAmbro}) and names of persons (\ref{ex:tChi.tWrmi}).

\begin{exe}
\ex \label{ex:nAmbro}
\gll
nɤʑo nɤ-mbro nɯ tɕʰi ŋu \\
\textsc{2sg} 2sg.poss-horse \textsc{dem} what be:\textsc{fact} \\
\glt `Who is your horse?' (about a sentient horse, 2003smanmi-tamu, 53)
\end{exe}  

\begin{exe}
\ex \label{ex:tChi.tWrmi}
\gll tɕʰi tɯ-rmi? \\
what 2-be.called:\textsc{fact} \\
\glt `What is your name?' (heard in context)
\end{exe}  

As in many languages, this interrogative pronoun (instead of the pronoun \japhug{ɕɯ}{who}) is also used in questions about classification of persons (\citealt{idiatov07nonselective}), including social affiliation (\ref{ex:tChi.WrWG}, and \ref{ex:tChi.kWNu} above) and biological affiliation (\ref{ex:tChi.tosci}).

\begin{exe}
\ex \label{ex:tChi.WrWG}
\gll ɯtɤz nɯʑo tɕʰi ɯ-rɯɣ tɯ-ŋu-nɯ? \\
finally \textsc{2pl} what  3sg.poss-race  2-be:\textsc{fact}-\textsc{pl} \\
\glt `Finally, what race (of being) are you?' (smanmi2003, 172)
\end{exe}  

There is no specific interrogative pronoun to ask about manner like English `how', and Japhug expresses this meaning by combining \forme{tɕʰi} with the verbs \japhug{fse}{be like...} or \japhug{stu}{do like...}, as in examples (\ref{ex:tChi.tAtWfsendZi}), (\ref{ex:tChi.atAfsej}) and (\ref{ex:tChi.Zo.tuwGBzu}).

\begin{exe}
\ex \label{ex:tChi.tAtWfsendZi}
\gll a-ʁi, ki kɯ-fse tɤjpɣom kɯ-wxti nɯtɕu, kɤ-ɕe tɕʰi tɤ-tɯ-fse-ndʑi?? \\
\textsc{1sg.poss}-younger.sibling this \textsc{nmlz}:S/A-be.like ice \textsc{nmlz}:S/A-be.big \textsc{dem:loc} inf-go what \textsc{pfv}-2-be.like-\textsc{du} \\
\glt `Sister, how did you cross such a big block of ice?' (stodtWphu2005, 156)
\end{exe}  

   \begin{exe}
\ex \label{ex:tChi.atAfsej}
\gll  kɤ-pʰɣo tɕʰi a-tɤ-fse-j    \\
\textsc{inf}-flee what \textsc{irr-pfv}-be.like-\textsc{1pl} \\
\glt  `How will we flee?' (Norbzang 69)
\end{exe} 

\begin{exe}
\ex \label{ex:tChi.Zo.tuwGBzu}
\gll nɤ-smɤn tɤ-sɯ-βzu-t-a ri maka mɯ́j-pʰɤn, tɕe tɕʰi ʑo tú-wɣ-stu pʰɤn \\
\textsc{3sg.poss}-medicine \textsc{pfv-caus}-make-\textsc{pst:tr-1sg} but at.all \textsc{neg:sens}-be.efficient \textsc{lnk} what \textsc{emph} \textsc{ipfv-inv}-do.like be.efficient:\textsc{fact} \\
\glt `I had medicine made for you but it does not work, how should we do for it to work?' (nyima wodzer 2002, 22) 
\end{exe}  

 
The pronoun \japhug{tɕʰi}{what} on its own can occur in questions about the reason or the purpose of a particular state of affair, as in (\ref{ex:tChi.apWNua}) and (\ref{ex:tChi.YWtWnAre}).

\begin{exe}
\ex \label{ex:tChi.apWNua}
\gll  aʑo tɕʰi a-pɯ-ŋu-a? \\
\textsc{1sg} what \textsc{irr-ipfv}-be-1sg \\
\glt `How can it be me?' (2003sras, 61)
\end{exe}  

\begin{exe}
\ex \label{ex:tChi.YWtWnAre}
\gll  a-tɤɕime, tɕʰi ɲɯ-tɯ-nɤre ŋu? \\
 \textsc{1sg.poss}-lady what \textsc{sens}-2-laugh be:\textsc{fact} \\
 \glt `My lady, why are you laughing?'  (Not `what are you laughing at ?', 2002qaCpa, 102)
\end{exe}  

When referring to purpose or reason, it is possible to combine  \japhug{tɕʰi}{what} with the nouns \japhug{ɯ-spa}{its material} and \japhug{ɯ-ndʐa}{its reason} (as the pronoun \japhug{tɕʰindʐa}{why})  respectively, as in (\ref{ex:tChi.Wspa.pWNu}) and (\ref{ex:tChi.YWtWɣAwu}). Note that examples (\ref{ex:tChi.YWtWnAre}) and (\ref{ex:tChi.YWtWɣAwu}) are from the same story, just a few lines away, in the same context; the construction in (\ref{ex:tChi.YWtWɣAwu}) is a more explicit variant of that in (\ref{ex:tChi.YWtWnAre}).

\begin{exe}
\ex \label{ex:tChi.Wspa.pWNu}
\gll tɕe tɕʰi ɯ-spa pɯ-ŋu mɤ-xsi ma tɕe nɯ kɯ-fse pjɤ-tu  \\
\textsc{lnk} what \textsc{3sg.poss}-material \textsc{pst.ipfv}-be \textsc{neg-genr}:know \textsc{lnk} \textsc{lnk} \textsc{dem} \textsc{nmlz}:S/A-be.like \textsc{ifr.ipfv}-exist \\
\glt `It is not known what it was for, but there was something like that.' (hist140522 GJW, 18)
\end{exe}  

\begin{exe}
\ex \label{ex:tChi.YWtWɣAwu}
\gll tɕʰindʐa ɲɯ-tɯ-ɣɤwu ŋu? \\
why \textsc{sens}-2-cry be:\textsc{fact} \\
\glt `Why are you crying?' (2002qaCpa, 94)
\end{exe} 

The pronoun \forme{tɕʰi} takes case marking with genitive \forme{ɣɯ} and the instrumental/ergative \forme{kɯ}, as in (\ref{ex:tChi.kW}).

\begin{exe}
\ex \label{ex:tChi.kW}
\gll tɕe tɕʰi kɯ tu-sɯ-βze ŋu mɤxsi ma nɯ kɯ-fse nɯ, sɯku ri ku-ndzoʁ ŋu \\
\textsc{lnk} what \textsc{erg} \textsc{ipfv}-\textsc{caus}-make[III] be:\textsc{fact} \textsc{neg}-\textsc{genr}-know \textsc{lnk} \textsc{dem} \textsc{nmlz}:S/A-be.like \textsc{dem} top.of.trees \textsc{loc} \textsc{ipfv}-\textsc{anticaus}:attach be:\textsc{fact} \\
\glt `I don't what it (the wasp) uses to make it (its nest), it is attached on trees.' (26-ndzWrnaR, 55)
\end{exe} 
In combination with the adverb \forme{jarma} / \japhug{jamar}{about}, it can be used to indicate a quantity, instead of \japhug{tʰɤstɯɣ}{how many} (section \ref{sec:thAstWG}).

\begin{exe}
\ex \label{ex:tChi.jamar}
\gll tu-ɕtʂam-a tɕe tɕʰi jamar ʑo ɣɤʑu kɯ? \\
\textsc{ipfv}-measure[III]-\textsc{1sg} \textsc{lnk} what about \textsc{emph} exist:\textsc{sens} \textsc{sfp} \\
\glt `I will measure it with a scoop to see how much (gold) there is.' (140512alibaba-zh, 59)
\end{exe}  

\begin{exe}
\ex \label{ex:tChi.jamar.kondza}
\gll kʰɯtsa ɯ-ŋgɯ tɯ-ci tu-rku-nɯ tɕe, nɯnɯtɕu tɤŋe nɯ pjɯ-sɯ-ntɕʰɤr-nɯ tɕe, tɕe tɕʰi jamar ko-ndza nɯnɯ, nɯnɯ ɯ-ŋgɯ nɯtɕu pjɯ-ru-nɯ tɕe,  nɯnɯ tu-rtoʁ-nɯ pjɤ-ŋgrɤl.   \\
bowl \textsc{3sg}-inside \textsc{indef.poss}-water \textsc{ipfv}-put.in-\textsc{pl} \textsc{lnk} \textsc{dem:loc} sun \textsc{dem} \textsc{ipfv-caus}-illuminate-\textsc{pl} \textsc{lnk} \textsc{lnk} what about \textsc{ifr}-eat \textsc{dem} \textsc{dem} \textsc{3sg}-inside \textsc{ipfv}:\textsc{down}-look.at-\textsc{pl} dem \textsc{ipfv}-see-\textsc{pl} \textsc{ifr.ipfv}-be.usually.the.case \\
\glt `They used to put water in a bowl and let the sunlight reflect into it; they could see how much (of the sun) had been occulted (`eaten' by the eclipse).' (29-mWBZi, 130)
\end{exe}  

\begin{exe}
\ex
\gll  zgo 	tʰɤstɯɣ 	ja-nnɯ-pɣaʁ-ndʑi, 	tɯ-ci 	tɕʰi 	jarma 	ja-nnɯ-pjɤl-ndʑi 	mɤ-xsi 	ma,       \\
 mountain how.many \textsc{pfv}:3\fl3'-\textsc{auto}-turn.over-\textsc{du} \textsc{indef.poss}-water what about \textsc{pfv}:3\fl3'-\textsc{auto}-cross-\textsc{du} \textsc{neg-genr}:know \textsc{lnk} \\
\glt `It is not known how many mountains and rivers they crossed.'  (2002qajdo, 50)
\end{exe}  

It is possible to combine \forme{tɕʰi jamar} with a adjective to express approximate comparison, as in (\ref{ex:tChi.kWzri}).

\begin{exe}
\ex \label{ex:tChi.kWzri}
\gll lɯlu ɣɯ tɕe ɯʑo ɯ-pʰoŋbu tɕʰi kɯ-zri jamar ɯ-jme nɯ kɯnɤ zri ri \\
cat \textsc{gen} \textsc{lnk} \textsc{3sg} \textsc{3sg.poss}-body what \textsc{nmlz}:S/A-be.long about \textsc{3sg.poss}-tail \textsc{dem} also be.long\textsc{fact} but \\
\glt `The cat, its body is about as long as its tail, but...' (27-qartshAz, 219)
\end{exe}  

In correlative clauses, the pronoun \japhug{tɕʰi}{what} can also be used to refer to a quantity without the adverb \japhug{jamar}{about} (example \ref{ex:tChi.tAkWsci}).

\begin{exe}
\ex \label{ex:tChi.tAkWsci}
\gll  
tɤ-rɟit tɕʰi tɤ-kɯ-sci nɯ ʑo ɣɯ-tɕɤt kɯ-ra pjɤ-ɕti tɕe,   \\
\textsc{indef.poss}-child what \textsc{pfv-nmlz:S/A}-be.born \textsc{dem} \textsc{emph} \textsc{inv}-take.out:\textsc{fact} \textsc{inf:stat}-have.to \textsc{ipfv.ifr}-be:\textsc{affirm} \textsc{lnk} \\
\glt `However many children were born, one had to raise them.' (tApAtso kAnWBdaR I, 9)
\end{exe}  

However, in independent interrogative clauses, \japhug{tɕʰi}{what} cannot refer to quantities. Sentence (\ref{ex:tChi.tosci}) thus can only mean `Was it a boy or a girl' not `How many children did she have?'.

\begin{exe}
\ex \label{ex:tChi.tosci}
\gll  ɯ-rɟit tɕʰi to-sci \\
\textsc{3sg.poss}-child what \textsc{ifr}-be.born \\
\glt `Was it a boy or a girl?'
\end{exe}  


The interrogative \japhug{tɕʰi}{what} occurs in topicalized clauses with an adjective stative verb in perfective form, meaning `as for how X it becomes' as in examples (\ref{ex:tChi.nWjpum}) and (\ref{ex:tChi.tAmbro}).

\begin{exe}
\ex \label{ex:tChi.nWjpum}
\gll tɕʰi nɯ-jpum ki ɕaŋtaʁ ɲɯ-jpum mɯ́j-cʰa \\
what \textsc{pfv}-be.thick \textsc{dem}.\textsc{prox} above \textsc{ipfv}-be.thick \textsc{neg}:\textsc{sens}-can \\
\glt `As for how thick it can grow, it cannot grow thicker than this.' (16-CWrNgo, 154)
\end{exe}


\begin{exe}
\ex \label{ex:tChi.tAmbro}
\gll tɕʰi tɤ-mbro, ʁnɯ-rtsɤɣ ɕaŋtaʁ tu-mbro mɯ́j-cʰa.  \\
what \textsc{pfv}-be.tall two-stairs above \textsc{ipfv}-be.tall \textsc{neg}:\textsc{sens}-can \\
\glt `As for how tall it can grow, it cannot grow taller than two stairs.' (07-paXCi, 8)
\end{exe}

\subsection{\japhug{ɕɯ}{who}}
The interrogative pronoun \japhug{ɕɯ}{who} occurs in questions about the identification of a human referent. It can occur in all syntactic roles, and does not have special ergative or genitive forms (see examples \ref{ex:CW.kW.tWwGmbi} and \ref{ex:CW.GW}). It is the probable cognate of a etymon widespread in the Trans-Himalayan family (for instance, Tibetan \tibet{སུ་}{su}{who}).

\begin{exe}
\ex  \label{ex:CW.tWNu}
\gll ma-tɯ-nɯqaɟy ma ɕɯ tɯ-ŋu mɤ-xsi \\
\textsc{neg:imp}-2-fish \textsc{lnk} who 2-be:\textsc{fact} \textsc{neg-genr}:know   \\
\glt `Don't fish, I don't who you are.' (gesar, 369)
\end{exe}  

\begin{exe}
\ex  \label{ex:CW.kW.tWwGmbi}
\gll  mɤ-ta-mbi nɤʑo qaɕpa ɕɯ kɯ tɯ́-wɣ-mbi    \\
\textsc{neg}-1\fl2-give:\textsc{fact} \textsc{2sg} frog who \textsc{erg} 2-\textsc{inv}-give:\textsc{fact}  \\
\glt `We won't give her to you, who would give her to you, a frog?'   (2002 qaCpa, 09)
\end{exe} 
 
\begin{exe}
\ex  \label{ex:CW.GW}
\gll  ɕɯ ɣɯ ʑo ɲɯ-kʰam-a ra kɯɣe?    \\
who \textsc{gen} \textsc{emph} \textsc{ipfv}-give:III-\textsc{1sg} \textsc{sfp} \\
\glt `Whom should I give (her) to (in marriage)?' (140508 benling gaoqiang de si xiongdi-zh, 222)
\end{exe}  

The pronoun  \japhug{ɕɯ}{who} can be used in one context with non-human referents, when asking about which object (out of two or more) has the highest value as to a property described by the main verb, as in (\ref{ex:CW.kW.YWzrindZi}); in this construction, the verb receives non-singular indexation (§ XXX), such as the dual  \forme{-ndʑi} in this example. Concerning the use of the ergative \forme{kɯ} in this sentence see \citet{jacques16comparative} and § XXX.  

\begin{exe}
\ex  \label{ex:CW.kW.YWzrindZi}
\gll nɯ nɤ-ku ɯ-tɯ-rɲɟi nɯ, aki ɕe-tɕi tɕe, mbro ɯ-jme cʰonɤ tú-wɣ-sɤfsu, ɕɯ kɯ ɲɯ-zri-ndʑi kɯ \\
\textsc{dem} \textsc{2sg}.\textsc{poss}-head \textsc{3sg}.\textsc{poss}-\textsc{nmlz}:\textsc{degree}-be.long  \textsc{sfp} down go:\textsc{fact}-\textsc{1du} \textsc{lnk} horse \textsc{3sg}.\textsc{poss}-tail \textsc{comit}  \textsc{ipfv}-\textsc{inv}-compare who \textsc{erg} \textsc{sens}-be.long-\textsc{du} \textsc{sfp} \\
\glt `Your hair is very long, let us go downstairs, and compare it with a horse's tail.' (2002qaCpa, 292)
\end{exe}  

Forms related to \japhug{ɕɯ}{who} in Japhug include the indefinite pronoun \japhug{ɕɯmɤɕɯ}{whoever, anybody} (\ref{sex:CWmACW}) and \japhug{ɕɯŋarɯra}{each better than the other} (XXX).

\subsection{\japhug{tʰɤstɯɣ}{how many} and \japhug{tʰɤjtɕu}{when}} \label{sec:thAstWG}
To ask about precise quantities, \japhug{tʰɤstɯɣ}{how many} (or `how much') occurs rather than \forme{tɕʰi jamar} as seen above (section \ref{ex:tChi.jamar}).

\begin{exe}
\ex \label{ex:thAstWG.tWkhAm}
 \gll    nɤʑo 	tʰɤstɯɣ 	tɯ-kʰɤm?    \\
 you how.much 2-give[III]:\textsc{fact}  \\
\glt  `How much (money) do you give (for it)?' (Bargaining, 13)
\end{exe} 

It can be used for any countable quantity, including for people, as in (\ref{ex:thAstWG.tWtunW}).

\begin{exe}
\ex \label{ex:thAstWG.tWtunW}
 \gll
tsʰupa tʰɤstɯɣ tɯ-tu-nɯ ŋu? \\
village how.much 2-exist:\textsc{fact}-\textsc{pl} be:\textsc{fact} \\
\glt `How many (people) are you in the village?' (conversation, 140501)
\end{exe} 

The pronoun \forme{tʰɤstɯɣ} has a conjunct form \forme{tʰɤstɯ-} when used with counted nouns (in \ref{ex:thAstWmaR}, with the classifier \japhug{X-maʁ}{size of shoes} from Chinese \zh{码} \forme{mǎ}, see § \ref{sec:other.numeral.prefixes}).

 \begin{exe}
\ex \label{ex:thAstWmaR}
 \gll   nɤ-xtsa nɯ tʰɤstɯ-maʁ tu-tɯ-ŋge ŋu   \\
\textsc{2sg.poss}-shoe \textsc{dem} how.many-size \textsc{ipfv}-2-wear[III] be:\textsc{fact} \\ 
\glt `What is the size of your shoes?'  (Conversation, 2015)
\end{exe} 

Combined with the noun \japhug{tɤ-rʑaʁ}{time}, 	\forme{tʰɤstɯɣ} can be used to ask about a length of time (\ref{ex:thAstWG}).

\begin{exe}
\ex \label{ex:thAstWG}
 \gll   nɤʑo 	tɤ-rʑaʁ 	tʰɤstɯɣ 	jamar 	tɤ-tsu tɕe 	kɤ-tɯ-spa-t?  \\
 you \textsc{indef.poss}-time how.many about \textsc{pfv}-pass \textsc{lnk} \textsc{pfv}-2-be.able-\textsc{pst:tr} \\
\glt   `How long did you need to learn it?' (elicited)
\end{exe} 

The phrase \forme{tɤ-rʑaʁ tʰɤstɯɣ} (or alternatively \forme{tɯtsʰot tʰɤstɯɣ}) in collocation with the verb \japhug{zɣɯt}{reach}, is also employed for asking about clock time, as in (\ref{ex:thAstWG.kozGWt}) (see § \ref{sec:hours}) or dates. %The nouns \japhug{tɤ-rʑaʁ}{time} and \japhug{tɯtsʰot}{time, hour, clock} are not even obligatory, as shown by  

 \begin{exe}
\ex \label{ex:thAstWG.kozGWt}
 \gll   tɤ-rʑaʁ 	tʰɤstɯɣ ko-zɣɯt? \\
  \textsc{indef.poss}-time how.many  \textsc{ifr}-reach \\
  \glt `What is the time?' (heard in context)
  \end{exe} 
    
Questions about time can also be expressed by the pronoun \japhug{tʰɤjtɕu}{when}, as in  (\ref{ex:thAjtCu}) and (\ref{ex:thAjtCu.GW}).

\begin{exe}
\ex \label{ex:thAjtCu}
\gll  tʰɤjtɕu 	lɤ-tɯ-nɯɣe 	pɯ-ŋu 	ra 	nɤ?    \\
 when \textsc{pfv}-2-come.back[II] \textsc{pst.ipfv}-be \textsc{pl} \textsc{sfp} \\
\glt  `When did you come back home?' (taRrdo conversation, 01)
\end{exe} 

As shown by (\ref{ex:thAjtCu.GW}), \japhug{tʰɤjtɕu}{when} can be used with the genitive \forme{ɣɯ}.

\begin{exe}
\ex \label{ex:thAjtCu.GW}
\gll <jipiao> nɯ, tʰɤjtɕu ɣɯ tɤ-tɯ-χtɯ-t? \\
plane.ticket \textsc{dem} when \textsc{gen} \textsc{pfv}-2-buy-\textsc{pst}:\textsc{tr} \\
\glt `Your plane ticket, for what date did you buy it?' (conversation, 2014.03.19)
\end{exe} 

The element \ipa{tʰɤ-} in the pronouns \japhug{tʰɤjtɕu}{when}  and \japhug{tʰɤstɯɣ}{how many} is the \textit{status constructus} form of proto-Japhug \forme{*tʰi}, the inherited form of the pronoun `what' (see § \ref{sec:tChi}). The element \forme{-tɕu} in \japhug{tʰɤjtɕu}{when} is related to the locative \forme{tɕu} (see § XXX).

\subsection{\japhug{ŋotɕu}{where}} \label{sec:NotCu}

The interrogative pronoun \japhug{ŋotɕu}{where} and its reduced form \forme{ŋoj} can be used to ask either about a location (\ref{ex:NotCu.kutWrAZi}), a direction towards (examples \ref{ex:NotCu.tWCe} and \ref{ex:Noj.nari}) or from (\ref{ex:NotCu.jAtWGenW}) a certain place. The second syllable of this pronoun \forme{-tɕu} comes from the locative postposition \forme{tɕu}, but the first part is etymologically obscure.
 
\begin{exe}
\ex \label{ex:NotCu.kutWrAZi}
\gll     ŋotɕu ku-tɯ-rɤʑi?   \\
  where \textsc{pres.egoph}-2-stay \\
\glt `Where are you?" (Conversation, 2005)
\end{exe} 

\begin{exe}
\ex \label{ex:NotCu.tWCe}
\gll   ŋotɕu tɯ-ɕe? \\
 where 2-go:\textsc{fact} \\
\glt `Where are you going to?' (Common greeting used when one meets someone on the road)
 \end{exe} 
 
\begin{exe}
\ex \label{ex:Noj.nari}
\gll     qala ŋoj nɯ-ari  \\
  rabbit where \textsc{pfv:west}-go[II] \\
\glt `Where did the rabbit go?'  (qala2002, 21)
\end{exe} 

\begin{exe}
\ex \label{ex:NotCu.jAtWGenW}
\gll  nɯʑɤra ŋotɕu jɤ-tɯ-ɣe-nɯ? ŋotɕu ɕ-pɯ-tɯ-tu-nɯ? \\
\textsc{2pl} where \textsc{pfv}-2-come[II]-\textsc{pl} where \textsc{transloc-pfv}-2-exist-\textsc{pl} \\
\glt `Where are you from? Where have you been?' (2003sras, 57)
\end{exe} 

With the determiner \forme{nɯ}, the pronoun \forme{ŋotɕu} means `which (of several places)', as in (\ref{ex:NotCu.nW.Nu}) and (\ref{ex:NotCu.nW.Wku.Nu}).

\begin{exe}
\ex \label{ex:NotCu.nW.Nu}
\gll kʰa raŋri ɣɯ ʑo ɯ-ftaʁ pjɤ-tu ɕti ma, tɕe ŋotɕu nɯ ŋu, ŋotɕu nɯ maʁ mɯ-pjɤ-saχsɤl. \\
house each \textsc{gen} \textsc{emph} \textsc{3sg.poss}-mark \textsc{ifr.ipfv}-exist be:\textsc{affirm:fact} \textsc{lnk} \textsc{lnk} where \textsc{dem} be:\textsc{fact}  where \textsc{dem} be:\textsc{fact} \textsc{neg-ifr.ipfv}-be.clear \\
\glt `There was a mark on each of the houses, and one could not tell which (house) was (Alibaba's) and which was not.' (140512 alibaba-zh, 189-190)
\end{exe} 

\begin{exe}
\ex \label{ex:NotCu.nW.Wku.Nu}
\gll qaprɤftsa nɯnɯ, cici jɤ-ari tɕe ɯ-ku ju-z-mɤke, cici tɕe ɯ-jme ju-zmɤke ɲɯ-ɕti tɕe
ŋotɕu nɯ ɯ-ku ŋu, ŋotɕu ɯ-jme ŋu, mɯ́j-saχsɤl \\
centipede \textsc{dem} sometimes \textsc{pfv}-go \textsc{lnk} \textsc{3sg.poss}-head \textsc{ipfv-caus}-be.first[III] \textsc{lnk} \textsc{3sg.poss}-tail \textsc{ipfv-caus}-be.first[III] sens-be:affirm lnk where \textsc{dem} \textsc{3sg.poss}-head be:\textsc{fact} where \textsc{3sg.poss}-head be:\textsc{fact} \textsc{neg.sens}-be/clear \\
\glt `The centipede, when it moves, sometimes its head goes first, sometimesits tail goes first, it is not each to tell which is its head and which is its tail.' (21-qaprAftsa, 12)
\end{exe} 


With generic nouns such as \japhug{tɯrme}{person}, \forme{ŋotɕu} can serve as prenominal determiner to mean `a person from where', as in (\ref{ex:NotCu.tWrme}).

\begin{exe}
\ex \label{ex:NotCu.tWrme}
\gll ŋotɕu tɯrme tɯ-ŋu? \\
where person 2-be:\textsc{fact} \\
\glt `Where are you from?' (2011-05-nyima, 83)
\end{exe} 

In participial relatives with subject participle (in \forme{kɯ-}, see § XXX), \japhug{ŋotɕu}{where} can occur to express relativization of locative adjuncts, as in  (\ref{ex:NotCu.kWtu}); see § XXX for a discussion of the other available constructions.

\begin{exe}
\ex \label{ex:NotCu.kWtu}
\gll kɯ-me nɯra qʰe me,  ŋotɕu kɯ-tu nɯ qʰe kɯ-dɯ\redp{}dɤn tu-ɬoʁ ŋu. \\
\textsc{nmlz}:S/A-not.exist dem:pl lnk not.exist:\textsc{fact} where \textsc{nmlz}:S/A-exist \textsc{dem} \textsc{lnk} \textsc{nmlz}:S/A-\textsc{emph}\redp{}be.many \textsc{ipfv}-come.out be:\textsc{fact} \\
\glt `In (places) where it is not found, there is none, but in (places) where it is found, it grows in great number.' (21-jmAGni, 91)
\end{exe} 

The pronoun \japhug{ŋotɕu}{where} is not exclusively used in question about place or direction, we also find it in the expression in (\ref{ex:NotCu.YWNgrAl}).

 \begin{exe}
\ex \label{ex:NotCu.YWNgrAl}
\gll     kɯki 	ŋotɕu 	ɲɯ-ŋgrɤl?   \\
 this where \textsc{ipfv}-be.usually.the.case \\
\glt `How could this be possible?'  (qajdoskAt 2002, 32)
\end{exe} 

This sentence is used to express indignation (as in Chinese \zh{哪有这样的道理?}).\footnote{In the story from which it is quoted, the husband says this sentence after his wife, quoting the words of a raven, says that she will have luck, not her husband, who thus reacts in anger. }



\section{Indefinite pronouns} \label{sec:indef.pro}
 Japhug has a handful of indefinite pronouns, indicated in Table \ref{tab:indef.pronoun}. They do not form a complete paradigm, and other constructions, in particular generic nouns and free relatives occur to express meanings for which no indefinite pronoun exists (see § XXX).

There are no negative indefinite pronouns, and indefinite pronouns are almost never under the scope of negation (except in translations from Chinese). They also never occur as standard of comparison.\footnote{Examples such as `In Freiburg the weather is better than anywhere in Germany' (\citealt[2]{haspelmath97indef}) would not be expressible with an indefinite pronoun, see § XXX.}
 

\begin{table}[H] \centering
\caption{Indefinite pronouns }\label{tab:indef.pronoun}
\begin{tabular}{lllllll} \lsptoprule
\japhug{ci}{one, someone} \\
\forme{tʰɯci}, \japhug{tʰɯtʰɤci}{something} \\
\japhug{tsʰitsuku}{whatever} \\
\japhug{ɕɯmɤɕɯ}{whoever, anybody} \\
\japhug{ciscʰiz}{somewhere} \\ 
\lspbottomrule
\end{tabular}
\end{table}

\subsection{\japhug{ci}{someone} } \label{sec:ci.someone} 
There is no distinct indefinite pronoun `someone' in Japhug, but the numeral \japhug{ci}{one}, which has many additional functions (indefinite article, modifier and pronoun § \ref{sec:indef.article}, § \ref{sec:other.pro}, § \ref{sec:partitive.pronouns}, § \ref{sec:identity.modifier}, § \ref{sec:one.to.ten} and § XXX), can express this meaning as in (\ref{ci.kW.thaGWt}) and (\ref{ci.kW.tWrdoR}).

\begin{exe}
\ex \label{ci.kW.thaGWt}
\gll 
ɯ-lɤcu nɯtɕu qaʑo kɤtsa ci, ci kɯ kɤ-ntsɣe tha-ɣɯt ɲɯ-ŋu. \\
\textsc{3sg}.\textsc{poss}-upstream \textsc{dem}:\textsc{loc} sheep parent.and.child \textsc{indef} one \textsc{erg} \textsc{inf}-sell \textsc{pfv}:3\fl{}3'-bring \textsc{sens}-be \\
\glt `Upstream from there, (there was) a ewe and her young, that someone had brought them to sell.' (2003kandZislama, 202)
\end{exe}

\begin{exe}
\ex \label{ci.kW.tWrdoR}
\gll 
 tɯ-xpa tɕe ci kɯ tɯ-rdoʁ pjɤ-sat. \\
 one-year \textsc{lnk} one \textsc{erg} one-piece \textsc{ifr}-kill \\
 \glt `One year, someone killed one of them (wils geese).' (22-qomndroN, 43)
\end{exe}

This use of \forme{ci} is rare. The preferred construction to express the meaning `someone' involves the combination of the generic noun \japhug{tɯrme}{person} with the indefinite \forme{ci} (§ \ref{sec:tWrme.indefinite}).

\subsection{\japhug{tʰɯci}{something} } \label{sec:thWci} 
The indefinite pronoun \japhug{tʰɯci}{something} derives from the \textit{status constructus} of the proto-Japhug pronoun \forme{*tʰi} `what' (see \ref{sec:tChi} above) with the indefinite determiner and numeral \japhug{ci}{one}. Note that vowel alternation bleeds the sound change \ipa{*tʰi} \fl{}  \ipa{tsʰi}, otherwise a form such as $\dagger$\forme{tsʰɯci} would have been expected. Its reduplicated form \forme{tʰɯtʰɤci} has an irregular vocalism \ipa{ɤ} ($\dagger$\forme{tʰɯtʰɯci} would have been expected instead).

 It can designate specific referents, whose nature is known to the speaker but unknown to the addressee (as in \ref{ex:thWthAci.Zo.pjWtu}),\footnote{Example (\ref{ex:thWthAci.Zo.pjWtu}) is from a tale about a rabbit tricking a snow leopard; the difference of knowledge between the speaker and the addressee concerning the nature of the `something' is crucial to the plot. }.

\begin{exe}
\ex  \label{ex:thWthAci.Zo.pjWtu}
\gll tu-nɯsman-a jɤɣ ri, mɤʑɯ ɯ-ftɕaka tsuku pjɯ-tu ra wo, tɕe tʰɯtʰɤci ʑo pjɯ-tu ra \\
\textsc{ipfv}-treat-\textsc{1sg} be.possible:\textsc{fact} but yet \textsc{3sg.poss}-manner some \textsc{ipfv}-exist have.to:\textsc{fact} \textsc{sfp} \textsc{lnk} something \textsc{emph} \textsc{ipfv}-exist have.to:\textsc{fact} \\
\glt `I can treat (your illness), but yet another method is needed, something (else) is needed.'  (140427 qala cho kWrtsAg, 48-49)
\end{exe}

The pronoun \forme{tʰɯci} also occurs to refer to things whose name is unknown to the speaker (as in \ref{ex:gser.zhwa} and \ref{ex:thWci.khWtsa}), even if he/she may have seen the object.
 
\begin{exe}
\ex \label{ex:gser.zhwa}
\gll tɕe nɯ nɯ-rte nɯ tɕʰi ŋu ma tʰɯci ci ``-ʑa" tu-ti ŋu, χsɤrʑa! \\
\textsc{lnk} \textsc{dem} \textsc{3pl.poss}-hat \textsc{dem} what be:\textsc{fact} \textsc{lnk} something \textsc{indef} ... \textsc{ipfv}-say be:\textsc{fact} golden.hat \\
\glt `How is their hat (called), something in `ʑa'.... yes, \tibet{གསེར་ཞྭ་}{gser.ʑʷa}{golden hat}!' (30-mboR, 102)
\end{exe}

\begin{exe}
\ex \label{ex:thWci.khWtsa}
\gll  tɕe tɤ-ndʑɯɣ nɯ kɯnɤ, tʰɯci kʰɯtsa kɯ-fse ɯ-ŋgɯ tu-rku-nɯ tɕe   \\
\textsc{lnk} \textsc{indef.poss}-resin \textsc{dem} also something bowl \textsc{nmlz}:S/Abe.like \textsc{3sg}-inside \textsc{ipfv}-put.in-\textsc{pl} \textsc{lnk}   \\
\glt `The resin, people put it into something like a bowl.'' (07-tAtho, 44)
\end{exe}

It is also used for non-specific referents whose nature is entirely unknown, as in  (\ref{ex:thWthAci.tannWrkunW}) and (\ref{ex:thWmqlaR}).

\begin{exe}
\ex \label{ex:thWthAci.tannWrkunW}
\gll   tɕe mɤʑɯ tʰɯtʰɤci ta-nnɯ-rku-nɯ kɯma  \\
\textsc{lnk} yet something \textsc{pfv}:3\fl3'-\textsc{auto}-put.in-\textsc{pl} \textsc{sfp} \\
\glt `They also probably gave them something else.' (02-deluge2012, 120)
 \end{exe}
 
  \begin{exe}
\ex \label{ex:thWmqlaR}
\gll 
 tʰɯ-mqlaʁ tʰɯ-mqlaʁ ma tʰɯci fse ci ndʐa cʰɯ-ɕe ɕti \\
 \textsc{imp}:swallow  \textsc{imp}:swallow \textsc{lnk} something be.like:\textsc{fact} \textsc{indef} reason \textsc{ipfv:downstream}-go be:\textsc{affirm:fact} \\
\glt `Swallow it, swallow it, it comes down (into your throat) for some reason.' (2005-stod-kunbzang, 87)
  \end{exe}

The reduplicated form \forme{tʰɯtʰɤci}, especially in combination with \japhug{fse}{be like}, can also mean `whatever (happened)', as in (\ref{ex:thWthAci.kWfse}).
 
 \begin{exe}
\ex \label{ex:thWthAci.kWfse}
\gll  slama ra ɣɯ tʰɯtʰɤci kɯ-fse, kɤ-rɤ-βzjoz ra ɲɯ-stu mɯ́j-stu-nɯ, nɯ-stu ɲɯ-nɤma-nɯ mɯ́j-nɤma-nɯ,  nɯnɯra nɯ-pʰama ra nɯ-ɕki kɯ-rɤfɕɤt ɲɯ-ra. \\
student \textsc{pl} \textsc{gen} something \textsc{nmlz}:S/A-be.like \textsc{inf-antipass}-learn \textsc{pl} \textsc{sens}-try.hard-\textsc{pl} \textsc{neg:sens}-try.hard-\textsc{pl} \textsc{3sg.poss}-right \textsc{sens}-do-\textsc{pl} \textsc{neg:sens}-do-\textsc{pl} \textsc{dem:pl} \textsc{3pl.poss}-parent \textsc{pl} \textsc{3pl-dat} \textsc{genr}:S/P-tell \textsc{sens}-have.to \\
\glt `One has to tell the parents whatever concerns the students, whether they study seriously and try hard or not.'   (150901 tshuBdWnskAt, 18)
 \end{exe}
  
The non-reduplicated form \forme{tʰɯci} occurs in a correlative construction with the form \forme{mɯci} to mean `this and that', an expression that is used especially in reporting speech from another person when the speaker does not want to bother reporting in details the exact words that have been said.

\begin{exe}
\ex \label{ex:thWci.mWci}
\gll 
tʰɯci nɤme-a ra, mɯci nɤ-me-a ra \\
something do[III]:fact-\textsc{1sg} have.to:\textsc{fact} something do[III]:fact-\textsc{1sg} have.to:\textsc{fact} \\
\glt `I have to do this and that (so I cannot do X)' (elicitation)
 \end{exe}
 
The pronoun \japhug{tʰɯci}{something}  can also occur as head of a relative clause as in (\ref{ex:thWci.khWtsa}) above with the relative \forme{tʰɯci kʰɯtsa kɯ-fse} `something which is like a bowl'). This use is most common in texts translated from Chinese, with the indefinite article \japhug{ci}{one} (§ \ref{sec:indef.article}) following relative clause, as in (\ref{ex:thWci.akAspa}). 

\begin{exe}
\ex \label{ex:thWci.akAspa}
\gll  laχɕi ci pjɯ-βzjoz-a, tʰɯci a-kɤ-spa ci a-pɯ-tu ɲɯ-ra  \\
 trade \textsc{indef} \textsc{ipfv}-learn-\textsc{1sg} something \textsc{1sg.poss-nmlz:P}-be.able \textsc{indef} \textsc{irr-pfv}-exist \textsc{sens}-have.to \\
 \glt `I have to learn a trade, to have something I am able to do.' (150902 luban-zh, 12)
\end{exe}
 
With stative verbs in the relative as in (\ref{ex:thWci.kApGWlu}), this construction has a low degree meaning `a little X'.

\begin{exe}
\ex \label{ex:thWci.kApGWlu}
\gll   tɕe kɯ-wɣrum ɯ-ŋgɯz kɯnɤ tʰɯci kɯ-ɤpɣɯlu kɯ-fse ci ŋu tɕe, \\
\textsc{lnk} \textsc{nmlz}:S/A-be.white \textsc{3sg}.\textsc{poss}-inside:\textsc{loc} also something \textsc{nmlz}:S/A-greyish \textsc{nmlz}:S/A-be.like \textsc{indef} be:\textsc{fact} \textsc{lnk} \\
\glt `(Silver) is white with a little greyish colour.' (30-Com, 176)
\end{exe}
  
 The reduplicated form of the the indefinite pronoun \forme{tʰɯtʰɤci.totia} can be used as an interrogative pronoun, as in (\ref{ex:thWthAci.totia}). This construction is similar in meaning to Chinese \ch{一些什么}{yīxiēshénme}{what kinds of things}, and is attested in particular with the verbs \japhug{ti}{say} and \japhug{ra}{have to, need}. By using this form, the speaker implies that the addressee necessarily knows the answer to the question. For instance,  in (\ref{ex:thWthAci.totia}), a sentence from a text enumerating the mountain names in Kamnyu, the names had been written before hand on a piece of paper, and I was reading them one by one to Tshendzin; given the fact that the name had been written down, it was obvious that I necessarily knew the answer to that question.
  
 \begin{exe}
\ex \label{ex:thWthAci.totia}
 \gll  nɯ ɯ-pa tʰɯtʰɤci to-ti-a? \\
 \textsc{dem} \textsc{3sg}.\textsc{poss}-down something \textsc{ifr}-say-\textsc{1sg} \\
 \glt `What did I say after that?' (140522 Kamnyu zgo, 58)
\end{exe}

 There are very marginal examples of \japhug{tʰɯci}{something} used as an indefinite prenominal determiner (§ \ref{sec:indefinite}).

The pronoun \japhug{tʰɯci}{something} can take various modifiers, for instance the identity modifier \japhug{kɯmaʁ}{other} (§ \ref{sec:identity.modifier})  as in (\ref{ex:kWmaR.thWci}). 
 
\begin{exe}
\ex \label{ex:kWmaR.thWci}
\gll    ki mbro ki ɲɯ-kɤ-ntsɣe tɕe, [kɯmaʁ tʰɯci] ɲɯ-kɤ-sɤndu to-nɯkrɤz-ndʑi \\
\textsc{dem:prox} horse \textsc{dem:prox} \textsc{ipfv-inf}-sell \textsc{lnk} other  something   \textsc{ipfv-inf}-exchange \textsc{ifr}-discuss-\textsc{du} \\
 \glt `They discussed about selling their horse, and exchanging it for something else.' (150822 laoye zuoshi zongshi duide-zh, 41)
\end{exe}

No example of \forme{tʰɯci} with topic markers contributing to mark definiteness such as \forme{nɯ} or \forme{iɕqʰa} (§ \ref{sec:definiteness}) have been found in the corpus.

\subsection{\japhug{tsʰitsuku}{whatever}} \label{sec:tshitsuku}
The pronoun \japhug{tsʰitsuku}{whatever} combines the  interrogative pronoun \japhug{tsʰi}{what} (replaced by \japhug{tɕʰi}{what}, a borrowing from Tibetan in Kamnyu Japhug, but still attested in Mangi village, see \ref{sec:tChi} above) with the mid-scalar quantifier  \japhug{tsuku}{some} (see § \ref{sec:tsuku}; also found as a partitive pronoun, § \ref{sec:partitive.pronouns}).  Unlike  \japhug{tʰɯci}{something}, is not used for specific referents.  Example (\ref{ex:tshitsuku.kuwGsqa}) illustrates its most common use. The variant form \forme{tʰitsuku}, without the sound change \forme{*tʰi} \fl{} \forme{tsʰi} is also used by speakers of the Kamnyu dialect.

\begin{exe}
\ex \label{ex:tshitsuku.kuwGsqa}
\gll  
kɤ-nɯβlɯ tɕe ɕkrɤz wuma ʑo pe ma nɯnɯ, nɯnɯ ɣɯ ɯ-smɯmba nɯ sɤɕke, tɕendɤre tsʰitsuku kú-wɣ-sqa tɕe, ʑaʑa ʑo ku-ɣɤ-smi cʰa, tsʰitsuku tú-wɣ-sɯ-ɤla tɕe, ʑaʑa tu-sɯ-ɤle cʰa. \\
\textsc{inf}-burn \textsc{lnk} oak really \textsc{emph} be.good:\textsc{fact} \textsc{lnk} \textsc{dem} \textsc{dem} \textsc{gen} \textsc{3sg.poss}-flame \textsc{dem} burning \textsc{lnk} whatever \textsc{ipfv-inv}-cook \textsc{lnk} soon \textsc{emph}  \textsc{ipfv-caus}-be.cooked can:\textsc{fact} whatever \textsc{ipfv-inv-caus}-be.boiling \textsc{lnk} soon  \textsc{ipfv-caus-caus}-be.boiling[III] can:\textsc{fact} \\
\glt `For burning, oak is very good, the flames (from its wood) are very hot, whatever one cooks, it cooks it quickly, whatever one boils, it boils it quickly.' (08-CkrAz, 4-5)
\end{exe}
%nɤʑo kɯ rcanɯ, tɯ-tso ɯ-tɯ-me nɯ, maka /ji/ tɕi-rca jɤ-ɣi tɕe, nɯ sɤznɤ tshitsuku a-pɯ-tɯ-mtɤm tɕe a-pɯ-tɯ-nɯtɯtso ɲɯ-mna
%140510_fengwang, 15
 
In many cases, it is better translated as `all kinds of things', as in (\ref{ex:tshitsuku.YWznAme}).

\begin{exe}
\ex \label{ex:tshitsuku.YWznAme}
\gll  
tɕe nɯtɕu kɯnɤ ɯ-jaʁ ɯ-ntsi tɤɲi pjɯ-sɤtse  ɯ-jaʁ ɯ-ntsi kɯ tsʰitsuku ɲɯ-z-nɤme qhe, ʑara nɯ-ndzɤtsʰi tu-βze, fsapaʁ ra nɯ-ndzɤtsʰi ɲɯ-βze \\
\textsc{lnk} \textsc{dem:loc} also \textsc{3sg.poss}-hand \textsc{3sg.poss}-one.of.a.pair staff \textsc{ipfv}-plant[III]  \textsc{3sg.poss}-hand \textsc{3sg.poss}-one.of.a.pair \textsc{erg} whatever \textsc{ipfv-caus}-do[III] \textsc{lnk} \textsc{3pl} \textsc{3pl.poss}-food \textsc{ipfv}-make[III] animal \textsc{pl} \textsc{3pl.poss}-food \textsc{ipfv}-make[III]  \\
\glt `Even like that, she supports herself with a staff in one hand, and with the other hand she does all kinds of things, makes their food, she makes food for the animals.' (14-tApitaRi, 54)
\end{exe}
%tshitsuku ɲɯ́-wɣ-mbi, ɲɯ́-wɣ-jtshi, tú-wɣ-raχtɕɤz tɕe, ʑɯrɯʑɤri tɕe tɕendɤre ku-kɯ-nɯfse ɲɯ-ŋu

As other indefinite pronouns, \japhug{tsʰitsuku}{whatever} is not normally used with negation, but such sentences do occur in the corpus in translations from Chinese, as (\ref{ex:tshitsuku.mWtoti}). They are not not idiomatic Japhug, and even only marginally grammatical.

\begin{exe}
\ex \label{ex:tshitsuku.mWtoti}
\gll   tsʰitsuku mɯ-to-ti, qʰe tɕendɤre kɯ-rŋgɯ jo-nɯɕe qʰe ko-nɯ-rŋgɯ. \\
whatever \textsc{neg-ifr}-say \textsc{lnk} \textsc{lnk} \textsc{nmlz}:S/A-lay.down \textsc{ifr}-go.back \textsc{lnk} \textsc{ifr-auto}-lay.down \\
\glt `He did not said anything, went back to sleep and laid down in bed.' (150902 qixian-zh, 91)
\end{exe}

 \subsection{\japhug{ɕɯmɤɕɯ}{whoever, anybody}} \label{sex:CWmACW}
 There is no indefinite pronoun for human referents `somebody' in Japhug  corresponding to \japhug{tʰɯci}{something} -- a generic noun with the indefinite determiner \japhug{ci}{one} such as \forme{tɯrme ci} `a man' is used instead. There is nevertheless a `free choice' pronoun \japhug{ɕɯmɤɕɯ}{whoever, anybody} (see \citealt[48-52]{haspelmath97indef} on the differences with universal quantifiers), which however is not very common. As example (\ref{ex:CWmACW.kW}) shows, it can take the ergative \forme{kɯ}, and the verb receives plural indexation (see § XXX for the use of the plural for indefinite referents). 
 
 \begin{exe}
\ex \label{ex:CWmACW.kW}
\gll tɕaχkɤr kʰɯtsa nɯ ʁo tʰam qʰe ɕɯmɤɕɯ kɯ ku-nɯ-ntɕʰoz-nɯ ɕti \\
tin bowl \textsc{dem} \textsc{advers} now \textsc{lnk} anybody \textsc{erg} \textsc{ipfv-auto}-use-\textsc{pl} be:\textsc{affirm:fact} \\
\glt `Now anybody can use tin bowls.' (unlike before, when only important people could use it, 160702 khWtsa, 26)
 \end{exe}
 
  \subsection{\japhug{ciscʰiz}{somewhere}}
The indefinite pronoun \japhug{ciscʰiz}{somewhere} comprises the indefinite \japhug{ci}{one} and the approximate locative \forme{(s)cʰiz} (see § XXX). It occurs with or without the locative postposition \forme{ri}, as in (\ref{ex:cischiz}) and (\ref{ex:cischiz.ri}). It can refer to static location, or motion from or towards a direction.

 \begin{exe}
\ex \label{ex:cischiz}
\gll
ciscʰiz, tɤtsʰoʁ ɯ-taʁ kɯ-fse, tɤ-jtsi ɯ-taʁ kɯ-fse, nɯnɯra, nɯnɯtɕu kú-wɣ-βraʁ tɕe, \\
somewhere nail \textsc{3sg-on} \textsc{nmlz}:S/A-be.like, \textsc{indef.poss}-pillar \textsc{3sg-on} \textsc{nmlz}:S/A-be.like,  \textsc{dem:pl} \textsc{dem:loc} \textsc{ipfv-inv}-attach \textsc{lnk} \\
\glt `One attaches (their noseband) somewhere, like on a nail, on a pillar.' (150902 kAxtCAr, 6)
 \end{exe}
 
 \begin{exe}
\ex \label{ex:cischiz.ri}
\gll nɯnɯ ciscʰiz ri tú-wɣ-z-nɯndzɯ tɕe ɲɯ́-wɣ-ta.\\
\textsc{dem} somewhere  \textsc{loc} \textsc{ipfv-inv-caus}-be.vertical \textsc{lnk} \textsc{ipfv:west-inv}-put\\
\glt `One puts it vertically somewhere.' (14-tasa, 62)
 \end{exe}
  
 
\subsection{Interrogative pronouns used as indefinites} \label{sec:interrogative.indef}
Non-specific indefinite referents can be expressed by interrogative pronouns in Japhug. One type of construction where this function is attested is correlatives, as in (\ref{ex:NotCu.lAtWrNgW}) and (\ref{ex:thAjtCu.fsaN}).
 
\begin{exe}
\ex \label{ex:NotCu.lAtWrNgW}
\gll a-pɯwɯ, ŋotɕu lɤ-tɯ-rŋgɯ ʑo qhe, nɯtɕu rɤʑi-tɕi ŋu ma, \\
\textsc{1sg}-donkey where \textsc{pfv:upstream}-2-lay.down \textsc{emph} \textsc{lnk} \textsc{dem:loc} stay:\textsc{fact}-\textsc{1du} be:\textsc{fact} because \\
\glt `My donkey, we will stay wherever you lay down.' (28-qAjdoskAt, 38)
\end{exe}  
 
\begin{exe}
\ex \label{ex:thAjtCu.fsaN}
\gll tʰɤjtɕu fsaŋ kɤ-ta tɤ-ra ʑo tɕe nɯnɯ tu-βlɯ-nɯ tɕe, \\
when fumigation \textsc{inf}-put \textsc{pfv}-have.to \textsc{emph} \textsc{lnk} \textsc{dem} \textsc{ipfv}-burn-\textsc{pl} \textsc{lnk} \\
\glt `Whenever there is need to make fumigations, they burn it.' (15-YaBrWG, 31)
\end{exe}  

This meaning also occurs in infinitival subordinate clauses, in particular in the expression \forme{tɕʰi kɤ-cʰa} `do whatever X can to Y', as in example (\ref{ex:tChi.kAcha.Zo}).

\begin{exe}
\ex \label{ex:tChi.kAcha.Zo}
\gll  tɕʰi kɤ-cʰa ʑo cʰɯ-pʰɯt-nɯ, \\
what \textsc{inf}-can \textsc{emph} \textsc{ipfv}-remove-\textsc{pl} \\
\glt `People do whatever they can to remove (this plant).' (12-Zmbroko, 119)
\end{exe}

The most common construction to express unspecified referents is built by combining an interrogative pronoun, the verb verb with partial reduplication on the last syllable of the stem, and in most cases the autobenefactive \forme{nɯ-} prefix (this use of the autobenefactive reminds of its occurrence in concessive conditionals, see § XXX).  

With \japhug{tɕʰi}{what}, this construction expresses the meaning `whatever; no matter what' in intransitive subject (\ref{ex:tChi.pWnWNWNu}), object (\ref{ex:tChi.tAtWnWtWtWt}) or semi-object (\ref{ex:tChi.kWstWstua}, see § XXX) functions.

\begin{exe}
\ex \label{ex:tChi.pWnWNWNu}
\gll lú-wɣ-sti tɕe tɕe nɯ ɯ-ŋgɯ tɕʰi pɯ-nɯ-ŋɯ\redp{}ŋu nɯ ɲɯ-mɲɤt mɯ́j-cʰa \\
\textsc{ipfv-inv}-block \textsc{lnk} \textsc{lnk} \textsc{dem} \textsc{3sg}-inside what \textsc{pst.ipfv}-\textsc{auto}-be \textsc{dem} \textsc{ipfv}-be.spoiled \textsc{neg:sens}-can \\
\glt `One seals (its opening) and whatever (food) is inside will not be spoiled.' (150828 kodAt, 14)
\end{exe}  

\begin{exe}
\ex \label{ex:tChi.tAtWnWtWtWt}
\gll tɕʰi tɤ-tɯ-nɯ-tɯ\redp{}tɯt ʑo ju-ɣi ɕti \\
what \textsc{pfv}-2-\textsc{auto}-say[II] \textsc{emph} \textsc{ipfv}-come be:\textsc{affirm:fact} \\
\glt  `Whatever you say will come.' (2003twxtsa, 117)
\end{exe}  

\begin{exe}
\ex \label{ex:tChi.kWstWstua}
\gll nɤʑo tɕʰi kɯ-stɯ\redp{}stu-a ʑo ŋu \\
\textsc{2sg} what 2\fl1-do.like-\textsc{1sg} \textsc{emph} be:\textsc{fact} \\
\glt `Whatever you do to me (will be fine).' (28-qAjdoskAt, 40)
\end{exe}


With \japhug{ɕɯ}{who}, the construction means `whoever; regardless of who; no matter who'. Examples are found with the non-specific referent in intransitive subject (\ref{ex:CW.pWnWNWNu}), transitive subject (\ref{ex:CW.kW.panWmtWmtonW}) or oblique argument (\ref{ex:CW.GW.nWnWkhWkhota}) functions. Note that it often occurs with plural indexation.

\begin{exe}
\ex \label{ex:CW.pWnWNWNu}
\gll tɯsqar nɯ kɯrɯ tɯrme ra mɤ-kɯ-rga maka ʑo me, ɕɯ pɯ-nɯ-ŋɯ\redp{}ŋu ʑo, tɯsqar a-pɯ-tu qʰe, tɕendɤre, nɯ-kɤ-ndza tu-rtaʁ ɕti, \\
tsampa \textsc{dem} Tibetan person \textsc{pl} \textsc{neg}-\textsc{nmlz}:S/A-like at.all \textsc{emph} not.exist:\textsc{fact} who  \textsc{pst.ipfv-auto}-be \textsc{emph} tsampa \textsc{irr}-\textsc{ipfv}-exist \textsc{lnk} \textsc{lnk} \textsc{3pl.poss}-\textsc{nmlz}:P-eat \textsc{ipfv}-be.enough be:\textsc{affirm}:\textsc{fact} \\
\glt `Among Tibetan people, everybody likes tsampa (`there is no one who does not like it'), no matter who, if they have tsampa, they have enough to eat.' (2002tWsqar2, 9)
\end{exe}

\begin{exe}
\ex \label{ex:CW.kW.panWmtWmtonW}
\gll tɕe ɕɯ kɯ pa-nɯ-mtɯ\redp{}mto-nɯ ʑo kɯki ɣɯ, nɯ-kʰa ɣɯ nɯ-mɯntoʁ nɯ cʰondɤre nɯ-ɕoŋpʰu nɯra tɕe, mɤʑɯ nɯ-<cai> nɯra, pjɯ-ɣɤmɯ-nɯ tɕe, \\
\textsc{lnk} who \textsc{erg} \textsc{pfv}:3\fl3'-see-\textsc{pl} \textsc{emph} \textsc{dem.prox} \textsc{gen} \textsc{3pl.poss}-house \textsc{gen} \textsc{3pl.poss}-flower \textsc{dem} \textsc{comit} \textsc{3pl.poss}-tree \textsc{dem:pl} \textsc{lnk} yet \textsc{3pl.poss}-vegetable \textsc{dem:pl} \textsc{ipfv}-praise-\textsc{pl} \textsc{lnk} \\
\glt `Whoever saw it, the flowers and the trees and the vegetables of their house, they praised it.' (150824 yuanding-zh, 30)
\end{exe}

\begin{exe}
\ex \label{ex:CW.GW.nWnWkhWkhota}
\gll tɤɕime ri tɯ-rdoʁ ma me, tɕendɤre nɯʑo ɕɯ ɣɯ nɯ-nɯ-kʰɯ\redp{}kʰo-t-a ʑo mɯ́j-nɯtɯtʂaŋ ɕti tɕe, \\
lady also one-piece apart.from not.exist:\textsc{fact} \textsc{lnk} \textsc{2pl} who \textsc{gen} \textsc{pfv}-\textsc{auto}-give-\textsc{pst:tr-1sg} \textsc{emph} \textsc{neg:sens}-be.fair be:\textsc{affirm:fact} \textsc{lnk} \\
\glt `There is only one princess, and regardless of whom among you all I give her hand to, it will be unfair.' (140508 benling gaoqiang de si xiongdi-zh, 227)
\end{exe}

Examples of this construction are also found with the pronoun \japhug{ŋotɕu}{where}, with the meaning `no matter where, wherever' (location or direction from or to).

\begin{exe}
\ex \label{ex:NotCu.nWnWlhWlhoR}
\gll ŋotɕu nɯ-ɬɯ\redp{}ɬoʁ ʑo wuma ʑo sɤɣdɯɣ \\
where \textsc{pfv}-come.out \textsc{emph} really \textsc{emph} be.annoying:\textsc{fact} \\
\glt `No matter where it grows, it is very annoying.' (5-khArWm, 19)
\end{exe}

\begin{exe}
\ex \label{ex:nWGtWta}
\gll 
ŋotɕu 	nɯ́-wɣ-tɯ\redp{}ta 	ʑo 	kɯpɤz 	ɲɯ-βze 	ɲɯ-ɕti\\
 where \textsc{ipfv-inv}-\textsc{indefinite}\textasciitilde{}put \textsc{emph} type.of.bug \textsc{ipfv}-grow \textsc{sens}-be.\textsc{assert}\\
\glt `Bugs will grow wherever you put (the meat).' (28-kWpAz, 48)
\end{exe}
 
 The pronoun \forme{ŋotɕu} in \textit{status constructus} form \forme{ŋɤtɕɯ-} occurs in the delocutive expression \japhug{ŋɤtɕɯkɤti,kʰɯ}{obey to everything} in a compound with the infinitive \forme{kɤ-ti} of the verb \japhug{ti}{say}, and in collocation with \japhug{kʰɯ}{agree}, as in (\ref{ex:NAtCWkAti}).\footnote{the causative \japhug{ŋɤtɕɯkɤti,sɯkʰɯ}{cause to obey to everything} also exists.} 
This expression originates presumably from a phrase such as `agree (\forme{kʰɯ}) to whatever (\forme{ŋotɕu}) he says (\forme{ti})', though the pronoun \japhug{tɕʰi}{what}, not \japhug{ŋotɕu}{where} is used in Japhug in the construction meaning `whatever' as in examples (\ref{ex:tChi.pWnWNWNu}) to (\ref{ex:tChi.kWstWstua}) above.

 \begin{exe}
\ex \label{ex:NAtCWkAti}
\gll  ɯ-tɕɯ kɯβde nɯra wuma ʑo ŋɤtɕɯkɤti pjɤ-kʰɯ-nɯ  \\
3sg.poss-son four dem:pl really emph obey.to.everything(1) \textsc{ifr.ipfv}-obey.to.everything(2)-\textsc{pl} \\
\glt `His four sons were very obedient.' (140508 benling gaoqiang de si xiongdi-zh, 15)
\end{exe} 

Good examples of this construction are not found in the corpus with the other interrogative pronouns, \japhug{tʰɤjtɕu}{when} or \japhug{tʰɤstɯɣ}{how many}, but they can also be used in the same way.

No example of multiple partitive use of interrogatives (as in French \textit{qui apportait un fromage, qui un sac de noix}, \citealt[177]{haspelmath97indef} ) is attested in the data at hand; mid-scalar quantifiers such as \japhug{tsuku}{some} occur instead as partitive pronouns (§ \ref{sec:partitive.pronouns}). 

\subsection{Generic nouns as indefinite pronouns} \label{sec:tWrme.indefinite}
The generic noun \japhug{tɯrme}{person} can be used in the meaning `someone' or `other people', especially in genitival constructions as in (\ref{ex:tWrme.WkhApa.zW}).
 
\begin{exe}
\ex \label{ex:tWrme.WkhApa.zW}
\gll tɯrme ɯ-kʰɤpa zɯ, ki kɯ-fse tɯ-rʑaʁ lu-znɯfsoʁspat-a ku-omdzɯ-a. \\
people \textsc{3sg}.\textsc{poss}-yard \textsc{loc} \textsc{dem}.\textsc{prox} \textsc{nmlz}:S/A-be.like one-night \textsc{ipfv}-do.the.whole.night-\textsc{1sg} \textsc{ipfv}-sit-\textsc{1sg} \\
\glt `I would spend an entire night from dusk till dawn sitting in someone's animal yard.' (2010-histoire09, 34)
\end{exe} 

\section{Quantifiers} \label{sec:quantifiers.pronouns} \label{sec:aRandWndAt}


\subsection{Universal quantifiers}
Several quantifiers meaning `all' exist in Japhug (§ \ref{sec:universal.quant} and XXX). Among them, \japhug{kɤsɯfse}{all} can be used in the meaning `everybody', as in example (\ref{ex:kAsWfse.kW}).

\begin{exe}
\ex \label{ex:kAsWfse.kW}
\gll kɤsɯfse kɯ ʑo ta-nɯ maʁ \\
all \textsc{erg} \textsc{emph} put:\textsc{fact}-\textsc{pl} not.be:\textsc{fact} \\
\glt `Not everybody puts it.' (160706 thotsi, 21)
\end{exe}

The less common form \japhug{mɲɯrɯri}{everybody, each person} (from Tibetan \tibet{མི་རེ་རེ་}{mi.re.re}{each man} also serves as a universal quantifier, as in (\ref{ex:mYWrWri}).

\begin{exe}
\ex \label{ex:mYWrWri}
\gll mɲɯrɯri kɯ `nɯnɯ ɲɯ-pe' ntsɯ to-ti-nɯ \\
everybody \textsc{erg} \textsc{dem} \textsc{sens}-be.good always \textsc{ifr}-say-\textsc{pl} \\
\glt `Everybody said `It is nice!'' (140521 huangdi de xinzhuang, 214)
\end{exe}

The interrogative pronoun \japhug{tɕʰi}{what}, appears with the plural demonstrative determiner \forme{kɯra} to mean `everything', as in example (\ref{ex:tChi.kWra}). It is not possible to express meanings such as `everybody' or `everywhere'  by combining the other pronouns \japhug{ɕɯ}{who} or \japhug{ŋotɕu} with the same demonstrative.

\begin{exe}
\ex \label{ex:tChi.kWra}
\gll ɯʑo tɕʰi kɯra ko-tso \\
\textsc{3sg} what \textsc{dem:prox:pl} \textsc{ifr}-understand \\
\glt `He understood everything.' (2002qajdoskAt, 115)
\end{exe}

There are two words, \forme{aʁɤndɯndɤt} and \forme{ŋotɕuŋɤndɤt}, which can be translated as `everywhere'.

The word \japhug{aʁɤndɯndɤt}{everywhere} is mainly used adverbially with the emphatic \forme{ʑo} as in (\ref{ex:aRAndWndAt.ʑo}), but there are examples where it occurs with the locative postposition \forme{ri} as (\ref{ex:aRAndWndAt.ri}) like a locative noun phrase, an observation suggesting that it can be analyzed as a pronoun, though no sentences with \japhug{aʁɤndɯndɤt}{everywhere} as subject (like `everywhere is quiet') are found in the corpus.

 \begin{exe}
\ex \label{ex:aRAndWndAt.ʑo}
\gll  aʁɤndɯndɤt ʑo kʰa ra cʰɯ-rɤpɯ. tɯ-ji ɯ-ngɯ ra cʰɯ-rɤpɯ, \\
everywhere \textsc{emph} house \textsc{pl} \textsc{ipfv}-litter \textsc{indef.poss}-field \textsc{3sg}-inside \textsc{pl} \textsc{ipfv}-litter \\
\glt `Mice have litter everywhere, in the house, in the fields.' (27-spjaNkW, 166)
\end{exe} 

 \begin{exe}
\ex \label{ex:aRAndWndAt.ri}
\gll nɯ fse ʑo aʁɤndɯndɤt ri tu-nnɯ-ɬoʁ qʰe, ɯ-zrɤm nɯra kɯ-tu maŋe. \\
\textsc{dem} be.like:\textsc{fact} \textsc{emph} everywhere \textsc{loc} \textsc{ipfv}-\textsc{auto}-come.out \textsc{lnk} \textsc{3sg.poss}-root \textsc{dem:pl} \textsc{nmlz}:S/A-exist not.exist:\textsc{sens} \\
\glt `It grows simply like that everywhere, it has no roots.' (20-sWrna,76)
\end{exe} 

When \japhug{aʁɤndɯndɤt}{everywhere} occurs under the scope of negation, it never expresses the meaning `nowhere', as shown by (\ref{ex:aRAndWndAt.me}) and (\ref{ex:aRAndWndAt.juCenW}).

\begin{exe}
\ex \label{ex:aRAndWndAt.me}
\gll stɤmku nɯra, tɯ-ci ɯ-rkɯ nɯra tu ma aʁɤndɯndɤt sthɯci me \\
plain \textsc{dem:pl} \textsc{indef.poss}-water \textsc{3sg.poss}-side  \textsc{dem:pl} exist:\textsc{fact} \textsc{lnk} everywhere so.much not.exist:\textsc{fact} \\
\glt `It is found in plains, or next to rivers, but it is not found everywhere.' (14-sWNgWJu, 53)
\end{exe} 

\begin{exe}
\ex \label{ex:aRAndWndAt.juCenW}
\gll tɕe ɕɤr tɕe cʰɯ-nɯ-ɬoʁ-nɯ tɕe, aʁɤndɯndɤt ju-ɕe-nɯ mɤ-kɯ-kʰɯ \\
\textsc{lnk} night \textsc{lnk} \textsc{ipfv:downstream-auto}-come.out-\textsc{pl} \textsc{lnk} everywhere \textsc{ipfv}-go-\textsc{pl} \textsc{neg}-\textsc{nmlz}:S/A-be.\textsc{possible} \\
\glt `(They make it) to prevent (animals) from coming out at night and going everywhere.' (150902 mkhoN, 21)
\end{exe} 

The word  \japhug{ŋotɕuŋɤndɤt}{everywhere} is semantically very close to \japhug{aʁɤndɯndɤt}{everywhere} but rarer; it may also be translated as `in all kinds of places'. It contains a partially reduplicated form of the interrogative pronoun \japhug{ŋotɕu}{where} (\ref{sec:NotCu}).

 \begin{exe}
\ex \label{ex:NotCuNondAt}
\gll ɕkrɤz ɯ-ŋgɯ tɕi ɲɯ-ɬoʁ, tɯrgi ɯ-ŋgɯ tɕi ɲɯ-ɬoʁ, ʑmbri ɯ-ŋgɯ tɕi ɲɯ-ɬoʁ,  mbraj ɯ-ŋgɯ tɕi ɲɯ-ɬoʁ, tɕe sɤjku sɯŋgɯ nɯra tɕi ɲɯ-ɬoʁ, tɕe ŋotɕuŋondɤt ʑo ɣɤʑu ɕti ri, stɤmku me, sɯŋgɯ ʁɟa ʑo tu-ɬoʁ ɲɯ-ŋu. \\
oak \textsc{3sg-inside} also \textsc{sens}-come.out fir \textsc{3sg-inside} also \textsc{sens}-come.out willow \textsc{3sg-inside} also \textsc{sens}-come.out red.birch \textsc{3sg-inside} also \textsc{sens}-come.out \textsc{lnk} birch  forest \textsc{dem:pl} also \textsc{sens}-come.out \textsc{lnk} everywhere \textsc{emph} exist:\textsc{sens} be:\textsc{affirm}  \textsc{lnk} plain whether  forest completely \textsc{emph} \textsc{ipfv}-come.out \textsc{sens}-be \\
\glt `(This mushroom) grows among oaks, among firs, among willows, among red or white birch forests, you find it everywhere, whether on plains or in forest.' (23-mbrAZim, 233-238)
\end{exe} 

\subsection{Partitive pronouns} \label{sec:partitive.pronouns}
The mid-scalar quantifier \japhug{tsuku}{some} is used both as a noun determiner (§ \ref{sec:tsuku}) and as a partitive pronoun, taking case markers and determiners. This construction expresses a meaning close to that obtained by combining a relative clause with an existential verb (`there is someone who...', § XXX), as can be seen in (\ref{ex:zgri.mWrkuj}) where both constructions are used one after the other. 

\begin{exe}
\ex \label{ex:zgri.mWrkuj}
\gll tsuku kɯ zgri tu-ti-nɯ ŋu, tsuku kɯ mɯrkuj tu-ti-nɯ ŋu. mɯrkuj tu-kɯ-ti tɕi tu, zgri tu-kɯ-ti tɕi tu ma, \\
some \textsc{erg} plant.name \textsc{ipfv}-say-\textsc{pl} some \textsc{erg} plant.name \textsc{ipfv}-say-\textsc{pl} plant.name  \textsc{ipfv}-\textsc{nmlz}:S/A-say also exist:\textsc{fact} plant.name  \textsc{ipfv}-\textsc{nmlz}:S/A-say also exist:\textsc{fact}  \textsc{lnk} \\
\glt  `Some call it \forme{zgri}, some call it \forme{mɯrkuj}; there are people who call it \forme{mɯrkuj}, and also people who call it \forme{zgri}.' (19-qachGa mWntoR, 168)
\end{exe}
 
The quantifier \japhug{tsuku}{some} used as a pronoun generally refers to humans in the corpus, but (\ref{ex:tsuku.YaR}) shows that it can also denote plants for instance. 

\begin{exe}
\ex \label{ex:tsuku.YaR}
\gll tsuku ɲaʁ, tsuku aqarŋɯrŋe, \\
some be.black:\textsc{fact} some be.light.yellow:\textsc{fact} \\
\glt `Some are black, some are light yellow.' (140505 stonka mWntoR, 5)
\end{exe}

Numerals (in particular \japhug{ci}{one}) and also counted nouns (§ \ref{sec:CN.quantifier}) can be used without head noun with a partitive meaning `one of (a group)' as in (\ref{ex:ci.GW})  and (\ref{ex:ci.thWkWrgAz}).

\begin{exe}
\ex \label{ex:ci.GW}
\gll ci ɣɯ 	tɤ-tɕɯ,  ci ɣɯ tɕʰeme tɯ\redp{}tɤ-tu nɤ, ʁzɤmi ku-kɤ-sɯ-βzu \\
one \textsc{gen} \textsc{indef}.\textsc{poss}-son one \textsc{gen} \textsc{indef}.\textsc{poss}-son \textsc{cond}\redp{}\textsc{pfv}-exist \textsc{lnk} husband.and.wife \textsc{ipfv}-\textsc{inf}-\textsc{caus}-make \\
\glt `If one of them has a boy, and the other one has a girl, let us make them husband and wife.' (zrAntCW 5)
\end{exe}

\begin{exe}
\ex \label{ex:ci.thWkWrgAz}
\gll ci tʰɯ-kɯ-rgɯ\redp{}rgɤz ɲɯ-ɕti tɕe, ci kɯ-xtɕɯ\redp{}xtɕi ɲɯ-ɕti tɕe, \\
one \textsc{pfv}-\textsc{nmlz}:S/A-\textsc{emph}\redp{}be.old C-be.\textsc{affirm} \textsc{lnk} one \textsc{pfv}-\textsc{nmlz}:S/A-\textsc{emph}\redp{}be.old C-be.\textsc{affirm} \textsc{lnk}  \\
\glt `One of them has grown very old, and one of them is very small.' (2011-05-nyima, 140)
\end{exe}


\subsection{Distributive pronouns} \label{sec:distributive.pronouns}
The pronoun \japhug{ʑaka}{each his own} and its variant \japhug{ʑakastaka}{each his own} occur as pronouns, especially as possessors in an possessive existential construction. It can be correlated with a third singular \forme{ɯ-} (\ref{ex:Zaka.WmdoR}) or a third plural \forme{nɯ-} (\ref{ex:Zaka.nWrmi}) prefix on the possessum.


\begin{exe}
\ex \label{ex:Zaka.WmdoR}
\gll tɕe nɯnɯ li qaʑo nɯ kɯ-ɲaʁ tu, kɯ-wɣrum tu, kɯ-ɤɣɯnɯɕɯr kɯ-fse tu, kɯ-ɤrŋɯlɯz tu, tɕe nɯnɯ ʑaka ɯ-mdoʁ tu ma \\
\textsc{lnk} \textsc{dem} again sheep \textsc{dem} \textsc{nmlz}:S/A-be.black exist:\textsc{fact} \textsc{nmlz}:S/A-be.white exist:\textsc{fact} \textsc{nmlz}:S/A-be.reddish exist:\textsc{fact}  \textsc{nmlz}:S/A-be.blueish exist:\textsc{fact} \textsc{lnk} \textsc{dem} each.his.own \textsc{3sg}.\textsc{poss}-colour  exist:\textsc{fact} \textsc{lnk} \\
\glt `There are black sheep, white ones, reddish ones, blueish ones, each has his own colour (they come in all types of colors).' (05-qaZo, 64-66)
\end{exe}

\begin{exe}
\ex \label{ex:Zaka.nWrmi}
\gll li ʑaka nɯ-rmi tu, \\
again each.his.own \textsc{3pl}.\textsc{poss}-name exist:\textsc{fact} \\
\glt `Each have their own names.' (150903 tWmNu, 11)
\end{exe}

The pronoun \forme{ʑaka} is built by combining the \textit{status constructus} of the pronominal root \forme{-ʑo} (§ \ref{sec:pers.pronouns}) with the root \forme{-ka} found in the distributive modifier \japhug{tɯka}{each} (which follows possessums, see § \ref{sec:raNri}).


\section{Identity pronoun} \label{sec:other.pro}
The words \japhug{kɯmaʁ}{other} and \japhug{kɯɕte}{other} occur as prenominal determiners (see § \ref{sec:identity.modifier}, also for a discussion on the etymology of the former), but it can also be used as a pronoun and take determiners as in (\ref{ex:kWmaR.nWra}).

\begin{exe}
\ex \label{ex:kWmaR.nWra}
\gll ma kɯmaʁ nɯra aj mɯ́j-sɯχsal-a ri, tɤkʰepɣɤtɕɯ nɯ sɯχsal-a  \\
\textsc{lnk} other \textsc{dem:pl} \textsc{1sg} \textsc{neg}:\textsc{sens}-recognize but bird.sp \textsc{dem} recognize:\textsc{fact}-\textsc{1sg} \\
\glt `The other ones I don't recognize them, but the \forme{tɤkʰepɣɤtɕɯ} bird, I do recognize it.' (23-scuz, 46)
\end{exe}

The interpretation of both \japhug{kɯmaʁ}{other} and \japhug{kɯɕte}{other} can be locative  `somewhere else' as in (\ref{ex:kWmaR.nWtWtat}), when the main verb (\japhug{ta}{put} in this example) selects a goal or a locative adjunct (since locative noun phrases are often unmarked, see § XXX).

\begin{exe}
\ex \label{ex:kWmaR.nWtWtat}
\gll kɯmaʁ/kɯɕte   nɯ-tɯ-ta-t ŋu ɯ-maʁ? \\
other  \textsc{pfv}-2-put-\textsc{tr}:\textsc{pst} be:\textsc{fact} \textsc{qu}-not.be:\textsc{fact} \\
\glt `Did you put it somewhere else?' (elicitation)
\end{exe}

Adding the indefinite determiner \japhug{ci}{one} is necessary in this context to convey the meaning `something else':

\begin{exe}
\ex \label{ex:kWmaR.ci.nWtWtat}
\gll kɯmaʁ/kɯɕte ci nɯ-tɯ-ta-t ŋu ɯ-maʁ? \\
other \textsc{indef} \textsc{pfv}-2-put-\textsc{tr}:\textsc{pst} be:\textsc{fact} \textsc{qu}-not.be:\textsc{fact} \\
\glt `Did you put something else?' (elicitation)
\end{exe}

Example (\ref{ex:kWmaR.ci.juGWt}) illustrates that both \forme{kɯmaʁ} and \forme{kɯmaʁ ci} can occur in the meaning `another one' in some contexts (here with the verb \japhug{ɕar}{search}).

 \begin{exe}
\ex \label{ex:kWmaR.ci.juGWt}
\gll  χsɯ-sŋi mɤ-kɯ-tsu qʰe li kɯmaʁ ci ju-ɣɯt qʰe, li ɯ-zda ɲɯ-nɯ-ɕar ɲɯ-ɕti.
tɕe nɯnɯ maka kɯjka nɯ ŋɤn ma, ɯ-zda nɯ nɯ-me ɯ-qʰu mɤ-kɯ-nɤrʑaʁ tɕe kɯmaʁ ɲɯ-ɕar ɲɯ-ɕti tɕe mɯ́j-pe tu-ti-nɯ ŋgrɤl. \\
three-day \textsc{neg}-\textsc{inf}:\textsc{stat}-pass \textsc{lnk} again other \textsc{indef} \textsc{ipfv}-bring \textsc{lnk} again \textsc{3sg}.\textsc{poss}-companion \textsc{ipfv}-\textsc{auto}-search \textsc{sens}-be.affirm \textsc{lnk} \textsc{dem} completely pyrrhocorax \textsc{dem} be.evil:\textsc{fact} \textsc{lnk} \textsc{3sg}.\textsc{poss}-companion \textsc{dem}  \textsc{pfv}-not.exist \textsc{3sg}.\textsc{poss}-after \textsc{neg}-\textsc{inf}:\textsc{stat}-spend.time \textsc{lnk} other \textsc{ipfv}-search \textsc{sens}-be.affirm \textsc{lnk} \textsc{neg}:\textsc{sens}-be.good \textsc{ipfv}-say-\textsc{pl} be.usually.the.case:\textsc{fact} \\
\glt `Not even three days (after hunters kill its mate, the pyrrhocorax) brings another one, it looks for another mate. People say that the pyrrhorax is not nice, because not long after its mate has died, it looks for another one, it is not good.' (22-CAGpGa, 84)
\end{exe}

The indefinite \japhug{ci}{one} combined with the demonstrative determiner \forme{nɯ} (or \forme{nɯnɯ}) has the meaning `the other one' (the definite counterpart of \japhug{kɯmaʁ}{other}), as in (\ref{ex:ci.nW.kW}) and (\ref{ex:tWrdoR.ci.nW}).

\begin{exe}
\ex \label{ex:ci.nW.kW}
 \gll tɕe ɯ-jaʁ kɯ ki tu-ste lu-z-naʁje ɲɯ-ŋu ri, tɕe ci nɯ kɯ ɯ-jaʁ ku-mtsɯɣ ɲɯ-ɕti qʰe, \\
 \textsc{lnk} \textsc{3sg.poss}-hand \textsc{dem:prox} \textsc{ipfv}-do.like[III]  \textsc{ipfv}-reach.into[III] \textsc{sens}-be but \textsc{lnk} \textsc{indef} \textsc{dem} \textsc{erg} \textsc{3sg.poss}-hand  \textsc{ipfv}-bite \textsc{sens}-be:\textsc{affirm:fact} \textsc{lnk} \\
\glt `(The cat) reaches with its paw (into the whole), but the other one (the weasel) bites its paw.' (27-spjaNkW)
\end{exe}

\begin{exe}
\ex \label{ex:tWrdoR.ci.nW}
 \gll 
ɯ-me ʁnɯz pjɤ-tu tɕe, tɯ-rdoʁ nɯ χsɤrlɤsmɤn pjɤ-rmi, ci nɯ rŋɯlɤsmɤn pjɤ-rmi tɕe, \\
\textsc{3sg.poss}-daughter two \textsc{ifr.ipfv}-exist \textsc{lnk} one-piece \textsc{dem} gser.la.sman \textsc{ifr.ipfv}-be.called \textsc{indef} \textsc{dem} dngul.la.sman \textsc{ifr.ipfv}-be.called \textsc{lnk} \\
\glt `He had two daughters, one of them was called Gser.la.sman, and the other Dngul.la.sman.' (2003-kWBRa, 1-2)
\end{exe}

Alternatively to the construction in (\ref{ex:tWrdoR.ci.nW}) with \japhug{tɯ-rdoʁ}{one piece} and \forme{ci nɯ} to express the meaning one of them .... and the other ...', it is possible to use \forme{ci nɯ} two times in the same sentence to refer to more than one distinct persons or animals, as in (\ref{ex:ci.nW.2}).

\begin{exe}
\ex \label{ex:ci.nW.2}
 \gll tɕe ci nɯnɯ ju-ɕe ɯ-kʰɯkʰa ci nɯ kɯ ɯ-pu tu-ndze, cʰɯ-rɤɕi. \\
\textsc{lnk} \textsc{indef} \textsc{dem} \textsc{ipfv}-go \textsc{3sg}-while \textsc{indef} \textsc{dem} \textsc{erg} \textsc{3sg.poss}-intestine \textsc{ipfv}-eat[III] \textsc{ipfv}-pull \\
\glt `While one of the two (the prey) is (still) going, the other one (the predator) eats and pulls its intestine.'  (20-RmbroN, 76)
\end{exe}

The dual \forme{ci nɯni} `the other two' and plural \forme{ci nɯra} `the other ones' are also attested, as in (\ref{ex:ci.nWni}), showing that \forme{ci} is here completely bleached of numeral meaning.

 \begin{exe}
\ex \label{ex:ci.nWni}
 \gll  ci nɯni ɣɯ nɯ, ndʑi-ta-mar rɟɤɣi pjɤ-ŋu tɕe tɕe nɯnɯ ɯʑo kɯ to-ndza \\
\textsc{indef} \textsc{dem:du} \gen \textsc{dem} \textsc{3du}.\textsc{poss}-\textsc{indef}.\textsc{poss}-butter tsampa \textsc{ipfv}.\textsc{ifr}-be:\textsc{fact} \textsc{lnk} \textsc{lnk} \textsc{dem} \textsc{3sg} \textsc{erg} \textsc{ifr}-eat \\
\glt `The tsampa of the other two (sisters) was butter tsampa, and she ate it.' (2003-kWBRa, 20)
\end{exe}

It is also possible in this function to use other modifiers such as numerals, as in (\ref{ex:ci.XsWm}) with \japhug{χsɯm}{three}. 

 \begin{exe}
\ex \label{ex:ci.XsWm}
 \gll iɕqʰa ci χsɯm nɯ mɯ-jo-ɣi-nɯ kɯ  \\
 the.aforementioned \textsc{indef} three \textsc{dem} \textsc{neg}-\textsc{ifr}-come-\textsc{pl} \textsc{erg} \\
\glt `The three other ones, without coming, (said...)' (140515 congming de wusui xiaohai-zh, 45)
 \end{exe}

As a prenominal determiner, \forme{ci} also has the meaning `the other X' (see § \ref{sec:identity.modifier}).


\section{Demonstrative pronouns} \label{sec:demonstrative.pronouns}
There are two basic demonstratives in Japhug, the proximal \japhug{ki}{this} and the distal one \japhug{nɯ}{that}, which also occur as demonstrative determiners (see § \ref{sec:demonstrative.determiners}). Table \ref{tab:dem.pronoun} illustrates the various demonstrative pronouns that are derived from these basic forms, with Reduplicated and Emphatic forms. There is in addition a Cataphoric pronoun \forme{nɤki}, discussed in section \ref{sec:cataph.pron}.

Plural and dual forms, as in the case of determiners, are formed by adding \forme{-ra} and \forme{-ni} suffixes (see § \ref{sec:number.determiners}) to the demonstrative root, which undergoes \textit{status constructus} change \ipa{i} \fl{} \ipa{ɯ} in the case of proximal demonstratives. Plural forms are given in the table; dual forms are attested but rare and can be predicted (\japhug{kɯni}{these two} etc).

\begin{table}
\caption{Demonstrative pronouns}\label{tab:dem.pronoun}
\begin{tabular}{lllll} 
\lsptoprule
&Base form & Reduplicated & Emphatic \\
\midrule
\textsc{prox.sg} & \forme{ki} & \forme{kɯki} &  \forme{ɯkɯki}  \\
\textsc{dist.sg} & \forme{nɯ} &  \forme{nɯnɯ} & \forme{ɯnɯnɯ} \\
\midrule
\textsc{prox.pl} & \forme{kɯra} & \forme{kɯkɯra} &  \forme{ɯkɯkɯra}  \\
\textsc{dist.pl} & \forme{nɯra} &  \forme{nɯnɯra} & \forme{ɯnɯnɯra} \\
\lspbottomrule
\end{tabular}
\end{table}

\subsection{Anaphoric demonstrative pronouns} \label{sec:anaphoric.demonstrative.pro}

The basic demonstratives \forme{ki} and \forme{nɯ} are less often used as pronouns that the other ones (they mainly occur as determiners). They nevertheless do occur in all syntactic functions, including object (in particular with the verb \japhug{ti}{say}, as in \ref{ex:nW.toti}, where it refers to words that have been previously told to another animal), XXX and extended object (in particular with the verb \japhug{stu}{do like} as in \ref{ex:ki.tuste}).

\begin{exe}
\ex \label{ex:nW.toti}
 \gll  li nɯ to-ti ri, \\
 again \textsc{dem} \textsc{ifr}-say \textsc{lnk} \\
\glt `(Gesar) said the same thing to the (snow leopard).' (gesar, 286)
\end{exe}

\begin{exe}
\ex \label{ex:ki.tuste}
 \gll ɯ-mu nɯ ku-rqoʁ tɕe ki tu-ste tɕe \\
\textsc{3sg.poss}-mother  \textsc{dem} \textsc{ipfv}-hug \textsc{lnk} \textsc{dem:prox} \textsc{ipfv}-do.like[III] \textsc{lnk} \\
\glt `It hugs its mother like that.' (19-GzW, 30)
\end{exe}

The distal demonstratives \forme{nɯ} and \forme{nɯnɯ} serve as anaphoric pronouns with any type of referent, including humans, but are most appropriate for abstract concepts, inanimate objects or plants as in (\ref{ex:nWnW.kW.smi}), though as mentioned in section \ref{sec:pers.pronouns},  third person pronouns such as \japhug{ɯʑo}{he} can also have inanimate antecedents.

\begin{exe}
\ex \label{ex:nWnW.kW.smi}
 \gll tʂʰa kɤ-nɯ-ta tɤ-ra, smi kɤ-βlɯ tɤ-ra pɯ-nɯ-ŋu, tʰamaka sko-nɯ pɯ-nɯ-ŋu, tɕe \textbf{nɯnɯ} kɯ smi tu-sɯ-tɕɤt-nɯ. \\
 tea \textsc{inf}-\textsc{auto}-put \textsc{pfv}-have.to fire  \textsc{pfv}-burn \textsc{pfv}-have.to \textsc{pst.ipfv-auto}-be tobacco smoke:\textsc{fact}-\textsc{pl} \textsc{pst.ipfv-auto}-be \textsc{lnk} \textsc{dem} \textsc{erg} fire \textsc{ipfv}-\textsc{caus}-take.out-\textsc{pl} \\
 \glt `When they need to boil tea, to make a fire or smoke tobacco, people light up the fire with it.' (15-babW, 226-229)
\end{exe}

When a third person mentioned in a discussion is present, the pronoun \japhug{ɯʑo}{he} is not the optimal way of referring to him/her, and a proximal demonstrative, in particular the reduplicated \japhug{kɯki}{this one} is used instead. It can occur to present someone to someone else (\ref{ex:kWki.aslama}) (note that a similar usage exists in Western languages such as English in the same context) and even to talk about the actions of this person, as in  (\ref{ex:kWki.kW.taBzu}) and (\ref{ex:kWki.nW.ftsWntCi}).

\begin{exe}
\ex \label{ex:kWki.aslama}
 \gll kɯki a-slama ŋu \\
\textsc{dem.prox} \textsc{1sg.poss}-student be:\textsc{fact} \\
\glt `This a (former) student of mine.' (conversation 140510, 17)
\end{exe}

\begin{exe}
\ex \label{ex:kWki.kW.taBzu}
 \gll  kɯki kɯ ta-βzu? \\
 \textsc{dem.prox} \textsc{erg} \textsc{pfv}:3\fl3'-make \\
 \glt `Did she make it?' (conversation 140510, 152)
\end{exe}

As other pronouns (see § \ref{sec:pers.pronouns}), demonstrative pronouns can take the demonstrative determiner \forme{nɯ}, as in (\ref{ex:kWki.nW.ftsWntCi}).

\begin{exe}
\ex \label{ex:kWki.nW.ftsWntCi}
 \gll mɯ\redp{}mɤ-pɯ-jɤɣ tɕe mɤ-ɣi-tɕi ma \textbf{kɯki} nɯ fstɯn-tɕi ra ma tɕi-βɣe ɯ-ku thɯ-kɯ-ɣɤrndi  \\
\textsc{cond}\redp{}\textsc{neg}-\textsc{pst}.\textsc{ipfv}-be.acceptable \textsc{lnk} \textsc{neg}-come:\textsc{fact}-\textsc{1du} \textsc{lnk} \textsc{dem:prox} \textsc{dem} serve:\textsc{fact}-\textsc{1du} have.to:\textsc{fact} \textsc{lnk} \textsc{1du.poss}-orphan \textsc{3sg.poss}-head \textsc{pfv}-\textsc{nmlz}:S/A-support \\
\glt `If it is not possible (to take the old man with us) we will not come, as we have to serve him, he is the one who adopted us orphans when we were in dire straits.' (The old man is presumably present when this sentence is uttered; 2003nyima2, 122)
\end{exe}

The emphatic demonstrative pronouns (which are also used as determiners, see § XXX) are built by combining the reduplicated forms of demonstratives with the third person possessive prefix \forme{ɯ-}. They are about fifty time less common than corresponding reduplicated forms, but their function is essentially the same. In example (\ref{ex:WnWnW.kW}), \forme{ɯnɯnɯ} is an anaphoric pronoun whose antecedent is present in the immediately preceding clause.

\begin{exe}
\ex \label{ex:WnWnW.kW}
 \gll
tɕe ɯ-rqʰu kɯ-fse ci ɣɤʑu tɕe, \textbf{ɯnɯnɯ} kɯ ɯ-rdu nɯ tu-ɕɯ-fkaβ kɯ-fse ɲɯ-ŋu. \\
\textsc{lnk} \textsc{3sg.poss}-hull \textsc{nmlz}:S/A-be.like \textsc{indef}  exist:\textsc{sens} \textsc{lnk} \textsc{dem:emph:distal} \textsc{erg} \textsc{3sg.poss}-eyeball \textsc{dem} \textsc{ipfv}-\textsc{caus}-cover  \textsc{nmlz}:S/A-be.like \textsc{sens}-be \\
\glt `It has something like a membrane, and it covers its eyeball with it.' (description of the nictitating membrane of birds, 140513 sWNgWrmABja, 9)
\end{exe}

\subsection{Cataphoric pronoun} \label{sec:cataph.pron}
The demonstrative \forme{nɤki} stands out among other demonstrative pronouns in that it is most specifically used for cataphoric referents. It occurs especially when the speaker hesitates and uses it as a filler, followed by a clause with the same verb (examples \ref{ex:nAki.YWNu} and \ref{ex:nAki.YAXtAr}) or just with the same auxiliary (\ref{ex:nAki.Nu.Ci}).

\begin{exe}
\ex \label{ex:nAki.YWNu}
 \gll
qra nɯ kɯ, mbala na-lɤt nɤ tɕe \textbf{nɤki} ɲɯ-ŋu, jla ɲɯ-ŋu, \\
female.yak \textsc{dem} \textsc{erg} male.young.bovid \textsc{pfv}:3\fl3' \textsc{lnk} \textsc{lnk} \textsc{dem:cataph} \textsc{sens}-be male.hybrid.yak \textsc{sens}-be \\
\glt `When a female yak has a young (with a bull), it is..., it is a hybrid yak.' (05-qambrW, 64)
\end{exe}

\begin{exe}
\ex \label{ex:nAki.YAXtAr}
 \gll tɕe nɯ tɯ-ci ɣɯ ɯ-taʁ nɯnɯtɕu, \textbf{nɤki} ɲɤ-χtɤr, iɕqʰa <yujinxiang> kɤ-ti mɯntoʁ nɯ ɣɯ  ɯ-jwaʁ nɯ ɲɤ-χtɤr.  \\
\textsc{lnk} \textsc{dem} \textsc{indef.poss}-water \textsc{gen} \textsc{3sg}-on \textsc{dem:loc} \textsc{dem:cataph} \textsc{ifr}-spread the.aforementionned tulip \textsc{nmlz}:P-say flower \textsc{dem} \textsc{gen} \textsc{3sg.poss}-leaf \textsc{dem} \textsc{ifr}-spread \\
\glt She spilled on the water...  she spilled the petals of the flower called `tulip'.' (150818 muzhi guniang, 69)
\end{exe}

\begin{exe}
\ex \label{ex:nAki.Nu.Ci}
 \gll 
tɕe nɯ-nɯŋa ra nɤki ŋu ɕi, tɕe ɲɯ-tɯ-nɤm qhe, tɕe ʑara ku-nɯ-nɯɣi-nɯ ŋu ɕi? \\
\textsc{lnk} \textsc{2sg}.\textsc{poss}-cow \textsc{dem}:\textsc{cataph} be:\textsc{fact} \textsc{qu} \textsc{lnk} \textsc{ipfv:east}-2-chase[III] \textsc{lnk} \textsc{lnk} \textsc{3pl} \textsc{ipfv:west}-auto-come.back-\textsc{pl} be:\textsc{fact} \textsc{qu} \\
\glt `And your cows, are they (still) like that, do you chase them, or do they come back home on their own?' (taRrdo conversation, 28-29)
\end{exe}
It is also used  when the speaker alerts the addressee that a long description follows as in (\ref{ex:nAki.tustunW}), as in English `(he said) the following'. Given the fact the Japhug is strictly verb-final and has pre-verbal complements (see § XXX), this is a strategy employed to avoid relegating the main verb to the end of the description.

 \begin{exe}
\ex \label{ex:nAki.tustunW}
 \gll
kʰopi kɯ ɴqiazwɤr ci ɲɯ-mɯm rca ɲɯ-saχaʁ ʑo tɕe \textbf{nɤki} tu-stu-nɯ ɲɯ-ŋu ɲɯ-ti, ɲɯ-pʰɯt-nɯ qʰe kɯ-zri... ki jamar ʑo kɯ-zri ɲɯ-pʰɯt-nɯ qʰe nɤki, ɯ-ku ɯ-mtɯ kɯ-fse nɯtɕu kú-wɣ-ndo qʰe tɕe ɯ-pa nɯ, ɯ-jwaʁ nɯ cʰɯ-χɕoʁ-nɯ ɲɯ-ŋu...  \\
p.n. \textsc{erg} bitter.wormwood \textsc{indef} \textsc{sens}-be.tasty \textsc{unexpect} \textsc{sens}-be.extremely \textsc{emph} \textsc{lnk} \textsc{dem:cataph} \textsc{ipfv}-do.like-\textsc{pl} \textsc{sens}-be \textsc{sens}-say \textsc{ipfv}-take.out-\textsc{pl} \textsc{lnk} \textsc{nmlz}:S/A-be.long \textsc{dem:prox} about \textsc{emph} \textsc{nmlz}:S/A-be.long \textsc{ipfv}-take.out-\textsc{pl} \textsc{lnk} \textsc{dem:cataph} \textsc{3sg.poss}-head  \textsc{3sg.poss}-crest \textsc{nmlz}:S/A-be.like  \textsc{dem:loc} \textsc{ipfv-inv}-take \textsc{lnk} \textsc{lnk} \textsc{3sg.poss}-under \textsc{dem}  \textsc{3sg.poss}-leaf \textsc{dem} \textsc{ipfv:downstream}-take.out-\textsc{pl}  \textsc{sens}-be \\
\glt `Kebei says that bitter wormwood is very tasty, and that they prepare it in the following way: they pluck (wormwoods) that are this big, take it by something that looks like a crest on the top, and prune away the leaves under it... (continued by several paragraphs)' (conversation 140510)
\end{exe}

The pronoun \forme{nɤki} is also used as a determiner (§ XXX) and the speech filler \forme{nɤkinɯ} (§ XXX) derives from the combination of  \forme{nɤki}  with the determiner \forme{nɯ}. There are no plural or dual forms of \forme{nɤki}, but it can be combined with dual or plural determiners as in (\ref{ex:nAki.nWra}).

\begin{exe}
\ex \label{ex:nAki.nWra}
\gll  tɕe nɤki nɯra, mkʰɤrmaŋ ra pjɤ-rɯsɯso-nɯ tɕe, \\
\textsc{lnk} \textsc{dem:cataph} \textsc{dem}:\textsc{pl} people \textsc{pl} \textsc{ifr}-think-\textsc{pl} \textsc{lnk} \\
\glt `And these, the people thought about it.' (150829 jidian-zh, 138)
\end{exe}

\subsection{Medial demonstrative} \label{sec:medial.dem.pro}
In addition to its function as a cataphoric pronoun, \forme{nɤki} also occurs as a medial demonstrative, in examples such as (\ref{ex:nAki.nWtɕu}) and (\ref{ex:nAki.nW.aZWG}), where it means `your place, near you', as opposed to `here'.

\begin{exe}
\ex \label{ex:nAki.nWtɕu}
\gll kutɕu ko-qanɯ pʰoʁpʰoʁ ʑo, nɤki nɯtɕu ɯ-kó-qanɯ? \\
here \textsc{ifr}-be.dark \textsc{idph}:II:completely \textsc{emph} \textsc{dem}:\textsc{medial} \textsc{dem}:\textsc{loc} \textsc{qu}-\textsc{ifr}-be.dark \\
\glt `Here it is already dark, is it (also) dark in your place?' (conversation, 14.12.24, referring to the time lag between Paris and Mbarkham)
\end{exe}

\begin{exe}
\ex \label{ex:nAki.nW.aZWG}
\gll  ki kɯra ɲɯ-kʰam-a tɕe nɤki nɯ aʑɯɣ nɯ-kʰɤm je \\
\textsc{dem}:\textsc{prox} \textsc{dem}:\textsc{prox}:\textsc{pl}  \textsc{ipfv}-give-\textsc{1sg} \textsc{lnk} \textsc{dem}:\textsc{medial} \textsc{dem} \textsc{1sg}:\textsc{gen} \textsc{imp}-give \textsc{sfp} \\
\glt `I give (you) these (toys), give me that one.'  (Norbzang2012, 135)
\end{exe}

The medial demonstrative \forme{nɤki} can be historically analyzed as a combination of the proximal demonstrative \forme{ki} with the second person possessive \forme{nɤ-}. However, equivalent dual or plural forms such as  $\dagger$\forme{ndʑiki} or $\dagger$\forme{nɯki} are completely impossible (the equivalent meaning can only be expressed with the dative, using a form such as \forme{ndʑiʑo ndʑi-pʰe} `at your$_{du}$ place', § \ref{sec:dative} ).


It is likely that the cataphoric demonstrative use of \forme{nɤki} derives  from its function as a medial demonstrative, suggesting the historical pathway in (\ref{ex:nAki.hist}).

\begin{exe}
\ex \label{ex:nAki.hist}
\glt \textsc{2sg}+\textsc{dem}:\textsc{prox} $\Rightarrow$ \textsc{dem}:\textsc{medial} $\Rightarrow$ \textsc{dem}:\textsc{cataphoric} $\Rightarrow$ \textsc{speech filler}
\end{exe}


 %Pronouns, Demonstratives and Indefinites
%\chapter{Numerals and counted nouns}

\section{Plain numerals} \label{sec:plain.numerals}
Unlike some languages of the Sino-Tibetan family which have exotic numeral systems (\citealt{mazaudon02nombre}), Japhug displays a strict decimal system, without evidence for vigesimal features or substractive numerals.


\subsection{Numerals 1-10}  \label{sec:one.to.ten}
The basic numerals from one to ten are indicated in Table \ref{tab:numerals.under.10}. The numeral \japhug{ci}{one} is identical to the indefinite determiner (§ XXX and § \ref{sec:other.pro}). Some dialects of Japhug other than the Kamnyu variety use \japhug{tɤɣ}{one} instead. Apart from \japhug{ci}{one} and \japhug{sqi}{ten}, these numerals have clear cognates in languages outside of the Gyalrongic group, even in Tibetan and Chinese; Table \ref{tab:numerals.under.10} includes the Tibetan equivalent of these numerals (the numerals that are \textit{not} cognate with their Japhug equivalent are indicated between brackets).

The numerals from 2 to 9 have a prefix, uvular \forme{χ-/ʁ-} in `two' and `three' and velar \forme{kɯ-} from `four' to `nine'. These prefixes do not appear in some derived forms such as teens (Table XXX below) or approximate numerals (\ref{sec:approx.numerals}).

The numeral \japhug{ʁnɯz}{two} is etymologically related to the dual clitic \forme{ni} (§ XXX), though the latter lacks the uvular prefix and the \forme{-z} suffix (the vowel different is expected as proto-Gyalrong \forme{*-is} yields Japhug \forme{-ɯz}, see § XXX). A relation with \japhug{kɯɕnɯz}{seven} (implying a former base five system) is possible but if true goes back to proto-ST and is irrelevant to the synchronic grammar of Japhug.

While the other numerals are native Gyalrong words, \japhug{χsɯm}{three} might be a borrowing from Tibetan \tibet{གསུམ་}{gsum}{three}, and occurs with the same form in obvious compound loans  such as \japhug{kɯmtɕʰoχsɯm}{triratna} from \tibet{དཀོན་མཆོག་གསུམ་}{dkon.mtɕʰog.gsum}{triratna}. This idea is apparently confirmed by the alternative forms \forme{-fsum} and \forme{fsɯ-} for `three' found in teens (§ \ref{sec:teens}) and decades  (§ \ref{sec:decades}). Alternatively, it is possible that the native word and the borrowing have the same form by coincidence.

The numeral \japhug{kɯngɯt}{nine} has a coda \forme{-t} which is not found in Situ and languages outside of Gyalrongic, suggesting analogical spreading of the coda from \japhug{kɯrcat}{eight}. The same analogy independently occurred in the Siyuewu dialect of Khroskyabs, where  `nine' is \forme{ŋgə́d} (\citealt[174]{lai17khroskyabs}).

\begin{table}
\caption{Basic numerals in Japhug and Tibetan}  \label{tab:numerals.under.10} \centering \label{tab:numerals}
\begin{tabular}{lllllll}
\lsptoprule
& Japhug & Tibetan  \\
1	&	\forme{ci} or \forme{tɤɣ} & \tibet{གཅིག་}{gtɕig}{one} \\
2	&	\forme{ʁnɯz}  & \tibet{གཉིས་}{gɲis}{two} \\
3	&	\forme{χsɯm}  & \tibet{གསུམ་}{gsum}{three} \\
4	&	\forme{kɯβde} & \tibet{བཞི་}{bʑi}{four} \\
5	&	\forme{kɯmŋu}  & \tibet{ལྔ་}{lŋa}{five} \\
6	&	\forme{kɯtʂɤɣ}  & \tibet{དྲུག་}{drug}{six} \\
7	&	\forme{kɯɕnɯz} & (\tibet{བདུན་}{bdun}{seven}) \\
8	&	\forme{kɯrcat}  & \tibet{བརྒྱད་}{brgʲad}{eight} \\
9	&	\forme{kɯngɯt}  & \tibet{དགུ་}{dgu}{nine} \\
10	&	\forme{sqi}  & (\tibet{བཅུ་}{btɕu}{ten}) \\
\lspbottomrule
\end{tabular}
\end{table}		

Numerals from 1 to 99 are a subclass of unpossessible nouns (§ \ref{sec:unpossessible.nouns}), and cannot take possessive prefixes; they differ in this regard from the higher numerals (§ \ref{sec.hundred.plus}, § \ref{sec:approx.numerals}).

\subsection{Numerals 11-19} \label{sec:teens}
The numerals 11-19, listed in Table \ref{tab:teens}, serve as the basis for building all following numerals between 21 and 99, by replacing the \forme{-sqi} element of the decade numeral (Table XXX) by the appropriate form. Table \ref{tab:teens} also illustrates the formation of the numerals 21 to 29 from \japhug{ɣnɤsqi} {twenty}. 

\begin{table}
\caption{Numerals 11-29}  \label{tab:teens} \centering
\begin{tabular}{lllllll}
\lsptoprule
10 & \forme{sqi} &	20	&	\forme{ɣnɤsqi}  \\	
\midrule
11 & \forme{sqaptɯɣ} &	21	&	\forme{ɣnɤsqaptɯɣ}  \\	
12 & \forme{sqamnɯz} &	22	&	\forme{ɣnɤsqamnɯz}  \\	
13 & \forme{sqafsum} &	23	&	\forme{ɣnɤsqafsum}  \\	
14 & \forme{sqaβde} &	24	&	\forme{ɣnɤsqaβde}  \\	
15 & \forme{sqamŋu} &	25	&	\forme{ɣnɤsqamŋu}  \\	
16 & \forme{sqaprɤɣ} &	26	&	\forme{ɣnɤsqaprɤɣ}  \\	
17 & \forme{sqaɕnɯz} &	27	&	\forme{ɣnɤsqaɕnɯz}  \\	
18 & \forme{sqarcat} &	28	&	\forme{ɣnɤsqarcat}  \\	
19 & \forme{sqangɯt} &	29	&	\forme{ɣnɤsqangɯt}  \\	
\lspbottomrule
\end{tabular}
\end{table}		
 
The numerals 11-19 present three morphological changes in comparison with the basic numerals 1-9.

First, the form \japhug{sqi}{ten} alternates with \forme{sqa-}. The origin of this Ablaut is unknown, though it could be a type of \textit{status constructus} (\ref{sec:status.constructus}); some Gyalrongic languages, such as Khroskyabs have a similar alternation (\citealt[175-6]{lai17khroskyabs}). 

Second, the velar \forme{kɯ-} and uvular \forme{χ-/ʁ-} prefixes found in the base numerals are lost in all teens.

Third, a labial element \ipa{p} (\japhug{sqaptɯɣ}{eleven}, \japhug{sqaprɤɣ}{sixteen}), \ipa{m} (\japhug{sqamnɯz}{twelve}), or \ipa{w} (\japhug{sqafsum}{thirteen}) is inserted between the \forme{sqa-} and the following numeral root. It does not occur in 17, 18 and 19 (which already have a cluster), 14 and 15 (which have a cluster with a labial as first element).

The form \japhug{sqaptɯɣ}{eleven} contains an ablauted form of \japhug{tɤɣ}{one} as second element. The cluster \forme{-pt-} in this word is the only case in the language of a \ipa{p} followed by an obstruent. 

In \japhug{sqamnɯz}{twelve}, the labial linker is nasalized by the following \forme{n}. This is not a synchronic rule: for instance, a noun \japhug{ɕnaβndʑɣi}{snotty-nosed kid} has \forme{β} allomorph of \ipa{w} before a prenasalized obstruent (§ \ref{sec:subject.verb.compounds}). However, there are other cases of nasalization of labial consonants to \ipa{m} before nasal or prenasalized consonants in Japhug (see § XXX).

In \japhug{sqaprɤɣ}{sixteen}, not only the prefix \forme{kɯ-} is lost, the \forme{tʂ} affricate of the base form 	\japhug{kɯtʂɤɣ}{six} is replaced by \ipa{r}, preceded by the linking element \forme{-p-}. This \ipa{tʂ} \tld{} \ipa{r} alternation is evidence for a sound change \forme{*tr-} \fl{} \ipa{tʂ} (see § \ref{sec:second.member.alternation} for additional evidence).  The numeral \japhug{kɯtʂɤɣ}{six} contains two etymological prefixes, \forme{kɯ-} and a prefix \forme{*t-} that has fused with the root as \forme{-tʂɤɣ}. This \forme{*t-} prefix is possibly cognate with the \forme{d-} of its Tibetan cognate  \tibet{དྲུག་}{drug}{six} .


\subsection{Decades} \label{sec:decades}
The numerals for decades (Table \ref{sec:numeral.prefixes}) are relatively straightforward. With the exception of \japhug{ɣnɤsqi}{twenty} and \japhug{fsɯsqi}{thirty}, they are predictable by combining \japhug{sqi}{ten} to the corresponding numeral prefix (§ \ref{sec:numeral.prefixes}).

The element \forme{ɣnɤ-} in \japhug{ɣnɤsqi}{twenty} is related to the numeral  \japhug{ʁnɯz}{two}, but present a velar \forme{ɣ-} prefix instead of the uvular \forme{ʁ-}, and has a different vowel. The adverb \japhug{ʁnaʁna}̌{both} is also relatable, but the alternations are not explainable from a synchronic point of view.

\begin{table}
\caption{Decades}  \label{tab:decades} \centering
\begin{tabular}{lllllll}
\lsptoprule
10	&	\forme{sqi} \\			
20	&	\forme{ɣnɤsqi} \\		
30	&	\forme{fsɯsqi}  \\		
40	&	\forme{kɯβdɤ-sqi}  \\	
50	&	\forme{kɯmŋɤ-sqi}  \\	
60	&	\forme{kɯtʂɤ-sqi}  \\	
70	&	\forme{kɯɕnɤ-sqi}  \\	
80	&	\forme{kɯrcɤ-sqi}  \\	
90	&	\forme{kɯngɯ-sqi}  \\	
\lspbottomrule
\end{tabular}
\end{table}		

Other numerals under one hundred are built by combining the forms in Table \ref{sec:teens} and \ref{tab:decades}. For instance, 37 can be obtained by putting together \japhug{fsɯsqi}{thirty} and \japhug{sqaɕnɯz}{seventeen} as \forme{fsɯ-sqa-ɕnɯz}.

\subsection{Hundred and above} \label{sec.hundred.plus}
 There are two ways of expressing numbers above 99 in Japhug. First, the noun-like numeral \japhug{ɣurʑa}{one hundred} can serve on its own or as a postnominal modifier, and be followed by another numeral to express a number between 101 and 199, as in (\ref{ex:hundred}).

\begin{exe}
\ex \label{ex:hundred} 
\gll aʑo 	kɯ-fse 	kɯ-cʰɯ\redp{}cʰa 	ʑo 	ʁʑɯnɯ 	ɣurʑa 	kɯrcat 	ra \\
\textsc{1sg} \textsc{nmlz}:S/A-be.like  \textsc{nmlz}:S/A-\textsc{emph}\redp{}can \textsc{emph} young.man hundred eight need:\textsc{fact} \\
\glt `I need one hundred and eight able young men like me.' (Norbzang, 16)
\end{exe}

The numeral \japhug{ɣurʑa}{one hundred} cannot be combined with unit numerals to express numbers between 200 and 900. The counted noun \japhug{tɯ-ri}{one hundred} is used for this purpose, as in \ref{ex:three.hundreds} (see § \ref{sec:numeral.prefixes} on the numeral prefixes). The two suppletive roots for hundreds are shared with Pumi (\forme{ɕí} `hundred' vs prefixed \forme{-ɻɛj}, see \citealt[101]{daudey14grammar}; evidence for cognacy with \forme{ɣurʑa} and \forme{-ri} respectively is presented in \citealt{jacques17num}).

\begin{exe}
\ex \label{ex:three.hundreds}
\gll χsɯ-ri 	jamar 	ndɤre 	tu-nɯ 	ko, 	tɯ-tɯpʰu 	nɯ \\
three-hundred about \textsc{lnk} exist:\textsc{fact-pl} \textsc{sfp} one-hive \textsc{dem} \\
\glt There are about three hundred of them, in one hive. (Bees, 48)
\end{exe}
 
  Numerals above the hundreds are all borrowed from Tibetan: \japhug{stoŋtsu}{thousand}, \japhug{kʰrɯtsu} {ten thousand}, \japhug{mbɯmχtɤr}{hundred thousand} from \tibet{སྟོང་ཚོ་}{stoŋ.tsʰo}{thousand}, \tibet{ཁྲི་ཚོ་}{kʰri.tsʰo}{ten thousand} and \tibet{འབུམ་ཐེར་}{ⁿbum.tʰer}{hundred thousand} respectively.  
  
Unlike numerals under 100, \japhug{ɣurʑa}{one hundred} and above are APNs and can take a third person possessive prefix \forme{ɯ-} to express an approximate number (§ \ref{sec:approx.numerals}).
 
 \subsection{Use of the numerals}  \label{sec:uses.numerals}
 Japhug numerals can occur on their own when counting (\forme{ci, ʁnɯz, χsɯ, kɯβde...}) or be used as postnominal attributive modifiers. The noun can be elided when the context is clear, especially when the same referent occurs in the previous proposition as in (\ref{ex:WrJit.kWngWt}  with \japhug{ɯ-rɟit}{her children}) and (\ref{ex:kWBde.kW} with <cai> `dish').

\begin{exe}
\ex \label{ex:WrJit.kWngWt} 
\gll
ɯ-rɟit kɯngɯt tɤ-tu ri, kɯtʂɤɣ nɯ-si \\
\textsc{3sg.poss}-child nine \textsc{pfv}-exist \textsc{lnk} six \textsc{pfv}-die \\
\glt `She had nine children, but six of them died.' (14-tApi taRi, 17)
\end{exe}

\begin{exe}
\ex \label{ex:kWBde.kW} 
\gll <cai> χsɯm tu-sɯ-lɤt-i tɕe tɕe  kɯβde nɯ kɯ χsɯm tu-ndza-j kɯ-fse. \\
dish three \textsc{ipfv}-\textsc{caus}-throw-\textsc{1pl} \textsc{lnk} \textsc{lnk} four \textsc{dem} \textsc{erg} three \textsc{ipfv}-eat-\textsc{1pl} \textsc{nmlz}:S/A-be.like \\
\glt `We used to order three dishes, and the four of us would eat them.' (140501 jingli, 92-3)
\end{exe}		

Numerals can also be modifiers of dual and plural pronouns as in (\ref{ex:iZo.kWBde}). Since pronouns are never obligatory in Japhug (§ XXX), it is also possible to use a bare numeral in core argument function with first or second person indexation on the verb, as in (\ref{ex:kWBde.kW}), where the verb \forme{tu-ndza-j} `we eat' of the second proposition has the \textsc{1pl} \forme{-j} suffix indexing the transitive subject, coreferent with the ergatively-marked phrase \forme{kɯβde nɯ kɯ} `the four' (standing for \forme{iʑo kɯβde nɯ kɯ} `the four of us').

\begin{exe}
\ex \label{ex:iZo.kWBde} 
\gll tɕe iʑo kɯβde nɯ tɯtɯrca ku-rɤʑi-j tɕe, \\
\textsc{lnk} \textsc{1pl} four \textsc{dem}  together \textsc{ipfv}-stay-\textsc{1pl} \textsc{lnk} \\
\glt `The four of us were living together.' (140501 jingli, 85)
\end{exe}		

In some constructions, bare numerals can have a specific meaning. For instance, with the verb \japhug{pa}{pass X years} (an intransitive counterpart of the transitive \japhug{pa}{do}), numerals obligatorily refer to years, as in (\ref{ex:40.topa}). It is not possible in this construction to replace the bare numeral by the counted noun  \japhug{tɯ-xpa}{one year}.

\begin{exe}
\ex \label{ex:40.topa} 
\gll tɕiʑo ni kɤ-amɯfse-tɕi nɯ jinde kɯβdɤsqi ɯ-ro to-pa \\
\textsc{1du} \textsc{du} \textsc{pfv}-know.each.other-\textsc{1du} \textsc{dem} nowadays forty \textsc{3sg.poss}-excess \textsc{ifr}-pass.X.years \\
\glt  `We have known each other for more than forty years.' (`More than forty years passed', 12-BzaNsa, 3)
 \end{exe}		
 
\section{Approximate numerals} \label{sec:approx.numerals}
To express an approximate number, it is possible in Japhug to use the adverb \japhug{jamar}{about} (from Tibetan \tibet{ཡར་མར་}{jar.mar}{about, up and down} and/or to combine adjacent numerals in a row as in (\ref{ex:RnWz.XsWm.kWBde}).

\begin{exe}
\ex \label{ex:RnWz.XsWm.kWBde}
\gll tɯ-kʰɤl nɯtɕu ʁnɯz, χsɯm kɯβde jamar ku-ndzoʁ. \\
one-place \textsc{dem}:\textsc{loc} two three four about \textsc{ipfv}-\textsc{anticaus}:attach \\
\glt `Two, three or four (of its flowers) grow in one place.' (16-RlWmsWsi, 9)
\end{exe}

However, Japhug also has five morphological devices to build approximate numerals (illustrated in Tables \ref{tab:approx.num.1to10} and XXX).

First, for numerals under seven (Table \ref{tab:approx.num.1to10}), one can build approximate numerals by prefixing a \forme{la-} or \forme{lɤ-} element to one (or two) numeral root(s). Not all possibilities are attested (for instance there is no such approximate numeral $\dagger$\forme{lɤtʂɤɣ} derived from only \japhug{kɯtʂɤɣ}{six}).  Prefixation of \forme{la-} / \forme{lɤ-} occurs with other morphological changes: (i) loss of the velar \forme{kɯ-} prefix (but not the uvular one in `two' and `three', § \ref{sec:one.to.ten}) (ii) loss of the \forme{m-} preinitial in  \japhug{lɤŋu}{about five}, but not of the \forme{*t-} prefix of \forme{kɯtʂɤɣ} (§ \ref{sec:teens}) in \japhug{lɤŋɤtʂɤɣ}{five or six} (otherwise $\dagger$\forme{lɤŋɤrɤɣ} would be have been found). The \forme{la-} / \forme{lɤ-} prefix is probably historically related to the \forme{-lɤ-} element found in \textit{dvandva} collectives (§ \ref{sec:dvandva.coll}).

Second, some approximate numerals are built by compounding two numeral roots (in some cases with the  \forme{la-} / \forme{lɤ-} prefix, Table \ref{tab:approx.num.1to10}). The first numeral undergoes \textit{status constructus} vowel change (§ \ref{sec:status.constructus}), with loss of the codas \forme{-z} and \forme{-t} (§ \ref{sec:loss.codas.compounds}). In the case of \japhug{ɕnɤcat}{seven or eight} (illustrated by example \ref{ex:CnAcat.ci}), the form \forme{ɕnɤ-} is irregular ($\dagger$\forme{ɕnɯ-} would be expected instead).

\begin{exe}
\ex \label{ex:CnAcat.ci} 
\gll ɯʑo nɯnɯ ɕnɤcat ci tɤ-kɤ-sɯpa jamar ʑo qarma wxti ri, \\
\textsc{3sg} \textsc{dem} seven.or.eight one \textsc{pfv}-\textsc{nmlz}:P-\textsc{caus}-do about \textsc{emph} crossoptilon be.big:\textsc{fact} but \\
\glt `Although the crossoptilon is as big as about seven or eight of them (weasels) put together.' (27-spjaNkW, 56)
\end{exe}

Third, for decades (Table \ref{tab:approx.decades}), approximate forms can be formed using the same rule as decades from 40 to 90 (§ \ref{sec:decades}), by combining the \textit{status constructus} of the under ten numeral with the root \japhug{sqi}{ten}, for instance  \japhug{lɤŋɤsqi}{about fifty}   from	 \japhug{lɤŋu}{about five} 
 like \japhug{kɯmŋɤsqi}{fifty} from \japhug{kɯmŋu}{five}. These approximate numerals are rare and not attested in the corpus.

Fourth, an alternative way of producing approximate decade numerals is to add the numeral prefix \forme{tɯ-} (§ \ref{sec:numeral.prefixes}) to a decade numeral, as for instance \japhug{tɯɣnɤsqi}{about twenty} from \japhug{ɣnɤsqi}{twenty}. Tshendzin considers approximate numerals formed this way to be acceptable for all decades from 20 to 90, but only \japhug{tɯɣnɤsqi}{about twenty} is attested in the corpus in example (\ref{ex:tWGnAsqi}).

\begin{exe}
\ex \label{ex:tWGnAsqi}
 \gll tɯɣnɤsqi jamar tɯtɯrca ju-ɣi-nɯ ŋgrɤl \\
 about.twenty about together \textsc{ipfv}-come-\textsc{pl} be.usually.the.case:\textsc{fact} \\
\glt  `They come in groups of about twenty individuals.' (23-qapGAmtWmtW, 105)
\end{exe}

\begin{table}
\caption{Approximate numerals in Japhug (one to ten)} \label{tab:approx.num.1to10} \centering
\begin{tabular}{llllll}
\lsptoprule
Approximate Numeral & Base Numerals \\
\midrule
\japhug{laʁnɯz}{a few} & \japhug{ʁnɯz}{two} \\
\japhug{laʁnɯχsɯm}{two or three}  & 	\japhug{ʁnɯz}{two} \\
&\japhug{χsɯm}{three} \\
\japhug{lɤβdelɤŋu}{four or five}  & 		\japhug{kɯβde}{four} \\
 & 		\japhug{kɯmŋu}{five} \\
 \japhug{lɤŋu}{about five}   & 		\japhug{kɯmŋu}{five} \\
\japhug{lɤŋɤtʂɤɣ}{five or six}  & 	\japhug{kɯmŋu}{five} \\
&\japhug{kɯtʂɤɣ}{six} \\
\japhug{ɕnɤcat}{seven or eight}  & 	\japhug{kɯɕnɯz}{seven} \\
 & 	\japhug{kɯrcat}{eight} \\
\japhug{kɯngɯsqi}{nine or ten}  & 	\japhug{kɯngɯt}{nine} \\
& 	\japhug{sqi}{ten} \\
\lspbottomrule
\end{tabular}
\end{table}

\begin{table}
\caption{Approximate numerals in Japhug (decades)} \label{tab:approx.decades} \centering
\begin{tabular}{llllll}
\lsptoprule
Approximate Numeral & Base Form \\
\midrule
\japhug{tɯɣnɤsqi}{about twenty} & \japhug{ɣnɤsqi}{twenty} \\
\japhug{tɯfsɯsqi}{about thirty}  & 	\japhug{fsɯsqi}{thirty} \\
\midrule
 \japhug{lɤŋɤsqi}{about fifty}   & 		\japhug{lɤŋu}{about five} \\
\japhug{lɤŋɤtʂɤsqi}{fifty or sixty}  & 	\japhug{lɤŋɤtʂɤɣ}{five or six}  \\
\japhug{ɕnɤcɤsqi}{seventy or eighty}  & 	\japhug{ɕnɤcat}{seven or eight} \\
\lspbottomrule
\end{tabular}
\end{table}

 Five, in the case of numerals above 99, approximate numerals are built by prefixing a third singular \forme{ɯ-} prefix, as \japhug{ɯ-ɣurʑa}{several hundreds} (example \ref{ex:WGurZa}), \japhug{ɯ-stoŋtsu}{several thousands} (\ref{ex:WstoNtsu}) and higher numerals.
 
\begin{exe}
\ex \label{ex:WGurZa}
\gll  ɯ-ɣurʑa, χsɯ-ri jamar ndɤre tu-nɯ ko, tɯ-tɯpʰu nɯ  \\
\textsc{3sg.poss}-hundred three-hundred about \textsc{lnk} exist:\textsc{fact}-\textsc{pl} \textsc{sfp} one-hive \textsc{dem} \\
\glt `In one hive, there are about several hundreds, about three hundred of them.' (26-GZo, 51-2)
\end{exe}

\begin{exe}
\ex \label{ex:WstoNtsu}
\gll  tɯ-ŋga tɯ-rdoʁ nɯ ɯ-stoŋtsu ɯ-phɯ kɯ-fse ŋu ma \\
\textsc{indef.poss}-clothes one-piece \textsc{dem} \textsc{3sg.poss}-thousand \textsc{3sg.poss}-price \textsc{nmlz}:S/A-be.like be:\textsc{fact} \textsc{lnk} \\ 
\glt `One piece of clothes (made from it), its price is several thousand renminbi.' (05-qaZo, 81)
\end{exe}

\section{Counted nouns} \label{sec:counted.nouns}
Counted nouns (henceforth CN) are a subclass of nouns that differs from IPN, APN and UPN in that they take a numeral prefix (whose paradigms are described in §\ref{sec:numeral.prefixes}) or alternatively a \textsc{3sg} possessive prefix (§ \ref{sec:CN.quantifier} and §\ref{sec:CN.time}).

\subsection{Numeral prefixes} \label{sec:numeral.prefixes}
Numerals are postnominal when used as attributive modifiers (sec § \ref{sec:uses.numerals}, § XXX), but occur before the noun roots in \textit{dvigu} compounds (§ \ref{sec:bahuvrihi.n.n}), and the bound forms deriving from numerals are strictly prefixal. The form with numeral prefix `one' is the citation form of counted nouns.

In this section, numeral prefixes are described following several categories (1-10, 11-99, approximate numerals and prefixes derived from nouns) and irregular forms are discussed in a separated subsection (§ \ref{sec:irregular.numeral.prefixes}). A final subsection presents historical hypotheses to account for the numeral prefixal paradigm and its relationship to that of other Gyalrongic languages.

\subsubsection{Numeral prefixes between 1 and 10}
The paradigm of regular numeral prefixes from 1 to 10 in Kamnyu Japhug is indicated in Table \ref{tab:num.prefix.1.to.10}. The prefixed are derived from the corresponding numeral by \textit{status constructus} (§ \ref{sec:status.constructus}), with loss of the coda and vowel alternation. Vowel alternation is optional for \japhug{kɯβde}{four} and  \japhug{kɯmŋu}{five}. The alternation between \japhug{tɤɣ}{one} and \forme{tɯ-} is irregular. The form \forme{tɤ-}, which is otherwise attested in another  paradigm (§ \ref{sec:irregular.numeral.prefixes}), should be the regular form. In the corpus,  the coda \forme{-z} of the numeral \japhug{kɯɕnɯz}{seven} seems to be preserved in one example (\ref{ex:kWCnWztWphu}), but this is an incorrect form, with a slight pause of hesitation between the numeral of the following noun.

\begin{exe}
\ex \label{ex:kWCnWztWphu}
 \gll kɯɕnɯz... -tɯpʰu ʑo pjɤ-tu \\
 seven -types \textsc{emph} \textsc{ifr}.\textsc{ipfv}-exist \\
 \glt `There were five types (of meals on the table).' (140504 baixuegongzhu-zh, 57)
\end{exe}


 \begin{table}
\caption{1-10 regular numeral prefixes in Japhug}  \label{tab:num.prefix.1.to.10} \centering
\begin{tabular}{lllllll}
\toprule
Numeral & Free form &  \forme{-sŋi} `day'   \\
\midrule
 1	&	\forme{tɤɣ}  &	\forme{tɯ-sŋi}  &	\\
2	&	\forme{ʁnɯz}  &	\forme{ʁnɯ-sŋi}  &	\\
3	&	\forme{χsɯm}  &	\forme{χsɯ-sŋi}  &	\\
4	&	\forme{kɯβde}  &	\forme{kɯβde-sŋi}, \forme{kɯβdɤ-sŋi}  &	\\
5	&	\forme{kɯmŋu}  &	\forme{kɯmŋu-sŋi}, \forme{kɯmŋɤ-sŋi}  &	\\
6	&	\forme{kɯtʂɤɣ}  &	\forme{kɯtʂɤ-sŋi}  &	\\
7	&	\forme{kɯɕnɯz}  &	\forme{kɯɕnɯ-sŋi}  &	\\
8	&	\forme{kɯrcat}  &	\forme{kɯrcɤ-sŋi}  &	\\
9	&	\forme{kɯngɯt}  &	\forme{kɯngɯ-sŋi}  &	\\
10	&	\forme{sqi}  &	\forme{sqɯ-sŋi}  &\\
\bottomrule
\end{tabular}
\end{table}

Note that the form of the numeral prefixes differ in some cases from the corresponding numeral prefixes in decades (cf table \ref{tab:decades}), compare \japhug{kɯɕnɯ-sŋi}{seven days} with \japhug{kɯɕnɤ-sqi}{seventy} or  \japhug{χsɯ-sŋi}{three days} with \japhug{fsɯ-sqi}{seventy}.

\subsubsection{Numeral prefixes between 11 and 99}
Above ten, numerals present some variation.  Table   \ref{tab:num.prefix.11.to.20} show the most common forms of the numerals between 11 and 20 (from which all numerals between 11 and 99 can be generated following the rules described in § \ref{sec:decades}), but many cases without vowel alternation or with preservation of coda \forme{-z} (\ref{ex:GnAsqamnWzpArme}) or \forme{-ɣ} (\ref{ex:GnAsqaptWWGrZaR}) are attested.

 \begin{table}
\caption{11-20 numeral prefixes in Japhug}  \label{tab:num.prefix.11.to.20} \centering
\begin{tabular}{lllllll}
\toprule
Numeral & Free form &  \forme{-sŋi} `day'   \\
\midrule
11	&	\forme{sqaptɯɣ}  &	\forme{sqaptɯ-sŋi}  &	\\
12	&	\forme{sqamnɯz}  &	\forme{sqamnɯ-sŋi}  &	\\
13	&	\forme{sqafsum}  &	\forme{sqafsum-sŋi}  &	\\
14	&	\forme{sqaβde}  &	\forme{sqaβde-sŋi}  &	\\
15	&	\forme{sqamŋu}  &	\forme{sqamŋu-sŋi}  &	\\
16	&	\forme{sqaprɤɣ}  &	\forme{sqaprɤ-sŋi}  &	\\
17	&	\forme{sqaɕnɯz}  &	\forme{sqaɕnɯ-sŋi}  &	\\
18	&	\forme{sqarcat}  &	\forme{sqarcɤ-sŋi}  &	\\
19	&	\forme{sqangɯt}  &	\forme{sqangɯ-sŋi}  &	\\
20	&	\forme{ɣnɤsqi}  &	\forme{ɣnɤsqɯ-sŋi}   &	\\
\bottomrule
\end{tabular}
\end{table}

In example (\ref{ex:fsWsqildZa}), we observe the two alternative forms \forme{-sqɯ-} and \forme{-sqi-} for the decades in the same sentence. While for the numeral ten only the prefix \forme{sqɯ-} (or its variant \forme{sqɤ-}, see Table XXX below) is found, for decades between 20 and 90 vowel alternation is optional and there is free variation between the two forms. In the corpus, we find 15 examples of \forme{-sqi-} and 14 of  \forme{-sqɯ-}, suggesting that both are about equally common.

\begin{exe}
\ex \label{ex:fsWsqildZa}
\gll tɯ-pʰɯ nɯ tɕe rcanɯ, li, fsɯsqi-ldʑa jamar, kɯβdɤsqɯ-ldʑa jamar tu. \\
one-tree \textsc{dem} \textsc{lnk} \textsc{unexpectedly} again thirty-\textsc{cl} about forty-\textsc{cl} about exist:\textsc{fact} \\
\glt `On one tree, there are about thirty or forty (branches).'   (14-sWNgWJu, 200)
\end{exe}

\begin{exe}
\ex \label{ex:GnAsqamnWzpArme}
\gll  ma ɯ-me kɯnɤ ɣnɤsqamnɯz-pɤrme tʰɯ-azɣɯt, \\
\textsc{lnk} \textsc{3sg}.\textsc{poss}-daughter also twenty.two-year.old \textsc{pfv}-reach \\
\glt `Even his daughter is now twenty-two.' (14-tApitaRi, 317)
\end{exe}

Example (\ref{ex:GnAsqaptWWGrZaR}) illustrates three alternative forms with the counted noun \japhug{tɤ-rʑaʁ}{one night}: \forme{ɣnɤsqaptɯ-rʑaʁ} with regular loss of coda, \forme{ɣnɤsqaptɯɣ-rʑaʁ} with preservation of the coda,   and \forme{ɣnɤsqamnɯz} \forme{tɤ-rʑaʁ} as two words, the numeral being a kind of prenominal modifier. The first form is regular, while the other ones each are \textit{hapax legomena}.

\begin{exe}
\ex \label{ex:GnAsqaptWWGrZaR}
\gll  tɕe nɯ ɣnɤsqaptɯɣ-rʑaʁ tu-tsu ɲɯ-ra. ``tɕe ɣnɤsqaptɯ-rʑaʁ tu-tsu tɕe ɲɯ-ʁaʁ ŋu" ɲɯ-ti-nɯ ri, aʑɯɣ nɯ ɣnɤsqamnɯz tɤ-rʑaʁ mɤɕtʂa mɯ-nɯ-ʁaʁ. \\
\textsc{lnk} \textsc{dem} twenty.one-night \textsc{ipfv}-pass \textsc{sens}-have.to  
\textsc{lnk}  twenty.one-night \textsc{ipfv}-pass  \textsc{lnk} \textsc{ipfv}-hatch be:\textsc{fact} \textsc{sens}-say-\textsc{pl} \textsc{lnk} \textsc{1sg}:\textsc{gen} \textsc{dem} twenty.two one-night until \textsc{neg}-\textsc{pfv}-hatch \\
\glt `(Eggs) need twenty-two days to hatch; people say `They hatch in twenty-two days' but mine only hatch after twenty two days.' (150819 kumpGa, 34-36)
\end{exe}


\subsubsection{Approximate numeral prefixes}
Approximate numerals also have corresponding prefixal forms. Table \ref{tab:approx.num.prefixes} presents the forms attested in the corpus.

 \begin{table}
\caption{Approximate numeral prefixes in Japhug} \label{tab:approx.num.prefixes} \centering
\begin{tabular}{llllll}
\lsptoprule
Approximate Numeral & Approximate Numeral ɓrefix \\
\midrule
\japhug{laʁnɯz}{a few} & \forme{laʁnɯ-} \\
\japhug{lɤβdelɤŋu}{four or five}  & 		\forme{lɤβdelɤŋu-}  \\
 \japhug{lɤŋu}{about five}   & 		\forme{lɤŋu-}  \\
\japhug{lɤŋɤtʂɤɣ}{five or six}  & 	\forme{lɤŋɤtʂɤ-}, \forme{lɤŋɤtʂɤɣ-} \\
\japhug{ɕnɤcat}{seven or eight}  & 	\forme{ɕnɤcɤ-} \\
\lspbottomrule
\end{tabular}
\end{table}

By far the most commonly used approximate numeral prefix is \forme{laʁnɯ-X}, whose meaning is not  `one or two' as could have been expected, but  `a few' (see example \ref{ex:laʁnWxpa} -- life expectancy of goats and sheep is much above four years).

\begin{exe}
\ex \label{ex:laʁnWxpa}
\gll tsʰɤt qaʑo nɯnɯ tɕe laʁnɯ-xpa ma cʰɯ-mdɯ mɯ́j-ŋgrɤl ma tɕe cʰɯ-rgɤz ɕti \\
goat sheep \textsc{dem} \textsc{lnk} a.few-year apart.from \textsc{ipfv}-live.up.to \textsc{neg}:\textsc{sens}-be.usually.the.case \textsc{lnk} \textsc{lnk} \textsc{ipfv}-be.old be.\textsc{affirm}:\textsc{fact} \\
\glt `Goats and sheep only live for a few years, and then become old.' (05-qaZo, 144)
\end{exe}

\subsubsection{Other numeral prefixes} \label{sec:other.numeral.prefixes}
In addition to the numerals mentioned above, some nouns can appear as prefixes of counted nouns, in particular the numerals above 99 and the participle of the stative verb \japhug{antɕʰɯ}{be many}. XXXXX  \ref{sec:thAstWG} \forme{tʰɤstɯ-} \japhug{tʰɤstɯɣ}{how many} 

\subsubsection{Irregular forms} \label{sec:irregular.numeral.prefixes}
A handful of CNs, in particular \japhug{tɤ-rʑaʁ}{one night}, have an alternative paradigm with \ipa{ɤ} instead of \forme{ɯ} in the prefixes, as shown in Table \ref{tab:num.prefix.tArZaR}). There is however some degree of variation, and using the regular paradigm is not considered erroneous. In the corpus, the numeral `one' form \japhug{tɯ-rʑaʁ}{one night} is actually considerably more common than \forme{tɤ-rʑaʁ}, possibly because of the homophony with the IPN \japhug{tɤ-rʑaʁ}{time} found in examples such as (\ref{ex:tArZaR.tArYJi}). For other numerals, the forms in Table (\ref{tab:num.prefix.tArZaR}) are considerably more common than the regular ones.

\begin{exe}
\ex \label{ex:tWrZaR}
\gll tɕe qarma nɯ, tɯ-rʑaʁ tɕe kɯβde kɯmŋu jamar pjɯ-sat-nɯ, tɯ-rdoʁ, tɯrme tɯ-rdoʁ kɯ \\
\textsc{lnk} crossoptilon \textsc{dem} one-night \textsc{lnk} four five about \textsc{ipfv}-kill-\textsc{pl}, one-\textsc{cl} person one-\textsc{cl} \textsc{erg} \\
\glt `Crossoptilons, in one night, each of (the hunters) can kill four or five of them.' (23-qapGAmtWmtW, 163)
\end{exe}

\begin{exe}
\ex \label{ex:tArZaR.tArYJi}
\gll tɤ-rʑaʁ tɤ-rɲɟi tɕe, nɯ-ji ra kɯ-dɯ\redp{}dɤn kɯ-jɯ\redp{}jom lo-pɣaʁ-nɯ, ɕoŋtɕa kɯ-dɯ\redp{}dɤn pjɤ-pʰɯt-nɯ \\
\textsc{indef}.\textsc{poss}-time \textsc{pfv}-be.long \textsc{lnk} \textsc{3pl}.\textsc{poss}-field \textsc{pl} \textsc{nmlz}:S/A-\textsc{emph}\redp{}be.many \textsc{nmlz}:S/A-\textsc{emph}\redp{}be.broad \textsc{ifr}-turn.over-\textsc{pl} timber \textsc{nmlz}:S/A-\textsc{emph}\redp{}be.many \textsc{ifr}-remove-\textsc{pl} \\
\glt `After some time, (those people) had ploughed many broad fields for them, and chopped a lot of timber.' (2002qajdoskAt, 90)
\end{exe}

 \begin{table}
\caption{Irregular numeral prefixes in Japhug}  \label{tab:num.prefix.tArZaR} \centering
\begin{tabular}{lllllll}
\lsptoprule
Numeral & Free form  &  \forme{-rʑaʁ} `night' \\
\midrule
 1	&	\forme{tɤɣ}  &		\forme{tɤ-rʑaʁ}  &	\\
2	&	\forme{ʁnɯz}  &		\forme{ʁnɤ-rʑaʁ}  &	\\
3	&	\forme{χsɯm}  &		\forme{χsɤ-rʑaʁ}  &	\\
4	&	\forme{kɯβde}  &		\forme{kɯβdɤ-rʑaʁ}  &	\\
5	&	\forme{kɯmŋu}  &		\forme{kɯmŋɤ-rʑaʁ}  &	\\
6	&	\forme{kɯtʂɤɣ}  &		\forme{kɯtʂɤ-rʑaʁ}  &	\\
7	&	\forme{kɯɕnɯz}  &		\forme{kɯɕnɤ-rʑaʁ}  &	\\
8	&	\forme{kɯrcat}  &		\forme{kɯrcɤ-rʑaʁ}  &	\\
9	&	\forme{kɯngɯt}  &		\forme{kɯngɤ-rʑaʁ}  &	\\
10	&	\forme{sqi}  &	\forme{sqɤ-rʑaʁ}  &	\\
\lspbottomrule
\end{tabular}
\end{table}

Apart from  \japhug{tɤ-rʑaʁ}{one night}, CNs following the paradigm in Table (\ref{tab:num.prefix.tArZaR}) are very rare. The CN \japhug{tɯ-tɣa}{one span} has the irregular form \japhug{χsɤ-tɣa}{three spans} (competing with regular \forme{χsɯ-tɣa}) as in (\ref{ex:XsAtGa}), but not for other numerals. 

\begin{exe}
\ex \label{ex:XsAtGa}
\gll nɯ χsɤ-tɣa kɯβde-tɣa jamar tu-rɲɟi cʰa. \\
\textsc{dem} three-span four-span about \textsc{ipfv}-be.long can:\textsc{fact} \\
\glt `It can grow three or four spans long.' (14-sWNgWJu, 194)
\end{exe}

Another unrelated irregularity concerns the CN \japhug{tɯ-ɣjɤn}{one time}, which presents two numeral `three' forms; while the regular form \japhug{χsɯ-ɣjɤn}{three times} is common, some speakers also use the monosyllabic fused form \japhug{χsjɤn}{three times}.

\subsubsection{Historical perspectives on the numeral prefixal paradigm}

\subsection{CNs as quantifiers} \label{sec:CN.quantifier}

\subsection{CNs and semantic classes} \label{sec:CN.classification}
\subsection{Derivation from CNs to IPNs}  

\section{Counting time} \label{sec:time}
\subsection{Time CNs} \label{sec:CN.time}

\subsection{Time ordinals} \label{sec:time.ordinals}

\subsection{Other nouns expressing time} \label{sec:nominal.time}
\section{Basic arithmetic operations} \label{sec:arithmetic}
 %Numerals and classifiers
%\chapter{The noun phrase} \label{chap:noun.phrase}

\section{Independent words vs clitics}
The present chapter deals with grammatical elements that are independent words rather than affixes, like those described in chapter \ref{chap:nominal.morphology}. Since some scholars such as 
\citet{jackson98morphology, jackson14morpho} treat the postpositions and the number modifiers as clitics rather than independent words as is done in the present work, a justification of my analysis is necessary.

The postpositions \japhug{kɯ}{ergative} and  \japhug{ɣɯ}{genitive} do have some clitic-like characteristics: they cannot be used without a preceding noun phrase (or a subordinate clause in some cases, see § XXX), are unstressed, and in the case of the genitive have special irregular forms with pronouns (§ \ref{sec:pronouns.gen}).

However, a pause can occur between these postpositions (\ref{ex:kW.nAmWmnW}) and the noun phrase they follow. For instance, in example (\ref{ex:kW.nAmWmnW}), a two second pause (with an inspiration) is found between the phrase \forme{nɯŋa ra} and the following ergative \forme{kɯ}. 

\begin{exe}
\ex \label{ex:kW.nAmWmnW}
\gll tɕe tɯrtsi nɯ pjɯ́-wɣ-βzu tɕe, nɯŋa ra, kɯ nɤ-mɯm-nɯ cʰo wuma ʑo ɣɯ-ɕɯ-fka-nɯ \\
\textsc{lnk} cow.food \textsc{dem} \textsc{ipfv}-\textsc{inv}-make \textsc{lnk} cow \textsc{pl} \textsc{erg} \textsc{trop}-be.tasty:\textsc{fact}-\textsc{pl} \textsc{comit} really \textsc{emph} \textsc{inv}-\textsc{caus}-be.satiated:\textsc{fact}-\textsc{pl} \\
\glt `They make cow food with flour, the cows find it tasty, and it satisfies their hunger.' (140513 tWrtsi, 15)
\end{exe}


Such cases are by no means exceptional; at least 54+35 examples of ergative and genitive preceded by a pause are attested in the corpus (they can be found by searching \forme{kɯ} or \forme{ɣɯ}  preceded by a comma). Most of these cases are found in sentences where the speaker hesitates, and are especially common in texts translated from Chinese.

\section{Postpositions} \label{ex:postpositions}

\subsection{Absolutive} \label{sec:absolutive}
\subsubsection{Intransitive subject}
\subsubsection{Object}
\subsubsection{Semi-object}
\subsubsection{Theme}
\subsubsection{Essive} \label{sec:essive.abs}
%tsuku kɯ paʁndza ɲɯ-nɯ-phɯt-nɯ ɲɯ-ŋu ri,
\subsubsection{Goal} \label{absolutive.goal}
\subsubsection{Locative adjunct} \label{absolutive.locative}

\subsection{Ergative} \label{sec:erg.kW}
\subsubsection{Transitive subject} \label{sec:A.kW}
\subsubsection{Instrumental} \label{sec:instr.kW}

%manner
%kumpɣa cho khɯna ni li tɤ-mqe tɤ-ndɯt kɯ jo-ɣi-ndʑi tɕe,
\subsubsection{Causee} \label{sec:causee.kW}
\subsubsection{Comparee marker} \label{sec:comparee.kW}

%mahi nɯnɯ kɯ aʑo sɤz cha
\subsubsection{Partitive} \label{sec:partitive.kW}
\subsubsection{Oblique argument} \label{sec:oblique.kW}
 The transitive verb \japhug{kʰɤt}{do repeatedly, do a long time} and its causative form \japhug{sɯ-kʰɤt}{cause to do repeatedly, cause to do a long time} occur in a construction with instrumental-like noun phrases marked with the ergative \forme{kɯ}, indicating the action which is performed repeatedly or done over a long time. These noun phrases can include either an action nominal derived from a verb with the prefix \forme{tɯ-} (§ XXX) as in (\ref{ex:tWqioR.kW}), or an underived action noun, as in (\ref{ex:tama.kW.takhAt}) and (\ref{ex:khAcAl.kW.takhata}).  
 
  \begin{exe}
\ex \label{ex:tWqioR.kW}
\gll tɯ-qioʁ kɯ tó-wɣ-sɯ-kʰɤt ʑo tɕe, tɕe nóʁmɯz nɤ tɯɣ nɯnɯ ló-wɣ-sɯ-tɕɤt  \\
\textsc{nmlz:action}-vomit \textsc{erg} \textsc{ifr-inv-caus}-do.a.long.time \textsc{emph} \textsc{lnk} \textsc{lnk} only.then \textsc{lnk} poison \textsc{dem} \textsc{ifr-inv-caus}-take.out \\
\glt `(The medicine) caused (Gesar) to vomit a long time until he expelled the poison.' (Gesar, 266)
\end{exe}

  \begin{exe}
\ex \label{ex:tama.kW.takhAt}
\gll ta-ma kɯ ta-kʰɤt ʑo  \\
\textsc{indef.poss}-work \textsc{erg} \textsc{pfv}:3$\rightarrow$3'-do.a.long.time \textsc{emph} \\
\glt `He did a lot of work.' (elicited)
\end{exe}

Example (\ref{ex:khAcAl.kW.takhata}), with the verb \japhug{kʰɤt}{do repeatedly, do a long time}  taking \textsc{1sg}\fl{}3 indexation (§ XXX), shows that the ergative phrase cannot be analyzed as a transitive subject; moreover, the fact that adding the causative in this case would imply a real causative interpretation (`cause X to repeatedly') also indicates that this phrase is not an instrumental adjunct (see § \ref{sec:instr.kW}).

  \begin{exe}
\ex \label{ex:khAcAl.kW.takhata}
\gll kʰɤcɤl kɯ tɤ-kʰat-a ʑo \\
conversation \textsc{erg} \textsc{pfv}-do.a.long.time-\textsc{1sg} \textsc{emph} \\
\glt `I have a long conversation.' (elicited)
\end{exe}

No other verb takes this type of oblique ergative phrase.

\subsection{Genitive} \label{sec:genitive}
With the exception of particular forms for some pronouns (§ \ref{sec:pronouns.gen}), the genitive postposition has the invariant form \forme{ɣɯ} in Kamnyu Japhug. Like the ergative \forme{kɯ}, it is likely borrowed from the Amdo clitic \forme{-ɣə/-kə} (\citealt[62]{haller04themchen}). It is used in possessive contructions, but also expresses beneficiary and recipient.

\subsubsection{Possession} \label{sec:gen.possession}
The genitive \forme{ɣɯ} occurs in various type of possessive constructions, including genitival noun complements and possessive existential predicates (§ XXX).

Inside the noun phrase, the genitive occurs between possessor and possessum, and a possessive prefix is found on the possessum (§ \ref{ex:prefix.expression.of.possession}), as in (\ref{ex:GZAndza.GW.WjwaR}).  

\begin{exe}
\ex \label{ex:GZAndza.GW.WjwaR}
\gll ri ɣʑɤndza ɣɯ ɯ-jwaʁ nɯra mɤ-wxti ri, ɲaʁ ʑo qhe, \\
\textsc{lnk} Agastache.rugosa \textsc{gen} \textsc{3sg}.\textsc{poss}-leaf \textsc{dem}:\textsc{pl} \textsc{neg}-be.big:\textsc{fact} \textsc{lnk} be.black:\textsc{fact} \textsc{emph} \textsc{lnk} \\
\glt `The leaves of the Agastache rugosa are not large and quite dark in colour.' (11-qarGW, 137)
\end{exe}

Genitival phrases without possessive prefix on the possessum are rare but do exist, in particular when the possessum is a noun borrowed from Chinese and non-fully nativized like \ch{国语}{guóyǔ}{national language} in (\ref{ex:iZo.GW.guoyu}).  

\begin{exe}
\ex \label{ex:iZo.GW.guoyu}
\gll iʑo ɣɯ <guoyu> ɲɯ-ŋu tɕe, nɯnɯ kɤsɯfse ɣɯ ji-rju ɲɯ-ŋu tɕe, \\
\textsc{1pl} \textsc{gen} national.language \textsc{sens}-be \textsc{lnk} dem all \textsc{gen} \textsc{1pl}.\textsc{poss}-speech \textsc{sens}-be \textsc{lnk} \\
\glt `(Chinese) is our national language, it is everybody's language.' (150901 tshuBdWnskAt, 15-16)
\end{exe}

For singular noun possessors, the presence or not of a third person possessive prefix \forme{ɯ-} is not always easy to tell from recordings, as due to the external sandhi (§ XXX), \forme{ɣɯ ɯ-} merges as \ipa{ɣɯ} when no pause occurs between the two. In careful speech, the third person prefix is clearly audible.

Nominal modifiers can sometimes be marked like possessors, with the genitive and/or with a possessive prefix on the following head noun, see § \ref{sec:gen.other}. 

The genitive can also appear between a noun phrase and a relator noun, and even be followed by focus markers in this position, as in (\ref{ex:GW.kWnA.WrkW.ri}).

\begin{exe}
\ex \label{ex:GW.kWnA.WrkW.ri}
\gll   tɯ-ci kɯ-wxti ɣɯ kɯnɤ ɯ-rkɯ ri nɯra tu ŋgrɤl.  \\
\textsc{indef}.\textsc{poss}-water \textsc{nmlz}:S/A-be.big \textsc{gen} also \textsc{3sg}.\textsc{poss}-side \textsc{loc} \textsc{dem}:\textsc{pl} exist:\textsc{fact} be.usually.the.case:\textsc{fact} \\
\glt `(Dragonflies) are also found near rivers.' (26-quspunmbro, 7)
\end{exe}

In these constructions, the genitive is always optional, and the prefix on the possessum suffices to express possession, as in (\ref{ex:paXCi.WjwaR}) (see § \ref{ex:prefix.expression.of.possession}).

\begin{exe}
\ex \label{ex:paXCi.WjwaR}
\gll paχɕi ɯ-jwaʁ tsa fse ri, nɯ sɤznɤ artɯm,\\
apple \textsc{3sg}.\textsc{poss}-leaf a.little be.like:fact \textsc{lnk} \textsc{dem} \textsc{comp} be.round:\textsc{fact} \\
\glt `(Its leaves) are a little like the leaves of an apple tree, but more round.' (09-mi, 15)
\end{exe}

When the possessum is elided however, the genitive postposition becomes obligatory, as in (\ref{ex:baigua.GW.sAz}).

\begin{exe}
\ex \label{ex:baigua.GW.sAz}
\gll ɯ-rɣi nɯnɯ, nɤki, <beigua> ɣɯ sɤz ɲɯ-jaʁjɯ. \\
\textsc{3sg}.\textsc{poss}-seed \textsc{dem} \textsc{filler}  pumpkin \textsc{gen} \textsc{comp} \textsc{sens}-be.thick.and.strong \\
\glt `Its seeds are thicker than those of the pumpkin.' (16-CWrNgo, 130)
\end{exe}

While there are transitive and semi-transitive verbs expressing possession (§ XXX), the most common possessive construction involves an existential verb taking the possessum as subject, with the possessor marked by a possessive prefix on the possessum, and optionally with the genitive, as in (\ref{ex:phu.nW.GW.WRrW.GAZu}). 

\begin{exe}
\ex \label{ex:phu.nW.GW.WRrW.GAZu}
\gll qartsʰaz pʰu nɯ ɣɯ ɯ-ʁrɯ ɣɤʑu \\
deer male \textsc{dem} \textsc{gen} \textsc{3sg}.\textsc{poss}-horn exist:\textsc{sens} \\
\glt `The male deer has horns.' (27-qartshAz, 32)
\end{exe}

This construction is also used for abstract possession, as in (\ref{ex:aZWG.aBlu.tu}).

\begin{exe}
\ex \label{ex:aZWG.aBlu.tu}
\gll aʑɯɣ a-βlu tu \\
\textsc{1sg}:\textsc{gen} \textsc{1sg}.\textsc{poss}-stratagem exist:\textsc{fact} \\
\glt `I have an idea.' (140507 tangguowu, 29)
\end{exe}

The causative verbs \japhug{ɣɤtu}{cause to have} and \japhug{ɣɤme}{cause not to have, destroy} derived from \japhug{tu}{exist} and \japhug{me}{not exist} respectively (see § XXX) select an oblique argument with the genitive, as in (\ref{ex:WZo.GW.tuGAtea}). Although this argument could be considered to be a type of beneficiary (§ \ref{sec:other.uses.poss.prefixes}), we observe here stability in case marking of the possessor between the base construction and the derived causative one.

\begin{exe} 
\ex \label{ex:WZo.GW.tuGAtea} 
\gll ɯʑo kɯ maka kɤ-ntɕʰoz mɤ-kɯ-ɤrɕo kɯ-fse ʑo tɯrɟɯ laχtɕʰa ɯʑo ɣɯ tu-ɣɤte-a jɤɣ \\ 
\textsc{3sg}.\textsc{poss} \textsc{erg} at.all \textsc{inf}-use \textsc{neg}-\textsc{inf}:\textsc{stat}-be.finished \textsc{inf}:\textsc{stat}-be.like \textsc{emph} wealth thing \textsc{3sg}.\textsc{poss} \textsc{gen} \textsc{ipfv}-\textsc{caus}-exist[III]-\textsc{1sg} be.possible:\textsc{fact} \\ 
\glt `(If someone saves me), I will make him have more wealth and riches than he can ever use.' (140512 yufu yu mogui, 84) 
\end{exe} 
%ma aʑo a-kɤ-cha,  a-kɤ-cha kɯ-tu nɯra, a-kɤ-spa, tu-βze-a kɤ-cha nɯra lonba ʑo nɤʑɯɣ tɤ-ɣɤtu-t-a ɕti tɕe,

Not all combinations of existential verbs and genitival phrases are existential possessive constructions. For instance, in (\ref{ex:BZW.GW.WqiW}), the second clause could appear to contain a possessive construction meaning `the mouse only has half of it', but the context makes it clear that a different interpretation is necessary (see § XXX on this use of the existential verbs).

\begin{exe}
\ex \label{ex:BZW.GW.WqiW}
\gll qamtɕɯr nɯ ɯ-mtɕʰi nɯnɯ βʑɯ sɤznɤ mɤʑɯ ʑo amtɕoʁ tɕe nɯ βʑɯ ɣɯ ɯ-qiɯ kɯnɤ me \\
shrew \textsc{dem} \textsc{3sg}.\textsc{poss}-mouth \textsc{dem} mouse \textsc{comp} yet \textsc{emph} be.pointy \textsc{lnk} \textsc{dem} mouse \textsc{gen} \textsc{3sg}.\textsc{poss}-half even not.exist:\textsc{fact} \\
\glt `The shrew's mouth is even sharper than that of the mouse, and (its size) is not even half that of the mouse.' (27-spjaNkW, 204-205)
\end{exe}

 
\subsubsection{Recipient and beneficiary} \label{sec:gen.beneficiary}
 
The genitive can be used to mark the recipient by the indirective verb \japhug{kʰo}{give, pass over}, as in (\ref{ex:aZWG.nWkhAm}) and (\ref{ex:GW.anWtWkhAm}).   

\begin{exe}
\ex \label{ex:aZWG.nWkhAm}
 \gll ɕɯ ʑo stu kɯ-mɤku pɯ-tɯ-mto-t nɯnɯ, laχtɕha pɯ-nnɯ-ŋu, tɯrme pɯ-nnɯ-ŋu nɯ, aʑɯɣ nɯ-kʰɤm tɕe tɕendɤre, aʑo ɲɯ-ta-lɤt jɤɣ \\
 who \textsc{emph} most \textsc{nmlz}:S/A-be.first \textsc{pfv}-2-see-\textsc{pst}:\textsc{tr} \textsc{dem}  thing \textsc{pst}.\textsc{ipfv}-\textsc{auto}-be   person \textsc{pst}.\textsc{ipfv}-\textsc{auto}-be \textsc{dem} \textsc{1sg}:\textsc{gen} \textsc{imp}-give[III] \textsc{lnk} \textsc{lnk} \textsc{1sg} \textsc{ipfv}-1\fl{}2-release be.possible:\textsc{fact} \\
 \glt `Give me the first thing you see (when you go back home), be it a person or an object, and I will release you.' (140506 shizi he huichang de bailingniao-zh, 50-52)
\end{exe}

\begin{exe}
\ex \label{ex:GW.anWtWkhAm}
 \gll jɤ-tsɯm tɕe iɕqʰa nɯ kɯβʁa nɯ ɣɯ a-nɯ-tɯ-kʰɤm \\
 \textsc{imp}-take.away \textsc{lnk} the.aforementioned \textsc{dem} noble \textsc{dem} \textsc{gen} \textsc{irr}-\textsc{pfv}-2-give[III] \\
 \glt  Take it and give it to the nobleman.' (150831 renshen wawa-zh, 43)
\end{exe}
 
The recipient of the verb  \japhug{kʰo}{give, pass over} can alternatively also be marked by a possessive prefix on the IPN \japhug{tɯ-jaʁ} (with the meaning  hand over', \ref{sec:semi.grammaticalized.relator}) or, with the dative relator nouns \forme{ɯ-ɕki} or \forme{ɯ-pʰe} (§ \ref{sec:dative}).

The genitive is selected by a few intransitive modal verbs to indicate the experiencer/beneficiary, in particular  \japhug{ra}{need, have to}, \japhug{ʁzi}{be necessary}, as in (\ref{ex:aZWG.WCArW}) and (\ref{ex:aZWG.Rzi}).

\begin{exe}
\ex \label{ex:aZWG.WCArW}
 \gll aʑɯɣ ɯ-ɕɤrɯ ra \\
 \textsc{1sg:gen} \textsc{3sg.poss}-bone have.to:\textsc{fact} \\
\glt `I want its bones.' (07-deluge, 9)
\end{exe}

\begin{exe}
\ex \label{ex:aZWG.Rzi}
 \gll aʑɯɣ wuma ʑo ʁzi ɲɯ-ŋu, a-kɤ-ntɕʰoz sna ɲɯ-ŋu \\
  \textsc{1sg:gen} really \textsc{emph} be.necessary:\textsc{fact} \textsc{sens}-be \textsc{1sg}.\textsc{poss}-nmlz:P-use be.good:\textsc{fact}  \textsc{sens}-be \\
  \glt `It will be useful for me, it will have good use of it.'  (150902 hailibu-zh, 44-45)
\end{exe}

The experiencer/beneficiary can also be marked by possessive prefixes on the subject, without genitive, as in 
(\ref{ex:arNWl.mAra}) (see also § \ref{sec:other.uses.poss.prefixes} for additional examples).

\begin{exe}
\ex \label{ex:arNWl.mAra}
 \gll aʑo a-rŋɯl a-χsɤr ra mɤ-ra \\
 \textsc{1sg} \textsc{1sg}.\textsc{poss}-silver \textsc{1sg}.\textsc{poss}-gold \textsc{pl} \textsc{neg}-have.to:\textsc{fact} \\
 \glt `I don't  need silver or gold.' (2014-kWlAG, 367)
\end{exe}

The genitive also occurs with beneficiaries/maleficiaries as adjuncts, not selected by the main verb, with transitive verbs such as \japhug{nɤma}{do} (\ref{ex:tChi.tunAmea}) and \japhug{wum}{collect} (\ref{ex:WZAG.pjAmaR}) or stative intransitive verbs such as \japhug{pe}{be good} as in (\ref{ex:aZWG.mApe}).

\begin{exe}
\ex \label{ex:tChi.tunAmea}
\gll nɤʑɯɣ tɕʰi tu-nɤme-a ra, tɤ-ti  \\
\textsc{2sg}:\textsc{gen} what \textsc{ipfv}-do[III]-\textsc{1sg} have.to:\textsc{fact} \textsc{imp}-say \\
\glt `Tell me what I shall do for you.' (140511 alading-zh, 175)
\end{exe}

\begin{exe}
\ex \label{ex:aZWG.mApe}
\gll  ɯ-fso tʰɯ-wxti tɕe aʑɯɣ mɤ-pe \\ 
\textsc{3sg}.\textsc{poss}-tomorrow \textsc{pfv}-be.big \textsc{lnk} \textsc{1sg}:\textsc{gen} \textsc{neg}-be.good:\textsc{fact} \\
\glt `In the future, when he will have grown up, he will cause me trouble.' (`he will not be good to me', 2011-05-nyima, 22)
\end{exe}

The beneficiary adjunct is not necessarily contiguous with the verb on which it depends, as in (\ref{ex:iZora.GW.tChi.tufsej}) where the genitive phrase \forme{iʑora ɣɯ} `for us, on our behalf'  is separated from the verb \japhug{tʰu}{ask} by a lengthy complement comprising two clauses.

\begin{exe}
\ex \label{ex:iZora.GW.tChi.tufsej}
\gll  iʑora ɣɯ [tɕʰi tu-fse-j tɕe ji-tɯ-ci ɣɤʑu] tu-tɯ-tʰe ɯ-tɯ́-cʰa? \\
\textsc{1pl} \textsc{gen} what \textsc{ipfv}-be.like-\textsc{1pl} \textsc{lnk} \textsc{1pl}.\textsc{poss}-\textsc{indef}.\textsc{poss}-water exist:\textsc{sens} \textsc{ipfv}-2-ask[III] \textsc{qu}-2-can:\textsc{fact} \\
\glt `Can you ask on our behalf how we should do to have water?' (2005tamukatsa, 14)
\end{exe}

Beneficiary genitive phrases can occur as predicates with a copula as \japhug{ɯʑɤɣ}{\textsc{3sg}:\textsc{gen}} in (\ref{ex:WZAG.pjAmaR}).

 \begin{exe}
\ex \label{ex:WZAG.pjAmaR}
\gll   tʰoʁtɤm ka-wum tɕe, ɯʑɤɣ pjɤ-maʁ kɯ, tɕoχtsi rɟɤlpu ɣɯ ku-wum,  \\
taxes \textsc{pfv}:3\fl{}3'-collect \textsc{lnk} \textsc{3sg}:\textsc{gen} \textsc{ifr}.\textsc{ipfv}-not.be \textsc{erg} p.n. king \textsc{gen} \textsc{ipfv}-collect \\
\glt `The taxes that he had collected were not for himself, he was collecting them for the king of Cogtse.' (150901 NAjstsa, 28)
\end{exe}

In this use too, it is alternatively possible to indicate the beneficiary as a possessive prefix on the object, without genitive postposition, as in (\ref{ex:atWci.tArke}).

 \begin{exe}
\ex \label{ex:atWci.tArke}
\gll   χsɤr kʰɯtsa ɯ-ŋgɯ nɯtɕu a-tɯ-ci ci tɤ-rke ma wuma ɲɯ-ɕpaʁ-a \\
gold bowl \textsc{3sg}.\textsc{poss}-inside \textsc{dem}:\textsc{loc} \textsc{1sg}.\textsc{poss}-\textsc{indef}.\textsc{poss}-water a.little \textsc{imp}-put.in[III] \textsc{lnk} really \textsc{sens}-be.thirsty-\textsc{1sg} \\
\glt  `Please pour some water in the golden bowl for me, I am thirsty.' (140428 mu e guniang-zh, 47)
\end{exe}

The genitive is also attested with a noun-verb collocations (§ XXX), like \japhug{ɯ-kɤrnoʁ+mtɕɯr}{feel dizzy}, in which the possessor  of the noun is an experiencer as in (\ref{ex:fsapaR.GW.kWnA}). This example also illustrates the use of the genitive followed by a focus marker, as (\ref{ex:GW.kWnA.WrkW.ri}) above.

\begin{exe}
\ex \label{ex:fsapaR.GW.kWnA}
\gll tɕeri fsapaʁ ɣɯ kɯnɤ ɯ-kɤrnoʁ ɲɯ-mtɕɯr ɲɯ-ŋu \\
\textsc{lnk} animal \textsc{gen} also \textsc{3sg}.\textsc{poss}-head \textsc{sens}-turn \textsc{sens}-be \\
\glt `But animals too can feel dizzy.' (29-tAmtshAzkAkWndo, 71)
\end{exe}


\subsubsection{Other uses} \label{sec:gen.other}
The genitive \forme{ɣɯ} occurs with various types of noun complements which are semantically neither possessive or beneficiaries/recipients. 

Nouns used as prenominal modifiers are in rare cases followed by a genitive postposition before the head noun. If the head noun is an APN, the presence of a third singular possessive prefix \forme{ɯ-} is optional, as shown by examples such as (\ref{ex:χsAr.GW.khWtsa}) and (\ref{ex:ftsoR.kWngWt.WphW}). 

\begin{exe}
\ex \label{ex:χsAr.GW.khWtsa}
\gll  χsɤr ɣɯ, nɤkinɯ, kʰɯtsa ci to-nɯ-ndo. \\
gold \textsc{gen} \textsc{filler} bowl \textsc{indef} \textsc{ifr}-\textsc{auto}-take \\
\glt `He took a golden bowl' (140508 shier ge tiaowu de gongzhu-zh, 158)
\end{exe}

This type of construction is most common in texts translated from Chinese, but does also occur in more spontaneous material as in (\ref{ex:ftsoR.kWngWt.WphW}), with a complex modifier \forme{ftsoʁ kɯngɯt ɯ-pʰɯ} `the price of nine female hybrid yaks'.

\begin{exe}
\ex \label{ex:ftsoR.kWngWt.WphW}
\gll tɕendɤre ɯ-jaʁ nɯtɕu [ftsoʁ kɯngɯt ɯ-pʰɯ] ɣɯ srɯnloʁ pjɤ-k-ɤ-rku-ci \\
\textsc{lnk} \textsc{3sg}.\textsc{poss} \textsc{dem}:\textsc{loc} female.hybrid.yak nine \textsc{3sg}.\textsc{poss}-price \textsc{gen} ring \textsc{ifr}.\textsc{ipfv}-\textsc{evd}-pass-put.in-\textsc{evd} \\
\glt `She had a ring worth nine female hybrid yak in her hand.' (2003gesar, 239)
\end{exe}

In a construction with a prenominal modifier marker in the genitive, even when a possessive prefix is present on the head noun (in particular when it is an IPN), that prefix does not necessarily refer to the modifier. For instance, in (\ref{ex:tWpAlAskAr.GW.nWmgozmArAB}), the third plural possessive prefix \forme{nɯ-} on \forme{nɯ-mgozmɤrɤβ} `their vegetables' refers to the people eating the vegetable, not the the modifier \japhug{tɯxpalɤskɤr}{the whole year} (on whose formation see § \ref{sec:dvandva.coll}) which would require a third singular prefix instead (an option which is also attested with this noun). Alternatively, it is also possible to have an indefinite possessive prefix on the head noun if it is an IPN, as in (\ref{ex:tWxpa.GW.tWGli}). Note that both options are attested in the construction with a prenominal modifier without the genitive, as seen in § \ref{sec:possessive.prefixes.prenominal}.
 
 \begin{exe}
\ex \label{ex:tWpAlAskAr.GW.nWmgozmArAB}
\gll  tɕe nɯnɯ tɯxpalɤskɤr ɣɯ nɯ-mgozmɤrɤβ nɯ nɯ ma pjɤ-me.  \\
\textsc{lnk} \textsc{dem} whole.year \textsc{gen} \textsc{3pl}.\textsc{poss}-vegetable \textsc{dem} \textsc{dem} apart.from \textsc{ifr}.\textsc{ipfv}-not.exist \\
\glt `It was the only vegetable that they had the whole year.' 
\end{exe}

\begin{exe}
\ex \label{ex:tWxpa.GW.tWGli}
\gll  tɯ-xpa ɣɯ tɯ-ɣli nɯ cʰɯ́-wɣ-tɕɤt tú-wɣ-rmbɯ  \\
one-year \textsc{gen} \textsc{indef}.\textsc{poss}-manure \textsc{dem} \textsc{ipfv}:\textsc{downstream}-\textsc{inv}-take.out \textsc{ipfv}-\textsc{inv}-heap \\
\glt `People take out (from the stable) the whole year's manure and heap it up.' (2010-tArAku)
\end{exe}

Some apparently unclassifiable uses of the genitive can be accounted for to some extent by assuming the elision of a head noun.  For instance, in (\ref{ex:iZora.GW}), the phrase \forme{iʑora ɣɯ}, meaning `in our language', can be explained as coming from \forme{iʑora ɣɯ ji-skɤt} `our language' used as a absolutive locative phrase (§ \ref{absolutive.locative}) `in our language', with elision of the head noun. This example does not illustrate a distinct function of the genitive, it is simply a particular case of possessive.

\begin{exe}
\ex \label{ex:iZora.GW}
\gll  <longtoutan> nɯ kupa-skɤt ɕti. tɕe iʑora ɣɯ tɕʰi tu-kɯ-ti ŋu mɤ-xsi. \\
pl.n. \textsc{dem} Chinese-language be.\textsc{affirm}:\textsc{fact} \textsc{lnk} \textsc{1pl} \textsc{gen} what \textsc{ipfv}-\textsc{genr}-say be:\textsc{fact} \textsc{neg}-\textsc{genr}:know \\
\glt `Longtoutan is a Chinese word; I don't know how it is said in our (language).'  (150820 qaprANar, 32)
\end{exe}

The same is true of the use of the genitive with the verb \japhug{mŋɤm}{be paintful} and its causative \japhug{ɕɯmŋɤm}{cause to be paintful}, which take a body part (not the person or animal feeling pain) as their subject and object respectively. In (\ref{ex:aZWG.taCWmNAm}), the genitive first person \japhug{aʑɯɣ}{\textsc{1sg:gen}} is not an oblique argument or even a malefactive adjunct. Rather, its presence implies an elided noun \japhug{a-βri}{my body} (`he caused pain to my body'). It is however likely that sentences like this are the pivot constructions which made possible the reanalysis of possessive genitive phrases as benefactive/malefactive adjuncts.

\begin{exe}
\ex \label{ex:aZWG.taCWmNAm}
\gll aʑɯɣ ta-ɕɯ-mŋɤm, aʑɯɣ a-laχtɕʰa ra ja-nɯ-tsɯm-nɯ \\
\textsc{1sg}:\textsc{gen} \textsc{pfv}:3\fl{}3'-\textsc{caus}-be.paintful \textsc{1sg}:\textsc{gen} \textsc{1sg}.\textsc{poss}-thing \textsc{pl} \textsc{pfv}:3\fl{}3'-\textsc{vert}-take.away-\textsc{pl} \\
\glt `He hurt me and took away my things.' (140426 luozi he qiangdao, 35)
\end{exe}

The genitive can also occur between prenominal relatives (§ XXX) and their head noun. In this construction the head noun generally does not take a possessive prefix. This type of relative is particularly common in story translated from Chinese, where it calques the prenominal relatives in \zh{的} <de>, as in (\ref{ex:makWra.GW.sAtCha}). The same situation has been observed in Khroskyabs (\citealt[640-643]{lai17khroskyabs}).

\begin{exe}
\ex \label{ex:makWra.GW.sAtCha}
\gll [kɯ-ɣɤndʐo ri kɯ-me], [kɯ-sɤ-mtsɯr ri kɯ-me], [kɤ-nɯsɯmɯzdɯɣ ri mɤ-kɯ-ra] ɣɯ sɤtɕʰa nɯtɕu jo-ɕe-ndʑi ɲɯ-ŋu. \\
\textsc{nmlz}:S/A-be.cold also \textsc{nmlz}:S/A-not.exist \textsc{nmlz}:S/A-\textsc{deexp}-be.hungry also \textsc{nmlz}:S/A-not.exist inf-worry also \textsc{neg}-\textsc{nmlz}:S/A-have.to \textsc{gen} place \textsc{dem}:\textsc{loc} \textsc{ifr}-go-\textsc{du} \textsc{sens}-be \\
\glt  `The two of them went to a place where they was cold cold and hunger, and where one did not need to worry.' (140519 mai huochai de xiao nvhai-zh, 182-183)
\end{exe}

However, this type of relative is also attested, though rarer, in non-translated texts, for instance in (\ref{ex:tWxpa.tukWlhoR.GW.sWjno}) with intransitive subject relativization (see § XXX for additional examples).
%kɯki aʑo ɕkom tu-mtshi-a ki ɣɯ ɯ-χpi ci pjɯ-fɕat-a

\begin{exe}
\ex \label{ex:tWxpa.tukWlhoR.GW.sWjno}
\gll  tɕe [tɯ-xpa tu-kɯ-ɬoʁ] ɣɯ sɯjno nɯ ŋu tɕe, \\
\textsc{lnk} one-year \textsc{ipfv}-\textsc{nmlz}:S/A-come.out \textsc{gen} plant \textsc{dem} be:\textsc{fact} \textsc{lnk} \\
\glt  `It is an annual plant.' (18-NGolo, 105)
\end{exe}

Genitival prenominal relative clauses are to be distinguished from relatives as possessors, as in (\ref{ex:tWCGA.kWmNAm.GW.WrJAnNgo}), where the possessum  \japhug{ɯ-rɟɤŋgo}{its radiating pain} is not an argument of the relative \forme{tɯ-ɕɣa kɯ-mŋɤm} `a tooth that hurts'. 

\begin{exe}
\ex \label{ex:tWCGA.kWmNAm.GW.WrJAnNgo}
\gll tɯ-ɕɣa a-tɤ-mŋɤm tɕe tɕe tɤ-rca tɯ-ɣmba, tɯ-ku nɯra tu-mŋɤm ɲɯ-ŋu tɕe,  nɯnɯ ``[tɯ-ɕɣa kɯ-mŋɤm] ɣɯ ɯ-rɟɤŋgo ɣɤʑu" tu-kɯ-ti ŋu. \\
\textsc{genr}.\textsc{poss}-tooth \textsc{irr}-\textsc{pfv}-be.painful \textsc{lnk} \textsc{lnk} \textsc{indef}.\textsc{poss}-following \textsc{genr}.\textsc{poss}-cheek \textsc{genr}.\textsc{poss}-head \textsc{dem}:\textsc{pl} \textsc{ipfv}-be.painful \textsc{sens}-be \textsc{lnk} \textsc{dem} \textsc{indef}.\textsc{poss}-tooth \textsc{nmlz}:S/A-be.painful \textsc{gen} \textsc{3sg}.\textsc{poss}-radiating.pain exist:\textsc{sens} \textsc{ipfv}-\textsc{genr}-say be:\textsc{fact} \\
\glt `When one has a toothache, and that one feels pain in one's cheek or a headache, one says `the toothache has a radiating pain.'' (140516 WrJANgo, 3)
\end{exe}

Adnominal complement clauses (§ XXX) can also take a genitive marker, as in (\ref{ex:mWjnaXtChWG.GW.WtCha}).

\begin{exe}
\ex \label{ex:mWjnaXtChWG.GW.WtCha}
\gll [<donggua> cʰo <qiezi> ni tɕʰi ʑo mɯ́j-naχtɕɯɣ] ɣɯ ɯ-tɕʰa a-jɤ-tɯ-ɣɯt ra \\
gourd \textsc{comit} eggplant \textsc{du} what \textsc{emph} \textsc{neg}:\textsc{sens}-be.the.same \textsc{gen} \textsc{3sg}.\textsc{poss}-information \textsc{irr}-\textsc{pfv}-2-bring have.to:\textsc{fact} \\
\glt `(Go there and come back to) tell me in what way gourd and eggplant differ from each other.' (2010-02-yitian bi yitian-zh, 7)
\end{exe}

Some relative clauses can take possessors marked in the genitive, relating to an argument within the relative, as in  (\ref{ex:stu.WkAnWmga}) and (\ref{ex:slama.ra.GW}). It is debatable whether the genitival phrase belongs to the relative in this type of construction.

\begin{exe}
\ex \label{ex:stu.WkAnWmga}
 \gll tɕe paʁ ɣɯ [stu ɯ-kɤ-nɯmga], iʑora ji-kɤ-nɯmga nɯ ɯ-ɕa ŋu tɕe \\
 \textsc{lnk} pig \textsc{gen} most \textsc{3sg}.\textsc{poss}-\textsc{nmlz}:P-want.from \textsc{1pl} \textsc{1pl}.\textsc{poss}-\textsc{nmlz}:P-want.from \textsc{dem} \textsc{3sg}.\textsc{poss}-meat be:\textsc{fact} \textsc{lnk} \\
\glt  `What is most wanted from pigs, what we want from them is their meat.' (05-paR, 13)
\end{exe}

\begin{exe}
\ex \label{ex:slama.ra.GW}
\gll  slama ra ɣɯ [tʰɯtʰɤci kɯ-fse], nɯ kɤ-rɤ-βzjoz ra ɲɯ-stu mɯ́j-stu nɯ, nɯ-stu ɲɯ-nɤma-nɯ mɯ́j-nɤma-nɯ,  nɯnɯra nɯ-pʰama ra nɯ-ɕki kɯ-rɤ-fɕɤt ɲɯ-ra. \\
student \textsc{pl} \textsc{gen} something \textsc{nmlz}:S/A-be.like \textsc{dem}  \textsc{inf}-\textsc{antipass}-learn \textsc{pl} \textsc{sens}-be.assiduous \textsc{sens}-be.assiduous \textsc{dem} \textsc{3pl}.\textsc{poss}-truth \textsc{sens}-work-\textsc{pl} \textsc{neg:sens}-work-\textsc{pl} \textsc{dem}:\textsc{pl} \textsc{3pl}.\textsc{poss}-parent \textsc{pl} \textsc{3pl}.\textsc{poss}-\textsc{dat} \textsc{genr}:S/P-\textsc{antipass}-tell:\textsc{fact} \textsc{sens}-have.to \\
\glt `One had to tell the parents all kinds of things concerning the students, whether they try hard or not, whether they work seriously or not.' (150901 tshuBdWnskAt, 18-20)
\end{exe}


\subsection{Locative} \label{sec:locative}
 

\subsubsection{Core locative postpositions} \label{sec:core.locative}


Example (\ref{ex:nAmkha.zW.pjAnWlhoRnW}) shows the locative \forme{zɯ} expressing static location (`in their hands') and motion from a place (`from the sky').

\begin{exe}
\ex \label{ex:nAmkha.zW.pjAnWlhoRnW}
\gll tɯmɯkɤrŋi ɯ-me kɯɕnɯz nɯ nɯ-jaʁ zɯ, nɤkinɯ, <shanzi> kɯ-mpɕɯ\redp{}mpɕɤr ʑo, qale ɯ-sɤ-lɤt nɯ pjɤ-k-ɤsɯ-ndo-nɯ-ci tɕe,  tɯmɯ nɤmkʰa zɯ pjɤ-nɯ-ɬoʁ-nɯ. \\
heaven \textsc{3sg}.\textsc{poss}-daughter seven \textsc{dem} \textsc{3pl}.\textsc{poss}-hand \textsc{loc} \textsc{filler} fan \textsc{nmlz}:S/A-\textsc{emph}\redp{}be.beautiful \textsc{emph} wind \textsc{3sg}.\textsc{poss}-\textsc{nmlz}:\textsc{oblique}-throw \textsc{dem} \textsc{ifr}.\textsc{ipfv}-\textsc{evd}-\textsc{prog}-hold-\textsc{pl}-\textsc{evd} \textsc{lnk} sky sky \textsc{loc} \textsc{ifr}:\textsc{down}-\textsc{auto}-come.out-\textsc{pl} \\
\glt `The seven daughters of heaven, holding beautiful fans in their hands, came down from the sky.' (150828 niulang-zh, 48)
\end{exe}

%tɕeri nɯtɕu nɯ tɕi kɯnɤ tɕiʑo kɤndʑɯβzaŋsa nɯ mɯ-pɯ-nɯ-qia-tɕi 

\begin{exe}
\ex \label{ex:tWxsoz.ri}
\gll tɯ-xsoz ri tɕe jo-ɣi ri, ɯ-βri tɤʑri ɣɤʑu, ɲɯ-ɤci. \\
one-morning \textsc{loc} \textsc{lnk} \textsc{ifr}-come \textsc{lnk} \textsc{3sg}.\textsc{poss}-body dew exist:\textsc{sens} \textsc{sens}-be.wet \\
\glt `One morning, (our hen) came (back from the forest, where it was laying eggs) and its body was wet from the dew.' (150819 kumpGa, 21-22)
\end{exe}



\subsubsection{Approximate locative}

%pa jɯl pɕoʁ nɤki, tɯji ɯ-rkɯ nɯra maka me.
%rɯŋgu kɯnɤ lɤchu koŋla ʑo kɯ-ɣɤndʐo mɤɕtʂa thi nɯra me
%15-babW, 116-7 (6:20)
\subsubsection{Traces of the locative suffix \forme{*-j}} \label{sec:locative.j}
Situ has a locative suffix \forme{-j}, also used in the possessive construction (\citealt[325-330]{linxr93jiarongen}), which has disappeared in Japhug, though a few traces remain.


The form \forme{tɕe}, which is mainly used as a linker (§ XXX) and also occurs as a topic marker (§ \ref{sec:tCe.topic}), probably originates from the combination of the locative postposition \forme{tɕu} and the locative suffix \forme{*-j}, with vowel merger at a stage preceding the sound change \forme{*o} $\rightarrow$ \forme{u}  (\forme{*tɕo-j} $\rightarrow$ \forme{tɕe}; see § XXX for a discussion of these sound changes).

The former locative meaning of \forme{tɕe} is still indirectly visible in constructions such as \forme{ɯ-sɯm tɕe} `in his opinion, in his mind' as in (\ref{ex:asWm.tCe}) and (\ref{ex:WsWm.tCe}), where the \forme{tɕe} is neither a linker nor a topic marker (it cannot be replaced here by \forme{nɯ} § \ref{sec:nW.topic}, for instance). 

\begin{exe}
\ex \label{ex:asWm.tCe}
\gll aʑo a-sɯm tɕe, nɯ-ʁrɯ ʑo ɣɤʑu ɕti tɕe \\
\textsc{1sg} \textsc{1sg}.\textsc{poss}-mind \textsc{lnk} \textsc{3pl}.\textsc{poss}-horn \textsc{emph} exist:\textsc{sens} be.\textsc{affirm}:\textsc{fact} \textsc{lnk} \\
\glt `In my opinion, (since) they have horns, (they should be able to fight the predators off).' (20-RmbroN, 64)
\end{exe}

In (\ref{ex:WsWm.tCe}), the second \forme{tɕe} may be analyzed as a topic marker (§ \ref{sec:tCe.topic}).

\begin{exe}
\ex \label{ex:WsWm.tCe}
\gll 
tɕe ɯʑo ɯ-sɯm tɕe tɕe tu-tsɯm tɕe tɕendɤre iʑora ji-sɤtɕha ra lonba ʑo tɯ-ci ɲɯ-sɯ-ɤβze to-ʁmɯɣ. \\
\textsc{lnk} \textsc{3sg} \textsc{3sg}.\textsc{poss}-mind \textsc{lnk} \textsc{lnk} \textsc{ipfv}:\textsc{up}-take \textsc{lnk} \textsc{lnk} \textsc{1pl} \textsc{1pl}.\textsc{poss}-place \textsc{pl} all \textsc{emph} \textsc{indef}.\textsc{poss}-water \textsc{ipfv}-\textsc{caus}-become[III] \textsc{ifr}-have.the.intention \\
\glt `In his mind, (the snake) wanted to take the water upwards and transform our whole area into water.' (150820 qaprANar, 20)
\end{exe}

Another trace of the locative suffix \forme{*-j} is found in the linker \japhug{qʰe}{then} (§ XXX) and the time ordinal \japhug{qʰuj}{this afternoon} (\ref{sec:time.ordinals}), combining the relator noun \japhug{ɯ-qʰu}{after} (as in \japhug{saχsɯ ɯ-qʰu}{after lunch, afternoon}) with the coda \forme{*-j}, in the former with vowel fusion (an earlier lexicalization), and the latter without fusion. The form  \japhug{qʰuj}{this afternoon}  shows that the suffix \forme{*-j} was still productive in Japhug after the sound change \forme{*o} $\rightarrow$ \forme{u} took place.



\subsection{Comitative} \label{sec:comitative} 
Postpositional phrases with the comitative postposition \japhug{cʰo}{and, with} and its variants \forme{cʰondɤre} and \forme{cʰonɤ} (comprising the linkers \forme{nɤ} and \forme{ndɤre}, see § XXX) are selected as oblique arguments by a handful of verbs, including \japhug{naχtɕɯɣ}{be the same} (§ \ref{sec:identity.modifier}) and \japhug{amɯmi}{be in good terms with}, as shown in (\ref{ex:cho.kWnaXtCWG}).

\begin{exe}
\ex \label{ex:cho.kWnaXtCWG}
\gll [ɯʑo cʰo] kɯ-naχtɕɯɣ [sɯjno, xɕaj ma mɤ-kɯ-ndza nɯra cʰonɤ] amɯmi-nɯ tɕe, \\
\textsc{3sg} \textsc{comit} \textsc{nmlz}:S/A-be.the.same vegetables grass apart.from \textsc{neg}-\textsc{nmlz}:S/A-eat \textsc{dem}:\textsc{pl} \textsc{comit} be.in.good.terms:\textsc{fact}-\textsc{pl} \textsc{lnk} \\
\glt `(The rabbit) is in good terms with (the animals) which eat only grass and vegetables like him.' (04-qala2, 8)
\end{exe}

Postpositional phrases in \forme{cʰo} are oblique arguments in the sense that they are relativized using the oblique participle (§ XXX). However, verbs that select \forme{cʰo} phrases index not only the intransitive subject proper, but the sum of the subject and the \forme{cʰo} phrase, which can be in the dual as in (\ref{ex:cho.YWnaXtCWGndZi}) (the white birch and the red birch) or in the plural (\ref{ex:cho.kWnaXtCWG}) (the rabbit and the other animals). 

\begin{exe}
\ex \label{ex:cho.YWnaXtCWGndZi}
\gll tɕe ɯ-rqʰu nɯ ɣɯrni laʁma ɯ-ŋgɯ nɯ [sɤjku cʰo] ɲɯ-naχtɕɯɣ-ndʑi ri\\
\textsc{lnk} \textsc{3sg}.\textsc{poss}-bark \textsc{dem} be.red:\textsc{fact} apart.from.the.fact \textsc{3sg}.\textsc{poss}-inside \textsc{dem} birch \textsc{comit} \textsc{sens}-be.the.same-\textsc{du} \textsc{lnk} \\
\glt `Apart from the fact that its bark is red, it is identical in the inside with the birch.' (06-mbrAj, 13)
\end{exe}

The verb \japhug{naχtɕɯɣ}{be the same} with a \forme{cʰo} phrase can be used in an equative construction (see § XXX).

 Apart from the function presented above, \japhug{cʰo}{and, with} is commonly used to link together two nouns inside a single noun phrase, as in (\ref{ex:awW.cho.aRi}). In this case too, the main verb of the clause indexes the whole noun phrase, comprising the sum of referents designated by the nouns linked by \forme{cʰo}.

\begin{exe}
\ex \label{ex:awW.cho.aRi}
\gll a-wɯ cʰo a-ʁi ni cʰɯ-ɣi-ndʑi ra ma ʑɤni-sti kɤ-rɤʑi mɤ-cʰa-ndʑi tɕe, \\
\textsc{1sg}.\textsc{poss}-grand.father \textsc{comit} \textsc{1sg}.\textsc{poss}-younger.sibling \textsc{du} \textsc{ipfv}:\textsc{downstream}-come-\textsc{du} have.to:\textsc{fact} \textsc{lnk} \textsc{3du}-alone \textsc{inf}-stay \textsc{neg}-can:\textsc{fact}-\textsc{du} \textsc{lnk} \\ 
\glt `My grandfather and my younger brother have to come, they cannot stay by themselves.' (2011-05-nyima, 209)
\end{exe}

The marker \forme{cʰo} can also link verb phrases and even entire clauses (see § XXX and \citealt[313]{jacques14linking}).

Given the apparently equal status of the two linked nouns in(\ref{ex:awW.cho.aRi}), in particular with regard to indexation, it is legitimate to wonder whether analyzing it as a postposition makes more sense than considering it to be a coordinator (§ \ref{sec:coordinator}). There are two arguments supporting the postposition analysis. First, \forme{cʰo} necessarily follows a noun phrase (or at the very least a demonstrative pronoun), but does not require to be followed by another noun as in (\ref{ex:cho.YWnaXtCWGndZi}) above. Second, phrases comprising \forme{cʰo} and the preceding noun are relativized using the oblique participle (see § XXX).

A \forme{cʰo} phrase can be followed by the associative plural marker \forme{ra} (§ \ref{sec:number.determiners}) as in (\ref{ex:cho.ra.kW}) to mean `et caetera', and the whole phrase can taken case marking such as ergative.

\begin{exe}
\ex \label{ex:cho.ra.kW}
\gll tɕeri ɯʑo ndɤre, qajdo cʰo ra kɯ ndɤ tú-wɣ-ndza ɕti \\
but \textsc{3sg} \textsc{advers} crow \textsc{comit} \textsc{pl} \textsc{erg} \textsc{advers} \textsc{ipfv}-\textsc{inv}-eat be.\textsc{affirm}:\textsc{fact} \\
\glt `But it is eaten by crows and other (animals).' (26-NalitCaRmbWm, 140)
\end{exe}

A postpositional comitative phrase can also serve as a dual or plural possessor, as if from a complex noun phrase `X \forme{cʰo} Y' with elided Y element, as in (\ref{ex:cho.ndZime}).\footnote{Note that (\ref{ex:cho.ndZime}) does not mean `There is his wife and their daughter' (dual indexation would be expected on the verb).}

\begin{exe}
\ex \label{ex:cho.ndZime}
\gll tɕe ɯ-rʑaβ cʰo ndʑi-me ci tu tɕe, \\
\textsc{lnk} \textsc{3sg}.\textsc{poss}-wife \textsc{comit} \textsc{3du}.\textsc{poss}-daughter one exist:\textsc{fact} \textsc{lnk} \\
\glt `He and his wife have a daughter.' (14-tApitaRi, 313)
\end{exe}

\subsection{Standard marker} \label{sec:comparative} 
Japhug has several postpositions that are mainly used to mark the standard in the comparative construction. The most common one is \japhug{sɤz}{compared with}, but the variants \forme{staʁ}, \forme{sɤznɤ}, \forme{staʁnɤ}, \forme{sɯstaʁ} and \forme{sɯχta} are also attested. Their relative frequency appears to be speaker-dependent, and no meaningful difference could be detected between them. 

In the comparative construction (§ XXX), the comparee is the intransitive subject of the main verb (the parameter, generally an adjectival stative verb) and is indexed on the verb. The comparee is either in the absolutive or in the ergative (§ \ref{sec:comparee.kW}). The standard is necessarily marked by one of the postpositions listed above, and cannot be indexed on the main verb. Neither the standard not the comparee are required to be overt. An adjectival stative verb with a standard postpositional phrase as in  (\ref{ex:sAznA.YWwxti})  is a well-formed comparative construction. Examples like (\ref{ex:aZo.YWwxti}) with overt comparee and standard are rarer.

\begin{exe}
\ex \label{ex:sAznA.YWwxti}
\gll  qandʑɣi sɤznɤ ɲɯ-wxti, qaliaʁ sɤznɤ ɲɯ-xtɕi \\
falcon \textsc{comp} \textsc{sens}-be.big eagle \textsc{comp} \textsc{sens}-be.small \\
\glt `It is bigger than a falcon, and smaller than an eagle.' (2011-08-kuwu, 40-41)
\end{exe}

\begin{exe}
\ex \label{ex:aZo.YWwxti}
\gll ɯʑo nɯ aʑo sɤz tɯ-xpa wxti  \\
\textsc{3sg} \textsc{dem} \textsc{1sg} \textsc{comp} one-year be.big:\textsc{fact} \\
\glt `She is one year older than me.' (12-BzaNsa, 94)
\end{exe}

The standard marker \forme{sɤz} (and its variants) also occurs in a  construction expressing progressive increase throughout the time, where a time counted noun like \japhug{tɯ-sŋi}{one day} or \japhug{tɯ-xpa}{one year} is followed by the standard marker and then repeated, as \forme{tɯ-xpa sɤz tɯ-xpa} `more each year' in (\ref{ex:tWxpa.sAz.tWxpa}). This construction, although attested in non-translated texts, is more common in texts from Chinese, where it calques the construction \ch{一年比一年}{yīnián bǐ yīnián}{more each year}. The more idiomatic Japhug construction to express the same meaning is through partial reduplication of the first syllable of the main verb (see § XXX).
 
 \begin{exe}
 \ex \label{ex:tWxpa.sAz.tWxpa}
 \gll nɯ-jwaʁ nɯ, [...] tɯ-xpa sɤz tɯ-xpa lu-dɤn ŋu ma \\
 \textsc{3pl}.\textsc{poss}-leaf \textsc{dem} { } one-year \textsc{comp} one-year \textsc{ipfv}-be.many be:\textsc{fact} \textsc{lnk} \\
\glt  `There are more needles (leaves) each year.' (08-saCW, 17)
\end{exe}
 
 The standard markers can also be used with subordinate clauses (§ XXX). The standard marker with the distal demonstrative \forme{nɯ sɤznɤ} has the meaning `rather than that, could ... as well' as in (\ref{ex:nW.sAznA.arca}).  
 
 \begin{exe}
 \ex \label{ex:nW.sAznA.arca}
 \gll  nɤ-mu kɤ-fsraŋ mɤ-tɯ-cʰa tɕe, nɯ sɤznɤ, a-rca jɤ-ɣi tɕe, a-rca, nɤki, laχɕi pɯ-βzjoz \\
 \textsc{2sg}.\textsc{poss}-mother \textsc{inf}-protect \textsc{neg}-2-can:\textsc{fact} \textsc{lnk} \textsc{dem} \textsc{comp} \textsc{1sg}.\textsc{poss}-following \textsc{imp}-come \textsc{lnk} \textsc{1sg}.\textsc{poss}-following \textsc{filler} trade \textsc{imp}-learn \\
\glt `You cannot save your mother, rather than that, come with me to learn  some abilities.' (150826 baoliandeng-zh, 142-143)
\end{exe}

This phrase can also be used as a scalar marker `even' with scope on the following clause, as in (\ref{ex:nW.sAznA.chaa}). See § XXX for a more detailed discussion. % nɯ sɤznɤ ... ʁo alala ri
 

 \begin{exe}
 \ex \label{ex:nW.sAznA.chaa}
 \gll ki kɤ-rtsi kɯ-tu me nɤ, aʑo nɯ sɤznɤ, nɤkinɯ, kʰa kɯ-qanɯ\redp{}nɯ ɯ-ŋgɯ zɯ, nɤkinɯ, tɯ-ɕpɤβ kɯβde-rzɯɣ tɤ-kɤ-lɤt nɯnɯ ku-sɤlɤɣi-a cʰa-a ɕti nɤ! \\
 \textsc{dem}.\textsc{prox} \textsc{inf}-count \textsc{nmlz}:S/A-exist not.exist:\textsc{fact} \textsc{sfp} \textsc{1sg} \textsc{dem} \textsc{comp} \textsc{filler} house   \textsc{nmlz}:S/A-\textsc{emph}\redp{}be.dark \textsc{3sg}.\textsc{poss}-inside \textsc{loc}  \textsc{filler} \textsc{indef}.\textsc{poss}-corpse four-section \textsc{pfv}-\textsc{nmlz}:P-throw \textsc{dem} \textsc{ipfv}-combine-\textsc{1sg} can:\textsc{fact}-\textsc{1sg} be.\textsc{affirm}:\textsc{fact} \textsc{sfp}  \\
\glt  `(What you ask) is nothing, I am even able to put together a body cut into four sections in a dark house.' (140512 alibaba-zh, 170)
\end{exe}

\subsection{Exceptive} \label{sec:exceptive} %\japhug{ma}{apart from} laʁma mɯma
The exceptive postposition \japhug{ma}{apart from} and its reduplicated variant \forme{mɯma} are not selected by any verb, and only used in adjunct postpositional phrases as in (\ref{ex:kWm.ci.mWma}).

 \begin{exe}
 \ex \label{ex:kWm.ci.mWma}
 \gll kɯm ci mɯma nɯnɯ tɕe znde ʁɟa ʑo ɕti \\
 door one apart.from \textsc{dem} \textsc{lnk} wall completely \textsc{emph} be.\textsc{affirm}:\textsc{fact} \\
 \glt `Apart from one door, there are walls everywhere.' (2011-11-kha, 40)
\end{exe}

The exceptive \japhug{ma}{apart from} is used in particular in restrictive focus constructions (§ \ref{sec:restrictive.focus}).

When the scope of the restrictive construction is on an entire clause rather than a single noun phrase, the clause is followed by the linker \forme{ma} (homophonous with the exceptive) and an exceptive phrase limited to the demonstrative pronoun \forme{nɯ} (here in resumptive use, coreferent with the entire preceding clause) and the postposition \forme{ma}, as in (\ref{ex:ra.ma.nW.ma}). The first \forme{ma} in this construction is not to be analyzed as the postposition: while it is possible to reduplicated the second one as in \forme{ma nɯ mɯma} (example \ref{ex:mAspea.ma.nW.ma}), reduplication of the first \forme{ma} is not attested.

 \begin{exe}
 \ex \label{ex:ra.ma.nW.ma}
 \gll [aʑɯɣ ɯ-ɕa ra] ma nɯ ma kɯ-ra me \\
 \textsc{1sg}.\textsc{gen} \textsc{3sg}.\textsc{poss}-meat have.to:fact \textsc{lnk} \textsc{dem} apart.from \textsc{nmlz}:S/A-have.to not.exist:\textsc{fact} \\
 \glt `I want its meat, and nothing else.' (02-deluge2012, 14)
\end{exe}

 \begin{exe}
 \ex \label{ex:mAspea.ma.nW.ma}
 \gll tɤ-pɤtso kɯ-ɣɤwu ʑo kɤ-nɯɕpɯz mɤ-spe-a ma nɯ mɯma spe-a \\
 \textsc{indef}.\textsc{poss}-child \textsc{nmlz}:S/A-cry \textsc{emph} \textsc{inf}-imitate \textsc{neg}-be.able[III]:\textsc{fact}-\textsc{1sg} \textsc{lnk} \textsc{dem} apart.from be.able[III]:\textsc{fact}-\textsc{1sg}  \\
\glt  `I cannot imitate a baby crying, but apart form that I can imitate (all animal sounds).' (27-kikakCi, 143)
\end{exe}

\subsection{Terminative} \label{sec:terminative}  

The postposition \japhug{mɤɕtʂa}{until} is used after noun phrases to indicate temporal (\ref{ex:RnWpArme.mACtsxa}) or locative (\ref{ex:akW.mACtsxa})  limit. It can be used in opposition with the egressive postpositions (see \ref{ex:tWmAmke.mACtsxa} in § \ref{sec:egressive}).

\begin{exe}
\ex \label{ex:RnWpArme.mACtsxa}
 \gll tɤ-pɤtso kɯ-dɤn nɯra tɕe, tɯ-pɤrme, ʁnɯ-pɤrme jamar mɤɕtʂa tɯ-nɯ ku-tsʰi-nɯ. \\
 \textsc{indef}.\textsc{poss}-child \textsc{nmlz}:S/A-be.many \textsc{dem}:\textsc{pl} \textsc{lnk} one-year.old two-year.old about until \textsc{indef}.\textsc{poss}-breast \textsc{ipfv}-drink-\textsc{pl} \\
 \glt `In (families where) children are many, (mothers) breastfeed (the children) until (they are) one or two years old.' (140426 tApAtso kAnWBdaR, 13)
\end{exe}

\begin{exe}
\ex \label{ex:akW.mACtsxa}
 \gll akɯ mɤɕtʂa ɣɯ-ku-ta-lɤt \\ 
east until \textsc{cisloc}-\textsc{ipfv}:\textsc{east}-1\fl{}2-release \\
\glt  `I come with you (see you off) until the (land of the) east.' (28-smAnmi, 220)
\end{exe}

In combination with the demonstrative \japhug{nɯ}{that}, \japhug{mɤɕtʂa}{until} means `otherwise', as in (\ref{ex:nW.mACtsxa}) (see also § XXX).

\begin{exe}
\ex \label{ex:nW.mACtsxa}
 \gll   kɤ-sɤŋo ʁɟa qʰe,  nɯ-mtɕʰi kɤ-χpjɤt ʁɟa kɯ kú-wɣ-spa ɕti.  nɯ mɤɕtʂa mɤ-kʰɯ. \\
 \textsc{inf}-hear completely \textsc{lnk} \textsc{3pl}.\textsc{poss}-mouth \textsc{inf}-observe completely \textsc{erg} \textsc{ipfv}-\textsc{inv}-be.able be.\textsc{affirm}:\textsc{fact} \textsc{dem} until \textsc{neg}-be.possible:\textsc{fact} \\
\glt `(In order to learn the Tshobdun language, since it has no writing system), one has no choice but to listen and observe people's mouth to learn it, otherwise it is not possible.' (150901 tshuBdWnskAt, 41-44)
\end{exe}

The postposition \japhug{mɤɕtʂa}{until} also occurs with subordinate temporal clauses, as shown in § XXX.

\subsection{Egressive} \label{sec:egressive}  
There are six egressive postpositions in Japhug, which are built by combining the root \forme{ɕaŋ-} with either the root of locative relator nouns (§ \ref{sec:relator.location}) or locational adverbs (§ XXX) as shown in Table \ref{tab:egressive}. There is a one-to-one relationship between the orientations of these postpositions and the six definite orientations found in verb morphology (§ XXX).

The egressive postpositions are mainly used with noun phrases of location expressing a length (\ref{ex:CaNtaR.mAzri}) or a reference point marking a limit (\ref{ex:praRwW.CaNdi}). However, \japhug{ɕaŋtaʁ}{up from} and \japhug{ɕaŋpa}{down from} can also follow noun phrases referring to time reference or durations, as in (\ref{ex:kWmNArZaR}) or more generally any quantity (\ref{ex:XsWm.CaNtaR}). No examples of these postpositions following finite subordinate clauses have been found.

\begin{exe}
\ex \label{ex:CaNtaR.mAzri}
 \gll  tɯ-tɣa ɕaŋtaʁ mɤ-zri. \\
 one-span up.from \textsc{neg}-be.long:\textsc{fact} \\
 \glt `It is at most one handspan long.' (28-tshAwAre, 51)
 \end{exe}
 
\begin{exe}
\ex \label{ex:praRwW.CaNdi}
 \gll ma kɯtɕɯmke nɯnɯtɕu, akɯ ku-ru tɕe, praʁwɯ ɕaŋdi sɤ-mto, 
andi tɕe tɕe, ɕɯfco ɕaŋkɯ nɯ sɤ-mto tɕe, \\
\textsc{lnk} pl.n. \textsc{dem}:\textsc{loc} east \textsc{ipfv}:\textsc{east}-look.at \textsc{lnk} pl.n. west.from \textsc{deexp}-see:\textsc{fact} west \textsc{lnk} \textsc{lnk} pl.n. east.from \textsc{dem} \textsc{deexp}-see:\textsc{fact} \textsc{lnk} \\
\glt `In Kuchumke, looking towards the east, (the areas) to the west of Praqwu are visible, and in the west, (the areas) to the east of Shyufkyo are visible.' (150904 tshAcim, 30)
\end{exe}

\begin{exe}
\ex \label{ex:kWmNArZaR}
 \gll kɯmŋɤsqɤ-rʑaʁ ɕaŋtaʁ cʰɯ-mdɯ-nɯ mɤ-ŋgrɤl tu-ti-nɯ ɲɯ-ŋu.  \\
 fifty-day up.from \textsc{ipfv}-live.up.to-\textsc{pl} \textsc{neg}-be.usually.the.case:\textsc{fact} \textsc{ipfv}-say-\textsc{pl} \textsc{sens}-be \\
\glt `They cannot live more than fifty days, it is said.' (26-GZo, 41)
\end{exe}

\begin{exe}
\ex \label{ex:XsWm.CaNtaR}
 \gll ɯ-pɯ nɯnɯ χsɯm ɕaŋtaʁ tu mɯ́j-ŋgrɤl \\
\textsc{3sg}.\textsc{poss}-young \textsc{dem} three up.from exist \textsc{neg}:\textsc{sens}-be.usually.the.case \\
\glt  `It does not usually have more than three offsprings.' (2011-08-kuwu, 14)
\end{exe}

The postposition \japhug{ɕaŋtaʁ}{up from}  is by far more common than all the other ones, and is often combined with a negative predicate to mean `at most', as shown by (\ref{ex:CaNtaR.mAzri}), (\ref{ex:kWmNArZaR}) and (\ref{ex:XsWm.CaNtaR}) above and (\ref{ex:tWfsu.CaNtaR}) below. In addition to the locational, temporal and quantitative meanings presented above, it can be used in a more abstract and qualitative sense, as in (\ref{ex:BGAru.ci.CaNtaR}).

\begin{exe}
\ex \label{ex:tWfsu.CaNtaR}
 \gll tɯrme tɯ-fsu ɕaŋtaʁ tu-mbro mɤ-cʰa. \\
people \textsc{genr}.\textsc{poss}-same.size up.from \textsc{ipfv}:\textsc{up}-be.high \textsc{neg}-can:\textsc{fact} \\
\glt `It grows at most as tall as a person.'(11-qarGW, 29)
\end{exe}

 \begin{exe}
\ex \label{ex:BGAru.ci.CaNtaR}
 \gll  wortɕʰi ʑo βɣɤru ci ɕaŋtaʁ ʑo tɤ-sɯ-ɤwɯwum-nɯ tɕe, \\
 please \textsc{emph} miller \textsc{indef} up.from \textsc{emph} \textsc{ifr}-\textsc{caus}-\textsc{recip}:gather-\textsc{pl} \textsc{lnk} \\
\glt `Please gather (everybody) up from the miller (the lowliest of servants).' (2003kandZislama, 174)
\end{exe}

\begin{table}
\caption{Egressive postpositions} \label{tab:egressive} \centering
\begin{tabular}{llllll}
\lsptoprule
Posposition & Relator noun & Locational adverb\\
\midrule
\japhug{ɕaŋtaʁ}{up from} & \japhug{ɯ-taʁ}{up, top}& \\
\japhug{ɕaŋpa}{down from} & \japhug{ɯ-pa}{down, bottom}& \\
\japhug{ɕaŋlo}{upstream from} & & \japhug{alo}{upstream} \\
\japhug{ɕaŋtʰi}{downstream from} & & \japhug{atʰi}{upstream} \\
\japhug{ɕaŋkɯ}{east from} & & \japhug{akɯ}{east} \\
\japhug{ɕaŋdi}{west from} & & \japhug{andi}{west} \\
\lspbottomrule
\end{tabular}
\end{table}

The egressive postpositions can be used in contrast with the terminative \japhug{mɤɕtʂa}{until}, as in (\ref{ex:tWmAmke.mACtsxa}).

\begin{exe}
\ex \label{ex:tWmAmke.mACtsxa}
 \gll tɯ-mke ɕaŋpa tɕe tɕe ki tɯ-mɤmke mɤɕtʂa kɯ-zɣɯt kɯ-rɲɟi pjɯ-ŋu ra.  \\
\textsc{indef}.\textsc{poss}-neck down.from \textsc{lnk} \textsc{lnk} \textsc{dem}.\textsc{prox} \textsc{indef}.\textsc{poss}-ankle until \textsc{nmlz}:S/A-reach  \textsc{nmlz}:S/A-be.long \textsc{ipfv}-be have.to:\textsc{fact} \\  
\glt  `(Tibetan clothes) have to be long (enough) so as to reach the ankle down from the neck.' 
\end{exe}

As other postpositional phrases, egressive phrases followed by demonstratives (§ \ref{sec:demonstrative.determiners}) mean `the person(s)/thing(s) from X', as in (\ref{ex:TshuBdWn.CaNlo}).

 \begin{exe}
\ex \label{ex:TshuBdWn.CaNlo}
 \gll iʑora kɯ, nɤki, tsʰuβdɯn ɕaŋlo nɯra `stɤtpa-pɯ' tu-ti-j ŋu. \\
 \textsc{1pl} \textsc{erg} \textsc{filler} pl.n. upstream.from \textsc{dem}:\textsc{pl} pl.n.-person \textsc{ipfv}-say-\textsc{1pl} be:\textsc{fact} \\
\glt `We call the people (who live) in Tshobdun and further upstream `Stotpa'.' (23-tCAphW, 14)
\end{exe}

The postposition \japhug{ɕaŋlo}{upstream from} is homophonous with, and historically related to, the noun \japhug{ɕaŋlo}{seating place for old people and ladies in the kitchen} (§ XXX).
 
 \subsection{Other temporal postpositions} \label{sec:temporal.postpositions}
%    \japhug{ʁaz}{xxxxx}
Apart from the locative, terminative and egressive postpositions, a certain number of specifically temporal postpositions are found in Japhug, including \japhug{ɕɯŋgɯ}{before},  \japhug{pɕɯntɕɤt}{since}, \japhug{jɤz}{when} and \japhug{ɕaŋpɕi}{from ... on}. All can be used with noun phrases and subordinate clauses; the latter use is studied in the section on temporal clauses (§ XXX).

The postposition  \japhug{ɕɯŋgɯ}{before} is an ancient compound containing as first element the \textit{status constructus} of a root cognate to Tangut \tangut{𗪘}{2104}{śji}{1.10} `formerly, before' and the relator noun \japhug{ɯ-ŋgɯ}{inside} (§ \ref{sec:relator.location}) as second element. It contrasts with  \japhug{ɯ-qʰu}{after} (\ref{sec:relator.temporal}), and can follow a noun phrase referring to a point in time, as in the common expression \japhug{saχsɯ ɕɯŋgɯ}{before lunch} (with the noun  \japhug{saχsɯ}{lunch}), or a duration, as in (\ref{ex:sqamNusNi.CWNgW}) and (\ref{ex:XsArZaR.CWNgW}). The latter example shows that \japhug{ɕɯŋgɯ}{before} can be used to refer to events occurring \textit{after} the current temporal point of reference (in \ref{ex:XsArZaR.CWNgW}, before three days from the present in the story).

\begin{exe}
\ex \label{ex:sqamNusNi.CWNgW}
 \gll tɤte sqamŋu-sŋi ɕɯŋgɯ nɯtɕu tɕe, nɤki, nɯ-rlaʁ tɕe nɯ-me pɯ-ŋu ɲɯ-ŋu, tɕʰeme nɯ, \\
 that.is fifteen-day before \textsc{dem}.\textsc{loc} \textsc{lnk} \textsc{filler} \textsc{pfv}-disappear \textsc{lnk} \textsc{pfv}-not.exist \textsc{pst}.\textsc{ipfv}-be \textsc{sens}-be girl \textsc{dem} \\
 \glt `That is, fifteen days before, she had disappeared, that girl.' (tWxtsa, 15)
\end{exe}


\begin{exe}
\ex \label{ex:XsArZaR.CWNgW}
 \gll χsɯ-sŋi χsɤ-rʑaʁ mɤɕtʂa a-mɤ-tɤ-tɯ-rɤru ra ma tɕe mɤ-pʰɤn nɯra to-ti. matɕi tɕetʰa χsɯ-sŋi χsɤ-rʑaʁ ɕɯŋgɯ ɯʑo ɲɯ-pʰɣo pjɤ-ra lo \\
 three-day  three-night until \textsc{irr}-\textsc{neg}-\textsc{pfv}-2-get.up have.to:\textsc{fact} \textsc{lnk} \textsc{lnk}  \textsc{neg}-be.efficient:\textsc{fact} \textsc{dem}:\textsc{pl} \textsc{ifr}-say because later   three-day  three-night before \textsc{3sg} \textsc{ipfv}-flee \textsc{ifr}.\textsc{ipfv}-have.to \textsc{sfp} \\
 \glt `(The rabbit) said `Don't get up until three days and three nights (have passed)', because (the rabbit was buying time) and had to flee before (the end of) these three days and nights.' (140427 qala cho kWrtsag, 30-31)
\end{exe}

The postposition   \japhug{ɕɯŋgɯ}{before} however most commonly occurs with subordinate clauses, and requires a finite verb in the imperfective form (§ XXX). It is attested following personal pronouns, as in (\ref{ex:aʑo.CWNgW}),  \japhug{ɕɯŋgɯ}{before} refers to an action concerning the referent of the pronoun, whose nature can be determined from the context, as if the main verb of a subordinate clause had been elided.

\begin{exe}
\ex \label{ex:aʑo.CWNgW}
 \gll  aʑo ɕɯŋgɯ a-pi ra atu rɤʑi-nɯ tɕe, nɯnɯra ɣɯ nɯ-rmi tɤ-z-mɤke qʰe, \\
 \textsc{1sg} before \textsc{1sg}.\textsc{poss}-elder.sibling \textsc{pl} up.there stay:\textsc{fact}-\textsc{pl} \textsc{lnk} \textsc{dem}:\textsc{pl} \textsc{gen} \textsc{3pl}.\textsc{poss}-name \textsc{imp}-\textsc{caus}-be.first[III] \textsc{lnk} \\
 \glt `Before (you choose a name for) me, my elder brothers up there, (choose) their names first.' (Gesar, 124)
\end{exe}


The common adverb \japhug{kɯɕɯŋgɯ}{in former times} comes from the combination of the \textit{status constructus} of the proximal demonstrative \japhug{ki}{this} (§ \ref{sec:demonstrative.pronouns}) with the postposition \japhug{ɕɯŋgɯ}{before}. The phrases \forme{ki ɕɯŋgɯ} `before this' \forme{nɯ ɕɯŋgɯ} `before that' with the demonstratives \japhug{ki}{this} and  \japhug{nɯ}{that} are also attested.

The postposition \japhug{pɕɯntɕɤt}{since} (from \tibet{ཕྱིན་ཆད་}{pʰʲin.tɕʰad}{thereafter}) can follow a date (\ref{ex:69nian.pCintCAt}) or a subordinate clause (see § XXX), and is most often used with the relator noun \japhug{ɯ-qʰu}{after} to mean `from that time on' as in (\ref{ex:nW.Wqhu.pCintCAt}). Another postposition,  \japhug{ɕaŋpɕi}{from ... on} (combining the \forme{ɕaŋ-} element found in egressive postpositions § \ref{sec:egressive}  with \forme{-pɕi} from Tibetan \tibet{ཕྱི་}{pʰʲi}{later}) can be used like \japhug{pɕɯntɕɤt}{since} following  \japhug{ɯ-qʰu}{after} as in (\ref{ex:Wqhu.CaNpCi}), or after temporal clauses (§ XXX).

 \begin{exe}
\ex \label{ex:69nian.pCintCAt}
 \gll tɕe tʰam kɯβdesqi ɯ-ro to-pa ma <liu.jiu.nian> pɕɯntɕɤt \\
 \textsc{lnk} now fourty \textsc{3sg}.\textsc{poss}-excess \textsc{ifr}-pass.X.years \textsc{lnk}  1969 since \\
 \glt `Now it has been fourty years (we have known each other), since 1969.' (12-BzaNsa, 13)
 \end{exe}
 
  \begin{exe}
\ex \label{ex:nW.Wqhu.pCintCAt}
 \gll  tɕiʑo pɯ-ari-tɕi ŋu, mtsʰukʰa pɯ-ftɕɤt-tɕi ŋu, nɯ ɯ-qʰu pɕɯntɕɤt tɕe,  nɯ-ʁgra nɯ nɯ-me ŋu \\
 \textsc{1du} \textsc{pfv}:\textsc{down}-go[II]-\textsc{1du} be:\textsc{fact} lake pfv-subdue-1du be:\textsc{fact} \textsc{dem} \textsc{3sg}.\textsc{poss}-after since \textsc{lnk} \textsc{2pl}.\textsc{poss}-enemy \textsc{dem} \textsc{pfv}-not.exist be:\textsc{fact} \\
 \glt `We went down (into the lake), subdued the (demons in) the lake, and from that time on, your enemy is no more.' (Nyima.'Odzer2003.2, 109-110)
 \end{exe}
 
 \begin{exe}
\ex \label{ex:Wqhu.CaNpCi}
\gll  tɕe tɤ-mu nɯ, nɯ ɯ-qhu ɕaŋpɕi ʑo kɤ-rɯndzɤqʰɤjɯ ta-znɯna \\
\textsc{lnk} \textsc{indef}.\textsc{poss}-mother \textsc{dem} \textsc{dem} \textsc{3sg}.\textsc{poss}-after from.that.time.on \textsc{emph} \textsc{inf}-eat.without.sharing \textsc{pfv}:3\fl{}3'-stop \\
\glt `From that (time) on, the mother stopped to eat on her own without sharing.' (tWJo2005, 53) 
\end{exe}

The postposition \japhug{jɤz}{when} and its variant \japhug{jɤznɤ}{when} follows either subordinate clauses (§ XXX), temporal adverbs (\ref{ex:jWfCWr.jAznA}) or the noun \japhug{ɯ-ŋgu}{beginning} (from \tibet{འགོ་}{go}{head, beginning}) as in (\ref{ex:jWfCWr.jAznA}); it is not attested with other noun phrases.

\begin{exe}
\ex \label{ex:jWfCWr.jAznA}
\gll  jɯfɕɯr nɯtɕu iɕqha tɤtɕɯpɯ nɯnɯra, jɯfɕɯr jɤznɤ tu-ndze-a tɕe pɯ-apa wo ri \\ 
yesterday \textsc{dem}:\textsc{loc} the.aforementioned boy \textsc{dem}:\textsc{pl} yesterday when \textsc{ipfv}-eat[III]-\textsc{1sg}  \textsc{lnk} \textsc{pst}.\textsc{ipfv}-be.correct \textsc{sfp} \textsc{lnk} \\
\glt  `I should have eaten these boys yesterday.' (160705 poucet5-v2, 36)
\end{exe}

 \begin{exe}
\ex \label{ex:WNgu.jAznA}
\gll  tɕe ɯ-ŋgu jɤznɤ tɕe, ɯ-mɤlɤjaʁ nɯra mɯ-cʰɯ-pʰaʁ-nɯ tɕe,  tɕe nɯ ɣɯ ɯ-ndʐi kɯ-fsɯ\redp{}fse nɯ pjɯ-qaʁ-nɯ tɕe tɕe \\
\textsc{lnk} \textsc{3sg}.\textsc{poss}-beginning when \textsc{lnk} \textsc{3sg}.\textsc{poss}-limb \textsc{dem}:\textsc{pl} \textsc{neg}-\textsc{ipfv}-cut-\textsc{pl} \textsc{lnk} \textsc{lnk} \textsc{dem} \textsc{gen} \textsc{3sg}.\textsc{poss}-skin \textsc{nmlz}:S/A-\textsc{emph}\redp{}be.like \textsc{dem} \textsc{ipfv}-remove.skin-\textsc{pl} \textsc{lnk} \textsc{lnk} \\
\glt `In the beginning, they don't cut off the limbs (from the cattle's body), and take out the skin (in such a way as to preserve its shape) exactly like (that of the living animal).' (06-BGa, 94)
\end{exe}

 
\section{Relator nouns}  \label{sec:relator.nouns}  
\subsection{Dative} \label{sec:dative} 
Two dative markers are attested in Japhug, \forme{ɯ-ɕki} and \forme{ɯ-pʰe}; some speakers like Tshendzin prefer the former (as in \ref{ex:WCki.zW}, \ref{ex:WCki.toti}), but most speakers I have recorded favour the latter (for instance, Kunbzang Mtsho who tells the story from which \ref{ex:Wphe.toti} is taken).


The dative can be followed by the locative postpositions \forme{zɯ} and \forme{tɕu}, as in (\ref{ex:WCki.zW}), (\ref{ex:nWCki.zYArNo}) and (\ref{ex:slANe.ZNgri.ra.nWphe}).


\begin{exe}
\ex \label{ex:WCki.zW}
\gll nɯ mbro tɯ-skɤt kɯ-tso nɯnɯ ɯ-ɕki zɯ .... to-ti \\
\textsc{dem} horse \textsc{indef}.\textsc{poss}-speech \textsc{nmlz}:S/A-understand \textsc{dem} \textsc{3sg}.\textsc{poss}-\textsc{dat} \textsc{loc} { } \textsc{ifr}-say \\
\glt `She said ... to the horse who could understand speech' (2003kAndzWsqhaj, 25)
\end{exe}

The dative is used to mark the recipient or addressee. It occurs with indirective verbs of speech such as \japhug{ti}{say} (\ref{ex:WCki.zW}, \ref{ex:WCki.toti} and \ref{ex:Wphe.toti}), \japhug{fɕɤt}{tell} and \japhug{tʰu}{ask} (\ref{ex:nWCki.tAthe}), and also with some intransitive verbs of speech such as \japhug{rɯɕmi}{speak} (\ref{ex:WCki.torWCmi}).

\begin{exe}
\ex \label{ex:WCki.toti}
\gll iɕqʰa srɯnmɯ nɯ kɯ, [...] smɤnmimitoʁ kuɕana ɯ-ɕki `nɤʑo tɕʰi ɯ-rɯɣ tɯ-ŋu' to-ti ri, \\
the.aforementioned râkshasî \textsc{dem} \textsc{erg} { } p.n. p.n. \textsc{3sg}.\textsc{poss}-\textsc{dat} \textsc{2sg} what \textsc{3sg}.\textsc{poss}-race 2-be:\textsc{fact} \textsc{ifr}-say \textsc{lnk} \\
\glt `The râkshasî asked Smanmi Metog Koshana, `What type of being are you?' (28-smAnmi, 378)
\end{exe}

\begin{exe}
\ex \label{ex:Wphe.toti}
\gll tɕe tɤ-tɕɯ nɯ kɯ ɯ-wa ɯ-pʰe nɯra pɯ-kɯ-fse nɯra to-ti ɲɯ-ŋu \\
\textsc{lnk} \textsc{indef}.\textsc{poss}-son \textsc{dem} \textsc{erg} \textsc{3sg}.\textsc{poss}-father \textsc{3sg}.\textsc{poss}-\textsc{dat} \textsc{dem}.\textsc{pl} \textsc{pfv}-\textsc{nmlz}:S/A-be.like \textsc{dem}.\textsc{pl}  \textsc{ifr}-say \textsc{sens}-be \\
\glt `The boy told his father the things that had happened.' (qachGa2012, 175)
\end{exe}

\begin{exe}
\ex \label{ex:nWCki.tAthe}
\gll nɤʑo ɯ-mɤ-ɲɯ-tɯ-stu nɤ, ʑara nɯ-ɕki tɤ-tʰe jɤɣ \\
\textsc{2sg} \textsc{qu}-\textsc{neg}-\textsc{sens}-2-believe \textsc{lnk} \textsc{3pl} \textsc{3pl}.\textsc{poss}-\textsc{dat} \textsc{imp}-ask[III] be.possible:\textsc{fact} \\
\glt `If you don't believe it, ask them!' (140508 shier ge tiaowu de gongzhu-zh, 190)
\end{exe}

\begin{exe}
\ex \label{ex:WCki.torWCmi}
\gll nɯ tɤ-pɤtso nɯ ɯ-ɕki to-rɯɕmi. \\
\textsc{dem} \textsc{indef}.\textsc{poss}-child \textsc{dem} \textsc{3sg}.\textsc{poss}-\textsc{dat} \textsc{ifr}-speak \\
\glt  `It spoke to the child.' (150831 renshen wawa-zh, 36)
\end{exe}

It also occurs with verbs of giving to mark the recipient as in (\ref{ex:Wphe.tokho}) and (\ref{ex:WCki.YAkho}) with the verb \japhug{kʰo}{give, pass over}, but also the source as in (\ref{ex:nWCki.zYArNo}) with verbs such as \japhug{rŋo}{borrow from} and \japhug{sɤmbi}{ask for}.

\begin{exe}
\ex \label{ex:Wphe.tokho}
\gll laʁjɯɣ nɯ ɯ-taʁ nɯ ɯ-pʰe to-kʰo tɕe,  \\
staff \textsc{dem} \textsc{3sg}.\textsc{poss}-on \textsc{dem} \textsc{3sg}.\textsc{poss}-\textsc{dat} \textsc{ifr}:\textsc{up}-give \textsc{lnk} \\
\glt `He gave the staff to the (thief) who was on (the tiger).' (khu2012, 15)
\end{exe}

\begin{exe}
\ex \label{ex:WCki.YAkho}
\gll  ɯ-nmaʁ ɯ-ɕki ɲɤ-kʰo tɕe,  \\
\textsc{3sg}.\textsc{poss}-husband \textsc{3sg}.\textsc{poss}-\textsc{dat} \textsc{ifr}-give \textsc{lnk} \\
\glt `She gave it to her husband.' (qajdoskAt, 71)
\end{exe}

\begin{exe}
\ex \label{ex:nWCki.zYArNo}
\gll 
kɯ-rɤrma ra nɯ-ɕki nɯtɕu, kuxtɕo ci z-ɲɤ-rŋo, \\
\textsc{nmlz}:S/A-work \textsc{pl} \textsc{3pl}.\textsc{poss}-\textsc{dat} \textsc{dem}:\textsc{loc} basket \textsc{indef} \textsc{transloc}-\textsc{ifr}-borrow \\
\glt `(The snow leopard) borrowed a basket from the workers.' (qala2002, 43)
\end{exe}

With the verb \japhug{kʰo}{give, pass over} the recipient is more often encoded with the genitive or a possessive prefix on the theme (§ \ref{sec:gen.beneficiary}) or with the semi-grammaticalized noun \japhug{tɯ-jaʁ}{hand} (§ \ref{sec:semi.grammaticalized.relator}).

 
The semi-transitive verb \japhug{ru}{look at} can mark its goal with the dative, as in (\ref{ex:WCki.Cturu}); this is however optional, as this verbs also takes goals in the absolutive (§ \ref{absolutive.goal}) or locative (§ \ref{sec:locative}).

\begin{exe}
\ex \label{ex:WCki.Cturu}
\gll tɕe tɤŋe nɯ nɯɣ-me tɕe, tɕe tɤŋe ɯ-ɕki ʁɟa ʑo ɕ-tu-ru tɕe, tɯʑo tɯ-ɕki maka ʑo mɤ-ru \\
\textsc{lnk} sun \textsc{dem} \textsc{appl}-be.afraid[III]:\textsc{fact}-\textsc{1sg} \textsc{lnk} \textsc{lnk} sun \textsc{3sg}.\textsc{poss}-\textsc{dat} completely \textsc{emph} \textsc{transloc}-\textsc{ipfv}:\textsc{up}-look.at \textsc{lnk} \textsc{genr} \textsc{genr}.\textsc{poss}-\textsc{dat} at.all \textsc{emph} \textsc{neg}-look.at:\textsc{fact} \\
\glt `(If the yeti catches you), it is afraid of the sun, it looks at the sun the whole time, and does not look at you.' (140510 mYWrgAt, 13)
\end{exe}

The dative \forme{ɯ-ɕki} derives from a relator noun meaning `side', `near' or `at X's place' (with or without motion). These locative meanings are still marginally present in Japhug in examples like (\ref{ex:WCki.loc}), (\ref{ex:slANe.ZNgri.ra.nWphe}) and (\ref{ex:WCki.kunWrAZi}).

\begin{exe}
\ex \label{ex:WCki.loc}
\gll  ɯ-rte nɯ ɯ-rna ɯ-ɕki pɯ-kɯ-ɴqoʁ nɯnɯ pjɤ-mɟa tɕe ɯ-ku ɯ-taʁ to-ta \\
\textsc{3sg.poss}-hat \textsc{dem} \textsc{3sg.poss}-ear \textsc{3sg}-\textsc{dat} \textsc{pfv:down-nmlz}:S/A-hang \textsc{dem} \textsc{ifr:down}-take \textsc{lnk} \textsc{3sg.poss}-head \textsc{3sg}-on \textsc{ifr}-put \\
\glt `He took the hard that was hanging on his ear and put it on his head.' (140505 liuhaohan zoubian tianxia, 164)
\end{exe}

\begin{exe}
\ex \label{ex:slANe.ZNgri.ra.nWphe}
\gll   tɤŋe cʰo slɤŋe ʑŋgri ra nɯ-pʰe nɯtɕu kɤ-nɤɕqa a-pɯ-tɯ-cʰa ra ma, \\
sun \textsc{comit} moon star \textsc{pl} \textsc{3pl}.\textsc{poss}-\textsc{dat} \textsc{dem}:\textsc{loc} \textsc{inf}-bear \textsc{irr}-\textsc{pfv}-2-can have.to:\textsc{fact} \textsc{lnk} \\
\glt `(When you are) by the sun, the moon and the stars, you will have to bear (the heat and the cold), otherwise...' (2003kandZislama, 53)
\end{exe}
 
\begin{exe}
\ex \label{ex:WCki.kunWrAZi}
\gll  li tɕɤtu tɤ-ɣe qʰe, ɯ-wa ɯ-ɕki ku-nɯ-rɤʑi, tɕɤki pɯ-ari qʰe, ɯ-wɯ ɯ-wi ni ndʑi-ɕki ju-nɯ-ɕe qʰe, \\
again up.there \textsc{pfv}:\textsc{up}-go[II] \textsc{lnk} \textsc{3sg}.\textsc{poss}-father \textsc{3sg}.\textsc{poss}-\textsc{dat} \textsc{ipfv}-\textsc{auto}-stay, down.there \textsc{pfv}-go[II] \textsc{lnk} \textsc{3sg}.\textsc{poss}-grand.father \textsc{3sg}.\textsc{poss}-grand.mother \textsc{du} \textsc{3du}.\textsc{poss}-\textsc{dat} \textsc{ipfv}-\textsc{auto}-go \textsc{lnk}  \\
\glt `When she comes up there, she stays at her father's house, and when she goes down there, she goes to her grandparent's place.' (14-tApitaRi, 305)
\end{exe}

\subsection{Secutive} \label{sec:secutive} 
The secutive relator noun \japhug{ɯ-rca}{following} is used with verbs of motion such as \japhug{gi}{come} to express the meaning `follow', `come/go with' as in (\ref{ex:nArca.Gia}).

\begin{exe}
\ex \label{ex:nArca.Gia}
 \gll  aʑo kɯnɤ nɤ-rca ɣi-a ɕti \\
 \textsc{1sg} also \textsc{2sg}.\textsc{poss}-following come:\textsc{fact}-\textsc{1sg} be.\textsc{affirm}:\textsc{fact} \\
\glt `I am coming/going with you.' (2011-05-nyima, 171)
\end{exe}

The secutive can have a meaning similar to that of the comitative adverb (§ \ref{sec:comitative.adverb}) `together with X', as in (\ref{ex:WBGi.Wrca}).

\begin{exe}
\ex \label{ex:WBGi.Wrca}
 \gll  pɤnmawombɤr ɣɯ ɯ-ɕɤrɯ ɯ-βɣi ɯ-rca tsʰɯntsʰɯn ʑo ta-wum-nɯ ɲɯ-ŋu \\ 
p.n. \textsc{gen} \textsc{3sg}.\textsc{poss}-bone \textsc{3sg}.\textsc{poss}-ash \textsc{3sg}.\textsc{poss}-following \textsc{idph}:II:neat \textsc{emph} \textsc{pfv}:3\fl{}3'-collect-\textsc{pl} \textsc{sens}-be  \\
\glt `They collected all of Padma 'Od-'bar's bones together with his ashes.' (Norbzang2005, 410)
\end{exe}

The secutive phrase can follow  (\ref{ex:WBGi.Wrca}), or precede (\ref{ex:tWjAGAt.Wrca}) the noun phrase it accompanies.

\begin{exe}
\ex \label{ex:tWjAGAt.Wrca}
 \gll   tɯ-jɤɣɤt ɯ-rca tɤ-se cʰɯ-nɯ-ɬoʁ \\
 \textsc{indef}.\textsc{poss}-feces \textsc{3sg}.\textsc{poss}-following \textsc{indef}.\textsc{poss}-blood \textsc{ipfv}:\textsc{downstream}-\textsc{auto}-come.out \\
 \glt `(In the case of this disease), blood comes out together with the feces.' 
 \end{exe}

The secutive with third person singular possessive \forme{ɯ-rca} is also used as a linker meaning `in addition' (see § XXX). With the indefinite possessive prefix \forme{tɤ-rca} and  \forme{tɯ-tɯ-rca}, the secutive appears in adverbial function with the meaning `together' (§ XXX), though in examples such as  (\ref{ex:tArAku.tArca}) the form  \forme{tɤ-rca} retains its nominal status.


\begin{exe}
\ex \label{ex:tArAku.tArca}
 \gll ɕoʁ nɯnɯ tɤ-rɤku tɤ-rca ŋu, sɯjno maʁ. \\
 buckwheat \textsc{dem} \textsc{indef}.\textsc{poss}-crops \textsc{indef}.\textsc{poss}-following be:fact grass not.be:fact \\
 \glt `Buckwheat (belongs) with the crops, it is not a (type of) grass.' (13-NanWkWmtsWG, 68)
\end{exe}

In addition, the unexpected focus marker \forme{rcanɯ} (§ \ref{sec:unexpected}) and the dubitative sentence final particle \forme{rca} (§ XXX) are historically related to the secutive.

\subsection{Deputative} \label{sec:deputative} 
The IPN \forme{ɯ-tsʰɤt} has two meanings. First, it can serve as a deputative relator noun `instead of, on behalf of' as in (\ref{ex:nWtAsno.WtshAt}) and (\ref{ex:nWsi.WtshAt}). No verb selects this relator noun. 

The deputative adjunct can correspond to the intransitive subject (as in \ref{ex:nWtAsno.WtshAt}, with the verb \japhug{tu}{exist}), the transitive subject (as in \ref{ex:aZo.nAtshAt}, with \japhug{ɣɯjtsi}{support}) or the object.

\begin{exe}
\ex \label{ex:nWtAsno.WtshAt}
\gll nɯʑora ɣɯ nɯ-tɤ-sno kɯ-fse ɯ-tsʰɤt nɯ, tɕiʑo ɣɯ, tɕi-xɕɤndʑu χsɯ-ldʑa pɯ-tu tɕe, nɯnɯ lɤ-nɯ-βlɯ-tɕi ɕti wo \\
\textsc{2pl} \textsc{gen} \textsc{2pl}.\textsc{poss}-\textsc{indef}.\textsc{poss}-saddle \textsc{nmlz}:S/A-be.like \textsc{3sg}.\textsc{poss}-instead.of \textsc{dem} \textsc{1du} \textsc{gen} \textsc{1du}.\textsc{poss}-twig three-long.object \textsc{pst}.\textsc{ipfv}-exist \textsc{lnk} \textsc{dem} \textsc{pfv}-\textsc{auto}-burn-\textsc{1du} be.\textsc{affirm}:\textsc{fact} \textsc{sfp} \\
\glt `Instead of a saddle like yours, we had three twigs, this is what we burned.' (Kubzang2003, 203)
\end{exe}

\begin{exe}
\ex \label{ex:aZo.nAtshAt}
\gll aʑo nɤ-tsʰɤt, nɤki, si nɯ tu-ɣɯjtsi-a jɤɣ \\
\textsc{1sg} \textsc{2sg}.\textsc{poss}-instead.of \textsc{filler} tree \textsc{dem} \textsc{ipfv}-support-\textsc{1sg} be.possible:\textsc{fact} \\
\glt `I can support the tree for you/instead of you (while you fetch it).' (150830 afanti, 136)
\end{exe}

 The noun phrase headed by \forme{ɯ-tsʰɤt} can be either an adjunct as in (\ref{ex:nWtAsno.WtshAt}) and (\ref{ex:aZo.nAtshAt}), the object of the verb \japhug{βzu}{make}, or a nominal predicate with a copula as in (\ref{ex:nWsi.WtshAt}) and (\ref{ex:aZo.atshAt}).  In the latter case, to express the meaning `do to $X$ instead of to $Y$', a biclausal construction `do to $X$, ($X$) is instead of $Y$' is used as in (\ref{ex:aZo.atshAt}).

\begin{exe}
\ex \label{ex:nWsi.WtshAt}
\gll si maŋe tɕe tɕe nɯnɯtɕu tɕe, nɯ-si ɯ-tsʰɤt ɲɯ-ŋu  \\
tree not.exist:\textsc{sens} \textsc{lnk} \textsc{lnk} \textsc{dem}:\textsc{loc} \textsc{lnk} \textsc{3pl}.\textsc{poss}-wood \textsc{3sg}.\textsc{poss}-instead.of \textsc{sens}-be \\
\glt `There no trees, there (dung) is used to replace the firewood.' (05-tamar, 10-11)
\end{exe}


\begin{exe}
\ex \label{ex:aZo.atshAt}
\gll nɯ tɤ-nɯ-ndɤm tɕe aʑo a-tsʰɤt ŋu tɕe \\
\textsc{dem} \textsc{imp}-\textsc{auto}-take[III] \textsc{lnk} \textsc{1sg} \textsc{1sg}.\textsc{poss}-instead.of be:\textsc{fact} \textsc{lnk} \\
\glt `Take these instead of me (as a compensation).' (2003kAndzwsqhaj2, 141)
\end{exe}

The examples (\ref{ex:aZo.nAtshAt}) and (\ref{ex:aZo.atshAt}) also show that the relator noun \japhug{ɯ-tsʰɤt}{instead of} can occur with a first or second person possessive prefix.

Second, \forme{ɯ-tsʰɤt} also means `with proper measure', mainly occurring in adverbial function as in (\ref{ex:WtshAt.tsa}) or in collocation with the verb \japhug{βzu}{make} in the sense `do with proper measure' as in (\ref{ex:WtshAt.tusWBzunW}). 

\begin{exe}
\ex \label{ex:WtshAt.tsa}
\gll rkaŋraŋ ɯ-tsʰɤt tsa ɲɯ-kɯ-nɤɕtʂaʁli-a-nɯ raʁmaʁ ma  \\
p.n. \textsc{3sg}.\textsc{poss}-proper.measure a.little \textsc{ipfv}-2\fl{}1-torture-\textsc{1sg}-\textsc{pl} \textsc{sfp} \textsc{lnk}  \\
\glt `Rkangrang, your torturing of me should have a limit.' 
\end{exe}

\begin{exe}
\ex \label{ex:WtshAt.tusWBzunW}
\gll ɯ-tsʰɤt tu-sɯ-βzu-nɯ mɯ́j-kʰɯ ma, nɯ-kɤ-kʰo nɯ mɯ-tʰa-ɕkɯt mɤɕtʂa tu-ndze ɲɯ-ɕti. \\
\textsc{3sg}.\textsc{poss}-proper.measure \textsc{ipfv}-\textsc{caus}-make-\textsc{pl} \textsc{neg}:\textsc{sens}-be.possible \textsc{pfv}-\textsc{nmlz}:P-give \textsc{dem} \textsc{neg}-\textsc{pfv}:3\fl{}3'-eat.completely until \textsc{ipfv}-eat[III] \textsc{sens}-be.\textsc{affirm} \\
\glt `They cannot make (the monkey eat) with measure, as it continues eating the (food) that is given to it until there is none.' (19-GzW, 60)
\end{exe}

In some contexts as in (\ref{ex:nWnW.WtshAt}), \japhug{ɯ-tsʰɤt}{proper measure} in adverbial used is better translated as `depending on the circumstances'.\footnote{This example is taken from a text describing goats and sheep; goats are called \forme{tsʰɤt} in Japhug, but it is clear from the context that \forme{ɯ-tsʰɤt} cannot be the possessed form of this noun. }

\begin{exe}
\ex \label{ex:nWnW.WtshAt}
\gll tɕe nɯnɯ ɯ-tsʰɤt nɯnɯ ɯ-pɯ ci ci ʁnɯz tu, ci ci tɯ-rdoʁ ma me tɕe núndʐa ɲɯ-ŋu. \\
\textsc{lnk} \textsc{dem} \textsc{3sg}.\textsc{poss}-proper.measure \textsc{dem}  \textsc{3sg}.\textsc{poss}-young once once two exist:\textsc{fact} once once one-piece apart.from not.exist:\textsc{fact} \textsc{lnk} for.this.reason \textsc{sens}-be \\
\glt `This is why, depending on the circumstances, sometimes (the goat) has two youngs, sometimes only one.' (05-qaZo, 28)
\end{exe}

The IPN  \forme{ɯ-tsʰɤt} (at least in the meaning `proper measure') is borrowed from \tibet{ཚད་}{tsʰad}{measure, limit}. It occurs as second element in the compound \japhug{xtɤtsʰɤt}{restraint of one's appetite}(with the \textit{status constructus} \forme{xtɤ-} of \japhug{tɯ-xtu}{belly}).

\subsection{Locative relator nouns} \label{sec:relator.location}
The dearth of specific locative postpositions (§ \ref{sec:locative}) other than the egressive ones (§ \ref{sec:egressive}) in Japhug is compensated by the existence of many relator nouns expressing various types of location, as shown in Table \ref{tab:relator.location}. 

As other relator nouns, they can follow noun phrases (including headless relative clauses), with the possessive prefix agreeing in person and number with the preceding constituant, but can also occur on their own if the referent is definite, as \japhug{ɯ-taʁ}{on} in (\ref{ex:WtaR.zW.kAmdzW}) and (\ref{ex:WtaR.nWtCu.YAXtAr}) below.

\begin{exe}
\ex \label{ex:WtaR.zW.kAmdzW}
\gll  ɯ-taʁ zɯ kɤ-amdzɯ nɤ tɤ-lu pa-tɕɤt ɲɯ-ŋu\\
\textsc{3sg}.\textsc{poss}-on \textsc{loc} \textsc{pfv}-sit \textsc{lnk} \textsc{indef}.\textsc{poss}-milk \textsc{pfv}:3\fl{}3'-take.out \textsc{sens}-be\\
\glt `She sat on him and milked.' (Kunbzang2005, 81)
\end{exe}

 

\subsubsection{The tridimensional system}

The first six nouns in Table \ref{tab:relator.location} are derived from location adverbs (§ XXX). The vertical dimension relator nouns \japhug{ɯ-taʁ}{on} and \japhug{ɯ-pa}{below} are simply built by adding a possessive prefix to the root of the adverb, while the other ones combine the status constructus of the adverbial root (\forme{lɤ-}, \forme{tʰɤ-}, \forme{kɤ-}, \forme{ndɤ-} from \forme{lo}, \forme{tʰi}, \forme{kɯ} and \forme{ndi} respectively) with a suffix \forme{-cu}. This sexpartite system comprising three pairs of elements along three dimensions (vertical, fluvial, solar) is the same as found in verbal morphological (§ XXX) and also egressive postpositions (§ \ref{sec:egressive}).


\begin{table}
\caption{Locative relator nouns in Japhug} \label{tab:relator.location}
\begin{tabular}{lllll}
\lsptoprule
& Lexical origin \\
\midrule
\japhug{ɯ-taʁ}{on, above} & \japhug{taʁ}{up}\\
\japhug{ɯ-pa}{below, under} & \japhug{pa}{down}\\
\japhug{ɯ-lɤcu}{upstream of} & \japhug{lo}{upstream}\\
\japhug{ɯ-tʰɤcu}{downstream of} & \japhug{thi}{downstream}\\
\japhug{ɯ-kɤcu}{east of} & \japhug{kɯ}{east}\\
\japhug{ɯ-ndɤcu}{west of} & \japhug{ndi}{west}\\
\midrule
\japhug{ɯ-ku}{top of} \\
\japhug{ɯ-ʁɤri}{before, in front of} \\
\japhug{ɯ-qʰu}{after, behind} \\
\japhug{ɯ-ŋgɯ}{inside} \\
\japhug{ɯ-pɕi}{outside} &  \tibet{ཕྱི་}{pʰʲi}{outside}\\
\japhug{ɯ-rkɯ}{side} \\
\japhug{ɯ-χcɤl}{middle, center} & \tibet{དཀྱིལ་}{dkʲil}{middle}\\
\japhug{ɯ-pɤrtʰɤβ}{between} & \tibet{བར་}{bar}{middle, between}\\
\japhug{ɯ-tʰɤβ}{between} \\
\japhug{ɯ-mŋu}{opening, edge, border} \\
\japhug{ɯ-ndo}{edge, border} \\
\lspbottomrule
\end{tabular}
\end{table}


\subsubsection{Other locative relator nouns}

Outside of the tridimensional system, other locative relator nouns also occur in antithetic pairs, in particular \japhug{ɯ-ʁɤri}{before, in front of}  vs \japhug{ɯ-qʰu}{after, behind}  (the latter also occurs as a temporal relator noun, cf § \ref{sec:relator.temporal}), \japhug{ɯ-ŋgɯ}{inside} vs \japhug{ɯ-pɕi}{outside}  and  \japhug{ɯ-mŋu}{opening, edge, border}  vs \japhug{ɯ-ndo}{edge, border}.

A few of these relator nouns are borrowed from Tibetan, as indicated in Table \ref{tab:relator.location}), but most of them are native Gyalrong words. In particular \japhug{ɯ-ʁɤri}{before, in front of} is one of the very rare cases of a disyllabic words that has a cognate in Tangut sharing both syllables (\tangut{𘁞𗙷}{5416-567}{ɣwə-rjir}{2.25-2.74} `before').\footnote{The syllable \forme{ʁɤ-} = \tangut{𘁞}{5416}{ɣwə}{2.25} may be a fossilized allomorph of \japhug{tɯ-ku}{head} with a uvular as in the Stau cognate \stau{ʁə}{head}.} The relator noun \japhug{ɯ-ku}{top of} (as in \ref{ex:Wku.ri}) is transparently grammaticalized from the body part \japhug{tɯ-ku}{head}.

\begin{exe}
\ex \label{ex:Wku.ri}
 \gll si ɣɯ ɯ-mat kɯ\redp{}kɯ-tu nɯ ɯ-ku ri ɕ-ku-zo ɲɯ-ŋu tɕe. \\
 tree \textsc{gen} \textsc{3sg}.\textsc{poss}-fruit \textsc{total}\redp{}\textsc{nmlz}:S/A-exist \textsc{dem} \textsc{3sg}.\textsc{poss}-top \textsc{loc} \textsc{transloc}-\textsc{ipfv}-land \textsc{sens}-be \textsc{lnk} \\
 \glt `It lands on the top of all trees that have fruits.' (24-ZmbrWpGa, 43)
\end{exe}

Some relator nouns other than \japhug{ɯ-ku}{top of} have non-grammaticalized uses; for instance \japhug{ɯ-ŋgɯ}{inside} still occurs as a noun meaning  `internal part, inside' in (\ref{ex:WNgW.nW.so}).

\begin{exe}
\ex \label{ex:WNgW.nW.so}
 \gll tɕe nɯnɯ kɯ-spoʁ ŋu, ɯ-ŋgɯ nɯ so tɕe  \\
 \textsc{lnk} \textsc{dem} \textsc{nmlz}:S/A-have.a.hole be:\textsc{fact} \textsc{3sg}.\textsc{poss}-inside \textsc{dem} be.hollow:\textsc{fact} \textsc{lnk} \\
 \glt `Its (inside) has a hole, its inside is hollow.' (12-Zmbroko, 23)
\end{exe}

The meaning of the relator nouns \forme{ɯ-mŋu} and \forme{ɯ-ndo} requires a specific description. These nouns are not antithetic to another in all cases. The basic (non-grammaticalized) meaning of  \forme{ɯ-mŋu}  is the border of the opening or mouth of a container / bag (\ref{ex:khWtsa.WmNu}), or the shoreline (of a lake), as in (\ref{ex:mtshu.WmNu}).

\begin{exe}
\ex \label{ex:khWtsa.WmNu}
\gll  tɕendɤre nɯnɯ kʰɯtsa ɯ-mŋu jamar kɯ-wxti tu,\\
\textsc{lnk} \textsc{dem} bowl \textsc{3sg}.\textsc{poss}-border about \textsc{nmlz}:S/A-be.big exist:\textsc{fact}\\
\glt  `Some are about as big as the mouth of a bowl.' (22-BlamajmAG, 130)
\end{exe}

\begin{exe}
\ex \label{ex:mtshu.WmNu}
\gll   tɕendre pɣɤtɕɯ nɯ mtsʰu ɯ-mŋu nɯtɕu `ʂɯt' to-ti to-nɯ-ɬoʁ.  \\
\textsc{lnk} bird \textsc{dem} lake \textsc{3sg}.\textsc{poss}-border \textsc{dem}:\textsc{loc} \textsc{idph}.I:sound \textsc{ifr}-say \textsc{ifr}:\textsc{up}-\textsc{auto}-come.out \\
\glt  `The bird came out of the shore of the lake with a noise.' (2014-kWlAG, 556)
\end{exe}

The basic meaning of \forme{ɯ-ndo} includes `extremity' (for instance, of a limb as in \ref{ex:WRar.Wndo}) and also `end' in both the locative and temporal sense (\ref{ex:Wndo.tCe}).

\begin{exe}
\ex \label{ex:WRar.Wndo}
\gll ɯ-ʁar ɯ-ndo nɯra hanɯni ɲɯ-ɲaʁ. \\ 
\textsc{3sg}.\textsc{poss}-wing \textsc{3sg}.\textsc{poss}-border \textsc{dem}:\textsc{pl} a.little \textsc{sens}-be.black \\
\glt  `The extremities of its wings are a bit black.' (23-scuz, 133)
\end{exe}

\begin{exe}
\ex \label{ex:Wndo.tCe}
\gll  ɯ-ndo tɕe maka kɯ-tu ɲɯ-me ɲɯ-ŋu tɕe,  \\
\textsc{3sg}.\textsc{poss}-border \textsc{lnk} at.all \textsc{nmlz}:S/A-exist \textsc{ipfv}-not.exist \textsc{sens}-be \textsc{lnk} \\
\glt `In the end, nothing is left.' (04-xiaocunzhuang-zh, 63)
\end{exe}

A contrast between \forme{ɯ-mŋu} and \forme{ɯ-ndo} occurs in their uses as relator nouns, indicating opposite extremities or sides. In the case of fields (as in \ref{ex:tWji.WmNu.Wndo}), \forme{ɯ-mŋu} designates the higher side of the field (towards the mountain), while \forme{ɯ-ndo} refers to the side closer to the river (all arable lands in Gyalrong area lie in narrow valleys).

\begin{exe}
\ex \label{ex:tWji.WmNu.Wndo}
\gll  qaʑmbri nɯ, tɯ-ji ɯ-ndo, tɯ-ji ɯ-mŋu nɯra aʁɤndɯndɤt ʑo tu-ɬoʁ ɕti \\
vine \textsc{dem} \textsc{indef}.\textsc{poss}-field \textsc{3sg}.\textsc{poss}-border \textsc{indef}.\textsc{poss}-field \textsc{3sg}.\textsc{poss}-border \textsc{dem}:\textsc{pl} everywhere \textsc{emph} \textsc{ipfv}:\textsc{up}-come.out be.\textsc{affirm}:\textsc{fact} \\
\glt `The vine, it grows everywhere, on both sides of the fields.' (06-qaZmbri, 12)
\end{exe}

In the case of clothes, \forme{ɯ-ndo} refers to the lower opening (towards the feet), as opposed to the collar, as in (\ref{ex:Wndo.ri.kulAtnW}).

\begin{exe}
\ex \label{ex:Wndo.ri.kulAtnW}
\gll tɕe nɯ ɯ-ndʐi nɯnɯ pjɯ-χtsɤβ-nɯ tɕe tɕe tɯ-ŋga ɯ-ndo ri ku-lɤt-nɯ, tɯ-ŋga ɯ-kuŋa, ɯ-pɤloʁ ɯ-ku, ɯ-ndo nɯra ku-lɤt-nɯ,  tu-sɯ-fskɤr-nɯ ŋu.   \\
\textsc{lnk} \textsc{dem} \textsc{3sg}.\textsc{poss}-skin \textsc{dem} \textsc{ipfv}-tan-\textsc{pl} \textsc{lnk} \textsc{lnk} \textsc{indef}.\textsc{poss}-clothes \textsc{3sg}.\textsc{poss}-border \textsc{loc} \textsc{ipfv}-throw-\textsc{pl} \textsc{indef}.\textsc{poss}-clothes \textsc{3sg}.\textsc{poss}-collar \textsc{3sg}.\textsc{poss}-sleeves \textsc{3sg}.\textsc{poss}-top \textsc{3sg}.\textsc{poss}-border  \textsc{dem}:\textsc{pl}  \textsc{ipfv}-throw-\textsc{pl} \textsc{ipfv}-\textsc{caus}-surround-\textsc{pl} be:\textsc{fact} \\
\glt `They tan its hide (of the otter) and put it on the lower opening of the clothes, on the collar of clothes, the cuffs of the sleeves and the lower opening, and make it around (these openings).' (28-qapar, 96)
\end{exe} 

The relator noun \forme{ɯ-mŋu}  can also designates the top extremity of stairs, as in (\ref{ex:rJAskAt.WmNu}).

\begin{exe}
\ex \label{ex:rJAskAt.WmNu}
\gll  cɯŋglɯɣ nɯ rɟɤskɤt ɯ-mŋu zɯ na-ta ɲɯ-ŋu \\
 pestle \textsc{dem} stairs \textsc{3sg}.\textsc{poss}-border \textsc{loc} \textsc{pfv}3\fl{}3'-put \textsc{sens}-be \\
 \glt `He put the pestle on the top of the stairs.' (tWJo2005, 49)
\end{exe} 

The superlative derivation (§ \ref{sec:superlative.XCWX}) can be applied to most locative relator nouns, as \forme{ɯ-mŋuɕɯmŋu} from  \forme{ɯ-mŋu} in (\ref{ex:WmNuCWmNu}), where it means that the liquid completely fills the bowl to the point of touching the border of its mouth, flowing out at the slightest motion.

\begin{exe}
\ex \label{ex:WmNuCWmNu}
\gll  tʂʰa tɤ́-wɣ-rku tɕe kʰɯtsa ɯ-mŋuɕɯmŋu stʰɯci tu-zɣɯt mɤ-ra ma kɤ-ndo tɕe sɤ-ɕke \\
tea \textsc{pfv}-\textsc{inv}-put.in \textsc{lnk} bowl \textsc{3sg}.\textsc{poss}-border:\textsc{superlative} so.much \textsc{ipfv}:\textsc{up}-reach \textsc{neg}-have.to:\textsc{fact} \textsc{lnk} \textsc{inf}-take \textsc{lnk} \textsc{deexp}-burn:\textsc{fact} \\
\glt `When one pours tea, it should not reach the limit of the mouth of the bowl, otherwise it will be burning when one holds it.' (elicited)
\end{exe} 

In addition to the relator nouns described above, the IPN \japhug{ɯ-stu}{straight ahead} is mainly used as an adverb, but can also serve as a relator noun to indicate the goal of a motion verb as in (\ref{ex:Wstu.Zo.YAGi}).

\begin{exe}
\ex \label{ex:Wstu.Zo.YAGi}
\gll  tɕe pʰaʁrgot ri li ɯʑo ɯ-stu ʑo ɲɤ-ɣi qʰe,  ɕɤmɯɣdɯ kɤ-lɤt mɯ-pjɤ-nɤz qʰe pʰaʁrgot jo-nɯ-ɕe. \\
\textsc{lnk} boar also again \textsc{3sg} \textsc{3sg}.\textsc{poss}-straight \textsc{emph} \textsc{ifr}:\textsc{west}-come \textsc{lnk} gun \textsc{inf}-throw \textsc{neg}-\textsc{ifr}.\textsc{ipfv}-dare \textsc{lnk} boar \textsc{ifr}-\textsc{auto}-go \\
\glt `The boar came directly at him, but he did not dare to shoot and the boar went away.' (150829 phaRrgot, 8)
\end{exe} 
 
 
 
\subsubsection{Locative relator nouns and locative postpositions} \label{sec:relator.postposition.location}
 All locative relator nouns can be also used with the locative postpositions \forme{zɯ}, \forme{ri}, \forme{tɕu} and the fused forms of the latter two with the demonstratives \forme{nɯre} and \forme{nɯtɕu} (§ \ref{sec:locative}). Example (\ref{ex:WtaR.ri.Wpa.ri}) illustrates the use of \japhug{ɯ-taʁ}{on, above} and \japhug{ɯ-pa}{below, under} with the locative \forme{ri} expressing both static position (with the existential verb \japhug{tu}{exist}) and motion towards (with the verb \japhug{lɤt}{throw}, here specifically meaning `direct water').

\begin{exe}
\ex \label{ex:WtaR.ri.Wpa.ri}
\gll ɯ-tʰɤcu maŋtʰi qʰajŋgɯ nɯ kɯ, nɤki, βɣa ɯ-pa ri tɕʰɯŋkʰɤr tu tɕe, tɕʰɯŋkʰɤr ɯ-taʁ ri cʰɯ-lɤt tɕe, tɕʰɯŋkʰɤr ɯ-taʁ nɯre ri βɣɤrnɤjwaʁ kɤ-ti tu tɕe \\
\textsc{3sg}.\textsc{poss}-downstream downstream water.trough \textsc{dem} \textsc{erg} \textsc{filler} mill \textsc{3sg}.\textsc{poss}-under \textsc{loc} water.wheel exist:\textsc{fact} \textsc{lnk} water.wheel \textsc{3sg}.\textsc{poss}-on \textsc{loc} \textsc{ipfv}:\textsc{downstream}-throw \textsc{lnk} \textsc{lnk} water.wheel \textsc{dem}:\textsc{loc} \textsc{loc} blades \textsc{nmlz}:P-say exist:\textsc{fact} \textsc{lnk} \\
\glt `The inferior water trough -- under the mill there is a water wheel -- (the water trough) directs (the water) onto that water wheel -- on the water wheel there are things called `blades'.' (06-BGa, 27-31)
\end{exe}

Example (\ref{ex:Wpa.ri.tulhoR}) illustrates \japhug{ɯ-pa}{below, under} with the locative \forme{ri} expressing motion from a place.

\begin{exe}
\ex \label{ex:Wpa.ri.tulhoR}
\gll ɯ-tʰoʁ ɯ-pa ri tu-ɬoʁ nɯra mɯ́j-cʰa \\
\textsc{3sg}.\textsc{poss}-earth \textsc{3sg}.\textsc{poss}-below \textsc{loc} \textsc{ipfv}:\textsc{up}-come.out \textsc{dem}:\textsc{pl} \textsc{neg}:\textsc{sens}-can \\
\glt `(Its shoots) cannot come out from under the ground.'  (15-babW, 45)
\end{exe}

Examples (\ref{ex:khri.WtaR.zW}) and (\ref{ex:WtaR.zW.kAmdzW}) above show the combination of \japhug{ɯ-taʁ}{on, above} with the locative \forme{zɯ}, also for static position and motion.

\begin{exe}
\ex \label{ex:khri.WtaR.zW}
\gll χsɤr kʰri ɯ-taʁ zɯ pjɤ-rɤʑi tɕe ɯ-tɯ-ɣɤχsrɯ pjɤ-saχaʁ ʑo. \\
gold bed \textsc{3sg}.\textsc{poss}-on \textsc{loc} \textsc{ifr}.\textsc{ipfv}-stay \textsc{lnk} \textsc{3sg}.\textsc{poss}-\textsc{nmlz}:\textsc{degree}-be.handsome \textsc{ifr}.\textsc{ipfv}-be.extremely \textsc{emph} \\
\glt `He was sitting on the golden bed, very handsome.' (2014-kWlAG, 409)
\end{exe}

The uses are attested with the locative \forme{tɕu}, as shown by (\ref{ex:WtaR.nWtCu.pjArAZi}) and (\ref{ex:WtaR.nWtCu.YAXtAr}). No clear criterion accounting for the presence or absence of these locative postpositions in combination with the relator nouns has been found.

\begin{exe}
\ex \label{ex:WtaR.nWtCu.pjArAZi}
\gll  rɟɤmtsʰu ɣɯ ɯ-rkɯ qambɯt ɯ-taʁ nɯtɕu pjɤ-rɤʑi. \\
ocean \textsc{gen} \textsc{3sg}.\textsc{poss}-side sand \textsc{3sg}.\textsc{poss}-on \textsc{dem}:\textsc{loc} \textsc{ifr}.\textsc{ipfv}-stay \\
\glt `He stayed on the beach.' (140511 xinbada-zh, 167)
\end{exe}

\begin{exe}
\ex \label{ex:WtaR.nWtCu.YAXtAr}
\gll tɕe ɯ-taʁ nɯtɕu pɣɤmuj tɯ-spra nɯ ɲɤ-χtɤr. \\
\textsc{lnk} \textsc{3sg}.\textsc{poss}-on \textsc{dem}:\textsc{loc} feather one-handful \textsc{ifr}-scatter \\
\glt `He scattered a handful of feathers on it.' (28-smAnmi, 327)
\end{exe}

There is one case of a fossilized \forme{zɯ} locative with the relator noun \japhug{ɯ-ŋgɯ}{inside}, the form \japhug{ɯ-ŋgɯz}{inside, among}, which is used in particular to single out an element for a group (as in \ref{ex:kAndZWRi.nWNgWz}) or to describe an itermediate colour (with stative verbs, as in example \ref{ex:arNi.WNgWz}).

\begin{exe}
\ex \label{ex:kAndZWRi.nWNgWz} 
\gll kɤndʑɯʁi nɯ-ŋgɯz stu kɯ-xtɕi nɯnɯ kɯ ... nɯra ntsɯ tu-ti pjɤ-ŋu. \\
\textsc{coll}:siblings \textsc{3pl}.\textsc{poss}-among:\textsc{loc} most \textsc{nmlz}:S/A-be.small \textsc{dem} \textsc{erg} { } \textsc{dem}:\textsc{pl} always \textsc{ipfv}-say \textsc{ifr}.\textsc{ipfv}-be \\
\glt `The youngest among the sisters was always saying ...' (150828 donglang, 26)
  \end{exe}
  
  \begin{exe}
\ex \label{ex:arNi.WNgWz} 
\gll nɯ ɯ-mdoʁ nɯ aj kɤ-ti mɯ́j-spe-a ma arŋi ɯ-ŋgɯz kɯnɤ pɣi kɯ-fse   \\
\textsc{dem} \textsc{3sg}.\textsc{poss}-colour \textsc{dem} \textsc{1sg} \textsc{inf}-say \textsc{neg}:\textsc{sens}-be.able[III]-\textsc{1sg} \textsc{lnk} be.green:\textsc{fact} \textsc{3sg}.\textsc{poss}-inside:\textsc{loc} also be.grey:\textsc{fact} \textsc{nmlz}:S/A-be.like \\
\glt `I cannot say its colour, it is somewhere between green and grey.' (06-qaZmbri, 56)
    \end{exe}
    
 
 \subsection{Temporal relator nouns} \label{sec:relator.temporal}
\subsection{Semi-grammaticalized relator nouns} \label{sec:semi.grammaticalized.relator} 
The noun \japhug{tɯ-jaʁ}{hand} occurs with several verbs in fixed collocation, the recipient of the action being indexed by the possessive prefix on this noun.

It is found with \japhug{kʰo}{give} to express the meaning `hand over to' as in (\ref{ex:ajaR.tAkhAm}) and (\ref{ex:nAjaR.YWkhoj}).

\begin{exe}
\ex \label{ex:ajaR.tAkhAm}
\gll nɤki tɤtʂu nɯ a-jaʁ tɤ-kʰɤm! \\
\textsc{dem}:\textsc{medial} lamp \textsc{dem}, \textsc{1sg}.\textsc{poss}-hand \textsc{imp}:\textsc{up}-give[III] \\
\glt `Hand over to me (up here) that lamp.' (140511 alading-zh, 122)
\end{exe}

\begin{exe}
\ex \label{ex:nAjaR.YWkhoj}
\gll kɯki mbro ki nɤ-jaʁ ɲɯ-kʰo-j ŋu \\
\textsc{dem}.\textsc{prox} horse \textsc{dem}.\textsc{prox} \textsc{2sg}.\textsc{poss}-hand \textsc{ipfv}-give-\textsc{1pl} be:\textsc{fact} \\
\glt  `(If you succeed), we will give you this horse.'  (X1-qachGa, 62)
\end{exe}

The collocation of \forme{tɯ-jaʁ} with the intransitive verb \japhug{zɣɯt}{reach, arrive} means `receive' or `obtain', as in (\ref{ex:ajaR.anWzGWt}). With the causative form \forme{sɤzɣɯt}, the collocation means `get' (with volition and controlability) as in (\ref{ex:WjaR.junWsAzGWt}) -- the recipient marked by the possessive prefix on  \forme{tɯ-jaʁ} is the same referent as the transitive subject of the main verb.

\begin{exe}
\ex \label{ex:ajaR.anWzGWt}
\gll iɕqʰa tɤtʂu nɯ a-jaʁ a-nɯ-zɣɯt ra \\
the.aforementioned lamp \textsc{dem} \textsc{1sg}.\textsc{poss}-hand \textsc{irr}-\textsc{pfv}-reach have.to:\textsc{fact} \\
\glt   `I have to obtain this lamp.' (140511 alading-zh, 212)
\end{exe}

\begin{exe}
\ex \label{ex:WjaR.junWsAzGWt}
\gll tɕʰi ra na-sɯso ʑo nɯ, ɯ-jaʁ ju-nɯ-sɯ-ɤzɣɯt pjɤ-cʰa.  \\
what \textsc{pl} \textsc{pfv}:3\fl{}3'-think \textsc{emph} \textsc{dem} \textsc{3sg}.\textsc{poss}-hand \textsc{ipfv}-\textsc{auto}-\textsc{caus}-reach \textsc{ifr}.\textsc{ipfv}-can \\
\glt  `He was able to get whatever he wanted.' (140508 benling gaoqiang de si xiongdi, 47)
\end{exe}

The noun \japhug{tɯ-jaʁ}{hand} the verb \japhug{ɣi}{come} also means `obtain' or `find', as in XXXXXX

\section{Noun modifiers and determiners}
This section discusses all nouns modifiers and determiners except relative clauses (§ XXX) and complement clauses (§ XXX). 
 
\subsection{Number}  \label{sec:number.determiners}
Japhug has two number markers, the dual \forme{ni} and the plural \forme{ra}. These clitics are not obligatory for non-singular arguments (even in the case of human referents), and do not necessary trigger plural or dual agreement on the verb. 

\subsubsection{Dual} \label{sec:dual.determiners}
The dual \forme{ni} is historically related to the numeral \forme{ʁnɯz} (§ \ref{sec:one.to.ten}), but their relationship is synchronically opaque. It combines with the proximal and distal demonstratives \forme{ki} and \forme{nɯ} respectively to form the dual demonstratives \forme{kɯni} and \forme{nɯni} (§ \ref{sec:demonstrative.pronouns}, § \ref{sec:demonstrative.determiners}).

There is no semantic restriction on the use of \forme{ni}, it most often occurs with human referents (\ref{ex:awW.cho.aRi.ni}, \ref{ex:Wmu.Wwa.ni}, \ref{ex:tCiZo.ni}, \ref{ex:ni.ndZisroR}), but is also commonly attested with animals (\ref{ex:ʁnWz.ni}) inanimate objects (\ref{ex:ni.RnaRna}), and placenames (\ref{ex:rgWnba.ni}).

\begin{exe}
\ex \label{ex:rgWnba.ni}
\gll prɤɕta cʰo rgɯnba ni ndʑi-pɤrtʰɤβ ri ŋu \\
pl.n. \textsc{comit} monastery \textsc{du} \textsc{3du}.\textsc{poss}-between \textsc{loc} be:\textsc{fact} \\
\glt `It is between Prashta and the monastery.' (140522 Kamnyu zgo, 115)
\end{exe}

The dual can follow the numeral \japhug{ʁnɯz}{two}, as in (\ref{ex:ʁnWz.ni}). This combination is however very rare (only 13 examples in the corpus out of hundreds of dual \forme{ni}). The opposite order (dual followed by numeral) is not grammatical.

\begin{exe}
\ex \label{ex:ʁnWz.ni}
\gll mbɣɤru nɯ jla ʁnɯz ni ndʑi-tʰɤβ ri ɲɯ-ɕe tɕe \\
plough.beam \textsc{dem} hybrid.yak two \textsc{3du}.\textsc{poss}-between \textsc{loc} \textsc{ipfv}:\textsc{west}-go \textsc{lnk} \textsc{lnk} \\
\glt `The beam of the plough goes between the two hybrid yaks.' 
\end{exe}


The adverb \japhug{ʁnaʁna}{both} (§ XXX) commonly co-occurs with dual, as in (\ref{ex:ni.RnaRna}).
%tɤ-pi ʁnaʁna ʑo pɯ́-wɣ-sat-ndʑi ɲɯ-ŋu. 

\begin{exe}
\ex \label{ex:ni.RnaRna}
\gll zaŋ cʰo raʁ ni ʁnaʁna ʑo ʁja ku-te ɲɯ-ŋu \\
copper \textsc{comit} brass \textsc{du} both \textsc{emph} verdigris \textsc{ipfv}-put[III] \textsc{sens}-be \\
\glt `Both copper and brass can get verdigris.' (30-Com, 101)
\end{exe}

The marker \forme{ni} can appear with a noun phrase comprising two nouns (each with singular referents) linked by the comitative \forme{cʰo} (§ \ref{sec:comitative}).

\begin{exe}
\ex \label{ex:awW.cho.aRi.ni}
\gll  tɕe a-wɯ cʰo a-ʁi ni pjɯ-tɯ-sat mɤ-jɤɣ \\
\textsc{lnk} \textsc{1sg}.\textsc{poss}-grandfather \textsc{comit} \textsc{1sg}.\textsc{poss}-younger.sibling \textsc{du} \textsc{ipfv}-2-kill \textsc{neg}-be.possible:\textsc{fact} \\
\glt `You cannot kill my grandfather and my younger brother.' (2011-05-nyima, 133)
\end{exe}

The dual can also be used with noun dyads (§ \ref{sec:dyads}), as in (\ref{ex:Wmu.Wwa.ni}). 

\begin{exe}
\ex \label{ex:Wmu.Wwa.ni}
\gll   ɯ-mu ɯ-wa ni kɯ ɲɯ-z-nɤja-ndʑi qʰe \\
\textsc{3sg}.\textsc{poss}-mother \textsc{3sg}.\textsc{poss}-father \textsc{du} \textsc{erg} \textsc{ipfv}-\textsc{caus}-be.a.pity-\textsc{du} \textsc{lnk} \\
\glt `Her parents would not be parted from her.' (14-tApitaRi, 305)
\end{exe}

The third person dual pronoun \forme{ʑɤni} is build by combining the pronominal root \forme{-ʑo-} with the dual \forme{ni} (§ \ref{sec:pers.pronouns}), and is not attested in combination with the dual. The first and second dual pronouns \forme{tɕiʑo} and \forme{ndʑiʑo}, do occur with the dual marker as in (\ref{ex:tCiZo.ni}), though examples are very rare.

\begin{exe}
\ex \label{ex:tCiZo.ni}
\gll  tɕiʑo ni wuma ʑo pɯ-amɯmi-tɕi tɕe \\
\textsc{1du} \textsc{du} really \textsc{emph} \textsc{pst}.\textsc{ipfv}-be.in.good.terms-\textsc{1du} \textsc{lnk} \\
\glt `We were in harmony together.' (140512 fushang he yaomo-zh, 85)
\end{exe}

Noun phrases with the dual \forme{ni} are always correlated with a dual prefix on the following noun in possessive constructions or with relator nouns, as in (\ref{ex:rgWnba.ni}), (\ref{ex:ʁnWz.ni}) and (\ref{ex:ni.ndZisroR}). Not a single example of a noun phrase in \forme{ni} followed by a noun with singular of plural possessive prefix is found in the corpus.

\begin{exe}
\ex \label{ex:ni.ndZisroR} 
\gll ɯ-pi ni ndʑi-sroʁ ko-ri tɕe \\
\textsc{3sg}.\textsc{poss}-elder.sibling \textsc{du} \textsc{3du}.\textsc{poss}-life \textsc{ifr}-save \textsc{lnk} \\
\glt `He saved the life of his two brothers.' (qachGa 2012, 139)
\end{exe}

The marker \forme{ni} is not obligatory with dual referents, in particular when the numeral \japhug{ʁnɯz}{two} is present. An overt noun phrase without dual marking can trigger indexation on the verb, especially with collectives expressing a pair of individuals as \japhug{ʁzɤmi}{husband and wife} in (\ref{ex:RjWmbrWg.RzAmi}), but also with other types of noun phrases as in (\ref{ex:nW.talWlAtndZi}).

\begin{exe}
\ex \label{ex:RjWmbrWg.RzAmi}
\gll  kɯɕɯŋgɯ tɕe tɕe atu <qinghai> ʑɴɢɯloʁ nɯtɕu tɕe, ʁjɯmbrɯɣ ʁzɤmi ci pjɤ-tu-ndʑi tɕe,  \\
in.former.times \textsc{lnk}  \textsc{lnk} up.there p.n. p.n. \textsc{dem}:\textsc{loc} \textsc{lnk} dragon husband.and.wife one \textsc{ifr}.\textsc{ipfv}-exist-\textsc{du} \textsc{lnk} \\
\glt `In former times, in Qinghai, in the Mgolog area, there was a couple of dragons.' (150820 qaprANar, 44)
\end{exe}

\begin{exe}
\ex \label{ex:nW.talWlAtndZi}
\gll  ʁdɯxpanaχpu ɯ-tɕɯ cʰo aʑo a-tɕɯ nɯ tɤ-alɯlɤt-ndʑi tɕe, \\
p.n. \textsc{3sg}.\textsc{poss}-son \textsc{comit} \textsc{1sg} \textsc{1sg}.\textsc{poss}-son \textsc{dem} \textsc{pfv}-fight-\textsc{du} \textsc{lnk} \\
\glt `The son of Gdugpa Nagpo and my son were fighting.' (28-smAnmi, 280)
\end{exe}

Such examples are however surprisingly rare in the corpus; dual indexation is most often correlated with a dual marker on the corresponding noun phrase, if overt.

The numeral \japhug{ʁnɯz}{two} without the dual also triggers dual indexation, as in (\ref{ex:RnWz.tundZi}).

\begin{exe}
\ex \label{ex:RnWz.tundZi}
\gll   sɯŋgɯ zɯ tɯrme wuma ʑo kɯ-wxti ʁnɯz tu-ndʑi tɕe\\
forest \textsc{loc} person really \textsc{emph} \textsc{nmlz}:S/A-be.big two exist:\textsc{fact}-\textsc{du} \textsc{lnk}\\
\glt `In the forest, there are two giants.'  (140428 yonggan de xiaocaifeng, 172)
\end{exe}

Dual marking on a noun phrase is not necessarily correlated with dual indexation on the verb, especially, but not exclusively, with inanimate referents, as in (\ref{ex:ni.tomto}). This question is studied in more detail in § XXX.

\begin{exe}
\ex \label{ex:ni.tomto}
\gll  ɯ-mɲaʁ χcʰoʁe ni to-mto. \\
\textsc{3sg}.\textsc{poss}-eye left.and.right \textsc{du} \textsc{pfv}-have.sight \\
\glt `His left and right eyes recovered sight.' (140517 mogui de jing, 105)
\end{exe}

However, a noun phrase with \forme{ni} is never correlated with a plural indexation marker on the verb. Apparent exceptions are either speech errors (a topic treated in § XXX), or cases of ambiguous indexation, as in (\ref{ex:paznAkharnW}).

\begin{exe}
\ex \label{ex:paznAkharnW}
 \gll  nɤ-pi ni kɯ nɤʑo nɯɣi kɤ-sɯso kɯ ʁmaʁ χsɯ-tɤxɯr kɯ pa-z-nɤkʰar-nɯ ɕti tɕe, \\
 \textsc{2sg}.\textsc{poss}-elder.sibling \textsc{du} \textsc{erg} \textsc{2sg} come.back:\textsc{fact} \textsc{inf}-think \textsc{erg} soldier three-round \textsc{erg} \textsc{pfv}:3\fl{}3'-\textsc{caus}-surround-\textsc{pl} be.\textsc{affirm}:\textsc{fact} \textsc{lnk} \\
 \glt `Your two elder brothers, thinking that you are coming back, had (the palace) guarded on all sides by three rows of soldiers.' (qachGa2012, 157)
\end{exe}

Example (\ref{ex:paznAkharnW}) is not completely straightforward, and deserves a detailed comment. The form \forme{paznɤkʰarnɯ} can be parsed as either \forme{pɯ-az-nɤkʰar-nɯ} \textsc{pst}.\textsc{ipfv}-\textsc{prog}-surround-\textsc{pl} `They were guarding it' with vowel fusion (§ XXX) or \forme{pa-z-nɤkʰar-nɯ} \textsc{pfv}:3\fl{}3'-\textsc{caus}-surround-\textsc{pl} `(He/they) had them guard it'. Context makes it clear here that the second option is the correct one, in particular because in the same passage in another version of the same story, we find the verb \forme{pa-sɯ-lɤt} \textsc{pfv}:3\fl{}3'-\textsc{caus}-throw `he had (them) make' with the perfective 3\fl{}3' form of a causative verb (\citealt[242]{jacques16complementation}, § XXX). Moreover, while the phrase \forme{nɤ-pi ni kɯ}  `your two elder brothers' could in principle belong to the infinitival clause in \forme{kɤ-sɯso}\footnote{Incidentally, note that this infinitival clause contains another complement in Hybrid Reported Speech, see § XXX.}, it is clear from the context and the explanations provided by native speakers that \forme{nɤ-pi ni kɯ} is the causer, and \forme{ʁmaʁ χsɯ-tɤxɯr kɯ} `three rows of soldiers' is the causee (also marked by the ergative, see § \ref{sec:causee.kW}). 

We thus observe plural indexation \forme{-nɯ} on the main verb \forme{pa-z-nɤkʰar-nɯ}, while the subject \forme{nɤ-pi ni kɯ}  has a dual marker. However, this is neither a counterexample to the number indexation rule stated above, nor a speech error: rather, it is a consequence of the fact that causees rather than causers can trigger number indexation on the verb in specific cases (see § XXX).

\subsubsection{Plural} \label{sec:plural.determiners}
The plural marker \forme{ra}, like the dual, follows the noun and most of its modifiers, and fuses with the demonstratives \forme{ki} and \forme{nɯ} respectively to build the plural demonstratives \forme{kɯra} and \forme{nɯra} (§ \ref{sec:demonstrative.pronouns}, § \ref{sec:demonstrative.determiners}). The etymology of the plural marker \forme{ra} is unknown, but a potential cognate exists in Pumi (\forme{=ɹə}, (\citealt[135]{daudey14grammar}; Japhug \forme{-a} regularly corresponds to Pumi \forme{-ə} in the native vocabulary). It should not be confused with the auxiliary verb \japhug{ra}{have to, need} (§ XXX), though there are cases where some ambiguity may occur (§ XXX).

Like the dual \forme{ni}, the plural \forme{ra} is compatible with both animate and inanimate referents, as in (\ref{ex:si.ra.cho}) and (\ref{ex:rdAstaR.ra}). It can be a plain marker of plurality as in (\ref{ex:si.ra.cho}).

\begin{exe}
\ex \label{ex:si.ra.cho}
\gll kɯmaʁ si ra cʰo nɯ-mdoʁ mɤ-naχtɕɯɣ \\
other tree \textsc{pl} \textsc{comit} \textsc{3pl}.\textsc{poss}-colour \textsc{neg}-be.the.same:\textsc{fact} \\
\glt `Its colour is different from that of the other trees.' (11-qrontshom, 56)
\end{exe} 

The marker \forme{ra} is also often an associative plural, understandable as `and other things', as in (\ref{ex:rdAstaR.ra}).

\begin{exe}
\ex \label{ex:rdAstaR.ra}
\gll rdɤstaʁ ra pjɯ-tʂaβ-nɯ qʰe tɯrme tu-xtsɯɣ ɲɯ-ŋu \\
stone \textsc{pl} \textsc{ipfv}-cause.to.fall-\textsc{pl} \textsc{lnk} people \textsc{ipfv}-hit \textsc{sens}-be \\
\glt `(Goats and sheep, as they climb high) cause stones (and other things) to fall and these hit people.' (tshAt-qaZo-kAlAG, 4)
\end{exe} 

The plural can follow numerals (even without head noun) to express an approximative number, as in (\ref{ex:XsWm.kWBde}).\footnote{Note that in (\ref{ex:XsWm.kWBde}) \forme{ci ci} is the expression for `sometimes', not used as a numeral, see § XXX.} 

\begin{exe}
\ex \label{ex:XsWm.kWBde}
\gll ci ci χsɯm kɯβde ra ɲɯ-lɤt ɲɯ-ŋgrɤl. tsuku tɕe ʁnɯz jamar ma mɯ́j-lɤt,\\
one one three four \textsc{pl} \textsc{sens}-throw \textsc{sens}-be.usually.the.case. some \textsc{lnk} two about apart.from \textsc{neg}:\textsc{sens}-throw \\
\glt  `Sometimes (dogs) have three or four (litters), some only have two.' (05-khWna, 22)
\end{exe} 

The plural marker \forme{ra} can also indicate approximate location, with or without locative markers. In (\ref{ex:kha.ra}), we find approximate location \forme{ra} in \forme{kʰa ra} `(everywhere) in the house, around the house' and \forme{tɯ-ji ɯ-ngɯ ra} `in the fields', and in (\ref{ex:nWrNa.ra}) with body parts.

This use of \forme{ra} can convey a meaning of distributed location, and is often combined with the adverb \japhug{aʁɤndɯndɤt}{everywhere} (§ \ref{sec:aRandWndAt}). It is reminiscent of plural markers in Kirghiz and Old Japanese, which combine collective, hypocoristic and approximate locative meanings (see \citealt[195]{antonov07ra}).

\begin{exe}
\ex \label{ex:kha.ra}
\gll βʑɯ nɯ wuma ʑo ŋɤn tɕe, tɕendɤre aʁɤndɯndɤt ʑo kʰa ra cʰɯ-rɤpɯ. tɯ-ji ɯ-ngɯ ra cʰɯ-rɤpɯ, \\
mouse \textsc{dem} really \textsc{emph} be.evil:\textsc{fact} \textsc{lnk} \textsc{lnk} everywhere \textsc{emph} house \textsc{pl} \textsc{ipfv}-bear.young \textsc{indef}.\textsc{poss}-field \textsc{3sg}.\textsc{poss}-inside \textsc{pl}  \textsc{ipfv}-bear.young \\
\glt `The mouse is fierce, it has youngs everywhere in the house, and has youngs in the fields.' (27-spjaNkW, 166)
\end{exe} 

\begin{exe}
\ex \label{ex:nWrNa.ra}
\gll nɯ-βri ra ɲɯ-ɬoʁ, nɯ-mke nɯra ɲɯ-ɬoʁ nɯ-rŋa ra brɤβbrɤβ ʑo ɲɯ-ɬoʁ ɲɯ-ŋu. \\
\textsc{3pl}.\textsc{poss}-body \textsc{pl} \textsc{ipfv}-come.out \textsc{3pl}.\textsc{poss}-neck \textsc{dem:pl} \textsc{ipfv}-come.out \textsc{3pl}.\textsc{poss}-face \textsc{pl} \textsc{idph}:II:covered.by.tiny.bumps \textsc{emph} \textsc{ipfv}-come.out  \textsc{sens}-be \\
\glt `(People who suffer from this disease have little blisters) appearing on their body, on their neck and all over their face.' (27-kharwut, 58)
\end{exe} 

The marker \forme{ra} even occurs with referents which are clearly singular, not only in the approximative location function, but also in examples such as (\ref{ex:tAwi.ra}) where the reason for the presence of \forme{ra} is less immediately obvious. In (\ref{ex:tAwi.ra}), a sentence taken from the translation of Rotkäppchen into Japhug (from Chinese, though here the presence of \forme{ra} cannot be due to calque), the function of the plural on the phrase \forme{tɤ-wi ra} `the grandmother' is more subtle: it conveys the idea idea that the impersonation takes on several aspects of the grandmother, not only her physical appearance, but also her voice, as implied by the second clause. 

\begin{exe}
\ex \label{ex:tAwi.ra}
\gll  qapar nɯ kɯ li, [...] tɤ-wi ra to-nɯɕpɯz tɕe, tɕe ɯ-skɤt ra cʰɤ-sɯ-ɤmtɕoʁ ʑo tɕe nɯra to-ti. \\
dhole \textsc{dem} \textsc{erg} again { } \textsc{indef}.\textsc{poss}-grandmother \textsc{pl} \textsc{ifr}-impersonate \textsc{lnk} \textsc{lnk} \textsc{3sg}.\textsc{poss}-voice \textsc{pl} \textsc{ifr}-\textsc{caus}-be.sharp \textsc{emph} \textsc{lnk} \textsc{dem}:\textsc{pl} \textsc{ifr}-say \\
\glt `The wolf was pretending to be the grandmother, and said these (words) with a sharp voice.' (140428 xiaohongmao-zh, 95-96)
\end{exe} 

Just like noun phrases with dual \forme{ni} correlate with dual possessive prefixe (see \ref{ex:ni.ndZisroR} in § \ref{sec:dual.determiners}), those with plural \forme{ra} can only be coreferent with a plural possessive prefix, as \forme{nɯ-} in (\ref{ex:si.ra.nWmat}).

\begin{exe}
\ex \label{ex:si.ra.nWmat}
 \gll  sɯku tɕe tʰɣe kɯ-fse, kɯmaʁ si ra nɯ-mat nɯra ɕ-pjɯ-nɯ-pʰɯt tɕe tu-ndze ɲɯ-ŋu.\\
tree \textsc{lnk} acorn \textsc{nmlz}:S/A-be.like other tree \textsc{pl} \textsc{3pl}.\textsc{poss}-fruit \textsc{dem}:\textsc{pl} \textsc{transloc}-\textsc{ipfv}:\textsc{down}-\textsc{auto}-pluck \textsc{lnk} \textsc{ipfv}-eat[III] \textsc{sens}-be \\
\glt `On the trees, (the bear) plucks acorn or fruits from other trees to eat.' (21-pri, 44)
\end{exe}

Apparent counterexamples such as (\ref{ex:WtaR.ra.Wmat}), where \forme{ra} is followed by a noun with the singular possessive prefix \forme{ɯ-}, occur when the preceding noun phrase is not the possessor of the following noun. For instance, in (\ref{ex:WtaR.ra.Wmat}) \forme{ra} has the vague locative function, and the phrase \forme{tɯ-ŋga ɯ-taʁ ra} `on the clothes' is not the possessor of \japhug{ɯ-mat}{its fruits}, it is a locative adjunct.

\begin{exe}
\ex \label{ex:WtaR.ra.Wmat}
 \gll tɯ-ŋga ɯ-taʁ ra ɯ-mat bɤbɤβ ʑo ku-ndzoʁ. \\
 \textsc{indef}.\textsc{poss}-clothes \textsc{3sg}.\textsc{poss}-on \textsc{pl} \textsc{3sg}.\textsc{poss}-fruit \textsc{idph}:II:in.clusters \textsc{emph} \textsc{ipfv}-\textsc{anticaus}:attach \\
\glt `Its seeds attach on clothes in clusters.' (18-qromJoR, 169)
\end{exe}

The plural \forme{ra} very commonly occurs with headless relatives, with or without a demonstrative, as in (\ref{ex:nW.tCaGi}), where we find both relatives followed by \forme{nɯnɯra} and another one followed by \forme{ra}.

\begin{exe}
\ex \label{ex:nW.tCaGi}
\gll [kɤ-ti mɤ-kɯ-pe kɯ-fse tu-kɯ-ti] nɯnɯra tɕe, [[kɤ-nɯtsɯ kɯ-ra] ra kɯnɤ tu-kɯ-ti] nɯnɯra, 
tɯrme ra kɯnɤ, tɕaɣi tu-sɤrmi-nɯ ŋgrɤl.  \\
\textsc{inf}-say \textsc{neg}-\textsc{nmlz}.S/A-be.good \textsc{nmlz}.S/A-be.like \textsc{ipfv}-\textsc{nmlz}.S/A-say \textsc{dem}:\textsc{pl} \textsc{lnk} \textsc{inf}-hide \textsc{nmlz}.S/A-have.to \textsc{pl} also \textsc{ipfv}-\textsc{nmlz}.S/A-say \textsc{dem}:\textsc{pl} people \textsc{pl} also  parrot \textsc{ipfv}-call-\textsc{pl} be.usually.the.case:\textsc{fact} \\
\glt `Those who say things that one should not say, who say even what should be concealed, even (if they are) people, they call them `parrots'. (24-qro, 125)
\end{exe} 

%ɯʑo sɤz pɣɤtɕɯ kɯ-xtɕi nɯra tu-ndze ɲɯ-ŋu. tɕe nɯnɯ tu-ti-nɯ ɲɯ-ŋu tɕe ɯ-mɤ-ŋu ma,
%ta-ndza ra pɯ́-wɣ-mto me
 
The plural \forme{ra} also occurs between auxiliaries and the preceding complement clause with a verb in finite (\ref{ex:GWkWnWru.ra}) or non-finite (\ref{ex:kAnAjaR.ra}) form, with a vague implication that additional related actions are concerned.

\begin{exe}
\ex \label{ex:GWkWnWru.ra}
 \gll li tɯ-ji ɯ-ŋgɯ ra ɣɯ-ku-nɯru ra ŋgrɤl. \\
 again \textsc{indef}.\textsc{poss}-field \textsc{3sg}.\textsc{poss}-inside \textsc{pl} \textsc{cisloc}-\textsc{ipfv}-eat.crops \textsc{pl} be.usually.the.case:\textsc{fact} \\
\glt `It also (usually) comes to eat crops in the fields.' (24-ZmbrWpGa, 37)
\end{exe}

\begin{exe}
\ex \label{ex:kAnAjaR.ra}
 \gll  ɣɤmdzu tɕe nɯnɯ kɤ-nɤjaʁ ra mɤ-sɤ-nɤz tɕe \\
be.thorny:\textsc{fact} \textsc{lnk} \textsc{dem} \textsc{inf}-touch \textsc{pl} \textsc{neg}-\textsc{deexp}-dare:\textsc{fact} \textsc{lnk} \\
\glt `It is thorny and one does not dare to touch it with the hand.' (11-qrontshom, 91)
\end{exe}

The marker \forme{ra} following a locative noun or adverb can have the meaning `the people/things from X', as in (\ref{ex:alo.ra}), without the need to add a demonstrative (cf \ref{ex:aki.nW} § \ref{sec:demonstrative.determiners}).

\begin{exe}
\ex \label{ex:alo.ra}
 \gll alo ra ɲɯ-mbɣom-nɯ qʰe \\
 upstream \textsc{pl} \textsc{sens}-be.in.a.hurry-\textsc{pl} \textsc{lnk} \\
 \glt `Those in the village, they (do things) in hurry.' (conversation140510 tshering, 175)
\end{exe}

\subsection{Demonstratives} \label{sec:demonstrative.determiners}
Japhug demonstrative determiners are formally identical  to the demonstrative pronouns (§ \ref{sec:demonstrative.pronouns}). They distinguish between proximal and distal demonstratives with different roots, and fuse with the dual and plural markers studied in § \ref{sec:number.determiners}; the proximal \forme{ki} undergoes change to \forme{kɯ-} in those fused forms.

As with the demonstrative pronouns, there are three sets of demonstratives, the base form, the reduplicated one (obtained by reduplicating the first syllable), and the emphatic one, with added \forme{ɯ-} prefix. Note that the latter two sets are not attested in the dual for determiners in the corpus, but the forms exist and are easily deducible from the corresponding plural ones. In addition, there is a medial demonstrative \forme{nɤki} which occurs in prenominal position.

\begin{table}
\caption{Demonstrative determiners}\label{tab:dem.determiners}
\begin{tabular}{ll|l|ll} 
\lsptoprule
&Base form & Reduplicated & Emphatic \\
\midrule
\textsc{prox.sg} & \forme{ki} & \forme{kɯki} &  \forme{ɯkɯki}  \\
\textsc{dist.sg} & \forme{nɯ} &  \forme{nɯnɯ} & \forme{ɯnɯnɯ} \\
\midrule
\textsc{prox.pl} & \forme{kɯni} & X &  X \\
\textsc{dist.pl} & \forme{nɯni} &  X & X \\
\midrule
\textsc{prox.pl} & \forme{kɯra} & \forme{kɯkɯra} &  \forme{ɯkɯkɯra}  \\
\textsc{dist.pl} & \forme{nɯra} &  \forme{nɯnɯra} & \forme{ɯnɯnɯra} \\
\midrule
\textsc{medial} &  \forme{nɤki} \\
\lspbottomrule
\end{tabular}
\end{table}

In Japhug, as in other Gyalrong languages, demonstrative determiners can be either/both pre- and postnominal, as shown by an example such as (\ref{ex:ki.srWnloRpW.ki}) with the proximal \forme{ki} both before and after the noun \japhug{srɯnloʁpɯ}{little ring}.

\begin{exe}
\ex \label{ex:ki.srWnloRpW.ki}
 \gll aʑo ɣɯ-ɕaβ-a tɤ-ŋu tɕe, ki srɯnloʁ-pɯ ki ɲɯ-ɕtʰɯz-a tɕe,  \\
 \textsc{1sg} \textsc{inv}-catch.up:\textsc{fact}-\textsc{1sg} \textsc{pfv}-be \textsc{lnk} \textsc{dem}.\textsc{prox} ring-\textsc{dim} \textsc{dem}.\textsc{prox} \textsc{ipfv}:\textsc{west}-turn.toward-\textsc{1sg} \\
\glt `When (the râkshasas) will be about to catch up with me, I will  turn this little ring towards west (in their direction).' (28-smAnmi, 222)
\end{exe}

All possible combinations of base demonstratives (B) and reduplicated demonstratives (R) are attested as pre- or postnominal determiners:

\begin{itemize}
\item BNB: \forme{ki} N \forme{ki}, \forme{nɯ} N \forme{nɯ} (\ref{ex:ki.srWnloRpW.ki})
\item RNB: \forme{kɯki} N \forme{ki}, \forme{nɯnɯ} N \forme{nɯ} (\ref{ex:kWki.tAYi.ki})
\item BNR: \forme{ki} N \forme{kɯki}, \forme{nɯ} N \forme{nɯnɯ} (\ref{ex:ki.rgAtpu.kWki})
\item RNR: \forme{kɯki} N \forme{kɯki}, \forme{nɯnɯ} N \forme{nɯnɯ} (\ref{ex:kWki.qingjiao.kWki})
\end{itemize}  

The types BNB and RNB, with the postnominal determiner as a base demonstrative, are by far the most common ones in the corpus.

\begin{exe}
\ex \label{ex:kWki.tAYi.ki}
 \gll  aʑo kɯki tɤɲi ki lu-nɤkʰɯkʰrɯt-a tɕe \\
 \textsc{1sg} \textsc{dem}.\textsc{prox} staff \textsc{dem}.\textsc{prox} \textsc{ipfv}:\textsc{upstream}-drag-\textsc{1sg} \textsc{lnk} \\
 \glt `I will drag along this staff (on the ground).' (Kunbzang2003, 225)
\end{exe}
 

\begin{exe}
\ex \label{ex:kWki.qingjiao.kWki}
 \gll iɕqʰa kɯki <qingjiao> kɯki tɕe, ɯ-qa kɯ-wɣrum ɲɯ-ŋu. \\
 the.aforementioned \textsc{dem}.\textsc{prox} plant.name \textsc{dem}.\textsc{prox} \textsc{lnk} \textsc{3sg}.\textsc{poss}-root \textsc{nmlz}:S/A-be.white \textsc{sens}-be \\
 \glt `This (plant that is called) \textit{qingjiao} (in Chinese), its root is white (unlike the other \textit{qingjiao} whose root is red).' (17-ndZWnW, 81)
\end{exe}

\begin{exe}
\ex \label{ex:ki.rgAtpu.kWki}
 \gll ki rgɤtpu kɯki kɯ, iɕqʰa, qaʑo nɯ to-mtsʰi qʰe, li tʂu kɯ-wxti nɯtɕu jo-ɕe tɕe, \\
\textsc{dem}.\textsc{prox} old.man \textsc{dem}.\textsc{prox} \textsc{erg} the.aforementioned sheep \textsc{dem} \textsc{ifr}-lead \textsc{lnk} again road \textsc{nmlz}:S/A-be.big \textsc{dem}:\textsc{loc} \textsc{ifr}-go \textsc{lnk} \\
\glt `The old man, leading the sheep, went to the big road.' (150822 laoye zuoshi zongshi duide-zh, 101
\end{exe}

The emphatic form is only used prenominally as in (\ref{ex:WkWki.arZaB.kWki}) to differentiate in case of confusion -- in this case, because the story is about two persons designated by the term \japhug{tɤ-rʑaβ}{wife}, even if they have different possessors (\textsc{3sg} vs \textsc{1sg}).

\begin{exe}
\ex \label{ex:WkWki.arZaB.kWki}
 \gll   nɯ ɯ-rʑaβ nɯ kɯ, ɯkɯki a-rʑaβ kɯki, kɯki ɕkom ki na-sɯ-ɤβzu tɕe, \\
 \textsc{dem} \textsc{3sg}.\textsc{poss}-wife \textsc{dem} \textsc{erg} \textsc{dem}.\textsc{prox}.\textsc{emph} \textsc{1sg}.\textsc{poss}-wide \textsc{dem}.\textsc{prox} \textsc{dem}.\textsc{prox} muntjac \textsc{dem}.\textsc{prox}  \textsc{pfv}:3\fl{}3'-\textsc{caus}-become \textsc{lnk} \\
\glt `His wife turned this wife of mine into this muntjac.' (140512 fushang he yaomo-zh, 187)
\end{exe}

When the postnominal demonstrative is in plural or dual form, the prenominal one is generally unmarked for number, as in (\ref{ex:kWki.tCheme.kWra}).

\begin{exe}
\ex \label{ex:kWki.tCheme.kWra}
 \gll kɯki tɕʰeme kɯra nɯ-rca aʑo tu-ɕe-a ɲɯ-ntsʰi ma mɯ́j-pe \\
 \textsc{dem} girl \textsc{dem}:\textsc{pl} \textsc{3pl}.\textsc{poss}-following \textsc{1sg} \textsc{ipfv}:\textsc{up}-go-\textsc{1sg} \textsc{sens}-have.better apart.from \textsc{neg}:\textsc{sens}-be.good \\
 \glt `I have no other choice but to go (to heaven) with these girls.' (31-deluge, 61)
 \end{exe}

However, there are also a few examples with plural marking on both pre- and postnominal demonstratives, as in (\ref{ex:nWnWra.pGa.nWra}), a remarkable phenomenon given the fact that the number markers are strictly postnominal. Plural marking on the prenominal demonstrative with a singular postnominal demonstrative is not attested.
 
\begin{exe}
\ex \label{ex:nWnWra.pGa.nWra}
 \gll nɯnɯra pɣa nɯra lonba ʑo ɲɤ-me-nɯ tɕe, ʁʑɯnɯ sqaptɯɣ ɲɤ-k-ɤpa-nɯ-ci. \\
 \textsc{dem.pl} bird  \textsc{dem.pl}  all \textsc{emph} \textsc{ifr}-not.exist \textsc{lnk} young.man eleven \textsc{ifr}-\textsc{evd}-become-\textsc{pl}-\textsc{evd} \\
 \glt `All those birds disappeared, and became eleven young men.' (140520 ye tiane-zh, 121)
\end{exe}

Proximal prenominal demonstratives can be combined with the postnominal \forme{nɯ}, as in (\ref{ex:kWki.Xpi.nW}), where the latter one is used as a topic marker. The opposite combination, a distal prenominal demonstrative with proximal postnominal one, is not attested in the corpus and presumably agrammatical.

\begin{exe}
\ex \label{ex:kWki.Xpi.nW}
 \gll kɯki χpi nɯ pɯpɯŋu nɤ,  \\
 \textsc{dem}.\textsc{prox} story \textsc{dem} \textsc{top} \textsc{lnk} \\
 \glt `As far as this story goes,' (11 examples in the corpus)
\end{exe}

The medial demonstrative \forme{nɤki}, used to designate referents closer to the addressee than the speaker, is found as a pronoun (§ \ref{sec:medial.dem.pro}), but also occurs as a prenominal determiner, with or without postnominal demonstrative (either proximal or distal), as in (\ref{ex:nAki.nAtAYi}) and (\ref{ex:nAki.nAtAri}). It is frequently used with a noun taking a second person possessive prefix -- note that the first syllable \forme{nɤ-} of the demonstrative \forme{nɤki} itself probably originates from the second singular possessive, as proposed in § \ref{sec:medial.dem.pro}.

\begin{exe}
\ex \label{ex:nAki.nAtAYi}
 \gll nɤki nɯ-tɤɲi ɯ-taʁ kɤ-rɤt nɯ ɯβrɤ-kɯ-z-nɤmɲo-a-nɯ \\
 \textsc{dem}:\textsc{medial} \textsc{2pl}.\textsc{poss}-staff \textsc{3sg}.\textsc{poss}-on \textsc{nmlz}:P-write \textsc{dem} \textsc{pot}-2\fl{}1-\textsc{caus}-watch-\textsc{1sg}-\textsc{pl} \\
 \glt `Would you show me what is written on that staff of yours?' (2003sras, 61)
\end{exe}

\begin{exe}
\ex \label{ex:nAki.nAtAri}
 \gll nɤki nɤ-tɤ-ri nɯ ŋotɕu pɯ-tu \\
 \textsc{dem}:\textsc{medial} \textsc{2sg}.\textsc{poss}-\textsc{indef}.\textsc{poss}-thread \textsc{dem} where \textsc{pst}.\textsc{ipfv}-exist \\
\glt `That thread of yours, where is it from?' (Norbzang2005, 180)
\end{exe}

The relative position of prenominal demonstrative and other pronominal elements is not free. The aforementioned topic marker \forme{iɕqʰa} strictly occurs before prenominal demonstratives (as in \ref{ex:kWki.qingjiao.kWki} and \ref{ex:kWki.XsAr.pGAtCW} respectively), while nominal modifiers such as \japhug{χsɤr}{gold} in \ref{ex:kWki.XsAr.pGAtCW} appear closer to the noun. Pronouns coreferent with a possessive prefix on the head noun, however, can be placed either after (\ref{ex:nW.aZo.aCArW.nW}) or before (\ref{ex:aZo.ki.aku.ki}) prenominal demonstratives.

\begin{exe}
\ex \label{ex:kWki.XsAr.pGAtCW}
 \gll kɯki χsɤr pɣɤtɕɯ ki nɤ-jaʁ ɲɯ-kham-a ŋu \\
\textsc{dem}.\textsc{prox} gold bird \textsc{dem}.\textsc{prox} \textsc{1sg}.\textsc{poss}-hand \textsc{ipfv}-give[III]-\textsc{1sg} be:\textsc{fact} \\
\glt `(If you succeed) I will give you this golden bird.' (2012qachGa, 46)
\end{exe}

\begin{exe}
\ex \label{ex:nW.aZo.aCArW.nW}
 \gll nɯtɕu a-tɯrsa ŋu, tɕe nɤʑo kɯ [nɯ aʑo a-ɕɤrɯ nɯnɯra] a-tɤ-tɯ-tɕɤt tɕe, \\
 \textsc{dem}:\textsc{loc} \textsc{1sg}.\textsc{poss}-tomb be:\textsc{fact} \textsc{lnk} \textsc{2sg} \textsc{erg} \textsc{dem} \textsc{1sg} \textsc{1sg}.\textsc{poss}-bone \textsc{dem}:\textsc{pl} \textsc{irr}-\textsc{pfv}-2-take.out \textsc{lnk} \\
\glt `My tomb is there, if you take out my bones (from it),' (150907 niexiaoqian-zh, 109)
\end{exe}

\begin{exe}
\ex \label{ex:aZo.ki.aku.ki}
 \gll  kɯki, aʑo [ki a-ku ki] pɯ-pʰɯt ra \\
 \textsc{dem}.\textsc{prox} \textsc{1sg}  \textsc{dem}.\textsc{prox} \textsc{1sg}.\textsc{poss}-head  \textsc{dem}.\textsc{prox} \textsc{imp}-cut have.to:\textsc{fact} \\
 \glt `Please behead me!' (140507 jinniao-zh, 292)
\end{exe}

The principles governing the presence and absence of the demonstrative determiners, and the choice of the various patterns described above, is particularly complex to describe and will be a topic for future research, when a larger corpus of texts will become available. While the proximal demonstratives always have some deictic function (although it may not be always appropriate to translate them with a demonstrative in other languages such as English), the distal demonstratives clearly contribute to marking topic (§ \ref{sec:topic}) and definiteness (§ \ref{sec:definiteness}), and disentangling these various functions is a complex matter.

The demonstratives \forme{nɯ} and \forme{nɯnɯ} are particularily common after relative clauses (either participial § XXX or finite ones § XXX) and complement clauses (§ XXX) but arguments against analyzing them as subordinators (like English `that') are presented in § XXX. 

Following locative adverbs (or locative postpositional phrases), the distal and proximal demonstratives can be used to express the meaning `the one/those X' as in (\ref{ex:athi.ki}) and (\ref{ex:aki.nW}). Note that the number determiner \forme{ra} can also be used in the same way (example \ref{ex:alo.ra} § \ref{sec:plural.determiners}) even without being combined with a demonstrative.

\begin{exe}
\ex \label{ex:athi.ki}
\gll amaŋ amaŋ, atʰi ki kɯ `a-βɣo mɤ-a<nɯ>tɯɣ-a tɕe a-scawa' ɲɯ-sɯsɤm ɲɯ-ŋu ɣe \\
\textsc{interj}:\textsc{surprise} \textsc{interj}:\textsc{surprise}  downstream \textsc{dem}.\textsc{prox} \textsc{erg} \textsc{1sg}.\textsc{poss}-uncle \textsc{neg}-<auto>meet:\textsc{fact}-\textsc{1sg} lnk 1sg.poss-poor.of \textsc{sens}-think[III] \textsc{sens}-be \textsc{sfp} \\
\glt `The one down there, he is thinking `Poor of me, I will not meet my lama', isn't he?' (2003kandZislama, 203)
\end{exe}

\begin{exe}
\ex \label{ex:aki.nW}
\gll a-pa, aki nɯ staʁlupa kɤ-βde ɯ-spa nɯ mɤ-nɯ-xsi ri, \\
\textsc{1sg}.\textsc{poss}-father down \textsc{dem} born.in.the.year.of.the.tiger \textsc{nmlz}:P-throw.away \textsc{3sg}.\textsc{poss}-material \textsc{dem} \textsc{neg}-\textsc{auto}-\textsc{genr}:know \textsc{lnk} \\
\glt `Father, the one down there, I don't know if he is a (boy) born in the year of the Tiger, to be thrown (in the lake), but...' (2011-05-nyima, 154)
\end{exe}



\subsection{Quantifiers} \label{sec:quantifiers.determiners}


\subsubsection{Universal quantifiers} \label{sec:universal.quant}
The determiner \japhug{tʰamtɕɤt}{all}, from Tibetan \tibet{ཐམས་ཅད་}{tʰams.cad}{all}, is strictly postnominal, as in (\ref{si.thamCAt.kW}). It cannot be used as a pronoun, and there are no examples in the corpus of \japhug{tʰamtɕɤt}{all} following a personal pronoun.

\begin{exe}
\ex \label{si.thamCAt.kW}
 \gll   sɯŋgɯ kɤ-kɯ-nɯχtɕɤn tɕe tɕe si tʰamtɕɤt kɯ nɯnɯ pjɯ-kɯ-sat kɯ-ŋgrɤl ɲɯ-ŋu. \\
 forest \textsc{pfv}-\textsc{nmlz}:S/A-be.dangerous \textsc{lnk} \textsc{lnk} tree all \textsc{erg} \textsc{dem} \textsc{ipfv}-\textsc{genr}:S/P-kill  \textsc{nmlz}:S/A-be.usually.the.case \textsc{sens}-be \\
 \glt `The fierce/dangerous forest, it was (a place where) all the tree would could to kill (people thrown into it).' (28-smAnmi, 191)
\end{exe}

It can also follow a headless relative clause, as in (\ref{WkWndza.thamCAt.nWnWra}), and be followed by demonstratives.

\begin{exe}
\ex \label{WkWndza.thamCAt.nWnWra}
 \gll nɯ-zda rɯdaʁ ɯ-kɯ-ndza tʰamtɕɤt nɯnɯra nɯ-rmi lonba kɯrŋi tu-kɯ-ti ŋu \\
 \textsc{3pl}.\textsc{poss}-companion animal \textsc{3sg}.\textsc{poss}-\textsc{nmlz}:S/A-eat all \textsc{dem}:\textsc{pl}  \textsc{3pl}.\textsc{poss}-name all beast \textsc{ipfv}-\textsc{genr}-say be:\textsc{fact} \\
 \glt `All those which eat the other animals, their name, all of them, is `beast'.'  (150822 kWrNi, 8)
 \end{exe}
 
The combination of a demonstrative such as \japhug{nɯ}{this} with  \japhug{tʰamtɕɤt}{all} does not mean `all of this', but `so much, so many', as in (\ref{kha.nW.thamCAt}) (see § XXX for more examples of this construction).

\begin{exe}
\ex \label{kha.nW.thamCAt}
 \gll iʑora tʰɯ-dɤn-i qʰe, kʰa nɯ tʰamtɕɤt mɯ-ɲɯ-ɤmɯ-xtɕʰɯt-i qʰe \\ 
 \textsc{1pl} \textsc{pfv}-be.many-\textsc{1pl} \textsc{lnk} house \textsc{dem} all \textsc{neg}-\textsc{sens}-\textsc{recip}-have.enough.place-\textsc{1pl} \textsc{lnk} \\
 \glt `There was now more of us (than before), and so many of us could not fit in the house.' (14-tApitaRi, 103-104)
 \end{exe}
 
 Another universal quantifier,  \japhug{kɤsɯfse}{all}, is common as a pronoun (§ \ref{sec:quantifiers.pronouns}) or as an adverb (§ XXX). It is potentially analyzable as a determiner in examples like  (\ref{ex:kAtsa.ra.kAsWfse.kW}) where it follows the noun phrase \forme{kɤtsa ra} `parents and children', and takes the ergative \forme{kɯ}.
 
 \begin{exe}
\ex \label{ex:kAtsa.ra.kAsWfse.kW}
\gll  kɤtsa ra kɤsɯfse kɯ wuma ʑo pjɤ-nɯ-rga-nɯ  \\
parents.and.children \textsc{pl} all \textsc{erg} really \textsc{emph} \textsc{ifr}.\textsc{ipfv}-\textsc{appl}-like-\textsc{pl} \\
\glt `Everybody in the family liked her very much.' (140429 qingwa wangzi, 5)
  \end{exe}
  
A third universal quantifier \japhug{rmɯrmi}{all, all kinds of}, a borrowing from Situ (meaning `everybody'), is attested but only rarely used, in examples such as (\ref{rmWrmi.GW.nWrmi}).
  
\begin{exe}
\ex \label{rmWrmi.GW.nWrmi}
\gll  maka tɤ-rɤku rmɯrmi ɣɯ nɯ-rmi nɯ to-nɤrmi ri maka kɯm mɯ-pjɤ-ɲɟɯ   \\
at.all \textsc{indef}.\textsc{poss}-crops all \textsc{gen} \textsc{3pl}.\textsc{poss}-name \textsc{ifr}-say.name \textsc{lnk} at.all door \textsc{neg}-\textsc{ifr}-\textsc{anticaus}:open \\
\glt `He said the names of all kinds of crops, but the door did not open.' (140512 alibaba-zh, 107)
\end{exe}  

 A fourth universal quantifier, \japhug{lonba}{all} (from Tibetan \tibet{ལོན་པ་}{lon.pa}{reached, enough, completed}), exists in Japhug, but it is not used as a noun determiner and only occurs as an adverb (§ XXX).
% aʑo a-ʁi nɯra lonba aʑo kɯ tɤ-nɤpɯpa-t-a
% 140426 tApAtso kAnWBdaR4, 1
% 
% tɕe, ɯ-tɯ-mbro nɯnɯ cho nɯra
%lonba qaɕti cho naχtɕɯɣ-ndʑi ʑo
%09-sArsi, 17

An alternative construction with a meaning similar to a universal quantifier is the totalitative reduplication (§ XXX) \japhug{kɯ\redp{}kɯ-tu}{all who exist} of the participle of the existential verb \japhug{tu}{exist}, in a post-nominal or head-internal relative clauses, as in (\ref{ex:Wzda.kWkWtu.kW}). 

 \begin{exe}
\ex \label{ex:Wzda.kWkWtu.kW}
\gll tɕe [ɯ-zda ra kɯ\redp{}kɯ-tu] kɯ nɯ-rʑaβ na-nɯ-ɕar-nɯ ɲɯ-ŋu \\
\textsc{lnk} \textsc{3sg}.\textsc{poss}-companion \textsc{pl} \textsc{total}\redp{}\textsc{nmlz}:S/A-exist \textsc{erg} \textsc{3pl}.\textsc{poss}-wife \textsc{pfv}:3\fl{}3'-\textsc{auto}-look.for-\textsc{pl} \textsc{sens}-be \\
\glt `All of his companions took (other women) as their wives.' (Norbzang2005, 57)
  \end{exe}

\subsubsection{Mid-scalar quantifier} \label{sec:tsuku}
The quantifier \japhug{tsuku}{some} is generally used as a pronoun (§ \ref{sec:partitive.pronouns}), but it does occur as a prenominal determiner as in (\ref{ex:tsuku.tWrme}), or a postnominal one as in (\ref{ex:kWmtChW.tsuku}) and (\ref{ex:kWmWrkW.tsuku}). It is most often used in the corpus with human referents, but is compatible with inanimate objects, as shown by (\ref{ex:kWmtChW.tsuku}).

\begin{exe}
\ex \label{ex:tsuku.tWrme}
\gll tsuku tɯrme ra kú-wɣ-mtsɯɣ-nɯ tɕe mɯ́j-ʁdɯɣ, tsuku tɯrme ra [...] kú-wɣ-mtsɯɣ-nɯ tɕe tɕe, wuma ʑo cʰɯ́-wɣ-z-nɯɣmbɤβ-nɯ qʰe ɲɯ́-wɣ-z-nɯtɯfɕɤl-nɯ qʰe ku-rŋgɯ-nɯ ɲɯ-ra.\\
some people \textsc{pl} \textsc{ipfv}-\textsc{inv}-bite-\textsc{pl} \textsc{lnk} \textsc{neg}.\textsc{sens}-be.serious some people \textsc{pl} { } \textsc{ipfv}-\textsc{inv}-bite-\textsc{pl} \textsc{lnk} \textsc{lnk} really \textsc{emph} \textsc{ipfv}-\textsc{inv}-\textsc{caus}-swell-\textsc{pl} \textsc{lnk}  \textsc{ipfv}-\textsc{inv}-\textsc{caus}-have.diarrhea-\textsc{pl} \textsc{lnk} \textsc{ipfv}-lie.down-\textsc{pl} \textsc{sens}-have.to\\
\glt `Some people, when they are stung (by bees) are fine, other people, when they are stung, it causes them swelling and diarrhea and they have to lie down.' (26-ndzWrnaR, 65-67)
\end{exe}

\begin{exe}
\ex \label{ex:kWmtChW.tsuku}
\gll  `pjɯ-nɯβle-a ɲɯ-ra' ɲɤ-sɯso tɕe, kɯmtɕʰɯ tsuku ɲɤ-kʰo tɕe, \\
\textsc{ipfv}-cheat[III]-\textsc{1sg} \textsc{sens}-have.to \textsc{ifr}-think \textsc{lnk} toy some \textsc{ifr}-give \textsc{lnk} \\
\glt `She thought `Let's cheat him' and gave him some toys.' (Norbzang2012, 134)
\end{exe}

\begin{exe}
\ex \label{ex:kWmWrkW.tsuku}
\gll ri kɯ-mɯrkɯ tsuku pjɤ-tu-nɯ tɕe tɕe, \\
\textsc{lnk} \textsc{nmlz}:S/A-steal some \textsc{ifr}.\textsc{ipfv}-exist-\textsc{pl} \textsc{lnk} \textsc{lnk} \\
\glt `There were some thieves.' (X1-khu, 7)
\end{exe}

Note in (\ref{ex:tsuku.tWrme.tWrdoR}) the combination of the quantifier \japhug{tsuku}{some} with the counted noun \japhug{tɯ-rdoʁ}{one piece}, which expresses here a partitive meaning (thirteen of fifteen children for each of them, § \ref{sec:ICN}).

\begin{exe}
\ex \label{ex:tsuku.tWrme.tWrdoR}
\gll tsuku tɯrme tɯ-rdoʁ ɣɯ ɯ-rɟit, sqafsum jamar, sqamŋu jamar tu-kɯ-tu pjɤ-tu. \\
 some person one-piece \textsc{gen} \textsc{3sg}.\textsc{poss}-offspring thirteen about fifteen about \textsc{ipfv}-\textsc{genr}:S/A-exist \textsc{ifr}.\textsc{ipfv}-exist    \\
\glt  `Some (women) had thirteen or fifteen children.' (140426 tApAtso kAnWBdaR, 88)
\end{exe}
 
\subsubsection{Distributive quantifier} \label{sec:raNri}
 Although distributive meaning is generally expressed in Japhug with a counted noun (see in particular § \ref{sec:ICN} and § \ref{sec:CCN}), the postnominal determiner \japhug{raŋri}{each} and its variant \japhug{rɯri}{each} (from Tibetan \tibet{རང་རེ་}{raŋ.re}{each}) can also express distributive meaning, as in (\ref{ex:tWtWpW.raNri}). 
 
\begin{exe}
\ex \label{ex:tWtWpW.raNri}
\gll paʁ rcanɯ, tɯ-tɯpɯ raŋri kɯ ʑo pjɯ-χsu-nɯ ra. \\
pig \textsc{unexpected} one-household each \textsc{erg} \textsc{emph} \textsc{ipfv}-raise-\textsc{pl} have.to:\textsc{fact} \\
 \glt `Each single household has to raise pigs.' (05-paR, 4)
 \end{exe}
 
 It can also be used with numerals, as in (\ref{ex:sqamNu.raNri}), where it refers specifically to days.

 \begin{exe}
\ex \label{ex:sqamNu.raNri}
\gll  sqamŋu raŋri ʑo zgo tu-ɕe pɯ-ŋu ɲɯ-ŋu, \\
fifteen \textsc{each} \textsc{emph} mountain \textsc{ipfv}:\textsc{up}-go \textsc{pst}.\textsc{ipfv}-be \textsc{sens}-be \\
\glt `Every fifteen days, she would go up the mountain.' (Norbzang2005, 57)
  \end{exe}

With a noun phrase with the quantifier \japhug{raŋri}{each} occurs in a clause with a counted noun, the scope of the two quantifiers is ambiguous, as in (\ref{ex:rirAB.raNri}).

\begin{exe}
\ex \label{ex:rirAB.raNri}
\gll rirɤβ raŋri χsɯ-tɤxɯr a-tɤ-tɯ-sɯ-lɤt tɕe, \\
mountain each three-round \textsc{irr}-\textsc{pfv}-2-\textsc{caus}-throw \textsc{lnk} \\
\glt `Drag her three times around each mountain.' (Kunbzang2005, 421)
 \end{exe}
 
When a noun with the determiner \forme{raŋri} is possessor, its possessum is often  followed by the distributive determiner \japhug{tɯka}{each} (and its reduplicated variant \forme{tɯkaka}) as in (\ref{ex:Wmat.raNri}).
 
  \begin{exe}
\ex \label{ex:Wmat.raNri}
\gll   iɕqʰa ɯ-mat raŋri ʑo nɯ ɯ-ru tɯka ntsɯ tu. \\
the.aforementioned \textsc{3sg}.\textsc{poss}-fruit each \textsc{emph} \textsc{dem} \textsc{3sg}.\textsc{poss}-stalk own always exist:\textsc{fact} \\
\glt `Each of its fruits has its own stalk.' (17-thowum, 34)
  \end{exe}
  
The  determiner \japhug{tɯka}{each} can also be used without a possessor in  \forme{raŋri}, for example with the distributive pronouns \japhug{ʑaka}{each his own} and \japhug{ʑakastaka}{each his own} (§ \ref{sec:distributive.pronouns})  as in (\ref{ex:nWkho.tWka}).

   \begin{exe}
\ex \label{ex:nWkho.tWka}
\gll      ʑakastaka nɯ-kʰo tɯka pjɤ-tu tɕe \\
each.his.own \textsc{3pl}.\textsc{poss}-room each \textsc{ifr}.\textsc{ipfv}-exist \textsc{lnk} \\
\glt `Each of them had her own room.' (140508 shie ge tiaowu de gongzhu, 85)
   \end{exe}
   
In addition to possessors, \japhug{tɯka}{each} also follows objects, with broad scope on the whole action (\ref{ex:pCaR.tWka}).

    \begin{exe}
\ex \label{ex:pCaR.tWka}
\gll  pɕaʁ tɯka to-βzu-nɯ tɕe jo-nɯ-ɕe-nɯ.  \\
reverence each \textsc{ifr}-make-\textsc{pl} \textsc{lnk} \textsc{ifr}-\textsc{vert}-go-\textsc{pl} \\
\glt `They made a reverence each and went back.' (28-smAnmi, 176)
   \end{exe}
 
\subsection{Indefinite and definite markers} \label{sec:indefinite.markers}

\subsubsection{Indefinite article} \label{sec:indef.article}
The form \japhug{ci}{one} has among its many functions (in addition to pronoun, numeral and adverb, see § \ref{sec:ci.someone}, § \ref{sec:other.pro}, § \ref{sec:partitive.pronouns}, § \ref{sec:identity.modifier}, § \ref{sec:one.to.ten} and § XXX) that of singular indefinite article, as in (\ref{ex:ci.indef}) and (\ref{ex:ci.chAGi}). It is typically used to introduce a new referent in a story.

\begin{exe}
\ex \label{ex:ci.indef}
\gll tɕʰeme kɯ-mpɕɯ\redp{}mpɕɤr ci ɲɤ-nɯ-ɬoʁ \\
girl \textsc{nmlz}:S/A-\textsc{emph}\redp{}beautiful \textsc{indef} \textsc{ifr}-\textsc{auto}-come.out \\
\glt `A very beautiful girl appeared (out of it).' (The flood, 39)
\end{exe}

\begin{exe}
\ex \label{ex:ci.chAGi}
\gll tɕɤlo tɕe tɤ-tɕɯ ci cʰɤ-ɣi qʰe, \\
upstream \textsc{lnk} \textsc{indef}.\textsc{poss}-son \textsc{indef} \textsc{ifr}:\textsc{downstream}-come \textsc{lnk} \\
\glt `A boy came from upstream.' (2003-kWBRa, 41)
\end{exe}

Although \forme{ci} can be used as a partitive pronoun `one of them' (§ \ref{sec:partitive.pronouns}), as a postnominal determiner it does not have partitive meaning. To express a meaning such as `one of the boys', a CN such as \japhug{tɯ-rdoʁ}{one piece} is used instead (§ \ref{sec:ICN}). 

Note that when used as a prenominal modifier, \forme{ci} has a completely different (definite) meaning `the other X' (§ \ref{sec:identity.modifier}). 

There are no dual or plural indefinite articles in Japhug. The plural marker \forme{ra} can occur after the indefinite \forme{ci}, but with a vague associative meaning `and other things' as in (\ref{ex:ci.ra}).

\begin{exe}
\ex \label{ex:ci.ra}
 \gll  ndʑi-tɕɯ ci, ndʑi-me ci ra to-tu. \\
 \textsc{3du}.\textsc{poss}-son \textsc{indef}  \textsc{3du}.\textsc{poss}-girl \textsc{indef} \textsc{pl} \textsc{ifr}-exist \\
 \glt  `They$_{du}$ had a boy and a girl (etc).' (150827 tianluo-zh, 155)
\end{exe}

 The indefinite \forme{ci} is not obligatory for indefinite referents (whether specific or non-specific), and bare NPs can used as \japhug{fsapaʁ}{animal} and \japhug{qapar}{dhole} in example (\ref{ex:ci.ra2}).
 

\begin{exe}
\ex \label{ex:ci.ra2}
 \gll  fsapaʁ nɯ-me, a-pɯ-si qhe, `nɯ qapar kɯ ta-ndza ŋu ma' tu-ti-nɯ ɕti ma, \\
 animal \textsc{pfv}-not.exist \textsc{irr}-\textsc{pfv}-die \textsc{lnk} \textsc{dem} dhole \textsc{erg} \textsc{pfv}:3\fl{}3'-eat be:\textsc{fact} \textsc{sfp} \textsc{ipfv}-say-\textsc{pl} be.\textsc{affirm}:\textsc{fact} \textsc{lnk}  \\
 \glt `When an animal disappears, dies, people say `A dhole ate it.' (28-qapar, 
\end{exe}


\subsubsection{Indefinite pronoun as modifier} \label{sec:indefinite}
The indefinite pronoun \japhug{tʰɯci}{something} (§ \ref{sec:thWci}) has marginal uses as a prenominal indefinite modifier, as in  (\ref{ex:thWci.laXCi}), (\ref{ex:thWci.WjmNo}) and (\ref{ex:laXtCha.ci.nWnW}) below. 

\begin{exe}
\ex \label{ex:thWci.laXCi}
\gll   tʰɯci laχɕi ci ɕ-pɯ-nɯ-βzjoz-nɯ tɕe, jɤ-ɕe-nɯ ra \\
something trade \textsc{indef} \textsc{transloc-imp-auto}-learn-\textsc{pl} \textsc{lnk} \textsc{imp}-go-\textsc{pl} have.to:\textsc{fact} \\
\glt `Go and learn some trade!' (140508 benling gaoqiang de si xiongdi-zh, 29)
 \end{exe}
 
 This construction arose perhaps from the use of the pronoun \forme{tʰɯci} as head of a postnominal relative clause with the verb \japhug{fse}{be like}, as illustrated by examples like (\ref{ex:thWci.kAnWsaXCWB}) or (\ref{ex:thWci.akAspa}) in § \ref{sec:thWci}. Turning the verb \japhug{fse}{be like} to a finite form as in (\ref{ex:thWci.WjmNo}) could cause the indefinite \forme{tʰɯci}, head of the relative in (\ref{ex:thWci.kAnWsaXCWB}), to be reanalyzed as the prenominal modifier of the immediately adjacent noun in (\ref{ex:thWci.WjmNo}).

 \begin{exe}
\ex \label{ex:thWci.kAnWsaXCWB}
\gll nɯra [tʰɯci [kɤ-nɯsaχɕɯβ kɯ-fse]] pɯ-ŋu wo.  \\
\textsc{dem}:\textsc{pl} something \textsc{inf}-have.a.contest \textsc{nmlz}:S/A-be.like \textsc{pst}.\textsc{ipfv}-be \textsc{sfp} \\
\glt `It was like a kind of contest.' (160706 thotsi, 16)
 \end{exe}
 
\begin{exe}
\ex \label{ex:thWci.WjmNo}
\gll [tʰɯci ɯ-jmŋo] ci ʑo pɯ-fse ri \\
something \textsc{3sg}.\textsc{poss}-dream one \textsc{emph} \textsc{pst}.\textsc{ipfv}-be.like \textsc{lnk} \\
\glt `It looked like (he had had) some dream.' (Lobzang2005, 74)
 \end{exe}
 
 
\subsubsection{The marking of definiteness} \label{sec:definiteness}
Japhug has no dedicated definite determiner, but  \forme{nɯ} and \forme{nɯnɯ}  as demonstrative determiners (\ref{sec:demonstrative.determiners}) and as topic markers (\ref{sec:topic}) and the prenominal aforementioned topic marker \forme{iɕqʰa} (§ \ref{sec:iCqha}) are generally used with definite referents.  

Example (\ref{ex:ci.joGi}) illustrates a typical example with the determiner \forme{nɯ}; the indefinite article \forme{ci} (§ \ref{sec:indef.article}) occurs in the first introduction of a new referent in the story as in the first clause of example (\ref{ex:ci.joGi}), but on the following occurrence of the same noun \forme{nɯ} is found.

\begin{exe}
\ex \label{ex:ci.joGi}
 \gll  tɕe qajdo ci jo-ɣi tɕe, tɕe qajdo nɯ kɯ `mo laz tu, pʰo laz me' to-ti. \\
 \textsc{lnk} crow \textsc{indef} \textsc{ifr}-come \textsc{lnk} \textsc{lnk} crow \textsc{dem} \textsc{erg} girl karma exist:\textsc{fact} boy karma not.exist:\textsc{fact} \textsc{ifr}-say \\
 \glt `A crow came. The crow said: `The girl will have chance, the boy won't.'' (28-qAjdoskAt, 8)
\end{exe}

However, although nouns phrases followed by \forme{nɯ} and \forme{nɯnɯ} more often than not denote definite referents, these determiners cannot be analyzed as definite articles, as noun phrases with \forme{nɯ} or \forme{nɯnɯ} can in certain cases have indefinite referents. 

A very clear case of use of \forme{nɯ} with an indefinite referent occurs on nouns serving as heads of head-internal relative clauses. A well-attested typological generalization is that in this type of relative clauses, definiteness marking is agrammatical (see \citealt{basilico96internally} and § XXX). In Khroskyabs, \citet[636]{lai17khroskyabs} reports that the definiteness marker \forme{=tə} is indeed not accepted on the head noun of head-internal relatives. In Japhug however, \forme{nɯ} does occur in such a syntactic context. For instance, in (\ref{ex:tAnmaR.nW.kW}), the head \forme{tɤ-nmaʁ nɯ kɯ} is subject of the participle \japhug{ɲɯ-kɯ-nɯ-ɕar}{looking for}, and is embedded in the participial relative clause indicated in brackets -- the presence of the ergative \forme{kɯ} precludes to analyze it as a post-nominal relative (§ XXX). From the meaning of the sentence the head \japhug{tɤ-nmaʁ}{husband} is clearly indefinite non-specific non-generic  (see \citealt[286-291]{lehmann84relativsatz}). The fact that it takes the marker \forme{nɯ} shows that this marker, unlike Khroskyabs \forme{=tə}, is not primarily marking definiteness.

\begin{exe}
\ex \label{ex:tAnmaR.nW.kW}
 \gll tɕeri [tɤ-nmaʁ nɯ kɯ ɯ-rʑaʁ kɯ-ɤntɕʰɯ ɲɯ-kɯ-nɯ-ɕar], aʁɤndɯndɤt tɤndɤɣri tu-kɯ-βzu pjɤ-tu.  \\
but  \textsc{indef}.\textsc{poss}-husband \textsc{dem} \textsc{erg} \textsc{3sg}.\textsc{poss}-wife  \textsc{nmlz}:S/A-be.many \textsc{ipfv}-\textsc{nmlz}:S/A-\textsc{auto}-search everywhere  illegitimate.child  \textsc{ipfv}-\textsc{nmlz}:S/A-make \textsc{ifr}.\textsc{ipfv}-exist \\
\glt `However there were husbands who were looking for several women and had illegitimate children.' (140427 tAndAGri, 3)
\end{exe}

Other cases of indefinite noun phrase with \forme{nɯ} are observed with left-dislocated topics. In example (\ref{ex:RnWz.nWnW}), we find a type of tail-head linkeage  (§ XXX) where both the noun phrase \japhug{spjaŋkɯ ʁnɯz}{two wolves} and the verb \japhug{ɲɤ-k-ɤtɯɣ-ci}{he met} are repeated; in the second occurrence, the noun phrase is topicalized and is followed by the topic marker \forme{nɯnɯ}, with a slight pause of hesitation. The determiner \forme{nɯnɯ} in this clause, unlike \forme{nɯ} in (\ref{ex:ci.joGi}), does not mark definiteness: that clause cannot be understood as `He met the two wolves'.

\begin{exe} 
\ex \label{ex:RnWz.nWnW} 
 \gll spjaŋkɯ ʁnɯz ɲɤ-k-ɤtɯɣ-ci. spjaŋkɯ ʁnɯz nɯnɯ, tɕendɤre ɲɤ-k-ɤtɯɣ-ci tɕe iɕqʰa, kɯ-rɤ-ntɕʰa nɯ wuma ʑo ɲɤ-mu. \\ 
 wolf two \textsc{ifr}-\textsc{evd}-meet-\textsc{evd}  wolf two \textsc{dem} \textsc{lnk} \textsc{ifr}-\textsc{evd}-meet-\textsc{evd} \textsc{lnk} the.aforementioned \textsc{nmlz}:S/A-\textsc{a.pass}:\textsc{n.hum}-kill \textsc{dem} really \textsc{emph} \textsc{ifr}-be.afraid \\ 
 \glt `He$_i$ (the butcher) met two wolves. He$_i$ met two wolves, and the butcher$_i$ was very much afraid.' (150902 liaozhai lang-zh, 7-8)
\end{exe}

The determiners \forme{nɯ} or \forme{nɯnɯ} are not attested in the corpus with the indefinite singular article \forme{ci} if both have scope on the same noun. In all cases with \forme{ci} followed by \forme{nɯ} (other than the identity pronoun in § \ref{sec:other.pro}), or of \forme{nɯ} followed by \forme{ci} in the corpus, they belong to different constituents. For instance, in (\ref{ex:ci.YAZGAsAphAr}), \forme{ci} is in adverbial use (`a little, once', see § XXX) and does not belong to the preceding noun phrase.  

\begin{exe}
\ex \label{ex:ci.YAZGAsAphAr}
\gll [tɕʰeme nɯ] ci ɲɤ-ʑɣɤ-sɤpʰɤr qʰe  \\
girl \textsc{dem} one \textsc{ifr}-\textsc{refl}-shake \textsc{lnk} \\
\glt `The girl shook herself.' (02-deluge2012, 125)
\end{exe}

In (\ref{ex:laXtCha.ci.nWnW}) although \forme{nɯnɯ} follows \forme{ci}, it has scope over the both preceding phrases, which are left-dislocated and followed by a pause.

\begin{exe}
\ex \label{ex:laXtCha.ci.nWnW}
\gll  kɤ-xtɕɤr tɕe nɯnɯ tɕe tɕe iɕqʰa, [[tʰɯci tɯmbri tɤ-ri kɯ-fse kɯ] [laχtɕʰa ci] nɯnɯ], ci kú-wɣ-sɯ-pa tɕe, kú-wɣ-xtɕɤr, \\
\textsc{inf}-attach \textsc{lnk} \textsc{dem} \textsc{lnk} \textsc{lnk} the.aforementioned something rope \textsc{indef}.\textsc{poss}-thread \textsc{nmlz}:S/A-be.like \textsc{erg} thing \textsc{indef} \textsc{dem} one \textsc{ipfv}-\textsc{inv}-\textsc{caus}-do \textsc{lnk} \textsc{ipfv}-\textsc{inf}-attach \\
\glt ``To attach' (means), to put together, attach something with something like a rope or a thread.'  (150902 kAxtCAr, 2-3)
\end{exe}

The aforementioned topic marker \forme{iɕqʰa} (§ \ref{sec:iCqha}) is almost always used with definite referents when prenominal, as in (\ref{ex:RnWz.nWnW}) above, and is the closest candidate to be analyzed as a definiteness marker in Japhug. It does occur with non-specific generic referents as in (\ref{ex:lWlAmu}), including some that are very clearly indefinite as in (\ref{ex:lApWG}); note the absence of postnominal determiner \forme{nɯ} (\ref{ex:lApWG}).

\begin{exe}
\ex \label{ex:lWlAmu}
 \gll iɕqʰa lɯlɤmu nɯ tʰɯ-rɤpɯ tɕe tɕe ɯ-sŋi tɕe kɤ-nɯ-rŋgɯ nɯ stʰɯci mɯ́j-tsu ma ɯ-pɯ ra χse ɲɯ-ra tɕe, \\
 the.aforementioned female.cat \textsc{dem} \textsc{ipfv}-bear.young \textsc{lnk} \textsc{lnk} \textsc{3sg}.\textsc{poss}-day \textsc{lnk} \textsc{inf}-\textsc{auto}-lie.down \textsc{dem} so.much \textsc{neg}:\textsc{sens}-have.time.to \\
 \glt `A/the female cat (unlike male cats), when it had had youngs, does not have time to sleep during the day, as it has to feed its youngs.' (21-lWLU, 
\end{exe}

\begin{exe}
\ex \label{ex:lApWG}
\gll  iɕqʰa lɤpɯɣ ɯ-rɣi ʑo fse. \\
the.aforementioned radish \textsc{3sg}.\textsc{poss}-seed \textsc{emph} be.like:\textsc{fact} \\
\glt `It looks like a radish seed.' (hist-26-qro-fourmi, 61)
\end{exe}

In  (\ref{ex:laXtCha.ci.nWnW}), \forme{iɕqʰa}  also precedes two phrases involving indefinite referents, but  there is a marked pause, and this is a case of \forme{iɕqʰa} in its function as speech filler (see § XXX).

\subsubsection{Absence of definiteness marking}
Like many languages (\citealt[130]{creissels06sgit1}), Japhug uses bare nouns without any definiteness marking. Bare nouns are most often non-referential, as \japhug{tɕʰeme}{girl} in (\ref{ex:tCheme.tWtAtu}).

\begin{exe}
\ex \label{ex:tCheme.tWtAtu}
\gll ʁnaʁna tɕʰeme tɯ\redp{}tɤ-tu nɤ, kɤndʑɯsqʰaj tu-kɤ-sɯ-βzu \\
both girl \textsc{cond}\redp{}\textsc{pfv}-exist \textsc{lnk} \textsc{coll}:sister \textsc{ipfv}-\textsc{inf}-\textsc{caus}-make \\
\glt `If both of them have girls, let them be sisters.' (zrAntCW, 4)
\end{exe}

Bare nouns are less common with referential nouns (except in answers to questions), but examples can be found, as \japhug{qacʰɣa}{fox} in (\ref{ex:qachGa.kW}).

\begin{exe}
\ex \label{ex:qachGa.kW}
\gll qacʰɣa 	kɯ maχtɕɯ tɤ-tɯt-a nɯ mɤ-tɯ-ste ti ɲɯ-ŋu \\
fox \textsc{erg} I.told.you.so \textsc{pfv}-say[II]-\textsc{1sg} \textsc{dem} \textsc{neg}-2-do.like[III]:\textsc{fact} say:\textsc{fact} \textsc{sens}-be \\
\glt `The fox says: `You do not do as I told you to." (2003qachGa, 44)
\end{exe}

Personal names generally occur as bare nouns, without any definiteness marker as in (\ref{ex:WrJAnpanma}), but there are no constraints against co-occurrence of personal names with the determiner \forme{nɯ} either (see § \ref{sec:personal.names.modifiers}).

\begin{exe}
\ex \label{ex:WrJAnpanma}
\gll  ɯrɟɤnpanma kɯ ʁlaŋsaŋtɕhin ɯ-ɕki  \\
 Padmasambhava \textsc{erg} Gesar \textsc{3sg}-\textsc{dat} \\
\glt `Padmasambhava (told) Gesar.' (Gesar, 2)
\end{exe}

 \subsection{Topic markers} \label{sec:topic}
 
  \subsubsection{Delimitative topic} \label{sec:delimitative}
The delimitative topic marker \forme{pɯ\redp{}pɯ-ŋu nɤ} `as for..., concerning...' is transparently derived from the past imperfective of the verb `be' in conditional form `if it was...' (with reduplication of the first syllable, see § XXX), as other copulas such as affirmative \japhug{ɕti}{be} and \japhug{maʁ}{not be} in (\ref{ex:pWpWmaʁ}).

\begin{exe}
\ex \label{ex:pWpWmaʁ}
\gll nɯnɯ koŋla ʑo tɤɕime pɯ\redp{}pɯ-maʁ nɤ \\
\textsc{dem} really \textsc{emph} princess \textsc{cond}\redp{}\textsc{pst.ipfv}-not.be lnk \\
\glt `If she was not really a princess,' (140519 wandou gongzhu, 71)
\end{exe}

The delimitative construction generally has scope over a noun phrase, which can have an additional demonstrative \forme{nɯ} as topicalizer as in (\ref{ex:nW.pWpWNunA}) (see § \ref{sec:nW.topic}).

\begin{exe}
\ex \label{ex:nW.pWpWNunA}
\gll a-mu nɯ pɯ\redp{}pɯ-ŋu nɤ, qhlɯ ʁdɯxpakɤrpu ɣɯ ɯ-me stu kɯ-xtɕi nɯ a-mu ɲɯ-pe, \\
\textsc{1sg}.\textsc{poss}-mother \textsc{dem} \textsc{cond}\redp{}\textsc{pst.ipfv}-be \textsc{lnk} nâga p.n \textsc{gen} \textsc{3sg}.\textsc{poss}-daughter most \textsc{nmlz}:S/A-be.small \textsc{dem} \textsc{1sg}.\textsc{poss}-mother \textsc{sens}-be.good \\
\glt `As for my mother, the daughter of the Nâga Gdugpa dkarpo is good to be my mother.' (Gesar, 5)
\end{exe}

In this construction, the verb is in the process of becoming grammaticalized as a topic particle. It is possible to find examples where the verb still takes person indexation in the delimitative construction when the topicalized element is a first or second person pronoun, as in (\ref{ex:pWpWNuanA}). 

\begin{exe}
\ex \label{ex:pWpWNuanA}
\gll aʑo pɯ\redp{}pɯ-ŋu-a nɤ, kɤndʑɯʁi kɯmŋu tu-j, \\
\textsc{1sg} \textsc{cond}\redp{}\textsc{pst.ipfv}-be-\textsc{1sg} \textsc{lnk} siblings five exist:\textsc{fact}-\textsc{1sg} \\
\glt `Concerning me, we are five brothers and sisters.' (hist140501 tshering skyid, 1)
\end{exe}

However, there are also examples with first or second person pronoun without indexation on the delimitative marker, as in (\ref{ex:pWpWNunA}), (\ref{ex:pWpWNunA2}) and (\ref{ex:pWpWNunA3}), where a first person singular form \forme{pɯ\redp{}pɯ-ŋu-a nɤ} or second person \forme{pɯ\redp{}pɯ-tɯ-ŋu nɤ} would have been expected. Such examples show that \forme{pɯpɯŋunɤ} has ceased to be analyzed as a verb form at least in these cases. Moreover, third person plural and dual indexation is hardly ever found in the delimitative construction.

\begin{exe}
\ex \label{ex:pWpWNunA}
\gll nɤʑo pɯpɯŋunɤ, ɬɤndʐi ra ɣɯ nɯ-kɯ-βʁa, nɯ-rɟɤlpu tɯ-ŋu \\
\textsc{2sg} as.for demon \textsc{pl} \textsc{gen} \textsc{3pl.poss}-\textsc{nmlz}:S/A-be.victorious \textsc{3pl.poss}-king 2-be:\textsc{fact} \\
\glt `You, you are the king of the demons.' (hist140512 fushang he yaomo-zh, 61)
\end{exe}

\begin{exe}
\ex \label{ex:pWpWNunA2}
\gll  aʑo kɯ-fse pɯpɯŋunɤ, ɕɯŋgɯ sɤ-xtɕɯ\redp{}xtɕi nɯtɕu, χpɯn lɤ-kɤ-ta, \\
\textsc{1sg} \textsc{nmlz}:S/A-be.like as.for  before \textsc{conv}-\redp{}be.small \textsc{dem}:\textsc{loc} monk \textsc{pfv}:\textsc{upstream}-\textsc{nmlz}:P-put \\
\glt `For instance me, (I was) sent to become monk early in my childhood.' (160721 XpWN, 7)
  \end{exe}

\begin{exe}
\ex \label{ex:pWpWNunA3}
\gll aʑo pɯpɯŋunɤ, nɯnɯ [...] aʑo ɣɯ a-ndʐa nɯ tu-o<nɯ>lɯlat-a pɯ-ŋu tɕe, \\
\textsc{1sg} as.for \textsc{dem} { } \textsc{1sg} \textsc{gen} \textsc{1sg}.\textsc{poss}-reason \textsc{dem} \textsc{ipfv}-<\textsc{auto}>fight-\textsc{1sg} \textsc{pst}.\textsc{ipfv}-be \textsc{lnk} \\
\glt  As for me, I was fighting for my own sake.' (140512 abide he mogui-zh, 92)
 \end{exe}
 
A short form \forme{ŋunɤ} instead of \forme{pɯpɯŋu nɤ} is also attested, as in (\ref{ex:WNga.ra.NunA}).

\begin{exe}
\ex \label{ex:WNga.ra.NunA}
\gll ma ɯ-ŋga ra ŋunɤ, maka wuma ʑo ko-ɴqhi ma. \\
\textsc{lnk} \textsc{3sg}.\textsc{poss}-clothes \textsc{pl} as.for at.all really \textsc{emph} \textsc{ifr}-be.dirty \textsc{lnk} \\
\glt `As for his clothes, they had become very dirty.' (conversation 140510)
\end{exe}
 
The delimitative topic  construction is appropriate to introduce the main topic of a following discourse (as in \ref{ex:pWpWNuanA} and \ref{ex:pWpWNunA2}), but can be used for contrastive topics, as in example (\ref{ex:pWpWNunA3}) where the speaker expresses a contrast between his and the addresses action (`you, you were fighting for the sake of other people').


 \subsubsection{Aforementioned topic} \label{sec:iCqha}
 The marker \japhug{iɕqʰa}{the aforementioned}  is used on referents that have been previously mentioned in the same story, usually only a few sentences back. It is strictly prenominal. 
 
Example (\ref{ex:iCqha.aforementioned}) illustrates the most typical use of this marker. Sentence (\ref{ex:kAtWm}) introduces a new reference, \japhug{kɤtɯm}{ball of thread} marked with the indefinite article \forme{ci} (§ \ref{sec:indef.article}). Three clauses later in (\ref{ex:iCqha.kAtWm}), the same referent occurs again with two topic markers, the postnominal \textit{nɯ} and the prenominal \textit{iɕqʰa}.
 
 
\begin{exe}
\ex \label{ex:iCqha.aforementioned}
\begin{xlist}
\ex \label{ex:kAtWm}
\gll `razri \textbf{kɤtɯm} \textbf{ci} ɲɯ-ra, taqaβ ci ɲɯ-ra' to-ti qʰe   \\
 thread ball \textsc{indef} \textsc{sens}-need needle \textsc{indef} \textsc{sens}-need \textsc{ifr}-say \textsc{lnk}  \\
\glt `He told (Rgyabza) `I need a ball of thread and a needle.''  
\ex  
\gll tɕendɤre ɲɤ-kʰo qʰe,  \\
\textsc{lnk} \textsc{ifr}-give \textsc{lnk}   \\
\glt `She gave it to him.'
\ex 
\gll  tɕe ɯ-ndzɤtsʰi ka-tsɯm-nɯ nɯtɕu qʰe tɕe,   \\
 \textsc{lnk} \textsc{3sg}.\textsc{poss}-meal \textsc{pfv}:3\fl{}3'-bring-\textsc{pl} \textsc{dem}:\textsc{loc}  \textsc{lnk} \textsc{lnk}    \\
\glt `When they brought his meal,'
\ex \label{ex:iCqha.kAtWm}
\gll   \textbf{iɕqʰa} \textbf{kɤtɯm} \textbf{nɯ} ɯʑo kɯ ko-ndo, \\
   the.aforementioned ball \textsc{dem} \textsc{3sg} \textsc{erg} \textsc{ifr}-take \\
\glt `he took the ball of thread, and...' (Gesar 270-272)
\end{xlist}
\end{exe}
 
A systematic study of the use of the topic marker \forme{iɕqʰa} in Japhug must overcome two inherent difficulties. First, this topic marker is homophonous with (and historically related to) the speech filler \forme{iɕqʰa} (§ XXX) and with the adverb \japhug{iɕqʰa}{just now}, which can also precede noun phrases. Listening to the sound files can help distinguishing between the three, as the speech filler is always followed by a pause (and optionally by the demonstrative \forme{nɯ}), but there are still ambiguous sentences (see below). Second, \forme{iɕqʰa} occurs on nouns designating entities that the speaker considers to have been previously referred to in the conversation, even if they are not present in the same recording. 

For instance in (\ref{ex:iCqha.pɣArnoR}) the noun \japhug{pɣɤrnoʁ}{a species of fungus} is used with \forme{iɕqʰa}, although this name does not occur before in the same text; it was however mentioned the day before in another recording.

\begin{exe}
\ex \label{ex:iCqha.pɣArnoR}
\gll nɯ zdɯmqe cʰo iɕqʰa, pɣɤrnoʁ nɯni ndʑi-tsʰɯɣa wuma ʑo naχtɕɯɣ. \\
\textsc{dem} fungi.sp. \textsc{comit} the.aforementioned fungi.sp. \textsc{dem}:\textsc{du} \textsc{3du}.\textsc{poss}-form really \textsc{emph} be:identical:\textsc{fact} \\
\glt `The \forme{zdɯmqe} and the \forme{pɣɤrnoʁ} are very similar.' (23-mbrAZim, 82)
\end{exe}

 
The topic marker \forme{iɕqʰa} transparently comes from the adverb \japhug{iɕqʰa}{just now} (§ XXX). The pivot constructions that allowed reanalysis from adverb to prenominal topic marker are very probably headless relatives (§ XXX) as in  (\ref{ex:iCqha.tAtWta}), or complement clauses as in (\ref{ex:iCqha.ZnWzmWnmuta}). 

\begin{exe}
\ex \label{ex:iCqha.tAtWta}
 \gll  [iɕqʰa tɤ-tɯt-a] nɯ tú-wɣ-stu qʰe, \\
 just.now \textsc{ifr}-say[II]-\textsc{1sg} \textsc{dem} \textsc{ipfv}-\textsc{inv}-do.like \textsc{lnk} \\
\glt `One does as I just said, and...' (2002tWsqar, 139)
\end{exe}

\begin{exe}
\ex \label{ex:iCqha.ZnWzmWnmuta}
 \gll iɕqʰa [ʑ-nɯ-z-mɯnmu-t-a] nɯ mɯ-pjɤ-pe rcama.  \\
the.aforementioned  \textsc{transloc}-\textsc{pfv}-\textsc{caus}-move-\textsc{pst}:\textsc{tr}-\textsc{1sg} \textsc{dem} \textsc{neg}-\textsc{ifr}.\textsc{ipfv}-be.good \textsc{fsp} \\
\glt `It was probably not a good thing that I had moved them (as I said above).' (150819 kumpGa, 45)
 \end{exe}
 
 These sentences are still synchronically ambiguous in Japhug; in  (\ref{ex:iCqha.ZnWzmWnmuta}) the context makes it clear that \forme{iɕqʰa} is the topic marker (since the fact of having moved (the eggs) had been told a few sentences back) and not an adverb `just now' with a temporal reference in the past, as the meaning would be `it was probably not a good thing that I had just moved them' (an impossible interpretation in this context, since this sentence is an explanation why several eggs had not given chicks, several days after they had been brought to another place). However, extracted from the context, both interpretation would be equally possible for (\ref{ex:iCqha.ZnWzmWnmuta}), and correspond to two distinct syntactic structures.

With postnominal (§ XXX) or left-headed head-internal relative clauses (§ XXX) as in (\ref{ex:tWrpa.thafse}), \forme{iɕqʰa} can also be ambiguous. Since the adverb \japhug{iɕqʰa}{just now} can occur both before the object (\ref{ex:tWrpa.thWfseta}) or before the verb (\ref{ex:tWrpa.thWfseta2}) in an independent clause, a relative such as (\ref{ex:tWrpa.thafse}) can be either interpreted `the axe (mentioned above) that he had whetted' (with the topic marker \forme{iɕqʰa} outside of the relative clause, having scope on its head) and `the axe that he had just whetted' with the adverb \japhug{iɕqʰa}{just now} inside the relative clause.

 \begin{exe}
\ex \label{ex:tWrpa.thafse}
 \gll  tɕendɤre <luban> kɯ iɕqʰa [tɯrpa tʰa-fse] nɯ to-ndo tɕe, \\
 \textsc{lnk} p.n. \textsc{erg} the.aforementioned axe \textsc{pfv}:3\fl{}3'-whet \textsc{dem} \textsc{ifr}-take \textsc{lnk} \\
 \glt `Luban took the axe that he had whetted.' (150902 luban-zh, 90)
 \end{exe}

  \begin{exe}
  \ex 
  \begin{xlist}
\ex \label{ex:tWrpa.thWfseta}
 \gll   iɕqʰa tɯrpa tʰɯ-fse-t-a \\
just.now axe \textsc{pfv}-whet-\textsc{pst}:\textsc{tr}-\textsc{1sg} \\
\ex \label{ex:tWrpa.thWfseta2}
 \gll   tɯrpa  iɕqʰa tʰɯ-fse-t-a \\
 axe just.now \textsc{pfv}-whet-\textsc{pst}:\textsc{tr}-\textsc{1sg} \\
 \glt `I just whetted a/the axe.' (elicited)
 \end{xlist}
 \end{exe}

The use of \forme{iɕqʰa} as a topic marker with nouns (as in \ref{ex:iCqha.kAtWm} above) probably took place by reanalysis of the adverb in headless or postnominal relatives, or in complment clauses as above, then generalized to all noun phrases even those without subordinate clause.

\subsubsection{The demonstrative \forme{nɯ} as a topic marker} \label{sec:nW.topic}
The postnominal determiner \forme{nɯ} and its reduplicated form \forme{nɯnɯ} is one of the most common words in Japhug, and has a considerable number of functions. It is used as a demonstrative (\ref{sec:demonstrative.determiners}), contributes to expressing definiteness (\ref{sec:definiteness}) and could be argued to be a subordinator (an analysis not adopted in the present work, see § XXX).

In addition, it is commonly used to mark topic: left-dislocated noun phrases generally (though not compulsorily) take this determiner. For instance, in texts presenting animals or plants, their name on first occurrence is left dislocated and followed by the determiner \forme{nɯ}, as in (\ref{ex:qawWz.nW}).

\begin{exe}
\ex \label{ex:qawWz.nW}
\gll  qawɯz nɯ, (qawɯz nɯ pɯ-tɯ-mto-t, ɣe?)  qawɯz nɯnɯ, nɤki, kɯɕɯŋgɯ tɕe, \\
Edelweiss \textsc{dem} Edelweiss \textsc{dem} \textsc{pfv}-2-see-\textsc{pst}:\textsc{tr} \textsc{sfp} Edelweiss \textsc{dem} \textsc{filler} before \textsc{lnk} \\
\glt `The edelweiss, (you saw Edelweiss before, right?)... The edelweiss, in former times,' (15-babW, 177)
\end{exe}

In its function as a topicalizer, the determiner \forme{nɯ} can follow a noun with postnominal demonstratives, as in (\ref{ex:kWki.nW}). However, due to the difficulty of systematically sorting out the topicalization and demonstrative functions of this marker, I do not attempt to reflect this distinction in the glosses, and use  \textsc{dem} everywhere.

\begin{exe}
\ex \label{ex:kWki.nW}
\gll tɕeri kɯki mɯntoʁ kɯki nɯ pɯpɯŋunɤ, wuma ʑo kɯ-ʑru, kɯ-pe, \\
\textsc{lnk} \textsc{dem}.\textsc{prox} flower \textsc{dem}.\textsc{prox} \textsc{dem} as.far really \textsc{emph} \textsc{nmlz}:S/A-be.strong \textsc{nmlz}:S/A-be.good \\ 
\glt `But concerning this flower, so precious and nice' (150820 meili de meiguihua, 58)
\end{exe}

\subsubsection{The linker \forme{tɕe} as a topic marker} \label{sec:tCe.topic}
The word \forme{tɕe}, which originates from a locative postposition (§\ref{sec:locative.j}), is mainly used in Japhug as a linker (§ XXX), one of the most common words in the corpus.

In addition, it can serve as a topic marker, following left-dislocated noun or postpositional phrases (\ref{ex:tsuku.kW.tCe}).

\begin{exe}
\ex \label{ex:tsuku.kW.tCe}
\gll tsuku kɯ tɕe lɤpɯɣ ra mbɯsɯt chɯ-lɤt-nɯ tɕe nɯra ɲɯ-rku-nɯ ɲɯ-ŋu \\
some \textsc{erg} \textsc{lnk} radish \textsc{pl} grating \textsc{ipfv}-throw-\textsc{pl} \textsc{lnk} \textsc{dem}.\textsc{pl} \textsc{ipfv}-put.in-\textsc{pl} \textsc{sens}-be \\
\glt `Some people, they grate radish and use it as filling (for the sausage).' (05-paR, 77)
\end{exe}

 \subsection{Focus markers} \label{sec:focus}
   \subsubsection{Unexpected focus} \label{sec:unexpected}
  The unexpected/high degree marker \forme{rcanɯ} or \forme{rca}, which was grammaticalized from the  secutive relator noun \japhug{ɯ-rca}{following} (§ \ref{sec:secutive}). It indicates that the phrase or clause preceding it is topical, and the situation or action described by the predicate that follows is unexpected (\ref{ex:nAZo.rcanW}), intensifies to a noticeable (and not foreseeable) extent (\ref{ex:tokAnWmqajndZic.tCe.rcanW}) or occurs with a remarkably high degree or intensity, with  (\ref{ex:mbro.rcanW}) or without (\ref{ex:apWme.rcanW}) surprise.

\begin{exe}
\ex \label{ex:nAZo.rcanW}
 \gll  wo nɤʑo rcanɯ tɕʰi ɲɯ-tɯ-nɤme ŋu ma,  aʑo tɯ-mɯ kɯ pɯ-kɯ-sɯ-χtɕi-a, tɤndʐo nɯ! \\
 \textsc{interj} \textsc{2sg} \textsc{foc}:\textsc{unexp} what \textsc{sens}-2-do[III] be:\textsc{fact} \textsc{lnk} \textsc{1sg} \textsc{indef}.\textsc{poss}-sky \textsc{erg} \textsc{pfv}-2\fl{}1-\textsc{caus}-wash-\textsc{1sg} cold \textsc{sfp} \\
\glt `You, what are you doing, you caused me to be drenched by the rain.' (kWlAG2014, 157) \\
\end{exe}

\begin{exe}
\ex \label{ex:tokAnWmqajndZic.tCe.rcanW}
 \gll to-k-ɤnɯmqaj-ndʑi-ci tɕe rcanɯ, ʑɯrɯʑɤri tɕe ko-k-ɤndɯndo-ndʑi-ci, \\
 \textsc{ifr}-\textsc{evd}-\textsc{recip}:scold-\textsc{du-evd} \textsc{lnk}  \textsc{foc}:\textsc{unexp} progressively \textsc{lnk}   \textsc{ifr}-\textsc{evd}-\textsc{recip}:take-\textsc{du-evd} \\
 \glt `They scolded each other and progressively started to fight, ' (lWlu2002, 52)
\end{exe}

 \begin{exe}
\ex \label{ex:mbro.rcanW}
 \gll mbro rcanɯ ɯ-xɕɤt kɯ-tɯ\redp{}tu ʑo nɯ-ntsʰɤr ɲɯ-nu, \\
 horse \textsc{foc}:\textsc{unexp} \textsc{3sg}.\textsc{poss}-strength \textsc{nmlz}:S/A-\textsc{emph}\redp{}exist \textsc{emph} \textsc{pfv}-neigh \textsc{sens}-be \\ 
 \glt `The horse neighed with all his strength.' (qachGa2003, 158)
\end{exe}

The marker \forme{rcanɯ} is particularly common in the degree construction with a \forme{tɯ-} degree nominal (§ XXX), as in (\ref{ex:apWme.rcanW}). In this particular construction,  \forme{rcanɯ} does not necessarily express unexpectedness.

\begin{exe}
\ex \label{ex:apWme.rcanW}
 \gll  tɕe nɯnɯ lɯlu a-pɯ-me rcanɯ, βʑɯ ɯ-tɯ-ŋɤn saχaʁ. \\
 \textsc{lnk} \textsc{dem} cat \textsc{irr}-\textsc{ipfv}-not.exist \textsc{foc}:\textsc{unexp} mouse
 \textsc{3sg}.\textsc{poss}-\textsc{nmlz}:\textsc{degree}-be.evil be.extremely:\textsc{fact} \\ 
 \glt `If there are no cats, the mice are extremely fierce (cause a lot of damages).' (21-lWlu, 32) 
\end{exe}

 \subsubsection{Additive and scalar focus marker \forme{kɯnɤ} } \label{sec:kWnA}
The additive and scalar focus marker \japhug{kɯnɤ}{also, even} follows the constituent over which it has scope, which can be noun phrases, postpositional phrases but also subordinate clauses (these are treated in § XXX). The stress is on the first syllable (\forme{kɯ́nɤ}) and the vowel on the second syllable is often elited (a pronunciation \forme{kɯn} is often heard).

The marker \forme{kɯnɤ} expresses both additive focus, as in (\ref{ex:aZo.kWNA.staRlupa}), and scalar focus, as in (\ref{ex:WNgWz.kWnA.tunAndWtnW}) in affirmative sentences. It is also compatible with negative verb forms, as in (\ref{ex:tWrdoR.kWnA}), expressing the meaning `not even' (see also \japhug{cinɤ}{(not) even one} in § \ref{sec:cinA}).

\begin{exe}
\ex \label{ex:aZo.kWNA.staRlupa}
\gll aʑo kɯnɤ staʁlupa ŋu-a tɕe \\
\textsc{1sg} also born.in.the.tiger.year be:\textsc{fact}-\textsc{1sg} \textsc{lnk} \\
\glt `Me too (like you), I am of the Tiger year.' (2011-05-nyima, 168)
\end{exe}

\begin{exe}
\ex \label{ex:WNgWz.kWnA.tunAndWtnW}
\gll ʑara ʑo ɯ-ŋgɯz kɯnɤ tu-nɤndɯt-nɯ tɕe nɯ kɯ-βʁa ɣɤʑu, kɯ-nŋo ɣɤʑu qʰe, \\
\textsc{3pl} \textsc{emph} \textsc{3sg}.\textsc{poss}-among:\textsc{loc} also \textsc{ipfv}-fight-\textsc{pl} \textsc{lnk} \textsc{dem} \textsc{nmlz}:S/A-win \textsc{sens}:exist \textsc{nmlz}:S/A-lose  \textsc{sens}:exist \textsc{lnk} \\
\glt `Even among themselves, they fight, and there are winners and losers.' (20-sWNgi, 62-63)
\end{exe}
 
\begin{exe}
\ex \label{ex:tWrdoR.kWnA}
\gll tɯ-sŋi mɯntoʁ tɯ-rdoʁ kɯnɤ ci ci tɕe mɯ́j-stʰɯt \\
one-day flower one-piece also once once \textsc{lnk} \textsc{neg}:\textsc{sens}-finish \\
\glt `Sometimes one cannot finish even one pattern (on the belt) in one day.' (2011-06-thaXtsa, 47)
\end{exe}

As an additive focus marker, \forme{kɯnɤ} can be repeated on all the nouns designating the members of a group sharing a particular property, in the construction $X$ \forme{kɯnɤ}, $Y$ \forme{kɯnɤ}  `both $X$ and $Y$', as in (\ref{ex:Dpalcan.kWnA}).

\begin{exe}
\ex \label{ex:Dpalcan.kWnA}
 \gll a-pɯ-ŋu tɕe, aʑo kɯnɤ taʁrdo rɟitpa a-pɯ-ŋu-a, χpɤltɕin kɯnɤ taʁrdo rɟitpa a-pɯ-ŋu, ... nɯ tɕi-rɟit nɯni tɕe taʁrdo rɟitpa ma nɯ ma kɯmaʁ rɟitpa nɯ kɤ-rtsi me.  \\
 \textsc{irr}-\textsc{ipfv}-be \textsc{lnk} \textsc{1sg} also pl.n. lineage  \textsc{irr}-\textsc{ipfv}-be-\textsc{1sg}  p.n. also pl.n. lineage  \textsc{irr}-\textsc{ipfv}-be { } \textsc{dem} \textsc{1du}.\textsc{poss}-offspring \textsc{dem}:\textsc{du} \textsc{lnk} pl.n. lineage \textsc{lnk} \textsc{dem} apart.from other lineage \textsc{dem} \textsc{nmlz}:O-count not.exist:\textsc{fact} \\
 \glt `For instance suppose that both Dpalcan and I were from Taqrdo lineage, then our two children would only count as members of the Taqrdo lineage and no other lineage.' (140426 rJitpa, 13-15)
\end{exe}

The scope of  \forme{kɯnɤ} is generally exclusively on the constituent that it immediately follows, but there are cases where the scope is more extensive. In (\ref{ex:aZo.kWnA.akAsWso}), \forme{kɯnɤ} occurs between the pronoun \forme{aʑo} and the following participial verb form, which bears a \textsc{1sg} possessive prefix \forme{a-} coreferent with that pronoun (see also \ref{ex:aZWG.kWnA} below). The semantic scope of \forme{kɯnɤ} here is on the whole relative \forme{aʑo a-kɤ-sɯso} `(the things) that I want' rather than exclusively on the pronoun \forme{aʑo}.

\begin{exe}
\ex \label{ex:aZo.kWnA.akAsWso}
 \gll aʑo kɯnɤ a-kɤ-sɯso nɯ tɤ-stu-nɯ ra \\
 \textsc{1sg} also \textsc{1sg}.\textsc{poss}-\textsc{nmlz}:O-think \textsc{dem} \textsc{imp}-do.like-\textsc{pl} have.to:\textsc{fact} \\
 \glt `(I will do as you say, but) do also the things I want.' (2003kAndzwsqhaj2, 47)
\end{exe}

The focus marker \forme{kɯnɤ} is found with nouns or pronouns in core argument function, including S (\ref{ex:kWnA.nArca}), O (\ref{ex:nWXpWm.kWnA}), and semi-objects (\ref{ex:kWnA.mAsna}).  Examples with transitive subjects are presented below (\ref{ex:nWra.kWnA} and \ref{ex:Wzda.ra.kWnA}).

 \begin{exe}
\ex \label{ex:kWnA.nArca}
\gll aʑo kɯnɤ nɤ-rca ɣi-a ɕti  \\
\textsc{1sg} also \textsc{2sg}.\textsc{poss}-following come:\textsc{fact}-\textsc{1sg} be.\textsc{affirm}:\textsc{fact} \\
\glt `I am coming with you too.' (2011-05-nyima, 171)
 \end{exe}
 
   \begin{exe}
\ex \label{ex:nWXpWm.kWnA}
\gll    ma nɯ-χpɯm kɯnɤ kʰro mɤ-kɯ-fkaβ kɯ-fse ku-rɤʑi-nɯ  \\
lnk 3pl.poss-knee also much \textsc{neg}-\textsc{nmlz}:S/A-cover \textsc{nmlz}:S/A-be.like \textsc{ipfv}-stay-\textsc{pl} \\
\glt `(Gents) would (wear trousers that did) not cover much even their knees.'  (30-rkAsnom, 5) 
  \end{exe}
  
  \begin{exe}
 \ex \label{ex:kWnA.mAsna}
 \gll   ɯ-ru nɯra laʁdɯn ɯ-jɯ kɯnɤ mɤ-sna, ma mɤ-ngɯt. \\
 \textsc{3sg}.\textsc{poss}-trunk \textsc{dem}:\textsc{pl} tool \textsc{3sg}.\textsc{poss}-handle also \textsc{neg}-be.worth \textsc{lnk}  \textsc{neg}-be.strong:\textsc{fact} \\
 \glt `(The wood from) its trunk is not even good (enough to be used to make) tool handles, as it is not strong.'  (17-xCAj, 79)
  \end{exe}

It also occurs with all types of oblique arguments and adjuncts, including genitive (\ref{ex:aZWG.kWnA}), dative (\forme{ɯ-ɕki} \ref{ex:nWCki.kWnA}),  locational adjuncts in \forme{tɕu} (\ref{ex:kutCu.kWnA}) or \forme{ri} (\ref{ex:ri.kWnA}), temporal adjuncts (\ref{ex:ftCAXcAl.kWnA}) or adjuncts expressing manner or cause (\ref{ex:nWtCu.kWnA}).  
  
   \begin{exe}
\ex \label{ex:aZWG.kWnA}
\gll aʑɯɣ kɯnɤ a-mpʰrɯmɯ a-pɯ-tɯ-sɯ-re ɯ-tɯ́-cʰa \\
\textsc{1sg}:\textsc{gen} also \textsc{1sg}.\textsc{poss}-divination \textsc{irr}-\textsc{pfv}-2-\textsc{caus}-look[III] \textsc{qu}-2-can:\textsc{fact} \\
\glt `Can you ask (the monk) to make a divination for me too?' (The divination, 31)
\end{exe}  
  
   \begin{exe}
\ex \label{ex:nWCki.kWnA}
\gll  tɯ-pi ɣɯ ɯ-nmaʁ ra nɯ-ɕki kɯnɤ `a-pi' tu-kɯ-ti ɕti ma nɯ ma kupa kɯ-fse ʑaka ɯ-rmi me. \\
\textsc{genr}.\textsc{poss}-elder.sibling \textsc{gen} \textsc{3sg}.\textsc{poss}-husband \textsc{pl} \textsc{3pl}.\textsc{poss}-\textsc{dat} also \textsc{1sg}.\textsc{poss}-elder.sibling \textsc{ipfv}-\textsc{genr}-say be.\textsc{affirm}:\textsc{fact} \textsc{lnk} \textsc{dem} apart.from Chinese \textsc{nmlz}:S/A-be.like each \textsc{3sg}.\textsc{poss}-name not.exist:\textsc{fact} \\
\glt  `One calls one's sister's husband (and others from his family) `my elder brother', there are no other special terms as in Chinese.' (140425 kWmdza05)
\end{exe}


  \begin{exe}
\ex \label{ex:kutCu.kWnA}
\gll  kutɕu kɯnɤ nɯ ɲɯ-fse, jɯfɕɯndʐi ra kɯ-xtɕɯ\redp{}xtɕi tɤ-ɣɤndʐo kɯ-fse ri, ɕɤxɕo tɕe kɯ-xtɕɯ\redp{}xtɕi ɲɯ-ʑi kɯ-fse \\
here also \textsc{dem} \textsc{sens}-be.like a.few.days.ago \textsc{nmlz}:S/A-\textsc{emph}\redp{}be.small \textsc{pfv}-be.cold \textsc{nmlz}:S/A-be.like \textsc{lnk} the.last.days \textsc{lnk} \textsc{nmlz}:S/A-\textsc{emph}\redp{}be.small \textsc{sens}-subside \textsc{nmlz}:S/A-be.like \\
\glt `It is like that here too, a few days ago the weather became a little cold, but the last days it has eased a bit.' (conversation, 141027)
  \end{exe}
  
    \begin{exe}
\ex \label{ex:ri.kWnA}
\gll   maldzɯ nɯ, nɯ ɯ-tʰɤcu tsa ri kɯnɤ ɣɤʑu. qarɣɤpɤt ɯ-rca ri kɯnɤ tu-ɬoʁ ɲɯ-ŋu. \\
plant.name \textsc{dem} \textsc{dem} \textsc{3sg}.\textsc{poss}-downstream a.little \textsc{loc} also exist:\textsc{sens} plant.name \textsc{3sg}.\textsc{poss}-among \textsc{loc} also \textsc{ipfv}-come.out \textsc{sens}-be \\
\glt `The \forme{maldzɯ} plant, it is also found in places of slightly lower altitude, but grows also in the same places as  \forme{qarɣɤpɤt} plants.' (18-qromJoR, 81-82)
    \end{exe}
    
\begin{exe}
\ex \label{ex:ftCAXcAl.kWnA}
\gll   kukutɕu ftɕɤχcɤl kɯnɤ <baonuanyi> tu-tɯ-ŋge pɯ-ɕti. \\
  here mid.summer also warm.clothes \textsc{ipfv}-2-wear[III] \textsc{pst}.\textsc{ipfv}-be.\textsc{affirm} \\
  \glt `Here you were wearing warm clothes even in mid summer.' (conversation, 141017)
    \end{exe}
    
    \begin{exe}
\ex \label{ex:nWtCu.kWnA}
\gll    tɕe nɯtɕu kɯnɤ ɯ-jaʁ ɯ-ntsi tɤɲi pjɯ-sɤtse, ɯ-jaʁ ɯ-ntsi kɯ tsʰitsuku ɲɯ-z-nɤme qʰe, \\
\textsc{lnk} \textsc{dem}:\textsc{loc} also \textsc{3sg}.\textsc{poss}-hand \textsc{3sg}.\textsc{poss}-one.of.a.pair erg various.things \textsc{ipfv}-\textsc{caus}-do[III] \textsc{lnk}  \\
\glt `Even like that (despite the pain in her legs), she props herself with a cane using one hand, and does all kinds of things with her other hand.' (14-tApitaRi, 52)
\end{exe}

Although \japhug{kɯnɤ}{also, even} can be combined with most postpositions and relator nouns as shown by the examples above, it is however incompatible with the ergative \forme{kɯ}. For instance, in  (\ref{ex:nWra.kWnA}), although the demonstrative pronoun \forme{nɯra} `they, those' in the second clause is the subject of the transitive verb \japhug{ndza}{eat}, it does not take the ergative \forme{kɯ} as would be expected (§ \ref{sec:A.kW}). The same applies to \forme{ɯ-zda ra} `his companions', subject of the transitive verb \forme{na-nɯ-ɕar-nɯ} `they looked for themselves' in (\ref{ex:Wzda.ra.kWnA}), 

  \begin{exe}
\ex \label{ex:nWra.kWnA}
\gll ɯ-pɯ nɯra li ju-ɣi-nɯ qʰe, nɯra kɯnɤ ɣɯ-tu-ndza-nɯ. \\
\textsc{3sg}.\textsc{poss}-young \textsc{dem}:\textsc{pl} again \textsc{ipfv}-come-\textsc{pl} \textsc{lnk} \textsc{dem}:\textsc{pl} also \textsc{cisloc}-\textsc{ipfv}-eat-\textsc{pl} \\
\glt `Its youngs also come and they too eat it.' (20-sWNgi, 59-60)
  \end{exe}
  
    \begin{exe}
\ex \label{ex:Wzda.ra.kWnA}
\gll   ɯ-zda ra kɯnɤ nɯ-rʑaβ tɯka na-nɯ-ɕar-nɯ ɲɯ-ŋu \\
\textsc{3sg}.\textsc{poss}-companion \textsc{pl} also \textsc{3sg}.\textsc{poss}-wife each \textsc{pfv}:3\fl{}3'-\textsc{auto}-search \textsc{sens}-be \\
\glt `His companions also took each a wife for himself (among the women of the island).' (2005Norbzang, 44)
    \end{exe}
    
The combinations $\dagger$\forme{kɯ kɯnɤ} or $\dagger$\forme{kɯnɤ kɯ} are unattested, and not accepted by native speakers. The contrast between absolutive and ergative noun phrases is therefore neutralized in additive or scalar focus with \forme{kɯnɤ}. Note that other focus markers, such as \forme{ri} and \forme{tɕi} (see \ref{ex:tCi.ndze} in § \ref{sec:ri.additive}) differ from \forme{kɯnɤ} in this regard.

Four distinct facts converge to suggest that the first syllable of \forme{kɯnɤ} is historically related to the ergative postposition \forme{kɯ}: (i) the incompatibility of co-occurrence of \forme{kɯnɤ} and \forme{kɯ}; (ii) the stress on the first syllable in \forme{kɯ́nɤ}; (iii) the similar \forme{-nɤ} element in the other scalar focus marker \japhug{cinɤ}{(not) even one} (§ \ref{sec:cinA}) (iv) the existence of the linker \forme{nɤ}, possibly of Tibetan origin (§ XXX). A detailed examination of this topic is however impossible on the basis Japhug-internal evidence, and will require extensive syntactic comparison between Gyalrong languages.

 \subsubsection{Correlative additive focus markers \forme{ri} and \forme{tɕi}} \label{sec:ri.additive} 
 The additive focus markers \forme{ri} and \forme{tɕi}  are used in enumerations, repeated after each noun referring to  members of a group, to focus on the fact that their referents share a common property (or properties that are semantically close enough), as in (\ref{ex:ri.kWsthWci.WWmpCar}) and (\ref{ex:tCi.tulhoR.cha}) (see additional examples in \citealt[313-314]{jacques14linking}).
 
 \begin{exe}
\ex \label{ex:ri.kWsthWci.WWmpCar}
 \gll  a-rʑaβ ri kɯstʰɯci ɲɯ-mpɕɤr, a-mbro ri kɯstʰɯci ɲɯ-ʑru, a-pɣɤtɕɯ ri kɯstʰɯci ɲɯ-mpɕɤr tɕe, \\
 \textsc{1sg}.\textsc{poss}-wife also so.much \textsc{sens}-be.beautiful  \textsc{1sg}.\textsc{poss}-horse also so.much \textsc{sens}-be.strong  \textsc{1sg}.\textsc{poss}-bird also so.much \textsc{sens}-be.beautiful \textsc{lnk} \\
 \glt `My wife is so beautiful, my horse so strong, my bird so beautiful.' (2003qachga, 116)
 \end{exe}
 
  \begin{exe}
\ex \label{ex:tCi.tulhoR.cha}
 \gll  ɴqiaβ tɕi tu-ɬoʁ cʰa, zrɯ tɕi tu-ɬoʁ cʰa, \\
 dark.side.of.the.mountain also \textsc{ipfv}-come.out can:\textsc{fact}   sunny.side.of.the.mountain also \textsc{ipfv}-come.out can:\textsc{fact}  \\
 \glt `It can grow in both the dark and the sunny sides of the mountains.' (17-thowum, 14)
  \end{exe}
  
The correlative focus markers \forme{ri} and \forme{tɕi} can occur after any noun phrase or postpositional phrase, including with the ergative  \forme{kɯ} as shown by (\ref{ex:tCi.ndze}), unlike the marker \japhug{kɯnɤ}{even, also} (see examples \ref{ex:nWra.kWnA} and \ref{ex:Wzda.ra.kWnA}, § \ref{sec:kWnA}).
  
  \begin{exe}
\ex \label{ex:tCi.ndze}
 \gll paʁ kɯ tɕi ndze, nɯŋa kɯ tɕi ndze, jla kɯ tɕi ndze.   \\
 pig \textsc{erg} also eat[III]:\textsc{fact}  cow \textsc{erg} also eat[III]:\textsc{fact}  hybrid.yak \textsc{erg} also eat[III]:\textsc{fact}  \\
 \glt `Pigs eat it, cows eat it, hybrid yaks eat it.' (18-NGolo, 171)
  \end{exe}

The focus markers \forme{ri} and \forme{tɕi} can have scope on only part of the noun/propositional phrase, and even on the relator nouns as in (\ref{ex:WNgW.tCi}).

   \begin{exe}
\ex \label{ex:WNgW.tCi}
 \gll   sɤtɕʰa ɯ-ŋgɯ tɕi ɣɤʑu, sɤtɕʰa ɯ-taʁ tɕi ʑo ɣɤʑu \\
 ground \textsc{3sg}.\textsc{poss}-inside also exist:\textsc{sens}  ground \textsc{3sg}.\textsc{poss}-inside also \textsc{emph} exist:\textsc{sens} \\
 \glt `It is found both inside the ground, and on the ground.' (25-GdAso, 17)
    \end{exe}
    
Alternatively, it is possible to enumerate distinct related properties of the same referent using \forme{ri} (this usage is not found with \forme{tɕi}), but that marker still follows the noun phrase (correlative \forme{ri} can follow verbs, but only in a specific construction, see \ref{ex:ri.kWmWm.ri} below). In this case the referent cannot be elided, and must be repeated in both clauses, at least as a third person pronoun \forme{ɯʑo} as in (\ref{ex:WlWz.ri.pjAxtCi}). 

  \begin{exe}
\ex \label{ex:WlWz.ri.pjArZi}
 \gll pʰaʁrgot nɯnɯ ɯʑo ri pjɤ-rʑi, ɯʑo ri pjɤ-tsʰu tɕe \\
 boar \textsc{dem} \textsc{3sg} also \textsc{ifr}.\textsc{ipfv}-be.heavy \textsc{3sg} also \textsc{ifr}.\textsc{ipfv}-be.fat \textsc{lnk} \\ 
\glt  `The boar, it was heavy and fat.' (140428 yonggan de xiaocaifeng-zh, 244)
 \end{exe}

A variant of this construction is found with internally-headed relative clauses in apposition, taking the third person pronoun \forme{ɯʑo} as head, as in (\ref{ex:WZo.ri.kWwxti}).

\begin{exe}
\ex \label{ex:WZo.ri.kWwxti}
\gll  [ɯʑo ri kɯ-wxti], [ɯʑo ri kɯ-sɤjlɯ\redp{}jloʁ] ci pjɤ-ŋu. \\
\textsc{3sg} also \textsc{nmlz}:S/A-be.big \textsc{3sg} also \textsc{nmlz}:S/A-\textsc{emph}\redp{}be.big \textsc{indef} \textsc{ifr}.\textsc{ipfv}-be \\
\glt `(The toad) was a big and disgusting (creature).' (150818 muzhi guniang, 86)
\end{exe}

 
The correlative construction can involve the possessor of an IPN, as in (\ref{ex:WlWz.ri.pjAxtCi}), where in the first clause the referent `the girl' is possessor of the intransitive subject (literally `her age was small', § XXX) and in second it corresponds to the intransitive subject, realized as a third person pronoun \forme{ɯʑo} `she'.

  \begin{exe}
\ex \label{ex:WlWz.ri.pjAxtCi}
 \gll tɕʰeme nɯ ɯ-lɯz ri pjɤ-xtɕi, ɯʑo ri pjɤ-mpɕɤr,  \\
 girl \textsc{dem} \textsc{3sg}.\textsc{poss}-age also \textsc{ifr}.\textsc{ipfv}-be.small \textsc{3sg} also \textsc{ifr}.\textsc{ipfv}-be.beautiful \\
\glt `The girl was young and beautiful.' (150909 hua pi-zh, 10)
 \end{exe}
 
 More complex correlations, involving different subjects and predicates related to another referent, are also possible as shown by example (\ref{ex:lWlu.kW}), where \forme{ri} occurs after the intransitive subject \japhug{tɯ-ci}{water}, after the transitive subject \japhug{lɯlu}{cat} with the ergative and after the finite verb \japhug{tu-ɕe}{it goes up} (on which see below and refer to § XXX).
 
 \begin{exe}
\ex   \label{ex:lWlu.kW}
\gll <yancong> ku-kɯ-rɤloʁ tɕe ɯ-taʁ tɯ-ci ri mɯ́j-ɣi lɯlu kɯ ri mɯ-ɲɯ́-wɣ-ɕaβ qapri tu-ɕe ri mɯ́j-cʰa tɕe \\
 chimney \textsc{ipfv}-\textsc{genr}:S/P-make.a.nest \textsc{lnk} \textsc{3sg}.\textsc{poss}-on \textsc{indef}.\textsc{poss}-water also \textsc{neg}:\textsc{sens}-come cat \textsc{erg} also \textsc{neg}-\textsc{ipfv}-\textsc{inv}-catch snake \textsc{ipfv}:\textsc{up}-go also \textsc{neg}:\textsc{sens}-can \textsc{lnk} \\
 \glt `(The sparrows) make their nest in the chimney, (because) water cannot come up there, the cats cannot catch them, and the snakes cannot go up there.' (22-kumpGatCW, 69)
 \end{exe}
 
 The marker \forme{ri} is homophonous with the locative \forme{ri} (§ \ref{sec:locative}), and in cases with an enumeration of locative adjuncts, there can be ambiguity between the two. In (\ref{ex:Xcha.ri.ci}), \forme{ri} is analyzed as a locative because of the position of the determiner \forme{ci}, and also because it can be replaced with other locative postpositions.
 
 \begin{exe}
\ex \label{ex:Xcha.ri.ci}
\gll   χcʰa ri ci, ɯ-ʁe ri ci ɯ-jme cʰɯ-ɬoʁ ɲɯ-ŋu. \\
right \textsc{loc} one  \textsc{3sg}.\textsc{poss}-left \textsc{loc} one \textsc{3sg}.\textsc{poss}-tail \textsc{ipfv}:\textsc{downstream}-come.out \textsc{sens}-be \\
\glt `It has one tail on the right, and one on the left.' (26-qro, 116)
\end{exe}

The marker \forme{ri} can follow verbs only if combined with an existential verb, a copula or a modal auxiliary verb as main predicate (meaning `both $X$ and $Y$' with positive copulas, and `neither $X$ nor $Y$' with negative ones). In this type of construction, verbs are mostly in non-finite form, as in (\ref{ex:ri.kWmWm.ri}). Examples with finite verbs however do exist; this topic is treated in § XXX. %ɲɯ-ɣɤwu ri kɯ-maʁ, ɲɯ-nɤre ri kɯ-maʁ kɯ-fse ɲɤ-k-ɤβzu-ci  ; tu-rɯɕmi ri mɤ-kɯ-khɯ, chɯ-nɯrɤɣo ri mɤ-kɯ-khɯ ci ɲɤ-k-ɤβzu-ci. ; tu-ndzur ri pjɤ-maʁ, ku-omdzɯ ri pjɤ-maʁ.

 \begin{exe}
\ex \label{ex:ri.kWmWm.ri}
 \gll   nɯ pɯ́-wɣ-ta ri  kɯroz kɯ-mɯm ri maŋe, kɯroz mɤ-kɯ-ɣɤ-mɲɤt ri maŋe qʰe, \\
 \textsc{dem} \textsc{pfv}-\textsc{inv}-put \textsc{lnk} specially \textsc{nmlz}:S/A-be.tasty also not.exist:\textsc{sens} specially \textsc{neg}-\textsc{nmlz}:S/A-\textsc{facil}-be.spoiled also not.exist:\textsc{sens} \textsc{lnk} \\
 \glt `When if one puts (a seal on the bread), there is nothing especially tasty about it, and nothing special concerning the preservation (of the bread).' (160706 thotsi, 27)
  \end{exe}
  

  
 \subsubsection{Scalar focus marker \forme{cinɤ}} \label{sec:cinA} 
 The focus marker \japhug{cinɤ}{(not) even one} exclusively occurs with a negative verb. Like \japhug{kɯnɤ}{also, even}, this marker has stress on the first syllable \forme{cínɤ}, which is obviously related to the numeral \japhug{ci}{one} (§ \ref{sec:one.to.ten}, § \ref{sec:indef.article}).
 
 The marker \forme{cinɤ} has scope over the constituent that immediately precedes it, generally a noun phrase including or consisting of a CN, as in (\ref{ex:tWrdoR.cinA3}), but also object and subject participial relative clauses as in (\ref{ex:zrWG.kAmto.cinA}), (\ref{ex:WrNa.WkWru.cinA}) and (\ref{ex:lukWpGaR.nW.cinA}).
 
 \begin{exe}
\ex \label{ex:tWrdoR.cinA3}
\gll tsuku kɯ qʰe tɯ-rdoʁ cinɤ mɤ-kɯ-mto tu. \\
some erg lnk one-piece even neg-nmlz:S/A-see exist:fact \\
\glt `There are some people who (cannot) even find a single one.' (20-grWBgrWB, 36)
 \end{exe} 

 \begin{exe}
\ex \label{ex:zrWG.kAmto.cinA}
\gll  ma tɕe jinde nɯ zrɯɣ kɤ-mto cinɤ maŋe. \\
\textsc{lnk} \textsc{lnk} nowadays \textsc{dem} louse \textsc{nmlz:P}-see even not.exist:\textsc{sens} \\
\glt `Nowadays there isn't even a single louse to be seen/one cannot even see a single louse.' (21-mdzadi, 77)
\end{exe} 

\begin{exe}
\ex \label{ex:WrNa.WkWru.cinA}
\gll ɯ-rŋa ɯ-kɯ-ru cinɤ ʑo pjɤ-me \\
3sg.poss-face 3sg.poss-nmlz:S/A-look even \textsc{emph} \textsc{ipfv}.\textsc{ifr}-not.exist \\
\glt `Not even one (of the thieves) looked at it/The (thieves) did not even so much as looked at it.' (140426 luozi he qiangdao)
\end{exe}

\begin{exe}
\ex \label{ex:lukWpGaR.nW.cinA}
\gll tɕe ɯ-ɲɯ-kɯ-ɣɤ-rkɯn nɯ ɲɯ-dɤn ma lu-kɯ-pɣaʁ nɯ tɯ-rdoʁ cinɤ ʑo maŋe \\
\textsc{lnk} \textsc{3sg}.\textsc{poss}-\textsc{ipfv}-\textsc{nmlz}:S/A-\textsc{caus}-be.few \textsc{dem} \textsc{sens}-be.many \textsc{lnk} \textsc{ipfv}:\textsc{upstream}-\textsc{nmlz}:S/A-plough \textsc{dem} one-piece even \textsc{emph} not.exist:\textsc{sens} \\
\glt `A lot of people diminish their fields, and not a single of them opens new fields.' (150903 friche, 6)
\end{exe}

In the case of relative clauses before \forme{cinɤ}, there is some ambiguity as to whether the scope of the focus marker is on the head of the relative or on the main verb of the relative clause, hence the two proposed translations above for (\ref{ex:zrWG.kAmto.cinA}) and (\ref{ex:WrNa.WkWru.cinA}).

It is not possible to use \forme{cinɤ} with scope over transitive subjects, followed by the ergative.

The form \forme{cinɤ} also occurs in the expression \forme{ŋu cinɤ maʁ kɯ} `in any case it is not', as in (\ref{ex:Nu.cinA.maR.kW}), literally `It is not even the case that...' ; in this construction, only the first verb \japhug{ŋu}{be} receives person indexation, as shown by (\ref{ex:Nua.cinA.maR.kW}). In addition to \japhug{ŋu}{be}, a few other verbs such as \japhug{fse}{be like} can occur with \forme{ci nɤ maʁ kɯ} `anyway X does not' .

 \begin{exe}
\ex \label{ex:Nu.cinA.maR.kW}
\gll qajdo kɯ tɕʰi mɤ-nɯ-ti ɕti nɤ, a-tɤ-nɯ-ti ma ŋu cinɤ maʁ kɯ, nɯ sɤznɤ kɯ-scɯ-scit rɤʑi-tɕi \\
crow \textsc{erg} what \textsc{neg}-\textsc{auto}-say:\textsc{fact} be.\textsc{affirm}:\textsc{fact} \textsc{lnk} \textsc{irr}-\textsc{pfv}-\textsc{auto}-say \textsc{lnk} be:\textsc{fact} even not.be:\textsc{fact} \textsc{sfp} \textsc{dem} \textsc{comp} \textsc{nmlz}:S/A-\textsc{emph}\redp{}happy stay:\textsc{fact}-\textsc{1du} \\
\glt `What would not a crow say (a crow tells only lies), let it say as it wants, in any case it is not (true), let us rather live (together) happily.' (28-qAjdoskAt, 28)
\end{exe} 

 \begin{exe}
\ex \label{ex:Nua.cinA.maR.kW}
\gll  kɯ-mɯrkɯ ŋu-a cinɤ maʁ kɯ  \\
\textsc{nmlz}:S/A-steal be:\textsc{fact}-\textsc{1sg} even not.be \textsc{sfp} \\
\glt `Anyway it is not me who is the thief.' (elicited)
\end{exe}

\subsubsection{Restrictive focus} \label{sec:restrictive.focus} 
 The most common way to express restrictive focus in Japhug is to combine the exceptive \japhug{ma}{apart from} (and its reduplicated variant \forme{mɯma} § \ref{sec:exceptive}) with a negative predicate. This can be a verb with a negative prefix as in (\ref{ex:XsArZaR}), or a negative existential verb as in (\ref{ex:Wmi.Wntsi.ma.me}).
 
 \begin{exe}
\ex  \label{ex:XsArZaR}
\gll   χsɤ-rʑaʁ ma mɯ-pɯ-tsu-a ɲɤ-sɯso ri χsɯ-xpa pjɤ-tsu tɕe,  \\
three-day apart.from \textsc{neg}-\textsc{pfv}-pass-\textsc{1sg} \textsc{ifr}-think \textsc{lnk} three-year \textsc{ifr}-pass \textsc{lnk} \\
\glt `He thought that he had spent only three days, but three years had passed.' (2011-4-smanmi, 178)
  \end{exe}
  
  \begin{exe}
\ex  \label{ex:Wmi.Wntsi.ma.me}
\gll  rkoŋɟɤl nɯnɯ, ɯ-mi ɯ-ntsi nɯ ma me kʰi.   \\
one.legged.demon \textsc{dem} \textsc{3sg}.\textsc{poss}-leg \textsc{3sg}.\textsc{poss}-one.of.a.pair \textsc{dem} apart.from not.exist:\textsc{fact} \textsc{hearsay} \\
\glt  `It is said that one-legged demons only had one leg.' (140510 rkoNJAl, 4)
  \end{exe}
  
The restrictive focus construction implies the presence of a noun phrase with a numeral or a CN when the restriction bears on the quantity, but restriction can also be qualitative, without quantifier, as in (\ref{ex:karGi.Zo.kWfse.ma.me}).

\begin{exe}
\ex \label{ex:karGi.Zo.kWfse.ma.me}
 \gll   ɯ-mat nɯnɯ na-lɤt ɕɯmɯma nɤ kɯ-ndɯ\redp{}ndɯβ ʑo ma me, karɣi ʑo kɯ-fse ma me  \\
 \textsc{3sg}.\textsc{poss}-fruit \textsc{dem} \textsc{pfv}:3\fl{}3'-throw just \textsc{lnk}  \textsc{nmlz}:S/A-\textsc{emph}\redp{}small \textsc{emph} apart.from not.exist:\textsc{fact} turnip.seed \textsc{emph} \textsc{nmlz}:S/A-be.like apart.from not.exist:\textsc{fact} \\
 \glt  `When the fruit of (xanthoxyllum) has just come out, there is only something very small, only like a turnip seed.'  (07-tCGom, 7)
  \end{exe}
  
The restrictive focus construction can be combined with a scalar focus in \forme{kɯnɤ} (see §  \ref{sec:kWnA}), as in (\ref{ex:ma.kWme.kWnA}). In this example, \forme{kɯnɤ} has scope over the subordinate clause \forme{stɯsti ma kɯ-me}, which is ambiguous between a participial headless relative (§ XXX) `consisting of only a female all alone' and a manner infinitival clause (§ XXX; in this case the gloss of \forme{kɯ-me} would be \textsc{inf}:\textsc{stat}-not.exist) `even (when) there is only a female all alone'.

  \begin{exe}
\ex \label{ex:ma.kWme.kWnA}
\gll  mu ma, stɯsti ma kɯ-me kɯnɤ cʰɯ-rɤŋgɯm ɲɯ-ɕti. \\
female apart.from alone apart.from \textsc{nmlz}:S/A-not.exist also \textsc{ipfv}-lay.eggs \textsc{sens}-be.\textsc{affirm} \\
\glt `Even only a female (hen) alone does lay eggs.' (150819 kumpGa, 11)
\end{exe}
   
A second possibility to express restrictive focus is the use of the adverb \japhug{ʁɟa}{completely, all} (§ XXX) with scope on a  noun phrase rather than the whole clause as in (\ref{ex:RJa.tunWndze}).\footnote{The form \forme{ʁɟa} possibly originates from the first syllable of Tibetan \tibet{གཡའ་མ་}{gja.ma}{stone slab}, through a meaning `bare rock'.}  

\begin{exe}
\ex \label{ex:stAmku.RJa}
\gll alo mbroχpa ra tɕe tɕe nɤki qra cʰo qambrɯ ra ɣɯ nɯ-ɣli nɯnɯ
tɕe nɯ tu-wum-nɯ, tu-sɯɣ-rom-nɯ mbroχpa sɤtɕʰa tɕe stɤmku ʁɟa ɲɯ-ɕti ma si maŋe tɕe tɕe    \\
upstream nomad \textsc{pl} \textsc{lnk} \textsc{lnk} \textsc{filler} female.yak \textsc{comit} male.yak \textsc{pl} \textsc{gen} \textsc{3pl}.\textsc{poss}-dung \textsc{dem} \textsc{lnk} \textsc{dem} \textsc{ipfv}-gather-\textsc{pl} \textsc{ipfv}-\textsc{caus}-be.dry-\textsc{pl} nomad place \textsc{lnk} grassland completely \textsc{sens}-be.\textsc{affirm} \textsc{lnk} tree not.exist:\textsc{sens} \textsc{lnk} \textsc{lnk}  \\
\glt `Upstream, in the nomad areas, they gather and dry yak dung, as in nomad places there is only grassland, there no trees.' (05-tamar, 7-10)
\end{exe}

The adverb \forme{ʁɟa} (here used rather as a noun modifier) is related to the denominal verb \japhug{aʁɟa}{be bald, be bare} (see § XXX on the \forme{a-} derivation), which can be applied to nouns such as \japhug{stɤmku}{grassland} and \japhug{zgo}{mountain}.
 
\begin{exe}
\ex \label{ex:RJa.tunWndze}
 \gll qajɯ ʁɟa tu-nɯ-ndze, ma nɯ ma tɤ-rɤku kɯ-fse ra ndze mɤ-ŋgrɤl. \\
 bug completely \textsc{ipfv}-\textsc{auto}-eat[III] \textsc{lnk} \textsc{dem} apart.from \textsc{indef}.\textsc{poss}-harvest \textsc{nmlz}:S/A-be.like \textsc{pl} eat[III]:\textsc{fact} \textsc{neg}-be.usually.the.case:\textsc{fact} \\ 
\glt `It only eats insects, it does not eat cultivated plants.' (140511 qamtsWrmdzu, 16)
\end{exe}

While in (\ref{ex:RJa.tunWndze})  and (\ref{ex:stAmku.RJa}) it remains ambiguous whether \forme{ʁɟa} forms a syntactic constituent with the previous nouns or the following verb, in (\ref{ex:RJa.kW}) the presence of the ergative makes it clear that \forme{ʁɟa} is not a clausal adverb, and belongs to the postpositional phrase headed by \forme{kɯ}.

\begin{exe}
\ex \label{ex:RJa.kW}
 \gll [tɤ-lu cʰo tɯkrimgo ʁɟa kɯ] cʰɯ-z-ɣɤ-wxti-nɯ. \\
 \textsc{indef}.\textsc{poss}-milk \textsc{comit} doughnut completely \textsc{erg} \textsc{ipfv}-\textsc{caus}-\textsc{caus}-be.big-\textsc{pl} \\
\glt `They (used to) raise up (the babies) by feeding them milk and doughnuts only.' (140426 tApAtso kAnWBdaR, 102)
\end{exe}

The same applies to (\ref{ex:Wru.RJa.nW}), where the presence of the demonstrative \forme{nɯ} after \forme{ʁɟa} shows that it belongs to the same noun phrase.

\begin{exe}
\ex \label{ex:Wru.RJa.nW}
 \gll ɯ-rdoʁ nɯ-me tɕe, [ɯ-ru ʁɟa nɯ], pɯ-kɤ-tɤβ nɯnɯ, taʁndzɤr ɯ-ŋgɯ tú-wɣ-rku tɕe, \\
 \textsc{3sg}.\textsc{poss}-grain \textsc{pfv}-not.exist \textsc{lnk} \textsc{3sg}.\textsc{poss}-stalk completely \textsc{dem} \textsc{pfv}-\textsc{nmlz}:P-thresh \textsc{dem} feeding.emmer \textsc{3sg}.\textsc{poss}-inside \textsc{ipfv}-\textsc{inv}-put.in \textsc{lnk} \\
 \glt `When all the grains have been removed, the bare stalks, the one that have been threshed, one puts them in a feeding emmer.' (140513 tWrtsi, 5)
\end{exe}

A reduplicated emphatic form \forme{ʁɟɯ\redp{}ʁɟa} is also found as in (\ref{ex:RJWRJa.kW})

\begin{exe}
\ex \label{ex:RJWRJa.kW}
 \gll χtɕɤnzɤn ʁɟɯ\redp{}ʁɟa kɯ ʑo pɯ́-wɣ-nɤjo ɕti ɲɯ-ŋu.  \\
beast \textsc{emph}\redp{}completely \textsc{erg} \textsc{emph} \textsc{pst}.\textsc{ipfv}-\textsc{inv}-wait be.\textsc{affirm}:\textsc{fact} \textsc{sens}-be \\
\glt `It was all wild beasts waiting for him (there).' (Norbzang2005, 308)
 \end{exe}
 
A third option to express restrictive focus is the IPN \forme{ɯ-jlu}, which is used in the meaning `uncooked' as a property IPN (§ \ref{sec:property.nouns}), but has become grammaticalized as a restrictive marker `exclusively, without anything else' (presumably from an intermediate meaning `plain'), as in (\ref{ex:Wjlu.Zo}).

\begin{exe}
\ex \label{ex:Wjlu.Zo}
 \gll srɤz nɯ kɯ tɕʰoz ɯ-jlu ʑo pjɯ-nɯjɤntɤn pɯ-ɕti ma jɯm nɯ mɯ-pjɤ-ɕar ɲɯ-ŋu, \\
prince \textsc{dem} \textsc{erg}  religion \textsc{3sg}.\textsc{poss}-exclusively \textsc{emph} \textsc{ipfv}-be.assiduous.in  \textsc{pst}.\textsc{ipfv}-be.\textsc{affirm} \textsc{lnk} wife \textsc{dem} \textsc{neg}-\textsc{ifr}.\textsc{ipfv}-look.for \textsc{sens}-be \\
 \glt `The prince was focused exclusively in the study of religion, and was not looking for a wife.' (sras2003, 3)
 \end{exe}

\subsection{Identity modifiers} \label{sec:identity.modifier}
There is no specific identity modifier `the same' in Japhug. The only way to express this meaning is to use the S-participle of the verb \japhug{naχtɕɯɣ}{be the same} (a denumeral verb of Tibetan origin, § \ref{sec:tibetan.numerals}, see also § \ref{sec:comitative} on the syntax of this stative verb and § XXX on its derivation) in a relative clause, as in (\ref{ex:tArmi.kWnaXtCWG}) (a possessor relative, § XXX). This participle is also used adverbially (see § XXX).

\begin{exe}
\ex \label{ex:tArmi.kWnaXtCWG}
\gll tɤ-rmi kɯ-naχtɕɯɣ pjɤ-dɤn wo kɤmɲɯ, nɤki kɯrɯ ra tɕe. \\
\textsc{indef}.\textsc{poss}-name \textsc{nmlz}:S/A-be.the.same \textsc{ifr}.\textsc{ipfv}-be.many \textsc{sfp} pl.n. \textsc{filler} Tibetan \textsc{pl} \textsc{lnk} \\
\glt `There were many people who had identical names, in Kamnyu, among the Tibetans.' (140522 tshupa, 161)
\end{exe}


There are two prenominal modifiers expressing non-identity in Japhug: \japhug{kɯmaʁ}{other} and the numeral \japhug{ci}{one}, which in prenominal position means `the other one' (in postnominal position, it is used as an indefinite article, see § \ref{sec:indef.article}). Both of these words can also be used as pronouns, though \forme{ci} requires to be combined with the demonstrative \forme{nɯ} in this usage (see § \ref{sec:other.pro}).

The modifier \forme{kɯmaʁ} is prenominal in its meaning `other', as in (\ref{ex:kWmaR.tWrme}). 

\begin{exe}
\ex \label{ex:kWmaR.tWrme}
\gll tɯ-zda nɯ ma kɯmaʁ tɯrme a-pɯ-me tɕe, kʰa ra aʁɤndɯndɤt ɲɯ-ɤ<nɯ>ɣro ɲɯ-ŋu ɲɯ-ti. \\
\textsc{genr}.\textsc{poss}-companion \textsc{dem} apart.from other person \textsc{irr}-\textsc{ipfv}-not.exist \textsc{lnk} house \textsc{pl} everywhere \textsc{ipfv}-<\textsc{auto}>play \textsc{sens}-be \textsc{sens}-say \\
\glt `(Our neighbour) says that if there are no other persons apart from family members, (the monkey) would play everywhere in the house.' (19-GzW2, 10)
\end{exe}

There are apparent examples of \japhug{kɯmaʁ}{other} in postnominal position, as in (\ref{ex:kWmaR.taXtW}) and (\ref{ex:kWmaR.tanWsWBzu}), but in such sentences \forme{kɯmaʁ} is a preverbal adverb, not a noun modifier, with a slightly different meaning `anew'. In (\ref{ex:kWmaR.taXtW}), the usage of \forme{kɯmaʁ} is very similar to its Chinese equivalent \ch{另外}{lìngwài}{other} in the corresponding Chinese sentence \zh{阿兰另外给我买了一部手机}, where the preverbal position of \ch{另外}{lìngwài}{other} clearly shows that it is not a noun modifier. 

\begin{exe}
\ex \label{ex:kWmaR.taXtW}
\gll <alan> kɯ a-<dianhua> kɯmaʁ ta-χtɯ \\
p.n. \textsc{erg} \textsc{1sg}.\textsc{poss}-phone other \textsc{pfv}:3\fl{}3'-buy \\
\glt `Alan bought me a new phone.' (conversation, 17-03-27)
\end{exe}

\begin{exe}
\ex \label{ex:kWmaR.tanWsWBzu}
\gll a-ʁi kɯ kʰa kɯmaʁ ta-nɯ-sɯ-βzu qʰe, \\
\textsc{1sg}.\textsc{poss}-younger.sibling \textsc{erg} house other \textsc{pfv}:3\fl{}3'-\textsc{auto}-\textsc{caus}-make \textsc{lnk} \\
\glt `My brother made himself a new house.' (14-tApitaRi, 304)
\end{exe}

The identity determiner \japhug{kɯmaʁ}{other} is grammaticalized from the S-participle of the verb \japhug{maʁ}{not be}, \forme{kɯ-maʁ} `who/which is not X', which is still widely used, as in (\ref{ex:tChWrtsAm.kWmaR}) and (\ref{ex:sthWci.kWmaR}).


%\begin{exe}
%\ex \label{ex:Wstu.kWmaR}
%\gll ɯ-stu kɯ-maʁ me, kɯki mɤ-kɯ-pe me \\
%\textsc{3sg}.\textsc{poss}-truth \textsc{nmlz}:S/A-not.be not.exist:\textsc{fact} dem.\textsc{prox} \textsc{neg}-\textsc{nmlz}:S/A-be.good not.exist:\textsc{fact} \\
%\glt ` (28-smAnmi, 16)
%\end{exe}

\begin{exe}
\ex \label{ex:tChWrtsAm.kWmaR}
\gll mɤʑɯ [tɕʰɯrtsɤm kɯ-maʁ] nɯnɯ tɕe, tú-wɣ-χtɕi ma nɯ ma kɤ-sqa (mɤ-ra) \\
yet type.of.tsampa \textsc{nmlz}:S/A-not.be \textsc{dem} \textsc{lnk} \textsc{ipfv}-\textsc{inv}-wash \textsc{lnk} \textsc{dem} apart.from \textsc{inf}-boil \textsc{neg}-have.to:\textsc{fact} \\
\glt `The tsampa that is not `chu.rtsam', one needs to wash it, but not to boil it.' (2002tWsqar, 112)
\end{exe}

\begin{exe}
\ex \label{ex:sthWci.kWmaR}
\gll  [ɯ-rkɯ wuma ʑo stʰɯci kɯ-maʁ] nɯtɕu tɤ-ri ci kú-wɣ-lɤt \\
\textsc{3sg}.\textsc{poss}-side really \textsc{emph} so.much \textsc{nmlz}:S/A-not.be \textsc{dem}:\textsc{loc} \textsc{indef}.\textsc{poss}-thread once \textsc{ipfv}-\textsc{inv}-throw \\
\glt `One sews a thread at a place which is not too much on the border (of the patch)'. (12-kAtsxWb, 16)
\end{exe}

The modifier \forme{ci} differs from \forme{kɯmaʁ} in that it is necessarily definite, meaning `the other one', as in (\ref{ex:ci.rWdaR}), where it refers to an animal that it chased by lions, which was previously mentioned in the text.

\begin{exe}
\ex \label{ex:ci.rWdaR}
\gll ʑɯrɯʑɤri qʰe ci rɯdaʁ nɯ dɯxpa ma nɯ-kɤ-ndza ɯ-spa ɲɯ-ɕti qʰe, qʰe pjɯ-ndʐaβ qʰe mɯ-ɲɯ-cʰa qʰe, \\
progressively \textsc{lnk} one animal \textsc{dem} poor.of \textsc{lnk} \textsc{3pl}.\textsc{poss}-\textsc{nmlz}:P-eat \textsc{3sg}.\textsc{poss}-material \textsc{sens}-be.affirm \textsc{lnk} \textsc{lnk} \textsc{ipfv}-\textsc{anticaus}:make.fall \textsc{lnk} \textsc{neg}-\textsc{ipfv}-can \textsc{lnk} \\
\glt `The other animal, poor of him, it is their prey, progressively it falls down and cannot stand it anymore.' (20-sWNgi, 43)
\end{exe}

Interestingly, the determiner \forme{ci} does not have scope over other noun modifiers. For instance, in (\ref{ex:ci.tCheme.kWNAn}), the noun \japhug{tɕʰeme}{woman} occurs with an attributive adjective in participial form \forme{kɯ-ŋɤn} `who is evil' (a relative clause, see § \ref{sec:attributes}), but the meaning is not `the other evil woman' as could have been expected (since the woman who is the subject of the sentence is, by contrast, a kind person), and rather must be `the other woman, the evil one'. There is no pause in the recording that could lead us to suppose that \forme{kɯ-ŋɤn} here is an apposition -- it is rather a postnominal relative.

\begin{exe}
\ex \label{ex:ci.tCheme.kWNAn}
\gll nɤki, tɕʰeme nɯ ɯ-ɕki ɯ-kɯ-sɤja jo-ɕe, ci tɕʰeme kɯ-ŋɤn nɯ ɯ-ɕki. \\
\textsc{filler} women \textsc{dem} \textsc{3sg}-\textsc{dat} \textsc{3sg}.\textsc{poss}-\textsc{nmlz}:S/A-give.back \textsc{ifr}-go one woman \textsc{nmlz}:S/A-be.evil \textsc{dem} \textsc{3sg}-\textsc{dat} \\
\glt `She went to give it back to the woman, the other one, the evil woman.' (140515 jiesu de laoren, 90)
\end{exe}

%a-ʁi kɯnɤ tɯrme kha kɤ-sɯxɕe mɯ́j-khɯ qhe,

%ci qhɤjmbaʁ nɯ kɯ-jaʁ kɯ-fse nɯnɯ 
%mtshalu ɯ-cu tɕe nɤki,
%tɯ-mgo zmɤrɤβ kú-wɣ-nɯ-lɤt sna.
%16-RlWmsWsi
%li ci /ɯt/ ɯ-tɯphu nɯ tɤpu qhɤjmbaʁ tu-ti-nɯ ŋu tɕe,
\subsection{Attributes} \label{sec:attributes}
%\ref{sec:property.nouns}
\subsubsection{Attributive postnominal modifiers} \label{ex:attributive.postnominal}
In addition to the postnominal markers studied above (numeral and number § \ref{sec:number.determiners}, demonstratives § \ref{sec:demonstrative.determiners}, quantifiers § \ref{sec:quantifiers.determiners}, definiteness markers § \ref{sec:indef.article}, topic and focus markers), there are a certain number of nouns that can serve a post-nominal modifiers.

An entire class of such nouns consists of the privative nouns in \forme{-lu} `...less', described in § \ref{sec:privative}.

The word \japhug{wuma}{real, really} from Tibetan \tibet{ངོ་མ་}{ŋo.ma}{real, true} is generally used adverbially as an intensifier, in particular with stative verbs (§ XXX), but also occurs as a postnominal modifier meaning ` real', its original meaning, as in (\ref{ex:lhAndzxi.wuma}).

\begin{exe}
\ex \label{ex:lhAndzxi.wuma}
\gll ɬɤndʐi wuma nɯ nɤʑo ɲɯ-tɯ-ŋu ma aʑo ɬɤndʐi ɲɯ-maʁ-a \\
demon real \textsc{dem} \textsc{2sg} \textsc{sens}-2-be \textsc{lnk} \textsc{1sg} demon \textsc{sens}-not.be-\textsc{1sg} \\
\glt `You are the real demon, not me.' (2002lhandzi, 12)
\end{exe}
%ɯ-qa nɯ qarŋe, tɤtsoʁ wuma nɯ. 

%\japhug{ɕɯŋarɯra}{each better than the other} XXXX
% rɟɤlpu ɕɯŋarɯra kɯ ta-tʰu-nɯ ɕti ri, mɯ-tɤ-nɤla-j ɕti tɕe,
% 2003 qachga, 71

\subsubsection{Attributive prenominal modifiers}

\subsubsection{Participial relatives}
%Postnominal or head-internal? definiteness wuma ʑo ... kɯ-

 %rɟɤlpu nɤrɯβzaŋ nɯ kɯ nɤki stu kɯ-mna tɕheme nɯ ɲɤ-nɯ-ɕar ɲɯ-ŋu
 
 
 %ɯʑo sɤz rɯdaʁ kɯ-xtɕi nɯra tu-ndze
 \section{Noun coordination}
\subsection{Coordination or dependency} \label{sec:coordinator}
The closest thing to a noun coordinator in Japhug is the comitative marker \forme{cʰo}, which is argued to be a postposition in § \ref{sec:comitative}.


%tɤ-rpi nɤ tɤ-rpi ʑo ɲɯ-sɯβzu-nɯ ɲɯ-ŋu tɕe,
%2003kandZislama, 112
\subsection{Bare coordination}

\subsubsection{Enumeration} \label{sec:noun.enumeration}
%mbro, jla, nɯŋa, mbala, tshɤt, qaʑo, paʁ, nɯra nɯtɕu ʁɟa z-ɲɯ́-wɣ-lɤɣ pɯ-ŋu.
%qhe tshɤt qaʑo ra ɣɯ nɯ-ndza nɯra ɲɯ-sna. 

\subsubsection{Noun dyads} \label{sec:dyads}
Noun dyads are a pair of nouns occurring in a fixed order, without intervening linker or postposition, and sharing their number and case markers. A good example is provided by the expression `parents' comprising the kinship terms \japhug{tɤ-mu}{mother} and \japhug{tɤ-wa}{father}, as in (\ref{ex:amu.awa.ni.GW}). Note that while number and case markers are shared by both nouns, each of them takes its own possessive prefix, and both prefixes are coreferent. 

\begin{exe}
\ex \label{ex:amu.awa.ni.GW}
 \gll nɯ a-mu a-wa ni ɣɯ ŋu \\
 \textsc{dem}  \textsc{1sg}.\textsc{poss}-mother \textsc{1sg}.\textsc{poss}-father \textsc{du} \textsc{gen} be:\textsc{fact} \\
 \glt `This is for my parents.' (meimei de gushi)
\end{exe}

The dyad for `parents' has a honorific variant, originally used for noblemen in the traditional society. It comprises the terms \japhug{tɤ-pa}{father} and \japhug{tɤ-ma}{mother}, which are borrowed from Tibetan \tibet{ཨ་ཕ་}{ʔa.pʰa}{father}  and \tibet{ཨ་མ་}{ʔa.ma}{mother} respectively. Interesting, the honorific expression follows the `father-mother' order (as in example \ref{ex:apa.ama}), while the native one puts `mother' in the first place.

\begin{exe}
\ex \label{ex:apa.ama}
 \gll nɯ kɯ-fse a-pa a-ma ni kɯ ɲɯ-ti-ndʑi tɕe \\
 \textsc{dem} \textsc{nmlz}:S/A-be.like \textsc{1sg}.\textsc{poss}-father  \textsc{1sg}.\textsc{poss}-mother \textsc{du} \textsc{erg} \textsc{sens}-say-\textsc{du} \textsc{lnk} \\
 \glt `My parents say this.' (2003nyima2, 94)
\end{exe}

Other common dyads include \forme{rgɤtpu rgɤnmɯ} `old man(men) and woman(women)', \forme{tɤ-tɕɯ tɕʰeme} `boy(s) and girl(s)' with APNs. They are most commonly used as collectives with indefinite referents as in (\ref{ex:tAtCW.tCheme.tWsAmdzW}), but are also attested with definite ones, as in (\ref{ex:rgAtpu.rgAnmW}).

\begin{exe}
\ex \label{ex:tAtCW.tCheme.tWsAmdzW}
 \gll  tɤ-tɕɯ tɕʰeme tɯ-sɤ-ɤmdzɯ ʑaka tu \\
 \textsc{indef}.\textsc{poss}-son girl \textsc{genr}.\textsc{poss}-\textsc{nmlz}:\textsc{oblique}-sit each \textsc{exist}:fact \\
\glt `Gents and ladies each have (different) seating places.' (31-khAjmu, 10)
\end{exe}

\begin{exe}
\ex \label{ex:rgAtpu.rgAnmW}
 \gll rgɤtpu rgɤnmɯ ni kɯ kɯki tɤ-pɤtso χsɯm ki kɤsɯfse ʑo cʰɤ-ɣɤ-wxti-ndʑi. \\
 old.man old.woman \textsc{du} \textsc{erg} \textsc{dem}.\textsc{prox} \textsc{indef}.\textsc{poss}-child three \textsc{dem.prox} all \textsc{emph} \textsc{ifr}-\textsc{caus}-be.big-\textsc{du} \\
\glt `The old man and the old woman raised all these three children.' (140514 huishuohua de niao, 60)
\end{exe}

\section{Apposition} \label{sec:apposition}

\section{The structure of the noun phrase}


\section{Nominal predicates} \label{sec:nominal.predicates}
 %The noun phrase
%\chapter{The noun phrase} \label{chap:noun.phrase}


\section{Noun modifiers and determiners}
This section discusses all nouns modifiers and determiners except relative clauses (§ XXX) and complement clauses (§ XXX). 
 
\subsection{Number}  \label{sec:number.determiners}
Japhug has two number markers, the dual \forme{ni} and the plural \forme{ra}. These clitics are not obligatory for non-singular arguments (even in the case of human referents), and do not necessary trigger plural or dual agreement on the verb. 

\subsubsection{Dual} \label{sec:dual.determiners}
The dual \forme{ni} is historically related to the numeral \forme{ʁnɯz} (§ \ref{sec:one.to.ten}), but their relationship is synchronically opaque. It combines with the proximal and distal demonstratives \forme{ki} and \forme{nɯ} respectively to form the dual demonstratives \forme{kɯni} and \forme{nɯni} (§ \ref{sec:demonstrative.pronouns}, § \ref{sec:demonstrative.determiners}).

There is no semantic restriction on the use of \forme{ni}, it most often occurs with human referents (\ref{ex:awW.cho.aRi.ni}, \ref{ex:Wmu.Wwa.ni}, \ref{ex:tCiZo.ni}, \ref{ex:ni.ndZisroR}), but is also commonly attested with animals (\ref{ex:ʁnWz.ni}) inanimate objects (\ref{ex:ni.RnaRna}), and placenames (\ref{ex:rgWnba.ni}).

\begin{exe}
\ex \label{ex:rgWnba.ni}
\gll prɤɕta cʰo rgɯnba ni ndʑi-pɤrtʰɤβ ri ŋu \\
pl.n. \textsc{comit} monastery \textsc{du} \textsc{3du}.\textsc{poss}-between \textsc{loc} be:\textsc{fact} \\
\glt `It is between Prashta and the monastery.' (140522 Kamnyu zgo, 115)
\end{exe}

The dual can follow the numeral \japhug{ʁnɯz}{two}, as in (\ref{ex:ʁnWz.ni}). This combination is however very rare (only 13 examples in the corpus out of hundreds of dual \forme{ni}). The opposite order (dual followed by numeral) is not grammatical.

\begin{exe}
\ex \label{ex:ʁnWz.ni}
\gll mbɣɤru nɯ jla ʁnɯz ni ndʑi-tʰɤβ ri ɲɯ-ɕe tɕe \\
plough.beam \textsc{dem} hybrid.yak two \textsc{3du}.\textsc{poss}-between \textsc{loc} \textsc{ipfv}:\textsc{west}-go \textsc{lnk} \textsc{lnk} \\
\glt `The beam of the plough goes between the two hybrid yaks.' 
\end{exe}


The adverb \japhug{ʁnaʁna}{both} (§ XXX) commonly co-occurs with dual, as in (\ref{ex:ni.RnaRna}).
%tɤ-pi ʁnaʁna ʑo pɯ́-wɣ-sat-ndʑi ɲɯ-ŋu. 

\begin{exe}
\ex \label{ex:ni.RnaRna}
\gll zaŋ cʰo raʁ ni ʁnaʁna ʑo ʁja ku-te ɲɯ-ŋu \\
copper \textsc{comit} brass \textsc{du} both \textsc{emph} verdigris \textsc{ipfv}-put[III] \textsc{sens}-be \\
\glt `Both copper and brass can get verdigris.' (30-Com, 101)
\end{exe}

The marker \forme{ni} can appear with a noun phrase comprising two nouns (each with singular referents) linked by the comitative \forme{cʰo} (§ \ref{sec:comitative}).

\begin{exe}
\ex \label{ex:awW.cho.aRi.ni}
\gll  tɕe a-wɯ cʰo a-ʁi ni pjɯ-tɯ-sat mɤ-jɤɣ \\
\textsc{lnk} \textsc{1sg}.\textsc{poss}-grandfather \textsc{comit} \textsc{1sg}.\textsc{poss}-younger.sibling \textsc{du} \textsc{ipfv}-2-kill \textsc{neg}-be.possible:\textsc{fact} \\
\glt `You cannot kill my grandfather and my younger brother.' (2011-05-nyima, 133)
\end{exe}

The dual can also be used with noun dyads (§ \ref{sec:dyads}), as in (\ref{ex:Wmu.Wwa.ni}). 

\begin{exe}
\ex \label{ex:Wmu.Wwa.ni}
\gll   ɯ-mu ɯ-wa ni kɯ ɲɯ-z-nɤja-ndʑi qʰe \\
\textsc{3sg}.\textsc{poss}-mother \textsc{3sg}.\textsc{poss}-father \textsc{du} \textsc{erg} \textsc{ipfv}-\textsc{caus}-be.a.pity-\textsc{du} \textsc{lnk} \\
\glt `Her parents would not be parted from her.' (14-tApitaRi, 305)
\end{exe}

The third person dual pronoun \forme{ʑɤni} is build by combining the pronominal root \forme{-ʑo-} with the dual \forme{ni} (§ \ref{sec:pers.pronouns}), and is not attested in combination with the dual. The first and second dual pronouns \forme{tɕiʑo} and \forme{ndʑiʑo}, do occur with the dual marker as in (\ref{ex:tCiZo.ni}), though examples are very rare.

\begin{exe}
\ex \label{ex:tCiZo.ni}
\gll  tɕiʑo ni wuma ʑo pɯ-amɯmi-tɕi tɕe \\
\textsc{1du} \textsc{du} really \textsc{emph} \textsc{pst}.\textsc{ipfv}-be.in.good.terms-\textsc{1du} \textsc{lnk} \\
\glt `We were in harmony together.' (140512 fushang he yaomo-zh, 85)
\end{exe}

Noun phrases with the dual \forme{ni} are always correlated with a dual prefix on the following noun in possessive constructions or with relator nouns, as in (\ref{ex:rgWnba.ni}), (\ref{ex:ʁnWz.ni}) and (\ref{ex:ni.ndZisroR}). Not a single example of a noun phrase in \forme{ni} followed by a noun with singular of plural possessive prefix is found in the corpus.

\begin{exe}
\ex \label{ex:ni.ndZisroR} 
\gll ɯ-pi ni ndʑi-sroʁ ko-ri tɕe \\
\textsc{3sg}.\textsc{poss}-elder.sibling \textsc{du} \textsc{3du}.\textsc{poss}-life \textsc{ifr}-save \textsc{lnk} \\
\glt `He saved the life of his two brothers.' (qachGa 2012, 139)
\end{exe}

The marker \forme{ni} is not obligatory with dual referents, in particular when the numeral \japhug{ʁnɯz}{two} is present. An overt noun phrase without dual marking can trigger indexation on the verb, especially with collectives expressing a pair of individuals as \japhug{ʁzɤmi}{husband and wife} in (\ref{ex:RjWmbrWg.RzAmi}), but also with other types of noun phrases as in (\ref{ex:nW.talWlAtndZi}).

\begin{exe}
\ex \label{ex:RjWmbrWg.RzAmi}
\gll  kɯɕɯŋgɯ tɕe tɕe atu <qinghai> ʑɴɢɯloʁ nɯtɕu tɕe, ʁjɯmbrɯɣ ʁzɤmi ci pjɤ-tu-ndʑi tɕe,  \\
in.former.times \textsc{lnk}  \textsc{lnk} up.there p.n. p.n. \textsc{dem}:\textsc{loc} \textsc{lnk} dragon husband.and.wife one \textsc{ifr}.\textsc{ipfv}-exist-\textsc{du} \textsc{lnk} \\
\glt `In former times, in Qinghai, in the Mgolog area, there was a couple of dragons.' (150820 qaprANar, 44)
\end{exe}

\begin{exe}
\ex \label{ex:nW.talWlAtndZi}
\gll  ʁdɯxpanaχpu ɯ-tɕɯ cʰo aʑo a-tɕɯ nɯ tɤ-alɯlɤt-ndʑi tɕe, \\
p.n. \textsc{3sg}.\textsc{poss}-son \textsc{comit} \textsc{1sg} \textsc{1sg}.\textsc{poss}-son \textsc{dem} \textsc{pfv}-fight-\textsc{du} \textsc{lnk} \\
\glt `The son of Gdugpa Nagpo and my son were fighting.' (28-smAnmi, 280)
\end{exe}

Such examples are however surprisingly rare in the corpus; dual indexation is most often correlated with a dual marker on the corresponding noun phrase, if overt.

The numeral \japhug{ʁnɯz}{two} without the dual also triggers dual indexation, as in (\ref{ex:RnWz.tundZi}).

\begin{exe}
\ex \label{ex:RnWz.tundZi}
\gll   sɯŋgɯ zɯ tɯrme wuma ʑo kɯ-wxti ʁnɯz tu-ndʑi tɕe\\
forest \textsc{loc} person really \textsc{emph} \textsc{nmlz}:S/A-be.big two exist:\textsc{fact}-\textsc{du} \textsc{lnk}\\
\glt `In the forest, there are two giants.'  (140428 yonggan de xiaocaifeng, 172)
\end{exe}

Dual marking on a noun phrase is not necessarily correlated with dual indexation on the verb, especially, but not exclusively, with inanimate referents, as in (\ref{ex:ni.tomto}). This question is studied in more detail in § XXX.

\begin{exe}
\ex \label{ex:ni.tomto}
\gll  ɯ-mɲaʁ χcʰoʁe ni to-mto. \\
\textsc{3sg}.\textsc{poss}-eye left.and.right \textsc{du} \textsc{pfv}-have.sight \\
\glt `His left and right eyes recovered sight.' (140517 mogui de jing, 105)
\end{exe}

However, a noun phrase with \forme{ni} is never correlated with a plural indexation marker on the verb. Apparent exceptions are either speech errors (a topic treated in § XXX), or cases of ambiguous indexation, as in (\ref{ex:paznAkharnW}).

\begin{exe}
\ex \label{ex:paznAkharnW}
 \gll  nɤ-pi ni kɯ nɤʑo nɯɣi kɤ-sɯso kɯ ʁmaʁ χsɯ-tɤxɯr kɯ pa-z-nɤkʰar-nɯ ɕti tɕe, \\
 \textsc{2sg}.\textsc{poss}-elder.sibling \textsc{du} \textsc{erg} \textsc{2sg} come.back:\textsc{fact} \textsc{inf}-think \textsc{erg} soldier three-round \textsc{erg} \textsc{pfv}:3\fl{}3'-\textsc{caus}-surround-\textsc{pl} be.\textsc{affirm}:\textsc{fact} \textsc{lnk} \\
 \glt `Your two elder brothers, thinking that you are coming back, had (the palace) guarded on all sides by three rows of soldiers.' (qachGa2012, 157)
\end{exe}

Example (\ref{ex:paznAkharnW}) is not completely straightforward, and deserves a detailed comment. The form \forme{paznɤkʰarnɯ} can be parsed as either \forme{pɯ-az-nɤkʰar-nɯ} \textsc{pst}.\textsc{ipfv}-\textsc{prog}-surround-\textsc{pl} `They were guarding it' with vowel fusion (§ XXX) or \forme{pa-z-nɤkʰar-nɯ} \textsc{pfv}:3\fl{}3'-\textsc{caus}-surround-\textsc{pl} `(He/they) had them guard it'. Context makes it clear here that the second option is the correct one, in particular because in the same passage in another version of the same story, we find the verb \forme{pa-sɯ-lɤt} \textsc{pfv}:3\fl{}3'-\textsc{caus}-throw `he had (them) make' with the perfective 3\fl{}3' form of a causative verb (\citealt[242]{jacques16complementation}, § XXX). Moreover, while the phrase \forme{nɤ-pi ni kɯ}  `your two elder brothers' could in principle belong to the infinitival clause in \forme{kɤ-sɯso}\footnote{Incidentally, note that this infinitival clause contains another complement in Hybrid Reported Speech, see § XXX.}, it is clear from the context and the explanations provided by native speakers that \forme{nɤ-pi ni kɯ} is the causer, and \forme{ʁmaʁ χsɯ-tɤxɯr kɯ} `three rows of soldiers' is the causee (also marked by the ergative, see § \ref{sec:causee.kW}). 

We thus observe plural indexation \forme{-nɯ} on the main verb \forme{pa-z-nɤkʰar-nɯ}, while the subject \forme{nɤ-pi ni kɯ}  has a dual marker. However, this is neither a counterexample to the number indexation rule stated above, nor a speech error: rather, it is a consequence of the fact that causees rather than causers can trigger number indexation on the verb in specific cases (see § XXX).

\subsubsection{Plural} \label{sec:plural.determiners}
The plural marker \forme{ra}, like the dual, follows the noun and most of its modifiers, and fuses with the demonstratives \forme{ki} and \forme{nɯ} respectively to build the plural demonstratives \forme{kɯra} and \forme{nɯra} (§ \ref{sec:demonstrative.pronouns}, § \ref{sec:demonstrative.determiners}). The etymology of the plural marker \forme{ra} is unknown, but a potential cognate exists in Pumi (\forme{=ɹə}, (\citealt[135]{daudey14grammar}; Japhug \forme{-a} regularly corresponds to Pumi \forme{-ə} in the native vocabulary). It should not be confused with the auxiliary verb \japhug{ra}{have to, need} (§ XXX), though there are cases where some ambiguity may occur (§ XXX).

Like the dual \forme{ni}, the plural \forme{ra} is compatible with both animate and inanimate referents, as in (\ref{ex:si.ra.cho}) and (\ref{ex:rdAstaR.ra}). It can be a plain marker of plurality as in (\ref{ex:si.ra.cho}).

\begin{exe}
\ex \label{ex:si.ra.cho}
\gll kɯmaʁ si ra cʰo nɯ-mdoʁ mɤ-naχtɕɯɣ \\
other tree \textsc{pl} \textsc{comit} \textsc{3pl}.\textsc{poss}-colour \textsc{neg}-be.the.same:\textsc{fact} \\
\glt `Its colour is different from that of the other trees.' (11-qrontshom, 56)
\end{exe} 

The marker \forme{ra} is also often an associative plural, understandable as `and other things', as in (\ref{ex:rdAstaR.ra}).

\begin{exe}
\ex \label{ex:rdAstaR.ra}
\gll rdɤstaʁ ra pjɯ-tʂaβ-nɯ qʰe tɯrme tu-xtsɯɣ ɲɯ-ŋu \\
stone \textsc{pl} \textsc{ipfv}-cause.to.fall-\textsc{pl} \textsc{lnk} people \textsc{ipfv}-hit \textsc{sens}-be \\
\glt `(Goats and sheep, as they climb high) cause stones (and other things) to fall and these hit people.' (tshAt-qaZo-kAlAG, 4)
\end{exe} 

The plural can follow numerals (even without head noun) to express an approximative number, as in (\ref{ex:XsWm.kWBde}).\footnote{Note that in (\ref{ex:XsWm.kWBde}) \forme{ci ci} is the expression for `sometimes', not used as a numeral, see § XXX.} 

\begin{exe}
\ex \label{ex:XsWm.kWBde}
\gll ci ci χsɯm kɯβde ra ɲɯ-lɤt ɲɯ-ŋgrɤl. tsuku tɕe ʁnɯz jamar ma mɯ́j-lɤt,\\
one one three four \textsc{pl} \textsc{sens}-throw \textsc{sens}-be.usually.the.case. some \textsc{lnk} two about apart.from \textsc{neg}:\textsc{sens}-throw \\
\glt  `Sometimes (dogs) have three or four (litters), some only have two.' (05-khWna, 22)
\end{exe} 

The plural marker \forme{ra} can also indicate approximate location, with or without locative markers. In (\ref{ex:kha.ra}), we find approximate location \forme{ra} in \forme{kʰa ra} `(everywhere) in the house, around the house' and \forme{tɯ-ji ɯ-ngɯ ra} `in the fields', and in (\ref{ex:nWrNa.ra}) with body parts.

This use of \forme{ra} can convey a meaning of distributed location, and is often combined with the adverb \japhug{aʁɤndɯndɤt}{everywhere} (§ \ref{sec:aRandWndAt}). It is reminiscent of plural markers in Kirghiz and Old Japanese, which combine collective, hypocoristic and approximate locative meanings (see \citealt[195]{antonov07ra}).

\begin{exe}
\ex \label{ex:kha.ra}
\gll βʑɯ nɯ wuma ʑo ŋɤn tɕe, tɕendɤre aʁɤndɯndɤt ʑo kʰa ra cʰɯ-rɤpɯ. tɯ-ji ɯ-ngɯ ra cʰɯ-rɤpɯ, \\
mouse \textsc{dem} really \textsc{emph} be.evil:\textsc{fact} \textsc{lnk} \textsc{lnk} everywhere \textsc{emph} house \textsc{pl} \textsc{ipfv}-bear.young \textsc{indef}.\textsc{poss}-field \textsc{3sg}.\textsc{poss}-inside \textsc{pl}  \textsc{ipfv}-bear.young \\
\glt `The mouse is fierce, it has youngs everywhere in the house, and has youngs in the fields.' (27-spjaNkW, 166)
\end{exe} 

\begin{exe}
\ex \label{ex:nWrNa.ra}
\gll nɯ-βri ra ɲɯ-ɬoʁ, nɯ-mke nɯra ɲɯ-ɬoʁ nɯ-rŋa ra brɤβbrɤβ ʑo ɲɯ-ɬoʁ ɲɯ-ŋu. \\
\textsc{3pl}.\textsc{poss}-body \textsc{pl} \textsc{ipfv}-come.out \textsc{3pl}.\textsc{poss}-neck \textsc{dem:pl} \textsc{ipfv}-come.out \textsc{3pl}.\textsc{poss}-face \textsc{pl} \textsc{idph}:II:covered.by.tiny.bumps \textsc{emph} \textsc{ipfv}-come.out  \textsc{sens}-be \\
\glt `(People who suffer from this disease have little blisters) appearing on their body, on their neck and all over their face.' (27-kharwut, 58)
\end{exe} 

The marker \forme{ra} even occurs with referents which are clearly singular, not only in the approximative location function, but also in examples such as (\ref{ex:tAwi.ra}) where the reason for the presence of \forme{ra} is less immediately obvious. In (\ref{ex:tAwi.ra}), a sentence taken from the translation of Rotkäppchen into Japhug (from Chinese, though here the presence of \forme{ra} cannot be due to calque), the function of the plural on the phrase \forme{tɤ-wi ra} `the grandmother' is more subtle: it conveys the idea idea that the impersonation takes on several aspects of the grandmother, not only her physical appearance, but also her voice, as implied by the second clause. 

\begin{exe}
\ex \label{ex:tAwi.ra}
\gll  qapar nɯ kɯ li, [...] tɤ-wi ra to-nɯɕpɯz tɕe, tɕe ɯ-skɤt ra cʰɤ-sɯ-ɤmtɕoʁ ʑo tɕe nɯra to-ti. \\
dhole \textsc{dem} \textsc{erg} again { } \textsc{indef}.\textsc{poss}-grandmother \textsc{pl} \textsc{ifr}-impersonate \textsc{lnk} \textsc{lnk} \textsc{3sg}.\textsc{poss}-voice \textsc{pl} \textsc{ifr}-\textsc{caus}-be.sharp \textsc{emph} \textsc{lnk} \textsc{dem}:\textsc{pl} \textsc{ifr}-say \\
\glt `The wolf was pretending to be the grandmother, and said these (words) with a sharp voice.' (140428 xiaohongmao-zh, 95-96)
\end{exe} 

Just like noun phrases with dual \forme{ni} correlate with dual possessive prefixe (see \ref{ex:ni.ndZisroR} in § \ref{sec:dual.determiners}), those with plural \forme{ra} can only be coreferent with a plural possessive prefix, as \forme{nɯ-} in (\ref{ex:si.ra.nWmat}).

\begin{exe}
\ex \label{ex:si.ra.nWmat}
 \gll  sɯku tɕe tʰɣe kɯ-fse, kɯmaʁ si ra nɯ-mat nɯra ɕ-pjɯ-nɯ-pʰɯt tɕe tu-ndze ɲɯ-ŋu.\\
tree \textsc{lnk} acorn \textsc{nmlz}:S/A-be.like other tree \textsc{pl} \textsc{3pl}.\textsc{poss}-fruit \textsc{dem}:\textsc{pl} \textsc{transloc}-\textsc{ipfv}:\textsc{down}-\textsc{auto}-pluck \textsc{lnk} \textsc{ipfv}-eat[III] \textsc{sens}-be \\
\glt `On the trees, (the bear) plucks acorn or fruits from other trees to eat.' (21-pri, 44)
\end{exe}

Apparent counterexamples such as (\ref{ex:WtaR.ra.Wmat}), where \forme{ra} is followed by a noun with the singular possessive prefix \forme{ɯ-}, occur when the preceding noun phrase is not the possessor of the following noun. For instance, in (\ref{ex:WtaR.ra.Wmat}) \forme{ra} has the vague locative function, and the phrase \forme{tɯ-ŋga ɯ-taʁ ra} `on the clothes' is not the possessor of \japhug{ɯ-mat}{its fruits}, it is a locative adjunct.

\begin{exe}
\ex \label{ex:WtaR.ra.Wmat}
 \gll tɯ-ŋga ɯ-taʁ ra ɯ-mat bɤbɤβ ʑo ku-ndzoʁ. \\
 \textsc{indef}.\textsc{poss}-clothes \textsc{3sg}.\textsc{poss}-on \textsc{pl} \textsc{3sg}.\textsc{poss}-fruit \textsc{idph}:II:in.clusters \textsc{emph} \textsc{ipfv}-\textsc{anticaus}:attach \\
\glt `Its seeds attach on clothes in clusters.' (18-qromJoR, 169)
\end{exe}

The plural \forme{ra} very commonly occurs with headless relatives, with or without a demonstrative, as in (\ref{ex:nW.tCaGi}), where we find both relatives followed by \forme{nɯnɯra} and another one followed by \forme{ra}.

\begin{exe}
\ex \label{ex:nW.tCaGi}
\gll [kɤ-ti mɤ-kɯ-pe kɯ-fse tu-kɯ-ti] nɯnɯra tɕe, [[kɤ-nɯtsɯ kɯ-ra] ra kɯnɤ tu-kɯ-ti] nɯnɯra, 
tɯrme ra kɯnɤ, tɕaɣi tu-sɤrmi-nɯ ŋgrɤl.  \\
\textsc{inf}-say \textsc{neg}-\textsc{nmlz}.S/A-be.good \textsc{nmlz}.S/A-be.like \textsc{ipfv}-\textsc{nmlz}.S/A-say \textsc{dem}:\textsc{pl} \textsc{lnk} \textsc{inf}-hide \textsc{nmlz}.S/A-have.to \textsc{pl} also \textsc{ipfv}-\textsc{nmlz}.S/A-say \textsc{dem}:\textsc{pl} people \textsc{pl} also  parrot \textsc{ipfv}-call-\textsc{pl} be.usually.the.case:\textsc{fact} \\
\glt `Those who say things that one should not say, who say even what should be concealed, even (if they are) people, they call them `parrots'. (24-qro, 125)
\end{exe} 

%ɯʑo sɤz pɣɤtɕɯ kɯ-xtɕi nɯra tu-ndze ɲɯ-ŋu. tɕe nɯnɯ tu-ti-nɯ ɲɯ-ŋu tɕe ɯ-mɤ-ŋu ma,
%ta-ndza ra pɯ́-wɣ-mto me
 
The plural \forme{ra} also occurs between auxiliaries and the preceding complement clause with a verb in finite (\ref{ex:GWkWnWru.ra}) or non-finite (\ref{ex:kAnAjaR.ra}) form, with a vague implication that additional related actions are concerned.

\begin{exe}
\ex \label{ex:GWkWnWru.ra}
 \gll li tɯ-ji ɯ-ŋgɯ ra ɣɯ-ku-nɯru ra ŋgrɤl. \\
 again \textsc{indef}.\textsc{poss}-field \textsc{3sg}.\textsc{poss}-inside \textsc{pl} \textsc{cisloc}-\textsc{ipfv}-eat.crops \textsc{pl} be.usually.the.case:\textsc{fact} \\
\glt `It also (usually) comes to eat crops in the fields.' (24-ZmbrWpGa, 37)
\end{exe}

\begin{exe}
\ex \label{ex:kAnAjaR.ra}
 \gll  ɣɤmdzu tɕe nɯnɯ kɤ-nɤjaʁ ra mɤ-sɤ-nɤz tɕe \\
be.thorny:\textsc{fact} \textsc{lnk} \textsc{dem} \textsc{inf}-touch \textsc{pl} \textsc{neg}-\textsc{deexp}-dare:\textsc{fact} \textsc{lnk} \\
\glt `It is thorny and one does not dare to touch it with the hand.' (11-qrontshom, 91)
\end{exe}

The marker \forme{ra} following a locative noun or adverb can have the meaning `the people/things from X', as in (\ref{ex:alo.ra}), without the need to add a demonstrative (cf \ref{ex:aki.nW} § \ref{sec:demonstrative.determiners}).

\begin{exe}
\ex \label{ex:alo.ra}
 \gll alo ra ɲɯ-mbɣom-nɯ qʰe \\
 upstream \textsc{pl} \textsc{sens}-be.in.a.hurry-\textsc{pl} \textsc{lnk} \\
 \glt `Those in the village, they (do things) in hurry.' (conversation140510 tshering, 175)
\end{exe}

\subsection{Demonstratives} \label{sec:demonstrative.determiners}
Japhug demonstrative determiners are formally identical  to the demonstrative pronouns (§ \ref{sec:demonstrative.pronouns}). They distinguish between proximal and distal demonstratives with different roots, and fuse with the dual and plural markers studied in § \ref{sec:number.determiners}; the proximal \forme{ki} undergoes change to \forme{kɯ-} in those fused forms.

As with the demonstrative pronouns, there are three sets of demonstratives, the base form, the reduplicated one (obtained by reduplicating the first syllable), and the emphatic one, with added \forme{ɯ-} prefix. Note that the latter two sets are not attested in the dual for determiners in the corpus, but the forms exist and are easily deducible from the corresponding plural ones. In addition, there is a medial demonstrative \forme{nɤki} which occurs in prenominal position.

\begin{table}
\caption{Demonstrative determiners}\label{tab:dem.determiners}
\begin{tabular}{ll|l|ll} 
\lsptoprule
&Base form & Reduplicated & Emphatic \\
\midrule
\textsc{prox.sg} & \forme{ki} & \forme{kɯki} &  \forme{ɯkɯki}  \\
\textsc{dist.sg} & \forme{nɯ} &  \forme{nɯnɯ} & \forme{ɯnɯnɯ} \\
\midrule
\textsc{prox.pl} & \forme{kɯni} & X &  X \\
\textsc{dist.pl} & \forme{nɯni} &  X & X \\
\midrule
\textsc{prox.pl} & \forme{kɯra} & \forme{kɯkɯra} &  \forme{ɯkɯkɯra}  \\
\textsc{dist.pl} & \forme{nɯra} &  \forme{nɯnɯra} & \forme{ɯnɯnɯra} \\
\midrule
\textsc{medial} &  \forme{nɤki} \\
\lspbottomrule
\end{tabular}
\end{table}

In Japhug, as in other Gyalrong languages, demonstrative determiners can be either/both pre- and postnominal, as shown by an example such as (\ref{ex:ki.srWnloRpW.ki}) with the proximal \forme{ki} both before and after the noun \japhug{srɯnloʁpɯ}{little ring}.

\begin{exe}
\ex \label{ex:ki.srWnloRpW.ki}
 \gll aʑo ɣɯ-ɕaβ-a tɤ-ŋu tɕe, ki srɯnloʁ-pɯ ki ɲɯ-ɕtʰɯz-a tɕe,  \\
 \textsc{1sg} \textsc{inv}-catch.up:\textsc{fact}-\textsc{1sg} \textsc{pfv}-be \textsc{lnk} \textsc{dem}.\textsc{prox} ring-\textsc{dim} \textsc{dem}.\textsc{prox} \textsc{ipfv}:\textsc{west}-turn.toward-\textsc{1sg} \\
\glt `When (the râkshasas) will be about to catch up with me, I will  turn this little ring towards west (in their direction).' (28-smAnmi, 222)
\end{exe}

All possible combinations of base demonstratives (B) and reduplicated demonstratives (R) are attested as pre- or postnominal determiners:

\begin{itemize}
\item BNB: \forme{ki} N \forme{ki}, \forme{nɯ} N \forme{nɯ} (\ref{ex:ki.srWnloRpW.ki})
\item RNB: \forme{kɯki} N \forme{ki}, \forme{nɯnɯ} N \forme{nɯ} (\ref{ex:kWki.tAYi.ki})
\item BNR: \forme{ki} N \forme{kɯki}, \forme{nɯ} N \forme{nɯnɯ} (\ref{ex:ki.rgAtpu.kWki})
\item RNR: \forme{kɯki} N \forme{kɯki}, \forme{nɯnɯ} N \forme{nɯnɯ} (\ref{ex:kWki.qingjiao.kWki})
\end{itemize}  

The types BNB and RNB, with the postnominal determiner as a base demonstrative, are by far the most common ones in the corpus.

\begin{exe}
\ex \label{ex:kWki.tAYi.ki}
 \gll  aʑo kɯki tɤɲi ki lu-nɤkʰɯkʰrɯt-a tɕe \\
 \textsc{1sg} \textsc{dem}.\textsc{prox} staff \textsc{dem}.\textsc{prox} \textsc{ipfv}:\textsc{upstream}-drag-\textsc{1sg} \textsc{lnk} \\
 \glt `I will drag along this staff (on the ground).' (Kunbzang2003, 225)
\end{exe}
 

\begin{exe}
\ex \label{ex:kWki.qingjiao.kWki}
 \gll iɕqʰa kɯki <qingjiao> kɯki tɕe, ɯ-qa kɯ-wɣrum ɲɯ-ŋu. \\
 the.aforementioned \textsc{dem}.\textsc{prox} plant.name \textsc{dem}.\textsc{prox} \textsc{lnk} \textsc{3sg}.\textsc{poss}-root \textsc{nmlz}:S/A-be.white \textsc{sens}-be \\
 \glt `This (plant that is called) \textit{qingjiao} (in Chinese), its root is white (unlike the other \textit{qingjiao} whose root is red).' (17-ndZWnW, 81)
\end{exe}

\begin{exe}
\ex \label{ex:ki.rgAtpu.kWki}
 \gll ki rgɤtpu kɯki kɯ, iɕqʰa, qaʑo nɯ to-mtsʰi qʰe, li tʂu kɯ-wxti nɯtɕu jo-ɕe tɕe, \\
\textsc{dem}.\textsc{prox} old.man \textsc{dem}.\textsc{prox} \textsc{erg} the.aforementioned sheep \textsc{dem} \textsc{ifr}-lead \textsc{lnk} again road \textsc{nmlz}:S/A-be.big \textsc{dem}:\textsc{loc} \textsc{ifr}-go \textsc{lnk} \\
\glt `The old man, leading the sheep, went to the big road.' (150822 laoye zuoshi zongshi duide-zh, 101
\end{exe}

The emphatic form is only used prenominally as in (\ref{ex:WkWki.arZaB.kWki}) to differentiate in case of confusion -- in this case, because the story is about two persons designated by the term \japhug{tɤ-rʑaβ}{wife}, even if they have different possessors (\textsc{3sg} vs \textsc{1sg}).

\begin{exe}
\ex \label{ex:WkWki.arZaB.kWki}
 \gll   nɯ ɯ-rʑaβ nɯ kɯ, ɯkɯki a-rʑaβ kɯki, kɯki ɕkom ki na-sɯ-ɤβzu tɕe, \\
 \textsc{dem} \textsc{3sg}.\textsc{poss}-wife \textsc{dem} \textsc{erg} \textsc{dem}.\textsc{prox}.\textsc{emph} \textsc{1sg}.\textsc{poss}-wide \textsc{dem}.\textsc{prox} \textsc{dem}.\textsc{prox} muntjac \textsc{dem}.\textsc{prox}  \textsc{pfv}:3\fl{}3'-\textsc{caus}-become \textsc{lnk} \\
\glt `His wife turned this wife of mine into this muntjac.' (140512 fushang he yaomo-zh, 187)
\end{exe}

When the postnominal demonstrative is in plural or dual form, the prenominal one is generally unmarked for number, as in (\ref{ex:kWki.tCheme.kWra}).

\begin{exe}
\ex \label{ex:kWki.tCheme.kWra}
 \gll kɯki tɕʰeme kɯra nɯ-rca aʑo tu-ɕe-a ɲɯ-ntsʰi ma mɯ́j-pe \\
 \textsc{dem} girl \textsc{dem}:\textsc{pl} \textsc{3pl}.\textsc{poss}-following \textsc{1sg} \textsc{ipfv}:\textsc{up}-go-\textsc{1sg} \textsc{sens}-have.better apart.from \textsc{neg}:\textsc{sens}-be.good \\
 \glt `I have no other choice but to go (to heaven) with these girls.' (31-deluge, 61)
 \end{exe}

However, there are also a few examples with plural marking on both pre- and postnominal demonstratives, as in (\ref{ex:nWnWra.pGa.nWra}), a remarkable phenomenon given the fact that the number markers are strictly postnominal. Plural marking on the prenominal demonstrative with a singular postnominal demonstrative is not attested.
 
\begin{exe}
\ex \label{ex:nWnWra.pGa.nWra}
 \gll nɯnɯra pɣa nɯra lonba ʑo ɲɤ-me-nɯ tɕe, ʁʑɯnɯ sqaptɯɣ ɲɤ-k-ɤpa-nɯ-ci. \\
 \textsc{dem.pl} bird  \textsc{dem.pl}  all \textsc{emph} \textsc{ifr}-not.exist \textsc{lnk} young.man eleven \textsc{ifr}-\textsc{evd}-become-\textsc{pl}-\textsc{evd} \\
 \glt `All those birds disappeared, and became eleven young men.' (140520 ye tiane-zh, 121)
\end{exe}

Proximal prenominal demonstratives can be combined with the postnominal \forme{nɯ}, as in (\ref{ex:kWki.Xpi.nW}), where the latter one is used as a topic marker. The opposite combination, a distal prenominal demonstrative with proximal postnominal one, is not attested in the corpus and presumably agrammatical.

\begin{exe}
\ex \label{ex:kWki.Xpi.nW}
 \gll kɯki χpi nɯ pɯpɯŋu nɤ,  \\
 \textsc{dem}.\textsc{prox} story \textsc{dem} \textsc{top} \textsc{lnk} \\
 \glt `As far as this story goes,' (11 examples in the corpus)
\end{exe}

The medial demonstrative \forme{nɤki}, used to designate referents closer to the addressee than the speaker, is found as a pronoun (§ \ref{sec:medial.dem.pro}), but also occurs as a prenominal determiner, with or without postnominal demonstrative (either proximal or distal), as in (\ref{ex:nAki.nAtAYi}) and (\ref{ex:nAki.nAtAri}). It is frequently used with a noun taking a second person possessive prefix -- note that the first syllable \forme{nɤ-} of the demonstrative \forme{nɤki} itself probably originates from the second singular possessive, as proposed in § \ref{sec:medial.dem.pro}.

\begin{exe}
\ex \label{ex:nAki.nAtAYi}
 \gll nɤki nɯ-tɤɲi ɯ-taʁ kɤ-rɤt nɯ ɯβrɤ-kɯ-z-nɤmɲo-a-nɯ \\
 \textsc{dem}:\textsc{medial} \textsc{2pl}.\textsc{poss}-staff \textsc{3sg}.\textsc{poss}-on \textsc{nmlz}:P-write \textsc{dem} \textsc{pot}-2\fl{}1-\textsc{caus}-watch-\textsc{1sg}-\textsc{pl} \\
 \glt `Would you show me what is written on that staff of yours?' (2003sras, 61)
\end{exe}

\begin{exe}
\ex \label{ex:nAki.nAtAri}
 \gll nɤki nɤ-tɤ-ri nɯ ŋotɕu pɯ-tu \\
 \textsc{dem}:\textsc{medial} \textsc{2sg}.\textsc{poss}-\textsc{indef}.\textsc{poss}-thread \textsc{dem} where \textsc{pst}.\textsc{ipfv}-exist \\
\glt `That thread of yours, where is it from?' (Norbzang2005, 180)
\end{exe}

The relative position of prenominal demonstrative and other pronominal elements is not free. The aforementioned topic marker \forme{iɕqʰa} strictly occurs before prenominal demonstratives (as in \ref{ex:kWki.qingjiao.kWki} and \ref{ex:kWki.XsAr.pGAtCW} respectively), while nominal modifiers such as \japhug{χsɤr}{gold} in \ref{ex:kWki.XsAr.pGAtCW} appear closer to the noun. Pronouns coreferent with a possessive prefix on the head noun, however, can be placed either after (\ref{ex:nW.aZo.aCArW.nW}) or before (\ref{ex:aZo.ki.aku.ki}) prenominal demonstratives.

\begin{exe}
\ex \label{ex:kWki.XsAr.pGAtCW}
 \gll kɯki χsɤr pɣɤtɕɯ ki nɤ-jaʁ ɲɯ-kham-a ŋu \\
\textsc{dem}.\textsc{prox} gold bird \textsc{dem}.\textsc{prox} \textsc{1sg}.\textsc{poss}-hand \textsc{ipfv}-give[III]-\textsc{1sg} be:\textsc{fact} \\
\glt `(If you succeed) I will give you this golden bird.' (2012qachGa, 46)
\end{exe}

\begin{exe}
\ex \label{ex:nW.aZo.aCArW.nW}
 \gll nɯtɕu a-tɯrsa ŋu, tɕe nɤʑo kɯ [nɯ aʑo a-ɕɤrɯ nɯnɯra] a-tɤ-tɯ-tɕɤt tɕe, \\
 \textsc{dem}:\textsc{loc} \textsc{1sg}.\textsc{poss}-tomb be:\textsc{fact} \textsc{lnk} \textsc{2sg} \textsc{erg} \textsc{dem} \textsc{1sg} \textsc{1sg}.\textsc{poss}-bone \textsc{dem}:\textsc{pl} \textsc{irr}-\textsc{pfv}-2-take.out \textsc{lnk} \\
\glt `My tomb is there, if you take out my bones (from it),' (150907 niexiaoqian-zh, 109)
\end{exe}

\begin{exe}
\ex \label{ex:aZo.ki.aku.ki}
 \gll  kɯki, aʑo [ki a-ku ki] pɯ-pʰɯt ra \\
 \textsc{dem}.\textsc{prox} \textsc{1sg}  \textsc{dem}.\textsc{prox} \textsc{1sg}.\textsc{poss}-head  \textsc{dem}.\textsc{prox} \textsc{imp}-cut have.to:\textsc{fact} \\
 \glt `Please behead me!' (140507 jinniao-zh, 292)
\end{exe}

The principles governing the presence and absence of the demonstrative determiners, and the choice of the various patterns described above, is particularly complex to describe and will be a topic for future research, when a larger corpus of texts will become available. While the proximal demonstratives always have some deictic function (although it may not be always appropriate to translate them with a demonstrative in other languages such as English), the distal demonstratives clearly contribute to marking topic (§ \ref{sec:topic}) and definiteness (§ \ref{sec:definiteness}), and disentangling these various functions is a complex matter.

The demonstratives \forme{nɯ} and \forme{nɯnɯ} are particularily common after relative clauses (either participial § XXX or finite ones § XXX) and complement clauses (§ XXX) but arguments against analyzing them as subordinators (like English `that') are presented in § XXX. 

Following locative adverbs or locative postpositional phrases, the distal and proximal demonstratives can be used to express the meaning `the one/those X' as in (\ref{ex:athi.ki}) and (\ref{ex:aki.nW}). Note that the number determiner \forme{ra} can also be used in the same way (example \ref{ex:alo.ra} § \ref{sec:plural.determiners}) even without being combined with a demonstrative.

\begin{exe}
\ex \label{ex:athi.ki}
\gll amaŋ amaŋ, atʰi ki kɯ `a-βɣo mɤ-a<nɯ>tɯɣ-a tɕe a-scawa' ɲɯ-sɯsɤm ɲɯ-ŋu ɣe \\
\textsc{interj}:\textsc{surprise} \textsc{interj}:\textsc{surprise}  downstream \textsc{dem}.\textsc{prox} \textsc{erg} \textsc{1sg}.\textsc{poss}-uncle \textsc{neg}-<auto>meet:\textsc{fact}-\textsc{1sg} lnk 1sg.poss-poor.of \textsc{sens}-think[III] \textsc{sens}-be \textsc{sfp} \\
\glt `The one down there, he is thinking `Poor of me, I will not meet my lama', isn't he?' (2003kandZislama, 203)
\end{exe}

\begin{exe}
\ex \label{ex:aki.nW}
\gll a-pa, aki nɯ staʁlupa kɤ-βde ɯ-spa nɯ mɤ-nɯ-xsi ri, \\
\textsc{1sg}.\textsc{poss}-father down \textsc{dem} born.in.the.year.of.the.tiger \textsc{nmlz}:P-throw.away \textsc{3sg}.\textsc{poss}-material \textsc{dem} \textsc{neg}-\textsc{auto}-\textsc{genr}:know \textsc{lnk} \\
\glt `Father, the one down there, I don't know if he is a (boy) born in the year of the Tiger, to be thrown (in the lake), but...' (2011-05-nyima, 154)
\end{exe}

Note however that demonstratives or number markers are not absolutely necessary in such a context. A few (rare) examples of locative postpositional phrases meaning `the one at/in/from' without any modifier can be found, as in (\ref{ex:sWkAku.nWtCu}), where the postpositional phrase is directly followed by the dative, here used in its locative meaning `by (the side of), near, at' (§ \ref{sec:dative}). In this example, the phrase \forme{sɯkɤku nɯtɕu} does not mean  on the treetop', but `the man who is on the treetop'.\footnote{The story from which this example is taken is about three thieves who mistakenly steal a tiger during the night, believing it was an ox; one of the three thieves flees on the top of a tree -- his manner of fleeing being here the characteristic distinguishing him from the other two thieves.}

\begin{exe}
\ex \label{ex:sWkAku.nWtCu}
\gll  [sɯkɤku nɯtɕu] ɯ-pʰe nɯtɕu lo-zɣɯt-ndʑi tɕe  tɕe,\\
treetop \textsc{dem}:\textsc{loc} \textsc{3sg}.\textsc{poss}-\textsc{dat} \textsc{dem}:\textsc{loc}  \textsc{ifr}:\textsc{upstream}-reach-\textsc{du} \textsc{lnk} \textsc{lnk} \\
\glt `(The tiger and the fox) arrived at (the tree where the one who was) on the treetop (was).' (2012-x1-khu, 47)
\end{exe}

\subsection{Quantifiers} \label{sec:quantifiers.determiners}
This section only discusses universal, mid-scalar and specifically distributive quantifiers; numerals and counted nouns, which also serve as quantifiers (in particular distributive ones) are described in chapter \ref{chap:numerals}.

\subsubsection{Universal quantifiers} \label{sec:universal.quant}
The determiner \japhug{tʰamtɕɤt}{all}, from Tibetan \tibet{ཐམས་ཅད་}{tʰams.cad}{all}, is strictly postnominal, as in (\ref{si.thamCAt.kW}). It cannot be used as a pronoun, and there are no examples in the corpus of \japhug{tʰamtɕɤt}{all} following a personal pronoun.

\begin{exe}
\ex \label{si.thamCAt.kW}
 \gll   sɯŋgɯ kɤ-kɯ-nɯχtɕɤn tɕe tɕe si tʰamtɕɤt kɯ nɯnɯ pjɯ-kɯ-sat kɯ-ŋgrɤl ɲɯ-ŋu. \\
 forest \textsc{pfv}-\textsc{nmlz}:S/A-be.dangerous \textsc{lnk} \textsc{lnk} tree all \textsc{erg} \textsc{dem} \textsc{ipfv}-\textsc{genr}:S/P-kill  \textsc{nmlz}:S/A-be.usually.the.case \textsc{sens}-be \\
 \glt `The fierce/dangerous forest, it was (a place where) all the tree would could to kill (people thrown into it).' (28-smAnmi, 191)
\end{exe}

It can also follow a headless relative clause, as in (\ref{WkWndza.thamCAt.nWnWra}), and be followed by demonstratives.

\begin{exe}
\ex \label{WkWndza.thamCAt.nWnWra}
 \gll nɯ-zda rɯdaʁ ɯ-kɯ-ndza tʰamtɕɤt nɯnɯra nɯ-rmi lonba kɯrŋi tu-kɯ-ti ŋu \\
 \textsc{3pl}.\textsc{poss}-companion animal \textsc{3sg}.\textsc{poss}-\textsc{nmlz}:S/A-eat all \textsc{dem}:\textsc{pl}  \textsc{3pl}.\textsc{poss}-name all beast \textsc{ipfv}-\textsc{genr}-say be:\textsc{fact} \\
 \glt `All those which eat the other animals, their name, all of them, is `beast'.'  (150822 kWrNi, 8)
 \end{exe}
 
The combination of a demonstrative such as \japhug{nɯ}{this} with  \japhug{tʰamtɕɤt}{all} does not mean `all of this', but `so much, so many', as in (\ref{kha.nW.thamCAt}) (see § XXX for more examples of this construction).

\begin{exe}
\ex \label{kha.nW.thamCAt}
 \gll iʑora tʰɯ-dɤn-i qʰe, kʰa nɯ tʰamtɕɤt mɯ-ɲɯ-ɤmɯ-xtɕʰɯt-i qʰe \\ 
 \textsc{1pl} \textsc{pfv}-be.many-\textsc{1pl} \textsc{lnk} house \textsc{dem} all \textsc{neg}-\textsc{sens}-\textsc{recip}-have.enough.place-\textsc{1pl} \textsc{lnk} \\
 \glt `There was now more of us (than before), and so many of us could not fit in the house.' (14-tApitaRi, 103-104)
 \end{exe}
 
 Another universal quantifier,  \japhug{kɤsɯfse}{all}, is common as a pronoun (§ \ref{sec:quantifiers.pronouns}) or as an adverb (§ XXX). It is potentially analyzable as a determiner in examples like  (\ref{ex:kAtsa.ra.kAsWfse.kW}) where it follows the noun phrase \forme{kɤtsa ra} `parents and children', and takes the ergative \forme{kɯ}.
 
 \begin{exe}
\ex \label{ex:kAtsa.ra.kAsWfse.kW}
\gll  kɤtsa ra kɤsɯfse kɯ wuma ʑo pjɤ-nɯ-rga-nɯ  \\
parents.and.children \textsc{pl} all \textsc{erg} really \textsc{emph} \textsc{ifr}.\textsc{ipfv}-\textsc{appl}-like-\textsc{pl} \\
\glt `Everybody in the family liked her very much.' (140429 qingwa wangzi, 5)
  \end{exe}
  
A third universal quantifier \japhug{rmɯrmi}{all, all kinds of}, a borrowing from Situ (meaning `everybody'), is attested but only rarely used, in examples such as (\ref{rmWrmi.GW.nWrmi}).
  
\begin{exe}
\ex \label{rmWrmi.GW.nWrmi}
\gll  maka tɤ-rɤku rmɯrmi ɣɯ nɯ-rmi nɯ to-nɤrmi ri maka kɯm mɯ-pjɤ-ɲɟɯ   \\
at.all \textsc{indef}.\textsc{poss}-crops all \textsc{gen} \textsc{3pl}.\textsc{poss}-name \textsc{ifr}-say.name \textsc{lnk} at.all door \textsc{neg}-\textsc{ifr}-\textsc{anticaus}:open \\
\glt `He said the names of all kinds of crops, but the door did not open.' (140512 alibaba-zh, 107)
\end{exe}  

 A fourth universal quantifier, \japhug{lonba}{all} (from Tibetan \tibet{ལོན་པ་}{lon.pa}{reached, enough, completed}), exists in Japhug, but it is not used as a noun determiner and only occurs as an adverb (§ XXX).
% aʑo a-ʁi nɯra lonba aʑo kɯ tɤ-nɤpɯpa-t-a
% 140426 tApAtso kAnWBdaR4, 1
% 
% tɕe, ɯ-tɯ-mbro nɯnɯ cho nɯra
%lonba qaɕti cho naχtɕɯɣ-ndʑi ʑo
%09-sArsi, 17

An alternative construction with a meaning similar to a universal quantifier is the totalitative reduplication (§ XXX) \japhug{kɯ\redp{}kɯ-tu}{all who exist} of the participle of the existential verb \japhug{tu}{exist}, in a post-nominal or head-internal relative clauses, as in (\ref{ex:Wzda.kWkWtu.kW}). 

 \begin{exe}
\ex \label{ex:Wzda.kWkWtu.kW}
\gll tɕe [ɯ-zda ra kɯ\redp{}kɯ-tu] kɯ nɯ-rʑaβ na-nɯ-ɕar-nɯ ɲɯ-ŋu \\
\textsc{lnk} \textsc{3sg}.\textsc{poss}-companion \textsc{pl} \textsc{total}\redp{}\textsc{nmlz}:S/A-exist \textsc{erg} \textsc{3pl}.\textsc{poss}-wife \textsc{pfv}:3\fl{}3'-\textsc{auto}-look.for-\textsc{pl} \textsc{sens}-be \\
\glt `All of his companions took (other women) as their wives.' (Norbzang2005, 57)
  \end{exe}

\subsubsection{Mid-scalar quantifier} \label{sec:tsuku}
The quantifier \japhug{tsuku}{some} is generally used as a pronoun (§ \ref{sec:partitive.pronouns}), but it does occur as a prenominal determiner as in (\ref{ex:tsuku.tWrme}), or a postnominal one as in (\ref{ex:kWmtChW.tsuku}) and (\ref{ex:kWmWrkW.tsuku}). It is most often used in the corpus with human referents, but is compatible with inanimate objects, as shown by (\ref{ex:kWmtChW.tsuku}).

\begin{exe}
\ex \label{ex:tsuku.tWrme}
\gll tsuku tɯrme ra kú-wɣ-mtsɯɣ-nɯ tɕe mɯ́j-ʁdɯɣ, tsuku tɯrme ra [...] kú-wɣ-mtsɯɣ-nɯ tɕe tɕe, wuma ʑo cʰɯ́-wɣ-z-nɯɣmbɤβ-nɯ qʰe ɲɯ́-wɣ-z-nɯtɯfɕɤl-nɯ qʰe ku-rŋgɯ-nɯ ɲɯ-ra.\\
some people \textsc{pl} \textsc{ipfv}-\textsc{inv}-bite-\textsc{pl} \textsc{lnk} \textsc{neg}.\textsc{sens}-be.serious some people \textsc{pl} { } \textsc{ipfv}-\textsc{inv}-bite-\textsc{pl} \textsc{lnk} \textsc{lnk} really \textsc{emph} \textsc{ipfv}-\textsc{inv}-\textsc{caus}-swell-\textsc{pl} \textsc{lnk}  \textsc{ipfv}-\textsc{inv}-\textsc{caus}-have.diarrhea-\textsc{pl} \textsc{lnk} \textsc{ipfv}-lie.down-\textsc{pl} \textsc{sens}-have.to\\
\glt `Some people, when they are stung (by bees) are fine, other people, when they are stung, it causes them swelling and diarrhea and they have to lie down.' (26-ndzWrnaR, 65-67)
\end{exe}

\begin{exe}
\ex \label{ex:kWmtChW.tsuku}
\gll  `pjɯ-nɯβle-a ɲɯ-ra' ɲɤ-sɯso tɕe, kɯmtɕʰɯ tsuku ɲɤ-kʰo tɕe, \\
\textsc{ipfv}-cheat[III]-\textsc{1sg} \textsc{sens}-have.to \textsc{ifr}-think \textsc{lnk} toy some \textsc{ifr}-give \textsc{lnk} \\
\glt `She thought `Let's cheat him' and gave him some toys.' (Norbzang2012, 134)
\end{exe}

\begin{exe}
\ex \label{ex:kWmWrkW.tsuku}
\gll ri kɯ-mɯrkɯ tsuku pjɤ-tu-nɯ tɕe tɕe, \\
\textsc{lnk} \textsc{nmlz}:S/A-steal some \textsc{ifr}.\textsc{ipfv}-exist-\textsc{pl} \textsc{lnk} \textsc{lnk} \\
\glt `There were some thieves.' (X1-khu, 7)
\end{exe}

Note in (\ref{ex:tsuku.tWrme.tWrdoR}) the combination of the quantifier \japhug{tsuku}{some} with the counted noun \japhug{tɯ-rdoʁ}{one piece}, which expresses here a partitive meaning (thirteen of fifteen children for each of them, § \ref{sec:ICN}).

\begin{exe}
\ex \label{ex:tsuku.tWrme.tWrdoR}
\gll tsuku tɯrme tɯ-rdoʁ ɣɯ ɯ-rɟit, sqafsum jamar, sqamŋu jamar tu-kɯ-tu pjɤ-tu. \\
 some person one-piece \textsc{gen} \textsc{3sg}.\textsc{poss}-offspring thirteen about fifteen about \textsc{ipfv}-\textsc{genr}:S/A-exist \textsc{ifr}.\textsc{ipfv}-exist    \\
\glt  `Some (women) had thirteen or fifteen children.' (140426 tApAtso kAnWBdaR, 88)
\end{exe}
 
\subsubsection{Distributive quantifier} \label{sec:raNri}
 Although distributive meaning is generally expressed in Japhug with a counted noun (see in particular § \ref{sec:ICN} and § \ref{sec:CCN}), the postnominal determiner \japhug{raŋri}{each} and its variant \japhug{rɯri}{each} (from Tibetan \tibet{རང་རེ་}{raŋ.re}{each} and \tibet{རེ་རེ་}{re.re}{each}) can also express distributive meaning, as in (\ref{ex:tWtWpW.raNri}). 
 
\begin{exe}
\ex \label{ex:tWtWpW.raNri}
\gll paʁ rcanɯ, tɯ-tɯpɯ raŋri kɯ ʑo pjɯ-χsu-nɯ ra. \\
pig \textsc{unexpected} one-household each \textsc{erg} \textsc{emph} \textsc{ipfv}-raise-\textsc{pl} have.to:\textsc{fact} \\
 \glt `Each single household has to raise pigs.' (05-paR, 4)
 \end{exe}
 
 It can also be used with numerals, as in (\ref{ex:sqamNu.raNri}), where it refers specifically to days.

 \begin{exe}
\ex \label{ex:sqamNu.raNri}
\gll  sqamŋu raŋri ʑo zgo tu-ɕe pɯ-ŋu ɲɯ-ŋu, \\
fifteen \textsc{each} \textsc{emph} mountain \textsc{ipfv}:\textsc{up}-go \textsc{pst}.\textsc{ipfv}-be \textsc{sens}-be \\
\glt `Every fifteen days, she would go up the mountain.' (Norbzang2005, 57)
  \end{exe}

With a noun phrase with the quantifier \japhug{raŋri}{each} occurs in a clause with a counted noun, the scope of the two quantifiers is ambiguous, as in (\ref{ex:rirAB.raNri}).

\begin{exe}
\ex \label{ex:rirAB.raNri}
\gll rirɤβ raŋri χsɯ-tɤxɯr a-tɤ-tɯ-sɯ-lɤt tɕe, \\
mountain each three-round \textsc{irr}-\textsc{pfv}-2-\textsc{caus}-throw \textsc{lnk} \\
\glt `Drag her three times around each mountain.' (Kunbzang2005, 421)
 \end{exe}
 
When a noun with the determiner \forme{raŋri} is possessor, its possessum is often  followed by the distributive determiner \japhug{tɯka}{each} (and its reduplicated variant \forme{tɯkaka}) as in (\ref{ex:Wmat.raNri}).
 
  \begin{exe}
\ex \label{ex:Wmat.raNri}
\gll   iɕqʰa ɯ-mat raŋri ʑo nɯ ɯ-ru tɯka ntsɯ tu. \\
the.aforementioned \textsc{3sg}.\textsc{poss}-fruit each \textsc{emph} \textsc{dem} \textsc{3sg}.\textsc{poss}-stalk own always exist:\textsc{fact} \\
\glt `Each of its fruits has its own stalk.' (17-thowum, 34)
  \end{exe}
  
The  determiner \japhug{tɯka}{each} can also be used without a possessor in  \forme{raŋri}, for example with the distributive pronouns \japhug{ʑaka}{each his own} and \japhug{ʑakastaka}{each his own} (§ \ref{sec:distributive.pronouns})  as in (\ref{ex:nWkho.tWka}).

   \begin{exe}
\ex \label{ex:nWkho.tWka}
\gll      ʑakastaka nɯ-kʰo tɯka pjɤ-tu tɕe \\
each.his.own \textsc{3pl}.\textsc{poss}-room each \textsc{ifr}.\textsc{ipfv}-exist \textsc{lnk} \\
\glt `Each of them had her own room.' (140508 shie ge tiaowu de gongzhu, 85)
   \end{exe}
   
In addition to possessors, \japhug{tɯka}{each} also follows objects, with broad scope on the whole action (\ref{ex:pCaR.tWka}).

    \begin{exe}
\ex \label{ex:pCaR.tWka}
\gll  pɕaʁ tɯka to-βzu-nɯ tɕe jo-nɯ-ɕe-nɯ.  \\
reverence each \textsc{ifr}-make-\textsc{pl} \textsc{lnk} \textsc{ifr}-\textsc{vert}-go-\textsc{pl} \\
\glt `They made a reverence each and went back.' (28-smAnmi, 176)
   \end{exe}
 
\subsection{Indefinite and definite markers} \label{sec:indefinite.markers}

\subsubsection{Indefinite article} \label{sec:indef.article}
The form \japhug{ci}{one} has among its many functions (in addition to pronoun, numeral and adverb, see § \ref{sec:ci.someone}, § \ref{sec:other.pro}, § \ref{sec:partitive.pronouns}, § \ref{sec:identity.modifier}, § \ref{sec:one.to.ten} and § XXX) that of singular indefinite article, as in (\ref{ex:ci.indef}) and (\ref{ex:ci.chAGi}). It is typically used to introduce a new referent in a story.

\begin{exe}
\ex \label{ex:ci.indef}
\gll tɕʰeme kɯ-mpɕɯ\redp{}mpɕɤr ci ɲɤ-nɯ-ɬoʁ \\
girl \textsc{nmlz}:S/A-\textsc{emph}\redp{}beautiful \textsc{indef} \textsc{ifr}-\textsc{auto}-come.out \\
\glt `A very beautiful girl appeared (out of it).' (The flood, 39)
\end{exe}

\begin{exe}
\ex \label{ex:ci.chAGi}
\gll tɕɤlo tɕe tɤ-tɕɯ ci cʰɤ-ɣi qʰe, \\
upstream \textsc{lnk} \textsc{indef}.\textsc{poss}-son \textsc{indef} \textsc{ifr}:\textsc{downstream}-come \textsc{lnk} \\
\glt `A boy came from upstream.' (2003-kWBRa, 41)
\end{exe}

Although \forme{ci} can be used as a partitive pronoun `one of them' (§ \ref{sec:partitive.pronouns}), as a postnominal determiner it does not have partitive meaning. To express a meaning such as `one of the boys', a CN such as \japhug{tɯ-rdoʁ}{one piece} is used instead (§ \ref{sec:ICN}). 

Note that when used as a prenominal modifier, \forme{ci} has a completely different (definite) meaning `the other X' (§ \ref{sec:identity.modifier}). However, the indefinite \forme{ci} is attested in prenominal position if preceded by the prenominal identity modifier \japhug{kɯmaʁ}{other}, as in (\ref{ex:kWmaR.ci.nArZaB}), though the exact syntactic analysis of such sentences may require more research (it is possible that \forme{nɤ-rʑaβ} here is an essive adjunct § \ref{sec:essive.abs}, and does not belong to the same constituent as \forme{kɯmaʁ ci}).

\begin{exe}
\ex \label{ex:kWmaR.ci.nArZaB}
\gll  nɤʑo kɯmaʁ ci nɤ-rʑaβ nɯ-nɯ-ɕar kɯ mna  \\
\textsc{2sg} other \textsc{indef} \textsc{2sg}.\textsc{poss}-wife \textsc{imp}-\textsc{auto}-search \textsc{erg} be.better:\textsc{fact} \\
\glt `You should look for another wife.' (150909 xiaocui, 163)
\end{exe}

There are no dual or plural indefinite articles in Japhug. The plural marker \forme{ra} can occur after the indefinite \forme{ci}, but with a vague associative meaning `and other things' as in (\ref{ex:ci.ra}).

\begin{exe}
\ex \label{ex:ci.ra}
 \gll  ndʑi-tɕɯ ci, ndʑi-me ci ra to-tu. \\
 \textsc{3du}.\textsc{poss}-son \textsc{indef}  \textsc{3du}.\textsc{poss}-girl \textsc{indef} \textsc{pl} \textsc{ifr}-exist \\
 \glt  `They$_{du}$ had a boy and a girl (etc).' (150827 tianluo-zh, 155)
\end{exe}

 The indefinite \forme{ci} is not obligatory for indefinite referents (whether specific or non-specific), and bare NPs can used as \japhug{fsapaʁ}{animal} and \japhug{qapar}{dhole} in example (\ref{ex:ci.ra2}).
 

\begin{exe}
\ex \label{ex:ci.ra2}
 \gll  fsapaʁ nɯ-me, a-pɯ-si qhe, `nɯ qapar kɯ ta-ndza ŋu ma' tu-ti-nɯ ɕti ma, \\
 animal \textsc{pfv}-not.exist \textsc{irr}-\textsc{pfv}-die \textsc{lnk} \textsc{dem} dhole \textsc{erg} \textsc{pfv}:3\fl{}3'-eat be:\textsc{fact} \textsc{sfp} \textsc{ipfv}-say-\textsc{pl} be.\textsc{affirm}:\textsc{fact} \textsc{lnk}  \\
 \glt `When an animal disappears, dies, people say `A dhole ate it.' (28-qapar, 
\end{exe}


\subsubsection{Indefinite pronoun as modifier} \label{sec:indefinite}
The indefinite pronoun \japhug{tʰɯci}{something} (§ \ref{sec:thWci}) has marginal uses as a prenominal indefinite modifier, as in  (\ref{ex:thWci.laXCi}), (\ref{ex:thWci.WjmNo}) and (\ref{ex:laXtCha.ci.nWnW}) below. 

\begin{exe}
\ex \label{ex:thWci.laXCi}
\gll   tʰɯci laχɕi ci ɕ-pɯ-nɯ-βzjoz-nɯ tɕe, jɤ-ɕe-nɯ ra \\
something trade \textsc{indef} \textsc{transloc-imp-auto}-learn-\textsc{pl} \textsc{lnk} \textsc{imp}-go-\textsc{pl} have.to:\textsc{fact} \\
\glt `Go and learn some trade!' (140508 benling gaoqiang de si xiongdi-zh, 29)
 \end{exe}
 
 This construction arose perhaps from the use of the pronoun \forme{tʰɯci} as head of a postnominal relative clause with the verb \japhug{fse}{be like}, as illustrated by examples like (\ref{ex:thWci.kAnWsaXCWB}) or (\ref{ex:thWci.akAspa}) in § \ref{sec:thWci}. Turning the verb \japhug{fse}{be like} to a finite form as in (\ref{ex:thWci.WjmNo}) could cause the indefinite \forme{tʰɯci}, head of the relative in (\ref{ex:thWci.kAnWsaXCWB}), to be reanalyzed as the prenominal modifier of the immediately adjacent noun in (\ref{ex:thWci.WjmNo}).

 \begin{exe}
\ex \label{ex:thWci.kAnWsaXCWB}
\gll nɯra [tʰɯci [kɤ-nɯsaχɕɯβ kɯ-fse]] pɯ-ŋu wo.  \\
\textsc{dem}:\textsc{pl} something \textsc{inf}-have.a.contest \textsc{nmlz}:S/A-be.like \textsc{pst}.\textsc{ipfv}-be \textsc{sfp} \\
\glt `It was like a kind of contest.' (160706 thotsi, 16)
 \end{exe}
 
\begin{exe}
\ex \label{ex:thWci.WjmNo}
\gll [tʰɯci ɯ-jmŋo] ci ʑo pɯ-fse ri \\
something \textsc{3sg}.\textsc{poss}-dream one \textsc{emph} \textsc{pst}.\textsc{ipfv}-be.like \textsc{lnk} \\
\glt `It looked like (he had had) some dream.' (Lobzang2005, 74)
 \end{exe}
 
 
\subsubsection{The marking of definiteness} \label{sec:definiteness}
Japhug has no dedicated definite determiner, but  \forme{nɯ} and \forme{nɯnɯ}  as demonstrative determiners (\ref{sec:demonstrative.determiners}) and as topic markers (\ref{sec:topic}) and the prenominal aforementioned topic marker \forme{iɕqʰa} (§ \ref{sec:iCqha}) are generally used with definite referents.  

Example (\ref{ex:ci.joGi}) illustrates a typical example with the determiner \forme{nɯ}; the indefinite article \forme{ci} (§ \ref{sec:indef.article}) occurs in the first introduction of a new referent in the story as in the first clause of example (\ref{ex:ci.joGi}), but on the following occurrence of the same noun \forme{nɯ} is found.

\begin{exe}
\ex \label{ex:ci.joGi}
 \gll  tɕe qajdo ci jo-ɣi tɕe, tɕe qajdo nɯ kɯ `mo laz tu, pʰo laz me' to-ti. \\
 \textsc{lnk} crow \textsc{indef} \textsc{ifr}-come \textsc{lnk} \textsc{lnk} crow \textsc{dem} \textsc{erg} girl karma exist:\textsc{fact} boy karma not.exist:\textsc{fact} \textsc{ifr}-say \\
 \glt `A crow came. The crow said: `The girl will have chance, the boy won't.'' (28-qAjdoskAt, 8)
\end{exe}

However, although nouns phrases followed by \forme{nɯ} and \forme{nɯnɯ} more often than not denote definite referents, these determiners cannot be analyzed as definite articles, as noun phrases with \forme{nɯ} or \forme{nɯnɯ} can in certain cases have indefinite referents. 

A very clear case of use of \forme{nɯ} with an indefinite referent occurs on nouns serving as heads of head-internal relative clauses. A well-attested typological generalization is that in this type of relative clauses, definiteness marking is agrammatical (see \citealt{basilico96internally} and § XXX). In Khroskyabs, \citet[636]{lai17khroskyabs} reports that the definiteness marker \forme{=tə} is indeed not accepted on the head noun of head-internal relatives. In Japhug however, \forme{nɯ} does occur in such a syntactic context. For instance, in (\ref{ex:tAnmaR.nW.kW}), the head \forme{tɤ-nmaʁ nɯ kɯ} is subject of the participle \japhug{ɲɯ-kɯ-nɯ-ɕar}{looking for}, and is embedded in the participial relative clause indicated in brackets -- the presence of the ergative \forme{kɯ} precludes to analyze it as a post-nominal relative (§ XXX). From the meaning of the sentence the head \japhug{tɤ-nmaʁ}{husband} is clearly indefinite non-specific non-generic  (see \citealt[286-291]{lehmann84relativsatz}). The fact that it takes the marker \forme{nɯ} shows that this marker, unlike Khroskyabs \forme{=tə}, is not primarily marking definiteness.

\begin{exe}
\ex \label{ex:tAnmaR.nW.kW}
 \gll tɕeri [tɤ-nmaʁ nɯ kɯ ɯ-rʑaʁ kɯ-ɤntɕʰɯ ɲɯ-kɯ-nɯ-ɕar], aʁɤndɯndɤt tɤndɤɣri tu-kɯ-βzu pjɤ-tu.  \\
but  \textsc{indef}.\textsc{poss}-husband \textsc{dem} \textsc{erg} \textsc{3sg}.\textsc{poss}-wife  \textsc{nmlz}:S/A-be.many \textsc{ipfv}-\textsc{nmlz}:S/A-\textsc{auto}-search everywhere  illegitimate.child  \textsc{ipfv}-\textsc{nmlz}:S/A-make \textsc{ifr}.\textsc{ipfv}-exist \\
\glt `However there were husbands who were looking for several women and had illegitimate children.' (140427 tAndAGri, 3)
\end{exe}

Other cases of indefinite noun phrase with \forme{nɯ} are observed with left-dislocated topics. In example (\ref{ex:RnWz.nWnW}), we find a type of tail-head linkeage  (§ XXX) where both the noun phrase \japhug{spjaŋkɯ ʁnɯz}{two wolves} and the verb \japhug{ɲɤ-k-ɤtɯɣ-ci}{he met} are repeated; in the second occurrence, the noun phrase is topicalized and is followed by the topic marker \forme{nɯnɯ}, with a slight pause of hesitation. The determiner \forme{nɯnɯ} in this clause, unlike \forme{nɯ} in (\ref{ex:ci.joGi}), does not mark definiteness: that clause cannot be understood as `He met the two wolves'.

\begin{exe} 
\ex \label{ex:RnWz.nWnW} 
 \gll spjaŋkɯ ʁnɯz ɲɤ-k-ɤtɯɣ-ci. spjaŋkɯ ʁnɯz nɯnɯ, tɕendɤre ɲɤ-k-ɤtɯɣ-ci tɕe iɕqʰa, kɯ-rɤ-ntɕʰa nɯ wuma ʑo ɲɤ-mu. \\ 
 wolf two \textsc{ifr}-\textsc{evd}-meet-\textsc{evd}  wolf two \textsc{dem} \textsc{lnk} \textsc{ifr}-\textsc{evd}-meet-\textsc{evd} \textsc{lnk} the.aforementioned \textsc{nmlz}:S/A-\textsc{a.pass}:\textsc{n.hum}-kill \textsc{dem} really \textsc{emph} \textsc{ifr}-be.afraid \\ 
 \glt `He$_i$ (the butcher) met two wolves. He$_i$ met two wolves, and the butcher$_i$ was very much afraid.' (150902 liaozhai lang-zh, 7-8)
\end{exe}

The determiners \forme{nɯ} or \forme{nɯnɯ} are followed by the indefinite singular article \forme{ci} in the corpus if both have scope on the same noun. In all cases with \forme{ci} followed by \forme{nɯ} (other than the identity pronoun in § \ref{sec:other.pro}), or of \forme{nɯ} followed by \forme{ci} in the corpus, they belong to different constituents. For instance, in (\ref{ex:ci.YAZGAsAphAr}), \forme{ci} is in adverbial use (`a little, once', see § XXX) and does not belong to the preceding noun phrase.  

\begin{exe}
\ex \label{ex:ci.YAZGAsAphAr}
\gll [tɕʰeme nɯ] ci ɲɤ-ʑɣɤ-sɤpʰɤr qʰe  \\
girl \textsc{dem} one \textsc{ifr}-\textsc{refl}-shake \textsc{lnk} \\
\glt `The girl shook herself.' (02-deluge2012, 125)
\end{exe}

In (\ref{ex:laXtCha.ci.nWnW}) although \forme{nɯnɯ} follows \forme{ci}, it has scope over the both preceding phrases, which are left-dislocated and followed by a pause.

\begin{exe}
\ex \label{ex:laXtCha.ci.nWnW}
\gll  kɤ-xtɕɤr tɕe nɯnɯ tɕe tɕe iɕqʰa, [[tʰɯci tɯmbri tɤ-ri kɯ-fse kɯ] [laχtɕʰa ci] nɯnɯ], ci kú-wɣ-sɯ-pa tɕe, kú-wɣ-xtɕɤr, \\
\textsc{inf}-attach \textsc{lnk} \textsc{dem} \textsc{lnk} \textsc{lnk} the.aforementioned something rope \textsc{indef}.\textsc{poss}-thread \textsc{nmlz}:S/A-be.like \textsc{erg} thing \textsc{indef} \textsc{dem} one \textsc{ipfv}-\textsc{inv}-\textsc{caus}-do \textsc{lnk} \textsc{ipfv}-\textsc{inf}-attach \\
\glt ``To attach' (means), to put together, attach something with something like a rope or a thread.'  (150902 kAxtCAr, 2-3)
\end{exe}

However, the indefinite article \forme{ci} can be followed by \forme{nɯ} as in (\ref{ex:kWrZi.ci.nW}).

\begin{exe}
\ex \label{ex:kWrZi.ci.nW}
\gll nɯ, kɯ-rʑi ci nɯ ɣɤʑu maʁ kɯ \\
\textsc{dem} \textsc{nmlz}:S/A-be.heavy \textsc{indef} \textsc{dem} exist:\textsc{sens} not.be:\textsc{fact} \textsc{sfp} \\
\glt `This is not a difficult thing (to do).' (divination2005, 15)
\end{exe}

The aforementioned topic marker \forme{iɕqʰa} (§ \ref{sec:iCqha}) is almost always used with definite referents when prenominal, as in (\ref{ex:RnWz.nWnW}) above, and is the closest candidate to be analyzed as a definiteness marker in Japhug. It does occur with non-specific generic referents as in (\ref{ex:lWlAmu}), including some that are very clearly indefinite as in (\ref{ex:lApWG}); note the absence of postnominal determiner \forme{nɯ} (\ref{ex:lApWG}).

\begin{exe}
\ex \label{ex:lWlAmu}
 \gll iɕqʰa lɯlɤmu nɯ tʰɯ-rɤpɯ tɕe tɕe ɯ-sŋi tɕe kɤ-nɯ-rŋgɯ nɯ stʰɯci mɯ́j-tsu ma ɯ-pɯ ra χse ɲɯ-ra tɕe, \\
 the.aforementioned female.cat \textsc{dem} \textsc{ipfv}-bear.young \textsc{lnk} \textsc{lnk} \textsc{3sg}.\textsc{poss}-day \textsc{lnk} \textsc{inf}-\textsc{auto}-lie.down \textsc{dem} so.much \textsc{neg}:\textsc{sens}-have.time.to \\
 \glt `A/the female cat (unlike male cats), when it had had youngs, does not have time to sleep during the day, as it has to feed its youngs.' (21-lWLU, 
\end{exe}

\begin{exe}
\ex \label{ex:lApWG}
\gll  iɕqʰa lɤpɯɣ ɯ-rɣi ʑo fse. \\
the.aforementioned radish \textsc{3sg}.\textsc{poss}-seed \textsc{emph} be.like:\textsc{fact} \\
\glt `It looks like a radish seed.' (hist-26-qro-fourmi, 61)
\end{exe}

In  (\ref{ex:laXtCha.ci.nWnW}), \forme{iɕqʰa}  also precedes two phrases involving indefinite referents, but  there is a marked pause, and this is a case of \forme{iɕqʰa} in its function as speech filler (see § XXX).

\subsubsection{Absence of definiteness marking}
Like many languages (\citealt[130]{creissels06sgit1}), Japhug uses bare nouns without any definiteness marking. Bare nouns are most often non-referential, as \japhug{tɕʰeme}{girl} in (\ref{ex:tCheme.tWtAtu}).

\begin{exe}
\ex \label{ex:tCheme.tWtAtu}
\gll ʁnaʁna tɕʰeme tɯ\redp{}tɤ-tu nɤ, kɤndʑisqʰaj tu-kɤ-sɯ-βzu \\
both girl \textsc{cond}\redp{}\textsc{pfv}-exist \textsc{lnk} \textsc{coll}:sister \textsc{ipfv}-\textsc{inf}-\textsc{caus}-make \\
\glt `If both of them have girls, let them be sisters.' (zrAntCW, 4)
\end{exe}

Bare nouns are less common with referential nouns (except in answers to questions), but examples can be found, as \japhug{qacʰɣa}{fox} in (\ref{ex:qachGa.kW}).

\begin{exe}
\ex \label{ex:qachGa.kW}
\gll qacʰɣa 	kɯ maχtɕɯ tɤ-tɯt-a nɯ mɤ-tɯ-ste ti ɲɯ-ŋu \\
fox \textsc{erg} I.told.you.so \textsc{pfv}-say[II]-\textsc{1sg} \textsc{dem} \textsc{neg}-2-do.like[III]:\textsc{fact} say:\textsc{fact} \textsc{sens}-be \\
\glt `The fox says: `You do not do as I told you to." (2003qachGa, 44)
\end{exe}

Personal names generally occur as bare nouns, without any definiteness marker as in (\ref{ex:WrJAnpanma}), but there are no constraints against co-occurrence of personal names with the determiner \forme{nɯ} either (see § \ref{sec:personal.names.modifiers}).

\begin{exe}
\ex \label{ex:WrJAnpanma}
\gll  ɯrɟɤnpanma kɯ ʁlaŋsaŋtɕhin ɯ-ɕki  \\
 Padmasambhava \textsc{erg} Gesar \textsc{3sg}-\textsc{dat} \\
\glt `Padmasambhava (told) Gesar.' (Gesar, 2)
\end{exe}

 \subsection{Topic markers} \label{sec:topic}
 
  \subsubsection{Delimitative topic} \label{sec:delimitative}
The delimitative topic marker \forme{pɯ\redp{}pɯ-ŋu nɤ} `as for..., concerning...' is transparently derived from the past imperfective of the verb `be' in conditional form `if it was...' (with reduplication of the first syllable, see § XXX), as other copulas such as affirmative \japhug{ɕti}{be} and \japhug{maʁ}{not be} in (\ref{ex:pWpWmaʁ}).

\begin{exe}
\ex \label{ex:pWpWmaʁ}
\gll nɯnɯ koŋla ʑo tɤɕime pɯ\redp{}pɯ-maʁ nɤ \\
\textsc{dem} really \textsc{emph} princess \textsc{cond}\redp{}\textsc{pst.ipfv}-not.be lnk \\
\glt `If she was not really a princess,' (140519 wandou gongzhu, 71)
\end{exe}

The delimitative construction generally has scope over a noun phrase, which can have an additional demonstrative \forme{nɯ} as topicalizer as in (\ref{ex:nW.pWpWNunA}) (see § \ref{sec:nW.topic}).

\begin{exe}
\ex \label{ex:nW.pWpWNunA}
\gll a-mu nɯ pɯ\redp{}pɯ-ŋu nɤ, qhlɯ ʁdɯxpakɤrpu ɣɯ ɯ-me stu kɯ-xtɕi nɯ a-mu ɲɯ-pe, \\
\textsc{1sg}.\textsc{poss}-mother \textsc{dem} \textsc{cond}\redp{}\textsc{pst.ipfv}-be \textsc{lnk} nâga p.n \textsc{gen} \textsc{3sg}.\textsc{poss}-daughter most \textsc{nmlz}:S/A-be.small \textsc{dem} \textsc{1sg}.\textsc{poss}-mother \textsc{sens}-be.good \\
\glt `As for my mother, the daughter of the Nâga Gdugpa dkarpo is good to be my mother.' (Gesar, 5)
\end{exe}

In this construction, the verb is in the process of becoming grammaticalized as a topic particle. It is possible to find examples where the verb still takes person indexation in the delimitative construction when the topicalized element is a first or second person pronoun, as in (\ref{ex:pWpWNuanA}). 

\begin{exe}
\ex \label{ex:pWpWNuanA}
\gll aʑo pɯ\redp{}pɯ-ŋu-a nɤ, kɤndʑiʁi kɯmŋu tu-j, \\
\textsc{1sg} \textsc{cond}\redp{}\textsc{pst.ipfv}-be-\textsc{1sg} \textsc{lnk} siblings five exist:\textsc{fact}-\textsc{1sg} \\
\glt `Concerning me, we are five brothers and sisters.' (hist140501 tshering skyid, 1)
\end{exe}

However, there are also examples with first or second person pronoun without indexation on the delimitative marker, as in (\ref{ex:pWpWNunA}), (\ref{ex:pWpWNunA2}) and (\ref{ex:pWpWNunA3}), where a first person singular form \forme{pɯ\redp{}pɯ-ŋu-a nɤ} or second person \forme{pɯ\redp{}pɯ-tɯ-ŋu nɤ} would have been expected. Such examples show that \forme{pɯpɯŋunɤ} has ceased to be analyzed as a verb form at least in these cases. Moreover, third person plural and dual indexation is hardly ever found in the delimitative construction.

\begin{exe}
\ex \label{ex:pWpWNunA}
\gll nɤʑo pɯpɯŋunɤ, ɬɤndʐi ra ɣɯ nɯ-kɯ-βʁa, nɯ-rɟɤlpu tɯ-ŋu \\
\textsc{2sg} as.for demon \textsc{pl} \textsc{gen} \textsc{3pl.poss}-\textsc{nmlz}:S/A-be.victorious \textsc{3pl.poss}-king 2-be:\textsc{fact} \\
\glt `You, you are the king of the demons.' (hist140512 fushang he yaomo-zh, 61)
\end{exe}

\begin{exe}
\ex \label{ex:pWpWNunA2}
\gll  aʑo kɯ-fse pɯpɯŋunɤ, ɕɯŋgɯ sɤ-xtɕɯ\redp{}xtɕi nɯtɕu, χpɯn lɤ-kɤ-ta, \\
\textsc{1sg} \textsc{nmlz}:S/A-be.like as.for  before \textsc{conv}-\redp{}be.small \textsc{dem}:\textsc{loc} monk \textsc{pfv}:\textsc{upstream}-\textsc{nmlz}:P-put \\
\glt `For instance me, (I was) sent to become monk early in my childhood.' (160721 XpWN, 7)
  \end{exe}

\begin{exe}
\ex \label{ex:pWpWNunA3}
\gll aʑo pɯpɯŋunɤ, nɯnɯ [...] aʑo ɣɯ a-ndʐa nɯ tu-o<nɯ>lɯlat-a pɯ-ŋu tɕe, \\
\textsc{1sg} as.for \textsc{dem} { } \textsc{1sg} \textsc{gen} \textsc{1sg}.\textsc{poss}-reason \textsc{dem} \textsc{ipfv}-<\textsc{auto}>fight-\textsc{1sg} \textsc{pst}.\textsc{ipfv}-be \textsc{lnk} \\
\glt  As for me, I was fighting for my own sake.' (140512 abide he mogui-zh, 92)
 \end{exe}
 
A short form \forme{ŋunɤ} instead of \forme{pɯpɯŋu nɤ} is also attested, as in (\ref{ex:WNga.ra.NunA}).

\begin{exe}
\ex \label{ex:WNga.ra.NunA}
\gll ma ɯ-ŋga ra ŋunɤ, maka wuma ʑo ko-ɴqhi ma. \\
\textsc{lnk} \textsc{3sg}.\textsc{poss}-clothes \textsc{pl} as.for at.all really \textsc{emph} \textsc{ifr}-be.dirty \textsc{lnk} \\
\glt `As for his clothes, they had become very dirty.' (conversation 140510)
\end{exe}
 
The delimitative topic  construction is appropriate to introduce the main topic of a following discourse (as in \ref{ex:pWpWNuanA} and \ref{ex:pWpWNunA2}), but can be used for contrastive topics, as in example (\ref{ex:pWpWNunA3}) where the speaker expresses a contrast between his and the addresses action (`you, you were fighting for the sake of other people').


 \subsubsection{Aforementioned topic} \label{sec:iCqha}
 The marker \japhug{iɕqʰa}{the aforementioned}  is used on referents that have been previously mentioned in the same story, usually only a few sentences back. It is strictly prenominal. 
 
Example (\ref{ex:iCqha.aforementioned}) illustrates the most typical use of this marker. Sentence (\ref{ex:kAtWm}) introduces a new reference, \japhug{kɤtɯm}{ball of thread} marked with the indefinite article \forme{ci} (§ \ref{sec:indef.article}). Three clauses later in (\ref{ex:iCqha.kAtWm}), the same referent occurs again with two topic markers, the postnominal \textit{nɯ} and the prenominal \textit{iɕqʰa}.
 
 
\begin{exe}
\ex \label{ex:iCqha.aforementioned}
\begin{xlist}
\ex \label{ex:kAtWm}
\gll `razri \textbf{kɤtɯm} \textbf{ci} ɲɯ-ra, taqaβ ci ɲɯ-ra' to-ti qʰe   \\
 thread ball \textsc{indef} \textsc{sens}-need needle \textsc{indef} \textsc{sens}-need \textsc{ifr}-say \textsc{lnk}  \\
\glt `He told (Rgyabza) `I need a ball of thread and a needle.''  
\ex  
\gll tɕendɤre ɲɤ-kʰo qʰe,  \\
\textsc{lnk} \textsc{ifr}-give \textsc{lnk}   \\
\glt `She gave it to him.'
\ex 
\gll  tɕe ɯ-ndzɤtsʰi ka-tsɯm-nɯ nɯtɕu qʰe tɕe,   \\
 \textsc{lnk} \textsc{3sg}.\textsc{poss}-meal \textsc{pfv}:3\fl{}3'-bring-\textsc{pl} \textsc{dem}:\textsc{loc}  \textsc{lnk} \textsc{lnk}    \\
\glt `When they brought his meal,'
\ex \label{ex:iCqha.kAtWm}
\gll   \textbf{iɕqʰa} \textbf{kɤtɯm} \textbf{nɯ} ɯʑo kɯ ko-ndo, \\
   the.aforementioned ball \textsc{dem} \textsc{3sg} \textsc{erg} \textsc{ifr}-take \\
\glt `he took the ball of thread, and...' (Gesar 270-272)
\end{xlist}
\end{exe}
 
A systematic study of the use of the topic marker \forme{iɕqʰa} in Japhug must overcome two inherent difficulties. First, this topic marker is homophonous with (and historically related to) the speech filler \forme{iɕqʰa} (§ XXX) and with the adverb \japhug{iɕqʰa}{just now}, which can also precede noun phrases. Listening to the sound files can help distinguishing between the three, as the speech filler is always followed by a pause (and optionally by the demonstrative \forme{nɯ}), but there are still ambiguous sentences (see below). Second, \forme{iɕqʰa} occurs on nouns designating entities that the speaker considers to have been previously referred to in the conversation, even if they are not present in the same recording. 

For instance in (\ref{ex:iCqha.pɣArnoR}) the noun \japhug{pɣɤrnoʁ}{a species of fungus} is used with \forme{iɕqʰa}, although this name does not occur before in the same text; it was however mentioned the day before in another recording.

\begin{exe}
\ex \label{ex:iCqha.pɣArnoR}
\gll nɯ zdɯmqe cʰo iɕqʰa, pɣɤrnoʁ nɯni ndʑi-tsʰɯɣa wuma ʑo naχtɕɯɣ. \\
\textsc{dem} fungi.sp. \textsc{comit} the.aforementioned fungi.sp. \textsc{dem}:\textsc{du} \textsc{3du}.\textsc{poss}-form really \textsc{emph} be:identical:\textsc{fact} \\
\glt `The \forme{zdɯmqe} and the \forme{pɣɤrnoʁ} are very similar.' (23-mbrAZim, 82)
\end{exe}

 
The topic marker \forme{iɕqʰa} transparently comes from the adverb \japhug{iɕqʰa}{just now} (§ XXX). The pivot constructions that allowed reanalysis from adverb to prenominal topic marker are very probably headless relatives (§ XXX) as in  (\ref{ex:iCqha.tAtWta}), or complement clauses as in (\ref{ex:iCqha.ZnWzmWnmuta}). 

\begin{exe}
\ex \label{ex:iCqha.tAtWta}
 \gll  [iɕqʰa tɤ-tɯt-a] nɯ tú-wɣ-stu qʰe, \\
 just.now \textsc{ifr}-say[II]-\textsc{1sg} \textsc{dem} \textsc{ipfv}-\textsc{inv}-do.like \textsc{lnk} \\
\glt `One does as I just said, and...' (2002tWsqar, 139)
\end{exe}

\begin{exe}
\ex \label{ex:iCqha.ZnWzmWnmuta}
 \gll iɕqʰa [ʑ-nɯ-z-mɯnmu-t-a] nɯ mɯ-pjɤ-pe rcama.  \\
the.aforementioned  \textsc{transloc}-\textsc{pfv}-\textsc{caus}-move-\textsc{pst}:\textsc{tr}-\textsc{1sg} \textsc{dem} \textsc{neg}-\textsc{ifr}.\textsc{ipfv}-be.good \textsc{fsp} \\
\glt `It was probably not a good thing that I had moved them (as I said above).' (150819 kumpGa, 45)
 \end{exe}
 
 These sentences are still synchronically ambiguous in Japhug; in  (\ref{ex:iCqha.ZnWzmWnmuta}) the context makes it clear that \forme{iɕqʰa} is the topic marker (since the fact of having moved (the eggs) had been told a few sentences back) and not an adverb `just now' with a temporal reference in the past, as the meaning would be `it was probably not a good thing that I had just moved them' (an impossible interpretation in this context, since this sentence is an explanation why several eggs had not given chicks, several days after they had been brought to another place). However, extracted from the context, both interpretation would be equally possible for (\ref{ex:iCqha.ZnWzmWnmuta}), and correspond to two distinct syntactic structures.

With postnominal (§ XXX) or left-headed head-internal relative clauses (§ XXX) as in (\ref{ex:tWrpa.thafse}), \forme{iɕqʰa} can also be ambiguous. Since the adverb \japhug{iɕqʰa}{just now} can occur both before the object (\ref{ex:tWrpa.thWfseta}) or before the verb (\ref{ex:tWrpa.thWfseta2}) in an independent clause, a relative such as (\ref{ex:tWrpa.thafse}) can be either interpreted `the axe (mentioned above) that he had whetted' (with the topic marker \forme{iɕqʰa} outside of the relative clause, having scope on its head) and `the axe that he had just whetted' with the adverb \japhug{iɕqʰa}{just now} inside the relative clause.

 \begin{exe}
\ex \label{ex:tWrpa.thafse}
 \gll  tɕendɤre <luban> kɯ iɕqʰa [tɯrpa tʰa-fse] nɯ to-ndo tɕe, \\
 \textsc{lnk} p.n. \textsc{erg} the.aforementioned axe \textsc{pfv}:3\fl{}3'-whet \textsc{dem} \textsc{ifr}-take \textsc{lnk} \\
 \glt `Luban took the axe that he had whetted.' (150902 luban-zh, 90)
 \end{exe}

  \begin{exe}
  \ex 
  \begin{xlist}
\ex \label{ex:tWrpa.thWfseta}
 \gll   iɕqʰa tɯrpa tʰɯ-fse-t-a \\
just.now axe \textsc{pfv}-whet-\textsc{pst}:\textsc{tr}-\textsc{1sg} \\
\ex \label{ex:tWrpa.thWfseta2}
 \gll   tɯrpa  iɕqʰa tʰɯ-fse-t-a \\
 axe just.now \textsc{pfv}-whet-\textsc{pst}:\textsc{tr}-\textsc{1sg} \\
 \glt `I just whetted a/the axe.' (elicited)
 \end{xlist}
 \end{exe}

The use of \forme{iɕqʰa} as a topic marker with nouns (as in \ref{ex:iCqha.kAtWm} above) probably took place by reanalysis of the adverb in headless or postnominal relatives, or in complment clauses as above, then generalized to all noun phrases even those without subordinate clause.

 \subsubsection{Adversative topic} \label{sec:adversative.topic}
There are two adversative topic markers in Japhug \forme{ʁo} and \forme{ndɤre}. The former is similar in meaning to Mandarin \ch{倒}{dào}{instead, on the other hand}, and occurs in contexts with a strong adversative meaning `however, but, on the other hand' as in (\ref{ex:Ro.pWtu}).

\begin{exe}
\ex \label{ex:Ro.pWtu}
\gll jinde ku-nɯ-tu ɕi kɯma mɤ-xsi ma kɯɕɯŋgɯ ʁo pɯ-tu, \\
nowadays \textsc{dubit}-\textsc{auto}-exist \textsc{qu} \textsc{sfp} \textsc{neg}-\textsc{genr}:know \textsc{lnk} in.former.times \textsc{top}.\textsc{advers} \textsc{ipfv}.\textsc{pst}-exist \\
\glt `It is not clear whether it is still to be found nowadays, but it did exist in former times.' (23-scuz, 30)
\end{exe}

The marker \forme{ʁo} also occurs in two constructions meaning `of course'. First, it is found in the `X \forme{ʁo} X' construction meaning `of course (it is)  X', as in (\ref{ex:nAZo.Ro.nAZo}), the answer to the question in (\ref{ex:nABJu.YWCara.Ci}) which presents two alternatives.

\begin{exe}
\ex
\begin{xlist}
\ex  \label{ex:nABJu.YWCara.Ci}
\gll `a-tɤɕime, nɤ-βɟu ɲɯ-ɕar-a ɕi, aʑo tu-ozgrɯ-a' nɯra to-ti, `ma nɤ-pi ɣɯ aʑo tɤ-azgrɯ-a ɕti' to-ti  \\
\textsc{1sg}.\textsc{poss}-lady \textsc{2sg}.\textsc{poss}-mat \textsc{ipfv}-search-\textsc{1sg} \textsc{qu} \textsc{1sg} \textsc{ipfv}-bow-\textsc{1sg} \textsc{dem}:\textsc{pl} \textsc{ifr}-say \textsc{lnk} \textsc{1sg}.\textsc{poss}-elder.sibling \textsc{gen} \textsc{1sg} \textsc{pfv}-bow-\textsc{1sg} be.\textsc{affirm}:\textsc{fact} \textsc{ifr}-say \\
\glt `He said `My lady, should I look for a cushion for you, or should I bow (for you to sit on my back)', and he said  `Because I bow for your elder sister (to sit).'
\ex  \label{ex:nAZo.Ro.nAZo}
\gll  `nɤʑo ʁo nɤʑo ma, a-βɟu ɲɯ-tɯ-ɕar kɯ-ɤtsɯtsu me' to-ti.   \\
\textsc{2sg} \textsc{top}.\textsc{advers} \textsc{2sg} \textsc{lnk} \textsc{1sg}.\textsc{poss}-mat \textsc{ipfv}-2-search \textsc{inf}.\textsc{stat}-have.time  not.exist:\textsc{fact} \textsc{ifr}-say \\
\glt  `Of course (I will sit on) you, there is no time to look for a mat for me.' (2014-kWlAG, 195)
\end{xlist}
\end{exe}

Second, \forme{ʁo} is commonly used with the adverb \japhug{lɯski}{of course}, as in (\ref{ex:Ro.lWski}), not necessarily with any adversative meaning.

\begin{exe}
\ex \label{ex:Ro.lWski}
\gll  pɣɤɲaʁ kɤ-ti ci tu tɕe, nɯnɯ ʁo lɯski li nɯ pɣa ŋu \\
pheasant \textsc{nmlz}:P-say \textsc{indef} exist:\textsc{fact} \textsc{lnk} \textsc{dem} \textsc{top}.\textsc{advers} of.course again \textsc{dem} bird be:\textsc{fact} \\
\glt `There is a bird called \forme{pɣɤɲaʁ} (\textit{Pucrasia macrolopha}), this one, of course (since its name contains \japhug{pɣa}{bird}, § \ref{sec:subject.verb.compounds}, Table \ref{tab:subj.v.compounds}) is also a bird (like those previously discussed).' (23-pGAYaR, 2)
\end{exe}

The marker \forme{ndɤre} presents a milder adversative meaning `as far as X is concerned, unlike some other (people)' as in (\ref{ex:aZo.ndAre.rgaa}).  

\begin{exe}
\ex \label{ex:aZo.ndAre.rgaa}
\gll tsuku kɯ-rga tu, tsuku mɤ-kɯ-rga tu. aʑo ndɤre rga-a. \\
some \textsc{nmlz}:S/A-like exist:\textsc{fact} some \textsc{neg}-\textsc{nmlz}:S/A-like exist:\textsc{fact} \textsc{1sg} \textsc{top.advers} like:\textsc{fact}-\textsc{1sg} \\
\glt `Some like it, some don't; as far as I am concerned, I like it.' (07-tCGom2, 8)
\end{exe}

In (\ref{ex:nW.ndAre.wuma}), the use of \forme{ndɤre} suggests the meaning `as opposed to other possible missions'.

\begin{exe}
\ex \label{ex:nW.ndAre.wuma}
\gll a a-pa, nɯ ndɤre wuma ʑo ɴqa, sɤɣʑɯr. \\
\textsc{interj} \textsc{1sg}.\textsc{poss}-father \textsc{dem} \textsc{top.advers} really \textsc{emph} be.difficult:\textsc{fact} be.dangerous:\textsc{fact} \\
\glt `Ah father, this (mission on which you send me) is very difficult and dangerous indeed.' (28-smAnmi, 72)
\end{exe}

In (\ref{ex:jWGmWr.ndAre}), \forme{ndɤre} has a clear adversative meaning `this evening, on the other hand' (as opposed to the previous evenings).

\begin{exe}
\ex \label{ex:jWGmWr.ndAre}
\gll jɯfɕɯr tɯrmɯ tɕe nɤ-pi tɯlɤt nɯ ɯ-taʁ ko-ɴqoʁ-a ri mɯ-tɤ́-wɣ-tsɯm-a tɕe,
jɯɣmɯr ndɤre nɤʑo tu-kɯ-tsɯm-a ra ma tɕe kutɕu aʑo-sti ma maŋe-a tɕe, \\
 yesterday dusk \textsc{lnk} \textsc{2sg}.\textsc{poss}-elder.sibling  second.sibling \textsc{dem} \textsc{3sg}.\textsc{poss}-on \textsc{ifr}-hang-\textsc{1sg} \textsc{lnk} \textsc{neg}-\textsc{pfv:up}-\textsc{inv}-take.away-\textsc{1sg} \textsc{lnk} this.evening \textsc{top.advers}  \textsc{2sg} \textsc{ipfv}:\textsc{up}-2$\rightarrow$1-take.away-\textsc{1sg} have.to:\textsc{fact} \textsc{lnk} \textsc{lnk} here \textsc{1sg}-alone apart.from  not.exist:\textsc{sens}-\textsc{1sg} \textsc{lnk}  \\
\glt  `Yesterday at dusk I clung onto your second eldest sister but she did not take me away, this evening take me away, I am all alone here.' (07-deluge, 56-57)
\end{exe}

The phrase \forme{nɯ sɤznɤ} `even, rather than that etc' (§ \ref{sec:comparative}) is also used as an adversative topic marker similar to \forme{ʁo} (see in particular example \ref{ex:nW.sAznA.YWwxti}).

\subsubsection{The demonstrative \forme{nɯ} as a topic marker} \label{sec:nW.topic}
The postnominal determiner \forme{nɯ} and its reduplicated form \forme{nɯnɯ} is one of the most common words in Japhug, and has a considerable number of functions. It is used as a demonstrative (\ref{sec:demonstrative.determiners}), contributes to expressing definiteness (\ref{sec:definiteness}) and could be argued to be a subordinator (an analysis not adopted in the present work, see § XXX).

In addition, it is commonly used to mark topic: left-dislocated noun phrases generally (though not compulsorily) take this determiner. For instance, in texts presenting animals or plants, their name on first occurrence is left dislocated and followed by the determiner \forme{nɯ}, as in (\ref{ex:qawWz.nW}).

\begin{exe}
\ex \label{ex:qawWz.nW}
\gll  qawɯz nɯ, (qawɯz nɯ pɯ-tɯ-mto-t, ɣe?)  qawɯz nɯnɯ, nɤki, kɯɕɯŋgɯ tɕe, \\
Edelweiss \textsc{dem} Edelweiss \textsc{dem} \textsc{pfv}-2-see-\textsc{pst}:\textsc{tr} \textsc{sfp} Edelweiss \textsc{dem} \textsc{filler} before \textsc{lnk} \\
\glt `The edelweiss, (you saw Edelweiss before, right?)... The edelweiss, in former times,' (15-babW, 177)
\end{exe}

It also occurs with personal pronouns, as in (\ref{ex:mWNi.zW}), a sentence where the narrator talks about his personal situation, as opposed to that of his parents who were mentioned in the previous lines.

\begin{exe}
\ex \label{ex:mWNi.zW}
\gll aʑo nɯ, mɯŋi zɯ kɯ-rɤ-βzjoz ɲɯ-ɕe-a pɯ-ŋu. \\
\textsc{1sg} \textsc{dem} pl.n. \textsc{loc} \textsc{nmlz}:S/A-\textsc{antipass}-learn \textsc{ipfv}:\textsc{west}-go-\textsc{1sg} \textsc{pst}.\textsc{ipfv}-be \\
\glt `As for me, I was going to school in Mungi.' (2010-09, 22)
\end{exe}

In its function as a topicalizer, the determiner \forme{nɯ} can follow a noun with postnominal demonstratives, as in (\ref{ex:kWki.nW}). However, due to the difficulty of systematically sorting out the topicalization and demonstrative functions of this marker, I do not attempt to reflect this distinction in the glosses, and use  \textsc{dem} everywhere.

\begin{exe}
\ex \label{ex:kWki.nW}
\gll tɕeri kɯki mɯntoʁ kɯki nɯ pɯpɯŋunɤ, wuma ʑo kɯ-ʑru, kɯ-pe, \\
\textsc{lnk} \textsc{dem}.\textsc{prox} flower \textsc{dem}.\textsc{prox} \textsc{dem} as.far really \textsc{emph} \textsc{nmlz}:S/A-be.strong \textsc{nmlz}:S/A-be.good \\ 
\glt `But concerning this flower, so precious and nice' (150820 meili de meiguihua, 58)
\end{exe}

\subsubsection{The linker \forme{tɕe} as a topic marker} \label{sec:tCe.topic}
The word \forme{tɕe}, which originates from a locative postposition (§\ref{sec:locative.j}), is mainly used in Japhug as a linker (§ XXX) and as a postposition (§ \ref{sec:tCe.postposition}), one of the most common words in the corpus.

In addition, it can serve as a topic marker, following left-dislocated noun or postpositional phrases (\ref{ex:tsuku.kW.tCe}).

\begin{exe}
\ex \label{ex:tsuku.kW.tCe}
\gll tsuku kɯ tɕe lɤpɯɣ ra mbɯsɯt chɯ-lɤt-nɯ tɕe nɯra ɲɯ-rku-nɯ ɲɯ-ŋu \\
some \textsc{erg} \textsc{lnk} radish \textsc{pl} grating \textsc{ipfv}-throw-\textsc{pl} \textsc{lnk} \textsc{dem}.\textsc{pl} \textsc{ipfv}-put.in-\textsc{pl} \textsc{sens}-be \\
\glt `Some people, they grate radish and use it as filling (for the sausage).' (05-paR, 77)
\end{exe}

 \subsection{Focus markers} \label{sec:focus}
   \subsubsection{Unexpected focus} \label{sec:unexpected}
  The unexpected/high degree marker \forme{rcanɯ} or \forme{rca}, which was grammaticalized from the  secutive relator noun \japhug{ɯ-rca}{following} (§ \ref{sec:secutive}). It indicates that the phrase or clause preceding it is topical, and the situation or action described by the predicate that follows is unexpected (\ref{ex:nAZo.rcanW}), intensifies to a noticeable (and not foreseeable) extent (\ref{ex:tokAnWmqajndZic.tCe.rcanW}) or occurs with a remarkably high degree or intensity, with  (\ref{ex:mbro.rcanW}) or without (\ref{ex:apWme.rcanW}) surprise.

\begin{exe}
\ex \label{ex:nAZo.rcanW}
 \gll  wo nɤʑo rcanɯ tɕʰi ɲɯ-tɯ-nɤme ŋu ma,  aʑo tɯ-mɯ kɯ pɯ-kɯ-sɯ-χtɕi-a, tɤndʐo nɯ! \\
 \textsc{interj} \textsc{2sg} \textsc{foc}:\textsc{unexp} what \textsc{sens}-2-do[III] be:\textsc{fact} \textsc{lnk} \textsc{1sg} \textsc{indef}.\textsc{poss}-sky \textsc{erg} \textsc{pfv}-2\fl{}1-\textsc{caus}-wash-\textsc{1sg} cold \textsc{sfp} \\
\glt `You, what are you doing, you caused me to be drenched by the rain.' (kWlAG2014, 157) \\
\end{exe}

\begin{exe}
\ex \label{ex:tokAnWmqajndZic.tCe.rcanW}
 \gll to-k-ɤnɯmqaj-ndʑi-ci tɕe rcanɯ, ʑɯrɯʑɤri tɕe ko-k-ɤndɯndo-ndʑi-ci, \\
 \textsc{ifr}-\textsc{evd}-\textsc{recip}:scold-\textsc{du-evd} \textsc{lnk}  \textsc{foc}:\textsc{unexp} progressively \textsc{lnk}   \textsc{ifr}-\textsc{evd}-\textsc{recip}:take-\textsc{du-evd} \\
 \glt `They scolded each other and progressively started to fight, ' (lWlu2002, 52)
\end{exe}

 \begin{exe}
\ex \label{ex:mbro.rcanW}
 \gll mbro rcanɯ ɯ-xɕɤt kɯ-tɯ\redp{}tu ʑo nɯ-ntsʰɤr ɲɯ-nu, \\
 horse \textsc{foc}:\textsc{unexp} \textsc{3sg}.\textsc{poss}-strength \textsc{nmlz}:S/A-\textsc{emph}\redp{}exist \textsc{emph} \textsc{pfv}-neigh \textsc{sens}-be \\ 
 \glt `The horse neighed with all his strength.' (qachGa2003, 158)
\end{exe}

The marker \forme{rcanɯ} is particularly common in the degree construction with a \forme{tɯ-} degree nominal (§ XXX), as in (\ref{ex:apWme.rcanW}). In this particular construction,  \forme{rcanɯ} does not necessarily express unexpectedness.

\begin{exe}
\ex \label{ex:apWme.rcanW}
 \gll  tɕe nɯnɯ lɯlu a-pɯ-me rcanɯ, βʑɯ ɯ-tɯ-ŋɤn saχaʁ. \\
 \textsc{lnk} \textsc{dem} cat \textsc{irr}-\textsc{ipfv}-not.exist \textsc{foc}:\textsc{unexp} mouse
 \textsc{3sg}.\textsc{poss}-\textsc{nmlz}:\textsc{degree}-be.evil be.extremely:\textsc{fact} \\ 
 \glt `If there are no cats, the mice are extremely fierce (cause a lot of damages).' (21-lWlu, 32) 
\end{exe}

 \subsubsection{Additive and scalar focus marker \forme{kɯnɤ} } \label{sec:kWnA}
The additive and scalar focus marker \japhug{kɯnɤ}{also, even} follows the constituent over which it has scope, which can be noun phrases, postpositional phrases but also subordinate clauses (these are treated in § XXX). As with other function words with the syllable \forme{nɤ} as last element (§ XXX), the stress is on the first syllable (\forme{kɯ́nɤ}) and the vowel on the second syllable is often elited (a pronunciation \forme{kɯn} is often heard). 

The marker \forme{kɯnɤ} expresses both additive focus, as in (\ref{ex:aZo.kWNA.staRlupa}), and scalar focus, as in (\ref{ex:WNgWz.kWnA.tunAndWtnW}) in affirmative sentences. It is also compatible with negative verb forms, as in (\ref{ex:tWrdoR.kWnA}), expressing the meaning `not even' (see also \japhug{cinɤ}{(not) even one} in § \ref{sec:cinA}).

\begin{exe}
\ex \label{ex:aZo.kWNA.staRlupa}
\gll aʑo kɯnɤ staʁlupa ŋu-a tɕe \\
\textsc{1sg} also born.in.the.tiger.year be:\textsc{fact}-\textsc{1sg} \textsc{lnk} \\
\glt `Me too (like you), I am of the Tiger year.' (2011-05-nyima, 168)
\end{exe}

\begin{exe}
\ex \label{ex:WNgWz.kWnA.tunAndWtnW}
\gll ʑara ʑo ɯ-ŋgɯz kɯnɤ tu-nɤndɯt-nɯ tɕe nɯ kɯ-βʁa ɣɤʑu, kɯ-nŋo ɣɤʑu qʰe, \\
\textsc{3pl} \textsc{emph} \textsc{3sg}.\textsc{poss}-among:\textsc{loc} also \textsc{ipfv}-fight-\textsc{pl} \textsc{lnk} \textsc{dem} \textsc{nmlz}:S/A-win \textsc{sens}:exist \textsc{nmlz}:S/A-lose  \textsc{sens}:exist \textsc{lnk} \\
\glt `Even among themselves, they fight, and there are winners and losers.' (20-sWNgi, 62-63)
\end{exe}
 
\begin{exe}
\ex \label{ex:tWrdoR.kWnA}
\gll tɯ-sŋi mɯntoʁ tɯ-rdoʁ kɯnɤ ci ci tɕe mɯ́j-stʰɯt \\
one-day flower one-piece also once once \textsc{lnk} \textsc{neg}:\textsc{sens}-finish \\
\glt `Sometimes one cannot finish even one pattern (on the belt) in one day.' (2011-06-thaXtsa, 47)
\end{exe}

As an additive focus marker, \forme{kɯnɤ} can be repeated on all the nouns designating the members of a group sharing a particular property, in the construction $X$ \forme{kɯnɤ}, $Y$ \forme{kɯnɤ}  `both $X$ and $Y$', as in (\ref{ex:Dpalcan.kWnA}).

\begin{exe}
\ex \label{ex:Dpalcan.kWnA}
 \gll a-pɯ-ŋu tɕe, aʑo kɯnɤ taʁrdo rɟitpa a-pɯ-ŋu-a, χpɤltɕin kɯnɤ taʁrdo rɟitpa a-pɯ-ŋu, ... nɯ tɕi-rɟit nɯni tɕe taʁrdo rɟitpa ma nɯ ma kɯmaʁ rɟitpa nɯ kɤ-rtsi me.  \\
 \textsc{irr}-\textsc{ipfv}-be \textsc{lnk} \textsc{1sg} also pl.n. lineage  \textsc{irr}-\textsc{ipfv}-be-\textsc{1sg}  p.n. also pl.n. lineage  \textsc{irr}-\textsc{ipfv}-be { } \textsc{dem} \textsc{1du}.\textsc{poss}-offspring \textsc{dem}:\textsc{du} \textsc{lnk} pl.n. lineage \textsc{lnk} \textsc{dem} apart.from other lineage \textsc{dem} \textsc{nmlz}:O-count not.exist:\textsc{fact} \\
 \glt `For instance suppose that both Dpalcan and I were from Taqrdo lineage, then our two children would only count as members of the Taqrdo lineage and no other lineage.' (140426 rJitpa, 13-15)
\end{exe}

The scope of  \forme{kɯnɤ} is generally exclusively on the constituent that it immediately follows, but there are cases where the scope is more extensive. In (\ref{ex:aZo.kWnA.akAsWso}), \forme{kɯnɤ} occurs between the pronoun \forme{aʑo} and the following participial verb form, which bears a \textsc{1sg} possessive prefix \forme{a-} coreferent with that pronoun (see also \ref{ex:aZWG.kWnA} below). The semantic scope of \forme{kɯnɤ} here is on the whole relative \forme{aʑo a-kɤ-sɯso} `(the things) that I want' rather than exclusively on the pronoun \forme{aʑo}.

\begin{exe}
\ex \label{ex:aZo.kWnA.akAsWso}
 \gll aʑo kɯnɤ a-kɤ-sɯso nɯ tɤ-stu-nɯ ra \\
 \textsc{1sg} also \textsc{1sg}.\textsc{poss}-\textsc{nmlz}:O-think \textsc{dem} \textsc{imp}-do.like-\textsc{pl} have.to:\textsc{fact} \\
 \glt `(I will do as you say, but) do also the things I want.' (2003kAndzwsqhaj2, 47)
\end{exe}

The focus marker \forme{kɯnɤ} is found with nouns or pronouns in core argument function, including S (\ref{ex:kWnA.nArca}), O (\ref{ex:nWXpWm.kWnA}), and semi-objects (\ref{ex:kWnA.mAsna}).  Examples with transitive subjects are presented below (\ref{ex:nWra.kWnA} and \ref{ex:Wzda.ra.kWnA}).

 \begin{exe}
\ex \label{ex:kWnA.nArca}
\gll aʑo kɯnɤ nɤ-rca ɣi-a ɕti  \\
\textsc{1sg} also \textsc{2sg}.\textsc{poss}-following come:\textsc{fact}-\textsc{1sg} be.\textsc{affirm}:\textsc{fact} \\
\glt `I am coming with you too.' (2011-05-nyima, 171)
 \end{exe}
 
   \begin{exe}
\ex \label{ex:nWXpWm.kWnA}
\gll    ma nɯ-χpɯm kɯnɤ kʰro mɤ-kɯ-fkaβ kɯ-fse ku-rɤʑi-nɯ  \\
lnk 3pl.poss-knee also much \textsc{neg}-\textsc{nmlz}:S/A-cover \textsc{nmlz}:S/A-be.like \textsc{ipfv}-stay-\textsc{pl} \\
\glt `(Gents) would (wear trousers that did) not cover much even their knees.'  (30-rkAsnom, 5) 
  \end{exe}
  
  \begin{exe}
 \ex \label{ex:kWnA.mAsna}
 \gll   ɯ-ru nɯra laʁdɯn ɯ-jɯ kɯnɤ mɤ-sna, ma mɤ-ngɯt. \\
 \textsc{3sg}.\textsc{poss}-trunk \textsc{dem}:\textsc{pl} tool \textsc{3sg}.\textsc{poss}-handle also \textsc{neg}-be.worth \textsc{lnk}  \textsc{neg}-be.strong:\textsc{fact} \\
 \glt `(The wood from) its trunk is not even good (enough to be used to make) tool handles, as it is not strong.'  (17-xCAj, 79)
  \end{exe}

It also occurs with all types of oblique arguments and adjuncts, including genitive (\ref{ex:aZWG.kWnA}), dative (\forme{ɯ-ɕki} \ref{ex:nWCki.kWnA}),  locational adjuncts in \forme{tɕu} (\ref{ex:kutCu.kWnA}) or \forme{ri} (\ref{ex:ri.kWnA}), temporal adjuncts (\ref{ex:ftCAXcAl.kWnA}) or adjuncts expressing manner or cause (\ref{ex:nWtCu.kWnA2}).  
  
   \begin{exe}
\ex \label{ex:aZWG.kWnA}
\gll aʑɯɣ kɯnɤ a-mpʰrɯmɯ a-pɯ-tɯ-sɯ-re ɯ-tɯ́-cʰa \\
\textsc{1sg}:\textsc{gen} also \textsc{1sg}.\textsc{poss}-divination \textsc{irr}-\textsc{pfv}-2-\textsc{caus}-look[III] \textsc{qu}-2-can:\textsc{fact} \\
\glt `Can you ask (the monk) to make a divination for me too?' (The divination, 31)
\end{exe}  
  
   \begin{exe}
\ex \label{ex:nWCki.kWnA}
\gll  tɯ-pi ɣɯ ɯ-nmaʁ ra nɯ-ɕki kɯnɤ `a-pi' tu-kɯ-ti ɕti ma nɯ ma kupa kɯ-fse ʑaka ɯ-rmi me. \\
\textsc{genr}.\textsc{poss}-elder.sibling \textsc{gen} \textsc{3sg}.\textsc{poss}-husband \textsc{pl} \textsc{3pl}.\textsc{poss}-\textsc{dat} also \textsc{1sg}.\textsc{poss}-elder.sibling \textsc{ipfv}-\textsc{genr}-say be.\textsc{affirm}:\textsc{fact} \textsc{lnk} \textsc{dem} apart.from Chinese \textsc{nmlz}:S/A-be.like each \textsc{3sg}.\textsc{poss}-name not.exist:\textsc{fact} \\
\glt  `One calls one's sister's husband (and others from his family) `my elder brother', there are no other special terms as in Chinese.' (140425 kWmdza05)
\end{exe}


  \begin{exe}
\ex \label{ex:kutCu.kWnA}
\gll  kutɕu kɯnɤ nɯ ɲɯ-fse, jɯfɕɯndʐi ra kɯ-xtɕɯ\redp{}xtɕi tɤ-ɣɤndʐo kɯ-fse ri, ɕɤxɕo tɕe kɯ-xtɕɯ\redp{}xtɕi ɲɯ-ʑi kɯ-fse \\
here also \textsc{dem} \textsc{sens}-be.like a.few.days.ago \textsc{nmlz}:S/A-\textsc{emph}\redp{}be.small \textsc{pfv}-be.cold \textsc{nmlz}:S/A-be.like \textsc{lnk} the.last.days \textsc{lnk} \textsc{nmlz}:S/A-\textsc{emph}\redp{}be.small \textsc{sens}-subside \textsc{nmlz}:S/A-be.like \\
\glt `It is like that here too, a few days ago the weather became a little cold, but the last days it has eased a bit.' (conversation, 141027)
  \end{exe}
  
    \begin{exe}
\ex \label{ex:ri.kWnA}
\gll   maldzɯ nɯ, nɯ ɯ-tʰɤcu tsa ri kɯnɤ ɣɤʑu. qarɣɤpɤt ɯ-rca ri kɯnɤ tu-ɬoʁ ɲɯ-ŋu. \\
plant.name \textsc{dem} \textsc{dem} \textsc{3sg}.\textsc{poss}-downstream a.little \textsc{loc} also exist:\textsc{sens} plant.name \textsc{3sg}.\textsc{poss}-among \textsc{loc} also \textsc{ipfv}-come.out \textsc{sens}-be \\
\glt `The \forme{maldzɯ} plant, it is also found in places of slightly lower altitude, but grows also in the same places as  \forme{qarɣɤpɤt} plants.' (18-qromJoR, 81-82)
    \end{exe}
    
\begin{exe}
\ex \label{ex:ftCAXcAl.kWnA}
\gll   kukutɕu ftɕɤχcɤl kɯnɤ <baonuanyi> tu-tɯ-ŋge pɯ-ɕti. \\
  here mid.summer also warm.clothes \textsc{ipfv}-2-wear[III] \textsc{pst}.\textsc{ipfv}-be.\textsc{affirm} \\
  \glt `Here you were wearing warm clothes even in mid summer.' (conversation, 141017)
    \end{exe}
    
    \begin{exe}
\ex \label{ex:nWtCu.kWnA2}
\gll    tɕe nɯtɕu kɯnɤ ɯ-jaʁ ɯ-ntsi tɤɲi pjɯ-sɤtse, ɯ-jaʁ ɯ-ntsi kɯ tsʰitsuku ɲɯ-z-nɤme qʰe, \\
\textsc{lnk} \textsc{dem}:\textsc{loc} also \textsc{3sg}.\textsc{poss}-hand \textsc{3sg}.\textsc{poss}-one.of.a.pair erg various.things \textsc{ipfv}-\textsc{caus}-do[III] \textsc{lnk}  \\
\glt `Even like that (despite the pain in her legs), she props herself with a cane using one hand, and does all kinds of things with her other hand.' (14-tApitaRi, 52)
\end{exe}

Although \japhug{kɯnɤ}{also, even} can be combined with most postpositions and relator nouns as shown by the examples above, it is however incompatible with the ergative \forme{kɯ}. For instance, in  (\ref{ex:nWra.kWnA}), although the demonstrative pronoun \forme{nɯra} `they, those' in the second clause is the subject of the transitive verb \japhug{ndza}{eat}, it does not take the ergative \forme{kɯ} as would be expected (§ \ref{sec:A.kW}). The same applies to \forme{ɯ-zda ra} `his companions', subject of the transitive verb \forme{na-nɯ-ɕar-nɯ} `they looked for themselves' in (\ref{ex:Wzda.ra.kWnA}), 

  \begin{exe}
\ex \label{ex:nWra.kWnA}
\gll ɯ-pɯ nɯra li ju-ɣi-nɯ qʰe, nɯra kɯnɤ ɣɯ-tu-ndza-nɯ. \\
\textsc{3sg}.\textsc{poss}-young \textsc{dem}:\textsc{pl} again \textsc{ipfv}-come-\textsc{pl} \textsc{lnk} \textsc{dem}:\textsc{pl} also \textsc{cisloc}-\textsc{ipfv}-eat-\textsc{pl} \\
\glt `Its youngs also come and they too eat it.' (20-sWNgi, 59-60)
  \end{exe}
  
    \begin{exe}
\ex \label{ex:Wzda.ra.kWnA}
\gll   ɯ-zda ra kɯnɤ nɯ-rʑaβ tɯka na-nɯ-ɕar-nɯ ɲɯ-ŋu \\
\textsc{3sg}.\textsc{poss}-companion \textsc{pl} also \textsc{3sg}.\textsc{poss}-wife each \textsc{pfv}:3\fl{}3'-\textsc{auto}-search \textsc{sens}-be \\
\glt `His companions also took each a wife for himself (among the women of the island).' (2005Norbzang, 44)
    \end{exe}
    
The combinations $\dagger$\forme{kɯ kɯnɤ} or $\dagger$\forme{kɯnɤ kɯ} are unattested, and not accepted by native speakers. The contrast between absolutive and ergative noun phrases is therefore neutralized in additive or scalar focus with \forme{kɯnɤ}. Note that other focus markers, such as \forme{ri} and \forme{tɕi} (see \ref{ex:tCi.ndze} in § \ref{sec:ri.additive}) differ from \forme{kɯnɤ} in this regard.

Four distinct facts converge to suggest that the first syllable of \forme{kɯnɤ} is historically related to the ergative postposition \forme{kɯ}: (i) the incompatibility of co-occurrence of \forme{kɯnɤ} and \forme{kɯ}; (ii) the stress on the first syllable in \forme{kɯ́nɤ}; (iii) the similar \forme{-nɤ} element in the other scalar focus marker \japhug{cinɤ}{(not) even one} (§ \ref{sec:cinA}) (iv) the existence of the linker \forme{nɤ}, possibly of Tibetan origin (§ XXX). A detailed examination of this topic is however impossible on the basis Japhug-internal evidence, and will require extensive syntactic comparison between Gyalrong languages.

The adverb \japhug{tɤmtɯkɯnɤ}{specially, on purpose} (§ XXX) appears to be a lexicalized combination of the noun \japhug{tɤ-mtɯ}{knot} and the focus marker \forme{kɯnɤ}.

 \subsubsection{Correlative additive focus markers \forme{ri} and \forme{tɕi}} \label{sec:ri.additive} 
 The additive focus markers \forme{ri} and \forme{tɕi}  are used in enumerations, repeated after each noun referring to  members of a group, to focus on the fact that their referents share a common property (or properties that are semantically close enough), as in (\ref{ex:ri.kWsthWci.WWmpCar}) and (\ref{ex:tCi.tulhoR.cha}) (see additional examples in \citealt[313-314]{jacques14linking}).
 
 \begin{exe}
\ex \label{ex:ri.kWsthWci.WWmpCar}
 \gll  a-rʑaβ ri kɯstʰɯci ɲɯ-mpɕɤr, a-mbro ri kɯstʰɯci ɲɯ-ʑru, a-pɣɤtɕɯ ri kɯstʰɯci ɲɯ-mpɕɤr tɕe, \\
 \textsc{1sg}.\textsc{poss}-wife also so.much \textsc{sens}-be.beautiful  \textsc{1sg}.\textsc{poss}-horse also so.much \textsc{sens}-be.strong  \textsc{1sg}.\textsc{poss}-bird also so.much \textsc{sens}-be.beautiful \textsc{lnk} \\
 \glt `My wife is so beautiful, my horse so strong, my bird so beautiful.' (2003qachga, 116)
 \end{exe}
 
  \begin{exe}
\ex \label{ex:tCi.tulhoR.cha}
 \gll  ɴqiaβ tɕi tu-ɬoʁ cʰa, zrɯ tɕi tu-ɬoʁ cʰa, \\
 dark.side.of.the.mountain also \textsc{ipfv}-come.out can:\textsc{fact}   sunny.side.of.the.mountain also \textsc{ipfv}-come.out can:\textsc{fact}  \\
 \glt `It can grow in both the dark and the sunny sides of the mountains.' (17-thowum, 14)
  \end{exe}
  
The correlative focus markers \forme{ri} and \forme{tɕi} can occur after any noun phrase or postpositional phrase, including with the ergative  \forme{kɯ} as shown by (\ref{ex:tCi.ndze}), unlike the marker \japhug{kɯnɤ}{even, also} (see examples \ref{ex:nWra.kWnA} and \ref{ex:Wzda.ra.kWnA}, § \ref{sec:kWnA}).
  
  \begin{exe}
\ex \label{ex:tCi.ndze}
 \gll paʁ kɯ tɕi ndze, nɯŋa kɯ tɕi ndze, jla kɯ tɕi ndze.   \\
 pig \textsc{erg} also eat[III]:\textsc{fact}  cow \textsc{erg} also eat[III]:\textsc{fact}  hybrid.yak \textsc{erg} also eat[III]:\textsc{fact}  \\
 \glt `Pigs eat it, cows eat it, hybrid yaks eat it.' (18-NGolo, 171)
  \end{exe}

The focus markers \forme{ri} and \forme{tɕi} can have scope on only part of the noun/propositional phrase, and even on the relator nouns as in (\ref{ex:WNgW.tCi}).

   \begin{exe}
\ex \label{ex:WNgW.tCi}
 \gll   sɤtɕʰa ɯ-ŋgɯ tɕi ɣɤʑu, sɤtɕʰa ɯ-taʁ tɕi ʑo ɣɤʑu \\
 ground \textsc{3sg}.\textsc{poss}-inside also exist:\textsc{sens}  ground \textsc{3sg}.\textsc{poss}-inside also \textsc{emph} exist:\textsc{sens} \\
 \glt `It is found both inside the ground, and on the ground.' (25-GdAso, 17)
    \end{exe}
    
Alternatively, it is possible to enumerate distinct related properties of the same referent using \forme{ri} (this usage is not found with \forme{tɕi}), but that marker still follows the noun phrase (correlative \forme{ri} can follow verbs, but only in a specific construction, see \ref{ex:ri.kWmWm.ri} below). In this case the referent cannot be elided, and must be repeated in both clauses, at least as a third person pronoun \forme{ɯʑo} as in (\ref{ex:WlWz.ri.pjAxtCi}). 

  \begin{exe}
\ex \label{ex:WlWz.ri.pjArZi}
 \gll pʰaʁrgot nɯnɯ ɯʑo ri pjɤ-rʑi, ɯʑo ri pjɤ-tsʰu tɕe \\
 boar \textsc{dem} \textsc{3sg} also \textsc{ifr}.\textsc{ipfv}-be.heavy \textsc{3sg} also \textsc{ifr}.\textsc{ipfv}-be.fat \textsc{lnk} \\ 
\glt  `The boar, it was heavy and fat.' (140428 yonggan de xiaocaifeng-zh, 244)
 \end{exe}

A variant of this construction is found with internally-headed relative clauses in apposition, taking the third person pronoun \forme{ɯʑo} as head, as in (\ref{ex:WZo.ri.kWwxti}).

\begin{exe}
\ex \label{ex:WZo.ri.kWwxti}
\gll  [ɯʑo ri kɯ-wxti], [ɯʑo ri kɯ-sɤjlɯ\redp{}jloʁ] ci pjɤ-ŋu. \\
\textsc{3sg} also \textsc{nmlz}:S/A-be.big \textsc{3sg} also \textsc{nmlz}:S/A-\textsc{emph}\redp{}be.big \textsc{indef} \textsc{ifr}.\textsc{ipfv}-be \\
\glt `(The toad) was a big and disgusting (creature).' (150818 muzhi guniang, 86)
\end{exe}

 
The correlative construction can involve the possessor of an IPN, as in (\ref{ex:WlWz.ri.pjAxtCi}), where in the first clause the referent `the girl' is possessor of the intransitive subject (literally `her age was small', § XXX) and in second it corresponds to the intransitive subject, realized as a third person pronoun \forme{ɯʑo} `she'.

  \begin{exe}
\ex \label{ex:WlWz.ri.pjAxtCi}
 \gll tɕʰeme nɯ ɯ-lɯz ri pjɤ-xtɕi, ɯʑo ri pjɤ-mpɕɤr,  \\
 girl \textsc{dem} \textsc{3sg}.\textsc{poss}-age also \textsc{ifr}.\textsc{ipfv}-be.small \textsc{3sg} also \textsc{ifr}.\textsc{ipfv}-be.beautiful \\
\glt `The girl was young and beautiful.' (150909 hua pi-zh, 10)
 \end{exe}
 
 More complex correlations, involving different subjects and predicates related to another referent, are also possible as shown by example (\ref{ex:lWlu.kW}), where \forme{ri} occurs after the intransitive subject \japhug{tɯ-ci}{water}, after the transitive subject \japhug{lɯlu}{cat} with the ergative and after the finite verb \japhug{tu-ɕe}{it goes up} (on which see below and refer to § XXX).
 
 \begin{exe}
\ex   \label{ex:lWlu.kW}
\gll <yancong> ku-kɯ-rɤloʁ tɕe ɯ-taʁ tɯ-ci ri mɯ́j-ɣi lɯlu kɯ ri mɯ-ɲɯ́-wɣ-ɕaβ qapri tu-ɕe ri mɯ́j-cʰa tɕe \\
 chimney \textsc{ipfv}-\textsc{genr}:S/P-make.a.nest \textsc{lnk} \textsc{3sg}.\textsc{poss}-on \textsc{indef}.\textsc{poss}-water also \textsc{neg}:\textsc{sens}-come cat \textsc{erg} also \textsc{neg}-\textsc{ipfv}-\textsc{inv}-catch snake \textsc{ipfv}:\textsc{up}-go also \textsc{neg}:\textsc{sens}-can \textsc{lnk} \\
 \glt `(The sparrows) make their nest in the chimney, (because) water cannot come up there, the cats cannot catch them, and the snakes cannot go up there.' (22-kumpGatCW, 69)
 \end{exe}
 
 The marker \forme{ri} is homophonous with the locative \forme{ri} (§ \ref{sec:locative}), and in cases with an enumeration of locative adjuncts, there can be ambiguity between the two. In (\ref{ex:Xcha.ri.ci}), \forme{ri} is analyzed as a locative because of the position of the determiner \forme{ci}, and also because it can be replaced with other locative postpositions.
 
 \begin{exe}
\ex \label{ex:Xcha.ri.ci}
\gll   χcʰa ri ci, ɯ-ʁe ri ci ɯ-jme cʰɯ-ɬoʁ ɲɯ-ŋu. \\
right \textsc{loc} one  \textsc{3sg}.\textsc{poss}-left \textsc{loc} one \textsc{3sg}.\textsc{poss}-tail \textsc{ipfv}:\textsc{downstream}-come.out \textsc{sens}-be \\
\glt `It has one tail on the right, and one on the left.' (26-qro, 116)
\end{exe}

The marker \forme{ri} can follow verbs only if combined with an existential verb, a copula or a modal auxiliary verb as main predicate (meaning `both $X$ and $Y$' with positive copulas, and `neither $X$ nor $Y$' with negative ones). In this type of construction, verbs are mostly in non-finite form, as in (\ref{ex:ri.kWmWm.ri}). Examples with finite verbs however do exist; this topic is treated in § XXX. %ɲɯ-ɣɤwu ri kɯ-maʁ, ɲɯ-nɤre ri kɯ-maʁ kɯ-fse ɲɤ-k-ɤβzu-ci  ; tu-rɯɕmi ri mɤ-kɯ-khɯ, chɯ-nɯrɤɣo ri mɤ-kɯ-khɯ ci ɲɤ-k-ɤβzu-ci. ; tu-ndzur ri pjɤ-maʁ, ku-omdzɯ ri pjɤ-maʁ.

 \begin{exe}
\ex \label{ex:ri.kWmWm.ri}
 \gll   nɯ pɯ́-wɣ-ta ri  kɯroz kɯ-mɯm ri maŋe, kɯroz mɤ-kɯ-ɣɤ-mɲɤt ri maŋe qʰe, \\
 \textsc{dem} \textsc{pfv}-\textsc{inv}-put \textsc{lnk} specially \textsc{nmlz}:S/A-be.tasty also not.exist:\textsc{sens} specially \textsc{neg}-\textsc{nmlz}:S/A-\textsc{facil}-be.spoiled also not.exist:\textsc{sens} \textsc{lnk} \\
 \glt `When if one puts (a seal on the bread), there is nothing especially tasty about it, and nothing special concerning the preservation (of the bread).' (160706 thotsi, 27)
  \end{exe}
  

  
 \subsubsection{Scalar focus marker \forme{cinɤ}} \label{sec:cinA} 
 The focus marker \japhug{cinɤ}{(not) even one} exclusively occurs with a negative verb. Like \japhug{kɯnɤ}{also, even}, this marker has stress on the first syllable \forme{cínɤ}, which is obviously related to the numeral \japhug{ci}{one} (§ \ref{sec:one.to.ten}, § \ref{sec:indef.article}).
 
 The marker \forme{cinɤ} has scope over the constituent that immediately precedes it, generally a noun phrase including or consisting of a CN, as in (\ref{ex:tWrdoR.cinA3}), but also object and subject participial relative clauses as in (\ref{ex:zrWG.kAmto.cinA}), (\ref{ex:WrNa.WkWru.cinA}) and (\ref{ex:lukWpGaR.nW.cinA}).
 
 \begin{exe}
\ex \label{ex:tWrdoR.cinA3}
\gll tsuku kɯ qʰe tɯ-rdoʁ cinɤ mɤ-kɯ-mto tu. \\
some erg lnk one-piece even neg-nmlz:S/A-see exist:fact \\
\glt `There are some people who (cannot) even find a single one.' (20-grWBgrWB, 36)
 \end{exe} 

 \begin{exe}
\ex \label{ex:zrWG.kAmto.cinA}
\gll  ma tɕe jinde nɯ zrɯɣ kɤ-mto cinɤ maŋe. \\
\textsc{lnk} \textsc{lnk} nowadays \textsc{dem} louse \textsc{nmlz:P}-see even not.exist:\textsc{sens} \\
\glt `Nowadays there isn't even a single louse to be seen/one cannot even see a single louse.' (21-mdzadi, 77)
\end{exe} 

\begin{exe}
\ex \label{ex:WrNa.WkWru.cinA}
\gll ɯ-rŋa ɯ-kɯ-ru cinɤ ʑo pjɤ-me \\
3sg.poss-face 3sg.poss-nmlz:S/A-look even \textsc{emph} \textsc{ipfv}.\textsc{ifr}-not.exist \\
\glt `Not even one (of the thieves) looked at it/The (thieves) did not even so much as looked at it.' (140426 luozi he qiangdao)
\end{exe}

\begin{exe}
\ex \label{ex:lukWpGaR.nW.cinA}
\gll tɕe ɯ-ɲɯ-kɯ-ɣɤ-rkɯn nɯ ɲɯ-dɤn ma lu-kɯ-pɣaʁ nɯ tɯ-rdoʁ cinɤ ʑo maŋe \\
\textsc{lnk} \textsc{3sg}.\textsc{poss}-\textsc{ipfv}-\textsc{nmlz}:S/A-\textsc{caus}-be.few \textsc{dem} \textsc{sens}-be.many \textsc{lnk} \textsc{ipfv}:\textsc{upstream}-\textsc{nmlz}:S/A-plough \textsc{dem} one-piece even \textsc{emph} not.exist:\textsc{sens} \\
\glt `A lot of people diminish their fields, and not a single of them opens new fields.' (150903 friche, 6)
\end{exe}

In the case of relative clauses before \forme{cinɤ}, there is some ambiguity as to whether the scope of the focus marker is on the head of the relative or on the main verb of the relative clause, hence the two proposed translations above for (\ref{ex:zrWG.kAmto.cinA}) and (\ref{ex:WrNa.WkWru.cinA}).

It is not possible to use \forme{cinɤ} with scope over transitive subjects, followed by the ergative.

The form \forme{cinɤ} also occurs in the expression \forme{ŋu cinɤ maʁ kɯ} `in any case it is not', as in (\ref{ex:Nu.cinA.maR.kW}), literally `It is not even the case that...' ; in this construction, only the first verb \japhug{ŋu}{be} receives person indexation, as shown by (\ref{ex:Nua.cinA.maR.kW}). In addition to \japhug{ŋu}{be}, a few other verbs such as \japhug{fse}{be like} can occur with \forme{ci nɤ maʁ kɯ} `anyway X does not' .

 \begin{exe}
\ex \label{ex:Nu.cinA.maR.kW}
\gll qajdo kɯ tɕʰi mɤ-nɯ-ti ɕti nɤ, a-tɤ-nɯ-ti ma ŋu cinɤ maʁ kɯ, nɯ sɤznɤ kɯ-scɯ-scit rɤʑi-tɕi \\
crow \textsc{erg} what \textsc{neg}-\textsc{auto}-say:\textsc{fact} be.\textsc{affirm}:\textsc{fact} \textsc{lnk} \textsc{irr}-\textsc{pfv}-\textsc{auto}-say \textsc{lnk} be:\textsc{fact} even not.be:\textsc{fact} \textsc{sfp} \textsc{dem} \textsc{comp} \textsc{nmlz}:S/A-\textsc{emph}\redp{}happy stay:\textsc{fact}-\textsc{1du} \\
\glt `What would not a crow say (a crow tells only lies), let it say as it wants, in any case it is not (true), let us rather live (together) happily.' (28-qAjdoskAt, 28)
\end{exe} 

 \begin{exe}
\ex \label{ex:Nua.cinA.maR.kW}
\gll  kɯ-mɯrkɯ ŋu-a cinɤ maʁ kɯ  \\
\textsc{nmlz}:S/A-steal be:\textsc{fact}-\textsc{1sg} even not.be \textsc{sfp} \\
\glt `Anyway it is not me who is the thief.' (elicited)
\end{exe}

\subsubsection{Restrictive focus} \label{sec:restrictive.focus} 
 The most common way to express restrictive focus in Japhug is to combine the exceptive \japhug{ma}{apart from} (and its reduplicated variant \forme{mɯma} § \ref{sec:exceptive}) with a negative predicate. This can be a verb with a negative prefix as in (\ref{ex:XsArZaR}), or a negative existential verb as in (\ref{ex:Wmi.Wntsi.ma.me}).
 
 \begin{exe}
\ex  \label{ex:XsArZaR}
\gll   χsɤ-rʑaʁ ma mɯ-pɯ-tsu-a ɲɤ-sɯso ri χsɯ-xpa pjɤ-tsu tɕe,  \\
three-day apart.from \textsc{neg}-\textsc{pfv}-pass-\textsc{1sg} \textsc{ifr}-think \textsc{lnk} three-year \textsc{ifr}-pass \textsc{lnk} \\
\glt `He thought that he had spent only three days, but three years had passed.' (2011-4-smanmi, 178)
  \end{exe}
  
  \begin{exe}
\ex  \label{ex:Wmi.Wntsi.ma.me}
\gll  rkoŋɟɤl nɯnɯ, ɯ-mi ɯ-ntsi nɯ ma me kʰi.   \\
one.legged.demon \textsc{dem} \textsc{3sg}.\textsc{poss}-leg \textsc{3sg}.\textsc{poss}-one.of.a.pair \textsc{dem} apart.from not.exist:\textsc{fact} \textsc{hearsay} \\
\glt  `It is said that one-legged demons only had one leg.' (140510 rkoNJAl, 4)
  \end{exe}
  
In the case of restrictive focalization on locative or temporal phrases,   the terminative \japhug{mɤɕtʂa}{until} (§ \ref{sec:terminative}) occurs instead of the exceptive.
  
\begin{exe}
\ex  \label{ex:nWnWtCu.mACtsxa.pjAme}
\gll  nɯnɯtɕu mɤɕtʂa mtsʰalu pjɤ-me qʰe \\
\textsc{dem}:\textsc{loc} until nettle  \textsc{ifr.ipfv}-not.exist \textsc{lnk} \\
\glt `It was only there that there was nettle.' =  `There was no nettle until there.' (140520 ye tian'e, 319)
\end{exe}
    
The restrictive focus construction implies the presence of a noun phrase with a numeral or a CN when the restriction bears on the quantity, but restriction can also be qualitative, without quantifier, as in (\ref{ex:karGi.Zo.kWfse.ma.me}).

\begin{exe}
\ex \label{ex:karGi.Zo.kWfse.ma.me}
 \gll   ɯ-mat nɯnɯ na-lɤt ɕɯmɯma nɤ kɯ-ndɯ\redp{}ndɯβ ʑo ma me, karɣi ʑo kɯ-fse ma me  \\
 \textsc{3sg}.\textsc{poss}-fruit \textsc{dem} \textsc{pfv}:3\fl{}3'-throw just \textsc{lnk}  \textsc{nmlz}:S/A-\textsc{emph}\redp{}small \textsc{emph} apart.from not.exist:\textsc{fact} turnip.seed \textsc{emph} \textsc{nmlz}:S/A-be.like apart.from not.exist:\textsc{fact} \\
 \glt  `When the fruit of (xanthoxyllum) has just come out, there is only something very small, only like a turnip seed.'  (07-tCGom, 7)
  \end{exe}
  
The restrictive focus construction can be combined with a scalar focus in \forme{kɯnɤ} (see §  \ref{sec:kWnA}), as in (\ref{ex:ma.kWme.kWnA}). In this example, \forme{kɯnɤ} has scope over the subordinate clause \forme{stɯsti ma kɯ-me}, which is ambiguous between a participial headless relative (§ XXX) `consisting of only a female all alone' and a manner infinitival clause (§ XXX; in this case the gloss of \forme{kɯ-me} would be \textsc{inf}:\textsc{stat}-not.exist) `even (when) there is only a female all alone'.

  \begin{exe}
\ex \label{ex:ma.kWme.kWnA}
\gll  mu ma, stɯsti ma kɯ-me kɯnɤ cʰɯ-rɤŋgɯm ɲɯ-ɕti. \\
female apart.from alone apart.from \textsc{nmlz}:S/A-not.exist also \textsc{ipfv}-lay.eggs \textsc{sens}-be.\textsc{affirm} \\
\glt `Even only a female (hen) alone does lay eggs.' (150819 kumpGa, 11)
\end{exe}
   
A second possibility to express restrictive focus is the use of the adverb \japhug{ʁɟa}{completely, all} (§ XXX) with scope on a  noun phrase rather than the whole clause as in (\ref{ex:RJa.tunWndze}).\footnote{The form \forme{ʁɟa} possibly originates from the first syllable of Tibetan \tibet{གཡའ་མ་}{gja.ma}{stone slab}, through a meaning `bare rock'.}  

\begin{exe}
\ex \label{ex:stAmku.RJa}
\gll alo mbroχpa ra tɕe tɕe nɤki qra cʰo qambrɯ ra ɣɯ nɯ-ɣli nɯnɯ
tɕe nɯ tu-wum-nɯ, tu-sɯɣ-rom-nɯ mbroχpa sɤtɕʰa tɕe stɤmku ʁɟa ɲɯ-ɕti ma si maŋe tɕe tɕe    \\
upstream nomad \textsc{pl} \textsc{lnk} \textsc{lnk} \textsc{filler} female.yak \textsc{comit} male.yak \textsc{pl} \textsc{gen} \textsc{3pl}.\textsc{poss}-dung \textsc{dem} \textsc{lnk} \textsc{dem} \textsc{ipfv}-gather-\textsc{pl} \textsc{ipfv}-\textsc{caus}-be.dry-\textsc{pl} nomad place \textsc{lnk} grassland completely \textsc{sens}-be.\textsc{affirm} \textsc{lnk} tree not.exist:\textsc{sens} \textsc{lnk} \textsc{lnk}  \\
\glt `Upstream, in the nomad areas, they gather and dry yak dung, as in nomad places there is only grassland, there no trees.' (05-tamar, 7-10)
\end{exe}

The adverb \forme{ʁɟa} (here used rather as a noun modifier) is related to the denominal verb \japhug{aʁɟa}{be bald, be bare} (see § XXX on the \forme{a-} derivation), which can be applied to nouns such as \japhug{stɤmku}{grassland} and \japhug{zgo}{mountain}.
 
\begin{exe}
\ex \label{ex:RJa.tunWndze}
 \gll qajɯ ʁɟa tu-nɯ-ndze, ma nɯ ma tɤ-rɤku kɯ-fse ra ndze mɤ-ŋgrɤl. \\
 bug completely \textsc{ipfv}-\textsc{auto}-eat[III] \textsc{lnk} \textsc{dem} apart.from \textsc{indef}.\textsc{poss}-harvest \textsc{nmlz}:S/A-be.like \textsc{pl} eat[III]:\textsc{fact} \textsc{neg}-be.usually.the.case:\textsc{fact} \\ 
\glt `It only eats insects, it does not eat cultivated plants.' (140511 qamtsWrmdzu, 16)
\end{exe}

While in (\ref{ex:RJa.tunWndze})  and (\ref{ex:stAmku.RJa}) it remains ambiguous whether \forme{ʁɟa} forms a syntactic constituent with the previous nouns or the following verb, in (\ref{ex:RJa.kW}) the presence of the ergative makes it clear that \forme{ʁɟa} is not a clausal adverb, and belongs to the postpositional phrase headed by \forme{kɯ}.

\begin{exe}
\ex \label{ex:RJa.kW}
 \gll [tɤ-lu cʰo tɯkrimgo ʁɟa kɯ] cʰɯ-z-ɣɤ-wxti-nɯ. \\
 \textsc{indef}.\textsc{poss}-milk \textsc{comit} doughnut completely \textsc{erg} \textsc{ipfv}-\textsc{caus}-\textsc{caus}-be.big-\textsc{pl} \\
\glt `They (used to) raise up (the babies) by feeding them milk and doughnuts only.' (140426 tApAtso kAnWBdaR, 102)
\end{exe}

The same applies to (\ref{ex:Wru.RJa.nW}), where the presence of the demonstrative \forme{nɯ} after \forme{ʁɟa} shows that it belongs to the same noun phrase.

\begin{exe}
\ex \label{ex:Wru.RJa.nW}
 \gll ɯ-rdoʁ nɯ-me tɕe, [ɯ-ru ʁɟa nɯ], pɯ-kɤ-tɤβ nɯnɯ, taʁndzɤr ɯ-ŋgɯ tú-wɣ-rku tɕe, \\
 \textsc{3sg}.\textsc{poss}-grain \textsc{pfv}-not.exist \textsc{lnk} \textsc{3sg}.\textsc{poss}-stalk completely \textsc{dem} \textsc{pfv}-\textsc{nmlz}:P-thresh \textsc{dem} feeding.emmer \textsc{3sg}.\textsc{poss}-inside \textsc{ipfv}-\textsc{inv}-put.in \textsc{lnk} \\
 \glt `When all the grains have been removed, the bare stalks, the one that have been threshed, one puts them in a feeding emmer.' (140513 tWrtsi, 5)
\end{exe}

A reduplicated emphatic form \forme{ʁɟɯ\redp{}ʁɟa} is also found as in (\ref{ex:RJWRJa.kW})

\begin{exe}
\ex \label{ex:RJWRJa.kW}
 \gll χtɕɤnzɤn ʁɟɯ\redp{}ʁɟa kɯ ʑo pɯ́-wɣ-nɤjo ɕti ɲɯ-ŋu.  \\
beast \textsc{emph}\redp{}completely \textsc{erg} \textsc{emph} \textsc{pst}.\textsc{ipfv}-\textsc{inv}-wait be.\textsc{affirm}:\textsc{fact} \textsc{sens}-be \\
\glt `It was all wild beasts waiting for him (there).' (Norbzang2005, 308)
 \end{exe}
 
A third option to express restrictive focus is the IPN \forme{ɯ-jlu}, which is used in the meaning `uncooked' as a property IPN (§ \ref{sec:property.nouns}), but has become grammaticalized as a restrictive marker `exclusively, without anything else' (presumably from an intermediate meaning `plain'), as in (\ref{ex:Wjlu.Zo}).

\begin{exe}
\ex \label{ex:Wjlu.Zo}
 \gll srɤz nɯ kɯ tɕʰoz ɯ-jlu ʑo pjɯ-nɯjɤntɤn pɯ-ɕti ma jɯm nɯ mɯ-pjɤ-ɕar ɲɯ-ŋu, \\
prince \textsc{dem} \textsc{erg}  religion \textsc{3sg}.\textsc{poss}-exclusively \textsc{emph} \textsc{ipfv}-be.assiduous.in  \textsc{pst}.\textsc{ipfv}-be.\textsc{affirm} \textsc{lnk} wife \textsc{dem} \textsc{neg}-\textsc{ifr}.\textsc{ipfv}-look.for \textsc{sens}-be \\
 \glt `The prince was focused exclusively in the study of religion, and was not looking for a wife.' (sras2003, 3)
 \end{exe}

For the expression of restrictive focus with temporal noun phrases or clauses, the postposition \japhug{kóʁmɯz}{only after} can also be used, especially with the demonstrative in the expression \japhug{nɯ kóʁmɯz nɤ}{only then} (§ \ref{sec:temporal.postpositions}, § XXX).



\subsection{Identity modifiers} \label{sec:identity.modifier}
There is no specific identity modifier `the same' in Japhug. The only way to express this meaning is to use the S-participle of the verb \japhug{naχtɕɯɣ}{be the same} (a denumeral verb of Tibetan origin, § \ref{sec:tibetan.numerals}, see also § \ref{sec:comitative} on the syntax of this stative verb and § XXX on its derivation) in a relative clause, as in (\ref{ex:tArmi.kWnaXtCWG}) (a possessor relative, § XXX). This participle is also used adverbially (see § XXX).

\begin{exe}
\ex \label{ex:tArmi.kWnaXtCWG}
\gll tɤ-rmi kɯ-naχtɕɯɣ pjɤ-dɤn wo kɤmɲɯ, nɤki kɯrɯ ra tɕe. \\
\textsc{indef}.\textsc{poss}-name \textsc{nmlz}:S/A-be.the.same \textsc{ifr}.\textsc{ipfv}-be.many \textsc{sfp} pl.n. \textsc{filler} Tibetan \textsc{pl} \textsc{lnk} \\
\glt `There were many people who had identical names, in Kamnyu, among the Tibetans.' (140522 tshupa, 161)
\end{exe}


There are two prenominal modifiers expressing non-identity in Japhug: \japhug{kɯmaʁ}{other} and the numeral \japhug{ci}{one}, which in prenominal position means `the other one' (in postnominal position, it is used as an indefinite article, see § \ref{sec:indef.article}). Both of these words can also be used as pronouns, though \forme{ci} requires to be combined with the demonstrative \forme{nɯ} in this usage (see § \ref{sec:other.pro}).

The modifier \forme{kɯmaʁ} is prenominal in its meaning `other', as in (\ref{ex:kWmaR.tWrme}). 

\begin{exe}
\ex \label{ex:kWmaR.tWrme}
\gll tɯ-zda nɯ ma kɯmaʁ tɯrme a-pɯ-me tɕe, kʰa ra aʁɤndɯndɤt ɲɯ-ɤ<nɯ>ɣro ɲɯ-ŋu ɲɯ-ti. \\
\textsc{genr}.\textsc{poss}-companion \textsc{dem} apart.from other person \textsc{irr}-\textsc{ipfv}-not.exist \textsc{lnk} house \textsc{pl} everywhere \textsc{ipfv}-<\textsc{auto}>play \textsc{sens}-be \textsc{sens}-say \\
\glt `(Our neighbour) says that if there are no other persons apart from family members, (the monkey) would play everywhere in the house.' (19-GzW2, 10)
\end{exe}

There are apparent examples of \japhug{kɯmaʁ}{other} in postnominal position, as in (\ref{ex:kWmaR.taXtW}) and (\ref{ex:kWmaR.tanWsWBzu}), but in such sentences \forme{kɯmaʁ} is a preverbal adverb, not a noun modifier, with a slightly different meaning `anew'. In (\ref{ex:kWmaR.taXtW}), the usage of \forme{kɯmaʁ} is very similar to its Chinese equivalent \ch{另外}{lìngwài}{other} in the corresponding Chinese sentence \zh{阿兰另外给我买了一部手机}, where the preverbal position of \ch{另外}{lìngwài}{other} clearly shows that it is not a noun modifier. 

\begin{exe}
\ex \label{ex:kWmaR.taXtW}
\gll <alan> kɯ a-<dianhua> kɯmaʁ ta-χtɯ \\
p.n. \textsc{erg} \textsc{1sg}.\textsc{poss}-phone other \textsc{pfv}:3\fl{}3'-buy \\
\glt `Alan bought me a new phone.' (conversation, 17-03-27)
\end{exe}

\begin{exe}
\ex \label{ex:kWmaR.tanWsWBzu}
\gll a-ʁi kɯ kʰa kɯmaʁ ta-nɯ-sɯ-βzu qʰe, \\
\textsc{1sg}.\textsc{poss}-younger.sibling \textsc{erg} house other \textsc{pfv}:3\fl{}3'-\textsc{auto}-\textsc{caus}-make \textsc{lnk} \\
\glt `My brother made himself a new house.' (14-tApitaRi, 304)
\end{exe}

The identity determiner \japhug{kɯmaʁ}{other} is grammaticalized from the S-participle of the verb \japhug{maʁ}{not be}, \forme{kɯ-maʁ} `who/which is not X' (see also § \ref{sec:lexicalized.subject.participle}), which is still widely used, as in (\ref{ex:tChWrtsAm.kWmaR}) and (\ref{ex:sthWci.kWmaR}).


%\begin{exe}
%\ex \label{ex:Wstu.kWmaR}
%\gll ɯ-stu kɯ-maʁ me, kɯki mɤ-kɯ-pe me \\
%\textsc{3sg}.\textsc{poss}-truth \textsc{nmlz}:S/A-not.be not.exist:\textsc{fact} dem.\textsc{prox} \textsc{neg}-\textsc{nmlz}:S/A-be.good not.exist:\textsc{fact} \\
%\glt ` (28-smAnmi, 16)
%\end{exe}

\begin{exe}
\ex \label{ex:tChWrtsAm.kWmaR}
\gll mɤʑɯ [tɕʰirtsɤm kɯ-maʁ] nɯnɯ tɕe, tú-wɣ-χtɕi ma nɯ ma kɤ-sqa (mɤ-ra) \\
yet type.of.tsampa \textsc{nmlz}:S/A-not.be \textsc{dem} \textsc{lnk} \textsc{ipfv}-\textsc{inv}-wash \textsc{lnk} \textsc{dem} apart.from \textsc{inf}-boil \textsc{neg}-have.to:\textsc{fact} \\
\glt `The tsampa that is not `chu.rtsam', one needs to wash it, but not to boil it.' (2002tWsqar, 112)
\end{exe}

\begin{exe}
\ex \label{ex:sthWci.kWmaR}
\gll  [ɯ-rkɯ wuma ʑo stʰɯci kɯ-maʁ] nɯtɕu tɤ-ri ci kú-wɣ-lɤt \\
\textsc{3sg}.\textsc{poss}-side really \textsc{emph} so.much \textsc{nmlz}:S/A-not.be \textsc{dem}:\textsc{loc} \textsc{indef}.\textsc{poss}-thread once \textsc{ipfv}-\textsc{inv}-throw \\
\glt `One sews a thread at a place which is not too much on the border (of the patch)'. (12-kAtsxWb, 16)
\end{exe}


The modifier \forme{ci} differs from \forme{kɯmaʁ} in that it is necessarily definite, meaning `the other one', as in (\ref{ex:ci.rWdaR}), where it refers to an animal that it chased by lions, which was previously mentioned in the text.

\begin{exe}
\ex \label{ex:ci.rWdaR}
\gll ʑɯrɯʑɤri qʰe ci rɯdaʁ nɯ dɯxpa ma nɯ-kɤ-ndza ɯ-spa ɲɯ-ɕti qʰe, qʰe pjɯ-ndʐaβ qʰe mɯ-ɲɯ-cʰa qʰe, \\
progressively \textsc{lnk} other.one animal \textsc{dem} poor.of \textsc{lnk} \textsc{3pl}.\textsc{poss}-\textsc{nmlz}:P-eat \textsc{3sg}.\textsc{poss}-material \textsc{sens}-be.affirm \textsc{lnk} \textsc{lnk} \textsc{ipfv}-\textsc{anticaus}:make.fall \textsc{lnk} \textsc{neg}-\textsc{ipfv}-can \textsc{lnk} \\
\glt `The other animal, poor of him, it is their prey, progressively it falls down and cannot stand it anymore.' (20-sWNgi, 43)
\end{exe}

Interestingly, the determiner \forme{ci} does not have scope over other noun modifiers. For instance, in (\ref{ex:ci.tCheme.kWNAn}), the noun \japhug{tɕʰeme}{woman} occurs with an attributive adjective in participial form \forme{kɯ-ŋɤn} `who is evil' (a relative clause, see § \ref{sec:attributes}), but the meaning is not `the other evil woman' as could have been expected (since the woman who is the subject of the sentence is, by contrast, a kind person), and rather must be `the other woman, the evil one'. There is no pause in the recording that could lead us to suppose that \forme{kɯ-ŋɤn} here is an apposition -- it is rather a postnominal relative.

\begin{exe}
\ex \label{ex:ci.tCheme.kWNAn}
\gll nɤki, tɕʰeme nɯ ɯ-ɕki ɯ-kɯ-sɤja jo-ɕe, ci tɕʰeme kɯ-ŋɤn nɯ ɯ-ɕki. \\
\textsc{filler} women \textsc{dem} \textsc{3sg}-\textsc{dat} \textsc{3sg}.\textsc{poss}-\textsc{nmlz}:S/A-give.back \textsc{ifr}-go other.one woman \textsc{nmlz}:S/A-be.evil \textsc{dem} \textsc{3sg}-\textsc{dat} \\
\glt `She went to give it back to the woman, the other one, the evil woman.' (140515 jiesu de laoren-zh, 90)
\end{exe}

%a-ʁi kɯnɤ tɯrme kha kɤ-sɯxɕe mɯ́j-khɯ qhe,

%ci qhɤjmbaʁ nɯ kɯ-jaʁ kɯ-fse nɯnɯ 
%mtshalu ɯ-cu tɕe nɤki,
%tɯ-mgo zmɤrɤβ kú-wɣ-nɯ-lɤt sna.
%16-RlWmsWsi
%li ci /ɯt/ ɯ-tɯphu nɯ tɤpu qhɤjmbaʁ tu-ti-nɯ ŋu tɕe,
\subsection{Attributes} \label{sec:attributes}
Japhug has several sub-parts of speech which could be described as `adjectives': stative verbs, adverbs and nouns which express properties (rather than actions or entities). Property words used as noun modifiers are collectively designated by the term \textit{attributes}. This heterogenous class excludes the property nouns described in § \ref{sec:property.nouns}, which are the syntactic heads of the noun phrase.

Three types of attributes are distinguished: attributive postnominal (noun or adverb) modifiers, prenominal modifiers and participial \forme{kɯ-} relatives (mainly postnominal or head-internal). The constructions mentioned in this section are all described in more details elsewhere in the grammar, and for this reason the discussion is kept brief.

\subsubsection{Attributive postnominal modifiers} \label{ex:attributive.postnominal}
In addition to the postnominal markers studied above (numeral and number § \ref{sec:number.determiners}, demonstratives § \ref{sec:demonstrative.determiners}, quantifiers § \ref{sec:quantifiers.determiners}, definiteness markers § \ref{sec:indef.article}, topic and focus markers), there are a certain number of nouns that can serve a post-nominal modifiers.

An entire class of such nouns consists of the privative nouns in \forme{-lu} `...less', described in § \ref{sec:privative}.

The word \japhug{wuma}{real, really} from Tibetan \tibet{ངོ་མ་}{ŋo.ma}{real, true} is generally used adverbially as an intensifier, in particular with stative verbs (§ XXX), but also occurs as a postnominal modifier meaning ` real', its original meaning, as in (\ref{ex:lhAndzxi.wuma} and (\ref{ex:tAtsoR.wuma}).

\begin{exe}
\ex \label{ex:lhAndzxi.wuma}
\gll ɬɤndʐi wuma nɯ nɤʑo ɲɯ-tɯ-ŋu ma aʑo ɬɤndʐi ɲɯ-maʁ-a \\
demon real \textsc{dem} \textsc{2sg} \textsc{sens}-2-be \textsc{lnk} \textsc{1sg} demon \textsc{sens}-not.be-\textsc{1sg} \\
\glt `You are the real demon, not me.' (2002lhandzi, 12)
\end{exe}

\begin{exe}
\ex \label{ex:tAtsoR.wuma}
\gll ɯ-qa nɯ qarŋe, tɤtsoʁ wuma nɯ. \\
\textsc{3sg}.\textsc{poss}-root \textsc{dem} be.yellow:\textsc{fact} silverweed real \textsc{dem} \\
\glt `Its root is yellow, the real silverweed.' (19-khWlu, 74)
\end{exe}

In addition, some adverbs, which are more often used with scope over the whole clause, can occur as postnominal modifiers, in particular the comitative adverbs (§ \ref{sec:comitative.adverb}). In (\ref{ex:CWNArWra.kW}), we clearly see that the adverb \japhug{ɕɯŋarɯra}{each better than the other} is a postnominal modifier of \japhug{rɟɤlpu}{king} and not a sentential adverb, as it is followed by the ergative \forme{kɯ}.

\begin{exe}
\ex \label{ex:CWNArWra.kW}
\gll   rɟɤlpu ɕɯŋarɯra kɯ ta-tʰu-nɯ ɕti ri, mɯ-tɤ-nɤla-j ɕti tɕe, \\
king each.better.than.the.other \textsc{erg} \textsc{pfv}:3\fl{}3'-ask-\textsc{pl} be.\textsc{affirm}:\textsc{fact} \textsc{lnk} \textsc{neg}-\textsc{pfv}-agree-\textsc{1sg} \\
\glt `(Many) kings, all better than the other, asked for (my daughters in marriage), but we did not agree.' (2003qachGa, 71)
\end{exe}

\subsubsection{Attributive prenominal modifiers}   \label{ex:attributive.prenominal}
There are two types of prenominal modifiers, the possessors (which are followed by a possessum taking a possessive prefix coreferent with the possessor, § \ref{ex:prefix.expression.of.possession}), but also   are essentially placenames and other unpossessible nouns (§ \ref{sec:place.names}) and nouns expressing the material from which an object is a made, as \japhug{χsɤr}{gold}, \japhug{rŋɯl}{silver} and \japhug{si}{wood} 
in (\ref{ex:rNWl.rJAskAt}).

\begin{exe}
\ex \label{ex:rNWl.rJAskAt}
\gll a-tɤɕime, nɤʑo rŋɯl rɟɤskɤt ɯ-taʁ tɯ-ɕe ɕi, χsɤr rɟɤskɤt ɯ-taʁ tɯ-ɕe ɕi, ɕom rɟɤskɤt ɯ-taʁ tɯ-nɯ-ɕe? \\
\textsc{1sg}.\textsc{poss}-lady \textsc{2sg} silver stair \textsc{3sg}.\textsc{poss}-on 2-go:\textsc{fact}  \textsc{qu} gold stair \textsc{3sg}.\textsc{poss}-on 2-go:\textsc{fact} \textsc{qu} wood stair \textsc{3sg}.\textsc{poss}-on 2-\textsc{auto}-go:\textsc{fact} \\
\glt `My lady, will you go on the silver stairs, the golden stairs or the wooden stairs?' (2014-kWlAG, 369)
\end{exe}

When a prenominal modifier is present, possessive prefixes on the head noun can be neutralized to indefinite possessor, or the modifier and the head noun can be fused as a compound noun, with the possessive prefixes occurring in leftmost position (see the discussion in § \ref{sec:possessive.prefixes.prenominal}).


\subsubsection{Participial relatives} \label{ex:attributive.participles.stative.verbs}
Most words expressing properties in Japhug are a subclass of stative verbs (§ XXX), and cannot serve as attributes without being embedded into a relative clause. Since intransitive subjects can only be relativized using \forme{kɯ-} participial relative clauses (§ XXX), attributive adjectival stative verbs are always in this form, as \forme{kɯ-pe} `good one, which is good' in (\ref{ex:wuma.Zo.kWpe}); the relative  \forme{wuma ʑo tɕʰeme kɯ-pe} in this example is head-internal (§ XXX), as shown by the position of the intensifier \forme{wuma ʑo}, and literally means `(a) woman who is/was really nice.'

\begin{exe}
   \ex  \label{ex:wuma.Zo.kWpe}
\gll  ɯ-rʑaβ βdaʁmu nɯ [wuma ʑo tɕʰeme kɯ-pe] ci pjɤ-ŋu. \\
\textsc{3sg}.\textsc{poss}-wife lady \textsc{dem} really \textsc{emph} woman \textsc{nmlz}:S/A-be.good \textsc{indef} \textsc{ifr}.\textsc{ipfv}-be \\
\glt `His wife, the queen, was a very nice woman.' (28-smAnmi, 4)
\end{exe}  

In the case of shorter relative clauses it is not always clear whether we have a head-internal, or a postnominal one (§ XXX). Prenominal relatives with an adjectival stative verb such as \forme{stu kɯ-mna} `the best one, the leader' in (\ref{ex:stu.kWmna.tCheme}) are very rare.


\begin{exe}
\ex \label{ex:stu.kWmna.tCheme}
\gll  rɟɤlpu nɤrɯβzaŋ nɯ kɯ, nɤki, [stu kɯ-mna tɕʰeme] nɯ ɲɤ-nɯ-ɕar ɲɯ-ŋu \\
king p.n. \textsc{dem} \textsc{erg} filler most \textsc{nmlz}:S/A-be.better woman \textsc{dem} \textsc{ifr}-\textsc{auto}-search \textsc{sens}-be \\
\glt `King Norbzang chose for himself the woman leader.' (Norbzang2012, 41)
\end{exe} 


Attributive relative clauses may contain comparative constructions as in (\ref{ex:WZo.sAz.rWdaR.kWxtCi}), or semi-objects; a more detailed account is presented in § XXX.

\begin{exe}
\ex \label{ex:WZo.sAz.rWdaR.kWxtCi}
\gll ɯʑo sɤz rɯdaʁ kɯ-xtɕi nɯra tu-ndze \\
\textsc{3sg} \textsc{comp} animal \textsc{nmlz}:S/A-be.small \textsc{dem}:\textsc{pl} \textsc{ipfv}-eat[III] \\
\glt `It eats the animals that are smaller than itself.' (20-sWNgi, 23)
\end{exe}

As shown in § XXX, in head-internal relative clauses the same determiner can appear on the head noun and repeated after the whole relative. The same is found with adjectival participial relatives, as in (\ref{ex:qajW.ci.kArNi.ci}) with the indefinite \japhug{ci}{one}, though this usage is rare, in most cases only one of the two determiners is used (either the one inside the relative or the external one, § XXX).

\begin{exe}
\ex \label{ex:qajW.ci.kArNi.ci}
\gll tɕe [qajɯ ci kɯ-ɤrŋi] ci ŋu. \\
\textsc{lnk} bug \textsc{indef} \textsc{nmlz}:S/A-be.green \textsc{indef} be:\textsc{fact} \\
\glt `It is a green/black bug.' (26-zrWGndza)
\end{exe}

Subject prenominal relatives are almost not attested with stative verbs, but are found with some intransitive dynamic verbs, as in the lexicalized expression in (\ref{ex:kWrla.kha}), where prenominal position is required.

\begin{exe}
\ex \label{ex:kWrla.kha}
\gll kɯ-rlaʁ kʰa \\
\textsc{nmlz}:S/A-disappear house \\
\glt `An abandonned house.' 
\end{exe}

 \section{Noun coordination}
Japhug lacks a dedicated noun coordinator. Nouns can be either coordinated by using the position \forme{cʰo} (§ \ref{sec:coordinator.cho}), or by  juxtaposition without any linking element (§ \ref{sec:bare.coordination}). 

\subsection{Coordination or embedded phrase} \label{sec:coordinator.cho}
The closest thing to a noun coordinator in Japhug is the comitative marker \forme{cʰo}; it can be used both to connect finite clauses (§ XXX) or nouns as in (\ref{ex:qandZGi.cho.qaliaR}), with the plural indexation on the verb reflecting the whole transitive subject constituent \forme{qandʑɣi cʰo qaliaʁ ra kɯ}.

\begin{exe}
\ex \label{ex:qandZGi.cho.qaliaR}
 \gll  qandʑɣi cʰo qaliaʁ ra kɯ cʰɯ-nɯ-tsɯm-nɯ tu-ndza-nɯ ŋgrɤl. \\
falcon \textsc{comit} eagle \textsc{pl} \textsc{erg} \textsc{ipfv}:\textsc{downstream}-\textsc{vert}-take.away-\textsc{pl} \textsc{ipfv}-eat-\textsc{pl} be.usually.the.case:\textsc{fact} \\
\glt `Falcons and eagles take them (the moles) and eat them.' (28-qapar, 201)
\end{exe}

There is however evidence that \forme{cʰo} is a postposition: first, some verbs select an oblique argument with the comitative (§ \ref{sec:comitative}), and second, while \forme{cʰo}  cannot be used without a preceding noun phrase (or clause), the following noun is optional. For these reasons, rather than assuming a `flat' structure as in Figure \ref{fig:qanZGi}, I consider \forme{qandʑɣi cʰo} to be a postpositional phrase  used as an adnominal modifier of the noun \japhug{qaliaʁ}{eagle}, as in Figure \ref{fig:qanZGi2}.

\begin{figure} 
\caption{\forme{cʰo} as a coordinator} \label{fig:qanZGi} \centering
\Tree [.PostP [.NP  [.N' [.N \forme{qandʑɣi} ] [.Coord \forme{cʰo} ]  [.N \forme{qaliaʁ} ] ] [.D \forme{ra} ] ] [.Post \forme{kɯ} ] ]
\end{figure}

\begin{figure} 
\caption{\forme{cʰo} as a postposition} \label{fig:qanZGi2} \centering
\Tree [.PostP [.NP  [.N' [.PostP [.N \forme{qandʑɣi} ] [.Post \forme{cʰo} ] ]  [.NP \forme{qaliaʁ} ] ] [.D \forme{ra} ] ] [.Post \forme{kɯ} ] ]
\end{figure}

Even without number marking, the constituent comprising the noun and the \forme{cʰo} postpositional phrase is indexed with non-singular indexation, dual in the case of (\ref{ex:dpalcan.cho.alan}).

 \begin{exe}
\ex \label{ex:dpalcan.cho.alan}
 \gll  χpɤltɕin cʰo alan kɯ ko-ndo-ndʑi tɕe, \\
 n.p. \textsc{comit} n.p. \textsc{erg} \textsc{ifr}-take-\textsc{du} \textsc{lnk} \\
 \glt `Dpalcan and Alan caught (one).' (24-qro, 101)
 \end{exe}
 
\subsection{Bare coordination} \label{sec:bare.coordination}

\subsubsection{Enumeration} \label{sec:noun.enumeration}
Enumerations are the listing of a series of nouns, often with a specific (rising) intonation and a pause between item, and without any coordinating element including the postposition \forme{cʰo}. In Japhug, quite lengthy enumerations are attested in the corpus, as shown by (\ref{ex:mbro.etc}) with seven nouns.

\begin{exe}
\ex \label{ex:mbro.etc}
 \gll mbro, jla, nɯŋa, mbala, tsʰɤt, qaʑo, paʁ, nɯra nɯtɕu ʁɟa z-ɲɯ́-wɣ-lɤɣ pɯ-ŋu. \\
 horse hybrid.yak cow bull goat sheep pig \textsc{dem}:\textsc{pl} \textsc{dem}:\textsc{loc} completely \textsc{transloc}-\textsc{ipfv}-\textsc{inv}-graze \textsc{pst}.\textsc{ipfv}-be \\
\glt `People used to graze there horses, hybrid yaks, cows, bulls, goats, sheep ans pigs.' (140522 Kamnyu zgo, 156)
\end{exe}

Enumerations with only two or three nouns and without specific intonation are also found, as in (\ref{ex:tshAt.qaZo}). When the order of the nouns is rigid (which is not the case in \ref{ex:tshAt.qaZo}, since \forme{qaʑo tsʰɤt} `sheep and goats' is also attested), the construction belongs to a distinct category: that of noun dyads (§ \ref{sec:dyads}).

\begin{exe}
\ex \label{ex:tshAt.qaZo}
 \gll  qʰe tsʰɤt qaʑo ra ɣɯ nɯ-ndza nɯra ɲɯ-sna.  \\
 \textsc{lnk} goat sheep \textsc{pl} \textsc{gen} \textsc{3pl}.\textsc{poss}-food \textsc{dem}:\textsc{pl} \textsc{sens}-be.good \\
\glt `It is good as fodder for goats and sheep.' (16-RlWmsWsi, 65)
\end{exe}

\subsubsection{Noun dyads} \label{sec:dyads}
Noun dyads are a pair of nouns occurring in a fixed order, without intervening linker or postposition, and sharing their number and case markers. A good example is provided by the expression `parents' comprising the kinship terms \japhug{tɤ-mu}{mother} and \japhug{tɤ-wa}{father}, as in (\ref{ex:amu.awa.ni.GW}). Note that while number and case markers are shared by both nouns, each of them takes its own possessive prefix, and both prefixes are coreferent. 

\begin{exe}
\ex \label{ex:amu.awa.ni.GW}
 \gll nɯ a-mu a-wa ni ɣɯ ŋu \\
 \textsc{dem}  \textsc{1sg}.\textsc{poss}-mother \textsc{1sg}.\textsc{poss}-father \textsc{du} \textsc{gen} be:\textsc{fact} \\
 \glt `This is for my parents.' (meimei de gushi)
\end{exe}

The dyad for `parents' has a honorific variant, originally used for noblemen in the traditional society. It comprises the terms \japhug{tɤ-pa}{father} and \japhug{tɤ-ma}{mother}, which are borrowed from Tibetan \tibet{ཨ་ཕ་}{ʔa.pʰa}{father}  and \tibet{ཨ་མ་}{ʔa.ma}{mother} respectively. Interesting, the honorific expression follows the `father-mother' order (as in example \ref{ex:apa.ama}), while the native one puts `mother' in the first place.

\begin{exe}
\ex \label{ex:apa.ama}
 \gll nɯ kɯ-fse a-pa a-ma ni kɯ ɲɯ-ti-ndʑi tɕe \\
 \textsc{dem} \textsc{nmlz}:S/A-be.like \textsc{1sg}.\textsc{poss}-father  \textsc{1sg}.\textsc{poss}-mother \textsc{du} \textsc{erg} \textsc{sens}-say-\textsc{du} \textsc{lnk} \\
 \glt `My parents say this.' (2003nyima2, 94)
\end{exe}

Other common dyads include \forme{rgɤtpu rgɤnmɯ} `old man(men) and woman(women)', \forme{tɤ-tɕɯ tɕʰeme} `boy(s) and girl(s)' with APNs. They are most commonly used as collectives with indefinite referents as in (\ref{ex:tAtCW.tCheme.tWsAmdzW}), but are also attested with definite ones, as in (\ref{ex:rgAtpu.rgAnmW}).

\begin{exe}
\ex \label{ex:tAtCW.tCheme.tWsAmdzW}
 \gll  tɤ-tɕɯ tɕʰeme tɯ-sɤ-ɤmdzɯ ʑaka tu \\
 \textsc{indef}.\textsc{poss}-son girl \textsc{genr}.\textsc{poss}-\textsc{nmlz}:\textsc{obl}-sit each \textsc{exist}:fact \\
\glt `Gents and ladies each have (different) seating places.' (31-khAjmu, 10)
\end{exe}

\begin{exe}
\ex \label{ex:rgAtpu.rgAnmW}
 \gll rgɤtpu rgɤnmɯ ni kɯ kɯki tɤ-pɤtso χsɯm ki kɤsɯfse ʑo cʰɤ-ɣɤ-wxti-ndʑi. \\
 old.man old.woman \textsc{du} \textsc{erg} \textsc{dem}.\textsc{prox} \textsc{indef}.\textsc{poss}-child three \textsc{dem.prox} all \textsc{emph} \textsc{ifr}-\textsc{caus}-be.big-\textsc{du} \\
\glt `The old man and the old woman raised all these three children.' (140514 huishuohua de niao-zh, 60)
\end{exe}

Another type of noun dyad comprises two abstract nouns, which can be used as manner adjuncts with the ergative (see example \ref{ex:tAmqe.tAndWt.kW} § \ref{sec:manner.nominal.kW}) or in the degree construction as in (\ref{ex:tAre.tAJaR}) with the dyad \forme{tɤ-re tɤ-ɟaʁ} `chatting and laughing' (only \japhug{tɤ-re}{laugh} exists as an independent word). Some of these dyads are nominalized forms of bipartite verbs (§ XXX).

\begin{exe}
\ex \label{ex:tAre.tAJaR}
 \gll  nɯtɕu rcanɯ maka tɤ-re tɤ-ɟaʁ pjɤ-saχaʁ \\
 dem:loc unexpectedly completely \textsc{indef}.\textsc{poss}-laugh \textsc{indef}.\textsc{poss}-laugh \textsc{indef}.\textsc{poss}-chatting.and.laughing \textsc{ifr}.\textsc{ipfv}-be.extremely \\
 \glt `There, (the râkshasî) were chatting and laughing a lot.' (2011-05-nyima, 34)
\end{exe}

\section{Word order in the noun phrase}
The relative order of the elements in the noun phrase in Japhug is relatively rigid.  Examples of relatively complex nouns phrases such as (\ref{ex:N.Adj.Num}) and (\ref{ex:Dem.N.Num.Dem}) illustrate the orders N(oun)-Adj(ective)-Num(eral) and Dem(onstrative)-N(oun)-Num(eral)-Dem(onstrative) respectively, considering here attributive adjectival stative verbs in participial form to be the main `adjective' class in Japhug, see § XXX and § \ref{ex:attributive.participles.stative.verbs}).
 
\begin{exe}
\ex \label{ex:N.Adj.Num}
 \gll sɯŋgɯ zɯ, [tɯrme wuma ʑo kɯ-wxti ʁnɯz] tu-ndʑi tɕe \\
 forest \textsc{loc} people really \textsc{emph} \textsc{nmlz}:S/A-be.big two exist:\textsc{fact}-\textsc{du} \textsc{lnk} \\
 \glt `In the forest, there are two very big persons.' (140428 yonggan de xiaocaifeng-zh, 171)
\end{exe}

\begin{exe}
\ex \label{ex:Dem.N.Num.Dem}
 \gll [ɯkɯki ɕnat ʁnɯz kɯni] kɯ, \\
 \textsc{dem}.\textsc{prox} heddle two \textsc{dem}.\textsc{prox}:\textsc{du} \textsc{erg} \\
 \glt `(The weaving is done) with these two heddles.' (vid-20140429090403, 47)
\end{exe}

From such examples, it can be extrapolated that the most basic word order in the noun phrase in Japhug is Dem-N-Adj-Num-Dem. Although no noun phrase in the corpus presents all five elements, it is easy to elicitate such an example. This order is not unusually crosslinguistically: \citet{cinque05universal20} notes that the orders  Dem-N-Adj-Num and N-Adj-Num-Dem order are both widely attested.

Although the indefinite marker \japhug{ci}{one} and the topic \forme{nɯ} can appear between the noun and the adjective (being here a head-internal participial relative clause, § \ref{ex:attributive.participles.stative.verbs}), as shown by examples (\ref{ex:qajW.ci.kArNi.ci}) above and (\ref{ex:GJW.ci.kWmbWmbro}) below, numerals are not attested in this position.

\begin{exe}
\ex \label{ex:GJW.ci.kWmbWmbro}
 \gll ɣɟɯ ci kɯ-mbɯ\redp{}mbro ʑo pjɤ-tu ɲɯ-ŋu tɕe   \\
 watchtower \textsc{indef} \textsc{nmlz}:S/A-\textsc{emph}\redp{}be.high \textsc{emph} \textsc{ifr}.\textsc{ipfv}-exist \textsc{sens}-be \textsc{lnk} \\
\glt  `There was a very big tower.' (Norbzang2012, 54)
\end{exe}

There are other attributes than adjectival stative verbs in Japhug (§ \ref{ex:attributive.postnominal} and § \ref{ex:attributive.prenominal}), in particular prenominal attributes as \japhug{rŋɯl}{silver} in (\ref{ex:rNWl.qhoRqhoR}), but it is problematic to use such examples as evidence for a Adj-N-Num-Dem, given the fact that prenominal attributes are always essentially nouns used as modifiers.

 \begin{exe}
\ex \label{ex:rNWl.qhoRqhoR}
\gll  rŋɯl qʰoʁqʰoʁ χsɯm nɯ ɲɤ-nɯ-ɬoʁ. \\
silver ingot three \textsc{dem} \textsc{ifr}-\textsc{auto}-come.out \\
\glt `The three silver ingots had come out.' (28-qAjdoskAt, 178)
\end{exe}

The position of the prenominal demonstrative can alternatively be filled by the identity modifiers \japhug{kɯmaʁ}{other} or \japhug{ci}{the other one} (§ \ref{sec:identity.modifier}). These elements can be preceded by either a comitative \forme{cʰo} phrase (§ \ref{sec:coordinator.cho}) or by the aforementioned topic marker \forme{iɕqʰa} (§ \ref{sec:iCqha}, cf example \ref{ex:iCqha.ci.rJAlpu}), the leftmost element of a noun phrase.

 \begin{exe}
\ex \label{ex:iCqha.ci.rJAlpu}
\gll rɟɤlpu ɯ-tɕɯ nɯ kɯ, iɕqʰa ci rɟɤlpu nɯnɯ, <xila> rɟɤlpu nɯ ɯ-ɕki, \\
king \textsc{3sg}.\textsc{poss}-son \textsc{dem} \textsc{erg} the.aforementioned other.one king \textsc{dem} pl.n. king \textsc{dem} \textsc{3sg}.\textsc{poss}-\textsc{dat} \\
\glt `The king's son (told) the other king, the king of Greece' (140518 huifei de muma-zh, 177
\end{exe}


%{Nominal predicates} \label{sec:nominal.predicates}
 %Ideophones
%\include{chapters/4-01} %The verbal template
%\chapter{Person indexation} \label{chap:indexation}
In Japhug, person indexation is the defining feature of finite verbs, as opposed to non-finite verbs (§\ref{chap:non-finite}) and other parts of speech. Japhug finite verb forms index one or two arguments, depending on the transitivity of the verb, using a combination of prefixes, suffixes and stem alternation. No verb indexes more than two arguments. The indexation system is very close to a canonical direct-inverse system (§\ref{sec:direct-inverse}).

This chapter first presents intransitive and transitive conjugations, investigates the issue of agreement mismatch, and then discusses the origin of person indexation affixes. In addition, it documents the analogical extension of person indexation suffixes to non-finite verb forms in some specific contexts.

\section{Intransitive verbs} \label{sec:intr.indexation}
Intransitive verbs comprise dynamic, stative and semi-transitive verbs. All of the verbs have in common the property of indexing one argument, the intransitive subject, which when overt is in absolutive form (§\ref{sec:absolutive.S}).

\subsection{The intransitive paradigm} \label{sec:intransitive.paradigm}
Table \ref{tab:intransitive.indexation} illustrates the paradigm of intransitive verbs in Kamnyu Japhug, using the verb \japhug{ɕe}{go} in the Factual non-past\footnote{This TAM category is chosen to illustrate the paradigms due to the fact that it does not bear any orientation prefix, but at the same time presents stem alternation in the transitive paradigm.} as an example. Other Japhug dialects have slightly different indexation suffixes, a question discussed in §\ref{sec:indexation.suffixes.history} with comparative evidence from other Gyalrong languages.

There is no stem alternation related to person indexation in the intransitive paradigm in any Japhug dialect. The invariable stem is represented with the symbol \ro{} in Table \ref{tab:intransitive.indexation}.\footnote{This notation follows the Kirantological tradition (for instance \citealt{driem93dumi}). }

\begin{table}[H] \centering
\caption{The intransitive conjugation in Japhug}\label{tab:intransitive.indexation}
\begin{tabular}{lllllllll} \lsptoprule
Person & Form & \japhug{ɕe}{go} (Factual non-past) \\
\midrule
\textsc{1sg} & \ro{}-\forme{a} & \forme{ɕe-a} \\
\textsc{1du} & \ro{}-\forme{tɕi} & \forme{ɕe-tɕi} \\
\textsc{1pl} & \ro{}-\forme{ji} & \forme{ɕe-j} \\
\midrule
\textsc{2sg} & \forme{tɯ}-\ro{} & \forme{tɯ-ɕe} \\
\textsc{2du} & \forme{tɯ}-\ro{}-\forme{ndʑi} & \forme{tɯ-ɕe-ndʑi} \\
\textsc{2pl} & \forme{tɯ}-\ro{}-\forme{nɯ} & \forme{tɯ-ɕe-nɯ} \\
\midrule
\textsc{3sg} & \ro{} & \forme{ɕe} \\
\textsc{3du} & \ro{}-\forme{ndʑi} & \forme{ɕe-ndʑi} \\
\textsc{3pl} & \ro{}-\forme{nɯ} & \forme{ɕe-nɯ} \\
\midrule
generic & \forme{kɯ}-\ro{} & \forme{kɯ-ɕe} \\
\lspbottomrule
\end{tabular}
\end{table}

In the intransitive paradigm, five suffixes and two prefixes are found. The stress is always on the last syllable of the verb stem, and all person indexation suffixes, including \forme{-a}, are unstressed and sometimes are even devoiced (§XXX). Unlike other languages of the Trans-Himalayan, such as Khaling (where the dual inclusive and the third dual are homophonous, see \citealt[1113]{jacques12khaling}), in Japhug all slots in the intransitive paradigm are distinct, without ambiguity.  

\subsubsection{First person}

First person subjects are indexed by a set of three suffixes marking both person and number: \forme{-a}, \forme{-tɕi} and \forme{-ji} respectively for first singular, dual and plural. As in the pronominal paradigms (§\ref{sec:pers.pronouns}), there is no inclusive/exclusive distinction in Japhug.

The \textsc{1sg} \forme{-a} suffix is the only suffix in Japhug with a vowel other than \forme{ɯ} (or \forme{i} after palatal and alveolo-palatal consonants, §\ref{sec:W.i.contrast}), and is the only indexation suffix that can be followed by another indexation suffix in the transitive paradigm (§\ref{sec:double.number.indexation}). The \forme{-a} \textsc{1sg} person index is among the suffixes revealing the underlying form of the codas: \forme{-β}, \forme{-ɣ}, \forme{-ʁ}, \forme{-z}, which become unvoiced in some contexts (§XXX) are realized as voiced (see for instance in Table \ref{tab:verb.stem.1sg} below; \forme{-β} is realized \phonet{-w-} in this context, see §XXX), but the coda \forme{-t} remains unvoiced (for instance \forme{scit-a} be.happy-\textsc{1sg} `I am happy'). The codas are resyllabified; for instance \forme{scit-a} is syllabified as \forme{sci/ta}).

 Some verb stems (independently of transitivity) undergo predictable phonological alterations when followed by \forme{-a}. With verb stems whose last syllable is an open syllable,  the \forme{-a} suffix merges with the vowel of the last syllable. With closed syllable verb stem in \forme{-ɤC} (C representing a coda), the \textsc{1sg} suffix causes vowel assimilation. These phonological rules are presented in Table \ref{tab:verb.stem.1sg}.

When the verb stem ends in \forme{-a}, the \textsc{1sg} suffix merges with the stem as \phonet{a} in Kamnyu Japhug, resulting in homophony between the \textsc{1sg} and the \textsc{3sg} forms. The surface form \phonet{rga} corresponds to both \textsc{1sg} \japhug{rga-a}{I like it} and \textsc{3sg} \japhug{rga}{he likes it}. The fused and invisible suffix is systematically indicated in the orthography used in this grammar. In some dialects of Japhug, a long vowel occurs in the \textsc{1sg}, which thus remains distinct from the \textsc{3sg}.

When the verb stem ends in the mid-high vowels \forme{-e} and \forme{-o}, these vowels become the corresponding high vowels \forme{-i} and \forme{-u} when followed by \textsc{1sg} in Kamnyu Japhug. This alternation does not occur in all Japhug dialects.

With verb stem ending in \forme{-ɤt}, \forme{-ɤn}, \forme{-ɤβ}, \forme{-ɤm}, \forme{-ɤr}, \forme{-ɤl} and \forme{-ɤz},  the \textsc{1sg} suffix causes non-optional vowel assimilation \forme{-ɤC-a} $\Rightarrow$ \ipa{-aCa}; Table \ref{tab:verb.stem.1sg} provides examples for all rhymes of this type. In the orthography employed in this grammar, these forms are transcribed as \forme{aC-a} rather than the underlying \forme{ɤC-a} (\forme{jɣat-a} rather than \forme{jɣɤt-a}), to indicate the fact that \forme{ɤ} $\Rightarrow$ \forme{a} assimilation is obligatory in this context.

\begin{table}
\caption{Predictable phonological alternations on the verb stem caused by the \forme{-a} \textsc{1sg} suffix in Kamnyu Japhug} \label{tab:verb.stem.1sg}
\begin{tabular}{llllll}
\lsptoprule
Rhyme of the  & Result of  &Examples \\
last syllable & fusion with  \\
of the verb stem & the \textsc{1sg} suffix \\
\midrule
\forme{-e} & \phonet{-ia} & \forme{ɕe-a} $\Rightarrow$ \phonet{ɕia} `I will go there' \\
\forme{-o} & \phonet{-ua} & \forme{tso-a} $\Rightarrow$ \phonet{tsua} `I understand it' \\
\forme{-a} & \phonet{-a} & \forme{rga-a} $\Rightarrow$ \phonet{rga} `I like it' \\
\midrule
\forme{-ɤβ} & \phonet{-awa} & \forme{tʰɯ-rdɤβ-a} $\Rightarrow$ \phonet{tʰɯrdawa} `I lost money' \\
\forme{-ɤm} & \phonet{-ama} & \forme{mtsʰɤm-a} $\Rightarrow$ \phonet{mtsʰama} `I hear it' \\
\forme{-ɤt} & \phonet{-ata} & \forme{jɣɤt-a} $\Rightarrow$ \phonet{jɣata} `I will come back' \\
\forme{-ɤn} & \phonet{-ana} & \forme{tu-nɯsmɤn-a} $\Rightarrow$ \phonet{tunɯsmana} \\
&&  `I will will treat it' \\
\forme{-ɤr} & \phonet{-ara} & \forme{pɯ-atɤr-a} $\Rightarrow$ \phonet{patara} `I fell down' \\
\forme{-ɤl} & \phonet{-ala} & \forme{nɯ-nɯtɯfɕɤl-a} $\Rightarrow$ \phonet{nɯ-nɯtɯfɕal-a}\\
&& `I had diarrhea' \\
\forme{-ɤz} & \phonet{-aza} & \forme{mkʰɤz-a} $\Rightarrow$ \phonet{mkʰaza} `I am expert at it' \\
\lspbottomrule
\end{tabular}
\end{table}

The first dual \forme{-tɕi} suffix (\forme{-tsə} in some dialects of Japhug, §\ref{sec:indexation.suffixes.history}) only causes regular devoicing assimilation on the coda of the verb stem: \forme{-z}, \forme{-r}, \forme{-ɣ}, \forme{-ʁ} are realized as \forme{-s}, \forme{-ʂ}, \forme{-x}, \forme{-χ} when followed by \forme{-tɕi} (for instance \forme{mkʰɤz-tɕi} is pronounced \phonet{mkʰɛ́stɕi}). The labial coda \forme{-β} is not affected. 

The first plural \forme{-ji} has two allomorphs, \forme{-j} and \forme{-i}. The first one occurs on verb stems ending in open syllables, for instance \japhug{ɕe-j} `we (will) go', and the second follows verb stems in closed syllables, such as \forme{scit-i} `we are happy', with resyllabification of the coda (\forme{sci/ti}). Like the \forme{-a} suffix discussed above, the suffix \forme{-i} reveals the underlying form of the codas. The contrast between \forme{-ɯ} and \forme{-i} is neutralized when followed by the \textsc{1pl} suffix:; for instance, the last syllable of \forme{smi tʰɯ-βlɯ-j} `we made a fire'  and \forme{lɤpɯɣ pɯ-βli-j} `we planted radish' is considered to be homophonous by Tshendzin (§XXX).

\subsubsection{Non-first person}

Second and third person forms have the same set of suffixes (zero, \forme{-ndʑi} and \forme{-nɯ} for singular,dual and plural respectively) and only differ by the presence of a \forme{tɯ-} prefix in second person forms. Unlike in Situ (\citealt[197-208]{linxr93jiarong}), there is no distinct second person suffix in the \textsc{2sg}.

The non-first person dual and plural suffixes \forme{-ndʑi} and \forme{-nɯ} (some Japhug dialects have \forme{-ndzə} in the dual instead, see §\ref{sec:indexation.suffixes.history}) nasalize the coda \forme{-t} to \phonet{n}, which is not audible before \forme{-ndʑi} and results in a geminate in the plural. For instance, \forme{scit-ndʑi} and \forme{scit-nɯ} are realized as \phonet{scíndʑi} and \phonet{scínnɯ} respectively. The vowel \forme{-i} and \forme{-ɯ} is often elided, resulting in apparent \forme{-n} codas. The contrast between the codas \forme{-n} and \forme{-t} is neutralized in these forms: the last two syllables of both \forme{tu-nɤndɯt-nɯ} \textsc{ipfv}-fight-\textsc{pl} `they fight (over it)' and \forme{pjɯ-ndɯn-nɯ}  \textsc{ipfv}-read-\textsc{pl} `they read/recite it' are thus realized as \phonet{-ndɯ́nnɯ}.

The second person \forme{tɯ-} prefix fuses with the initial \forme{a-} of contracting verbs (§XXX). The result of vowel fusion is \forme{tɯ-a-} $\Rightarrow$ \phonet{ta} in the Factual Non-past (\japhug{tɯ-atɤr}{you will fall down}) or the Past Perfective (\japhug{jɤ-tɯ-ari}{you went there}), but \forme{tɯ-ɤ-} $\Rightarrow$ \phonet{tɤ} in Irrealis, Imperative, Imperfective or Prohibitive (\japhug{ma-tɤ-tɯ-ɤɕqʰe}{don't cough}) forms. Some irregular verbs have unpredictable second person forms (§\ref{sec:intr.person.irregular}). The generic intransitive subject prefix \forme{kɯ-} (also used for the object of transitive verbs, see \ref{sec:indexation.generic.tr}) follows the same rules of vowel fusion as the second person prefix.


\subsection{Irregular intransitive verbs} \label{sec:intr.person.irregular}
In comparison with Zbu (\citealt{gong18these}), Japhug only has very few irregular verbs. Irregularities related to person marking in Japhug all involve the prefixes.

The second person forms of sensory existential verbs \japhug{ɣɤʑu}{exist} and \japhug{maŋe}{not exist} are infixed rather prefixed. The infixed forms are \forme{ɣɤtɤʑu} and \forme{mataŋe} respectively, as in (\ref{ex:GAtAZu})  (from \citealt[91]{jacques12agreement}) and  (\ref{ex:kAmtshAm.mataNe}).

\begin{exe}
\ex \label{ex:GAtAZu}
\gll iɕqʰa tɯrme ra nɯ-rca ɣɤ<tɤ>ʑu \\
the.aforementioned person \textsc{pl} \textsc{3pl}.\textsc{poss}-following <2sg>exist:\textsc{sens} \\
\glt `(I saw) you among these people.' (elicited)
\end{exe}

\begin{exe}
\ex \label{ex:kAmtshAm.mataNe}
\gll kɤ-mtsʰɤm maka ma<ta>ŋe tɕe, nɤ-kɯ-mŋɤm tu ɯβrɤ-ŋu ma, mɤ-kɯ-pe tu ɯβrɤ-ŋu ma nɯra nɯ-sɯso-t-a. \\
\textsc{inf}-hear at.all <\textsc{2sg}>not.exist:\textsc{sens} \textsc{lnk} \textsc{2sg}.\textsc{poss}-\textsc{nmlz}:S/A-hurt exist:\textsc{fact} \textsc{opt}-be:\textsc{fact} \textsc{sfp} \textsc{2sg}.\textsc{poss}-\textsc{neg}-\textsc{nmlz}:S/A-be.good:\textsc{fact} exist:\textsc{fact} \textsc{opt}-be:\textsc{fact} \textsc{sfp} \textsc{dem}:\textsc{pl} \textsc{pfv}-think-\textsc{pst}:\textsc{tr}-\textsc{1sg} \\
\glt `(I) have not heard at all about you (for some time), I was wondering whether you have some disease, whether something bad happened to you.' (phone conversation, 16-12-28)
\end{exe}

These are not the only infixed forms in the paradigm of these verbs: the generic person \forme{kɯ-} is also infixed (\forme{ɣɤkɤʑu}, \forme{makaŋe}) as is the spontaneous-autobenefactive \forme{nɯ-} (§XXX).

The verb \japhug{zɣɯt}{reach, arrive} has in part of its paradigm forms that are identical to those of contracting verbs (§XXX). In the Past Perfective, it has two alternative second person forms in free variation, the regular \forme{jɤ-tɯ-zɣɯt} and the form \forme{jɤ-tɯ-azɣɯt} with an additional \forme{a-}, illustrated by (\ref{ex:jAtWzGWt.tCe}) and  (\ref{ex:jAtazGWt.mACtsxa}) respectively, coming from two versions of the same story by the same speaker. 

\begin{exe}
\ex \label{ex:jAtazGWt.mACtsxa}
\gll  a-rkɯ mɯ-jɤ-tɯ-azɣɯt mɤɕtʂa mɯ-pɯ-ta-mtsʰɤm tɕe \\
\textsc{1sg}.\textsc{poss}-side \textsc{neg}-\textsc{pfv}-2-arrive until \textsc{neg}-\textsc{pfv}-1\fl{}2-hear \textsc{lnk} \\
\glt `I did not feel your (presence) until you arrived near me.' (Norbzang2012, 260)
\end{exe}

\begin{exe}
\ex \label{ex:jAtWzGWt.tCe}
\gll jɤ-tɯ-zɣɯt tɕe, nɤki, aʑo a-kʰa a-jɤ-tɯ-z-mɤke ma nɤj nɤ-kʰa a-mɤ-jɤ-tɯ-z-mɤke ra mɯ-tɤ-tɯ-tɯt \\
\textsc{pfv}-2-arrive \textsc{lnk} \textsc{filler} \textsc{1sg} \textsc{1sg}.\textsc{poss}-house \textsc{irr}-\textsc{pfv}-2-\textsc{caus}-be.first[III] \textsc{lnk} 
\textsc{2sg} \textsc{2sg}.\textsc{poss}-house \textsc{irr}-\textsc{neg}-\textsc{pfv}-2-\textsc{caus}-be.first[III] have.to:\textsc{fact} \textsc{neg}-\textsc{pfv}-2-say[III] \\
\glt `You did not say ``When you arrive, don't go first to your house, come to my house first.'' (Norbzang2005, 261)
\end{exe}

The paradigm of this verb otherwise includes non-optional contracting (\forme{jɤ-azɣɯt} `he arrived') and non-contracting forms (the immediate converb \forme{ju-tɯ-zɣɯt} `as soon as X arrived', §\ref{sec:immediate.converb}).

\subsection{Semi-transitive verbs} \label{sec:semi.transitive}
Semi-transitive verbs have the same paradigm as plain intransitive verbs, and lack the morphological properties of transitive verbs (§\ref{sec:transitivity.morphology}). Their intransitive subject is in absolutive form. However, they take a semi-object (§\ref{sec:semi.object}), also in absolutive form, as \japhug{paχɕi}{apple} in (\ref{ex:paXCi.ci.taroa}). These semi-objects do present some objectal properties (§XXX).

\begin{exe}
\ex \label{ex:paXCi.ci.taroa}
\gll  tɕe aʑo tʰam kɯki, paχɕi ci tɤ-aro-a tɕe tɕendɤre, 
[...] nɯʑora kɯnɤ ta-sɯ-ɤʁe-nɯ ra \\
\textsc{lnk} \textsc{1sg} now \textsc{dem}.\textsc{prox} apple \textsc{indef} \textsc{pfv}-have-\textsc{1sg} \textsc{lnk} \textsc{lnk} { } \textsc{2pl} also 1\fl{}2-\textsc{caus}-have.to.eat:\textsc{fact}-\textsc{pl} have.to:\textsc{fact} \\
\glt `Now that I have (was given) this apple, I will give it to you also to eat.' (150904 zhongli-zh, 35)
\end{exe}

Unlike transitive verbs, which can index the number of the object if the subject is \textsc{1sg} (§\ref{sec:double.number.indexation}), semi-transitive verbs cannot stack a person index after the \textsc{1sg} \forme{-a}. For instance, in (\ref{ex:XsWm.aroa}), although the object is plural, a form such as $\dagger$\forme{aroa-a-nɯ} with the \forme{-nɯ} plural prefix is strictly prohibited. 

\begin{exe}
\ex   \label{ex:XsWm.aroa}
 \gll aʑo tɤ-rɟit χsɯm aro-a   \\
I \textsc{indef.poss}-child three have:\textsc{fact}-\textsc{1sg} \\
 \glt `I have three children.' (elicited)
\end{exe} 

The subject of some semi-transitive verbs, in particular \japhug{tso}{know, understand} and \japhug{ʑɣɤpa}{pretend}, can be optionally marked with the ergative like a transitive subject (§\ref{sec:S.kW}), as \forme{tɤ-mu nɯ kɯ} in (\ref{ex:kW.mWpjAtso}) and \forme{βdaʁmu nɯ kɯ} in (\ref{ex:kW.toZGApa}). 

\begin{exe}
\ex   \label{ex:kW.mWpjAtso}
 \gll  tɕendɤre [tɤ-mu nɯ kɯ] ɕɯ ŋu nɯ maka mɯ-pjɤ-tso tɕeri \\
\textsc{lnk} \textsc{indef}.\textsc{poss}-mother \textsc{dem} \textsc{erg} who be:\textsc{fact} \textsc{dem} at.all \textsc{neg}-\textsc{ifr}.\textsc{ipfv}-know \textsc{lnk} \\
\glt `The old woman did not realize who it was.' (2002qaCpa, 242)
\end{exe}

\begin{exe}
\ex   \label{ex:kW.toZGApa}
 \gll iɕqʰa βdaʁmu nɯ kɯ [wuma ʑo ɯ-sɯm kɯ-sna] to-ʑɣɤpa \\
 the.aforementioned lady \textsc{dem} \textsc{erg} really \textsc{emph} \textsc{3sg}.\textsc{poss}-mind nmlz:S/A-be.good \textsc{ifr}-pretend \\
 \glt `The lady pretended to be a good person.' (140520 ye tiane-zh, 44)
\end{exe}


Some semi-transitive verbs can take both nominal semi-object and complement clauses. For instance, \forme{tso} (which can be translated as `know', `understand' or `realize' depending on the context) occurs with nouns referring to speech or meaning as semi-object (as in \ref{ex:apWtWtso.smWlAm}), finite relative clauses (\ref{ex:rCanW.mWkAtsoa}) and also participial clauses (\ref{ex:kWNu.kutsoa}; see §XXX concerning the analysis of such clauses).

\begin{exe}
\ex   \label{ex:apWtWtso.smWlAm}
 \gll pja mɯndʐamɯχtɕɯɣ nɯ-skɤt a-pɯ-tɯ-tso smɯlɤm! \\
 bird all.type \textsc{3pl}.\textsc{poss}-speech \textsc{irr}-\textsc{pfv}-2-understand wish \\
 \glt `May you understand the speech of all the species of birds!'(2003kandZislama, 85)
\end{exe}

\begin{exe}
\ex   \label{ex:rCanW.mWkAtsoa}
 \gll ci ta-pa-tɕi, nɯstʰɯci tɤ-nɤrʑaʁ ri, [nɯstʰɯci nɤ-ku ʑru rcanɯ] mɯ-kɤ-tso-a \\
 one \textsc{pfv}:3\fl{}3'-do-\textsc{du} so.much \textsc{pfv}-pass(time) so.much \textsc{2sg}.\textsc{poss}-head be.strong:\textsc{fact} \textsc{foc}:\textsc{unexp} \textsc{neg}-\textsc{pfv}-know-\textsc{1sg} \\
\glt `So much time has passed since we have married, I did not realize that your hair was so long.' (Kunbzang2003, 467)
\end{exe}

\begin{exe}
\ex   \label{ex:kWNu.kutsoa}
 \gll  tɕe [tɕʰi ɯ-skɤt kɯ-ŋu ra] ku-tso-a ɲɯ-ra ma tu-tɯ-ti stɯsti, mɯ́j-ɕɯftaʁ-a ɲɯ-ti \\
\textsc{lnk} what \textsc{3sg}.\textsc{poss}-speech \textsc{nmlz}:S/A-be \textsc{pl} \textsc{ipfv}-understand-\textsc{1sg} \textsc{sens}-have.to \textsc{lnk} \textsc{ipfv}-2-say alone \textsc{neg}:\textsc{sens}-remember-\textsc{1sg} \textsc{sens}-say \\
 \glt `He says: `I need to understand what it is about (what objects these words refer to), otherwise if you only speak (if you only explain orally) I won't remember.'' (conversation 14-05-10, 79)
\end{exe}
%\begin{exe}
%\ex   \label{ex:ŋundZi.mWkAtsoa}
% \gll pʰu ŋu ɕi, pʰu ci mu ŋu-ndʑi nɯ mɯ-kɤ-tso-a ma \\
% male be:\textsc{fact} \textsc{qu} male \textsc{indef} female be:\textsc{fact}-\textsc{du} \textsc{dem} \textsc{neg}-\textsc{pfv}-know-\textsc{1sg} \\
% \glt `I did not get to know whether it is the male that is like that, or both whether both the male and the female are.' (24-ZmbrWpGa, 81)
%\end{exe} 
Among semi-transitive verbs, we find the following subclasses:

\begin{itemize}
\item Verbs of cognition and perception: \japhug{tso}{know, understand}, \japhug{sɤŋo}{listen}
\item Verbs of evaluation: \japhug{rga}{like}, \japhug{stu}{believe}
\item Modal verbs: \japhug{cʰa}{can}
\item Verbs of possession:  \japhug{aro}{have}
\item Copulas: \japhug{ŋu}{be}, \japhug{maʁ}{not be}
\item Verbs of assignation: \japhug{rmi}{be called}, \japhug{artsi}{count as}, \japhug{fse}{be like}
\item Verbs requiring an argument expressing time: \japhug{acʰɤt}{have X years of difference}, \japhug{tsu}{pass X time}
\item Verbs of pretense:  \japhug{ʑɣɤpa}{pretend}
\item Verbs of obtention: \japhug{aʁe}{have to eat}, \japhug{βɟɤt}{get, obtain}
\item Some adjectival stative verbs: \japhug{mkʰɤz}{be expert}, \japhug{pʰɤn}{be efficient}
\end{itemize}

Most semi-transitive verbs are underived bare roots. The only obviously derived verbs are \japhug{ʑɣɤpa}{pretend}, which comes from the reflexive (§XXX) of the verb \japhug{pa}{do} and \japhug{artsi}{count as}, passive of \japhug{rtsi}{count}. The verb \japhug{aro}{have} might be denominal from \japhug{tɤ-ro}{surplus, leftover}.

Most semi-transitive verbs do not usually take a human semi-object, so that sentences with a first or second person semi-object are generally clumsy to build. For some of the verbs above, applicative forms are used when a first or second person object is needed, for instance \japhug{nɯrga}{like} and \japhug{nɤstu}{believe}. The verbs \japhug{stu}{believe} and \japhug{nɤstu}{believe} differ in that the semi-object of the former refers to words (in general, a complement clause; `believe that X') while the object of the latter is a person (`believe him'). 

However, examples with subjects and semi-objects both either first or second person are attested. For instance, (\ref{ex:WYWfsea}) shows a very spontaneous use of a \textsc{2sg} semi-object with a \textsc{1sg} subject with the verb \japhug{fse}{be like}. Only the subject is indexed (with the suffix \forme{-a}) and the use of the transitive \forme{ta-} 1\fl{}2 portmanteau prefix (§\ref{sec:indexation.local}) here would be nonsensical. 

\begin{exe}
\ex \label{ex:WYWfsea}
\gll a-ʁi, nɤʑo ɯ-ɲɯ-fse-a? \\
\textsc{1sg}.\textsc{poss}-younger.sibling \textsc{2sg} \textsc{qu}-\textsc{sens}-be.like-\textsc{1sg} \\
\glt `Sister, do I look like you?' (2014-kWlAG, 475)
\end{exe}


Some semi-transitive verbs are labile; some have a transitive counterpart, while other ones have a plain intransitive one (§\ref{sec:semi.tr.labile}). The meaning of the verb also slightly changes depending on transitivity (for instance, \forme{rga} means `like' when semi-transitive, and `be happy' when stative intransitive).

The copulas are a distinct subclass of semi-transitive verbs, in that their semi-object is a predicative noun. Bare predicative nouns without any verbal element do occur in the corpus, but are rare (§XXX). Person indexation on copula generally follows the subject, but we do find examples in which the indexation agrees with the predicative noun, as in (\ref{ex:stAmku.nWra.NunW}), where the verb has plural indexation like the predicative noun phrase \forme{stɤmku nɯra}, whereas the subject is in the singular. Note that the plural \forme{nɯra} here cannot be interpreted as approximate location (§\ref{sec:plural.determiners}), because in the next sentence the grasslands are analogically referred to by the plural demonstrative pronoun \forme{nɯra} (§\ref{sec:anaphoric.demonstrative.pro}).

\begin{exe}
\ex \label{ex:stAmku.nWra.NunW}
\gll  stu ɯ-sɤ-dɤn nɯ [stɤmku nɯra] ŋu-nɯ. tɕe nɯra nɯ-ŋgɯ tɕe tu-ɬoʁ ŋu tɕe \\
most \textsc{3sg}.\textsc{poss}-\textsc{nmlz}:\textsc{obl}-be.many \textsc{dem} grassland \textsc{dem}:\textsc{pl} be:\textsc{fact}-\textsc{pl} \textsc{lnk} \textsc{dem}:\textsc{pl} \textsc{3pl}.\textsc{poss}-inside \textsc{loc} \textsc{ipfv}-come.out be:\textsc{fact} \textsc{lnk} \\
\\
\glt `The place where it is most numerous is the grasslands, and it grows in these' (19-qachGa mWntoR, 24-25)
\end{exe}

\subsection{Intransitive verbs with oblique arguments} \label{sec:intr.goal}
Semi-transitive verbs have to be distinguished from motion verbs (or perception verbs) with a goal (§\ref{absolutive.goal}), such as \japhug{ɕe}{go}, \japhug{ɣi}{come} or \japhug{ru}{look at}. These verbs are morphologically intransitive, lacking the morphological characteristics of transitive verbs (§\ref{sec:transitivity.morphology}).

With these verbs, the goal can occur in absolutive form, and superficially resembles a semi-object, as \japhug{sɯŋgɯ}{forest} in (\ref{ex:sWNgW.joCendZi}). Indexation obligatorily occurs with the subject (for example, the \textsc{3du} form in \ref{ex:sWNgW.joCendZi}), never with the goal. As in the case of semi-transitive verbs, number stacking on the 1sg \forme{-a} is not possible (§\ref{sec:semi.transitive}, example \ref{ex:XsWm.aroa}).

\begin{exe}
\ex   \label{ex:sWNgW.joCendZi}
 \gll ʁnɯz ni, [sɯŋgɯ] jo-ɕe-ndʑi. \\
two \textsc{du} forest \textsc{ifr}-go-\textsc{du} \\
\glt `Two (men) went into the forest.' (26-tAGe, 1)
\end{exe}

However, unlike semi-objects, these goals can optionally take locative postpositions, such as \forme{zɯ} in (\ref{ex:sWNgW.zW.joCe}).

\begin{exe}
\ex   \label{ex:sWNgW.zW.joCe}
 \gll tɤ-pɤtso nɯnɯ li [sɯŋgɯ zɯ] jo-ɕe. \\
 \textsc{indef}.\textsc{poss}-child \textsc{dem} again forest \textsc{loc} \textsc{ifr}-go \\
 \glt `The child went again into the forest.' (140428 yonggan de xiaocaifen-zh, 230)
\end{exe}

Dative marking on the goals is also well-attested, as in (\ref{ex:sWNgW.WCki.joCe}) -- with motion verbs, it translates as `towards X'.

\begin{exe}
\ex   \label{ex:sWNgW.WCki.joCe}
 \gll tɕhemɤpɯ nɯ kɯ ɯ-wa cʰo ɯ-pi nɯra ɲɤ-βde tɕe, sɯŋgɯ ɯ-ɕki tɕe jo-ɕe. \\
girl \textsc{dem} \textsc{erg} \textsc{3sg}.\textsc{poss}-father \textsc{comit} \textsc{3sg}.\textsc{poss}-elder.sibling \textsc{dem}:\textsc{pl} \textsc{ifr}-leave \textsc{lnk} forest \textsc{3sg}.\textsc{poss}-\textsc{dat} \textsc{loc} \textsc{ifr}-go \\
\glt `The girl left her father and her brothers, and went toward the forest.' (140506 shizi he huichang de bailingniao, 76)
\end{exe}

The subject of intransitive verbs with goals is in absolutive form, except when shared with a transitive verb in another clause, as \forme{tɕhemɤpɯ nɯ kɯ} in (\ref{ex:sWNgW.WCki.joCe}), which owes its ergative marking to the transitive verb \forme{ɲɤ-βde} `She left them'. The verb \japhug{rpu}{bump} (which takes as goal the surface of physical contact) however can take ergative subjects, as it is labile and can be conjugated transitively (§\ref{sec:goal.labile}).

Some verbs, such as \japhug{atɯɣ}{meet}, select an oblique comitative argument in \forme{cʰo} (§\ref{sec:comitative}).

\subsection{Intrinsically non-singular subjects}
Some intransitive verbs have an intrinsic reciprocal meaning, and do not occur in singular form. This category includes most derived reciprocal verbs (§XXX), but also some historical reciprocal verbs that are synchronically non-analyzable such as \japhug{amɯmi}{be in good terms}, and denominal verbs in \forme{a-} like \japhug{anɯmqaj}{fight} or \japhug{aɕɣa}{be of the same age} (§XXX). Example (\ref{ex:atAtAnWmqajnW}) provides some examples of verbs of this type. In the corpus, these verbs only occur in dual or plural form.

\begin{exe}
\ex   \label{ex:atAtAnWmqajnW}
 \gll tɕe a-pi a-ʁi ra kutɕu a-nɯ-tɯ-ɤnɯɣro-nɯ, ci ci a-tɤ-tɯ-ɤnɯmqaj-nɯ, ci ci a-tɤ-tɯ-ɤmɯmi-nɯ qʰe a-kɤ-tɯ-nɯ-rɤʑi-nɯ, \\
\textsc{lnk} \textsc{1sg}.\textsc{poss}-elder.sibling  \textsc{1sg}.\textsc{poss}-younger.sibling \textsc{pl}  here \textsc{irr}-\textsc{pfv}-2-<\textsc{auto}>play-\textsc{pl} once once \textsc{irr}-\textsc{pfv}-2-fight-\textsc{pl} once once \textsc{irr}-\textsc{pfv}-2-be.in.good.terms-\textsc{pl} \textsc{lnk} \textsc{irr}-\textsc{pfv}-2-\textsc{auto}-stay-\textsc{pl}\\
\glt `Brothers, stay here and play, fight from time to time, reconcile with each other from time to time.' (2003kandzwsqhaj, 43)
\end{exe}

The subject of these verbs can be a noun phrase comprising a comitative postpositional phrase in \forme{cʰo} (see §\ref{sec:comitative} and §\ref{sec:coordinator.cho}); number indexation on the verb reflects the addition of the added number of all nominals in the noun phrase. For example, in (\ref{ex:cho.pjAkAmWmindZi}) and (\ref{ex:cho.aCGAtCi}), the verbs have dual indexation, referring to the total number of individuals in the subject noun phrase connected by the comitative \forme{cʰo}.

\begin{exe}
\ex   \label{ex:cho.pjAkAmWmindZi}
 \gll  <maerjina> nɯ cʰo <alibaba> ni wuma ʑo pjɤ-k-ɤmɯmi-ndʑi tɕe \\
p.n. \textsc{dem} \textsc{comit} p.n. \textsc{du} really \textsc{emph} \textsc{pst}.\textsc{ifr}-\textsc{evd}-be.in.good.terms-\textsc{du} \textsc{lnk} \\
\glt `Maerjina and Alibaba were in very good terms.' (140512 alibaba-zh, 306)
\end{exe}

\begin{exe}
\ex   \label{ex:cho.aCGAtCi}
 \gll  nɤj nɤ-mu cʰo aʑo ni aɕɣa-tɕi \\
 \textsc{2sg} \textsc{2sg}.\textsc{poss}-mother \textsc{comit} \textsc{1sg} \textsc{du} be.of.the.same.age:\textsc{fact}-\textsc{1du} \\
 \glt `I have the same age as your mother.' (`You mother and I have the same age') (elicited)
\end{exe} 

The verb \japhug{acʰɤt}{have X years of difference} has an intrinsically non-singular subject, and is at the same time semi-transitive (§\ref{sec:semi.transitive}), taking as semi-object a temporal noun phrase expressing the age difference between the members of the group referred to by the subject, for instance \forme{ʁnɯ-pɤrme nɤ ʁnɯ-pɤrme} `two years each' in (\ref{ex:RnWpArme.machAti}).

\begin{exe}
\ex   \label{ex:RnWpArme.machAti}
 \gll tɕe iʑo kɤndʑiʁi ra ʁnɯ-pɤrme nɤ ʁnɯ-pɤrme ntsɯ ma mɤ-acʰɤt-i \\
 \textsc{lnk} \textsc{1pl} \textsc{coll}:sibling \textsc{pl} two-year \textsc{lnk} two-year always apart.from \textsc{neg}-differ.in.age:\textsc{fact}-\textsc{1pl} \\
\glt `We brother and sisters were born in two years intervals.' (if ranked by birth order, each couple of adjacent sibling differ in age from each other by two years each) (14-tApitaRi, 243)
 \end{exe}
 
The verb \japhug{alɯlɤt}{fight}, historically the reciprocal of \japhug{lɤt}{throw, release} (§XXX), almost always has non-singular indexation, as in (\ref{ex:qro.ni.YAlWlAtndZi}).


\begin{exe}
\ex   \label{ex:qro.ni.YAlWlAtndZi}
 \gll  tɕeki qro ni ɲɯ-ɤlɯlɤt-ndʑi \\
 down ant \textsc{du} \textsc{sens}-fight-\textsc{du} \\
\glt `Down there two ants are fighting.' (conversation140501-01)
\end{exe}

However, examples with singular indexation are also attested, for instance (\ref{ex:tAtalWlAt}) and (\ref{ex:tutalWlAt}) with \textsc{2sg} form. Both are from texts translated from Chinese, but were not considered infelicitous by Tshendzin. The singular verb forms might be due to calquing, but are not radically ungrammatical.\footnote{The Chinese original sentences of examples (\ref{ex:tAtalWlAt}) and (\ref{ex:tutalWlAt}) are \ch{你是为众人的利益而战}{nǐ shì wèi zhòngrén de lìyì érzhàn}{You were fighting for the interest of the people} and \ch{你不该和舅舅动手}{nǐ bùgāi hé jiùjiù dòngshǒu}{You should not get into a fight with your uncle} respectively.  } In (\ref{ex:tutalWlAt}), instead of using dual indexation, the person with whom the subject fights is marked with the dative.

\begin{exe}
\ex   \label{ex:tAtalWlAt}
 \gll  ki ɕɯŋgɯ ki pɯpɯŋunɤ, nɤʑo kɯ iɕqʰa mkʰɤrmaŋ ɣɯ nɯ-ndʐa kɯ tɤ-tɯ-alɯlɤt pɯ-ŋu tɕe, \\
 \textsc{dem}.\textsc{prox} before \textsc{dem}.\textsc{prox} \textsc{top} \textsc{2sg} \textsc{erg} \textsc{filler} people \textsc{gen} \textsc{3pl}.\textsc{poss}-reason \textsc{erg} \textsc{pfv}-2-fight \textsc{pst}.\textsc{ipfv}-be \textsc{lnk} \\
\glt `The previous time, you fought for the sake of the people.' (140512 abide he mogui-zh, 89)
 \end{exe}

\begin{exe}
\ex   \label{ex:tutalWlAt}
 \gll  tɕe nɤʑo nɤ-rpɯ ɯ-ɕki, nɤkinɯ, tu-tɯ-ɤlɯlɤt ndɤre mɤ-pe \\
 \textsc{lnk} \textsc{2sg} \textsc{2sg}.\textsc{poss}-MB \textsc{3sg}.\textsc{poss}-\textsc{dat} \textsc{filler} \textsc{ipfv}-2-fight \textsc{lnk} \textsc{neg}-be.good:\textsc{fact} \\
 \glt `It is not good for you to fight with your uncle.' (150826 baoliandeng-zh, 185)
  \end{exe}
  
\subsection{Invariable intransitive verbs}
%thɯ-nɯɕe ma, mɤ-tɯ-ra

\section{Transitive verbs}

\subsection{The morphological marking of transitivity in Japhug} \label{sec:transitivity.morphology}

%The past tense \forme{-t} suffix

\subsection{The direct-inverse system} \label{sec:direct-inverse}

\subsubsection{Mixed configurations} \label{sec:indexation.moxed}

\subsubsection{Non-local configurations} \label{sec:indexation.non.local}

\subsubsection{Local configurations} \label{sec:indexation.local}

\subsubsection{Generic indexation} \label{sec:indexation.generic.tr}

\subsubsection{Double number indexation}  \label{sec:double.number.indexation}

\begin{landscape}
\begin{table}[H]
\caption{Japhug transitive and intransitive paradigms}\label{tab:japhug.tr}
\resizebox{\columnwidth}{!}{
\begin{tabular}{l|l|l|l|l|l|l|l|l|l|l|}
\textsc{} & 	\textsc{1sg} & 	  \textsc{1du} & 	\textsc{1pl} & 	\textsc{2sg} & 	\textsc{2du} & 	\textsc{2pl} & 	\textsc{3sg} & 	\textsc{3du} & 	\textsc{3pl} & 	\textsc{3'} \\ 	
\hline
\textsc{1sg} & \multicolumn{3}{c|}{\grise{}} &	\forme{} & 	\forme{} & 	\forme{} & 	\forme{\sigc{}-a}   & 	 \forme{\sigc{}-a-ndʑi} & 	 \forme{\sigc{}-a-nɯ} & 	\grise{} \\	
\cline{8-10}
\textsc{1du} & 	\multicolumn{3}{c|}{\grise{}} &	\forme{ta-\siga{}} & 	\forme{ta-\siga{}-ndʑi} & 	\forme{ta-\siga{}-nɯ} & 	\multicolumn{3}{c|}{ \forme{\siga{}-tɕi}}  & 	\grise{} \\	
\cline{8-10}
\textsc{1pl} & 	\multicolumn{3}{c|}{\grise{}} & 	  & 	&  & 	\multicolumn{3}{c|}{ \forme{\siga{}-ji}}  & 	\grise{} \\	
\hline
\textsc{2sg} & 	\forme{kɯ-\siga{}-a} & 	\forme{} & 	\forme{} & 	\multicolumn{3}{c|}{\grise{}}&	\multicolumn{3}{c|}{\forme{tɯ-\sigc{}}} & 	\grise{} \\	
\cline{2-2}
\cline{8-10}
\textsc{2du} & 	\forme{kɯ-\siga{}-a-ndʑi} & 	\forme{kɯ-\siga{}-tɕi} & 	\forme{kɯ-\siga{}-ji} & 	\multicolumn{3}{c|}{\grise{}} &	\multicolumn{3}{c|}{\forme{tɯ-\siga{}-ndʑi}} & 	\grise{} \\	
\cline{2-2}
\cline{8-10}
\textsc{2pl} & 	\forme{kɯ-\siga{}-a-nɯ} & 	\forme{} & 	\forme{} & 	\multicolumn{3}{c|}{\grise{}}&	\multicolumn{3}{c|}{\forme{tɯ-\siga{}-nɯ}} & 	\grise{} \\	
\hline
\textsc{3sg} &  	\forme{wɣɯ́-\siga{}-a} & 	\forme{} & 	\forme{} & 	\forme{} & 	\forme{} & 	\forme{} & \multicolumn{3}{c|}{\grise{}} &	\forme{\sigc{}} \\ 	
\cline{2-2}
\cline{11-11}
\textsc{3du} &  	\forme{wɣɯ́-\siga{}-a-ndʑi} & 	 \forme{wɣɯ́-\siga{}-tɕi} & 		\forme{wɣɯ́-\siga{}-ji} & 	\forme{tɯ́-wɣ-\siga{}} & 	\forme{tɯ́-wɣ-\siga{}-ndʑi} & 	\forme{tɯ́-wɣ-\siga{}-nɯ} & 	\multicolumn{3}{c|}{\grise{}} &	\forme{\siga{}-ndʑi} \\ 
\cline{2-2}	
\cline{11-11}
\textsc{3pl} &  	\forme{wɣɯ́-\siga{}-a-nɯ} & 	\forme{} & 	\forme{} & 	\forme{} & 	\forme{} & 	\forme{} & \multicolumn{3}{c|}{\grise{}} &	\forme{\siga{}-nɯ} \\ 	
\hline
\textsc{3'} & 	\multicolumn{6}{c|}{\grise{}} &	\forme{wɣɯ́-\siga{}} & 	\forme{wɣɯ́-\siga{}-ndʑi} & 	\forme{wɣɯ́-\siga{}-nɯ} & 	\grise{} \\	
	\hline	\hline
\textsc{intr}&\forme{\siga{}-a}&\forme{\siga{}-tɕi}&\forme{\siga{}-ji}&\forme{tɯ-\siga{}}&\forme{tɯ-\siga{}-ndʑi}&\forme{tɯ-\siga{}-nɯ}&\forme{\siga{}}&\forme{\siga{}-ndʑi} &\forme{\siga{}-nɯ}& 	\grise{} \\	
\hline
\end{tabular}}
\end{table}


\begin{table}[H]
\caption{The paradigm of the verb \japhug{mto}{see} in the Factual non-past}\label{tab:mto.paradigm}
\resizebox{\columnwidth}{!}{
\begin{tabular}{l|l|l|l|l|l|l|l|l|l|l|}
\textsc{} & 	\textsc{1sg} & 	  \textsc{1du} & 	\textsc{1pl} & 	\textsc{2sg} & 	\textsc{2du} & 	\textsc{2pl} & 	\textsc{3sg} & 	\textsc{3du} & 	\textsc{3pl} & 	\textsc{3'} \\ 	
\hline
\textsc{1sg} & \multicolumn{3}{c|}{\grise{}} &	\forme{} & 	\forme{} & 	\forme{} & 	\forme{mtam-a}   & 	 \forme{mtam-a-ndʑi} & 	 \forme{mtam-a-nɯ} & 	\grise{} \\	
\cline{8-10}
\textsc{1du} & 	\multicolumn{3}{c|}{\grise{}} &	\forme{ta-mto} & 	\forme{ta-mto-ndʑi} & 	\forme{ta-mto-nɯ} & 	\multicolumn{3}{c|}{ \forme{mto-tɕi}}  & 	\grise{} \\	
\cline{8-10}
\textsc{1pl} & 	\multicolumn{3}{c|}{\grise{}} & 	  & 	&  & 	\multicolumn{3}{c|}{ \forme{mto-j}}  & 	\grise{} \\	
\hline
\textsc{2sg} & 	\forme{kɯ-mto-a} & 	\forme{} & 	\forme{} & 	\multicolumn{3}{c|}{\grise{}}&	\multicolumn{3}{c|}{\forme{tɯ-mtɤm}} & 	\grise{} \\	
\cline{2-2}
\cline{8-10}
\textsc{2du} & 	\forme{kɯ-mto-a-ndʑi} & 	\forme{kɯ-mto-tɕi} & 	\forme{kɯ-mto-j} & 	\multicolumn{3}{c|}{\grise{}} &	\multicolumn{3}{c|}{\forme{tɯ-mto-ndʑi}} & 	\grise{} \\	
\cline{2-2}
\cline{8-10}
\textsc{2pl} & 	\forme{kɯ-mto-a-nɯ} & 	\forme{} & 	\forme{} & 	\multicolumn{3}{c|}{\grise{}}&	\multicolumn{3}{c|}{\forme{tɯ-mto-nɯ}} & 	\grise{} \\	
\hline
\textsc{3sg} &  	\forme{wɣɯ́-mto-a} & 	\forme{} & 	\forme{} & 	\forme{} & 	\forme{} & 	\forme{} & \multicolumn{3}{c|}{\grise{}} &	\forme{mtɤm} \\ 	
\cline{2-2}
\cline{11-11}
\textsc{3du} &  	\forme{wɣɯ́-mto-a-ndʑi} & 	 \forme{wɣɯ́-mto-tɕi} & 		\forme{wɣɯ́-mto-j} & 	\forme{tɯ́-wɣ-mto} & 	\forme{tɯ́-wɣ-mto-ndʑi} & 	\forme{tɯ́-wɣ-mto-nɯ} & 	\multicolumn{3}{c|}{\grise{}} &	\forme{mto-ndʑi} \\ 
\cline{2-2}	
\cline{11-11}
\textsc{3pl} &  	\forme{wɣɯ́-mto-a-nɯ} & 	\forme{} & 	\forme{} & 	\forme{} & 	\forme{} & 	\forme{} & \multicolumn{3}{c|}{\grise{}} &	\forme{mto-nɯ} \\ 	
\hline
\textsc{3'} & 	\multicolumn{6}{c|}{\grise{}} &	\forme{wɣɯ́-mto} & 	\forme{wɣɯ́-mto-ndʑi} & 	\forme{wɣɯ́-mto-nɯ} & 	\grise{} \\	
\hline
\end{tabular}}
\end{table}
\end{landscape}


\subsection{Ditransitive verbs}

\subsubsection{Indirective}
\subsubsection{Secundative}
%including stu
\subsubsection{Causative}
\subsection{The function of the direct/inverse contrast in non-local configurations}

\subsection{An irregular verb}

\section{Labile verbs}
\subsection{Transitive-intransitive labile verbs}
\subsection{Transitive-intransitive labile verbs with oblique arguments} \label{sec:goal.labile}
%rpu
\subsection{Semi-transitive labile verbs}\label{sec:semi.tr.labile}
%sɤŋo, tso me, rga
\section{Agreement mismatch}
\subsection{Optional number indexation} \label{sec:optional.indexation}
%"ma nɯra aʑo a-pi ŋu-nɯ wo" to-ti.

\subsection{Plural as honorific} \label{sec:honorific.indexation}
%tɕendɤre, wo nɯ kɯ-fse ci tɯ-ŋu, nɯ mɤ-kɯ-naχtɕɯɣ ci tɯ-ŋu tɯ-ŋu-nɯ 
\subsection{Partitive indexation} \label{sec:partitive.indexation}
Unlike with second or third persons, with first persons number indexation is absolutely compulsory. Apparent examples of mismatch however do exist, but are confined to a very specific partitive use of dual of plural number. With the interrogative pronoun \japhug{ɕɯ}{who} (§\ref{sec:CW.pronoun}), in particular, indexation on the verb can be non-singular with the specific partitive meaning `who among $X$', in particular in comparative constructions as in (\ref{ex:CW.kW.YWmpCArtCi}), with \textsc{1du} indexation on the verb although this sentence implies that only one of the two sisters is the most beautiful (see §XXX on this comparative construction, and \citet{jacques16comparative} and §\ref{sec:comparee.kW} on the use of the ergative here).

\begin{exe}
\ex   \label{ex:CW.kW.YWmpCArtCi}
 \gll  a-ʁi, nɤki tɕetʰi tɕe, tɯ-ci ɯ-ŋgɯ ɕ-pɯ-ru tɕe, ɕɯ kɯ ɲɯ-mpɕɤr-tɕi kɯ? \\
 \textsc{1sg}.\textsc{poss}-younger.sibling \textsc{filler} downstream \textsc{loc} \textsc{indef}.\textsc{poss}-water \textsc{3sg}.\textsc{poss}-inside \textsc{transloc}-\textsc{imp}:\textsc{down}-look \textsc{lnk} who \textsc{erg} \textsc{sens}-be.beautiful-\textsc{1sg} \textsc{sfp} \\
 \glt `Sister, go and look down there in the water, who is the most beautiful of us?' (2014-kWlAG, 477)
\end{exe} 


%ɕɯ kɯ nɯ stu ʑo, nɤkinɯ, kɯ-sɤmtshɤr, ʑɯmkhɤm ɯ-ku kɯ-rkɯn kɯ-fse kɤ-ɣɯt kɯ-cha nɯnɯ kɯ,
%kɯki @nuoha nɯ nɯ-rʑaβ ku-tɯ-nɯ-sɯβzu-nɯ jɤɣ" to-ti
%tɯ-rdoʁ kɯ a-sci a-thɯ-ndo-nɯ ntshi ɲɯ-sɯsɤm pjɤ-ŋu 

Another type of partitive indexation is found with \textsc{1pl} pronouns and third person indexation, as in (\ref{ex:iZora.tutinW}), with two verbs in \textsc{3pl}\fl{}3 form and the \textsc{1pl} pronoun in topicalized position meaning `some among us'.

\begin{exe}
\ex   \label{ex:iZora.tutinW}
 \gll  iʑora tɕe ɕkɤpʰɤr tu-ti-nɯ tsuku kɯ ɕkɤjwaʁ tu-ti-nɯ ŋu ma \\
 \textsc{1pl} \textsc{lnk} wild.chives \textsc{ipfv}-say-\textsc{pl} some \textsc{erg}  wild.chives \textsc{ipfv}-say-\textsc{pl} be:\textsc{fact} \textsc{lnk} \\
\glt `Among us, some call it \forme{ɕkɤpʰɤr}, some \forme{ɕkɤjwaʁ}.'(07-Cku, 82)
\end{exe} 

\subsection{First person and generic}

Another type of agreement mismatch observed with first person concerns \textsc{1pl} and generic person. In (\ref{ex:tWZAra.pWxtCij}), the adjectival stative verb \japhug{xtɕi}{be small} bears \textsc{1pl} indexation, but the corresponding overt pronoun in the sentence is the generic person \japhug{tɯʑɤra}{one} (§\ref{sec:genr.pro}). The generic form \forme{pɯ-kɯ-xtɕi}, as in (\ref{ex:pWkWxtCi.tCe.pWwGmto}) would be possible in the exactly the same context, clearly including the first person.

\begin{exe}
\ex   \label{ex:tWZAra.pWxtCij}
 \gll tɯʑɤra pɯ-xtɕi-j tɕe, \\
 \textsc{genr} \textsc{pst}.\textsc{ipfv}-be.small-\textsc{1pl} \textsc{lnk} \\
 \glt `When we were young.' (17-ndZWnW, 52)
\end{exe}

\begin{exe}
\ex   \label{ex:pWkWxtCi.tCe.pWwGmto}
 \gll tɕe jinde aj pɯ-mto-t-a me ri, pɯ-kɯ-xtɕi tɕe pɯ́-wɣ-mto \\
 \textsc{lnk} now \textsc{1sg} \textsc{pfv}-see-\textsc{pst}:\textsc{tr}-\textsc{1sg} not.exist:\textsc{fact} \textsc{lnk} \textsc{pst}.\textsc{ipfv}-\textsc{genr}:S/P-be.small \textsc{lnk} \textsc{pfv}-\textsc{inv}-see \\
\glt `I have not seen it recently, but when we were young, we did see it.' (22-qomndroN, 35)
\end{exe}

Conversely, in (\ref{ex:iZo.ci.YWwGphWt}), the \textsc{1pl} pronoun \forme{iʑo} seems to agree with a verb in transitive subject generic form (\ref{sec:indexation.generic.tr}). However, here the use of the generic form with the imperfective on the verb implies a slight deontic or gnomic meaning; the same configuration is also found in procedural texts as in (\ref{ex:iZo.luWGnWBzu}).

\begin{exe}
\ex   \label{ex:iZo.ci.YWwGphWt}
 \gll  iʑo kɯ-mɤku pɤjkʰu, ɯ-ɕɣa kɯ-mtɕoʁ nɯ ci ɲɯ́-wɣ-pʰɯt \\
 \textsc{1pl} \textsc{nmlz}:S/A-be.first still \textsc{3sg}.\textsc{poss}-tooth \textsc{nmlz}:S/A-be.sharp \textsc{dem} a.little \textsc{ipfv}-\textsc{inv}-take.out \\
\glt `Let us first take out its sharp teeth.' (150908 menglang-zh, 80)
\end{exe}

\begin{exe}
\ex   \label{ex:iZo.luWGnWBzu}
 \gll iʑo kɯrɯ ra, nɤkinɯ, qajɣi lú-wɣ-nɯ-βzu tɕe \\
\textsc{1pl} Tibetan \textsc{pl} \textsc{filler} bread \textsc{ipfv}-\textsc{inv}-\textsc{auto}-make \textsc{lnk} \\
\glt `We Tibetans, when we make bread,' (160706 thotsi, 1)
\end{exe}
\section{The historical relationshop between person indexation suffixes and possessive prefixes} \label{sec:indexation.suffixes.history}

\section{The origin of portmanteau prefixes}

\section{Person indexation on non-finite predicative words} \label{sec:non.finite.indexation}
 %Person indexation
%\chapter{Orientation and associated motion}
\section{Associated motion}

\subsection{Motion verbs and AM prefixes}
In Japhug, there is no constraint against AM prefixes occurring on motion verbs with the same deixis. Examples (\ref{ex:GWjuGinW}) and (\ref{ex:CpjACe}) respectively illustrate the cislocative on the verb \japhug{ɣi}{come} and the translocative on the verb \japhug{ɕe}{go}. Such examples are not common enough to allow a clear analysis of the semantic value of the redundant AM in these examples.

\begin{exe}
\ex \label{ex:GWjuGinW}
 \gll <jiazhang> ra ju-ɣi-nɯ tɕe <laoshi> ɯ-ɕki, tɯ-ɕki ʑo ɣɯ-ju-ɣi-nɯ ɕti netɕi? \\
 parents \textsc{pl} \textsc{ipfv}-come-\textsc{pl} \textsc{lnk} teacher \textsc{3sg}.\textsc{poss}-\textsc{dat} \textsc{genr}.\textsc{poss}-\textsc{dat} \textsc{emph} \textsc{cisloc}-\textsc{ipfv}-come-\textsc{pl} be.\textsc{affirm}:\textsc{fact} \textsc{sfp} \\
 \glt `The parents come, come to the teachers (us).' (conversation140501 01, 60)
\end{exe}

\begin{exe}
\ex \label{ex:CpjACe}
 \gll li nɤki iɕqʰa nɯ tɤjlu kɤ-rku ɯ-ŋgɯ zɯ ɕ-pjɤ-ɕe \\
 again \textsc{dem} the.aforementioned \textsc{dem} flour \textsc{nmlz}:P-put.in \textsc{3sg}.\textsc{poss}-inside \textsc{loc} \textsc{transloc}-\textsc{ifr}:\textsc{down}-go \\
 \glt `He went into the bag of flour.' (140519 chou xiaoya-zh, 145)
\end{exe}

The opposite combinations, namely cislocative with \japhug{ɕe}{go} and translocative with \japhug{ɣi}{come}, are not grammatical. 

\subsection{Orientation and AM}
In Japhug, AM markers only specify deixis and the temporal relation between motion event and verbal action, but are neutral as regards the orientation of the motion event.

Orientation and AM markers occupy distinct prefixal slots. Non-orientable verbs (verbs expressing actions other than motion, manipulation, sight or actions with a single direction, see § XXX) select one or two lexicalized orientations (see § XXX). For instance, the verb \japhug{mɯrkɯ}{steal} occurs with the oriental `up' (with the orientation prefixes \forme{tɤ-}, \forme{ta-}, \forme{tu-}, \forme{to-}). 

When non-orientable verbs occurs with AM, the verb normally keeps the lexicalized orientation prefix, as in \ref{ex:CtumWrki}, where \japhug{mɯrkɯ}{steal} is used with the \forme{tu-} `up' prefix; the orientation prefix is thus irrelevant to the motion event. 

\begin{exe}
\ex \label{ex:CtumWrki}
 \gll kɯ-nŋo nɯ qʰe ci ci ɕ-tu-mɯrki kɯ-fse ma nɯ ma mɯ-ɲɯ-ɤʁe. \\
\textsc{nmlz}:S/A-be.defeated \textsc{dem} \textsc{lnk} one one \textsc{transloc}-\textsc{ipfv}-steal[III] \textsc{nmlz}:S/A-be.like apart.from \textsc{dem} apart.from \textsc{neg}-\textsc{sens}-have.to.eat \\
\glt `The (lion) which is defeated steals a little out of it, but apart from that has nothing to eat.' (20-sWNgi, 65)
\end{exe}

In exceptional cases, however, it is possible to use the orientation prefix corresponding to direction of the motion event, as in (\ref{ex:ZlumWrkia}), where the `upstream' prefix occurs with \japhug{mɯrkɯ}{steal}, to express the fact that the character steals from a place located downstream to bring it upstream; however, it is unclear even in this case if the presence of AM is solely responsible for the occurrence of the `upstream' prefix on this verb form, as the same verb with the `upstream' prefix  without AM marker is found in the same story (see \ref{ex:lomWrkW}).\footnote{The first occurrences of the verb \japhug{mɯrkɯ}{steal} with the  `upstream' prefix in the story have the translocative prefix, but that AM marker is elided in the second half of the story.  }

\begin{exe}
\ex \label{ex:ZlumWrkia}
 \gll tɕetʰi tɤmuj jlɤrɯcɤrna ɣɯ ɯ-pʰe nɯtɕu kɯ-mɯrkɯ chɯ-ɕe-a ŋu tɕe. tsʰɤt ɯ-ʁrɯ ɣɯ ɯ-ci nɯnɯ ʑ-lu-mɯrki-a ri a-qʰu zɯ lɤ-ɣe-nɯ tɕe \\
 downstream p.n. p.n. gen \textsc{3sg}.\textsc{poss}-\textsc{dat} \textsc{dem}:\textsc{loc}  \textsc{nmlz}:S/A-steal  \textsc{ipfv}:\textsc{downstream}-go-\textsc{1sg} be:\textsc{fact} \textsc{lnk}  goat \textsc{3sg}.\textsc{poss}-horn \textsc{gen} \textsc{3sg}.\textsc{poss}-water \textsc{dem} \textsc{transloc}-\textsc{ipfv}:\textsc{upstream}-streal[III]-\textsc{1sg} \textsc{lnk} \textsc{1sg}.\textsc{poss}-after \textsc{loc} \textsc{pfv}:\textsc{upstream}-come[II]-\textsc{pl} \textsc{lnk}  \\
 \glt `(Tomorrow  morning) I will go downstream to steal from Tamuj Jlarukyarna, I will steal the water from the goat's horn, but when (the mountain god) comes after me...' (25-kAmYW-XpAltCin, 31-32)
\end{exe}

\begin{exe}
\ex \label{ex:lomWrkW}
 \gll lo-mɯrkɯ pjɤ-cʰa tɕe lo-ɣɯt ri \\
 \textsc{ifr}:\textsc{upstream}-steal \textsc{ifr}-can \textsc{lnk} \textsc{ifr}:\textsc{upstream}-bring \textsc{lnk} \\
\glt `He was able to steal it and brought it upstream.' (02-montagnes-kamnyu-cz, 31)
\end{exe}

\subsection{Echo phenomena} \label{sec:AM.echo}
Previous literature on AM has reported the existence of `echo phenomena' in the use of AM markers (\citealt[251]{wilkins91associated.motion}, \citealt[681-683]{vuillermet12eseejja}, \citealt[128-130]{rose15am}, \citealt[11]{guillaume16am}), namely that the same motion event can be expressed by more than one AM marker. This phenomenon is common in Japhug narratives. Two subtypes of echo can be distinguished.

First, in examples such as (\ref{ex:CtAru}) and (\ref{ex:GWYWsloR}), a motion verb (\japhug{ɕe}{go} and \japhug{ɣi}{come} respectively) is followed by a verb with an AM prefix with the same deixis, though only one motion event took place.

\begin{exe}
\ex \label{ex:CtAru}
\gll tɕʰi ɯ-taʁ to-ɕe tɕe ɕ-tɤ-ru   \\
stairs \textsc{3sg}.\textsc{poss}-on \textsc{ifr}:\textsc{up}-go \textsc{lnk}  \textsc{transloc}-\textsc{up}:\textsc{pfv}-look \\
\glt `He went up the stairs and looked up.'  (08-kWqhi, 18)
\end{exe}

\begin{exe}
\ex \label{ex:GWYWsloR}
\gll kʰa mɯ-pɯ-rɤʑi tɕe tɕe, ftɕar nɯ wuma ʑo βɣɯz pjɤ-rɯŋɯŋɤn tɕe maka,
kɯmtʰoʁ ra kɯnɤ ju-ɣi ɣɯ-ɲɯ-sloʁ pjɤ-ŋu. \\
house \textsc{neg}-\textsc{pst}.\textsc{ipfv}-stay \textsc{lnk} \textsc{lnk} summer \textsc{dem} really \textsc{emph} badger \textsc{ifr}.\textsc{ipfv}-cause.damage \textsc{lnk} completely threshold \textsc{pl} also \textsc{ipfv}-come \textsc{cisloc}-\textsc{ipfv}-dig.up \textsc{ifr}.\textsc{ipfv}-be \\
\glt `He was not home, and that summer badgers were causing a lot of damages, they came and even dug up  the threshold of the house.'  (27-spjaNkW, 107)
\end{exe}

Second, we also find cases such as (\ref{ex:GWtaBzu}) without a motion verb, but with two verbs redundantly prefixed with the same AM marker (here \forme{ɣɯ-}).

\begin{exe}
\ex \label{ex:GWtaBzu}
\gll  tɕe a-kʰa ra ɣɯ-ta-rɤroʁrɯz, 	a-mgo  ra ɣɯ-ta-βzu ŋu ɕi \\
\textsc{lnk} \textsc{1sg}.\textsc{poss}-house \textsc{pl} \textsc{cisloc}-\textsc{pfv}:3$\rightarrow$3'-tidy 
 \textsc{1sg}.\textsc{poss}-food \textsc{pl} \textsc{cisloc}-\textsc{pfv}:3$\rightarrow$3'-make be:\textsc{fact} \textsc{qu} \\ 
\glt `Is it (the neighbour's wife who took pity on me), and came to tidy my house and make food for me?'  (150827 tianluo, 76)
\end{exe}


Echo AM is required in serial verb constructions (\citealt[253-255]{jacques16complementation}, § XXX), as shown in (\ref{ex:CkunWrtCe}), where the verbs \japhug{stu}{do like} and the \japhug{nɯrtɕa}{tease} share the same person (3$\rightarrow$3'), TAM (imperfective) and AM (translocative) markers.

\begin{exe}
\ex \label{ex:CkunWrtCe}
\gll  kɯra ɕ-tu-ste tɕe ɕ-ku-nɯrtɕe ra pjɤ-ŋu. \\
\textsc{dem}:\textsc{prox}:\textsc{pl} \textsc{transloc}-\textsc{ipfv}-do.like[III] \textsc{lnk}  \textsc{transloc}-\textsc{ipfv}-tease[III] \textsc{pl} \textsc{ifr}.\textsc{ipfv}-be \\
\glt `(The mouse) went and teased (the cat) like that.' (150902 dashu, 31)
\end{exe}

\subsection{Associated motion vs motion verb construction}
To express the meaning of motion prior to an action, associated motion prefixes are nearly two times as common as corresponding motion verb constructions (henceforth MVC) in the Japhug corpus. There is however a clear semantic difference between the two constructions, which was briefly described in \citet{jacques13harmonization}, but is presented here in more detail.

AM and MVC differ from each other in that in the former, the completion of both motion event and verbal action is presupposed (AM is monoactional), whereas in the case of the latter, the two can be separated. This mono- vs. pluractionality contrast is most conspicuous in past perfective forms, and can be observed in four types of constructions: concessives (with negation of the verbal action), interrogatives, conditionals and complement clauses. 

Another difference between MVC and AM is the fact that while MVC require a volitional verb in the purposive complete, there is no such requirement for the AM markers.

\subsubsection{Concessive} \label{sec:am.concessive}
A MVC  with the motion verb in perfective form can be followed by a clause negating the purposive action, as in (\ref{ex:nAkWrtoR}). In this example, only the motion is realized, while the action expressed by the verb \japhug{rtoʁ}{look} could not be accomplished.

\begin{exe}
\ex \label{ex:nAkWrtoR}
\gll nɤ-kɯ-rtoʁ jɤ-ɣe-a ri, mɯ-nɯ-atɯɣ-tɕi, mɯ-pɯ-ta-mto. \\
\textsc{1sg.poss}-\textsc{nmlz}:S/A-see \textsc{pfv}-come[II]-\textsc{1sg} \textsc{lnk} \textsc{neg}-\textsc{pfv}-meet-\textsc{1du} \textsc{neg}-\textsc{pfv}-1\fl2-see \\
\glt `I came to see you but I did not see you.' 
\end{exe}

With the corresponding AM verb form \japhug{ɣɯ-jɤ-ta-rtoʁ}{I came to see you}, negating the action of the verb is self-contradictory and nonsensical, and a sentence such as (\ref{ex:GWjAtartoR}) is incorrect.

\begin{exe}
\ex \label{ex:GWjAtartoR}
\gll $\dagger$ɣɯ-jɤ-ta-rtoʁ ri mɯ-pɯ-ta-mto \\
\textsc{cisloc}-\textsc{pfv}-1\fl2-look \textsc{lnk} \textsc{neg}-\textsc{pfv}-1\fl2-see \\
\glt Intended meaning: `I came to see you but I did not see you.' 
\end{exe}

Additional minimal pairs of the same type are presented in \citet[202-203]{jacques13harmonization}.

Example (\ref{ex:mWjsWntsGe}) from a conversation illustrates this property also with a manipulative verb \japhug{ɣɯt}{bring}: the action of the purposive complement  \japhug{kɤ-ntsɣe}{to sell} is negated in the following clause (with an abilitative \forme{sɯ-}, see § XXX).

 \begin{exe}
\ex \label{ex:mWjsWntsGe}
 \gll   sɤnɤmmtsʰu kɯ kɤ-ntsɣe cʰɤ-ɣɯt ri mɯ́j-sɯ-ntsɣe ndɤre, \\
 p.n. \textsc{erg} \textsc{inf}-sell \textsc{ifr}:\textsc{downstream}-bring \textsc{lnk} \textsc{neg}:\textsc{sens}-\textsc{abil}-sell \textsc{lnk} \\
\glt `Bsod.nams.mtsho brought them (to Mbarkham) to sell, but could not sell it.' (conversation, 14.05.10)
 \end{exe}

 

\subsubsection{Interrogative} \label{sec:am.interrogative}
In interrogative clauses, MVCs are required to express meanings such as `What/who have you come/gone to X', as in example (\ref{ex:tChi.WkWpa}), an example which occurs nine times in the corpus.

\begin{exe}
\ex \label{ex:tChi.WkWpa}
\gll tɕʰi ɯ-kɯ-pa jɤ-tɯ-ɣe? \\
what \textsc{3sg.poss}-do \textsc{pfv}-2-come[II] \\
\glt `What did you come to do?' (nine examples in the corpus)
\end{exe}

The difference between MVC and AM in interrogatives can be illustrated by comparing the minimal pair  (\ref{ex:tChi.WkWndza}) and (\ref{ex:tChi.GWtAtWndzat}). Example (\ref{ex:tChi.WkWndza}), which has the same structure as (\ref{ex:tChi.WkWpa}), implies that the addressee has not eaten yet, while (\ref{ex:tChi.GWtAtWndzat}) with associated motion can only be used if the food ingestion has already taken place, and requires a different translation.

\begin{exe}
\ex \label{ex:tChi.WkWndza}
\gll tɕʰi ɯ-kɯ-ndza jɤ-tɯ-ɣe? \\
what \textsc{3sg.poss}-eat \textsc{pfv}-2-come[II] \\
\glt `What have you come to eat?' (elicited)
\end{exe}

\begin{exe}
\ex \label{ex:tChi.GWtAtWndzat}
\gll tɕʰi ɣɯ-tɤ-tɯ-ndza-t \\
what \textsc{cisloc}-\textsc{pfv}-2-eat-\textsc{pst:tr}    \\
\glt `What did you eat upon coming here?' (elicited)
\end{exe}

\subsubsection{Conditional} \label{sec:am.conditional}
The presuppositional difference between MVC and AM is also perceptible in the protasis of conditional clauses. 

With MVC in the protasis as in (\ref{ex:mWmAjAtWGe}), there is no presupposition that the verbal action took place, only the motion event constitutes a condition to the state of affair described in the apodosis.

\begin{exe}
\ex \label{ex:mWmAjAtWGe}
\gll nɤ-wa ɯ-kɯ-rtoʁ mɯ\redp{}mɤ-jɤ-tɯ-ɣe nɤ aʑo mɯ-pɯ-kɯ-mto-a. \\
\textsc{1sg.poss}-father \textsc{3sg.poss-}\textsc{nmlz}:S/A-look \textsc{cond}\redp{}\textsc{neg}-\textsc{pfv}-2-come[II] \textsc{lnk} \textsc{1sg} \textsc{neg}-\textsc{pfv}-2\fl{}1-\textsc{1sg} \\
\glt `If you had not come to see your father, you would not have seen me.' (you saw me, but your father was not here)
\end{exe}

By contrast, with AM, the verbal action necessarily took place, as in example (\ref{ex:mWmAGWjAtWrtoR}).

\begin{exe}
\ex \label{ex:mWmAGWjAtWrtoR}
\gll nɤ-wa  mɯ\redp{}mɤ-ɣɯ-jɤ-tɯ-rtoʁ nɤ pɯ-sɤzdɯxpa \\
\textsc{1sg.poss}-father \textsc{cond}\redp{}\textsc{neg}-\textsc{cisloc}-\textsc{pfv}-2-look \textsc{lnk} \textsc{pst.ipfv}-be.pitiful \\ 
\glt `If you had not come to see your father, he would have felt sorry.' (but you did saw him, so he does not feel sorry)
\end{exe}

\subsubsection{Complement clauses} \label{sec:am.conditional}
In complement clauses, verbs with AM prefixes are attested, and complement taking verbs always have scope over both the action of the verb and motion event.

 
In (\ref{ex:mACWkAtshi}), the modal verb \japhug{cʰa}{can} and the double negations (with the specific meaning `cannot help', § XXX) have scope over both the motion event and the verbal action -- this example is taken from a passage in a story where the king reproaches a small child, who just returned from a mission he himself send him to accomplish, not to have first come to greet him on his return home; the child says these words to justify why he first went to see his mother before greeting the king -- from this context it is clear that both the motion event (to him mother's house, explaining the child's failure to go to see the king) and the action `drink milk' (the reason for that motion event) are equally important to the plot and inseparable. 

\begin{exe}
\ex \label{ex:mACWkAtshi}
\gll  tɯ-nɯ ɯ-kɯ-tsʰi ɲɯ-ɕti-a tɕe, jɤ-azɣɯt-a tɕe, tɯ-nɯ ci mɤ-ɕɯ-kɤ-tsʰi nɯ mɯ́j-cʰa-a \\
\textsc{indef}.\textsc{poss}-breast \textsc{3sg}.\textsc{poss}-\textsc{nmlz}:S/A-drink \textsc{sens}-be.\textsc{affirm}-\textsc{1sg} \textsc{lnk} \textsc{pfv}-arrive-\textsc{1sg} \textsc{lnk} \textsc{indef}.\textsc{poss}-breast  \textsc{indef} \textsc{neg}-\textsc{transloc}-\textsc{inf}-drink \textsc{dem} \textsc{neg}:\textsc{sens}-can-\textsc{1sg} \\
\glt `I am (a toddler) who (still) drinks (his mother's) milk, when I arrived, I could not help but go to drink milk.'  (Norbzang, 262)
 \end{exe}
 
 In (\ref{ex:CWkAmWrkW.mAtWcha}), the negated modal verb has also on the action of both the main verb and the motion event -- the guards would prevent the main character not only to steal, but also to the where the object to be stolen is found. 
 
\begin{exe}
\ex \label{ex:CWkAmWrkW.mAtWcha}
\gll ʁmaʁ χsɯ-tɤkʰar kɯ ɲɯ-ɤz-nɤkʰar-nɯ ɕti tɕe, ɕɯ-kɤ-mɯrkɯ mɤ-tɯ-cʰa  \\
solider three-rounds \textsc{erg} \textsc{sens}-\textsc{prog}-surround-\textsc{pl} be.\textsc{affirm}:\textsc{fact} \textsc{lnk}  \textsc{transloc}-\textsc{inf}-steal \textsc{neg}-2-can:\textsc{fact} \\
\glt `Three rounds of soldiers will be surrounding it, you will not be able to (go there and) steal it.' (2003qachga, 55)
   \end{exe}
   
 Examples (\ref{ex:GWkAcW}) and (\ref{ex:CWkAmtChot}) illustrate the scope of aspectual  auxiliary verbs (here  \japhug{atsu}{have the time to} and \japhug{mda}{be time to}) on both motion event and verbal action.  In (\ref{ex:CWkAmtChot}), note that the infinitive form with AM \japhug{ɕɯ-kɤ-mtɕʰot}{go and make offerings}  translates the Chinese festival \ch{清明节}{qīngmíngjié}{Tomb-Sweeping Day} (using a verb borrowed from Tibetan  \tibet{མཆོད་}{mtɕʰod}{make offerings}). There was no motion verb in the original text.
 
\begin{exe}
\ex \label{ex:GWkAcW}
\gll qʰe potɯrʑi kɯ nɤ-kɯm ɣɯ-kɤ-cɯ mɤ-atsu ma \\
\textsc{lnk} p.n. \textsc{erg} \textsc{2sg}.\textsc{poss}-door \textsc{cisloc}-\textsc{inf}-open \textsc{neg}-have.the.time.to \textsc{lnk} \\
\glt `Bod.rje does not have time to come and open the door for you.' (2010 meimei de gushi, 21)
\end{exe} 
  
\begin{exe}
\ex \label{ex:CWkAmtChot}
\gll tɯrsa ɕɯ-kɤ-mtɕʰot to-mda ɲɯ-ŋu \\
grave \textsc{transloc}-\textsc{inf}-make.offerings \textsc{ifr}-be.time.to \textsc{sens}-be \\
\glt `It was the time to (go and) make offerings for the graves.' (160630 abao-zh, 70)
 \end{exe} 
 
The same scopal effect also applies to  verbs with AM in complement clauses selected by a verb in the protasis, as in (\ref{ex:CWkAru}) and (\ref{ex:CWkAmWrkW}): the realization of the verbal action (in addition to that of the motion event) belongs to the condition.

\begin{exe}
\ex \label{ex:CWkAru}
\gll ɕɯ-kɤ-ru mɯ\redp{}mɤ-pɯ-tɯ-cha ŋu nɤ nɤ-srɤm nɤ-sroʁ lɤt-i \\
\textsc{transloc}-\textsc{inf}-bring \textsc{cond}\redp{}\textsc{neg}-2-can:\textsc{fact} be:\textsc{fact} \textsc{lnk} \textsc{1sg.poss}-root \textsc{1sg.poss}-life throw:\textsc{fact}-\textsc{1pl} \\ 
\glt `If you are not able to bring it here, we will have your root and your life.' (Norbzang, 10)
\end{exe}

\begin{exe}
\ex \label{ex:CWkAmWrkW}
\gll nɤʑo ɕɯ-kɤ-mɯrkɯ a-pɯ-tɯ-cʰa nɤ aʑo cʰɯ-sɯ-jɣat-a jɤɣ \\
\textsc{2sg} \textsc{transloc}-\textsc{inf}-steal \textsc{irr}-\textsc{ipfv}-2-can \textsc{lnk} \textsc{1sg} \textsc{ipfv}-\textsc{caus}-go.back-\textsc{1sg} be.agreed:\textsc{fact} \\
\glt `If you succeed stealing it (after having gone there), I can cause him to go back there.' (02-montagnes-kamnyu, 46)
\end{exe}

 
By contrast, in  (\ref{ex:kWrAma.kACe}), in the case of the infinitival complement \forme{kɯ-rɤma kɤ-ɕe} `go to work' with a purposive clause \forme{kɯ-rɤma} (§ XXX), the main verb \japhug{mda}{be time to} only has scope over the motion event expressed by the verb \japhug{ɕe}{go} -- the time that is indicated by the stars refers to the beginning of the journey to work, not the start of the work itself. Compare this example in particular with (\ref{ex:CWkAmtChot}) above, with the same auxiliary verb.
 
 \begin{exe}
\ex \label{ex:kWrAma.kACe}
\gll  tɕe kɯɕɯŋgɯ tɕe tɯtsʰot pɯ-me tɕe  nɯnɯ cʰɯ-ɬoʁ lu-ɕqʰlɤt nɯra ɕ-tu-kɯ-ru tɕe, nɯnɯ kɤ-rɤru mda mɤ-mda cʰondɤre kɯ-rɤma kɤ-ɕe mda mɤ-mda nɯtɕu ɕ-tu-kɯ-ru pɯ-ŋgrɤl. \\
 \textsc{lnk} long.ago \textsc{lnk} clock \textsc{pst}.\textsc{ipfv}-not.exist \textsc{lnk} \textsc{dem} \textsc{ipfv}:\textsc{downstream}-come.out \textsc{ipfv}:\textsc{upstream}-disappear \textsc{dem}:\textsc{pl} \textsc{transloc}-\textsc{ipfv}:up-\textsc{genr}:S/P-look \textsc{lnk} \textsc{dem} \textsc{inf}-get.up be.time:\textsc{fact} \textsc{neg}-be.time:\textsc{fact} \textsc{comit} \textsc{nmlz}:S/A-work \textsc{inf}-go be.time:\textsc{fact} \textsc{neg}-be.time:\textsc{fact} \textsc{dem}:\textsc{loc} \textsc{transloc}-\textsc{ipfv}:up-\textsc{genr}:S/P-look \textsc{pst}.\textsc{ipfv}-be.usually.the.case  \\
 \glt  `In former times, there was no clock, and people used to watch when (these stars) came out or disappeared (to know) whether it was time to get up or go to work.' (29-LAntshAm, 66)
  \end{exe}
  
\subsubsection{Volitionality and controllability}
An additional difference between AM and MVC has to do with volitionality and/or controllability. In the case of a MVC, the purpose following the motion event is always necessarily volitional and controllable. By contrast, in the case of AM, it is possible to find examples where the verbal actional expresses a non-controllable event, such as the action of finding a lost object in example (\ref{ex:CpjAmto}) with echo phenomenon (cf § \ref{sec:AM.echo}). Note that there are no examples of the non-volitional verb \japhug{mto}{see} with the MVC in the corpus (the volitional \japhug{rtoʁ}{see, look} or \japhug{ru}{look} occur  instead). 

\begin{exe}
\ex  \label{ex:CpjAmto}
\gll  nɯɕɯmɯma ʑo tɯ-ci ɯ-ŋgɯ pjɤ-ɕe qʰe iɕqʰa tɤɕime kɯ ɯ-sɤcɯ pɯ-kɤ-nɯ-ɕlɯɣ nɯ ɕ-pjɤ-mto. \\
immediately \textsc{emph} \textsc{indef}.\textsc{poss}-water \textsc{3sg}.\textsc{poss}-inside \textsc{ifr}:\textsc{down}-go \textsc{lnk} the.aforementioned lady \textsc{erg} \textsc{3sg}.\textsc{poss}-key \textsc{pfv}:\textsc{down}-\textsc{nmlz}:P-\textsc{auto}-drop \textsc{dem} \textsc{transloc}-\textsc{ifr}-see \\
\glt `He went immediately into the water and found there the key that the lady had dropped by mistake.' (140510 fengwang, 118)
\end{exe}

 %Orientation and associated motion
%\chapter{Non-finite verbal morphology}

\section{Participles}
Participles are nominalized verb forms that keep some verbal characteristics: they can serve as predicates of subordinate clauses (relative or complement clauses), take TAM, polarity and associated motion marking, and preserve the verb's argument structure.

Participles differ from finite verbs in three ways. First, they cannot serve as the predicate of a main clause. Second, they are not compatible with the personal prefixes and suffixes of the intransitive and transitive conjugations (including direct/inverse marking and past transitive \forme{-t-}, § XXX).\footnote{Japhug is identical in this regard to Tshobdun and Zbu, but crucially differs from Situ, where nominalized forms in \forme{kə-} can bear indexation suffixes (\citet{jackson06guanxiju}, \citet{jacksonlin07}) } Rather, like nouns, they can take a possessive prefix which can be coreferent with one the arguments. Due to the general impossibility of stacking possessive prefixes (§ \ref{sec:possessive.paradigm}), at most only one argument can be indexed this way. Third, there are restrictions on TAM marking on participles: they have at most three forms (neutral, perfective and imperfective), and completely lack inferential (§ XXX), egophoric (§ XXX), sensory (§ XXX) and progressive forms (§ XXX).

There are three participles in Japhug; the subject S/A participle in \forme{kɯ-}, the object participle in \forme{kɤ-} and the oblique participle in \forme{sɤ-}. 

Complex participial forms, including negative, associated motion or TAM prefixes are possible, as shown by example \ref{ex:WGWjAkWqru}. However, never more than four inflexional prefixes are found; forms with all five prefixal slots filled (such as $\dagger$\forme{ɯ-ɣɯ-jɤ-kɯ-qru}) are not accepted by Tshendzin.

 \begin{exe}
\ex \label{ex:WGWjAkWqru}
\gll ɯ-ɣɯ-jɤ-kɯ-qru tɤ-tɕɯ  \\
  \textsc{3sg}-\textsc{cisloc}-\textsc{pfv}-\textsc{nmlz}:S/A-meet \textsc{indef}.\textsc{poss}-boy   \\
\glt `The boy who had come to look for her.' (The three sisters, 231)
 \end{exe}

Table \ref{tab:template.nmlz} summarizes the template of participial verb forms; more details are provided on possible and attested forms for each participle type in the following sections.

\begin{table}[h]
\caption{The template of participial verb forms in Japhug} \centering \label{tab:template.nmlz}
\resizebox{\columnwidth}{!}{
\begin{tabular}{lllllll}
\toprule
-5 & -4&-3 &-2&-1& \ro{} \\
possessive & negative&associated   & TAM & participle prefix &enlarged  \\
prefix & prefix &motion prefix  &orientation&&stem\\
\bottomrule
\end{tabular}}
\end{table}

Stem alternation is reduced in participle forms: stem 3 (§ XXX) never occurs. The few verbs that have a distinct stem 2 (\japhug{ɕe}{go}, \japhug{ɣi}{come}, \japhug{ti}{say} and derived forms) however, use this stem in subject and object participles with perfective orientational prefixes (§ XXX), in forms like \forme{jɤ-kɯ-ɣe} \textsc{pfv}-\textsc{nmlz}:S/A-come[II] `the one who came'
or \forme{tɤ-kɤ-tɯt} \textsc{pfv}-\textsc{nmlz}:P-say[II] `what was said'.
 

\subsection{Subject participles} \label{sec:subject.participles}
The subject participle, built by adding the prefix \forme{kɯ-} to the verb stem, designates an entity corresponding to the intransitive subject (\ref{ex:kWsi}, § \ref{sec:absolutive.S} and XXX), a possessor of the subject, or the transitive subject (\ref{ex:WkWndza}, § \ref{sec:A.kW}, § XXX) of the base verb. 

 \begin{exe}
\ex \label{ex:kWsi}
\gll kɯ-si  \\
  \textsc{nmlz}:S/A-die \\
 \glt  `The dead one' (many attestations)
\end{exe}

 \begin{exe} 
\ex \label{ex:WkWndza}
\gll ɯ-kɯ-ndza \\
  \textsc{3sg}-\textsc{nmlz}:S/A-eat \\
 \glt  `The one who eats it.' (many attestations)
\end{exe}

The subject participle \forme{kɯ-} prefix is historically related to that of object participles (§ \ref{sec:object.participle}), velar infinitives (§ \ref{sec:velar.inf}) and deverbal nouns in \forme{x-/ɣ-} (§ \ref{sec:G.nmlz}), and has cognates elsewhere in the family (§ \ref{sec:velar.nmlz.history}).

In this section, I discuss first morphological issues (possessive prefixes § \ref{sec:subject.participle.ambiguities}, other prefixes § \ref{sec:subject.participle.other.prefixes} and ambiguous forms § \ref{sec:subject.participle.ambiguities}), and then present the various functions of subject participles, including participial relatives (§ \ref{sec:subject.participle.subject.relative} and § \ref{sec:subject.participle.other.relative}), complementation strategies (\ref{sec:subject.participle.complementation}), adverbials (§ \ref{sec:subject.participle.adverbial}), as well as the case of lexicalized participles (§ \ref{sec:lexicalized.subject.participle}).
 
 

\subsubsection{Possessive prefixes on subject participles}  \label{sec:subject.participle.possessive}

In the case of transitive verbs, a possessive prefix coreferent with the object is obligatory when no overt object is present (\textsc{3sg} \forme{ɯ-} in \ref{ex:WkWndza}), and when no other prefix is added to the participle.

When a polarity or orientation prefix is present, the possessive prefix is optional, as shown by forms like \forme{mɤ-kɯ-ndza} `the one which does not eat (it)' in (\ref{ex:mAkWndza}), as opposed to \forme{ɯ-mɤ-kɯ-mto} `the one who does not see it' in (\ref{ex:WmAkWmto}) with both possessive \forme{ɯ-} and the negative prefix \forme{mɤ-}.

 \begin{exe} 
\ex \label{ex:mAkWndza}
\gll  tɤ-mtʰɯm ʁɟa ʑo ma nɯ ma, nɤki, tɯjpu mɤ-kɯ-ndza ci tu tɕe, \\
\textsc{indef}.\textsc{poss}-meat completely \textsc{emph} \textsc{lnk} \textsc{dem} apart.from \textsc{filler} flour.based.food \textsc{neg}-\textsc{nmlz}:S/A-eat \textsc{indef} exist:\textsc{fact} \textsc{lnk} \\
\glt  `There is (an animal like the mouse) which only eats meat, not food made from flour.' (27-spjaNkW, 202-2063)
\end{exe}

 \begin{exe} 
\ex \label{ex:WmAkWmto}
\gll  li nɯnɯ kɯnɤ ɯ-kɯ-mto ɣɤʑu, ɯ-mɤ-kɯ-mto ɣɤʑu. \\
again \textsc{dem} also \textsc{3sg}.\textsc{poss}-\textsc{nmlz}:S/A-see exist:\textsc{sens} \textsc{3sg}.\textsc{poss}-\textsc{neg}-\textsc{nmlz}:S/A-see exist:\textsc{sens} \\
\glt `There are (people) who see (find) it, and people who don't.' (20-sWrna, 20)
\end{exe}

In the case of ditransitive verbs, the possessive prefix striclty refer to the object. With indirective verbs like \japhug{tʰu}{ask}, the possessive prefix is necessarily the theme, never the recipient. The form in (\ref{ex:AkWthu}) thus cannot be interpreted as meaning `the one who asks me (about it)'; the correct construction would be (\ref{ex:ACki.kWthu}), with the recipient in the dative case.

\begin{exe}
\ex \label{ex:AkWthu}
\gll a-kɯ-tʰu  \\
\textsc{1sg.poss}-\textsc{nmlz}:S/A-ask \\
\glt `The one asking for me (in marriage).' (elicited)
\ex \label{ex:ACki.kWthu}
\gll a-ɕki ɯ-kɯ-tʰu  \\
\textsc{1sg.poss}-\textsc{dat} \textsc{3sg}.\textsc{poss}-\textsc{nmlz}:S/A-ask \\ 
\glt `The one who asks me.' 
\end{exe}

With secundative verbs (§ XXX), the possessive prefix of the subject participle is obligatorily coreferent with the recipient, not the theme, as in (\ref{ex:nAkWmbi}).

\begin{exe}
\ex \label{ex:nAkWmbi}
\gll nɯ ma nɤ-kɯ-mbi me \\
\textsc{dem} apart.from \textsc{2sg}.\textsc{poss}-\textsc{nmlz}:S/A-give not.exist:\textsc{fact} \\
\glt `Nobody will give you another (daughter in marriage).' (2002qaCpa, 57)
\end{exe}
With intransitive verbs, including adjectival stative verbs (§ XXX), a possessive prefix can also be added. In the case of semi-transitive verbs (§ XXX), the possessive can refer to the semi-object (§ \ref{sec:semi.object}), as in example (\ref{ex:WkWrga.pWdAn}).

 \begin{exe} 
\ex \label{ex:WkWrga.pWdAn}
\gll  nɯ ɕɯŋgɯ tɕe, ɯ-kɯ-rga pɯ-dɤn. \\
\textsc{dem} before \textsc{lnk} \textsc{3sg}.\textsc{poss}-\textsc{nmlz}:S/A-like \textsc{pst}.\textsc{ipfv}-be.many \\
\glt  `Before, there used to be many people who liked it.' (12-Zmbroko, 112)
\end{exe}

It can also refer to the beneficiary (which is normally marked with genitive or possessive prefixes, see § \ref{sec:other.uses.poss.prefixes} and § \ref{sec:gen.beneficiary}), as in (\ref{ex:tWZo.tWkWpe}) and (\ref{ex:aZo.akWra}).

 \begin{exe} 
\ex \label{ex:tWZo.tWkWpe}
\gll  kɯ-pe tú-wɣ-nɤma tɕe li tɯʑo tɯ-kɯ-pe tu \\
\textsc{nmlz}:S/A-be.good \textsc{ipfv}-\textsc{inv}-make \textsc{lnk} again \textsc{genr} \textsc{genr}.\textsc{poss}-\textsc{nmlz}:S/A-be.good exist:\textsc{fact} \\
\glt  `If one does good things, one will also have good things.' (140518 mao he laoshu-zh, 124)
\end{exe}

 \begin{exe} 
\ex \label{ex:aZo.akWra}
\gll  aʑo a-kɯ-ra nɯra a-tɤ-tɯ-ste qʰendɤre aʑo nɯnɯ, nɤki, ku-nɤtsi-a jɤɣ \\
\textsc{1sg} \textsc{1sg}.\textsc{poss}-\textsc{nmlz}:S/A-have.to \textsc{dem}:\textsc{pl} \textsc{irr}-\textsc{pfv}-2-do.like[III] \textsc{lnk} \textsc{dem} \textsc{filler} \textsc{ipfv}-hide[III]-\textsc{1sg} be.possible:\textsc{fact}  \\
\glt  `If you do the things I need, I will keep it secret.'  (2014-kWlAG, 247)
\end{exe}

Since participles are also noun-like, the possessive prefixes can be real possessive, and be preceded with a genitive phrase as in (\ref{ex:tCaXpa.ra.GW.nWkWmna}) with \forme{nɯ-kɯ-mna} `the best among them' = `their chief'.

 \begin{exe} 
\ex \label{ex:tCaXpa.ra.GW.nWkWmna}
\gll tɕaχpa ra ɣɯ nɯ-kɯ-mna nɯ wuma ʑo pjɤ-nɯrɤŋom. \\
bandit \textsc{pl} \textsc{gen} \textsc{3pl}.\textsc{poss}-\textsc{nmlz}:S/A-be.better \textsc{dem} really \textsc{emph} \textsc{ifr}-be.upset \\
\glt `The chief of the bandits was very upset.' (140512 alibaba-zh, 195)
\end{exe}

This construction is used as a type of superlative, as in (\ref{ex:thamtCAt.GW.nWkWmpCAr}), where \forme{pɣa tʰamtɕɤt ɣɯ nɯ-kɯ-mpɕɤr nɯ}, literally meaning `the beautiful one (among/of) all birds' is to be understood as `the most beautiful of all birds.' (see § XXX on superlative constructions).

 \begin{exe} 
\ex \label{ex:thamtCAt.GW.nWkWmpCAr}
\gll tɕe pɣa tʰamtɕɤt ɣɯ nɯ-kɯ-mpɕɤr nɯ rmɤβja ɲɯ-ŋu.  \\
\textsc{lnk} bird all \textsc{gen} \textsc{3pl}.\textsc{poss}-\textsc{nmlz}:S/A-be.beautiful \textsc{dem} peacock \textsc{sens}-be \\
\glt `The peacock is the most beautiful of all birds.' (24-ZmbrWpGa, 84)
\end{exe}

\subsubsection{Associated motion, polarity and orientation prefixes on subject participles}  \label{sec:subject.participle.other.prefixes}
Of all non-finite verb forms, subject participles allow the richest possible combinations of inflexional prefixes: associated motion (§ \ref{sec:associated.motion}, example \ref{ex:WCWkWphWt}) below with the translocative \forme{ɕɯ-}), polarity (§ XXX, see \ref{ex:WmAkWmto} above) and orientation prefixes marking TAM (§ XXX) all can be prefixed. 
 
\begin{exe}
\ex \label{ex:WCWkWphWt}
 \gll tɕeri nɯra ɯ-ɕɯ-kɯ-pʰɯt ra kɯ-tu me ma,   \\
 \textsc{lnk} \textsc{dem}:\textsc{pl} \textsc{3sg}.\textsc{poss}-\textsc{transloc}-\textsc{nmlz}:S/A-cut \textsc{pl} \textsc{nmlz}:S/A-exist not.exist:\textsc{fact} \textsc{lnk} \\
 \glt `But nobody goes to collect (its stalks).' (11-paRzwamWntoR, 90)
\end{exe}

Most examples in the corpus have one or two prefixes, either combining a possessive prefix with another prefix (as in \ref{ex:WmAkWmto} and \ref{ex:WCWkWphWt}), or combining a negative prefix with an orientation prefix, as in (\ref{ex:mWnWkWsna}).

 \begin{exe}
\ex \label{ex:mWnWkWsna}
 \gll tɕe kʰa ɣɯ ɯ-ndzɤtsʰi ɯ-ro nɯ-kɯ-ri nɯra, mɯ-nɯ-kɯ-sna nɯra, nɯra paʁ kɯ ʁɟa tu-ndze ɲɯ-ŋu \\
 \textsc{lnk} house \textsc{gen} \textsc{3sg}.\textsc{poss}-food \textsc{3sg}.\textsc{poss}-excess \textsc{pfv}-\textsc{nmlz}:S/A-left \textsc{dem}:\textsc{pl}  \textsc{neg}-\textsc{pfv}-\textsc{nmlz}:S/A-be.good \textsc{dem}:\textsc{pl} \textsc{dem}:\textsc{pl} pig \textsc{erg} completely  \textsc{ipfv}-eat[III] \textsc{sens}-be \\
 \glt  `Food from the house that has been left over, or which is not good any more, pigs eat all of it.' (05-paR, 33)
\end{exe}

Subject participles with three prefixes before the participle prefix \forme{kɯ-} are possible, but attestations are extremely rare. Example (\ref{ex:WGWjAkWqru}) above shows the combination of a possessive, an associated motion and an orientation prefixes (\forme{ɯ-ɣɯ-jɤ-kɯ-qru} `the one who had come to meet/look for her'), and (\ref{ex:WmApjWnWfkAB}) below that of a possessive, a polarity and an orientation prefixes.

\begin{exe}
\ex \label{ex:WmApjWnWfkAB}
 \gll ɯ-pjɯ-kɯ-nɯ-fkaβ tu, ɯ-mɤ-pjɯ-kɯ-nɯ-fkaβ tu ri nɯ kɯ-fse tu-nɯ-ndza-nɯ ɕti. \\
 \textsc{3sg}.\textsc{poss}-\textsc{ipfv}-\textsc{nmlz}:S/A-\textsc{auto}-cover exist:\textsc{fact}  \textsc{3sg}.\textsc{poss}-\textsc{neg}-\textsc{ipfv}-\textsc{nmlz}:S/A-\textsc{auto}-cover exist:\textsc{fact} \textsc{lnk} \textsc{dem} \textsc{nmlz}:S/A-be.like \textsc{ipfv}-\textsc{auto}-eat-\textsc{pl} be.\textsc{affirm}:\textsc{fact} \\
 \glt `There are people who cover it (with a lid while cooking), and people who don't, they eat it like that.' (23-mbrAZim, 22-23)
\end{exe}

There are no constraints on the number of derivational prefixes in participial forms. The derivational prefixes are all closer to the verb root than the participle prefix \forme{kɯ-}, and thus follow it as shown by (\ref{ex:WmApjWnWfkAB}), where the autobenefactive \forme{-nɯ-}, the leftmost of all derivational prefixes (§ XXX), is placed after \forme{kɯ-}. 

Two of the four series of orientation prefixes are possible with subject participles. With series A prefixes (\forme{tɤ-} `up', \forme{pɯ-} `down' etc, § XXX), the participle of dynamic verbs is perfective as \forme{tʰɯ-kɯ-ɣe} `the one who came' in (\ref{ex:WkWntsGe.thWkWGe}), and take stem II (§ XXX). With series B prefixes (\forme{tu-} `up', \forme{pjɯ-} `down' etc, § XXX), it has a habitual imperfective meaning with dynamic verbs as \forme{ju-kɯ-ɣi} `the one who (usually) comes' in (\ref{ex:WkWndza.jukWGi}).\footnote{These two examples also illustrate the use of subject participles as purposive complements with the forms \forme{ɯ-kɯ-ntsɣe} and \forme{ɯ-kɯ-ndza} (see § \ref{sec:subject.participle.complementation}, § XXX).} The prefixes \forme{ɲɯ-}and \forme{ku-} do appear on subject participles, but only to express imperfective: there are no egophoric (§ XXX) or sensory (§ XXX) subject partiples.

\begin{exe}
\ex \label{ex:WkWntsGe.thWkWGe}
\gll iɕqʰa qaʑo ɯ-kɯ-ntsɣe tʰɯ-kɯ-ɣe nɯ ɯ-pʰe \\
the.aforementioned sheep \textsc{3sg}.\textsc{poss}-\textsc{nmlz}:S/A-sell \textsc{pfv}:\textsc{downstream}-\textsc{nmlz}:S/A-come[II] \textsc{dem} \textsc{3sg}.\textsc{poss}-\textsc{dat} \\
\glt  `(He told) the person who had come to sell the sheep.' (2003kandZislama, 212)
\end{exe}

\begin{exe}
\ex \label{ex:WkWndza.jukWGi}
\gll ɯ-kɯ-ndza ju-kɯ-ɣi nɯ pɣa ci ɲɯ-ŋu \\
\textsc{3sg}.\textsc{poss}-\textsc{nmlz}:S/A-eat \textsc{ipfv}-\textsc{nmlz}:S/A-come dem bird indef sens-be \\
\glt   `The one who comes to eat (the fruits) is a bird.' (2012qachGa, 22)
\end{exe}

The participles of stative verbs with series A and B orientation prefixes have an inchoative meaning, exactly like their finite counterpart (§ XXX and § XXX).  In (\ref{ex:YWkWjpum}) for instance, the imperfective participle \forme{ɲɯ-kɯ-jpum} from \japhug{jpum}{be thick} means `the one which becomes thicker', as opposed to the basic participle \forme{kɯ-jpum} `the thick one'.

\begin{exe}
\ex \label{ex:YWkWjpum}
 \gll ndʑu ɯ-ku jamar ɲɯ-kɯ-jpum ɣɤʑu nɤ, kɯ-wxti.  \\
 chopsticks \textsc{3sg}.\textsc{poss}-head about \textsc{ipfv}-\textsc{nmlz}:S/A-be.thick exist:\textsc{sens} \textsc{sfp} \textsc{nmlz}:S/A-be.big \\
 \glt  `There are (maggots) that grow as thick as the tip of a chopstick, the big ones.' (25-akWzgumba, 80)
\end{exe}

Imperfective participles of stative adjectival verbs are also appropriate to describe the gradient variation of a property across space rather than time. For instance, in (\ref{ex:YWkWjpum2}), the imperfective subject participles \forme{ku-kɯ-xtsʰɯm} and \forme{ɲɯ-kɯ-jpum} are used not to indicate a change across time, but to describe the shape of the gourd, which is progressively thinner towards the top and thicker towards the bottom.

\begin{exe}
\ex \label{ex:YWkWjpum2}
 \gll  tɕe ɯ-mat nɯnɯ, ɯ-taʁ ku-kɯ-xtsʰɯm, ɯ-pa ɲɯ-kɯ-jpum ci cʰɯ-βze ɲɯ-ŋu tɕe, nɯ <hulu> tu-sɤrmi-nɯ. \\
 \textsc{lnk} \textsc{3sg}.\textsc{poss}-fruit \textsc{dem} \textsc{3sg}.\textsc{poss}-up \textsc{ipfv}-\textsc{nmlz}:S/A-be.thin \textsc{3sg}.\textsc{poss}-down \textsc{ipfv}-\textsc{nmlz}:S/A-be.thick \textsc{indef} \textsc{ipfv}-make[III]  \textsc{sens}-be \textsc{lnk} gourd \textsc{ipfv}-call-\textsc{pl} \\
 \glt `It grows a fruit that is thinner (in diameter) on the upper part, and thicker on the lower part, people call it `gourd'.' (150825 huluwa, 3)
\end{exe}

The past imperfective of stative verbs is built using the series A prefix \forme{pɯ-} as in the corresponding finite forms (§ XXX). For instance, the past imperfective participle of \japhug{ŋu}{be} is \forme{pɯ-kɯ-ŋu} `the one who used to be ....', as in (\ref{ex:pWkWNu}).

\begin{exe}
\ex \label{ex:pWkWNu}
 \gll  ɯʑɤɣ nɯ ɕɯŋgɯ ɯ-nmaʁ pɯ-kɯ-ŋu tsʰɯraŋ nɯ pjɤ-mto \\
 \textsc{3sg}:\textsc{gen} \textsc{dem} before \textsc{3sg}.\textsc{poss}-husband \textsc{pst}.\textsc{ipfv}-\textsc{nmlz}:S/A-be p.n. \textsc{dem} \textsc{ifr}-see \\
\glt `She saw Tshering, who used to be her husband.' (qajdoskAt2002, 101)
\end{exe}

Subject participles can undergo totalitative reduplication (§ XXX), which applies to the first syllable of the word, whether it is the participle \forme{kɯ-} or an orientation prefix as in (\ref{ex:jWjAkWGe}), meaning `all of those who/that X'.

\begin{exe}
\ex \label{ex:jWjAkWGe}
\gll tɕe nɯnɯ ɯ-taʁ jɯ\redp{}jɤ-kɯ-ɣe nɯ ku-ndɤm ɲɯ-ŋu. \\
\textsc{lnk} \textsc{dem} \textsc{3sg}.\textsc{poss}-on \textsc{total}\redp{}\textsc{pfv}-\textsc{nmlz}:S/A-come[II] \textsc{ipfv}-take[III] \textsc{sens}-be \\
\glt `(The spider) catches all of the (insects) that have come on (the web).' (26-mYaRmtsaR, 108)
\end{exe}

The totalitative subject participle of the existential verb \japhug{tu}{exist} can take a possessive prefix, which is interpreted as a possessor, as in \forme{a-kɯ\redp{}kɯ-tu} \textsc{1sg}.\textsc{poss}-\textsc{total}\redp{}\textsc{nmlz}:S/A-exist `everything that I have'. No other totalitative verb form allows possessor prefixation.

\subsubsection{Ambiguities}  \label{sec:subject.participle.ambiguities}
The subject participle \forme{kɯ-} prefix is homophonous with the generic person marker for intransitive subject and object (§ XXX; note that these two prefixes are probably historically related). In the case of intransitive verbs, some subject participles are therefore homophonous with generic person forms. 

For instance, the past imperfective generic \forme{pɯ-kɯ-ŋu} `one used to be' in (\ref{ex:pWkWNu.genr}) is identical to the past imperfective participle \forme{pɯ-kɯ-ŋu} `the one who used to be ....', discussed above (example \ref{ex:pWkWNu} in § \ref{sec:subject.participle.other.prefixes}). In this example, it is obvious that \forme{kɯ-} is the generic person marker because the verb \forme{pɯ-kɯ-rga} `one used to be' occurs as main verb; outside of any context,  \forme{tɤ-pɤtso pɯ-kɯ-ŋu} could be understood as a relative clause `the one who used to be a child', but this is not the meaning of this sentence. 

\begin{exe}
\ex \label{ex:pWkWNu.genr}
 \gll tɕeri tɤ-pɤtso pɯ-kɯ-ŋu tɕe, nɯ kɤ-ndza wuma ʑo pɯ-kɯ-rga. \\
 \textsc{lnk} \textsc{indef}.\textsc{poss}-child \textsc{pst}.\textsc{ipfv}-\textsc{genr}:S/P-be \textsc{lnk} \textsc{dem} \textsc{inf}-eat really \textsc{emph} \textsc{pst}.\textsc{ipfv}-\textsc{genr}:S/P-like \\
 \glt `When (we) were children, (we) used to like eating it.' (12-ndZiNgri, 137-138)
\end{exe}

More generally, the factual, imperfective, past imperfective and perfective forms of intransitive verbs in generic person forms are homophonous with unmarked, imperfective, past imperfective and perfective participles respectively. In the case of transitive verbs, the subject participle can be identical to the object generic form. For instance, the participle \forme{nɯ-tu-kɯ-ndza} `the one who eats them' in (\ref{ex:tukWndza.nmlz}) only differs from the generic \forme{tu-kɯ-ndza} `it eats us/people' in \ref{ex:tukWndza.genr}) by the possessive prefix \forme{nɯ-}, and that prefix being optional, there are forms that are really ambiguous between participle and generic. 

\begin{exe}
\ex \label{ex:tukWndza.nmlz}
 \gll nɯ ɯ-rkɯ jɤ-azɣɯt-nɯ tɕe, ʑara nɯ-tu-kɯ-ndza srɯnmɯ ci pjɤ-tu, \\
 \textsc{dem} \textsc{3sg}.\textsc{poss}-side \textsc{pfv}-reach-\textsc{pl} \textsc{lnk} \textsc{3pl} \textsc{3pl}.\textsc{poss}-\textsc{ipfv}-\textsc{nmlz}:S/A-eat râkshasî \textsc{indef} \textsc{ifr}.\textsc{ipfv}-be \\
\glt `There was a râkshasî who ate those who had come near her.' (Kunbzang2012, 255)
\end{exe}

\begin{exe}
\ex \label{ex:tukWndza.genr}
 \gll tɕe ndzɤpri kɤ-ti nɯ tɕe tɯrme tu-kɯ-ndza ɲɯ-ŋgrɤl  \\
 \textsc{lnk} brown.bear \textsc{nmlz}:P-say \textsc{dem} \textsc{lnk} people \textsc{ipfv}-\textsc{genr}:S/P \textsc{sens}-be.usually.the.case \\
\glt `The brown bear, it eats people.' (21-pri, 94)
\end{exe}
 
The irregular generic \forme{tu-kɯ-ti} `one says' of the verb \japhug{ti}{say} is also identical with the participle `the one who says'.

The \textsc{2sg}\fl{}\textsc{1sg} form of transitive verbs in \forme{-a}, due to the vowel fusion rule \ipa{-a-a} \fl{} \forme{-a}, are also superficially identical to subject participles, for instance \forme{tu-kɯ-ndza-a} `you eat me' is pronounced \phonet{tukɯndza} exactly like the generic and the participle \forme{tu-kɯ-ndza} in the Kamnyu dialect (in the dialects of Japhug where this vowel fusion does not occur, the forms remain distinct).

\begin{exe}
\ex \label{ex:tukWndzaa}
 \gll nɯ kóʁmɯz nɤ tu-kɯ-ndza-a \\
 \textsc{dem} only.after \textsc{lnk} \textsc{ipfv}-2\fl{}1-eat-\textsc{1sg} \\
 \glt `Eat me only after (having taken out the thorn on my foot).' (140426 lang yisheng-zh, 16)
\end{exe} 

\subsubsection{Subject relative clauses}  \label{sec:subject.participle.subject.relative}
The most common use of subject participles is to build participial relative clauses whose head noun is the subject; it is the only way to relativize the subject in Japhug (§ XXX). Headless relatives are most common (§ XXX), but when the head noun is overt, the relative can be either prenominal, postnominal or head-internal. With intransitive verbs the difference between postnominal or head-internal relatives is often difficult to ascertain, and many examples are ambiguous; for instance in (\ref{ex:tCheme.RnWz.kWrWCmi}), the relative clause could be argued to be postnominal (limited to the participle \forme{kɯ-rɯɕmi} `speaking') or head-internal (including \forme{tɕʰeme ʁnɯz} `two girls', and possibly even the previous adjunct).

\begin{exe}
\ex \label{ex:tCheme.RnWz.kWrWCmi}
 \gll  kʰa ɯ-ŋgɯ nɯtɕu tɕʰeme ʁnɯz kɯ-rɯɕmi pjɤ-tu. \\
 house \textsc{3sg}.\textsc{poss}-inside \textsc{dem}:\textsc{loc} girl two \textsc{nmlz}:S/A-speak \textsc{ifr}.\textsc{ipfv}-exist \\
\glt  `There were two girls speaking in the house.' (150909 xiaocui-zh, 157)
\end{exe}

With transitive verbs, subject head-internal relatives can be distinguished from postnominal ones  by the presence of the ergative \forme{kɯ} on the head noun (§ XXX), as in (\ref{ex:WrdWrdoR.kW.thotsi.WkWta}).

\begin{exe}
\ex \label{ex:WrdWrdoR.kW.thotsi.WkWta}
 \gll [tsuku ɯ-rdɯ\redp{}rdoʁ kɯ ʑo tʰotsi ɯ-kɯ-ta] ɣɤʑu. \\
 some \textsc{3sg}.\textsc{poss}-piece \textsc{erg} \textsc{emph} seal \textsc{3sg}.\textsc{poss}-\textsc{nmlz}:S/A-put exist:\textsc{sens} \\
 \glt `There are people who put a seal (on their bread).'
\end{exe}

%pɣɤtɕɯ nɯ kɯnɤ tɯ-ɣjɤn cinɤ ʑo tɤ-kɯ-mbri kɯ-me,
%nɯ to-ɣɤscɤscɤt ʑo to-mbri ɲɯ-ŋu,

Prenominal relatives are relatively rare with intransitive verbs, but commonly occur with transitive verbs, as in (\ref{ex:tWnW.WkWtshi}). Note the presence of indefinite person possessive marking on the head noun \japhug{tɤ-pɤtso}{child} in this example; unlike in Situ (\citealt{jacksonlin07}), the head noun of prenominal relatives in Japhug does not take a third person singular prefix as in a possessive construction (in which case the form $\dagger$\forme{ɯ-pɤtso} would have been found).

\begin{exe}
\ex \label{ex:tWnW.WkWtshi}
 \gll  [tɯ-nɯ ɯ-kɯ-tsʰi] tɤ-pɤtso ɣɯ ɯ-kɯ-mŋɤm ɲɯ-ŋu tɕe, \\
 \textsc{indef}.\textsc{poss}-breast \textsc{3sg}.\textsc{poss}-\textsc{nmlz}:S/A-drink \textsc{indef}.\textsc{poss}-child \textsc{gen} \textsc{3sg}.\textsc{poss}-\textsc{nmlz}:S/A-hurt \textsc{sens}-be \textsc{lnk} \\
 \glt `It is a disease of infants (who still drink mother milk).' (25-kACAl, 61)
\end{exe}

There are nevertheless prenominal subject relative clauses marked with the genitive \forme{ɣɯ}, especially common in texts translated from Chinese (due to calquing with \zh{的} <de>-relatives, § XXX), but also attested in natural speech, as in (\ref{ex:kWsAndza.GW}).

\begin{exe}
\ex \label{ex:kWsAndza.GW}
 \gll nɯnɯ kɯ-sɤ-ndza ɣɯ rɯdaʁ nɯnɯ tɕe kɯrŋi tu-kɯ-ti ŋu.  \\
 \textsc{dem} \textsc{nmlz}:S/A-\textsc{apass}-eat \textsc{gen} animal \textsc{dem} \textsc{lnk} beast \textsc{ipfv}-\textsc{genr}-say be:\textsc{fact} \\
 \glt `Animals eating (other animals) are called `beasts'.' (150822 kWrNi, 6)
\end{exe}

\subsubsection{Other relative clauses}  \label{sec:subject.participle.other.relative}
In addition to subject relativization, the subject participle is also used in possessor relatives, when the relativized element is the possessor of the subject (§ XXX).  The head-internal clause in (\ref{ex:kWkWtu.head.internal}) is such a possessor relative; its head noun \japhug{si}{tree}, possessor of the subject \japhug{ɯ-mat}{its fruits},  is marked with the genitive, showing that it belongs to the relative.  

\begin{exe}
\ex \label{ex:kWkWtu.head.internal}
 \gll si ɣɯ ɯ-mat kɯ\redp{}kɯ-tu nɯ ɯ-ku ri ɕ-ku-zo ɲɯ-ŋu tɕe. \\
 tree \textsc{gen} \textsc{3sg}.\textsc{poss}-fruit \textsc{total}\redp{}\textsc{nmlz}:S/A-exist \textsc{dem} \textsc{3sg}.\textsc{poss}-top \textsc{loc} \textsc{transloc}-\textsc{ipfv}-land \textsc{sens}-be \textsc{lnk} \\
 \glt `It lands on the top of all trees that have fruits.' (24-ZmbrWpGa, 43)
\end{exe}

 Headless possessor relative clauses, such as  \forme{nɯ-mtɕʰi mɤ-kɯ-pe} `those with a foul mouth' in (\ref{ex:nWmtChi.mAkWpe}), are even more common.

\begin{exe}
\ex \label{ex:nWmtChi.mAkWpe}
 \gll nɯ-mtɕʰi mɤ-kɯ-pe, kɤ-nɯtsɯ kɯ-ra ra kɯnɤ tu-kɯ-nɯ-ti nɯnɯra tɕaɣi tu-sɤrmi-nɯ ŋgrɤl. \\
 \textsc{3pl}.\textsc{poss}-mouth \textsc{neg}-\textsc{nmlz}:S/A-be.good \textsc{inf}-hide \textsc{nmlz}:S/A-have.to \textsc{pl} also \textsc{ipfv}-nmlz:S/A-\textsc{auto}-say \textsc{dem}:\textsc{pl} parrot \textsc{ipfv}-call-\textsc{pl} be.usually.the.case:\textsc{fact} \\
 \glt `Those with a foul mouth, who say things that should be hidden, are called `parrots'. (24-qro, 130)
\end{exe}


\subsubsection{Complementation strategies}  \label{sec:subject.participle.complementation}
Subject participles are also required in several types of complement clauses and complementation strategies (§ XXX on the difference between the two types).

The most common complementation construction involving subject participles occurs with the motion verbs \japhug{ɕe}{go} and \japhug{ɣi}{come}. These verbs have purposive clauses which compulsorily take a verb in subject participle form (§ XXX).  When the verb in the purposive clause is transitive, the participle has a possessive prefix coreferent with the object as in the case of relative clauses (\ref{sec:subject.participle.possessive}), and the subject can either take absolutive marking following the motion verb (which is morphologically intransitive), as in (\ref{ex:WkWCar.chACe}), or ergative  marking following the verb of the purposive clause as in (\ref{ex:WkWCar.loCenW}). This case marking difference can be analysed as reflecting distinct clausal structures: in (\ref{ex:WkWCar.loCenW}), the subject \forme{tɤ-rɟit ra} `the children' belongs to the purposive clauses, whereas in (\ref{ex:WkWCar.chACe}), the subject \forme{tɤ-mu nɯ} lies outside of it.

\begin{exe}
\ex \label{ex:WkWCar.chACe}
 \gll lo-fsoʁ tɕe tɕe tɤ-mu nɯ [ɯ-tɕɯ ɯ-kɯ-ɕar] cʰɤ-ɕe tɕe,\\
 \textsc{ifr}-be.bright \textsc{lnk} \textsc{lnk} \textsc{indef}.\textsc{poss}-mother \textsc{dem} \textsc{3sg}.\textsc{poss}-son \textsc{3sg}.\textsc{poss}-\textsc{nmlz}:S/A-search \textsc{ifr}:\textsc{downstream}-go \textsc{lnk} \\
\glt `When the sun came up (in the morning), the mother went to look for her son.' (2012tWJo, 33)
\end{exe}

\begin{exe}
\ex \label{ex:WkWCar.loCenW}
 \gll ɯ-fso-soz tɕe, [tɤ-rɟit ra kɯ nɯ ɯ-kɯ-ɕar] jo-ɕe-nɯ ɲɯ-ŋu tɕe \\
 \textsc{3sg}.\textsc{poss}-tomorrow-morning \textsc{lnk} \textsc{indef}.\textsc{poss}-child \textsc{pl} \textsc{erg} \textsc{dem} \textsc{3sg}.\textsc{poss}-\textsc{nmlz}:S/A-search \textsc{ifr}:\textsc{upstream}-go \textsc{sens}-be \textsc{lnk} \\
\glt  `The morning of the next day, the children went (there) to look for him.'  (Norbzang, 325)
\end{exe}

The goal of the motion verb can however occur within the purposive clause, as in (\ref{ex:khapa.WkWnnAjo}), where the subject in ergative form \forme{ɯ-wa nɯ kɯ} `his father' is stranded from the transitive verb \forme{ɯ-kɯ-n-nɤjo} by the goal \forme{kʰapa tɕe} `downstairs'.

\begin{exe}
\ex \label{ex:khapa.WkWnnAjo}
 \gll [ɯ-wa nɯ kɯ kʰapa tɕe ɯ-kɯ-n-nɤjo] pjɤ-ɣi.  \\
  \textsc{3sg}.\textsc{poss}-father \textsc{dem} \textsc{erg} downstairs \textsc{loc}    \textsc{3sg}.\textsc{poss}-\textsc{nmlz}:S/A-\textsc{auto}-wait \textsc{ifr}:\textsc{down}-come \\
  \glt `His father came downstairs to wait for him.' (140506 loBzi, 5)
\end{exe}

There is obligatory coreference between the subject of the motion verb and that of the purposive clause in this construction; to express coreference with the \textit{object} of the purposive clause (in the case of a transitive verb), the object participle is used instead (§ XXX, § XXX). 

Motion verb with purposive clauses have some semantic overlap with the corresponding associated motion prefixes (§ \ref{sec:am.prefixes}); the functional difference between the two construction is discussed in § \ref{sec:am.vs.mvc}.

Some transitive and semi-transitive verbs take object complement clauses (§ XXX) requiring a subject participle. This group includes the verb \japhug{sɯχsɤl}{recognize, notice} as in (\ref{ex:tAkWnWCpWz.pjAsWXsAl}) and the verbs of pretence \japhug{ʑɣɤpa}{pretend} and \japhug{nɯɕpɯz}{pretend, disguise as, imitate} as in (\ref{ex:kukWtshi.tonWCpWznW}). 

\begin{exe}
\ex \label{ex:tAkWnWCpWz.pjAsWXsAl}
 \gll tɕe nɯnɯ kɯ [qaʑo tɤ-kɯ-nɯɕpɯz] nɯ pjɤ-sɯχsɤl. \\
 \textsc{lnk} \textsc{dem} \textsc{erg} sheep \textsc{pfv}-\textsc{nmlz}:S/A-disguised \textsc{dem} \textsc{ifr}-recognize \\
\glt `He (the shepherd boy) had noticed the (nobleman) disguised as a sheep.' (40513 mutong de disheng-zh, 63)
\end{exe}

\begin{exe}
\ex \label{ex:kukWtshi.tonWCpWznW}
 \gll  ʑara kɯ [cʰa nɯ ku-kɯ-tsʰi] to-nɯɕpɯz-nɯ, \\
\textsc{3pl} \textsc{erg} alcohol \textsc{dem} \textsc{ipfv}-\textsc{nmlz}:S/A-drink \textsc{ifr}-pretend-\textsc{pl} \\
\glt  `They pretended to drink the alcohol.'  (Norbzang 2012, 91)
\end{exe}

The status of the clauses with subject participles occurring with these three verbs, though superficially similar to the purposive clauses, is however entirely distinct: these clauses are note specific constructions, but simply subject relative clauses in object or semi-object position. The difference with purposive clauses can be shown by three pieces of evidence. 

First, the three verbs in question can take nouns as objects (as shown by \ref{ex:qaɕpa.tonWCpWz} and \ref{ex:tAkWnWCpWz.pjAsWXsAl}) instead of clauses with subject participles. 

\begin{exe}
\ex \label{ex:qaɕpa.tonWCpWz}
 \gll  qaɕpa to-nɯɕpɯz, qaɕpa ɯ-rqʰu to-ŋga, \\
frog \textsc{ifr}-pretend frog \textsc{3sg}.\textsc{poss}-skin \textsc{ifr}-wear \\
\glt `He disguised as a frog, he wore a frog's skin.' (2002 qaCpa, 10)
\end{exe}
 
 Second, these verbs can occur with a participial clause whose subject is overt and distinct from the subject of the verb of the matrix clause, as in (\ref{ex:tApAtso.kWGAwu.kAnWCpWz}) where  \japhug{tɤ-pɤtso}{child} is the subject of the verb \japhug{ɣɤwu}{cry} in the participial clause, but not the subject of  \japhug{nɯɕpɯz}{pretend, disguise as, imitate} (see § XXX for additional examples). Such a subject mismatch would be completely ungrammatical with a purposive clause.
 
\begin{exe}
\ex \label{ex:tApAtso.kWGAwu.kAnWCpWz}
 \gll   [tɤ-pɤtso kɯ-ɣɤwu] ʑo kɤ-nɯɕpɯz mɤ-spe-a ma nɯ mɯma spe-a \\
 \textsc{indef}.\textsc{poss}-child \textsc{nmlz}:S/A-cry \textsc{emph} \textsc{inf}-imitate \textsc{neg}-be.able[III]:\textsc{fact}-\textsc{1sg} \textsc{lnk} \textsc{dem} apart.from be.able[III]:\textsc{fact}-\textsc{1sg} \\
\glt `I cannot imitate a child crying, but apart from that I can imitate (anything).' (27-kikakCi, 143)
\end{exe}

Third, we find examples like (\ref{ex:Wmi.kWmNAm.tonWCpWznW}) where the subject of the verb in the main clause is not coreferent with the subject of the participial clause but with the possessor of the subject; these are in fact explainable as cases of possessor relative clauses (§ \ref{sec:subject.participle.other.relative}).

\begin{exe}
\ex \label{ex:Wmi.kWmNAm.tonWCpWznW}
 \gll  tɕe [ɯ-mi kɯ-mŋɤm] to-nɯɕpɯz  \\
 \textsc{lnk} \textsc{3sg}.\textsc{poss}-leg \textsc{nmlz}:S/A-hurt \textsc{ifr}-pretend \\
 \glt `He pretended to have a pain in the leg.' (140426 lang yisheng-zh, 9)
\end{exe}

In some constructions, subject participial clauses can occur instead of infinitive clauses; For instance, the imperfective \forme{kɤ-} infinitive + existential verb construction expressing impossibility (§ \ref{sec:inf.exist}) has a variant with imperfective subject participles, as in (\ref{ex:tukWGi.YAGAme}). 

 \begin{exe}
\ex \label{ex:tukWGi.YAGAme}
 \gll   tu-kɯ-ɣi ɲɤ-ɣɤ-me qʰe,  \\
  \textsc{ipfv}:\textsc{up}-\textsc{nmlz}:S/A-come \textsc{ifr}-\textsc{caus}-not.exist \textsc{lnk} \\
  \glt `She made it impossible for her to come out (again).' (2003-kWBRa, 97)
 \end{exe}
 
\subsubsection{Adverbials} \label{sec:subject.participle.adverbial}
Subject participles used as adverbs are very common, and in a great variety of constructions.

Reduplicated subject participles occur as adverbs of degree, for instance \japhug{kɯ-xtɕɯ\redp{}xtɕi}{a little} from \japhug{xtɕi}{be small} in examples such as (\ref{ex:kWxtCWxtCi.wxti}).

\begin{exe}
\ex \label{ex:kWxtCWxtCi.wxti}
 \gll βʑɯ sɤz kɯ-xtɕɯ-xtɕi wxti. \\
 mouse \textsc{comp} \textsc{nmlz}:S/A-\textsc{emph}\redp{}be.small be.big:\textsc{fact} \\
 \glt `It is a little bigger than a mouse.' (21-GzWLa, 4)
\end{exe}

In some cases, it is debatable whether the \forme{kɯ-} prefixed verb is a subject participle or a \forme{kɯ-} infinitive (§ \ref{sec:velar.inf}). For instance, in (\ref{ex:mAkWmbrAt.YWrAma}) \forme{mɤ-kɯ-mbrɤt} `without stop' is considered to be a subject participle serving as a manner adverb, but it could be possible to propose an alternative analysis as a non-human infinitive.

\begin{exe}
\ex \label{ex:mAkWmbrAt.YWrAma}
 \gll nɯ maka mɤ-kɯ-mbrɤt ʑo ɲɯ-rɤma ɲɯ-ɕti tɕe,  \\
 \textsc{dem} at.all  \textsc{neg}-\textsc{nmlz}:S/A-\textsc{acaus}:break \textsc{emph} \textsc{ipfv}-work \textsc{sens}-be.\textsc{affirm} \textsc{lnk} \\
 \glt `It works without stopping at all.' (26-GZo, 69)
\end{exe}

The existential verb \japhug{tu}{exist} in participial form \forme{kɯ-tu} following a noun or a pronoun can mean `in the presence of...' as in (\ref{ex:Zara.kWtu.Zo}).

\begin{exe}
\ex \label{ex:Zara.kWtu.Zo}
\gll ʑara kɯ-tu ʑo to-sɤrɯru tɕe, \\
\textsc{3pl} \textsc{nmlz}:S/A-exist \textsc{emph} \textsc{ifr}-compare \textsc{lnk} \\
\glt `He compared (his testimony with theirs) in their presence.' (150909 xifangping-zh, 155)
\end{exe}

\subsubsection{Lexicalized subject participles} \label{sec:lexicalized.subject.participle}
A certain number of subject participles have developed specialized meanings and can be considered to have been lexicalized. Some of these lexicalized participles are formally identical to the regular participle (Table  \ref{tab:lexicalized.S.nmlz}, for instance the noun \japhug{kɯcʰi}{candy} in (\ref{ex:akWchi})  as compared to the non-lexicalized participle \forme{kɯ-cʰi} `the one that is sweet' in (\ref{ex:kWchi.tu}). For such nouns, lexicalization is shown by the meaning specialization and the inability to take orientation, associated motion and polarity prefixes (but not possessive prefixes, as shown by he prefix \forme{a-} on \japhug{kɯcʰi}{candy} in \ref{ex:akWchi}).

\begin{exe}
\ex \label{ex:akWchi}
 \gll aʑo a-ŋgra a-kɯcʰi ci tɤ-χti ra \\
 \textsc{1sg} \textsc{1sg}.\textsc{poss}-salary \textsc{1sg}.\textsc{poss}-candy \textsc{indef} \textsc{imp}-buy[III] have.to:\textsc{fact} \\
\glt `Give me a candy as a reward.' (140515 congming de wusui xiaohai, 82)
\end{exe}

\begin{exe}
\ex \label{ex:kWchi.tu}
 \gll tɕe nɯnɯ li tú-wɣ-ndza tɕe, kɯ-cʰi tu, mɤ-kɯ-cʰi tu. \\
\textsc{lnk} \textsc{dem} \textsc{ipfv}-\textsc{inv}-eat \textsc{lnk} \textsc{nmlz}:S/A-be.sweet  \textsc{neg}-\textsc{nmlz}:S/A-be.sweet exist:fact \\
\glt `When one eats them, some are sweet, some are not.' (08-rasti, 55)
\end{exe}

\begin{table}[H]
\caption{Lexicalized subject participles} \label{tab:lexicalized.S.nmlz} \centering
\begin{tabular}{llll}
\lsptoprule
Noun & Base verb \\
\midrule
\japhug{kɯβʁa}{noble} & \japhug{βʁa}{prevail, win}  \\
\japhug{kɯspoʁ}{hole} & \japhug{spoʁ}{have a hole}  \\
 \japhug{kɯcʰi}{candy} & \japhug{cʰi}{be sweet} \\
 \japhug{kɯmŋɤm}{ailment} & \japhug{mŋɤm}{hurt, feel pain} \\
\lspbottomrule
\end{tabular}
\end{table}

In addition, we find nouns in \forme{kɯ-} that can be suspected to be former lexicalized participles, such as \japhug{kɯjŋu}{oath}, which appears to contain the root of the verb   \japhug{ŋu}{be}, though the segment \forme{-j-} cannot be accounted for at the present moment.

Nominalizations with the \forme{x-/ɣ-} prefix (§ \ref{sec:G.nmlz}) are ancient lexicalized subject participles that have undergone a syllable reduction rule (§ XXX) and have become completely separated from their base verb synchronically.

\subsection{Object participles} \label{sec:object.participle}
The object participle is a nominalized form which refers to an entity corresponding to the object (\ref{sec:absolutive.P}) or semi-object (§ \ref{sec:semi.object}) of the base verb. All transitive and semi-transitive verbs (except for \japhug{kɤtɯpa}{tell}, § XXX) can build an object participle by adding the prefix \forme{kɤ-} (for instance \forme{kɤ-ndza} from the verb \japhug{ndza}{eat} in \ref{ex:kAndza}). This form is homophonous with, and historically related to the velar infinitive (§ \ref{sec:velar.inf}, § \ref{sec:velar.nmlz.history}).

 \begin{exe} 
\ex \label{ex:kAndza}
\gll kɤ-ndza \\
   \textsc{nmlz}:P-eat \\
 \glt  `The one that is eaten.' (many attestations)
 \end{exe}
 
 In this section, I discuss first morphological issues (possessive prefixes § \ref{sec:object.participle.possessive}  and other prefixes § \ref{sec:object.participle.other.prefixes})
 
\subsubsection{Possessive prefixes on object participles}  \label{sec:object.participle.possessive} 
Unlike subject participles, object participles never require a possessive prefix. An optional possessive prefix coreferent with the transitive subject, as in (\ref{ex:akAsWz}), can however be added.
  
  \begin{exe}
\ex \label{ex:akAsWz}
\gll a-kɤ-sɯz    \\
   \textsc{1sg-nmlz}:P-know \\
 \glt  `The one that I know.' (many attestations)
 \end{exe}

In the case of semi-transitive verbs, the possessive prefix is also coreferent with the subject, as in the form \forme{ɯ-kɤ-rga} `the one that he likes' in (\ref{ex:stu.WkArga}), build in the same way as the object participle of the transitive (tropative § XXX) verb \japhug{nɤmɯm}{find tasty}.

\begin{exe}
\ex \label{ex:stu.WkArga}
\gll ri nɯnɯ stu ɯ-kɤ-rga, ɯ-kɤ-nɤ-mɯm pjɤ-ɕti. \\
\textsc{lnk} \textsc{dem}  most \textsc{3sg}.\textsc{poss}-\textsc{nmlz}:P-like most \textsc{3sg}.\textsc{poss}-\textsc{nmlz}:P-\textsc{trop}-be.tasty ifr.\textsc{ipfv}-be.\textsc{affirm} \\
\glt `But it was what he liked most, what he found most tasty.' (160703 poucet3, 74)
\end{exe}

In addition to semi-transitive verbs, the complement-taking verb \japhug{cʰa}{can} has object participles taking possessive prefixes meaning `the one that X can Y', X being the subject (marked by the possessive prefix), and Y the verb in the complement clause, which can be overt or not as in (\ref{ex:nWmAkAcha}), where \forme{nɯ-mɤ-kɤ-cʰa} stands for \forme{kɤ-ndo nɯ-mɤ-kɤ-cha} `the one(s) that they are able to catch'.
 
\begin{exe}
\ex  \label{ex:nWmAkAcha}
\gll tɕe nɯ-mɤ-kɤ-cʰa nɯ kʰɯna χsɯm pɯ-tu qʰe, nɯra kɯ rcanɯ ɕlaʁ ʑo ku-ndo-nɯ ɲɯ-ɕti. \\
\textsc{lnk} \textsc{3pl}.\textsc{poss}-\textsc{neg}-\textsc{inf}-can \textsc{dem} dog three \textsc{pst}.\textsc{ipfv}-exist \textsc{lnk} \textsc{dem}:\textsc{pl} \textsc{erg} unexpectedly \textsc{ideo}.I:immediately \textsc{emph} \textsc{ipfv}-catch-\textsc{pl} \textsc{sens}-be.\textsc{affirm} \\
\glt `The (rats) that they (the people) had been unable to (catch), there were three dogs, these (dogs) caught them at once.' (150831 BZW kAnArRaR, 48)
\end{exe}

\subsubsection{Associated motion, polarity and orientation prefixes on object participles}  \label{sec:object.participle.other.prefixes}

%tɕheme nɯ kɯ iɕqha, ɯ-jaʁ /mei/ @meihua, mɯntoʁ pɯ-kɤ-ɤsɯ-ndo nɯ pjɤ-ɣɤrɤt. 

\subsection{Oblique participles}
The \forme{sɤ}-prefix (and its allomorphs \forme{sɤz}- and \forme{z}-) is used for non-core argument nominalization, in particular recipient of indirective verbs (§ \ref{sec:gen.beneficiary}, § \ref{sec:dative}), instruments (\ref{sec:instr.kW}), place and time adjuncts. It takes a possessive prefix which can be coreferent with any core argument (subject or object).

   \begin{exe}
\ex \label{ex:come}
\gll ɯ-sɤ-ɣi    \\
   \textsc{3sg-nmlz:oblique}-come \\
 \glt  `The place/moment where/when it comes.' (elicited)
 \end{exe}
 
 %nɯnɯ ɯ-sɤtɕha, tɤjmɤɣ ɯ-sɤ-tu ɯ-sɤ-me ɣɤʑu.
 %ɯnɯnɯ kɯ ɯ-sɤ-ɕmi tu-sɯ-βzu-nɯ pɯ-ŋgrɤl 
 %tɕe saŋdi nɯ tɕe, nɯnɯ /nɤki/ si ɯ-sɤ-ta ɯ-rkoz ʑo pjɤ-ŋu.
 
 
 
\subsubsection{Associated motion, polarity and orientation prefixes} \label{sec:oblique.participle.orientation}
Unlike subject and object participles, oblique participles can only take series B orientation prefixes. 

It is thus not possible to have perfective or past imperfective oblique participles, and alternative strategies are used to express the corresponding meanings. For instance, from the verb \japhug{sqa}{cook}, the form $\dagger$\forme{ɯ-pɯ-sɤ-sqa} (intended meaning: `the thing that has been used to cook') is incorrect, and the solution to circumvent this morphological constraint is to combine the plain oblique participle \forme{ɯ-sɤ-sqa} with \forme{pɯ-kɯ-ŋu} (the past imperfective subject participle of \japhug{ŋu}{be}) and with the phrase \forme{nɯ ɕɯŋgɯ} `before that', as in (\ref{ex:WsAsqa.pWkWNu}).

\begin{exe}
\ex \label{ex:WsAsqa.pWkWNu}
\gll  nɯ ɕɯŋgɯ ɯ-sɤ-sqa pɯ-kɯ-ŋu ɯ-ŋgɯ (tu-rku-nɯ) \\
\textsc{dem} before \textsc{3sg}.\textsc{poss}-\textsc{nmlz}:\textsc{oblique}-cook \textsc{pst}.\textsc{ipfv}-\textsc{nmlz}:S/A-be \textsc{3sg}.\textsc{poss}-inside \textsc{ipfv}-put.in-\textsc{pl} \\
\glt `(They put it) in the (pan) that had been used before to cook (the barley grains).' (31-cha, 64)
\end{exe}


\section{Infinitives}

\subsection{Velar infinitives} \label{sec:velar.inf}


\begin{exe}
\ex \label{ex:CWkAXtW}
\gll a-mgɯr ɲɯ-mŋɤm tɕe ɕɯ-kɤ-χtɯ mɯ́j-cʰa-a \\
\textsc{1sg}.\textsc{poss}-back \textsc{sens}-hurt \textsc{lnk} \textsc{transloc}-\textsc{inf}-buy \textsc{neg}:\textsc{sens}-can-\textsc{1sg} \\
\glt `My back hurts and I cannot go to  buy (apples).' (conversation, 30-04-2018)
\end{exe}

\begin{exe}
\ex \label{mWpjWkAlhoR.ftCaka}
 \gll  tɤ-se mɯ-pjɯ-kɤ-ɬoʁ ftɕaka tu-βze-a tu-mdzoz-a pɯ-ŋu ma, \\
 indef.\textsc{poss}-blood \textsc{neg}-\textsc{ipfv}-\textsc{inf}-come.out manner \textsc{ipfv}-make[III]-\textsc{1sg} \textsc{ipfv}-avoid-\textsc{1sg} \textsc{pst}.\textsc{ipfv}-be lnk \\
\glt `I avoided by all means to let the blood come out.' (24-pGArtsAG, 57)
 \end{exe}

\subsubsection{Doubly prefixed velar infinitives with negative existential verb} \label{sec:inf.exist}
The infinitive in \forme{kɤ-} can even take two prefixes in a construction combining the negative existential verb \japhug{me}{not exist} with a verb in the infinitive prefixed with a series B orientation prefix and a possessive prefix coreferent with the subject,\footnote{The assertion in \citet[228]{jacques16complementation} that infinitives cannot take possessive prefixes is thus wrong.} meaning `have no way to X, be completely unable to X', as in (\ref{ex:ndZijukACe}) and (\ref{ex:WpjWkAnWZWB}). Note that since in both of these examples, the verbs are intransitive and lack an object participle, the \forme{kɤ-} form can only be analyzed as an infinitive here. This construction is also possible with transitive verbs, in which case the possessive prefix corresponds to the transitive subject.

\begin{exe}
\ex \label{ex:ndZijukACe}
\gll tɕe ndʑi-ju-kɤ-ɕe pjɤ-me \\
\textsc{lnk} \textsc{3du}.\textsc{poss}-\textsc{ipfv}-\textsc{inf}-go \textsc{ifr}.\textsc{ipfv}-not.exist \\
\glt `They could not go.' (150908 menglang-zh, 46)
\end{exe}

\begin{exe}
\ex \label{ex:WpjWkAnWZWB}
\gll tɯ-rʑaʁ nɯ ɯ-pjɯ-kɤ-nɯʑɯβ pjɤ-me matɕi, \\
one-night \textsc{dem} \textsc{3sg}.\textsc{poss}-\textsc{ipfv}-\textsc{inf}-sleep \textsc{ifr}.\textsc{ipfv}-not.exist \textsc{lnk} \\
\glt `He could not sleep the whole night, because...' (150831 BZW kAnArRaR, 12)
\end{exe}

A derived construction involves the causative \japhug{ɣɤme}{cause not to exist, suppress} with doubly prefixed infinitives to `make it impossible for X to Y' as in (\ref{ex:apjWkAnWZWB.naGAme}).

\begin{exe}
\ex \label{ex:apjWkAnWZWB.naGAme}
\gll a-pjɯ-kɤ-nɯʑɯβ na-ɣɤ-me \\
\textsc{1sg}.\textsc{poss}-\textsc{ipfv}-\textsc{inf}-sleep \textsc{pfv}:3\fl{}3'-\textsc{caus}-not.exist \\
\glt `He made me unable to sleep.' (elicited)
\end{exe}

%cʰa a-ku-kɤ-tsʰi na-ɣɤme
There is a variant of this construction with imperfective subject participles in \forme{kɯ-} instead of infinitives (§ \ref{sec:subject.participle.complementation}).

\subsection{Dental infinitives} \label{sec:dental.inf}
\subsection{Bare infinitives} \label{sec:bare.inf}
\section{Degree nominals} \label{sec:degree.nominals}

\section{Other deverbal nouns}

\subsection{Nominalization \forme{-z} suffix} \label{sec:z.nmlz}
Japhug has two inalienably possessed nouns derived from verbs by means of a nominalizing \forme{-z}; both take the indefinite possessive prefix \forme{tɤ-} (§ \ref{sec:inalienably.possessed}).

The first one, \japhug{tɤ-scoz}{letter, writing}, originates from the verb \japhug{sco}{see off, accompany}, though it might be a loan from Situ (\citealt{jacques03s.houzhui}). 

The second one, \japhug{tɤ-rkuz}{parting present}, a biactantial IPN whose possessor corresponds to the recipient (§ \ref{sec:biactantial.ipn}), comes from \japhug{rku}{put in}. The etymological relationship  between these two words is obvious when the use of  \forme{rku} in the sense of `give as a present to take away' (put in someone's luggage) is considered, as in (\ref{ex:kWki.nArkuz.Nu}).

\begin{exe}
\ex \label{ex:kWki.nArkuz.Nu}
\gll tɕe tó-wɣ-z-rɤŋgat tɕe, tɕendɤre nɯnɯ kɯ, iɕqʰa nɯ,  tɯ-ci tɯ-tɤ-ste to-rku. tɕe `kɯki nɤ-rkuz ŋu' to-ti. \\
\textsc{lnk} \textsc{ifr}-\textsc{inv}-\textsc{caus}-prepare.to.leave \textsc{lnk} \textsc{lnk} \textsc{dem} \textsc{erg} \textsc{filler} \textsc{dem} \textsc{indef}.\textsc{poss}-water \textsc{one}-\textsc{indef}.\textsc{poss}-bladder \textsc{ifr}-put.in \textsc{lnk} \textsc{dem}.\textsc{prox} \textsc{2sg}.\textsc{poss}-present be:\textsc{fact} \textsc{ifr}-say \\
\glt `He prepared his departure, and gave him a bladder full of water to take with him, and said `this is your departing present'.' (28-smAnmi.txt, 264-265)
\end{exe}


The verb \japhug{rku}{put in} can even occur with its derived noun \japhug{tɤ-rkuz}{parting present} in the \textit{figura etymologica} construction in (\ref{ex:arkuz.tarku}) (the verb \japhug{βzu}{make} can alternatively be used instead of \japhug{rku}{put in}).

\begin{exe}
\ex \label{ex:arkuz.tarku}
\gll a-me kɯ a-rkuz rŋɯl ta-rku \\
\textsc{1sg}.\textsc{poss}-daughter \textsc{erg} \textsc{1sg}.\textsc{poss}-present money \textsc{pfv}:3\fl{}3'-put.in \\
\glt `My daughter gave me some money (as present for my departure) (elicited)
\end{exe}

The \forme{-z} nominalizing suffix, though rare in Japhug, is of Sino-Tibetan origin. In Situ, the corresponding nominalizing \forme{-s} suffix is much more common (\citealt{jacques03s.houzhui}), and Tibetan and Chinese have traces of a cognate suffix (\citealt{jacques16ssuffixes}).


\subsection{Nominalization \forme{ɣ-/x-} prefix} \label{sec:G.nmlz}
A handful of nouns, most of them IPNs, are derives from intransitive verbs by means of a velar prefix  \forme{ɣ}- or \forme{x}- (harmonizing in voicing with the initial consonant of the stem). These nouns are lexicalized ancient subject participles (§ \ref{sec:subject.participles}) which underwent the same phonological change as that observed with the velar animal class prefix (§ \ref{sec:velar.class.prefix}), that has a syllabic allomorph \forme{kɯ-} and reduced allomorphs \forme{ɣ}- or \forme{x}-. 

The reduced \forme{ɣ}- / \forme{x}- prefix only derives nouns from intransitive verbs with monosyllabic stems, without consonant clusters. 


\begin{table}[H]
\caption{Irregular subject nominalizations in \forme{ɣ}- and \forme{x}-} \label{tab:irregular.nmlz} \centering
\begin{tabular}{llll}
\lsptoprule
Noun & Base verb & Reference \\
\midrule
\japhug{ɣndʑɤβ}{disastrous fire} & \japhug{ndʑɤβ}{burn} \\
\japhug{ɯ-ɣɲaʁ}{disaster}& \japhug{ɲaʁ}{be black} \\
\japhug{ɯ-ɣɲɟɯ}{orifice} & \japhug{ɲɟɯ}{be opened} \\
\japhug{ɯ-xso}{empty, normal} &\japhug{so}{be empty} &  \ref{sec:property.nouns} \\
\japhug{ɯ-ɣrom}{dried thing} & \japhug{rom}{be dry} \\
\lspbottomrule
\end{tabular}
\end{table}

The noun \japhug{tɯ-xpa}{one year}, although derived from the verb \japhug{pa}{pass X years} and having an additional \forme{x-} element, does not belong to this category, see  § \ref{sec:num.prefix.paradigm.history} and § \ref{sec:CN.verbs}.


\section{Converbs}
\subsection{Gerund} \label{sec:gerund}
%nɤʑo laχtɕha kɤ-fkur tu-tɯ-ŋke ɲɯ-ŋu tɕe,
\subsection{Purposive} \label{sec:purposive.converb}


With intransitive and semi-transitive verbs, the possessive prefix is coreferent to the intransitive subject, as in (\ref{ex:aYWsAstWstu}).

\begin{exe}
\ex \label{ex:aYWsAstWstu}
\gll  tɕe nɯnɯ a-ɲɯ-sɤ-stɯ\redp{}stu nɯra tu-nɤme pjɤ-ŋu \\
\textsc{lnk} \textsc{dem} \textsc{1sg}-\textsc{ipfv}-\textsc{purp}-believe \textsc{dem}:\textsc{pl} \textsc{ipfv}-make[III] \textsc{ifr}.\textsc{ipfv}-be \\
\glt `He was doing these things so that I would believe (in his predictions).' (150904 yaoshu-zh, 104)
\end{exe}

\subsection{Immediate} \label{sec:immediate.converb}

\section{Historical perspectives} \label{sec:nmlz.historical.perspectives}

\subsection{Velar non-finite prefixes} \label{sec:velar.nmlz.history}

\subsection{Sigmatic non-finite prefixes} \label{sec:sigmatic.nmlz.history} %Non-finite verbal morphology
%\chapter{Voice derivations}

\section{Fossil derivations}

\subsection{Applicative \forme{-t}}
Beside the productive prefixal \forme{nɯ-} applicative (sec XXX), Japhug has vestigial traces of a \forme{-t} applicative suffix, better attested in Kiranti and West Himalayish languages (see \citealt{michailovsky85dental}, \citet{jacques15derivational.khaling} and \citealt{jacques16ssuffixes} for comparative studies of this suffix). Only two examples of this derivation exist in Japhug: \japhug{ɣɯt}{bring} and \japhug{mdɯt}{strongly wish for}. 

The verb of manipulation \japhug{ɣɯt}{bring} derives from the motion verb \japhug{ɣi}{come}; the vowel alternation is regular as pre-Japhug \forme{*i} changes to \ipa{ɯ} in closed syllables. With a motion verb such as `come', the effect of the applicative (\ref{ex:bring1}) is similar to a causative  (\ref{ex:bring2}). 

\begin{exe}
\ex \label{ex:bring1}
\glt `come with X' $\rightarrow$ `bring'
\ex \label{ex:bring2}
\glt `cause X to come' $\rightarrow$ `bring'
\end{exe}

The transitive verb \japhug{mdɯt}{strongly wish for} is historically related to the verb \japhug{mdɯ}{live up to}, and constitutes another example of the \forme{-t} applicative, though it is less immediately obvious than in the case of \japhug{ɣɯt}{bring} because each of the verbs has undergone semantic specialization after the derivation took palce.

The verb \forme{mdɯ} is semi-transitive (sec XXX), and takes as its semi-object the lifespan; it can be applied to plants, animals and humans, as shown by examples (\ref{ex:chWmdW}) and (\ref{ex:chWmdWa}). It selects the  downstream' series of directional prefixes (\forme{tʰɯ-}, \forme{cʰɯ-}).

 \begin{exe}
\ex \label{ex:chWmdW}
\gll tɕe nɯŋa ɯʑo nɯnɯ, sqamŋu-xpa jamar cʰɯ-mdɯ ɲɯ-ŋgrɤl\\
\textsc{lnk} cow \textsc{3sg} \textsc{dem} fifteen-year about \textsc{ipfv}-live.up.to \textsc{sens}-be.usually.the.case \\
\glt `A cow itself can live up to fifteen years.' (05-qaZo, 142)
\end{exe}

 \begin{exe}
\ex \label{ex:chWmdWa}
\gll ``nɤʑo nɯ kʰrɯtsu-xpa a-tʰɯ-tɯ-mdɯ ra nɤ" to-ti ɲɯ-ŋu. tɕe ``aʑo kɯnɤ kʰrɯtsu cʰondɤre tɯ-rʑaʁ nɯnɯ cʰɯ-mdɯ-a ra" to-ti \\
\textsc{2sg} \textsc{dem} ten.thousand-year \textsc{irr}-\textsc{pfv}-2-live.up.to have.to:\textsc{fact} \textsc{sfp} \textsc{ifr}-say \textsc{sens}-be \textsc{lnk} \textsc{1sg} also  ten.thousand-year \textsc{comit} one-day \textsc{dem} \textsc{ipfv}-live.up.to-1sg have.to:\textsc{fact} \textsc{ifr}-say \\
\glt `He said: `May you live ten thousand years! I want to live one thousand years and one more day.' (150830 afanti-zh, 64)
\end{exe}

The meaning `live until/up to' is however a semantic innovation in Japhug: its Situ cognate \forme{mdə́} means `reach' as a motion verb. Japhug has restricted the meaning of this verb to a very specific context.

The verb \japhug{mdɯt}{strongly wish for} is morphologically transitive, and can take as its object an infinitive complement as in (\ref{ex:chWmdWta}). It shares with \japhug{mdɯ}{live up to} the `downstream' directional prefixes (\forme{cʰɯ-}).

 \begin{exe}
\ex \label{ex:chWmdWta}
\gll aʑo kɯrɯ-skɤt kɤ-βzjoz nɯ cʰɯ-mdɯt-a ʑo ɕti \\
\textsc{1sg} Tibetan-language \textsc{inf}-learn \textsc{dem} \textsc{ipfv}-strongly.wish \textsc{emph} be.\textsc{affirm}:\textsc{fact} \\
\glt `I strong wish to learn Tibetan/Gyalrong.' (elicited)
\end{exe}

The precise meaning of \forme{mdɯt}  is to wish for something that one is confident he can realize. If one accepts that the idea the original meaning of Japhug \japhug{mdɯ}{live up to} was `reach' as in Situ, the meaning `wish for' of the verb \forme{mdɯt} has the same relationship to that of the base verb as English `reach for' (`reach for the stars') to the verb `reach', with a conative interpretation `try to reach'.  The addition of the suffix \forme{-t} turns the semi-transitive (morphologically intransitive) \forme{mdɯ} into a transitive verb whose A corresponds to the S of the base verb. This applicative derivation from a semi-transitive verb is not unique in Japhug; the transitive verb \japhug{nɯrga}{like} from the verb \japhug{rga}{like, be happy} with the \forme{nɯ-} applicative is another similar example (see section XXX).


 %Voice derivations
%\include{chapters/4-06} %Denominal derivations
% \chapter{Tense, aspect, modality and evidentiality} \label{chap:tame}

  \section{Non-past tenses}
 \subsection{Dubitative}
 The dubitative \forme{ku-} is formally identical to imperfective `toward east' orientation prefix (§ XXX) and to the egophoric (§ XXX). It always occurs with the autobenefactive \forme{-nɯ-} prefix (§ XXX) and with the polar question \forme{ɕi} particule (§ XXX, see example \ref{ex:kunWZru.Ci.kWma}), the interrogative \forme{kɯ} (§ XXX, \ref{ex:CW.ci.kunWNu.kW}) or the alternative polar question construction (combining a positive followed by the equivalent negative verb form as in \ref{ex:kunWphAn.mWkunWphAn}, see § XXX).
 
 \begin{exe}
\ex \label{ex:kunWZru.Ci.kWma}
 \gll  tɕe lu-kɤ-nɯ-ji nɯ kɯ ʑru tu-ti-nɯ ɲɯ-ŋu tɕe mɤ-xsi. ku-nnɯ-ʑru ɕi kɯma. \\
 \textsc{lnk} \textsc{ipfv}-\textsc{nmlz}:P-\textsc{auto}-plant \textsc{dem} \textsc{erg} be.strong:\textsc{fact} \textsc{ipfv}-say-\textsc{pl} \textsc{sens}-be \textsc{lnk} \textsc{neg}-\textsc{genr}:know:\textsc{fact}  \textsc{dubit}-\textsc{auto}-be.strong \textsc{qu} \textsc{sfp} \\
 \glt `The cultivated (variety of Angelica) is better (than the wild one), they say, I don't know, maybe it is better.' (17-ndZWnW, 34)
 \end{exe}
 

  \begin{exe}
\ex \label{ex:kunWphAn.mWkunWphAn}
 \gll  ku-nɯ-pʰɤn mɯ-ku-nɯ-pʰɤn mɤ-xsi ma \\
 \textsc{dubit}-\textsc{auto}-be.efficient \textsc{neg}-\textsc{dubit}-\textsc{auto}-be.efficient \textsc{neg}-\textsc{genr}:know:\textsc{fact} \textsc{sfp} \\
\glt `I don't know whether it efficient or not (as medicine).' (19-GzW, 108)
  \end{exe}

In addition, dubitative verb forms are followed either by the sentence final particles \forme{kɯma} or \forme{kɯɣe} (§ XXX) as in (\ref{ex:kunWZru.Ci.kWma}) or a verb form such as \japhug{mɤ-xsi}{one does not know} (§ \ref{ex:kunWphAn.mWkunWphAn}).

The dubitative is mainly used to express doubts while reporting opinions from other people (as in \ref{ex:kunWZru.Ci.kWma} and \ref{ex:kunWphAn.mWkunWphAn}), but with the interrogative \forme{kɯ} as in (\ref{ex:CW.ci.kunWNu.kW}), its meaning is rather that of emphasis on the fact that the speaker has no clue about the answer to the question (as in French \textit{donc...bien} in `\textit{Qui donc cela peut-il bien être?}').

  \begin{exe}
\ex \label{ex:CW.ci.kunWNu.kW}
 \gll wo, nɯ ɕɯ ci ku-nɯ-ŋu kɯ?  \\
 \textsc{interj} \textsc{dem} who \textsc{indef} \textsc{dubit}-\textsc{auto}-be \textsc{qu} \\
\glt `Who on earth is it (who does all) that?' (2014-kWLAG, 619)
 \end{exe}
 
\subsection{Inferential} 

\subsubsection{Inferential with first person}
The inferential is not rare with first person subject, but occurs specifically to express that the speaker did not realize that the action took (or did not take) place at the time, as in (\ref{ex:mWtosAlata}), a sentence said after the speaker (Tshendzin) noticed that she forgot to put the water to boil.

\begin{exe}
\ex \label{ex:mWtosAlata}
\gll tɯ-ci mɯ-to-sɯ-ɤla-t-a \\
\textsc{indef}.\textsc{poss}-water \textsc{neg}-\textsc{ifr}-caus-be.boiling-\textsc{tr}:\textsc{pst}-\textsc{1sg} \\
\glt `I did not put the water to boil.' (Conversation, 01-05-2018, Tshendzin)
\end{exe}


Example (\ref{ex:mWtozwara}) illustrate the differential use of perfective and inferential with first person subject: the speaker did put the water on the oven, but forgot to open the oven, hence the use of the inferential for the second verb.

\begin{exe}
\ex \label{ex:mWtozwara}
\gll tɯ-ci kɤ-ta-t-a ri, <dian> mɯ-to-zwar-a \\
\textsc{indef}.\textsc{poss}-water \textsc{pfv}-put-\textsc{pst}:\textsc{tr}-\textsc{1sg} \textsc{lnk} electricity \textsc{neg}-\textsc{ifr}-burn-\textsc{1sg} \\
\glt `I put the water (on the oven), but did not open the electricity.' (Conversation, 04-05-2018, Tshendzin)
\end{exe}

%\begin{exe}
%\ex \label{ex:mWtotata}
%\gll mɯ-to-ta-t-a \\
%\textsc{neg}-\textsc{ifr}-put-\textsc{tr}:\textsc{pst}-\textsc{1sg} \\
%\glt `I did not put (the tea on the oven).' (Conversation, 28-04-2018, Dpalcan)
%\end{exe}

The inferential with first person is particularly common with verbs expressing uncontrollable and non-volitional actions, such as (\ref{ex:kAnWZWB.kordala}). 

\begin{exe}
\ex \label{ex:kAnWZWB.kordala}
\gll kɤ-nɯʑɯβ ko-rdal-a \\
\textsc{inf}-sleep \textsc{ifr}-overshoot-\textsc{1sg} \\
\glt `I overslept.' (elicitation)
\end{exe}

The inferential does not however express by itself non-volitionality; the autobenefactive-spontaneous prefix \forme{-nɯ-} (§ XXX) is used in conjunction with the inferential to insist on the non-volitional character of a particular action, as in (\ref{ex:YAnWjmWta}).

\begin{exe}
\ex \label{ex:YAnWjmWta}
\gll ɲɤ-nɯ-jmɯt-a \\
 \textsc{ifr}-\textsc{auto}-forget-\textsc{1sg} \\
\glt `I forgot.' (many attestations)
\end{exe}

  \section{Auxiliary TAM categories}
  
  \subsection{Apprehensive}
  
\begin{exe}
\ex 
\gll   mɯ-ɕɯ-kʰɯ kɯ nɯ-sɯso-t-a \\
\textsc{neg}-\textsc{apprehensive}-be.possible \textsc{sfp} \textsc{pfv}-think-\textsc{pst}:\textsc{tr}-\textsc{1sg} \\
\glt `I thought it would not work.'  (conversation, 02-05-2018; after opening a washing machine, expecting it would not open)
\end{exe}
 %TAME
%\include{chapters/4-08} %Other verbal categories
%\chapter{Relative clauses}
%negative with maka
%ɯʑo kɯ ta-tɯt maka kɯ-tu me
%He said nothing 什么也没有说
%\subsection{Competing constructions}
%
%\subsubsection{Relatives}
%nɤʑo	nɯ-nɯ-ɣɤwu	ma,	nɤ-kɯ-nɯɣmu	me	ma	ma-ta-mbi
%Frog 38
%
%
%iɕqha tɯrme kɯngo ɣɤʑu tɕe, ɯkɯrtoʁ jaria wo!
%Someone was ill \wav{8_kWngoGAZu}
%
%
%	kɤ-mɯnmu	kɯ́nɤ	mɯ-pjɤ-mɯnmu				
%	Raven 58
%
%
%A	78	nɤj	thɤstɯɣ	tɯ-chɯ-cha	ʑo	nɤ,	a-tɤ-tɯ-nɯ-rke	qhe	nɯnɯ	nɤʑo	ɣɯ	nɤ-rkus	ŋu"	to-ti,
%
%
%tɕhi kɯfse pɯtɯnɯmbɣom kɯ́nɤ, zgo nɯ kutɯpɣaʁ ndɤre ra
%\wav{8_kutWnWmbGom}
%
%nɯ apɯŋu ma dianhua ɯkɯlɤt ɣɤʑu!
%\wav{WkWlAt}
%%Negation
%ɯ-ɕɯ-kɯ-phɯt ra kɯ-tu me %Relativization
%\chapter{Relative clauses}
%negative with maka
%ɯʑo kɯ ta-tɯt maka kɯ-tu me
%He said nothing 什么也没有说
%\subsection{Competing constructions}
%
%\subsubsection{Relatives}
%nɤʑo	nɯ-nɯ-ɣɤwu	ma,	nɤ-kɯ-nɯɣmu	me	ma	ma-ta-mbi
%Frog 38
%
%
%iɕqha tɯrme kɯngo ɣɤʑu tɕe, ɯkɯrtoʁ jaria wo!
%Someone was ill \wav{8_kWngoGAZu}
%
%
%	kɤ-mɯnmu	kɯ́nɤ	mɯ-pjɤ-mɯnmu				
%	Raven 58
%
%
%A	78	nɤj	thɤstɯɣ	tɯ-chɯ-cha	ʑo	nɤ,	a-tɤ-tɯ-nɯ-rke	qhe	nɯnɯ	nɤʑo	ɣɯ	nɤ-rkus	ŋu"	to-ti,
%
%
%tɕhi kɯfse pɯtɯnɯmbɣom kɯ́nɤ, zgo nɯ kutɯpɣaʁ ndɤre ra
%\wav{8_kutWnWmbGom}
%
%nɯ apɯŋu ma dianhua ɯkɯlɤt ɣɤʑu!
%\wav{WkWlAt}
%%Negation
%ɯ-ɕɯ-kɯ-phɯt ra kɯ-tu me

\begin{exe}
\ex \label{ex:qala.kW.WkWmWrRWz}
 \gll  ɯ-rqo nɯtɕu iɕqʰa, [qala kɯ ɯ-kɯ-mɯrʁɯz] kɯ-fse ci pjɤ-tu.  \\
 \textsc{3sg}.\textsc{poss}-throat \textsc{dem}:\textsc{loc} \textsc{filler} rabbit \textsc{erg} \textsc{3sg}.\textsc{poss}-\textsc{nmlz}:S/A-scratch \textsc{nmlz}:S/A-be.like \textsc{indef} \textsc{ifr}.\textsc{ipfv}-be \\
\glt `In his throat; there was something like a rabbit scratching him.' (150909 hua pi-zh, 192)
\end{exe} 

%tɕe khru kɯ thɯ-kɤ-sɯ-lɤt laʁdɯn kɯnɤ tu ma 
%prenominal %Complementation
%\chapter{Temporal and conditional clauses} \label{chap:temporal.conditional}

\subsubsection{Immediate succession}
\begin{exe}
\ex \label{ex:pjWtWqlWt} 
\gll pjɯ-tɯ-qlɯt qʰe pjɯ-ɴɢlɯt ɲɯ-ɕti. \\
\textsc{ipfv-conv:imm}-break \textsc{lnk} \textsc{ipfv}-\textsc{acaus}:break \textsc{sens}-be.\textsc{affirm} \\
\glt `It breaks as soon as one breaks it.' (07-Zmbri, 6)
\end{exe}
pjɯ́-wɣ-qlɯt nɯɕɯmɯma pjɯ-ɴɢlɯt ɕti.
\subsubsection{Opportunity}
The postposition \japhug{ʁaz}{while ... still} has a meaning close to that of Chinese \ch{趁着}{chènzhe}{while ... still}, 

\begin{exe}
\ex \label{ex:GWrNi.RaznA}
\gll iɕqʰa tɯ-ndʐi nɯ nɤki, ɣɯrŋi ʁaznɤ nɯnɯ nɯnɯtɕu pjɯ-tʂɯβ-nɯ tɕe tɕe \\
the.aforementioned \textsc{indef}.\textsc{poss}-skin \textsc{dem} \textsc{filler} be.wet:\textsc{fact} while \textsc{dem} \textsc{dem}:\textsc{loc} \textsc{ipfv}-sew-\textsc{pl} \textsc{lnk} \textsc{lnk} \\
\glt `They would sew the skin while it was still wet.' (06-BGa, 53)
\end{exe}
XXXXX ta-ʁa + z %Temporal and conditional clauses
%\chapter{Temporal and conditional clauses} \label{chap:temporal.conditional}

\subsubsection{Immediate succession}
\begin{exe}
\ex \label{ex:pjWtWqlWt} 
\gll pjɯ-tɯ-qlɯt qʰe pjɯ-ɴɢlɯt ɲɯ-ɕti. \\
\textsc{ipfv-conv:imm}-break \textsc{lnk} \textsc{ipfv}-\textsc{acaus}:break \textsc{sens}-be.\textsc{affirm} \\
\glt `It breaks as soon as one breaks it.' (07-Zmbri, 6)
\end{exe}
pjɯ́-wɣ-qlɯt nɯɕɯmɯma pjɯ-ɴɢlɯt ɕti.
\subsubsection{Opportunity}
The postposition \japhug{ʁaz}{while ... still} has a meaning close to that of Chinese \ch{趁着}{chènzhe}{while ... still}, 

\begin{exe}
\ex \label{ex:GWrNi.RaznA}
\gll iɕqʰa tɯ-ndʐi nɯ nɤki, ɣɯrŋi ʁaznɤ nɯnɯ nɯnɯtɕu pjɯ-tʂɯβ-nɯ tɕe tɕe \\
the.aforementioned \textsc{indef}.\textsc{poss}-skin \textsc{dem} \textsc{filler} be.wet:\textsc{fact} while \textsc{dem} \textsc{dem}:\textsc{loc} \textsc{ipfv}-sew-\textsc{pl} \textsc{lnk} \textsc{lnk} \\
\glt `They would sew the skin while it was still wet.' (06-BGa, 53)
\end{exe}
XXXXX ta-ʁa + z %Comparison
%\include{chapters/5-06} %Other subordinate clauses
%\include{chapters/5-07} %Sentence final particules
%\include{chapters/6} %Japhug in Sino-Tibetan perspective

% % copy the lines above and adapt as necessary

%%%%%%%%%%%%%%%%%%%%%%%%%%%%%%%%%%%%%%%%%%%%%%%%%%%%
%%%                                              %%%
%%%             Backmatter                       %%%
%%%                                              %%%
%%%%%%%%%%%%%%%%%%%%%%%%%%%%%%%%%%%%%%%%%%%%%%%%%%%%

%\input{localseealso.tex} 
% There is normally no need to change the backmatter section
\input{backmatter.tex} 
\end{document} 

% you can create your book by running
% xelatex main.tex
%
% you can also try a simple 
% make
% on the commandline
