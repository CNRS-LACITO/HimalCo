%%%%%%%%%%%%%%%%%%%%%%%%%%%%%%%%%%%%%%%%%%%%%%%%%%%%
%%%                                              %%%
%%%     Language Science Press Master File       %%%
%%%         follow the instructions below        %%%
%%%                                              %%%
%%%%%%%%%%%%%%%%%%%%%%%%%%%%%%%%%%%%%%%%%%%%%%%%%%%%
 
% Everything following a % is ignored
% Some lines start w--ith %. Remove the % to include them

\documentclass[output=book,
  nonflat,
  modfonts,
  colorlinks,
showindex
		  ]{langsci/langscibook}    
  
%%%%%%%%%%%%%%%%%%%%%%%%%%%%%%%%%%%%%%%%%%%%%%%%%%%%
%%%                                              %%%
%%%          additional packages                 %%%
%%%                                              %%%
%%%%%%%%%%%%%%%%%%%%%%%%%%%%%%%%%%%%%%%%%%%%%%%%%%%%

% put all additional commands you need in the 
% following files. {I}f you do not know what this might 
% mean, you can safely ignore this section

\input{localmetadata.tex}
\input{localpackages.tex}
\input{localhyphenation.tex}
\bibliography{bibliogj} 

%%%%%%%%%%%%%%%%%%%%%%%%%%%%%%%%%%%%%%%%%%%%%%%%%%%%
%%%                                              %%%
%%%             Frontmatter                      %%%
%%%                                              %%%
%%%%%%%%%%%%%%%%%%%%%%%%%%%%%%%%%%%%%%%%%%%%%%%%%%%% 
%\usepackage{lipsum}
\begin{document}     
%add all your local new commands to this file

\newcommand{\smiley}{:)}

\renewbibmacro*{index:name}[5]{%
  \usebibmacro{index:entry}{#1}
    {\iffieldundef{usera}{}{\thefield{usera}\actualoperator}\mkbibindexname{#2}{#3}{#4}{#5}}}

% \newcommand{\noop}[1]{}

\makeatletter
\def\blx@maxline{77}
\makeatother

\newcommand{\appref}[1]{Appendix \ref{#1}}
\newcommand{\fnref}[1]{Appendix \ref{#1}}
\newcommand{\regel}[1]{#1}
\newcommand{\vernacular}[1]{\emph{#1}}
\newcommand{\gloss}[1]{#1}

\newfontfamily\phon[Mapping=tex-text,Ligatures=Common,Scale=MatchLowercase]{Charis SIL} 
\newfontfamily\tibetain{Microsoft Himalaya} 
\newcommand{\ipa}[1]{\mbox{\phon/#1/}}  
\newcommand{\phonet}[1]{\mbox{\phon[#1]}}  
\newcommand{\ipab}[1]{{\phon#1}}  
\newcommand{\forme}[1]{\textit{\phon#1}}  
\newcommand{\japhug}[2]{\textit{\phon#1} `#2'}  
\newcommand{\tibet}[3]{{\tibetain#1} \textit{\phon#2} `#3'}  
\newcommand{\deux}[1]{\ipa{#1}\addtocounter{2clusters}{1}}
\newcommand{\trois}[1]{\ipa{#1}\addtocounter{3clusters}{1}}
 
\newcommand{\grise}[1]{\cellcolor{lightgray}\textbf{#1}}
\newfontfamily\cn[Mapping=tex-text,Ligatures=Common,Scale=MatchUppercase]{SimSun}%pour le chinois
\newcommand{\zh}[1]{{\cn #1}}


\newcommand{\tib}[1]{\cellcolor{lightgray}\textbf{#1}}
\newcommand{\idph}[1]{\cellcolor{gray}\textbf{#1}}
\newcommand{\tld}{\textasciitilde{}}

\XeTeXlinebreaklocale "zh" %使用中文换行
\XeTeXlinebreakskip = 0pt plus 1pt %
 %CIRCG
 \newcommand{\resetcounters}[2]{
\newcounter{#1}
\newcounter{#2}
\setcounter{#1}{\value{2clusters}}
\setcounter{#2}{\value{3clusters}}
 \setcounter{2clusters}{0}
  \setcounter{3clusters}{0}
}
 \newcommand{\addition}[2]{\ADD{\value{#1}}{\value{#2}}{\solution}\solution}
 
 

\maketitle                
\frontmatter
% %% uncomment if you have preface and/or acknowledgements

\currentpdfbookmark{Contents}{name} % adds a PDF bookmark
\tableofcontents
% \include{chapters/preface}
% \include{chapters/acknowledgments}
% \include{chapters/abbreviations} 
\mainmatter         
  

%%%%%%%%%%%%%%%%%%%%%%%%%%%%%%%%%%%%%%%%%%%%%%%%%%%%
%%%                                              %%%
%%%             Chapters                         %%%
%%%                                              %%%
%%%%%%%%%%%%%%%%%%%%%%%%%%%%%%%%%%%%%%%%%%%%%%%%%%%%
%\chapter{The Japhug language}
This paper focuses on the Japhug language (local name \ipa{kɯrɯ skɤt}) of Kamnyu village (\ipa{kɤmɲɯ}, Chinese \textit{Ganmuniao} \zh{干木鸟}) in Gdong-brgyad area (\ipa{ʁdɯrɟɤt}, Chinese  \textit{Longerjia} \zh{龙尔甲}), Mbarkhams county (Chinese \textit{Maerkang} \zh{马尔康}), Rngaba prefecture, Sichuan province, China.
 
 Japhug belongs to the Sino-Tibetan family, and is one of the four Rgyalrong languages, alongside Tshobdun, Zbu and Situ.\footnote{See  \citet{jackson00sidaba} for an overview of the Rgyalrong group, whose closest relatives include Khroskyabs (\citealt{lai15person}) and Horpa (\citealt{jackson07shangzhai}). A text collection of Japhug with sound files is included in the Pangloss archive (\citealt{michailovsky14pangloss}). A short grammar (\citealt{jacques08}), a series of articles on morphosyntax (see for instance  \citealt{jacques13harmonization} and
 \citealt{jacques14antipassive}) and a dictionary (\citealt{jacques15japhug}) are available but little has been published specifically on its phonology. } 
 
The description is based on the author’s fieldwork, and the word list and the short story in the appendix have been provided by Tshendzin (Chenzhen \zh{陈珍}, female, born 1950), a retired schoolteacher (a native speaker of Japhug, bilingual in Sichuan Mandarin since childhood).  %The Japhug language
%\chapter{Corpus/02.tex}
  %The Japhug Corpus
\chapter{Phonology} \label{sec:phonology}

  

 Japhug has a highly  developed system of ideophones (\citealt{japhug14ideophones}), which present unusual phonological features, in particular rare clusters. In the following discussion, phonemes or clusters found exclusively on  ideophones will be treated separately. In addition, about a quarter of the Japhug vocabulary is borrowed from Tibetan, and these loanwords, like the ideophones, fill some gaps in the phonotactic distribution of vowels and consonants (on gap-filling by loanwords see \citealt[63-64]{martinet05economie}).  These cases are carefully distinguished from the native vocabulary in the analyses that follow, in order to bring out the phonotactics of inherited Japhug vocabulary.

%\begin{exe} 
% \ex 
%\gll   nɯ kɯ pʰɤn, ɯ-pʰɤntʰoʁ tu kɤ-ti ɲɯ-ŋu.\\ 
% \textsc{dem} \textsc{erg} be.efficient 3\textsc{sg}.\textsc{poss}-advantage    exist \textsc{inf}-say \textsc{testim}-be\\ 
% \glt  it is more efficient, more advantageous, it is said.
%\end{exe} 

\section{Consonants}

\subsection{Consonant clusters}
%\&[ \t]*\\ipa\{([^}]*)\} *\&[\t ]*([^\&]*)
%\& \\japhug{\1}{\2} 
%regex for converting tables

\newcounter{2clusters}
\newcounter{3clusters}

 \begin{table}
 \caption{List of consonant clusters with \ipa{w} as a first element (15+8)}  \centering \label{prein.w}
\begin{tabular}{l|lll|lll|lll|lllllll}
  \lsptoprule
%\ipa{p}  & 	  & 	  & 	  & 	 \\
%\ipa{pʰ}  & 	  & 	  & 	  & 	 \\
%\ipa{b}  & 	  & 	  & 	  & 	 \\
%\ipa{mb}  & 	  & 	  & 	  & 	 \\
%\ipa{m}  & 	  & 	  & 	  & 	 \\
\ipa{t}  & 	 \deux{wt}  &  	 \japhug{ɯ-ftaʁ}{sign} \\
%\ipa{tʰ}  & 	  & 	  & 	  & 	 \\
\ipa{d}  & 	 \deux{wd}  & 	\japhug{βdɯt}{demon} \\
%%\ipa{nd}  & 	  & 	  & 	  & 	 \\
%%\ipa{n}  & 	  & 	  & 	  & 	 \\
\ipa{ts}  & 	 \deux{wts}  & 	\japhug{ftsoʁ}{female hybrid yak} \\
\ipa{tsʰ}  & 	 \deux{wtsʰ}  & 	\japhug{ftsʰi}{it is not serious (disease)} \\
%%\ipa{dz}  & 	  & 	  & 	  & 	 \\
%%\ipa{ndz}  & 	  & 	  & 	  & 	 \\
\ipa{s}  & 	 \deux{ws}  & 	\japhug{fsaŋ}{fumigation} \\
\ipa{z}  & 	 \deux{wz} \tib{}  & 	\japhug{βzaŋsa}{friend} \\
%\ipa{ɬ}  & 	  & 	  & 	  & 	 \\
\ipa{tɕ}  & 	 \deux{wtɕ}  & 	\japhug{ftɕar}{summer} \\
\ipa{tɕʰ}  & 	 \deux{wtɕʰ}  & 	\japhug{ftɕʰur}{he pours it down} \\
%\ipa{dʑ}  & 	  & 	  & 	  & 	 \\
%\ipa{ndʑ}  & 	  & 	  & 	  & 	 \\
\ipa{ɕ}  & 	 \deux{wɕ}  & 	\japhug{fɕaʁ}{he repents for it} \\
\ipa{ʑ}  & 	 \deux{wʑ}  & 	\japhug{βʑar}{buzzard} \\
\ipa{tʂ}  & 	 \deux{wtʂ}  & 	\japhug{ftʂi}{he melts it} \\
%\ipa{tʂʰ}  & 	  & 	  & 	  & 	 \\
%\ipa{dʐ}  & 	  & 	  & 	  & 	 \\
%\ipa{ndʐ}  & 	  & 	  & 	  & 	 \\
%\ipa{ʂ}  & 	  & 	  & 	  & 	 \\
\ipa{c}  & 	 \deux{wc}  & 	\japhug{tɯ-fcaʁ}{dorsal mat} \\
%\ipa{cʰ}  & 	  & 	  & 	  & 	 \\
\ipa{ɟ}  & 	 \deux{wɟ}  & 	\japhug{βɟi}{he runs after it} \\
%\ipa{ɲɟ}  & 	  & 	  & 	  & 	 \\
%\ipa{ɲ}  & 	  & 	  & 	  & 	 \\
\ipa{k}  & 	 \deux{wk}  & 	\japhug{fka}{order} \\
%\ipa{kʰ}  & 	  & 	  & 	  & 	 \\
\ipa{g}  & 	 \deux{wg} \tib{} & 	\japhug{βgoz}{he prepares it} \\
\midrule
&	\trois{wxt}  &	\japhug{wxti}{it is big} \\
&	\trois{wst} \tib{} &	\japhug{fstɯn}{he serves him} \\
&	\trois{wrt}  \tib{} &	\japhug{frtɤn}{he is trustworthy} \\
&	\trois{wsk}  \tib{} &	\japhug{fskɤr}{he goes around it} \\
&	\trois{wzg}  \tib{} &	\japhug{βzgɤr}{he delays it} \\
&	\trois{wzd}  \tib{} &	\japhug{βzdɯ-nɯ}{they collect it} \\
&	\trois{wzɟ}  \tib{} &	\japhug{βzɟɯr}{he transforms it} \\
&	\trois{wrɟ}  \tib{} &	\japhug{βrɟaŋ}{he stretches it (skin)} \\					
\lspbottomrule
\end{tabular} 
\end{table}
\resetcounters{2wC}{3wC}

 \begin{table}
 \caption{List of consonant clusters with \ipa{s} or \ipa{z} as a first element (23+0)}  \centering \label{prein.sz}
\begin{tabular}{l|ll}
  \lsptoprule
\ipa{p}  & 	 \deux{sp}  & \japhug{spoz}{incense} 	  \\
%\ipa{pʰ}  & 	  & 	  & 	  & 	  & 	  & 	  \\
\ipa{b}  & 	 	 \deux{zb}  & \japhug{zbaʁ}{dry} \\
\ipa{mb}  & 	 \deux{zmb}  & \japhug{tɤzmbɯr}{silt}  \\
\ipa{m}  & 	 \deux{sm}  & \japhug{smar}{river} \\
& 	 \deux{zm}  & \japhug{zmɤrɤβ}{he eat it with} \\
\ipa{t}  & 	 \deux{st}  & \japhug{staχpɯ}{pea} 	  \\
\ipa{tʰ}  & 	 \deux{stʰ}  & \japhug{stʰaβ}{he touches it} 	  \\
\ipa{d}  & 	   \deux{zd}  & \japhug{zdɯm}{cloud} \\
\ipa{nd}  & 	 	 \deux{znd}  & \japhug{znde}{wall} \\
\ipa{n}  & 	 \deux{sn}  & \japhug{sna}{it is worth} \\
&	 \deux{zn}  & \japhug{znɤje}{he feels sorry} \\
%\ipa{ts}  & 	  & 	  & 	  & 	  & 	  & 	  \\
%\ipa{tsʰ}  & 	  & 	  & 	  & 	  & 	  & 	  \\
%\ipa{dz}  & 	  & 	  & 	  & 	  & 	  & 	  \\
%\ipa{ndz}  & 	  & 	  & 	  & 	  & 	  & 	  \\
%\ipa{s}  & 	  & 	  & 	  & 	  & 	  & 	  \\
%\ipa{z}  & 	  & 	  & 	  & 	  & 	  & 	  \\
%\ipa{ɬ}  & 	  & 	  & 	  & 	  & 	  & 	  \\
%\ipa{tɕ}  & 	  & 	  & 	  & 	  & 	  & 	  \\
%\ipa{tɕʰ}  & 	  & 	  & 	  & 	  & 	  & 	  \\
%\ipa{dʑ}  & 	  & 	  & 	  & 	  & 	  & 	  \\
%\ipa{ndʑ}  & 	  & 	  & 	  & 	  & 	  & 	  \\
%\ipa{ɕ}  & 	  & 	  & 	  & 	  & 	  & 	  \\
%\ipa{ʑ}  & 	  & 	  & 	  & 	  & 	  & 	  \\
%\ipa{tʂ}  & 	  & 	  & 	  & 	  & 	  & 	  \\
%\ipa{tʂʰ}  & 	  & 	  & 	  & 	  & 	  & 	  \\
%\ipa{dʐ}  & 	  & 	  & 	  & 	  & 	  & 	  \\
%\ipa{ndʐ}  & 	  & 	  & 	  & 	  & 	  & 	  \\
%\ipa{ʂ}  & 	  & 	  & 	  & 	  & 	  & 	  \\
\ipa{c}  & 	 \deux{sc}  & \japhug{scoʁ}{scoop} 	  \\
\ipa{cʰ}  & 	 \deux{scʰ}  & \japhug{scʰɤt}{it recedes (water)} 	  \\
\ipa{ɟ}  & 	   	 \deux{zɟ}  & \japhug{kɯ-nɯzɟɯ}{suffering losses} \\
\ipa{ɲɟ}  & 		 \deux{zɲɟ}  & \japhug{zɲɟa}{plant sp.} \\
\ipa{ɲ}  & 	 \deux{sɲ}  & \japhug{sɲaŋne}{fasting} 	  \\
%\midrule  \\
\ipa{k}  & 	 \deux{sk}  & \japhug{skɤm}{ox}  	  \\
\ipa{kʰ}  & 	 \deux{skʰ}  & \japhug{rɟɤskʰi}{pan}   \\
\ipa{g}  & 		 \deux{zg}  & \japhug{zga}{sauce} \\
\ipa{ŋg}  & 		  	 \deux{zŋg}  & \japhug{kɤ-ɤkʰɤzŋga}{to call} \\
\ipa{ŋ}  & 	 \deux{sŋ}  & \japhug{sŋaʁ}{he curses him} 	  \\
%\ipa{x}  & 	  & 	  & 	  & 	  & 	  & 	  \\
\ipa{q}  & 	 \deux{sq}  & \japhug{sqamnɯz}{twelve} 	  \\
\ipa{qʰ}  & 	 \deux{sqʰ}  & \japhug{sqʰi}{tripod} 	  \\
%\ipa{ɴɢ}  & 	  & 	  & 	  & 	  & 	  & 	  \\
%\ipa{χ}  & 	  & 	  & 	  & 	  & 	  & 	  \\
\lspbottomrule
\end{tabular} 
\end{table}
\resetcounters{2szC}{3szC}


\begin{table}
 \caption{List of consonant clusters with \ipa{l}  as a first element (17+1)} \label{prein.l}  \centering
\begin{tabular}{l|lll}
\lsptoprule
\ipa{p}   & 	 	 \deux{lp}   & \japhug{tɯ-lpɤɣ}{one piece}  \\ 
%\ipa{pʰ}   & 	 	   & 	 	   & 	 	   \\ 
%\ipa{b}   & 	 	   & 	 	   & 	 	   \\ 
%\ipa{mb}   & 	 	   & 	 	   & 	 	   \\ 
\ipa{m}   & 	 	 \deux{lm}   & \japhug{tɤlmɯz}{straw covering the balcony}  \\ 
\ipa{t}   & 	 	 \deux{lt}   & \japhug{ltɤβ}{he  folds it}  \\ 
\ipa{tʰ}   & 	 	 \deux{ltʰ} \idph{}   & \japhug{ltʰɯmɯmi}{coming slowly (sleep)}  \\ 
\ipa{d}   & 	 	 \deux{ld}   & \japhug{ldɯɣi}{bharal  }  \\ 
%\ipa{nd}   & 	 	   & 	 	   & 	 	   \\ 
\ipa{n}   & 	 	 \deux{ln}   & \japhug{lni}{it withers   }  \\ 
\ipa{ts}   & 	 	 \deux{lts}   & \japhug{ɕɤltsaʁ}{leather coat}  \\ 
\ipa{tsʰ}   & 	 	 \deux{ltsʰ} \idph{}   & \japhug{ltshɤltshɤt}{small and weak}  \\ 
%\ipa{dz}   & 	 	   & 	 	   & 	 	   \\ 
%\ipa{ndz}   & 	 	   & 	 	   & 	 	   \\ 
%\ipa{s}   & 	 	   & 	 	   & 	 	   \\ 
%\ipa{z}   & 	 	   & 	 	   & 	 	   \\ 
%\ipa{ɬ}   & 	 	   & 	 	   & 	 	   \\ 
\ipa{tɕ}   & 	 	 \deux{ltɕ}   & \japhug{rtɤltɕaʁ}{horse whip   }  \\ 
\ipa{tɕʰ}   & 	 	 \deux{ltɕʰ} \idph{}   & \japhug{ltɕʰɤltɕʰɤt}{hanging (of fluffy objects)   }  \\ 
\ipa{dʑ}   & 	 	 \deux{ldʑ} \tib{}   & \japhug{ldʑaŋkɯ}{green  }  \\ 
%\ipa{ndʑ}   & 	 	   & 	 	   & 	 	   \\ 
%\ipa{ɕ}   & 	 	   & 	 	   & 	 	   \\ 
%\ipa{ʑ}   & 	 	   & 	 	   & 	 	   \\ 
%\ipa{tʂ}   & 	 	   & 	 	   & 	 	   \\ 
%\ipa{tʂʰ}   & 	 	   & 	 	   & 	 	   \\ 
\ipa{dʐ}   & 	 	 \deux{ldʐ} \idph{}   & \japhug{ldʐaŋldʐaŋ}{hanging (big object)}  \\ 
%\ipa{ndʐ}   & 	 	   & 	 	   & 	 	   \\ 
%\ipa{ʂ}   & 	 	   & 	 	   & 	 	   \\ 
\ipa{c}   & 	 	 \deux{lc} \idph{}  & \japhug{lcɯɣlcɯɣ}{drenching}  \\ 
\ipa{cʰ}   & 	 	 \deux{lcʰ}   & \japhug{tɯ-lcʰɯɣ}{section (of a bag)}  \\ 
%\ipa{ɟ}   & 	 	   & 	 	   & 	 	   \\ 
%\ipa{ɲɟ}   & 	 	   & 	 	   & 	 	   \\ 
%\ipa{ɲ}   & 	 	   & 	 	   & 	 	   \\ 
%\ipa{k}   & 	 	   & 	 	   & 	 	   \\ 
%\ipa{kʰ}   & 	 	   & 	 	   & 	 	   \\ 
%\ipa{g}   & 	 	   & 	 	   & 	 	   \\ 
%\ipa{ŋg}   & 	 	   & 	 	   & 	 	   \\ 
\ipa{ŋ}   & 	 	 \deux{lŋ} \idph{}   & \japhug{lŋɤlŋɤt}{hanging (fruit)}  \\ 
\ipa{x}   & 	 	 \deux{lx} \idph{}   & \japhug{lxɤβlxɤβ}{thick (clothes)}  \\ 
\ipa{q}   & 	 	\deux{lq}   & 	 \japhug{lqɤnɤlqɤt}{toddling}	   \\ 
%\ipa{qʰ}   & 	 	   & 	 	   & 	 	   \\ 
%\ipa{ɴɢ}   & 	 	   & 	 	   & 	 	   \\ 
%\ipa{χ}   & 	 	   & 	 	   & 	 	   \\ 
% & 	 & 	 & 	 \\ 
\midrule
&\trois{lpɕ}	&\japhug{qalpɕa}{it opens (fern leaf)} \\
\lspbottomrule
\end{tabular}
\end{table}
\resetcounters{2lC}{3lC} %deux

  \begin{table}
 \caption{List of consonant clusters with \ipa{r} and \ipa{ʂ}  as a first element (35+0)} \label{prein.r}  \centering
\begin{tabular}{l|lll|lll|lll|l}
\lsptoprule
\ipa{p}   & \deux{ʂp}   & \japhug{tɯ-rpa}{axe}  \\ 
\ipa{pʰ}   & \deux{ʂpʰ} \idph{}   & \japhug{rpʰɤβrpʰɤβ}{flapping wings}  \\ 
%\ipa{b}   &   \\ 
\ipa{mb}   & \deux{rmb}   & \japhug{armbat}{near}  \\ 
\ipa{m}   & \deux{rm}   & \japhug{rmɤβja}{peacock}  \\ 
\ipa{t}   & \deux{ʂt}   & \japhug{rtalu}{horse year}  \\ 
\ipa{tʰ}   & \deux{ʂtʰ}   & \japhug{ɯ-pɤrtʰɤβ}{middle}  \\ 
\ipa{d}   & \deux{rd}   & \japhug{rdɤstaʁ}{stone}  \\ 
\ipa{nd}   & \deux{rnd}   & \japhug{rnde}{he finds it}  \\ 
\ipa{n}   & \deux{rn}   & \japhug{rnaʁ}{it is deep}  \\ 
\ipa{ts}   & \deux{ʂts}   & \japhug{rtsot}{vengeance}  \\ 
\ipa{tsʰ}   & \deux{ʂtsʰ}   & \japhug{rtsʰom}{it has a crack (bucket)}  \\ 
\ipa{dz}   & \deux{rdz} \idph{}   & \japhug{rdzardza}{insolent}  \\ 
\ipa{ndz}   & \deux{rndz}   & \japhug{rndzɤkɤŋe}{shade of the mountain}  \\ 
\ipa{s}   & \deux{ʂs} \idph{}   & \japhug{rsɯβrsɯβ}{hairy}  \\ 
\ipa{z}   & \deux{rz}   & \japhug{tɯ-rzɯɣ}{one section}  \\ 
%\ipa{ɬ}   &  &  &  \\ 
\ipa{tɕ}   & \deux{ʂtɕ}   & \japhug{nɯrtɕe}{he teases him}  \\ 
\ipa{tɕʰ}   & \deux{ʂtɕʰ}   & \japhug{rtɕʰɯʁjɯ}{caterpillar}  \\ 
%\ipa{dʑ}   &  &  &  \\ 
\ipa{ndʑ}   & \deux{rndʑ}   & \japhug{cɯrndʑi}{sand}  \\ 
\ipa{ɕ}   & \deux{ʂɕ}   & \japhug{rɕɯβrɕɯβ}{rough}  \\ 
\ipa{ʑ}   & \deux{rʑ}   & \japhug{tɤ-rʑaβ}{wife}  \\ 
%\ipa{tʂ}   &  &  &  \\ 
%\ipa{tʂʰ}   &  &  &  \\ 
%\ipa{dʐ}   &  &  &  \\ 
%\ipa{ndʐ}   &  &  &  \\ 
%\ipa{ʂ}   &  &  &  \\ 
\ipa{c}   & \deux{ʂc}   & \japhug{tɤ-rcoʁ}{mud}  \\ 
\ipa{cʰ}   & \deux{ʂcʰ}   & \japhug{ɯ-rcʰarcʰɤβ}{interstice}  \\ 
\ipa{ɟ}   & \deux{rɟ}   & \japhug{rɟaʁ}{he dances}  \\ 
\ipa{ɲɟ}   & \deux{rɲɟ}   & \japhug{rɲɟaʁlo}{bolt}  \\ 
\ipa{ɲ}   & \deux{ʂɲ} \idph{}   & \japhug{ʂɲoʁʂɲoʁ}{long and thin}  \\ 
    & \deux{rɲ} \tib{}   & \japhug{rɲaŋ}{it is ancient}  \\ 
\ipa{k}   & \deux{ʂk}   & \japhug{rko}{it is hard}  \\ 
\ipa{kʰ}   & \deux{ʂkʰ}   & \japhug{tɤ-rkʰom}{feather rachis}  \\ 
\ipa{g}   & \deux{rg}   & \japhug{rga}{he likes it}  \\ 
\ipa{ŋg}   & \deux{rŋg}   & \japhug{rŋgɤm}{hard piece}  \\ 
\ipa{ŋ}   & \deux{rŋ}   & \japhug{tɯ-rŋa}{face}  \\ 
%\ipa{x}   &  &  &  \\ 
\ipa{q}   & \deux{ʂq}   & \japhug{rqoʁ}{he hugs him}  \\ 
\ipa{qʰ}   & \deux{ʂqʰ}   & \japhug{tɤ-rqʰu}{bark, skin}  \\ 
\ipa{ɴɢ}   & \deux{rɴɢ}   & \japhug{ɕɯrɴɢo}{Anisodus tanguticus}  \\ 
\ipa{χ}   & \deux{ʂχ}   & \japhug{ʂχɯʂχi}{with big nostrils}  \\ 
\lspbottomrule
\end{tabular}
\end{table}
\resetcounters{2rC}{3rC} %deux 

   \begin{table}
 \caption{List of consonant clusters with \ipa{ɕ} and \ipa{ʑ}  as a first element (18+0)} \label{prein.C.Z}  \centering
\begin{tabular}{l|ll}
\lsptoprule
\ipa{p} & \deux{ɕp} & \japhug{ɕpaʁ}{he is thirsty} \\ 
\ipa{pʰ} & \deux{ɕpʰ} & \japhug{ɕpʰɤt}{he patches it} \\ 
%\ipa{b} & & & \\ 
\ipa{mb} & \deux{ʑmb} & \japhug{ʑmbɤr}{ulcer} \\ 
\ipa{m} & \deux{ɕm} & \japhug{ɕmi}{he mixes it} \\ 
\ipa{t} & \deux{ɕt} & \japhug{ɕte}{he contaminates him} \\ 
\ipa{tʰ} & \deux{ɕtʰ} & \japhug{ɕtʰɯz}{he turns towards} \\ 
\ipa{d} & \deux{ʑd} \idph{} & \japhug{ʑdɯɣʑdɯɣ}{strong, tough} \\ 
%\ipa{nd} & & & \\ 
\ipa{n} & \deux{ɕn} & \japhug{ɕnat}{weaving implement} \\ 
  & \deux{ʑn} & \japhug{ʑ-nɯ-ɕar}{go and look for it} \\ 
%\ipa{ts} & & & \\ 
%\ipa{tsʰ} & & & \\ 
%\ipa{dz} & & & \\ 
%\ipa{ndz} & & & \\ 
%\ipa{s} & & & \\ 
%\ipa{z} & & & \\ 
%\ipa{ɬ} & & & \\ 
%\ipa{tɕ} & & & \\ 
%\ipa{tɕʰ} & & & \\ 
%\ipa{dʑ} & & & \\ 
%\ipa{ndʑ} & & & \\ 
%\ipa{ɕ} & & & \\ 
%\ipa{ʑ} & & & \\ 
\ipa{tʂ} & \deux{ɕtʂ} \idph{} & \japhug{ɕtʂaŋlaŋ}{hanging and swinging} \\ 
%\ipa{tʂʰ} & & & \\ 
%\ipa{dʐ} & & & \\ 
%\ipa{ndʐ} & & & \\ 
%\ipa{ʂ} & & & \\ 
%\ipa{c} & & & \\ 
%\ipa{cʰ} & & & \\ 
%\ipa{ɟ} & & & \\ 
%\ipa{ɲɟ} & & & \\ 
%\ipa{ɲ} & & & \\ 
\ipa{k} &   \deux{ɕk} & \japhug{ɕkom}{muntjac} \\ 
\ipa{kʰ} &   \deux{ɕkʰ} & \japhug{ɕkʰo-nɯ}{they spread it} \\ 
\ipa{g} &   \deux{ʑg} & \japhug{ʑgaʁ}{exactly} \\ 
\ipa{ŋg} &   \deux{ʑŋg} & \japhug{ʑŋgu}{he crosses river on boat} \\ 
\ipa{ŋ} &   \deux{ɕŋ} \idph{} & \japhug{ɕŋaʁɕŋaʁ}{bright yellow} \\ 
%\ipa{x} & 	 & & \\ 
\ipa{q} &   \deux{ɕq} & \japhug{ɕqɤjɤr}{cross-eyed} \\ 
\ipa{qʰ} &   \deux{ɕqʰ} & \japhug{ɕqʰaloʁ}{latch} \\ 
\ipa{ɴɢ} &   \deux{ʑɴɢ} & \japhug{ʑɴɢɯloʁ}{walnut} \\ 
%\ipa{χ} & 	 & & \\ 
\lspbottomrule
\end{tabular}
\end{table}
\resetcounters{2CZC}{3CZC} %deux 


   \begin{table}
 \caption{List of consonant clusters with \ipa{j}  as a first element (12+1)} \label{prein.j}  \centering
\begin{tabular}{l|lll|lll|l}
\lsptoprule
\ipa{p}   & 	 	 \deux{jp}   & \japhug{jpum}{it is thick}  \\ 
%\ipa{pʰ}   & 	 	   & 	 	   & 	 	   \\ 
%\ipa{b}   & 	 	   & 	 	   & 	 	   \\ 
%\ipa{mb}   & 	 	   & 	 	   & 	 	   \\ 
\ipa{m}   & 	 	 \deux{jm}   & \japhug{jmɯt}{he forgets it}  \\ 
\ipa{t}   & 	 	 \deux{jt}   & \japhug{ajtɯ}{it accumulates}  \\ 
%\ipa{tʰ}   & 	 	   & 	 	   & 	 	   \\ 
%\ipa{d}   & 	 	   & 	 	   & 	 	   \\ 
%\ipa{nd}   & 	 	   & 	 	   & 	 	   \\ 
\ipa{n}   & 	 	 \deux{jn}   & \japhug{jnom}{it is flexible}  \\ 
\ipa{ts}   & 	 	 \deux{jts}   & \japhug{tɤ-jtsi}{pillar}  \\ 
\ipa{tsʰ}   & 	 	 \deux{jtsʰ}   & \japhug{jtsʰi}{he gives him to drink}  \\ 
%\ipa{dz}   & 	 	   & 	 	   & 	 	   \\ 
%\ipa{ndz}   & 	 	   & 	 	   & 	 	   \\ 
%\ipa{s}   & 	 	   & 	 	   & 	 	   \\ 
%\ipa{z}   & 	 	   & 	 	   & 	 	   \\ 
%\ipa{ɬ}   & 	 	   & 	 	   & 	 	   \\ 
%\ipa{tɕ}   & 	 	   & 	 	   & 	 	   \\ 
%\ipa{tɕʰ}   & 	 	   & 	 	   & 	 	   \\ 
%\ipa{dʑ}   & 	 	   & 	 	   & 	 	   \\ 
%\ipa{ndʑ}   & 	 	   & 	 	   & 	 	   \\ 
%\ipa{ɕ}   & 	 	   & 	 	   & 	 	   \\ 
%\ipa{ʑ}   & 	 	   & 	 	   & 	 	   \\ 
%\ipa{tʂ}   & 	 	   & 	 	   & 	 	   \\ 
\ipa{tʂʰ}   & 	 	 \deux{jtʂʰ}   & \japhug{qajtʂʰa}{vulture}  \\ 
%\ipa{dʐ}   & 	 	   & 	 	   & 	 	   \\ 
\ipa{ndʐ}   & 	 	 \deux{jndʐ}   & \japhug{jndʐɤz}{it is thick (powder)}  \\ 
%\ipa{ʂ}   & 	 	   & 	 	   & 	 	   \\ 
%\ipa{c}   & 	 	   & 	 	   & 	 	   \\ 
%\ipa{cʰ}   & 	 	   & 	 	   & 	 	   \\ 
%\ipa{ɟ}   & 	 	   & 	 	   & 	 	   \\ 
%\ipa{ɲɟ}   & 	 	   & 	 	   & 	 	   \\ 
%\ipa{ɲ}   & 	 	   & 	 	   & 	 	   \\ 
\ipa{k}   & 		 \deux{jk}   & \japhug{tɤ-jkɯz}{secret}  \\ 
%\ipa{kʰ}   & 		   & 		   & 		   \\ 
%\ipa{g}   & 		   & 		   & 		   \\ 
%\ipa{ŋg}   & 		   & 		   & 		   \\ 
\ipa{ŋ}   & 		 \deux{jŋ}   & \japhug{tɤ-jŋoʁ}{hook}  \\ 
%\ipa{x}   & 		   & 		   & 		   \\ 
\ipa{q}   & 		 \deux{jq}   & \japhug{jqe}{he is able to lift it}  \\ 
%\ipa{qʰ}   & 		   & 		   & 		   \\ 
%\ipa{ɴɢ}   & 		   & 		   & 		   \\ 
\ipa{χ}   & 		 \deux{jχ}   & \japhug{ajχoʁ}{it is flat (belly)}  \\ 
&\trois{jmŋ} & \japhug{tɯ-jmŋo}{dream (n)} \\  
\lspbottomrule
\end{tabular}
\end{table}
   \resetcounters{2jC}{3jC} 

 \begin{table}
 \caption{List of consonant clusters with  \ipa{x} and \ipa{ɣ} as a first element (23+0)} \label{prein.x}  \centering
\begin{tabular}{l|lll}
\lsptoprule
\ipa{p}	 & 	 	 \deux{xp}	 & \japhug{tɯ-xpa}{one year} \\ 
%\ipa{pʰ}	 & 		 & 		 & 		 \\ 
%\ipa{b}	 & 		 & 		 & 		 \\ 
\ipa{mb}	 & 	 	 \deux{ɣmb}	 & \japhug{tɯ-ɣmba}{cheek}  \\ 
\ipa{m}	 & 	 	 \deux{ɣm}	 & \japhug{tɯ-ɣmaz}{wound}  \\ 
\ipa{t}	 & 	 	 \deux{xt}	 & \japhug{xtɯt}{wild cat}  \\ 
\ipa{tʰ}	 & 	 	 \deux{xtʰ}	 & \japhug{xtʰom}{he puts it horizontally}  \\ 
\ipa{d}	 & 	 	 \deux{ɣd}	 & \japhug{ɣdɤso}{species of grub}  \\ 
\ipa{nd}	 & 	 	 \deux{ɣnd}	 & \japhug{ɣnde}{he hits with a hammer}  \\ 
\ipa{n}	 & 	 	 \deux{ɣn}	 & \japhug{ɣnɤsqi}{twenty}  \\ 
\ipa{ts}	 & 	 	 \deux{xts}	 & \japhug{xtsɤɕna}{tip of boot}  \\ 
\ipa{tsʰ}	 & 	 	 \deux{xtsʰ}	 & \japhug{xtsʰɯm}{it is thin}  \\ 
%\ipa{dz} 	 & 		 & 		 & 		 \\ 
%\ipa{ndz}	 & 		 & 		 & 		 \\ 
\ipa{s}	 & 	 	 \deux{xs}	 & \japhug{xsar}{goral}  \\ 
\ipa{z}	 & 	 	 \deux{ɣz}	 & \japhug{ɣzɯ}{monkey}  \\ 
%\ipa{ɬ} 	 & 		 & 		 & 		 \\ 
\ipa{tɕ}	 & 	 	 \deux{xtɕ}	 & \japhug{xtɕi}{it is small}  \\ 
\ipa{tɕʰ}	 & 	 	 \deux{xtɕʰ}	 & \japhug{xtɕʰɯt}{it can hold}  \\ 
%\ipa{dʑ} 	 & 		 & 		 & 		 \\ 
\ipa{ndʑ}	 & 	 	 \deux{ɣndʑ}	 & \japhug{ɣndʑɤβ}{fire}  \\ 
\ipa{ɕ}	 & 	 	 \deux{xɕ}	 & \japhug{xɕaj}{grass}  \\ 
\ipa{ʑ}	 & 	 	 \deux{ɣʑ}	 & \japhug{ɣʑo}{bee}  \\ 
\ipa{tʂ}	 & 	 	 \deux{xtʂ}	 & \japhug{nɤxtʂi}{he will bring it with him}  \\ 
%\ipa{tʂʰ}	 & 		 & 		 & 		 \\ 
%\ipa{dʐ}	 & 		 & 		 & 		 \\ 
%\ipa{ndʐ}	 & 		 & 		 & 		 \\ 
\ipa{ʂ}	 & 	 	 \deux{xʂ} \idph{}	 & \japhug{xʂɤxʂɤt}{long and thin}  \\ 
\ipa{c}	 & 	 	 \deux{xc}	 & \japhug{xcat}{many}  \\ 
\ipa{cʰ}	 & 	 	 \deux{xcʰ}	 & \japhug{tɤlɤxcʰi}{curdled milk}  \\ 
\ipa{ɟ}	 & 	 	 \deux{ɣɟ}	 & \japhug{ɣɟaβ}{he will churn (milk)}  \\ 
%\ipa{ɲɟ}	 &	 & 	 	 	 & 	 	 	 \\ 
\ipa{ɲ}	 & 	 	 \deux{ɣɲ}	 & \japhug{ɯ-ɣɲaʁ}{disaster}  \\ 
\lspbottomrule
\end{tabular}
\end{table}
\resetcounters{2xGC}{3xGC} %deux 


 \begin{table}
 \caption{List of consonant clusters with  \ipa{χ} and \ipa{ʁ} as a first element (25+0)} \label{prein.X.R}  \centering
\begin{tabular}{l|llllllll}
\lsptoprule
\ipa{p}	 &	   \deux{χp} \tib{}	 & \japhug{χpi}{story}  &	   		 \\
\ipa{pʰ}	 &	 	 \deux{χpʰ}	 & \japhug{taχpʰe}{slap}  &	   	 \\
\ipa{b}	 &	\deux{ʁb}  \idph{}	 & \japhug{ʁbɤʁbɤβ}{thick and big}  \\
\ipa{mb}	 &	 	  \deux{ʁmb}	 & \japhug{aʁmbɯm}{concave}  \\
\ipa{m}	 &	 	 \deux{ʁm}	 & \japhug{ʁmaʁ}{army}  \\
\ipa{t}	 &	 	 \deux{χt}	 & \japhug{χtɤrma}{offerings}  &	   	 \\
\ipa{tʰ}	 &	 	 \deux{χtʰ}	 & \japhug{naχthɤβ}{he seizes the opportunity}  &	  	 \\
\ipa{d}	 &	 	 \deux{ʁd} \tib{} 	 & \japhug{ʁdɯɣ}{it is serious}  \\
\ipa{nd}	 &	 	 \deux{ʁnd}	 & \japhug{ʁndɤr}{it scatters}  \\
\ipa{n}	 &	 \deux{ʁn}	 & \japhug{ʁnaʁna}{both}  \\
\ipa{ts}	 &	 	 \deux{χts}	 & \japhug{χtso}{it is clean}  &	   	 \\
\ipa{tsʰ}	 &	 	 \deux{χtsʰ} \idph{}	 & \japhug{χtsʰɤχtsʰɤt}{small and active}  &	  	 \\
%\ipa{dz} 	 &	 	    \\
\ipa{ndz}	 &		 \deux{ʁndz}	 & \japhug{ʁndzɤr}{he cuts it (with scissors)}  \\
\ipa{s}	 &	 	 \deux{χs}	 & \japhug{χsɤr}{gold}  &	  	 \\
\ipa{z}	 &	 \deux{ʁz}  \tib{}	 &	   \japhug{ʁzɤβ}{he is careful in}   		 \\
%\ipa{ɬ} 	 &	 	    \\
\ipa{tɕ}	 &	 	 \deux{χtɕ}  \tib{} 	 & \japhug{χtɕoŋ}{rheumatism}  &	   	 \\
%\ipa{tɕʰ}	 &	   		 \\
%\ipa{dʑ} 	 &	 	    \\
%\ipa{ndʑ}	 &		   	 \\
\ipa{ɕ}	 &	 	 \deux{χɕ}	 & \japhug{χɕu}{it is strong}  &	  	 \\
\ipa{ʑ}	 &	\ipa{ʁʑ}  &\japhug{ʁʑɯnɯ}{young man}   		 \\
\ipa{tʂ}	 &	 	 \deux{χtʂ}  \tib{}	 & \japhug{χtʂɯɣdʑa}{butter tea}  &	  	 \\
%\ipa{tʂʰ}	 &		   	 \\
%\ipa{dʐ}	 &		   	 \\
%\ipa{ndʐ}	 &		   	 \\
\ipa{ʂ}	 &	 	 \deux{χʂ} \idph{}	 & \japhug{χʂɤχʂɤt}{light (clothes)}  &	   	 \\
\ipa{c}	 &	 	 \deux{χc} \tib{}	 & \japhug{χcoŋkroŋ}{cross-legged (sitting)}  &	  	 \\
\ipa{cʰ}	 &	 	 \deux{χcʰ}	 & \japhug{χcʰa}{right}  &	  	 \\
\ipa{ɟ}	 &		 \deux{ʁɟ}	 & \japhug{ʁɟa}{completely}  \\
\ipa{ɲɟ}	 &		 \deux{ʁɲɟ}	 & \japhug{ʁɲɟiʁɲɟi}{enormous}  \\
\ipa{ɲ}	 &	 	\deux{χɲ} \idph{}	 & \japhug{χɲɤχɲɤr}{without energy} \\
& \deux{ʁɲ}\tib{}	 & \japhug{ʁɲɤrpa}{steward (monastery)}  \\
\lspbottomrule
\end{tabular}
\end{table}
\resetcounters{2XRC}{3XRC} %deux 

 \begin{table} %ɴqiaβ
 \caption{List of consonant clusters with a homorganic nasal as  first element (14+1)} \label{prein.nasal}  \centering
\begin{tabular}{l|lll}
\lsptoprule
\ipa{p} 	 &	 \deux{mp} 	 & \japhug{mpɯ}{it is soft} \\	
\ipa{pʰ} 	 &	 \deux{mpʰ} 	 & \japhug{mpʰɯl}{it reproduces} \\	
%\ipa{b} 	 &	 	 &	 	 &	 	 \\	
%\ipa{mb} 	 &	 	 &	 	 &	 	 \\	
%\ipa{m} 	 &	 	 &	 	 &	 	 \\	
\ipa{t} 	 &	 \deux{nt} 	 & \japhug{ntaw}{it is stable} \\	
\ipa{tʰ} 	 &	 \deux{ntʰ} 	 & \japhug{ntʰɤβ}{it is caught between} \\	
%\ipa{d} 	 &	 	 &	 	 &	 	 \\	
%\ipa{nd} 	 &	 	 &	 	 &	 	 \\	
%\ipa{n} 	 &	 	 &	 	 &	 	 \\	
\ipa{ts} 	 &	 \deux{nts} 	 & \japhug{ntsɯ}{always} \\	
\ipa{tsʰ} 	 &	 \deux{ntsʰ} 	 & \japhug{ntsʰɤr}{it neighs} \\	
%\ipa{dz} 	 &	 	 &	 	 &	 	 \\	
%\ipa{ndz} 	 &	 	 &	 	 &	 	 \\	
%\ipa{s} 	 &	 	 &	 	 &	 	 \\	
%\ipa{z} 	 &	 	 &	 	 &	 	 \\	
%\ipa{ɬ} 	 &	 	 &	 	 &	 	 \\	
%\ipa{tɕ} 	 &	 	 &	 	 &	 	 \\	
\ipa{tɕʰ} 	 &	 \deux{ntɕʰ} 	 & \japhug{ntɕʰoz}{he uses it} \\	
%\ipa{dʑ} 	 &	 	 &	 	 &	 	 \\	
%\ipa{ndʑ} 	 &	 	 &	 	 &	 	 \\	
%\ipa{ɕ} 	 &	 	 &	 	 &	 	 \\	
%\ipa{ʑ} 	 &	 	 &	 	 &	 	 \\	
\ipa{tʂ} 	 &	 \deux{ntʂ} 	 & \japhug{ntʂu-nɯ}{they weed} \\	
%\ipa{tʂʰ} 	 &	 	 &	 	 &	 	 \\	
%\ipa{dʐ} 	 &	 	 &	 	 &	 	 \\	
%\ipa{ndʐ} 	 &	 	 &	 	 &	 	 \\	
%\ipa{ʂ} 	 &	 	 &	 	 &	 	 \\	
\ipa{c} 	 &	 \deux{ɲc} 	 & \japhug{ɲcɤr}{he presses on} \\	
\ipa{cʰ} 	 &	 \deux{ɲcʰ} 	 & \japhug{ɲcʰoʁ}{it shrinks} \\	
%\ipa{ɟ} 	 &	 	 &	 	 &	 	 \\	
%\ipa{ɲɟ} 	 &	 	 &	 	 &	 	 \\	
%\ipa{ɲ} 	 &	 	 &	 	 &	 	 \\	
\ipa{k} 	 &	 \deux{ŋk} 	 & \japhug{ŋke}{he walks} \\	
\ipa{kʰ} 	 &	 \deux{ŋkʰ} 	 & \japhug{ŋkʰor}{he arrives} \\	
%\ipa{g} 	 &	 	 &	 	 &	 	 \\	
%\ipa{ŋg} 	 &	 	 &	 	 &	 	 \\	
%\ipa{ŋ} 	 &	 	 &	 	 &	 	 \\	
%\ipa{x} 	 &	 	 &	 	 &	 	 \\	
\ipa{q} 	 &	   \deux{ɴq} 	 & \japhug{ɴqa}{it is difficult} \\	
\ipa{qʰ}   	 &	   \deux{ɴqʰ} 	 & \japhug{ɴqʰi}{it is dirty} \\	
%\ipa{ɴɢ}  	 &	 	 &		 &		 \\	
\midrule
&\trois{mpɕ} &\japhug{mpɕɤr}{it is beautiful} \\
\lspbottomrule
\end{tabular}
\end{table}		
\resetcounters{2NC}{3NC}

 \begin{table} 
 \caption{List of consonant clusters with a non-homorganic nasal as  first element (24+0)} \label{prein.nh.nasal}  \centering
\begin{tabular}{l|lll}
\lsptoprule
%\ipa{p} & & & \\
%\ipa{pʰ} & & & \\
%\ipa{b} &\\
\ipa{mb} &  \deux{nb} 	& \japhug{anbaʁ}{he hides}  \\
%\ipa{m} & & & \\
\ipa{t} & \deux{mt} & \japhug{tɤ-mtɯ}{knot} \\
\ipa{tʰ} & \deux{mtʰ}\tib{} & \japhug{mtʰɯ}{spell} \\
%\ipa{d} & & & \\
\ipa{nd} & \deux{md} & \japhug{mda}{it reaches} \\
\ipa{n} & \deux{mn} & \japhug{mna}{it heals} \\
\ipa{ts} & \deux{mts} & \japhug{tɤ-mtsɯ}{button} \\
\ipa{tsʰ} & \deux{mtsʰ} & \japhug{mtsʰɤm}{he hears} \\
%\ipa{dz} & & & \\
\ipa{ndz} & \deux{mdz} & \japhug{mdzadi}{flea} \\
%\ipa{s} & & & \\
%\ipa{z} & & & \\
%\ipa{ɬ} & & & \\
\ipa{tɕ} & \deux{mtɕ} & \japhug{mtɕoʁ}{it is sharp} \\
\ipa{tɕʰ} & \deux{mtɕʰ} & \japhug{tɤ-mtɕʰo}{wedge} \\
%\ipa{dʑ} & & & \\
\ipa{ndʑ} & \deux{mdʑ} & \japhug{tɯ-mdʑu}{tongue} \\
%\ipa{ɕ} & & & \\
%\ipa{ʑ} & & & \\
\ipa{tʂ} & \deux{mtʂ} & \japhug{kɯ-ɤrɤmtʂɯmtʂaj}{sticky} \\
%\ipa{tʂʰ} & & & \\
%\ipa{dʐ} & & & \\
\ipa{ndʐ} & \deux{mdʐ} & \japhug{mdʐɯɕɯɣ}{bedbug} \\
%\ipa{ʂ} & & & \\
\ipa{c} & \deux{mc} & \japhug{tɤmcar}{tongs} \\
\ipa{cʰ} & \deux{mcʰ} & \japhug{tɯ-mcʰi}{gall} \\
%\ipa{ɟ} & & & \\
\ipa{ɲɟ} & \deux{mɟ} & \japhug{tɯ-mɟa}{jaw} \\
\ipa{ɲ} & \deux{mɲ} & \japhug{mɲɤm}{species of tree} \\
\ipa{k} & \deux{mk} & \japhug{tɯ-mke}{neck} \\
\ipa{kʰ} & \deux{mkʰ} & \japhug{mkʰɤz}{he is expert} \\
%\ipa{g} & \\
\ipa{ŋg} & \deux{mg} & \japhug{tɯ-mga}{advantage} \\
&\deux{ng} 	& \japhug{ngɯt}{it is strong}\\\
\ipa{ŋ} & \deux{mŋ} & \japhug{mŋɤm}{it hurts} \\
& \deux{nŋ} & \japhug{nŋo-nɯ}{they will lose} \\
%\ipa{x} & & & \\
%\ipa{q} & & & \\
%\ipa{qʰ} & & & \\
\ipa{ɴɢ} & \deux{mɢ} & \japhug{tamɢom}{clamp} \\
\lspbottomrule
\end{tabular}
\end{table}		
\resetcounters{2mnC}{3mnC}

 \begin{table}
 \caption{List of consonant clusters ending in   \ipa{w} (10+0)} \label{med.w}  \centering
 \begin{tabular}{l|lll}
\lsptoprule
%\ipa{p}   &    &    &   \\
%\ipa{pʰ}   &    &    &   \\
%\ipa{b}   &    &    &   \\
%\ipa{mb}   &    &    &   \\
%\ipa{m}   &    &    &   \\
%\ipa{w}   &    &    &   \\
%\ipa{t}   &    &    &   \\
%\ipa{tʰ}   &    &    &   \\
\ipa{d}   & \deux{dw}\idph{}   & \japhug{dwaŋdwaŋ}{out of his head} \\
%\ipa{nd}   &    &    &   \\
%\ipa{n}   &    &    &   \\
%\ipa{ts}   &    &    &   \\
%\ipa{tsʰ}   &    &    &   \\
%\ipa{dz}   &    &    &   \\
%\ipa{ndz}   &    &    &   \\
%\ipa{s}   &    &    &   \\
\ipa{z}   & \deux{zw}   & \japhug{zwɤr}{mugwort} \\
\ipa{l}   & \deux{lw}   & \japhug{lwɤz}{he will be sick again} \\
%\ipa{ɬ}   &    &    &   \\
%\ipa{tɕ}   &    &    &   \\
%\ipa{tɕʰ}   &    &    &   \\
%\ipa{dʑ}   &    &    &   \\
%\ipa{ndʑ}   &    &    &   \\
%\ipa{ɕ}   &    &    &   \\
%\ipa{ʑ}   &    &    &   \\
%\ipa{tʂ}   &    &    &   \\
%\ipa{tʂʰ}   &    &    &   \\
%\ipa{dʐ}   &    &    &   \\
%\ipa{ndʐ}   &    &    &   \\
\ipa{r}   & \deux{rw}\tib{}   & \japhug{rwa}{yak felt tent} \\
\ipa{ʂ}   & \deux{ʂw} \tib{}  & \japhug{aɣɯʂwaŋ}{it comes in pairs} \\
%\ipa{c}   &    &    &   \\
%\ipa{cʰ}   &    &    &   \\
%\ipa{ɟ}   &    &    &   \\
%\ipa{ɲɟ}   &    &    &   \\
%\ipa{ɲ}   &    &    &   \\
\ipa{j}   & \deux{jw}   & \japhug{jwajwa}{very thin} \\
\ipa{k}   & \deux{kw}\tib{}   & \japhug{kwitsɯt}{cupboard} \\
%\ipa{kʰ}   &     &    &   \\
%\ipa{g}   &    &    &   \\
%\ipa{ŋg}   &    &    &   \\
%\ipa{ŋ}   &    &    &   \\
\ipa{x}   & \deux{xw}\idph{}   & \japhug{xwɤrnɤxwɤr}{rotating quickly} \\
%\ipa{ɣ}   &    &    &   \\
%\ipa{q}   &    &    &   \\
%\ipa{qʰ}   &    &    &   \\
%\ipa{ɴɢ}   &    &    &   \\
\ipa{χ}   & \deux{χw} \tib{}  & \japhug{χwɤr}{Hor (name)} \\
%\ipa{ʁ}   &    &    &   \\
\ipa{h}   & \deux{hw} \idph{}  & \japhug{hwɤrhwɤr}{wide-mouthed} \\
\lspbottomrule
\end{tabular}
\end{table}		
\resetcounters{2Cw}{3Cw}

  \begin{table}
 \caption{List of consonant clusters ending in  \ipa{j} (20+18)} \label{med.j}  \centering
 \begin{tabular}{l|lll}
\lsptoprule
\ipa{p}   &    \deux{pj}   & \japhug{pjalu}{year of the cock} \\  
%\ipa{pʰ}   &       &    & \\  
\ipa{b}   &    \deux{bj}\idph{}   & \japhug{bjɯbjɯɣ}{hanging in great number} \\  
\ipa{mb}   &    \deux{mbj}   & \japhug{mbjom}{it is fast} \\  
%\ipa{m}   &       &    & \\  
\ipa{w}   &    \deux{wj}   & \japhug{tɕʰɯβja}{duck} \\  
%\ipa{t}   &       &    & \\  
%\ipa{tʰ}   &       &    & \\  
\ipa{d}   &    \deux{dj} \idph{}  & \japhug{dioʁdioʁ}{evenly mixed} \\  
\ipa{nd}   &    \deux{ndj} \idph{}  & \japhug{ndiɤndiɤt}{gracious} \\  
%\ipa{n}   &       &    & \\  
\ipa{ts}   &     \deux{tsj}   & \japhug{tsiaŋnɤtsiaŋ}{very tall, moving} \\  
%\ipa{tsʰ}   &       &    & \\  
%\ipa{dz}   &       &    & \\  
\ipa{ndz}   &    \deux{ndzj}   & \japhug{ndziaʁ}{it is tight (knot)} \\  
\ipa{s}   &    \deux{sj} \idph{}  & \japhug{sjaŋnɤsjaŋ}{shaking one's head} \\  
\ipa{z}   &    \deux{zj} \idph{}  & \japhug{zjaŋzjaŋ}{big} \\  
\ipa{l}   &    \deux{lj}   & \japhug{qaliaʁ}{eagle} \\  
%\ipa{ɬ}   &       &    & \\  
%\ipa{tɕ}   &       &    & \\  
%\ipa{tɕʰ}   &       &    & \\  
%\ipa{dʑ}   &       &    & \\  
%\ipa{ndʑ}   &       &    & \\  
%\ipa{ɕ}   &       &    & \\  
%\ipa{ʑ}   &       &    & \\  
%\ipa{tʂ}   &       &    & \\  
%\ipa{tʂʰ}   &       &    & \\  
%\ipa{dʐ}   &       &    & \\  
%\ipa{ndʐ}   &       &    & \\  
\ipa{r}   &    \deux{rj}   & \japhug{tɯ-rju}{word} \\  
%\ipa{ʂ}   &       &    & \\  
%\ipa{c}   &       &    & \\  
%\ipa{cʰ}   &       &    & \\  
%\ipa{ɟ}   &       &    & \\  
%\ipa{ɲɟ}   &       &    & \\  
%\ipa{ɲ}   &       &    & \\  
%\ipa{j}   &       &    & \\  
\ipa{k}   &    \deux{kj}   & \japhug{pa-kio}{he caused it to slip} \\  
\ipa{kʰ}   &       \deux{kʰj} \idph{} & \japhug{kʰiɤt}{gliding} \\  
%\ipa{g}   &       &    & \\  
\ipa{ŋg}   &    \deux{ŋgj}   & \japhug{ŋgio}{he slips} \\  
%\ipa{ŋ}   &       &    & \\  
%\ipa{x}   &       &    & \\  
\ipa{ɣ}   &    \deux{ɣj}   & \japhug{tɯ-ɣjɤn}{one time} \\  
\ipa{q}   &    \deux{qj}   & \japhug{qiaβ}{it is bitter} \\  
\ipa{qʰ}   &    \deux{qʰj} \idph{}  & \japhug{qʰiɯqʰiɯ}{blunt (colour)} \\  
\ipa{ɴɢ}   &    \deux{ɴɢj}   & \japhug{ɴɢia}{it will come loose} \\  
%\ipa{χ}   &       &    & \\  
\ipa{ʁ}   &    \deux{ʁj}   & \japhug{ʁjit}{he thinks about him} \\ 
%\ipa{h}   &       &    & \\  
\hline
&    \trois{wsj}    & \japhug{tɤ-fsjit}{whistle} \\ 
&    \trois{wzj}  \tib{}   & \japhug{βzjoz}{he learns it} \\ 
\hline
&    \trois{spj}    & \japhug{spjaŋkɯ}{wolf} \\ 
&    \trois{spʰj}    & \japhug{spʰjar}{he dries it} \\ 
&    \trois{stj}  \idph{}   & \japhug{stiaŋnɤstiaŋ}{jumping} \\ 
&    \trois{sqʰj}    & \japhug{sqʰiar}{he stretches it} \\ 
\hline
&    \trois{ltʰj}  \idph{}   & \japhug{ltʰiɤltʰiɤt}{well-ironed (clothes)} \\ 
&    \trois{lbj} \idph{}   & \japhug{lbjɯlbjɯɣ}{hanging} \\ 
\hline
&    \trois{ʂpj}    & \japhug{rpjɯ}{it is spoiled (milk)} \\ 
&    \trois{rmbj}    & \japhug{tɤ-rmbja}{flash of lightning} \\ 
&    \trois{ʂtsj}    & \japhug{rtsiaʁ}{it is steep (road)} \\ 
&    \trois{ʂqʰj}    & \japhug{ɯ-rqʰioʁ}{its notch} \\ 
&    \trois{rɴɢj}    & \japhug{arɤrɴɢioʁ}{having a notch} \\ 
\hline
&    \trois{χtsj}    & \japhug{χtsiɯ}{pint} \\ 
&    \trois{χpj} \tib{}   & \japhug{χpjɤt}{he  observes it} \\ 
&    \trois{χsj}    & \japhug{ɯ-χsjɯβ}{its slough} \\ 
\hline
&    \trois{mpj}    & \japhug{mpja}{it is warm} \\ 
&    \trois{mtsj}    & \japhug{ɯ-mtsioʁ}{its beak} \\ 
\lspbottomrule
\end{tabular}
\end{table}		
\resetcounters{2Cj}{3Cj}

\begin{table}
 \caption{Count of consonant clusters} \label{tab:clusters.tot}  \centering
\begin{tabular}{lrrrr}
  \lsptoprule	
type &CC& CCC& total\\		
\midrule
\ipab{wC}  & 	\arabic{2wC}  & \arabic{3wC}  &   \addition{2wC}{3wC}  & 	\\	
\ipab{s/zC}  & 	\arabic{2szC}  & \arabic{3szC}  &   \addition{2szC}{3szC}  & 	\\	
\ipab{lC}  & 	\arabic{2lC}  & \arabic{3lC}  &   \addition{2lC}{3lC}  & 	\\	
\ipab{ʂ/rC}  & 	\arabic{2rC}  & \arabic{3rC}  &   \addition{2rC}{3rC}  & 	\\	
\ipab{jC}  & 	\arabic{2jC}  & \arabic{3jC}  &   \addition{2jC}{3jC}  & 	\\	
\ipab{ɕ/ʑC}  & 	\arabic{2CZC}  & \arabic{3CZC}  &   \addition{2CZC}{3CZC}  & 	\\	
\ipab{x/ɣC}  & 	\arabic{2xGC}  & \arabic{3xGC}  &   \addition{2xGC}{3xGC}  & 	\\	
\ipab{χ/ʁC}  & 	\arabic{2XRC}  & \arabic{3XRC}  &   \addition{2XRC}{3XRC}  & 	\\	
\ipab{NC}  & \arabic{2NC}  & \arabic{3NC}  &   \addition{2NC}{3NC}  & 	\\	
\ipab{m/nC}  & \arabic{2mnC}  & \arabic{3mnC}  &   \addition{2mnC}{3mnC}  & 	\\	
\midrule
\ipab{Cɕ}  & 	2  & 	  & 	  2& 	\\	
\midrule
\ipab{Cw}  & 	 \arabic{2Cw}  & \arabic{3Cw}  &   \addition{2Cw}{3Cw}  & 	\\
\ipab{Cj}  & 	 \arabic{2Cj}  & \arabic{3Cj}  &   \addition{2Cj}{3Cj}  & 	\\
%\ipab{Cl; Cr}  & 	 \arabic{2Clr}  & \arabic{3Clr}  &   \addition{2Clr}{3Clr}  & 	\\
%\ipab{Cɣ; Cʁ} & \arabic{2Cg}  & \arabic{3Cg}  &   \addition{2Cg}{3Cg}  & 	\\
%\midrule
%total & \totdeux & \tottrois & \ADD{\totdeux}{\tottrois}{\total}\total \\
\lspbottomrule
\end{tabular}
\end{table}

\subsection{Syllabic contraints} 
\subsubsection{Uvular harmony} \label{sec:uvular.harmony}
Velars and uvulars do not coexist well within the same syllable in Japhug. There are no syllables of the type $\dagger$\forme{QMVɣ} or $\dagger$\forme{KMVʁ} (where K and Q  represent any velar and uvular initial consonants respectively): the initial and the coda have to be both velars, or both uvulars. Thus, syllables such as \japhug{qraʁ}{ploughshare} and \japhug{krɤɣ}{shear, mow} are possible, but not $\dagger$\forme{kraʁ} or $\dagger$\forme{qrɤɣ}. 

With preinitials and medial consonants, the constraint depends on the context. Several cases have to be distinguished.

First, the uvular coda \ipa{-ʁ} is compatible with the velar medial \ipa{-ɣ-}, as shown by examples such as \japhug{pɣaʁ}{turn over}; the opposite case is not attested, but given the relative rarity of medial \ipa{-ʁ-}, this may be fortuitous.

Second, the uvular preinitial  \ipa{ʁ-} does occur with velar initials in Tibetan loanwords, as in \japhug{ʁgra}{enemy} (from Tibetan \tibet{དགྲ་}{dgra}{enemy}).

Third, a uvular preinitial with the velar medial \ipa{-ɣ-} is only found in the dialectal word \japhug{tɯ-χpɣi}{thigh}.

These constraints do not apply across syllables, as shown by words such as \japhug{koʁmɯz}{just as}, in which the \ipa{ʁ} is the preinitial of the second syllable.

The discrimination of uvulars and velars is due to a recent sound change that occurred in Japhug and affected both native words and Tibetan loanwords, the uvularization of velar initial consonants in syllables with uvular \ipa{-ʁ}. This sound change explains for instance why Japhug words such as \japhug{tɯ-qʰoχpa}{organs, state of mind} (phonologically \ipa{qʰoʁ.pa} with internal sandhi) from Tibetan \tibet{ཁོག་པ་}{kʰog.pa}{insides}\footnote{See section \ref{sec:body.part} for an account of the prefix \forme{tɯ-}.} has a uvular \ipa{qʰ-} corresponding to a velar \ipa{kʰ-} in Tibetan: dorsal codas (transcribed as \forme{-g}) are realized as uvulars after \forme{a} and \forme{o} in most Tibetan varieties (\citealt{gong16amdo}), so that a correspondence of Tibetan \forme{-ag} and \forme{-og} to Japhug \ipa{-aʁ} and \ipa{-oʁ} is expected. At an earlier stage, \forme{tɯ-qʰoχpa} has probably been borrowed as *\forme{tɯ-kʰoʁpa} and the sound law *\textsc{velar} \fl{} \textsc{uvular} /\_V\forme{ʁ} applied to it as to the rest of the vocabulary.
 %Phonology
%\chapter{Nominal morphology}
This chapter does not treat of grammatical categories expressed by independent words or clitics, such as number and grammatical relations (discussed in XXX and XXX), and focuses on possessive prefixes, compounding and  noun derivations. Nominalization (including lexicalized deverbal nouns) and denominal verbalization are treated in chapters XXX and XXX respectively.

\section{Possessive prefixes}  \label{sec:possessive.prefixes}
Japhug nouns are divided into two main categories, inalienably possessed nouns, which require a possessive prefix (\ref{sec:inalienably.possessed}) and common nouns which can occur with or without possessive prefix. 

\subsection{Possessive paradigm} \label{sec:possessive.paradigm}
The paradigm of possessive prefixes in Japhug is indicated in Table \ref{tab:possessive.prefixes}. It presents obvious commonalities with the personal pronouns (section \ref{sec:pers.pronouns}) and the indexation suffixes (section XXX), a question studied in more detail in XXX. 

\begin{table}[h] \centering
\caption{Possessive prefixes }\label{tab:possessive.prefixes}
\begin{tabular}{lllllllll} \lsptoprule
 Prefix & Person \\
\midrule
\forme{a-}  &		1\sg{} \\
\forme{nɤ-}  &			2\sg{} \\
\forme{ɯ-}  &			3\sg{} \\
\midrule
\forme{tɕi-}  &			1\du{} \\
\forme{ndʑi-}  &		2/3\du{} \\	
\midrule
\forme{i-}  &			1\pl{} \\
\forme{nɯ-}  &			2/3\pl{} \\
\midrule
\forme{tɯ-/tɤ-}  &			indefinite \\
\forme{tɯ-}  &			generic \\
\lspbottomrule
\end{tabular}
\end{table}

In the possessive paradigm, the contrast between second and third person is neutralized in the dual and plural, while it is preserved in pronouns and person indexation.

Unlike languages like Situ which have two series of possessive pronouns with the same initial consonant but distinct vocalism (\citealt[168-169]{linxr93jiarong}),\footnote{\citet[118-119]{prins16kyomkyo} analyzes the vowel as part of the nominal root.} Japhug preserves the vowel contrast \ipa{ɯ} vs \ipa{ɤ} only with the indefinite possessor form of inalienably possessed nouns; with definite possessors, the contrast is neutralized.

Possession cannot be expressed without a possessive prefix on the possessee. Possessive prefixes can be used on any noun, including recent borrowings from Chinese (or quasi-code switching), as \zh{老家} \forme{lǎojiā} `place of origin; old house' in (\ref{ex:aʑo.GW.alaojia}).

\begin{exe}
\ex \label{ex:aʑo.GW.alaojia}
\gll
aʑo ɣɯ a-<laojia> ɣɯ ɯ-lɤcu nɯre ri ku-rɤʑi-nɯ ŋu \\
\textsc{1sg} \textsc{gen} \textsc{1sg.poss}-old.house \textsc{gen} \textsc{3sg.poss}-upstream there \textsc{loc} \textsc{ipfv}-stay-\textsc{pl} be:\textsc{fact} \\
\glt `They live in a place upstream from my old house.' (14-tApitaRi, 238)
\end{exe}

In the case of first or second person possessors, it is possible to have simply a possessive prefix on the noun, (\forme{a-ɣɲi}, \forme{a-mbro} and \forme{a-ʁgra} in \ref{ex:ambro}), a personal pronoun and a possessive prefix (same person and number, as in \ref{ex:aʑo.ambro}) or even a pronoun, the genitive clitic \forme{ɣɯ} and a possessive prefix as in (\ref{ex:aʑo.GW.alaojia}).

 \begin{exe}
\ex \label{ex:ambro} 
\gll a-ɣɲi ci tɯ\redp{}tɯ-ŋu nɤ, a-mbro ɯ-lwa ɯ-taʁ kɤ-zo, a-ʁgra ci tɯ\redp{}tɯ-ŋu nɤ, a-mbro ɯ-jme ɯ-taʁ kɤ-zo \\
\textsc{1sg.poss}-friend \textsc{indef} \textsc{cond}\redp{}2-be:\textsc{fact} \textsc{lnk} \textsc{1sg.poss}-horse \textsc{3sg.poss}-mane \textsc{3sg}-on \textsc{imp}-land \textsc{1sg.poss}-enemy \textsc{indef} \textsc{cond}\redp{}2-be:\textsc{fact} \textsc{lnk} \textsc{1sg.poss}-horse \textsc{3sg.poss}-tail  \textsc{3sg}-on \textsc{imp}-land  \\
\glt `If you are my friend, land on my horse's mane, if you are my enemy, land on my horse's tail.' (2002qaCpa, 196)
\end{exe}
\begin{exe}
\ex \label{ex:aʑo.ambro}
\gll aʑo a-mbro nɤrwɯrɯnbotɕʰi ŋu, tɯ-sŋi χpaχtsʰɤt ci ɲɯ́-wɣ-tsɯm-a cʰa \\
\textsc{1sg} \textsc{1sg.poss}-horse p.n. be:\textsc{fact} one-day yojana \textsc{indef} i\textsc{pfv:west}-\textsc{inv}-take.away-\textsc{1sg} can:\textsc{fact} \\
\glt `My horse is Norbu Rinpoche, he can make me cross one yojana per day.' (2003smanmi2, 54)
\end{exe}

Possessive prefixes are also used to express beneficiaries as in (\ref{ex:ambro.tARndo.kWtso}), a topic discussed in section XXX.

 \begin{exe}
\ex \label{ex:ambro.tARndo.kWtso}
\gll a-mbro taʁndo kɯ-tso ci tɕi ra \\
\textsc{1sg.poss}-horse speech \textsc{nmls}:S/A-understand one also have.to:\textsc{fact} \\
\glt `I also need a horse who understands speech.' (2003kAndzwsqhaj2, 52)
\end{exe}

%aʑɯɣ a-kɯ-ra, aʑɯɣ kɯ-ra

\subsection{Inalienably possessed nouns} \label{sec:inalienably.possessed}

\subsection{Indefinite vs generic possessor}

\subsection{How did words like `sky', `earth' and `water' become inalienably possessed in Japhug?}

\section{Status constructus}

\subsection{Second member of compounds}
Irregular forms for second members of compounds are very rare.  The noun \japhug{ftɕɤru}{path in the middle of the fields} is a compound of \japhug{ftɕar}{summer} and \japhug{tʂu}{path} (such paths are made during summer to allow workers to work in the field without damaging the crops). The first element \forme{ftɕɤ-} is the \textit{status constructus} of \forme{ftɕar} (with loss of final consonant) and the form \forme{-ru} for the second member of the compound is a clue that \forme{tʂu} comes from earlier \forme{*t-ro} with a dental stop+\ipa{r} cluster changing to a retroflex affricate (see section XXX) -- the \forme{*t-} element being prefixal (perhaps a fossilized indefinite possessor prefix).

\section{Compounding}
\subsection{Noun-Noun compounds}
\subsection{Verb-Verb compounds}
\subsection{Adverb-Verb compounds}
\subsection{Noun-Verb compounds}
\subsection{Verb-Noun compounds}
Verb-Noun compounds are extremely rare in Japhug, as they are in general in Trans-Himalayan languages other than Chinese.  A possible example is \japhug{ndzɤpri}{brown bear}, compound of \japhug{pri}{bear} and \japhug{ndza}{eat} -- as shown by (\ref{ex:ndzApri}) from a text about bears, it is considered by some native speakers of Japhug as a man eater, though this explanation could be folk-etymology.

\begin{exe}
\ex \label{ex:ndzApri}
\gll tɕe ndzɤpri kɤ-ti nɯ tɕe tɯrme tu-kɯ-ndza ɲɯ-ŋgrɤl \\
\textsc{lnk} brown.bear \textsc{nmlz}:S-say \textsc{dem} \textsc{lnk} people \textsc{ipfv}-\textsc{genr}:S/P-eat \textsc{sens}-be.usually.the.case \\
\glt `The brown bear, it eats people.' (21-pri, 94)
\end{exe} 


We find several examples of nominal compounds whose structure is \forme{tɤ-}+Verb+Noun, where the verb is an adjectival stative verb. This category includes \japhug{tɤqiaβjmɤɣ}{lactarius sp.}, literally `bitter mushroom', from the noun \japhug{tɤjmɤɣ}{mushroom} and the verb \japhug{qiaβ}{be bitter}, and \japhug{tɤmbextsa}{type of shoes} from \japhug{tɯ-xtsa}{shoe} and \japhug{mbe}{be old}. These should not be analyzed as Verb-Noun compounds however, as the first element originates from a nominalized form of the verb (such as \japhug{tɤ-mbe}{old thing}, see section XXX on this derivation): they really are a subtype of Noun-Noun compounds.

\section{Noun class prefixes}
\section{Nominal derivations}
\subsection{Superlative}
While there is no adjectival superlative derivation in Japhug (superlative constructions are synthetic, see section XXX), we find nevertheless a derivation applied to locative nouns, expressing the most extreme location. As shown in Table \ref{tab:superlative.n}, it is built by adding an element \forme{-ɕɯ-} followed by a complete copy of the root of the noun; the resulting noun is still an inalienably possessed locative noun. ɛxample (\ref{ex:WqaCWqa}) illustrate the use of one of these forms.

\begin{table}
\caption{Superlative noun derivation} \label{tab:superlative.n}
\begin{tabular}{l|lll}
 \lsptoprule
\japhug{tɯ-ku}{head, top} & \japhug{ɯ-kuɕɯku}{the highest place} \\
\japhug{tɯ-qa}{root, paw, bottom} & \japhug{ɯ-qaɕɯqa}{the deepest place} \\
\japhug{tɯ-rkɯ}{side} & \japhug{ɯ-rkɯɕɯrkɯ}{the place most on the side} \\
 \lspbottomrule
\end{tabular}
\end{table}

\begin{exe}
\ex \label{ex:WqaCWqa}
\gll rɟɤmtsʰu ɯ-qaɕɯqa pjɯ-ɕe tɕe, nɯnɯ ɯ-kɤ-nɤ-mɯm nɯra ɕ-tu-nɯ-tɕɤt ɲɯ-ŋu. \\
ocean \textsc{3sg.poss}-bottom:\textsc{super} \textsc{ipfv}:\textsc{down}-go  \textsc{lnk} \textsc{dem} \textsc{3sg.poss}-\textsc{nmlz}:P-\textsc{trop}-be.tasty \textsc{dem:pl} \textsc{transloc-ipfv}-\textsc{auto}-take.out \textsc{sens}-be \\
\glt `(The sperm whale) goes to the lowest depths of the ocean and catches the things it likes to eat.' (160703 jingyu, 24)
\end{exe}
\section{Denominal adverbs}
\subsection{Comitative adverbs}
\subsection{Time adverbs}
 %Nominal morphology
\chapter{Pronouns and Demonstratives}
\section{Personal pronouns} \label{sec:pers.pronouns}

%\\ipa\{([\w-]*)\}
%\1

%(\d)\\(\w\w)\{\}
%\\textsc{\1\2}

The pronominal system of Japhug distinguishes singular, dual and plural. Alongside the free pronouns, a system of pronominal prefixes is used not only to express possession on noun (see § XXX for an account of the possessive constructions), but also appears in various constructions in the verbal system. These prefixes do not distinguish the second and the third person in the dual and plural forms; their use is described in section \ref{sec:possessive.paradigm}.

\begin{table}[h] \centering
\caption{Pronouns and possessive prefixes }\label{tab:pronoun}
\begin{tabular}{lllllllll} \lsptoprule
 Free pronoun & Prefix & \\
\midrule
 \forme{aʑo},    \forme{aj} &	\forme{a-}  &		1\sg{} \\
\forme{nɤʑo},  \forme{nɤj} &	\forme{nɤ-}  &			2\sg{} \\
\forme{ɯʑo}  &	\forme{ɯ-}  &			3\sg{} \\
\midrule
\forme{tɕiʑo}  &	\forme{tɕi-}  &			1\du{} \\
\forme{ndʑiʑo}  &	\forme{ndʑi-}  &		2\du{} \\	
\forme{ʑɤni}  &	\forme{ndʑi-}  &		3\du{} \\	
\midrule
\forme{iʑo}, \forme{iʑora},   \forme{iʑɤra}   &	\forme{i-}  &			1\pl{} \\
\forme{nɯʑo}, \forme{nɯʑora},   \forme{nɯʑɤra}  &	\forme{nɯ-}  &			2\pl{} \\
\forme{ʑara}  &	\forme{nɯ-}  &			3\pl{} \\
\lspbottomrule
\end{tabular}
\end{table}

Free pronouns and possessive prefixes are remarkably similar in Kamnyu Japhug. In the eastern Japhug dialects, a different \textsc{1sg} pronoun distinct from the possessive prefix  is used: \forme{ŋa} (possibly borrowed from Situ \forme{ŋā}). In the table above, we observe that all the pronouns except the third person dual and plural are formed by adding the root \forme{-ʑo} to the corresponding possessive prefix. 

The first and second person singular pronouns  \forme{aʑo} and \forme{nɤʑo} also have the shorter monosyllabic forms \forme{aj} and \forme{nɤj} respectively. These short forms are considerably less common in stories (in the reported speech of the characters), but appear frequently in free conversations.

Japhug lacks any inclusive / exclusive distinction, unlike other Gyalrongic languages such as Tshobdun, Situ or Khroskyabs (see \citealt{jackson98morphology}, \citealt[177]{linxr93jiarongen}, \citealt[92]{prins16kyomkyo}, \citealt[170]{lai17khroskyabs}). Example  (\ref{ex:tCiZo.CetCi}) shows the dual pronoun \japhug{tɕiʑo}{we (dual)} in inclusive use (it is clear from the context that the son tells his mother to come with him), and (\ref{ex:tCiZo.tCitAYi}) illustrates the same pronoun in exclusive use. Similar pairs of examples can be found with the first plural pronoun \japhug{iʑo}{we (plural} and its variants.

\begin{exe}
\ex \label{ex:tCiZo.CetCi}
\gll a-mu tɕetʰa tɕiʑo kɯnɤ ɕe-tɕi \\
\textsc{1sg.poss}-mother later \textsc{1du} also go:\textsc{fact}-\textsc{1du} \\
\glt `Mother, you and I will go too.' (2003tWxtsa, 138)
\end{exe}

\begin{exe}
\ex \label{ex:tCiZo.tCitAYi}
\gll nɯʑora ɣɯ nɯ-ɕɤmɯɣdɯ cʰo kɯ-fse nɯ ɯ-tsʰɤt nɯ, tɕiʑo ɣɯ tɕi-tɤɲi tɯ-ldʑa pɯ-tu tɕe, nɯ kɤ-nɯ-tʰɯ-tɕi ɕti wo \\
\textsc{2pl} \textsc{gen} \textsc{2pl.poss}-gun \textsc{comit} \textsc{nmlz}:S/A-be.like \textsc{dem} \textsc{3sg}-instead \textsc{dem} \textsc{1du} \textsc{gen} \textsc{1du.poss}-staff one-long.object \textsc{pst.ipfv}-exist \textsc{lnk} \textsc{dem} \textsc{pfv}-\textsc{auto}-spread-\textsc{1du} be:\textsc{affirm}:\textsc{fact} \textsc{sfp} \\
\glt `Instead of guns and other things like you, we only had a staff, and we used it as a bridge (to cross the river).' (2003kunbzang, 164)
\end{exe}

Third person pronouns can be used with inanimate referents, as the third person dual \forme{ʑɤni} in example (\ref{ex:rNgW}).

\begin{exe}
\ex \label{ex:rNgW}
\gll tɕe rŋgɯ nɯ  to-k-ɤmɯrpu-ndʑi-ci tɕe, tɕendɤre ʑɤni pjɤ-nɯ-ɴɢrɯ-ndʑi   \\
\textsc{lnk} boulder \textsc{dem} \textsc{ifr-evd}-bump.into:\textsc{recip}-\textsc{du-evd} \textsc{lnk} \textsc{lnk} \textsc{3du} \textsc{ifr-auto}-crush-\textsc{du} \\
\glt `The boulders bumped into each other and they were pulverized.' (smanmi4.82-83)
\end{exe}

In some contexts, demonstrative pronouns rather than person pronouns are used to refer to a third person, even human (see § \ref{sec:demonstrative.pronouns}).

Personal pronouns are not used as head of relative clauses (as in Chinese \zh{……的你} `you who are ...'), though there are case of relativization of first or second person possessor, as in (\ref{ex:amu.kWme}) (see § XXX).

\begin{exe}
\ex \label{ex:amu.kWme}
\gll aʑo nɯ a-mu kɯ-me ŋu-a tɕe tɕe \\
1sg \textsc{dem} \textsc{1sg.poss}-mother \textsc{nmlz}:S/A-exist be:\textsc{fact-1sg} \textsc{lnk} \textsc{lnk} \\
\glt `I am someone who does not have a mother.' (2003Nyimawodzer2, 12)
\end{exe}

Personal pronouns can take determiners, in particular the demonstrative \forme{nɯ} as in (\ref{ex:amu.kWme}), numerals (§ \ref{sec:uses.numerals}) and can also precede a noun in apposition, in expressions such as \forme{iʑo kɯrɯ} `we, Tibetans' (\ref{ex:iZo.kWrW}) or \forme{nɤʑo qaɕpa} `you frog' in (\ref{ex:nAZo.qaCpa}).

\begin{exe}
\ex \label{ex:iZo.kWrW}
\gll iʑo kɯrɯ tɕe pɤjka tu-nɯ-ti-j ŋu tɕe, \\
\textsc{1pl} Tibetan \textsc{lnk} species.of.squash \textsc{ipfv}-\textsc{auto}-say-\textsc{1pl} \textsc{lnk} \\
\glt `We Tibetans call it \forme{pɤjka}.' (16-CWrNgo, 71)
\end{exe}

\begin{exe}
\ex  \label{ex:nAZo.qaCpa}
\gll  nɯ-nɯ-nɤre ma nɤʑo qaɕpa nɤ-rʑaβ nɤ-kɯ-mbi kɯ-tu me   \\
\textsc{imp-auto}-laugh \textsc{lnk} \textsc{2sg} frog \textsc{2sg.poss}-wife \textsc{2sg.poss}-\textsc{nmlz}:S/A-give \textsc{nmlz}:S/A-exist not.exist:\textsc{fact} \\
\glt `Laugh as you wish, nobody will give you a wife, you frog.'   (2002 qaCpa, 176)
\end{exe} 

Personal pronouns occur as member of compounds only in two constructions: with the root \forme{-sɯso} `as X wish' (from the verb \japhug{sɯso}{think}), as in example (\ref{ex:ʑara.sWso}) and with \forme{-sti} `alone'. These constructions is discussed in more detail in § XXX and § \ref{sec:stWsti}.

\begin{exe}
\ex \label{ex:ʑara.sWso}
\gll a-zda ra ʑara-sɯso tu-nɯ-nɤŋkɯŋke-nɯ ɲɯ-kʰɯ \\
\textsc{1sg.poss}-companion \textsc{pl} \textsc{3pl}-as.wish \textsc{ipfv}-\textsc{auto}-go.here.and.there-\textsc{pl} \textsc{sens}-be.possible \\
\glt `The other (snakes) can go here and there as they wish.' (The divination, 43)
\end{exe}

As in most languages with polypersonal indexation, pronouns (especially first and second person pronouns) are never obligatory, and a finite verb form without overt argument NPs is a perfectly well-formed sentence (see § XXX). 


Japhug presents a very common subtype of split ergativity: the ergative \forme{kɯ} being obligatory on transitive subject third person pronouns (except in the case of the emphatic use of pronouns, § \ref{sec:pronouns.emph}) but optional on first and second person pronouns (§\ref{sec:absolutive.A} and § \ref{sec:A.kW}).

\section{Generic pronoun}  \label{sec:genr.pro}
The generic pronoun \japhug{tɯʑo}{one} has the same morphological structure as personal pronouns as seen in the previous section, combining the generic possessive prefix \ipa{tɯ-} with the pronominal root \forme{-ʑo}. Note that this generic possessive has to be strictly distinguished from the homophonous indefinite possessive prefix \forme{tɯ-} (see § \ref{sec:indef.genr.poss}).

In Japhug, sentences have at most one generic human referent(§ XXX). If this referent is core argument, the verb has generic indexation (\forme{kɯ-} for S/P, \forme{wɣ-} for A, as in the following examples; see also section XXX). The generic argument can be realized as the generic pronoun \forme{tɯʑo} as in (\ref{ex:pjWkWZGAGANgi}) or by a generic noun (such as \japhug{tɯrme}{person}, see § XXX).

\begin{exe}
\ex \label{ex:pjWkWZGAGANgi}
\gll tɯ-zda pjɯ́-wɣ-z-ɣɤtɕa, \textbf{tɯʑo}  ntsɯ  pjɯ-kɯ-ʑɣɤ-ɣɤŋgi   	tɕe,  pɯ-kɯ-nɯ-ɣɤtɕa 	kɯ́nɤ   	pjɯ-kɯ-ʑɣɤ-ɣɤŋgi   	tɕe,    ɯ-mbrɤzɯ   	kɯ-tu   	me  	tu-kɯ-ti   	ɲɯ-ŋu.   \\
\textsc{genr.poss}-companion \textsc{ipfv-inv-caus}-be.wrong oneself always \textsc{ipfv-genr:S/P-refl}-be.right \textsc{lnk} \textsc{pfv-genr:S/P-auto}-be.wrong also \textsc{ipfv-genr:S/P-refl}-be.right lnk \textsc{3sg.poss}-result \textsc{nmlz:S/A}-have  not.exist:\textsc{fact} \textsc{ipfv-genr}-say \textsc{sens}-be \\
\glt  `If one considers that one's companion is wrong, and always considers himself to be right even if one is wrong, there is can be no good result.' (Mouse and sparrow, 80-82)
\end{exe} 

The generic pronoun can occur before a noun with the generic possessive as in \forme{tɯʑo tɯ-skɤt}  `one's language' in example (\ref{ex:tWZo.tWskAt}); this contributes to disambiguating between the indefinite possessive and the generic possessive in the case of inalienably possessed nouns (thus on its own \forme{tɯ-skɤt} can mean either `a language' or `one's language').

\begin{exe}
\ex \label{ex:tWZo.tWskAt}
\gll tɕendɤre tɯʑo tɯ-skɤt ʑara ɣɯ-sɯxɕɤt ɲɯ-ra, ʑara nɯ-skɤt tɯʑo kɯ-sɯxɕɤt ɲɯ-ra \\
\textsc{lnk} \textsc{genr} \textsc{genr.poss}-language \textsc{3pl} \textsc{inv}-teach \textsc{sens}-have.to \textsc{3pl} \textsc{3pl.poss}-language \textsc{genr} \textsc{genr}:S/P-teach \textsc{sens}-have.to \\
\glt `One has to teach them one's language, and they have to teach you their language.'  (150901 tshuBdWnskAt, 29)
\end{exe} 

When occurring in A function, the generic pronoun \forme{tɯʑo} obligatorily receives the ergative \forme{kɯ} as in (\ref{ex:tWZo.kW}) (note that in example \ref{ex:tWZo.tWskAt}, although the generic referent is A in the first clause, \forme{tɯʑo} does not take ergative because it is a determiner of \forme{tɯ-skɤt}). 

\begin{exe}
\ex \label{ex:tWZo.kW}
\gll tɯʑo kɯ tɯ-χti ɲɯ́-wɣ-nɯ-ɕar kɯ-maʁ kɯ,  tɯ-pʰama ra kɯ tɯ-χti ɲɯ-ɕar-nɯ tɕe tɯ-sɯm pɯ-a<nɯ>ri nɤ ju-kɯ-ɕe,
mɯ-pɯ-a<nɯ>ri nɤ ju-kɯ-ɕe pɯ-ra \\
\textsc{genr} \textsc{erg} \textsc{genr.poss}-spouse \textsc{ipfv-inv-auto}-search \textsc{inf:stat}-not.be \textsc{erg} \textsc{genr.poss}-parent \textsc{pl} \textsc{erg} \textsc{genr.poss}-spouse \textsc{ipfv}-search-\textsc{pl} \textsc{lnk} \textsc{genr.poss}-mind \textsc{pst.ipfv}-<\textsc{auto}>go[II] \textsc{lnk} \textsc{ipfv}-\textsc{genr}:S/P-go, \textsc{neg-pst.ipfv}-<\textsc{auto}>go[II] \textsc{lnk} \textsc{ipfv-genr}:S/P-go \textsc{pst.ipfv}-have.to \\
\glt `One could not choose one's spouse, one's parents chose one's spouse, and one had to go whether one liked it or not.' (14-tApitaRi, 212-215)
\end{exe} 

Other cases like dative are treated like inalienably possessed nouns (see § XXX); for instance, when the generic argument is in the dative, the forms \forme{tɯ-ɕki} or \forme{tɯ-pʰe} `to one' occur, with no indexation on the verb selecting this dative argument, as in example (\ref{ex:tWZo.tWCki}).

\begin{exe}
\ex \label{ex:tWZo.tWCki}
\gll
tɯ-ɲi ɣɯ ɯ-rɟit nɯra kɯ tɯʑo tɯ-ɕki ``a-rpɯ", tɤ-tɕɯ pɯ-kɯ-ŋu nɤ ``a-rpɯ" tu-ti-nɯ, tɕʰeme pɯ-kɯ-ŋu nɤ ``a-ɬaʁ" tu-ti-nɯ kɯ-ra ŋu \\
\textsc{genr.poss}-FZ \textsc{gen} \textsc{3sg.poss}-child \textsc{dem:pl} \textsc{erg} \textsc{genr} \textsc{genr}-dat \textsc{1sg.poss}-MB \textsc{indef.poss}-son \textsc{pst.ipfv-genr}:S/P-be if \textsc{1sg.poss}-MB \textsc{ipfv}-say-\textsc{pl} girl \textsc{pst.ipfv-genr}:S/P-be  if \textsc{1sg.poss}-MZ \textsc{ipfv}-say-\textsc{pl} \textsc{inf.stat}-have.to be:\textsc{fact} \\
\glt `One's father's sister's children have to call oneself `my mather uncle' if one is a boy, `my mather aunt' if one is a girl.'  (see § XXX about Omaha kinship, 140425kWmdza03, 1)
\end{exe} 

As examples (\ref{ex:pjWkWZGAGANgi}) to (\ref{ex:tWZo.tWCki}) illustrate, generic agreement between pronoun, possessive prefix and verb indexation is very systematic, and suffers no exception.

Due to the constraint against more than one generic argument per clause (§ XXX), the only case that the generic pronoun can appear two times in the same clause occurs in reflexive constructions, as in (\ref{ex:tWZo.kW.tWZo}).

\begin{exe}
\ex \label{ex:tWZo.kW.tWZo}
\gll tɯʑo kɯ tɯʑo tu-kɯ-nɯ-ʑɣɤ-βri ra kɤ-ti ɲɯ-ŋu \\
\textsc{genr} \textsc{erg} \textsc{genr} \textsc{ipfv}-\textsc{genr}:S/P-\textsc{auto}-\textsc{refl}-protect have.to:\textsc{fact} \textsc{inf}-say \textsc{sens}-be \\
\glt `One has to protect oneself.' (04-qala1, 25)
\end{exe} 

\section{Genitive forms} \label{sec:pronouns.gen}
The form of pronouns and personal prefixes undergoes few morphophonological changes in combination with postpositions and relational nouns. However, in combination with the genitive postposition \forme{ɣɯ} (cf \ref{sec:genitive}), some  personal pronouns have special forms indicated in Table  \ref{tab:pronoun.gen}.

\begin{table}[h] \centering
\caption{Pronouns and possessive prefixes }\label{tab:pronoun.gen}
\begin{tabular}{lllllllll} \lsptoprule
 Free pronoun & Genitive & \\
\midrule
 \forme{aʑo}  &	\forme{aʑɯɣ}  &		\textsc{1sg} \\ 
\forme{nɤʑo}  &	\forme{nɤʑɯɣ}  &			\textsc{2sg} \\ 
\forme{ɯʑo}  &	\forme{ɯʑɤɣ}  &			\textsc{3sg} \\ 
\forme{tɕiʑo}  &	\forme{tɕiʑɤɣ}  &			\textsc{1du} \\ 
\forme{ndʑiʑo}  &	\forme{ndʑiʑɤɣ}  &		\textsc{2du} \\	 
\forme{ʑɤni}  &	\forme{ʑɤniɣɯ}  &		\textsc{3du} \\	 
\forme{iʑo}  &	\forme{iʑɤɣ}, 	\forme{iʑɤra ɣɯ}   &			\textsc{1pl} \\ 
\forme{nɯʑo}  &	\forme{nɯʑɤɣ}, 	\forme{nɯʑɤra ɣɯ}  &			\textsc{2pl} \\ 
\forme{ʑara}  &	\forme{ʑaraɣ},   \forme{ʑara ɣɯ}&			\textsc{3pl}  \\  
\lspbottomrule
\end{tabular}
\end{table}

While some degree of variation exists with dual and plural pronouns (for instance the regular \forme{iʑo ɣɯ} is found alongside \forme{iʑɤɣ} and \forme{iʑɤra ɣɯ}), for the singular pronouns only one form is attested.

\begin{exe}
\ex
\gll aʑɯɣ 	ndʐa 	ŋu 	ɕi, 	nɤʑɯɣ 	ndʐa 	ŋu, 	aj 	mɯ́j-tso-a   \\
\textsc{1sg:gen} reason be:\textsc{fact} \textsc{qu} \textsc{2sg:gen} reason be:\textsc{fact} \textsc{1sg} \textsc{neg:sens}-understand-\textsc{1sg} \\
\glt  `I don't know if it is because of me, or because of you.' (that the phone line is not working well) (phone conversation, 2011) %\wav{8_ndzxa})
\end{exe} 

In the genitive forms of the pronouns, the vowel of the genitive marker is generally dropped, and the pronominal root \forme{-ʑo} undergoes vowel change to \forme{-ʑɯɣ} (in the case of first and second person) and \forme{-ʑɤɣ} (in other forms). Note that \forme{ʑaraɣ} is the only case of the rhyme \ipa{aɣ} in Japhug.

When genitive pronouns occur as determiners of nouns (including in the possessive existential construction, see § XXX), these nouns almost always take a possessive prefix coreferent with the genitive pronoun, as in (\ref{ex:tɕithAfkAlAGi}).

\begin{exe}
\ex \label{ex:tɕithAfkAlAGi}
\gll 
tɕiʑɤɣ tɕi-tʰɤfkɤlɤɣi tɯ-ɕkat pɯ-tu tɕe, nɯ kɤ-nɯ-χtɤr-tɕi ɕti wo \\
\textsc{1du:gen} 2du-plant.ash one-load \textsc{pst.ipfv}-exist \textsc{lnk} \textsc{dem} \textsc{pfv-auto}-spread-\textsc{1du} be:\textsc{affirm}:\textsc{fact} \textsc{sfp} \\
\glt `We had one load of plant ash, and spill it there.' (2003kunbzang, 171)
\end{exe} 

The genitive pronouns can be used as possessive pronouns (`mine', `my own' etc) and take the determiner \forme{nɯ} and the plural \forme{ra}, as in (\ref{ex:aZWG.nW}) and (\ref{ex:WZAG.nWra}).

\begin{exe}
\ex \label{ex:aZWG.nW}
\gll ``tɕe ɣnɤsqaptɯ-rʑaʁ tu-tsu tɕe ɲɯ-ʁaʁ ŋu" ɲɯ-ti-nɯ ri, aʑɯɣ nɯ ɣnɤsqamnɯz tɤ-rʑaʁ mɤɕtʂa mɯ-nɯ-ʁaʁ. \\
\textsc{lnk} eleven-night \textsc{ipfv}-pass \textsc{lnk} \textsc{ipfv}-hatch be:\textsc{fact} \textsc{sens}-say-\textsc{pl} \textsc{lnk} \textsc{1sg:gen} \textsc{dem} twelve one-night  until \textsc{neg-pfv}-hatch \\
\glt `People say that (chicken eggs) hatch after eleven days, mine took twelve days to hatch.' (150819 kumpGa)
\end{exe} 

\begin{exe}
\ex \label{ex:WZAG.nWra}
\gll ɯʑɤɣ nɯra tu-nɯ-ɣɤ-βdi tɕe, ɕɯ-sɤ-sqɤr mɤ-ra \\
\textsc{3sg:gen} \textsc{dem:pl} \textsc{ipfv-auto-caus}-be.good \textsc{lnk} \textsc{transloc-antipass}-hire \textsc{neg}-have.to:\textsc{fact} \\
\glt `He repairs his own (machines) himself, he does not need to ask other people.' (14-tApitaRi, 168)
\end{exe} 

\section{The emphatic use of pronouns} \label{sec:pronouns.emph}
In addition to their referential and anaphoric functions, pronouns in Japhug can be used in an emphatic way in combination with the particle \forme{ʑo}, as in  (\ref{ex:WZo.Zo}).

\begin{exe}
\ex \label{ex:WZo.Zo}
\gll aʑo ɯʑo ʑo kɤ-mto mɯ-pɯ-rɲo-t-a. \\
\textsc{1sg} \textsc{3sg} \textsc{emph} \textsc{inf}-see \textsc{neg-pfv}-experience-\textsc{pst:tr-1sg} \\
\glt `I never saw it itself.' (24-kWmu, 7)
\end{exe} 

In combination with the autobenefactive \forme{nɯ-} on the verb, pronouns express the meaning `do X on one's own'. In the case of transitive verbs, third person pronouns in this function does not take the ergative even if the referent is the transitive subject (example \ref{ex:pjWnWtCAtnW}, where \japhug{tɕɤt}{take out} is transitive).

\begin{exe}
\ex
\gll tɕe ɲɯ-tɯ-nɤm qhe, tɕe ʑara ku-nɯ-nɯɣi-nɯ ŋu ɕi? \\
\textsc{lnk} \textsc{ipfv:east}-2-chase[III] \textsc{lnk} \textsc{lnk} \textsc{3pl} \textsc{ipfv:west}-auto-come.back-\textsc{pl} be:\textsc{fact} \textsc{qu} \\
\glt `Do you chase them, or do they come back home on their own?' (taRrdo conversation, 29)
\end{exe} 

\begin{exe}
\ex \label{ex:pjWnWtCAtnW}
\gll tɕe lu-nɯ-rɤji-nɯ tɕe, nɯ-kɤ-ndza nɯra ʑara pjɯ-nɯ-tɕɤt-nɯ pjɤ-ŋu tɕe \\
\textsc{lnk} \textsc{ipfv-auto}-plant.crops-\textsc{pl} \textsc{lnk} \textsc{3pl.poss-nmlz:P}-eat \textsc{dem:pl} \textsc{3pl} \textsc{ipfv-auto}-take.out-pl \textsc{ifr.ipfv}-be \textsc{lnk} \\
\glt `They planted crops, and earned their food on their own.' (about lepers, who were settled in the special place by the government, 25-khArWm, 70)
\end{exe}

This construction is also attested with first of second person pronouns, as in (\ref{ex:aZo.Zo}).

\begin{exe}
\ex \label{ex:aZo.Zo}
\gll aʑo ʑo nɯnɯ ɕ-pjɯ-sat-a ra \\
\textsc{1sg} \textsc{emph} \textsc{dem} \textsc{transloc-ipfv}-kill-\textsc{1sg} have.to:\textsc{fact} \\
\glt `I have to kill her myself.' (140504 baixuegongzhu, 117)
\end{exe}

The emphatic pronoun \japhug{raŋ}{oneself} borrowed from Tibetan \tibet{རང་}{raŋ}{oneself}, can also be used with any person, though this usage is not very common. It can occur with the autobenefactive (\ref{ex:aZo.raN}) or without it (\ref{ex:aZo.raN.Zo}).

\begin{exe}
\ex \label{ex:aZo.raN}
\gll
nɤʑo tu-tɯ-ti mɤ-ra ma aʑo raŋ tu-nɯ-ti-a jɤɣ \\
\textsc{2sg} \textsc{ipfv}-2-say \textsc{neg}-have.too:\textsc{fact} \textsc{lnk} \textsc{1sg} oneself \textsc{ipfv}-\textsc{auto}-say-\textsc{1sg} be.possible:\textsc{fact} \\
\glt `You don't need to say it, I can say it myself.' (elicited)
\end{exe}

\begin{exe}
\ex \label{ex:aZo.raN.Zo}
\gll aʑo raŋ ʑo ju-ɕe-a ra \\
\textsc{1sg} oneself \textsc{emph} \textsc{ipfv}-go-\textsc{1sg} have.to:\textsc{fact} \\
\glt `I have to go there myself.' (150830 afanti-zh, 96)
\end{exe}
\section{Interrogative pronouns}
The interrogative pronouns in Japhug are indicated in Table \ref{tab:interrog.pronoun}. These pronouns are used in independent interrogative clauses (\ref{ex:tChi.pWNu}), in subordinate clauses (\ref{ex:tChi.kWNu}), and also in correlatives (\ref{ex:NotCu.WsAzrAZi}), and also occur to express non-specific referents (these uses are described in section  \ref{sec:interrogative.indef}, after the indefinite pronouns).

\begin{exe}
\ex \label{ex:tChi.pWNu}
\gll
tɕe mɤʑɯ tɕʰi pɯ-ŋu? \\
\textsc{lnk} yet what \textsc{pst.ipfv}-be \\
\glt `What was there (after this one)?' (12-ndZiNgri, 100)
\end{exe}  

\begin{exe}
\ex \label{ex:tChi.kWNu}
\gll ɯʑo tɕʰi kɯ-ŋu nɯ ko-tso-nɯ tɕe tɕe cʰɤ́-wɣ-tɕɤt \\
\textsc{3sg} what \textsc{nmlz}:S/A-be \textsc{dem} \textsc{ifr}-understand-\textsc{pl} \textsc{lnk} \textsc{lnk} \textsc{ifr:downstream-inv}-take.out \\
\glt `They understood what he was, and expelled him (from their group).' (140427 hanya yu gezi-zh, 19)
\end{exe}  

\begin{exe}
\ex \label{ex:NotCu.WsAzrAZi}
\gll 
ɯ-pʰoŋbu tɕʰi kɯ-fse nɯ, ŋotɕu ɯ-sɤz-rɤʑi nɯnɯ ɣɯ kɯ-nɯtsa kɯ-fse ɲɯ-ɕti tɕe \\
\textsc{3sg}.\textsc{poss}-body what \textsc{nmlz}:S/A-be.like \textsc{dem} where \textsc{3sg-nmlz:oblique}-remain \textsc{dem} \textsc{gen}  nmlz:S/A-fit \textsc{nmlz}:S/A-be.like \textsc{sens}-be:\textsc{affirm} \textsc{lnk} \\
\glt `The way its body is like is well-fitted to the place where it lives.' (19-rNamoN, 24)
\end{exe}  

\begin{table}[h] \centering
\caption{Interrogative pronouns }\label{tab:interrog.pronoun}
\begin{tabular}{lllllllll} \lsptoprule
\japhug{tɕʰi}{what} \\
\japhug{ɕɯ}{who} \\
\japhug{tʰɤstɯɣ}{how many} \\
\japhug{tʰɤjtɕu}{when} \\
\japhug{ŋotɕu}{where}, \japhug{ŋoj}{where} \\
\japhug{tɕʰindʐa}{why} \\
\lspbottomrule
\end{tabular}
\end{table}

In addition to these pronouns, some indefinite pronouns are also marginally used in questions, see for instance (\ref{ex:thWthAci.totia}) in § \ref{sec:thWci}.

\subsection{\japhug{tɕʰi}{what}} \label{sec:tChi}
The interrogative pronoun  `what' considerably varies across Japhug dialects. In Kamnyu we find \forme{tɕʰi}, apparently borrowed from Tibetan \forme{tɕʰi}. Neighbouring dialects of Gdongbrgyad area have either \forme{tsʰi} (in Mangi) or \forme{tʰi} (in Rqaco), which represents the original Rgyalrongic root for this interrogative pronoun (cognate with Tibetan \tibet{ཆི་}{tɕʰi}{what} and Limbu \forme{the}). Even in the Kamnyu dialect, the form \forme{tsʰi-} is directly attested in the indefinite \japhug{tsʰitsuku}{some} (\ref{sec:tshitsuku}). Mangi Japhug shares with Kamnyu the sound change \forme{*tʰi} \fl{} \forme{tsʰi} which also affects the verb \japhug{tsʰi}{drink} (this sound change occurred after the pronoun  \forme{*tʰi} underwent \textit{status constructus} alternation to \forme{tʰɯ-} and was used to build the indefinite pronoun \japhug{tʰɯci}{something}, see \ref{sec:thWci}). Note that Kamnyu Japhug \japhug{tɕʰi}{what} is homophonous with the noun \japhug{tɕʰi}{tree-trunk stairs} attested for instance in example (\ref{ex:tChi.tukWndW}) -- Japhug texts readers have to be aware of potential ambiguities.

\begin{exe}
\ex \label{ex:tChi.tukWndW}
\gll
ɕom ɣɟɯ kɯ-mbɯ\redp{}mbro ʑo,  ɯ-ɣmbaj zɯ tɕʰi tu-kɯ-ndɯ ci pɯ-tu ɲɯ-ŋu \\
iron tower \textsc{nmlz}:S/A-\textsc{emph}\redp{}be.high \textsc{emph} \textsc{3sg}.\textsc{poss}-side \textsc{loc} treetrunk.stairs \textsc{ipfv}:\textsc{up}-\textsc{nmlz}:S/A-\textsc{acaus}:spread \textsc{indef} \textsc{pst}.\textsc{ipfv}-be \textsc{sens}-be \\
\glt `There was a huge iron toward, with a tree-trunk stairs on its side.' (Norbzang05, 65)
\end{exe}  

The Eastern dialects of Gsardzong and Datshang have \forme{xto} instead, a word of unknown etymology.

In the Kamnyu dialect, \japhug{tɕʰi}{what} is by far the most common interrogative pronoun in the corpus. In interrogative clauses, it can be used to ask about objects, non-human animals (\ref{ex:nAmbro}) and names of persons (\ref{ex:tChi.tWrmi}).

\begin{exe}
\ex \label{ex:nAmbro}
\gll
nɤʑo nɤ-mbro nɯ tɕʰi ŋu \\
\textsc{2sg} 2sg.poss-horse \textsc{dem} what be:\textsc{fact} \\
\glt `Who is your horse?' (about a sentient horse, 2003smanmi-tamu, 53)
\end{exe}  

\begin{exe}
\ex \label{ex:tChi.tWrmi}
\gll tɕʰi tɯ-rmi? \\
what 2-be.called:\textsc{fact} \\
\glt `What is your name?' (heard in context)
\end{exe}  

As in many languages, this interrogative pronoun (instead of the pronoun \japhug{ɕɯ}{who}) is also used in questions about classification of persons (\citealt{idiatov07nonselective}), including social affiliation (\ref{ex:tChi.WrWG}, and \ref{ex:tChi.kWNu} above) and biological affiliation (\ref{ex:tChi.tosci}).

\begin{exe}
\ex \label{ex:tChi.WrWG}
\gll ɯtɤz nɯʑo tɕʰi ɯ-rɯɣ tɯ-ŋu-nɯ? \\
finally \textsc{2pl} what  3sg.poss-race  2-be:\textsc{fact}-\textsc{pl} \\
\glt `Finally, what race (of being) are you?' (smanmi2003, 172)
\end{exe}  

There is no specific interrogative pronoun to ask about manner like English `how', and Japhug expresses this meaning by combining \forme{tɕʰi} with the verbs \japhug{fse}{be like...} or \japhug{stu}{do like...}, as in examples (\ref{ex:tChi.tAtWfsendZi}), (\ref{ex:tChi.atAfsej}) and (\ref{ex:tChi.Zo.tuwGBzu}).

\begin{exe}
\ex \label{ex:tChi.tAtWfsendZi}
\gll a-ʁi, ki kɯ-fse tɤjpɣom kɯ-wxti nɯtɕu, kɤ-ɕe tɕʰi tɤ-tɯ-fse-ndʑi?? \\
\textsc{1sg.poss}-younger.sibling this \textsc{nmlz}:S/A-be.like ice \textsc{nmlz}:S/A-be.big \textsc{dem:loc} inf-go what \textsc{pfv}-2-be.like-\textsc{du} \\
\glt `Sister, how did you cross such a big block of ice?' (stodtWphu2005, 156)
\end{exe}  

   \begin{exe}
\ex \label{ex:tChi.atAfsej}
\gll  kɤ-pʰɣo tɕʰi a-tɤ-fse-j    \\
\textsc{inf}-flee what \textsc{irr-pfv}-be.like-\textsc{1pl} \\
\glt  `How will we flee?' (Norbzang 69)
\end{exe} 

\begin{exe}
\ex \label{ex:tChi.Zo.tuwGBzu}
\gll nɤ-smɤn tɤ-sɯ-βzu-t-a ri maka mɯ́j-pʰɤn, tɕe tɕʰi ʑo tú-wɣ-stu pʰɤn \\
\textsc{3sg.poss}-medicine \textsc{pfv-caus}-make-\textsc{pst:tr-1sg} but at.all \textsc{neg:sens}-be.efficient \textsc{lnk} what \textsc{emph} \textsc{ipfv-inv}-do.like be.efficient:\textsc{fact} \\
\glt `I had medicine made for you but it does not work, how should we do for it to work?' (nyima wodzer 2002, 22) 
\end{exe}  

 
The pronoun \japhug{tɕʰi}{what} on its own can occur in questions about the reason or the purpose of a particular state of affair, as in (\ref{ex:tChi.apWNua}) and (\ref{ex:tChi.YWtWnAre}).

\begin{exe}
\ex \label{ex:tChi.apWNua}
\gll  aʑo tɕʰi a-pɯ-ŋu-a? \\
\textsc{1sg} what \textsc{irr-ipfv}-be-1sg \\
\glt `How can it be me?' (2003sras, 61)
\end{exe}  

\begin{exe}
\ex \label{ex:tChi.YWtWnAre}
\gll  a-tɤɕime, tɕʰi ɲɯ-tɯ-nɤre ŋu? \\
 \textsc{1sg.poss}-lady what \textsc{sens}-2-laugh be:\textsc{fact} \\
 \glt `My lady, why are you laughing?'  (Not `what are you laughing at ?', 2002qaCpa, 102)
\end{exe}  

When referring to purpose or reason, it is possible to combine  \japhug{tɕʰi}{what} with the nouns \japhug{ɯ-spa}{its material} and \japhug{ɯ-ndʐa}{its reason} (as the pronoun \japhug{tɕʰindʐa}{why})  respectively, as in (\ref{ex:tChi.Wspa.pWNu}) and (\ref{ex:tChi.YWtWɣAwu}). Note that examples (\ref{ex:tChi.YWtWnAre}) and (\ref{ex:tChi.YWtWɣAwu}) are from the same story, just a few lines away, in the same context; the construction in (\ref{ex:tChi.YWtWɣAwu}) is a more explicit variant of that in (\ref{ex:tChi.YWtWnAre}).

\begin{exe}
\ex \label{ex:tChi.Wspa.pWNu}
\gll tɕe tɕʰi ɯ-spa pɯ-ŋu mɤ-xsi ma tɕe nɯ kɯ-fse pjɤ-tu  \\
\textsc{lnk} what \textsc{3sg.poss}-material \textsc{pst.ipfv}-be \textsc{neg-genr}:know \textsc{lnk} \textsc{lnk} \textsc{dem} \textsc{nmlz}:S/A-be.like \textsc{ifr.ipfv}-exist \\
\glt `It is not known what it was for, but there was something like that.' (hist140522 GJW, 18)
\end{exe}  

\begin{exe}
\ex \label{ex:tChi.YWtWɣAwu}
\gll tɕʰindʐa ɲɯ-tɯ-ɣɤwu ŋu? \\
why \textsc{sens}-2-cry be:\textsc{fact} \\
\glt `Why are you crying?' (2002qaCpa, 94)
\end{exe} 

The pronoun \forme{tɕʰi} takes case marking with genitive \forme{ɣɯ} and the instrumental/ergative \forme{kɯ}, as in (\ref{ex:tChi.kW}).

\begin{exe}
\ex \label{ex:tChi.kW}
\gll tɕe tɕʰi kɯ tu-sɯ-βze ŋu mɤxsi ma nɯ kɯ-fse nɯ, sɯku ri ku-ndzoʁ ŋu \\
\textsc{lnk} what \textsc{erg} \textsc{ipfv}-\textsc{caus}-make[III] be:\textsc{fact} \textsc{neg}-\textsc{genr}-know \textsc{lnk} \textsc{dem} \textsc{nmlz}:S/A-be.like \textsc{dem} top.of.trees \textsc{loc} \textsc{ipfv}-\textsc{anticaus}:attach be:\textsc{fact} \\
\glt `I don't what it (the wasp) uses to make it (its nest), it is attached on trees.' (26-ndzWrnaR, 55)
\end{exe} 
In combination with the adverb \forme{jarma} / \japhug{jamar}{about}, it can be used to indicate a quantity, instead of \japhug{tʰɤstɯɣ}{how many} (section \ref{sec:thAstWG}).

\begin{exe}
\ex \label{ex:tChi.jamar}
\gll tu-ɕtʂam-a tɕe tɕʰi jamar ʑo ɣɤʑu kɯ? \\
\textsc{ipfv}-measure[III]-\textsc{1sg} \textsc{lnk} what about \textsc{emph} exist:\textsc{sens} \textsc{sfp} \\
\glt `I will measure it with a scoop to see how much (gold) there is.' (140512alibaba-zh, 59)
\end{exe}  

\begin{exe}
\ex \label{ex:tChi.jamar.kondza}
\gll kʰɯtsa ɯ-ŋgɯ tɯ-ci tu-rku-nɯ tɕe, nɯnɯtɕu tɤŋe nɯ pjɯ-sɯ-ntɕʰɤr-nɯ tɕe, tɕe tɕʰi jamar ko-ndza nɯnɯ, nɯnɯ ɯ-ŋgɯ nɯtɕu pjɯ-ru-nɯ tɕe,  nɯnɯ tu-rtoʁ-nɯ pjɤ-ŋgrɤl.   \\
bowl \textsc{3sg}-inside \textsc{indef.poss}-water \textsc{ipfv}-put.in-\textsc{pl} \textsc{lnk} \textsc{dem:loc} sun \textsc{dem} \textsc{ipfv-caus}-illuminate-\textsc{pl} \textsc{lnk} \textsc{lnk} what about \textsc{ifr}-eat \textsc{dem} \textsc{dem} \textsc{3sg}-inside \textsc{ipfv}:\textsc{down}-look.at-\textsc{pl} dem \textsc{ipfv}-see-\textsc{pl} \textsc{ifr.ipfv}-be.usually.the.case \\
\glt `They used to put water in a bowl and let the sunlight reflect into it; they could see how much (of the sun) had been occulted (`eaten' by the eclipse).' (29-mWBZi, 130)
\end{exe}  

\begin{exe}
\ex
\gll  zgo 	tʰɤstɯɣ 	ja-nnɯ-pɣaʁ-ndʑi, 	tɯ-ci 	tɕʰi 	jarma 	ja-nnɯ-pjɤl-ndʑi 	mɤ-xsi 	ma,       \\
 mountain how.many \textsc{pfv}:3\fl3'-\textsc{auto}-turn.over-\textsc{du} \textsc{indef.poss}-water what about \textsc{pfv}:3\fl3'-\textsc{auto}-cross-\textsc{du} \textsc{neg-genr}:know \textsc{lnk} \\
\glt `It is not known how many mountains and rivers they crossed.'  (2002qajdo, 50)
\end{exe}  

It is possible to combine \forme{tɕʰi jamar} with a adjective to express approximate comparison, as in (\ref{ex:tChi.kWzri}).

\begin{exe}
\ex \label{ex:tChi.kWzri}
\gll lɯlu ɣɯ tɕe ɯʑo ɯ-pʰoŋbu tɕʰi kɯ-zri jamar ɯ-jme nɯ kɯnɤ zri ri \\
cat \textsc{gen} \textsc{lnk} \textsc{3sg} \textsc{3sg.poss}-body what \textsc{nmlz}:S/A-be.long about \textsc{3sg.poss}-tail \textsc{dem} also be.long\textsc{fact} but \\
\glt `The cat, its body is about as long as its tail, but...' (27-qartshAz, 219)
\end{exe}  

In correlative clauses, the pronoun \japhug{tɕʰi}{what} can also be used to refer to a quantity without the adverb \japhug{jamar}{about} (example \ref{ex:tChi.tAkWsci}).

\begin{exe}
\ex \label{ex:tChi.tAkWsci}
\gll  
tɤ-rɟit tɕʰi tɤ-kɯ-sci nɯ ʑo ɣɯ-tɕɤt kɯ-ra pjɤ-ɕti tɕe,   \\
\textsc{indef.poss}-child what \textsc{pfv-nmlz:S/A}-be.born \textsc{dem} \textsc{emph} \textsc{inv}-take.out:\textsc{fact} \textsc{inf:stat}-have.to \textsc{ipfv.ifr}-be:\textsc{affirm} \textsc{lnk} \\
\glt `However many children were born, one had to raise them.' (tApAtso kAnWBdaR I, 9)
\end{exe}  

However, in independent interrogative clauses, \japhug{tɕʰi}{what} cannot refer to quantities. Sentence (\ref{ex:tChi.tosci}) thus can only mean `Was it a boy or a girl' not `How many children did she have?'.

\begin{exe}
\ex \label{ex:tChi.tosci}
\gll  ɯ-rɟit tɕʰi to-sci \\
\textsc{3sg.poss}-child what \textsc{ifr}-be.born \\
\glt `Was it a boy or a girl?'
\end{exe}  


The interrogative \japhug{tɕʰi}{what} occurs in topicalized clauses with an adjective stative verb in perfective form, meaning `as for how X it becomes' as in examples (\ref{ex:tChi.nWjpum}) and (\ref{ex:tChi.tAmbro}).

\begin{exe}
\ex \label{ex:tChi.nWjpum}
\gll tɕʰi nɯ-jpum ki ɕaŋtaʁ ɲɯ-jpum mɯ́j-cʰa \\
what \textsc{pfv}-be.thick \textsc{dem}.\textsc{prox} above \textsc{ipfv}-be.thick \textsc{neg}:\textsc{sens}-can \\
\glt `As for how thick it can grow, it cannot grow thicker than this.' (16-CWrNgo, 154)
\end{exe}


\begin{exe}
\ex \label{ex:tChi.tAmbro}
\gll tɕʰi tɤ-mbro, ʁnɯ-rtsɤɣ ɕaŋtaʁ tu-mbro mɯ́j-cʰa.  \\
what \textsc{pfv}-be.tall two-stairs above \textsc{ipfv}-be.tall \textsc{neg}:\textsc{sens}-can \\
\glt `As for how tall it can grow, it cannot grow taller than two stairs.' (07-paXCi, 8)
\end{exe}

\subsection{\japhug{ɕɯ}{who}}
The interrogative pronoun \japhug{ɕɯ}{who} occurs in questions about the identification of a human referent. It can occur in all syntactic roles, and does not have special ergative or genitive forms (see examples \ref{ex:CW.kW.tWwGmbi} and \ref{ex:CW.GW}). It is the probable cognate of a etymon widespread in the Trans-Himalayan family (for instance, Tibetan \tibet{སུ་}{su}{who}).

\begin{exe}
\ex  \label{ex:CW.tWNu}
\gll ma-tɯ-nɯqaɟy ma ɕɯ tɯ-ŋu mɤ-xsi \\
\textsc{neg:imp}-2-fish \textsc{lnk} who 2-be:\textsc{fact} \textsc{neg-genr}:know   \\
\glt `Don't fish, I don't who you are.' (gesar, 369)
\end{exe}  

\begin{exe}
\ex  \label{ex:CW.kW.tWwGmbi}
\gll  mɤ-ta-mbi nɤʑo qaɕpa ɕɯ kɯ tɯ́-wɣ-mbi    \\
\textsc{neg}-1\fl2-give:\textsc{fact} \textsc{2sg} frog who \textsc{erg} 2-\textsc{inv}-give:\textsc{fact}  \\
\glt `We won't give her to you, who would give her to you, a frog?'   (2002 qaCpa, 09)
\end{exe} 
 
\begin{exe}
\ex  \label{ex:CW.GW}
\gll  ɕɯ ɣɯ ʑo ɲɯ-kʰam-a ra kɯɣe?    \\
who \textsc{gen} \textsc{emph} \textsc{ipfv}-give:III-\textsc{1sg} \textsc{sfp} \\
\glt `Whom should I give (her) to (in marriage)?' (140508 benling gaoqiang de si xiongdi-zh, 222)
\end{exe}  

The pronoun  \japhug{ɕɯ}{who} can be used in one context with non-human referents, when asking about which object (out of two or more) has the highest value as to a property described by the main verb, as in (\ref{ex:CW.kW.YWzrindZi}); in this construction, the verb receives non-singular indexation (§ XXX), such as the dual  \forme{-ndʑi} in this example. Concerning the use of the ergative \forme{kɯ} in this sentence see \citet{jacques16comparative} and § XXX.  

\begin{exe}
\ex  \label{ex:CW.kW.YWzrindZi}
\gll nɯ nɤ-ku ɯ-tɯ-rɲɟi nɯ, aki ɕe-tɕi tɕe, mbro ɯ-jme cʰonɤ tú-wɣ-sɤfsu, ɕɯ kɯ ɲɯ-zri-ndʑi kɯ \\
\textsc{dem} \textsc{2sg}.\textsc{poss}-head \textsc{3sg}.\textsc{poss}-\textsc{nmlz}:\textsc{degree}-be.long  \textsc{sfp} down go:\textsc{fact}-\textsc{1du} \textsc{lnk} horse \textsc{3sg}.\textsc{poss}-tail \textsc{comit}  \textsc{ipfv}-\textsc{inv}-compare who \textsc{erg} \textsc{sens}-be.long-\textsc{du} \textsc{sfp} \\
\glt `Your hair is very long, let us go downstairs, and compare it with a horse's tail.' (2002qaCpa, 292)
\end{exe}  

Forms related to \japhug{ɕɯ}{who} in Japhug include the indefinite pronoun \japhug{ɕɯmɤɕɯ}{whoever, anybody} (\ref{sex:CWmACW}) and \japhug{ɕɯŋarɯra}{each better than the other} (XXX).

\subsection{\japhug{tʰɤstɯɣ}{how many} and \japhug{tʰɤjtɕu}{when}} \label{sec:thAstWG}
To ask about precise quantities, \japhug{tʰɤstɯɣ}{how many} (or `how much') occurs rather than \forme{tɕʰi jamar} as seen above (section \ref{ex:tChi.jamar}).

\begin{exe}
\ex \label{ex:thAstWG.tWkhAm}
 \gll    nɤʑo 	tʰɤstɯɣ 	tɯ-kʰɤm?    \\
 you how.much 2-give[III]:\textsc{fact}  \\
\glt  `How much (money) do you give (for it)?' (Bargaining, 13)
\end{exe} 

It can be used for any countable quantity, including for people, as in (\ref{ex:thAstWG.tWtunW}).

\begin{exe}
\ex \label{ex:thAstWG.tWtunW}
 \gll
tsʰupa tʰɤstɯɣ tɯ-tu-nɯ ŋu? \\
village how.much 2-exist:\textsc{fact}-\textsc{pl} be:\textsc{fact} \\
\glt `How many (people) are you in the village?' (conversation, 140501)
\end{exe} 

The pronoun \forme{tʰɤstɯɣ} has a conjunct form \forme{tʰɤstɯ-} when used with counted nouns (in \ref{ex:thAstWmaR}, with the classifier \japhug{X-maʁ}{size of shoes} from Chinese \zh{码} \forme{mǎ}, see § \ref{sec:other.numeral.prefixes}).

 \begin{exe}
\ex \label{ex:thAstWmaR}
 \gll   nɤ-xtsa nɯ tʰɤstɯ-maʁ tu-tɯ-ŋge ŋu   \\
\textsc{2sg.poss}-shoe \textsc{dem} how.many-size \textsc{ipfv}-2-wear[III] be:\textsc{fact} \\ 
\glt `What is the size of your shoes?'  (Conversation, 2015)
\end{exe} 

Combined with the noun \japhug{tɤ-rʑaʁ}{time}, 	\forme{tʰɤstɯɣ} can be used to ask about a length of time (\ref{ex:thAstWG}).

\begin{exe}
\ex \label{ex:thAstWG}
 \gll   nɤʑo 	tɤ-rʑaʁ 	tʰɤstɯɣ 	jamar 	tɤ-tsu tɕe 	kɤ-tɯ-spa-t?  \\
 you \textsc{indef.poss}-time how.many about \textsc{pfv}-pass \textsc{lnk} \textsc{pfv}-2-be.able-\textsc{pst:tr} \\
\glt   `How long did you need to learn it?' (elicited)
\end{exe} 

The phrase \forme{tɤ-rʑaʁ tʰɤstɯɣ} (or alternatively \forme{tɯtsʰot tʰɤstɯɣ}) in collocation with the verb \japhug{zɣɯt}{reach}, is also employed for asking about clock time, as in (\ref{ex:thAstWG.kozGWt}) (see § \ref{sec:hours}) or dates. %The nouns \japhug{tɤ-rʑaʁ}{time} and \japhug{tɯtsʰot}{time, hour, clock} are not even obligatory, as shown by  

 \begin{exe}
\ex \label{ex:thAstWG.kozGWt}
 \gll   tɤ-rʑaʁ 	tʰɤstɯɣ ko-zɣɯt? \\
  \textsc{indef.poss}-time how.many  \textsc{ifr}-reach \\
  \glt `What is the time?' (heard in context)
  \end{exe} 
    
Questions about time can also be expressed by the pronoun \japhug{tʰɤjtɕu}{when}, as in  (\ref{ex:thAjtCu}) and (\ref{ex:thAjtCu.GW}).

\begin{exe}
\ex \label{ex:thAjtCu}
\gll  tʰɤjtɕu 	lɤ-tɯ-nɯɣe 	pɯ-ŋu 	ra 	nɤ?    \\
 when \textsc{pfv}-2-come.back[II] \textsc{pst.ipfv}-be \textsc{pl} \textsc{sfp} \\
\glt  `When did you come back home?' (taRrdo conversation, 01)
\end{exe} 

As shown by (\ref{ex:thAjtCu.GW}), \japhug{tʰɤjtɕu}{when} can be used with the genitive \forme{ɣɯ}.

\begin{exe}
\ex \label{ex:thAjtCu.GW}
\gll <jipiao> nɯ, tʰɤjtɕu ɣɯ tɤ-tɯ-χtɯ-t? \\
plane.ticket \textsc{dem} when \textsc{gen} \textsc{pfv}-2-buy-\textsc{pst}:\textsc{tr} \\
\glt `Your plane ticket, for what date did you buy it?' (conversation, 2014.03.19)
\end{exe} 

The element \ipa{tʰɤ-} in the pronouns \japhug{tʰɤjtɕu}{when}  and \japhug{tʰɤstɯɣ}{how many} is the \textit{status constructus} form of proto-Japhug \forme{*tʰi}, the inherited form of the pronoun `what' (see § \ref{sec:tChi}). The element \forme{-tɕu} in \japhug{tʰɤjtɕu}{when} is related to the locative \forme{tɕu} (see § XXX).

\subsection{\japhug{ŋotɕu}{where}} \label{sec:NotCu}

The interrogative pronoun \japhug{ŋotɕu}{where} and its reduced form \forme{ŋoj} can be used to ask either about a location (\ref{ex:NotCu.kutWrAZi}), a direction towards (examples \ref{ex:NotCu.tWCe} and \ref{ex:Noj.nari}) or from (\ref{ex:NotCu.jAtWGenW}) a certain place. The second syllable of this pronoun \forme{-tɕu} comes from the locative postposition \forme{tɕu}, but the first part is etymologically obscure.
 
\begin{exe}
\ex \label{ex:NotCu.kutWrAZi}
\gll     ŋotɕu ku-tɯ-rɤʑi?   \\
  where \textsc{pres.egoph}-2-stay \\
\glt `Where are you?" (Conversation, 2005)
\end{exe} 

\begin{exe}
\ex \label{ex:NotCu.tWCe}
\gll   ŋotɕu tɯ-ɕe? \\
 where 2-go:\textsc{fact} \\
\glt `Where are you going to?' (Common greeting used when one meets someone on the road)
 \end{exe} 
 
\begin{exe}
\ex \label{ex:Noj.nari}
\gll     qala ŋoj nɯ-ari  \\
  rabbit where \textsc{pfv:west}-go[II] \\
\glt `Where did the rabbit go?'  (qala2002, 21)
\end{exe} 

\begin{exe}
\ex \label{ex:NotCu.jAtWGenW}
\gll  nɯʑɤra ŋotɕu jɤ-tɯ-ɣe-nɯ? ŋotɕu ɕ-pɯ-tɯ-tu-nɯ? \\
\textsc{2pl} where \textsc{pfv}-2-come[II]-\textsc{pl} where \textsc{transloc-pfv}-2-exist-\textsc{pl} \\
\glt `Where are you from? Where have you been?' (2003sras, 57)
\end{exe} 

With the determiner \forme{nɯ}, the pronoun \forme{ŋotɕu} means `which (of several places)', as in (\ref{ex:NotCu.nW.Nu}) and (\ref{ex:NotCu.nW.Wku.Nu}).

\begin{exe}
\ex \label{ex:NotCu.nW.Nu}
\gll kʰa raŋri ɣɯ ʑo ɯ-ftaʁ pjɤ-tu ɕti ma, tɕe ŋotɕu nɯ ŋu, ŋotɕu nɯ maʁ mɯ-pjɤ-saχsɤl. \\
house each \textsc{gen} \textsc{emph} \textsc{3sg.poss}-mark \textsc{ifr.ipfv}-exist be:\textsc{affirm:fact} \textsc{lnk} \textsc{lnk} where \textsc{dem} be:\textsc{fact}  where \textsc{dem} be:\textsc{fact} \textsc{neg-ifr.ipfv}-be.clear \\
\glt `There was a mark on each of the houses, and one could not tell which (house) was (Alibaba's) and which was not.' (140512 alibaba-zh, 189-190)
\end{exe} 

\begin{exe}
\ex \label{ex:NotCu.nW.Wku.Nu}
\gll qaprɤftsa nɯnɯ, cici jɤ-ari tɕe ɯ-ku ju-z-mɤke, cici tɕe ɯ-jme ju-zmɤke ɲɯ-ɕti tɕe
ŋotɕu nɯ ɯ-ku ŋu, ŋotɕu ɯ-jme ŋu, mɯ́j-saχsɤl \\
centipede \textsc{dem} sometimes \textsc{pfv}-go \textsc{lnk} \textsc{3sg.poss}-head \textsc{ipfv-caus}-be.first[III] \textsc{lnk} \textsc{3sg.poss}-tail \textsc{ipfv-caus}-be.first[III] sens-be:affirm lnk where \textsc{dem} \textsc{3sg.poss}-head be:\textsc{fact} where \textsc{3sg.poss}-head be:\textsc{fact} \textsc{neg.sens}-be/clear \\
\glt `The centipede, when it moves, sometimes its head goes first, sometimesits tail goes first, it is not each to tell which is its head and which is its tail.' (21-qaprAftsa, 12)
\end{exe} 


With generic nouns such as \japhug{tɯrme}{person}, \forme{ŋotɕu} can serve as prenominal determiner to mean `a person from where', as in (\ref{ex:NotCu.tWrme}).

\begin{exe}
\ex \label{ex:NotCu.tWrme}
\gll ŋotɕu tɯrme tɯ-ŋu? \\
where person 2-be:\textsc{fact} \\
\glt `Where are you from?' (2011-05-nyima, 83)
\end{exe} 

In participial relatives with subject participle (in \forme{kɯ-}, see § XXX), \japhug{ŋotɕu}{where} can occur to express relativization of locative adjuncts, as in  (\ref{ex:NotCu.kWtu}); see § XXX for a discussion of the other available constructions.

\begin{exe}
\ex \label{ex:NotCu.kWtu}
\gll kɯ-me nɯra qʰe me,  ŋotɕu kɯ-tu nɯ qʰe kɯ-dɯ\redp{}dɤn tu-ɬoʁ ŋu. \\
\textsc{nmlz}:S/A-not.exist dem:pl lnk not.exist:\textsc{fact} where \textsc{nmlz}:S/A-exist \textsc{dem} \textsc{lnk} \textsc{nmlz}:S/A-\textsc{emph}\redp{}be.many \textsc{ipfv}-come.out be:\textsc{fact} \\
\glt `In (places) where it is not found, there is none, but in (places) where it is found, it grows in great number.' (21-jmAGni, 91)
\end{exe} 

The pronoun \japhug{ŋotɕu}{where} is not exclusively used in question about place or direction, we also find it in the expression in (\ref{ex:NotCu.YWNgrAl}).

 \begin{exe}
\ex \label{ex:NotCu.YWNgrAl}
\gll     kɯki 	ŋotɕu 	ɲɯ-ŋgrɤl?   \\
 this where \textsc{ipfv}-be.usually.the.case \\
\glt `How could this be possible?'  (qajdoskAt 2002, 32)
\end{exe} 

This sentence is used to express indignation (as in Chinese \zh{哪有这样的道理?}).\footnote{In the story from which it is quoted, the husband says this sentence after his wife, quoting the words of a raven, says that she will have luck, not her husband, who thus reacts in anger. }



\section{Indefinite pronouns} \label{sec:indef.pro}
 Japhug has a handful of indefinite pronouns, indicated in Table \ref{tab:indef.pronoun}. They do not form a complete paradigm, and other constructions, in particular generic nouns and free relatives occur to express meanings for which no indefinite pronoun exists (see § XXX).

There are no negative indefinite pronouns, and indefinite pronouns are almost never under the scope of negation (except in translations from Chinese). They also never occur as standard of comparison.\footnote{Examples such as `In Freiburg the weather is better than anywhere in Germany' (\citealt[2]{haspelmath97indef}) would not be expressible with an indefinite pronoun, see § XXX.}
 

\begin{table}[H] \centering
\caption{Indefinite pronouns }\label{tab:indef.pronoun}
\begin{tabular}{lllllll} \lsptoprule
\japhug{ci}{one, someone} \\
\forme{tʰɯci}, \japhug{tʰɯtʰɤci}{something} \\
\japhug{tsʰitsuku}{whatever} \\
\japhug{ɕɯmɤɕɯ}{whoever, anybody} \\
\japhug{ciscʰiz}{somewhere} \\ 
\lspbottomrule
\end{tabular}
\end{table}

\subsection{\japhug{ci}{someone} } \label{sec:ci.someone} 
There is no distinct indefinite pronoun `someone' in Japhug, but the numeral \japhug{ci}{one}, which has many additional functions (indefinite article, modifier and pronoun § \ref{sec:indef.article}, § \ref{sec:other.pro}, § \ref{sec:partitive.pronouns}, § \ref{sec:identity.modifier}, § \ref{sec:one.to.ten} and § XXX), can express this meaning as in (\ref{ci.kW.thaGWt}) and (\ref{ci.kW.tWrdoR}).

\begin{exe}
\ex \label{ci.kW.thaGWt}
\gll 
ɯ-lɤcu nɯtɕu qaʑo kɤtsa ci, ci kɯ kɤ-ntsɣe tha-ɣɯt ɲɯ-ŋu. \\
\textsc{3sg}.\textsc{poss}-upstream \textsc{dem}:\textsc{loc} sheep parent.and.child \textsc{indef} one \textsc{erg} \textsc{inf}-sell \textsc{pfv}:3\fl{}3'-bring \textsc{sens}-be \\
\glt `Upstream from there, (there was) a ewe and her young, that someone had brought them to sell.' (2003kandZislama, 202)
\end{exe}

\begin{exe}
\ex \label{ci.kW.tWrdoR}
\gll 
 tɯ-xpa tɕe ci kɯ tɯ-rdoʁ pjɤ-sat. \\
 one-year \textsc{lnk} one \textsc{erg} one-piece \textsc{ifr}-kill \\
 \glt `One year, someone killed one of them (wils geese).' (22-qomndroN, 43)
\end{exe}

This use of \forme{ci} is rare. The preferred construction to express the meaning `someone' involves the combination of the generic noun \japhug{tɯrme}{person} with the indefinite \forme{ci} (§ \ref{sec:tWrme.indefinite}).

\subsection{\japhug{tʰɯci}{something} } \label{sec:thWci} 
The indefinite pronoun \japhug{tʰɯci}{something} derives from the \textit{status constructus} of the proto-Japhug pronoun \forme{*tʰi} `what' (see \ref{sec:tChi} above) with the indefinite determiner and numeral \japhug{ci}{one}. Note that vowel alternation bleeds the sound change \ipa{*tʰi} \fl{}  \ipa{tsʰi}, otherwise a form such as $\dagger$\forme{tsʰɯci} would have been expected. Its reduplicated form \forme{tʰɯtʰɤci} has an irregular vocalism \ipa{ɤ} ($\dagger$\forme{tʰɯtʰɯci} would have been expected instead).

 It can designate specific referents, whose nature is known to the speaker but unknown to the addressee (as in \ref{ex:thWthAci.Zo.pjWtu}),\footnote{Example (\ref{ex:thWthAci.Zo.pjWtu}) is from a tale about a rabbit tricking a snow leopard; the difference of knowledge between the speaker and the addressee concerning the nature of the `something' is crucial to the plot. }.

\begin{exe}
\ex  \label{ex:thWthAci.Zo.pjWtu}
\gll tu-nɯsman-a jɤɣ ri, mɤʑɯ ɯ-ftɕaka tsuku pjɯ-tu ra wo, tɕe tʰɯtʰɤci ʑo pjɯ-tu ra \\
\textsc{ipfv}-treat-\textsc{1sg} be.possible:\textsc{fact} but yet \textsc{3sg.poss}-manner some \textsc{ipfv}-exist have.to:\textsc{fact} \textsc{sfp} \textsc{lnk} something \textsc{emph} \textsc{ipfv}-exist have.to:\textsc{fact} \\
\glt `I can treat (your illness), but yet another method is needed, something (else) is needed.'  (140427 qala cho kWrtsAg, 48-49)
\end{exe}

The pronoun \forme{tʰɯci} also occurs to refer to things whose name is unknown to the speaker (as in \ref{ex:gser.zhwa} and \ref{ex:thWci.khWtsa}), even if he/she may have seen the object.
 
\begin{exe}
\ex \label{ex:gser.zhwa}
\gll tɕe nɯ nɯ-rte nɯ tɕʰi ŋu ma tʰɯci ci ``-ʑa" tu-ti ŋu, χsɤrʑa! \\
\textsc{lnk} \textsc{dem} \textsc{3pl.poss}-hat \textsc{dem} what be:\textsc{fact} \textsc{lnk} something \textsc{indef} ... \textsc{ipfv}-say be:\textsc{fact} golden.hat \\
\glt `How is their hat (called), something in `ʑa'.... yes, \tibet{གསེར་ཞྭ་}{gser.ʑʷa}{golden hat}!' (30-mboR, 102)
\end{exe}

\begin{exe}
\ex \label{ex:thWci.khWtsa}
\gll  tɕe tɤ-ndʑɯɣ nɯ kɯnɤ, tʰɯci kʰɯtsa kɯ-fse ɯ-ŋgɯ tu-rku-nɯ tɕe   \\
\textsc{lnk} \textsc{indef.poss}-resin \textsc{dem} also something bowl \textsc{nmlz}:S/Abe.like \textsc{3sg}-inside \textsc{ipfv}-put.in-\textsc{pl} \textsc{lnk}   \\
\glt `The resin, people put it into something like a bowl.'' (07-tAtho, 44)
\end{exe}

It is also used for non-specific referents whose nature is entirely unknown, as in  (\ref{ex:thWthAci.tannWrkunW}) and (\ref{ex:thWmqlaR}).

\begin{exe}
\ex \label{ex:thWthAci.tannWrkunW}
\gll   tɕe mɤʑɯ tʰɯtʰɤci ta-nnɯ-rku-nɯ kɯma  \\
\textsc{lnk} yet something \textsc{pfv}:3\fl3'-\textsc{auto}-put.in-\textsc{pl} \textsc{sfp} \\
\glt `They also probably gave them something else.' (02-deluge2012, 120)
 \end{exe}
 
  \begin{exe}
\ex \label{ex:thWmqlaR}
\gll 
 tʰɯ-mqlaʁ tʰɯ-mqlaʁ ma tʰɯci fse ci ndʐa cʰɯ-ɕe ɕti \\
 \textsc{imp}:swallow  \textsc{imp}:swallow \textsc{lnk} something be.like:\textsc{fact} \textsc{indef} reason \textsc{ipfv:downstream}-go be:\textsc{affirm:fact} \\
\glt `Swallow it, swallow it, it comes down (into your throat) for some reason.' (2005-stod-kunbzang, 87)
  \end{exe}

The reduplicated form \forme{tʰɯtʰɤci}, especially in combination with \japhug{fse}{be like}, can also mean `whatever (happened)', as in (\ref{ex:thWthAci.kWfse}).
 
 \begin{exe}
\ex \label{ex:thWthAci.kWfse}
\gll  slama ra ɣɯ tʰɯtʰɤci kɯ-fse, kɤ-rɤ-βzjoz ra ɲɯ-stu mɯ́j-stu-nɯ, nɯ-stu ɲɯ-nɤma-nɯ mɯ́j-nɤma-nɯ,  nɯnɯra nɯ-pʰama ra nɯ-ɕki kɯ-rɤfɕɤt ɲɯ-ra. \\
student \textsc{pl} \textsc{gen} something \textsc{nmlz}:S/A-be.like \textsc{inf-antipass}-learn \textsc{pl} \textsc{sens}-try.hard-\textsc{pl} \textsc{neg:sens}-try.hard-\textsc{pl} \textsc{3sg.poss}-right \textsc{sens}-do-\textsc{pl} \textsc{neg:sens}-do-\textsc{pl} \textsc{dem:pl} \textsc{3pl.poss}-parent \textsc{pl} \textsc{3pl-dat} \textsc{genr}:S/P-tell \textsc{sens}-have.to \\
\glt `One has to tell the parents whatever concerns the students, whether they study seriously and try hard or not.'   (150901 tshuBdWnskAt, 18)
 \end{exe}
  
The non-reduplicated form \forme{tʰɯci} occurs in a correlative construction with the form \forme{mɯci} to mean `this and that', an expression that is used especially in reporting speech from another person when the speaker does not want to bother reporting in details the exact words that have been said.

\begin{exe}
\ex \label{ex:thWci.mWci}
\gll 
tʰɯci nɤme-a ra, mɯci nɤ-me-a ra \\
something do[III]:fact-\textsc{1sg} have.to:\textsc{fact} something do[III]:fact-\textsc{1sg} have.to:\textsc{fact} \\
\glt `I have to do this and that (so I cannot do X)' (elicitation)
 \end{exe}
 
The pronoun \japhug{tʰɯci}{something}  can also occur as head of a relative clause as in (\ref{ex:thWci.khWtsa}) above with the relative \forme{tʰɯci kʰɯtsa kɯ-fse} `something which is like a bowl'). This use is most common in texts translated from Chinese, with the indefinite article \japhug{ci}{one} (§ \ref{sec:indef.article}) following relative clause, as in (\ref{ex:thWci.akAspa}). 

\begin{exe}
\ex \label{ex:thWci.akAspa}
\gll  laχɕi ci pjɯ-βzjoz-a, tʰɯci a-kɤ-spa ci a-pɯ-tu ɲɯ-ra  \\
 trade \textsc{indef} \textsc{ipfv}-learn-\textsc{1sg} something \textsc{1sg.poss-nmlz:P}-be.able \textsc{indef} \textsc{irr-pfv}-exist \textsc{sens}-have.to \\
 \glt `I have to learn a trade, to have something I am able to do.' (150902 luban-zh, 12)
\end{exe}
 
With stative verbs in the relative as in (\ref{ex:thWci.kApGWlu}), this construction has a low degree meaning `a little X'.

\begin{exe}
\ex \label{ex:thWci.kApGWlu}
\gll   tɕe kɯ-wɣrum ɯ-ŋgɯz kɯnɤ tʰɯci kɯ-ɤpɣɯlu kɯ-fse ci ŋu tɕe, \\
\textsc{lnk} \textsc{nmlz}:S/A-be.white \textsc{3sg}.\textsc{poss}-inside:\textsc{loc} also something \textsc{nmlz}:S/A-greyish \textsc{nmlz}:S/A-be.like \textsc{indef} be:\textsc{fact} \textsc{lnk} \\
\glt `(Silver) is white with a little greyish colour.' (30-Com, 176)
\end{exe}
  
 The reduplicated form of the the indefinite pronoun \forme{tʰɯtʰɤci.totia} can be used as an interrogative pronoun, as in (\ref{ex:thWthAci.totia}). This construction is similar in meaning to Chinese \ch{一些什么}{yīxiēshénme}{what kinds of things}, and is attested in particular with the verbs \japhug{ti}{say} and \japhug{ra}{have to, need}. By using this form, the speaker implies that the addressee necessarily knows the answer to the question. For instance,  in (\ref{ex:thWthAci.totia}), a sentence from a text enumerating the mountain names in Kamnyu, the names had been written before hand on a piece of paper, and I was reading them one by one to Tshendzin; given the fact that the name had been written down, it was obvious that I necessarily knew the answer to that question.
  
 \begin{exe}
\ex \label{ex:thWthAci.totia}
 \gll  nɯ ɯ-pa tʰɯtʰɤci to-ti-a? \\
 \textsc{dem} \textsc{3sg}.\textsc{poss}-down something \textsc{ifr}-say-\textsc{1sg} \\
 \glt `What did I say after that?' (140522 Kamnyu zgo, 58)
\end{exe}

 There are very marginal examples of \japhug{tʰɯci}{something} used as an indefinite prenominal determiner (§ \ref{sec:indefinite}).

The pronoun \japhug{tʰɯci}{something} can take various modifiers, for instance the identity modifier \japhug{kɯmaʁ}{other} (§ \ref{sec:identity.modifier})  as in (\ref{ex:kWmaR.thWci}). 
 
\begin{exe}
\ex \label{ex:kWmaR.thWci}
\gll    ki mbro ki ɲɯ-kɤ-ntsɣe tɕe, [kɯmaʁ tʰɯci] ɲɯ-kɤ-sɤndu to-nɯkrɤz-ndʑi \\
\textsc{dem:prox} horse \textsc{dem:prox} \textsc{ipfv-inf}-sell \textsc{lnk} other  something   \textsc{ipfv-inf}-exchange \textsc{ifr}-discuss-\textsc{du} \\
 \glt `They discussed about selling their horse, and exchanging it for something else.' (150822 laoye zuoshi zongshi duide-zh, 41)
\end{exe}

No example of \forme{tʰɯci} with topic markers contributing to mark definiteness such as \forme{nɯ} or \forme{iɕqʰa} (§ \ref{sec:definiteness}) have been found in the corpus.

\subsection{\japhug{tsʰitsuku}{whatever}} \label{sec:tshitsuku}
The pronoun \japhug{tsʰitsuku}{whatever} combines the  interrogative pronoun \japhug{tsʰi}{what} (replaced by \japhug{tɕʰi}{what}, a borrowing from Tibetan in Kamnyu Japhug, but still attested in Mangi village, see \ref{sec:tChi} above) with the mid-scalar quantifier  \japhug{tsuku}{some} (see § \ref{sec:tsuku}; also found as a partitive pronoun, § \ref{sec:partitive.pronouns}).  Unlike  \japhug{tʰɯci}{something}, is not used for specific referents.  Example (\ref{ex:tshitsuku.kuwGsqa}) illustrates its most common use. The variant form \forme{tʰitsuku}, without the sound change \forme{*tʰi} \fl{} \forme{tsʰi} is also used by speakers of the Kamnyu dialect.

\begin{exe}
\ex \label{ex:tshitsuku.kuwGsqa}
\gll  
kɤ-nɯβlɯ tɕe ɕkrɤz wuma ʑo pe ma nɯnɯ, nɯnɯ ɣɯ ɯ-smɯmba nɯ sɤɕke, tɕendɤre tsʰitsuku kú-wɣ-sqa tɕe, ʑaʑa ʑo ku-ɣɤ-smi cʰa, tsʰitsuku tú-wɣ-sɯ-ɤla tɕe, ʑaʑa tu-sɯ-ɤle cʰa. \\
\textsc{inf}-burn \textsc{lnk} oak really \textsc{emph} be.good:\textsc{fact} \textsc{lnk} \textsc{dem} \textsc{dem} \textsc{gen} \textsc{3sg.poss}-flame \textsc{dem} burning \textsc{lnk} whatever \textsc{ipfv-inv}-cook \textsc{lnk} soon \textsc{emph}  \textsc{ipfv-caus}-be.cooked can:\textsc{fact} whatever \textsc{ipfv-inv-caus}-be.boiling \textsc{lnk} soon  \textsc{ipfv-caus-caus}-be.boiling[III] can:\textsc{fact} \\
\glt `For burning, oak is very good, the flames (from its wood) are very hot, whatever one cooks, it cooks it quickly, whatever one boils, it boils it quickly.' (08-CkrAz, 4-5)
\end{exe}
%nɤʑo kɯ rcanɯ, tɯ-tso ɯ-tɯ-me nɯ, maka /ji/ tɕi-rca jɤ-ɣi tɕe, nɯ sɤznɤ tshitsuku a-pɯ-tɯ-mtɤm tɕe a-pɯ-tɯ-nɯtɯtso ɲɯ-mna
%140510_fengwang, 15
 
In many cases, it is better translated as `all kinds of things', as in (\ref{ex:tshitsuku.YWznAme}).

\begin{exe}
\ex \label{ex:tshitsuku.YWznAme}
\gll  
tɕe nɯtɕu kɯnɤ ɯ-jaʁ ɯ-ntsi tɤɲi pjɯ-sɤtse  ɯ-jaʁ ɯ-ntsi kɯ tsʰitsuku ɲɯ-z-nɤme qhe, ʑara nɯ-ndzɤtsʰi tu-βze, fsapaʁ ra nɯ-ndzɤtsʰi ɲɯ-βze \\
\textsc{lnk} \textsc{dem:loc} also \textsc{3sg.poss}-hand \textsc{3sg.poss}-one.of.a.pair staff \textsc{ipfv}-plant[III]  \textsc{3sg.poss}-hand \textsc{3sg.poss}-one.of.a.pair \textsc{erg} whatever \textsc{ipfv-caus}-do[III] \textsc{lnk} \textsc{3pl} \textsc{3pl.poss}-food \textsc{ipfv}-make[III] animal \textsc{pl} \textsc{3pl.poss}-food \textsc{ipfv}-make[III]  \\
\glt `Even like that, she supports herself with a staff in one hand, and with the other hand she does all kinds of things, makes their food, she makes food for the animals.' (14-tApitaRi, 54)
\end{exe}
%tshitsuku ɲɯ́-wɣ-mbi, ɲɯ́-wɣ-jtshi, tú-wɣ-raχtɕɤz tɕe, ʑɯrɯʑɤri tɕe tɕendɤre ku-kɯ-nɯfse ɲɯ-ŋu

As other indefinite pronouns, \japhug{tsʰitsuku}{whatever} is not normally used with negation, but such sentences do occur in the corpus in translations from Chinese, as (\ref{ex:tshitsuku.mWtoti}). They are not not idiomatic Japhug, and even only marginally grammatical.

\begin{exe}
\ex \label{ex:tshitsuku.mWtoti}
\gll   tsʰitsuku mɯ-to-ti, qʰe tɕendɤre kɯ-rŋgɯ jo-nɯɕe qʰe ko-nɯ-rŋgɯ. \\
whatever \textsc{neg-ifr}-say \textsc{lnk} \textsc{lnk} \textsc{nmlz}:S/A-lay.down \textsc{ifr}-go.back \textsc{lnk} \textsc{ifr-auto}-lay.down \\
\glt `He did not said anything, went back to sleep and laid down in bed.' (150902 qixian-zh, 91)
\end{exe}

 \subsection{\japhug{ɕɯmɤɕɯ}{whoever, anybody}} \label{sex:CWmACW}
 There is no indefinite pronoun for human referents `somebody' in Japhug  corresponding to \japhug{tʰɯci}{something} -- a generic noun with the indefinite determiner \japhug{ci}{one} such as \forme{tɯrme ci} `a man' is used instead. There is nevertheless a `free choice' pronoun \japhug{ɕɯmɤɕɯ}{whoever, anybody} (see \citealt[48-52]{haspelmath97indef} on the differences with universal quantifiers), which however is not very common. As example (\ref{ex:CWmACW.kW}) shows, it can take the ergative \forme{kɯ}, and the verb receives plural indexation (see § XXX for the use of the plural for indefinite referents). 
 
 \begin{exe}
\ex \label{ex:CWmACW.kW}
\gll tɕaχkɤr kʰɯtsa nɯ ʁo tʰam qʰe ɕɯmɤɕɯ kɯ ku-nɯ-ntɕʰoz-nɯ ɕti \\
tin bowl \textsc{dem} \textsc{advers} now \textsc{lnk} anybody \textsc{erg} \textsc{ipfv-auto}-use-\textsc{pl} be:\textsc{affirm:fact} \\
\glt `Now anybody can use tin bowls.' (unlike before, when only important people could use it, 160702 khWtsa, 26)
 \end{exe}
 
  \subsection{\japhug{ciscʰiz}{somewhere}} \label{sec:cischiz}
The indefinite pronoun \japhug{ciscʰiz}{somewhere} comprises the indefinite \japhug{ci}{one} and the approximate locative \forme{(s)cʰiz} (see § \ref{sec:approximate.locative}). It occurs with or without the locative postposition \forme{ri}, as in (\ref{ex:cischiz}) and (\ref{ex:cischiz.ri}). It can refer to static location, or motion from or towards a direction.

 \begin{exe}
\ex \label{ex:cischiz}
\gll
ciscʰiz, tɤtsʰoʁ ɯ-taʁ kɯ-fse, tɤ-jtsi ɯ-taʁ kɯ-fse, nɯnɯra, nɯnɯtɕu kú-wɣ-βraʁ tɕe, \\
somewhere nail \textsc{3sg-on} \textsc{nmlz}:S/A-be.like, \textsc{indef.poss}-pillar \textsc{3sg-on} \textsc{nmlz}:S/A-be.like,  \textsc{dem:pl} \textsc{dem:loc} \textsc{ipfv-inv}-attach \textsc{lnk} \\
\glt `One attaches (their noseband) somewhere, like on a nail, on a pillar.' (150902 kAxtCAr, 6)
 \end{exe}
 
 \begin{exe}
\ex \label{ex:cischiz.ri}
\gll nɯnɯ ciscʰiz ri tú-wɣ-z-nɯndzɯ tɕe ɲɯ́-wɣ-ta.\\
\textsc{dem} somewhere  \textsc{loc} \textsc{ipfv-inv-caus}-be.vertical \textsc{lnk} \textsc{ipfv:west-inv}-put\\
\glt `One puts it vertically somewhere.' (14-tasa, 62)
 \end{exe}
  
 
\subsection{Interrogative pronouns used as indefinites} \label{sec:interrogative.indef}
Non-specific indefinite referents can be expressed by interrogative pronouns in Japhug. One type of construction where this function is attested is correlatives, as in (\ref{ex:NotCu.lAtWrNgW}) and (\ref{ex:thAjtCu.fsaN}).
 
\begin{exe}
\ex \label{ex:NotCu.lAtWrNgW}
\gll a-pɯwɯ, ŋotɕu lɤ-tɯ-rŋgɯ ʑo qhe, nɯtɕu rɤʑi-tɕi ŋu ma, \\
\textsc{1sg}-donkey where \textsc{pfv:upstream}-2-lay.down \textsc{emph} \textsc{lnk} \textsc{dem:loc} stay:\textsc{fact}-\textsc{1du} be:\textsc{fact} because \\
\glt `My donkey, we will stay wherever you lay down.' (28-qAjdoskAt, 38)
\end{exe}  
 
\begin{exe}
\ex \label{ex:thAjtCu.fsaN}
\gll tʰɤjtɕu fsaŋ kɤ-ta tɤ-ra ʑo tɕe nɯnɯ tu-βlɯ-nɯ tɕe, \\
when fumigation \textsc{inf}-put \textsc{pfv}-have.to \textsc{emph} \textsc{lnk} \textsc{dem} \textsc{ipfv}-burn-\textsc{pl} \textsc{lnk} \\
\glt `Whenever there is need to make fumigations, they burn it.' (15-YaBrWG, 31)
\end{exe}  

This meaning also occurs in infinitival subordinate clauses, in particular in the expression \forme{tɕʰi kɤ-cʰa} `do whatever X can to Y', as in example (\ref{ex:tChi.kAcha.Zo}).

\begin{exe}
\ex \label{ex:tChi.kAcha.Zo}
\gll  tɕʰi kɤ-cʰa ʑo cʰɯ-pʰɯt-nɯ, \\
what \textsc{inf}-can \textsc{emph} \textsc{ipfv}-remove-\textsc{pl} \\
\glt `People do whatever they can to remove (this plant).' (12-Zmbroko, 119)
\end{exe}

The most common construction to express unspecified referents is built by combining an interrogative pronoun, the verb verb with partial reduplication on the last syllable of the stem, and in most cases the autobenefactive \forme{nɯ-} prefix (this use of the autobenefactive reminds of its occurrence in concessive conditionals, see § XXX).  

With \japhug{tɕʰi}{what}, this construction expresses the meaning `whatever; no matter what' in intransitive subject (\ref{ex:tChi.pWnWNWNu}), object (\ref{ex:tChi.tAtWnWtWtWt}) or semi-object (\ref{ex:tChi.kWstWstua}, see § XXX) functions.

\begin{exe}
\ex \label{ex:tChi.pWnWNWNu}
\gll lú-wɣ-sti tɕe tɕe nɯ ɯ-ŋgɯ tɕʰi pɯ-nɯ-ŋɯ\redp{}ŋu nɯ ɲɯ-mɲɤt mɯ́j-cʰa \\
\textsc{ipfv-inv}-block \textsc{lnk} \textsc{lnk} \textsc{dem} \textsc{3sg}-inside what \textsc{pst.ipfv}-\textsc{auto}-be \textsc{dem} \textsc{ipfv}-be.spoiled \textsc{neg:sens}-can \\
\glt `One seals (its opening) and whatever (food) is inside will not be spoiled.' (150828 kodAt, 14)
\end{exe}  

\begin{exe}
\ex \label{ex:tChi.tAtWnWtWtWt}
\gll tɕʰi tɤ-tɯ-nɯ-tɯ\redp{}tɯt ʑo ju-ɣi ɕti \\
what \textsc{pfv}-2-\textsc{auto}-say[II] \textsc{emph} \textsc{ipfv}-come be:\textsc{affirm:fact} \\
\glt  `Whatever you say will come.' (2003twxtsa, 117)
\end{exe}  

\begin{exe}
\ex \label{ex:tChi.kWstWstua}
\gll nɤʑo tɕʰi kɯ-stɯ\redp{}stu-a ʑo ŋu \\
\textsc{2sg} what 2\fl1-do.like-\textsc{1sg} \textsc{emph} be:\textsc{fact} \\
\glt `Whatever you do to me (will be fine).' (28-qAjdoskAt, 40)
\end{exe}


With \japhug{ɕɯ}{who}, the construction means `whoever; regardless of who; no matter who'. Examples are found with the non-specific referent in intransitive subject (\ref{ex:CW.pWnWNWNu}), transitive subject (\ref{ex:CW.kW.panWmtWmtonW}) or oblique argument (\ref{ex:CW.GW.nWnWkhWkhota}) functions. Note that it often occurs with plural indexation.

\begin{exe}
\ex \label{ex:CW.pWnWNWNu}
\gll tɯsqar nɯ kɯrɯ tɯrme ra mɤ-kɯ-rga maka ʑo me, ɕɯ pɯ-nɯ-ŋɯ\redp{}ŋu ʑo, tɯsqar a-pɯ-tu qʰe, tɕendɤre, nɯ-kɤ-ndza tu-rtaʁ ɕti, \\
tsampa \textsc{dem} Tibetan person \textsc{pl} \textsc{neg}-\textsc{nmlz}:S/A-like at.all \textsc{emph} not.exist:\textsc{fact} who  \textsc{pst.ipfv-auto}-be \textsc{emph} tsampa \textsc{irr}-\textsc{ipfv}-exist \textsc{lnk} \textsc{lnk} \textsc{3pl.poss}-\textsc{nmlz}:P-eat \textsc{ipfv}-be.enough be:\textsc{affirm}:\textsc{fact} \\
\glt `Among Tibetan people, everybody likes tsampa (`there is no one who does not like it'), no matter who, if they have tsampa, they have enough to eat.' (2002tWsqar2, 9)
\end{exe}

\begin{exe}
\ex \label{ex:CW.kW.panWmtWmtonW}
\gll tɕe ɕɯ kɯ pa-nɯ-mtɯ\redp{}mto-nɯ ʑo kɯki ɣɯ, nɯ-kʰa ɣɯ nɯ-mɯntoʁ nɯ cʰondɤre nɯ-ɕoŋpʰu nɯra tɕe, mɤʑɯ nɯ-<cai> nɯra, pjɯ-ɣɤmɯ-nɯ tɕe, \\
\textsc{lnk} who \textsc{erg} \textsc{pfv}:3\fl3'-see-\textsc{pl} \textsc{emph} \textsc{dem.prox} \textsc{gen} \textsc{3pl.poss}-house \textsc{gen} \textsc{3pl.poss}-flower \textsc{dem} \textsc{comit} \textsc{3pl.poss}-tree \textsc{dem:pl} \textsc{lnk} yet \textsc{3pl.poss}-vegetable \textsc{dem:pl} \textsc{ipfv}-praise-\textsc{pl} \textsc{lnk} \\
\glt `Whoever saw it, the flowers and the trees and the vegetables of their house, they praised it.' (150824 yuanding-zh, 30)
\end{exe}

\begin{exe}
\ex \label{ex:CW.GW.nWnWkhWkhota}
\gll tɤɕime ri tɯ-rdoʁ ma me, tɕendɤre nɯʑo ɕɯ ɣɯ nɯ-nɯ-kʰɯ\redp{}kʰo-t-a ʑo mɯ́j-nɯtɯtʂaŋ ɕti tɕe, \\
lady also one-piece apart.from not.exist:\textsc{fact} \textsc{lnk} \textsc{2pl} who \textsc{gen} \textsc{pfv}-\textsc{auto}-give-\textsc{pst:tr-1sg} \textsc{emph} \textsc{neg:sens}-be.fair be:\textsc{affirm:fact} \textsc{lnk} \\
\glt `There is only one princess, and regardless of whom among you all I give her hand to, it will be unfair.' (140508 benling gaoqiang de si xiongdi-zh, 227)
\end{exe}

Examples of this construction are also found with the pronoun \japhug{ŋotɕu}{where}, with the meaning `no matter where, wherever' (location or direction from or to).

\begin{exe}
\ex \label{ex:NotCu.nWnWlhWlhoR}
\gll ŋotɕu nɯ-ɬɯ\redp{}ɬoʁ ʑo wuma ʑo sɤɣdɯɣ \\
where \textsc{pfv}-come.out \textsc{emph} really \textsc{emph} be.annoying:\textsc{fact} \\
\glt `No matter where it grows, it is very annoying.' (5-khArWm, 19)
\end{exe}

\begin{exe}
\ex \label{ex:nWGtWta}
\gll 
ŋotɕu 	nɯ́-wɣ-tɯ\redp{}ta 	ʑo 	kɯpɤz 	ɲɯ-βze 	ɲɯ-ɕti\\
 where \textsc{ipfv-inv}-\textsc{indefinite}\textasciitilde{}put \textsc{emph} type.of.bug \textsc{ipfv}-grow \textsc{sens}-be.\textsc{assert}\\
\glt `Bugs will grow wherever you put (the meat).' (28-kWpAz, 48)
\end{exe}
 
 The pronoun \forme{ŋotɕu} in \textit{status constructus} form \forme{ŋɤtɕɯ-} occurs in the delocutive expression \japhug{ŋɤtɕɯkɤti,kʰɯ}{obey to everything} in a compound with the infinitive \forme{kɤ-ti} of the verb \japhug{ti}{say}, and in collocation with \japhug{kʰɯ}{agree}, as in (\ref{ex:NAtCWkAti}).\footnote{the causative \japhug{ŋɤtɕɯkɤti,sɯkʰɯ}{cause to obey to everything} also exists.} 
This expression originates presumably from a phrase such as `agree (\forme{kʰɯ}) to whatever (\forme{ŋotɕu}) he says (\forme{ti})', though the pronoun \japhug{tɕʰi}{what}, not \japhug{ŋotɕu}{where} is used in Japhug in the construction meaning `whatever' as in examples (\ref{ex:tChi.pWnWNWNu}) to (\ref{ex:tChi.kWstWstua}) above.

 \begin{exe}
\ex \label{ex:NAtCWkAti}
\gll  ɯ-tɕɯ kɯβde nɯra wuma ʑo ŋɤtɕɯkɤti pjɤ-kʰɯ-nɯ  \\
3sg.poss-son four dem:pl really emph obey.to.everything(1) \textsc{ifr.ipfv}-obey.to.everything(2)-\textsc{pl} \\
\glt `His four sons were very obedient.' (140508 benling gaoqiang de si xiongdi-zh, 15)
\end{exe} 

Good examples of this construction are not found in the corpus with the other interrogative pronouns, \japhug{tʰɤjtɕu}{when} or \japhug{tʰɤstɯɣ}{how many}, but they can also be used in the same way.

No example of multiple partitive use of interrogatives (as in French \textit{qui apportait un fromage, qui un sac de noix}, \citealt[177]{haspelmath97indef} ) is attested in the data at hand; mid-scalar quantifiers such as \japhug{tsuku}{some} occur instead as partitive pronouns (§ \ref{sec:partitive.pronouns}). 

\subsection{Generic nouns as indefinite pronouns} \label{sec:tWrme.indefinite}
The generic noun \japhug{tɯrme}{person} can be used in the meaning `someone' or `other people', especially in genitival constructions as in (\ref{ex:tWrme.WkhApa.zW}).
 
\begin{exe}
\ex \label{ex:tWrme.WkhApa.zW}
\gll tɯrme ɯ-kʰɤpa zɯ, ki kɯ-fse tɯ-rʑaʁ lu-znɯfsoʁspat-a ku-omdzɯ-a. \\
people \textsc{3sg}.\textsc{poss}-yard \textsc{loc} \textsc{dem}.\textsc{prox} \textsc{nmlz}:S/A-be.like one-night \textsc{ipfv}-do.the.whole.night-\textsc{1sg} \textsc{ipfv}-sit-\textsc{1sg} \\
\glt `I would spend an entire night from dusk till dawn sitting in someone's animal yard.' (2010-histoire09, 34)
\end{exe} 

\section{Quantifiers} \label{sec:quantifiers.pronouns} \label{sec:aRandWndAt}


\subsection{Universal quantifiers}
Several quantifiers meaning `all' exist in Japhug (§ \ref{sec:universal.quant} and XXX). Among them, \japhug{kɤsɯfse}{all} can be used in the meaning `everybody', as in example (\ref{ex:kAsWfse.kW}).

\begin{exe}
\ex \label{ex:kAsWfse.kW}
\gll kɤsɯfse kɯ ʑo ta-nɯ maʁ \\
all \textsc{erg} \textsc{emph} put:\textsc{fact}-\textsc{pl} not.be:\textsc{fact} \\
\glt `Not everybody puts it.' (160706 thotsi, 21)
\end{exe}

The less common form \japhug{mɲɯrɯri}{everybody, each person} (from Tibetan \tibet{མི་རེ་རེ་}{mi.re.re}{each man} also serves as a universal quantifier, as in (\ref{ex:mYWrWri}).

\begin{exe}
\ex \label{ex:mYWrWri}
\gll mɲɯrɯri kɯ `nɯnɯ ɲɯ-pe' ntsɯ to-ti-nɯ \\
everybody \textsc{erg} \textsc{dem} \textsc{sens}-be.good always \textsc{ifr}-say-\textsc{pl} \\
\glt `Everybody said `It is nice!'' (140521 huangdi de xinzhuang, 214)
\end{exe}

The interrogative pronoun \japhug{tɕʰi}{what}, appears with the plural demonstrative determiner \forme{kɯra} to mean `everything', as in example (\ref{ex:tChi.kWra}). It is not possible to express meanings such as `everybody' or `everywhere'  by combining the other pronouns \japhug{ɕɯ}{who} or \japhug{ŋotɕu} with the same demonstrative.

\begin{exe}
\ex \label{ex:tChi.kWra}
\gll ɯʑo tɕʰi kɯra ko-tso \\
\textsc{3sg} what \textsc{dem:prox:pl} \textsc{ifr}-understand \\
\glt `He understood everything.' (2002qajdoskAt, 115)
\end{exe}

There are two words, \forme{aʁɤndɯndɤt} and \forme{ŋotɕuŋɤndɤt}, which can be translated as `everywhere'.

The word \japhug{aʁɤndɯndɤt}{everywhere} is mainly used adverbially with the emphatic \forme{ʑo} as in (\ref{ex:aRAndWndAt.ʑo}), but there are examples where it occurs with the locative postposition \forme{ri} as (\ref{ex:aRAndWndAt.ri}) like a locative noun phrase, an observation suggesting that it can be analyzed as a pronoun, though no sentences with \japhug{aʁɤndɯndɤt}{everywhere} as subject (like `everywhere is quiet') are found in the corpus.

 \begin{exe}
\ex \label{ex:aRAndWndAt.ʑo}
\gll  aʁɤndɯndɤt ʑo kʰa ra cʰɯ-rɤpɯ. tɯ-ji ɯ-ngɯ ra cʰɯ-rɤpɯ, \\
everywhere \textsc{emph} house \textsc{pl} \textsc{ipfv}-litter \textsc{indef.poss}-field \textsc{3sg}-inside \textsc{pl} \textsc{ipfv}-litter \\
\glt `Mice have litter everywhere, in the house, in the fields.' (27-spjaNkW, 166)
\end{exe} 

 \begin{exe}
\ex \label{ex:aRAndWndAt.ri}
\gll nɯ fse ʑo aʁɤndɯndɤt ri tu-nnɯ-ɬoʁ qʰe, ɯ-zrɤm nɯra kɯ-tu maŋe. \\
\textsc{dem} be.like:\textsc{fact} \textsc{emph} everywhere \textsc{loc} \textsc{ipfv}-\textsc{auto}-come.out \textsc{lnk} \textsc{3sg.poss}-root \textsc{dem:pl} \textsc{nmlz}:S/A-exist not.exist:\textsc{sens} \\
\glt `It grows simply like that everywhere, it has no roots.' (20-sWrna,76)
\end{exe} 

When \japhug{aʁɤndɯndɤt}{everywhere} occurs under the scope of negation, it never expresses the meaning `nowhere', as shown by (\ref{ex:aRAndWndAt.me}) and (\ref{ex:aRAndWndAt.juCenW}).

\begin{exe}
\ex \label{ex:aRAndWndAt.me}
\gll stɤmku nɯra, tɯ-ci ɯ-rkɯ nɯra tu ma aʁɤndɯndɤt sthɯci me \\
plain \textsc{dem:pl} \textsc{indef.poss}-water \textsc{3sg.poss}-side  \textsc{dem:pl} exist:\textsc{fact} \textsc{lnk} everywhere so.much not.exist:\textsc{fact} \\
\glt `It is found in plains, or next to rivers, but it is not found everywhere.' (14-sWNgWJu, 53)
\end{exe} 

\begin{exe}
\ex \label{ex:aRAndWndAt.juCenW}
\gll tɕe ɕɤr tɕe cʰɯ-nɯ-ɬoʁ-nɯ tɕe, aʁɤndɯndɤt ju-ɕe-nɯ mɤ-kɯ-kʰɯ \\
\textsc{lnk} night \textsc{lnk} \textsc{ipfv:downstream-auto}-come.out-\textsc{pl} \textsc{lnk} everywhere \textsc{ipfv}-go-\textsc{pl} \textsc{neg}-\textsc{nmlz}:S/A-be.\textsc{possible} \\
\glt `(They make it) to prevent (animals) from coming out at night and going everywhere.' (150902 mkhoN, 21)
\end{exe} 

The word  \japhug{ŋotɕuŋɤndɤt}{everywhere} is semantically very close to \japhug{aʁɤndɯndɤt}{everywhere} but rarer; it may also be translated as `in all kinds of places'. It contains a partially reduplicated form of the interrogative pronoun \japhug{ŋotɕu}{where} (\ref{sec:NotCu}).

 \begin{exe}
\ex \label{ex:NotCuNondAt}
\gll ɕkrɤz ɯ-ŋgɯ tɕi ɲɯ-ɬoʁ, tɯrgi ɯ-ŋgɯ tɕi ɲɯ-ɬoʁ, ʑmbri ɯ-ŋgɯ tɕi ɲɯ-ɬoʁ,  mbraj ɯ-ŋgɯ tɕi ɲɯ-ɬoʁ, tɕe sɤjku sɯŋgɯ nɯra tɕi ɲɯ-ɬoʁ, tɕe ŋotɕuŋondɤt ʑo ɣɤʑu ɕti ri, stɤmku me, sɯŋgɯ ʁɟa ʑo tu-ɬoʁ ɲɯ-ŋu. \\
oak \textsc{3sg-inside} also \textsc{sens}-come.out fir \textsc{3sg-inside} also \textsc{sens}-come.out willow \textsc{3sg-inside} also \textsc{sens}-come.out red.birch \textsc{3sg-inside} also \textsc{sens}-come.out \textsc{lnk} birch  forest \textsc{dem:pl} also \textsc{sens}-come.out \textsc{lnk} everywhere \textsc{emph} exist:\textsc{sens} be:\textsc{affirm}  \textsc{lnk} plain whether  forest completely \textsc{emph} \textsc{ipfv}-come.out \textsc{sens}-be \\
\glt `(This mushroom) grows among oaks, among firs, among willows, among red or white birch forests, you find it everywhere, whether on plains or in forest.' (23-mbrAZim, 233-238)
\end{exe} 

\subsection{Partitive pronouns} \label{sec:partitive.pronouns}
The mid-scalar quantifier \japhug{tsuku}{some} is used both as a noun determiner (§ \ref{sec:tsuku}) and as a partitive pronoun, taking case markers and determiners. This construction expresses a meaning close to that obtained by combining a relative clause with an existential verb (`there is someone who...', § XXX), as can be seen in (\ref{ex:zgri.mWrkuj}) where both constructions are used one after the other. 

\begin{exe}
\ex \label{ex:zgri.mWrkuj}
\gll tsuku kɯ zgri tu-ti-nɯ ŋu, tsuku kɯ mɯrkuj tu-ti-nɯ ŋu. mɯrkuj tu-kɯ-ti tɕi tu, zgri tu-kɯ-ti tɕi tu ma, \\
some \textsc{erg} plant.name \textsc{ipfv}-say-\textsc{pl} some \textsc{erg} plant.name \textsc{ipfv}-say-\textsc{pl} plant.name  \textsc{ipfv}-\textsc{nmlz}:S/A-say also exist:\textsc{fact} plant.name  \textsc{ipfv}-\textsc{nmlz}:S/A-say also exist:\textsc{fact}  \textsc{lnk} \\
\glt  `Some call it \forme{zgri}, some call it \forme{mɯrkuj}; there are people who call it \forme{mɯrkuj}, and also people who call it \forme{zgri}.' (19-qachGa mWntoR, 168)
\end{exe}
 
The quantifier \japhug{tsuku}{some} used as a pronoun generally refers to humans in the corpus, but (\ref{ex:tsuku.YaR}) shows that it can also denote plants for instance. 

\begin{exe}
\ex \label{ex:tsuku.YaR}
\gll tsuku ɲaʁ, tsuku aqarŋɯrŋe, \\
some be.black:\textsc{fact} some be.light.yellow:\textsc{fact} \\
\glt `Some are black, some are light yellow.' (140505 stonka mWntoR, 5)
\end{exe}

Numerals (in particular \japhug{ci}{one}) and also counted nouns (§ \ref{sec:CN.quantifier}) can be used without head noun with a partitive meaning `one of (a group)' as in (\ref{ex:ci.GW})  and (\ref{ex:ci.thWkWrgAz}).

\begin{exe}
\ex \label{ex:ci.GW}
\gll ci ɣɯ 	tɤ-tɕɯ,  ci ɣɯ tɕʰeme tɯ\redp{}tɤ-tu nɤ, ʁzɤmi ku-kɤ-sɯ-βzu \\
one \textsc{gen} \textsc{indef}.\textsc{poss}-son one \textsc{gen} \textsc{indef}.\textsc{poss}-son \textsc{cond}\redp{}\textsc{pfv}-exist \textsc{lnk} husband.and.wife \textsc{ipfv}-\textsc{inf}-\textsc{caus}-make \\
\glt `If one of them has a boy, and the other one has a girl, let us make them husband and wife.' (zrAntCW 5)
\end{exe}

\begin{exe}
\ex \label{ex:ci.thWkWrgAz}
\gll ci tʰɯ-kɯ-rgɯ\redp{}rgɤz ɲɯ-ɕti tɕe, ci kɯ-xtɕɯ\redp{}xtɕi ɲɯ-ɕti tɕe, \\
one \textsc{pfv}-\textsc{nmlz}:S/A-\textsc{emph}\redp{}be.old C-be.\textsc{affirm} \textsc{lnk} one \textsc{pfv}-\textsc{nmlz}:S/A-\textsc{emph}\redp{}be.old C-be.\textsc{affirm} \textsc{lnk}  \\
\glt `One of them has grown very old, and one of them is very small.' (2011-05-nyima, 140)
\end{exe}


\subsection{Distributive pronouns} \label{sec:distributive.pronouns}
The pronoun \japhug{ʑaka}{each his own} and its variant \japhug{ʑakastaka}{each his own} occur as pronouns, especially as possessors in an possessive existential construction. It can be correlated with a third singular \forme{ɯ-} (\ref{ex:Zaka.WmdoR}) or a third plural \forme{nɯ-} (\ref{ex:Zaka.nWrmi}) prefix on the possessum.


\begin{exe}
\ex \label{ex:Zaka.WmdoR}
\gll tɕe nɯnɯ li qaʑo nɯ kɯ-ɲaʁ tu, kɯ-wɣrum tu, kɯ-ɤɣɯnɯɕɯr kɯ-fse tu, kɯ-ɤrŋɯlɯz tu, tɕe nɯnɯ ʑaka ɯ-mdoʁ tu ma \\
\textsc{lnk} \textsc{dem} again sheep \textsc{dem} \textsc{nmlz}:S/A-be.black exist:\textsc{fact} \textsc{nmlz}:S/A-be.white exist:\textsc{fact} \textsc{nmlz}:S/A-be.reddish exist:\textsc{fact}  \textsc{nmlz}:S/A-be.blueish exist:\textsc{fact} \textsc{lnk} \textsc{dem} each.his.own \textsc{3sg}.\textsc{poss}-colour  exist:\textsc{fact} \textsc{lnk} \\
\glt `There are black sheep, white ones, reddish ones, blueish ones, each has his own colour (they come in all types of colors).' (05-qaZo, 64-66)
\end{exe}

\begin{exe}
\ex \label{ex:Zaka.nWrmi}
\gll li ʑaka nɯ-rmi tu, \\
again each.his.own \textsc{3pl}.\textsc{poss}-name exist:\textsc{fact} \\
\glt `Each have their own names.' (150903 tWmNu, 11)
\end{exe}

The pronoun \forme{ʑaka} is built by combining the \textit{status constructus} of the pronominal root \forme{-ʑo} (§ \ref{sec:pers.pronouns}) with the root \forme{-ka} found in the distributive modifier \japhug{tɯka}{each} (which follows possessums, see § \ref{sec:raNri}).


\section{Identity pronoun} \label{sec:other.pro}
The words \japhug{kɯmaʁ}{other} and \japhug{kɯɕte}{other} occur as prenominal determiners (see § \ref{sec:identity.modifier}, also for a discussion on the etymology of the former), but it can also be used as a pronoun and take determiners as in (\ref{ex:kWmaR.nWra}).

\begin{exe}
\ex \label{ex:kWmaR.nWra}
\gll ma kɯmaʁ nɯra aj mɯ́j-sɯχsal-a ri, tɤkʰepɣɤtɕɯ nɯ sɯχsal-a  \\
\textsc{lnk} other \textsc{dem:pl} \textsc{1sg} \textsc{neg}:\textsc{sens}-recognize but bird.sp \textsc{dem} recognize:\textsc{fact}-\textsc{1sg} \\
\glt `The other ones I don't recognize them, but the \forme{tɤkʰepɣɤtɕɯ} bird, I do recognize it.' (23-scuz, 46)
\end{exe}

The interpretation of both \japhug{kɯmaʁ}{other} and \japhug{kɯɕte}{other} can be locative  `somewhere else' as in (\ref{ex:kWmaR.nWtWtat}), when the main verb (\japhug{ta}{put} in this example) selects a goal or a locative adjunct (since locative noun phrases are often unmarked, see § XXX).

\begin{exe}
\ex \label{ex:kWmaR.nWtWtat}
\gll kɯmaʁ/kɯɕte   nɯ-tɯ-ta-t ŋu ɯ-maʁ? \\
other  \textsc{pfv}-2-put-\textsc{tr}:\textsc{pst} be:\textsc{fact} \textsc{qu}-not.be:\textsc{fact} \\
\glt `Did you put it somewhere else?' (elicitation)
\end{exe}

Adding the indefinite determiner \japhug{ci}{one} is necessary in this context to convey the meaning `something else':

\begin{exe}
\ex \label{ex:kWmaR.ci.nWtWtat}
\gll kɯmaʁ/kɯɕte ci nɯ-tɯ-ta-t ŋu ɯ-maʁ? \\
other \textsc{indef} \textsc{pfv}-2-put-\textsc{tr}:\textsc{pst} be:\textsc{fact} \textsc{qu}-not.be:\textsc{fact} \\
\glt `Did you put something else?' (elicitation)
\end{exe}

Example (\ref{ex:kWmaR.ci.juGWt}) illustrates that both \forme{kɯmaʁ} and \forme{kɯmaʁ ci} can occur in the meaning `another one' in some contexts (here with the verb \japhug{ɕar}{search}).

 \begin{exe}
\ex \label{ex:kWmaR.ci.juGWt}
\gll  χsɯ-sŋi mɤ-kɯ-tsu qʰe li kɯmaʁ ci ju-ɣɯt qʰe, li ɯ-zda ɲɯ-nɯ-ɕar ɲɯ-ɕti.
tɕe nɯnɯ maka kɯjka nɯ ŋɤn ma, ɯ-zda nɯ nɯ-me ɯ-qʰu mɤ-kɯ-nɤrʑaʁ tɕe kɯmaʁ ɲɯ-ɕar ɲɯ-ɕti tɕe mɯ́j-pe tu-ti-nɯ ŋgrɤl. \\
three-day \textsc{neg}-\textsc{inf}:\textsc{stat}-pass \textsc{lnk} again other \textsc{indef} \textsc{ipfv}-bring \textsc{lnk} again \textsc{3sg}.\textsc{poss}-companion \textsc{ipfv}-\textsc{auto}-search \textsc{sens}-be.affirm \textsc{lnk} \textsc{dem} completely pyrrhocorax \textsc{dem} be.evil:\textsc{fact} \textsc{lnk} \textsc{3sg}.\textsc{poss}-companion \textsc{dem}  \textsc{pfv}-not.exist \textsc{3sg}.\textsc{poss}-after \textsc{neg}-\textsc{inf}:\textsc{stat}-spend.time \textsc{lnk} other \textsc{ipfv}-search \textsc{sens}-be.affirm \textsc{lnk} \textsc{neg}:\textsc{sens}-be.good \textsc{ipfv}-say-\textsc{pl} be.usually.the.case:\textsc{fact} \\
\glt `Not even three days (after hunters kill its mate, the pyrrhocorax) brings another one, it looks for another mate. People say that the pyrrhorax is not nice, because not long after its mate has died, it looks for another one, it is not good.' (22-CAGpGa, 84)
\end{exe}

The indefinite \japhug{ci}{one} combined with the demonstrative determiner \forme{nɯ} (or \forme{nɯnɯ}) has the meaning `the other one' (the definite counterpart of \japhug{kɯmaʁ}{other}), as in (\ref{ex:ci.nW.kW}) and (\ref{ex:tWrdoR.ci.nW}).

\begin{exe}
\ex \label{ex:ci.nW.kW}
 \gll tɕe ɯ-jaʁ kɯ ki tu-ste lu-z-naʁje ɲɯ-ŋu ri, tɕe ci nɯ kɯ ɯ-jaʁ ku-mtsɯɣ ɲɯ-ɕti qʰe, \\
 \textsc{lnk} \textsc{3sg.poss}-hand \textsc{dem:prox} \textsc{ipfv}-do.like[III]  \textsc{ipfv}-reach.into[III] \textsc{sens}-be but \textsc{lnk} \textsc{indef} \textsc{dem} \textsc{erg} \textsc{3sg.poss}-hand  \textsc{ipfv}-bite \textsc{sens}-be:\textsc{affirm:fact} \textsc{lnk} \\
\glt `(The cat) reaches with its paw (into the whole), but the other one (the weasel) bites its paw.' (27-spjaNkW)
\end{exe}

\begin{exe}
\ex \label{ex:tWrdoR.ci.nW}
 \gll 
ɯ-me ʁnɯz pjɤ-tu tɕe, tɯ-rdoʁ nɯ χsɤrlɤsmɤn pjɤ-rmi, ci nɯ rŋɯlɤsmɤn pjɤ-rmi tɕe, \\
\textsc{3sg.poss}-daughter two \textsc{ifr.ipfv}-exist \textsc{lnk} one-piece \textsc{dem} gser.la.sman \textsc{ifr.ipfv}-be.called \textsc{indef} \textsc{dem} dngul.la.sman \textsc{ifr.ipfv}-be.called \textsc{lnk} \\
\glt `He had two daughters, one of them was called Gser.la.sman, and the other Dngul.la.sman.' (2003-kWBRa, 1-2)
\end{exe}

Alternatively to the construction in (\ref{ex:tWrdoR.ci.nW}) with \japhug{tɯ-rdoʁ}{one piece} and \forme{ci nɯ} to express the meaning one of them .... and the other ...', it is possible to use \forme{ci nɯ} two times in the same sentence to refer to more than one distinct persons or animals, as in (\ref{ex:ci.nW.2}).

\begin{exe}
\ex \label{ex:ci.nW.2}
 \gll tɕe ci nɯnɯ ju-ɕe ɯ-kʰɯkʰa ci nɯ kɯ ɯ-pu tu-ndze, cʰɯ-rɤɕi. \\
\textsc{lnk} \textsc{indef} \textsc{dem} \textsc{ipfv}-go \textsc{3sg}-while \textsc{indef} \textsc{dem} \textsc{erg} \textsc{3sg.poss}-intestine \textsc{ipfv}-eat[III] \textsc{ipfv}-pull \\
\glt `While one of the two (the prey) is (still) going, the other one (the predator) eats and pulls its intestine.'  (20-RmbroN, 76)
\end{exe}

The dual \forme{ci nɯni} `the other two' and plural \forme{ci nɯra} `the other ones' are also attested, as in (\ref{ex:ci.nWni}), showing that \forme{ci} is here completely bleached of numeral meaning.

 \begin{exe}
\ex \label{ex:ci.nWni}
 \gll  ci nɯni ɣɯ nɯ, ndʑi-ta-mar rɟɤɣi pjɤ-ŋu tɕe tɕe nɯnɯ ɯʑo kɯ to-ndza \\
\textsc{indef} \textsc{dem:du} \gen \textsc{dem} \textsc{3du}.\textsc{poss}-\textsc{indef}.\textsc{poss}-butter tsampa \textsc{ipfv}.\textsc{ifr}-be:\textsc{fact} \textsc{lnk} \textsc{lnk} \textsc{dem} \textsc{3sg} \textsc{erg} \textsc{ifr}-eat \\
\glt `The tsampa of the other two (sisters) was butter tsampa, and she ate it.' (2003-kWBRa, 20)
\end{exe}

It is also possible in this function to use other modifiers such as numerals, as in (\ref{ex:ci.XsWm}) with \japhug{χsɯm}{three}. 

 \begin{exe}
\ex \label{ex:ci.XsWm}
 \gll iɕqʰa ci χsɯm nɯ mɯ-jo-ɣi-nɯ kɯ  \\
 the.aforementioned \textsc{indef} three \textsc{dem} \textsc{neg}-\textsc{ifr}-come-\textsc{pl} \textsc{erg} \\
\glt `The three other ones, without coming, (said...)' (140515 congming de wusui xiaohai-zh, 45)
 \end{exe}

As a prenominal determiner, \forme{ci} also has the meaning `the other X' (see § \ref{sec:identity.modifier}).


\section{Demonstrative pronouns} \label{sec:demonstrative.pronouns}
There are two basic demonstratives in Japhug, the proximal \japhug{ki}{this} and the distal one \japhug{nɯ}{that}, which also occur as demonstrative determiners (see § \ref{sec:demonstrative.determiners}). Table \ref{tab:dem.pronoun} illustrates the various demonstrative pronouns that are derived from these basic forms, with Reduplicated and Emphatic forms. There is in addition a Cataphoric pronoun \forme{nɤki}, discussed in section \ref{sec:cataph.pron}.

Plural and dual forms, as in the case of determiners, are formed by adding \forme{-ra} and \forme{-ni} suffixes (see § \ref{sec:number.determiners}) to the demonstrative root, which undergoes \textit{status constructus} change \ipa{i} \fl{} \ipa{ɯ} in the case of proximal demonstratives. Plural forms are given in the table; dual forms are attested but rare and can be predicted (\japhug{kɯni}{these two} etc).

\begin{table}
\caption{Demonstrative pronouns}\label{tab:dem.pronoun}
\begin{tabular}{lllll} 
\lsptoprule
&Base form & Reduplicated & Emphatic \\
\midrule
\textsc{prox.sg} & \forme{ki} & \forme{kɯki} &  \forme{ɯkɯki}  \\
\textsc{dist.sg} & \forme{nɯ} &  \forme{nɯnɯ} & \forme{ɯnɯnɯ} \\
\midrule
\textsc{prox.pl} & \forme{kɯra} & \forme{kɯkɯra} &  \forme{ɯkɯkɯra}  \\
\textsc{dist.pl} & \forme{nɯra} &  \forme{nɯnɯra} & \forme{ɯnɯnɯra} \\
\lspbottomrule
\end{tabular}
\end{table}

\subsection{Anaphoric demonstrative pronouns} \label{sec:anaphoric.demonstrative.pro}

The basic demonstratives \forme{ki} and \forme{nɯ} are less often used as pronouns that the other ones (they mainly occur as determiners). They nevertheless do occur in all syntactic functions, including object (in particular with the verb \japhug{ti}{say}, as in \ref{ex:nW.toti}, where it refers to words that have been previously told to another animal), XXX and extended object (in particular with the verb \japhug{stu}{do like} as in \ref{ex:ki.tuste}).

\begin{exe}
\ex \label{ex:nW.toti}
 \gll  li nɯ to-ti ri, \\
 again \textsc{dem} \textsc{ifr}-say \textsc{lnk} \\
\glt `(Gesar) said the same thing to the (snow leopard).' (gesar, 286)
\end{exe}

\begin{exe}
\ex \label{ex:ki.tuste}
 \gll ɯ-mu nɯ ku-rqoʁ tɕe ki tu-ste tɕe \\
\textsc{3sg.poss}-mother  \textsc{dem} \textsc{ipfv}-hug \textsc{lnk} \textsc{dem:prox} \textsc{ipfv}-do.like[III] \textsc{lnk} \\
\glt `It hugs its mother like that.' (19-GzW, 30)
\end{exe}

The distal demonstratives \forme{nɯ} and \forme{nɯnɯ} serve as anaphoric pronouns with any type of referent, including humans, but are most appropriate for abstract concepts, inanimate objects or plants as in (\ref{ex:nWnW.kW.smi}), though as mentioned in section \ref{sec:pers.pronouns},  third person pronouns such as \japhug{ɯʑo}{he} can also have inanimate antecedents.

\begin{exe}
\ex \label{ex:nWnW.kW.smi}
 \gll tʂʰa kɤ-nɯ-ta tɤ-ra, smi kɤ-βlɯ tɤ-ra pɯ-nɯ-ŋu, tʰamaka sko-nɯ pɯ-nɯ-ŋu, tɕe \textbf{nɯnɯ} kɯ smi tu-sɯ-tɕɤt-nɯ. \\
 tea \textsc{inf}-\textsc{auto}-put \textsc{pfv}-have.to fire  \textsc{pfv}-burn \textsc{pfv}-have.to \textsc{pst.ipfv-auto}-be tobacco smoke:\textsc{fact}-\textsc{pl} \textsc{pst.ipfv-auto}-be \textsc{lnk} \textsc{dem} \textsc{erg} fire \textsc{ipfv}-\textsc{caus}-take.out-\textsc{pl} \\
 \glt `When they need to boil tea, to make a fire or smoke tobacco, people light up the fire with it.' (15-babW, 226-229)
\end{exe}

When a third person mentioned in a discussion is present, the pronoun \japhug{ɯʑo}{he} is not the optimal way of referring to him/her, and a proximal demonstrative, in particular the reduplicated \japhug{kɯki}{this one} is used instead. It can occur to present someone to someone else (\ref{ex:kWki.aslama}) (note that a similar usage exists in Western languages such as English in the same context) and even to talk about the actions of this person, as in  (\ref{ex:kWki.kW.taBzu}) and (\ref{ex:kWki.nW.ftsWntCi}).

\begin{exe}
\ex \label{ex:kWki.aslama}
 \gll kɯki a-slama ŋu \\
\textsc{dem.prox} \textsc{1sg.poss}-student be:\textsc{fact} \\
\glt `This a (former) student of mine.' (conversation 140510, 17)
\end{exe}

\begin{exe}
\ex \label{ex:kWki.kW.taBzu}
 \gll  kɯki kɯ ta-βzu? \\
 \textsc{dem.prox} \textsc{erg} \textsc{pfv}:3\fl3'-make \\
 \glt `Did she make it?' (conversation 140510, 152)
\end{exe}

As other pronouns (see § \ref{sec:pers.pronouns}), demonstrative pronouns can take the demonstrative determiner \forme{nɯ}, as in (\ref{ex:kWki.nW.ftsWntCi}).

\begin{exe}
\ex \label{ex:kWki.nW.ftsWntCi}
 \gll mɯ\redp{}mɤ-pɯ-jɤɣ tɕe mɤ-ɣi-tɕi ma \textbf{kɯki} nɯ fstɯn-tɕi ra ma tɕi-βɣe ɯ-ku thɯ-kɯ-ɣɤrndi  \\
\textsc{cond}\redp{}\textsc{neg}-\textsc{pst}.\textsc{ipfv}-be.acceptable \textsc{lnk} \textsc{neg}-come:\textsc{fact}-\textsc{1du} \textsc{lnk} \textsc{dem:prox} \textsc{dem} serve:\textsc{fact}-\textsc{1du} have.to:\textsc{fact} \textsc{lnk} \textsc{1du.poss}-orphan \textsc{3sg.poss}-head \textsc{pfv}-\textsc{nmlz}:S/A-support \\
\glt `If it is not possible (to take the old man with us) we will not come, as we have to serve him, he is the one who adopted us orphans when we were in dire straits.' (The old man is presumably present when this sentence is uttered; 2003nyima2, 122)
\end{exe}

The emphatic demonstrative pronouns (which are also used as determiners, see § XXX) are built by combining the reduplicated forms of demonstratives with the third person possessive prefix \forme{ɯ-}. They are about fifty time less common than corresponding reduplicated forms, but their function is essentially the same. In example (\ref{ex:WnWnW.kW}), \forme{ɯnɯnɯ} is an anaphoric pronoun whose antecedent is present in the immediately preceding clause.

\begin{exe}
\ex \label{ex:WnWnW.kW}
 \gll
tɕe ɯ-rqʰu kɯ-fse ci ɣɤʑu tɕe, \textbf{ɯnɯnɯ} kɯ ɯ-rdu nɯ tu-ɕɯ-fkaβ kɯ-fse ɲɯ-ŋu. \\
\textsc{lnk} \textsc{3sg.poss}-hull \textsc{nmlz}:S/A-be.like \textsc{indef}  exist:\textsc{sens} \textsc{lnk} \textsc{dem:emph:distal} \textsc{erg} \textsc{3sg.poss}-eyeball \textsc{dem} \textsc{ipfv}-\textsc{caus}-cover  \textsc{nmlz}:S/A-be.like \textsc{sens}-be \\
\glt `It has something like a membrane, and it covers its eyeball with it.' (description of the nictitating membrane of birds, 140513 sWNgWrmABja, 9)
\end{exe}

The demonstrative \forme{nɯ} may anaphorically refer not to a particular entity, but to an entire situation, as in (\ref{ex:nW.pjArAZindZi}) where it occurs as an adjunct in absolutive form, meaning `this way, like that' (in another version of the same story, we find \forme{nɯ kɯ-fse} `like that' instead of \forme{nɯ} in the same context).

 \begin{exe}
\ex \label{ex:nW.pjArAZindZi}
\gll   nɯnɯ ɯ-mŋu nɯtɕu zɯ li, qapri, nɤki, kɯ-ɲaʁ nɯ kɯ kɯ-wɣrum nɯ ɯ-qiɯ ʑo cʰɤ-mqlaʁ tɕe nɯ pjɤ-rɤʑi-ndʑi. \\
\textsc{dem} \textsc{3sg}.\textsc{poss}-bank \textsc{dem}:\textsc{loc} \textsc{loc} again snake \textsc{filler} \textsc{nmlz}:S/A-be.black \textsc{dem} \textsc{erg}  \textsc{nmlz}:S/A-be.white \textsc{dem} \textsc{3sg}.\textsc{poss}-half \textsc{emph} \textsc{ifr}-swallow \textsc{lnk} \textsc{dem} \textsc{ifr}.\textsc{ipfv}-stay-\textsc{du} \\
\glt `On the bank (of the lake), there was again a black snake that had swallow half of a white snake, and they were staying (stuck) like that.' (28-smAnmi, 104)
\end{exe}

\subsection{Cataphoric pronoun} \label{sec:cataph.pron}
The demonstrative \forme{nɤki} stands out among other demonstrative pronouns in that it is most specifically used for cataphoric referents. It occurs especially when the speaker hesitates and uses it as a filler, followed by a clause with the same verb (examples \ref{ex:nAki.YWNu} and \ref{ex:nAki.YAXtAr}) or just with the same auxiliary (\ref{ex:nAki.Nu.Ci}).

\begin{exe}
\ex \label{ex:nAki.YWNu}
 \gll
qra nɯ kɯ, mbala na-lɤt nɤ tɕe \textbf{nɤki} ɲɯ-ŋu, jla ɲɯ-ŋu, \\
female.yak \textsc{dem} \textsc{erg} male.young.bovid \textsc{pfv}:3\fl3' \textsc{lnk} \textsc{lnk} \textsc{dem:cataph} \textsc{sens}-be male.hybrid.yak \textsc{sens}-be \\
\glt `When a female yak has a young (with a bull), it is..., it is a hybrid yak.' (05-qambrW, 64)
\end{exe}

\begin{exe}
\ex \label{ex:nAki.YAXtAr}
 \gll tɕe nɯ tɯ-ci ɣɯ ɯ-taʁ nɯnɯtɕu, \textbf{nɤki} ɲɤ-χtɤr, iɕqʰa <yujinxiang> kɤ-ti mɯntoʁ nɯ ɣɯ  ɯ-jwaʁ nɯ ɲɤ-χtɤr.  \\
\textsc{lnk} \textsc{dem} \textsc{indef.poss}-water \textsc{gen} \textsc{3sg}-on \textsc{dem:loc} \textsc{dem:cataph} \textsc{ifr}-spread the.aforementionned tulip \textsc{nmlz}:P-say flower \textsc{dem} \textsc{gen} \textsc{3sg.poss}-leaf \textsc{dem} \textsc{ifr}-spread \\
\glt She spilled on the water...  she spilled the petals of the flower called `tulip'.' (150818 muzhi guniang, 69)
\end{exe}

\begin{exe}
\ex \label{ex:nAki.Nu.Ci}
 \gll 
tɕe nɯ-nɯŋa ra nɤki ŋu ɕi, tɕe ɲɯ-tɯ-nɤm qhe, tɕe ʑara ku-nɯ-nɯɣi-nɯ ŋu ɕi? \\
\textsc{lnk} \textsc{2sg}.\textsc{poss}-cow \textsc{dem}:\textsc{cataph} be:\textsc{fact} \textsc{qu} \textsc{lnk} \textsc{ipfv:east}-2-chase[III] \textsc{lnk} \textsc{lnk} \textsc{3pl} \textsc{ipfv:west}-auto-come.back-\textsc{pl} be:\textsc{fact} \textsc{qu} \\
\glt `And your cows, are they (still) like that, do you chase them, or do they come back home on their own?' (taRrdo conversation, 28-29)
\end{exe}
It is also used  when the speaker alerts the addressee that a long description follows as in (\ref{ex:nAki.tustunW}), as in English `(he said) the following'. Given the fact the Japhug is strictly verb-final and has pre-verbal complements (see § XXX), this is a strategy employed to avoid relegating the main verb to the end of the description.

 \begin{exe}
\ex \label{ex:nAki.tustunW}
 \gll
kʰopi kɯ ɴqiazwɤr ci ɲɯ-mɯm rca ɲɯ-saχaʁ ʑo tɕe \textbf{nɤki} tu-stu-nɯ ɲɯ-ŋu ɲɯ-ti, ɲɯ-pʰɯt-nɯ qʰe kɯ-zri... ki jamar ʑo kɯ-zri ɲɯ-pʰɯt-nɯ qʰe nɤki, ɯ-ku ɯ-mtɯ kɯ-fse nɯtɕu kú-wɣ-ndo qʰe tɕe ɯ-pa nɯ, ɯ-jwaʁ nɯ cʰɯ-χɕoʁ-nɯ ɲɯ-ŋu...  \\
p.n. \textsc{erg} bitter.wormwood \textsc{indef} \textsc{sens}-be.tasty \textsc{unexpect} \textsc{sens}-be.extremely \textsc{emph} \textsc{lnk} \textsc{dem:cataph} \textsc{ipfv}-do.like-\textsc{pl} \textsc{sens}-be \textsc{sens}-say \textsc{ipfv}-take.out-\textsc{pl} \textsc{lnk} \textsc{nmlz}:S/A-be.long \textsc{dem:prox} about \textsc{emph} \textsc{nmlz}:S/A-be.long \textsc{ipfv}-take.out-\textsc{pl} \textsc{lnk} \textsc{dem:cataph} \textsc{3sg.poss}-head  \textsc{3sg.poss}-crest \textsc{nmlz}:S/A-be.like  \textsc{dem:loc} \textsc{ipfv-inv}-take \textsc{lnk} \textsc{lnk} \textsc{3sg.poss}-under \textsc{dem}  \textsc{3sg.poss}-leaf \textsc{dem} \textsc{ipfv:downstream}-take.out-\textsc{pl}  \textsc{sens}-be \\
\glt `Kebei says that bitter wormwood is very tasty, and that they prepare it in the following way: they pluck (wormwoods) that are this big, take it by something that looks like a crest on the top, and prune away the leaves under it... (continued by several paragraphs)' (conversation 140510)
\end{exe}

The pronoun \forme{nɤki} is also used as a determiner (§ XXX) and the speech filler \forme{nɤkinɯ} (§ XXX) derives from the combination of  \forme{nɤki}  with the determiner \forme{nɯ}. There are no plural or dual forms of \forme{nɤki}, but it can be combined with dual or plural determiners as in (\ref{ex:nAki.nWra}).

\begin{exe}
\ex \label{ex:nAki.nWra}
\gll  tɕe nɤki nɯra, mkʰɤrmaŋ ra pjɤ-rɯsɯso-nɯ tɕe, \\
\textsc{lnk} \textsc{dem:cataph} \textsc{dem}:\textsc{pl} people \textsc{pl} \textsc{ifr}-think-\textsc{pl} \textsc{lnk} \\
\glt `And these, the people thought about it.' (150829 jidian-zh, 138)
\end{exe}

\subsection{Medial demonstrative} \label{sec:medial.dem.pro}
In addition to its function as a cataphoric pronoun, \forme{nɤki} also occurs as a medial demonstrative, in examples such as (\ref{ex:nAki.nWtɕu}) and (\ref{ex:nAki.nW.aZWG}), where it means `your place, near you', as opposed to `here'.

\begin{exe}
\ex \label{ex:nAki.nWtɕu}
\gll kutɕu ko-qanɯ pʰoʁpʰoʁ ʑo, nɤki nɯtɕu ɯ-kó-qanɯ? \\
here \textsc{ifr}-be.dark \textsc{idph}:II:completely \textsc{emph} \textsc{dem}:\textsc{medial} \textsc{dem}:\textsc{loc} \textsc{qu}-\textsc{ifr}-be.dark \\
\glt `Here it is already dark, is it (also) dark in your place?' (conversation, 14.12.24, referring to the time lag between Paris and Mbarkham)
\end{exe}

\begin{exe}
\ex \label{ex:nAki.nW.aZWG}
\gll  ki kɯra ɲɯ-kʰam-a tɕe nɤki nɯ aʑɯɣ nɯ-kʰɤm je \\
\textsc{dem}:\textsc{prox} \textsc{dem}:\textsc{prox}:\textsc{pl}  \textsc{ipfv}-give-\textsc{1sg} \textsc{lnk} \textsc{dem}:\textsc{medial} \textsc{dem} \textsc{1sg}:\textsc{gen} \textsc{imp}-give \textsc{sfp} \\
\glt `I give (you) these (toys), give me that one.'  (Norbzang2012, 135)
\end{exe}

The medial demonstrative \forme{nɤki} can be historically analyzed as a combination of the proximal demonstrative \forme{ki} with the second person possessive \forme{nɤ-}. However, equivalent dual or plural forms such as  $\dagger$\forme{ndʑiki} or $\dagger$\forme{nɯki} are completely impossible (the equivalent meaning can only be expressed with the dative, using a form such as \forme{ndʑiʑo ndʑi-pʰe} `at your$_{du}$ place', § \ref{sec:dative} ).


It is likely that the cataphoric demonstrative use of \forme{nɤki} derives  from its function as a medial demonstrative, suggesting the historical pathway in (\ref{ex:nAki.hist}).

\begin{exe}
\ex \label{ex:nAki.hist}
\glt \textsc{2sg}+\textsc{dem}:\textsc{prox} $\Rightarrow$ \textsc{dem}:\textsc{medial} $\Rightarrow$ \textsc{dem}:\textsc{cataphoric} $\Rightarrow$ \textsc{speech filler}
\end{exe}

\subsection{Locative forms of the demonstrative pronouns} \label{sec:locative.pronoun}
The locative postposition \forme{tɕu} (§ \ref{sec:core.locative}) can be combined with the demonstrative pronouns \forme{nɯ} and \forme{ki} and their reduplicated and emphatic forms, as shown in Table \ref{tab:loc.dem.pronoun}.  

The locative pronouns in \forme{-tɕu} can be followed by the postposition \forme{zɯ} as in (\ref{ex:nWtCu.zW.mbro}), but not by the locative \forme{ri}.
 
\begin{exe}
\ex \label{ex:nWtCu.zW.mbro}
 \gll mbrosta ci tu tɕe, nɯtɕu zɯ mbro nɯ a-ja. \\
 stable \textsc{indef} exist:\textsc{fact} \textsc{lnk} \textsc{dem}:\textsc{loc} \textsc{loc} horse \textsc{dem} \textsc{pass}-keep.attached:\textsc{fact} \\
 \glt `There are stables (in this palace), and the horse is kept there.' (140507 jinniao, 174)
\end{exe} 

The proximal demonstrative \forme{ki} undergoes \textit{status constructus} (§ \ref{sec:status.constructus}) alternation and changes to \forme{kɯ-} when combined with the locative postpositions, with further assimilation to [\forme{u}] due to the regressive vowel harmony (§ XXX) when followed by  \forme{-tɕu}.

 In addition to the locative pronouns in  \forme{-tɕu}, there is an entirely parallel series of pronouns in \forme{-re}; this suffix is probably unrelated to the locative postposition \forme{ri}, and may rather reflect the plural marker \forme{ra} (which can be used to mark vague location, see § \ref{sec:plural.determiners}) with the proto-Gyalrong locative suffix \forme{*-j} and regular vowel fusion (§ \ref{sec:locative.j}). These locative pronouns are much less commonly used in the corpus than those of the  \forme{-tɕu} series.
 
\begin{table}
\caption{Locative demonstrative pronouns}\label{tab:loc.dem.pronoun}
\begin{tabular}{lllll} 
\lsptoprule
&Base form & Reduplicated & Emphatic \\
\midrule
\textsc{prox.sg} & \forme{kutɕu} & \forme{kukutɕu} &  X  \\
\textsc{dist.sg} & \forme{nɯtɕu} &  \forme{nɯnɯtɕu} & \forme{ɯnɯnɯtɕu} \\
\midrule
\textsc{prox.sg} & \forme{kɯre} & \forme{kɯkɯre} &  X  \\
\textsc{dist.sg} & \forme{nɯre} &  \forme{nɯnɯre} & \forme{ɯnɯnɯre} \\
\lspbottomrule
\end{tabular}
\end{table}

Locative pronouns in \forme{-re} can appear on their own as in (\ref{ex:kWre}), or with the locative postposition \forme{ri} as in (\ref{ex:ndWchu.kWre.ri}) -- never with the other postpositions \forme{zɯ} and \forme{tɕu}. 

\begin{exe}
\ex \label{ex:kWre}
 \gll aʑo mɯ-pɯ-rɤʑi-a, kɯre pɯ-ɕti-a. \\
 \textsc{1sg} \textsc{neg}-\textsc{pst}.\textsc{ipfv}-stay here \textsc{pst}.\textsc{ipfv}-be.\textsc{affirm}-\textsc{1sg} \\
\glt `I was not present (there), I was here.' (conversation140510 , 84)
\end{exe}

\begin{exe}
\ex \label{ex:ndWchu.kWre.ri}
 \gll tɕe kukutɕu, <zhuanmen>, nɤkinɯ, tɯ-ɕɣa ɯ-kɯ-nɯsmɤn, tɕɤtʰi, ndɯcʰu kɯre ri rɤʑi ma,
nɯnɯ wuma ʑo mkʰɤz tɕe, \\
\textsc{lnk} \textsc{dem}.\textsc{prox}:\textsc{loc} specially \textsc{filler} indef.poss-tooth 3sg.poss-nmlz:S/A-treat downstream west:\textsc{approx}.\textsc{loc} \textsc{dem}.\textsc{prox}:\textsc{loc} \textsc{loc} stay:\textsc{fact} \textsc{lnk} \textsc{dem} really \textsc{emph} be.expert:\textsc{fact} \textsc{lnk} \\
\glt `Here (in Mbarkham), there is someone who specially treats teeth in the west (of Mbarkham).' (27-tApGi, 139-140)
\end{exe}

 Locative adverbs, such as those based on the approximate locative \forme{-cʰu} (§ \ref{sec:approximate.locative}) can be combined with the locative pronouns in \forme{-re} as in (\ref{ex:ndWchu.kWre.ri}).

Both series of locative pronouns can express static location as in (\ref{ex:kWre}) and (\ref{ex:ndWchu.kWre.ri}) above and (\ref{ex:nWtCu.kurAzinW}) below, or motion from or towards a place as in (\ref{ex:kWre.nWnWGea}) and (\ref{ex:nWnWtCu.pjWwGlAt}).

\begin{exe}
\ex \label{ex:nWtCu.kurAzinW}
 \gll nɯtɕu ku-rɤʑi-nɯ ɲɯ-ŋu. \\
 \textsc{dem}:\textsc{loc} \textsc{ipfv}-stay-\textsc{pl} \textsc{sens}-be \\
 \glt `They live there.' (20-RmbroN, 4)
\end{exe}

\begin{exe}
\ex \label{ex:kWre.nWnWGea}
 \gll aʑo akɯ kɤntɕʰaʁ ri kɤ-ari-a tɕe, nɯ kóʁmɯz kɯre nɯ-nɯ-ɣe-a \\
 \textsc{1sg} east street \textsc{loc} \textsc{pfv}:\textsc{east}-go[II]-\textsc{1sg} \textsc{lnk} \textsc{dem} only.after \textsc{dem}.\textsc{prox}:\textsc{loc} \textsc{pfv}:\textsc{west}-\textsc{vert}-come[II]-\textsc{1sg} \\
\glt `I went there on the street, I just came back here.' (conversation, 2013-12-02)
\end{exe}

\begin{exe}
\ex \label{ex:nWnWtCu.pjWwGlAt}
 \gll βɣɤtu nɯ ɣɯ ɯ-χcɤl ri spoʁ. tɕe pjɯ-spoʁ ɲɯ-ŋu tɕe nɯnɯtɕu tɕe kɤ-ɣndʑɯr ɯ-spa nɯra pjɯ́-wɣ-lɤt. \\
upper.grindstone \textsc{dem} \textsc{gen} \textsc{3sg}.\textsc{poss}-middle \textsc{loc} have.a.hole:\textsc{fact} \textsc{lnk} \textsc{ipfv}-have.a.hole \textsc{sens}-be \textsc{lnk} \textsc{dem}:\textsc{loc} \textsc{lnk} \textsc{nmlz}:P-grind \textsc{3sg}.\textsc{poss}-material \textsc{dem}:\textsc{pl} \textsc{ipfv}-\textsc{inv}-throw \\
\glt `There is a hole in the middle of the upper grindstone, and this is where one pours (the grains) that are to be ground.' (160705 khABGa, 14)
\end{exe}

Apart from its locative uses, \forme{nɯtɕu} can express a temporal meaning `at that time' as in (\ref{ex:nWtCu.temporal}).

\begin{exe}
\ex \label{ex:nWtCu.temporal}
\gll <qidian> tɕe tɤ-mŋɤm ta-ʑa a-pɯ-ŋu tɕe, tɕe nɯnɯ tɯ-sŋi nɯ tu-mŋɤm, tɯ-rʑaʁ nɯ tu-mŋɤm tɕe,
ɯ-fso <qidian> mɤɕtʂa nɯ mɯ́j-ʑi tɕe \textbf{nɯtɕu} tɕe kɯ-xtɕɯ\redp{}xtɕi tɯ-ʑi ɲɯ-ʑe ɲɯ-ŋu tɕe \\
seven.o'clock \textsc{lnk} \textsc{pfv}-hurt \textsc{pfv}:3\fl{}3'-start \textsc{irr}-\textsc{ipfv}-be \textsc{lnk} \textsc{lnk} \textsc{dem} one-day \textsc{dem} \textsc{ipfv}-hurt  one-night \textsc{dem} \textsc{ipfv}-hurt  \textsc{lnk}  \textsc{3sg}.\textsc{poss}-tomorrow seven.o'clock until \textsc{dem} \textsc{neg}:\textsc{sens}-subside \textsc{lnk} \textsc{dem}:\textsc{loc} \textsc{lnk} \textsc{nmlz}:S/A-\textsc{emph}\redp{}be.small \textsc{inf}-subside \textsc{ipfv}-start[III] \textsc{sens}-be \textsc{lnk} \\
\glt `(For instance), if (the headache) starts at seven o'clock, it hurts for one day and one night, and subsides only in the next day at seven, \textbf{at that time} it starts to subside a little.' (24-pGArtsAG, 93-96)
\end{exe}

In addition, it can convey in some contexts a more abstract meaning like `in those circumstances', as in (\ref{ex:nWtCu.kWnA}).

\begin{exe}
\ex \label{ex:nWtCu.kWnA}
\gll tɕeri nɯtɕu kɯnɤ tɕiʑo kɤndʑɯβzaŋsa nɯ mɯ-pɯ-nɯ-qia-tɕi \\
but \textsc{dem}:\textsc{loc} also \textsc{1du} \textsc{coll}:friend \textsc{dem} \textsc{neg}-\textsc{pfv}-\textsc{auto}-tear.down-\textsc{1du} \\
\glt `But even in those circumstances (working in different places, and meeting only once a year), we did not lose our friendship.' (12-BzaNsa, 42)
\end{exe}

As for \forme{kutɕu}, it is almost exclusively used for spatial location; a metaphorical usage is attested in (\ref{ex:kutCu.tCe.nW.tutia}), where it means `in this story'.

\begin{exe}
\ex \label{ex:kutCu.tCe.nW.tutia}
\gll iɕqʰa nɤkinɯ <piqiu> nɯ ɯ-rmi kɤ-spa-t-a. nɤki, tsʰuβdɯn ra kɯ rgoŋlu tu-ti-nɯ ɲɯ-ŋu. iʑora, tɕe kutɕu tɕe nɯ tu-ti-a ŋu. \\
\textsc{filler} \textsc{filler} ball \textsc{dem} \textsc{3sg}.\textsc{poss}-name \textsc{pfv}-be.able-\textsc{pst}:\textsc{tr}-\textsc{1sg} \textsc{filler} pl.n. \textsc{pl} \textsc{erg} ball \textsc{ipfv}-say-\textsc{pl} \textsc{sens}-be \textsc{1pl} \textsc{lnk} \textsc{dem}.\textsc{prox}:\textsc{loc} \textsc{lnk} \textsc{dem} \textsc{ipfv}-say-\textsc{1pl} be:\textsc{fact} \\
\glt `I have learned how to say `ball', people from Tshobdun call it \forme{rgoŋlu}, we (do not have this word but) this is how I am going to say it here (in this story). (140514 huishuohua de niao, 4)
\end{exe}

The forms \forme{nɯtɕu} and \forme{nɯnɯtɕu} following a noun phrase result from the fusion of the postnominal demonstrative determiners \forme{nɯ} and \forme{nɯnɯ} with the postposition \forme{tɕu} (§ \ref{sec:core.locative}), and are not be to analyzed as locative pronouns.

 %Pronouns, Demonstratives and Indefinites
%\chapter{Numerals and counted nouns}

\section{Plain numerals}
Unlike some languages of the Sino-Tibetan family which have exotic numeral systems (\citealt{mazaudon02nombre}), Japhug displays a strict decimal system, without evidence for vigesimal features or substractive numerals.


\subsection{Numerals 1-10}
The basic numerals from one to ten are indicated in Table \ref{tab:numerals.under.10}. The numeral \japhug{ci}{one} is identical to the indefinite determiner (§ XXX and § \ref{sec:other.pro}). Some dialects of Japhug other than the Kamnyu variety use \japhug{tɤɣ}{one} instead. Apart from \japhug{ci}{one} and \japhug{sqi}{ten}, these numerals have clear cognates in languages outside of the Gyalrongic group, even in Tibetan and Chinese; Table \ref{tab:numerals.under.10} includes the Tibetan equivalent of these numerals (the numerals that are \textit{not} cognate with their Japhug equivalent are indicated between brackets).

The numerals from 2 to 9 have a prefix, uvular \forme{χ-/ʁ-} in `two' and `three' and velar \forme{kɯ-} from `four' to `nine'. These prefixes do not appear in some derived forms such as teens (Table XXX below) or approximate numerals (\ref{sec:approx.numerals}).

The numeral \japhug{ʁnɯz}{two} is etymologically related to the dual clitic \forme{ni} (§ XXX), though the latter lacks the uvular prefix and the \forme{-z} suffix (the vowel different is expected as proto-Gyalrong \forme{*-is} yields Japhug \forme{-ɯz}, see § XXX). A relation with \japhug{kɯɕnɯz}{seven} (implying a former base five system) is possible but if true goes back to proto-ST and is irrelevant to the synchronic grammar of Japhug.

While the other numerals are native Gyalrong words, \japhug{χsɯm}{three} might be a borrowing from Tibetan \tibet{གསུམ་}{gsum}{three}, and occurs with the same form in obvious compound loans  such as \japhug{kɯmtɕʰoχsɯm}{triratna} from \tibet{དཀོན་མཆོག་གསུམ་}{dkon.mtɕʰog.gsum}{triratna}. This idea is apparently confirmed by the alternative forms \forme{-fsum} and \forme{fsɯ-} for `three' found in teens (§ \ref{sec:teens}) and decades  (§ \ref{sec:decades}). Alternatively, it is possible that the native word and the borrowing have the same form by coincidence.

The numeral \japhug{kɯngɯt}{nine} has a coda \forme{-t} which is not found in Situ and languages outside of Gyalrongic, suggesting analogical spreading of the coda from \japhug{kɯrcat}{eight}. The same analogy independently occurred in the Siyuewu dialect of Khroskyabs, where  `nine' is \forme{ŋgə́d} (\citealt[174]{lai17khroskyabs}).

\begin{table}
\caption{Basic numerals in Japhug and Tibetan}  \label{tab:numerals.under.10} \centering \label{tab:numerals}
\begin{tabular}{lllllll}
\lsptoprule
& Japhug & Tibetan  \\
1	&	\forme{ci} or \forme{tɤɣ} & \tibet{གཅིག་}{gtɕig}{one} \\
2	&	\forme{ʁnɯz}  & \tibet{གཉིས་}{gɲis}{two} \\
3	&	\forme{χsɯm}  & \tibet{གསུམ་}{gsum}{three} \\
4	&	\forme{kɯβde} & \tibet{བཞི་}{bʑi}{four} \\
5	&	\forme{kɯmŋu}  & \tibet{ལྔ་}{lŋa}{five} \\
6	&	\forme{kɯtʂɤɣ}  & \tibet{དྲུག་}{drug}{six} \\
7	&	\forme{kɯɕnɯz} & (\tibet{བདུན་}{bdun}{seven}) \\
8	&	\forme{kɯrcat}  & \tibet{བརྒྱད་}{brgʲad}{eight} \\
9	&	\forme{kɯngɯt}  & \tibet{དགུ་}{dgu}{nine} \\
10	&	\forme{sqi}  & (\tibet{བཅུ་}{btɕu}{ten}) \\
\lspbottomrule
\end{tabular}
\end{table}		

Japhug numerals can be used either on their own or a postnominal modifiers. XXX


\subsection{Numerals 11-19} \label{sec:teens}
The numerals 11-19, listed in Table \ref{tab:teens}, serve as the basis for building all following numerals between 21 and 99, by replacing the \forme{-sqi} element of the decade numeral (Table XXX) by the appropriate form. Table \ref{tab:teens} also illustrates the formation of the numerals 21 to 29 from \japhug{ɣnɤsqi} {twenty}. 

\begin{table}
\caption{Numerals 11-29}  \label{tab:teens} \centering
\begin{tabular}{lllllll}
\lsptoprule
10 & \forme{sqi} &	20	&	\forme{ɣnɤsqi}  \\	
\midrule
11 & \forme{sqaptɯɣ} &	21	&	\forme{ɣnɤsqaptɯɣ}  \\	
12 & \forme{sqamnɯz} &	22	&	\forme{ɣnɤsqamnɯz}  \\	
13 & \forme{sqafsum} &	23	&	\forme{ɣnɤsqafsum}  \\	
14 & \forme{sqaβde} &	24	&	\forme{ɣnɤsqaβde}  \\	
15 & \forme{sqamŋu} &	25	&	\forme{ɣnɤsqamŋu}  \\	
16 & \forme{sqaprɤɣ} &	26	&	\forme{ɣnɤsqaprɤɣ}  \\	
17 & \forme{sqaɕnɯz} &	27	&	\forme{ɣnɤsqaɕnɯz}  \\	
18 & \forme{sqarcat} &	28	&	\forme{ɣnɤsqarcat}  \\	
19 & \forme{sqangɯt} &	29	&	\forme{ɣnɤsqangɯt}  \\	
\lspbottomrule
\end{tabular}
\end{table}		
 
The numerals 11-19 present three morphological changes in comparison with the basic numerals 1-9.

First, the form \japhug{sqi}{ten} alternates with \forme{sqa-}. The origin of this Ablaut is unknown, though it could be a type of \textit{status constructus} (\ref{sec:status.constructus}); some Gyalrongic languages, such as Khroskyabs have a similar alternation (\citealt[175-6]{lai17khroskyabs}). 

Second, the velar \forme{kɯ-} and uvular \forme{χ-/ʁ-} prefixes found in the base numerals are lost in all teens.

Third, a labial element \ipa{p} (\japhug{sqaptɯɣ}{eleven}, \japhug{sqaprɤɣ}{sixteen}), \ipa{m} (\japhug{sqamnɯz}{twelve}), or \ipa{w} (\japhug{sqafsum}{thirteen}) is inserted between the \forme{sqa-} and the following numeral root. It does not occur in 17, 18 and 19 (which already have a cluster), 14 and 15 (which have a cluster with a labial as first element).

The form \japhug{sqaptɯɣ}{eleven} contains an ablauted form of \japhug{tɤɣ}{one} as second element. The cluster \forme{-pt-} in this word is the only case in the language of a \ipa{p} followed by an obstruent. 

In \japhug{sqamnɯz}{twelve}, the labial linker is nasalized by the following \forme{n}. This is not a synchronic rule: for instance, a noun \japhug{ɕnaβndʑɣi}{snotty-nosed kid} has \forme{β} allomorph of \ipa{w} before a prenasalized obstruent (§ \ref{sec:subject.verb.compounds}). However, there are other cases of nasalization of labial consonants to \ipa{m} before nasal or prenasalized consonants in Japhug (see § XXX).

In \japhug{sqaprɤɣ}{sixteen}, not only the prefix \forme{kɯ-} is lost, the \forme{tʂ} affricate of the base form 	\japhug{kɯtʂɤɣ}{six} is replaced by \ipa{r}, preceded by the linking element \forme{-p-}. This \ipa{tʂ} \tld{} \ipa{r} alternation is evidence for a sound change \forme{*tr-} \fl{} \ipa{tʂ} (see § \ref{sec:second.member.alternation} for additional evidence).  The numeral \japhug{kɯtʂɤɣ}{six} contains two etymological prefixes, \forme{kɯ-} and a prefix \forme{*t-} that has fused with the root as \forme{-tʂɤɣ}. This \forme{*t-} prefix is possibly cognate with the \forme{d-} of its Tibetan cognate  \tibet{དྲུག་}{drug}{six} .


\subsection{Decades} \label{sec:decades}
The numerals for decades (Table \ref{sec:numeral.prefixes}) are relatively straightforward. With the exception of \japhug{ɣnɤsqi}{twenty} and \japhug{fsɯsqi}{thirty}, they are predictable by combining \japhug{sqi}{ten} to the corresponding numeral prefix (§ \ref{sec:numeral.prefixes}).

The element \forme{ɣnɤ-} in \japhug{ɣnɤsqi}{twenty} is related to the numeral  \japhug{ʁnɯz}{two}, but present a velar \forme{ɣ-} prefix instead of the uvular \forme{ʁ-}, and has a different vowel. The adverb \japhug{ʁnaʁna}̌{both} is also relatable, but the alternations are not explainable from a synchronic point of view.

\begin{table}
\caption{Decades}  \label{tab:decades} \centering
\begin{tabular}{lllllll}
\lsptoprule
10	&	\forme{sqi} \\			
20	&	\forme{ɣnɤsqi} \\		
30	&	\forme{fsɯsqi}  \\		
40	&	\forme{kɯβdɤ-sqi}  \\	
50	&	\forme{kɯmŋɤ-sqi}  \\	
60	&	\forme{kɯtʂɤ-sqi}  \\	
70	&	\forme{kɯɕnɤ-sqi}  \\	
80	&	\forme{kɯrcɤ-sqi}  \\	
90	&	\forme{kɯngɯ-sqi}  \\	
\lspbottomrule
\end{tabular}
\end{table}		

Other numerals under one hundred are built by combining the forms in Table \ref{sec:teens} and \ref{tab:decades}. For instance, 37 can be obtained by putting together \japhug{fsɯsqi}{thirty} and \japhug{sqaɕnɯz}{seventeen} as \forme{fsɯ-sqa-ɕnɯz}.

\subsection{Hundred and above}
 There are two ways of expressing numbers above 99 in Japhug. First, the noun-like numeral \japhug{ɣurʑa}{one hundred} can serve on its own or as a postnominal modifier, and be followed by another numeral to express a number between 101 and 199, as in (\ref{ex:hundred}).

\begin{exe}
\ex \label{ex:hundred} 
\gll aʑo 	kɯ-fse 	kɯ-cʰɯ\redp{}cʰa 	ʑo 	ʁʑɯnɯ 	ɣurʑa 	kɯrcat 	ra \\
\textsc{1sg} \textsc{nmlz}:S/A-be.like  \textsc{nmlz}:S/A-\textsc{emph}\redp{}can \textsc{emph} young.man hundred eight need:\textsc{fact} \\
\glt `I need one hundred and eight able young men like me.' (Norbzang, 16)
\end{exe}

The numeral \japhug{ɣurʑa}{one hundred} cannot be combined with unit numerals to express numbers between 200 and 900. The counted noun \japhug{tɯ-ri}{one hundred} is used for this purpose, as in \ref{ex:three.hundreds} (see § \ref{sec:numeral.prefixes} on the numeral prefixes). The two suppletive roots for hundreds are shared with Pumi (\forme{ɕí} `hundred' vs prefixed \forme{-ɻɛj}, see \citealt[101]{daudey14grammar}; evidence for cognacy with \forme{ɣurʑa} and \forme{-ri} is presented in \citealt{jacques17num}).

\begin{exe}
\ex \label{ex:three.hundreds}
\gll χsɯ-ri 	jamar 	ndɤre 	tu-nɯ 	ko, 	tɯ-tɯpʰu 	nɯ \\
three-hundred about \textsc{lnk} exist:\textsc{fact-pl} \textsc{sfp} one-hive \textsc{dem} \\
\glt There are about three hundred of them, in one hive. (Bees, 48)
\end{exe}
 
  Numerals above the hundreds are all borrowed from Tibetan: \japhug{stoŋtsu}{thousand}, \japhug{kʰrɯtsu} {ten thousand}, \japhug{mbɯmχtɤr}{hundred thousand} from \tibet{སྟོང་ཚོ་}{stoŋ.tsʰo}{thousand}, \tibet{ཁྲི་ཚོ་}{kʰri.tsʰo}{ten thousand} and \tibet{འབུམ་ཐེར་}{ⁿbum.tʰer}{hundred thousand} respectively.  
  
 
\section{Approximate numerals} \label{sec:approx.numerals}

\section{Counted nouns} \label{sec:counted.nouns}
\subsection{Numeral prefixes} \label{sec:numeral.prefixes}
\subsection{Time ordinals} \label{sec:time.ordinals}

\section{Basic arithmetic operations} \label{sec:arithmetic}
 %Numerals and classifiers
%\chapter{The noun phrase}

\section{Postpositions and relator nouns}

\subsection{Absolutive}
\subsubsection{Core argument}
\subsubsection{Essive}
%tsuku kɯ paʁndza ɲɯ-nɯ-phɯt-nɯ ɲɯ-ŋu ri,
\subsubsection{Locative adjunct}

\subsection{Ergative}
\subsubsection{Core argument}
\subsubsection{Instrumental}
\subsubsection{Comparee marker}

\subsection{Genitive}
\subsection{Locative}
\section{Noun modifiers and determiners}
This section discusses all nouns modifiers and determiners except relative clauses (§ XXX) and complement clauses (§ XXX). 

\subsection{Demonstratives}

\subsection{Quantifiers}
\subsubsection{Universal quantifiers} \label{sec:universal.quant}
\subsubsection{Mid-scalar quantifier} \label{sec:tsuku}
(\ref{sec:partitive.pronouns})
\subsection{Indefinite and definite markers} \label{sec:indefinite.markers}
Japhug has no definite article. The demonstrative \forme{nɯ} and topic markers such as \forme{iɕqʰa} (§ \ref{sec:iCqha})

Like many languages (\citealt[130]{creissels06sgit1}), Japhug  uses bare nouns without any definiteness marking. Bare nouns are most often non-referential, as \japhug{tɕʰeme}{girl} in (\ref{ex:tCheme.tWtAtu}).

\begin{exe}
\ex \label{ex:tCheme.tWtAtu}
\gll ʁnaʁna tɕʰeme tɯ\redp{}tɤ-tu nɤ, kɤndʑɯsqʰaj 	tu-kɤ-sɯ-βzu \\
both girl \textsc{cond}\redp{}\textsc{pfv}-exist \textsc{lnk} \textsc{coll}:sister \textsc{ipfv}-\textsc{inf}-\textsc{caus}-make \\
\glt `If both of them have girls, let them be sisters.' (zrAntCW, 4)
\end{exe}

Bare nouns are less common with referential nouns (except in answers to questions), but examples can be found, as \japhug{qacʰɣa}{fox} in (\ref{ex:qachGa.kW}).

\begin{exe}
\ex \label{ex:qachGa.kW}
\gll qacʰɣa 	kɯ 	maχtɕɯ tɤ-tɯt-a nɯ mɤ-tɯ-ste 	ti 	ɲɯ-ŋu  \\
fox \textsc{erg} I.told.you.so \textsc{pfv}-say[II]-\textsc{1sg} \textsc{dem} \textsc{neg}-2-do.like[III]:\textsc{fact} say:\textsc{fact} \textsc{sens}-be \\
\glt `The fox says: `You do not do as I told you to." (2003qachGa, 44)
\end{exe}

Personal names generally occur as bare nouns, without any definiteness marker as in (\ref{ex:WrJAnpanma}), but there are no constraints against co-occurrence of personal names with the determiner \forme{nɯ} either (see § \ref{sec:personal.names.modifiers}).

\begin{exe}
\ex \label{ex:WrJAnpanma}
\gll  ɯrɟɤnpanma kɯ ʁlaŋsaŋtɕhin ɯ-ɕki  \\
 Padmasambhava \textsc{erg} Gesar \textsc{3sg}-\textsc{dat} \\
\glt `Padmasambhava (told) Gesar.' (Gesar, 2)
\end{exe}

The form \japhug{ci}{one} has among its many functions (in addition to pronoun, numeral and adverb, see § \ref{sec:other.pro}, § \ref{sec:partitive.pronouns}, § \ref{sec:identity.modifier}, § \ref{sec:one.to.ten} and § XXX) that of indefinite article, as in (\ref{ex:ci.indef}) and (\ref{ex:ci.chAGi}). It is typically used to introduce a new referent in a story.

\begin{exe}
\ex \label{ex:ci.indef}
\gll tɕʰeme kɯ-mpɕɯ\redp{}mpɕɤr 	ci 	ɲɤ-nɯ-ɬoʁ \\
girl \textsc{nmlz}:S/A-\textsc{emph}\redp{}beautiful \textsc{indef} \textsc{ifr}-\textsc{auto}-come.out \\
\glt `A very beautiful girl appeared (out of it).' (The flood, 39)
\end{exe}

\begin{exe}
\ex \label{ex:ci.chAGi}
\gll tɕɤlo tɕe tɤ-tɕɯ ci cʰɤ-ɣi qʰe, \\
upstream \textsc{lnk} \textsc{indef}.\textsc{poss}-son \textsc{indef} \textsc{ifr}:\textsc{downstream}-come \textsc{lnk} \\
\glt `A boy came from upstream.' (2003-kWBRa, 41)
\end{exe}

Although \forme{ci} can be used as a partitive pronoun `one of them' (§ \ref{sec:partitive.pronouns}), as a postnominal determiner it does not have partitive meaning. To express a meaning such as `one of the boys', a CN such as \japhug{tɯ-rdoʁ}{one piece} is used instead (§ \ref{sec:ICN}). 

Note that when used as a prenominal modifier, \forme{ci} has a completely different (definite) meaning `the other X' (§ \ref{sec:identity.modifier}).

 \subsection{Topic and focus markers} \label{sec:topic}
 
 \subsubsection{Aforementioned topic} \label{sec:iCqha}
 The marker \japhug{iɕqʰa}{the aforementioned} (from the adverb \japhug{iɕqʰa}{just before}, see § XXX)  is used on referents that have been previously mentioned in the same story, usually only a few sentences back.
 
 
%  \begin{exe}
%\ex \label{ex:indef}
%\gll \textbf{``razri} 	\textbf{kɤtɯm} 	\textbf{ci} 	ɲɯ-ra, 	taqaβ 	ci 	ɲɯ-ra" to-ti qhe   \\
% thread ball \textsc{indef} \const{}-need needle \textsc{indef} \const{}-need \evd{}-say \coord{}  \\
%\glt He told (Rgyabza) ``I need a ball of thread and a needle''.
%\ex \label{ex:icqha}
%\gll tɕendɤre 	ɲo-kho 	qhe, 	tɕe 	ɯ-ndzɤtshi 	kɤ-tsɯm 	nɯ 	tɕu 	qhe 	tɕe, \textbf{iɕqʰa} 	\textbf{kɤtɯm} 	\textbf{nɯ }	ɯʑo 	kɯ 	ko-ndo, 	taqaβ-rna 	nɯ 	ɲɤ-rku qhe,  \\
% \coord{} \evd{}-give \coord{} \coord{} \textsc{3sg}.\poss{}-meal \inftv{}-bring \textsc{compl}  \loc{} \coord{} \coord{} the.aforementioned ball \topic{} he \erg{} \evd{}-take needle-ear \topic{} \evd{}put.in \coord{} \\
%\glt She gave it to him. While (people) brought his meal, he took the ball of thread and put it into the ear of the needle. (Gesar 270-2)
%\end{exe}
%The referent ``ball of thread'', first introduced in sentence \ref{ex:indef}, appears again two sentences later with both the topic markers nɯ and \textit{iɕqʰa}. 
%
%
%There seems to be a limit to the number of sentences that can separate a noun phrase in iɕqʰa from its preceding occurrence (probably no more than five-six), but this topic deserves of systematic study based on all available stories.
 
\subsection{Genitival phrases}
%
%tɕendɤre ɯ-jaʁ nɯtɕu ftsoʁ kɯngɯt ɯ-phɯ ɣɯ srɯnloʁ pjɤ-k-ɤrku-ci
%2003gesar, 239
%
%\subsection{Determiners} \label{sec:determiners}
%\japhug{ɕɯŋarɯra}{each better than the other}
% rɟɤlpu ɕɯŋarɯra kɯ ta-tʰu-nɯ ɕti ri, mɯ-tɤ-nɤla-j ɕti tɕe,
% 2003 qachga, 71
\subsection{Identity modifiers} \label{sec:identity.modifier}

%nɤki tɕheme nɯ ɯ-ɕki ɯ-kɯ-sɤja jo-ɕe, ci tɕheme kɯ-ŋɤn nɯ ɯ-ɕki.
\subsection{Attributes}

\section{The structure of the noun phrase}

\section{Nominal predicates}
 %The noun phrase
%\chapter{The noun phrase} \label{chap:noun.phrase}


\section{Noun modifiers and determiners}
This section discusses all nouns modifiers and determiners except relative clauses (§ XXX) and complement clauses (§ XXX). 
 
\subsection{Number}  \label{sec:number.determiners}
Japhug has two number markers, the dual \forme{ni} and the plural \forme{ra}. These clitics are not obligatory for non-singular arguments (even in the case of human referents), and do not necessary trigger plural or dual agreement on the verb. 

\subsubsection{Dual} \label{sec:dual.determiners}
The dual \forme{ni} is historically related to the numeral \forme{ʁnɯz} (§ \ref{sec:one.to.ten}), but their relationship is synchronically opaque. It combines with the proximal and distal demonstratives \forme{ki} and \forme{nɯ} respectively to form the dual demonstratives \forme{kɯni} and \forme{nɯni} (§ \ref{sec:demonstrative.pronouns}, § \ref{sec:demonstrative.determiners}).

There is no semantic restriction on the use of \forme{ni}, it most often occurs with human referents (\ref{ex:awW.cho.aRi.ni}, \ref{ex:Wmu.Wwa.ni}, \ref{ex:tCiZo.ni}, \ref{ex:ni.ndZisroR}), but is also commonly attested with animals (\ref{ex:ʁnWz.ni}) inanimate objects (\ref{ex:ni.RnaRna}), and placenames (\ref{ex:rgWnba.ni}).

\begin{exe}
\ex \label{ex:rgWnba.ni}
\gll prɤɕta cʰo rgɯnba ni ndʑi-pɤrtʰɤβ ri ŋu \\
pl.n. \textsc{comit} monastery \textsc{du} \textsc{3du}.\textsc{poss}-between \textsc{loc} be:\textsc{fact} \\
\glt `It is between Prashta and the monastery.' (140522 Kamnyu zgo, 115)
\end{exe}

The dual can follow the numeral \japhug{ʁnɯz}{two}, as in (\ref{ex:ʁnWz.ni}). This combination is however very rare (only 13 examples in the corpus out of hundreds of dual \forme{ni}). The opposite order (dual followed by numeral) is not grammatical.

\begin{exe}
\ex \label{ex:ʁnWz.ni}
\gll mbɣɤru nɯ jla ʁnɯz ni ndʑi-tʰɤβ ri ɲɯ-ɕe tɕe \\
plough.beam \textsc{dem} hybrid.yak two \textsc{3du}.\textsc{poss}-between \textsc{loc} \textsc{ipfv}:\textsc{west}-go \textsc{lnk} \textsc{lnk} \\
\glt `The beam of the plough goes between the two hybrid yaks.' 
\end{exe}


The adverb \japhug{ʁnaʁna}{both} (§ XXX) commonly co-occurs with dual, as in (\ref{ex:ni.RnaRna}).
%tɤ-pi ʁnaʁna ʑo pɯ́-wɣ-sat-ndʑi ɲɯ-ŋu. 

\begin{exe}
\ex \label{ex:ni.RnaRna}
\gll zaŋ cʰo raʁ ni ʁnaʁna ʑo ʁja ku-te ɲɯ-ŋu \\
copper \textsc{comit} brass \textsc{du} both \textsc{emph} verdigris \textsc{ipfv}-put[III] \textsc{sens}-be \\
\glt `Both copper and brass can get verdigris.' (30-Com, 101)
\end{exe}

The marker \forme{ni} can appear with a noun phrase comprising two nouns (each with singular referents) linked by the comitative \forme{cʰo} (§ \ref{sec:comitative}).

\begin{exe}
\ex \label{ex:awW.cho.aRi.ni}
\gll  tɕe a-wɯ cʰo a-ʁi ni pjɯ-tɯ-sat mɤ-jɤɣ \\
\textsc{lnk} \textsc{1sg}.\textsc{poss}-grandfather \textsc{comit} \textsc{1sg}.\textsc{poss}-younger.sibling \textsc{du} \textsc{ipfv}-2-kill \textsc{neg}-be.possible:\textsc{fact} \\
\glt `You cannot kill my grandfather and my younger brother.' (2011-05-nyima, 133)
\end{exe}

The dual can also be used with noun dyads (§ \ref{sec:dyads}), as in (\ref{ex:Wmu.Wwa.ni}). 

\begin{exe}
\ex \label{ex:Wmu.Wwa.ni}
\gll   ɯ-mu ɯ-wa ni kɯ ɲɯ-z-nɤja-ndʑi qʰe \\
\textsc{3sg}.\textsc{poss}-mother \textsc{3sg}.\textsc{poss}-father \textsc{du} \textsc{erg} \textsc{ipfv}-\textsc{caus}-be.a.pity-\textsc{du} \textsc{lnk} \\
\glt `Her parents would not be parted from her.' (14-tApitaRi, 305)
\end{exe}

The third person dual pronoun \forme{ʑɤni} is build by combining the pronominal root \forme{-ʑo-} with the dual \forme{ni} (§ \ref{sec:pers.pronouns}), and is not attested in combination with the dual. The first and second dual pronouns \forme{tɕiʑo} and \forme{ndʑiʑo}, do occur with the dual marker as in (\ref{ex:tCiZo.ni}), though examples are very rare.

\begin{exe}
\ex \label{ex:tCiZo.ni}
\gll  tɕiʑo ni wuma ʑo pɯ-amɯmi-tɕi tɕe \\
\textsc{1du} \textsc{du} really \textsc{emph} \textsc{pst}.\textsc{ipfv}-be.in.good.terms-\textsc{1du} \textsc{lnk} \\
\glt `We were in harmony together.' (140512 fushang he yaomo-zh, 85)
\end{exe}

Noun phrases with the dual \forme{ni} are always correlated with a dual prefix on the following noun in possessive constructions or with relator nouns, as in (\ref{ex:rgWnba.ni}), (\ref{ex:ʁnWz.ni}) and (\ref{ex:ni.ndZisroR}). Not a single example of a noun phrase in \forme{ni} followed by a noun with singular of plural possessive prefix is found in the corpus.

\begin{exe}
\ex \label{ex:ni.ndZisroR} 
\gll ɯ-pi ni ndʑi-sroʁ ko-ri tɕe \\
\textsc{3sg}.\textsc{poss}-elder.sibling \textsc{du} \textsc{3du}.\textsc{poss}-life \textsc{ifr}-save \textsc{lnk} \\
\glt `He saved the life of his two brothers.' (qachGa 2012, 139)
\end{exe}

The marker \forme{ni} is not obligatory with dual referents, in particular when the numeral \japhug{ʁnɯz}{two} is present. An overt noun phrase without dual marking can trigger indexation on the verb, especially with collectives expressing a pair of individuals as \japhug{ʁzɤmi}{husband and wife} in (\ref{ex:RjWmbrWg.RzAmi}), but also with other types of noun phrases as in (\ref{ex:nW.talWlAtndZi}).

\begin{exe}
\ex \label{ex:RjWmbrWg.RzAmi}
\gll  kɯɕɯŋgɯ tɕe tɕe atu <qinghai> ʑɴɢɯloʁ nɯtɕu tɕe, ʁjɯmbrɯɣ ʁzɤmi ci pjɤ-tu-ndʑi tɕe,  \\
in.former.times \textsc{lnk}  \textsc{lnk} up.there p.n. p.n. \textsc{dem}:\textsc{loc} \textsc{lnk} dragon husband.and.wife one \textsc{ifr}.\textsc{ipfv}-exist-\textsc{du} \textsc{lnk} \\
\glt `In former times, in Qinghai, in the Mgolog area, there was a couple of dragons.' (150820 qaprANar, 44)
\end{exe}

\begin{exe}
\ex \label{ex:nW.talWlAtndZi}
\gll  ʁdɯxpanaχpu ɯ-tɕɯ cʰo aʑo a-tɕɯ nɯ tɤ-alɯlɤt-ndʑi tɕe, \\
p.n. \textsc{3sg}.\textsc{poss}-son \textsc{comit} \textsc{1sg} \textsc{1sg}.\textsc{poss}-son \textsc{dem} \textsc{pfv}-fight-\textsc{du} \textsc{lnk} \\
\glt `The son of Gdugpa Nagpo and my son were fighting.' (28-smAnmi, 280)
\end{exe}

Such examples are however surprisingly rare in the corpus; dual indexation is most often correlated with a dual marker on the corresponding noun phrase, if overt.

The numeral \japhug{ʁnɯz}{two} without the dual also triggers dual indexation, as in (\ref{ex:RnWz.tundZi}).

\begin{exe}
\ex \label{ex:RnWz.tundZi}
\gll   sɯŋgɯ zɯ tɯrme wuma ʑo kɯ-wxti ʁnɯz tu-ndʑi tɕe\\
forest \textsc{loc} person really \textsc{emph} \textsc{nmlz}:S/A-be.big two exist:\textsc{fact}-\textsc{du} \textsc{lnk}\\
\glt `In the forest, there are two giants.'  (140428 yonggan de xiaocaifeng, 172)
\end{exe}

Dual marking on a noun phrase is not necessarily correlated with dual indexation on the verb, especially, but not exclusively, with inanimate referents, as in (\ref{ex:ni.tomto}). This question is studied in more detail in § XXX.

\begin{exe}
\ex \label{ex:ni.tomto}
\gll  ɯ-mɲaʁ χcʰoʁe ni to-mto. \\
\textsc{3sg}.\textsc{poss}-eye left.and.right \textsc{du} \textsc{pfv}-have.sight \\
\glt `His left and right eyes recovered sight.' (140517 mogui de jing, 105)
\end{exe}

However, a noun phrase with \forme{ni} is never correlated with a plural indexation marker on the verb. Apparent exceptions are either speech errors (a topic treated in § XXX), or cases of ambiguous indexation, as in (\ref{ex:paznAkharnW}).

\begin{exe}
\ex \label{ex:paznAkharnW}
 \gll  nɤ-pi ni kɯ nɤʑo nɯɣi kɤ-sɯso kɯ ʁmaʁ χsɯ-tɤxɯr kɯ pa-z-nɤkʰar-nɯ ɕti tɕe, \\
 \textsc{2sg}.\textsc{poss}-elder.sibling \textsc{du} \textsc{erg} \textsc{2sg} come.back:\textsc{fact} \textsc{inf}-think \textsc{erg} soldier three-round \textsc{erg} \textsc{pfv}:3\fl{}3'-\textsc{caus}-surround-\textsc{pl} be.\textsc{affirm}:\textsc{fact} \textsc{lnk} \\
 \glt `Your two elder brothers, thinking that you are coming back, had (the palace) guarded on all sides by three rows of soldiers.' (qachGa2012, 157)
\end{exe}

Example (\ref{ex:paznAkharnW}) is not completely straightforward, and deserves a detailed comment. The form \forme{paznɤkʰarnɯ} can be parsed as either \forme{pɯ-az-nɤkʰar-nɯ} \textsc{pst}.\textsc{ipfv}-\textsc{prog}-surround-\textsc{pl} `They were guarding it' with vowel fusion (§ XXX) or \forme{pa-z-nɤkʰar-nɯ} \textsc{pfv}:3\fl{}3'-\textsc{caus}-surround-\textsc{pl} `(He/they) had them guard it'. Context makes it clear here that the second option is the correct one, in particular because in the same passage in another version of the same story, we find the verb \forme{pa-sɯ-lɤt} \textsc{pfv}:3\fl{}3'-\textsc{caus}-throw `he had (them) make' with the perfective 3\fl{}3' form of a causative verb (\citealt[242]{jacques16complementation}, § XXX). Moreover, while the phrase \forme{nɤ-pi ni kɯ}  `your two elder brothers' could in principle belong to the infinitival clause in \forme{kɤ-sɯso}\footnote{Incidentally, note that this infinitival clause contains another complement in Hybrid Reported Speech, see § XXX.}, it is clear from the context and the explanations provided by native speakers that \forme{nɤ-pi ni kɯ} is the causer, and \forme{ʁmaʁ χsɯ-tɤxɯr kɯ} `three rows of soldiers' is the causee (also marked by the ergative, see § \ref{sec:causee.kW}). 

We thus observe plural indexation \forme{-nɯ} on the main verb \forme{pa-z-nɤkʰar-nɯ}, while the subject \forme{nɤ-pi ni kɯ}  has a dual marker. However, this is neither a counterexample to the number indexation rule stated above, nor a speech error: rather, it is a consequence of the fact that causees rather than causers can trigger number indexation on the verb in specific cases (see § XXX).

\subsubsection{Plural} \label{sec:plural.determiners}
The plural marker \forme{ra}, like the dual, follows the noun and most of its modifiers, and fuses with the demonstratives \forme{ki} and \forme{nɯ} respectively to build the plural demonstratives \forme{kɯra} and \forme{nɯra} (§ \ref{sec:demonstrative.pronouns}, § \ref{sec:demonstrative.determiners}). The etymology of the plural marker \forme{ra} is unknown, but a potential cognate exists in Pumi (\forme{=ɹə}, (\citealt[135]{daudey14grammar}; Japhug \forme{-a} regularly corresponds to Pumi \forme{-ə} in the native vocabulary). It should not be confused with the auxiliary verb \japhug{ra}{have to, need} (§ XXX), though there are cases where some ambiguity may occur (§ XXX).

Like the dual \forme{ni}, the plural \forme{ra} is compatible with both animate and inanimate referents, as in (\ref{ex:si.ra.cho}) and (\ref{ex:rdAstaR.ra}). It can be a plain marker of plurality as in (\ref{ex:si.ra.cho}).

\begin{exe}
\ex \label{ex:si.ra.cho}
\gll kɯmaʁ si ra cʰo nɯ-mdoʁ mɤ-naχtɕɯɣ \\
other tree \textsc{pl} \textsc{comit} \textsc{3pl}.\textsc{poss}-colour \textsc{neg}-be.the.same:\textsc{fact} \\
\glt `Its colour is different from that of the other trees.' (11-qrontshom, 56)
\end{exe} 

The marker \forme{ra} is also often an associative plural, understandable as `and other things', as in (\ref{ex:rdAstaR.ra}).

\begin{exe}
\ex \label{ex:rdAstaR.ra}
\gll rdɤstaʁ ra pjɯ-tʂaβ-nɯ qʰe tɯrme tu-xtsɯɣ ɲɯ-ŋu \\
stone \textsc{pl} \textsc{ipfv}-cause.to.fall-\textsc{pl} \textsc{lnk} people \textsc{ipfv}-hit \textsc{sens}-be \\
\glt `(Goats and sheep, as they climb high) cause stones (and other things) to fall and these hit people.' (tshAt-qaZo-kAlAG, 4)
\end{exe} 

The plural can follow numerals (even without head noun) to express an approximative number, as in (\ref{ex:XsWm.kWBde}).\footnote{Note that in (\ref{ex:XsWm.kWBde}) \forme{ci ci} is the expression for `sometimes', not used as a numeral, see § XXX.} 

\begin{exe}
\ex \label{ex:XsWm.kWBde}
\gll ci ci χsɯm kɯβde ra ɲɯ-lɤt ɲɯ-ŋgrɤl. tsuku tɕe ʁnɯz jamar ma mɯ́j-lɤt,\\
one one three four \textsc{pl} \textsc{sens}-throw \textsc{sens}-be.usually.the.case. some \textsc{lnk} two about apart.from \textsc{neg}:\textsc{sens}-throw \\
\glt  `Sometimes (dogs) have three or four (litters), some only have two.' (05-khWna, 22)
\end{exe} 

The plural marker \forme{ra} can also indicate approximate location, with or without locative markers. In (\ref{ex:kha.ra}), we find approximate location \forme{ra} in \forme{kʰa ra} `(everywhere) in the house, around the house' and \forme{tɯ-ji ɯ-ngɯ ra} `in the fields', and in (\ref{ex:nWrNa.ra}) with body parts.

This use of \forme{ra} can convey a meaning of distributed location, and is often combined with the adverb \japhug{aʁɤndɯndɤt}{everywhere} (§ \ref{sec:aRandWndAt}). It is reminiscent of plural markers in Kirghiz and Old Japanese, which combine collective, hypocoristic and approximate locative meanings (see \citealt[195]{antonov07ra}).

\begin{exe}
\ex \label{ex:kha.ra}
\gll βʑɯ nɯ wuma ʑo ŋɤn tɕe, tɕendɤre aʁɤndɯndɤt ʑo kʰa ra cʰɯ-rɤpɯ. tɯ-ji ɯ-ngɯ ra cʰɯ-rɤpɯ, \\
mouse \textsc{dem} really \textsc{emph} be.evil:\textsc{fact} \textsc{lnk} \textsc{lnk} everywhere \textsc{emph} house \textsc{pl} \textsc{ipfv}-bear.young \textsc{indef}.\textsc{poss}-field \textsc{3sg}.\textsc{poss}-inside \textsc{pl}  \textsc{ipfv}-bear.young \\
\glt `The mouse is fierce, it has youngs everywhere in the house, and has youngs in the fields.' (27-spjaNkW, 166)
\end{exe} 

\begin{exe}
\ex \label{ex:nWrNa.ra}
\gll nɯ-βri ra ɲɯ-ɬoʁ, nɯ-mke nɯra ɲɯ-ɬoʁ nɯ-rŋa ra brɤβbrɤβ ʑo ɲɯ-ɬoʁ ɲɯ-ŋu. \\
\textsc{3pl}.\textsc{poss}-body \textsc{pl} \textsc{ipfv}-come.out \textsc{3pl}.\textsc{poss}-neck \textsc{dem:pl} \textsc{ipfv}-come.out \textsc{3pl}.\textsc{poss}-face \textsc{pl} \textsc{idph}:II:covered.by.tiny.bumps \textsc{emph} \textsc{ipfv}-come.out  \textsc{sens}-be \\
\glt `(People who suffer from this disease have little blisters) appearing on their body, on their neck and all over their face.' (27-kharwut, 58)
\end{exe} 

The marker \forme{ra} even occurs with referents which are clearly singular, not only in the approximative location function, but also in examples such as (\ref{ex:tAwi.ra}) where the reason for the presence of \forme{ra} is less immediately obvious. In (\ref{ex:tAwi.ra}), a sentence taken from the translation of Rotkäppchen into Japhug (from Chinese, though here the presence of \forme{ra} cannot be due to calque), the function of the plural on the phrase \forme{tɤ-wi ra} `the grandmother' is more subtle: it conveys the idea idea that the impersonation takes on several aspects of the grandmother, not only her physical appearance, but also her voice, as implied by the second clause. 

\begin{exe}
\ex \label{ex:tAwi.ra}
\gll  qapar nɯ kɯ li, [...] tɤ-wi ra to-nɯɕpɯz tɕe, tɕe ɯ-skɤt ra cʰɤ-sɯ-ɤmtɕoʁ ʑo tɕe nɯra to-ti. \\
dhole \textsc{dem} \textsc{erg} again { } \textsc{indef}.\textsc{poss}-grandmother \textsc{pl} \textsc{ifr}-impersonate \textsc{lnk} \textsc{lnk} \textsc{3sg}.\textsc{poss}-voice \textsc{pl} \textsc{ifr}-\textsc{caus}-be.sharp \textsc{emph} \textsc{lnk} \textsc{dem}:\textsc{pl} \textsc{ifr}-say \\
\glt `The wolf was pretending to be the grandmother, and said these (words) with a sharp voice.' (140428 xiaohongmao-zh, 95-96)
\end{exe} 

Just like noun phrases with dual \forme{ni} correlate with dual possessive prefixe (see \ref{ex:ni.ndZisroR} in § \ref{sec:dual.determiners}), those with plural \forme{ra} can only be coreferent with a plural possessive prefix, as \forme{nɯ-} in (\ref{ex:si.ra.nWmat}).

\begin{exe}
\ex \label{ex:si.ra.nWmat}
 \gll  sɯku tɕe tʰɣe kɯ-fse, kɯmaʁ si ra nɯ-mat nɯra ɕ-pjɯ-nɯ-pʰɯt tɕe tu-ndze ɲɯ-ŋu.\\
tree \textsc{lnk} acorn \textsc{nmlz}:S/A-be.like other tree \textsc{pl} \textsc{3pl}.\textsc{poss}-fruit \textsc{dem}:\textsc{pl} \textsc{transloc}-\textsc{ipfv}:\textsc{down}-\textsc{auto}-pluck \textsc{lnk} \textsc{ipfv}-eat[III] \textsc{sens}-be \\
\glt `On the trees, (the bear) plucks acorn or fruits from other trees to eat.' (21-pri, 44)
\end{exe}

Apparent counterexamples such as (\ref{ex:WtaR.ra.Wmat}), where \forme{ra} is followed by a noun with the singular possessive prefix \forme{ɯ-}, occur when the preceding noun phrase is not the possessor of the following noun. For instance, in (\ref{ex:WtaR.ra.Wmat}) \forme{ra} has the vague locative function, and the phrase \forme{tɯ-ŋga ɯ-taʁ ra} `on the clothes' is not the possessor of \japhug{ɯ-mat}{its fruits}, it is a locative adjunct.

\begin{exe}
\ex \label{ex:WtaR.ra.Wmat}
 \gll tɯ-ŋga ɯ-taʁ ra ɯ-mat bɤbɤβ ʑo ku-ndzoʁ. \\
 \textsc{indef}.\textsc{poss}-clothes \textsc{3sg}.\textsc{poss}-on \textsc{pl} \textsc{3sg}.\textsc{poss}-fruit \textsc{idph}:II:in.clusters \textsc{emph} \textsc{ipfv}-\textsc{anticaus}:attach \\
\glt `Its seeds attach on clothes in clusters.' (18-qromJoR, 169)
\end{exe}

The plural \forme{ra} very commonly occurs with headless relatives, with or without a demonstrative, as in (\ref{ex:nW.tCaGi}), where we find both relatives followed by \forme{nɯnɯra} and another one followed by \forme{ra}.

\begin{exe}
\ex \label{ex:nW.tCaGi}
\gll [kɤ-ti mɤ-kɯ-pe kɯ-fse tu-kɯ-ti] nɯnɯra tɕe, [[kɤ-nɯtsɯ kɯ-ra] ra kɯnɤ tu-kɯ-ti] nɯnɯra, 
tɯrme ra kɯnɤ, tɕaɣi tu-sɤrmi-nɯ ŋgrɤl.  \\
\textsc{inf}-say \textsc{neg}-\textsc{nmlz}.S/A-be.good \textsc{nmlz}.S/A-be.like \textsc{ipfv}-\textsc{nmlz}.S/A-say \textsc{dem}:\textsc{pl} \textsc{lnk} \textsc{inf}-hide \textsc{nmlz}.S/A-have.to \textsc{pl} also \textsc{ipfv}-\textsc{nmlz}.S/A-say \textsc{dem}:\textsc{pl} people \textsc{pl} also  parrot \textsc{ipfv}-call-\textsc{pl} be.usually.the.case:\textsc{fact} \\
\glt `Those who say things that one should not say, who say even what should be concealed, even (if they are) people, they call them `parrots'. (24-qro, 125)
\end{exe} 

%ɯʑo sɤz pɣɤtɕɯ kɯ-xtɕi nɯra tu-ndze ɲɯ-ŋu. tɕe nɯnɯ tu-ti-nɯ ɲɯ-ŋu tɕe ɯ-mɤ-ŋu ma,
%ta-ndza ra pɯ́-wɣ-mto me
 
The plural \forme{ra} also occurs between auxiliaries and the preceding complement clause with a verb in finite (\ref{ex:GWkWnWru.ra}) or non-finite (\ref{ex:kAnAjaR.ra}) form, with a vague implication that additional related actions are concerned.

\begin{exe}
\ex \label{ex:GWkWnWru.ra}
 \gll li tɯ-ji ɯ-ŋgɯ ra ɣɯ-ku-nɯru ra ŋgrɤl. \\
 again \textsc{indef}.\textsc{poss}-field \textsc{3sg}.\textsc{poss}-inside \textsc{pl} \textsc{cisloc}-\textsc{ipfv}-eat.crops \textsc{pl} be.usually.the.case:\textsc{fact} \\
\glt `It also (usually) comes to eat crops in the fields.' (24-ZmbrWpGa, 37)
\end{exe}

\begin{exe}
\ex \label{ex:kAnAjaR.ra}
 \gll  ɣɤmdzu tɕe nɯnɯ kɤ-nɤjaʁ ra mɤ-sɤ-nɤz tɕe \\
be.thorny:\textsc{fact} \textsc{lnk} \textsc{dem} \textsc{inf}-touch \textsc{pl} \textsc{neg}-\textsc{deexp}-dare:\textsc{fact} \textsc{lnk} \\
\glt `It is thorny and one does not dare to touch it with the hand.' (11-qrontshom, 91)
\end{exe}

The marker \forme{ra} following a locative noun or adverb can have the meaning `the people/things from X', as in (\ref{ex:alo.ra}), without the need to add a demonstrative (cf \ref{ex:aki.nW} § \ref{sec:demonstrative.determiners}).

\begin{exe}
\ex \label{ex:alo.ra}
 \gll alo ra ɲɯ-mbɣom-nɯ qʰe \\
 upstream \textsc{pl} \textsc{sens}-be.in.a.hurry-\textsc{pl} \textsc{lnk} \\
 \glt `Those in the village, they (do things) in hurry.' (conversation140510 tshering, 175)
\end{exe}

\subsection{Demonstratives} \label{sec:demonstrative.determiners}
Japhug demonstrative determiners are formally identical  to the demonstrative pronouns (§ \ref{sec:demonstrative.pronouns}). They distinguish between proximal and distal demonstratives with different roots, and fuse with the dual and plural markers studied in § \ref{sec:number.determiners}; the proximal \forme{ki} undergoes change to \forme{kɯ-} in those fused forms.

As with the demonstrative pronouns, there are three sets of demonstratives, the base form, the reduplicated one (obtained by reduplicating the first syllable), and the emphatic one, with added \forme{ɯ-} prefix. Note that the latter two sets are not attested in the dual for determiners in the corpus, but the forms exist and are easily deducible from the corresponding plural ones. In addition, there is a medial demonstrative \forme{nɤki} which occurs in prenominal position.

\begin{table}
\caption{Demonstrative determiners}\label{tab:dem.determiners}
\begin{tabular}{ll|l|ll} 
\lsptoprule
&Base form & Reduplicated & Emphatic \\
\midrule
\textsc{prox.sg} & \forme{ki} & \forme{kɯki} &  \forme{ɯkɯki}  \\
\textsc{dist.sg} & \forme{nɯ} &  \forme{nɯnɯ} & \forme{ɯnɯnɯ} \\
\midrule
\textsc{prox.pl} & \forme{kɯni} & X &  X \\
\textsc{dist.pl} & \forme{nɯni} &  X & X \\
\midrule
\textsc{prox.pl} & \forme{kɯra} & \forme{kɯkɯra} &  \forme{ɯkɯkɯra}  \\
\textsc{dist.pl} & \forme{nɯra} &  \forme{nɯnɯra} & \forme{ɯnɯnɯra} \\
\midrule
\textsc{medial} &  \forme{nɤki} \\
\lspbottomrule
\end{tabular}
\end{table}

In Japhug, as in other Gyalrong languages, demonstrative determiners can be either/both pre- and postnominal, as shown by an example such as (\ref{ex:ki.srWnloRpW.ki}) with the proximal \forme{ki} both before and after the noun \japhug{srɯnloʁpɯ}{little ring}.

\begin{exe}
\ex \label{ex:ki.srWnloRpW.ki}
 \gll aʑo ɣɯ-ɕaβ-a tɤ-ŋu tɕe, ki srɯnloʁ-pɯ ki ɲɯ-ɕtʰɯz-a tɕe,  \\
 \textsc{1sg} \textsc{inv}-catch.up:\textsc{fact}-\textsc{1sg} \textsc{pfv}-be \textsc{lnk} \textsc{dem}.\textsc{prox} ring-\textsc{dim} \textsc{dem}.\textsc{prox} \textsc{ipfv}:\textsc{west}-turn.toward-\textsc{1sg} \\
\glt `When (the râkshasas) will be about to catch up with me, I will  turn this little ring towards west (in their direction).' (28-smAnmi, 222)
\end{exe}

All possible combinations of base demonstratives (B) and reduplicated demonstratives (R) are attested as pre- or postnominal determiners:

\begin{itemize}
\item BNB: \forme{ki} N \forme{ki}, \forme{nɯ} N \forme{nɯ} (\ref{ex:ki.srWnloRpW.ki})
\item RNB: \forme{kɯki} N \forme{ki}, \forme{nɯnɯ} N \forme{nɯ} (\ref{ex:kWki.tAYi.ki})
\item BNR: \forme{ki} N \forme{kɯki}, \forme{nɯ} N \forme{nɯnɯ} (\ref{ex:ki.rgAtpu.kWki})
\item RNR: \forme{kɯki} N \forme{kɯki}, \forme{nɯnɯ} N \forme{nɯnɯ} (\ref{ex:kWki.qingjiao.kWki})
\end{itemize}  

The types BNB and RNB, with the postnominal determiner as a base demonstrative, are by far the most common ones in the corpus.

\begin{exe}
\ex \label{ex:kWki.tAYi.ki}
 \gll  aʑo kɯki tɤɲi ki lu-nɤkʰɯkʰrɯt-a tɕe \\
 \textsc{1sg} \textsc{dem}.\textsc{prox} staff \textsc{dem}.\textsc{prox} \textsc{ipfv}:\textsc{upstream}-drag-\textsc{1sg} \textsc{lnk} \\
 \glt `I will drag along this staff (on the ground).' (Kunbzang2003, 225)
\end{exe}
 

\begin{exe}
\ex \label{ex:kWki.qingjiao.kWki}
 \gll iɕqʰa kɯki <qingjiao> kɯki tɕe, ɯ-qa kɯ-wɣrum ɲɯ-ŋu. \\
 the.aforementioned \textsc{dem}.\textsc{prox} plant.name \textsc{dem}.\textsc{prox} \textsc{lnk} \textsc{3sg}.\textsc{poss}-root \textsc{nmlz}:S/A-be.white \textsc{sens}-be \\
 \glt `This (plant that is called) \textit{qingjiao} (in Chinese), its root is white (unlike the other \textit{qingjiao} whose root is red).' (17-ndZWnW, 81)
\end{exe}

\begin{exe}
\ex \label{ex:ki.rgAtpu.kWki}
 \gll ki rgɤtpu kɯki kɯ, iɕqʰa, qaʑo nɯ to-mtsʰi qʰe, li tʂu kɯ-wxti nɯtɕu jo-ɕe tɕe, \\
\textsc{dem}.\textsc{prox} old.man \textsc{dem}.\textsc{prox} \textsc{erg} the.aforementioned sheep \textsc{dem} \textsc{ifr}-lead \textsc{lnk} again road \textsc{nmlz}:S/A-be.big \textsc{dem}:\textsc{loc} \textsc{ifr}-go \textsc{lnk} \\
\glt `The old man, leading the sheep, went to the big road.' (150822 laoye zuoshi zongshi duide-zh, 101
\end{exe}

The emphatic form is only used prenominally as in (\ref{ex:WkWki.arZaB.kWki}) to differentiate in case of confusion -- in this case, because the story is about two persons designated by the term \japhug{tɤ-rʑaβ}{wife}, even if they have different possessors (\textsc{3sg} vs \textsc{1sg}).

\begin{exe}
\ex \label{ex:WkWki.arZaB.kWki}
 \gll   nɯ ɯ-rʑaβ nɯ kɯ, ɯkɯki a-rʑaβ kɯki, kɯki ɕkom ki na-sɯ-ɤβzu tɕe, \\
 \textsc{dem} \textsc{3sg}.\textsc{poss}-wife \textsc{dem} \textsc{erg} \textsc{dem}.\textsc{prox}.\textsc{emph} \textsc{1sg}.\textsc{poss}-wide \textsc{dem}.\textsc{prox} \textsc{dem}.\textsc{prox} muntjac \textsc{dem}.\textsc{prox}  \textsc{pfv}:3\fl{}3'-\textsc{caus}-become \textsc{lnk} \\
\glt `His wife turned this wife of mine into this muntjac.' (140512 fushang he yaomo-zh, 187)
\end{exe}

When the postnominal demonstrative is in plural or dual form, the prenominal one is generally unmarked for number, as in (\ref{ex:kWki.tCheme.kWra}).

\begin{exe}
\ex \label{ex:kWki.tCheme.kWra}
 \gll kɯki tɕʰeme kɯra nɯ-rca aʑo tu-ɕe-a ɲɯ-ntsʰi ma mɯ́j-pe \\
 \textsc{dem} girl \textsc{dem}:\textsc{pl} \textsc{3pl}.\textsc{poss}-following \textsc{1sg} \textsc{ipfv}:\textsc{up}-go-\textsc{1sg} \textsc{sens}-have.better apart.from \textsc{neg}:\textsc{sens}-be.good \\
 \glt `I have no other choice but to go (to heaven) with these girls.' (31-deluge, 61)
 \end{exe}

However, there are also a few examples with plural marking on both pre- and postnominal demonstratives, as in (\ref{ex:nWnWra.pGa.nWra}), a remarkable phenomenon given the fact that the number markers are strictly postnominal. Plural marking on the prenominal demonstrative with a singular postnominal demonstrative is not attested.
 
\begin{exe}
\ex \label{ex:nWnWra.pGa.nWra}
 \gll nɯnɯra pɣa nɯra lonba ʑo ɲɤ-me-nɯ tɕe, ʁʑɯnɯ sqaptɯɣ ɲɤ-k-ɤpa-nɯ-ci. \\
 \textsc{dem.pl} bird  \textsc{dem.pl}  all \textsc{emph} \textsc{ifr}-not.exist \textsc{lnk} young.man eleven \textsc{ifr}-\textsc{evd}-become-\textsc{pl}-\textsc{evd} \\
 \glt `All those birds disappeared, and became eleven young men.' (140520 ye tiane-zh, 121)
\end{exe}

Proximal prenominal demonstratives can be combined with the postnominal \forme{nɯ}, as in (\ref{ex:kWki.Xpi.nW}), where the latter one is used as a topic marker. The opposite combination, a distal prenominal demonstrative with proximal postnominal one, is not attested in the corpus and presumably agrammatical.

\begin{exe}
\ex \label{ex:kWki.Xpi.nW}
 \gll kɯki χpi nɯ pɯpɯŋu nɤ,  \\
 \textsc{dem}.\textsc{prox} story \textsc{dem} \textsc{top} \textsc{lnk} \\
 \glt `As far as this story goes,' (11 examples in the corpus)
\end{exe}

The medial demonstrative \forme{nɤki}, used to designate referents closer to the addressee than the speaker, is found as a pronoun (§ \ref{sec:medial.dem.pro}), but also occurs as a prenominal determiner, with or without postnominal demonstrative (either proximal or distal), as in (\ref{ex:nAki.nAtAYi}) and (\ref{ex:nAki.nAtAri}). It is frequently used with a noun taking a second person possessive prefix -- note that the first syllable \forme{nɤ-} of the demonstrative \forme{nɤki} itself probably originates from the second singular possessive, as proposed in § \ref{sec:medial.dem.pro}.

\begin{exe}
\ex \label{ex:nAki.nAtAYi}
 \gll nɤki nɯ-tɤɲi ɯ-taʁ kɤ-rɤt nɯ ɯβrɤ-kɯ-z-nɤmɲo-a-nɯ \\
 \textsc{dem}:\textsc{medial} \textsc{2pl}.\textsc{poss}-staff \textsc{3sg}.\textsc{poss}-on \textsc{nmlz}:P-write \textsc{dem} \textsc{pot}-2\fl{}1-\textsc{caus}-watch-\textsc{1sg}-\textsc{pl} \\
 \glt `Would you show me what is written on that staff of yours?' (2003sras, 61)
\end{exe}

\begin{exe}
\ex \label{ex:nAki.nAtAri}
 \gll nɤki nɤ-tɤ-ri nɯ ŋotɕu pɯ-tu \\
 \textsc{dem}:\textsc{medial} \textsc{2sg}.\textsc{poss}-\textsc{indef}.\textsc{poss}-thread \textsc{dem} where \textsc{pst}.\textsc{ipfv}-exist \\
\glt `That thread of yours, where is it from?' (Norbzang2005, 180)
\end{exe}

The relative position of prenominal demonstrative and other pronominal elements is not free. The aforementioned topic marker \forme{iɕqʰa} strictly occurs before prenominal demonstratives (as in \ref{ex:kWki.qingjiao.kWki} and \ref{ex:kWki.XsAr.pGAtCW} respectively), while nominal modifiers such as \japhug{χsɤr}{gold} in \ref{ex:kWki.XsAr.pGAtCW} appear closer to the noun. Pronouns coreferent with a possessive prefix on the head noun, however, can be placed either after (\ref{ex:nW.aZo.aCArW.nW}) or before (\ref{ex:aZo.ki.aku.ki}) prenominal demonstratives.

\begin{exe}
\ex \label{ex:kWki.XsAr.pGAtCW}
 \gll kɯki χsɤr pɣɤtɕɯ ki nɤ-jaʁ ɲɯ-kham-a ŋu \\
\textsc{dem}.\textsc{prox} gold bird \textsc{dem}.\textsc{prox} \textsc{1sg}.\textsc{poss}-hand \textsc{ipfv}-give[III]-\textsc{1sg} be:\textsc{fact} \\
\glt `(If you succeed) I will give you this golden bird.' (2012qachGa, 46)
\end{exe}

\begin{exe}
\ex \label{ex:nW.aZo.aCArW.nW}
 \gll nɯtɕu a-tɯrsa ŋu, tɕe nɤʑo kɯ [nɯ aʑo a-ɕɤrɯ nɯnɯra] a-tɤ-tɯ-tɕɤt tɕe, \\
 \textsc{dem}:\textsc{loc} \textsc{1sg}.\textsc{poss}-tomb be:\textsc{fact} \textsc{lnk} \textsc{2sg} \textsc{erg} \textsc{dem} \textsc{1sg} \textsc{1sg}.\textsc{poss}-bone \textsc{dem}:\textsc{pl} \textsc{irr}-\textsc{pfv}-2-take.out \textsc{lnk} \\
\glt `My tomb is there, if you take out my bones (from it),' (150907 niexiaoqian-zh, 109)
\end{exe}

\begin{exe}
\ex \label{ex:aZo.ki.aku.ki}
 \gll  kɯki, aʑo [ki a-ku ki] pɯ-pʰɯt ra \\
 \textsc{dem}.\textsc{prox} \textsc{1sg}  \textsc{dem}.\textsc{prox} \textsc{1sg}.\textsc{poss}-head  \textsc{dem}.\textsc{prox} \textsc{imp}-cut have.to:\textsc{fact} \\
 \glt `Please behead me!' (140507 jinniao-zh, 292)
\end{exe}

The principles governing the presence and absence of the demonstrative determiners, and the choice of the various patterns described above, is particularly complex to describe and will be a topic for future research, when a larger corpus of texts will become available. While the proximal demonstratives always have some deictic function (although it may not be always appropriate to translate them with a demonstrative in other languages such as English), the distal demonstratives clearly contribute to marking topic (§ \ref{sec:topic}) and definiteness (§ \ref{sec:definiteness}), and disentangling these various functions is a complex matter.

The demonstratives \forme{nɯ} and \forme{nɯnɯ} are particularily common after relative clauses (either participial § XXX or finite ones § XXX) and complement clauses (§ XXX) but arguments against analyzing them as subordinators (like English `that') are presented in § XXX. 

Following locative adverbs (or locative postpositional phrases), the distal and proximal demonstratives can be used to express the meaning `the one/those X' as in (\ref{ex:athi.ki}) and (\ref{ex:aki.nW}). Note that the number determiner \forme{ra} can also be used in the same way (example \ref{ex:alo.ra} § \ref{sec:plural.determiners}) even without being combined with a demonstrative.

\begin{exe}
\ex \label{ex:athi.ki}
\gll amaŋ amaŋ, atʰi ki kɯ `a-βɣo mɤ-a<nɯ>tɯɣ-a tɕe a-scawa' ɲɯ-sɯsɤm ɲɯ-ŋu ɣe \\
\textsc{interj}:\textsc{surprise} \textsc{interj}:\textsc{surprise}  downstream \textsc{dem}.\textsc{prox} \textsc{erg} \textsc{1sg}.\textsc{poss}-uncle \textsc{neg}-<auto>meet:\textsc{fact}-\textsc{1sg} lnk 1sg.poss-poor.of \textsc{sens}-think[III] \textsc{sens}-be \textsc{sfp} \\
\glt `The one down there, he is thinking `Poor of me, I will not meet my lama', isn't he?' (2003kandZislama, 203)
\end{exe}

\begin{exe}
\ex \label{ex:aki.nW}
\gll a-pa, aki nɯ staʁlupa kɤ-βde ɯ-spa nɯ mɤ-nɯ-xsi ri, \\
\textsc{1sg}.\textsc{poss}-father down \textsc{dem} born.in.the.year.of.the.tiger \textsc{nmlz}:P-throw.away \textsc{3sg}.\textsc{poss}-material \textsc{dem} \textsc{neg}-\textsc{auto}-\textsc{genr}:know \textsc{lnk} \\
\glt `Father, the one down there, I don't know if he is a (boy) born in the year of the Tiger, to be thrown (in the lake), but...' (2011-05-nyima, 154)
\end{exe}



\subsection{Quantifiers} \label{sec:quantifiers.determiners}


\subsubsection{Universal quantifiers} \label{sec:universal.quant}
The determiner \japhug{tʰamtɕɤt}{all}, from Tibetan \tibet{ཐམས་ཅད་}{tʰams.cad}{all}, is strictly postnominal, as in (\ref{si.thamCAt.kW}). It cannot be used as a pronoun, and there are no examples in the corpus of \japhug{tʰamtɕɤt}{all} following a personal pronoun.

\begin{exe}
\ex \label{si.thamCAt.kW}
 \gll   sɯŋgɯ kɤ-kɯ-nɯχtɕɤn tɕe tɕe si tʰamtɕɤt kɯ nɯnɯ pjɯ-kɯ-sat kɯ-ŋgrɤl ɲɯ-ŋu. \\
 forest \textsc{pfv}-\textsc{nmlz}:S/A-be.dangerous \textsc{lnk} \textsc{lnk} tree all \textsc{erg} \textsc{dem} \textsc{ipfv}-\textsc{genr}:S/P-kill  \textsc{nmlz}:S/A-be.usually.the.case \textsc{sens}-be \\
 \glt `The fierce/dangerous forest, it was (a place where) all the tree would could to kill (people thrown into it).' (28-smAnmi, 191)
\end{exe}

It can also follow a headless relative clause, as in (\ref{WkWndza.thamCAt.nWnWra}), and be followed by demonstratives.

\begin{exe}
\ex \label{WkWndza.thamCAt.nWnWra}
 \gll nɯ-zda rɯdaʁ ɯ-kɯ-ndza tʰamtɕɤt nɯnɯra nɯ-rmi lonba kɯrŋi tu-kɯ-ti ŋu \\
 \textsc{3pl}.\textsc{poss}-companion animal \textsc{3sg}.\textsc{poss}-\textsc{nmlz}:S/A-eat all \textsc{dem}:\textsc{pl}  \textsc{3pl}.\textsc{poss}-name all beast \textsc{ipfv}-\textsc{genr}-say be:\textsc{fact} \\
 \glt `All those which eat the other animals, their name, all of them, is `beast'.'  (150822 kWrNi, 8)
 \end{exe}
 
The combination of a demonstrative such as \japhug{nɯ}{this} with  \japhug{tʰamtɕɤt}{all} does not mean `all of this', but `so much, so many', as in (\ref{kha.nW.thamCAt}) (see § XXX for more examples of this construction).

\begin{exe}
\ex \label{kha.nW.thamCAt}
 \gll iʑora tʰɯ-dɤn-i qʰe, kʰa nɯ tʰamtɕɤt mɯ-ɲɯ-ɤmɯ-xtɕʰɯt-i qʰe \\ 
 \textsc{1pl} \textsc{pfv}-be.many-\textsc{1pl} \textsc{lnk} house \textsc{dem} all \textsc{neg}-\textsc{sens}-\textsc{recip}-have.enough.place-\textsc{1pl} \textsc{lnk} \\
 \glt `There was now more of us (than before), and so many of us could not fit in the house.' (14-tApitaRi, 103-104)
 \end{exe}
 
 Another universal quantifier,  \japhug{kɤsɯfse}{all}, is common as a pronoun (§ \ref{sec:quantifiers.pronouns}) or as an adverb (§ XXX). It is potentially analyzable as a determiner in examples like  (\ref{ex:kAtsa.ra.kAsWfse.kW}) where it follows the noun phrase \forme{kɤtsa ra} `parents and children', and takes the ergative \forme{kɯ}.
 
 \begin{exe}
\ex \label{ex:kAtsa.ra.kAsWfse.kW}
\gll  kɤtsa ra kɤsɯfse kɯ wuma ʑo pjɤ-nɯ-rga-nɯ  \\
parents.and.children \textsc{pl} all \textsc{erg} really \textsc{emph} \textsc{ifr}.\textsc{ipfv}-\textsc{appl}-like-\textsc{pl} \\
\glt `Everybody in the family liked her very much.' (140429 qingwa wangzi, 5)
  \end{exe}
  
A third universal quantifier \japhug{rmɯrmi}{all, all kinds of}, a borrowing from Situ (meaning `everybody'), is attested but only rarely used, in examples such as (\ref{rmWrmi.GW.nWrmi}).
  
\begin{exe}
\ex \label{rmWrmi.GW.nWrmi}
\gll  maka tɤ-rɤku rmɯrmi ɣɯ nɯ-rmi nɯ to-nɤrmi ri maka kɯm mɯ-pjɤ-ɲɟɯ   \\
at.all \textsc{indef}.\textsc{poss}-crops all \textsc{gen} \textsc{3pl}.\textsc{poss}-name \textsc{ifr}-say.name \textsc{lnk} at.all door \textsc{neg}-\textsc{ifr}-\textsc{anticaus}:open \\
\glt `He said the names of all kinds of crops, but the door did not open.' (140512 alibaba-zh, 107)
\end{exe}  

 A fourth universal quantifier, \japhug{lonba}{all} (from Tibetan \tibet{ལོན་པ་}{lon.pa}{reached, enough, completed}), exists in Japhug, but it is not used as a noun determiner and only occurs as an adverb (§ XXX).
% aʑo a-ʁi nɯra lonba aʑo kɯ tɤ-nɤpɯpa-t-a
% 140426 tApAtso kAnWBdaR4, 1
% 
% tɕe, ɯ-tɯ-mbro nɯnɯ cho nɯra
%lonba qaɕti cho naχtɕɯɣ-ndʑi ʑo
%09-sArsi, 17

An alternative construction with a meaning similar to a universal quantifier is the totalitative reduplication (§ XXX) \japhug{kɯ\redp{}kɯ-tu}{all who exist} of the participle of the existential verb \japhug{tu}{exist}, in a post-nominal or head-internal relative clauses, as in (\ref{ex:Wzda.kWkWtu.kW}). 

 \begin{exe}
\ex \label{ex:Wzda.kWkWtu.kW}
\gll tɕe [ɯ-zda ra kɯ\redp{}kɯ-tu] kɯ nɯ-rʑaβ na-nɯ-ɕar-nɯ ɲɯ-ŋu \\
\textsc{lnk} \textsc{3sg}.\textsc{poss}-companion \textsc{pl} \textsc{total}\redp{}\textsc{nmlz}:S/A-exist \textsc{erg} \textsc{3pl}.\textsc{poss}-wife \textsc{pfv}:3\fl{}3'-\textsc{auto}-look.for-\textsc{pl} \textsc{sens}-be \\
\glt `All of his companions took (other women) as their wives.' (Norbzang2005, 57)
  \end{exe}

\subsubsection{Mid-scalar quantifier} \label{sec:tsuku}
The quantifier \japhug{tsuku}{some} is generally used as a pronoun (§ \ref{sec:partitive.pronouns}), but it does occur as a prenominal determiner as in (\ref{ex:tsuku.tWrme}), or a postnominal one as in (\ref{ex:kWmtChW.tsuku}) and (\ref{ex:kWmWrkW.tsuku}). It is most often used in the corpus with human referents, but is compatible with inanimate objects, as shown by (\ref{ex:kWmtChW.tsuku}).

\begin{exe}
\ex \label{ex:tsuku.tWrme}
\gll tsuku tɯrme ra kú-wɣ-mtsɯɣ-nɯ tɕe mɯ́j-ʁdɯɣ, tsuku tɯrme ra [...] kú-wɣ-mtsɯɣ-nɯ tɕe tɕe, wuma ʑo cʰɯ́-wɣ-z-nɯɣmbɤβ-nɯ qʰe ɲɯ́-wɣ-z-nɯtɯfɕɤl-nɯ qʰe ku-rŋgɯ-nɯ ɲɯ-ra.\\
some people \textsc{pl} \textsc{ipfv}-\textsc{inv}-bite-\textsc{pl} \textsc{lnk} \textsc{neg}.\textsc{sens}-be.serious some people \textsc{pl} { } \textsc{ipfv}-\textsc{inv}-bite-\textsc{pl} \textsc{lnk} \textsc{lnk} really \textsc{emph} \textsc{ipfv}-\textsc{inv}-\textsc{caus}-swell-\textsc{pl} \textsc{lnk}  \textsc{ipfv}-\textsc{inv}-\textsc{caus}-have.diarrhea-\textsc{pl} \textsc{lnk} \textsc{ipfv}-lie.down-\textsc{pl} \textsc{sens}-have.to\\
\glt `Some people, when they are stung (by bees) are fine, other people, when they are stung, it causes them swelling and diarrhea and they have to lie down.' (26-ndzWrnaR, 65-67)
\end{exe}

\begin{exe}
\ex \label{ex:kWmtChW.tsuku}
\gll  `pjɯ-nɯβle-a ɲɯ-ra' ɲɤ-sɯso tɕe, kɯmtɕʰɯ tsuku ɲɤ-kʰo tɕe, \\
\textsc{ipfv}-cheat[III]-\textsc{1sg} \textsc{sens}-have.to \textsc{ifr}-think \textsc{lnk} toy some \textsc{ifr}-give \textsc{lnk} \\
\glt `She thought `Let's cheat him' and gave him some toys.' (Norbzang2012, 134)
\end{exe}

\begin{exe}
\ex \label{ex:kWmWrkW.tsuku}
\gll ri kɯ-mɯrkɯ tsuku pjɤ-tu-nɯ tɕe tɕe, \\
\textsc{lnk} \textsc{nmlz}:S/A-steal some \textsc{ifr}.\textsc{ipfv}-exist-\textsc{pl} \textsc{lnk} \textsc{lnk} \\
\glt `There were some thieves.' (X1-khu, 7)
\end{exe}

Note in (\ref{ex:tsuku.tWrme.tWrdoR}) the combination of the quantifier \japhug{tsuku}{some} with the counted noun \japhug{tɯ-rdoʁ}{one piece}, which expresses here a partitive meaning (thirteen of fifteen children for each of them, § \ref{sec:ICN}).

\begin{exe}
\ex \label{ex:tsuku.tWrme.tWrdoR}
\gll tsuku tɯrme tɯ-rdoʁ ɣɯ ɯ-rɟit, sqafsum jamar, sqamŋu jamar tu-kɯ-tu pjɤ-tu. \\
 some person one-piece \textsc{gen} \textsc{3sg}.\textsc{poss}-offspring thirteen about fifteen about \textsc{ipfv}-\textsc{genr}:S/A-exist \textsc{ifr}.\textsc{ipfv}-exist    \\
\glt  `Some (women) had thirteen or fifteen children.' (140426 tApAtso kAnWBdaR, 88)
\end{exe}
 
\subsubsection{Distributive quantifier} \label{sec:raNri}
 Although distributive meaning is generally expressed in Japhug with a counted noun (see in particular § \ref{sec:ICN} and § \ref{sec:CCN}), the postnominal determiner \japhug{raŋri}{each} and its variant \japhug{rɯri}{each} (from Tibetan \tibet{རང་རེ་}{raŋ.re}{each}) can also express distributive meaning, as in (\ref{ex:tWtWpW.raNri}). 
 
\begin{exe}
\ex \label{ex:tWtWpW.raNri}
\gll paʁ rcanɯ, tɯ-tɯpɯ raŋri kɯ ʑo pjɯ-χsu-nɯ ra. \\
pig \textsc{unexpected} one-household each \textsc{erg} \textsc{emph} \textsc{ipfv}-raise-\textsc{pl} have.to:\textsc{fact} \\
 \glt `Each single household has to raise pigs.' (05-paR, 4)
 \end{exe}
 
 It can also be used with numerals, as in (\ref{ex:sqamNu.raNri}), where it refers specifically to days.

 \begin{exe}
\ex \label{ex:sqamNu.raNri}
\gll  sqamŋu raŋri ʑo zgo tu-ɕe pɯ-ŋu ɲɯ-ŋu, \\
fifteen \textsc{each} \textsc{emph} mountain \textsc{ipfv}:\textsc{up}-go \textsc{pst}.\textsc{ipfv}-be \textsc{sens}-be \\
\glt `Every fifteen days, she would go up the mountain.' (Norbzang2005, 57)
  \end{exe}

With a noun phrase with the quantifier \japhug{raŋri}{each} occurs in a clause with a counted noun, the scope of the two quantifiers is ambiguous, as in (\ref{ex:rirAB.raNri}).

\begin{exe}
\ex \label{ex:rirAB.raNri}
\gll rirɤβ raŋri χsɯ-tɤxɯr a-tɤ-tɯ-sɯ-lɤt tɕe, \\
mountain each three-round \textsc{irr}-\textsc{pfv}-2-\textsc{caus}-throw \textsc{lnk} \\
\glt `Drag her three times around each mountain.' (Kunbzang2005, 421)
 \end{exe}
 
When a noun with the determiner \forme{raŋri} is possessor, its possessum is often  followed by the distributive determiner \japhug{tɯka}{each} (and its reduplicated variant \forme{tɯkaka}) as in (\ref{ex:Wmat.raNri}).
 
  \begin{exe}
\ex \label{ex:Wmat.raNri}
\gll   iɕqʰa ɯ-mat raŋri ʑo nɯ ɯ-ru tɯka ntsɯ tu. \\
the.aforementioned \textsc{3sg}.\textsc{poss}-fruit each \textsc{emph} \textsc{dem} \textsc{3sg}.\textsc{poss}-stalk own always exist:\textsc{fact} \\
\glt `Each of its fruits has its own stalk.' (17-thowum, 34)
  \end{exe}
  
The  determiner \japhug{tɯka}{each} can also be used without a possessor in  \forme{raŋri}, for example with the distributive pronouns \japhug{ʑaka}{each his own} and \japhug{ʑakastaka}{each his own} (§ \ref{sec:distributive.pronouns})  as in (\ref{ex:nWkho.tWka}).

   \begin{exe}
\ex \label{ex:nWkho.tWka}
\gll      ʑakastaka nɯ-kʰo tɯka pjɤ-tu tɕe \\
each.his.own \textsc{3pl}.\textsc{poss}-room each \textsc{ifr}.\textsc{ipfv}-exist \textsc{lnk} \\
\glt `Each of them had her own room.' (140508 shie ge tiaowu de gongzhu, 85)
   \end{exe}
   
In addition to possessors, \japhug{tɯka}{each} also follows objects, with broad scope on the whole action (\ref{ex:pCaR.tWka}).

    \begin{exe}
\ex \label{ex:pCaR.tWka}
\gll  pɕaʁ tɯka to-βzu-nɯ tɕe jo-nɯ-ɕe-nɯ.  \\
reverence each \textsc{ifr}-make-\textsc{pl} \textsc{lnk} \textsc{ifr}-\textsc{vert}-go-\textsc{pl} \\
\glt `They made a reverence each and went back.' (28-smAnmi, 176)
   \end{exe}
 
\subsection{Indefinite and definite markers} \label{sec:indefinite.markers}

\subsubsection{Indefinite article} \label{sec:indef.article}
The form \japhug{ci}{one} has among its many functions (in addition to pronoun, numeral and adverb, see § \ref{sec:ci.someone}, § \ref{sec:other.pro}, § \ref{sec:partitive.pronouns}, § \ref{sec:identity.modifier}, § \ref{sec:one.to.ten} and § XXX) that of singular indefinite article, as in (\ref{ex:ci.indef}) and (\ref{ex:ci.chAGi}). It is typically used to introduce a new referent in a story.

\begin{exe}
\ex \label{ex:ci.indef}
\gll tɕʰeme kɯ-mpɕɯ\redp{}mpɕɤr ci ɲɤ-nɯ-ɬoʁ \\
girl \textsc{nmlz}:S/A-\textsc{emph}\redp{}beautiful \textsc{indef} \textsc{ifr}-\textsc{auto}-come.out \\
\glt `A very beautiful girl appeared (out of it).' (The flood, 39)
\end{exe}

\begin{exe}
\ex \label{ex:ci.chAGi}
\gll tɕɤlo tɕe tɤ-tɕɯ ci cʰɤ-ɣi qʰe, \\
upstream \textsc{lnk} \textsc{indef}.\textsc{poss}-son \textsc{indef} \textsc{ifr}:\textsc{downstream}-come \textsc{lnk} \\
\glt `A boy came from upstream.' (2003-kWBRa, 41)
\end{exe}

Although \forme{ci} can be used as a partitive pronoun `one of them' (§ \ref{sec:partitive.pronouns}), as a postnominal determiner it does not have partitive meaning. To express a meaning such as `one of the boys', a CN such as \japhug{tɯ-rdoʁ}{one piece} is used instead (§ \ref{sec:ICN}). 

Note that when used as a prenominal modifier, \forme{ci} has a completely different (definite) meaning `the other X' (§ \ref{sec:identity.modifier}). 

There are no dual or plural indefinite articles in Japhug. The plural marker \forme{ra} can occur after the indefinite \forme{ci}, but with a vague associative meaning `and other things' as in (\ref{ex:ci.ra}).

\begin{exe}
\ex \label{ex:ci.ra}
 \gll  ndʑi-tɕɯ ci, ndʑi-me ci ra to-tu. \\
 \textsc{3du}.\textsc{poss}-son \textsc{indef}  \textsc{3du}.\textsc{poss}-girl \textsc{indef} \textsc{pl} \textsc{ifr}-exist \\
 \glt  `They$_{du}$ had a boy and a girl (etc).' (150827 tianluo-zh, 155)
\end{exe}

 The indefinite \forme{ci} is not obligatory for indefinite referents (whether specific or non-specific), and bare NPs can used as \japhug{fsapaʁ}{animal} and \japhug{qapar}{dhole} in example (\ref{ex:ci.ra2}).
 

\begin{exe}
\ex \label{ex:ci.ra2}
 \gll  fsapaʁ nɯ-me, a-pɯ-si qhe, `nɯ qapar kɯ ta-ndza ŋu ma' tu-ti-nɯ ɕti ma, \\
 animal \textsc{pfv}-not.exist \textsc{irr}-\textsc{pfv}-die \textsc{lnk} \textsc{dem} dhole \textsc{erg} \textsc{pfv}:3\fl{}3'-eat be:\textsc{fact} \textsc{sfp} \textsc{ipfv}-say-\textsc{pl} be.\textsc{affirm}:\textsc{fact} \textsc{lnk}  \\
 \glt `When an animal disappears, dies, people say `A dhole ate it.' (28-qapar, 
\end{exe}


\subsubsection{Indefinite pronoun as modifier} \label{sec:indefinite}
The indefinite pronoun \japhug{tʰɯci}{something} (§ \ref{sec:thWci}) has marginal uses as a prenominal indefinite modifier, as in  (\ref{ex:thWci.laXCi}), (\ref{ex:thWci.WjmNo}) and (\ref{ex:laXtCha.ci.nWnW}) below. 

\begin{exe}
\ex \label{ex:thWci.laXCi}
\gll   tʰɯci laχɕi ci ɕ-pɯ-nɯ-βzjoz-nɯ tɕe, jɤ-ɕe-nɯ ra \\
something trade \textsc{indef} \textsc{transloc-imp-auto}-learn-\textsc{pl} \textsc{lnk} \textsc{imp}-go-\textsc{pl} have.to:\textsc{fact} \\
\glt `Go and learn some trade!' (140508 benling gaoqiang de si xiongdi-zh, 29)
 \end{exe}
 
 This construction arose perhaps from the use of the pronoun \forme{tʰɯci} as head of a postnominal relative clause with the verb \japhug{fse}{be like}, as illustrated by examples like (\ref{ex:thWci.kAnWsaXCWB}) or (\ref{ex:thWci.akAspa}) in § \ref{sec:thWci}. Turning the verb \japhug{fse}{be like} to a finite form as in (\ref{ex:thWci.WjmNo}) could cause the indefinite \forme{tʰɯci}, head of the relative in (\ref{ex:thWci.kAnWsaXCWB}), to be reanalyzed as the prenominal modifier of the immediately adjacent noun in (\ref{ex:thWci.WjmNo}).

 \begin{exe}
\ex \label{ex:thWci.kAnWsaXCWB}
\gll nɯra [tʰɯci [kɤ-nɯsaχɕɯβ kɯ-fse]] pɯ-ŋu wo.  \\
\textsc{dem}:\textsc{pl} something \textsc{inf}-have.a.contest \textsc{nmlz}:S/A-be.like \textsc{pst}.\textsc{ipfv}-be \textsc{sfp} \\
\glt `It was like a kind of contest.' (160706 thotsi, 16)
 \end{exe}
 
\begin{exe}
\ex \label{ex:thWci.WjmNo}
\gll [tʰɯci ɯ-jmŋo] ci ʑo pɯ-fse ri \\
something \textsc{3sg}.\textsc{poss}-dream one \textsc{emph} \textsc{pst}.\textsc{ipfv}-be.like \textsc{lnk} \\
\glt `It looked like (he had had) some dream.' (Lobzang2005, 74)
 \end{exe}
 
 
\subsubsection{The marking of definiteness} \label{sec:definiteness}
Japhug has no dedicated definite determiner, but  \forme{nɯ} and \forme{nɯnɯ}  as demonstrative determiners (\ref{sec:demonstrative.determiners}) and as topic markers (\ref{sec:topic}) and the prenominal aforementioned topic marker \forme{iɕqʰa} (§ \ref{sec:iCqha}) are generally used with definite referents.  

Example (\ref{ex:ci.joGi}) illustrates a typical example with the determiner \forme{nɯ}; the indefinite article \forme{ci} (§ \ref{sec:indef.article}) occurs in the first introduction of a new referent in the story as in the first clause of example (\ref{ex:ci.joGi}), but on the following occurrence of the same noun \forme{nɯ} is found.

\begin{exe}
\ex \label{ex:ci.joGi}
 \gll  tɕe qajdo ci jo-ɣi tɕe, tɕe qajdo nɯ kɯ `mo laz tu, pʰo laz me' to-ti. \\
 \textsc{lnk} crow \textsc{indef} \textsc{ifr}-come \textsc{lnk} \textsc{lnk} crow \textsc{dem} \textsc{erg} girl karma exist:\textsc{fact} boy karma not.exist:\textsc{fact} \textsc{ifr}-say \\
 \glt `A crow came. The crow said: `The girl will have chance, the boy won't.'' (28-qAjdoskAt, 8)
\end{exe}

However, although nouns phrases followed by \forme{nɯ} and \forme{nɯnɯ} more often than not denote definite referents, these determiners cannot be analyzed as definite articles, as noun phrases with \forme{nɯ} or \forme{nɯnɯ} can in certain cases have indefinite referents. 

A very clear case of use of \forme{nɯ} with an indefinite referent occurs on nouns serving as heads of head-internal relative clauses. A well-attested typological generalization is that in this type of relative clauses, definiteness marking is agrammatical (see \citealt{basilico96internally} and § XXX). In Khroskyabs, \citet[636]{lai17khroskyabs} reports that the definiteness marker \forme{=tə} is indeed not accepted on the head noun of head-internal relatives. In Japhug however, \forme{nɯ} does occur in such a syntactic context. For instance, in (\ref{ex:tAnmaR.nW.kW}), the head \forme{tɤ-nmaʁ nɯ kɯ} is subject of the participle \japhug{ɲɯ-kɯ-nɯ-ɕar}{looking for}, and is embedded in the participial relative clause indicated in brackets -- the presence of the ergative \forme{kɯ} precludes to analyze it as a post-nominal relative (§ XXX). From the meaning of the sentence the head \japhug{tɤ-nmaʁ}{husband} is clearly indefinite non-specific non-generic  (see \citealt[286-291]{lehmann84relativsatz}). The fact that it takes the marker \forme{nɯ} shows that this marker, unlike Khroskyabs \forme{=tə}, is not primarily marking definiteness.

\begin{exe}
\ex \label{ex:tAnmaR.nW.kW}
 \gll tɕeri [tɤ-nmaʁ nɯ kɯ ɯ-rʑaʁ kɯ-ɤntɕʰɯ ɲɯ-kɯ-nɯ-ɕar], aʁɤndɯndɤt tɤndɤɣri tu-kɯ-βzu pjɤ-tu.  \\
but  \textsc{indef}.\textsc{poss}-husband \textsc{dem} \textsc{erg} \textsc{3sg}.\textsc{poss}-wife  \textsc{nmlz}:S/A-be.many \textsc{ipfv}-\textsc{nmlz}:S/A-\textsc{auto}-search everywhere  illegitimate.child  \textsc{ipfv}-\textsc{nmlz}:S/A-make \textsc{ifr}.\textsc{ipfv}-exist \\
\glt `However there were husbands who were looking for several women and had illegitimate children.' (140427 tAndAGri, 3)
\end{exe}

Other cases of indefinite noun phrase with \forme{nɯ} are observed with left-dislocated topics. In example (\ref{ex:RnWz.nWnW}), we find a type of tail-head linkeage  (§ XXX) where both the noun phrase \japhug{spjaŋkɯ ʁnɯz}{two wolves} and the verb \japhug{ɲɤ-k-ɤtɯɣ-ci}{he met} are repeated; in the second occurrence, the noun phrase is topicalized and is followed by the topic marker \forme{nɯnɯ}, with a slight pause of hesitation. The determiner \forme{nɯnɯ} in this clause, unlike \forme{nɯ} in (\ref{ex:ci.joGi}), does not mark definiteness: that clause cannot be understood as `He met the two wolves'.

\begin{exe} 
\ex \label{ex:RnWz.nWnW} 
 \gll spjaŋkɯ ʁnɯz ɲɤ-k-ɤtɯɣ-ci. spjaŋkɯ ʁnɯz nɯnɯ, tɕendɤre ɲɤ-k-ɤtɯɣ-ci tɕe iɕqʰa, kɯ-rɤ-ntɕʰa nɯ wuma ʑo ɲɤ-mu. \\ 
 wolf two \textsc{ifr}-\textsc{evd}-meet-\textsc{evd}  wolf two \textsc{dem} \textsc{lnk} \textsc{ifr}-\textsc{evd}-meet-\textsc{evd} \textsc{lnk} the.aforementioned \textsc{nmlz}:S/A-\textsc{a.pass}:\textsc{n.hum}-kill \textsc{dem} really \textsc{emph} \textsc{ifr}-be.afraid \\ 
 \glt `He$_i$ (the butcher) met two wolves. He$_i$ met two wolves, and the butcher$_i$ was very much afraid.' (150902 liaozhai lang-zh, 7-8)
\end{exe}

The determiners \forme{nɯ} or \forme{nɯnɯ} are not attested in the corpus with the indefinite singular article \forme{ci} if both have scope on the same noun. In all cases with \forme{ci} followed by \forme{nɯ} (other than the identity pronoun in § \ref{sec:other.pro}), or of \forme{nɯ} followed by \forme{ci} in the corpus, they belong to different constituents. For instance, in (\ref{ex:ci.YAZGAsAphAr}), \forme{ci} is in adverbial use (`a little, once', see § XXX) and does not belong to the preceding noun phrase.  

\begin{exe}
\ex \label{ex:ci.YAZGAsAphAr}
\gll [tɕʰeme nɯ] ci ɲɤ-ʑɣɤ-sɤpʰɤr qʰe  \\
girl \textsc{dem} one \textsc{ifr}-\textsc{refl}-shake \textsc{lnk} \\
\glt `The girl shook herself.' (02-deluge2012, 125)
\end{exe}

In (\ref{ex:laXtCha.ci.nWnW}) although \forme{nɯnɯ} follows \forme{ci}, it has scope over the both preceding phrases, which are left-dislocated and followed by a pause.

\begin{exe}
\ex \label{ex:laXtCha.ci.nWnW}
\gll  kɤ-xtɕɤr tɕe nɯnɯ tɕe tɕe iɕqʰa, [[tʰɯci tɯmbri tɤ-ri kɯ-fse kɯ] [laχtɕʰa ci] nɯnɯ], ci kú-wɣ-sɯ-pa tɕe, kú-wɣ-xtɕɤr, \\
\textsc{inf}-attach \textsc{lnk} \textsc{dem} \textsc{lnk} \textsc{lnk} the.aforementioned something rope \textsc{indef}.\textsc{poss}-thread \textsc{nmlz}:S/A-be.like \textsc{erg} thing \textsc{indef} \textsc{dem} one \textsc{ipfv}-\textsc{inv}-\textsc{caus}-do \textsc{lnk} \textsc{ipfv}-\textsc{inf}-attach \\
\glt ``To attach' (means), to put together, attach something with something like a rope or a thread.'  (150902 kAxtCAr, 2-3)
\end{exe}

The aforementioned topic marker \forme{iɕqʰa} (§ \ref{sec:iCqha}) is almost always used with definite referents when prenominal, as in (\ref{ex:RnWz.nWnW}) above, and is the closest candidate to be analyzed as a definiteness marker in Japhug. It does occur with non-specific generic referents as in (\ref{ex:lWlAmu}), including some that are very clearly indefinite as in (\ref{ex:lApWG}); note the absence of postnominal determiner \forme{nɯ} (\ref{ex:lApWG}).

\begin{exe}
\ex \label{ex:lWlAmu}
 \gll iɕqʰa lɯlɤmu nɯ tʰɯ-rɤpɯ tɕe tɕe ɯ-sŋi tɕe kɤ-nɯ-rŋgɯ nɯ stʰɯci mɯ́j-tsu ma ɯ-pɯ ra χse ɲɯ-ra tɕe, \\
 the.aforementioned female.cat \textsc{dem} \textsc{ipfv}-bear.young \textsc{lnk} \textsc{lnk} \textsc{3sg}.\textsc{poss}-day \textsc{lnk} \textsc{inf}-\textsc{auto}-lie.down \textsc{dem} so.much \textsc{neg}:\textsc{sens}-have.time.to \\
 \glt `A/the female cat (unlike male cats), when it had had youngs, does not have time to sleep during the day, as it has to feed its youngs.' (21-lWLU, 
\end{exe}

\begin{exe}
\ex \label{ex:lApWG}
\gll  iɕqʰa lɤpɯɣ ɯ-rɣi ʑo fse. \\
the.aforementioned radish \textsc{3sg}.\textsc{poss}-seed \textsc{emph} be.like:\textsc{fact} \\
\glt `It looks like a radish seed.' (hist-26-qro-fourmi, 61)
\end{exe}

In  (\ref{ex:laXtCha.ci.nWnW}), \forme{iɕqʰa}  also precedes two phrases involving indefinite referents, but  there is a marked pause, and this is a case of \forme{iɕqʰa} in its function as speech filler (see § XXX).

\subsubsection{Absence of definiteness marking}
Like many languages (\citealt[130]{creissels06sgit1}), Japhug uses bare nouns without any definiteness marking. Bare nouns are most often non-referential, as \japhug{tɕʰeme}{girl} in (\ref{ex:tCheme.tWtAtu}).

\begin{exe}
\ex \label{ex:tCheme.tWtAtu}
\gll ʁnaʁna tɕʰeme tɯ\redp{}tɤ-tu nɤ, kɤndʑɯsqʰaj tu-kɤ-sɯ-βzu \\
both girl \textsc{cond}\redp{}\textsc{pfv}-exist \textsc{lnk} \textsc{coll}:sister \textsc{ipfv}-\textsc{inf}-\textsc{caus}-make \\
\glt `If both of them have girls, let them be sisters.' (zrAntCW, 4)
\end{exe}

Bare nouns are less common with referential nouns (except in answers to questions), but examples can be found, as \japhug{qacʰɣa}{fox} in (\ref{ex:qachGa.kW}).

\begin{exe}
\ex \label{ex:qachGa.kW}
\gll qacʰɣa 	kɯ maχtɕɯ tɤ-tɯt-a nɯ mɤ-tɯ-ste ti ɲɯ-ŋu \\
fox \textsc{erg} I.told.you.so \textsc{pfv}-say[II]-\textsc{1sg} \textsc{dem} \textsc{neg}-2-do.like[III]:\textsc{fact} say:\textsc{fact} \textsc{sens}-be \\
\glt `The fox says: `You do not do as I told you to." (2003qachGa, 44)
\end{exe}

Personal names generally occur as bare nouns, without any definiteness marker as in (\ref{ex:WrJAnpanma}), but there are no constraints against co-occurrence of personal names with the determiner \forme{nɯ} either (see § \ref{sec:personal.names.modifiers}).

\begin{exe}
\ex \label{ex:WrJAnpanma}
\gll  ɯrɟɤnpanma kɯ ʁlaŋsaŋtɕhin ɯ-ɕki  \\
 Padmasambhava \textsc{erg} Gesar \textsc{3sg}-\textsc{dat} \\
\glt `Padmasambhava (told) Gesar.' (Gesar, 2)
\end{exe}

 \subsection{Topic markers} \label{sec:topic}
 
  \subsubsection{Delimitative topic} \label{sec:delimitative}
The delimitative topic marker \forme{pɯ\redp{}pɯ-ŋu nɤ} `as for..., concerning...' is transparently derived from the past imperfective of the verb `be' in conditional form `if it was...' (with reduplication of the first syllable, see § XXX), as other copulas such as affirmative \japhug{ɕti}{be} and \japhug{maʁ}{not be} in (\ref{ex:pWpWmaʁ}).

\begin{exe}
\ex \label{ex:pWpWmaʁ}
\gll nɯnɯ koŋla ʑo tɤɕime pɯ\redp{}pɯ-maʁ nɤ \\
\textsc{dem} really \textsc{emph} princess \textsc{cond}\redp{}\textsc{pst.ipfv}-not.be lnk \\
\glt `If she was not really a princess,' (140519 wandou gongzhu, 71)
\end{exe}

The delimitative construction generally has scope over a noun phrase, which can have an additional demonstrative \forme{nɯ} as topicalizer as in (\ref{ex:nW.pWpWNunA}) (see § \ref{sec:nW.topic}).

\begin{exe}
\ex \label{ex:nW.pWpWNunA}
\gll a-mu nɯ pɯ\redp{}pɯ-ŋu nɤ, qhlɯ ʁdɯxpakɤrpu ɣɯ ɯ-me stu kɯ-xtɕi nɯ a-mu ɲɯ-pe, \\
\textsc{1sg}.\textsc{poss}-mother \textsc{dem} \textsc{cond}\redp{}\textsc{pst.ipfv}-be \textsc{lnk} nâga p.n \textsc{gen} \textsc{3sg}.\textsc{poss}-daughter most \textsc{nmlz}:S/A-be.small \textsc{dem} \textsc{1sg}.\textsc{poss}-mother \textsc{sens}-be.good \\
\glt `As for my mother, the daughter of the Nâga Gdugpa dkarpo is good to be my mother.' (Gesar, 5)
\end{exe}

In this construction, the verb is in the process of becoming grammaticalized as a topic particle. It is possible to find examples where the verb still takes person indexation in the delimitative construction when the topicalized element is a first or second person pronoun, as in (\ref{ex:pWpWNuanA}). 

\begin{exe}
\ex \label{ex:pWpWNuanA}
\gll aʑo pɯ\redp{}pɯ-ŋu-a nɤ, kɤndʑɯʁi kɯmŋu tu-j, \\
\textsc{1sg} \textsc{cond}\redp{}\textsc{pst.ipfv}-be-\textsc{1sg} \textsc{lnk} siblings five exist:\textsc{fact}-\textsc{1sg} \\
\glt `Concerning me, we are five brothers and sisters.' (hist140501 tshering skyid, 1)
\end{exe}

However, there are also examples with first or second person pronoun without indexation on the delimitative marker, as in (\ref{ex:pWpWNunA}), (\ref{ex:pWpWNunA2}) and (\ref{ex:pWpWNunA3}), where a first person singular form \forme{pɯ\redp{}pɯ-ŋu-a nɤ} or second person \forme{pɯ\redp{}pɯ-tɯ-ŋu nɤ} would have been expected. Such examples show that \forme{pɯpɯŋunɤ} has ceased to be analyzed as a verb form at least in these cases. Moreover, third person plural and dual indexation is hardly ever found in the delimitative construction.

\begin{exe}
\ex \label{ex:pWpWNunA}
\gll nɤʑo pɯpɯŋunɤ, ɬɤndʐi ra ɣɯ nɯ-kɯ-βʁa, nɯ-rɟɤlpu tɯ-ŋu \\
\textsc{2sg} as.for demon \textsc{pl} \textsc{gen} \textsc{3pl.poss}-\textsc{nmlz}:S/A-be.victorious \textsc{3pl.poss}-king 2-be:\textsc{fact} \\
\glt `You, you are the king of the demons.' (hist140512 fushang he yaomo-zh, 61)
\end{exe}

\begin{exe}
\ex \label{ex:pWpWNunA2}
\gll  aʑo kɯ-fse pɯpɯŋunɤ, ɕɯŋgɯ sɤ-xtɕɯ\redp{}xtɕi nɯtɕu, χpɯn lɤ-kɤ-ta, \\
\textsc{1sg} \textsc{nmlz}:S/A-be.like as.for  before \textsc{conv}-\redp{}be.small \textsc{dem}:\textsc{loc} monk \textsc{pfv}:\textsc{upstream}-\textsc{nmlz}:P-put \\
\glt `For instance me, (I was) sent to become monk early in my childhood.' (160721 XpWN, 7)
  \end{exe}

\begin{exe}
\ex \label{ex:pWpWNunA3}
\gll aʑo pɯpɯŋunɤ, nɯnɯ [...] aʑo ɣɯ a-ndʐa nɯ tu-o<nɯ>lɯlat-a pɯ-ŋu tɕe, \\
\textsc{1sg} as.for \textsc{dem} { } \textsc{1sg} \textsc{gen} \textsc{1sg}.\textsc{poss}-reason \textsc{dem} \textsc{ipfv}-<\textsc{auto}>fight-\textsc{1sg} \textsc{pst}.\textsc{ipfv}-be \textsc{lnk} \\
\glt  As for me, I was fighting for my own sake.' (140512 abide he mogui-zh, 92)
 \end{exe}
 
A short form \forme{ŋunɤ} instead of \forme{pɯpɯŋu nɤ} is also attested, as in (\ref{ex:WNga.ra.NunA}).

\begin{exe}
\ex \label{ex:WNga.ra.NunA}
\gll ma ɯ-ŋga ra ŋunɤ, maka wuma ʑo ko-ɴqhi ma. \\
\textsc{lnk} \textsc{3sg}.\textsc{poss}-clothes \textsc{pl} as.for at.all really \textsc{emph} \textsc{ifr}-be.dirty \textsc{lnk} \\
\glt `As for his clothes, they had become very dirty.' (conversation 140510)
\end{exe}
 
The delimitative topic  construction is appropriate to introduce the main topic of a following discourse (as in \ref{ex:pWpWNuanA} and \ref{ex:pWpWNunA2}), but can be used for contrastive topics, as in example (\ref{ex:pWpWNunA3}) where the speaker expresses a contrast between his and the addresses action (`you, you were fighting for the sake of other people').


 \subsubsection{Aforementioned topic} \label{sec:iCqha}
 The marker \japhug{iɕqʰa}{the aforementioned}  is used on referents that have been previously mentioned in the same story, usually only a few sentences back. It is strictly prenominal. 
 
Example (\ref{ex:iCqha.aforementioned}) illustrates the most typical use of this marker. Sentence (\ref{ex:kAtWm}) introduces a new reference, \japhug{kɤtɯm}{ball of thread} marked with the indefinite article \forme{ci} (§ \ref{sec:indef.article}). Three clauses later in (\ref{ex:iCqha.kAtWm}), the same referent occurs again with two topic markers, the postnominal \textit{nɯ} and the prenominal \textit{iɕqʰa}.
 
 
\begin{exe}
\ex \label{ex:iCqha.aforementioned}
\begin{xlist}
\ex \label{ex:kAtWm}
\gll `razri \textbf{kɤtɯm} \textbf{ci} ɲɯ-ra, taqaβ ci ɲɯ-ra' to-ti qʰe   \\
 thread ball \textsc{indef} \textsc{sens}-need needle \textsc{indef} \textsc{sens}-need \textsc{ifr}-say \textsc{lnk}  \\
\glt `He told (Rgyabza) `I need a ball of thread and a needle.''  
\ex  
\gll tɕendɤre ɲɤ-kʰo qʰe,  \\
\textsc{lnk} \textsc{ifr}-give \textsc{lnk}   \\
\glt `She gave it to him.'
\ex 
\gll  tɕe ɯ-ndzɤtsʰi ka-tsɯm-nɯ nɯtɕu qʰe tɕe,   \\
 \textsc{lnk} \textsc{3sg}.\textsc{poss}-meal \textsc{pfv}:3\fl{}3'-bring-\textsc{pl} \textsc{dem}:\textsc{loc}  \textsc{lnk} \textsc{lnk}    \\
\glt `When they brought his meal,'
\ex \label{ex:iCqha.kAtWm}
\gll   \textbf{iɕqʰa} \textbf{kɤtɯm} \textbf{nɯ} ɯʑo kɯ ko-ndo, \\
   the.aforementioned ball \textsc{dem} \textsc{3sg} \textsc{erg} \textsc{ifr}-take \\
\glt `he took the ball of thread, and...' (Gesar 270-272)
\end{xlist}
\end{exe}
 
A systematic study of the use of the topic marker \forme{iɕqʰa} in Japhug must overcome two inherent difficulties. First, this topic marker is homophonous with (and historically related to) the speech filler \forme{iɕqʰa} (§ XXX) and with the adverb \japhug{iɕqʰa}{just now}, which can also precede noun phrases. Listening to the sound files can help distinguishing between the three, as the speech filler is always followed by a pause (and optionally by the demonstrative \forme{nɯ}), but there are still ambiguous sentences (see below). Second, \forme{iɕqʰa} occurs on nouns designating entities that the speaker considers to have been previously referred to in the conversation, even if they are not present in the same recording. 

For instance in (\ref{ex:iCqha.pɣArnoR}) the noun \japhug{pɣɤrnoʁ}{a species of fungus} is used with \forme{iɕqʰa}, although this name does not occur before in the same text; it was however mentioned the day before in another recording.

\begin{exe}
\ex \label{ex:iCqha.pɣArnoR}
\gll nɯ zdɯmqe cʰo iɕqʰa, pɣɤrnoʁ nɯni ndʑi-tsʰɯɣa wuma ʑo naχtɕɯɣ. \\
\textsc{dem} fungi.sp. \textsc{comit} the.aforementioned fungi.sp. \textsc{dem}:\textsc{du} \textsc{3du}.\textsc{poss}-form really \textsc{emph} be:identical:\textsc{fact} \\
\glt `The \forme{zdɯmqe} and the \forme{pɣɤrnoʁ} are very similar.' (23-mbrAZim, 82)
\end{exe}

 
The topic marker \forme{iɕqʰa} transparently comes from the adverb \japhug{iɕqʰa}{just now} (§ XXX). The pivot constructions that allowed reanalysis from adverb to prenominal topic marker are very probably headless relatives (§ XXX) as in  (\ref{ex:iCqha.tAtWta}), or complement clauses as in (\ref{ex:iCqha.ZnWzmWnmuta}). 

\begin{exe}
\ex \label{ex:iCqha.tAtWta}
 \gll  [iɕqʰa tɤ-tɯt-a] nɯ tú-wɣ-stu qʰe, \\
 just.now \textsc{ifr}-say[II]-\textsc{1sg} \textsc{dem} \textsc{ipfv}-\textsc{inv}-do.like \textsc{lnk} \\
\glt `One does as I just said, and...' (2002tWsqar, 139)
\end{exe}

\begin{exe}
\ex \label{ex:iCqha.ZnWzmWnmuta}
 \gll iɕqʰa [ʑ-nɯ-z-mɯnmu-t-a] nɯ mɯ-pjɤ-pe rcama.  \\
the.aforementioned  \textsc{transloc}-\textsc{pfv}-\textsc{caus}-move-\textsc{pst}:\textsc{tr}-\textsc{1sg} \textsc{dem} \textsc{neg}-\textsc{ifr}.\textsc{ipfv}-be.good \textsc{fsp} \\
\glt `It was probably not a good thing that I had moved them (as I said above).' (150819 kumpGa, 45)
 \end{exe}
 
 These sentences are still synchronically ambiguous in Japhug; in  (\ref{ex:iCqha.ZnWzmWnmuta}) the context makes it clear that \forme{iɕqʰa} is the topic marker (since the fact of having moved (the eggs) had been told a few sentences back) and not an adverb `just now' with a temporal reference in the past, as the meaning would be `it was probably not a good thing that I had just moved them' (an impossible interpretation in this context, since this sentence is an explanation why several eggs had not given chicks, several days after they had been brought to another place). However, extracted from the context, both interpretation would be equally possible for (\ref{ex:iCqha.ZnWzmWnmuta}), and correspond to two distinct syntactic structures.

With postnominal (§ XXX) or left-headed head-internal relative clauses (§ XXX) as in (\ref{ex:tWrpa.thafse}), \forme{iɕqʰa} can also be ambiguous. Since the adverb \japhug{iɕqʰa}{just now} can occur both before the object (\ref{ex:tWrpa.thWfseta}) or before the verb (\ref{ex:tWrpa.thWfseta2}) in an independent clause, a relative such as (\ref{ex:tWrpa.thafse}) can be either interpreted `the axe (mentioned above) that he had whetted' (with the topic marker \forme{iɕqʰa} outside of the relative clause, having scope on its head) and `the axe that he had just whetted' with the adverb \japhug{iɕqʰa}{just now} inside the relative clause.

 \begin{exe}
\ex \label{ex:tWrpa.thafse}
 \gll  tɕendɤre <luban> kɯ iɕqʰa [tɯrpa tʰa-fse] nɯ to-ndo tɕe, \\
 \textsc{lnk} p.n. \textsc{erg} the.aforementioned axe \textsc{pfv}:3\fl{}3'-whet \textsc{dem} \textsc{ifr}-take \textsc{lnk} \\
 \glt `Luban took the axe that he had whetted.' (150902 luban-zh, 90)
 \end{exe}

  \begin{exe}
  \ex 
  \begin{xlist}
\ex \label{ex:tWrpa.thWfseta}
 \gll   iɕqʰa tɯrpa tʰɯ-fse-t-a \\
just.now axe \textsc{pfv}-whet-\textsc{pst}:\textsc{tr}-\textsc{1sg} \\
\ex \label{ex:tWrpa.thWfseta2}
 \gll   tɯrpa  iɕqʰa tʰɯ-fse-t-a \\
 axe just.now \textsc{pfv}-whet-\textsc{pst}:\textsc{tr}-\textsc{1sg} \\
 \glt `I just whetted a/the axe.' (elicited)
 \end{xlist}
 \end{exe}

The use of \forme{iɕqʰa} as a topic marker with nouns (as in \ref{ex:iCqha.kAtWm} above) probably took place by reanalysis of the adverb in headless or postnominal relatives, or in complment clauses as above, then generalized to all noun phrases even those without subordinate clause.

 \subsubsection{Adversative topic} \label{sec:adversative.topic}
There are two adversative topic markers in Japhug \forme{ʁo} and \forme{ndɤre}. The former is similar in meaning to Mandarin \ch{倒}{dào}{instead, on the other hand}, and occurs in contexts with a strong adversative meaning `however, but, on the other hand' as in (\ref{ex:Ro.pWtu}).

\begin{exe}
\ex \label{ex:Ro.pWtu}
\gll jinde ku-nɯ-tu ɕi kɯma mɤ-xsi ma kɯɕɯŋgɯ ʁo pɯ-tu, \\
nowadays \textsc{dubit}-\textsc{auto}-exist \textsc{qu} \textsc{sfp} \textsc{neg}-\textsc{genr}:know \textsc{lnk} in.former.times \textsc{top}.\textsc{advers} \textsc{ipfv}.\textsc{pst}-exist \\
\glt `It is not clear whether it is still to be found nowadays, but it did exist in former times.' (23-scuz, 30)
\end{exe}

The marker \forme{ʁo} also occurs in two constructions meaning `of course'. First, it is found in the `X \forme{ʁo} X' construction meaning `of course (it is)  X', as in (\ref{ex:nAZo.Ro.nAZo}), the answer to the question in (\ref{ex:nABJu.YWCara.Ci}) which presents two alternatives.

\begin{exe}
\ex
\begin{xlist}
\ex  \label{ex:nABJu.YWCara.Ci}
\gll `a-tɤɕime, nɤ-βɟu ɲɯ-ɕar-a ɕi, aʑo tu-ozgrɯ-a' nɯra to-ti, `ma nɤ-pi ɣɯ aʑo tɤ-azgrɯ-a ɕti' to-ti  \\
\textsc{1sg}.\textsc{poss}-lady \textsc{2sg}.\textsc{poss}-mat \textsc{ipfv}-search-\textsc{1sg} \textsc{qu} \textsc{1sg} \textsc{ipfv}-bow-\textsc{1sg} \textsc{dem}:\textsc{pl} \textsc{ifr}-say \textsc{lnk} \textsc{1sg}.\textsc{poss}-elder.sibling \textsc{gen} \textsc{1sg} \textsc{pfv}-bow-\textsc{1sg} be.\textsc{affirm}:\textsc{fact} \textsc{ifr}-say \\
\glt `He said `My lady, should I look for a cushion for you, or should I bow (for you to sit on my back)', and he said  `Because I bow for your elder sister (to sit).'
\ex  \label{ex:nAZo.Ro.nAZo}
\gll  `nɤʑo ʁo nɤʑo ma, a-βɟu ɲɯ-tɯ-ɕar kɯ-ɤtsɯtsu me' to-ti.   \\
\textsc{2sg} \textsc{top}.\textsc{advers} \textsc{2sg} \textsc{lnk} \textsc{1sg}.\textsc{poss}-mat \textsc{ipfv}-2-search \textsc{inf}.\textsc{stat}-have.time  not.exist:\textsc{fact} \textsc{ifr}-say \\
\glt  `Of course (I will sit on) you, there is no time to look for a mat for me.' (2014-kWlAG, 195)
\end{xlist}
\end{exe}

Second, \forme{ʁo} is commonly used with the adverb \japhug{lɯski}{of course}, as in (\ref{ex:Ro.lWski}), not necessarily with any adversative meaning.

\begin{exe}
\ex \label{ex:Ro.lWski}
\gll  pɣɤɲaʁ kɤ-ti ci tu tɕe, nɯnɯ ʁo lɯski li nɯ pɣa ŋu \\
pheasant \textsc{nmlz}:P-say \textsc{indef} exist:\textsc{fact} \textsc{lnk} \textsc{dem} \textsc{top}.\textsc{advers} of.course again \textsc{dem} bird be:\textsc{fact} \\
\glt `There is a bird called \forme{pɣɤɲaʁ} (\textit{Pucrasia macrolopha}), this one, of course (since its name contains \japhug{pɣa}{bird}, § \ref{sec:subject.verb.compounds}, Table \ref{tab:subj.v.compounds}) is also a bird (like those previously discussed).' (23-pGAYaR, 2)
\end{exe}

The marker \forme{ndɤre} presents a milder adversative meaning `as far as X is concerned, unlike some other (people)' as in (\ref{ex:aZo.ndAre.rgaa}).  

\begin{exe}
\ex \label{ex:aZo.ndAre.rgaa}
\gll tsuku kɯ-rga tu, tsuku mɤ-kɯ-rga tu. aʑo ndɤre rga-a. \\
some \textsc{nmlz}:S/A-like exist:\textsc{fact} some \textsc{neg}-\textsc{nmlz}:S/A-like exist:\textsc{fact} \textsc{1sg} \textsc{top.advers} like:\textsc{fact}-\textsc{1sg} \\
\glt `Some like it, some don't; as far as I am concerned, I like it.' (07-tCGom2, 8)
\end{exe}

In (\ref{ex:nW.ndAre.wuma}), the use of \forme{ndɤre} suggests the meaning `as opposed to other possible missions'.

\begin{exe}
\ex \label{ex:nW.ndAre.wuma}
\gll a a-pa, nɯ ndɤre wuma ʑo ɴqa, sɤɣʑɯr. \\
\textsc{interj} \textsc{1sg}.\textsc{poss}-father \textsc{dem} \textsc{top.advers} really \textsc{emph} be.difficult:\textsc{fact} be.dangerous:\textsc{fact} \\
\glt `Ah father, this (mission on which you send me) is very difficult and dangerous indeed.' (28-smAnmi, 72)
\end{exe}

In (\ref{ex:jWGmWr.ndAre}), \forme{ndɤre} has a clear adversative meaning `this evening, on the other hand' (as opposed to the previous evenings).

\begin{exe}
\ex \label{ex:jWGmWr.ndAre}
\gll jɯfɕɯr tɯrmɯ tɕe nɤ-pi tɯlɤt nɯ ɯ-taʁ ko-ɴqoʁ-a ri mɯ-tɤ́-wɣ-tsɯm-a tɕe,
jɯɣmɯr ndɤre nɤʑo tu-kɯ-tsɯm-a ra ma tɕe kutɕu aʑo-sti ma maŋe-a tɕe, \\
 yesterday dusk \textsc{lnk} \textsc{2sg}.\textsc{poss}-elder.sibling  second.sibling \textsc{dem} \textsc{3sg}.\textsc{poss}-on \textsc{ifr}-hang-\textsc{1sg} \textsc{lnk} \textsc{neg}-\textsc{pfv:up}-\textsc{inv}-take.away-\textsc{1sg} \textsc{lnk} this.evening \textsc{top.advers}  \textsc{2sg} \textsc{ipfv}:\textsc{up}-2$\rightarrow$1-take.away-\textsc{1sg} have.to:\textsc{fact} \textsc{lnk} \textsc{lnk} here \textsc{1sg}-alone apart.from  not.exist:\textsc{sens}-\textsc{1sg} \textsc{lnk}  \\
\glt  `Yesterday at dusk I clung onto your second eldest sister but she did not take me away, this evening take me away, I am all alone here.' (07-deluge, 56-57)
\end{exe}

\subsubsection{The demonstrative \forme{nɯ} as a topic marker} \label{sec:nW.topic}
The postnominal determiner \forme{nɯ} and its reduplicated form \forme{nɯnɯ} is one of the most common words in Japhug, and has a considerable number of functions. It is used as a demonstrative (\ref{sec:demonstrative.determiners}), contributes to expressing definiteness (\ref{sec:definiteness}) and could be argued to be a subordinator (an analysis not adopted in the present work, see § XXX).

In addition, it is commonly used to mark topic: left-dislocated noun phrases generally (though not compulsorily) take this determiner. For instance, in texts presenting animals or plants, their name on first occurrence is left dislocated and followed by the determiner \forme{nɯ}, as in (\ref{ex:qawWz.nW}).

\begin{exe}
\ex \label{ex:qawWz.nW}
\gll  qawɯz nɯ, (qawɯz nɯ pɯ-tɯ-mto-t, ɣe?)  qawɯz nɯnɯ, nɤki, kɯɕɯŋgɯ tɕe, \\
Edelweiss \textsc{dem} Edelweiss \textsc{dem} \textsc{pfv}-2-see-\textsc{pst}:\textsc{tr} \textsc{sfp} Edelweiss \textsc{dem} \textsc{filler} before \textsc{lnk} \\
\glt `The edelweiss, (you saw Edelweiss before, right?)... The edelweiss, in former times,' (15-babW, 177)
\end{exe}

In its function as a topicalizer, the determiner \forme{nɯ} can follow a noun with postnominal demonstratives, as in (\ref{ex:kWki.nW}). However, due to the difficulty of systematically sorting out the topicalization and demonstrative functions of this marker, I do not attempt to reflect this distinction in the glosses, and use  \textsc{dem} everywhere.

\begin{exe}
\ex \label{ex:kWki.nW}
\gll tɕeri kɯki mɯntoʁ kɯki nɯ pɯpɯŋunɤ, wuma ʑo kɯ-ʑru, kɯ-pe, \\
\textsc{lnk} \textsc{dem}.\textsc{prox} flower \textsc{dem}.\textsc{prox} \textsc{dem} as.far really \textsc{emph} \textsc{nmlz}:S/A-be.strong \textsc{nmlz}:S/A-be.good \\ 
\glt `But concerning this flower, so precious and nice' (150820 meili de meiguihua, 58)
\end{exe}

\subsubsection{The linker \forme{tɕe} as a topic marker} \label{sec:tCe.topic}
The word \forme{tɕe}, which originates from a locative postposition (§\ref{sec:locative.j}), is mainly used in Japhug as a linker (§ XXX), one of the most common words in the corpus.

In addition, it can serve as a topic marker, following left-dislocated noun or postpositional phrases (\ref{ex:tsuku.kW.tCe}).

\begin{exe}
\ex \label{ex:tsuku.kW.tCe}
\gll tsuku kɯ tɕe lɤpɯɣ ra mbɯsɯt chɯ-lɤt-nɯ tɕe nɯra ɲɯ-rku-nɯ ɲɯ-ŋu \\
some \textsc{erg} \textsc{lnk} radish \textsc{pl} grating \textsc{ipfv}-throw-\textsc{pl} \textsc{lnk} \textsc{dem}.\textsc{pl} \textsc{ipfv}-put.in-\textsc{pl} \textsc{sens}-be \\
\glt `Some people, they grate radish and use it as filling (for the sausage).' (05-paR, 77)
\end{exe}

 \subsection{Focus markers} \label{sec:focus}
   \subsubsection{Unexpected focus} \label{sec:unexpected}
  The unexpected/high degree marker \forme{rcanɯ} or \forme{rca}, which was grammaticalized from the  secutive relator noun \japhug{ɯ-rca}{following} (§ \ref{sec:secutive}). It indicates that the phrase or clause preceding it is topical, and the situation or action described by the predicate that follows is unexpected (\ref{ex:nAZo.rcanW}), intensifies to a noticeable (and not foreseeable) extent (\ref{ex:tokAnWmqajndZic.tCe.rcanW}) or occurs with a remarkably high degree or intensity, with  (\ref{ex:mbro.rcanW}) or without (\ref{ex:apWme.rcanW}) surprise.

\begin{exe}
\ex \label{ex:nAZo.rcanW}
 \gll  wo nɤʑo rcanɯ tɕʰi ɲɯ-tɯ-nɤme ŋu ma,  aʑo tɯ-mɯ kɯ pɯ-kɯ-sɯ-χtɕi-a, tɤndʐo nɯ! \\
 \textsc{interj} \textsc{2sg} \textsc{foc}:\textsc{unexp} what \textsc{sens}-2-do[III] be:\textsc{fact} \textsc{lnk} \textsc{1sg} \textsc{indef}.\textsc{poss}-sky \textsc{erg} \textsc{pfv}-2\fl{}1-\textsc{caus}-wash-\textsc{1sg} cold \textsc{sfp} \\
\glt `You, what are you doing, you caused me to be drenched by the rain.' (kWlAG2014, 157) \\
\end{exe}

\begin{exe}
\ex \label{ex:tokAnWmqajndZic.tCe.rcanW}
 \gll to-k-ɤnɯmqaj-ndʑi-ci tɕe rcanɯ, ʑɯrɯʑɤri tɕe ko-k-ɤndɯndo-ndʑi-ci, \\
 \textsc{ifr}-\textsc{evd}-\textsc{recip}:scold-\textsc{du-evd} \textsc{lnk}  \textsc{foc}:\textsc{unexp} progressively \textsc{lnk}   \textsc{ifr}-\textsc{evd}-\textsc{recip}:take-\textsc{du-evd} \\
 \glt `They scolded each other and progressively started to fight, ' (lWlu2002, 52)
\end{exe}

 \begin{exe}
\ex \label{ex:mbro.rcanW}
 \gll mbro rcanɯ ɯ-xɕɤt kɯ-tɯ\redp{}tu ʑo nɯ-ntsʰɤr ɲɯ-nu, \\
 horse \textsc{foc}:\textsc{unexp} \textsc{3sg}.\textsc{poss}-strength \textsc{nmlz}:S/A-\textsc{emph}\redp{}exist \textsc{emph} \textsc{pfv}-neigh \textsc{sens}-be \\ 
 \glt `The horse neighed with all his strength.' (qachGa2003, 158)
\end{exe}

The marker \forme{rcanɯ} is particularly common in the degree construction with a \forme{tɯ-} degree nominal (§ XXX), as in (\ref{ex:apWme.rcanW}). In this particular construction,  \forme{rcanɯ} does not necessarily express unexpectedness.

\begin{exe}
\ex \label{ex:apWme.rcanW}
 \gll  tɕe nɯnɯ lɯlu a-pɯ-me rcanɯ, βʑɯ ɯ-tɯ-ŋɤn saχaʁ. \\
 \textsc{lnk} \textsc{dem} cat \textsc{irr}-\textsc{ipfv}-not.exist \textsc{foc}:\textsc{unexp} mouse
 \textsc{3sg}.\textsc{poss}-\textsc{nmlz}:\textsc{degree}-be.evil be.extremely:\textsc{fact} \\ 
 \glt `If there are no cats, the mice are extremely fierce (cause a lot of damages).' (21-lWlu, 32) 
\end{exe}

 \subsubsection{Additive and scalar focus marker \forme{kɯnɤ} } \label{sec:kWnA}
The additive and scalar focus marker \japhug{kɯnɤ}{also, even} follows the constituent over which it has scope, which can be noun phrases, postpositional phrases but also subordinate clauses (these are treated in § XXX). As with other function words with the syllable \forme{nɤ} as last element (§ XXX), the stress is on the first syllable (\forme{kɯ́nɤ}) and the vowel on the second syllable is often elited (a pronunciation \forme{kɯn} is often heard). 

The marker \forme{kɯnɤ} expresses both additive focus, as in (\ref{ex:aZo.kWNA.staRlupa}), and scalar focus, as in (\ref{ex:WNgWz.kWnA.tunAndWtnW}) in affirmative sentences. It is also compatible with negative verb forms, as in (\ref{ex:tWrdoR.kWnA}), expressing the meaning `not even' (see also \japhug{cinɤ}{(not) even one} in § \ref{sec:cinA}).

\begin{exe}
\ex \label{ex:aZo.kWNA.staRlupa}
\gll aʑo kɯnɤ staʁlupa ŋu-a tɕe \\
\textsc{1sg} also born.in.the.tiger.year be:\textsc{fact}-\textsc{1sg} \textsc{lnk} \\
\glt `Me too (like you), I am of the Tiger year.' (2011-05-nyima, 168)
\end{exe}

\begin{exe}
\ex \label{ex:WNgWz.kWnA.tunAndWtnW}
\gll ʑara ʑo ɯ-ŋgɯz kɯnɤ tu-nɤndɯt-nɯ tɕe nɯ kɯ-βʁa ɣɤʑu, kɯ-nŋo ɣɤʑu qʰe, \\
\textsc{3pl} \textsc{emph} \textsc{3sg}.\textsc{poss}-among:\textsc{loc} also \textsc{ipfv}-fight-\textsc{pl} \textsc{lnk} \textsc{dem} \textsc{nmlz}:S/A-win \textsc{sens}:exist \textsc{nmlz}:S/A-lose  \textsc{sens}:exist \textsc{lnk} \\
\glt `Even among themselves, they fight, and there are winners and losers.' (20-sWNgi, 62-63)
\end{exe}
 
\begin{exe}
\ex \label{ex:tWrdoR.kWnA}
\gll tɯ-sŋi mɯntoʁ tɯ-rdoʁ kɯnɤ ci ci tɕe mɯ́j-stʰɯt \\
one-day flower one-piece also once once \textsc{lnk} \textsc{neg}:\textsc{sens}-finish \\
\glt `Sometimes one cannot finish even one pattern (on the belt) in one day.' (2011-06-thaXtsa, 47)
\end{exe}

As an additive focus marker, \forme{kɯnɤ} can be repeated on all the nouns designating the members of a group sharing a particular property, in the construction $X$ \forme{kɯnɤ}, $Y$ \forme{kɯnɤ}  `both $X$ and $Y$', as in (\ref{ex:Dpalcan.kWnA}).

\begin{exe}
\ex \label{ex:Dpalcan.kWnA}
 \gll a-pɯ-ŋu tɕe, aʑo kɯnɤ taʁrdo rɟitpa a-pɯ-ŋu-a, χpɤltɕin kɯnɤ taʁrdo rɟitpa a-pɯ-ŋu, ... nɯ tɕi-rɟit nɯni tɕe taʁrdo rɟitpa ma nɯ ma kɯmaʁ rɟitpa nɯ kɤ-rtsi me.  \\
 \textsc{irr}-\textsc{ipfv}-be \textsc{lnk} \textsc{1sg} also pl.n. lineage  \textsc{irr}-\textsc{ipfv}-be-\textsc{1sg}  p.n. also pl.n. lineage  \textsc{irr}-\textsc{ipfv}-be { } \textsc{dem} \textsc{1du}.\textsc{poss}-offspring \textsc{dem}:\textsc{du} \textsc{lnk} pl.n. lineage \textsc{lnk} \textsc{dem} apart.from other lineage \textsc{dem} \textsc{nmlz}:O-count not.exist:\textsc{fact} \\
 \glt `For instance suppose that both Dpalcan and I were from Taqrdo lineage, then our two children would only count as members of the Taqrdo lineage and no other lineage.' (140426 rJitpa, 13-15)
\end{exe}

The scope of  \forme{kɯnɤ} is generally exclusively on the constituent that it immediately follows, but there are cases where the scope is more extensive. In (\ref{ex:aZo.kWnA.akAsWso}), \forme{kɯnɤ} occurs between the pronoun \forme{aʑo} and the following participial verb form, which bears a \textsc{1sg} possessive prefix \forme{a-} coreferent with that pronoun (see also \ref{ex:aZWG.kWnA} below). The semantic scope of \forme{kɯnɤ} here is on the whole relative \forme{aʑo a-kɤ-sɯso} `(the things) that I want' rather than exclusively on the pronoun \forme{aʑo}.

\begin{exe}
\ex \label{ex:aZo.kWnA.akAsWso}
 \gll aʑo kɯnɤ a-kɤ-sɯso nɯ tɤ-stu-nɯ ra \\
 \textsc{1sg} also \textsc{1sg}.\textsc{poss}-\textsc{nmlz}:O-think \textsc{dem} \textsc{imp}-do.like-\textsc{pl} have.to:\textsc{fact} \\
 \glt `(I will do as you say, but) do also the things I want.' (2003kAndzwsqhaj2, 47)
\end{exe}

The focus marker \forme{kɯnɤ} is found with nouns or pronouns in core argument function, including S (\ref{ex:kWnA.nArca}), O (\ref{ex:nWXpWm.kWnA}), and semi-objects (\ref{ex:kWnA.mAsna}).  Examples with transitive subjects are presented below (\ref{ex:nWra.kWnA} and \ref{ex:Wzda.ra.kWnA}).

 \begin{exe}
\ex \label{ex:kWnA.nArca}
\gll aʑo kɯnɤ nɤ-rca ɣi-a ɕti  \\
\textsc{1sg} also \textsc{2sg}.\textsc{poss}-following come:\textsc{fact}-\textsc{1sg} be.\textsc{affirm}:\textsc{fact} \\
\glt `I am coming with you too.' (2011-05-nyima, 171)
 \end{exe}
 
   \begin{exe}
\ex \label{ex:nWXpWm.kWnA}
\gll    ma nɯ-χpɯm kɯnɤ kʰro mɤ-kɯ-fkaβ kɯ-fse ku-rɤʑi-nɯ  \\
lnk 3pl.poss-knee also much \textsc{neg}-\textsc{nmlz}:S/A-cover \textsc{nmlz}:S/A-be.like \textsc{ipfv}-stay-\textsc{pl} \\
\glt `(Gents) would (wear trousers that did) not cover much even their knees.'  (30-rkAsnom, 5) 
  \end{exe}
  
  \begin{exe}
 \ex \label{ex:kWnA.mAsna}
 \gll   ɯ-ru nɯra laʁdɯn ɯ-jɯ kɯnɤ mɤ-sna, ma mɤ-ngɯt. \\
 \textsc{3sg}.\textsc{poss}-trunk \textsc{dem}:\textsc{pl} tool \textsc{3sg}.\textsc{poss}-handle also \textsc{neg}-be.worth \textsc{lnk}  \textsc{neg}-be.strong:\textsc{fact} \\
 \glt `(The wood from) its trunk is not even good (enough to be used to make) tool handles, as it is not strong.'  (17-xCAj, 79)
  \end{exe}

It also occurs with all types of oblique arguments and adjuncts, including genitive (\ref{ex:aZWG.kWnA}), dative (\forme{ɯ-ɕki} \ref{ex:nWCki.kWnA}),  locational adjuncts in \forme{tɕu} (\ref{ex:kutCu.kWnA}) or \forme{ri} (\ref{ex:ri.kWnA}), temporal adjuncts (\ref{ex:ftCAXcAl.kWnA}) or adjuncts expressing manner or cause (\ref{ex:nWtCu.kWnA}).  
  
   \begin{exe}
\ex \label{ex:aZWG.kWnA}
\gll aʑɯɣ kɯnɤ a-mpʰrɯmɯ a-pɯ-tɯ-sɯ-re ɯ-tɯ́-cʰa \\
\textsc{1sg}:\textsc{gen} also \textsc{1sg}.\textsc{poss}-divination \textsc{irr}-\textsc{pfv}-2-\textsc{caus}-look[III] \textsc{qu}-2-can:\textsc{fact} \\
\glt `Can you ask (the monk) to make a divination for me too?' (The divination, 31)
\end{exe}  
  
   \begin{exe}
\ex \label{ex:nWCki.kWnA}
\gll  tɯ-pi ɣɯ ɯ-nmaʁ ra nɯ-ɕki kɯnɤ `a-pi' tu-kɯ-ti ɕti ma nɯ ma kupa kɯ-fse ʑaka ɯ-rmi me. \\
\textsc{genr}.\textsc{poss}-elder.sibling \textsc{gen} \textsc{3sg}.\textsc{poss}-husband \textsc{pl} \textsc{3pl}.\textsc{poss}-\textsc{dat} also \textsc{1sg}.\textsc{poss}-elder.sibling \textsc{ipfv}-\textsc{genr}-say be.\textsc{affirm}:\textsc{fact} \textsc{lnk} \textsc{dem} apart.from Chinese \textsc{nmlz}:S/A-be.like each \textsc{3sg}.\textsc{poss}-name not.exist:\textsc{fact} \\
\glt  `One calls one's sister's husband (and others from his family) `my elder brother', there are no other special terms as in Chinese.' (140425 kWmdza05)
\end{exe}


  \begin{exe}
\ex \label{ex:kutCu.kWnA}
\gll  kutɕu kɯnɤ nɯ ɲɯ-fse, jɯfɕɯndʐi ra kɯ-xtɕɯ\redp{}xtɕi tɤ-ɣɤndʐo kɯ-fse ri, ɕɤxɕo tɕe kɯ-xtɕɯ\redp{}xtɕi ɲɯ-ʑi kɯ-fse \\
here also \textsc{dem} \textsc{sens}-be.like a.few.days.ago \textsc{nmlz}:S/A-\textsc{emph}\redp{}be.small \textsc{pfv}-be.cold \textsc{nmlz}:S/A-be.like \textsc{lnk} the.last.days \textsc{lnk} \textsc{nmlz}:S/A-\textsc{emph}\redp{}be.small \textsc{sens}-subside \textsc{nmlz}:S/A-be.like \\
\glt `It is like that here too, a few days ago the weather became a little cold, but the last days it has eased a bit.' (conversation, 141027)
  \end{exe}
  
    \begin{exe}
\ex \label{ex:ri.kWnA}
\gll   maldzɯ nɯ, nɯ ɯ-tʰɤcu tsa ri kɯnɤ ɣɤʑu. qarɣɤpɤt ɯ-rca ri kɯnɤ tu-ɬoʁ ɲɯ-ŋu. \\
plant.name \textsc{dem} \textsc{dem} \textsc{3sg}.\textsc{poss}-downstream a.little \textsc{loc} also exist:\textsc{sens} plant.name \textsc{3sg}.\textsc{poss}-among \textsc{loc} also \textsc{ipfv}-come.out \textsc{sens}-be \\
\glt `The \forme{maldzɯ} plant, it is also found in places of slightly lower altitude, but grows also in the same places as  \forme{qarɣɤpɤt} plants.' (18-qromJoR, 81-82)
    \end{exe}
    
\begin{exe}
\ex \label{ex:ftCAXcAl.kWnA}
\gll   kukutɕu ftɕɤχcɤl kɯnɤ <baonuanyi> tu-tɯ-ŋge pɯ-ɕti. \\
  here mid.summer also warm.clothes \textsc{ipfv}-2-wear[III] \textsc{pst}.\textsc{ipfv}-be.\textsc{affirm} \\
  \glt `Here you were wearing warm clothes even in mid summer.' (conversation, 141017)
    \end{exe}
    
    \begin{exe}
\ex \label{ex:nWtCu.kWnA}
\gll    tɕe nɯtɕu kɯnɤ ɯ-jaʁ ɯ-ntsi tɤɲi pjɯ-sɤtse, ɯ-jaʁ ɯ-ntsi kɯ tsʰitsuku ɲɯ-z-nɤme qʰe, \\
\textsc{lnk} \textsc{dem}:\textsc{loc} also \textsc{3sg}.\textsc{poss}-hand \textsc{3sg}.\textsc{poss}-one.of.a.pair erg various.things \textsc{ipfv}-\textsc{caus}-do[III] \textsc{lnk}  \\
\glt `Even like that (despite the pain in her legs), she props herself with a cane using one hand, and does all kinds of things with her other hand.' (14-tApitaRi, 52)
\end{exe}

Although \japhug{kɯnɤ}{also, even} can be combined with most postpositions and relator nouns as shown by the examples above, it is however incompatible with the ergative \forme{kɯ}. For instance, in  (\ref{ex:nWra.kWnA}), although the demonstrative pronoun \forme{nɯra} `they, those' in the second clause is the subject of the transitive verb \japhug{ndza}{eat}, it does not take the ergative \forme{kɯ} as would be expected (§ \ref{sec:A.kW}). The same applies to \forme{ɯ-zda ra} `his companions', subject of the transitive verb \forme{na-nɯ-ɕar-nɯ} `they looked for themselves' in (\ref{ex:Wzda.ra.kWnA}), 

  \begin{exe}
\ex \label{ex:nWra.kWnA}
\gll ɯ-pɯ nɯra li ju-ɣi-nɯ qʰe, nɯra kɯnɤ ɣɯ-tu-ndza-nɯ. \\
\textsc{3sg}.\textsc{poss}-young \textsc{dem}:\textsc{pl} again \textsc{ipfv}-come-\textsc{pl} \textsc{lnk} \textsc{dem}:\textsc{pl} also \textsc{cisloc}-\textsc{ipfv}-eat-\textsc{pl} \\
\glt `Its youngs also come and they too eat it.' (20-sWNgi, 59-60)
  \end{exe}
  
    \begin{exe}
\ex \label{ex:Wzda.ra.kWnA}
\gll   ɯ-zda ra kɯnɤ nɯ-rʑaβ tɯka na-nɯ-ɕar-nɯ ɲɯ-ŋu \\
\textsc{3sg}.\textsc{poss}-companion \textsc{pl} also \textsc{3sg}.\textsc{poss}-wife each \textsc{pfv}:3\fl{}3'-\textsc{auto}-search \textsc{sens}-be \\
\glt `His companions also took each a wife for himself (among the women of the island).' (2005Norbzang, 44)
    \end{exe}
    
The combinations $\dagger$\forme{kɯ kɯnɤ} or $\dagger$\forme{kɯnɤ kɯ} are unattested, and not accepted by native speakers. The contrast between absolutive and ergative noun phrases is therefore neutralized in additive or scalar focus with \forme{kɯnɤ}. Note that other focus markers, such as \forme{ri} and \forme{tɕi} (see \ref{ex:tCi.ndze} in § \ref{sec:ri.additive}) differ from \forme{kɯnɤ} in this regard.

Four distinct facts converge to suggest that the first syllable of \forme{kɯnɤ} is historically related to the ergative postposition \forme{kɯ}: (i) the incompatibility of co-occurrence of \forme{kɯnɤ} and \forme{kɯ}; (ii) the stress on the first syllable in \forme{kɯ́nɤ}; (iii) the similar \forme{-nɤ} element in the other scalar focus marker \japhug{cinɤ}{(not) even one} (§ \ref{sec:cinA}) (iv) the existence of the linker \forme{nɤ}, possibly of Tibetan origin (§ XXX). A detailed examination of this topic is however impossible on the basis Japhug-internal evidence, and will require extensive syntactic comparison between Gyalrong languages.

 \subsubsection{Correlative additive focus markers \forme{ri} and \forme{tɕi}} \label{sec:ri.additive} 
 The additive focus markers \forme{ri} and \forme{tɕi}  are used in enumerations, repeated after each noun referring to  members of a group, to focus on the fact that their referents share a common property (or properties that are semantically close enough), as in (\ref{ex:ri.kWsthWci.WWmpCar}) and (\ref{ex:tCi.tulhoR.cha}) (see additional examples in \citealt[313-314]{jacques14linking}).
 
 \begin{exe}
\ex \label{ex:ri.kWsthWci.WWmpCar}
 \gll  a-rʑaβ ri kɯstʰɯci ɲɯ-mpɕɤr, a-mbro ri kɯstʰɯci ɲɯ-ʑru, a-pɣɤtɕɯ ri kɯstʰɯci ɲɯ-mpɕɤr tɕe, \\
 \textsc{1sg}.\textsc{poss}-wife also so.much \textsc{sens}-be.beautiful  \textsc{1sg}.\textsc{poss}-horse also so.much \textsc{sens}-be.strong  \textsc{1sg}.\textsc{poss}-bird also so.much \textsc{sens}-be.beautiful \textsc{lnk} \\
 \glt `My wife is so beautiful, my horse so strong, my bird so beautiful.' (2003qachga, 116)
 \end{exe}
 
  \begin{exe}
\ex \label{ex:tCi.tulhoR.cha}
 \gll  ɴqiaβ tɕi tu-ɬoʁ cʰa, zrɯ tɕi tu-ɬoʁ cʰa, \\
 dark.side.of.the.mountain also \textsc{ipfv}-come.out can:\textsc{fact}   sunny.side.of.the.mountain also \textsc{ipfv}-come.out can:\textsc{fact}  \\
 \glt `It can grow in both the dark and the sunny sides of the mountains.' (17-thowum, 14)
  \end{exe}
  
The correlative focus markers \forme{ri} and \forme{tɕi} can occur after any noun phrase or postpositional phrase, including with the ergative  \forme{kɯ} as shown by (\ref{ex:tCi.ndze}), unlike the marker \japhug{kɯnɤ}{even, also} (see examples \ref{ex:nWra.kWnA} and \ref{ex:Wzda.ra.kWnA}, § \ref{sec:kWnA}).
  
  \begin{exe}
\ex \label{ex:tCi.ndze}
 \gll paʁ kɯ tɕi ndze, nɯŋa kɯ tɕi ndze, jla kɯ tɕi ndze.   \\
 pig \textsc{erg} also eat[III]:\textsc{fact}  cow \textsc{erg} also eat[III]:\textsc{fact}  hybrid.yak \textsc{erg} also eat[III]:\textsc{fact}  \\
 \glt `Pigs eat it, cows eat it, hybrid yaks eat it.' (18-NGolo, 171)
  \end{exe}

The focus markers \forme{ri} and \forme{tɕi} can have scope on only part of the noun/propositional phrase, and even on the relator nouns as in (\ref{ex:WNgW.tCi}).

   \begin{exe}
\ex \label{ex:WNgW.tCi}
 \gll   sɤtɕʰa ɯ-ŋgɯ tɕi ɣɤʑu, sɤtɕʰa ɯ-taʁ tɕi ʑo ɣɤʑu \\
 ground \textsc{3sg}.\textsc{poss}-inside also exist:\textsc{sens}  ground \textsc{3sg}.\textsc{poss}-inside also \textsc{emph} exist:\textsc{sens} \\
 \glt `It is found both inside the ground, and on the ground.' (25-GdAso, 17)
    \end{exe}
    
Alternatively, it is possible to enumerate distinct related properties of the same referent using \forme{ri} (this usage is not found with \forme{tɕi}), but that marker still follows the noun phrase (correlative \forme{ri} can follow verbs, but only in a specific construction, see \ref{ex:ri.kWmWm.ri} below). In this case the referent cannot be elided, and must be repeated in both clauses, at least as a third person pronoun \forme{ɯʑo} as in (\ref{ex:WlWz.ri.pjAxtCi}). 

  \begin{exe}
\ex \label{ex:WlWz.ri.pjArZi}
 \gll pʰaʁrgot nɯnɯ ɯʑo ri pjɤ-rʑi, ɯʑo ri pjɤ-tsʰu tɕe \\
 boar \textsc{dem} \textsc{3sg} also \textsc{ifr}.\textsc{ipfv}-be.heavy \textsc{3sg} also \textsc{ifr}.\textsc{ipfv}-be.fat \textsc{lnk} \\ 
\glt  `The boar, it was heavy and fat.' (140428 yonggan de xiaocaifeng-zh, 244)
 \end{exe}

A variant of this construction is found with internally-headed relative clauses in apposition, taking the third person pronoun \forme{ɯʑo} as head, as in (\ref{ex:WZo.ri.kWwxti}).

\begin{exe}
\ex \label{ex:WZo.ri.kWwxti}
\gll  [ɯʑo ri kɯ-wxti], [ɯʑo ri kɯ-sɤjlɯ\redp{}jloʁ] ci pjɤ-ŋu. \\
\textsc{3sg} also \textsc{nmlz}:S/A-be.big \textsc{3sg} also \textsc{nmlz}:S/A-\textsc{emph}\redp{}be.big \textsc{indef} \textsc{ifr}.\textsc{ipfv}-be \\
\glt `(The toad) was a big and disgusting (creature).' (150818 muzhi guniang, 86)
\end{exe}

 
The correlative construction can involve the possessor of an IPN, as in (\ref{ex:WlWz.ri.pjAxtCi}), where in the first clause the referent `the girl' is possessor of the intransitive subject (literally `her age was small', § XXX) and in second it corresponds to the intransitive subject, realized as a third person pronoun \forme{ɯʑo} `she'.

  \begin{exe}
\ex \label{ex:WlWz.ri.pjAxtCi}
 \gll tɕʰeme nɯ ɯ-lɯz ri pjɤ-xtɕi, ɯʑo ri pjɤ-mpɕɤr,  \\
 girl \textsc{dem} \textsc{3sg}.\textsc{poss}-age also \textsc{ifr}.\textsc{ipfv}-be.small \textsc{3sg} also \textsc{ifr}.\textsc{ipfv}-be.beautiful \\
\glt `The girl was young and beautiful.' (150909 hua pi-zh, 10)
 \end{exe}
 
 More complex correlations, involving different subjects and predicates related to another referent, are also possible as shown by example (\ref{ex:lWlu.kW}), where \forme{ri} occurs after the intransitive subject \japhug{tɯ-ci}{water}, after the transitive subject \japhug{lɯlu}{cat} with the ergative and after the finite verb \japhug{tu-ɕe}{it goes up} (on which see below and refer to § XXX).
 
 \begin{exe}
\ex   \label{ex:lWlu.kW}
\gll <yancong> ku-kɯ-rɤloʁ tɕe ɯ-taʁ tɯ-ci ri mɯ́j-ɣi lɯlu kɯ ri mɯ-ɲɯ́-wɣ-ɕaβ qapri tu-ɕe ri mɯ́j-cʰa tɕe \\
 chimney \textsc{ipfv}-\textsc{genr}:S/P-make.a.nest \textsc{lnk} \textsc{3sg}.\textsc{poss}-on \textsc{indef}.\textsc{poss}-water also \textsc{neg}:\textsc{sens}-come cat \textsc{erg} also \textsc{neg}-\textsc{ipfv}-\textsc{inv}-catch snake \textsc{ipfv}:\textsc{up}-go also \textsc{neg}:\textsc{sens}-can \textsc{lnk} \\
 \glt `(The sparrows) make their nest in the chimney, (because) water cannot come up there, the cats cannot catch them, and the snakes cannot go up there.' (22-kumpGatCW, 69)
 \end{exe}
 
 The marker \forme{ri} is homophonous with the locative \forme{ri} (§ \ref{sec:locative}), and in cases with an enumeration of locative adjuncts, there can be ambiguity between the two. In (\ref{ex:Xcha.ri.ci}), \forme{ri} is analyzed as a locative because of the position of the determiner \forme{ci}, and also because it can be replaced with other locative postpositions.
 
 \begin{exe}
\ex \label{ex:Xcha.ri.ci}
\gll   χcʰa ri ci, ɯ-ʁe ri ci ɯ-jme cʰɯ-ɬoʁ ɲɯ-ŋu. \\
right \textsc{loc} one  \textsc{3sg}.\textsc{poss}-left \textsc{loc} one \textsc{3sg}.\textsc{poss}-tail \textsc{ipfv}:\textsc{downstream}-come.out \textsc{sens}-be \\
\glt `It has one tail on the right, and one on the left.' (26-qro, 116)
\end{exe}

The marker \forme{ri} can follow verbs only if combined with an existential verb, a copula or a modal auxiliary verb as main predicate (meaning `both $X$ and $Y$' with positive copulas, and `neither $X$ nor $Y$' with negative ones). In this type of construction, verbs are mostly in non-finite form, as in (\ref{ex:ri.kWmWm.ri}). Examples with finite verbs however do exist; this topic is treated in § XXX. %ɲɯ-ɣɤwu ri kɯ-maʁ, ɲɯ-nɤre ri kɯ-maʁ kɯ-fse ɲɤ-k-ɤβzu-ci  ; tu-rɯɕmi ri mɤ-kɯ-khɯ, chɯ-nɯrɤɣo ri mɤ-kɯ-khɯ ci ɲɤ-k-ɤβzu-ci. ; tu-ndzur ri pjɤ-maʁ, ku-omdzɯ ri pjɤ-maʁ.

 \begin{exe}
\ex \label{ex:ri.kWmWm.ri}
 \gll   nɯ pɯ́-wɣ-ta ri  kɯroz kɯ-mɯm ri maŋe, kɯroz mɤ-kɯ-ɣɤ-mɲɤt ri maŋe qʰe, \\
 \textsc{dem} \textsc{pfv}-\textsc{inv}-put \textsc{lnk} specially \textsc{nmlz}:S/A-be.tasty also not.exist:\textsc{sens} specially \textsc{neg}-\textsc{nmlz}:S/A-\textsc{facil}-be.spoiled also not.exist:\textsc{sens} \textsc{lnk} \\
 \glt `When if one puts (a seal on the bread), there is nothing especially tasty about it, and nothing special concerning the preservation (of the bread).' (160706 thotsi, 27)
  \end{exe}
  

  
 \subsubsection{Scalar focus marker \forme{cinɤ}} \label{sec:cinA} 
 The focus marker \japhug{cinɤ}{(not) even one} exclusively occurs with a negative verb. Like \japhug{kɯnɤ}{also, even}, this marker has stress on the first syllable \forme{cínɤ}, which is obviously related to the numeral \japhug{ci}{one} (§ \ref{sec:one.to.ten}, § \ref{sec:indef.article}).
 
 The marker \forme{cinɤ} has scope over the constituent that immediately precedes it, generally a noun phrase including or consisting of a CN, as in (\ref{ex:tWrdoR.cinA3}), but also object and subject participial relative clauses as in (\ref{ex:zrWG.kAmto.cinA}), (\ref{ex:WrNa.WkWru.cinA}) and (\ref{ex:lukWpGaR.nW.cinA}).
 
 \begin{exe}
\ex \label{ex:tWrdoR.cinA3}
\gll tsuku kɯ qʰe tɯ-rdoʁ cinɤ mɤ-kɯ-mto tu. \\
some erg lnk one-piece even neg-nmlz:S/A-see exist:fact \\
\glt `There are some people who (cannot) even find a single one.' (20-grWBgrWB, 36)
 \end{exe} 

 \begin{exe}
\ex \label{ex:zrWG.kAmto.cinA}
\gll  ma tɕe jinde nɯ zrɯɣ kɤ-mto cinɤ maŋe. \\
\textsc{lnk} \textsc{lnk} nowadays \textsc{dem} louse \textsc{nmlz:P}-see even not.exist:\textsc{sens} \\
\glt `Nowadays there isn't even a single louse to be seen/one cannot even see a single louse.' (21-mdzadi, 77)
\end{exe} 

\begin{exe}
\ex \label{ex:WrNa.WkWru.cinA}
\gll ɯ-rŋa ɯ-kɯ-ru cinɤ ʑo pjɤ-me \\
3sg.poss-face 3sg.poss-nmlz:S/A-look even \textsc{emph} \textsc{ipfv}.\textsc{ifr}-not.exist \\
\glt `Not even one (of the thieves) looked at it/The (thieves) did not even so much as looked at it.' (140426 luozi he qiangdao)
\end{exe}

\begin{exe}
\ex \label{ex:lukWpGaR.nW.cinA}
\gll tɕe ɯ-ɲɯ-kɯ-ɣɤ-rkɯn nɯ ɲɯ-dɤn ma lu-kɯ-pɣaʁ nɯ tɯ-rdoʁ cinɤ ʑo maŋe \\
\textsc{lnk} \textsc{3sg}.\textsc{poss}-\textsc{ipfv}-\textsc{nmlz}:S/A-\textsc{caus}-be.few \textsc{dem} \textsc{sens}-be.many \textsc{lnk} \textsc{ipfv}:\textsc{upstream}-\textsc{nmlz}:S/A-plough \textsc{dem} one-piece even \textsc{emph} not.exist:\textsc{sens} \\
\glt `A lot of people diminish their fields, and not a single of them opens new fields.' (150903 friche, 6)
\end{exe}

In the case of relative clauses before \forme{cinɤ}, there is some ambiguity as to whether the scope of the focus marker is on the head of the relative or on the main verb of the relative clause, hence the two proposed translations above for (\ref{ex:zrWG.kAmto.cinA}) and (\ref{ex:WrNa.WkWru.cinA}).

It is not possible to use \forme{cinɤ} with scope over transitive subjects, followed by the ergative.

The form \forme{cinɤ} also occurs in the expression \forme{ŋu cinɤ maʁ kɯ} `in any case it is not', as in (\ref{ex:Nu.cinA.maR.kW}), literally `It is not even the case that...' ; in this construction, only the first verb \japhug{ŋu}{be} receives person indexation, as shown by (\ref{ex:Nua.cinA.maR.kW}). In addition to \japhug{ŋu}{be}, a few other verbs such as \japhug{fse}{be like} can occur with \forme{ci nɤ maʁ kɯ} `anyway X does not' .

 \begin{exe}
\ex \label{ex:Nu.cinA.maR.kW}
\gll qajdo kɯ tɕʰi mɤ-nɯ-ti ɕti nɤ, a-tɤ-nɯ-ti ma ŋu cinɤ maʁ kɯ, nɯ sɤznɤ kɯ-scɯ-scit rɤʑi-tɕi \\
crow \textsc{erg} what \textsc{neg}-\textsc{auto}-say:\textsc{fact} be.\textsc{affirm}:\textsc{fact} \textsc{lnk} \textsc{irr}-\textsc{pfv}-\textsc{auto}-say \textsc{lnk} be:\textsc{fact} even not.be:\textsc{fact} \textsc{sfp} \textsc{dem} \textsc{comp} \textsc{nmlz}:S/A-\textsc{emph}\redp{}happy stay:\textsc{fact}-\textsc{1du} \\
\glt `What would not a crow say (a crow tells only lies), let it say as it wants, in any case it is not (true), let us rather live (together) happily.' (28-qAjdoskAt, 28)
\end{exe} 

 \begin{exe}
\ex \label{ex:Nua.cinA.maR.kW}
\gll  kɯ-mɯrkɯ ŋu-a cinɤ maʁ kɯ  \\
\textsc{nmlz}:S/A-steal be:\textsc{fact}-\textsc{1sg} even not.be \textsc{sfp} \\
\glt `Anyway it is not me who is the thief.' (elicited)
\end{exe}

\subsubsection{Restrictive focus} \label{sec:restrictive.focus} 
 The most common way to express restrictive focus in Japhug is to combine the exceptive \japhug{ma}{apart from} (and its reduplicated variant \forme{mɯma} § \ref{sec:exceptive}) with a negative predicate. This can be a verb with a negative prefix as in (\ref{ex:XsArZaR}), or a negative existential verb as in (\ref{ex:Wmi.Wntsi.ma.me}).
 
 \begin{exe}
\ex  \label{ex:XsArZaR}
\gll   χsɤ-rʑaʁ ma mɯ-pɯ-tsu-a ɲɤ-sɯso ri χsɯ-xpa pjɤ-tsu tɕe,  \\
three-day apart.from \textsc{neg}-\textsc{pfv}-pass-\textsc{1sg} \textsc{ifr}-think \textsc{lnk} three-year \textsc{ifr}-pass \textsc{lnk} \\
\glt `He thought that he had spent only three days, but three years had passed.' (2011-4-smanmi, 178)
  \end{exe}
  
  \begin{exe}
\ex  \label{ex:Wmi.Wntsi.ma.me}
\gll  rkoŋɟɤl nɯnɯ, ɯ-mi ɯ-ntsi nɯ ma me kʰi.   \\
one.legged.demon \textsc{dem} \textsc{3sg}.\textsc{poss}-leg \textsc{3sg}.\textsc{poss}-one.of.a.pair \textsc{dem} apart.from not.exist:\textsc{fact} \textsc{hearsay} \\
\glt  `It is said that one-legged demons only had one leg.' (140510 rkoNJAl, 4)
  \end{exe}
  
The restrictive focus construction implies the presence of a noun phrase with a numeral or a CN when the restriction bears on the quantity, but restriction can also be qualitative, without quantifier, as in (\ref{ex:karGi.Zo.kWfse.ma.me}).

\begin{exe}
\ex \label{ex:karGi.Zo.kWfse.ma.me}
 \gll   ɯ-mat nɯnɯ na-lɤt ɕɯmɯma nɤ kɯ-ndɯ\redp{}ndɯβ ʑo ma me, karɣi ʑo kɯ-fse ma me  \\
 \textsc{3sg}.\textsc{poss}-fruit \textsc{dem} \textsc{pfv}:3\fl{}3'-throw just \textsc{lnk}  \textsc{nmlz}:S/A-\textsc{emph}\redp{}small \textsc{emph} apart.from not.exist:\textsc{fact} turnip.seed \textsc{emph} \textsc{nmlz}:S/A-be.like apart.from not.exist:\textsc{fact} \\
 \glt  `When the fruit of (xanthoxyllum) has just come out, there is only something very small, only like a turnip seed.'  (07-tCGom, 7)
  \end{exe}
  
The restrictive focus construction can be combined with a scalar focus in \forme{kɯnɤ} (see §  \ref{sec:kWnA}), as in (\ref{ex:ma.kWme.kWnA}). In this example, \forme{kɯnɤ} has scope over the subordinate clause \forme{stɯsti ma kɯ-me}, which is ambiguous between a participial headless relative (§ XXX) `consisting of only a female all alone' and a manner infinitival clause (§ XXX; in this case the gloss of \forme{kɯ-me} would be \textsc{inf}:\textsc{stat}-not.exist) `even (when) there is only a female all alone'.

  \begin{exe}
\ex \label{ex:ma.kWme.kWnA}
\gll  mu ma, stɯsti ma kɯ-me kɯnɤ cʰɯ-rɤŋgɯm ɲɯ-ɕti. \\
female apart.from alone apart.from \textsc{nmlz}:S/A-not.exist also \textsc{ipfv}-lay.eggs \textsc{sens}-be.\textsc{affirm} \\
\glt `Even only a female (hen) alone does lay eggs.' (150819 kumpGa, 11)
\end{exe}
   
A second possibility to express restrictive focus is the use of the adverb \japhug{ʁɟa}{completely, all} (§ XXX) with scope on a  noun phrase rather than the whole clause as in (\ref{ex:RJa.tunWndze}).\footnote{The form \forme{ʁɟa} possibly originates from the first syllable of Tibetan \tibet{གཡའ་མ་}{gja.ma}{stone slab}, through a meaning `bare rock'.}  

\begin{exe}
\ex \label{ex:stAmku.RJa}
\gll alo mbroχpa ra tɕe tɕe nɤki qra cʰo qambrɯ ra ɣɯ nɯ-ɣli nɯnɯ
tɕe nɯ tu-wum-nɯ, tu-sɯɣ-rom-nɯ mbroχpa sɤtɕʰa tɕe stɤmku ʁɟa ɲɯ-ɕti ma si maŋe tɕe tɕe    \\
upstream nomad \textsc{pl} \textsc{lnk} \textsc{lnk} \textsc{filler} female.yak \textsc{comit} male.yak \textsc{pl} \textsc{gen} \textsc{3pl}.\textsc{poss}-dung \textsc{dem} \textsc{lnk} \textsc{dem} \textsc{ipfv}-gather-\textsc{pl} \textsc{ipfv}-\textsc{caus}-be.dry-\textsc{pl} nomad place \textsc{lnk} grassland completely \textsc{sens}-be.\textsc{affirm} \textsc{lnk} tree not.exist:\textsc{sens} \textsc{lnk} \textsc{lnk}  \\
\glt `Upstream, in the nomad areas, they gather and dry yak dung, as in nomad places there is only grassland, there no trees.' (05-tamar, 7-10)
\end{exe}

The adverb \forme{ʁɟa} (here used rather as a noun modifier) is related to the denominal verb \japhug{aʁɟa}{be bald, be bare} (see § XXX on the \forme{a-} derivation), which can be applied to nouns such as \japhug{stɤmku}{grassland} and \japhug{zgo}{mountain}.
 
\begin{exe}
\ex \label{ex:RJa.tunWndze}
 \gll qajɯ ʁɟa tu-nɯ-ndze, ma nɯ ma tɤ-rɤku kɯ-fse ra ndze mɤ-ŋgrɤl. \\
 bug completely \textsc{ipfv}-\textsc{auto}-eat[III] \textsc{lnk} \textsc{dem} apart.from \textsc{indef}.\textsc{poss}-harvest \textsc{nmlz}:S/A-be.like \textsc{pl} eat[III]:\textsc{fact} \textsc{neg}-be.usually.the.case:\textsc{fact} \\ 
\glt `It only eats insects, it does not eat cultivated plants.' (140511 qamtsWrmdzu, 16)
\end{exe}

While in (\ref{ex:RJa.tunWndze})  and (\ref{ex:stAmku.RJa}) it remains ambiguous whether \forme{ʁɟa} forms a syntactic constituent with the previous nouns or the following verb, in (\ref{ex:RJa.kW}) the presence of the ergative makes it clear that \forme{ʁɟa} is not a clausal adverb, and belongs to the postpositional phrase headed by \forme{kɯ}.

\begin{exe}
\ex \label{ex:RJa.kW}
 \gll [tɤ-lu cʰo tɯkrimgo ʁɟa kɯ] cʰɯ-z-ɣɤ-wxti-nɯ. \\
 \textsc{indef}.\textsc{poss}-milk \textsc{comit} doughnut completely \textsc{erg} \textsc{ipfv}-\textsc{caus}-\textsc{caus}-be.big-\textsc{pl} \\
\glt `They (used to) raise up (the babies) by feeding them milk and doughnuts only.' (140426 tApAtso kAnWBdaR, 102)
\end{exe}

The same applies to (\ref{ex:Wru.RJa.nW}), where the presence of the demonstrative \forme{nɯ} after \forme{ʁɟa} shows that it belongs to the same noun phrase.

\begin{exe}
\ex \label{ex:Wru.RJa.nW}
 \gll ɯ-rdoʁ nɯ-me tɕe, [ɯ-ru ʁɟa nɯ], pɯ-kɤ-tɤβ nɯnɯ, taʁndzɤr ɯ-ŋgɯ tú-wɣ-rku tɕe, \\
 \textsc{3sg}.\textsc{poss}-grain \textsc{pfv}-not.exist \textsc{lnk} \textsc{3sg}.\textsc{poss}-stalk completely \textsc{dem} \textsc{pfv}-\textsc{nmlz}:P-thresh \textsc{dem} feeding.emmer \textsc{3sg}.\textsc{poss}-inside \textsc{ipfv}-\textsc{inv}-put.in \textsc{lnk} \\
 \glt `When all the grains have been removed, the bare stalks, the one that have been threshed, one puts them in a feeding emmer.' (140513 tWrtsi, 5)
\end{exe}

A reduplicated emphatic form \forme{ʁɟɯ\redp{}ʁɟa} is also found as in (\ref{ex:RJWRJa.kW})

\begin{exe}
\ex \label{ex:RJWRJa.kW}
 \gll χtɕɤnzɤn ʁɟɯ\redp{}ʁɟa kɯ ʑo pɯ́-wɣ-nɤjo ɕti ɲɯ-ŋu.  \\
beast \textsc{emph}\redp{}completely \textsc{erg} \textsc{emph} \textsc{pst}.\textsc{ipfv}-\textsc{inv}-wait be.\textsc{affirm}:\textsc{fact} \textsc{sens}-be \\
\glt `It was all wild beasts waiting for him (there).' (Norbzang2005, 308)
 \end{exe}
 
A third option to express restrictive focus is the IPN \forme{ɯ-jlu}, which is used in the meaning `uncooked' as a property IPN (§ \ref{sec:property.nouns}), but has become grammaticalized as a restrictive marker `exclusively, without anything else' (presumably from an intermediate meaning `plain'), as in (\ref{ex:Wjlu.Zo}).

\begin{exe}
\ex \label{ex:Wjlu.Zo}
 \gll srɤz nɯ kɯ tɕʰoz ɯ-jlu ʑo pjɯ-nɯjɤntɤn pɯ-ɕti ma jɯm nɯ mɯ-pjɤ-ɕar ɲɯ-ŋu, \\
prince \textsc{dem} \textsc{erg}  religion \textsc{3sg}.\textsc{poss}-exclusively \textsc{emph} \textsc{ipfv}-be.assiduous.in  \textsc{pst}.\textsc{ipfv}-be.\textsc{affirm} \textsc{lnk} wife \textsc{dem} \textsc{neg}-\textsc{ifr}.\textsc{ipfv}-look.for \textsc{sens}-be \\
 \glt `The prince was focused exclusively in the study of religion, and was not looking for a wife.' (sras2003, 3)
 \end{exe}

For the expression of restrictive focus with temporal noun phrases or clauses, the postposition \japhug{kóʁmɯz}{only after} can also be used, especially with the demonstrative in the expression \japhug{nɯ kóʁmɯz nɤ}{only then} (§ \ref{sec:temporal.postpositions}, § XXX).



\subsection{Identity modifiers} \label{sec:identity.modifier}
There is no specific identity modifier `the same' in Japhug. The only way to express this meaning is to use the S-participle of the verb \japhug{naχtɕɯɣ}{be the same} (a denumeral verb of Tibetan origin, § \ref{sec:tibetan.numerals}, see also § \ref{sec:comitative} on the syntax of this stative verb and § XXX on its derivation) in a relative clause, as in (\ref{ex:tArmi.kWnaXtCWG}) (a possessor relative, § XXX). This participle is also used adverbially (see § XXX).

\begin{exe}
\ex \label{ex:tArmi.kWnaXtCWG}
\gll tɤ-rmi kɯ-naχtɕɯɣ pjɤ-dɤn wo kɤmɲɯ, nɤki kɯrɯ ra tɕe. \\
\textsc{indef}.\textsc{poss}-name \textsc{nmlz}:S/A-be.the.same \textsc{ifr}.\textsc{ipfv}-be.many \textsc{sfp} pl.n. \textsc{filler} Tibetan \textsc{pl} \textsc{lnk} \\
\glt `There were many people who had identical names, in Kamnyu, among the Tibetans.' (140522 tshupa, 161)
\end{exe}


There are two prenominal modifiers expressing non-identity in Japhug: \japhug{kɯmaʁ}{other} and the numeral \japhug{ci}{one}, which in prenominal position means `the other one' (in postnominal position, it is used as an indefinite article, see § \ref{sec:indef.article}). Both of these words can also be used as pronouns, though \forme{ci} requires to be combined with the demonstrative \forme{nɯ} in this usage (see § \ref{sec:other.pro}).

The modifier \forme{kɯmaʁ} is prenominal in its meaning `other', as in (\ref{ex:kWmaR.tWrme}). 

\begin{exe}
\ex \label{ex:kWmaR.tWrme}
\gll tɯ-zda nɯ ma kɯmaʁ tɯrme a-pɯ-me tɕe, kʰa ra aʁɤndɯndɤt ɲɯ-ɤ<nɯ>ɣro ɲɯ-ŋu ɲɯ-ti. \\
\textsc{genr}.\textsc{poss}-companion \textsc{dem} apart.from other person \textsc{irr}-\textsc{ipfv}-not.exist \textsc{lnk} house \textsc{pl} everywhere \textsc{ipfv}-<\textsc{auto}>play \textsc{sens}-be \textsc{sens}-say \\
\glt `(Our neighbour) says that if there are no other persons apart from family members, (the monkey) would play everywhere in the house.' (19-GzW2, 10)
\end{exe}

There are apparent examples of \japhug{kɯmaʁ}{other} in postnominal position, as in (\ref{ex:kWmaR.taXtW}) and (\ref{ex:kWmaR.tanWsWBzu}), but in such sentences \forme{kɯmaʁ} is a preverbal adverb, not a noun modifier, with a slightly different meaning `anew'. In (\ref{ex:kWmaR.taXtW}), the usage of \forme{kɯmaʁ} is very similar to its Chinese equivalent \ch{另外}{lìngwài}{other} in the corresponding Chinese sentence \zh{阿兰另外给我买了一部手机}, where the preverbal position of \ch{另外}{lìngwài}{other} clearly shows that it is not a noun modifier. 

\begin{exe}
\ex \label{ex:kWmaR.taXtW}
\gll <alan> kɯ a-<dianhua> kɯmaʁ ta-χtɯ \\
p.n. \textsc{erg} \textsc{1sg}.\textsc{poss}-phone other \textsc{pfv}:3\fl{}3'-buy \\
\glt `Alan bought me a new phone.' (conversation, 17-03-27)
\end{exe}

\begin{exe}
\ex \label{ex:kWmaR.tanWsWBzu}
\gll a-ʁi kɯ kʰa kɯmaʁ ta-nɯ-sɯ-βzu qʰe, \\
\textsc{1sg}.\textsc{poss}-younger.sibling \textsc{erg} house other \textsc{pfv}:3\fl{}3'-\textsc{auto}-\textsc{caus}-make \textsc{lnk} \\
\glt `My brother made himself a new house.' (14-tApitaRi, 304)
\end{exe}

The identity determiner \japhug{kɯmaʁ}{other} is grammaticalized from the S-participle of the verb \japhug{maʁ}{not be}, \forme{kɯ-maʁ} `who/which is not X', which is still widely used, as in (\ref{ex:tChWrtsAm.kWmaR}) and (\ref{ex:sthWci.kWmaR}).


%\begin{exe}
%\ex \label{ex:Wstu.kWmaR}
%\gll ɯ-stu kɯ-maʁ me, kɯki mɤ-kɯ-pe me \\
%\textsc{3sg}.\textsc{poss}-truth \textsc{nmlz}:S/A-not.be not.exist:\textsc{fact} dem.\textsc{prox} \textsc{neg}-\textsc{nmlz}:S/A-be.good not.exist:\textsc{fact} \\
%\glt ` (28-smAnmi, 16)
%\end{exe}

\begin{exe}
\ex \label{ex:tChWrtsAm.kWmaR}
\gll mɤʑɯ [tɕʰɯrtsɤm kɯ-maʁ] nɯnɯ tɕe, tú-wɣ-χtɕi ma nɯ ma kɤ-sqa (mɤ-ra) \\
yet type.of.tsampa \textsc{nmlz}:S/A-not.be \textsc{dem} \textsc{lnk} \textsc{ipfv}-\textsc{inv}-wash \textsc{lnk} \textsc{dem} apart.from \textsc{inf}-boil \textsc{neg}-have.to:\textsc{fact} \\
\glt `The tsampa that is not `chu.rtsam', one needs to wash it, but not to boil it.' (2002tWsqar, 112)
\end{exe}

\begin{exe}
\ex \label{ex:sthWci.kWmaR}
\gll  [ɯ-rkɯ wuma ʑo stʰɯci kɯ-maʁ] nɯtɕu tɤ-ri ci kú-wɣ-lɤt \\
\textsc{3sg}.\textsc{poss}-side really \textsc{emph} so.much \textsc{nmlz}:S/A-not.be \textsc{dem}:\textsc{loc} \textsc{indef}.\textsc{poss}-thread once \textsc{ipfv}-\textsc{inv}-throw \\
\glt `One sews a thread at a place which is not too much on the border (of the patch)'. (12-kAtsxWb, 16)
\end{exe}

The modifier \forme{ci} differs from \forme{kɯmaʁ} in that it is necessarily definite, meaning `the other one', as in (\ref{ex:ci.rWdaR}), where it refers to an animal that it chased by lions, which was previously mentioned in the text.

\begin{exe}
\ex \label{ex:ci.rWdaR}
\gll ʑɯrɯʑɤri qʰe ci rɯdaʁ nɯ dɯxpa ma nɯ-kɤ-ndza ɯ-spa ɲɯ-ɕti qʰe, qʰe pjɯ-ndʐaβ qʰe mɯ-ɲɯ-cʰa qʰe, \\
progressively \textsc{lnk} one animal \textsc{dem} poor.of \textsc{lnk} \textsc{3pl}.\textsc{poss}-\textsc{nmlz}:P-eat \textsc{3sg}.\textsc{poss}-material \textsc{sens}-be.affirm \textsc{lnk} \textsc{lnk} \textsc{ipfv}-\textsc{anticaus}:make.fall \textsc{lnk} \textsc{neg}-\textsc{ipfv}-can \textsc{lnk} \\
\glt `The other animal, poor of him, it is their prey, progressively it falls down and cannot stand it anymore.' (20-sWNgi, 43)
\end{exe}

Interestingly, the determiner \forme{ci} does not have scope over other noun modifiers. For instance, in (\ref{ex:ci.tCheme.kWNAn}), the noun \japhug{tɕʰeme}{woman} occurs with an attributive adjective in participial form \forme{kɯ-ŋɤn} `who is evil' (a relative clause, see § \ref{sec:attributes}), but the meaning is not `the other evil woman' as could have been expected (since the woman who is the subject of the sentence is, by contrast, a kind person), and rather must be `the other woman, the evil one'. There is no pause in the recording that could lead us to suppose that \forme{kɯ-ŋɤn} here is an apposition -- it is rather a postnominal relative.

\begin{exe}
\ex \label{ex:ci.tCheme.kWNAn}
\gll nɤki, tɕʰeme nɯ ɯ-ɕki ɯ-kɯ-sɤja jo-ɕe, ci tɕʰeme kɯ-ŋɤn nɯ ɯ-ɕki. \\
\textsc{filler} women \textsc{dem} \textsc{3sg}-\textsc{dat} \textsc{3sg}.\textsc{poss}-\textsc{nmlz}:S/A-give.back \textsc{ifr}-go one woman \textsc{nmlz}:S/A-be.evil \textsc{dem} \textsc{3sg}-\textsc{dat} \\
\glt `She went to give it back to the woman, the other one, the evil woman.' (140515 jiesu de laoren, 90)
\end{exe}

%a-ʁi kɯnɤ tɯrme kha kɤ-sɯxɕe mɯ́j-khɯ qhe,

%ci qhɤjmbaʁ nɯ kɯ-jaʁ kɯ-fse nɯnɯ 
%mtshalu ɯ-cu tɕe nɤki,
%tɯ-mgo zmɤrɤβ kú-wɣ-nɯ-lɤt sna.
%16-RlWmsWsi
%li ci /ɯt/ ɯ-tɯphu nɯ tɤpu qhɤjmbaʁ tu-ti-nɯ ŋu tɕe,
\subsection{Attributes} \label{sec:attributes}
%\ref{sec:property.nouns}
\subsubsection{Attributive postnominal modifiers} \label{ex:attributive.postnominal}
In addition to the postnominal markers studied above (numeral and number § \ref{sec:number.determiners}, demonstratives § \ref{sec:demonstrative.determiners}, quantifiers § \ref{sec:quantifiers.determiners}, definiteness markers § \ref{sec:indef.article}, topic and focus markers), there are a certain number of nouns that can serve a post-nominal modifiers.

An entire class of such nouns consists of the privative nouns in \forme{-lu} `...less', described in § \ref{sec:privative}.

The word \japhug{wuma}{real, really} from Tibetan \tibet{ངོ་མ་}{ŋo.ma}{real, true} is generally used adverbially as an intensifier, in particular with stative verbs (§ XXX), but also occurs as a postnominal modifier meaning ` real', its original meaning, as in (\ref{ex:lhAndzxi.wuma}).

\begin{exe}
\ex \label{ex:lhAndzxi.wuma}
\gll ɬɤndʐi wuma nɯ nɤʑo ɲɯ-tɯ-ŋu ma aʑo ɬɤndʐi ɲɯ-maʁ-a \\
demon real \textsc{dem} \textsc{2sg} \textsc{sens}-2-be \textsc{lnk} \textsc{1sg} demon \textsc{sens}-not.be-\textsc{1sg} \\
\glt `You are the real demon, not me.' (2002lhandzi, 12)
\end{exe}
%ɯ-qa nɯ qarŋe, tɤtsoʁ wuma nɯ. 

%\japhug{ɕɯŋarɯra}{each better than the other} XXXX
% rɟɤlpu ɕɯŋarɯra kɯ ta-tʰu-nɯ ɕti ri, mɯ-tɤ-nɤla-j ɕti tɕe,
% 2003 qachga, 71

\subsubsection{Attributive prenominal modifiers}

\subsubsection{Participial relatives}
%Postnominal or head-internal? definiteness wuma ʑo ... kɯ-

 %rɟɤlpu nɤrɯβzaŋ nɯ kɯ nɤki stu kɯ-mna tɕheme nɯ ɲɤ-nɯ-ɕar ɲɯ-ŋu
 
 
 %ɯʑo sɤz rɯdaʁ kɯ-xtɕi nɯra tu-ndze
 \section{Noun coordination}
\subsection{Coordination or dependency} \label{sec:coordinator}
The closest thing to a noun coordinator in Japhug is the comitative marker \forme{cʰo}, which is argued to be a postposition in § \ref{sec:comitative}.


%tɤ-rpi nɤ tɤ-rpi ʑo ɲɯ-sɯβzu-nɯ ɲɯ-ŋu tɕe,
%2003kandZislama, 112
\subsection{Bare coordination}

\subsubsection{Enumeration} \label{sec:noun.enumeration}
%mbro, jla, nɯŋa, mbala, tshɤt, qaʑo, paʁ, nɯra nɯtɕu ʁɟa z-ɲɯ́-wɣ-lɤɣ pɯ-ŋu.
%qhe tshɤt qaʑo ra ɣɯ nɯ-ndza nɯra ɲɯ-sna. 

\subsubsection{Noun dyads} \label{sec:dyads}
Noun dyads are a pair of nouns occurring in a fixed order, without intervening linker or postposition, and sharing their number and case markers. A good example is provided by the expression `parents' comprising the kinship terms \japhug{tɤ-mu}{mother} and \japhug{tɤ-wa}{father}, as in (\ref{ex:amu.awa.ni.GW}). Note that while number and case markers are shared by both nouns, each of them takes its own possessive prefix, and both prefixes are coreferent. 

\begin{exe}
\ex \label{ex:amu.awa.ni.GW}
 \gll nɯ a-mu a-wa ni ɣɯ ŋu \\
 \textsc{dem}  \textsc{1sg}.\textsc{poss}-mother \textsc{1sg}.\textsc{poss}-father \textsc{du} \textsc{gen} be:\textsc{fact} \\
 \glt `This is for my parents.' (meimei de gushi)
\end{exe}

The dyad for `parents' has a honorific variant, originally used for noblemen in the traditional society. It comprises the terms \japhug{tɤ-pa}{father} and \japhug{tɤ-ma}{mother}, which are borrowed from Tibetan \tibet{ཨ་ཕ་}{ʔa.pʰa}{father}  and \tibet{ཨ་མ་}{ʔa.ma}{mother} respectively. Interesting, the honorific expression follows the `father-mother' order (as in example \ref{ex:apa.ama}), while the native one puts `mother' in the first place.

\begin{exe}
\ex \label{ex:apa.ama}
 \gll nɯ kɯ-fse a-pa a-ma ni kɯ ɲɯ-ti-ndʑi tɕe \\
 \textsc{dem} \textsc{nmlz}:S/A-be.like \textsc{1sg}.\textsc{poss}-father  \textsc{1sg}.\textsc{poss}-mother \textsc{du} \textsc{erg} \textsc{sens}-say-\textsc{du} \textsc{lnk} \\
 \glt `My parents say this.' (2003nyima2, 94)
\end{exe}

Other common dyads include \forme{rgɤtpu rgɤnmɯ} `old man(men) and woman(women)', \forme{tɤ-tɕɯ tɕʰeme} `boy(s) and girl(s)' with APNs. They are most commonly used as collectives with indefinite referents as in (\ref{ex:tAtCW.tCheme.tWsAmdzW}), but are also attested with definite ones, as in (\ref{ex:rgAtpu.rgAnmW}).

\begin{exe}
\ex \label{ex:tAtCW.tCheme.tWsAmdzW}
 \gll  tɤ-tɕɯ tɕʰeme tɯ-sɤ-ɤmdzɯ ʑaka tu \\
 \textsc{indef}.\textsc{poss}-son girl \textsc{genr}.\textsc{poss}-\textsc{nmlz}:\textsc{oblique}-sit each \textsc{exist}:fact \\
\glt `Gents and ladies each have (different) seating places.' (31-khAjmu, 10)
\end{exe}

\begin{exe}
\ex \label{ex:rgAtpu.rgAnmW}
 \gll rgɤtpu rgɤnmɯ ni kɯ kɯki tɤ-pɤtso χsɯm ki kɤsɯfse ʑo cʰɤ-ɣɤ-wxti-ndʑi. \\
 old.man old.woman \textsc{du} \textsc{erg} \textsc{dem}.\textsc{prox} \textsc{indef}.\textsc{poss}-child three \textsc{dem.prox} all \textsc{emph} \textsc{ifr}-\textsc{caus}-be.big-\textsc{du} \\
\glt `The old man and the old woman raised all these three children.' (140514 huishuohua de niao, 60)
\end{exe}

\section{Apposition} \label{sec:apposition}

\section{The structure of the noun phrase}


\section{Nominal predicates} \label{sec:nominal.predicates}
 %Ideophones
%\include{chapters/4-01} %The verbal template
%\chapter{Person indexation} \label{chap:indexation}
In Japhug, person indexation is the defining feature of finite verbs, as opposed to non-finite verbs (§\ref{chap:non-finite}) and other parts of speech. Japhug finite verb forms index one or two arguments, depending on the transitivity of the verb, using a combination of prefixes, suffixes and stem alternation. No verb indexes more than two arguments. The indexation system is very close to a canonical direct-inverse system (§\ref{sec:direct-inverse}).

This chapter first presents intransitive and transitive conjugations, investigates the issue of optional number indexation, and then discusses the origin of person indexation affixes. In addition, it documents the analogical extension of person indexation suffixes to non-finite verb forms in some specific contexts.

\section{Intransitive verbs} \label{sec:intr.indexation}
Intransitive verbs comprise dynamic, stative and semi-transitive verbs. All of the verbs have in common the property of indexing one argument, the intransitive subject, which when overt is in absolutive form (§\ref{sec:absolutive.S}).

\subsection{The intransitive paradigm} \label{sec:intransitive.paradigm}
Table \ref{tab:intransitive.indexation} illustrates the paradigm of intransitive verbs in Kamnyu Japhug, using the verb \japhug{ɕe}{go} in the Factual non-past\footnote{This TAM category is chosen to illustrate the paradigms due to the fact that it does not bear any orientation prefix, but at the same time presents stem alternation in the transitive paradigm.} as an example. Other Japhug dialects have slightly different indexation suffixes, a question discussed in §\ref{sec:indexation.suffixes.history} with comparative evidence from other Gyalrong languages.

There is no stem alternation related to person indexation in the intransitive paradigm in any Japhug dialect. The invariable stem is represented with the symbol \ro{} in Table \ref{tab:intransitive.indexation}.\footnote{This notation follows the Kirantological tradition (for instance \citealt{driem93dumi}). }

\begin{table}[H] \centering
\caption{The intransitive conjugation in Japhug}\label{tab:intransitive.indexation}
\begin{tabular}{lllllllll} \lsptoprule
Person & Form & \japhug{ɕe}{go} (Factual non-past) \\
\midrule
\textsc{1sg} & \ro{}-\forme{a} & \forme{ɕe-a} \\
\textsc{1du} & \ro{}-\forme{tɕi} & \forme{ɕe-tɕi} \\
\textsc{1pl} & \ro{}-\forme{ji} & \forme{ɕe-j} \\
\midrule
\textsc{2sg} & \forme{tɯ}-\ro{} & \forme{tɯ-ɕe} \\
\textsc{2du} & \forme{tɯ}-\ro{}-\forme{ndʑi} & \forme{tɯ-ɕe-ndʑi} \\
\textsc{2pl} & \forme{tɯ}-\ro{}-\forme{nɯ} & \forme{tɯ-ɕe-nɯ} \\
\midrule
\textsc{3sg} & \ro{} & \forme{ɕe} \\
\textsc{3du} & \ro{}-\forme{ndʑi} & \forme{ɕe-ndʑi} \\
\textsc{3pl} & \ro{}-\forme{nɯ} & \forme{ɕe-nɯ} \\
\midrule
generic & \forme{kɯ}-\ro{} & \forme{kɯ-ɕe} \\
\lspbottomrule
\end{tabular}
\end{table}

In the intransitive paradigm, five suffixes and two prefixes are found. The stress is always on the last syllable of the verb stem, and all person indexation suffixes, including \forme{-a}, are unstressed and sometimes are even devoiced (§XXX). Unlike other languages of the Trans-Himalayan, such as Khaling (where the dual inclusive and the third dual are homophonous, see \citealt[1113]{jacques12khaling}), in Japhug all slots in the intransitive paradigm are distinct, without ambiguity.  

\subsubsection{First person}

First person subjects are indexed by a set of three suffixes marking both person and number: \forme{-a}, \forme{-tɕi} and \forme{-ji} respectively for first singular, dual and plural. As in the pronominal paradigms (§\ref{sec:pers.pronouns}), there is no inclusive/exclusive distinction in Japhug.

The \textsc{1sg} \forme{-a} suffix is the only suffix in Japhug with a vowel other than \forme{ɯ} (or \forme{i} after palatal and alveolo-palatal consonants, §\ref{sec:W.i.contrast}), and is the only indexation suffix that can be followed by another indexation suffix in the transitive paradigm (§\ref{sec:double.number.indexation}). The \forme{-a} \textsc{1sg} person index is among the suffixes revealing the underlying form of the codas: \forme{-β}, \forme{-ɣ}, \forme{-ʁ}, \forme{-z}, which become unvoiced in some contexts (§XXX) are realized as voiced (see for instance in Table \ref{tab:verb.stem.1sg} below; \forme{-β} is realized \phonet{-w-} in this context, see §XXX), but the coda \forme{-t} remains unvoiced (for instance \forme{scit-a} be.happy-\textsc{1sg} `I am happy'). The codas are resyllabified; for instance \forme{scit-a} is syllabified as \forme{sci/ta}).

 Some verb stems (independently of transitivity) undergo predictable phonological alterations when followed by \forme{-a}. With verb stems whose last syllable is an open syllable,  the \forme{-a} suffix merges with the vowel of the last syllable. With closed syllable verb stem in \forme{-ɤC} (C representing a coda), the \textsc{1sg} suffix causes vowel assimilation. These phonological rules are presented in Table \ref{tab:verb.stem.1sg}.

When the verb stem ends in \forme{-a}, the \textsc{1sg} suffix merges with the stem as \phonet{a} in Kamnyu Japhug, resulting in homophony between the \textsc{1sg} and the \textsc{3sg} forms. The surface form \phonet{rga} corresponds to both \textsc{1sg} \japhug{rga-a}{I like it} and \textsc{3sg} \japhug{rga}{he likes it}. The fused and invisible suffix is systematically indicated in the orthography used in this grammar. In some dialects of Japhug, a long vowel occurs in the \textsc{1sg}, which thus remains distinct from the \textsc{3sg}.

When the verb stem ends in the mid-high vowels \forme{-e} and \forme{-o}, these vowels become the corresponding high vowels \forme{-i} and \forme{-u} when followed by \textsc{1sg} in Kamnyu Japhug. This alternation does not occur in all Japhug dialects.

With verb stem ending in \forme{-ɤt}, \forme{-ɤn}, \forme{-ɤβ}, \forme{-ɤm}, \forme{-ɤr}, \forme{-ɤl} and \forme{-ɤz},  the \textsc{1sg} suffix causes non-optional vowel assimilation \forme{-ɤC-a} $\Rightarrow$ \ipa{-aCa}; Table \ref{tab:verb.stem.1sg} provides examples for all rhymes of this type. In the orthography employed in this grammar, these forms are transcribed as \forme{aC-a} rather than the underlying \forme{ɤC-a} (\forme{jɣat-a} rather than \forme{jɣɤt-a}), to indicate the fact that \forme{ɤ} $\Rightarrow$ \forme{a} assimilation is obligatory in this context.

\begin{table}
\caption{Predictable phonological alternations on the verb stem caused by the \forme{-a} \textsc{1sg} suffix in Kamnyu Japhug} \label{tab:verb.stem.1sg}
\begin{tabular}{llllll}
\lsptoprule
Rhyme of the  & Result of  &Examples \\
last syllable & fusion with  \\
of the verb stem & the \textsc{1sg} suffix \\
\midrule
\forme{-e} & \phonet{-ia} & \forme{ɕe-a} $\Rightarrow$ \phonet{ɕia} `I will go there' \\
\forme{-o} & \phonet{-ua} & \forme{tso-a} $\Rightarrow$ \phonet{tsua} `I understand it' \\
\forme{-a} & \phonet{-a} & \forme{rga-a} $\Rightarrow$ \phonet{rga} `I like it' \\
\midrule
\forme{-ɤβ} & \phonet{-awa} & \forme{tʰɯ-rdɤβ-a} $\Rightarrow$ \phonet{tʰɯrdawa} `I lost money' \\
\forme{-ɤm} & \phonet{-ama} & \forme{mtsʰɤm-a} $\Rightarrow$ \phonet{mtsʰama} `I hear it' \\
\forme{-ɤt} & \phonet{-ata} & \forme{jɣɤt-a} $\Rightarrow$ \phonet{jɣata} `I will come back' \\
\forme{-ɤn} & \phonet{-ana} & \forme{tu-nɯsmɤn-a} $\Rightarrow$ \phonet{tunɯsmana} \\
&&  `I will will treat it' \\
\forme{-ɤr} & \phonet{-ara} & \forme{pɯ-atɤr-a} $\Rightarrow$ \phonet{patara} `I fell down' \\
\forme{-ɤl} & \phonet{-ala} & \forme{nɯ-nɯtɯfɕɤl-a} $\Rightarrow$ \phonet{nɯ-nɯtɯfɕal-a}\\
&& `I had diarrhea' \\
\forme{-ɤz} & \phonet{-aza} & \forme{mkʰɤz-a} $\Rightarrow$ \phonet{mkʰaza} `I am expert at it' \\
\lspbottomrule
\end{tabular}
\end{table}

The first dual \forme{-tɕi} suffix (\forme{-tsə} in some dialects of Japhug, §\ref{sec:indexation.suffixes.history}) only causes regular devoicing assimilation on the coda of the verb stem: \forme{-z}, \forme{-r}, \forme{-ɣ}, \forme{-ʁ} are realized as \forme{-s}, \forme{-ʂ}, \forme{-x}, \forme{-χ} when followed by \forme{-tɕi} (for instance \forme{mkʰɤz-tɕi} is pronounced \phonet{mkʰɛ́stɕi}). The labial coda \forme{-β} is not affected. 

The first plural \forme{-ji} has two allomorphs, \forme{-j} and \forme{-i}. The first one occurs on verb stems ending in open syllables, for instance \japhug{ɕe-j} `we (will) go', and the second follows verb stems in closed syllables, such as \forme{scit-i} `we are happy', with resyllabification of the coda (\forme{sci/ti}). Like the \forme{-a} suffix discussed above, the suffix \forme{-i} reveals the underlying form of the codas. The contrast between \forme{-ɯ} and \forme{-i} is neutralized when followed by the \textsc{1pl} suffix:; for instance, the last syllable of \forme{smi tʰɯ-βlɯ-j} `we made a fire'  and \forme{lɤpɯɣ pɯ-βli-j} `we planted radish' is considered to be homophonous by Tshendzin (§XXX).

\subsubsection{Non-first person}

Second and third person forms have the same set of suffixes (zero, \forme{-ndʑi} and \forme{-nɯ} for singular,dual and plural respectively) and only differ by the presence of a \forme{tɯ-} prefix in second person forms. Unlike in Situ (\citealt[197-208]{linxr93jiarong}), there is no distinct second person suffix in the \textsc{2sg}.

The non-first person dual and plural suffixes \forme{-ndʑi} and \forme{-nɯ} (some Japhug dialects have \forme{-ndzə} in the dual instead, see §\ref{sec:indexation.suffixes.history}) nasalize the coda \forme{-t} to \phonet{n}, which is not audible before \forme{-ndʑi} and results in a geminate in the plural. For instance, \forme{scit-ndʑi} and \forme{scit-nɯ} are realized as \phonet{scíndʑi} and \phonet{scínnɯ} respectively. The vowel \forme{-i} and \forme{-ɯ} is often elided, resulting in apparent \forme{-n} codas. The contrast between the codas \forme{-n} and \forme{-t} is neutralized in these forms: the last two syllables of both \forme{tu-nɤndɯt-nɯ} \textsc{ipfv}-fight-\textsc{pl} `they fight (over it)' and \forme{pjɯ-ndɯn-nɯ}  \textsc{ipfv}-read-\textsc{pl} `they read/recite it' are thus realized as \phonet{-ndɯ́nnɯ}.

The second person \forme{tɯ-} prefix fuses with the initial \forme{a-} of contracting verbs (§XXX). The result of vowel fusion is \forme{tɯ-a-} $\Rightarrow$ \phonet{ta} in the Factual Non-past (\japhug{tɯ-atɤr}{you will fall down}) or the Past Perfective (\japhug{jɤ-tɯ-ari}{you went there}), but \forme{tɯ-ɤ-} $\Rightarrow$ \phonet{tɤ} in Irrealis, Imperative, Imperfective or Prohibitive (\japhug{ma-tɤ-tɯ-ɤɕqʰe}{don't cough}) forms. Some irregular verbs have unpredictable second person forms (§\ref{sec:intr.person.irregular}). The generic intransitive subject (and object) prefix \forme{kɯ-} follows the same rules of vowel fusion as the second person prefix.

\subsection{Irregular intransitive verbs} \label{sec:intr.person.irregular}
In comparison with Zbu (\citealt{gong18these}), Japhug only has very few irregular verbs. Irregularities related to person marking in Japhug all involve the prefixes.

The second person forms of sensory existential verbs \japhug{ɣɤʑu}{exist} and \japhug{maŋe}{not exist} are infixed rather prefixed. The infixed forms are \forme{ɣɤtɤʑu} and \forme{mataŋe} respectively, as in (\ref{ex:GAtAZu})  (from \citealt[91]{jacques12agreement}) and  (\ref{ex:kAmtshAm.mataNe}).

\begin{exe}
\ex \label{ex:GAtAZu}
\gll iɕqʰa tɯrme ra nɯ-rca ɣɤ<tɤ>ʑu \\
the.aforementioned person \textsc{pl} \textsc{3pl}.\textsc{poss}-following <2sg>exist:\textsc{sens} \\
\glt `(I saw) you among these people.' (elicited)
\end{exe}

\begin{exe}
\ex \label{ex:kAmtshAm.mataNe}
\gll kɤ-mtsʰɤm maka ma<ta>ŋe tɕe, nɤ-kɯ-mŋɤm tu ɯβrɤ-ŋu ma, mɤ-kɯ-pe tu ɯβrɤ-ŋu ma nɯra nɯ-sɯso-t-a. \\
\textsc{inf}-hear at.all <\textsc{2sg}>not.exist:\textsc{sens} \textsc{lnk} \textsc{2sg}.\textsc{poss}-\textsc{nmlz}:S/A-hurt exist:\textsc{fact} \textsc{opt}-be:\textsc{fact} \textsc{sfp} \textsc{2sg}.\textsc{poss}-\textsc{neg}-\textsc{nmlz}:S/A-be.good:\textsc{fact} exist:\textsc{fact} \textsc{opt}-be:\textsc{fact} \textsc{sfp} \textsc{dem}:\textsc{pl} \textsc{pfv}-think-\textsc{pst}:\textsc{tr}-\textsc{1sg} \\
\glt `(I) have not heard at all about you (for some time), I was wondering whether you have some disease, whether something bad happened to you.' (phone conversation, 16-12-28)
\end{exe}

These are not the only infixed forms in the paradigm of these verbs: the generic person \forme{kɯ-} is also infixed (\forme{ɣɤkɤʑu}, \forme{makaŋe}) as is the spontaneous-autobenefactive \forme{nɯ-} (§XXX).

The verb \japhug{zɣɯt}{reach, arrive} has in part of its paradigm forms that are identical to those of contracting verbs (§XXX). In the Past Perfective, it has two alternative second person forms in free variation, the regular \forme{jɤ-tɯ-zɣɯt} and the form \forme{jɤ-tɯ-azɣɯt} with an additional \forme{a-}, illustrated by (\ref{ex:jAtWzGWt.tCe}) and  (\ref{ex:jAtazGWt.mACtsxa}) respectively, coming from two versions of the same story by the same speaker. 

\begin{exe}
\ex \label{ex:jAtazGWt.mACtsxa}
\gll  a-rkɯ mɯ-jɤ-tɯ-azɣɯt mɤɕtʂa mɯ-pɯ-ta-mtsʰɤm tɕe \\
\textsc{1sg}.\textsc{poss}-side \textsc{neg}-\textsc{pfv}-2-arrive until \textsc{neg}-\textsc{pfv}-1\fl{}2-hear \textsc{lnk} \\
\glt `I did not feel your (presence) until you arrived near me.' (Norbzang2012, 260)
\end{exe}

\begin{exe}
\ex \label{ex:jAtWzGWt.tCe}
\gll jɤ-tɯ-zɣɯt tɕe, nɤki, aʑo a-kʰa a-jɤ-tɯ-z-mɤke ma nɤj nɤ-kʰa a-mɤ-jɤ-tɯ-z-mɤke ra mɯ-tɤ-tɯ-tɯt \\
\textsc{pfv}-2-arrive \textsc{lnk} \textsc{filler} \textsc{1sg} \textsc{1sg}.\textsc{poss}-house \textsc{irr}-\textsc{pfv}-2-\textsc{caus}-be.first[III] \textsc{lnk} 
\textsc{2sg} \textsc{2sg}.\textsc{poss}-house \textsc{irr}-\textsc{neg}-\textsc{pfv}-2-\textsc{caus}-be.first[III] have.to:\textsc{fact} \textsc{neg}-\textsc{pfv}-2-say[III] \\
\glt `You did not say ``When you arrive, don't go first to your house, come to my house first.'' (Norbzang2005, 261)
\end{exe}

The paradigm of this verb otherwise includes non-optional contracting (\forme{jɤ-azɣɯt} `he arrived') and non-contracting forms (the immediate converb \forme{ju-tɯ-zɣɯt} `as soon as X arrived', §\ref{sec:immediate.converb}).

\subsection{Semi-transitive verbs} \label{sec:semi.transitive}
Semi-transitive verbs have the same paradigm as plain intransitive verbs, and lack the morphological properties of transitive verbs (§\ref{sec:transitivity.morphology}). Their intransitive subject is in absolutive form. However, they take a semi-object (§\ref{sec:semi.object}), also in absolutive form, as \japhug{paχɕi}{apple} in (\ref{ex:paXCi.ci.taroa}). These semi-objects do present some objectal properties (§XXX).

\begin{exe}
\ex \label{ex:paXCi.ci.taroa}
\gll  tɕe aʑo tʰam kɯki, paχɕi ci tɤ-aro-a tɕe tɕendɤre, 
[...] nɯʑora kɯnɤ ta-sɯ-ɤʁe-nɯ ra \\
\textsc{lnk} \textsc{1sg} now \textsc{dem}.\textsc{prox} apple \textsc{indef} \textsc{pfv}-have-\textsc{1sg} \textsc{lnk} \textsc{lnk} { } \textsc{2pl} also 1\fl{}2-\textsc{caus}-have.to.eat:\textsc{fact}-\textsc{pl} have.to:\textsc{fact} \\
\glt `Now that I have (was given) this apple, I will give it to you also to eat.' (150904 zhongli-zh, 35)
\end{exe}

Unlike transitive verbs, which can index the number of the object if the subject is \textsc{1sg} (§\ref{sec:double.number.indexation}), semi-transitive verbs cannot stack a person index after the \textsc{1sg} \forme{-a}. For instance, in (\ref{ex:XsWm.aroa}), although the object is plural, a form such as $\dagger$\forme{aroa-a-nɯ} with the \forme{-nɯ} plural prefix is strictly prohibited. 

\begin{exe}
\ex   \label{ex:XsWm.aroa}
 \gll aʑo tɤ-rɟit χsɯm aro-a   \\
I \textsc{indef.poss}-child three have:\textsc{fact}-\textsc{1sg} \\
 \glt `I have three children.' (elicited)
\end{exe} 

Some semi-transitive verbs can take both nominal semi-object and complement clauses. For instance, \forme{tso} (which can be translated as `know', `understand' or `realize' depending on the context) occurs with nouns referring to speech or meaning as semi-object (as in \ref{ex:apWtWtso.smWlAm}), finite relative clauses (\ref{ex:rCanW.mWkAtsoa}) and also participial clauses (\ref{ex:kWNu.kutsoa}; see §XXX concerning the analysis of such clauses).

\begin{exe}
\ex   \label{ex:apWtWtso.smWlAm}
 \gll pja mɯndʐamɯχtɕɯɣ nɯ-skɤt a-pɯ-tɯ-tso smɯlɤm! \\
 bird all.type \textsc{3pl}.\textsc{poss}-speech \textsc{irr}-\textsc{pfv}-2-understand wish \\
 \glt `May you understand the speech of all the species of birds!'(2003kandZislama, 85)
\end{exe}

\begin{exe}
\ex   \label{ex:rCanW.mWkAtsoa}
 \gll ci ta-pa-tɕi, nɯstʰɯci tɤ-nɤrʑaʁ ri, [nɯstʰɯci nɤ-ku ʑru rcanɯ] mɯ-kɤ-tso-a \\
 one \textsc{pfv}:3\fl{}3'-do-\textsc{du} so.much \textsc{pfv}-pass(time) so.much \textsc{2sg}.\textsc{poss}-head be.strong:\textsc{fact} \textsc{foc}:\textsc{unexp} \textsc{neg}-\textsc{pfv}-know-\textsc{1sg} \\
\glt `So much time has passed since we have married, I did not realize that your hair was so long.' (Kunbzang2003, 467)
\end{exe}

\begin{exe}
\ex   \label{ex:kWNu.kutsoa}
 \gll  tɕe [tɕʰi ɯ-skɤt kɯ-ŋu ra] ku-tso-a ɲɯ-ra ma tu-tɯ-ti stɯsti, mɯ́j-ɕɯftaʁ-a ɲɯ-ti \\
\textsc{lnk} what \textsc{3sg}.\textsc{poss}-speech \textsc{nmlz}:S/A-be \textsc{pl} \textsc{ipfv}-understand-\textsc{1sg} \textsc{sens}-have.to \textsc{lnk} \textsc{ipfv}-2-say alone \textsc{neg}:\textsc{sens}-remember-\textsc{1sg} \textsc{sens}-say \\
 \glt `He says: `I need to understand what it is about (what objects these words refer to), otherwise if you only speak (if you only explain orally) I won't remember.'' (conversation 14-05-10, 79)
\end{exe}
%\begin{exe}
%\ex   \label{ex:ŋundZi.mWkAtsoa}
% \gll pʰu ŋu ɕi, pʰu ci mu ŋu-ndʑi nɯ mɯ-kɤ-tso-a ma \\
% male be:\textsc{fact} \textsc{qu} male \textsc{indef} female be:\textsc{fact}-\textsc{du} \textsc{dem} \textsc{neg}-\textsc{pfv}-know-\textsc{1sg} \\
% \glt `I did not get to know whether it is the male that is like that, or both whether both the male and the female are.' (24-ZmbrWpGa, 81)
%\end{exe} 
Among semi-transitive verbs, we find the following subclasses:

\begin{itemize}
\item Verbs of cognition and perception: \japhug{tso}{know, understand}, \japhug{sɤŋo}{listen}
\item Verbs of evaluation: \japhug{rga}{like}, \japhug{stu}{believe}
\item Modal verbs: \japhug{cʰa}{can}
\item Verbs of possession:  \japhug{aro}{have}
\item Verbs of pretense:  \japhug{ʑɣɤpa}{pretend}
\item Verbs of obtention: \japhug{aʁe}{have to eat}, \japhug{βɟɤt}{get, obtain}
\item Some adjectival stative verbs: \japhug{mkʰɤz}{be expert}, \japhug{pʰɤn}{be efficient}
\end{itemize}

Most semi-transitive verbs are underived bare roots. The only obviously derived verb is \japhug{ʑɣɤpa}{pretend}, which comes from the reflexive (§XXX) of the verb \japhug{pa}{do}. The verb \japhug{aro}{have} might be denominal from \japhug{tɤ-ro}{surplus, leftover}.

Semi-transitive verbs do not usually take a human semi-object, so that sentences with a first or second person semi-object are clumsy. For some of the verbs above, applicative forms are used when a first or second person object is needed, for instance \japhug{nɯrga}{like} and \japhug{nɤstu}{believe}. The verbs \japhug{stu}{believe} and \japhug{nɤstu}{believe} differ in that the semi-object of the former refers to words (in general, a complement clause; `believe that X') while the object of the latter is a person (`believe him').

Some semi-transitive verbs are labile; some have a transitive counterpart, while other ones have a plain intransitive one (§\ref{sec:semi.tr.labile}). The meaning of the verb also slightly changes depending on transitivity (for instance, \forme{rga} means `like' when semi-transitive, and `be happy' when stative intransitive).

The subject of some semi-transitive verbs, in particular \japhug{tso}{know, understand} and \japhug{ʑɣɤpa}{pretend}, can be optionally marked with the ergative like a transitive subject (§\ref{sec:S.kW}), as \forme{tɤ-mu nɯ kɯ} in (\ref{ex:kW.mWpjAtso}) and \forme{βdaʁmu nɯ kɯ} in (\ref{ex:kW.toZGApa}). 

\begin{exe}
\ex   \label{ex:kW.mWpjAtso}
 \gll  tɕendɤre [tɤ-mu nɯ kɯ] ɕɯ ŋu nɯ maka mɯ-pjɤ-tso tɕeri \\
\textsc{lnk} \textsc{indef}.\textsc{poss}-mother \textsc{dem} \textsc{erg} who be:\textsc{fact} \textsc{dem} at.all \textsc{neg}-\textsc{ifr}.\textsc{ipfv}-know \textsc{lnk} \\
\glt `The old woman did not realize who it was.' (2002qaCpa, 242)
\end{exe}

\begin{exe}
\ex   \label{ex:kW.toZGApa}
 \gll iɕqʰa βdaʁmu nɯ kɯ [wuma ʑo ɯ-sɯm kɯ-sna] to-ʑɣɤpa \\
 the.aforementioned lady \textsc{dem} \textsc{erg} really \textsc{emph} \textsc{3sg}.\textsc{poss}-mind nmlz:S/A-be.good \textsc{ifr}-pretend \\
 \glt `The lady pretended to be a good person.' (140520 ye tiane-zh, 44)
\end{exe}

\subsection{Intransitive verbs with oblique arguments} \label{sec:intr.goal}
Semi-transitive verbs have to be distinguished from motion verbs (or perception verbs) with a goal (§\ref{absolutive.goal}), such as \japhug{ɕe}{go}, \japhug{ɣi}{come} or \japhug{ru}{look at}. These verbs are morphologically intransitive, lacking the morphological characteristics of transitive verbs (§\ref{sec:transitivity.morphology}).

With these verbs, the goal can occur in absolutive form, and superficially resembles a semi-object, as \japhug{sɯŋgɯ}{forest} in (\ref{ex:sWNgW.joCendZi}). Indexation obligatorily occurs with the subject (for example, the \textsc{3du} form in \ref{ex:sWNgW.joCendZi}), never with the goal. As in the case of semi-transitive verbs, number stacking on the 1sg \forme{-a} is not possible (§\ref{sec:semi.transitive}, example \ref{ex:XsWm.aroa}).

\begin{exe}
\ex   \label{ex:sWNgW.joCendZi}
 \gll ʁnɯz ni, [sɯŋgɯ] jo-ɕe-ndʑi. \\
two \textsc{du} forest \textsc{ifr}-go-\textsc{du} \\
\glt `Two (men) went into the forest.' (26-tAGe, 1)
\end{exe}

However, unlike semi-objects, these goals can optionally take locative postpositions, such as \forme{zɯ} in (\ref{ex:sWNgW.zW.joCe}).

\begin{exe}
\ex   \label{ex:sWNgW.zW.joCe}
 \gll tɤ-pɤtso nɯnɯ li [sɯŋgɯ zɯ] jo-ɕe. \\
 \textsc{indef}.\textsc{poss}-child \textsc{dem} again forest \textsc{loc} \textsc{ifr}-go \\
 \glt `The child went again into the forest.' (140428 yonggan de xiaocaifen-zh, 230)
\end{exe}

Dative marking on the goals is also well-attested, as in (\ref{ex:sWNgW.WCki.joCe}) -- with motion verbs, it translates as `towards X'.

\begin{exe}
\ex   \label{ex:sWNgW.WCki.joCe}
 \gll tɕhemɤpɯ nɯ kɯ ɯ-wa cʰo ɯ-pi nɯra ɲɤ-βde tɕe, sɯŋgɯ ɯ-ɕki tɕe jo-ɕe. \\
girl \textsc{dem} \textsc{erg} \textsc{3sg}.\textsc{poss}-father \textsc{comit} \textsc{3sg}.\textsc{poss}-elder.sibling \textsc{dem}:\textsc{pl} \textsc{ifr}-leave \textsc{lnk} forest \textsc{3sg}.\textsc{poss}-\textsc{dat} \textsc{loc} \textsc{ifr}-go \\
\glt `The girl left her father and her brothers, and went toward the forest.' (140506 shizi he huichang de bailingniao, 76)
\end{exe}

The subject of intransitive verbs with goals is in absolutive form, except when shared with a transitive verb in another clause, as \forme{tɕhemɤpɯ nɯ kɯ} in (\ref{ex:sWNgW.WCki.joCe}), which owes its ergative marking to the transitive verb \forme{ɲɤ-βde} `She left them'. The verb \japhug{rpu}{bump} (which takes as goal the surface of physical contact) however can take ergative subjects, as it is labile and can be conjugated transitively (§\ref{sec:goal.labile}).

Some verbs, such as \japhug{atɯɣ}{meet}, select an oblique comitative argument in \forme{cʰo} (§\ref{sec:comitative}).

\subsection{Invariable intransitive verbs}
%thɯ-nɯɕe ma, mɤ-tɯ-ra
\section{Transitive verbs}

\subsection{The morphological marking of transitivity in Japhug} \label{sec:transitivity.morphology}

%The past tense \forme{-t} suffix

\subsection{The direct-inverse system} \label{sec:direct-inverse}

\subsubsection{Mixed configurations}

\subsubsection{Non-local configurations}

\subsubsection{Local configurations}

\subsubsection{Double number indexation}  \label{sec:double.number.indexation}

\begin{landscape}
\begin{table}[H]
\caption{Japhug transitive and intransitive paradigms}\label{tab:japhug.tr}
\resizebox{\columnwidth}{!}{
\begin{tabular}{l|l|l|l|l|l|l|l|l|l|l|}
\textsc{} & 	\textsc{1sg} & 	  \textsc{1du} & 	\textsc{1pl} & 	\textsc{2sg} & 	\textsc{2du} & 	\textsc{2pl} & 	\textsc{3sg} & 	\textsc{3du} & 	\textsc{3pl} & 	\textsc{3'} \\ 	
\hline
\textsc{1sg} & \multicolumn{3}{c|}{\grise{}} &	\forme{} & 	\forme{} & 	\forme{} & 	\forme{\sigc{}-a}   & 	 \forme{\sigc{}-a-ndʑi} & 	 \forme{\sigc{}-a-nɯ} & 	\grise{} \\	
\cline{8-10}
\textsc{1du} & 	\multicolumn{3}{c|}{\grise{}} &	\forme{ta-\siga{}} & 	\forme{ta-\siga{}-ndʑi} & 	\forme{ta-\siga{}-nɯ} & 	\multicolumn{3}{c|}{ \forme{\siga{}-tɕi}}  & 	\grise{} \\	
\cline{8-10}
\textsc{1pl} & 	\multicolumn{3}{c|}{\grise{}} & 	  & 	&  & 	\multicolumn{3}{c|}{ \forme{\siga{}-ji}}  & 	\grise{} \\	
\hline
\textsc{2sg} & 	\forme{kɯ-\siga{}-a} & 	\forme{} & 	\forme{} & 	\multicolumn{3}{c|}{\grise{}}&	\multicolumn{3}{c|}{\forme{tɯ-\sigc{}}} & 	\grise{} \\	
\cline{2-2}
\cline{8-10}
\textsc{2du} & 	\forme{kɯ-\siga{}-a-ndʑi} & 	\forme{kɯ-\siga{}-tɕi} & 	\forme{kɯ-\siga{}-ji} & 	\multicolumn{3}{c|}{\grise{}} &	\multicolumn{3}{c|}{\forme{tɯ-\siga{}-ndʑi}} & 	\grise{} \\	
\cline{2-2}
\cline{8-10}
\textsc{2pl} & 	\forme{kɯ-\siga{}-a-nɯ} & 	\forme{} & 	\forme{} & 	\multicolumn{3}{c|}{\grise{}}&	\multicolumn{3}{c|}{\forme{tɯ-\siga{}-nɯ}} & 	\grise{} \\	
\hline
\textsc{3sg} &  	\forme{wɣɯ́-\siga{}-a} & 	\forme{} & 	\forme{} & 	\forme{} & 	\forme{} & 	\forme{} & \multicolumn{3}{c|}{\grise{}} &	\forme{\sigc{}} \\ 	
\cline{2-2}
\cline{11-11}
\textsc{3du} &  	\forme{wɣɯ́-\siga{}-a-ndʑi} & 	 \forme{wɣɯ́-\siga{}-tɕi} & 		\forme{wɣɯ́-\siga{}-ji} & 	\forme{tɯ́-wɣ-\siga{}} & 	\forme{tɯ́-wɣ-\siga{}-ndʑi} & 	\forme{tɯ́-wɣ-\siga{}-nɯ} & 	\multicolumn{3}{c|}{\grise{}} &	\forme{\siga{}-ndʑi} \\ 
\cline{2-2}	
\cline{11-11}
\textsc{3pl} &  	\forme{wɣɯ́-\siga{}-a-nɯ} & 	\forme{} & 	\forme{} & 	\forme{} & 	\forme{} & 	\forme{} & \multicolumn{3}{c|}{\grise{}} &	\forme{\siga{}-nɯ} \\ 	
\hline
\textsc{3'} & 	\multicolumn{6}{c|}{\grise{}} &	\forme{wɣɯ́-\siga{}} & 	\forme{wɣɯ́-\siga{}-ndʑi} & 	\forme{wɣɯ́-\siga{}-nɯ} & 	\grise{} \\	
	\hline	\hline
\textsc{intr}&\forme{\siga{}-a}&\forme{\siga{}-tɕi}&\forme{\siga{}-ji}&\forme{tɯ-\siga{}}&\forme{tɯ-\siga{}-ndʑi}&\forme{tɯ-\siga{}-nɯ}&\forme{\siga{}}&\forme{\siga{}-ndʑi} &\forme{\siga{}-nɯ}& 	\grise{} \\	
\hline
\end{tabular}}
\end{table}


\begin{table}[H]
\caption{The paradigm of the verb \japhug{mto}{see} in the Factual non-past}\label{tab:mto.paradigm}
\resizebox{\columnwidth}{!}{
\begin{tabular}{l|l|l|l|l|l|l|l|l|l|l|}
\textsc{} & 	\textsc{1sg} & 	  \textsc{1du} & 	\textsc{1pl} & 	\textsc{2sg} & 	\textsc{2du} & 	\textsc{2pl} & 	\textsc{3sg} & 	\textsc{3du} & 	\textsc{3pl} & 	\textsc{3'} \\ 	
\hline
\textsc{1sg} & \multicolumn{3}{c|}{\grise{}} &	\forme{} & 	\forme{} & 	\forme{} & 	\forme{mtam-a}   & 	 \forme{mtam-a-ndʑi} & 	 \forme{mtam-a-nɯ} & 	\grise{} \\	
\cline{8-10}
\textsc{1du} & 	\multicolumn{3}{c|}{\grise{}} &	\forme{ta-mto} & 	\forme{ta-mto-ndʑi} & 	\forme{ta-mto-nɯ} & 	\multicolumn{3}{c|}{ \forme{mto-tɕi}}  & 	\grise{} \\	
\cline{8-10}
\textsc{1pl} & 	\multicolumn{3}{c|}{\grise{}} & 	  & 	&  & 	\multicolumn{3}{c|}{ \forme{mto-j}}  & 	\grise{} \\	
\hline
\textsc{2sg} & 	\forme{kɯ-mto-a} & 	\forme{} & 	\forme{} & 	\multicolumn{3}{c|}{\grise{}}&	\multicolumn{3}{c|}{\forme{tɯ-mtɤm}} & 	\grise{} \\	
\cline{2-2}
\cline{8-10}
\textsc{2du} & 	\forme{kɯ-mto-a-ndʑi} & 	\forme{kɯ-mto-tɕi} & 	\forme{kɯ-mto-j} & 	\multicolumn{3}{c|}{\grise{}} &	\multicolumn{3}{c|}{\forme{tɯ-mto-ndʑi}} & 	\grise{} \\	
\cline{2-2}
\cline{8-10}
\textsc{2pl} & 	\forme{kɯ-mto-a-nɯ} & 	\forme{} & 	\forme{} & 	\multicolumn{3}{c|}{\grise{}}&	\multicolumn{3}{c|}{\forme{tɯ-mto-nɯ}} & 	\grise{} \\	
\hline
\textsc{3sg} &  	\forme{wɣɯ́-mto-a} & 	\forme{} & 	\forme{} & 	\forme{} & 	\forme{} & 	\forme{} & \multicolumn{3}{c|}{\grise{}} &	\forme{mtɤm} \\ 	
\cline{2-2}
\cline{11-11}
\textsc{3du} &  	\forme{wɣɯ́-mto-a-ndʑi} & 	 \forme{wɣɯ́-mto-tɕi} & 		\forme{wɣɯ́-mto-j} & 	\forme{tɯ́-wɣ-mto} & 	\forme{tɯ́-wɣ-mto-ndʑi} & 	\forme{tɯ́-wɣ-mto-nɯ} & 	\multicolumn{3}{c|}{\grise{}} &	\forme{mto-ndʑi} \\ 
\cline{2-2}	
\cline{11-11}
\textsc{3pl} &  	\forme{wɣɯ́-mto-a-nɯ} & 	\forme{} & 	\forme{} & 	\forme{} & 	\forme{} & 	\forme{} & \multicolumn{3}{c|}{\grise{}} &	\forme{mto-nɯ} \\ 	
\hline
\textsc{3'} & 	\multicolumn{6}{c|}{\grise{}} &	\forme{wɣɯ́-mto} & 	\forme{wɣɯ́-mto-ndʑi} & 	\forme{wɣɯ́-mto-nɯ} & 	\grise{} \\	
\hline
\end{tabular}}
\end{table}
\end{landscape}


\subsection{Ditransitive verbs}


\subsection{The function of the direct/inverse contrast in non-local configurations}

\subsection{An irregular verb}

\section{Labile verbs}
\subsection{Transitive-intransitive labile verbs}
\subsection{Transitive-intransitive labile verbs with oblique arguments} \label{sec:goal.labile}
%rpu
\subsection{Semi-transitive labile verbs}\label{sec:semi.tr.labile}
%sɤŋo, tso me, rga
\section{Dual and plural indexation}

\section{The historical relationshop between person indexation suffixes and possessive prefixes} \label{sec:indexation.suffixes.history}

\section{The origin of portmanteau prefixes}

\section{Person indexation on non-finite predicative words} \label{sec:non.finite.indexation}
 %Person indexation
%\chapter{Orientation and associated motion}
\section{Associated motion}
\subsection{Associated motion vs motion verb construction}
To express the meaning of motion prior to an action, associated motion prefixes are nearly two times as common as corresponding motion verb constructions (henceforth MVC) in the Japhug corpus. There is however a clear semantic difference between the two constructions, which was briefly described in \citet{jacques13harmonization}, but is presented here in more detail.

AM and MVC differ from each other in that in the former, the completion of both motion event and verbal action is presupposed (AM is monoactional), whereas in the case of the latter, the two can be separated. This mono- vs. pluractionality contrast is most conspicuous in past perfective forms, and can be observed in four types of constructions: concessives (with negation of the verbal action), interrogatives, conditionals and complement clauses.

\subsubsection{Concessive} \label{sec:am.concessive}
A MVC  with the motion verb in perfective form can be followed by a clause negating the purposive action, as in (\ref{ex:nAkWrtoR}). In this example, only the motion is realized, while the action expressed by the verb \japhug{rtoʁ}{look} could not be accomplished.

\begin{exe}
\ex \label{ex:nAkWrtoR}
\gll nɤ-kɯ-rtoʁ jɤ-ɣe-a ri, mɯ-nɯ-atɯɣ-tɕi, mɯ-pɯ-ta-mto. \\
\textsc{1sg.poss}-\textsc{nmlz}:S/A-see \textsc{pfv}-come[II]-\textsc{1sg} \textsc{lnk} \textsc{neg}-\textsc{pfv}-meet-\textsc{1du} \textsc{neg}-\textsc{pfv}-1\fl2-see \\
\glt `I came to see you but I did not see you.' 
\end{exe}

With the corresponding AM verb form \japhug{ɣɯ-jɤ-ta-rtoʁ}{I came to see you}, negating the action of the verb is self-contradictory and nonsensical, and a sentence such as (\ref{ex:GWjAtartoR}) is incorrect.

\begin{exe}
\ex \label{ex:GWjAtartoR}
\gll $\dagger$ɣɯ-jɤ-ta-rtoʁ ri mɯ-pɯ-ta-mto \\
\textsc{cisloc}-\textsc{pfv}-1\fl2-look \textsc{lnk} \textsc{neg}-\textsc{pfv}-1\fl2-see \\
\glt Intended meaning: `I came to see you but I did not see you.' 
\end{exe}

Additional minimal pairs of the same type are presented in \citet[202-203]{jacques13harmonization}.

Example (\ref{ex:mWjsWntsGe}) from a conversation illustrates this property also with a manipulative verb \japhug{ɣɯt}{bring}: the action of the purposive complement  \japhug{kɤ-ntsɣe}{to sell} is negated in the following clause (with an abilitative \forme{sɯ-}, see § XXX).

 \begin{exe}
\ex \label{ex:mWjsWntsGe}
 \gll   sɤnɤmmtsʰu kɯ kɤ-ntsɣe cʰɤ-ɣɯt ri mɯ́j-sɯ-ntsɣe ndɤre, \\
 p.n. \textsc{erg} \textsc{inf}-sell \textsc{ifr}:\textsc{downstream}-bring \textsc{lnk} \textsc{neg}:\textsc{sens}-\textsc{abil}-sell \textsc{lnk} \\
\glt `Bsod.nams.mtsho brought them (to Mbarkham) to sell, but could not sell it.' (conversation, 14.05.10)
 \end{exe}

 

\subsubsection{Interrogative} \label{sec:am.interrogative}
In interrogative clauses, MVCs are required to express meanings such as `What/who have you come/gone to X', as in example (\ref{ex:tChi.WkWpa}), an example which occurs nine times in the corpus.

\begin{exe}
\ex \label{ex:tChi.WkWpa}
\gll tɕʰi ɯ-kɯ-pa jɤ-tɯ-ɣe? \\
what \textsc{3sg.poss}-do \textsc{pfv}-2-come[II] \\
\glt `What did you come to do?' (nine examples in the corpus)
\end{exe}

The difference between MVC and AM in interrogatives can be illustrated by comparing the minimal pair  (\ref{ex:tChi.WkWndza}) and (\ref{ex:tChi.GWtAtWndzat}). example (\ref{ex:tChi.WkWndza}), which has the same structure as (\ref{ex:tChi.WkWpa}), implies that the addressee has not eaten yet, while (\ref{ex:tChi.GWtAtWndzat}) with associated motion can only be used if the food ingestion has already taken place, and requires a different translation.

\begin{exe}
\ex \label{ex:tChi.WkWndza}
\gll tɕʰi ɯ-kɯ-ndza jɤ-tɯ-ɣe? \\
what \textsc{3sg.poss}-eat \textsc{pfv}-2-come[II] \\
\glt `What have you come to eat?' (elicited)
\end{exe}

\begin{exe}
\ex \label{ex:tChi.GWtAtWndzat}
\gll tɕʰi ɣɯ-tɤ-tɯ-ndza-t \\
what \textsc{cisloc}-\textsc{pfv}-2-eat-\textsc{pst:tr}    \\
\glt `What did you eat, after you came here?' (elicited)
\end{exe}

\subsubsection{Conditional} \label{sec:am.conditional}
The presuppositional difference between MVC and AM is also perceptible in the protasis of conditional clauses. 

With MVC in the protasis as in (\ref{ex:mWmAjAtWGe}), there is no presupposition that the verbal action took place, only the motion event constitutes a condition to the state of affair described in the apodosis.

\begin{exe}
\ex \label{ex:mWmAjAtWGe}
\gll nɤ-wa ɯ-kɯ-rtoʁ mɯ\redp{}mɤ-jɤ-tɯ-ɣe nɤ aʑo mɯ-pɯ-kɯ-mto-a. \\
\textsc{1sg.poss}-father \textsc{3sg.poss-}\textsc{nmlz}:S/A-look \textsc{cond}\redp{}\textsc{neg}-\textsc{pfv}-2-come[II] \textsc{lnk} \textsc{1sg} \textsc{neg}-\textsc{pfv}-2\fl{}1-\textsc{1sg} \\
\glt `If you had not come to see your father, you would not have seen me.' (you saw me, but your father was not here)
\end{exe}

By contrast, with AM, the verbal action necessarily took place, as in example (\ref{ex:mWmAGWjAtWrtoR}).

\begin{exe}
\ex \label{ex:mWmAGWjAtWrtoR}
\gll nɤ-wa  mɯ\redp{}mɤ-ɣɯ-jɤ-tɯ-rtoʁ nɤ pɯ-sɤzdɯxpa \\
\textsc{1sg.poss}-father \textsc{cond}\redp{}\textsc{neg}-\textsc{cisloc}-\textsc{pfv}-2-look \textsc{lnk} \textsc{pst.ipfv}-be.pitiful \\ 
\glt `If you had not come to see your father, he would have felt sorry.' (but you did saw him, so he does not feel sorry)
\end{exe}

\subsubsection{Complement clauses} \label{sec:am.conditional}
In complement clauses, verbs with AM prefixes are attested, and complement taking verbs always have scope over both the action of the verb and motion event.

 
In (\ref{ex:mACWkAtshi}), the modal verb \japhug{cʰa}{can} and the double negations (with the specific meaning `cannot help') have scope over both the motion event and the verbal action -- this example is taken from a passage in a story where the king reproaches a small child, who just returned from a mission he himself send him to accomplish, not to have first come to greet him on his return home; the child says these words to justify why he first went to see his mother before greeting the king -- from this context it is clear that both the motion event (to him mother's house, explaining the child's failure to go to see the king) and the action `drink milk' (the reason for that motion event) are equally important to the plot and inseparable. 

\begin{exe}
\ex \label{ex:mACWkAtshi}
\gll  tɯ-nɯ ɯ-kɯ-tsʰi ɲɯ-ɕti-a tɕe, jɤ-azɣɯt-a tɕe, tɯ-nɯ ci mɤ-ɕɯ-kɤ-tsʰi nɯ mɯ́j-cʰa-a \\
\textsc{indef}.\textsc{poss}-breast \textsc{3sg}.\textsc{poss}-\textsc{nmlz}:S/A-drink \textsc{sens}-be.\textsc{affirm}-\textsc{1sg} \textsc{lnk} \textsc{pfv}-arrive-\textsc{1sg} \textsc{lnk} \textsc{indef}.\textsc{poss}-breast  \textsc{indef} \textsc{neg}-\textsc{transloc}-\textsc{inf}-drink \textsc{dem} \textsc{neg}:\textsc{sens}-can-\textsc{1sg} \\
\glt `I am (a toddler) who (still) drinks (his mother's) milk, when I arrived, I could not help but go to drink milk.'  (Norbzang, 262)
 \end{exe}
 
 In (\ref{ex:CWkAmWrkW.mAtWcha}), the negated modal verb has also on the action of both the main verb and the motion event -- the guards would prevent the main character not only to steal, but also to the where the object to be stolen is found.
 
\begin{exe}
\ex \label{ex:CWkAmWrkW.mAtWcha}
\gll ʁmaʁ χsɯ-tɤkʰar kɯ ɲɯ-ɤz-nɤkʰar-nɯ ɕti tɕe, ɕɯ-kɤ-mɯrkɯ mɤ-tɯ-cʰa  \\
solider three-rounds \textsc{erg} \textsc{sens}-\textsc{prog}-surround-\textsc{pl} be.\textsc{affirm}:\textsc{fact} \textsc{lnk}  \textsc{transloc}-\textsc{inf}-steal \textsc{neg}-2-can:\textsc{fact} \\
\glt `Three rounds of soldiers will be surrounding it, you will not be able to (go there and) steal it.' (2003qachga, 55)
   \end{exe}
 
The same observation also applies to  verbs with AM in complement clauses selected by a verb in the protasis, as in (\ref{ex:CWkAru}) and (\ref{ex:CWkAmWrkW}): the realization of the verbal action (in addition to that of the motion event) belongs to the condition.

\begin{exe}
\ex \label{ex:CWkAru}
\gll ɕɯ-kɤ-ru mɯ\redp{}mɤ-pɯ-tɯ-cha ŋu nɤ nɤ-srɤm nɤ-sroʁ lɤt-i \\
\textsc{transloc}-\textsc{inf}-bring \textsc{cond}\redp{}\textsc{neg}-2-can:\textsc{fact} be:\textsc{fact} \textsc{lnk} \textsc{1sg.poss}-root \textsc{1sg.poss}-life throw:\textsc{fact}-\textsc{1pl} \\ 
\glt `If you are not able to bring it here, we will have your root and your life.' (Norbzang, 10)
\end{exe}

\begin{exe}
\ex \label{ex:CWkAmWrkW}
\gll nɤʑo ɕɯ-kɤ-mɯrkɯ a-pɯ-tɯ-cʰa nɤ aʑo cʰɯ-sɯ-jɣat-a jɤɣ \\
\textsc{2sg} \textsc{transloc}-\textsc{inf}-steal \textsc{irr}-\textsc{ipfv}-2-can \textsc{lnk} \textsc{1sg} \textsc{ipfv}-\textsc{caus}-go.back-\textsc{1sg} be.agreed:\textsc{fact} \\
\glt `If you succeed stealing it (after having gone there), I can cause him to go back there.' (02-montagnes-kamnyu, 46)
\end{exe}

By contrast, in  (\ref{ex:kWrAma.kACe}), in the case of the infinitival complement \forme{kɯ-rɤma kɤ-ɕe} `go to work' with a purposive clause \forme{kɯ-rɤma} (§ XXX), the main verb \japhug{mda}{be time to} only has scope over the motion event expressed by the verb \japhug{ɕe}{go} -- the time that is indicated by the stars refers to the beginning of the journey to work, not the start of the work itself.
 
 \begin{exe}
\ex \label{ex:kWrAma.kACe}
\gll  tɕe kɯɕɯŋgɯ tɕe tɯtsʰot pɯ-me tɕe  nɯnɯ cʰɯ-ɬoʁ lu-ɕqʰlɤt nɯra ɕ-tu-kɯ-ru tɕe, nɯnɯ kɤ-rɤru mda mɤ-mda cʰondɤre kɯ-rɤma kɤ-ɕe mda mɤ-mda nɯtɕu ɕ-tu-kɯ-ru pɯ-ŋgrɤl. \\
 \textsc{lnk} long.ago \textsc{lnk} clock \textsc{pst}.\textsc{ipfv}-not.exist \textsc{lnk} \textsc{dem} \textsc{ipfv}:\textsc{downstream}-come.out \textsc{ipfv}:\textsc{upstream}-disappear \textsc{dem}:\textsc{pl} \textsc{transloc}-\textsc{ipfv}:up-\textsc{genr}:S/P-look \textsc{lnk} \textsc{dem} \textsc{inf}-get.up be.time:\textsc{fact} \textsc{neg}-be.time:\textsc{fact} \textsc{comit} \textsc{nmlz}:S/A-work \textsc{inf}-go be.time:\textsc{fact} \textsc{neg}-be.time:\textsc{fact} \textsc{dem}:\textsc{loc} \textsc{transloc}-\textsc{ipfv}:up-\textsc{genr}:S/P-look \textsc{pst}.\textsc{ipfv}-be.usually.the.case  \\
 \glt  `In former times, there was no clock, and people used to watch when (these stars) came out or disappeared (to know) whether it was time to get up or go to work.' (29-LAntshAm, 66)
  \end{exe}
   %Orientation and associated motion
%\chapter{Non-finite verbal morphology}

\section{Participles}
Participles are nominalized verb forms that keep some verbal characteristics: they can serve as predicates of subordinate clauses (relative or complement clauses), take TAM, polarity and associated motion marking, and preserve the verb's argument structure.

Participles differ from finite verbs in three ways. First, they cannot serve as the predicate of a main clause. Second, they are not compatible with the personal prefixes and suffixes of the intransitive and transitive conjugations (including direct/inverse marking and past transitive \forme{-t-}, § XXX).\footnote{Japhug is identical in this regard to Tshobdun and Zbu, but crucially differs from Situ, where nominalized forms in \forme{kə-} can bear indexation suffixes (\citet{jackson06guanxiju}, \citet{jacksonlin07}) } Rather, like nouns, they can take a possessive prefix which can be coreferent with one the arguments. Due to the general impossibility of stacking possessive prefixes (§ \ref{sec:possessive.paradigm}), at most only one argument can be indexed this way. Third, there are restrictions on TAM marking on participles: they have at most three forms (neutral, perfective and imperfective), and completely lack inferential (§ XXX), egophoric (§ XXX), sensory (§ XXX) and progressive forms (§ XXX).

There are three participles in Japhug; the subject S/A participle in \forme{kɯ-}, the object participle in \forme{kɤ-} and the oblique participle in \forme{sɤ-}. 

Complex participial forms, including negative, associated motion or TAM prefixes are possible, as shown by example \ref{ex:WGWjAkWqru}. However, never more than four inflexional prefixes are found; forms with all five prefixal slots filled (such as $\dagger$\forme{ɯ-ɣɯ-jɤ-kɯ-qru}) are not accepted by Tshendzin.

 \begin{exe}
\ex \label{ex:WGWjAkWqru}
\gll ɯ-ɣɯ-jɤ-kɯ-qru tɤ-tɕɯ  \\
  \textsc{3sg}-\textsc{cisloc}-\textsc{pfv}-\textsc{nmlz}:S/A-meet \textsc{indef}.\textsc{poss}-boy   \\
\glt `The boy who had come to look for her.' (The three sisters, 231)
 \end{exe}

Table \ref{tab:template.nmlz} summarizes the template of participial verb forms; more details are provided on possible and attested forms for each participle type in the following sections.

\begin{table}[h]
\caption{The template of participial verb forms in Japhug} \centering \label{tab:template.nmlz}
\resizebox{\columnwidth}{!}{
\begin{tabular}{lllllll}
\toprule
-5 & -4&-3 &-2&-1& \ro{} \\
possessive & negative&associated   & TAM & participle prefix &enlarged  \\
prefix & prefix &motion prefix  &orientation&&stem\\
\bottomrule
\end{tabular}}
\end{table}

Stem alternation is reduced in participle forms: stem 3 (§ XXX) never occurs. The few verbs that have a distinct stem 2 (\japhug{ɕe}{go}, \japhug{ɣi}{come}, \japhug{ti}{say} and derived forms) however, use this stem in subject and object participles with perfective orientational prefixes (§ XXX), in forms like \forme{jɤ-kɯ-ɣe} \textsc{pfv}-\textsc{nmlz}:S/A-come[II] `the one who came'
or \forme{tɤ-kɤ-tɯt} \textsc{pfv}-\textsc{nmlz}:P-say[II] `what was said'.
 
\subsection{Subject participles} \label{sec:subject.participles}
The subject participle, built by adding the prefix \forme{kɯ-} to the verb stem, designates an entity corresponding to the intransitive subject (\ref{ex:kWsi}, § \ref{sec:absolutive.S} and XXX), a possessor of the subject, or the transitive subject (\ref{ex:WkWndza}, § \ref{sec:A.kW}, § XXX) of the base verb. 

 \begin{exe}
\ex \label{ex:kWsi}
\gll kɯ-si  \\
  \textsc{nmlz}:S/A-die \\
 \glt  `The dead one' (many attestations)
\end{exe}

 \begin{exe} 
\ex \label{ex:WkWndza}
\gll ɯ-kɯ-ndza \\
  \textsc{3sg}-\textsc{nmlz}:S/A-eat \\
 \glt  `The one who eats it.' (many attestations)
\end{exe}

With \forme{a-} initial verbs the \forme{kɯ-} prefix regularly merge with \forme{a-} as \forme{kɤ-}, a form which resembles an object participle (but there is no ambiguity since all \forme{a-} initial verbs are intransitive, § XXX). For instance, the subject participle of the semi-transitive \japhug{aro}{have} is \forme{kɯ-ɤro} \ipa{kɤro} `having, the one who has'.

The subject participle \forme{kɯ-} prefix is historically related to that of object participles (§ \ref{sec:object.participle}), velar infinitives (§ \ref{sec:velar.inf}) and deverbal nouns in \forme{x-/ɣ-} (§ \ref{sec:G.nmlz}), and has cognates elsewhere in the family (§ \ref{sec:velar.nmlz.history}).

In this section, I discuss first morphological issues (possessive prefixes § \ref{sec:subject.participle.ambiguities}, other prefixes § \ref{sec:subject.participle.other.prefixes} and ambiguous forms § \ref{sec:subject.participle.ambiguities}), and then present the various functions of subject participles, including participial relatives (§ \ref{sec:subject.participle.subject.relative} and § \ref{sec:subject.participle.other.relative}), complementation strategies (\ref{sec:subject.participle.complementation}), as well as the case of lexicalized participles (§ \ref{sec:lexicalized.subject.participle}).
 
Examples which could be potentially be viewed as subject participles in converbial use are analyzed as \forme{kɯ-} infinitives (§ \ref{sec:inf.converb}).

\subsubsection{Possessive prefixes on subject participles}  \label{sec:subject.participle.possessive}

In the case of transitive verbs, a possessive prefix coreferent with the object is obligatory when no overt object is present (\textsc{3sg} \forme{ɯ-} in \ref{ex:WkWndza}), and when no other prefix is added to the participle.

When another prefix (polarity, associated motion or orientation prefix) is present, the possessive prefix is optional, as shown by forms like \forme{mɤ-kɯ-ndza} `the one which does not eat (it)' in (\ref{ex:mAkWndza}), as opposed to \forme{ɯ-mɤ-kɯ-mto} `the one who does not see it' in (\ref{ex:WmAkWmto}) with both possessive \forme{ɯ-} and the negative prefix \forme{mɤ-}.

 \begin{exe} 
\ex \label{ex:mAkWndza}
\gll  tɤ-mtʰɯm ʁɟa ʑo ma nɯ ma, nɤki, tɯjpu mɤ-kɯ-ndza ci tu tɕe, \\
\textsc{indef}.\textsc{poss}-meat completely \textsc{emph} \textsc{lnk} \textsc{dem} apart.from \textsc{filler} flour.based.food \textsc{neg}-\textsc{nmlz}:S/A-eat \textsc{indef} exist:\textsc{fact} \textsc{lnk} \\
\glt  `There is (an animal like the mouse) which only eats meat, not food made from flour.' (27-spjaNkW, 202-2063)
\end{exe}

 \begin{exe} 
\ex \label{ex:WmAkWmto} 
\gll  li nɯnɯ kɯnɤ ɯ-kɯ-mto ɣɤʑu, ɯ-mɤ-kɯ-mto ɣɤʑu. \\
again \textsc{dem} also \textsc{3sg}.\textsc{poss}-\textsc{nmlz}:S/A-see exist:\textsc{sens} \textsc{3sg}.\textsc{poss}-\textsc{neg}-\textsc{nmlz}:S/A-see exist:\textsc{sens} \\
\glt `There are (people) who see (find) it, and people who don't.' (20-sWrna, 20)
\end{exe}

In the case of ditransitive verbs, the possessive prefix striclty refer to the object. With indirective verbs like \japhug{tʰu}{ask}, the possessive prefix is necessarily the theme, never the recipient. The form in (\ref{ex:AkWthu}) thus cannot be interpreted as meaning `the one who asks me (about it)'; the correct construction would be (\ref{ex:ACki.kWthu}), with the recipient in the dative case.

\begin{exe}
\ex \label{ex:AkWthu}
\gll a-kɯ-tʰu  \\
\textsc{1sg.poss}-\textsc{nmlz}:S/A-ask \\
\glt `The one asking for me (in marriage).' (elicited)
\ex \label{ex:ACki.kWthu}
\gll a-ɕki ɯ-kɯ-tʰu  \\
\textsc{1sg.poss}-\textsc{dat} \textsc{3sg}.\textsc{poss}-\textsc{nmlz}:S/A-ask \\ 
\glt `The one who asks me.' 
\end{exe}

With secundative verbs (§ XXX), the possessive prefix of the subject participle is obligatorily coreferent with the recipient, not the theme, as in (\ref{ex:nAkWmbi}).

\begin{exe}
\ex \label{ex:nAkWmbi}
\gll nɯ ma nɤ-kɯ-mbi me \\
\textsc{dem} apart.from \textsc{2sg}.\textsc{poss}-\textsc{nmlz}:S/A-give not.exist:\textsc{fact} \\
\glt `Nobody will give you another (daughter in marriage).' (2002qaCpa, 57)
\end{exe}
With intransitive verbs, including adjectival stative verbs (§ XXX), a possessive prefix can also be added. In the case of semi-transitive verbs (§ XXX), the possessive can refer to the semi-object (§ \ref{sec:semi.object}), as in example (\ref{ex:WkWrga.pWdAn}).

 \begin{exe} 
\ex \label{ex:WkWrga.pWdAn}
\gll  nɯ ɕɯŋgɯ tɕe, ɯ-kɯ-rga pɯ-dɤn. \\
\textsc{dem} before \textsc{lnk} \textsc{3sg}.\textsc{poss}-\textsc{nmlz}:S/A-like \textsc{pst}.\textsc{ipfv}-be.many \\
\glt  `Before, there used to be many people who liked it.' (12-Zmbroko, 112)
\end{exe}

It can also refer to the beneficiary (which is normally marked with genitive or possessive prefixes, see § \ref{sec:other.uses.poss.prefixes} and § \ref{sec:gen.beneficiary}), as in (\ref{ex:tWZo.tWkWpe}) and (\ref{ex:aZo.akWra}).

 \begin{exe} 
\ex \label{ex:tWZo.tWkWpe}
\gll  kɯ-pe tú-wɣ-nɤma tɕe li tɯʑo tɯ-kɯ-pe tu \\
\textsc{nmlz}:S/A-be.good \textsc{ipfv}-\textsc{inv}-make \textsc{lnk} again \textsc{genr} \textsc{genr}.\textsc{poss}-\textsc{nmlz}:S/A-be.good exist:\textsc{fact} \\
\glt  `If one does good things, one will also have good things.' (140518 mao he laoshu-zh, 124)
\end{exe}

 \begin{exe} 
\ex \label{ex:aZo.akWra}
\gll  aʑo a-kɯ-ra nɯra a-tɤ-tɯ-ste qʰendɤre aʑo nɯnɯ, nɤki, ku-nɤtsi-a jɤɣ \\
\textsc{1sg} \textsc{1sg}.\textsc{poss}-\textsc{nmlz}:S/A-have.to \textsc{dem}:\textsc{pl} \textsc{irr}-\textsc{pfv}-2-do.like[III] \textsc{lnk} \textsc{dem} \textsc{filler} \textsc{ipfv}-hide[III]-\textsc{1sg} be.possible:\textsc{fact}  \\
\glt  `If you do the things I need, I will keep it secret.'  (2014-kWlAG, 247)
\end{exe}

Since participles are also noun-like, the possessive prefixes can be real possessive, and be preceded with a genitive phrase as in (\ref{ex:tCaXpa.ra.GW.nWkWmna}) with \forme{nɯ-kɯ-mna} `the best among them' = `their chief'.

 \begin{exe} 
\ex \label{ex:tCaXpa.ra.GW.nWkWmna}
\gll tɕaχpa ra ɣɯ nɯ-kɯ-mna nɯ wuma ʑo pjɤ-nɯrɤŋom. \\
bandit \textsc{pl} \textsc{gen} \textsc{3pl}.\textsc{poss}-\textsc{nmlz}:S/A-be.better \textsc{dem} really \textsc{emph} \textsc{ifr}-be.upset \\
\glt `The chief of the bandits was very upset.' (140512 alibaba-zh, 195)
\end{exe}

This construction is used as a type of superlative, as in (\ref{ex:thamtCAt.GW.nWkWmpCAr}), where \forme{pɣa tʰamtɕɤt ɣɯ nɯ-kɯ-mpɕɤr nɯ}, literally meaning `the beautiful one (among/of) all birds' is to be understood as `the most beautiful of all birds.' (see § XXX on superlative constructions).

 \begin{exe} 
\ex \label{ex:thamtCAt.GW.nWkWmpCAr}
\gll tɕe pɣa tʰamtɕɤt ɣɯ nɯ-kɯ-mpɕɤr nɯ rmɤβja ɲɯ-ŋu.  \\
\textsc{lnk} bird all \textsc{gen} \textsc{3pl}.\textsc{poss}-\textsc{nmlz}:S/A-be.beautiful \textsc{dem} peacock \textsc{sens}-be \\
\glt `The peacock is the most beautiful of all birds.' (24-ZmbrWpGa, 84)
\end{exe}

\subsubsection{Associated motion, polarity and orientation prefixes on subject participles}  \label{sec:subject.participle.other.prefixes}
Of all non-finite verb forms, subject participles allow the richest possible combinations of inflexional prefixes: associated motion (§ \ref{sec:associated.motion}, example \ref{ex:WCWkWphWt}) below with the translocative \forme{ɕɯ-}), polarity (§ XXX, see \ref{ex:WmAkWmto} above) and orientation prefixes marking TAM (§ XXX) all can be prefixed. 
 
\begin{exe}
\ex \label{ex:WCWkWphWt}
 \gll tɕeri nɯra ɯ-ɕɯ-kɯ-pʰɯt ra kɯ-tu me ma,   \\
 \textsc{lnk} \textsc{dem}:\textsc{pl} \textsc{3sg}.\textsc{poss}-\textsc{transloc}-\textsc{nmlz}:S/A-cut \textsc{pl} \textsc{nmlz}:S/A-exist not.exist:\textsc{fact} \textsc{lnk} \\
 \glt `But nobody goes to collect (its stalks).' (11-paRzwamWntoR, 90)
\end{exe}

Two of the four series of orientation prefixes are possible with subject participles. With series A prefixes (\forme{tɤ-} `up', \forme{pɯ-} `down' etc, § XXX), the participle of dynamic verbs is perfective as \forme{tʰɯ-kɯ-ɣe} `the one who came' in (\ref{ex:WkWntsGe.thWkWGe}), and takes stem II (§ XXX). With series B prefixes (\forme{tu-} `up', \forme{pjɯ-} `down' etc, § XXX), it has a habitual imperfective meaning with dynamic verbs as \forme{ju-kɯ-ɣi} `the one who (usually) comes' in (\ref{ex:WkWndza.jukWGi}).\footnote{These two examples also illustrate the use of subject participles as purposive complements with the forms \forme{ɯ-kɯ-ntsɣe} and \forme{ɯ-kɯ-ndza} (see § \ref{sec:subject.participle.complementation}, § XXX).} The prefixes \forme{ɲɯ-}and \forme{ku-} do appear on subject participles, but only to express imperfective: there are no egophoric (§ XXX) or sensory (§ XXX) subject partiples.

\begin{exe}
\ex \label{ex:WkWntsGe.thWkWGe}
\gll iɕqʰa qaʑo ɯ-kɯ-ntsɣe tʰɯ-kɯ-ɣe nɯ ɯ-pʰe \\
the.aforementioned sheep \textsc{3sg}.\textsc{poss}-\textsc{nmlz}:S/A-sell \textsc{pfv}:\textsc{downstream}-\textsc{nmlz}:S/A-come[II] \textsc{dem} \textsc{3sg}.\textsc{poss}-\textsc{dat} \\
\glt  `(He told) the person who had come to sell the sheep.' (2003kandZislama, 212)
\end{exe}

\begin{exe}
\ex \label{ex:WkWndza.jukWGi}
\gll ɯ-kɯ-ndza ju-kɯ-ɣi nɯ pɣa ci ɲɯ-ŋu \\
\textsc{3sg}.\textsc{poss}-\textsc{nmlz}:S/A-eat \textsc{ipfv}-\textsc{nmlz}:S/A-come dem bird indef sens-be \\
\glt   `The one who comes to eat (the fruits) is a bird.' (2012qachGa, 22)
\end{exe}

The participles of stative verbs with series A and B orientation prefixes have an inchoative meaning, exactly like their finite counterpart (§ XXX and § XXX).  In (\ref{ex:YWkWjpum}) for instance, the imperfective participle \forme{ɲɯ-kɯ-jpum} from \japhug{jpum}{be thick} means `the one which becomes thicker', as opposed to the basic participle \forme{kɯ-jpum} `the thick one'.

\begin{exe}
\ex \label{ex:YWkWjpum}
 \gll ndʑu ɯ-ku jamar ɲɯ-kɯ-jpum ɣɤʑu nɤ, kɯ-wxti.  \\
 chopsticks \textsc{3sg}.\textsc{poss}-head about \textsc{ipfv}-\textsc{nmlz}:S/A-be.thick exist:\textsc{sens} \textsc{sfp} \textsc{nmlz}:S/A-be.big \\
 \glt  `There are (maggots) that grow as thick as the tip of a chopstick, the big ones.' (25-akWzgumba, 80)
\end{exe}

Imperfective participles of stative adjectival verbs are also appropriate to describe the gradient variation of a property across space rather than time. For instance, in (\ref{ex:YWkWjpum2}), the imperfective subject participles \forme{ku-kɯ-xtsʰɯm} and \forme{ɲɯ-kɯ-jpum} are used not to indicate a change across time, but to describe the shape of the gourd, which is progressively thinner towards the top and thicker towards the bottom.

\begin{exe}
\ex \label{ex:YWkWjpum2}
 \gll  tɕe ɯ-mat nɯnɯ, ɯ-taʁ ku-kɯ-xtsʰɯm, ɯ-pa ɲɯ-kɯ-jpum ci cʰɯ-βze ɲɯ-ŋu tɕe, nɯ <hulu> tu-sɤrmi-nɯ. \\
 \textsc{lnk} \textsc{3sg}.\textsc{poss}-fruit \textsc{dem} \textsc{3sg}.\textsc{poss}-up \textsc{ipfv}-\textsc{nmlz}:S/A-be.thin \textsc{3sg}.\textsc{poss}-down \textsc{ipfv}-\textsc{nmlz}:S/A-be.thick \textsc{indef} \textsc{ipfv}-make[III]  \textsc{sens}-be \textsc{lnk} gourd \textsc{ipfv}-call-\textsc{pl} \\
 \glt `It grows a fruit that is thinner (in diameter) on the upper part, and thicker on the lower part, people call it `gourd'.' (150825 huluwa, 3)
\end{exe}

The past imperfective of stative verbs is built using the series A prefix \forme{pɯ-} as in the corresponding finite forms (§ XXX). For instance, the past imperfective participle of \japhug{ŋu}{be} is \forme{pɯ-kɯ-ŋu} `the one who used to be ....', as in (\ref{ex:pWkWNu}).

\begin{exe}
\ex \label{ex:pWkWNu}
 \gll  ɯʑɤɣ nɯ ɕɯŋgɯ ɯ-nmaʁ pɯ-kɯ-ŋu tsʰɯraŋ nɯ pjɤ-mto \\
 \textsc{3sg}:\textsc{gen} \textsc{dem} before \textsc{3sg}.\textsc{poss}-husband \textsc{pst}.\textsc{ipfv}-\textsc{nmlz}:S/A-be p.n. \textsc{dem} \textsc{ifr}-see \\
\glt `She saw Tshering, who used to be her husband.' (qajdoskAt2002, 101)
\end{exe}

Most examples in the corpus have one or two prefixes, either combining a possessive prefix with another prefix (as in \ref{ex:WmAkWmto} and \ref{ex:WCWkWphWt}), or combining a negative prefix with an orientation prefix, as in (\ref{ex:mWnWkWsna}).

 \begin{exe}
\ex \label{ex:mWnWkWsna}
 \gll tɕe kʰa ɣɯ ɯ-ndzɤtsʰi ɯ-ro nɯ-kɯ-ri nɯra, mɯ-nɯ-kɯ-sna nɯra, nɯra paʁ kɯ ʁɟa tu-ndze ɲɯ-ŋu \\
 \textsc{lnk} house \textsc{gen} \textsc{3sg}.\textsc{poss}-food \textsc{3sg}.\textsc{poss}-excess \textsc{pfv}-\textsc{nmlz}:S/A-left \textsc{dem}:\textsc{pl}  \textsc{neg}-\textsc{pfv}-\textsc{nmlz}:S/A-be.good \textsc{dem}:\textsc{pl} \textsc{dem}:\textsc{pl} pig \textsc{erg} completely  \textsc{ipfv}-eat[III] \textsc{sens}-be \\
 \glt  `Food from the house that has been left over, or which is not good any more, pigs eat all of it.' (05-paR, 33)
\end{exe}

Subject participles with three prefixes before the participle prefix \forme{kɯ-} are possible, but attestations are rare. Example (\ref{ex:WGWjAkWqru}) above shows the combination of a possessive, an associated motion and an orientation prefixes (\forme{ɯ-ɣɯ-jɤ-kɯ-qru} `the one who had come to meet/look for her'), and (\ref{ex:WmApjWkWnWfkAB}) below that of a possessive, a polarity and an orientation prefixes.

\begin{exe}
\ex \label{ex:WmApjWkWnWfkAB}
 \gll ɯ-pjɯ-kɯ-nɯ-fkaβ tu, ɯ-mɤ-pjɯ-kɯ-nɯ-fkaβ tu ri nɯ kɯ-fse tu-nɯ-ndza-nɯ ɕti. \\
 \textsc{3sg}.\textsc{poss}-\textsc{ipfv}-\textsc{nmlz}:S/A-\textsc{auto}-cover exist:\textsc{fact}  \textsc{3sg}.\textsc{poss}-\textsc{neg}-\textsc{ipfv}-\textsc{nmlz}:S/A-\textsc{auto}-cover exist:\textsc{fact} \textsc{lnk} \textsc{dem} \textsc{nmlz}:S/A-be.like \textsc{ipfv}-\textsc{auto}-eat-\textsc{pl} be.\textsc{affirm}:\textsc{fact} \\
 \glt `There are people who cover it (with a lid while cooking), and people who don't, they eat it like that.' (23-mbrAZim, 22-23)
\end{exe}

In addition to imperfective orientation prefixes as in (\ref{ex:WmApjWkWnWfkAB}), it is possible for subject participles to combine possessive prefixes with \textit{perfective} orientation prefixes, as in (\ref{ex:WnWkWCar}). Subject participles are the only non-finite form attested with such a combination: the object participles do not allow combination of possessive and orientation prefixes (§ \ref{sec:object.participle.possessive}) and the oblique participles cannot take perfective orientation prefixes (§ \ref{sec:oblique.participle.orientation}).

 \begin{exe} 
\ex \label{ex:WnWkWCar}
\gll nɯnɯ ɯ-nɯ-kɯ-ɕar ɯ-pɯ-kɯ-mto nɯ ɣɯ ɲɯ-tʂaŋ ma nɯnɯ, nɤkinɯ, ɯ-ɕɯ-kɯ-βɟi nɯ ɣɯ mɯ́j-tʂaŋ \\
\textsc{dem} \textsc{3sg}.\textsc{poss}-\textsc{pfv}-\textsc{nmlz}:S/A-search \textsc{3sg}.\textsc{poss}-\textsc{pfv}-\textsc{nmlz}:S/A-see \textsc{dem} \textsc{gen} \textsc{sens}-be.fair \textsc{lnk} \textsc{dem} \textsc{filler} \textsc{3sg}.\textsc{poss}-\textsc{transloc}-\textsc{nmlz}:S/A-chase \textsc{dem} \textsc{gen} \textsc{neg}:\textsc{sens}-be.fair \\
\glt `It is fair that she would (be given) to the one who looked for her and found her, not to the ones chasing her.' (140517 buaishuohua, 123)
\end{exe}

The negative prefix has the form \forme{mɯ-} when occurring with a perfective orientation prefix as \forme{mɯ-nɯ-kɯ-sna} `the one that is not good anymore' in (\ref{ex:mWnWkWsna}) and \forme{mɤ-} when no orientation prefix is present (examples \ref{ex:WmAkWmto} and \ref{ex:mAkWndza} above). With the imperfective orientation prefixes, the allomorph \forme{mɤ-} occurs when preceded by a possessive prefix (§ \ref{ex:WmApjWkWnWfkAB}) and \forme{mɯ-} is found when no possessive prefix is present: compare the elicited forms (\ref{ex:WmAkukWtshi}) and (\ref{ex:mWkukWtshi}). The allomorphs of the negative prefix are not in free variation: forms such as $\dagger$\forme{ɯ-mɯ-ku-kɯ-tsʰi} or $\dagger$\forme{mɤ-ku-kɯ-tsʰi} would be incorrect in Kamnyu Japhug.

 \begin{exe} 
\ex \label{ex:WmAkukWtshi}
\gll ɯ-mɤ-ku-kɯ-tsʰi \\
\textsc{3sg}.\textsc{poss}-\textsc{neg}-\textsc{ipfv}-\textsc{nmlz}:S/A-drink \\
\ex \label{ex:mWkukWtshi}
\gll mɯ-ku-kɯ-tsʰi \\
\textsc{neg}-\textsc{ipfv}-\textsc{nmlz}:S/A-drink \\
\glt  `The one who drinks it' (elicited)
\end{exe}

There are no constraints on the number of derivational prefixes in participial forms. The derivational prefixes are all closer to the verb root than the participle prefix \forme{kɯ-}, and thus follow it as shown by (\ref{ex:WmApjWkWnWfkAB}), where the autobenefactive \forme{-nɯ-}, the leftmost of all derivational prefixes (§ XXX), is placed after \forme{kɯ-}. 

Aside from possessive, orientation, associated motion and polarity prefixes, subject participles can also receive the conative prefix \forme{jɯ-} as in (\ref{ex:jWtukWwGrum}).

 \begin{exe} 
\ex \label{ex:jWtukWwGrum}
\gll  kɯ-pɣi ci koŋla ʑo zɯmi jɯ-tu-kɯ-wɣrum kɯ-fse ci ɲɯ-ŋu.  \\
\textsc{nmlz}:S/A-be.grey \textsc{indef} completely \textsc{emph} almost \textsc{conat}-\textsc{ipfv}-\textsc{nmlz}:S/A-be.white \textsc{nmlz}:S/A-be.like \textsc{indef} \textsc{sens}-be \\
\glt `It is grey, almost like it is about to become white.'  (24-ZmbrWpGa, 34)
\end{exe}

Subject participles can undergo totalitative reduplication (§ XXX), which applies to the first syllable of the word, whether it is the participle \forme{kɯ-} or an orientation prefix as in (\ref{ex:jWjAkWGe}), meaning `all of those who/that X'.

\begin{exe}
\ex \label{ex:jWjAkWGe}
\gll tɕe nɯnɯ ɯ-taʁ jɯ\redp{}jɤ-kɯ-ɣe nɯ ku-ndɤm ɲɯ-ŋu. \\
\textsc{lnk} \textsc{dem} \textsc{3sg}.\textsc{poss}-on \textsc{total}\redp{}\textsc{pfv}-\textsc{nmlz}:S/A-come[II] \textsc{ipfv}-take[III] \textsc{sens}-be \\
\glt `(The spider) catches all of the (insects) that have come on (the web).' (26-mYaRmtsaR, 108)
\end{exe}

The totalitative subject participle of the existential verb \japhug{tu}{exist} can take a possessive prefix, which is interpreted as a possessor, as in \forme{a-kɯ\redp{}kɯ-tu} \textsc{1sg}.\textsc{poss}-\textsc{total}\redp{}\textsc{nmlz}:S/A-exist `everything that I have'. No other totalitative verb form allows possessor prefixation.

\subsubsection{Ambiguities}  \label{sec:subject.participle.ambiguities}
The subject participle \forme{kɯ-} prefix is homophonous with the generic person marker for intransitive subject and object (§ XXX; note that these two prefixes are probably historically related). In the case of intransitive verbs, some subject participles are therefore homophonous with generic person forms. 

For instance, the past imperfective generic \forme{pɯ-kɯ-ŋu} `one used to be' in (\ref{ex:pWkWNu.genr}) is identical to the past imperfective participle \forme{pɯ-kɯ-ŋu} `the one who used to be ....', discussed above (example \ref{ex:pWkWNu} in § \ref{sec:subject.participle.other.prefixes}). In this example, it is obvious that \forme{kɯ-} is the generic person marker because the verb \forme{pɯ-kɯ-rga} `one used to be' occurs as main verb; outside of any context,  \forme{tɤ-pɤtso pɯ-kɯ-ŋu} could be understood as a relative clause `the one who used to be a child', but this is not the meaning of this sentence. 

\begin{exe}
\ex \label{ex:pWkWNu.genr}
 \gll tɕeri tɤ-pɤtso pɯ-kɯ-ŋu tɕe, nɯ kɤ-ndza wuma ʑo pɯ-kɯ-rga. \\
 \textsc{lnk} \textsc{indef}.\textsc{poss}-child \textsc{pst}.\textsc{ipfv}-\textsc{genr}:S/P-be \textsc{lnk} \textsc{dem} \textsc{inf}-eat really \textsc{emph} \textsc{pst}.\textsc{ipfv}-\textsc{genr}:S/P-like \\
 \glt `When (we) were children, (we) used to like eating it.' (12-ndZiNgri, 137-138)
\end{exe}

More generally, the factual, imperfective, past imperfective and perfective forms of intransitive verbs in generic person forms are homophonous with unmarked, imperfective, past imperfective and perfective participles respectively. In the case of transitive verbs, the subject participle can be identical to the object generic form. For instance, the participle \forme{nɯ-tu-kɯ-ndza} `the one who eats them' in (\ref{ex:tukWndza.nmlz}) only differs from the generic \forme{tu-kɯ-ndza} `it eats us/people' in \ref{ex:tukWndza.genr}) by the possessive prefix \forme{nɯ-}, and that prefix being optional, there are forms that are really ambiguous between participle and generic. 

\begin{exe}
\ex \label{ex:tukWndza.nmlz}
 \gll nɯ ɯ-rkɯ jɤ-azɣɯt-nɯ tɕe, ʑara nɯ-tu-kɯ-ndza srɯnmɯ ci pjɤ-tu, \\
 \textsc{dem} \textsc{3sg}.\textsc{poss}-side \textsc{pfv}-reach-\textsc{pl} \textsc{lnk} \textsc{3pl} \textsc{3pl}.\textsc{poss}-\textsc{ipfv}-\textsc{nmlz}:S/A-eat râkshasî \textsc{indef} \textsc{ifr}.\textsc{ipfv}-be \\
\glt `There was a râkshasî who ate those who had come near her.' (Kunbzang2012, 255)
\end{exe}

\begin{exe}
\ex \label{ex:tukWndza.genr}
 \gll tɕe ndzɤpri kɤ-ti nɯ tɕe tɯrme tu-kɯ-ndza ɲɯ-ŋgrɤl  \\
 \textsc{lnk} brown.bear \textsc{nmlz}:P-say \textsc{dem} \textsc{lnk} people \textsc{ipfv}-\textsc{genr}:S/P \textsc{sens}-be.usually.the.case \\
\glt `The brown bear, it eats people.' (21-pri, 94)
\end{exe}
 
The irregular generic \forme{tu-kɯ-ti} `one says' of the verb \japhug{ti}{say} is also identical with the participle `the one who says'.

The \textsc{2sg}\fl{}\textsc{1sg} form of transitive verbs in \forme{-a}, due to the vowel fusion rule \ipa{-a-a} \fl{} \forme{-a}, are also superficially identical to subject participles, for instance \forme{tu-kɯ-ndza-a} `you eat me' is pronounced \phonet{tukɯndza} exactly like the generic and the participle \forme{tu-kɯ-ndza} in the Kamnyu dialect (in the dialects of Japhug where this vowel fusion does not occur, the forms remain distinct).

\begin{exe}
\ex \label{ex:tukWndzaa}
 \gll nɯ kóʁmɯz nɤ tu-kɯ-ndza-a \\
 \textsc{dem} only.after \textsc{lnk} \textsc{ipfv}-2\fl{}1-eat-\textsc{1sg} \\
 \glt `Eat me only after (having taken out the thorn on my foot).' (140426 lang yisheng-zh, 16)
\end{exe} 

In the case of stative verbs and some auxiliary verbs, the infinitive has in some cases the form \forme{kɯ-}, and there is thus ambiguity between infinitive and subject participial forms for these verbs (§ \ref{sec:velar.inf}).

\subsubsection{Subject relative clauses}  \label{sec:subject.participle.subject.relative}
The most common use of subject participles is to build participial relative clauses whose head noun is the subject; it is the only way to relativize the subject in Japhug (§ XXX). Headless relatives are most common (§ XXX), but when the head noun is overt, the relative can be either prenominal, postnominal or head-internal. With intransitive verbs the difference between postnominal or head-internal relatives is often difficult to ascertain, and many examples are ambiguous; for instance in (\ref{ex:tCheme.RnWz.kWrWCmi}), the relative clause could be argued to be postnominal (limited to the participle \forme{kɯ-rɯɕmi} `speaking') or head-internal (including \forme{tɕʰeme ʁnɯz} `two girls', and possibly even the previous adjunct).

\begin{exe}
\ex \label{ex:tCheme.RnWz.kWrWCmi}
 \gll  kʰa ɯ-ŋgɯ nɯtɕu tɕʰeme ʁnɯz kɯ-rɯɕmi pjɤ-tu. \\
 house \textsc{3sg}.\textsc{poss}-inside \textsc{dem}:\textsc{loc} girl two \textsc{nmlz}:S/A-speak \textsc{ifr}.\textsc{ipfv}-exist \\
\glt  `There were two girls speaking in the house.' (150909 xiaocui-zh, 157)
\end{exe}

With transitive verbs, subject head-internal relatives can be distinguished from postnominal ones  by the presence of the ergative \forme{kɯ} on the head noun (§ XXX), as in (\ref{ex:WrdWrdoR.kW.thotsi.WkWta}).

\begin{exe}
\ex \label{ex:WrdWrdoR.kW.thotsi.WkWta}
 \gll [tsuku ɯ-rdɯ\redp{}rdoʁ kɯ ʑo tʰotsi ɯ-kɯ-ta] ɣɤʑu. \\
 some \textsc{3sg}.\textsc{poss}-piece \textsc{erg} \textsc{emph} seal \textsc{3sg}.\textsc{poss}-\textsc{nmlz}:S/A-put exist:\textsc{sens} \\
 \glt `There are people who put a seal (on their bread).'
\end{exe}


Prenominal relatives are relatively rare with intransitive verbs, but commonly occur with transitive verbs, as in (\ref{ex:tWnW.WkWtshi}). Note the presence of indefinite person possessive marking on the head noun \japhug{tɤ-pɤtso}{child} in this example; unlike in Situ (\citealt{jacksonlin07}), the head noun of prenominal relatives in Japhug does not take a third person singular prefix as in a possessive construction (in which case the form $\dagger$\forme{ɯ-pɤtso} would have been found).

\begin{exe}
\ex \label{ex:tWnW.WkWtshi}
 \gll  [tɯ-nɯ ɯ-kɯ-tsʰi] tɤ-pɤtso ɣɯ ɯ-kɯ-mŋɤm ɲɯ-ŋu tɕe, \\
 \textsc{indef}.\textsc{poss}-breast \textsc{3sg}.\textsc{poss}-\textsc{nmlz}:S/A-drink \textsc{indef}.\textsc{poss}-child \textsc{gen} \textsc{3sg}.\textsc{poss}-\textsc{nmlz}:S/A-hurt \textsc{sens}-be \textsc{lnk} \\
 \glt `It is a disease of infants (who still drink mother milk).' (25-kACAl, 61)
\end{exe}

There are nevertheless prenominal genitival subject relative clauses, containing a subject participle, with the genitive \forme{ɣɯ} occurring between the relative clause and the head noun. This construction is especially common in texts translated from Chinese (due to calquing with \zh{的} <de>-relatives, § XXX), but also attested in natural speech, as in (\ref{ex:kWsAndza.GW}).

\begin{exe}
\ex \label{ex:kWsAndza.GW}
 \gll nɯnɯ kɯ-sɤ-ndza ɣɯ rɯdaʁ nɯnɯ tɕe kɯrŋi tu-kɯ-ti ŋu.  \\
 \textsc{dem} \textsc{nmlz}:S/A-\textsc{apass}-eat \textsc{gen} animal \textsc{dem} \textsc{lnk} beast \textsc{ipfv}-\textsc{genr}-say be:\textsc{fact} \\
 \glt `Animals eating (other animals) are called `beasts'.' (150822 kWrNi, 6)
\end{exe}

%qambrɯ kɯ-rɤpɯ ci a-jɤ-tɯ-ɣɯt-nɯ ra, tɕe tɯmɯ nɤmkha ɯ-kɯ-luj ɣɯ raz ci a-jɤ-tɯ-ɣɯt-nɯ ra.

When subject relative clauses contain a complement clause, the main verb of that clause can be in subject participle form (see examples \ref{ex:ndZikWsAndu} and \ref{ex:tAkWmbri.kWme} in § \ref{sec:subject.participle.complementation} and § XXX).
 
\subsubsection{Other relative clauses}  \label{sec:subject.participle.other.relative}
In addition to subject relativization, the subject participle is also used in possessor relatives, when the relativized element is the possessor of the subject (§ XXX).  The head-internal clause in (\ref{ex:kWkWtu.head.internal}) is such a possessor relative; its head noun \japhug{si}{tree}, possessor of the subject \japhug{ɯ-mat}{its fruits},  is marked with the genitive, showing that it belongs to the relative.  

\begin{exe}
\ex \label{ex:kWkWtu.head.internal}
 \gll si ɣɯ ɯ-mat kɯ\redp{}kɯ-tu nɯ ɯ-ku ri ɕ-ku-zo ɲɯ-ŋu tɕe. \\
 tree \textsc{gen} \textsc{3sg}.\textsc{poss}-fruit \textsc{total}\redp{}\textsc{nmlz}:S/A-exist \textsc{dem} \textsc{3sg}.\textsc{poss}-top \textsc{loc} \textsc{transloc}-\textsc{ipfv}-land \textsc{sens}-be \textsc{lnk} \\
 \glt `It lands on the top of all trees that have fruits.' (24-ZmbrWpGa, 43)
\end{exe}

 Headless possessor relative clauses, such as  \forme{nɯ-mtɕʰi mɤ-kɯ-pe} `those with a foul mouth' in (\ref{ex:nWmtChi.mAkWpe}), are even more common.

\begin{exe}
\ex \label{ex:nWmtChi.mAkWpe}
 \gll nɯ-mtɕʰi mɤ-kɯ-pe, kɤ-nɯtsɯ kɯ-ra ra kɯnɤ tu-kɯ-nɯ-ti nɯnɯra tɕaɣi tu-sɤrmi-nɯ ŋgrɤl. \\
 \textsc{3pl}.\textsc{poss}-mouth \textsc{neg}-\textsc{nmlz}:S/A-be.good \textsc{inf}-hide \textsc{nmlz}:S/A-have.to \textsc{pl} also \textsc{ipfv}-nmlz:S/A-\textsc{auto}-say \textsc{dem}:\textsc{pl} parrot \textsc{ipfv}-call-\textsc{pl} be.usually.the.case:\textsc{fact} \\
 \glt `Those with a foul mouth, who say things that should be hidden, are called `parrots'. (24-qro, 130)
\end{exe}

In (\ref{ex:WkWXsu.kWme}), we find a prenominal relative containing another verb in subject participle form that can be interpreted a possessor relativization as in (\ref{ex:kWkWtu.head.internal}): \forme{ɯ-kɯ-χsu kɯ-me}  `having no feeder'.

\begin{exe}
\ex \label{ex:WkWXsu.kWme}
 \gll  [ɯ-pɕi kɯ-rɤʑi] [ɯ-kɯ-χsu kɯ-me] lɯlu ɣɤʑu tɕe nɯnɯ kupa kɯ <yemao> tu-ti ŋu \\
 \textsc{3sg}.\textsc{poss}-outside \textsc{nmlz}:S/A-stay \textsc{3sg}.\textsc{poss}-\textsc{nmlz}:S/A-feed \textsc{nmzl}:S/A-not.exist cat exist:\textsc{sens} \textsc{lnk} \textsc{dem} Chinese \textsc{erg} wild.cat \textsc{ipfv}-say be:\textsc{fact} \\
 \glt `There are cats that live outside, that nobody feeds, Chinese people call them wild cats.' (21-lWlu, 2)
 \end{exe}

However, there are cases of relative clauses with the subject participle of the negative existential verb \forme{kɯ-me} and an \textit{intransitive} verb in participial (or finite) form in the preceding complement clause, for instance \forme{tɤ-kɯ-mbri} in (\ref{ex:tAkWmbri.kWme}). It is manifest that here the relativized element is neither the subject of \japhug{me}{not exist} nor a possessor, but rather the subject of the verb of the complement clause \japhug{mbri}{cry, sing, make noise}.

  \begin{exe}
\ex \label{ex:tAkWmbri.kWme}
 \gll  pɣɤtɕɯ nɯ kɯnɤ [[tɯ-ɣjɤn cinɤ ʑo tɤ-kɯ-mbri] kɯ-me], nɯ to-ɣɤscɤscɤt ʑo to-mbri ɲɯ-ŋu, \\
bird \textsc{dem} also one-time even.one \textsc{emph} \textsc{pfv}-\textsc{nmlz}:S/A-make.noise \textsc{nmlz}:S/A-not.exist \textsc{dem} \textsc{ifr}-do.quickly \textsc{emph} \textsc{ifr}-make.noise \textsc{sens}-be \\
\glt `Even the bird, who had not sung even once (since coming to the palace), immediately started singing.' (2012 qachGa, 170)
 \end{exe}

The clause \forme{tɯ-ɣjɤn cinɤ ʑo tɤ-kɯ-mbri kɯ-me} here is in fact the nominalized version of the postverbal negative construction  (§ XXX). In main clauses, this construction combines a negative existential verb in impersonal (third singular) form with a complement clause in finite form. In (\ref{ex:tAkWmbri.kWme}), we see that when the intransitive subject of a postverbal negative construction is nominalized, both the matrix verb \japhug{me}{not exist} and the verb of the complement clause \forme{tɤ-kɯ-mbri} occur in subject participle form. This construction, though superficially similar to that in (\ref{ex:WkWXsu.kWme}), is therefore distinct from it.
 
 There appear to be participial relative clauses in \forme{kɯ-} whose relativized element is neither the subject or the possessor of the subject. In the participial relative taking \forme{mɤ-kɯ-sɤ-mto } `the one that is not visible' as its main verb in (\ref{ex:mAkWsAmto.schiz}), the head \japhug{smar}{river} is not the possessor of the subject in the proper sense, but the possessor of a noun (\japhug{ɯ-βzɯr}{its side}) subject of a clause embedded within another clause (headed by the participle \forme{kɯ-fse} `that is like...') serving as the subject of \forme{mɤ-kɯ-sɤ-mto } `the one that is not visible'.  

  \begin{exe}
\ex \label{ex:mAkWsAmto.schiz}
 \gll  [maka tɕɤkɯ ku-kɯ-ru tɕe tɕɤndi smar [[ɯ-βzɯr tɕʰi kɯ-fse ŋu] kɯ-fse] mɤ-kɯ-sɤ-mto] ʑo scʰiz nɯ-azɣɯt ɲɯ-ŋu. \\
 at.all east \textsc{ipfv}:\textsc{east}-\textsc{genr}:S/P-look \textsc{lnk} west river \textsc{3sg}.\textsc{poss}-side what \textsc{nmlz}:S/A-be.like be:\textsc{fact} \textsc{nmlz}:S/A-be.like \textsc{neg}-\textsc{nmlz}:S/A-\textsc{deexp}-see \textsc{emph} \textsc{approx}.\textsc{loc} \textsc{pfv}:\textsc{west}-reach \textsc{sens}-be \\
\glt `He arrived at a river which was such that if one looked from one bank to the other side, what was on the other side was not at all visible.' (Divination 2005, 27)
 \end{exe}
 
 This particularly convoluted example is however not representative of what is usually found in the corpus.
\subsubsection{Complementation strategies}  \label{sec:subject.participle.complementation}
Subject participles are also required in several types of complement clauses and complementation strategies (§ XXX on the difference between the two types).

The most common complementation construction involving subject participles occurs with the motion verbs \japhug{ɕe}{go} and \japhug{ɣi}{come}. These verbs have purposive clauses which compulsorily take a verb in subject participle form (§ XXX).  When the verb in the purposive clause is transitive, the participle has a possessive prefix coreferent with the object as in the case of relative clauses (\ref{sec:subject.participle.possessive}), and the subject can either take absolutive marking following the motion verb (which is morphologically intransitive), as in (\ref{ex:WkWCar.chACe}), or ergative  marking following the verb of the purposive clause as in (\ref{ex:WkWCar.loCenW}). This case marking difference can be analysed as reflecting distinct clausal structures: in (\ref{ex:WkWCar.loCenW}), the subject \forme{tɤ-rɟit ra} `the children' belongs to the purposive clauses, whereas in (\ref{ex:WkWCar.chACe}), the subject \forme{tɤ-mu nɯ} lies outside of it.

\begin{exe}
\ex \label{ex:WkWCar.chACe}
 \gll lo-fsoʁ tɕe tɕe tɤ-mu nɯ [ɯ-tɕɯ ɯ-kɯ-ɕar] cʰɤ-ɕe tɕe,\\
 \textsc{ifr}-be.bright \textsc{lnk} \textsc{lnk} \textsc{indef}.\textsc{poss}-mother \textsc{dem} \textsc{3sg}.\textsc{poss}-son \textsc{3sg}.\textsc{poss}-\textsc{nmlz}:S/A-search \textsc{ifr}:\textsc{downstream}-go \textsc{lnk} \\
\glt `When the sun came up (in the morning), the mother went to look for her son.' (2012tWJo, 33)
\end{exe}

\begin{exe}
\ex \label{ex:WkWCar.loCenW}
 \gll ɯ-fso-soz tɕe, [tɤ-rɟit ra kɯ nɯ ɯ-kɯ-ɕar] jo-ɕe-nɯ ɲɯ-ŋu tɕe \\
 \textsc{3sg}.\textsc{poss}-tomorrow-morning \textsc{lnk} \textsc{indef}.\textsc{poss}-child \textsc{pl} \textsc{erg} \textsc{dem} \textsc{3sg}.\textsc{poss}-\textsc{nmlz}:S/A-search \textsc{ifr}:\textsc{upstream}-go \textsc{sens}-be \textsc{lnk} \\
\glt  `The morning of the next day, the children went (there) to look for him.'  (Norbzang, 325)
\end{exe}

The goal of the motion verb can however occur within the purposive clause, as in (\ref{ex:khapa.WkWnnAjo}), where the subject in ergative form \forme{ɯ-wa nɯ kɯ} `his father' is stranded from the transitive verb \forme{ɯ-kɯ-n-nɤjo} by the goal \forme{kʰapa tɕe} `downstairs'.

\begin{exe}
\ex \label{ex:khapa.WkWnnAjo}
 \gll [ɯ-wa nɯ kɯ kʰapa tɕe ɯ-kɯ-n-nɤjo] pjɤ-ɣi.  \\
  \textsc{3sg}.\textsc{poss}-father \textsc{dem} \textsc{erg} downstairs \textsc{loc}    \textsc{3sg}.\textsc{poss}-\textsc{nmlz}:S/A-\textsc{auto}-wait \textsc{ifr}:\textsc{down}-come \\
  \glt `His father came downstairs to wait for him.' (140506 loBzi, 5)
\end{exe}

There is obligatory coreference between the subject of the motion verb and that of the purposive clause in this construction; to express coreference with the \textit{object} of the purposive clause (in the case of a transitive verb), the object participle is used instead (§ XXX, § XXX). 

Motion verb with purposive clauses have some semantic overlap with the corresponding associated motion prefixes (§ \ref{sec:am.prefixes}); the functional difference between the two construction is discussed in § \ref{sec:am.vs.mvc}.

Some transitive and semi-transitive verbs take object complement clauses (§ XXX) requiring a subject participle. This group includes the verb \japhug{sɯχsɤl}{recognize, notice} as in (\ref{ex:tAkWnWCpWz.pjAsWXsAl}) and the verbs of pretence \japhug{ʑɣɤpa}{pretend} and \japhug{nɯɕpɯz}{pretend, disguise as, imitate} as in (\ref{ex:kukWtshi.tonWCpWznW}). 

\begin{exe}
\ex \label{ex:tAkWnWCpWz.pjAsWXsAl}
 \gll tɕe nɯnɯ kɯ [qaʑo tɤ-kɯ-nɯɕpɯz] nɯ pjɤ-sɯχsɤl. \\
 \textsc{lnk} \textsc{dem} \textsc{erg} sheep \textsc{pfv}-\textsc{nmlz}:S/A-disguised \textsc{dem} \textsc{ifr}-recognize \\
\glt `He (the shepherd boy) had noticed the (nobleman) disguised as a sheep.' (40513 mutong de disheng-zh, 63)
\end{exe}

\begin{exe}
\ex \label{ex:kukWtshi.tonWCpWznW}
 \gll  ʑara kɯ [cʰa nɯ ku-kɯ-tsʰi] to-nɯɕpɯz-nɯ, \\
\textsc{3pl} \textsc{erg} alcohol \textsc{dem} \textsc{ipfv}-\textsc{nmlz}:S/A-drink \textsc{ifr}-pretend-\textsc{pl} \\
\glt  `They pretended to drink the alcohol.'  (Norbzang 2012, 91)
\end{exe}

The status of the clauses with subject participles occurring with these three verbs, though superficially similar to the purposive clauses, is however entirely distinct: these clauses are note specific constructions, but simply subject relative clauses in object or semi-object position. The difference with purposive clauses can be shown by three pieces of evidence. 

First, the three verbs in question can take nouns as objects (as shown by \ref{ex:qaɕpa.tonWCpWz} and \ref{ex:tAkWnWCpWz.pjAsWXsAl}) instead of clauses with subject participles. 

\begin{exe}
\ex \label{ex:qaɕpa.tonWCpWz}
 \gll  qaɕpa to-nɯɕpɯz, qaɕpa ɯ-rqʰu to-ŋga, \\
frog \textsc{ifr}-pretend frog \textsc{3sg}.\textsc{poss}-skin \textsc{ifr}-wear \\
\glt `He disguised as a frog, he wore a frog's skin.' (2002 qaCpa, 10)
\end{exe}
 
 Second, these verbs can occur with a participial clause whose subject is overt and distinct from the subject of the verb of the matrix clause, as in (\ref{ex:tApAtso.kWGAwu.kAnWCpWz}) where  \japhug{tɤ-pɤtso}{child} is the subject of the verb \japhug{ɣɤwu}{cry} in the participial clause, but not the subject of  \japhug{nɯɕpɯz}{pretend, disguise as, imitate} (see § XXX for additional examples). Such a subject mismatch would be completely ungrammatical with a purposive clause.
 
\begin{exe}
\ex \label{ex:tApAtso.kWGAwu.kAnWCpWz}
 \gll   [tɤ-pɤtso kɯ-ɣɤwu] ʑo kɤ-nɯɕpɯz mɤ-spe-a ma nɯ mɯma spe-a \\
 \textsc{indef}.\textsc{poss}-child \textsc{nmlz}:S/A-cry \textsc{emph} \textsc{inf}-imitate \textsc{neg}-be.able[III]:\textsc{fact}-\textsc{1sg} \textsc{lnk} \textsc{dem} apart.from be.able[III]:\textsc{fact}-\textsc{1sg} \\
\glt `I cannot imitate a child crying, but apart from that I can imitate (anything).' (27-kikakCi, 143)
\end{exe}

Third, we find examples like (\ref{ex:Wmi.kWmNAm.tonWCpWznW}) where the subject of the verb in the main clause is not coreferent with the subject of the participial clause but with the possessor of the subject; these are in fact explainable as cases of possessor relative clauses (§ \ref{sec:subject.participle.other.relative}).

\begin{exe}
\ex \label{ex:Wmi.kWmNAm.tonWCpWznW}
 \gll  tɕe [ɯ-mi kɯ-mŋɤm] to-nɯɕpɯz  \\
 \textsc{lnk} \textsc{3sg}.\textsc{poss}-leg \textsc{nmlz}:S/A-hurt \textsc{ifr}-pretend \\
 \glt `He pretended to have a pain in the leg.' (140426 lang yisheng-zh, 9)
\end{exe}

In some constructions, subject participial clauses can occur instead of infinitive clauses; For instance, the imperfective \forme{kɤ-} infinitive + existential verb construction expressing impossibility (§ \ref{sec:inf.exist}) has a variant with imperfective subject participles, as in (\ref{ex:tukWGi.YAGAme}). 

 \begin{exe}
\ex \label{ex:tukWGi.YAGAme}
 \gll   tu-kɯ-ɣi ɲɤ-ɣɤ-me qʰe,  \\
  \textsc{ipfv}:\textsc{up}-\textsc{nmlz}:S/A-come \textsc{ifr}-\textsc{caus}-not.exist \textsc{lnk} \\
  \glt `She made it impossible for her to come out (again).' (2003-kWBRa, 97)
 \end{exe}
 
In particular, when a complement-taking verb is itself in the subject participle form, it is possible for the complement either to be in the expected form (infinitive or finite), or to be in subject participle form itself. For instance, in example (\ref{ex:ndZikWsAndu}) the subject participle \forme{kɯ-cʰa}  the one who can' takes a complement with a subject participle (\forme{ndʑi-kɯ-sɤndu}) instead of the expected \forme{kɤ-} infinitive (or finite clause). See also (\ref{ex:tAkWmbri.kWme}) in § \ref{sec:subject.participle.other.relative} for a similar case, without subject coreference between the matrix verb and the verb of the complement clause.

\begin{exe}
\ex \label{ex:ndZikWsAndu}
\gll  rŋɯl kɯ ndʑi-kɯ-sɤndu kɯ-cʰa kɯ-fse pɯ\redp{}pɯ-tu nɤ  \\
silver \textsc{erg} \textsc{3du-nmlz}:S/A-exchange \textsc{nmlz}:S/A-can \textsc{nmlz}:S/A-be.like 
\textsc{cond}\redp{}\textsc{pst.ipfv}-exist if \\
\glt `If there was someone who could redeem (the life of two brothers) with money, ...' (140507 jinniao-zh, 339)
\end{exe}
 
% kɯki ɯ-ku-kɯ-ndɯn kɯ-cʰa ci ŋu
However, this type of construction is potentially ambiguous, and pairs of verbs in subject participle form should not necessarily be analyzed as complement clauses embedded in relatives. In (\ref{ex:pWkWNGlWt.kWthW}), \forme{pɯ-kɯ-ɴɢlɯt} `(bone) that has been broken, fracture' is not a (subject) complement of \forme{kɯ-tʰɯ}  `the one that is serious'; rather, \forme{wuma ʑo pɯ-kɯ-ɴɢlɯt kɯ-tʰɯ} is simply a head-internal relative clause (§ XXX) with the subject participle (itself a headless relative clause) \forme{pɯ-kɯ-ɴɢlɯt} as its subject. 

\begin{exe}
\ex \label{ex:pWkWNGlWt.kWthW}
\gll  [wuma ʑo [pɯ-kɯ-ɴɢlɯt] kɯ-tʰɯ] nɯra qʰe ndɤre, tɕʰaχɕaŋ tu-te qʰe tɕe tu-xtɕɤr ŋu.  \\
really \textsc{emph} \textsc{pfv}-\textsc{nmlz}:S/A-\textsc{acaus}:break \textsc{nmlz}:S/A-be.serious \textsc{dem}:\textsc{pl} \textsc{lnk}   \textsc{lnk} splinter \textsc{ipfv}-put[III]   \textsc{lnk}  \textsc{lnk} \textsc{ipfv}-attach be:\textsc{fact} \\
\glt `The fractures that are serious, he puts on it a splinter and attaches it.' (140426 laxthab, 7)
\end{exe}

\subsubsection{Lexicalized subject participles} \label{sec:lexicalized.subject.participle}
A certain number of subject participles have developed specialized meanings and can be considered to have been lexicalized. Some of these lexicalized participles are formally identical to the regular participle (Table  \ref{tab:lexicalized.S.nmlz}, for instance the noun \japhug{kɯcʰi}{candy} in (\ref{ex:akWchi})  as compared to the non-lexicalized participle \forme{kɯ-cʰi} `the one that is sweet' in (\ref{ex:kWchi.tu}). For such nouns, lexicalization is shown by the meaning specialization and the inability to take orientation, associated motion and polarity prefixes (but not possessive prefixes, as shown by he prefix \forme{a-} on \japhug{kɯcʰi}{candy} in \ref{ex:akWchi}).

\begin{exe}
\ex \label{ex:akWchi}
 \gll aʑo a-ŋgra a-kɯcʰi ci tɤ-χti ra \\
 \textsc{1sg} \textsc{1sg}.\textsc{poss}-salary \textsc{1sg}.\textsc{poss}-candy \textsc{indef} \textsc{imp}-buy[III] have.to:\textsc{fact} \\
\glt `Give me a candy as a reward.' (140515 congming de wusui xiaohai-zh, 82)
\end{exe}

\begin{exe}
\ex \label{ex:kWchi.tu}
 \gll tɕe nɯnɯ li tú-wɣ-ndza tɕe, kɯ-cʰi tu, mɤ-kɯ-cʰi tu. \\
\textsc{lnk} \textsc{dem} again \textsc{ipfv}-\textsc{inv}-eat \textsc{lnk} \textsc{nmlz}:S/A-be.sweet exist:\textsc{fact} \textsc{neg}-\textsc{nmlz}:S/A-be.sweet exist:\textsc{fact} \\
\glt `When one eats them, some are sweet, some are not.' (08-rasti, 55)
\end{exe}

Table \ref{tab:lexicalized.S.nmlz} does not include the many names of profession / occupation built from the subject participles which are semantically transparent. We can distinguish two cases. 

First, labile verb derive participial forms such as \japhug{kɯ-lɤɣ}{shepherd} or \japhug{kɯ-mɯrkɯ}{thief} (from \japhug{lɤɣ}{graze} and \japhug{mɯrkɯ}{steal)}) without obligatory possessive prefix; the absence of these prefixes cannot be attributed to lexicalization, since these verbs can also be used intransitively (§ XXX). 

Second, plain transitive verbs have to undergo antipassive derivation (§ \ref{sec:antipassive}) for their subject participles to be usable as names of profession. For instance, \japhug{kɯ-rɤ-rɤt}{writer} and \japhug{kɯ-rɤ-tʂɯβ}{tailor} are from the \forme{rɤ-} non-human antipassive forms of \japhug{rɤt}{write} and \japhug{tʂɯβ}{sew}, while  \japhug{kɯ-sɤ-sɯxɕɤt}{teacher} comes from the \forme{sɤ-} human antipassive of \japhug{sɯxɕɤt}{teach}. Without antipassive prefix, the subject participles of (non-labile) transitive verbs require either an overt object or a definite and anaphorically recoverable object, and are not appropriate as names of professions. For instance, in (\ref{ex:tArmi.WkWrAt}), the participle \forme{ɯ-kɯ-rɤt} `the one writing it' is used with \japhug{tɤ-rmi}{name} as its object.

\begin{exe}
\ex \label{ex:tArmi.WkWrAt}
 \gll  [tɤ-rmi ɯ-kɯ-rɤt] tɤ-pɤtso nɯ ɯ-rkɯ ʑo, [...] pjɤ-zɣɯt tɕe, \\
 \textsc{indef}.\textsc{poss}-name \textsc{3sg}.\textsc{poss}-\textsc{nmlz}:S/A-write  \textsc{indef}.\textsc{poss}-child \textsc{dem} \textsc{3sg}.\textsc{poss}-side \textsc{emph} { } \textsc{ifr}-reach \textsc{lnk} \\
\glt `It arrived near the boy who wrote the names (of the contestants).' (150826 shier shengxiao, 110)
\end{exe}

\begin{table}[H]
\caption{Lexicalized subject participles} \label{tab:lexicalized.S.nmlz} \centering
\begin{tabular}{llll}
\lsptoprule
Noun & Base verb \\
\midrule
\japhug{kɯβʁa}{noble} & \japhug{βʁa}{prevail, win}  \\
\japhug{kɯspoʁ}{hole} & \japhug{spoʁ}{have a hole}  \\
 \japhug{kɯcʰi}{candy} & \japhug{cʰi}{be sweet} \\
 \japhug{kɯmŋɤm}{ailment} & \japhug{mŋɤm}{hurt, feel pain} \\
 \japhug{kɯŋu}{right thing} & \japhug{ŋu}{be} \\
 \japhug{kɯmaʁ}{bad thing} & \japhug{maʁ}{not be} \\
\lspbottomrule
\end{tabular}
\end{table}


In the case of  \japhug{kɯŋu}{right thing}  and  \japhug{kɯmaʁ}{bad thing}, lexicalization is very advanced, and the meaning of the noun is very distinct from the corresponding participles \japhug{kɯ-ŋu}{the one that is}  and  \japhug{kɯ-maʁ}{the one that is not} respectively. Examples such as (\ref{ex:kWNu.mAtWnAme}) and (\ref{ex:kWNu.mAtWnAme}) illustrate their use in collocation with verbs like \japhug{nɤma}{work, make} and \japhug{fse}{be like}.

\begin{exe}
\ex \label{ex:kWNu.mAtWnAme}
 \gll  mɤ-ti-a ma kɯŋu mɤ-tɯ-nɤme \\
\textsc{neg}-say:\textsc{fact}-\textsc{1sg} \textsc{lnk} right.thing \textsc{neg}-2-make[III]:\textsc{fact} \\
\glt `I won't say it, because you will not do the right thing.' (2005 Kunbzang, 397)
\end{exe}

\begin{exe}
\ex \label{ex:kWNu.mAfse}
 \gll  a-lɤ́-wɣ-ɕaβ-a tɕe tɕendɤre kɯŋu mɤ-fse \\
 \textsc{irr}-\textsc{pfv}-\textsc{inv}-catch.up-\textsc{1sg} \textsc{lnk} \textsc{lnk} right.thing \textsc{neg}-be.like:\textsc{fact} \\ 
\glt `If he catches up with me, (our enterprise) won't succeed.' (25-kAmYW-XpAltCin, 37)
\end{exe}

The participle \japhug{kɯ-maʁ}{the one that is not} has been independently grammaticalized as an identity pronoun/determined \japhug{kɯmaʁ}{other} (see § \ref{sec:other.pro} and § \ref{sec:identity.modifier}).

From the nouns \japhug{kɯŋu}{right thing}  and  \japhug{kɯmaʁ}{bad thing}, the intransitive verbs \japhug{rɯkɯŋu}{do the right thing, take good care of one's family} and \japhug{rɯkɯŋu}{do bad things, happen bad things, be clumsy} and the transitive verb \japhug{nɯkɯmaʁ}{make a mistake} have been derived by denominal derivation with \forme{rɯ-} and \forme{nɯ-} (§ XXX).

The subject participle \forme{kɯ-mpɕɤr} `the beautiful one' of the verb \japhug{mpɕɤr}{be beautiful} has a derived denominal transitive verb \japhug{nɯkɯmpɕɤr}{wear (on important occasions)} with highly derived semantics, reflecting the lexicalized use of the participle in the meaning `decoration' as in (\ref{ex:WkWmpCAr.tAkABzu}).

 \begin{exe}
\ex \label{ex:WkWmpCAr.tAkABzu}
 \gll tɕe li ɯ-kɯ-mpɕɤr kɯ-fse tɤ-kɤ-βzu ɲɯ-ŋu tɕe    \\
\textsc{lnk} again \textsc{3sg}.\textsc{poss}-\textsc{nmlz}:S/A-be.beautiful \textsc{nmlz}:S/A-be.like \textsc{pfv}-\textsc{nmlz}:P-make \textsc{sens}-be \textsc{lnk} \\
 \glt `(The seal on breads) is used for decoration.' (160706 WzbroN, 6)
\end{exe}

Several names of diseases only exist as intransitive verbs, and the disease itself or the person suffering from the disease can only be referred to by using a participial or infinitive form. In particular, the word \japhug{kɤ-kɯ-nɤndza}{leper} is the perfective subject participle of \japhug{nɤndza}{have leprosy}; this word has some degree of lexicalization (in particular, it is a common insult), but it behaves like a participle grammatically; in particular, it can undergo totalitative reduplication (§ XXX), as in (\ref{ex:kWkAkWnAndza}).

\begin{exe}
\ex \label{ex:kWkAkWnAndza}
 \gll nɯnɯ kɯ, nɯnɯtɕu kɯ\redp{}kɯ-rɤʑi nɯ to-ɣɤ-mna. to-ɣɤ-mna ɯ-qʰu tɕe tɕendɤre <quanxian> tɕe kɯ\redp{}kɤ-kɯ-nɤndza nɯ ɲɤ-ɣɤ-me \\
 \textsc{dem} \textsc{erg} \textsc{dem}:\textsc{loc} \textsc{total}\redp{}\textsc{nmlz}:S/A-stay \textsc{dem} \textsc{ifr}-\textsc{caus}-recover  \textsc{ifr}-\textsc{caus}-recover  \textsc{3sg}.\textsc{poss}-after lnk lnk all.the.district loc \textsc{total}\redp{}\textsc{pfv}-\textsc{nmlz}:S/A-have.leprosy \textsc{dem} \textsc{ifr}-\textsc{caus}-not.exist \\
\glt `He healed all those who were staying there (in the leper house). After he healed them, he had eradicated leprosy (removed all lepers) from our district.' (25-khArWm, 82)
\end{exe}

Other disease names such as \japhug{tɤkɤzbɣaʁ}{migraine} (as in \ref{ex:tAkAzbGaR}), although clearly the perfective participle or infinitive of a verb root \forme{*azbɣaʁ}, is hardly ever attested in finite form.

\begin{exe}
\ex \label{ex:tAkAzbGaR}
 \gll tɤkɤzbɣaʁ nɯ tɤ-mŋɤm qʰe, tɕe nɯ ɯ-qʰu nɤ, ŋgɯsqɤ-rʑaʁ ʑo mɯ-tu-mna \\
 migraine \textsc{dem} \textsc{pfv}-hurt \textsc{lnk} \textsc{lnk} \textsc{dem} \textsc{3sg}.\textsc{poss}-after \textsc{lnk} nine.or.ten-night \textsc{emph} \textsc{neg}-\textsc{ipfv}-recover \\
\glt `After the migraine starts, it does not recede until nine or ten days. (conversation taRrdo 2003, 9)
\end{exe}

In addition, we find nouns in \forme{kɯ-} that can be suspected to be former lexicalized participles, such as \japhug{kɯjŋu}{oath}, which appears to contain the root of the verb   \japhug{ŋu}{be}, though the segment \forme{-j-} cannot be accounted for at the present moment,\footnote{In any case, the Tangut cognate \tangut{𗡔}{4600}{ŋwụ}{1.58}  oath' shows that this derivation is very ancient and reflects a non-productive morphological process. } and \japhug{kɯmtɕʰɯ}{toy}, whose verbal root cannot be identified. The name \japhug{kɯsɤɣru}{mirror} (an archaic word in the process of being replaced by the Tibetan \japhug{χɕɤlzgoŋ}{mirror}) could also be a frozen subject participle  of the verb \japhug{ru}{look at}, but the nature of the prefix \forme{sɤɣ-} is unclear: it could be deexperiencer (§ XXX). Alternatively, \forme{sɤɣ-} could be analyzed as a frozen oblique participle (§ \ref{sec:lexicalized.oblique.participle}), but in this view the prefix \forme{kɯ-} would not be identifiable.

Lexicalized subject participles appearing in compounds are also found. Several cases must be distinguished. First, we find subject participles of transitive verbs as second member of a compound, with their object as first member, as kind a lexicalized headless relative clause, like \japhug{qalekɯtsʰi}{species of kite}, which combines  \japhug{qale}{wind} and the participle \forme{ɯ-kɯ-tsʰi}  of the transitive verb \japhug{tsʰi}{block}, literally `blocking the wind'   (§ \ref{sec:tatpurusha.n.n}), a designation referring to this bird's ability to apparently remain unmoving in the sky, as described in (\ref{ex:kAnWqambWmbjom.mWjCe}).

\begin{exe}
\ex \label{ex:kAnWqambWmbjom.mWjCe}
 \gll  kɤ-nɯqambɯmbjom mɯ́j-ɕe kɯ nɯnɯre ɯ-stu ri ku-rɤʑi tɕe, [...] ɯ-ʁar nɯ tu-sɤlqɤlqɤt nɤ tu-sɤlqɤlqɤt ŋgrɤl  \\
 \textsc{inf}-fly \textsc{neg}:\textsc{sens}-go \textsc{erg} there \textsc{3sg}.\textsc{poss}-place \textsc{loc} \textsc{ipfv}-stay \textsc{lnk} { } \textsc{3sg}.\textsc{poss}-wing \textsc{dem} \textsc{ipfv}-flap.slightly \textsc{lnk}  \textsc{ipfv}-flap.slightly be.usually.the.case:\textsc{fact} \\
 \glt `It does not move (flying) but remain there (in the sky) at his place, slightly flapping its wings.' (23-RmWrcWftsa, 40)
\end{exe}

A second type involves two participles in apposition, as \japhug{kɯrŋukɯɣndʑɯr}{harvestman}, build from the subject participles of \japhug{rŋu}{parch} and  \japhug{ɣndʑɯr}{grind} (§ \ref{sec:appositive.n.n}). Both verbs being transitive, the absence of a possessive prefix \forme{ɯ-} is an additional clue that the form is fully lexicalized.

Nominalizations with the \forme{x-/ɣ-} prefix (§ \ref{sec:G.nmlz}) are ancient lexicalized subject participles that have undergone a syllable reduction rule (§ XXX) and have become completely separated from their base verb synchronically.

\subsection{Object participles} \label{sec:object.participle}
The object participle is a nominalized form which refers to an entity corresponding to the object (\ref{sec:absolutive.P}) or semi-object (§ \ref{sec:semi.object}) of the base verb. All transitive and semi-transitive verbs (except for \japhug{kɤtɯpa}{tell}, § XXX) can build an object participle by adding the prefix \forme{kɤ-} (for instance \forme{kɤ-ndza} from the verb \japhug{ndza}{eat} in \ref{ex:kAndza}). This form is homophonous with, and historically related to the velar infinitive (§ \ref{sec:velar.inf}, § \ref{sec:velar.nmlz.history}).

 \begin{exe} 
\ex \label{ex:kAndza}
\gll kɤ-ndza \\
   \textsc{nmlz}:P-eat \\
 \glt  `The one that is eaten.' (many attestations)
 \end{exe}

In the case of secundative verbs (§ XXX), the object participle can either refer to the recipient or the theme, as in (\ref{ex:nWkAmbi}); this question is discussed in more detail in § \ref{sec:object.participle.relatives}.

  \begin{exe} 
\ex \label{ex:nWkAmbi}
\gll nɯ-kɤ-mbi \\
   \textsc{pfv}-\textsc{nmlz}:P-give \\
 \glt  `The one that he has given it to.'
 \glt `The one that has been given to him.'  (many attestations)
 \end{exe}

   
 In this section, I first describe the morphological properties of object participles (compatibility with possessive prefixes § \ref{sec:object.participle.possessive} and other prefixes § \ref{sec:object.participle.other.prefixes}). Then, I discuss several cases of ambiguity between object participles and other \forme{kɤ-} prefixed forms in § \ref{sec:object.participle.ambiguity} (see also § \ref{sec:infinitives.participles}). The uses of object participles to build relative clauses and complement clauses are described in  § \ref{sec:object.participle.relatives} and § \ref{sec:object.participles.complement}. Finally, I present a few cases of lexicalized object participles in § \ref{sec:lexicalized.object.participle}.
 
\subsubsection{Possessive prefixes on object participles}  \label{sec:object.participle.possessive} 
Unlike subject participles, object participles never require a possessive prefix. An optional possessive prefix coreferent with the transitive subject, as in (\ref{ex:akAsWz}), can however be added.
  
  \begin{exe}
\ex \label{ex:akAsWz}
\gll a-kɤ-sɯz    \\
   \textsc{1sg-nmlz}:P-know \\
 \glt  `The one that I know.' (many attestations)
 \end{exe}

In the case of semi-transitive verbs, the possessive prefix is also coreferent with the subject, as in the form \forme{ɯ-kɤ-rga} `the one that he likes' in (\ref{ex:stu.WkArga}), build in the same way as the object participle of the transitive (tropative § XXX) verb \japhug{nɤmɯm}{find tasty}.

\begin{exe}
\ex \label{ex:stu.WkArga}
\gll ri nɯnɯ stu ɯ-kɤ-rga, ɯ-kɤ-nɤ-mɯm pjɤ-ɕti. \\
\textsc{lnk} \textsc{dem}  most \textsc{3sg}.\textsc{poss}-\textsc{nmlz}:P-like \textsc{3sg}.\textsc{poss}-\textsc{nmlz}:P-\textsc{trop}-be.tasty \textsc{ifr}.\textsc{ipfv}-be.\textsc{affirm} \\
\glt `But it was what he liked most, what he found most tasty.' (160703 poucet3, 74)
\end{exe}

In addition to semi-transitive verbs, the complement-taking verb \japhug{cʰa}{can} has object participles taking possessive prefixes meaning `the one that $X$ can $Y$', $X$ being the subject (marked by the possessive prefix), and $Y$ the verb in the complement clause, which can be overt or not as in (\ref{ex:nWmAkAcha}), where \forme{nɯ-mɤ-kɤ-cʰa} stands for \forme{kɤ-ndo nɯ-mɤ-kɤ-cha} `the one(s) that they are able to catch'.
 
\begin{exe}
\ex  \label{ex:nWmAkAcha}
\gll tɕe nɯ-mɤ-kɤ-cʰa nɯ kʰɯna χsɯm pɯ-tu qʰe, nɯra kɯ rcanɯ ɕlaʁ ʑo ku-ndo-nɯ ɲɯ-ɕti. \\
\textsc{lnk} \textsc{3pl}.\textsc{poss}-\textsc{neg}-\textsc{inf}-can \textsc{dem} dog three \textsc{pst}.\textsc{ipfv}-exist \textsc{lnk} \textsc{dem}:\textsc{pl} \textsc{erg} unexpectedly \textsc{ideo}.I:immediately \textsc{emph} \textsc{ipfv}-catch-\textsc{pl} \textsc{sens}-be.\textsc{affirm} \\
\glt `The (rats) that they (the people) had been unable to (catch), there were three dogs, these (dogs) caught them at once.' (150831 BZW kAnArRaR, 48)
\end{exe}

\subsubsection{Associated motion, polarity and orientation prefixes on object participles}  \label{sec:object.participle.other.prefixes}
Object participles, like subject participles, are compatible with polarity (\ref{ex:amAkAsWz}), associated motion (\ref{ex:WCWkAnAma}) and orientation prefixes (\ref{ex:WCWkAnAma}).

\begin{exe}
\ex  \label{ex:amAkAsWz}
\gll tɕe aʑo a-mɤ-kɤ-sɯz tɤjmɤɣ nɯ kɤ-ndza mɤ-naz-a \\
\textsc{lnk} \textsc{1sg} \textsc{1sg}.\textsc{poss}-\textsc{neg}-\textsc{nmlz}:P-know mushroom \textsc{dem} \textsc{inf}-eat \textsc{neg}-dare:\textsc{fact}-\textsc{1sg}  \\
\glt `I do not dare to eat the mushrooms that I do not know.' (23-mbrAZim, 113)
\end{exe}

\begin{exe}
\ex  \label{ex:WCWkAnAma}
\gll ɯ-pɕi tɕe ɯ-ɕɯ-kɤ-nɤma ci pjɤ-tu tɕe, \\
\textsc{3sg}.\textsc{poss}-outside \textsc{lnk} \textsc{3sg}.\textsc{poss}-\textsc{transloc}-\textsc{nmlz}:P-work \textsc{indef} \textsc{ifr}.\textsc{ipfv}-exist \textsc{lnk} \\
\glt  `(The mouse) had something to do outside.' (140518 mao he laoshu-zh, 88)
\end{exe}

Associated motion and polarity prefixes on object participles co-occur with possessive prefixes, as shown by  (\ref{ex:amAkAsWz}) and  (\ref{ex:WCWkAnAma})  above, but orientation prefixes (whether perfective or imperfective) do not. This is an important difference between subject and object participles (§ \ref{sec:subject.participle.other.prefixes}). Object participles only have at most two prefixes.

\begin{exe}
\ex  \label{ex:pjWKAnWji}
\gll tɕe pɤjka wuma nɯnɯ tɕe, pjɯ-kɤ-nɯ-ji ŋu tɕe, \\
\textsc{lnk} gourd really \textsc{dem} \textsc{lnk} \textsc{ipfv}-\textsc{nmlz}:P-\textsc{auto}-plant be:\textsc{fact} \textsc{lnk} \\
\glt `The gourd proper is cultivated (it does not grow on its own).' (16-CWrNgo, 63)
\end{exe}

Finite relative clauses, instead of object participles, can be used to specify both TAM and the subject (§ XXX).

Unlike subject participle, object participles are attested with the progressive \forme{asɯ-} prefix, as in (\ref{ex:pWkASWndo}). It is the only non-finite form compatible with this prefix.  

\begin{exe}
\ex  \label{ex:pWkASWndo}
\gll  tɕʰeme nɯ kɯ iɕqʰa, ɯ-jaʁ <meihua>, mɯntoʁ pɯ-kɤ-ɤsɯ-ndo nɯ pjɤ-ɣɤrɤt.  \\
girl \textsc{dem} \textsc{erg} the.aforementioned \textsc{3sg}.\textsc{poss}-hand plum.blossom flower \textsc{pst}.\textsc{ipfv}-\textsc{nmlz}:P-\textsc{prog}-take \textsc{dem} \textsc{ifr}-throw \\
\glt `The girl threw down the plum blossom, the flower that she was holding in her hand.' (150907 yingning-zh, 30)
\end{exe}

These forms are rare and difficult to identify, as they are always ambiguous with object participles or infinitive of causativized verbs. In the case of (\ref{ex:pWkASWndo}), the context makes it clear that interpretation as the participle of a progressive form is the only possible, as the same verb with the progressive appears a few sentences before in (\ref{ex:meihua.ci.pjAkAsWndoci}).

\begin{exe}
\ex  \label{ex:meihua.ci.pjAkAsWndoci}
\gll   ɯ-jaʁ nɯtɕu, iɕqʰa, <meihua> ci pjɤ-k-ɤsɯ-ndo-ci, \\
\textsc{3sg}.\textsc{poss}-hand \textsc{dem}:\textsc{loc} \textsc{filler} plum.blossom \textsc{indef} \textsc{ifr}.\textsc{ipfv}-\textsc{evd}-\textsc{prog}-take-\textsc{evd} \\
 \glt `She was holding a plum blossom in her hand.' (150907 yingning-zh, 20)
\end{exe}

\subsubsection{Ambiguity} \label{sec:object.participle.ambiguity}
There is rampant ambiguity between object participles, \forme{kɤ-} infinitives and subject participles of passive verbs.  The question of the ambiguity between object participles and \forme{kɤ-} infinitives is discussed in § \ref{sec:velar.inf.ambiguity}.

The passive \forme{a-} merges with the subject participle as \ipa{kɤ}, homophonous with the infinitive and the object participle. Potentially ambiguous examples are very common. For instance, in (\ref{ex:kArku.passive}), the form \ipa{kɤrku} could be argued to be an object participle \forme{kɤ-rku} or a passive subject participle \forme{kɯ-ɤ-rku}; the second option is chosen here due to the semantics, which fits the passive \japhug{arku}{be put in, be located in} better (as this passive verb is in the process of becoming a locative existential verb, § XXX). In the absence of any argument in favour of the passive analysis, the ambiguous \ipa{kɤ-} forms are analyzed as object participles by default.

\begin{exe}
\ex \label{ex:kArku.passive}
 \gll  sɤtɕʰa ɯ-ŋgɯ kɯ-ɤ-rku <yangyu> cʰo lɤpɯɣ nɯra tu-ndze ŋgrɤl. \\
 earth \textsc{3sg}.\textsc{poss}-inside \textsc{nmlz}:S/A-\textsc{pass}-put.in potato \textsc{comit} radish \textsc{dem}:\textsc{pl} \textsc{ipfv}-eat[III] be.usually.the.case:\textsc{fact} \\
 \glt `It eats the radish and the potatoes that are in the ground.' (25-akWzgumba, 22)
\end{exe}

Due to the fact that passive verbs in Japhug are barely attested in perfective forms (§ XXX), participles with perfective prefixes can be considered to be object participles, especially in cases like (\ref{ex:YAXtAr.nWkAXtAr}), where the participle \forme{nɯ-kɤ-χtɤr}  `(those) that have been scattered' occurs in a sentence following the transitive form \forme{ɲɤ-χtɤr} `it scattered, it smashed'

\begin{exe}
\ex \label{ex:YAXtAr.nWkAXtAr}
 \gll    to-ɣi tɕe nɯ-ʑmbrɯ ɲɤ-χtɤr ʑo ɲɯ-ŋu tɕe, ɯ-zda ra nɯ-pʰe, nɤki,  ``nɯnɯ ʑmbrɯ nɯ-kɤ-χtɤr nɯ ɯ-taʁ kɤ-ɴqoʁ-nɯ ra'' to-ti tɕe,  \\
 \textsc{ifr}:\textsc{up}-come \textsc{lnk} \textsc{3pl}.\textsc{poss}-ship \textsc{ifr}-scatter \textsc{emph} \textsc{sens}-be \textsc{lnk} \textsc{3sg}.\textsc{poss}-companion pl \textsc{3pl}.\textsc{poss}-\textsc{dat} \textsc{filler} dem ship \textsc{pfv}-\textsc{nmlz}:P-scatter \textsc{dem} \textsc{3sg}.\textsc{poss}-on \textsc{imp}-hang-\textsc{pl} have.to:\textsc{fact} \textsc{ifr}-say \textsc{lnk} \\
 \glt `The (monster) came up and smashed their ship, and (Norbzang) said to his companions: ``Grab the (pieces of the) ship that have been scattered''.' (2012 Norbzang, 31-32)
\end{exe}

The same analysis as object participles, rather than passive subject participles is applied to examples of perfective \forme{kɤ-} forms also when the transitive verb is not found in finite form in a neighbouring sentence, such as (\ref{ex:pWkAprAt}).

\begin{exe}
\ex \label{ex:pWkAprAt}
 \gll fsapaʁ ɯ-ŋgo rcanɯ, pɯ-kɤ-prɤt ʑo tɤ-fse ɲɯ-ŋu. \\
 animals \textsc{3sg}.\textsc{poss}-disease unexpectedly \textsc{pfv}-\textsc{nmlz}:P-break \textsc{emph} \textsc{pfv}-be.like \textsc{sens}-be \\
 \glt  `It was like the disease of the cattle had been (suddenly) stopped.' (2003 kAndZislama, 190)
\end{exe}

A more marginal case of homophony occurs between object participles  and velar infinitives on the one hand, and several finite forms taking the series A orientation prefix \forme{kɤ-} on the other hand (§ \ref{sec:infinitives.participles}).

\subsubsection{Object relative clauses} \label{sec:object.participle.relatives}
Object participles can be used to build object relative clauses, but compete in this function with finite relatives (§ XXX). They differ in this regard from subject relatives, which are the only available construction to relativize transitive and intransitive subjects.

In object participial relatives, when the relativized element is overt, it is generally located before the participle, as in (\ref{ex:thWkAraGdWt}). Prenominal object participial relatives can be elicited, but are rarer in the corpus (\ref{ex:akAsWz.Cku} is such an example; see however the discussion about genitival relatives and example \ref{ex:tAkAsWBzu.GW.tWxtsa} below).

\begin{exe}
\ex \label{ex:thWkAraGdWt}
\gll  nɯŋa ɯ-ndʐi tʰɯ-kɤ-rɤɣdɯt, tʰɯ-kɤ-tʂɯβ nɯ ɯ-ŋgɯ nɯtɕu ko-ɕe  \\
cow \textsc{3sg.poss}-skin \textsc{pfv-nmlz:P}-skin \textsc{pfv-nmlz:P}-sew \textsc{dem} \textsc{3sg.poss}-inside \textsc{dem}:\textsc{loc} \textsc{evd:east}-go \\
\glt  `He went into the cow hide that had been skinned and sewed.'    (02-deluge2012, 32)
\end{exe}  

\begin{exe}
\ex \label{ex:akAsWz.Cku}
\gll   tɕe [aʑo a-kɤ-sɯz] ɕku nɯ nɯra ŋu \\
\textsc{lnk} \textsc{1sg} \textsc{1sg}.\textsc{poss}-\textsc{nmlz}:P-know allium dem \textsc{dem}:\textsc{pl} be:\textsc{fact} \\
\glt `These are the (plants belonging to the gender) \textit{allium} that I know about.' (07-Cku, 165)
\end{exe}  

When the relative only consists of the relativized element followed by the object participle as in (\ref{ex:thWkAraGdWt}), it is not possible to determine whether the relative is postnominal or hea-internal (§ XXX). In the case of longer relatives, in particular containing locative or instrumental adjuncts, it is possible to discriminate between the two types using the relative position of the relativized element and the adjunct. In (\ref{ex:qaR.thWkAsWBzu}) for instance, the head noun \japhug{qaʁ}{hoe} occurs between the instrumental adjunct \forme{qaʁ kɯ}  and the participle: this is an uncontroversial example of head-internal relative (§ XXX). There are no clear examples of postnominal object participial relatives in the corpus.

\begin{exe}
\ex \label{ex:qaR.thWkAsWBzu}
\gll   [rŋɯl kɯ qaʁ tʰɯ-kɤ-sɯ-βzu] nɯra ko-sɯ-ɤʑirja-nɯ. \\
silver \textsc{erg} hoe \textsc{pfv}-\textsc{nmlz}:P-\textsc{caus}-make \textsc{dem}:\textsc{pl} \textsc{ifr}-\textsc{caus}-be.align-\textsc{pl} \\
\glt `They aligned the hoe that had been made from silver.' (28-qAjdoskAt, 102)
\end{exe}  

As was described in § \ref{sec:object.participle.other.prefixes}, object participles, unlike subject and oblique participles, cannot combine possessive and orientation prefixes.

Object participles with orientation prefixes are used in relative clauses with indefinite subjects, or with definite third person subjects as in ( \ref{ex:qajGi.nWkAmbi}).  
 
\begin{exe}
\ex \label{ex:qajGi.nWkAmbi}
\gll     [ɬamu kɯ qajɣi nɯ-kɤ-mbi] nɯ tu-ndze pjɤ-ŋu \\
p.n. \textsc{erg} bread \textsc{pfv}-\textsc{nmlz}:S/A-give \textsc{dem} \textsc{ipfv}-eat[III] \textsc{ifr}.\textsc{ipfv}-be \\
\glt `(As) he was eating the (pieces of) bread that Lhamo had given him.' (2002 qajdoskAt, 111)
\end{exe}  

The only example of first or second person that could be interpreted as subject in a perfective object participle relative in the corpus is (\ref{ex:aZo.kW.pWkAsWrAt}), but in this example (translated from Chinese), the referent of the first person is a pen that has been used to write a poem; the ergative postpositional phrase \forme{aʑo kɯ} here can be either analyzed as an instrument (`the poem that has been written using me') or as a causee  (§ \ref{sec:causee.kW}, `the poem that he has made me write'), as shown by the presence of the causative \forme{sɯ-} prefix, not a subject.

\begin{exe}
\ex \label{ex:aZo.kW.pWkAsWrAt}
\gll   [aʑo kɯ pɯ-kɤ-sɯ-rɤt] nɯnɯ pjɯ-ndɯn ɲɯ-ŋu nétɕi \\
\textsc{1sg} \textsc{erg} \textsc{pfv}-\textsc{nmlz}:P-\textsc{caus}-write \textsc{dem} \textsc{ipfv}-read \textsc{sens}-be \textsc{sfp} \\
\glt (The pen said: the poet is reading the poem) that has been written using me.' (150818 bi he moshuihu-zh, 143)
\end{exe}  

When the subject is first or second person, an object participle with a possessive prefix  is used instead. In (\ref{ex:iZo.jikArku}) for instance, we find \forme{ji-kɤ-rku} `the thing that we give' (see § \ref{sec:z.nmlz} concerning the meaning of this verb) and \forme{nɤ-kɤ-sɯso} `the thing that you think / that you want' with a first plural and a second singular subject respectively.

\begin{exe}
\ex \label{ex:iZo.jikArku}
\gll   nɯ ma iʑo ji-kɤ-rku me,  atu spɣi tɤ-ɕe qʰe, laχtɕʰa ŋotɕu nɤ-kɤ-sɯso ʑo nɯnɯ, 
nɤ-mɲaʁ, nɤ-rna, nɤ-ɕna cʰo ra kɯ\redp{}kɯ-spoʁ nɯ ɯ-ŋgɯ tɕe a-kɤ-tɯ-rke qʰe, \\
\textsc{dem} apart.from \textsc{1pl} \textsc{1pl}.\textsc{poss}-\textsc{nmlz}:P-put.in not.exist:\textsc{fact} up.there granary imp:up-go lnk thing where \textsc{2sg}.\textsc{poss}-\textsc{nmlz}:P-think \textsc{emph} \textsc{dem} \textsc{2sg}.\textsc{poss}-eye  \textsc{2sg}.\textsc{poss}-ear  \textsc{2sg}.\textsc{poss}-nose \textsc{comit} \textsc{pl}  \textsc{total}\redp{}\textsc{nmlz}:S/A-have.a.hole \textsc{dem} \textsc{3sg}.\textsc{poss}-inside \textsc{loc} \textsc{irr}-\textsc{pfv}-2-put.in[III] \textsc{lnk} \\
\glt `We don't have anything else to give you as a departing present, go up there in the granary, and whatever you want, put it in all the holes (in your body), you eyes, you ears, you nose etc.' (31-deluge, 136)
\end{exe}  

When the subject is a definite third person, it is also possible to have a third person possessive prefix on the object participle, as in (\ref{ex:WkWnWmbrApW}) (or \ref{ex:WCWkAnAma} above).

\begin{exe}
\ex \label{ex:WkWnWmbrApW}
\gll  lɤ-fsoʁ ɯ-jɯja nɯ pjɯ-ru tɕe [ɯ-kɤ-nɯmbrɤpɯ] nɯ kʰu pɯ-ɕti ɲɯ-ŋu,  \\
\textsc{pfv}-be.clear    \textsc{3sg}-along  \textsc{dem} \textsc{ipfv:down}-look \textsc{lnk} \textsc{3sg-nmlz:P}-ride \textsc{dem} tiger \textsc{pst.ipfv}-be.\textsc{affirm}  \textsc{sens}-be \\
\glt `As the day broke, looking down, he (progressively realized that) what he was riding was a tiger.' (2005 khu, 20)
\end{exe}

Unlike in Tshobdun (\citealt[10]{jacksonlin07}), in Japhug object participial relatives with possessive prefixes are not restricted to generic state of affairs, but can refer to particular situations as in examples such as (\ref{ex:WkWnWmbrApW}) and (\ref{ex:nAkAti.nWra}).

\begin{exe}
\ex \label{ex:nAkAti.nWra}
\gll a-ɬaʁ, tɕe nɤ-kɯ-mŋɤm tɕʰi ɲɯ-fse ma [alo qʰaqʰu nɤ-kɤ-ti] nɯra tɤ-stu-t-a \\
\textsc{1sg}.\textsc{poss}-aunt \textsc{lnk} \textsc{2sg}.\textsc{poss}-\textsc{nmlz}:S/A-hurt what \textsc{sens}-be.like \textsc{lnk} upstream behind.the.house \textsc{2sg}.\textsc{poss}-\textsc{nmlz}:P-say \textsc{dem}:\textsc{pl} \textsc{pfv}-do.like-\textsc{pst}:\textsc{tr}-\textsc{1sg} \\
\glt `Stepmother, how do you feel, I did the things you said (about creating a lake) up there behind the house.' (28-smAnmi, 361)
\end{exe}

As mentioned above (example \ref{ex:nWkAmbi}), the object participles of secundative verbs can either refer to their object proper (the recipient, § XXX) or to the theme, which is not indexed on the verb but occurs in absolutive form (§ \ref{sec:theme.ditransitive}). In fact, in the corpus examples of theme relativization with the object participle are quite common (as \ref{ex:qajGi.nWkAmbi} above and \ref{ex:WkAmbi.maNe} and \ref{ex:pWkAsWxCAt.nWra} below), but recipient relativization is quite rare (\ref{ex:xCiri.nWkAsWxCAt}). Examples can however be elicited without difficulty.

\begin{exe}
\ex \label{ex:WkAmbi.maNe}
\gll nɯ ma ɯ-kɤ-mbi maŋe tɕe, ``a-me ta-mbi ra" to-ti tɕe, \\
\textsc{dem} apart.from \textsc{3sg}.\textsc{poss}-\textsc{nmlz}:P-give not.exist:\textsc{sens} \textsc{lnk} \textsc{1sg}.\textsc{poss}-daughter 1\fl{}2-give:\textsc{fact}  have.to:\textsc{fact} \textsc{ifr}-say \textsc{lnk} \\
\glt `He had nothing else to give him, and said `I give you my daughter'.' (2011-04-smanmi, 171)
\end{exe}

\begin{exe}
\ex \label{ex:pWkAsWxCAt.nWra}
\gll   tɕendɤre [tɯmɯkɤrŋi kɯ pɯ-kɤ-sɯxɕɤt] ra ɲɤ-nɤxtʂɯn tɕe tɕe nɯɕɯmɯma ʑo pjɤ-nɯ-ɕe. \\
\textsc{lnk} heaven \textsc{erg} \textsc{pfv}-\textsc{nmlz}:P-teach \textsc{pl} \textsc{ifr}-be.grateful \textsc{lnk} \textsc{lnk} immediately \textsc{emph} \textsc{ifr}:\textsc{down}-\textsc{vert}-go \\
\glt `(Pu'an) was thankful for the things that the god of heaven had taught him and went back (to earth) immediately.' (150827 taisui-zh, 135)
\end{exe}

\begin{exe}
\ex \label{ex:xCiri.nWkAsWxCAt}
\gll    iɕqʰa, [kɤntɕʰɯ-xpa ʑo, nɤki, xɕiri nɯ-kɤ-sɯxɕɤt] nɯ pjɤ-sat, \\
 the.aforementioned several-year \textsc{emph} filler weasel \textsc{pfv}-\textsc{nmlz}:P-teach \textsc{dem} \textsc{ifr}-kill \\
\glt `He killed the weasel that he had trained for several years.'  (140518 xuezhe he huangshulang-zh, 28)
\end{exe}

With indirective verbs, the object participle can only refer to the theme, as in (\ref{ex:nAkAthu.WGAZu}), for these verb the recipient must be relativized with the oblique participle  (§ \ref{sec:oblique.participle.relatives}).

\begin{exe}
\ex \label{ex:nAkAthu.WGAZu}
\gll nɤ-kɤ-tʰu ɯ-ɣɤʑu nɤ, tɤ-tʰe jɤɣ \\
\textsc{2sg}.\textsc{poss}-\textsc{nmlz}:P-ask \textsc{qu}-exist:\textsc{sens} \textsc{lnk} \textsc{imp}-ask[III] be.allowed:\textsc{fact} \\
\glt `If you have and questions, you can ask them.' (conversation 14-11-08)
\end{exe}

The semi-object of semi-transitive verbs (§ \ref{sec:semi.object}) can also be relativized with a object participial relative, as in (\ref{ex:stu.jikArga}).

\begin{exe}
\ex \label{ex:stu.jikArga}
\gll  iɕqʰa <macha> kɤ-ti nɯ [iʑora stu ji-kɤ-rga] ŋu \\
the.aforementioned macha.tea \textsc{nmlz}:P-say \textsc{dem} \textsc{1pl} most \textsc{1pl}.\textsc{poss}-\textsc{nmlz}:P-like be:\textsc{fact} \\
\glt `The (type of tea) called `macha' is what we like most.' (30-macha, 1)
\end{exe}

Secundative verbs undergoing antipassivization become semi-transitive verbs (§ XXX) with the theme remaining the semi-object. Like other semi-transitive verbs, these antipassive verbs can build an object participle, which can then be used to relativize the theme, as in (\ref{ex:nAkArAmbi}).

\begin{exe}
\ex \label{ex:nAkArAmbi}
\gll  nɤ-kɤ-rɤ-mbi nɯ tɕʰi pɯ-ŋu? \\
\textsc{2sg}.\textsc{poss}-\textsc{nmlz}:P-\textsc{apass}-give \textsc{dem} what \textsc{pst}.\textsc{ipfv}-be \\
\glt `What was it that you gave (to people)?' (elicited)
\end{exe}

Object participles also occur in genitival relatives (postnominal relative with the genitive postposition \forme{ɣɯ} occurring between the relative clause and the head noun, built like adnominal complement clauses, § XXX), as in example (\ref{ex:tAkAsWBzu.GW.tWxtsa}). This type of examples is frequently found in texts translated from Chinese, but unattested in the rest of the corpus for object relativization, and is a clear case of calque (§ XXX); although speakers do not find these examples to be ungrammatical, they cannot be considered to representative of the normal grammar of the language.

\begin{exe}
\ex \label{ex:tAkAsWBzu.GW.tWxtsa}
\gll  tɕe [<shuijing> kɯ tɤ-kɤ-sɯ-βzu] ɣɯ tɯ-xtsa nɯra jo-ɣɯt. \\
\textsc{lnk} crystal \textsc{erg} \textsc{pfv}-\textsc{nmlz}:P-\textsc{caus}-make \textsc{gen} \textsc{indef}.\textsc{poss}-shoe \textsc{dem}:\textsc{pl} \textsc{ifr}-bring \\
\glt `(The bird) brought shoes made of crystal.' (140504 huiguniang-zh, 162)
\end{exe}

\subsubsection{Other relative clauses}  \label{sec:object.participle.other.relative}
There is one case where an object participle to relativize a locative adjunct. The object participle of the perception verbs \japhug{mto}{see} and \japhug{mtsʰɤm}{hear} can be used to make headless locative relative clauses meaning `(a place) where $X$ can see/hear $Y$, in particular when occurring as the goal of a motion verb as in (\ref{ex:tCirna.mAkAmtshAm}) or (\ref{ex:WmAkAmto}). Note the optionality of the ergative on the nouns \japhug{tɕi-rna}{our ears} and \japhug{tɕi-mɲaʁ}{our eyes} in (\ref{ex:tCirna.mAkAmtshAm}). 

\begin{exe}
\ex \label{ex:tCirna.mAkAmtshAm}
\gll [tɕi-rna mɤ-kɤ-mtsʰɤm], [tɕi-mɲaʁ mɤ-kɤ-mto] a-jɤ-ɕe-ndʑi ra \\
\textsc{1du}.\textsc{poss}-ear \textsc{neg}-\textsc{nmlz}:P-hear \textsc{1du}.\textsc{poss}-eye \textsc{neg}-\textsc{nmlz}:P-see \textsc{irr}-\textsc{pfv}-go-\textsc{du} have.to:\textsc{fact} \\
\glt  `May they go away (to a place) where our ears cannot hear them, where our eyes cannot see them.' (2003-kWBRa, 23)
\end{exe}

\begin{exe}
\ex \label{ex:WmAkAmto}
\gll [iɕqʰa qaɕpa kɯ ɯ-mɤ-kɤ-mto] ʑo jo-ɕe  \\
the.aforementioned frog \textsc{erg} \textsc{3sg}.\textsc{poss}-\textsc{neg}-\textsc{nmlz}:P-see \textsc{emph} \textsc{ifr}-go   \\
\glt `She went to (a place) where the frog would not find her.' (150818 muzhi guniang-zh, 149)
\end{exe}

It is not possible in (\ref{ex:tCirna.mAkAmtshAm}) or (\ref{ex:WmAkAmto}) to replace the object participle by an oblique participle \forme{sɤ-}.

\subsubsection{Complement strategies} \label{sec:object.participles.complement}
While \forme{kɤ-} prefixed non-finite verb forms are very common in complement clauses, the near-totality of these forms are infinitives rather than object participles (§ \ref{sec:inf.complementation}), since there are no restrictions on intransitive verbs (§ \ref{sec:velar.inf}).

The only complement strategy where an object participle, rather than an infinitive, has to be posited occurs in the purposive clause of  motion verbs when the verb of the purposive clause is transitive and coreference occurs between its object (rather than subject) and the subject of the matrix motion verb, as \forme{kɤ-nɤkʰu} in (\ref{ex:kAnAkhu.juGi}).

\begin{exe}
\ex \label{ex:kAnAkhu.juGi}
\gll <xingqi> raŋri ʑo tɕe nɯnɯ sɤβʑɯ ɣɯ ɯ-kʰa nɯtɕu kɤ-nɤkʰu ju-ɣi pjɤ-ŋu  \\
week each \textsc{emph} \textsc{lnk} \textsc{dem} mouse \textsc{gen} \textsc{3sg.poss}-house \textsc{dem:loc} \textsc{nmlz:P}-invite \textsc{ipfv}-come \textsc{ipfv.ifr}-be \\
\glt `He would come to the mouse's house as a guest.' (150818 muzhi guniang-zh, 299).
\end{exe}

In purposive clauses, the rule is thus that the subject participle is used when there is subject-subject coreference (§ \ref{sec:subject.participle.complementation}), and the object participle in cases of object-subject coreference  (\citealt[248]{jacques16complementation}).
 
 Coreference between the P of the complement clause and the S of the matrix clause is possible only if the P has control over the action, something that is possible for only a few transitive verbs such as \japhug{nɤkʰu}{invite} (since the guest has the choice of accepting or refusing the invitation), explaining the rarity of this construction. 
 
 A possessive prefix coreferent with the transitive subject of \forme{nɤkʰu} can be optionally added on this object participle, as in (\ref{ex:akAnAkhu}).

\begin{exe}
\ex \label{ex:akAnAkhu}
\gll a-kɤ-nɤkʰu jɤ-ɣe  \\
 \textsc{1sg.poss-nmlz:P}-invite \textsc{pfv}-come[II] \\
\glt `He came to my house as a guest (following my invitation).' (elicited)
\end{exe}
 
\subsubsection{Lexicalized object participles} \label{sec:lexicalized.object.participle}
While some object participles are commonly used as headless relative clauses, few can be considered to be fully lexicalized. 

The verbs related to food ingestion such as \japhug{ndza}{eat}, \japhug{tsʰi}{drink}, \japhug{ndzɤtsʰi}{eat and drink}, \japhug{moʁ}{eat powdery food} have object participles such as \japhug{kɤ-ndza}{food}, \japhug{kɤ-tsʰi}{drink (n), beverage}, \japhug{kɤ-ndzɤtsʰi}{food and drink} and \japhug{kɤmoʁ}{dry tsampa}, which commonly occur in enumerations (§ \ref{sec:noun.enumeration}) with nouns not derived from verbs, as in (\ref{ex:WkAndza.WkAtshi}).  

\begin{exe}
\ex \label{ex:WkAndza.WkAtshi}
\gll  ɯ-kɤ-ndza ɯ-kɤ-tsʰi ɯ-tɯkrimgo ra to-ɣɯt qʰe, tɕendɤre, nɯra ɲɤ́-wɣ-mbi qʰe, \\
\textsc{3sg}.\textsc{poss}-\textsc{nmlz}:P-eat \textsc{3sg}.\textsc{poss}-\textsc{nmlz}:P-drink \textsc{3sg}.\textsc{poss}-butter.bread \textsc{pl} \textsc{ifr}:\textsc{up}-bring \textsc{lnk} \textsc{lnk} \textsc{dem}:\textsc{pl} \textsc{ifr}-\textsc{inv}-give \textsc{lnk} \\
 \glt  `She brought food, drinks and butter bread for her and gave them to her.' (2003-kWBRa, 70)
\end{exe}

In these enumerations, sometimes only the first element takes a possessive prefix, as in (\ref{ex:ndZikAndza.kAtshi}), where we find \forme{ndʑi-kɤ-ndza kɤ-tsʰi} instead of the equally possible \forme{ndʑi-kɤ-ndza ndʑi-kɤ-tsʰi} (however, if the first \forme{kɤ-} participle in the enumeration has no possessive prefix, the following participle cannot take one).

\begin{exe}
\ex \label{ex:ndZikAndza.kAtshi}
\gll  ndʑi-kɤ-ndza kɤ-tsʰi mɤ-mbrɤt, ndʑi-kɯ-ndzɤtsʰi a-pɯ-me smɯlɤm \\
\textsc{2du}.\textsc{poss}-\textsc{nmlz}:P-eat \textsc{nmlz}:P-drink \textsc{neg}-\textsc{acaus}:cut  \textsc{2du}.\textsc{poss}-\textsc{nmlz}:S/A-eat.and.drink \textsc{irr}-\textsc{ipfv}-not.exist prayer \\
\glt `May you never lack food or drink, may there nobody (coming to) eat you.' (2003kAndZWslama, 218)
\end{exe}

In these examples, the possessive prefix always refer to the person or animal ingesting the food (not the person giving the food), and although these forms are very common, since their semantics is completely predictable from the base verb, and since the possessive prefix behaves like that of a normal oblique participle, there is no specific reason to consider that they have become nouns and constitute lexical entries distinct from the verb (except in the case of \japhug{kɤmoʁ}{dry tsampa}, whose meaning has become more specific).

The forms \forme{kɤ-pa} and \forme{kɤ-stu} from the verbs from the verbs \japhug{pa}{do} and \japhug{stu}{do like} respectively, both meaning `manner, method (to solve a problem)' (like Chinese \ch{办法}{bànfǎ}{method}) are other potential candidates to be analyzed as lexicalized object participles (or infinitives).  They are particularly commonly used with existential verbs to mean `$X$ has (no/a) way to do it' ($X$ being referred to by the possessive prefix on \forme{kɤ-pa} or \forme{kɤ-stu}), as in (\ref{ex:akApa.maNe}).

\begin{exe}
\ex \label{ex:akApa.maNe}
\gll a-kɤpa maŋe \\
\textsc{1sg}.\textsc{poss}-method not.exist:\textsc{sens} \\
\glt `I have no way to do it.' (many attestations)
\end{exe}
 
However, collocation in texts of \forme{kɤ-pa} and \forme{kɤ-stu} with the finite forms of the verbs \japhug{pa}{do} and \japhug{stu}{do like}, as in (\ref{ex:nAkApa.tWtu}) and, suggest that these forms are still synchronically linked with these verbs, and that it may be more economical to analyze them as participles rather than derived nouns.
 
 \begin{exe}
\ex \label{ex:nAkApa.tWtu}
\gll  nɤ-kɤ-pa tɯ\redp{}tu nɤ,  tɤ-pe ma mtsʰoʁlaŋ tɤ-ɣe  \\
\textsc{2sg}.\textsc{poss}-\textsc{nmlz}:P-do \textsc{cond}\redp{}exist:\textsc{fact} \textsc{lnk} \textsc{imp}-do[III] \textsc{lnk} water.monster \textsc{pfv}:\textsc{up}-come[II] \\
\glt `If you have some way (to protect us), use it, because the water monster has come.' (Norbzang 2012, 27-28)
 \end{exe}
 
  \begin{exe}
\ex \label{ex:nAkAstu.WGAZu}
\gll   nɯnɯ ɯ-taʁ nɯtɕu nɤ-kɤ-stu ɯ-ɣɤʑu tɕe a-tɤ-tɯ-ste ma tɕe \\
\textsc{dem} \textsc{3sg}.\textsc{poss}-on \textsc{dem}:\textsc{loc} \textsc{2sg}.\textsc{poss}-\textsc{nmlz}:P-do.like \textsc{qu}-exist:\textsc{sens} \textsc{lnk} \textsc{irr}-\textsc{pfv}-2-do.like[III] \textsc{lnk} \textsc{lnk} \\
\glt `If you have a way to deal with him, use it.'   (25-kAmYW-XpAltCin, 37)
 \end{exe}
 
 %kɤpupu
\subsection{Oblique participles} \label{sec:oblique.participle}
The \forme{sɤ}-prefix (and its allomorphs \forme{sɤɣ}-, \forme{sɤz}- and \forme{z}-) is used for non-core argument nominalization, in particular recipient of indirective verbs (§ \ref{sec:gen.beneficiary}, § \ref{sec:dative}), instruments (\ref{sec:instr.kW}), place and time adjuncts, as in (\ref{ex:come}). It takes a possessive prefix which can be coreferent with any core argument (subject or object).

\begin{exe}
\ex \label{ex:come}
\gll ɯ-sɤ-ɣi \\
 \textsc{3sg-nmlz:oblique}-come \\
\glt  `The place/moment where/when it comes.' (elicited)
\end{exe}
 
 %nɯnɯ ɯ-sɤtɕha, tɤjmɤɣ ɯ-sɤ-tu ɯ-sɤ-me ɣɤʑu.
 %ɯnɯnɯ kɯ ɯ-sɤ-ɕmi tu-sɯ-βzu-nɯ pɯ-ŋgrɤl 
 %tɕe saŋdi nɯ tɕe, nɯnɯ /nɤki/ si ɯ-sɤ-ta ɯ-rkoz ʑo pjɤ-ŋu.
 \subsubsection{Allomorphy} \label{sec:oblique.participle.allomorphy}
The base form of the oblique participle is \forme{sɤ-}, but three additional allomorphs are also found: \forme{sɤz-}, \forme{z-} and \forme{sɤɣ-}.

The allomorphs \forme{sɤz-} and \forme{z-} occur in the same context, with non-monosyllabic verb stems, where the first syllable (either a productive or a frozen prefix) is sonorant-initial. These two allomorphs are completely interchangeable, without restriction on particular verbs or the function of the the relativized element (instrument, locative or temporal adjunct). For instance, the locative participle of \japhug{rɤʑi}{stay} is attested as both \japhug{ɯ-sɤz-rɤʑi} and \japhug{ɯ-z-rɤʑi} `the place when he/it stays' in the corpus, as shown by examples (\ref{ex:WsAzrAZi}) and (\ref{ex:WzrAZi}), a few sentences away from each other in the same story.

\begin{exe}
\ex \label{ex:WsAzrAZi}
\gll  tɕeri nɯnɯ sɤtɕʰa nɯ li iɕqʰa qapribɯxsi ɯ-sɤz-rɤʑi pjɤ-ɕti. \\
but \textsc{dem} place \textsc{dem} again the.aforementioned python \textsc{3sg}.\textsc{poss}-\textsc{nmlz}:\textsc{oblique}-stay \textsc{ifr}.\textsc{ipfv}-be.\textsc{affirm} \\
\glt `But that place was the abode of a python.' (140511 xinbada-zh, 92)
\end{exe}

\begin{exe}
\ex \label{ex:WzrAZi}
\gll tɕe <xinbaba> rcanɯ, maka nɯtɕu ɯ-z-rɤʑi ɯ-tɯ-sɤɣ-mu pjɤ-sɤre ʑo tɕe, \\
\textsc{lnk} Sinbad \textsc{unexpected} at.all \textsc{dem}:\textsc{loc} \textsc{3sg}.\textsc{poss}-\textsc{nmlz}:\textsc{oblique}-stay \textsc{3sg}.\textsc{poss}-\textsc{nmlz}:\textsc{degree}-\textsc{deexp}-be.afraid \textsc{ifr}.\textsc{ipfv}-be.ridiculous \textsc{emph} \textsc{lnk} \\
\glt `Sinbad, the place where he stayed was extremely terrifying.' (140511 xinbada-zh, 99)
\end{exe}

%tɕɤtʰi iʑo prɤscʰɯ ɯ-tʰɤcu lu-sɤ-nɯ-ɬoʁ nɯnɯre ri tɯ-ji ci tu tɕe, nɯnɯ tɤɕɤttɤku rmi.
%140522_kAmYW_tWji2, 1
%ɛexceptions: sɤ-nɯɕe, sɤ-nɯɬoʁ vs sɤɣ-nɯɬoʁ

 \subsubsection{Possessive prefixes} \label{sec:oblique.participle.possessive}
 
\subsubsection{Polarity and orientation prefixes} \label{sec:oblique.participle.orientation}
Unlike subject and object participles, the only prefixes (other than possessive prefixes) that oblique participles can take are the polarity prefixes and series B orientation prefixes.

It is thus not possible to have perfective or past imperfective oblique participles, and alternative strategies are used to express the corresponding meanings. For instance, from the verb \japhug{sqa}{cook}, the form $\dagger$\forme{ɯ-pɯ-sɤ-sqa} (intended meaning: `the thing that has been used to cook') is incorrect, and the solution to circumvent this morphological constraint is to combine the plain oblique participle \forme{ɯ-sɤ-sqa} with \forme{pɯ-kɯ-ŋu} (the past imperfective subject participle of \japhug{ŋu}{be}) and with the phrase \forme{nɯ ɕɯŋgɯ} `before that', as in (\ref{ex:WsAsqa.pWkWNu}).

\begin{exe}
\ex \label{ex:WsAsqa.pWkWNu}
\gll  nɯ ɕɯŋgɯ ɯ-sɤ-sqa pɯ-kɯ-ŋu ɯ-ŋgɯ (tu-rku-nɯ) \\
\textsc{dem} before \textsc{3sg}.\textsc{poss}-\textsc{nmlz}:\textsc{oblique}-cook \textsc{pst}.\textsc{ipfv}-\textsc{nmlz}:S/A-be \textsc{3sg}.\textsc{poss}-inside \textsc{ipfv}-put.in-\textsc{pl} \\
\glt `(They put it) in the (pan) that had been used before to cook (the barley grains).' (31-cha, 64)
\end{exe}

Negative forms of the oblique participle are not very common, but examples are found in the corpus (as in \ref{ex:WmAsApe}) and there is no difficulty to elicit them.

\begin{exe}
\ex \label{ex:WmAsApe}
\gll  qaʑmbri nɯ, nɤkinɯ, ɯ-sɤ-pe ra me, ɯ-mɤ-sɤ-pe ra me, \\
vine \textsc{dem} \textsc{filler} \textsc{3sg}.\textsc{poss}-\textsc{nmlz}:\textsc{oblique}-be.good \textsc{pl} not.exist:\textsc{fact}  \textsc{3sg}.\textsc{poss}-\textsc{neg}-\textsc{nmlz}:\textsc{oblique}-be.good \textsc{pl} not.exist:\textsc{fact} \\
\glt `The vine is neither an advantage nor a harm (to the plants on which it grows).' (06-qaZmbri, 17)
\end{exe}

\subsubsection{Oblique relative clauses} \label{sec:oblique.participle.relatives}

The relativized locative adjunct can also be the place of origin rather than the goal, as in (\ref{ex:jisAGi}).

\begin{exe}
\ex \label{ex:jisAGi}
\gll  iʑora nɯ ji-sɤ-ɣi nɯtɕu pjɤ-ŋu tɕe. \\
C \textsc{dem} \textsc{1pl}.\textsc{poss}-\textsc{nmlz}:\textsc{oblique}-come \textsc{dem}:\textsc{loc} \textsc{ifr}.\textsc{ipfv}-be \textsc{lnk} \\
\glt  `The place from where we come was there.' (2010-06, 3)
\end{exe}
Oblique participle keep the verb transitivity, and transitive verbs can take an overt object as \forme{qaj ɯ-sɤ-ji} `place for planting wheat' in (\ref{ex:qaj.WsAji}).

\begin{exe}
\ex \label{ex:qaj.WsAji}
\gll  qajsta nɯnɯ kɯɕɯŋgɯ qaj ɯ-sɤ-ji pjɤ-pe tɕe tɕe núndʐa qajsta tu-ti-nɯ ɲɯ-ŋu \\
pl.n. \textsc{dem} in.former.times wheat 3sg.poss-\textsc{nmlz}:\textsc{oblique}-plant \textsc{ifr}.\textsc{ipfv}-be.good \textsc{lnk} \textsc{lnk} for.this.reason pl.n. \textsc{ipfv}-say-\textsc{pl} \textsc{sens}-be \\
\glt `Qaysta, in former time it was a wheat field which was good, and for this reason, it is called `Qaysta' `the place of the wheat'.' (140522 kAmYW tWji2, 104)
\end{exe}

An antipassive form (\ref{sec:antipassive}) is necessary if there is no definite object. For instance in (\ref{ex:sAZrAji}), \forme{sɤz-rɤ-ji} `place for planting, field' is based on the \forme{rɤ-} antipassive of \japhug{ji}{plant}; this form, unlike \forme{ɯ-sɤ-ji} in (\ref{ex:qaj.WsAji}), is used without (and cannot occur with) any noun specifying the crop planted in the field.

\begin{exe}
\ex \label{ex:sAZrAji}
\gll  tɕe tɯmgri sɤz-rɤ-ji nɯ koŋla ɲɤ-ɣɤ-me-nɯ ma kʰa ʁɟa ʑo to-βzu-nɯ. \\
lnk pl.n. \textsc{nmlz}:\textsc{oblique}-\textsc{apass}-plant \textsc{dem} completely \textsc{ifr}-\textsc{caus}-not.exist-\textsc{pl} \textsc{lnk} house completely \textsc{emph} \textsc{ifr}-make-\textsc{pl} \\
\glt `They removed all the fields in Temgri, and built houses there (instead).' (140522 kAmYW tWji2, 18)
\end{exe}


\subsubsection{Ambiguity} \label{sec:oblique.participle.ambiguity}

\subsubsection{Lexicalized oblique participles} \label{sec:lexicalized.oblique.participle}
Nouns of instruments and of location, including placenames, are often made from oblique participles. 

The noun \japhug{sɤcɯ}{key}, although transparently originating from the instrumental use of the oblique participle of \japhug{cɯ}{open}, is lexicalized as shown by the fact that it cannot take orientation prefixes, and that it occurs in collocation with the auxiliary \forme{lɤt} to mean `lock (the door)' as in (\ref{ex:sAcW.malAt}).

\begin{exe}
\ex \label{ex:sAcW.malAt}
\gll   ɯ-ŋgɯ lɤ-ɣi jɤɣ ma sɤcɯ mɤ-a-lɤt \\
\textsc{3sg}.poss-inside \textsc{imp}:\textsc{upstream}-come be.allowed:\textsc{fact} \textsc{lnk} key \textsc{neg}-\textsc{pass}-throw \\
\glt `Come in, the door is not locked.' (140428 xiaohongmao, 78)
\end{exe}

Place names built from oblique participle include \forme{Znɤrɣɤma}, from the locative participle \forme{z-nɤrɣɤma} of the verb \japhug{nɤrɣɤma}{pray for rain} (see § XXX on the etymology of this verb), as it was the place where people use to perform this activity in Kamnyu in the traditional society, as explained in § XXX.

Another example is the uninhabited place called \forme{kɯlɤɣsɤmdzɯ}, a transparent combination  \japhug{kɯ-lɤɣ}{shepherd} (§ \ref{sec:lexicalized.subject.participle}) and \japhug{ɯ-sɤ-ɤmdzɯ}{sitting place} reflecting the use of this place (as described in \ref{ex:kWlAG.sAmdzW}).

\begin{exe}
\ex \label{ex:kWlAG.sAmdzW}
\gll    kɯ-xtɕɯ\redp{}xtɕi ci ʑo antɤm, tɕe nɯ kɯ-lɤɣ ra nɯtɕu ku-rɤʑi-nɯ pjɤ-ŋgrɤl \\
\textsc{inf}:\textsc{stat}-\textsc{emph}\redp{}be.small a.little \textsc{emph} be.flat:\textsc{fact}  \textsc{lnk} \textsc{dem} \textsc{nmlz}:S/A-graze \textsc{pl} \textsc{dem}:\textsc{loc} \textsc{ipfv}-stay-\textsc{pl} \textsc{ifr}.\textsc{ipfv}-be.usually.the.case \\
\glt `(The place called \forme{kɯlɤɣsɤmdzɯ}) is a bit flat, and shepherd used to stay there.' (140522 Kamnyu zgo, 281)
 \end{exe}
 
There are also case of nouns of instruments in \forme{sɤ-} whose base verb is not identifiable. For instance, the noun \japhug{sɤɕtɕɯɣ}{strap to carry children on the back}, which is glossed using an oblique participle as in (\ref{ex:tApAtso.WsAzbWwa}), is most certainly a frozen oblique participle, but there is no verb \forme{*ɕtɕɯɣ} in Japhug.

\begin{exe}
\ex \label{ex:tApAtso.WsAzbWwa}
\gll   tɤ-pɤtso ɯ-sɤz-bɯwa \\
\textsc{indef}.\textsc{poss}-child \textsc{3sg}.\textsc{poss}-\textsc{nmlz}:\textsc{oblique}-carry.on.the.back \\
\glt `Something used to carry children on the back' (definition given for the noun \forme{sɤɕtɕɯɣ})
\end{exe}

An even more pronounced type of lexicalization occurs when an oblique participle becomes member of a compound with \textit{status contructus} vowel alternation. In these cases, it is probable that the compound is genitival, rather than a reduced relative clause.

As first element of compound, we find the oblique participle \forme{ɯ-sɤ-qru} (from the verb \japhug{qru}{greet, welcome, receive}) in \textit{status constructus} combined with the noun \japhug{cʰa}{alcohol} into 
\japhug{sɤqrɤcʰa}{alcohol to treat the guests} (§ \ref{sec:tatpurusha.n.n}). 

As second element, the oblique participle \japhug{ɯ-sɤ-ɴɢɤt}{place where X part ways} (from the intransitive verb \japhug{ɴɢɤt}{part ways, part company}) is combined with the \textit{status constructus} of \japhug{tʂu}{path} into the compound \japhug{tʂɤsɤɴɢɤt}{crossroad}. 

\section{Infinitives}

 
\subsection{Velar infinitives} \label{sec:velar.inf}
The most common infinitives in Japhug are the velar infinitives, build from the stem I of the verb and prefixed either with \forme{kɤ-} or or \forme{kɯ-}; they are homophonous with participles and not always easily distinguishable from them (§ \ref{sec:infinitives.participles}). They are in particular the preferred citation form of the verbs (§ \ref{sec:inf.citation}), though not with all speakers.
 
The \forme{kɯ-} infinitives are found with stative verbs (including adjectives and existential verbs), impersonal modal verbs and some anticausative verbs (§ XXX); other verbs take the \forme{kɤ-} infinitives. 

Stative verbs in \forme{a-} have regular fusion of \forme{kɯ-} and \forme{a-} as \ipa{kɤ-}, and thus superficially appear to have \forme{kɤ-} infinitives (for instance, the infinitive of \japhug{arŋi}{be green} is \forme{kɯ-ɤrŋi} \ipa{kɤrŋi}).

\subsubsection{Infinitives vs participles} \label{sec:infinitives.participles}
It is not immediately obvious that a category of `velar infinitives' needs to be distinguished from participles in Japhug, as both are non-finite verbal categories prefixed in \forme{kɤ-} or \forme{kɯ-} (§ \ref{sec:object.participle.ambiguity} and § \ref{sec:subject.participle.ambiguities}). 

The necessity to set \forme{kɤ-} infinitives apart from object participles stems from the fact that the latter can only be built from transitive or semi-transitive verbs, while the former also occurs with strictly intransitive verbs. Thus, if one were to argue that all \forme{kɤ-}prefixed non-finite forms of transitive verbs are object participles, including in the case of complement clauses (for instance \forme{kɤ-ndza} in \ref{ex:kAndza.mWpWrYota}), one would not be able to account for the \forme{kɤ-}prefixed forms of intransitive verbs occurring in the same context such as \forme{kɤ-ɕe} in (\ref{ex:kACe.mWpWrYota}) -- even though \japhug{ɕe}{go} could be considered to be an indirect transitive verb (since it can take a goal, which can be relativized with a finite relative clause, § XXX), it is not possible to build a participial relative clause by prefixing \forme{kɤ-} on this verb (the oblique participle \forme{sɤ-} must be used instead, § \ref{sec:oblique.participle}).

\begin{exe}
\ex \label{ex:kAndza.mWpWrYota}
\gll aʑo kɤ-ndza mɯ-pɯ-rɲo-t-a \\
\textsc{1sg} ???-eat \textsc{neg}-\textsc{pfv}-experience-\textsc{pst}:\textsc{tr}-\textsc{1sg} \\
\glt `I never ate that.' (many attestations)
\end{exe}

\begin{exe}
\ex \label{ex:kACe.mWpWrYota}
\gll  aj kɤ-ɕe mɯ-pɯ-rɲo-t-a \\
\textsc{1sg} \textsc{inf}-go \textsc{neg}-\textsc{pfv}-experience-\textsc{pst}:\textsc{tr}-\textsc{1sg}  \\
\glt `I never went there.' (150820 ZNGWloR, 4)
\end{exe}

I therefore adopt the following criteria to discriminate between a \forme{kɤ-} infinitive and an object participle:  \forme{kɤ-} non-finite forms of (non-semi-transitive) intransitive verbs are infinitives; \forme{kɤ-} non-finite forms of transitive and semi-transitive verbs occurring in the same contexts as the infinitives of intransitive verbs are infinitives.

By systematically applying these criteria, we can identify three contexts where infinitives are attested: citation form (§ \ref{sec:inf.citation}), complementation (§ \ref{sec:inf.complementation}; there are however a few cases of object participles used in complement clauses, § \ref{sec:object.participles.complement}), and manner converbs (§ \ref{sec:inf.converb}). 

Distinguishing between subject participles and \forme{kɯ-} infinitives in Japhug is less straightforward, unlike in other Gyalrong languages such as Tshobdun for instance, \citealt{jackson14morpho}, since even stative verbs take the \forme{kɤ-} infinitive in complement clauses (§ \ref{sec:kW.infinitive}). The only clear contexts where \forme{kɯ-} infinitives do occur is that of citation forms (§ \ref{sec:inf.citation}) and complement clauses containing impersonal modal verbs (§ \ref{sec:kW.infinitive}). Converbs in \forme{kɯ-} are analyzed as infinitives rather than subject participle because they are only attested with stative verbs or other verbs taking the \forme{kɯ-} infinitives (§ \ref{sec:inf.converb}).

\subsubsection{Associated motion, polarity and orientation prefixes on infinitives}  \label{sec:infinitives.other.prefixes}
With the exception of the construction in § \ref{sec:inf.exist}, \forme{kɤ-} infinitives doe not take possessive prefixes. However, like participles, they are compatible with associated motion (\ref{ex:CWkAXtW}), 

\begin{exe}
\ex \label{ex:CWkAXtW}
\gll a-mgɯr ɲɯ-mŋɤm tɕe ɕɯ-kɤ-χtɯ mɯ́j-cʰa-a \\
\textsc{1sg}.\textsc{poss}-back \textsc{sens}-hurt \textsc{lnk} \textsc{transloc}-\textsc{inf}-buy \textsc{neg}:\textsc{sens}-can-\textsc{1sg} \\
\glt `My back hurts and I cannot go to  buy (apples).' (conversation, 30-04-2018)
\end{exe}

\begin{exe}
\ex \label{mWpjWkAlhoR.ftCaka}
 \gll  tɤ-se mɯ-pjɯ-kɤ-ɬoʁ ftɕaka tu-βze-a tu-mdzoz-a pɯ-ŋu ma, \\
 indef.\textsc{poss}-blood \textsc{neg}-\textsc{ipfv}-\textsc{inf}-come.out manner \textsc{ipfv}-make[III]-\textsc{1sg} \textsc{ipfv}-avoid-\textsc{1sg} \textsc{pst}.\textsc{ipfv}-be lnk \\
\glt `I avoided by all means to let the blood come out.' (24-pGArtsAG, 57)
 \end{exe}
 
\subsubsection{Ambiguity}  \label{sec:velar.inf.ambiguity}
Aside from the homophony between infinitives and participles discussed in § \ref{sec:infinitives.participles}, another type of ambiguity occurs with verbs selecting the orientation prefix `east', whose A form is \forme{kɤ-} (§ XXX). Intransitive verbs in imperative singular and perfective third singular forms (for instance \forme{kɤ-rŋgɯ} `he laid down') and transitive verbs without stem alternation in imperative singular (\forme{kɤ-tsʰi} `drink!', § XXX) have forms that are homophonous with the corresponding infinitives ( \forme{kɤ-rŋgɯ} `to lie down', \forme{kɤ-tsʰi}  to drink'), but these cases are never really ambiguous, as it is trivial to distinguish between a finite verb form and a non-finite one, for instance by changing from singular to dual or plural. 

For instance, in a particular context, if a form such as \forme{kɤ-rŋgɯ} can be changed to the corresponding plural \forme{kɤ-rŋgɯ-nɯ} (which can be either perfective \textsc{pfv}-lie.down-\textsc{pl} `they laid down' or imperative `lie down!'), it is possible to conclude that this \forme{kɤ-rŋgɯ} is necessarily finite (since non-finite verb forms in Japhug never take indexation affixes, § XXX) and cannot be an object participle or an infinitive.

\subsubsection{Citation form} \label{sec:inf.citation}

\subsubsection{Complementation}    \label{sec:inf.complementation}

\subsubsection{Doubly prefixed velar infinitives with negative existential verb} \label{sec:inf.exist}
The infinitive in \forme{kɤ-} can even take two prefixes in a construction combining the negative existential verb \japhug{me}{not exist} with a verb in the infinitive prefixed with a series B orientation prefix and a possessive prefix coreferent with the subject,\footnote{The assertion in \citet[228]{jacques16complementation} that infinitives cannot take possessive prefixes is thus wrong.} meaning `have no way to X, be completely unable to X', as in (\ref{ex:ndZijukACe}) and (\ref{ex:WpjWkAnWZWB}). Note that since in both of these examples, the verbs are intransitive and lack an object participle, the \forme{kɤ-} form can only be analyzed as an infinitive here. This construction is also possible with transitive verbs, in which case the possessive prefix corresponds to the transitive subject.

\begin{exe}
\ex \label{ex:ndZijukACe}
\gll tɕe ndʑi-ju-kɤ-ɕe pjɤ-me \\
\textsc{lnk} \textsc{3du}.\textsc{poss}-\textsc{ipfv}-\textsc{inf}-go \textsc{ifr}.\textsc{ipfv}-not.exist \\
\glt `They could not go.' (150908 menglang-zh, 46)
\end{exe}

\begin{exe}
\ex \label{ex:WpjWkAnWZWB}
\gll tɯ-rʑaʁ nɯ ɯ-pjɯ-kɤ-nɯʑɯβ pjɤ-me matɕi, \\
one-night \textsc{dem} \textsc{3sg}.\textsc{poss}-\textsc{ipfv}-\textsc{inf}-sleep \textsc{ifr}.\textsc{ipfv}-not.exist \textsc{lnk} \\
\glt `He could not sleep the whole night, because...' (150831 BZW kAnArRaR, 12)
\end{exe}

A derived construction involves the causative \japhug{ɣɤme}{cause not to exist, suppress} with doubly prefixed infinitives to `make it impossible for X to Y' as in (\ref{ex:apjWkAnWZWB.naGAme}).

\begin{exe}
\ex \label{ex:apjWkAnWZWB.naGAme}
\gll a-pjɯ-kɤ-nɯʑɯβ na-ɣɤ-me \\
\textsc{1sg}.\textsc{poss}-\textsc{ipfv}-\textsc{inf}-sleep \textsc{pfv}:3\fl{}3'-\textsc{caus}-not.exist \\
\glt `He made me unable to sleep.' (elicited)
\end{exe}

%cʰa a-ku-kɤ-tsʰi na-ɣɤme
There is a variant of this construction with imperfective subject participles in \forme{kɯ-} instead of infinitives (§ \ref{sec:subject.participle.complementation}).
\subsubsection{\forme{kɯ-} infinitives in complement clauses}    \label{sec:kW.infinitive}
Stative verbs, when occurring in a complement clause, generally take the \forme{kɤ-} infinitive, as in example  (\ref{ex:rYo}) and (\ref{ex:kAscit}). The main verb of the complement clauses in these examples have the \forme{kɤ-} infinitive, even though both \japhug{tu}{exist} and \japhug{scit}{be happy} are stative verbs and have a citation form with the \forme{kɯ-} prefix.

\begin{exe}
\ex \label{ex:rYo}
\gll   a-rŋɯl kɤ-tu pɯ-rɲo-t-a \\
\textsc{1sg.poss}-money \textsc{inf}-exist \textsc{pst:ipfv}-experience-\textsc{pst:tr-1sg} \\
\glt `I used to have money'. (elicited)
\end{exe}

\begin{exe}
\ex \label{ex:kAscit}
 \gll  kɤ-scit pjɤ-ŋgrɯ ɲɯ-ŋu  \\
 \textsc{inf}-be.happy \textsc{ifr}-succeed \textsc{sens}-be \\
 \glt `She succeeded in being happy.' (150818 muzhi guniang, 6)
 \end{exe} 
 
However, the conversion to \forme{kɤ-} infinitive only applies to stative verbs, not to  impersonal modal verbs such as \japhug{ra}{have to, need}. When the latter occur in a complement clause, as in example (\ref{ex:kAndza.kWra}), they always have the \forme{kɯ-} prefix.

\begin{exe}
\ex \label{ex:kAndza.kWra}
\gll  smɤn kɤ-ndza kɯ-ra pɯ-rɲo-t-a  \\ 
medicine \textsc{inf}-eat \textsc{inf}-have.to  \textsc{pst:ipfv}-experience-\textsc{pst:tr-1sg} \\
\glt `I used to have to take medicine.' (elicited)
\end{exe}


\subsubsection{Converbial use}    \label{sec:inf.converb}

Infinitives in \forme{kɯ-} (non-volitional or stative) also occur as converbs, though these forms could in principle also be analyzed as subject participles (\ref{sec:subject.participles}). For instance, in (\ref{ex:mAkWmbrAt.YWrAma}) \forme{mɤ-kɯ-mbrɤt} `without stop' is considered to be a non-volitional infinitive serving as a manner converb, but it could be possible to propose an alternative analysis as a \forme{kɯ-} subject participle `the one which does not stop' used adverbially.  The analysis as infinitives however better accounts for the fact that only the verbs whose infinitive is in \forme{kɯ-} have converbial forms in \forme{kɯ-}.

\begin{exe}
\ex \label{ex:mAkWmbrAt.YWrAma}
 \gll nɯ maka mɤ-kɯ-mbrɤt ʑo ɲɯ-rɤma ɲɯ-ɕti tɕe,  \\
 \textsc{dem} at.all  \textsc{neg}-\textsc{inf:n.vol}:S/A-\textsc{acaus}:break \textsc{emph} \textsc{ipfv}-work \textsc{sens}-be.\textsc{affirm} \textsc{lnk} \\
 \glt `It works without stopping at all.' (26-GZo, 69)
\end{exe}

Adjectives are also used adverbially, and can even become degree adverbs, as \japhug{kɯ-xtɕɯ\redp{}xtɕi}{a little} from \japhug{xtɕi}{be small} in examples such as (\ref{ex:kWxtCWxtCi.wxti}).

\begin{exe}
\ex \label{ex:kWxtCWxtCi.wxti}
 \gll βʑɯ sɤz kɯ-xtɕɯ\redp{}xtɕi wxti. \\
 mouse \textsc{comp} \textsc{inf:stat}-\textsc{emph}\redp{}be.small be.big:\textsc{fact} \\
 \glt `It is a little bigger than a mouse.' (21-GzWLa, 4)
\end{exe}

The existential verbs also occur in converbial use. For instance, \japhug{tu}{exist} in infinitive form \forme{kɯ-tu} following a noun or a pronoun can mean `in the presence of...' as in (\ref{ex:Zara.kWtu.Zo}).

\begin{exe}
\ex \label{ex:Zara.kWtu.Zo}
\gll ʑara kɯ-tu ʑo to-sɤrɯru tɕe, \\
\textsc{3pl} \textsc{inf:stat}-exist \textsc{emph} \textsc{ifr}-compare \textsc{lnk} \\
\glt `He compared (his testimony with theirs) in their presence.' (150909 xifangping-zh, 155)
\end{exe}
 
\subsection{Dental infinitives} \label{sec:dental.inf}
\subsection{Bare infinitives} \label{sec:bare.inf}
\section{Degree nominals} \label{sec:degree.nominals}

\section{Other deverbal nouns}

\subsection{Nominalization \forme{-z} suffix} \label{sec:z.nmlz}
Japhug has three inalienably possessed nouns derived from verbs by means of a nominalizing \forme{-z}, two of which take the indefinite possessive prefix \forme{tɤ-} (§ \ref{sec:inalienably.possessed}).

The first one, \japhug{tɤ-rkuz}{parting present}, a biactantial IPN whose possessor corresponds to the recipient (§ \ref{sec:biactantial.ipn}), comes from \japhug{rku}{put in}. The etymological relationship  between these two words is obvious when the use of  \forme{rku} in the sense of `give as a present to take away' (put in someone's luggage) is considered, as in (\ref{ex:kWki.nArkuz.Nu}).

\begin{exe}
\ex \label{ex:kWki.nArkuz.Nu}
\gll tɕe tó-wɣ-z-rɤŋgat tɕe, tɕendɤre nɯnɯ kɯ, iɕqʰa nɯ,  tɯ-ci tɯ-tɤ-ste to-rku. tɕe `kɯki nɤ-rkuz ŋu' to-ti. \\
\textsc{lnk} \textsc{ifr}-\textsc{inv}-\textsc{caus}-prepare.to.leave \textsc{lnk} \textsc{lnk} \textsc{dem} \textsc{erg} \textsc{filler} \textsc{dem} \textsc{indef}.\textsc{poss}-water \textsc{one}-\textsc{indef}.\textsc{poss}-bladder \textsc{ifr}-put.in \textsc{lnk} \textsc{dem}.\textsc{prox} \textsc{2sg}.\textsc{poss}-present be:\textsc{fact} \textsc{ifr}-say \\
\glt `He prepared his departure, and gave him a bladder full of water to take with him, and said `this is your departing present'.' (28-smAnmi.txt, 264-265)
\end{exe}


The verb \japhug{rku}{put in} can even occur with its derived noun \japhug{tɤ-rkuz}{parting present} in the \textit{figura etymologica} construction in (\ref{ex:arkuz.tarku}) (the verb \japhug{βzu}{make} can alternatively be used instead of \japhug{rku}{put in}).

\begin{exe}
\ex \label{ex:arkuz.tarku}
\gll a-me kɯ a-rkuz rŋɯl ta-rku \\
\textsc{1sg}.\textsc{poss}-daughter \textsc{erg} \textsc{1sg}.\textsc{poss}-present money \textsc{pfv}:3\fl{}3'-put.in \\
\glt `My daughter gave me some money (as present for my departure) (elicited)
\end{exe}

The second one, \japhug{ɯ-mɲoz}{preparation}, only occurs to express `preparation' (used in a collocation with the verb \japhug{βzu}{make} as in \ref{ex:WmYoz.tuwGBzu}), is derived from the transitive verb \japhug{mɲo}{prepare}, itself the irregular causative of \japhug{ɲo}{be prepared} (§ XXX). A \forme{-z}-less variant \japhug{ɯ-mɲo}{preparation} is also attested.

\begin{exe}
\ex \label{ex:WmYoz.tuwGBzu}
\gll  pjɯ-ɲɟo ɕɯŋgɯ tɕe ɯ-mɲoz tú-wɣ-βzu ra \\
\textsc{ipfv}-be.damaged before \textsc{lnk} \textsc{3sg}.\textsc{poss}-preparation \textsc{ipfv}-\textsc{inv}-make have.to:\textsc{fact} \\
\glt `One has to take preparations before it gets damaged.' (elicited)
\end{exe}

The third one, the noun \japhug{tɤ-scoz}{letter, writing} possibly originates from the verb \japhug{sco}{see off, accompany}, though it might be a loan from Situ (\citealt{jacques03s.houzhui}). 

The \forme{-z} nominalizing suffix, though rare in Japhug, is of Sino-Tibetan origin. In Situ, the corresponding nominalizing \forme{-s} suffix is much more common (\citealt{jacques03s.houzhui}), and Tibetan and Chinese have traces of a cognate suffix (\citealt{jacques16ssuffixes}).


\subsection{Nominalization \forme{ɣ-/x-} prefix} \label{sec:G.nmlz}
A handful of nouns, most of them IPNs, are derives from intransitive verbs by means of a velar prefix  \forme{ɣ}- or \forme{x}- (harmonizing in voicing with the initial consonant of the stem). These nouns are lexicalized ancient subject participles (§ \ref{sec:subject.participles}) which underwent the same phonological change as that observed with the velar animal class prefix (§ \ref{sec:velar.class.prefix}), that has a syllabic allomorph \forme{kɯ-} and reduced allomorphs \forme{ɣ}- or \forme{x}-. 

The reduced \forme{ɣ}- / \forme{x}- prefix only derives nouns from intransitive verbs with monosyllabic stems, without consonant clusters. 


\begin{table}[H]
\caption{Irregular subject nominalizations in \forme{ɣ}- and \forme{x}-} \label{tab:irregular.nmlz} \centering
\begin{tabular}{llll}
\lsptoprule
Noun & Base verb & Reference \\
\midrule
\japhug{ɣndʑɤβ}{disastrous fire} & \japhug{ndʑɤβ}{burn} \\
\japhug{ɯ-ɣɲaʁ}{disaster}& \japhug{ɲaʁ}{be black} \\
\japhug{ɯ-ɣɲɟɯ}{orifice} & \japhug{ɲɟɯ}{be opened} \\
\japhug{ɯ-xso}{empty, normal} &\japhug{so}{be empty} &  \ref{sec:property.nouns} \\
\japhug{ɯ-ɣrom}{dried thing} & \japhug{rom}{be dry} \\
\lspbottomrule
\end{tabular}
\end{table}

The noun \japhug{tɯ-xpa}{one year}, although derived from the verb \japhug{pa}{pass X years} and having an additional \forme{x-} element, does not belong to this category, see  § \ref{sec:num.prefix.paradigm.history} and § \ref{sec:CN.verbs}.


\section{Converbs}
\subsection{Gerund} \label{sec:gerund}
%nɤʑo laχtɕha kɤ-fkur tu-tɯ-ŋke ɲɯ-ŋu tɕe,
\subsection{Purposive} \label{sec:purposive.converb}


With intransitive and semi-transitive verbs, the possessive prefix is coreferent to the intransitive subject, as in (\ref{ex:aYWsAstWstu}).

\begin{exe}
\ex \label{ex:aYWsAstWstu}
\gll  tɕe nɯnɯ a-ɲɯ-sɤ-stɯ\redp{}stu nɯra tu-nɤme pjɤ-ŋu \\
\textsc{lnk} \textsc{dem} \textsc{1sg}-\textsc{ipfv}-\textsc{purp}-believe \textsc{dem}:\textsc{pl} \textsc{ipfv}-make[III] \textsc{ifr}.\textsc{ipfv}-be \\
\glt `He was doing these things so that I would believe (in his predictions).' (150904 yaoshu-zh, 104)
\end{exe}

\subsection{Immediate} \label{sec:immediate.converb}

\section{Historical perspectives} \label{sec:nmlz.historical.perspectives}

\subsection{Velar non-finite prefixes} \label{sec:velar.nmlz.history}

\subsection{Sigmatic non-finite prefixes} \label{sec:sigmatic.nmlz.history} %Non-finite verbal morphology
%\chapter{Voice derivations}

\subsection{Antipassive} 
%Special meaning of the Antipassive
%tɯ-mɯ wuma ʑo tɤ-me tɕe,  tɕendɤre tshitsuku ɕ-ku-sɤ-nɯrtɕa-nɯ tɕe, fsaŋ ra ɕ-pjɯ-ta-nɯ,  ɕ-pjɯ-rɟaʁ-nɯ nɯra tɕe,  tɕe tɯ-mɯ ku-sɯ-lɤt-nɯ pjɤ-ŋgrɤl.
\section{Fossil derivations}

\subsection{Applicative \forme{-t}}
Beside the productive prefixal \forme{nɯ-} applicative (sec XXX), Japhug has vestigial traces of a \forme{-t} applicative suffix, better attested in Kiranti and West Himalayish languages (see \citealt{michailovsky85dental}, \citet{jacques15derivational.khaling} and \citealt{jacques16ssuffixes} for comparative studies of this suffix). Only two examples of this derivation exist in Japhug: \japhug{ɣɯt}{bring} and \japhug{mdɯt}{strongly wish for}. 

The verb of manipulation \japhug{ɣɯt}{bring} derives from the motion verb \japhug{ɣi}{come}; the vowel alternation is regular as pre-Japhug \forme{*i} changes to \ipa{ɯ} in closed syllables. With a motion verb such as `come', the effect of the applicative (\ref{ex:bring1}) is similar to a causative  (\ref{ex:bring2}). 

\begin{exe}
\ex \label{ex:bring1}
\glt `come with X' $\rightarrow$ `bring'
\ex \label{ex:bring2}
\glt `cause X to come' $\rightarrow$ `bring'
\end{exe}

The transitive verb \japhug{mdɯt}{be resolved to, be determined to} is historically related to the verb \japhug{mdɯ}{live up to}, and constitutes another example of the \forme{-t} applicative, though it is less immediately obvious than in the case of \japhug{ɣɯt}{bring} because each of the verbs has undergone semantic specialization after the derivation took palce.

The verb \forme{mdɯ} is semi-transitive (sec XXX), and takes as its semi-object the lifespan; it can be applied to plants, animals and humans, as shown by examples (\ref{ex:chWmdW}) and (\ref{ex:chWmdWa}). It selects the  downstream' series of directional prefixes (\forme{tʰɯ-}, \forme{cʰɯ-}).

 \begin{exe}
\ex \label{ex:chWmdW}
\gll tɕe nɯŋa ɯʑo nɯnɯ, sqamŋu-xpa jamar cʰɯ-mdɯ ɲɯ-ŋgrɤl\\
\textsc{lnk} cow \textsc{3sg} \textsc{dem} fifteen-year about \textsc{ipfv}-live.up.to \textsc{sens}-be.usually.the.case \\
\glt `A cow itself can live up to fifteen years.' (05-qaZo, 142)
\end{exe}

 \begin{exe}
\ex \label{ex:chWmdWa}
\gll ``nɤʑo nɯ kʰrɯtsu-xpa a-tʰɯ-tɯ-mdɯ ra nɤ" to-ti ɲɯ-ŋu. tɕe ``aʑo kɯnɤ kʰrɯtsu cʰondɤre tɯ-rʑaʁ nɯnɯ cʰɯ-mdɯ-a ra" to-ti \\
\textsc{2sg} \textsc{dem} ten.thousand-year \textsc{irr}-\textsc{pfv}-2-live.up.to have.to:\textsc{fact} \textsc{sfp} \textsc{ifr}-say \textsc{sens}-be \textsc{lnk} \textsc{1sg} also  ten.thousand-year \textsc{comit} one-day \textsc{dem} \textsc{ipfv}-live.up.to-1sg have.to:\textsc{fact} \textsc{ifr}-say \\
\glt `He said: `May you live ten thousand years! I want to live one thousand years and one more day.' (150830 afanti-zh, 64)
\end{exe}

The meaning `live until/up to' is however a semantic innovation in Japhug: its Situ cognate \forme{mdə́} means `reach' as a motion verb. Japhug has restricted the meaning of this verb to a very specific context.

The verb \japhug{mdɯt}{be resolved to, be determined to} is morphologically transitive, and can take as its object an infinitive complement as in (\ref{ex:chWmdWta}). It shares with \japhug{mdɯ}{live up to} the `downstream' directional prefixes (\forme{cʰɯ-}).

 \begin{exe}
\ex \label{ex:chWmdWta}
\gll aʑo kɯrɯ-skɤt kɤ-βzjoz nɯ cʰɯ-mdɯt-a ʑo ɕti \\
\textsc{1sg} Tibetan-language \textsc{inf}-learn \textsc{dem} \textsc{ipfv}-be.determined \textsc{emph} be.\textsc{affirm}:\textsc{fact} \\
\glt `I am determined to learn Tibetan/Gyalrong.' (elicited)
\end{exe}

The precise meaning of \forme{mdɯt}  is to be determined to do something that one has confidence he can realize. If one accepts that the idea the original meaning of Japhug \japhug{mdɯ}{live up to} was `reach' as in Situ, the meaning `be determined to' of the verb \forme{mdɯt} has the same relationship to that of the base verb as English `reach for' (`reach for the stars') to the verb `reach', with a conative interpretation `try/strive to reach'.  The addition of the suffix \forme{-t} turns the semi-transitive (morphologically intransitive) \forme{mdɯ} into a transitive verb whose A corresponds to the S of the base verb. This applicative derivation from a semi-transitive verb is not unique in Japhug; the transitive verb \japhug{nɯrga}{like} from the verb \japhug{rga}{like, be happy} with the \forme{nɯ-} applicative is another similar example (see section XXX).


 %Voice derivations
%\include{chapters/4-06} %Denominal derivations
% \chapter{Tense, aspect, modality and evidentiality} \label{chap:tame}
 
  \section{Non-past tenses}
 \subsection{Dubitative}
 The dubitative \forme{ku-} is formally identical to imperfective `toward east' orientation prefix (§ XXX) and to the egophoric (§ XXX). It always occurs with the autobenefactive \forme{-nɯ-} prefix (§ XXX) and with the polar question \forme{ɕi} particule (§ XXX, see example \ref{ex:kunWZru.Ci.kWma}), the interrogative \forme{kɯ} (§ XXX, \ref{ex:CW.ci.kunWNu.kW}) or the alternative polar question construction (combining a positive followed by the equivalent negative verb form as in \ref{ex:kunWphAn.mWkunWphAn}, see § XXX).
 
 \begin{exe}
\ex \label{ex:kunWZru.Ci.kWma}
 \gll  tɕe lu-kɤ-nɯ-ji nɯ kɯ ʑru tu-ti-nɯ ɲɯ-ŋu tɕe mɤ-xsi. ku-nnɯ-ʑru ɕi kɯma. \\
 \textsc{lnk} \textsc{ipfv}-\textsc{nmlz}:P-\textsc{auto}-plant \textsc{dem} \textsc{erg} be.strong:\textsc{fact} \textsc{ipfv}-say-\textsc{pl} \textsc{sens}-be \textsc{lnk} \textsc{neg}-\textsc{genr}:know:\textsc{fact}  \textsc{dubit}-\textsc{auto}-be.strong \textsc{qu} \textsc{sfp} \\
 \glt `The cultivated (variety of Angelica) is better (than the wild one), they say, I don't know, maybe it is better.' (17-ndZWnW, 34)
 \end{exe}
 

  \begin{exe}
\ex \label{ex:kunWphAn.mWkunWphAn}
 \gll  ku-nɯ-pʰɤn mɯ-ku-nɯ-pʰɤn mɤ-xsi ma \\
 \textsc{dubit}-\textsc{auto}-be.efficient \textsc{neg}-\textsc{dubit}-\textsc{auto}-be.efficient \textsc{neg}-\textsc{genr}:know:\textsc{fact} sfp \\
\glt `I don't know whether it efficient or not (as medicine).' (19-GzW, 108)
  \end{exe}

In addition, dubitative verb forms are followed either by the sentence final particles \forme{kɯma} or \forme{kɯɣe} (§ XXX) as in (\ref{ex:kunWZru.Ci.kWma}) or a verb form such as \japhug{mɤ-xsi}{one does not know} (§ \ref{ex:kunWphAn.mWkunWphAn}).

The dubitative is mainly used to express doubts while reporting opinions from other people (as in \ref{ex:kunWZru.Ci.kWma} and \ref{ex:kunWphAn.mWkunWphAn}), but with the interrogative \forme{kɯ} as in (\ref{ex:CW.ci.kunWNu.kW}), its meaning is rather that of emphasis on the fact that the speaker has no clue about the answer to the question (as in French \textit{donc...bien} in `\textit{Qui donc cela peut-il bien être?}').

  \begin{exe}
\ex \label{ex:CW.ci.kunWNu.kW}
 \gll wo, nɯ ɕɯ ci ku-nɯ-ŋu kɯ?  \\
 \textsc{interj} \textsc{dem} who \textsc{indef} \textsc{dubit}-\textsc{auto}-be \textsc{qu} \\
\glt `Who on earth is it (who does all) that?' (2014-kWLAG, 619)
 \end{exe}
  %TAME
%\include{chapters/4-08} %Other verbal categories
%\chapter{Basic sentences}

\section{Subject-Verb order}
\section{Object-Verb order}
\section{Non-verbal predicates}
\subsection{Bare nominal predicates}
\subsection{Exclamative expressions}\\
\section{Preverbal adverbs}
\section{Postverbal elements}
 %Relativization
%\include{chapters/5-02} %Complementation
%\chapter{Temporal and conditional clauses} \label{chap:temporal.conditional}

\subsubsection{Immediate succession}
\begin{exe}
\ex \label{ex:pjWtWqlWt} 
\gll pjɯ-tɯ-qlɯt qʰe pjɯ-ɴɢlɯt ɲɯ-ɕti. \\
\textsc{ipfv-conv:imm}-break \textsc{lnk} \textsc{ipfv}-\textsc{acaus}:break \textsc{sens}-be.\textsc{affirm} \\
\glt `It breaks as soon as one breaks it.' (07-Zmbri, 6)
\end{exe}
pjɯ́-wɣ-qlɯt nɯɕɯmɯma pjɯ-ɴɢlɯt ɕti.
\subsubsection{Opportunity}
The postposition \japhug{ʁaz}{while ... still} has a meaning close to that of Chinese \ch{趁着}{chènzhe}{while ... still}, 

\begin{exe}
\ex \label{ex:GWrNi.RaznA}
\gll iɕqʰa tɯ-ndʐi nɯ nɤki, ɣɯrŋi ʁaznɤ nɯnɯ nɯnɯtɕu pjɯ-tʂɯβ-nɯ tɕe tɕe \\
the.aforementioned \textsc{indef}.\textsc{poss}-skin \textsc{dem} \textsc{filler} be.wet:\textsc{fact} while \textsc{dem} \textsc{dem}:\textsc{loc} \textsc{ipfv}-sew-\textsc{pl} \textsc{lnk} \textsc{lnk} \\
\glt `They would sew the skin while it was still wet.' (06-BGa, 53)
\end{exe}
XXXXX ta-ʁa + z %Temporal and conditional clauses
%\include{chapters/5-04} %Comparison
%\include{chapters/5-06} %Other subordinate clauses
%\include{chapters/5-07} %Sentence final particules
%\include{chapters/6} %Japhug in Sino-Tibetan perspective

% % copy the lines above and adapt as necessary

%%%%%%%%%%%%%%%%%%%%%%%%%%%%%%%%%%%%%%%%%%%%%%%%%%%%
%%%                                              %%%
%%%             Backmatter                       %%%
%%%                                              %%%
%%%%%%%%%%%%%%%%%%%%%%%%%%%%%%%%%%%%%%%%%%%%%%%%%%%%

%\input{localseealso.tex} 
% There is normally no need to change the backmatter section
\input{backmatter.tex} 
\end{document} 

% you can create your book by running
% xelatex main.tex
%
% you can also try a simple 
% make
% on the commandline
