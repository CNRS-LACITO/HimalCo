 \chapter{Tense, aspect, modality and evidentiality} \label{chap:tame}

  \section{Non-past tenses}
 \subsection{Dubitative}
 The dubitative \forme{ku-} is formally identical to imperfective `toward east' orientation prefix (§ XXX) and to the egophoric (§ XXX). It always occurs with the autobenefactive \forme{-nɯ-} prefix (§ XXX) and with the polar question \forme{ɕi} particule (§ XXX, see example \ref{ex:kunWZru.Ci.kWma}), the interrogative \forme{kɯ} (§ XXX, \ref{ex:CW.ci.kunWNu.kW}) or the alternative polar question construction (combining a positive followed by the equivalent negative verb form as in \ref{ex:kunWphAn.mWkunWphAn}, see § XXX).
 
 \begin{exe}
\ex \label{ex:kunWZru.Ci.kWma}
 \gll  tɕe lu-kɤ-nɯ-ji nɯ kɯ ʑru tu-ti-nɯ ɲɯ-ŋu tɕe mɤ-xsi. ku-nnɯ-ʑru ɕi kɯma. \\
 \textsc{lnk} \textsc{ipfv}-\textsc{nmlz}:P-\textsc{auto}-plant \textsc{dem} \textsc{erg} be.strong:\textsc{fact} \textsc{ipfv}-say-\textsc{pl} \textsc{sens}-be \textsc{lnk} \textsc{neg}-\textsc{genr}:know:\textsc{fact}  \textsc{dubit}-\textsc{auto}-be.strong \textsc{qu} \textsc{sfp} \\
 \glt `The cultivated (variety of Angelica) is better (than the wild one), they say, I don't know, maybe it is better.' (17-ndZWnW, 34)
 \end{exe}
 

  \begin{exe}
\ex \label{ex:kunWphAn.mWkunWphAn}
 \gll  ku-nɯ-pʰɤn mɯ-ku-nɯ-pʰɤn mɤ-xsi ma \\
 \textsc{dubit}-\textsc{auto}-be.efficient \textsc{neg}-\textsc{dubit}-\textsc{auto}-be.efficient \textsc{neg}-\textsc{genr}:know:\textsc{fact} sfp \\
\glt `I don't know whether it efficient or not (as medicine).' (19-GzW, 108)
  \end{exe}

In addition, dubitative verb forms are followed either by the sentence final particles \forme{kɯma} or \forme{kɯɣe} (§ XXX) as in (\ref{ex:kunWZru.Ci.kWma}) or a verb form such as \japhug{mɤ-xsi}{one does not know} (§ \ref{ex:kunWphAn.mWkunWphAn}).

The dubitative is mainly used to express doubts while reporting opinions from other people (as in \ref{ex:kunWZru.Ci.kWma} and \ref{ex:kunWphAn.mWkunWphAn}), but with the interrogative \forme{kɯ} as in (\ref{ex:CW.ci.kunWNu.kW}), its meaning is rather that of emphasis on the fact that the speaker has no clue about the answer to the question (as in French \textit{donc...bien} in `\textit{Qui donc cela peut-il bien être?}').

  \begin{exe}
\ex \label{ex:CW.ci.kunWNu.kW}
 \gll wo, nɯ ɕɯ ci ku-nɯ-ŋu kɯ?  \\
 \textsc{interj} \textsc{dem} who \textsc{indef} \textsc{dubit}-\textsc{auto}-be \textsc{qu} \\
\glt `Who on earth is it (who does all) that?' (2014-kWLAG, 619)
 \end{exe}
 
\subsection{Inferential} 


\ref{ex:mWtotata} said after realizing that he did not do it.

\begin{exe}
\ex \label{ex:mWtotata}
\gll mɯ-to-ta-t-a \\
\textsc{neg}-\textsc{ifr}-put-\textsc{tr}:\textsc{pst}-\textsc{1sg} \\
\glt `I did not put (the tea on the oven).' (Conversation, 28-04-2018, Dpalcan)
\end{exe}

\begin{exe}
\ex \label{ex:mWtosAlata}
\gll tɯ-ci mɯ-to-sɯ-ɤla-t-a \\
\textsc{indef}.\textsc{poss}-water \textsc{neg}-\textsc{ifr}-caus-be.boiling-\textsc{tr}:\textsc{pst}-\textsc{1sg} \\
\glt `I did not put the water to boil.' (Conversation, 01-05-2018, Tshendzin)
\end{exe}

\begin{exe}
\ex \label{ex:kAnWZWB.kordala}
\gll kɤ-nɯʑɯβ ko-rdal-a \\
\textsc{inf}-sleep \textsc{ifr}-overshoot-\textsc{1sg} \\
\glt `I overslept.' (elicitation)
\end{exe}

  \section{Auxiliary TAM categories}
  
  \subsection{Apprehensive}
  
\begin{exe}
\ex 
\gll   mɯ-ɕɯ-kʰɯ kɯ nɯ-sɯso-t-a \\
\textsc{neg}-\textsc{apprehensive}-be.possible \textsc{sfp} \textsc{pfv}-think-\textsc{pst}:\textsc{tr}-\textsc{1sg} \\
\glt `I thought it would not work.'  (conversation, 02-05-2018; after opening a washing machine, expecting it would not open)
\end{exe}
