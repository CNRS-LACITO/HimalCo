 \chapter{Tense, aspect, modality and evidentiality} \label{chap:tame}

  \section{Non-past tenses}
 \subsection{Dubitative}
 The dubitative \forme{ku-} is formally identical to imperfective `toward east' orientation prefix (§ XXX) and to the egophoric (§ XXX). It always occurs with the autobenefactive \forme{-nɯ-} prefix (§ XXX) and with the polar question \forme{ɕi} particule (§ XXX, see example \ref{ex:kunWZru.Ci.kWma}), the interrogative \forme{kɯ} (§ XXX, \ref{ex:CW.ci.kunWNu.kW}) or the alternative polar question construction (combining a positive followed by the equivalent negative verb form as in \ref{ex:kunWphAn.mWkunWphAn}, see § XXX).
 
 \begin{exe}
\ex \label{ex:kunWZru.Ci.kWma}
 \gll  tɕe lu-kɤ-nɯ-ji nɯ kɯ ʑru tu-ti-nɯ ɲɯ-ŋu tɕe mɤ-xsi. ku-nnɯ-ʑru ɕi kɯma. \\
 \textsc{lnk} \textsc{ipfv}-\textsc{nmlz}:P-\textsc{auto}-plant \textsc{dem} \textsc{erg} be.strong:\textsc{fact} \textsc{ipfv}-say-\textsc{pl} \textsc{sens}-be \textsc{lnk} \textsc{neg}-\textsc{genr}:know:\textsc{fact}  \textsc{dubit}-\textsc{auto}-be.strong \textsc{qu} \textsc{sfp} \\
 \glt `The cultivated (variety of Angelica) is better (than the wild one), they say, I don't know, maybe it is better.' (17-ndZWnW, 34)
 \end{exe}
 

  \begin{exe}
\ex \label{ex:kunWphAn.mWkunWphAn}
 \gll  ku-nɯ-pʰɤn mɯ-ku-nɯ-pʰɤn mɤ-xsi ma \\
 \textsc{dubit}-\textsc{auto}-be.efficient \textsc{neg}-\textsc{dubit}-\textsc{auto}-be.efficient \textsc{neg}-\textsc{genr}:know:\textsc{fact} \textsc{sfp} \\
\glt `I don't know whether it efficient or not (as medicine).' (19-GzW, 108)
  \end{exe}

In addition, dubitative verb forms are followed either by the sentence final particles \forme{kɯma} or \forme{kɯɣe} (§ XXX) as in (\ref{ex:kunWZru.Ci.kWma}) or a verb form such as \japhug{mɤ-xsi}{one does not know} (§ \ref{ex:kunWphAn.mWkunWphAn}).

The dubitative is mainly used to express doubts while reporting opinions from other people (as in \ref{ex:kunWZru.Ci.kWma} and \ref{ex:kunWphAn.mWkunWphAn}), but with the interrogative \forme{kɯ} as in (\ref{ex:CW.ci.kunWNu.kW}), its meaning is rather that of emphasis on the fact that the speaker has no clue about the answer to the question (as in French \textit{donc...bien} in `\textit{Qui donc cela peut-il bien être?}').

  \begin{exe}
\ex \label{ex:CW.ci.kunWNu.kW}
 \gll wo, nɯ ɕɯ ci ku-nɯ-ŋu kɯ?  \\
 \textsc{interj} \textsc{dem} who \textsc{indef} \textsc{dubit}-\textsc{auto}-be \textsc{qu} \\
\glt `Who on earth is it (who does all) that?' (2014-kWLAG, 619)
 \end{exe}
 
\subsection{Inferential} 

\subsubsection{Inferential with first person}
The inferential is not rare with first person subject, but occurs specifically to express that the speaker did not realize that the action took (or did not take) place at the time, as in (\ref{ex:mWtosAlata}), a sentence said after the speaker (Tshendzin) noticed that she forgot to put the water to boil.

\begin{exe}
\ex \label{ex:mWtosAlata}
\gll tɯ-ci mɯ-to-sɯ-ɤla-t-a \\
\textsc{indef}.\textsc{poss}-water \textsc{neg}-\textsc{ifr}-caus-be.boiling-\textsc{tr}:\textsc{pst}-\textsc{1sg} \\
\glt `I did not put the water to boil.' (Conversation, 01-05-2018, Tshendzin)
\end{exe}

Example (\ref{ex:mWtozwara}) illustrate the differential use of perfective and inferential with first person subject: the speaker did put the water on the oven, but forgot to open the oven, hence the use of the inferential for the second verb.

\begin{exe}
\ex \label{ex:mWtozwara}
\gll tɯ-ci kɤ-ta-t-a ri, <dian> mɯ-to-zwar-a \\
\textsc{indef}.\textsc{poss}-water \textsc{pfv}-put-\textsc{pst}:\textsc{tr}-\textsc{1sg} \textsc{lnk} electricity \textsc{neg}-\textsc{ifr}-burn-\textsc{1sg} \\
\glt `I put the water (on the oven), but did not open the electricity.' (Conversation, 04-05-2018, Tshendzin)
\end{exe}

%\begin{exe}
%\ex \label{ex:mWtotata}
%\gll mɯ-to-ta-t-a \\
%\textsc{neg}-\textsc{ifr}-put-\textsc{tr}:\textsc{pst}-\textsc{1sg} \\
%\glt `I did not put (the tea on the oven).' (Conversation, 28-04-2018, Dpalcan)
%\end{exe}

The inferential with first person can also be used when the speaker did the action he intended but on the wrong object, as in (\ref{ex:koGAdZama.YWmaR}), a sentence said after Tshendzin realized (by looking into the pot) that she mistakenly warmed the wrong pot (not the one containing nettles). Here \japhug{mtsʰalu}{nettle} is focalized using the copula \forme{ɲɯ-maʁ} (see § XXX).

\begin{exe}
\ex \label{ex:koGAdZama.YWmaR}
\gll  mtsʰalu ko-ɣɤ-ndʑam-a ɲɯ-maʁ \\
nettle \textsc{ifr}-\textsc{caus}-be.warm-\textsc{1sg} \textsc{sens}-not.be \\
\glt `It is not the nettles that I warmed.' (Conversation, 07-05-2018, Tshendzin)
\end{exe}

The inferential with first person is particularly common with verbs expressing uncontrollable and non-volitional actions, such as (\ref{ex:kAnWZWB.kordala}). 

\begin{exe}
\ex \label{ex:kAnWZWB.kordala}
\gll kɤ-nɯʑɯβ ko-rdal-a \\
\textsc{inf}-sleep \textsc{ifr}-overshoot-\textsc{1sg} \\
\glt `I overslept.' (elicitation)
\end{exe}

The inferential does not however express by itself non-volitionality; the autobenefactive-spontaneous prefix \forme{-nɯ-} (§ XXX) is used in conjunction with the inferential to insist on the non-volitional character of a particular action, as in (\ref{ex:YAnWjmWta}).

\begin{exe}
\ex \label{ex:YAnWjmWta}
\gll ɲɤ-nɯ-jmɯt-a \\
 \textsc{ifr}-\textsc{auto}-forget-\textsc{1sg} \\
\glt `I forgot.' (many attestations)
\end{exe}

  \section{Auxiliary TAM categories}
  
  \subsection{Apprehensive} \label{sec:apprehensive}
  
\begin{exe}
\ex 
\gll   mɯ-ɕɯ-kʰɯ kɯ nɯ-sɯso-t-a \\
\textsc{neg}-\textsc{apprehensive}-be.possible \textsc{sfp} \textsc{pfv}-think-\textsc{pst}:\textsc{tr}-\textsc{1sg} \\
\glt `I thought it would not work.'  (conversation, 02-05-2018; after opening a washing machine, expecting it would not open)
\end{exe}
