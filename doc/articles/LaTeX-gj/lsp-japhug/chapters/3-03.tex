\chapter{Numerals and counted nouns}

\section{Plain numerals}
Unlike some languages of the Sino-Tibetan family which have exotic numeral systems (\citealt{mazaudon02nombre}), Japhug displays a strict decimal system, without evidence for vigesimal features or substractive numerals.


\subsection{Numerals 1-10}
The basic numerals from one to ten are indicated in Table \ref{tab:numerals.under.10}. The numeral \japhug{ci}{one} is identical to the indefinite determiner (§ XXX and § \ref{sec:other.pro}). Some dialects of Japhug other than the Kamnyu variety use \japhug{tɤɣ}{one} instead. Apart from \japhug{ci}{one} and \japhug{sqi}{ten}, these numerals have clear cognates in languages outside of the Gyalrongic group, even in Tibetan and Chinese; Table \ref{tab:numerals.under.10} includes the Tibetan equivalent of these numerals (the numerals that are \textit{not} cognate with their Japhug equivalent are indicated between brackets).

The numerals from 2 to 9 have a prefix, uvular \forme{χ-/ʁ-} in `two' and `three' and velar \forme{kɯ-} from `four' to `nine'. These prefixes do not appear in some derived forms such as teens (Table XXX below) or approximate numerals (\ref{sec:approx.numerals}).

The numeral \japhug{ʁnɯz}{two} is etymologically related to the dual clitic \forme{ni} (§ XXX), though the latter lacks the uvular prefix and the \forme{-z} suffix (the vowel different is expected as proto-Gyalrong \forme{*-is} yields Japhug \forme{-ɯz}, see § XXX). A relation with \japhug{kɯɕnɯz}{seven} (implying a former base five system) is possible but if true goes back to proto-ST and is irrelevant to the synchronic grammar of Japhug.

While the other numerals are native Gyalrong words, \japhug{χsɯm}{three} might be a borrowing from Tibetan \tibet{གསུམ་}{gsum}{three}, and occurs with the same form in obvious compound loans  such as \japhug{kɯmtɕʰoχsɯm}{triratna} from \tibet{དཀོན་མཆོག་གསུམ་}{dkon.mtɕʰog.gsum}{triratna}. This idea is apparently confirmed by the alternative forms \forme{-fsum} and \forme{fsɯ-} for `three' found in teens (§ \ref{sec:teens}) and decades  (§ \ref{sec:decades}). Alternatively, it is possible that the native word and the borrowing have the same form by coincidence.

The numeral \japhug{kɯngɯt}{nine} has a coda \forme{-t} which is not found in Situ and languages outside of Gyalrongic, suggesting analogical spreading of the coda from \japhug{kɯrcat}{eight}. The same analogy independently occurred in the Siyuewu dialect of Khroskyabs, where  `nine' is \forme{ŋgə́d} (\citealt[174]{lai17khroskyabs}).

\begin{table}
\caption{Basic numerals in Japhug and Tibetan}  \label{tab:numerals.under.10} \centering \label{tab:numerals}
\begin{tabular}{lllllll}
\lsptoprule
& Japhug & Tibetan  \\
1	&	\forme{ci} or \forme{tɤɣ} & \tibet{གཅིག་}{gtɕig}{one} \\
2	&	\forme{ʁnɯz}  & \tibet{གཉིས་}{gɲis}{two} \\
3	&	\forme{χsɯm}  & \tibet{གསུམ་}{gsum}{three} \\
4	&	\forme{kɯβde} & \tibet{བཞི་}{bʑi}{four} \\
5	&	\forme{kɯmŋu}  & \tibet{ལྔ་}{lŋa}{five} \\
6	&	\forme{kɯtʂɤɣ}  & \tibet{དྲུག་}{drug}{six} \\
7	&	\forme{kɯɕnɯz} & (\tibet{བདུན་}{bdun}{seven}) \\
8	&	\forme{kɯrcat}  & \tibet{བརྒྱད་}{brgʲad}{eight} \\
9	&	\forme{kɯngɯt}  & \tibet{དགུ་}{dgu}{nine} \\
10	&	\forme{sqi}  & (\tibet{བཅུ་}{btɕu}{ten}) \\
\lspbottomrule
\end{tabular}
\end{table}		

Japhug numerals can be used either on their own or a postnominal modifiers. XXX


\subsection{Numerals 11-19} \label{sec:teens}
The numerals 11-19, listed in Table \ref{tab:teens}, serve as the basis for building all following numerals between 21 and 99, by replacing the \forme{-sqi} element of the decade numeral (Table XXX) by the appropriate form. Table \ref{tab:teens} also illustrates the formation of the numerals 21 to 29 from \japhug{ɣnɤsqi} {twenty}. 

\begin{table}
\caption{Numerals 11-29}  \label{tab:teens} \centering
\begin{tabular}{lllllll}
\lsptoprule
10 & \forme{sqi} &	20	&	\forme{ɣnɤsqi}  \\	
\midrule
11 & \forme{sqaptɯɣ} &	21	&	\forme{ɣnɤsqaptɯɣ}  \\	
12 & \forme{sqamnɯz} &	22	&	\forme{ɣnɤsqamnɯz}  \\	
13 & \forme{sqafsum} &	23	&	\forme{ɣnɤsqafsum}  \\	
14 & \forme{sqaβde} &	24	&	\forme{ɣnɤsqaβde}  \\	
15 & \forme{sqamŋu} &	25	&	\forme{ɣnɤsqamŋu}  \\	
16 & \forme{sqaprɤɣ} &	26	&	\forme{ɣnɤsqaprɤɣ}  \\	
17 & \forme{sqaɕnɯz} &	27	&	\forme{ɣnɤsqaɕnɯz}  \\	
18 & \forme{sqarcat} &	28	&	\forme{ɣnɤsqarcat}  \\	
19 & \forme{sqangɯt} &	29	&	\forme{ɣnɤsqangɯt}  \\	
\lspbottomrule
\end{tabular}
\end{table}		
 
The numerals 11-19 present three morphological changes in comparison with the basic numerals 1-9.

First, the form \japhug{sqi}{ten} alternates with \forme{sqa-}. The origin of this Ablaut is unknown, though it could be a type of \textit{status constructus} (\ref{sec:status.constructus}); some Gyalrongic languages, such as Khroskyabs have a similar alternation (\citealt[175-6]{lai17khroskyabs}). 

Second, the velar \forme{kɯ-} and uvular \forme{χ-/ʁ-} prefixes found in the base numerals are lost in all teens.

Third, a labial element \ipa{p} (\japhug{sqaptɯɣ}{eleven}, \japhug{sqaprɤɣ}{sixteen}), \ipa{m} (\japhug{sqamnɯz}{twelve}), or \ipa{w} (\japhug{sqafsum}{thirteen}) is inserted between the \forme{sqa-} and the following numeral root. It does not occur in 17, 18 and 19 (which already have a cluster), 14 and 15 (which have a cluster with a labial as first element).

The form \japhug{sqaptɯɣ}{eleven} contains an ablauted form of \japhug{tɤɣ}{one} as second element. The cluster \forme{-pt-} in this word is the only case in the language of a \ipa{p} followed by an obstruent. 

In \japhug{sqamnɯz}{twelve}, the labial linker is nasalized by the following \forme{n}. This is not a synchronic rule: for instance, a noun \japhug{ɕnaβndʑɣi}{snotty-nosed kid} has \forme{β} allomorph of \ipa{w} before a prenasalized obstruent (§ \ref{sec:subject.verb.compounds}). However, there are other cases of nasalization of labial consonants to \ipa{m} before nasal or prenasalized consonants in Japhug (see § XXX).

In \japhug{sqaprɤɣ}{sixteen}, not only the prefix \forme{kɯ-} is lost, the \forme{tʂ} affricate of the base form 	\japhug{kɯtʂɤɣ}{six} is replaced by \ipa{r}, preceded by the linking element \forme{-p-}. This \ipa{tʂ} \tld{} \ipa{r} alternation is evidence for a sound change \forme{*tr-} \fl{} \ipa{tʂ} (see § \ref{sec:second.member.alternation} for additional evidence).  The numeral \japhug{kɯtʂɤɣ}{six} contains two etymological prefixes, \forme{kɯ-} and a prefix \forme{*t-} that has fused with the root as \forme{-tʂɤɣ}. This \forme{*t-} prefix is possibly cognate with the \forme{d-} of its Tibetan cognate  \tibet{དྲུག་}{drug}{six} .


\subsection{Decades} \label{sec:decades}
The numerals for decades (Table \ref{sec:numeral.prefixes}) are relatively straightforward. With the exception of \japhug{ɣnɤsqi}{twenty} and \japhug{fsɯsqi}{thirty}, they are predictable by combining \japhug{sqi}{ten} to the corresponding numeral prefix (§ \ref{sec:numeral.prefixes}).

The element \forme{ɣnɤ-} in \japhug{ɣnɤsqi}{twenty} is related to the numeral  \japhug{ʁnɯz}{two}, but present a velar \forme{ɣ-} prefix instead of the uvular \forme{ʁ-}, and has a different vowel. The adverb \japhug{ʁnaʁna}̌{both} is also relatable, but the alternations are not explainable from a synchronic point of view.

\begin{table}
\caption{Decades}  \label{tab:decades} \centering
\begin{tabular}{lllllll}
\lsptoprule
10	&	\forme{sqi} \\			
20	&	\forme{ɣnɤsqi} \\		
30	&	\forme{fsɯsqi}  \\		
40	&	\forme{kɯβdɤ-sqi}  \\	
50	&	\forme{kɯmŋɤ-sqi}  \\	
60	&	\forme{kɯtʂɤ-sqi}  \\	
70	&	\forme{kɯɕnɤ-sqi}  \\	
80	&	\forme{kɯrcɤ-sqi}  \\	
90	&	\forme{kɯngɯ-sqi}  \\	
\lspbottomrule
\end{tabular}
\end{table}		

Other numerals under one hundred are built by combining the forms in Table \ref{sec:teens} and \ref{tab:decades}. For instance, 37 can be obtained by putting together \japhug{fsɯsqi}{thirty} and \japhug{sqaɕnɯz}{seventeen} as \forme{fsɯ-sqa-ɕnɯz}.

\subsection{Hundred and above}
 There are two ways of expressing numbers above 99 in Japhug. First, the noun-like numeral \japhug{ɣurʑa}{one hundred} can serve on its own or as a postnominal modifier, and be followed by another numeral to express a number between 101 and 199, as in (\ref{ex:hundred}).

\begin{exe}
\ex \label{ex:hundred} 
\gll aʑo 	kɯ-fse 	kɯ-cʰɯ\redp{}cʰa 	ʑo 	ʁʑɯnɯ 	ɣurʑa 	kɯrcat 	ra \\
\textsc{1sg} \textsc{nmlz}:S/A-be.like  \textsc{nmlz}:S/A-\textsc{emph}\redp{}can \textsc{emph} young.man hundred eight need:\textsc{fact} \\
\glt `I need one hundred and eight able young men like me.' (Norbzang, 16)
\end{exe}

The numeral \japhug{ɣurʑa}{one hundred} cannot be combined with unit numerals to express numbers between 200 and 900. The counted noun \japhug{tɯ-ri}{one hundred} is used for this purpose, as in \ref{ex:three.hundreds} (see § \ref{sec:numeral.prefixes} on the numeral prefixes). The two suppletive roots for hundreds are shared with Pumi (\forme{ɕí} `hundred' vs prefixed \forme{-ɻɛj}, see \citealt[101]{daudey14grammar}; evidence for cognacy with \forme{ɣurʑa} and \forme{-ri} is presented in \citealt{jacques17num}).

\begin{exe}
\ex \label{ex:three.hundreds}
\gll χsɯ-ri 	jamar 	ndɤre 	tu-nɯ 	ko, 	tɯ-tɯpʰu 	nɯ \\
three-hundred about \textsc{lnk} exist:\textsc{fact-pl} \textsc{sfp} one-hive \textsc{dem} \\
\glt There are about three hundred of them, in one hive. (Bees, 48)
\end{exe}
 
  Numerals above the hundreds are all borrowed from Tibetan: \japhug{stoŋtsu}{thousand}, \japhug{kʰrɯtsu} {ten thousand}, \japhug{mbɯmχtɤr}{hundred thousand} from \tibet{སྟོང་ཚོ་}{stoŋ.tsʰo}{thousand}, \tibet{ཁྲི་ཚོ་}{kʰri.tsʰo}{ten thousand} and \tibet{འབུམ་ཐེར་}{ⁿbum.tʰer}{hundred thousand} respectively.  
  
 
\section{Approximate numerals} \label{sec:approx.numerals}

\section{Counted nouns} \label{sec:counted.nouns}
\subsection{Numeral prefixes} \label{sec:numeral.prefixes}
\subsection{Time ordinals} \label{sec:time.ordinals}

\section{Basic arithmetic operations} \label{sec:arithmetic}
