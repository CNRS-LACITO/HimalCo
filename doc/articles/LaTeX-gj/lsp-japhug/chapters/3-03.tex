\chapter{Numerals and counted nouns}

\section{Plain numerals} \label{sec:plain.numerals}
Unlike some languages of the Sino-Tibetan family which have exotic numeral systems (\citealt{mazaudon02nombre}), Japhug displays a strict decimal system, without evidence for vigesimal features or substractive numerals.


\subsection{Numerals 1-10}  \label{sec:one.to.ten}
The basic numerals from one to ten are indicated in Table \ref{tab:numerals.under.10}. The numeral \japhug{ci}{one} is identical to the indefinite determiner (§ XXX and § \ref{sec:other.pro}). Some dialects of Japhug other than the Kamnyu variety use \japhug{tɤɣ}{one} instead. In calculations (see § \ref{sec:arithmetic}), the generic CN with numeral prefix `one' \japhug{tɯ-rdoʁ}{one piece} is used instead of \japhug{ci}{one} to express the number `one'.

Apart from \japhug{ci}{one} and \japhug{sqi}{ten}, these numerals have clear cognates in languages outside of the Gyalrongic group, even in Tibetan and Chinese; Table \ref{tab:numerals.under.10} includes the Tibetan equivalent of these numerals (the numerals that are \textit{not} cognate with their Japhug equivalent are indicated between brackets).

The numerals from 2 to 9 have a prefix, uvular \forme{χ-/ʁ-} in `two' and `three' and velar \forme{kɯ-} from `four' to `nine'. These prefixes do not appear in some derived forms such as teens (Table XXX below) or approximate numerals (\ref{sec:approx.numerals}).

The numeral \japhug{ʁnɯz}{two} is etymologically related to the dual clitic \forme{ni} (§ XXX), though the latter lacks the uvular prefix and the \forme{-z} suffix (the vowel different is expected as proto-Gyalrong \forme{*-is} yields Japhug \forme{-ɯz}, see § XXX). A relation with \japhug{kɯɕnɯz}{seven} (implying a former base five system) is possible but if true goes back to proto-ST and is irrelevant to the synchronic grammar of Japhug.

While the other numerals are native Gyalrong words, \japhug{χsɯm}{three} might be a borrowing from Tibetan \tibet{གསུམ་}{gsum}{three}, and occurs with the same form in obvious compound loans  such as \japhug{kɯmtɕʰoχsɯm}{triratna} from \tibet{དཀོན་མཆོག་གསུམ་}{dkon.mtɕʰog.gsum}{triratna}. This idea is apparently confirmed by the alternative forms \forme{-fsum} and \forme{fsɯ-} for `three' found in teens (§ \ref{sec:teens}) and decades  (§ \ref{sec:decades}). Alternatively, it is possible that the native word and the borrowing have the same form by coincidence.

The numeral \japhug{kɯngɯt}{nine} has a coda \forme{-t} which is not found in Situ and languages outside of Gyalrongic, suggesting analogical spreading of the coda from \japhug{kɯrcat}{eight}. The same analogy independently occurred in the Siyuewu dialect of Khroskyabs, where  `nine' is \forme{ŋgə́d} (\citealt[174]{lai17khroskyabs}).

\begin{table}
\caption{Basic numerals in Japhug and Tibetan}  \label{tab:numerals.under.10} \centering \label{tab:numerals}
\begin{tabular}{lllllll}
\lsptoprule
& Japhug & Tibetan  \\
1	&	\forme{ci} or \forme{tɤɣ} & \tibet{གཅིག་}{gtɕig}{one} \\
2	&	\forme{ʁnɯz}  & \tibet{གཉིས་}{gɲis}{two} \\
3	&	\forme{χsɯm}  & \tibet{གསུམ་}{gsum}{three} \\
4	&	\forme{kɯβde} & \tibet{བཞི་}{bʑi}{four} \\
5	&	\forme{kɯmŋu}  & \tibet{ལྔ་}{lŋa}{five} \\
6	&	\forme{kɯtʂɤɣ}  & \tibet{དྲུག་}{drug}{six} \\
7	&	\forme{kɯɕnɯz} & (\tibet{བདུན་}{bdun}{seven}) \\
8	&	\forme{kɯrcat}  & \tibet{བརྒྱད་}{brgʲad}{eight} \\
9	&	\forme{kɯngɯt}  & \tibet{དགུ་}{dgu}{nine} \\
10	&	\forme{sqi}  & (\tibet{བཅུ་}{btɕu}{ten}) \\
\lspbottomrule
\end{tabular}
\end{table}		

Numerals from 1 to 99 are a subclass of unpossessible nouns (§ \ref{sec:unpossessible.nouns}), and cannot take possessive prefixes; they differ in this regard from the higher numerals (§ \ref{sec.hundred.plus}, § \ref{sec:approx.numerals}).

\subsection{Numerals 11-19} \label{sec:teens}
The numerals 11-19, listed in Table \ref{tab:teens}, serve as the basis for building all following numerals between 21 and 99, by replacing the \forme{-sqi} element of the decade numeral (Table XXX) by the appropriate form. Table \ref{tab:teens} also illustrates the formation of the numerals 21 to 29 from \japhug{ɣnɤsqi} {twenty}. 

\begin{table}
\caption{Numerals 11-29}  \label{tab:teens} \centering
\begin{tabular}{lllllll}
\lsptoprule
10 & \forme{sqi} &	20	&	\forme{ɣnɤsqi}  \\	
\midrule
11 & \forme{sqaptɯɣ} &	21	&	\forme{ɣnɤsqaptɯɣ}  \\	
12 & \forme{sqamnɯz} &	22	&	\forme{ɣnɤsqamnɯz}  \\	
13 & \forme{sqafsum} &	23	&	\forme{ɣnɤsqafsum}  \\	
14 & \forme{sqaβde} &	24	&	\forme{ɣnɤsqaβde}  \\	
15 & \forme{sqamŋu} &	25	&	\forme{ɣnɤsqamŋu}  \\	
16 & \forme{sqaprɤɣ} &	26	&	\forme{ɣnɤsqaprɤɣ}  \\	
17 & \forme{sqaɕnɯz} &	27	&	\forme{ɣnɤsqaɕnɯz}  \\	
18 & \forme{sqarcat} &	28	&	\forme{ɣnɤsqarcat}  \\	
19 & \forme{sqangɯt} &	29	&	\forme{ɣnɤsqangɯt}  \\	
\lspbottomrule
\end{tabular}
\end{table}		
 
The numerals 11-19 present three morphological changes in comparison with the basic numerals 1-9.

First, the form \japhug{sqi}{ten} alternates with \forme{sqa-}. The origin of this Ablaut is unknown, though it could be a type of \textit{status constructus} (\ref{sec:status.constructus}); some Gyalrongic languages, such as Khroskyabs have a similar alternation (\citealt[175-6]{lai17khroskyabs}). 

Second, the velar \forme{kɯ-} and uvular \forme{χ-/ʁ-} prefixes found in the base numerals are lost in all teens.

Third, a labial element \ipa{p} (\japhug{sqaptɯɣ}{eleven}, \japhug{sqaprɤɣ}{sixteen}), \ipa{m} (\japhug{sqamnɯz}{twelve}), or \ipa{w} (\japhug{sqafsum}{thirteen}) is inserted between the \forme{sqa-} and the following numeral root. It does not occur in 17, 18 and 19 (which already have a cluster), 14 and 15 (which have a cluster with a labial as first element).

The form \japhug{sqaptɯɣ}{eleven} contains an ablauted form of \japhug{tɤɣ}{one} as second element. The cluster \forme{-pt-} in this word is the only case in the language of a \ipa{p} followed by an obstruent. 

In \japhug{sqamnɯz}{twelve}, the labial linker is nasalized by the following \forme{n}. This is not a synchronic rule: for instance, a noun \japhug{ɕnaβndʑɣi}{snotty-nosed kid} has \forme{β} allomorph of \ipa{w} before a prenasalized obstruent (§ \ref{sec:subject.verb.compounds}). However, there are other cases of nasalization of labial consonants to \ipa{m} before nasal or prenasalized consonants in Japhug (see § XXX).

In \japhug{sqaprɤɣ}{sixteen}, not only the prefix \forme{kɯ-} is lost, the \forme{tʂ} affricate of the base form 	\japhug{kɯtʂɤɣ}{six} is replaced by \ipa{r}, preceded by the linking element \forme{-p-}. This \ipa{tʂ} \tld{} \ipa{r} alternation is evidence for a sound change \forme{*tr-} \fl{} \ipa{tʂ} (see § \ref{sec:second.member.alternation} for additional evidence).  The numeral \japhug{kɯtʂɤɣ}{six} contains two etymological prefixes, \forme{kɯ-} and a prefix \forme{*t-} that has fused with the root as \forme{-tʂɤɣ}. This \forme{*t-} prefix is possibly cognate with the \forme{d-} of its Tibetan cognate  \tibet{དྲུག་}{drug}{six} .


\subsection{Decades} \label{sec:decades}
The numerals for decades (Table \ref{sec:numeral.prefixes}) are relatively straightforward. With the exception of \japhug{ɣnɤsqi}{twenty} and \japhug{fsɯsqi}{thirty}, they are predictable by combining \japhug{sqi}{ten} to the corresponding numeral prefix (§ \ref{sec:numeral.prefixes}).

The element \forme{ɣnɤ-} in \japhug{ɣnɤsqi}{twenty} is related to the numeral  \japhug{ʁnɯz}{two}, but present a velar \forme{ɣ-} prefix instead of the uvular \forme{ʁ-}, and has a different vowel. The adverb \japhug{ʁnaʁna}̌{both} is also relatable, but the alternations are not explainable from a synchronic point of view.

\begin{table}
\caption{Decades}  \label{tab:decades} \centering
\begin{tabular}{lllllll}
\lsptoprule
10	&	\forme{sqi} \\			
20	&	\forme{ɣnɤsqi} \\		
30	&	\forme{fsɯsqi}  \\		
40	&	\forme{kɯβdɤ-sqi}  \\	
50	&	\forme{kɯmŋɤ-sqi}  \\	
60	&	\forme{kɯtʂɤ-sqi}  \\	
70	&	\forme{kɯɕnɤ-sqi}  \\	
80	&	\forme{kɯrcɤ-sqi}  \\	
90	&	\forme{kɯngɯ-sqi}  \\	
\lspbottomrule
\end{tabular}
\end{table}		

Other numerals under one hundred are built by combining the forms in Table \ref{sec:teens} and \ref{tab:decades}. For instance, 37 can be obtained by putting together \japhug{fsɯsqi}{thirty} and \japhug{sqaɕnɯz}{seventeen} as \forme{fsɯ-sqa-ɕnɯz}.

\subsection{Hundred and above} \label{sec.hundred.plus}
 There are two ways of expressing numbers above 99 in Japhug. First, the noun-like numeral \japhug{ɣurʑa}{one hundred} can serve on its own or as a postnominal modifier, and be followed by another numeral to express a number between 101 and 199, as in (\ref{ex:hundred.and.eight}).

\begin{exe}
\ex \label{ex:hundred.and.eight}
\gll aʑo 	kɯ-fse 	kɯ-cʰɯ\redp{}cʰa 	ʑo 	ʁʑɯnɯ 	ɣurʑa 	kɯrcat 	ra \\
\textsc{1sg} \textsc{nmlz}:S/A-be.like  \textsc{nmlz}:S/A-\textsc{emph}\redp{}can \textsc{emph} young.man hundred eight need:\textsc{fact} \\
\glt `I need one hundred and eight able young men like me.' (Norbzang, 16)
\end{exe}

The numeral \japhug{ɣurʑa}{one hundred} cannot be combined with unit numerals to express numbers between 200 and 900. The counted noun \japhug{tɯ-ri}{one hundred} is used for this purpose, as in \ref{ex:three.hundreds} (see § \ref{sec:numeral.prefixes} on the numeral prefixes). The two suppletive roots for hundreds are shared with Pumi (\forme{ɕí} `hundred' vs prefixed \forme{-ɻɛj}, see \citealt[101]{daudey14grammar}; evidence for cognacy with \forme{ɣurʑa} and \forme{-ri} respectively is presented in \citealt{jacques17num}).

\begin{exe}
\ex \label{ex:three.hundreds}
\gll χsɯ-ri 	jamar 	ndɤre 	tu-nɯ 	ko, 	tɯ-tɯpʰu 	nɯ \\
three-hundred about \textsc{lnk} exist:\textsc{fact-pl} \textsc{sfp} one-hive \textsc{dem} \\
\glt There are about three hundred of them, in one hive. (Bees, 48)
\end{exe}
 
  Numerals above the hundreds are all borrowed from Tibetan: \japhug{stoŋtsu}{thousand}, \japhug{kʰrɯtsu} {ten thousand}, \japhug{mbɯmχtɤr}{hundred thousand} from \tibet{སྟོང་ཚོ་}{stoŋ.tsʰo}{thousand}, \tibet{ཁྲི་ཚོ་}{kʰri.tsʰo}{ten thousand} and \tibet{འབུམ་ཐེར་}{ⁿbum.tʰer}{hundred thousand} respectively.  

These numerals can take lower numerals between one and ten as modifiers to express multiplicative digits, as in (\ref{ex:stoNtsu.kWtsxAG}), where \japhug{kɯtʂɤɣ}{six} follows  \japhug{stoŋtsu}{thousand} to express `six thousands'. This use illustrates the difference between the numerals \japhug{stoŋtsu}{thousand} and above with \japhug{ɣurʑa}{hundred}, as when the latter is followed by a numeral, as \japhug{ɣurʑa kɯrcat}{one hundred and eight}, it can only be in additive, not multiplicative, relation to it (see example \ref{ex:hundred.and.eight} above).

\begin{exe}
\ex  \label{ex:stoNtsu.kWtsxAG}
 \gll   rŋɯl tɯ-xpa tɕe stoŋtsu kɯtʂɤɣ jarma ɲɯ-fsoʁ ɲɯ-cʰa \\
 money one-year \textsc{lnk} thousand six about \textsc{ipfv}-earn \textsc{sens}-can \\
 \glt `He can earn six thousands (renminbi) per year.' (14-tApitaRi, 178)
 \end{exe}
  
Unlike numerals under 100, \japhug{ɣurʑa}{one hundred} and above are APNs and can take a third person possessive prefix \forme{ɯ-} to express an approximate number (§ \ref{sec:approx.numerals}).
 
 \subsection{Use of the numerals}  \label{sec:uses.numerals}
 Japhug numerals can occur on their own when counting (\forme{ci, ʁnɯz, χsɯ, kɯβde...}) or be used as postnominal attributive modifiers. The noun can be elided when the context is clear, especially when the same referent occurs in the previous proposition as in (\ref{ex:WrJit.kWngWt}  with \japhug{ɯ-rɟit}{her children}) and (\ref{ex:kWBde.kW} with <cai> `dish').

\begin{exe}
\ex \label{ex:WrJit.kWngWt} 
\gll
ɯ-rɟit kɯngɯt tɤ-tu ri, kɯtʂɤɣ nɯ-si \\
\textsc{3sg.poss}-child nine \textsc{pfv}-exist \textsc{lnk} six \textsc{pfv}-die \\
\glt `She had nine children, but six of them died.' (14-tApi taRi, 17)
\end{exe}

\begin{exe}
\ex \label{ex:kWBde.kW} 
\gll <cai> χsɯm tu-sɯ-lɤt-i tɕe tɕe  kɯβde nɯ kɯ χsɯm tu-ndza-j kɯ-fse. \\
dish three \textsc{ipfv}-\textsc{caus}-throw-\textsc{1pl} \textsc{lnk} \textsc{lnk} four \textsc{dem} \textsc{erg} three \textsc{ipfv}-eat-\textsc{1pl} \textsc{nmlz}:S/A-be.like \\
\glt `We used to order three dishes, and the four of us would eat them.' (140501 jingli, 92-3)
\end{exe}		

Numerals can also be modifiers of dual and plural pronouns as in (\ref{ex:iZo.kWBde}). Since pronouns are never obligatory in Japhug (§ XXX), it is also possible to use a bare numeral in core argument function with first or second person indexation on the verb, as in (\ref{ex:kWBde.kW}), where the verb \forme{tu-ndza-j} `we eat' of the second proposition has the \textsc{1pl} \forme{-j} suffix indexing the transitive subject, coreferent with the ergatively-marked phrase \forme{kɯβde nɯ kɯ} `the four' (standing for \forme{iʑo kɯβde nɯ kɯ} `the four of us').

\begin{exe}
\ex \label{ex:iZo.kWBde} 
\gll tɕe iʑo kɯβde nɯ tɯtɯrca ku-rɤʑi-j tɕe, \\
\textsc{lnk} \textsc{1pl} four \textsc{dem}  together \textsc{ipfv}-stay-\textsc{1pl} \textsc{lnk} \\
\glt `The four of us were living together.' (140501 jingli, 85)
\end{exe}		

In some constructions, bare numerals can have a specific meaning. For instance, with the verb \japhug{pa}{pass X years} (an intransitive counterpart of the transitive \japhug{pa}{do}), numerals obligatorily refer to years, as in (\ref{ex:40.topa}). It is not possible in this construction to replace the bare numeral by the counted noun  \japhug{tɯ-xpa}{one year}.

\begin{exe}
\ex \label{ex:40.topa} 
\gll tɕiʑo ni kɤ-amɯfse-tɕi nɯ jinde kɯβdɤsqi ɯ-ro to-pa \\
\textsc{1du} \textsc{du} \textsc{pfv}-know.each.other-\textsc{1du} \textsc{dem} nowadays forty \textsc{3sg.poss}-excess \textsc{ifr}-pass.X.years \\
\glt  `We have known each other for more than forty years.' (`More than forty years passed', 12-BzaNsa, 3)
 \end{exe}		
  
\section{Approximate numerals} \label{sec:approx.numerals}
To express an approximate number, it is possible in Japhug to use the adverb \japhug{jamar}{about} (from Tibetan \tibet{ཡར་མར་}{jar.mar}{about, up and down} and/or to combine adjacent numerals in a row as in (\ref{ex:RnWz.XsWm.kWBde}).

\begin{exe}
\ex \label{ex:RnWz.XsWm.kWBde}
\gll tɯ-kʰɤl nɯtɕu ʁnɯz, χsɯm kɯβde jamar ku-ndzoʁ. \\
one-place \textsc{dem}:\textsc{loc} two three four about \textsc{ipfv}-\textsc{anticaus}:attach \\
\glt `Two, three or four (of its flowers) grow in one place.' (16-RlWmsWsi, 9)
\end{exe}

However, Japhug also has five morphological devices to build approximate numerals (illustrated in Tables \ref{tab:approx.num.1to10} and XXX).

First, for numerals under seven (Table \ref{tab:approx.num.1to10}), one can build approximate numerals by prefixing a \forme{la-} or \forme{lɤ-} element to one (or two) numeral root(s). Not all possibilities are attested (for instance there is no such approximate numeral $\dagger$\forme{lɤtʂɤɣ} derived from only \japhug{kɯtʂɤɣ}{six}).  Prefixation of \forme{la-} / \forme{lɤ-} occurs with other morphological changes: (i) loss of the velar \forme{kɯ-} prefix (but not the uvular one in `two' and `three', § \ref{sec:one.to.ten}) (ii) loss of the \forme{m-} preinitial in  \japhug{lɤŋu}{about five}, but not of the \forme{*t-} prefix of \forme{kɯtʂɤɣ} (§ \ref{sec:teens}) in \japhug{lɤŋɤtʂɤɣ}{five or six} (otherwise $\dagger$\forme{lɤŋɤrɤɣ} would be have been found). The \forme{la-} / \forme{lɤ-} prefix is probably historically related to the \forme{-lɤ-} element found in \textit{dvandva} collectives (§ \ref{sec:dvandva.coll}).

Second, some approximate numerals are built by compounding two numeral roots (in some cases with the  \forme{la-} / \forme{lɤ-} prefix, Table \ref{tab:approx.num.1to10}). The first numeral undergoes \textit{status constructus} vowel change (§ \ref{sec:status.constructus}), with loss of the codas \forme{-z} and \forme{-t} (§ \ref{sec:loss.codas.compounds}). In the case of \japhug{ɕnɤcat}{seven or eight} (illustrated by example \ref{ex:CnAcat.ci}), the form \forme{ɕnɤ-} is irregular ($\dagger$\forme{ɕnɯ-} would be expected instead).

\begin{exe}
\ex \label{ex:CnAcat.ci} 
\gll ɯʑo nɯnɯ ɕnɤcat ci tɤ-kɤ-sɯpa jamar ʑo qarma wxti ri, \\
\textsc{3sg} \textsc{dem} seven.or.eight one \textsc{pfv}-\textsc{nmlz}:P-\textsc{caus}-do about \textsc{emph} crossoptilon be.big:\textsc{fact} but \\
\glt `Although the crossoptilon is as big as about seven or eight of them (weasels) put together.' (27-spjaNkW, 56)
\end{exe}

Third, for decades (Table \ref{tab:approx.decades}), approximate forms can be formed using the same rule as decades from 40 to 90 (§ \ref{sec:decades}), by combining the \textit{status constructus} of the under ten numeral with the root \japhug{sqi}{ten}, for instance  \japhug{lɤŋɤsqi}{about fifty}   from	 \japhug{lɤŋu}{about five} 
 like \japhug{kɯmŋɤsqi}{fifty} from \japhug{kɯmŋu}{five}. These approximate numerals are rare and not attested in the corpus.

Fourth, an alternative way of producing approximate decade numerals is to add the numeral prefix \forme{tɯ-} (§ \ref{sec:numeral.prefixes}) to a decade numeral, as for instance \japhug{tɯɣnɤsqi}{about twenty} from \japhug{ɣnɤsqi}{twenty} (see example \ref{ex:tWGnAsqi}).

\begin{exe}
\ex \label{ex:tWGnAsqi}
 \gll tɯɣnɤsqi jamar tɯtɯrca ju-ɣi-nɯ ŋgrɤl \\
 about.twenty about together \textsc{ipfv}-come-\textsc{pl} be.usually.the.case:\textsc{fact} \\
\glt  `They come in groups of about twenty individuals.' (23-qapGAmtWmtW, 105)
\end{exe}

\begin{table}
\caption{Approximate numerals in Japhug (one to ten)} \label{tab:approx.num.1to10} \centering
\begin{tabular}{llllll}
\lsptoprule
Approximate Numeral & Base Numerals \\
\midrule
\japhug{laʁnɯz}{a few} & \japhug{ʁnɯz}{two} \\
\japhug{laʁnɯχsɯm}{two or three}  & 	\japhug{ʁnɯz}{two} \\
&\japhug{χsɯm}{three} \\
\japhug{lɤβdelɤŋu}{four or five}  & 		\japhug{kɯβde}{four} \\
 & 		\japhug{kɯmŋu}{five} \\
 \japhug{lɤŋu}{about five}   & 		\japhug{kɯmŋu}{five} \\
\japhug{lɤŋɤtʂɤɣ}{five or six}  & 	\japhug{kɯmŋu}{five} \\
&\japhug{kɯtʂɤɣ}{six} \\
\japhug{ɕnɤcat}{seven or eight}  & 	\japhug{kɯɕnɯz}{seven} \\
 & 	\japhug{kɯrcat}{eight} \\
\japhug{kɯngɯsqi}{nine or ten}  & 	\japhug{kɯngɯt}{nine} \\
& 	\japhug{sqi}{ten} \\
\lspbottomrule
\end{tabular}
\end{table}

\begin{table}
\caption{Approximate numerals in Japhug (decades)} \label{tab:approx.decades} \centering
\begin{tabular}{llllll}
\lsptoprule
Approximate Numeral & Base Form \\
\midrule
\japhug{tɯɣnɤsqi}{about twenty} & \japhug{ɣnɤsqi}{twenty} \\
\japhug{tɯfsɯsqi}{about thirty}  & 	\japhug{fsɯsqi}{thirty} \\
\japhug{tɯkɯβdɤsqi}{about forty} 	&	\japhug{kɯβdɤsqi}{forty}  \\	
\japhug{tɯkɯmŋɤsqi}{about fifty} 	&	\japhug{kɯmŋɤsqi}{fifty}  \\	
\japhug{tɯkɯtʂɤsqi}{about sixty} 	&	\japhug{kɯtʂɤsqi}{sixty}  \\	
\japhug{tɯkɯɕnɤsqi}{about seventy} 	&	\japhug{kɯɕnɤsqi}{seventy}  \\	
\japhug{tɯkɯrcɤsqi}{about eighty} 	&	\japhug{kɯrcɤsqi}{eighty}  \\	
\japhug{tɯkɯngɯsqi}{about ninety} 	&	\japhug{kɯngɯsqi}{ninety}  \\	
\midrule
 \japhug{lɤŋɤsqi}{about fifty}   & 		\japhug{lɤŋu}{about five} \\
\japhug{lɤŋɤtʂɤsqi}{fifty or sixty}  & 	\japhug{lɤŋɤtʂɤɣ}{five or six}  \\
\japhug{ɕnɤcɤsqi}{seventy or eighty}  & 	\japhug{ɕnɤcat}{seven or eight} \\
\lspbottomrule
\end{tabular}
\end{table}

 Five, in the case of numerals above 99, approximate numerals are built by prefixing a third singular \forme{ɯ-} prefix, as \japhug{ɯ-ɣurʑa}{several hundreds} (example \ref{ex:WGurZa}), \japhug{ɯ-stoŋtsu}{several thousands} (\ref{ex:WstoNtsu}) and higher numerals.
 
\begin{exe}
\ex \label{ex:WGurZa}
\gll  ɯ-ɣurʑa, χsɯ-ri jamar ndɤre tu-nɯ ko, tɯ-tɯpʰu nɯ  \\
\textsc{3sg.poss}-hundred three-hundred about \textsc{lnk} exist:\textsc{fact}-\textsc{pl} \textsc{sfp} one-hive \textsc{dem} \\
\glt `In one hive, there are about several hundreds, about three hundred of them.' (26-GZo, 51-2)
\end{exe}

\begin{exe}
\ex \label{ex:WstoNtsu}
\gll  tɯ-ŋga tɯ-rdoʁ nɯ ɯ-stoŋtsu ɯ-phɯ kɯ-fse ŋu ma \\
\textsc{indef.poss}-clothes one-piece \textsc{dem} \textsc{3sg.poss}-thousand \textsc{3sg.poss}-price \textsc{nmlz}:S/A-be.like be:\textsc{fact} \textsc{lnk} \\ 
\glt `One piece of clothes (made from it), its price is several thousand renminbi.' (05-qaZo, 81)
\end{exe}

\section{Counted nouns} \label{sec:counted.nouns}
Counted nouns (henceforth CN) are a subclass of nouns that differs from IPN, APN and UPN in that they take a numeral prefix (whose paradigms are described in §\ref{sec:numeral.prefixes}) or alternatively a \textsc{3sg} possessive prefix (§ \ref{sec:CN.quantifier} and §\ref{sec:CN.time}).

\subsection{Numeral prefixes} \label{sec:numeral.prefixes}
Numerals are postnominal when used as attributive modifiers (sec § \ref{sec:uses.numerals}, § XXX), but occur before the noun roots in \textit{dvigu} compounds (§ \ref{sec:bahuvrihi.n.n}), and the bound forms deriving from numerals are strictly prefixal. The form with numeral prefix `one' is the citation form of counted nouns.

In this section, numeral prefixes are described following several categories (1-10, 11-99, approximate numerals and prefixes derived from nouns) and irregular forms are discussed in a separated subsection (§ \ref{sec:irregular.numeral.prefixes}). A final subsection presents historical hypotheses to account for the numeral prefixal paradigm and its relationship to that of other Gyalrongic languages.

\subsubsection{Numeral prefixes between 1 and 10} \label{sec:num.prefixes.1.10}
The paradigm of regular numeral prefixes from 1 to 10 in Kamnyu Japhug is indicated in Table \ref{tab:num.prefix.1.to.10}. The prefixed are derived from the corresponding numeral by \textit{status constructus} (§ \ref{sec:status.constructus}), with loss of the coda and vowel alternation. Vowel alternation is optional for \japhug{kɯβde}{four} and  \japhug{kɯmŋu}{five}. The alternation between \japhug{tɤɣ}{one} and \forme{tɯ-} is irregular. The form \forme{tɤ-}, which is otherwise attested in another  paradigm (§ \ref{sec:irregular.numeral.prefixes}), should be the regular form. In the corpus,  the coda \forme{-z} of the numeral \japhug{kɯɕnɯz}{seven} seems to be preserved in one example (\ref{ex:kWCnWztWphu}), but this is an incorrect form, with a slight pause of hesitation between the numeral of the following noun.

\begin{exe}
\ex \label{ex:kWCnWztWphu}
 \gll kɯɕnɯz... -tɯpʰu ʑo pjɤ-tu \\
 seven -types \textsc{emph} \textsc{ifr}.\textsc{ipfv}-exist \\
 \glt `There were five types (of meals on the table).' (140504 baixuegongzhu-zh, 57)
\end{exe}


 \begin{table}
\caption{1-10 regular numeral prefixes in Japhug}  \label{tab:num.prefix.1.to.10} \centering
\begin{tabular}{lllllll}
\toprule
Numeral & Free form &  \forme{-sŋi} `day'   \\
\midrule
 1	&	\forme{tɤɣ}  &	\forme{tɯ-sŋi}  &	\\
2	&	\forme{ʁnɯz}  &	\forme{ʁnɯ-sŋi}  &	\\
3	&	\forme{χsɯm}  &	\forme{χsɯ-sŋi}  &	\\
4	&	\forme{kɯβde}  &	\forme{kɯβde-sŋi}, \forme{kɯβdɤ-sŋi}  &	\\
5	&	\forme{kɯmŋu}  &	\forme{kɯmŋu-sŋi}, \forme{kɯmŋɤ-sŋi}  &	\\
6	&	\forme{kɯtʂɤɣ}  &	\forme{kɯtʂɤ-sŋi}  &	\\
7	&	\forme{kɯɕnɯz}  &	\forme{kɯɕnɯ-sŋi}  &	\\
8	&	\forme{kɯrcat}  &	\forme{kɯrcɤ-sŋi}  &	\\
9	&	\forme{kɯngɯt}  &	\forme{kɯngɯ-sŋi}  &	\\
10	&	\forme{sqi}  &	\forme{sqɯ-sŋi}  &\\
\bottomrule
\end{tabular}
\end{table}

Note that the form of the numeral prefixes differ in some cases from the corresponding numeral prefixes in decades (cf table \ref{tab:decades}), compare \japhug{kɯɕnɯ-sŋi}{seven days} with \japhug{kɯɕnɤ-sqi}{seventy} or  \japhug{χsɯ-sŋi}{three days} with \japhug{fsɯ-sqi}{seventy}.

\subsubsection{Numeral prefixes between 11 and 99} \label{sec:num.prefixes.11.99}
Above ten, numerals present some variation.  Table   \ref{tab:num.prefix.11.to.20} show the most common forms of the numerals between 11 and 20 (from which all numerals between 11 and 99 can be generated following the rules described in § \ref{sec:decades}), but many cases without vowel alternation or with preservation of coda \forme{-z} (\ref{ex:GnAsqamnWzpArme}) or \forme{-ɣ} (\ref{ex:GnAsqaptWWGrZaR}) are attested.

 \begin{table}
\caption{11-20 numeral prefixes in Japhug}  \label{tab:num.prefix.11.to.20} \centering
\begin{tabular}{lllllll}
\toprule
Numeral & Free form &  \forme{-sŋi} `day'   \\
\midrule
11	&	\forme{sqaptɯɣ}  &	\forme{sqaptɯ-sŋi}  &	\\
12	&	\forme{sqamnɯz}  &	\forme{sqamnɯ-sŋi}  &	\\
13	&	\forme{sqafsum}  &	\forme{sqafsum-sŋi}  &	\\
14	&	\forme{sqaβde}  &	\forme{sqaβde-sŋi}  &	\\
15	&	\forme{sqamŋu}  &	\forme{sqamŋu-sŋi}  &	\\
16	&	\forme{sqaprɤɣ}  &	\forme{sqaprɤ-sŋi}  &	\\
17	&	\forme{sqaɕnɯz}  &	\forme{sqaɕnɯ-sŋi}  &	\\
18	&	\forme{sqarcat}  &	\forme{sqarcɤ-sŋi}  &	\\
19	&	\forme{sqangɯt}  &	\forme{sqangɯ-sŋi}  &	\\
20	&	\forme{ɣnɤsqi}  &	\forme{ɣnɤsqɯ-sŋi}   &	\\
\bottomrule
\end{tabular}
\end{table}

In example (\ref{ex:fsWsqildZa}), we observe the two alternative forms \forme{-sqɯ-} and \forme{-sqi-} for the decades in the same sentence. While for the numeral ten only the prefix \forme{sqɯ-} (or its variant \forme{sqɤ-}, see Table XXX below) is found, for decades between 20 and 90 vowel alternation is optional and there is free variation between the two forms. In the corpus, we find 15 examples of \forme{-sqi-} and 14 of  \forme{-sqɯ-}, suggesting that both are about equally common.

\begin{exe}
\ex \label{ex:fsWsqildZa}
\gll tɯ-pʰɯ nɯ tɕe rcanɯ, li, fsɯsqi-ldʑa jamar, kɯβdɤsqɯ-ldʑa jamar tu. \\
one-tree \textsc{dem} \textsc{lnk} \textsc{unexpectedly} again thirty-long.object about forty-long.object about exist:\textsc{fact} \\
\glt `On one tree, there are about thirty or forty (branches).'   (14-sWNgWJu, 200)
\end{exe}

\begin{exe}
\ex \label{ex:GnAsqamnWzpArme}
\gll  ma ɯ-me kɯnɤ ɣnɤsqamnɯz-pɤrme tʰɯ-azɣɯt, \\
\textsc{lnk} \textsc{3sg}.\textsc{poss}-daughter also twenty.two-year.old \textsc{pfv}-reach \\
\glt `Even his daughter is now twenty-two.' (14-tApitaRi, 317)
\end{exe}

Example (\ref{ex:GnAsqaptWWGrZaR}) illustrates three alternative forms with the counted noun \japhug{tɤ-rʑaʁ}{one night}: \forme{ɣnɤsqaptɯ-rʑaʁ} with regular loss of coda, \forme{ɣnɤsqaptɯɣ-rʑaʁ} with preservation of the coda,   and \forme{ɣnɤsqamnɯz} \forme{tɤ-rʑaʁ} as two words, the numeral being a kind of prenominal modifier. The first form is regular, while the other ones each are \textit{hapax legomena}.

\begin{exe}
\ex \label{ex:GnAsqaptWWGrZaR}
\gll  tɕe nɯ ɣnɤsqaptɯɣ-rʑaʁ tu-tsu ɲɯ-ra. ``tɕe ɣnɤsqaptɯ-rʑaʁ tu-tsu tɕe ɲɯ-ʁaʁ ŋu" ɲɯ-ti-nɯ ri, aʑɯɣ nɯ ɣnɤsqamnɯz tɤ-rʑaʁ mɤɕtʂa mɯ-nɯ-ʁaʁ. \\
\textsc{lnk} \textsc{dem} twenty.one-night \textsc{ipfv}-pass \textsc{sens}-have.to  
\textsc{lnk}  twenty.one-night \textsc{ipfv}-pass  \textsc{lnk} \textsc{ipfv}-hatch be:\textsc{fact} \textsc{sens}-say-\textsc{pl} \textsc{lnk} \textsc{1sg}:\textsc{gen} \textsc{dem} twenty.two one-night until \textsc{neg}-\textsc{pfv}-hatch \\
\glt `(Eggs) need twenty-two days to hatch; people say `They hatch in twenty-two days' but mine only hatch after twenty two days.' (150819 kumpGa, 34-36)
\end{exe}


\subsubsection{Approximate numeral prefixes} \label{sec:approximate.numeral.prefixes}
Approximate numerals also have corresponding prefixal forms. Table \ref{tab:approx.num.prefixes} presents the forms attested in the corpus.

 \begin{table}
\caption{Approximate numeral prefixes in Japhug} \label{tab:approx.num.prefixes} \centering
\begin{tabular}{llllll}
\lsptoprule
Approximate Numeral & Approximate Numeral ɓrefix \\
\midrule
\japhug{laʁnɯz}{a few} & \forme{laʁnɯ-} \\
\japhug{lɤβdelɤŋu}{four or five}  & 		\forme{lɤβdelɤŋu-}  \\
 \japhug{lɤŋu}{about five}   & 		\forme{lɤŋu-}  \\
\japhug{lɤŋɤtʂɤɣ}{five or six}  & 	\forme{lɤŋɤtʂɤ-}, \forme{lɤŋɤtʂɤɣ-} \\
\japhug{ɕnɤcat}{seven or eight}  & 	\forme{ɕnɤcɤ-} \\
\lspbottomrule
\end{tabular}
\end{table}

By far the most commonly used approximate numeral prefix is \forme{laʁnɯ-X}, whose meaning is not  `one or two' as could have been expected, but  `a few' (see example \ref{ex:laʁnWxpa} -- life expectancy of goats and sheep is much above four years).

\begin{exe}
\ex \label{ex:laʁnWxpa}
\gll tsʰɤt qaʑo nɯnɯ tɕe laʁnɯ-xpa ma cʰɯ-mdɯ mɯ́j-ŋgrɤl ma tɕe cʰɯ-rgɤz ɕti \\
goat sheep \textsc{dem} \textsc{lnk} a.few-year apart.from \textsc{ipfv}-live.up.to \textsc{neg}:\textsc{sens}-be.usually.the.case \textsc{lnk} \textsc{lnk} \textsc{ipfv}-be.old be.\textsc{affirm}:\textsc{fact} \\
\glt `Goats and sheep only live for a few years, and then become old.' (05-qaZo, 144)
\end{exe}

\subsubsection{Other numeral prefixes} \label{sec:other.numeral.prefixes}
In addition to the numerals mentioned above, all higher and compound numerals, an interrogative pronoun and a participle form can appear as prefixes of counted nouns.

The numerals above 99 (§ \ref{sec.hundred.plus}) occur as numeral prefixes without vowel alternation, as \japhug{ɣurʑa-xpa}{a hundred years} and \japhug{stoŋtsu-xpa}{a thousand years} (from \japhug{ɣurʑa}{hundred} and \japhug{stoŋtsu}{thousand}). 

The prefixal form of hundreds based on the CN \japhug{tɯ-ri}{one hundred} have double numeral prefixes, such as \japhug{χsɯ-ri-xpa}{three hundred years} in (\ref{ex:XsWrixpa}).

\begin{exe}
\ex \label{ex:XsWrixpa}
 \gll tɕendɤre χsɯ-ri-xpa tɤ-tsu tɕe, li kɯjŋu pɯ-ta-t-a tɕe, \\
 \textsc{lnk} three-hundred-year \textsc{pfv}-pass \textsc{lnk} again oath \textsc{pfv}-put-\textsc{tr}:\textsc{pst}-\textsc{1sg} \textsc{lnk} \\
\glt `Three hundred years passed, and I made another oath.' (140512 yufu yu mogui-zh, 92)
\end{exe}

In the case of complex numerals, such as the very common \japhug{ɣurʑa kɯrcat}{one hundred and eight}, only the last one undergoes \textit{status constructus}, as in (\ref{ex:GurZa.kWrCAJom}) and (\ref{ex:GurZa.kWrCAGdAt}) where we find \forme{ɣurʑa} as a distinct phonological word followed by the CN with the numeral prefix \forme{kɯrcɤ-} from \japhug{kɯrcat}{eight}.

\begin{exe}
\ex \label{ex:GurZa.kWrCAJom}
 \gll ɕommbri ɣurʑa kɯrcɤ-ɟom, [...] ɕomtsʰoʁ ɣurʑa kɯrcɤ-ldʑa ra \\
 iron.chain hundred eight-length.of.two.outstretched.arms ... iron.nails hundred eight-long.object have.to:\textsc{fact} \\
 \glt  `(I) need a chain of one hundred and eight fathoms, and one hundred and eight nails.' (2003tWxtsa, 21)
\end{exe}

\begin{exe}
\ex \label{ex:GurZa.kWrCAGdAt}
 \gll  ɣurʑa kɯrcɤ-ɣdɤt qapri nɯ a-tɤ-ɕe ra \\
 hundred eight-section snake \textsc{dem} \textsc{irr}-\textsc{pfv}:\textsc{up}-go have.to:\textsc{fact} \\
\glt `May the snake be cut into hundred and eight sections!' (Norbzang2012,  280)
\end{exe}

The interrogative pronoun \japhug{tʰɤstɯɣ}{how many} has the prefixal form \forme{tʰɤstɯ-} with CNs (§  \ref{sec:thAstWG}). The alternative form \forme{tʰɤstɤ-} is attested with the irregular  CN \japhug{tɤ-rʑaʁ}{one night} (§ \ref{sec:irregular.numeral.prefixes}).

The S-participle \forme{kɯ-ɤntɕʰɯ}  of the stative verb \japhug{antɕʰɯ}{be many},  has the prefixal form \forme{kɤntɕʰɯ-}, as in (\ref{ex:kAntChWtWpW}). This prefix is very common in the corpus.

\begin{exe}
\ex \label{ex:kAntChWtWpW}
 \gll  tɕe nɯnɯ kɤntɕhɯ-tɯpɯ ɣɯ nɯ-tɯrsa ɯ-sta ɲɯ-ŋu ma \\
 \textsc{lnk} dem many-household \textsc{dem} \textsc{3pl}.\textsc{poss}-grave \textsc{3sg}.\textsc{poss}-place \textsc{sens}-be \textsc{lnk} \\
 \glt `It is the grave-place of many families.' (140522 kAmYW tWji, 96)
\end{exe}

It is however not possible to convert any noun, pronoun or quantifier into a numeral prefix; with other words, it is necessary to convert the CN to an IPN, with a third singural \forme{ɯ-} prefix instead of the numeral prefixes, see § \ref{sec:CN.IPN}.

\subsubsection{Irregular forms} \label{sec:irregular.numeral.prefixes}
A handful of CNs, in particular \japhug{tɤ-rʑaʁ}{one night}, have an alternative paradigm with \ipa{ɤ} instead of \forme{ɯ} in the prefixes, as shown in Table \ref{tab:num.prefix.tArZaR}). There is however some degree of variation, and using the regular paradigm is not considered erroneous. In the corpus, the numeral `one' form \japhug{tɯ-rʑaʁ}{one night} is actually considerably more common than \forme{tɤ-rʑaʁ}, possibly because of the homophony with the IPN \japhug{tɤ-rʑaʁ}{time} found in examples such as (\ref{ex:tArZaR.tArYJi}). For other numerals, the forms in Table (\ref{tab:num.prefix.tArZaR}) are considerably more common than the regular ones. In addition, the \forme{ɤ} vocalism is also found with other numeral prefixes such as the interrogative (\japhug{tʰɤstɤ-rʑaʁ}{how many nights}).

\begin{exe}
\ex \label{ex:tWrZaR}
\gll tɕe qarma nɯ, tɯ-rʑaʁ tɕe kɯβde kɯmŋu jamar pjɯ-sat-nɯ, tɯ-rdoʁ, tɯrme tɯ-rdoʁ kɯ \\
\textsc{lnk} crossoptilon \textsc{dem} one-night \textsc{lnk} four five about \textsc{ipfv}-kill-\textsc{pl}, one-piece person one-piece \textsc{erg} \\
\glt `Crossoptilons, in one night, each of (the hunters) can kill four or five of them.' (23-qapGAmtWmtW, 163)
\end{exe}

\begin{exe}
\ex \label{ex:tArZaR.tArYJi}
\gll tɤ-rʑaʁ tɤ-rɲɟi tɕe, nɯ-ji ra kɯ-dɯ\redp{}dɤn kɯ-jɯ\redp{}jom lo-pɣaʁ-nɯ, ɕoŋtɕa kɯ-dɯ\redp{}dɤn pjɤ-pʰɯt-nɯ \\
\textsc{indef}.\textsc{poss}-time \textsc{pfv}-be.long \textsc{lnk} \textsc{3pl}.\textsc{poss}-field \textsc{pl} \textsc{nmlz}:S/A-\textsc{emph}\redp{}be.many \textsc{nmlz}:S/A-\textsc{emph}\redp{}be.broad \textsc{ifr}-turn.over-\textsc{pl} timber \textsc{nmlz}:S/A-\textsc{emph}\redp{}be.many \textsc{ifr}-remove-\textsc{pl} \\
\glt `After some time, (those people) had ploughed many broad fields for them, and chopped a lot of timber.' (2002qajdoskAt, 90)
\end{exe}

 \begin{table}
\caption{Irregular numeral prefixes in Japhug}  \label{tab:num.prefix.tArZaR} \centering
\begin{tabular}{lllllll}
\lsptoprule
Numeral & Free form  &  \forme{-rʑaʁ} `night' \\
\midrule
 1	&	\forme{tɤɣ}  &		\forme{tɤ-rʑaʁ}  &	\\
2	&	\forme{ʁnɯz}  &		\forme{ʁnɤ-rʑaʁ}  &	\\
3	&	\forme{χsɯm}  &		\forme{χsɤ-rʑaʁ}  &	\\
4	&	\forme{kɯβde}  &		\forme{kɯβdɤ-rʑaʁ}  &	\\
5	&	\forme{kɯmŋu}  &		\forme{kɯmŋɤ-rʑaʁ}  &	\\
6	&	\forme{kɯtʂɤɣ}  &		\forme{kɯtʂɤ-rʑaʁ}  &	\\
7	&	\forme{kɯɕnɯz}  &		\forme{kɯɕnɤ-rʑaʁ}  &	\\
8	&	\forme{kɯrcat}  &		\forme{kɯrcɤ-rʑaʁ}  &	\\
9	&	\forme{kɯngɯt}  &		\forme{kɯngɤ-rʑaʁ}  &	\\
10	&	\forme{sqi}  &	\forme{sqɤ-rʑaʁ}  &	\\
\lspbottomrule
\end{tabular}
\end{table}

Apart from  \japhug{tɤ-rʑaʁ}{one night}, CNs following the paradigm in Table (\ref{tab:num.prefix.tArZaR}) are very rare. The CNs \japhug{tɯ-tɣa}{one span} \japhug{tɯ-rtsɤɣ}{one story} have the irregular forms \japhug{χsɤ-tɣa}{three spans} and \japhug{χsɤ-rtsɤɣ}{three stories} (competing with regular \forme{χsɯ-tɣa} and \forme{χsɯ-rtsɤɣ}) as in (\ref{ex:XsAtGa}), but not for other numerals. 

\begin{exe}
\ex \label{ex:XsAtGa}
\gll nɯ χsɤ-tɣa kɯβde-tɣa jamar tu-rɲɟi cʰa. \\
\textsc{dem} three-span four-span about \textsc{ipfv}-be.long can:\textsc{fact} \\
\glt `It can grow three or four spans long.' (14-sWNgWJu, 194)
\end{exe}

Another unrelated irregularity concerns the CN \japhug{tɯ-ɣjɤn}{one time}: free variation between \forme{-jɤn} and \forme{-ɣjɤn} is observed for the numerals `two' and `three' (both \japhug{χsɯ-ɣjɤn}{three times} and \forme{χsɯ-jɤn} are attested).

The CN \japhug{tɯ-xpa}{one year} has some irregular forms, for instance the approximate numeral \japhug{ɕnɤcɤ-pa}{seven or eight years} (see § \ref{sec:num.prefix.paradigm.history}).

\subsubsection{Distributive numeral prefixes} \label{sec:numeral.prefixes.distributive}
The distributive form of CNs is built by reduplicating the numeral `one' prefix \forme{tɯ-}, as \forme{tɯ-tɯ-rdoʁ} from the generic CN \japhug{tɯ-rdoʁ}{one piece} (see § \ref{sec:CN.classification}) in example (\ref{ex:tWtWrdoR}).

 \begin{exe}
\ex  \label{ex:tWtWrdoR}
 \gll 
tɕeri li iɕqʰa sɯŋgi cʰo kɯrtsɤɣ nɯra kɯ ju-βɟi-nɯ ɲɯ-ŋu. ju-βɟi-nɯ qʰe tɕe kɯ-dɯ\redp{}dɤn ju-βɟi-nɯ nɤ ju-βɟi-nɯ, ju-βɟi-nɯ nɤ ju-βɟi-nɯ qʰe, ʑɯrɯʑɤri qʰe, tɕe tɯ-tɯ-rdoʁ nɯ ɲɯ-ʑɣɤ-qɤr-nɯ qʰe \\
but again the.aforementioned  lion \textsc{comit} leopard \textsc{dem}:\textsc{pl} \textsc{erg} \textsc{ipfv}-chase-\textsc{pl} \textsc{sens}-be \textsc{ipfv}-chase-\textsc{pl} \textsc{lnk} \textsc{lnk} \textsc{nmlz}:S/A-\textsc{emph}\redp{}be.many \textsc{ipfv}-chase-\textsc{pl} \textsc{lnk} \textsc{ipfv}-chase-\textsc{pl}  \textsc{ipfv}-chase-\textsc{pl} \textsc{lnk} \textsc{ipfv}-chase-\textsc{pl} \textsc{lnk} progressively \textsc{lnk} \textsc{lnk} one-one-piece \textsc{dem} \textsc{ipfv}-get.separated-\textsc{pl} \textsc{lnk} \\
\glt `But the lions, and leopard chase them, chase them in great number, and progressively, some of them get separated (from the herd).' (20-RmbroN, 57-59)
   \end{exe}

 Example (\ref{ex:tWtWxpa}) with \japhug{tɯ-tɯ-xpa}{some years} from \japhug{tɯ-xpa}{one year} clearly illustrates the difference with approximate numerals (§ \ref{sec:approx.numerals} and § \ref{sec:approximate.numeral.prefixes}), as the meaning of the distributive cannot be translated here as `a few years' -- the years when the income is good are not necessarily contiguous in time.

\begin{exe}
\ex  \label{ex:tWtWxpa}
 \gll   tɯ-tɯ-xpa tɕe a-pɯ-pe tɕe, kʰrɯtsu ɯ-ro jamar ɲɯ-fsoʁ ɲɯ-cʰa \\
one-one-year  \textsc{lnk} \textsc{irr}-\textsc{ipfv}-be.good \textsc{lnk} ten.thousand \textsc{3sg}.\textsc{poss}-excess about \textsc{ipfv}-earn \textsc{sens}-can \\ 
 \glt `Some years if (his income) is good, he can earn more than ten thousands.' (14-tApitaRi, 180)
 \end{exe}

An alternative distributive form involves partial reduplication of the stem of the CN and replacing the numeral prefix by a third singular possessive \forme{ɯ-}, as in the conversion from CN to IPN (§ \ref{sec:CN.IPN}). In (\ref{sec:WphWphW}), the CN \japhug{tɯ-pʰɯ}{one tree} is changed to \japhug{ɯ-pʰɯ\redp{}pʰɯ}{some trees}. 

\begin{exe}
\ex \label{sec:WphWphW}
\gll zgoku kɯ-mbro tɕe, ɕkrɤz kɯ-wxti ra nɯ ɯ-rcʰɤβ ri ɯ-pʰɯ\redp{}pʰɯ ʑo tu tɕe \\
mountain \textsc{nmzl}:S/A-be.tall \textsc{lnk} oak \textsc{nmzl}:S/A-be.big \textsc{pl} \textsc{dem} \textsc{3sg}.\textsc{poss}-between \textsc{loc} \textsc{3sg}.\textsc{poss}-\textsc{part}\redp{}tree \textsc{emph} exist:\textsc{fact} \textsc{lnk} \\
\glt `On high mountains, among big oaks, there are some (of these little trees).' (16-CWrNgo, 177)
\end{exe}   

This distributive form puts emphasis on the non-contiguousness and spread over distribution of the entities designated by the reduplicated CN: in (\ref{sec:WphWphW}) for instance, its presence implies that the little trees are not clustered, but rather scattered among the oaks.

\subsubsection{Historical perspectives on the numeral prefixal paradigm} \label{sec:num.prefix.paradigm.history}
Most non-Tibetan languages of Western Sichuan/Northern Yunnan have CNs (generally called `classifiers', see § \ref{sec:CN.classification}) with numeral-CN order (see for instance \citealt{zhang14classifiers}, \citealt[163-194]{michaud17yongning}).  It is striking that among the quasi-isolating languages (Lolo-Burmese, Naish), numeral paradigms are commonly a pocket of irregular morphology (\citealt{bradley05numerals}, \citealt{michaud11cl}); this is also true in some Hmong-Mien  languages (see \citealt{gerner10classifier.isolating}).

By contrast, while Japhug and the other Gyalrong languages have a richer morphology in general, the numeral prefixal paradigms are, with only few exceptions (§ \ref{sec:irregular.numeral.prefixes}), suspiciously regular. The historical interpretation of this observation is not completely straightforward (\citealt{jacques17num}), but in any case CNs are morphologically like \textit{karmadhāraya} compounds with a numeral as first element, undergoing \textit{status constructus} in the case of more integrated numerals (some of the numerals under 100, see § \ref{sec:num.prefixes.1.10},  § \ref{sec:num.prefixes.11.99} and § \ref{sec:approximate.numeral.prefixes}) and immune from it in the case of higher numerals (§ \ref{sec:other.numeral.prefixes}).

One morphological alternation common to all numeral prefixes under 10 is the loss of the codas, otherwise a rare phenomenon in noun compounds (see § \ref{sec:loss.codas.compounds}). For numerals above 10, there is some degree of free variation in the preservation of the codas (see for instance example \ref{ex:GnAsqaptWWGrZaR} p.\pageref{ex:GnAsqaptWWGrZaR}); the forms preserving codas are rarer, and may be ongoing analogical levelling.

Even in the case of lower numerals, there are indirect traces of the former existence of codas, in particular in irregularities (§ \ref{sec:irregular.numeral.prefixes}). A few CNs like \japhug{tɯ-xpa}{one year} and \japhug{tɯ-ɣjɤn}{one time} have velar fricative preinitials \forme{x-/ɣ-} which are not found in other forms deriving from the same root. 

In the case of \japhug{tɯ-xpa}{one year}, note the approximate numeral \japhug{ɕnɤcɤ-pa}{seven or eight years}, the adverb \japhug{pakuku}{every year} and the verb  \japhug{pa}{pass X years} (see example \ref{ex:40.topa} p.\pageref{ex:40.topa}) from which the CN \japhug{tɯ-xpa}{one year} is historically derived. As for \japhug{tɯ-ɣjɤn}{one time}, there are forms in the paradigm such as \japhug{χsɯ-jɤn}{three times} without the \forme{ɣ-}. 

A possible explanation for this \forme{x-/ɣ-} element is that it originates from the coda of the numeral `one' (although replaced by \japhug{ci}{one}, this former numeral is still found in the element \forme{-tɯɣ} in \japhug{sqaptɯɣ}{eleven}) through false segmentation (\forme{*tɯk-pa} $\rightarrow$ \forme{*tɯ-kpa} $\rightarrow$ \forme{*tɯ-xpa} `one year') and subsequent generalization to the whole paradigm.\footnote{Note that since proto-Gyalrong \forme{*kp-} regularly yields \forme{βɣ-} with metathesis (\citealt[272]{jacques04these}), this false segmentation must have occurred \textit{after} the \forme{*kp-} $\rightarrow$\forme{βɣ-} sound change, which is not shared with other Gyalrong languages. }


\subsection{CNs as quantifiers} \label{sec:CN.quantifier}
In Japhug, CNs are not required for numerals to serve as noun modifiers -- numerals  directly occur as postnominal modifiers (§ \ref{sec:uses.numerals}), and are not used either to express indefiniteness (this is rather the function of the modifier \japhug{ci}{one}, cf  § XXX).

When occurring as postnominal modifiers, CNs are essentially \textit{partitive} in meaning, referring to a certain number of individuals from a group. Thus, a phrase such as  \forme{tɯrme tɯ-rdoʁ} combining the noun \japhug{tɯrme}{person} with the generic CN \japhug{tɯ-rdoʁ}{one piece} (see §  \ref{sec:CN.classification} on CN selection) is generally either to be translated as a partitive `one of them'  as in (\ref{ex:tWrme.tWrdoR.RnWz}) or even, with a distributive meaning  `each of them' when another numeral or CN occurs in the same clause, as in examples (\ref{ex:tWrme.tWrdoR.kW}) and (\ref{ex:tWrme.tWrdoR2}) (see also \ref{ex:tWrZaR} p.\pageref{ex:tWrZaR}).

\begin{exe}
\ex \label{ex:tWrme.tWrdoR.RnWz}
\gll tɯrme ʁnɯz pjɤ-tu tɕe, tɯrme tɯ-rdoʁ nɯ rcanɯ, ɯ-stɤrju ʁɟa ʑo tu-βze tɕe, \\
people two \textsc{ifr}.\textsc{ipfv}-exist \textsc{lnk} people one-piece \textsc{dem} unexpectedly \textsc{3sg}.\textsc{poss}-truth completely \textsc{emph} \textsc{ipfv}-do[III] \textsc{lnk} \\
\glt `There were two persons, one of them always told the truth (and the other one was a liar).' (140427 yuanhou-zh, 2-3)
\end{exe} 

When the CN has a distributive meaning, indicating that each of the members of the group performs the same action, the verb can either receive plural (as in \ref{ex:tWrme.tWrdoR.kW} and \ref{ex:tWrZaR} p.\pageref{ex:tWrZaR}) or singular indexation (example \ref{ex:tWrme.tWrdoR2}). This is a particular case of fluid number indexation (see § XXX).

\begin{exe}
\ex \label{ex:tWrme.tWrdoR.kW}
\gll  tɕe tɕe tɯrme tɯ-rdoʁ kɯ kʰɯna ʁnɯz χsɯm jamar tu-ndo-nɯ \\
\textsc{lnk} \textsc{lnk}  person one-piece \textsc{erg} dog two three about \textsc{ipfv}-take-\textsc{pl} \\
\glt `Each of them (of the hunters) takes two or three dogs.' (150829 KAGWcAno)
\end{exe} 

\begin{exe}
\ex \label{ex:tWrme.tWrdoR2}
\gll tɯrme tɯ-rdoʁ kɯ cʰɤmdɤru tɯ-ldʑa tu-nɯ-ndɤm  \\
people one-piece \textsc{erg} drinking.straw one-\textsc{cl} \textsc{ipfv-auto}-take[III] \\
\glt `Each person takes one straw.' (30-tChorzi, 40)
\end{exe}

The partitive meaning is found even when the CN occurs on its own without overt head noun as in (\ref{ex:tWrdoR.cinA}).

 \begin{exe}
\ex \label{ex:tWrdoR.cinA}
\gll ɯ-tɕɯ kɯβde pɯ-tu ri, ɯʑo kɯ-fse kɯ-ɕqraʁ tɯ-rdoʁ cinɤ pɯ-me ɲɯ-ŋu 	\\
\textsc{3sg.poss}-son four \textsc{pst.ipfv}-exist but \textsc{3sg} \textsc{nmlz}:S/A-be.like \textsc{nmlz}:S/A-be.intelligent one-piece even \textsc{pst.ipfv}-not.exist \textsc{sens}-be \\
\glt `He had four sons, but not even one of them was smart like him.' (2005tAwakWcqraR, 3)
\end{exe} 

The only context where a noun+CN phrase such as \forme{tɯrme tɯ-rdoʁ} means `one person' is in restrictive constructions (`only one person'), as in (\ref{ex:tWrme.tWrdoR.ma.me}).  

\begin{exe}
\ex \label{ex:tWrme.tWrdoR.ma.me}
\gll tʰam tɯrme tɯ-rdoʁ ma me \\
now people one-piece apart.from not.exist:\textsc{fact} \\
\glt `Now there is only one person (in that place).' (140522 tshupa, 35)
\end{exe} 

As illustrated by the following pair of examples (from a similar episode in two traditional stories), in this construction both noun+CN  (\ref{ex:tWrme.tWrdoR.ma.maNetCi})  or plain numeral (\ref{ex:RnWz.ma.maNetCi}) can occur.

\begin{exe}
\ex \label{ex:tWrme.tWrdoR.ma.maNetCi}
\gll tɯrme ʁnɯ-rdoʁ ma maŋe-tɕi tɕe, kɤ-ɤnɯndzɤqɯqɤr mɤ-nɯ-cʰa-tɕi , \\
people two-piece apart.from not.exist:\textsc{sens}-\textsc{1du}  \textsc{inf}-\textsc{recip}:eat.on.one's.own \textsc{neg}-\textsc{auto}-can:\textsc{fact}-\textsc{1du} \\
\glt `There is only two of us, we cannot eat each on our own without sharing with each other.' (2003kunbzang, 100)
\end{exe} 

\begin{exe}
\ex \label{ex:RnWz.ma.maNetCi}
\gll  tɕiʑo ʁnɯz ma maŋe-tɕi tɕe, ʑaka kɤ-nɯ-βzu mɤ-rtaʁ-tɕi \\
1du two apart.from not.exist:\textsc{sens}-\textsc{1du} \textsc{lnk} each \textsc{inf}-\textsc{auto}-make \textsc{neg}-be.enough:\textsc{fact}-\textsc{1du} \\
\glt  `There is only two of us, there is not enough of us to act each on his own.' (2002qaCpa, 220)
\end{exe} 

In example (\ref{ex:tWrme.tWrdoR.Wndzxa}) also with noun+CN, there is no specific restrictive construction, but there is an implicit restrictive meaning (`because of (just) one person'). 

\begin{exe}
\ex \label{ex:tWrme.tWrdoR.Wndzxa}
\gll tɯrme tɯ-rdoʁ ɯ-ndʐa nɯ ʑo to-stu-nɯ ɕti ri, \\
people one-piece  \textsc{3sg}.\textsc{poss}-reason \textsc{dem} \textsc{emph} \textsc{ifr}-do.like-\textsc{pl} be.\textsc{affirm}:\textsc{fact} \textsc{lnk} \\
\glt `They did all that because of one person.' (2003smanmi-tamu, 101)
\end{exe} 

\subsubsection{Repetition of CNs}
%tɕeri tɯ-tɯ-rdoʁ tɯ-tɯ-rdoʁ ɲɯ-ŋu ma kɤndʑɯrɣa kɯ-fse kɯ-ɤrɤkhɯmkʰɤl kɯ-fse maŋe. 
\subsection{CNs and semantic classes} \label{sec:CN.classification}
\subsection{CNs and IPNs}   \label{sec:CN.IPN}
CNs and IPNs stand out among other nouns in having an obligatory prefix. Since the citation form of both classes of nouns -- the numeral `one' prefix \forme{tɯ-} and the indefinite possessor prefixes \forme{tɯ-/tɤ-} are homophonous, its is not unexpected that conversion occurs between the two classes. 

Given the fact that numeral prefixes are a closed class (see § \ref{sec:numeral.prefixes} and in particular § \ref{sec:other.numeral.prefixes}), when one needs to use a quantifier without numeral prefix equivalent, it is necessary to convert the CN into an IPN in third person singular form, with the quantifier before it. In example (\ref{ex:nW.thamtCAt.WtWphu}), the quantifier \japhug{nɯ tʰamtɕɤt}{that many} has no prefixal form, and therefore the CN \japhug{tɯ-tɯpʰu}{one type} is converted to an IPN in \textsc{3sg} possessive form \forme{ɯ-tɯpʰu}.

\begin{exe}
\ex \label{ex:nW.thamtCAt.WtWphu}
 \gll tɕe paʁ tɯ-ɣjɤn pjɯ́-wɣ-ntɕʰa nɯ nɯ tʰamtɕɤt ɯ-tɯpʰu ɲɯ-ɬoʁ ra \\ 
 \textsc{lnk} pig one-time \textsc{ipfv}-\textsc{inv}-butcher \textsc{dem} \textsc{dem} all \textsc{3sg}.\textsc{poss}-type \textsc{ipfv}-come.out have.to:\textsc{fact} \\
\glt `Each time one kills a pig, one will get that many types (of foodstuff from it).' (05-paR, 93)
\end{exe}

Clear example of conversion from IPN to CN is rarer. The IPN \japhug{ɯ-mdoʁ}{colour}  (from Tibetan \tibet{མདོག་}{mdog}{colour}) derives a CN with the numeral prefix \forme{kɤntɕʰɯ-} `many' (on which see § \ref{sec:other.numeral.prefixes}) in (\ref{ex:kAntChWmdoR}). This may be an example of Augenblicksbildung (due to the preceding CN \japhug{kɤntɕʰɯ-tɯpʰu}{many types}).

\begin{exe}
\ex \label{ex:kAntChWmdoR}
 \gll  tɕeri kɤntɕʰɯ-tɯpʰu, kɤntɕʰɯ-mdoʁ ɣɤʑu. \\
 but many-types many-colour exist:\textsc{sens} \\
\glt  `There are many types (of the mushrooms called \forme{tɯqejmɤɣ}), and with many colours.' (24-zwArqhAjmAG, 50)
\end{exe}



\section{Counting time} \label{sec:time}
\subsection{Time CNs} \label{sec:CN.time}
 
\subsection{Time ordinals} \label{sec:time.ordinals}

\subsection{Other time adverbs} \label{sec:time.adverbs}
\section{Basic arithmetic operations} \label{sec:arithmetic}
%gram140505_math.wav

Although Japhug lacks an elaborate mathematical vocabulary, is it possible to express at least the basic arithmetic operations without recourse to Chinese. 

The CN \japhug{tɯ-rdoʁ}{one piece} is used instead of the numeral \japhug{ci}{one} is calculations (see \ref{ex:tWrdoR.pjWwGta} below).

Addition is expressed by the construction in (\ref{ex:tWrdoR.pjWwGta}), with the verb \japhug{ta}{put} (or alternatively, \japhug{ɣɤjɯ}{add}, in the case below \forme{pjɯ́-wɣ-ɣɤjɯ}) and the locative noun \japhug{ɯ-taʁ}{on top of}, literally  `If one puts Y on the top of X, it makes Z' corresponding to $X+Y=Z$.  

\begin{exe}
\ex \label{ex:tWrdoR.pjWwGta}
 \gll kɯmŋu ɯ-taʁ tɯ-rdoʁ pjɯ́-wɣ-ta tɕe kɯtʂɤɣ tu-βze ŋu.  \\
 five \textsc{3sg}-on one-piece \textsc{ipfv}:\textsc{down}-\textsc{inv}-put \textsc{lnk} six \textsc{ipfv}-do[III] be:\textsc{fact} \\
 \glt `Five plus one equals six.' (elicitation)
\end{exe}