\chapter{Non-finite verbal morphology}

\section{Participles}
Participles are nominalized verb forms that keep some verbal characteristics: they can serve as predicates of subordinate clauses (relative or complement clauses), take TAM, polarity and associated motion marking, and preserve the verb's argument structure.

Participles differ from finite verbs in three ways. First, they cannot serve as the predicate of a main clause. Second, they are not compatible with the personal prefixes and suffixes of the intransitive and transitive conjugations (including direct/inverse marking and past transitive \forme{-t-}, § XXX).\footnote{Japhug is identical in this regard to Tshobdun and Zbu, but crucially differs from Situ, where nominalized forms in \forme{kə-} can bear indexation suffixes (\citet{jackson06guanxiju}, \citet{jacksonlin07}) } Rather, like nouns, they can take a possessive prefix which can be coreferent with one the arguments. Due to the general impossibility of stacking possessive prefixes (§ \ref{sec:possessive.paradigm}), at most only one argument can be indexed this way. Third, there are restrictions on TAM marking on participles.

There are three participles in Japhug; the subject S/A participle in \forme{kɯ-}, the object participle in \forme{kɤ-} and the oblique participle in \forme{sɤ-}. 

Complex participial forms, including negative, associated motion or TAM prefixes are possible, as shown by example \ref{ex:WGWjAkWqru}. However, never more than four inflexional prefixes are found; forms with all five prefixal slots filled (such as $\dagger$\forme{ɯ-ɣɯ-jɤ-kɯ-qru}) are not accepted by Tshendzin.

 \begin{exe}
\ex \label{ex:WGWjAkWqru}
\gll ɯ-ɣɯ-jɤ-kɯ-qru  	tɤ-tɕɯ  	   \\
  \textsc{3sg-cisloc-pfv-nmlz:}S/A-meet \textsc{indef.poss}-boy   \\
\glt `The boy who had come to look for her.' (The three sisters, 231)
 \end{exe}

Table \ref{tab:template.nmlz} summarizes the template of participial verb forms.

\begin{table}[h]
\caption{The template of participial verb forms in Japhug} \centering \label{tab:template.nmlz}
\resizebox{\columnwidth}{!}{
\begin{tabular}{lllllll}
\toprule
-5 & -4&-3 &-2&-1& \ro{} \\
possessive & negative&associated   & TAM & participle prefix &enlarged  \\
prefix & prefix &motion prefix  &orientation&&stem\\
\bottomrule
\end{tabular}}
\end{table}

Stem alternation is reduced in participle forms: stem 3 (§ XXX) never occurs. The few verbs that have a distinct stem 2 (\japhug{ɕe}{go}, \japhug{ɣi}{come}, \japhug{ti}{say} and derived forms) however, use this stem in subject and object participles with perfective orientational prefixes (§ XXX), in forms like \forme{jɤ-kɯ-ɣe} \textsc{pfv}-\textsc{nmlz}:S/A-come[II] `the one who came'
or \forme{tɤ-kɤ-tɯt} \textsc{pfv}-\textsc{nmlz}:P-say[II] `what was said'.
 

\subsection{Subject participles}
The subject participle, built by adding the prefix \forme{kɯ-} to the verb stem, designates an entity corresponding to the intransitive subject (\ref{ex:die}, § \ref{sec:absolutive.S} and XXX) or the transitive subject (\ref{ex:kill}, § \ref{sec:A.kW}, § XXX) of the verb. 

 \begin{exe}
\ex \label{ex:kWsi}
\gll kɯ-si    \\
  \textsc{nmlz}:S/A-die \\
 \glt  `The dead one' (many attestations)
\end{exe}

 \begin{exe} 
\ex \label{ex:WkWndza}
\gll ɯ-kɯ-ndza    \\
  \textsc{3sg}-\textsc{nmlz}:S/A-eat \\
 \glt  `The one who eats it.' (many attestations)
\end{exe}

In this section, I discuss first morphological issues (possessive prefixes § \ref{ex:subject.participle.ambiguities} and ambiguous forms § \ref{ex:subject.participle.ambiguities}), and then present the various functions of subject participles, including participial relatives (§ \ref{ex:subject.participle.subject.relative} and § \ref{ex:subject.participle.other.relative}), lexicalized  XXXXX

\subsubsection{Possessive prefixes on subject participles}  {ex:subject.participle.possessive}

In the case of transitive verbs, a possessive prefix coreferent with the object is obligatory when no overt object is present (\textsc{3sg} \forme{ɯ-} in \ref{ex:WkWndza}), and when no other prefix is added to the participle.

When a polarity or orientation prefix is present, the possessive prefix is optional, as shown by forms like \forme{mɤ-kɯ-ndza} `the one which does not eat (it)' in (\ref{ex:mAkWndza}), as opposed to \forme{ɯ-mɤ-kɯ-mto} `the one who does not see it' in (\ref{ex:WmAkWmto}) with both possessive \forme{ɯ-} and the negative prefix \forme{mɤ-}.

 \begin{exe} 
\ex \label{ex:mAkWndza}
\gll  tɤ-mtʰɯm ʁɟa ʑo ma nɯ ma, nɤki, tɯjpu mɤ-kɯ-ndza ci tu tɕe, \\
\textsc{indef}.\textsc{poss}-meat completely \textsc{emph} \textsc{lnk} \textsc{dem} apart.from \textsc{filler} flour.based.food \textsc{neg}-\textsc{nmlz}:S/A-eat \textsc{indef} exist:\textsc{fact} \textsc{lnk} \\
\glt  `There is (an animal like the mouse) which only eats meat, not food made from flour.' (27-spjaNkW, 202-2063)
\end{exe}

 \begin{exe} 
\ex \label{ex:WmAkWmto}
\gll  li nɯnɯ kɯnɤ ɯ-kɯ-mto ɣɤʑu, ɯ-mɤ-kɯ-mto ɣɤʑu. \\
again \textsc{dem} also \textsc{3sg}.\textsc{poss}-\textsc{nmlz}:S/A-see exist:\textsc{sens} \textsc{3sg}.\textsc{poss}-\textsc{neg}-\textsc{nmlz}:S/A-see exist:\textsc{sens} \\
\glt `There are (people) who see (find) it, and people who don't.' (20-sWrna, 20)
\end{exe}

With intransitive verbs, including adjectival stative verbs (§ XXX), a possessive prefix can also be added. In the case of semi-transitive verbs (§ XXX), the possessive can refer to the semi-object (§ \ref{sec:semi.object}), as in example (\ref{ex:WkWrga.pWdAn}).

 \begin{exe} 
\ex \label{ex:WkWrga.pWdAn}
\gll  nɯ ɕɯŋgɯ tɕe, ɯ-kɯ-rga pɯ-dɤn. \\
\textsc{dem} before \textsc{lnk} \textsc{3sg}.\textsc{poss}-\textsc{nmlz}:S/A-like \textsc{pst}.\textsc{ipfv}-be.many \\
\glt  `Before, there used to be many people who liked it.' (12-Zmbroko, 112)
\end{exe}

It can also refer to the beneficiary (which is normally marked with genitive or possessive prefixes, see § \ref{sec:other.uses.poss.prefixes} and § \ref{sec:gen.beneficiary}), as in (\ref{ex:tWZo.tWkWpe}) and (\ref{ex:aZo.akWra}).

 \begin{exe} 
\ex \label{ex:tWZo.tWkWpe}
\gll  kɯ-pe tú-wɣ-nɤma tɕe li tɯʑo tɯ-kɯ-pe tu \\
\textsc{nmlz}:S/A-be.good \textsc{ipfv}-\textsc{inv}-make \textsc{lnk} again \textsc{genr} \textsc{genr}.\textsc{poss}-\textsc{nmlz}:S/A-be.good exist:\textsc{fact} \\
\glt  `If one does good things, one will also have good things.' (140518 mao he laoshu, 124)
\end{exe}

 \begin{exe} 
\ex \label{ex:aZo.akWra}
\gll  aʑo a-kɯ-ra nɯra a-tɤ-tɯ-ste qʰendɤre aʑo nɯnɯ, nɤki, ku-nɤtsi-a jɤɣ \\
\textsc{1sg} \textsc{1sg}.\textsc{poss}-\textsc{nmlz}:S/A-have.to \textsc{dem}:\textsc{pl} \textsc{irr}-\textsc{pfv}-2-do.like[III] \textsc{lnk} \textsc{dem} \textsc{filler} \textsc{ipfv}-hide[III]-\textsc{1sg} be.possible:\textsc{fact}  \\
\glt  `If you do the things I need, I will keep it secret.'  (2014-kWlAG, 247)
\end{exe}

Since participles are also noun-like, the possessive prefixes can be real possessive, and be preceded with a genitive phrase as in (\ref{ex:WkWrga.pWdAn}) with \forme{nɯ-kɯ-mna} `the best among them' = `their chief'.

 \begin{exe} 
\ex \label{ex:WkWrga.pWdAn}
\gll tɕaχpa ra ɣɯ nɯ-kɯ-mna nɯ wuma ʑo pjɤ-nɯrɤŋom. \\
bandit \textsc{pl} \textsc{gen} \textsc{3pl}.\textsc{poss}-\textsc{nmlz}:S/A-be.better \textsc{dem} really \textsc{emph} \textsc{ifr}-be.upset \\
\glt `The chief of the bandits was very upset.' (140512 alibaba-zh, 195)
\end{exe}
\subsubsection{Ambiguities}  \label{ex:subject.participle.ambiguities}
\subsubsection{Subject relative clauses}  \label{ex:subject.participle.subject.relative}

\subsubsection{Other relative clauses}  \label{ex:subject.participle.other.relative}

\subsubsection{Lexicalized subject participles} \label{ex:lexicalized.subject.participle}

\subsubsection{Complementation strategies}  \label{ex:subject.participle.complementation}

\subsubsection{Adverbials} \label{ex:subject.participle.adverbial}

\subsection{Object participles}
The object participle corresponds to the object (\ref{sec:absolutive.P}) or semi-object (§ \ref{sec:semi.object}) of the verb. This form is homophonous with the velar infinitive (§ \ref{sec:velar.inf}).

 \begin{exe} 
\ex \label{ex:kill2}
\gll kɤ-sat    \\
   \textsc{nmlz}:P-kill \\
 \glt  `The one that is killed.' (elicited)
 \end{exe}
 
The object participle can appear with an optional possessive prefix coreferent with the transitive subject as in (\ref{ex:kill3}).
  
  \begin{exe}
\ex \label{ex:kill3}
\gll a-kɤ-sat    \\
   \textsc{1sg-nmlz}:P-kill \\
 \glt  `The one that I kill.' (elicited)
 \end{exe}

\subsection{Oblique participles}
The \forme{sɤ}-prefix (and its allomorphs \forme{sɤz}- and \forme{z}-) is used for non-core argument nominalization, in particular recipient of indirective verbs (§ \ref{sec:gen.beneficiary}, § \ref{sec:dative}), instruments (\ref{sec:instr.kW}), place and time adjuncts. It takes a possessive prefix which can be coreferent with any core argument (subject or object).

   \begin{exe}
\ex \label{ex:come}
\gll ɯ-sɤ-ɣi    \\
   \textsc{3sg-nmlz:oblique}-come \\
 \glt  `The place/moment where/when it comes.' (elicited)
 \end{exe}
 
 %nɯnɯ ɯ-sɤtɕha, tɤjmɤɣ ɯ-sɤ-tu ɯ-sɤ-me ɣɤʑu.
\section{Infinitives}

\subsection{Velar infinitives} \label{sec:velar.inf}
\subsection{Dental infinitives} \label{sec:dental.inf}
\subsection{Bare infinitives} \label{sec:bare.inf}
\section{Degree nominal}

\section{Other deverbal nouns}

\section{Converbs}
\subsection{Gerund} \label{sec:gerund}
\subsection{Purposive} \label{sec:purposive.converb}
\subsection{Immediate} \label{sec:immediate.converb}
