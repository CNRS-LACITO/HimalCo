\chapter{Non-finite verbal morphology}

\section{Participles}
Participles are nominalized verb forms that keep some verbal characteristics: they can serve as predicates of subordinate clauses (relative or complement clauses), take TAM, polarity and associated motion marking, and preserve the verb's argument structure.

Participles differ from finite verbs in three ways. First, they cannot serve as the predicate of a main clause. Second, they are not compatible with the personal prefixes and suffixes of the intransitive and transitive conjugations (including direct/inverse marking and past transitive \forme{-t-}, § XXX). Rather, like nouns, they can take a possessive prefix which can be coreferent with one the arguments. Due to the general impossibility of stacking possessive prefixes (§ \ref{sec:possessive.paradigm}), at most only one argument can be indexed this way. Third, there are restrictions on TAM marking on participles.

There are three participles in Japhug; the subject S/A participle in \forme{kɯ-}, the object participle in \forme{kɤ-} and the oblique participle in \forme{sɤ-}. 

Complex participial forms, including negative, associated motion or TAM prefixes are possible, as shown by example \ref{ex:WGWjAkWqru}. However, never more than four inflexional prefixes are found; forms with all five prefixal slots filled (such as $\dagger$\forme{ɯ-ɣɯ-jɤ-kɯ-qru}) are not accepted by Tshendzin.

 \begin{exe}
\ex \label{ex:WGWjAkWqru}
\gll ɯ-ɣɯ-jɤ-kɯ-qru  	tɤ-tɕɯ  	   \\
  \textsc{3sg-cisloc-pfv-nmlz:}S/A-meet \textsc{indef.poss}-boy   \\
\glt `The boy who had come to look for her.' (The three sisters, 231)
 \end{exe}

Table \ref{tab:template.nmlz} summarizes the template of participial verb forms.

\begin{table}[h]
\caption{The template of participial verb forms in Japhug} \centering \label{tab:template.nmlz}
\resizebox{\columnwidth}{!}{
\begin{tabular}{lllllll}
\toprule
-5 & -4&-3 &-2&-1& \ro{} \\
possessive & negative&associated   & TAM & participle prefix &enlarged  \\
prefix & prefix &motion prefix  &orientation&&stem\\
\bottomrule
\end{tabular}}
\end{table}

Stem alternation is reduced in participle form: stem 3 (§ XXX) never occurs. The few verbs that have a distinct stem 2, however, use this stem in subject and object participles with perfective prefixes: XXXXXXX

\subsection{Subject participles}
The subject participle is used to designate an entity corresponding to the intransitive subject (\ref{ex:die}, § \ref{sec:absolutive.S} and XXX) or the transitive subject (\ref{ex:kill}, § \ref{sec:A.kW}, § XXX) of the verb. In the case of transitive verbs, a possessive prefix coreferent with the object is obligatory when no overt object is present, and when no other prefix is added to the participle.

 \begin{exe}
\ex \label{ex:die}
\gll kɯ-si    \\
  \textsc{nmlz}:S/A-die \\
 \glt  `The dead one' (elicited)
\end{exe}

 \begin{exe} 
\ex \label{ex:kill}
\gll ɯ-kɯ-sat    \\
  \textsc{3sg}-\textsc{nmlz}:S/A-kill \\
 \glt  `The one who kills him.' (elicited)
\end{exe}

\subsubsection{Subject relative clauses}

\subsubsection{Lexicalized subject participles}

\subsubsection{Complementation strategies}

\subsubsection{Adverbials}

\subsection{Object participles}
The object participle corresponds to the object (\ref{sec:absolutive.P}) or semi-object (§ \ref{sec:semi.object}) of the verb. This form is homophonous with the velar infinitive (§ \ref{sec:velar.inf}).

 \begin{exe} 
\ex \label{ex:kill2}
\gll kɤ-sat    \\
   \textsc{nmlz}:P-kill \\
 \glt  `The one that is killed.' (elicited)
 \end{exe}
 
The object participle can appear with an optional possessive prefix coreferent with the transitive subject as in (\ref{ex:kill3}).
  
  \begin{exe}
\ex \label{ex:kill3}
\gll a-kɤ-sat    \\
   \textsc{1sg-nmlz}:P-kill \\
 \glt  `The one that I kill.' (elicited)
 \end{exe}

\subsection{Oblique participles}
The \forme{sɤ}-prefix (and its allomorphs \forme{sɤz}- and \forme{z}-) is used for non-core argument nominalization, in particular recipient of indirective verbs (§ \ref{sec:gen.beneficiary}, § \ref{sec:dative}), instruments (\ref{sec:instr.kW}), place and time adjuncts. It takes a possessive prefix which can be coreferent with any core argument (subject or object).

   \begin{exe}
\ex \label{ex:come}
\gll ɯ-sɤ-ɣi    \\
   \textsc{3sg-nmlz:oblique}-come \\
 \glt  `The place/moment where/when it comes.' (elicited)
 \end{exe}
\section{Infinitives}

\subsection{Velar infinitives} \label{sec:velar.inf}
\subsection{Dental infinitives} \label{sec:dental.inf}
\subsection{Bare infinitives} \label{sec:bare.inf}
\section{Degree nominal}

\section{Other deverbal nouns}