\chapter{Non-finite verbal morphology}

\section{Participles}
Participles are nominalized verb forms that keep some verbal characteristics: they can serve as predicates of subordinate clauses (relative or complement clauses), take TAM, polarity and associated motion marking, and preserve the verb's argument structure.

Participles differ from finite verbs in three ways. First, they cannot serve as the predicate of a main clause. Second, they are not compatible with the personal prefixes and suffixes of the intransitive and transitive conjugations (including direct/inverse marking and past transitive \forme{-t-}, § XXX).\footnote{Japhug is identical in this regard to Tshobdun and Zbu, but crucially differs from Situ, where nominalized forms in \forme{kə-} can bear indexation suffixes (\citet{jackson06guanxiju}, \citet{jacksonlin07}) } Rather, like nouns, they can take a possessive prefix which can be coreferent with one the arguments. Due to the general impossibility of stacking possessive prefixes (§ \ref{sec:possessive.paradigm}), at most only one argument can be indexed this way. Third, there are restrictions on TAM marking on participles: they have at most three forms (neutral, perfective and imperfective), and completely lack inferential (§ XXX), egophoric (§ XXX), sensory (§ XXX) and progressive forms (§ XXX).

There are three participles in Japhug; the subject S/A participle in \forme{kɯ-}, the object participle in \forme{kɤ-} and the oblique participle in \forme{sɤ-}. 

Complex participial forms, including negative, associated motion or TAM prefixes are possible, as shown by example \ref{ex:WGWjAkWqru}. However, never more than four inflexional prefixes are found; forms with all five prefixal slots filled (such as $\dagger$\forme{ɯ-ɣɯ-jɤ-kɯ-qru}) are not accepted by Tshendzin.

 \begin{exe}
\ex \label{ex:WGWjAkWqru}
\gll ɯ-ɣɯ-jɤ-kɯ-qru  	tɤ-tɕɯ  	   \\
  \textsc{3sg-cisloc-pfv-nmlz:}S/A-meet \textsc{indef.poss}-boy   \\
\glt `The boy who had come to look for her.' (The three sisters, 231)
 \end{exe}

Table \ref{tab:template.nmlz} summarizes the template of participial verb forms; more details are provided on possible and attested forms for each participle type in the following sections.

\begin{table}[h]
\caption{The template of participial verb forms in Japhug} \centering \label{tab:template.nmlz}
\resizebox{\columnwidth}{!}{
\begin{tabular}{lllllll}
\toprule
-5 & -4&-3 &-2&-1& \ro{} \\
possessive & negative&associated   & TAM & participle prefix &enlarged  \\
prefix & prefix &motion prefix  &orientation&&stem\\
\bottomrule
\end{tabular}}
\end{table}

Stem alternation is reduced in participle forms: stem 3 (§ XXX) never occurs. The few verbs that have a distinct stem 2 (\japhug{ɕe}{go}, \japhug{ɣi}{come}, \japhug{ti}{say} and derived forms) however, use this stem in subject and object participles with perfective orientational prefixes (§ XXX), in forms like \forme{jɤ-kɯ-ɣe} \textsc{pfv}-\textsc{nmlz}:S/A-come[II] `the one who came'
or \forme{tɤ-kɤ-tɯt} \textsc{pfv}-\textsc{nmlz}:P-say[II] `what was said'.
 

\subsection{Subject participles}
The subject participle, built by adding the prefix \forme{kɯ-} to the verb stem, designates an entity corresponding to the intransitive subject (\ref{ex:die}, § \ref{sec:absolutive.S} and XXX) or the transitive subject (\ref{ex:kill}, § \ref{sec:A.kW}, § XXX) of the verb. 

 \begin{exe}
\ex \label{ex:kWsi}
\gll kɯ-si    \\
  \textsc{nmlz}:S/A-die \\
 \glt  `The dead one' (many attestations)
\end{exe}

 \begin{exe} 
\ex \label{ex:WkWndza}
\gll ɯ-kɯ-ndza    \\
  \textsc{3sg}-\textsc{nmlz}:S/A-eat \\
 \glt  `The one who eats it.' (many attestations)
\end{exe}

In this section, I discuss first morphological issues (possessive prefixes § \ref{ex:subject.participle.ambiguities}, other prefixes § \ref{ex:subject.participle.other.prefixes} and ambiguous forms § \ref{ex:subject.participle.ambiguities}), and then present the various functions of subject participles, including participial relatives (§ \ref{ex:subject.participle.subject.relative} and § \ref{ex:subject.participle.other.relative}), complementation strategies (\ref{ex:subject.participle.complementation}), adverbials (§ \ref{ex:subject.participle.adverbial}), as well as the case of lexicalized participles (§ \ref{ex:lexicalized.subject.participle}).
 
 

\subsubsection{Possessive prefixes on subject participles}  \label{ex:subject.participle.possessive}

In the case of transitive verbs, a possessive prefix coreferent with the object is obligatory when no overt object is present (\textsc{3sg} \forme{ɯ-} in \ref{ex:WkWndza}), and when no other prefix is added to the participle.

When a polarity or orientation prefix is present, the possessive prefix is optional, as shown by forms like \forme{mɤ-kɯ-ndza} `the one which does not eat (it)' in (\ref{ex:mAkWndza}), as opposed to \forme{ɯ-mɤ-kɯ-mto} `the one who does not see it' in (\ref{ex:WmAkWmto}) with both possessive \forme{ɯ-} and the negative prefix \forme{mɤ-}.

 \begin{exe} 
\ex \label{ex:mAkWndza}
\gll  tɤ-mtʰɯm ʁɟa ʑo ma nɯ ma, nɤki, tɯjpu mɤ-kɯ-ndza ci tu tɕe, \\
\textsc{indef}.\textsc{poss}-meat completely \textsc{emph} \textsc{lnk} \textsc{dem} apart.from \textsc{filler} flour.based.food \textsc{neg}-\textsc{nmlz}:S/A-eat \textsc{indef} exist:\textsc{fact} \textsc{lnk} \\
\glt  `There is (an animal like the mouse) which only eats meat, not food made from flour.' (27-spjaNkW, 202-2063)
\end{exe}

 \begin{exe} 
\ex \label{ex:WmAkWmto}
\gll  li nɯnɯ kɯnɤ ɯ-kɯ-mto ɣɤʑu, ɯ-mɤ-kɯ-mto ɣɤʑu. \\
again \textsc{dem} also \textsc{3sg}.\textsc{poss}-\textsc{nmlz}:S/A-see exist:\textsc{sens} \textsc{3sg}.\textsc{poss}-\textsc{neg}-\textsc{nmlz}:S/A-see exist:\textsc{sens} \\
\glt `There are (people) who see (find) it, and people who don't.' (20-sWrna, 20)
\end{exe}

With intransitive verbs, including adjectival stative verbs (§ XXX), a possessive prefix can also be added. In the case of semi-transitive verbs (§ XXX), the possessive can refer to the semi-object (§ \ref{sec:semi.object}), as in example (\ref{ex:WkWrga.pWdAn}).

 \begin{exe} 
\ex \label{ex:WkWrga.pWdAn}
\gll  nɯ ɕɯŋgɯ tɕe, ɯ-kɯ-rga pɯ-dɤn. \\
\textsc{dem} before \textsc{lnk} \textsc{3sg}.\textsc{poss}-\textsc{nmlz}:S/A-like \textsc{pst}.\textsc{ipfv}-be.many \\
\glt  `Before, there used to be many people who liked it.' (12-Zmbroko, 112)
\end{exe}

It can also refer to the beneficiary (which is normally marked with genitive or possessive prefixes, see § \ref{sec:other.uses.poss.prefixes} and § \ref{sec:gen.beneficiary}), as in (\ref{ex:tWZo.tWkWpe}) and (\ref{ex:aZo.akWra}).

 \begin{exe} 
\ex \label{ex:tWZo.tWkWpe}
\gll  kɯ-pe tú-wɣ-nɤma tɕe li tɯʑo tɯ-kɯ-pe tu \\
\textsc{nmlz}:S/A-be.good \textsc{ipfv}-\textsc{inv}-make \textsc{lnk} again \textsc{genr} \textsc{genr}.\textsc{poss}-\textsc{nmlz}:S/A-be.good exist:\textsc{fact} \\
\glt  `If one does good things, one will also have good things.' (140518 mao he laoshu, 124)
\end{exe}

 \begin{exe} 
\ex \label{ex:aZo.akWra}
\gll  aʑo a-kɯ-ra nɯra a-tɤ-tɯ-ste qʰendɤre aʑo nɯnɯ, nɤki, ku-nɤtsi-a jɤɣ \\
\textsc{1sg} \textsc{1sg}.\textsc{poss}-\textsc{nmlz}:S/A-have.to \textsc{dem}:\textsc{pl} \textsc{irr}-\textsc{pfv}-2-do.like[III] \textsc{lnk} \textsc{dem} \textsc{filler} \textsc{ipfv}-hide[III]-\textsc{1sg} be.possible:\textsc{fact}  \\
\glt  `If you do the things I need, I will keep it secret.'  (2014-kWlAG, 247)
\end{exe}

Since participles are also noun-like, the possessive prefixes can be real possessive, and be preceded with a genitive phrase as in (\ref{ex:WkWrga.pWdAn}) with \forme{nɯ-kɯ-mna} `the best among them' = `their chief'.

 \begin{exe} 
\ex \label{ex:WkWrga.pWdAn}
\gll tɕaχpa ra ɣɯ nɯ-kɯ-mna nɯ wuma ʑo pjɤ-nɯrɤŋom. \\
bandit \textsc{pl} \textsc{gen} \textsc{3pl}.\textsc{poss}-\textsc{nmlz}:S/A-be.better \textsc{dem} really \textsc{emph} \textsc{ifr}-be.upset \\
\glt `The chief of the bandits was very upset.' (140512 alibaba-zh, 195)
\end{exe}

This construction is used as a type of superlative, as in (\ref{ex:thamtCAt.GW.nWkWmpCAr}), where \forme{pɣa tʰamtɕɤt ɣɯ nɯ-kɯ-mpɕɤr nɯ}, literally meaning `the beautiful one (among/of) all birds' is to be understood as `the most beautiful of all birds.' (see § XXX on superlative constructions).

 \begin{exe} 
\ex \label{ex:thamtCAt.GW.nWkWmpCAr}
\gll tɕe pɣa tʰamtɕɤt ɣɯ nɯ-kɯ-mpɕɤr nɯ rmɤβja ɲɯ-ŋu.  \\
\textsc{lnk} bird all \textsc{gen} \textsc{3pl}.\textsc{poss}-\textsc{nmlz}:S/A-be.beautiful \textsc{dem} peacock \textsc{sens}-be \\
\glt `The peacock is the most beautiful of all birds.' (24-ZmbrWpGa, 84)
\end{exe}

\subsubsection{Associated motion, polarity and orientation prefixes}  \label{ex:subject.participle.other.prefixes}
Of all non-finite verb forms, subject participles allow the richest possible combinations of inflexional prefixes: associated motion (§ \ref{sec:associated.motion}, example \ref{ex:WCWkWphWt}) below with the translocative \forme{ɕɯ-}), polarity (§ XXX, see \ref{ex:WmAkWmto} above) and orientation prefixes marking TAM (§ XXX) all can be prefixed. 
 
\begin{exe}
\ex \label{ex:WCWkWphWt}
 \gll tɕeri nɯra ɯ-ɕɯ-kɯ-pʰɯt ra kɯ-tu me ma,   \\
 \textsc{lnk} \textsc{dem}:\textsc{pl} \textsc{3sg}.\textsc{poss}-\textsc{transloc}-\textsc{nmlz}:S/A-cut \textsc{pl} \textsc{nmlz}:S/A-exist not.exist:\textsc{fact} \textsc{lnk} \\
 \glt `But nobody goes to collect (its stalks).' (11-paRzwamWntoR, 90)
\end{exe}

Most examples in the corpus have one or two prefixes, either combining a possessive prefix with another prefix (as in \ref{ex:WmAkWmto} and \ref{ex:WCWkWphWt}), or combining a negative prefix with an orientation prefix, as in (\ref{ex:mWnWkWsna}).

 \begin{exe}
\ex \label{ex:mWnWkWsna}
 \gll tɕe kʰa ɣɯ ɯ-ndzɤtsʰi ɯ-ro nɯ-kɯ-ri nɯra, mɯ-nɯ-kɯ-sna nɯra, nɯra paʁ kɯ ʁɟa tu-ndze ɲɯ-ŋu \\
 \textsc{lnk} house \textsc{gen} \textsc{3sg}.\textsc{poss}-food \textsc{3sg}.\textsc{poss}-excess \textsc{pfv}-\textsc{nmlz}:S/A-left \textsc{dem}:\textsc{pl}  \textsc{neg}-\textsc{pfv}-\textsc{nmlz}:S/A-be.good \textsc{dem}:\textsc{pl} \textsc{dem}:\textsc{pl} pig \textsc{erg} completely  \textsc{ipfv}-eat[III] \textsc{sens}-be \\
 \glt  `Food from the house that has been left over, or which is not good any more, pigs eat all of it.' (05-paR, 33)
\end{exe}

Subject participles with three prefixes before the participle prefix \forme{kɯ-} are possible, but attestations are extremely rare. Example (\ref{ex:WGWjAkWqru}) above shows the combination of a possessive, an associated motion and an orientation prefixes (\forme{ɯ-ɣɯ-jɤ-kɯ-qru} `the one who had come to meet/look for her'), and (\ref{ex:WmApjWnWfkAB}) below that of a possessive, a polarity and an orientation prefixes.

\begin{exe}
\ex \label{ex:WmApjWnWfkAB}
 \gll ɯ-pjɯ-kɯ-nɯ-fkaβ tu, ɯ-mɤ-pjɯ-kɯ-nɯ-fkaβ tu ri nɯ kɯ-fse tu-nɯ-ndza-nɯ ɕti. \\
 \textsc{3sg}.\textsc{poss}-\textsc{ipfv}-\textsc{nmlz}:S/A-\textsc{auto}-cover exist:\textsc{fact}  \textsc{3sg}.\textsc{poss}-\textsc{neg}-\textsc{ipfv}-\textsc{nmlz}:S/A-\textsc{auto}-cover exist:\textsc{fact} \textsc{lnk} \textsc{dem} \textsc{nmlz}:S/A-be.like \textsc{ipfv}-\textsc{auto}-eat-\textsc{pl} be.\textsc{affirm}:\textsc{fact} \\
 \glt `There are people who cover it (with a lid while cooking), and people who don't, they eat it like that.' (23-mbrAZim, 22-23)
\end{exe}

There are no constraints on the number of derivational prefixes in participial forms. The derivational prefixes are all closer to the verb root than the participle prefix \forme{kɯ-}, and thus follow it as shown by (\ref{ex:WmApjWnWfkAB}), where the autobenefactive \forme{-nɯ-}, the leftmost of all derivational prefixes (§ XXX), is placed after \forme{kɯ-}. 

Two of the four series of orientation prefixes are possible with subject participles. With series A prefixes (\forme{tɤ-} `up', \forme{pɯ-} `down' etc, § XXX), the participle of dynamic verbs is perfective as \forme{tʰɯ-kɯ-ɣe} `the one who came' in (\ref{ex:WkWntsGe.thWkWGe}). With series B prefixes (\forme{tu-} `up', \forme{pjɯ-} `down' etc, § XXX), it has a habitual imperfective meaning with dynamic verbs as \forme{ju-kɯ-ɣi} `the one who (usually) comes' in (\ref{ex:WkWndza.jukWGi}).\footnote{These two examples also illustrate the use of subject participles as purposive complements with the forms \forme{ɯ-kɯ-ntsɣe} and \forme{ɯ-kɯ-ndza} (see § \ref{ex:subject.participle.complementation}, § XXX).} The prefixes \forme{ɲɯ-}and \forme{ku-} do appear on subject participles, but only to express imperfective: there are no egophoric (§ XXX) or sensory (§ XXX) subject partiples.

\begin{exe}
\ex \label{ex:WkWntsGe.thWkWGe}
\gll iɕqʰa qaʑo ɯ-kɯ-ntsɣe tʰɯ-kɯ-ɣe nɯ ɯ-pʰe \\
the.aforementioned sheep \textsc{3sg}.\textsc{poss}-\textsc{nmlz}:S/A-sell \textsc{pfv}:\textsc{downstream}-\textsc{nmlz}:S/A-come[II] \textsc{dem} \textsc{3sg}.\textsc{poss}-\textsc{dat} \\
\glt  `(He told) the person who had come to sell the sheep.' (2003kandZislama, 212)
\end{exe}

\begin{exe}
\ex \label{ex:WkWndza.jukWGi}
\gll ɯ-kɯ-ndza ju-kɯ-ɣi nɯ pɣa ci ɲɯ-ŋu \\
\textsc{3sg}.\textsc{poss}-\textsc{nmlz}:S/A-eat \textsc{ipfv}-\textsc{nmlz}:S/A-come dem bird indef sens-be \\
\glt   `The one who comes to eat (the fruits) is a bird.' (2012qachGa, 22)
\end{exe}

The participles of stative verbs with series A and B orientation prefixes have an inchoative meaning, exactly like their finite counterpart (§ XXX and § XXX).  In (\ref{ex:YWkWjpum}) for instance, the imperfective participle \forme{ɲɯ-kɯ-jpum} from \japhug{jpum}{be thick} means `the one which becomes thicker', as opposed to the basic participle \forme{kɯ-jpum} `the thick one'.

\begin{exe}
\ex \label{ex:YWkWjpum}
 \gll ndʑu ɯ-ku jamar ɲɯ-kɯ-jpum ɣɤʑu nɤ, kɯ-wxti.  \\
 chopsticks \textsc{3sg}.\textsc{poss}-head about \textsc{ipfv}-\textsc{nmlz}:S/A-be.thick exist:\textsc{sens} \textsc{sfp} \textsc{nmlz}:S/A-be.big \\
 \glt  `There are (maggots) that grow as thick as the tip of a chopstick, the big ones.' (25-akWzgumba, 80)
\end{exe}

Imperfective participles of stative adjectival verbs are also appropriate to describe the gradient variation of a property across space rather than time. For instance, in (\ref{ex:YWkWjpum2}), the imperfective subject participles \forme{ku-kɯ-xtsʰɯm} and \forme{ɲɯ-kɯ-jpum} are used not to indicate a change across time, but to describe the shape of the gourd, which is progressively thinner towards the top and thicker towards the bottom.

\begin{exe}
\ex \label{ex:YWkWjpum2}
 \gll  tɕe ɯ-mat nɯnɯ, ɯ-taʁ ku-kɯ-xtsʰɯm, ɯ-pa ɲɯ-kɯ-jpum ci cʰɯ-βze ɲɯ-ŋu tɕe, nɯ <hulu> tu-sɤrmi-nɯ. \\
 \textsc{lnk} \textsc{3sg}.\textsc{poss}-fruit \textsc{dem} \textsc{3sg}.\textsc{poss}-up \textsc{ipfv}-\textsc{nmlz}:S/A-be.thin \textsc{3sg}.\textsc{poss}-down \textsc{ipfv}-\textsc{nmlz}:S/A-be.thick \textsc{indef} \textsc{ipfv}-make[III]  \textsc{sens}-be \textsc{lnk} gourd \textsc{ipfv}-call-\textsc{pl} \\
 \glt `It grows a fruit that is thinner (in diameter) on the upper part, and thicker on the lower part, people call it `gourd'.' (150825 huluwa, 3)
\end{exe}

The past imperfective of stative verbs is built using the series A prefix \forme{pɯ-} as in the corresponding finite form, as shown by XXXX

%ɯʑɤɣ nɯɕɯŋgɯ ɯ-nmaʁ pɯ-kɯ-ŋu tshɯraŋ nɯ pjɤ-mto tɕeri
%101

Subject participles can undergo totalitative reduplication (§ XXX), which applies to the first syllable of the word, whether it is the participle \forme{kɯ-} or an orientation prefix as in (\ref{ex:jWjAkWGe}), meaning `all of those who/that X'.

\begin{exe}
\ex \label{ex:jWjAkWGe}
\gll tɕe nɯnɯ ɯ-taʁ jɯ\redp{}jɤ-kɯ-ɣe nɯ ku-ndɤm ɲɯ-ŋu. \\
\textsc{lnk} \textsc{dem} \textsc{3sg}.\textsc{poss}-on \textsc{total}\redp{}\textsc{pfv}-\textsc{nmlz}:S/A-come[II] \textsc{ipfv}-take[III] \textsc{sens}-be \\
\glt `(The spider) catches all of the (insects) that have come on (the web).' (26-mYaRmtsaR, 108)
\end{exe}

\subsubsection{Ambiguities}  \label{ex:subject.participle.ambiguities}
\subsubsection{Subject relative clauses}  \label{ex:subject.participle.subject.relative}

%pɣɤtɕɯ nɯ kɯnɤ tɯ-ɣjɤn cinɤ ʑo tɤ-kɯ-mbri kɯ-me,
%nɯ to-ɣɤscɤscɤt ʑo to-mbri ɲɯ-ŋu,

\subsubsection{Other relative clauses}  \label{ex:subject.participle.other.relative}

\subsubsection{Complementation strategies}  \label{ex:subject.participle.complementation}
%tɤ-se mɯ-pjɯ-kɯ-ɬoʁ ftɕaka tu-βze-a tu-mdzoz-a pɯ-ŋu ma,
%24-pGArtsAG, 57
% tu-kɯ-ɣi ɲɤ-ɣɤme qhe, 
\subsubsection{Adverbials} \label{ex:subject.participle.adverbial}


\subsubsection{Lexicalized subject participles} \label{ex:lexicalized.subject.participle}
\japhug{kɯβʁa}{noble}, \japhug{kɯspoʁ}{hole}, \japhug{kɯcʰi}{candy}, \japhug{kɯmŋɤm}{ailment}, 

Nominalizations with the \forme{x-/ɣ-} prefix (§ \ref{sec:G.nmlz}) are ancient lexicalized subject participles that have undergone a syllable reduction rule (§ XXX) and have become completely separated from their base verb synchronically.

\subsection{Object participles}
The object participle corresponds to the object (\ref{sec:absolutive.P}) or semi-object (§ \ref{sec:semi.object}) of the verb. This form is homophonous with the velar infinitive (§ \ref{sec:velar.inf}).

 \begin{exe} 
\ex \label{ex:kill2}
\gll kɤ-sat    \\
   \textsc{nmlz}:P-kill \\
 \glt  `The one that is killed.' (elicited)
 \end{exe}
 
\subsubsection{Possessive prefixes on object participles}  \label{ex:object.participle.possessive} 
 
The object participle can appear with an optional possessive prefix coreferent with the transitive subject as in (\ref{ex:kill3}).
  
  \begin{exe}
\ex \label{ex:kill3}
\gll a-kɤ-sat    \\
   \textsc{1sg-nmlz}:P-kill \\
 \glt  `The one that I kill.' (elicited)
 \end{exe}

\subsection{Oblique participles}
The \forme{sɤ}-prefix (and its allomorphs \forme{sɤz}- and \forme{z}-) is used for non-core argument nominalization, in particular recipient of indirective verbs (§ \ref{sec:gen.beneficiary}, § \ref{sec:dative}), instruments (\ref{sec:instr.kW}), place and time adjuncts. It takes a possessive prefix which can be coreferent with any core argument (subject or object).

   \begin{exe}
\ex \label{ex:come}
\gll ɯ-sɤ-ɣi    \\
   \textsc{3sg-nmlz:oblique}-come \\
 \glt  `The place/moment where/when it comes.' (elicited)
 \end{exe}
 
 %nɯnɯ ɯ-sɤtɕha, tɤjmɤɣ ɯ-sɤ-tu ɯ-sɤ-me ɣɤʑu.
 %past imperfective:
 %nɯ ɕɯŋgɯ ɯ-sɤ-sqa pɯ-kɯ-ŋu ɯ-ŋgɯ
\section{Infinitives}

\subsection{Velar infinitives} \label{sec:velar.inf}
\subsection{Dental infinitives} \label{sec:dental.inf}
\subsection{Bare infinitives} \label{sec:bare.inf}
\section{Degree nominals} \label{sec:degree.nominals}

\section{Other deverbal nouns}

\subsection{Nominalization \forme{-z} suffix} \label{sec:z.nmlz}
\subsection{Nominalization \forme{ɣ-/x-} prefix} \label{sec:G.nmlz}
\begin{table}[H]
\caption{Irregular subject nominalizations in \ipa{ɣ}-- and \ipa{x}--} \label{tab:irregular.nmlz} \centering
\begin{tabular}{llll}
\lsptoprule
Noun & Base verb\\
\midrule
\japhug{ɣndʑɤβ}{disastrous fire} & \japhug{ndʑɤβ}{burn} \\
\japhug{ɯ-ɣɲaʁ}{disaster}& \japhug{ɲaʁ}{be black} \\
\japhug{ɯ-ɣɲɟɯ}{orifice} & \japhug{ɲɟɯ}{be opened} \\
\japhug{ɯ-xso}{empty, normal} &\japhug{so}{be empty} \\
\lspbottomrule
\end{tabular}
\end{table}

\section{Converbs}
\subsection{Gerund} \label{sec:gerund}
\subsection{Purposive} \label{sec:purposive.converb}
\subsection{Immediate} \label{sec:immediate.converb}
