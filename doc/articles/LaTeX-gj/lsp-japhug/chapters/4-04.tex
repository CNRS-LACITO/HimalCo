\chapter{Non-finite verbal morphology}

\section{Participles}
Participles are nominalized verb forms that keep some verbal characteristics: they can serve as predicates of subordinate clauses (relative or complement clauses), take TAM, polarity and associated motion marking, and preserve the verb's argument structure.

Participles differ from finite verbs in three ways. First, they cannot serve as the predicate of a main clause. Second, they are not compatible with the personal prefixes and suffixes of the intransitive and transitive conjugations (including direct/inverse marking and past transitive \forme{-t-}, § XXX).\footnote{Japhug is identical in this regard to Tshobdun and Zbu, but crucially differs from Situ, where nominalized forms in \forme{kə-} can bear indexation suffixes (\citet{jackson06guanxiju}, \citet{jacksonlin07}) } Rather, like nouns, they can take a possessive prefix which can be coreferent with one the arguments. Due to the general impossibility of stacking possessive prefixes (§\ref{sec:possessive.paradigm}), at most only one argument can be indexed this way. Third, there are restrictions on TAM marking on participles: they have at most three forms (neutral, perfective and imperfective), and completely lack inferential (§ XXX), egophoric (§ XXX), sensory (§ XXX) and progressive forms (§ XXX).

There are three participles in Japhug; the subject S/A participle in \forme{kɯ-}, the object participle in \forme{kɤ-} and the oblique participle in \forme{sɤ-}. 

Complex participial forms, including negative, associated motion or TAM prefixes are possible, as shown by example \ref{ex:WGWjAkWqru}. However, never more than four inflexional prefixes are found; forms with all five prefixal slots filled (such as $\dagger$\forme{ɯ-ɣɯ-jɤ-kɯ-qru}) are not accepted by Tshendzin.

 \begin{exe}
\ex \label{ex:WGWjAkWqru}
\gll ɯ-ɣɯ-jɤ-kɯ-qru tɤ-tɕɯ  \\
  \textsc{3sg}-\textsc{cisloc}-\textsc{pfv}-\textsc{nmlz}:S/A-meet \textsc{indef}.\textsc{poss}-boy   \\
\glt `The boy who had come to look for her.' (The three sisters, 231)
 \end{exe}

Table \ref{tab:template.nmlz} summarizes the template of participial verb forms; more details are provided on possible and attested forms for each participle type in the following sections.

\begin{table}[h]
\caption{The template of participial verb forms in Japhug} \centering \label{tab:template.nmlz}
\resizebox{\columnwidth}{!}{
\begin{tabular}{lllllll}
\toprule
-5 & -4&-3 &-2&-1& \ro{} \\
possessive & negative&associated   & TAM & participle prefix &enlarged  \\
prefix & prefix &motion prefix  &orientation&&stem\\
\bottomrule
\end{tabular}}
\end{table}

Stem alternation is reduced in participle forms: stem 3 (§ XXX) never occurs. The few verbs that have a distinct stem 2 (\japhug{ɕe}{go}, \japhug{ɣi}{come}, \japhug{ti}{say} and derived forms) however, use this stem in subject and object participles with perfective orientational prefixes (§ XXX), in forms like \forme{jɤ-kɯ-ɣe} \textsc{pfv}-\textsc{nmlz}:S/A-come[II] `the one who came'
or \forme{tɤ-kɤ-tɯt} \textsc{pfv}-\textsc{nmlz}:P-say[II] `what was said'.
 
\subsection{Subject participles} \label{sec:subject.participles}
The subject participle, built by adding the prefix \forme{kɯ-} to the verb stem, designates an entity corresponding to the intransitive subject (\ref{ex:kWsi}, §\ref{sec:absolutive.S} and XXX), a possessor of the subject, or the transitive subject (\ref{ex:WkWndza}, §\ref{sec:A.kW}, § XXX) of the base verb. 

 \begin{exe}
\ex \label{ex:kWsi}
\gll kɯ-si  \\
  \textsc{nmlz}:S/A-die \\
 \glt  `The dead one' (many attestations)
\end{exe}

 \begin{exe} 
\ex \label{ex:WkWndza}
\gll ɯ-kɯ-ndza \\
  \textsc{3sg}-\textsc{nmlz}:S/A-eat \\
 \glt  `The one who eats it.' (many attestations)
\end{exe}

With \forme{a-} initial verbs the \forme{kɯ-} prefix regularly merge with \forme{a-} as \forme{kɤ-}, a form which resembles an object participle. There is almost no ambiguity since all \forme{a-} initial verbs are intransitive, § XXX). The only exception are the semi-transitive verbs in \forme{a-}, such as \japhug{aro}{have}, whose subject participle \forme{kɯ-ɤro} `having, the one who has' and object participle \forme{kɤ-ɤro} both surface as \ipa{kɤro}.

The subject participle \forme{kɯ-} prefix is historically related to that of object participles (§\ref{sec:object.participle}), velar infinitives (§\ref{sec:velar.inf}) and deverbal nouns in \forme{x-/ɣ-} (§\ref{sec:G.nmlz}), and has cognates elsewhere in the family (§\ref{sec:velar.nmlz.history}).

In this section, I discuss first morphological issues (possessive prefixes §\ref{sec:subject.participle.ambiguities}, other prefixes §\ref{sec:subject.participle.other.prefixes} and ambiguous forms §\ref{sec:subject.participle.ambiguities}), and then present the various functions of subject participles, including participial relatives (§\ref{sec:subject.participle.subject.relative} and §\ref{sec:subject.participle.other.relative}), complementation strategies (§\ref{sec:subject.participle.complementation}), as well as the case of lexicalized participles (§\ref{sec:lexicalized.subject.participle}).
 
Examples which could be potentially be viewed as subject participles in converbial use are analyzed as \forme{kɯ-} infinitives (§\ref{sec:inf.converb}).

\subsubsection{Possessive prefixes on subject participles}  \label{sec:subject.participle.possessive}

In the case of transitive verbs, a possessive prefix coreferent with the object is obligatory when no overt object is present (\textsc{3sg} \forme{ɯ-} in \ref{ex:WkWndza}), and when no other prefix is added to the participle.

When another prefix (polarity, associated motion or orientation prefix) is present, the possessive prefix is optional, as shown by forms like \forme{mɤ-kɯ-ndza} `the one which does not eat (it)' in (\ref{ex:mAkWndza}), as opposed to \forme{ɯ-mɤ-kɯ-mto} `the one who does not see it' in (\ref{ex:WmAkWmto}) with both possessive \forme{ɯ-} and the negative prefix \forme{mɤ-}.

 \begin{exe} 
\ex \label{ex:mAkWndza}
\gll  tɤ-mtʰɯm ʁɟa ʑo ma nɯ ma, nɤki, tɯjpu mɤ-kɯ-ndza ci tu tɕe, \\
\textsc{indef}.\textsc{poss}-meat completely \textsc{emph} \textsc{lnk} \textsc{dem} apart.from \textsc{filler} flour.based.food \textsc{neg}-\textsc{nmlz}:S/A-eat \textsc{indef} exist:\textsc{fact} \textsc{lnk} \\
\glt  `There is (an animal like the mouse) which only eats meat, not food made from flour.' (27-spjaNkW, 202-2063)
\end{exe}

 \begin{exe} 
\ex \label{ex:WmAkWmto} 
\gll  li nɯnɯ kɯnɤ ɯ-kɯ-mto ɣɤʑu, ɯ-mɤ-kɯ-mto ɣɤʑu. \\
again \textsc{dem} also \textsc{3sg}.\textsc{poss}-\textsc{nmlz}:S/A-see exist:\textsc{sens} \textsc{3sg}.\textsc{poss}-\textsc{neg}-\textsc{nmlz}:S/A-see exist:\textsc{sens} \\
\glt `There are (people) who see (find) it, and people who don't.' (20-sWrna, 20)
\end{exe}

In the case of ditransitive verbs, the possessive prefix striclty refer to the object. With indirective verbs like \japhug{tʰu}{ask}, the possessive prefix is necessarily the theme, never the recipient. The form in (\ref{ex:AkWthu}) thus cannot be interpreted as meaning `the one who asks me (about it)'; the correct construction would be (\ref{ex:ACki.kWthu}), with the recipient in the dative case.

\begin{exe}
\ex \label{ex:AkWthu}
\gll a-kɯ-tʰu  \\
\textsc{1sg.poss}-\textsc{nmlz}:S/A-ask \\
\glt `The one asking for me (in marriage).' (elicited)
\ex \label{ex:ACki.kWthu}
\gll a-ɕki ɯ-kɯ-tʰu  \\
\textsc{1sg.poss}-\textsc{dat} \textsc{3sg}.\textsc{poss}-\textsc{nmlz}:S/A-ask \\ 
\glt `The one who asks me.' 
\end{exe}

With secundative verbs (§ XXX), the possessive prefix of the subject participle is obligatorily coreferent with the recipient, not the theme, as in (\ref{ex:nAkWmbi}).

\begin{exe}
\ex \label{ex:nAkWmbi}
\gll nɯ ma nɤ-kɯ-mbi me \\
\textsc{dem} apart.from \textsc{2sg}.\textsc{poss}-\textsc{nmlz}:S/A-give not.exist:\textsc{fact} \\
\glt `Nobody will give you another (daughter in marriage).' (2002qaCpa, 57)
\end{exe}
With intransitive verbs, including adjectival stative verbs (§ XXX), a possessive prefix can also be added. In the case of semi-transitive verbs (§ XXX), the possessive can refer to the semi-object (§\ref{sec:semi.object}), as in example (\ref{ex:WkWrga.pWdAn}).

 \begin{exe} 
\ex \label{ex:WkWrga.pWdAn}
\gll  nɯ ɕɯŋgɯ tɕe, ɯ-kɯ-rga pɯ-dɤn. \\
\textsc{dem} before \textsc{lnk} \textsc{3sg}.\textsc{poss}-\textsc{nmlz}:S/A-like \textsc{pst}.\textsc{ipfv}-be.many \\
\glt  `Before, there used to be many people who liked it.' (12-Zmbroko, 112)
\end{exe}

It can also refer to the beneficiary (which is normally marked with genitive or possessive prefixes, see §\ref{sec:other.uses.poss.prefixes} and §\ref{sec:gen.beneficiary}), as in (\ref{ex:tWZo.tWkWpe}) and (\ref{ex:aZo.akWra}).

 \begin{exe} 
\ex \label{ex:tWZo.tWkWpe}
\gll  kɯ-pe tú-wɣ-nɤma tɕe li tɯʑo tɯ-kɯ-pe tu \\
\textsc{nmlz}:S/A-be.good \textsc{ipfv}-\textsc{inv}-make \textsc{lnk} again \textsc{genr} \textsc{genr}.\textsc{poss}-\textsc{nmlz}:S/A-be.good exist:\textsc{fact} \\
\glt  `If one does good things, one will also have good things.' (140518 mao he laoshu-zh, 124)
\end{exe}

 \begin{exe} 
\ex \label{ex:aZo.akWra}
\gll  aʑo a-kɯ-ra nɯra a-tɤ-tɯ-ste qʰendɤre aʑo nɯnɯ, nɤki, ku-nɤtsi-a jɤɣ \\
\textsc{1sg} \textsc{1sg}.\textsc{poss}-\textsc{nmlz}:S/A-have.to \textsc{dem}:\textsc{pl} \textsc{irr}-\textsc{pfv}-2-do.like[III] \textsc{lnk} \textsc{dem} \textsc{filler} \textsc{ipfv}-hide[III]-\textsc{1sg} be.possible:\textsc{fact}  \\
\glt  `If you do the things I need, I will keep it secret.'  (2014-kWlAG, 247)
\end{exe}

Since participles are also noun-like, the possessive prefixes can be real possessive, and be preceded with a genitive phrase as in (\ref{ex:tCaXpa.ra.GW.nWkWmna}) with \forme{nɯ-kɯ-mna} `the best among them' = `their chief'.

 \begin{exe} 
\ex \label{ex:tCaXpa.ra.GW.nWkWmna}
\gll tɕaχpa ra ɣɯ nɯ-kɯ-mna nɯ wuma ʑo pjɤ-nɯrɤŋom. \\
bandit \textsc{pl} \textsc{gen} \textsc{3pl}.\textsc{poss}-\textsc{nmlz}:S/A-be.better \textsc{dem} really \textsc{emph} \textsc{ifr}-be.upset \\
\glt `The chief of the bandits was very upset.' (140512 alibaba-zh, 195)
\end{exe}

This construction is used as a type of superlative, as in (\ref{ex:thamtCAt.GW.nWkWmpCAr}), where \forme{pɣa tʰamtɕɤt ɣɯ nɯ-kɯ-mpɕɤr nɯ}, literally meaning `the beautiful one (among/of) all birds' is to be understood as `the most beautiful of all birds.' (see also \ref{ex:WsAsna} in §\ref{sec:other.oblique.participle.relatives} for a similar meaning with an oblique participle, and § XXX on superlative constructions).

 \begin{exe} 
\ex \label{ex:thamtCAt.GW.nWkWmpCAr}
\gll tɕe pɣa tʰamtɕɤt ɣɯ nɯ-kɯ-mpɕɤr nɯ rmɤβja ɲɯ-ŋu.  \\
\textsc{lnk} bird all \textsc{gen} \textsc{3pl}.\textsc{poss}-\textsc{nmlz}:S/A-be.beautiful \textsc{dem} peacock \textsc{sens}-be \\
\glt `The peacock is the most beautiful of all birds.' (24-ZmbrWpGa, 84)
\end{exe}

\subsubsection{Associated motion, polarity and orientation prefixes on subject participles}  \label{sec:subject.participle.other.prefixes}
Of all non-finite verb forms, subject participles allow the richest possible combinations of inflexional prefixes: associated motion (§\ref{sec:associated.motion}, example \ref{ex:WCWkWphWt}) below with the translocative \forme{ɕɯ-}), polarity (§ XXX, see \ref{ex:WmAkWmto} above) and orientation prefixes marking TAM (§ XXX) all can be prefixed. 
 
\begin{exe}
\ex \label{ex:WCWkWphWt}
 \gll tɕeri nɯra ɯ-ɕɯ-kɯ-pʰɯt ra kɯ-tu me ma,   \\
 \textsc{lnk} \textsc{dem}:\textsc{pl} \textsc{3sg}.\textsc{poss}-\textsc{transloc}-\textsc{nmlz}:S/A-cut \textsc{pl} \textsc{nmlz}:S/A-exist not.exist:\textsc{fact} \textsc{lnk} \\
 \glt `But nobody goes to collect (its stalks).' (11-paRzwamWntoR, 90)
\end{exe}

Two of the four series of orientation prefixes are possible with subject participles. With series A prefixes (\forme{tɤ-} `up', \forme{pɯ-} `down' etc, § XXX), the participle of dynamic verbs is perfective as \forme{tʰɯ-kɯ-ɣe} `the one who came' in (\ref{ex:WkWntsGe.thWkWGe}), and takes stem II (§ XXX). With series B prefixes (\forme{tu-} `up', \forme{pjɯ-} `down' etc, § XXX), it has a habitual imperfective meaning with dynamic verbs as \forme{ju-kɯ-ɣi} `the one who (usually) comes' in (\ref{ex:WkWndza.jukWGi}).\footnote{These two examples also illustrate the use of subject participles as purposive complements with the forms \forme{ɯ-kɯ-ntsɣe} and \forme{ɯ-kɯ-ndza} (see §\ref{sec:subject.participle.complementation}, § XXX).} The prefixes \forme{ɲɯ-}and \forme{ku-} do appear on subject participles, but only to express imperfective: there are no egophoric (§ XXX) or sensory (§ XXX) subject partiples.

\begin{exe}
\ex \label{ex:WkWntsGe.thWkWGe}
\gll iɕqʰa qaʑo ɯ-kɯ-ntsɣe tʰɯ-kɯ-ɣe nɯ ɯ-pʰe \\
the.aforementioned sheep \textsc{3sg}.\textsc{poss}-\textsc{nmlz}:S/A-sell \textsc{pfv}:\textsc{downstream}-\textsc{nmlz}:S/A-come[II] \textsc{dem} \textsc{3sg}.\textsc{poss}-\textsc{dat} \\
\glt  `(He told) the person who had come to sell the sheep.' (2003kandZislama, 212)
\end{exe}

\begin{exe}
\ex \label{ex:WkWndza.jukWGi}
\gll ɯ-kɯ-ndza ju-kɯ-ɣi nɯ pɣa ci ɲɯ-ŋu \\
\textsc{3sg}.\textsc{poss}-\textsc{nmlz}:S/A-eat \textsc{ipfv}-\textsc{nmlz}:S/A-come dem bird indef sens-be \\
\glt   `The one who comes to eat (the fruits) is a bird.' (2012qachGa, 22)
\end{exe}

The participles of stative verbs with series A and B orientation prefixes have an inchoative meaning, exactly like their finite counterpart (§ XXX and § XXX).  In (\ref{ex:YWkWjpum}) for instance, the imperfective participle \forme{ɲɯ-kɯ-jpum} from \japhug{jpum}{be thick} means `the one which becomes thicker', as opposed to the basic participle \forme{kɯ-jpum} `the thick one'.

\begin{exe}
\ex \label{ex:YWkWjpum}
 \gll ndʑu ɯ-ku jamar ɲɯ-kɯ-jpum ɣɤʑu nɤ, kɯ-wxti.  \\
 chopsticks \textsc{3sg}.\textsc{poss}-head about \textsc{ipfv}-\textsc{nmlz}:S/A-be.thick exist:\textsc{sens} \textsc{sfp} \textsc{nmlz}:S/A-be.big \\
 \glt  `There are (maggots) that grow as thick as the tip of a chopstick, the big ones.' (25-akWzgumba, 80)
\end{exe}

Imperfective participles of stative adjectival verbs are also appropriate to describe the gradient variation of a property across space rather than time. For instance, in (\ref{ex:YWkWjpum2}), the imperfective subject participles \forme{ku-kɯ-xtsʰɯm} and \forme{ɲɯ-kɯ-jpum} are used not to indicate a change across time, but to describe the shape of the gourd, which is progressively thinner towards the top and thicker towards the bottom.

\begin{exe}
\ex \label{ex:YWkWjpum2}
 \gll  tɕe ɯ-mat nɯnɯ, ɯ-taʁ ku-kɯ-xtsʰɯm, ɯ-pa ɲɯ-kɯ-jpum ci cʰɯ-βze ɲɯ-ŋu tɕe, nɯ <hulu> tu-sɤrmi-nɯ. \\
 \textsc{lnk} \textsc{3sg}.\textsc{poss}-fruit \textsc{dem} \textsc{3sg}.\textsc{poss}-up \textsc{ipfv}-\textsc{nmlz}:S/A-be.thin \textsc{3sg}.\textsc{poss}-down \textsc{ipfv}-\textsc{nmlz}:S/A-be.thick \textsc{indef} \textsc{ipfv}-make[III]  \textsc{sens}-be \textsc{lnk} gourd \textsc{ipfv}-call-\textsc{pl} \\
 \glt `It grows a fruit that is thinner (in diameter) on the upper part, and thicker on the lower part, people call it `gourd'.' (150825 huluwa, 3)
\end{exe}

The past imperfective of stative verbs is built using the series A prefix \forme{pɯ-} as in the corresponding finite forms (§ XXX). For instance, the past imperfective participle of \japhug{ŋu}{be} is \forme{pɯ-kɯ-ŋu} `the one who used to be ....', as in (\ref{ex:pWkWNu}).

\begin{exe}
\ex \label{ex:pWkWNu}
 \gll  ɯʑɤɣ nɯ ɕɯŋgɯ ɯ-nmaʁ pɯ-kɯ-ŋu tsʰɯraŋ nɯ pjɤ-mto \\
 \textsc{3sg}:\textsc{gen} \textsc{dem} before \textsc{3sg}.\textsc{poss}-husband \textsc{pst}.\textsc{ipfv}-\textsc{nmlz}:S/A-be p.n. \textsc{dem} \textsc{ifr}-see \\
\glt `She saw Tshering, who used to be her husband.' (qajdoskAt2002, 101)
\end{exe}

Most examples in the corpus have one or two prefixes, either combining a possessive prefix with another prefix (as in \ref{ex:WmAkWmto} and \ref{ex:WCWkWphWt}), or combining a negative prefix with an orientation prefix, as in (\ref{ex:mWnWkWsna}).

 \begin{exe}
\ex \label{ex:mWnWkWsna}
 \gll tɕe kʰa ɣɯ ɯ-ndzɤtsʰi ɯ-ro nɯ-kɯ-ri nɯra, mɯ-nɯ-kɯ-sna nɯra, nɯra paʁ kɯ ʁɟa tu-ndze ɲɯ-ŋu \\
 \textsc{lnk} house \textsc{gen} \textsc{3sg}.\textsc{poss}-food \textsc{3sg}.\textsc{poss}-excess \textsc{pfv}-\textsc{nmlz}:S/A-left \textsc{dem}:\textsc{pl}  \textsc{neg}-\textsc{pfv}-\textsc{nmlz}:S/A-be.good \textsc{dem}:\textsc{pl} \textsc{dem}:\textsc{pl} pig \textsc{erg} completely  \textsc{ipfv}-eat[III] \textsc{sens}-be \\
 \glt  `Food from the house that has been left over, or which is not good any more, pigs eat all of it.' (05-paR, 33)
\end{exe}

Subject participles with three prefixes before the participle prefix \forme{kɯ-} are possible, but attestations are rare. Example (\ref{ex:WGWjAkWqru}) above shows the combination of a possessive, an associated motion and an orientation prefixes (\forme{ɯ-ɣɯ-jɤ-kɯ-qru} `the one who had come to meet/look for her'), and (\ref{ex:WmApjWkWnWfkAB}) below that of a possessive, a polarity and an orientation prefixes.

\begin{exe}
\ex \label{ex:WmApjWkWnWfkAB}
 \gll ɯ-pjɯ-kɯ-nɯ-fkaβ tu, ɯ-mɤ-pjɯ-kɯ-nɯ-fkaβ tu ri nɯ kɯ-fse tu-nɯ-ndza-nɯ ɕti. \\
 \textsc{3sg}.\textsc{poss}-\textsc{ipfv}-\textsc{nmlz}:S/A-\textsc{auto}-cover exist:\textsc{fact}  \textsc{3sg}.\textsc{poss}-\textsc{neg}-\textsc{ipfv}-\textsc{nmlz}:S/A-\textsc{auto}-cover exist:\textsc{fact} \textsc{lnk} \textsc{dem} \textsc{nmlz}:S/A-be.like \textsc{ipfv}-\textsc{auto}-eat-\textsc{pl} be.\textsc{affirm}:\textsc{fact} \\
 \glt `There are people who cover it (with a lid while cooking), and people who don't, they eat it like that.' (23-mbrAZim, 22-23)
\end{exe}

In addition to imperfective orientation prefixes as in (\ref{ex:WmApjWkWnWfkAB}), it is possible for subject participles to combine possessive prefixes with \textit{perfective} orientation prefixes, as in (\ref{ex:WnWkWCar}). Subject participles are the only non-finite form attested with such a combination: the object participles do not allow combination of possessive and orientation prefixes (§\ref{sec:object.participle.possessive}) and the oblique participles cannot take perfective orientation prefixes (§\ref{sec:oblique.participle.orientation}).

 \begin{exe} 
\ex \label{ex:WnWkWCar}
\gll nɯnɯ ɯ-nɯ-kɯ-ɕar ɯ-pɯ-kɯ-mto nɯ ɣɯ ɲɯ-tʂaŋ ma nɯnɯ, nɤkinɯ, ɯ-ɕɯ-kɯ-βɟi nɯ ɣɯ mɯ́j-tʂaŋ \\
\textsc{dem} \textsc{3sg}.\textsc{poss}-\textsc{pfv}-\textsc{nmlz}:S/A-search \textsc{3sg}.\textsc{poss}-\textsc{pfv}-\textsc{nmlz}:S/A-see \textsc{dem} \textsc{gen} \textsc{sens}-be.fair \textsc{lnk} \textsc{dem} \textsc{filler} \textsc{3sg}.\textsc{poss}-\textsc{transloc}-\textsc{nmlz}:S/A-chase \textsc{dem} \textsc{gen} \textsc{neg}:\textsc{sens}-be.fair \\
\glt `It is fair that she would (be given) to the one who looked for her and found her, not to the ones chasing her.' (140517 buaishuohua, 123)
\end{exe}

The negative prefix has the form \forme{mɯ-} when occurring with a perfective orientation prefix as \forme{mɯ-nɯ-kɯ-sna} `the one that is not good anymore' in (\ref{ex:mWnWkWsna}) and \forme{mɤ-} when no orientation prefix is present (examples \ref{ex:WmAkWmto} and \ref{ex:mAkWndza} above). With the imperfective orientation prefixes, the allomorph \forme{mɤ-} occurs when preceded by a possessive prefix (§\ref{ex:WmApjWkWnWfkAB}) and \forme{mɯ-} is found when no possessive prefix is present: compare the elicited forms (\ref{ex:WmAkukWtshi}) and (\ref{ex:mWkukWtshi}). The allomorphs of the negative prefix are not in free variation: forms such as $\dagger$\forme{ɯ-mɯ-ku-kɯ-tsʰi} or $\dagger$\forme{mɤ-ku-kɯ-tsʰi} would be incorrect in Kamnyu Japhug.

 \begin{exe} 
\ex \label{ex:WmAkukWtshi}
\gll ɯ-mɤ-ku-kɯ-tsʰi \\
\textsc{3sg}.\textsc{poss}-\textsc{neg}-\textsc{ipfv}-\textsc{nmlz}:S/A-drink \\
\ex \label{ex:mWkukWtshi}
\gll mɯ-ku-kɯ-tsʰi \\
\textsc{neg}-\textsc{ipfv}-\textsc{nmlz}:S/A-drink \\
\glt  `The one who drinks it' (elicited)
\end{exe}

There are no constraints on the number of derivational prefixes in participial forms. The derivational prefixes are all closer to the verb root than the participle prefix \forme{kɯ-}, and thus follow it as shown by (\ref{ex:WmApjWkWnWfkAB}), where the autobenefactive \forme{-nɯ-}, the leftmost of all derivational prefixes (§ XXX), is placed after \forme{kɯ-}. 

Aside from possessive, orientation, associated motion and polarity prefixes, subject participles can also receive the conative prefix \forme{jɯ-} as in (\ref{ex:jWtukWwGrum}).

 \begin{exe} 
\ex \label{ex:jWtukWwGrum}
\gll  kɯ-pɣi ci koŋla ʑo zɯmi jɯ-tu-kɯ-wɣrum kɯ-fse ci ɲɯ-ŋu.  \\
\textsc{nmlz}:S/A-be.grey \textsc{indef} completely \textsc{emph} almost \textsc{conat}-\textsc{ipfv}-\textsc{nmlz}:S/A-be.white \textsc{nmlz}:S/A-be.like \textsc{indef} \textsc{sens}-be \\
\glt `It is grey, almost like it is about to become white.'  (24-ZmbrWpGa, 34)
\end{exe}

Subject participles can undergo totalitative reduplication (§ XXX), which applies to the first syllable of the word, whether it is the participle \forme{kɯ-} or an orientation prefix as in (\ref{ex:jWjAkWGe}), meaning `all of those who/that X'.

\begin{exe}
\ex \label{ex:jWjAkWGe}
\gll tɕe nɯnɯ ɯ-taʁ jɯ\redp{}jɤ-kɯ-ɣe nɯ ku-ndɤm ɲɯ-ŋu. \\
\textsc{lnk} \textsc{dem} \textsc{3sg}.\textsc{poss}-on \textsc{total}\redp{}\textsc{pfv}-\textsc{nmlz}:S/A-come[II] \textsc{ipfv}-take[III] \textsc{sens}-be \\
\glt `(The spider) catches all of the (insects) that have come on (the web).' (26-mYaRmtsaR, 108)
\end{exe}

The totalitative subject participle of the existential verb \japhug{tu}{exist} can take a possessive prefix, which is interpreted as a possessor, as in \forme{a-kɯ\redp{}kɯ-tu} \textsc{1sg}.\textsc{poss}-\textsc{total}\redp{}\textsc{nmlz}:S/A-exist `everything that I have'. No other totalitative verb form allows possessor prefixation.

\subsubsection{Ambiguities}  \label{sec:subject.participle.ambiguities}
The subject participle \forme{kɯ-} prefix is homophonous with the generic person marker for intransitive subject and object (§ XXX; note that these two prefixes are probably historically related). In the case of intransitive verbs, some subject participles are therefore homophonous with generic person forms. 

For instance, the past imperfective generic \forme{pɯ-kɯ-ŋu} `one used to be' in (\ref{ex:pWkWNu.genr}) is identical to the past imperfective participle \forme{pɯ-kɯ-ŋu} `the one who used to be ....', discussed above (example \ref{ex:pWkWNu} in §\ref{sec:subject.participle.other.prefixes}). In this example, it is obvious that \forme{kɯ-} is the generic person marker because the verb \forme{pɯ-kɯ-rga} `one used to be' occurs as main verb; outside of any context,  \forme{tɤ-pɤtso pɯ-kɯ-ŋu} could be understood as a relative clause `the one who used to be a child', but this is not the meaning of this sentence. 

\begin{exe}
\ex \label{ex:pWkWNu.genr}
 \gll tɕeri tɤ-pɤtso pɯ-kɯ-ŋu tɕe, nɯ kɤ-ndza wuma ʑo pɯ-kɯ-rga. \\
 \textsc{lnk} \textsc{indef}.\textsc{poss}-child \textsc{pst}.\textsc{ipfv}-\textsc{genr}:S/P-be \textsc{lnk} \textsc{dem} \textsc{inf}-eat really \textsc{emph} \textsc{pst}.\textsc{ipfv}-\textsc{genr}:S/P-like \\
 \glt `When (we) were children, (we) used to like eating it.' (12-ndZiNgri, 137-138)
\end{exe}

More generally, the factual, imperfective, past imperfective and perfective forms of intransitive verbs in generic person forms are homophonous with unmarked, imperfective, past imperfective and perfective participles respectively. In the case of transitive verbs, the subject participle can be identical to the object generic form. For instance, the participle \forme{nɯ-tu-kɯ-ndza} `the one who eats them' in (\ref{ex:tukWndza.nmlz}) only differs from the generic \forme{tu-kɯ-ndza} `it eats us/people' in \ref{ex:tukWndza.genr}) by the possessive prefix \forme{nɯ-}, and that prefix being optional, there are forms that are really ambiguous between participle and generic. 

\begin{exe}
\ex \label{ex:tukWndza.nmlz}
 \gll nɯ ɯ-rkɯ jɤ-azɣɯt-nɯ tɕe, ʑara nɯ-tu-kɯ-ndza srɯnmɯ ci pjɤ-tu, \\
 \textsc{dem} \textsc{3sg}.\textsc{poss}-side \textsc{pfv}-reach-\textsc{pl} \textsc{lnk} \textsc{3pl} \textsc{3pl}.\textsc{poss}-\textsc{ipfv}-\textsc{nmlz}:S/A-eat râkshasî \textsc{indef} \textsc{ifr}.\textsc{ipfv}-be \\
\glt `There was a râkshasî who ate those who had come near her.' (Kunbzang2012, 255)
\end{exe}

\begin{exe}
\ex \label{ex:tukWndza.genr}
 \gll tɕe ndzɤpri kɤ-ti nɯ tɕe tɯrme tu-kɯ-ndza ɲɯ-ŋgrɤl  \\
 \textsc{lnk} brown.bear \textsc{nmlz}:P-say \textsc{dem} \textsc{lnk} people \textsc{ipfv}-\textsc{genr}:S/P \textsc{sens}-be.usually.the.case \\
\glt `The brown bear, it eats people.' (21-pri, 94)
\end{exe}
 
The irregular generic \forme{tu-kɯ-ti} `one says' of the verb \japhug{ti}{say} is also identical with the participle `the one who says'.

The \textsc{2sg}\fl{}\textsc{1sg} form of transitive verbs in \forme{-a}, due to the vowel fusion rule \ipa{-a-a} \fl{} \forme{-a}, are also superficially identical to subject participles, for instance \forme{tu-kɯ-ndza-a} `you eat me' is pronounced \phonet{tukɯndza} exactly like the generic and the participle \forme{tu-kɯ-ndza} in the Kamnyu dialect (in the dialects of Japhug where this vowel fusion does not occur, the forms remain distinct).

\begin{exe}
\ex \label{ex:tukWndzaa}
 \gll nɯ kóʁmɯz nɤ tu-kɯ-ndza-a \\
 \textsc{dem} only.after \textsc{lnk} \textsc{ipfv}-2\fl{}1-eat-\textsc{1sg} \\
 \glt `Eat me only after (having taken out the thorn on my foot).' (140426 lang yisheng-zh, 16)
\end{exe} 

In the case of stative verbs and some auxiliary verbs, the infinitive has in some cases the form \forme{kɯ-}, and there is thus ambiguity between infinitive and subject participial forms for these verbs (§\ref{sec:velar.inf}).

\subsubsection{Subject relative clauses}  \label{sec:subject.participle.subject.relative}
The most common use of subject participles is to build participial relative clauses whose head noun is the subject; it is the only way to relativize the subject in Japhug (§ XXX). Headless relatives are most common (§ XXX), but when the head noun is overt, the relative can be either prenominal, postnominal or head-internal. With intransitive verbs the difference between postnominal or head-internal relatives is often difficult to ascertain, and many examples are ambiguous; for instance in (\ref{ex:tCheme.RnWz.kWrWCmi}), the relative clause could be argued to be postnominal (limited to the participle \forme{kɯ-rɯɕmi} `speaking') or head-internal (including \forme{tɕʰeme ʁnɯz} `two girls', and possibly even the previous adjunct).

\begin{exe}
\ex \label{ex:tCheme.RnWz.kWrWCmi}
 \gll  kʰa ɯ-ŋgɯ nɯtɕu tɕʰeme ʁnɯz kɯ-rɯɕmi pjɤ-tu. \\
 house \textsc{3sg}.\textsc{poss}-inside \textsc{dem}:\textsc{loc} girl two \textsc{nmlz}:S/A-speak \textsc{ifr}.\textsc{ipfv}-exist \\
\glt  `There were two girls speaking in the house.' (150909 xiaocui-zh, 157)
\end{exe}

Other examples such as (\ref{ex:kWm.WrkW.zW.pGa}) are unambiguously head-internal, since the locative adjunct \forme{kɯm ɯ-rkɯ zɯ} cannot belong to the matrix clause. This example additionally illustrates the necessity of using a subject relative clause with an existential verb to connect a noun with an postpositional phrase ($\dagger$\forme{kɯm ɯ-rkɯ zɯ pɣa} would not be a complete sentence).

\begin{exe}
\ex \label{ex:kWm.WrkW.zW.pGa}
 \gll  kumpɣa nɯnɯ tɕe [kɯm ɯ-rkɯ zɯ pɣa kɯ-tu] kɤ-ti ɲɯ-ŋu  \\
 hen \textsc{dem} \textsc{lnk} door \textsc{3sg}.\textsc{poss}-side \textsc{loc} bird \textsc{nmlz}:S/A-exist \textsc{inf}-say \textsc{sens}-be \\
 \glt `The word \japhug{kumpɣa}{hen} means `the bird that is next to the door'.' (22-kumpGa, 3). 
\end{exe}

With transitive verbs, subject head-internal relatives can be distinguished from postnominal ones  by the presence of the ergative \forme{kɯ} on the head noun (§ XXX), as in (\ref{ex:WrdWrdoR.kW.thotsi.WkWta}).

\begin{exe}
\ex \label{ex:WrdWrdoR.kW.thotsi.WkWta}
 \gll [tsuku ɯ-rdɯ\redp{}rdoʁ kɯ ʑo tʰotsi ɯ-kɯ-ta] ɣɤʑu. \\
 some \textsc{3sg}.\textsc{poss}-piece \textsc{erg} \textsc{emph} seal \textsc{3sg}.\textsc{poss}-\textsc{nmlz}:S/A-put exist:\textsc{sens} \\
 \glt `There are people who put a seal (on their bread).' (160706 thotsi, 20)
\end{exe}

Prenominal relatives are relatively rare with intransitive verbs, but commonly occur with transitive verbs, as in (\ref{ex:tWnW.WkWtshi}). Note the presence of indefinite person possessive marking on the head noun \japhug{tɤ-pɤtso}{child} in this example; unlike in Situ (\citealt{jacksonlin07}), the head noun of prenominal relatives in Japhug does not take a third person singular prefix as in a possessive construction (in which case the form $\dagger$\forme{ɯ-pɤtso} would have been found).

\begin{exe}
\ex \label{ex:tWnW.WkWtshi}
 \gll  [tɯ-nɯ ɯ-kɯ-tsʰi] tɤ-pɤtso ɣɯ ɯ-kɯ-mŋɤm ɲɯ-ŋu tɕe, \\
 \textsc{indef}.\textsc{poss}-breast \textsc{3sg}.\textsc{poss}-\textsc{nmlz}:S/A-drink \textsc{indef}.\textsc{poss}-child \textsc{gen} \textsc{3sg}.\textsc{poss}-\textsc{nmlz}:S/A-hurt \textsc{sens}-be \textsc{lnk} \\
 \glt `It is a disease of infants (who still drink mother milk).' (25-kACAl, 61)
\end{exe}

There are nevertheless prenominal genitival subject relative clauses, containing a subject participle, with the genitive \forme{ɣɯ} occurring between the relative clause and the head noun. This construction is especially common in texts translated from Chinese (due to calquing with \zh{的} <de>-relatives, § XXX), but also attested in natural speech, as in (\ref{ex:kWsAndza.GW}).

\begin{exe}
\ex \label{ex:kWsAndza.GW}
 \gll nɯnɯ kɯ-sɤ-ndza ɣɯ rɯdaʁ nɯnɯ tɕe kɯrŋi tu-kɯ-ti ŋu.  \\
 \textsc{dem} \textsc{nmlz}:S/A-\textsc{apass}-eat \textsc{gen} animal \textsc{dem} \textsc{lnk} beast \textsc{ipfv}-\textsc{genr}-say be:\textsc{fact} \\
 \glt `Animals eating (other animals) are called `beasts'.' (150822 kWrNi, 6)
\end{exe}

%qambrɯ kɯ-rɤpɯ ci a-jɤ-tɯ-ɣɯt-nɯ ra, tɕe tɯmɯ nɤmkha ɯ-kɯ-luj ɣɯ raz ci a-jɤ-tɯ-ɣɯt-nɯ ra.

When subject relative clauses contain a complement clause, the main verb of that clause can be in subject participle form (see examples \ref{ex:ndZikWsAndu} and \ref{ex:tAkWmbri.kWme} in §\ref{sec:subject.participle.complementation} and § XXX).
 
\subsubsection{Other relative clauses}  \label{sec:subject.participle.other.relative}
In addition to subject relativization, the subject participle is also used in possessor relatives, when the relativized element is the possessor of the subject (§ XXX).  The head-internal clause in (\ref{ex:kWkWtu.head.internal}) is such a possessor relative; its head noun \japhug{si}{tree}, possessor of the subject \japhug{ɯ-mat}{its fruits},  is marked with the genitive, showing that it belongs to the relative.  

\begin{exe}
\ex \label{ex:kWkWtu.head.internal}
 \gll si ɣɯ ɯ-mat kɯ\redp{}kɯ-tu nɯ ɯ-ku ri ɕ-ku-zo ɲɯ-ŋu tɕe. \\
 tree \textsc{gen} \textsc{3sg}.\textsc{poss}-fruit \textsc{total}\redp{}\textsc{nmlz}:S/A-exist \textsc{dem} \textsc{3sg}.\textsc{poss}-top \textsc{loc} \textsc{transloc}-\textsc{ipfv}-land \textsc{sens}-be \textsc{lnk} \\
 \glt `It lands on the top of all trees that have fruits.' (24-ZmbrWpGa, 43)
\end{exe}

 Headless possessor relative clauses, such as  \forme{nɯ-mtɕʰi mɤ-kɯ-pe} `those with a foul mouth' in (\ref{ex:nWmtChi.mAkWpe}), are even more common.

\begin{exe}
\ex \label{ex:nWmtChi.mAkWpe}
 \gll nɯ-mtɕʰi mɤ-kɯ-pe, kɤ-nɯtsɯ kɯ-ra ra kɯnɤ tu-kɯ-nɯ-ti nɯnɯra tɕaɣi tu-sɤrmi-nɯ ŋgrɤl. \\
 \textsc{3pl}.\textsc{poss}-mouth \textsc{neg}-\textsc{nmlz}:S/A-be.good \textsc{inf}-hide \textsc{nmlz}:S/A-have.to \textsc{pl} also \textsc{ipfv}-nmlz:S/A-\textsc{auto}-say \textsc{dem}:\textsc{pl} parrot \textsc{ipfv}-call-\textsc{pl} be.usually.the.case:\textsc{fact} \\
 \glt `Those with a foul mouth, who say things that should be hidden, are called `parrots'. (24-qro, 130)
\end{exe}

In (\ref{ex:WkWXsu.kWme}), we find a prenominal relative containing another verb in subject participle form that can be interpreted a possessor relativization as in (\ref{ex:kWkWtu.head.internal}): \forme{ɯ-kɯ-χsu kɯ-me}  `having no feeder'.

\begin{exe}
\ex \label{ex:WkWXsu.kWme}
 \gll  [ɯ-pɕi kɯ-rɤʑi] [ɯ-kɯ-χsu kɯ-me] lɯlu ɣɤʑu tɕe nɯnɯ kupa kɯ <yemao> tu-ti ŋu \\
 \textsc{3sg}.\textsc{poss}-outside \textsc{nmlz}:S/A-stay \textsc{3sg}.\textsc{poss}-\textsc{nmlz}:S/A-feed \textsc{nmzl}:S/A-not.exist cat exist:\textsc{sens} \textsc{lnk} \textsc{dem} Chinese \textsc{erg} wild.cat \textsc{ipfv}-say be:\textsc{fact} \\
 \glt `There are cats that live outside, that nobody feeds, Chinese people call them wild cats.' (21-lWlu, 2)
 \end{exe}

However, there are cases of relative clauses with the subject participle of the negative existential verb \forme{kɯ-me} and an \textit{intransitive} verb in participial (or finite) form in the preceding complement clause, for instance \forme{tɤ-kɯ-mbri} in (\ref{ex:tAkWmbri.kWme}). It is manifest that here the relativized element is neither the subject of \japhug{me}{not exist} nor a possessor, but rather the subject of the verb of the complement clause \japhug{mbri}{cry, sing, make noise}.

\begin{exe}
\ex \label{ex:tAkWmbri.kWme}
 \gll  pɣɤtɕɯ nɯ kɯnɤ [[tɯ-ɣjɤn cinɤ ʑo tɤ-kɯ-mbri] kɯ-me], nɯ to-ɣɤscɤscɤt ʑo to-mbri ɲɯ-ŋu, \\
bird \textsc{dem} also one-time even.one \textsc{emph} \textsc{pfv}-\textsc{nmlz}:S/A-make.noise \textsc{nmlz}:S/A-not.exist \textsc{dem} \textsc{ifr}-do.quickly \textsc{emph} \textsc{ifr}-make.noise \textsc{sens}-be \\
\glt `Even the bird, who had not sung even once (since coming to the palace), immediately started singing.' (2012 qachGa, 170)
 \end{exe}

The clause \forme{tɯ-ɣjɤn cinɤ ʑo tɤ-kɯ-mbri kɯ-me} here is in fact the nominalized version of the postverbal negative construction  (§ XXX). In main clauses, this construction combines a negative existential verb in impersonal (third singular) form with a complement clause in finite form. In (\ref{ex:tAkWmbri.kWme}), we see that when the intransitive subject of a postverbal negative construction is nominalized, both the matrix verb \japhug{me}{not exist} and the verb of the complement clause \forme{tɤ-kɯ-mbri} occur in subject participle form. This construction, though superficially similar to that in (\ref{ex:WkWXsu.kWme}), is therefore distinct from it.


 In the participial relative taking \forme{mɤ-kɯ-sɤ-mto} `the one that is not visible' as its main verb in (\ref{ex:mAkWsAmto.schiz}), the head \japhug{smar}{river} is not the possessor of the subject in the proper sense, but the possessor of a noun (\japhug{ɯ-βzɯr}{its side}) subject of a clause embedded within another clause (headed by the participle \forme{kɯ-fse} `that is like...') serving as the subject of \forme{mɤ-kɯ-sɤ-mto} `the one that is not visible'.  

  \begin{exe}
\ex \label{ex:mAkWsAmto.schiz}
 \gll  [maka tɕɤkɯ ku-kɯ-ru tɕe tɕɤndi smar [[ɯ-βzɯr tɕʰi kɯ-fse ŋu] kɯ-fse] mɤ-kɯ-sɤ-mto] ʑo scʰiz nɯ-azɣɯt ɲɯ-ŋu. \\
 at.all east \textsc{ipfv}:\textsc{east}-\textsc{genr}:S/P-look \textsc{lnk} west river \textsc{3sg}.\textsc{poss}-side what \textsc{nmlz}:S/A-be.like be:\textsc{fact} \textsc{nmlz}:S/A-be.like \textsc{neg}-\textsc{nmlz}:S/A-\textsc{deexp}-see \textsc{emph} \textsc{approx}.\textsc{loc} \textsc{pfv}:\textsc{west}-reach \textsc{sens}-be \\
\glt `He arrived at a river which was such that if one looked from one bank to the other side, what was on the other side was not at all visible.' (Divination 2005, 27)
 \end{exe}
 
 This particularly convoluted example is however not representative of what is usually found in the corpus.


In addition, there are also participial relative clauses in \forme{kɯ-} whose relativized element is neither the subject or the possessor of the subject. Prenominal relatives with \japhug{ɯ-stu}{place} as head noun can have a locative meaning, as in (\ref{ex:kWrAZi.Wstu}) and (\ref{ex:tAntAm.lAkWZa.Wstu}).
 
\begin{exe}
\ex \label{ex:kWrAZi.Wstu}
 \gll ɯʑo kɯ-rɤʑi ɯ-stu ʑo nɯ kú-wɣ-sɯ-ɤsɯɣ. \\
\textsc{3sg} \textsc{nmlz}:S/A-stay \textsc{3sg}.\textsc{poss}-place \textsc{emph} \textsc{dem} \textsc{ipfv}-\textsc{inv}-\textsc{caus}-be.tight \\
\glt `One presses the place (in the cow's hide) where (the bug) is.' (25-akWzgumba, 12)
\end{exe}

\begin{exe}
\ex \label{ex:tAntAm.lAkWZa.Wstu}
 \gll nɯnɯ tɯ-ɤntɤm lɤ-kɯ-ʑa ɯ-stu nɯ ɯ-mpʰɯsku tu-kɯ-ti ŋu. \\
\textsc{dem} \textsc{inf}:II-be.flat \textsc{pfv}:\textsc{upstream}-\textsc{nmlz}:S/A-start \textsc{3sg}.\textsc{poss}-place \textsc{dem} \textsc{3sg}.\textsc{poss}-rump \textsc{ipfv}-\textsc{genr}-say be:\textsc{fact} \\
\glt  (When one goes upstream), the place where (the slope on the mountain) starts to become flatter (the point of inflexion in the slope of the mountain) is called  the `rump' (of the mountain).' (150908 Wmphsku, 8)
  \end{exe}
  
  \subsubsection{Complementation strategies}  \label{sec:subject.participle.complementation}
Subject participles are also required in several types of complement clauses and complementation strategies (§ XXX on the difference between the two types).

The most common complementation construction involving subject participles occurs with the motion verbs \japhug{ɕe}{go} and \japhug{ɣi}{come}. These verbs have purposive clauses which compulsorily take a verb in \forme{kɯ-} subject participle form. This subject participle cannot take orientation, polarity and associated motion prefixes, and \citet{sun12complementation} posits for Tshobdun posits a distinct category of supine, a term which could also be appropriate for Japhug. However, given the existence of object participle purposive clauses (§\ref{sec:object.participles.complement}), I consider the supine to be only a specific use of the subject participle, rather than an independent category.

When the verb in the purposive clause is transitive, the participle has a possessive prefix coreferent with the object as in the case of relative clauses (\ref{sec:subject.participle.possessive}), and the subject can either take absolutive marking following the motion verb (which is morphologically intransitive), as in (\ref{ex:WkWCar.chACe}), or ergative  marking following the verb of the purposive clause as in (\ref{ex:WkWCar.loCenW}). This case marking difference can be analysed as reflecting distinct clausal structures: in (\ref{ex:WkWCar.loCenW}), the subject \forme{tɤ-rɟit ra} `the children' belongs to the purposive clauses, whereas in (\ref{ex:WkWCar.chACe}), the subject \forme{tɤ-mu nɯ} lies outside of it.

\begin{exe}
\ex \label{ex:WkWCar.chACe}
 \gll lo-fsoʁ tɕe tɕe tɤ-mu nɯ [ɯ-tɕɯ ɯ-kɯ-ɕar] cʰɤ-ɕe tɕe,\\
 \textsc{ifr}-be.bright \textsc{lnk} \textsc{lnk} \textsc{indef}.\textsc{poss}-mother \textsc{dem} \textsc{3sg}.\textsc{poss}-son \textsc{3sg}.\textsc{poss}-\textsc{nmlz}:S/A-search \textsc{ifr}:\textsc{downstream}-go \textsc{lnk} \\
\glt `When the sun came up (in the morning), the mother went to look for her son.' (2012tWJo, 33)
\end{exe}

\begin{exe}
\ex \label{ex:WkWCar.loCenW}
 \gll ɯ-fso-soz tɕe, [tɤ-rɟit ra kɯ nɯ ɯ-kɯ-ɕar] jo-ɕe-nɯ ɲɯ-ŋu tɕe \\
 \textsc{3sg}.\textsc{poss}-tomorrow-morning \textsc{lnk} \textsc{indef}.\textsc{poss}-child \textsc{pl} \textsc{erg} \textsc{dem} \textsc{3sg}.\textsc{poss}-\textsc{nmlz}:S/A-search \textsc{ifr}:\textsc{upstream}-go \textsc{sens}-be \textsc{lnk} \\
\glt  `The morning of the next day, the children went (there) to look for him.'  (Norbzang, 325)
\end{exe}

The goal of the motion verb can however occur within the purposive clause, as in (\ref{ex:khapa.WkWnnAjo}), where the subject in ergative form \forme{ɯ-wa nɯ kɯ} `his father' is stranded from the transitive verb \forme{ɯ-kɯ-n-nɤjo} by the goal \forme{kʰapa tɕe} `downstairs'.

\begin{exe}
\ex \label{ex:khapa.WkWnnAjo}
 \gll [ɯ-wa nɯ kɯ kʰapa tɕe ɯ-kɯ-n-nɤjo] pjɤ-ɣi.  \\
  \textsc{3sg}.\textsc{poss}-father \textsc{dem} \textsc{erg} downstairs \textsc{loc}    \textsc{3sg}.\textsc{poss}-\textsc{nmlz}:S/A-\textsc{auto}-wait \textsc{ifr}:\textsc{down}-come \\
  \glt `His father came downstairs to wait for him.' (140506 loBzi, 5)
\end{exe}

There is obligatory coreference between the subject of the motion verb and that of the purposive clause in this construction; to express coreference with the \textit{object} of the purposive clause (in the case of a transitive verb), the object participle is used instead (§ XXX, § XXX). 

Motion verb with purposive clauses have some semantic overlap with the corresponding associated motion prefixes (§\ref{sec:am.prefixes}); the functional difference between the two construction is discussed in §\ref{sec:am.vs.mvc}.

Some transitive and semi-transitive verbs take object complement clauses (§ XXX) requiring a subject participle. This group includes the verb \japhug{sɯχsɤl}{recognize, notice} as in (\ref{ex:tAkWnWCpWz.pjAsWXsAl}) and the verbs of pretence \japhug{ʑɣɤpa}{pretend} and \japhug{nɯɕpɯz}{pretend, disguise as, imitate} as in (\ref{ex:kukWtshi.tonWCpWznW}). 

\begin{exe}
\ex \label{ex:tAkWnWCpWz.pjAsWXsAl}
 \gll tɕe nɯnɯ kɯ [qaʑo tɤ-kɯ-nɯɕpɯz] nɯ pjɤ-sɯχsɤl. \\
 \textsc{lnk} \textsc{dem} \textsc{erg} sheep \textsc{pfv}-\textsc{nmlz}:S/A-disguised \textsc{dem} \textsc{ifr}-recognize \\
\glt `He (the shepherd boy) had noticed the (nobleman) disguised as a sheep.' (40513 mutong de disheng-zh, 63)
\end{exe}

\begin{exe}
\ex \label{ex:kukWtshi.tonWCpWznW}
 \gll  ʑara kɯ [cʰa nɯ ku-kɯ-tsʰi] to-nɯɕpɯz-nɯ, \\
\textsc{3pl} \textsc{erg} alcohol \textsc{dem} \textsc{ipfv}-\textsc{nmlz}:S/A-drink \textsc{ifr}-pretend-\textsc{pl} \\
\glt  `They pretended to drink the alcohol.'  (Norbzang 2012, 91)
\end{exe}

The status of the clauses with subject participles occurring with these three verbs, though superficially similar to the purposive clauses, is however entirely distinct: these clauses are note specific constructions, but simply subject relative clauses in object or semi-object position. The difference with purposive clauses can be shown by three pieces of evidence. 

First, the three verbs in question can take nouns as objects (as shown by \ref{ex:qaɕpa.tonWCpWz} and \ref{ex:tAkWnWCpWz.pjAsWXsAl}) instead of clauses with subject participles. 

\begin{exe}
\ex \label{ex:qaɕpa.tonWCpWz}
 \gll  qaɕpa to-nɯɕpɯz, qaɕpa ɯ-rqʰu to-ŋga, \\
frog \textsc{ifr}-pretend frog \textsc{3sg}.\textsc{poss}-skin \textsc{ifr}-wear \\
\glt `He disguised as a frog, he wore a frog's skin.' (2002 qaCpa, 10)
\end{exe}
 
 Second, these verbs can occur with a participial clause whose subject is overt and distinct from the subject of the verb of the matrix clause, as in (\ref{ex:tApAtso.kWGAwu.kAnWCpWz}) where  \japhug{tɤ-pɤtso}{child} is the subject of the verb \japhug{ɣɤwu}{cry} in the participial clause, but not the subject of  \japhug{nɯɕpɯz}{pretend, disguise as, imitate} (see § XXX for additional examples). Such a subject mismatch would be completely ungrammatical with a purposive clause.
 
\begin{exe}
\ex \label{ex:tApAtso.kWGAwu.kAnWCpWz}
 \gll   [tɤ-pɤtso kɯ-ɣɤwu] ʑo kɤ-nɯɕpɯz mɤ-spe-a ma nɯ mɯma spe-a \\
 \textsc{indef}.\textsc{poss}-child \textsc{nmlz}:S/A-cry \textsc{emph} \textsc{inf}-imitate \textsc{neg}-be.able[III]:\textsc{fact}-\textsc{1sg} \textsc{lnk} \textsc{dem} apart.from be.able[III]:\textsc{fact}-\textsc{1sg} \\
\glt `I cannot imitate a child crying, but apart from that I can imitate (anything).' (27-kikakCi, 143)
\end{exe}

Third, we find examples like (\ref{ex:Wmi.kWmNAm.tonWCpWznW}) where the subject of the verb in the main clause is not coreferent with the subject of the participial clause but with the possessor of the subject; these are in fact explainable as cases of possessor relative clauses (§\ref{sec:subject.participle.other.relative}).

\begin{exe}
\ex \label{ex:Wmi.kWmNAm.tonWCpWznW}
 \gll  tɕe [ɯ-mi kɯ-mŋɤm] to-nɯɕpɯz  \\
 \textsc{lnk} \textsc{3sg}.\textsc{poss}-leg \textsc{nmlz}:S/A-hurt \textsc{ifr}-pretend \\
 \glt `He pretended to have a pain in the leg.' (140426 lang yisheng-zh, 9)
\end{exe}

In some constructions, subject participial clauses can occur instead of infinitive clauses; For instance, the imperfective \forme{kɤ-} infinitive + existential verb construction expressing impossibility (§\ref{sec:inf.exist}) has a variant with imperfective subject participles, as in (\ref{ex:tukWGi.YAGAme}). 

 \begin{exe}
\ex \label{ex:tukWGi.YAGAme}
 \gll   tu-kɯ-ɣi ɲɤ-ɣɤ-me qʰe,  \\
  \textsc{ipfv}:\textsc{up}-\textsc{nmlz}:S/A-come \textsc{ifr}-\textsc{caus}-not.exist \textsc{lnk} \\
  \glt `She made it impossible for her to come out (again).' (2003-kWBRa, 97)
 \end{exe}
 
In particular, when a complement-taking verb is itself in the subject participle form, it is possible for the complement either to be in the expected form (infinitive or finite), or to be in subject participle form itself. For instance, in examples (\ref{ex:ndZikWsAndu}) and (\ref{ex:akWCWnNo})  the subject participle (in both cases \forme{kɯ-cʰa} ` the one who can') takes complements with a subject participle with a possessive prefix coreferent with the object (\forme{ndʑi-kɯ-sɤndu} and \forme{a-kɯ-ɕɯ-nŋo} respectively), instead of the expected \forme{kɤ-} infinitive (or finite clause). See also (\ref{ex:tAkWmbri.kWme}) in §\ref{sec:subject.participle.other.relative} for a similar case, without subject coreference between the matrix verb and the verb of the complement clause.

\begin{exe}
\ex \label{ex:ndZikWsAndu}
\gll  rŋɯl kɯ ndʑi-kɯ-sɤndu kɯ-cʰa kɯ-fse pɯ\redp{}pɯ-tu nɤ  \\
silver \textsc{erg} \textsc{3du-nmlz}:S/A-exchange \textsc{nmlz}:S/A-can \textsc{nmlz}:S/A-be.like 
\textsc{cond}\redp{}\textsc{pst.ipfv}-exist if \\
\glt `If there was someone who could redeem (the life of two brothers) with money, ...' (140507 jinniao-zh, 339)
\end{exe}

\begin{exe}
\ex \label{ex:akWCWnNo}
\gll  aʑo a-kɯ-ɕɯ-nŋo kɯ-cʰa me  \\
\textsc{1sg} \textsc{1sg}.\textsc{poss}-\textsc{nmlz}:S/A-\textsc{caus}-be.defeated \textsc{nmlz}:S/A-can not.exist:fact \\
 \glt `Nobody can defeat me.' (150821 edu de wangzi-zh, 5)
 \end{exe}
 
% kɯki ɯ-ku-kɯ-ndɯn kɯ-cʰa ci ŋu
However, this type of construction is potentially ambiguous, and pairs of verbs in subject participle form should not necessarily be analyzed as complement clauses embedded in relatives. In (\ref{ex:pWkWNGlWt.kWthW}), \forme{pɯ-kɯ-ɴɢlɯt} `(bone) that has been broken, fracture' is not a (subject) complement of \forme{kɯ-tʰɯ}  `the one that is serious'; rather, \forme{wuma ʑo pɯ-kɯ-ɴɢlɯt kɯ-tʰɯ} is simply a head-internal relative clause (§ XXX) with the subject participle (itself a headless relative clause) \forme{pɯ-kɯ-ɴɢlɯt} as its subject. 

\begin{exe}
\ex \label{ex:pWkWNGlWt.kWthW}
\gll  [wuma ʑo [pɯ-kɯ-ɴɢlɯt] kɯ-tʰɯ] nɯra qʰe ndɤre, tɕʰaχɕaŋ tu-te qʰe tɕe tu-xtɕɤr ŋu.  \\
really \textsc{emph} \textsc{pfv}-\textsc{nmlz}:S/A-\textsc{acaus}:break \textsc{nmlz}:S/A-be.serious \textsc{dem}:\textsc{pl} \textsc{lnk}   \textsc{lnk} splinter \textsc{ipfv}-put[III]   \textsc{lnk}  \textsc{lnk} \textsc{ipfv}-attach be:\textsc{fact} \\
\glt `The fractures that are serious, he puts on it a splinter and attaches it.' (140426 laxthab, 7)
\end{exe}

\subsubsection{Lexicalized subject participles} \label{sec:lexicalized.subject.participle}
A certain number of subject participles have developed specialized meanings and can be considered to have been lexicalized. Some of these lexicalized participles are formally identical to the regular participle (Table  \ref{tab:lexicalized.S.nmlz}, for instance the noun \japhug{kɯcʰi}{candy} in (\ref{ex:akWchi})  as compared to the non-lexicalized participle \forme{kɯ-cʰi} `the one that is sweet' in (\ref{ex:kWchi.tu}). For such nouns, lexicalization is shown by the meaning specialization and the inability to take orientation, associated motion and polarity prefixes (but not possessive prefixes, as shown by he prefix \forme{a-} on \japhug{kɯcʰi}{candy} in \ref{ex:akWchi}).

\begin{exe}
\ex \label{ex:akWchi}
 \gll aʑo a-ŋgra a-kɯcʰi ci tɤ-χti ra \\
 \textsc{1sg} \textsc{1sg}.\textsc{poss}-salary \textsc{1sg}.\textsc{poss}-candy \textsc{indef} \textsc{imp}-buy[III] have.to:\textsc{fact} \\
\glt `Give me a candy as a reward.' (140515 congming de wusui xiaohai-zh, 82)
\end{exe}

\begin{exe}
\ex \label{ex:kWchi.tu}
 \gll tɕe nɯnɯ li tú-wɣ-ndza tɕe, kɯ-cʰi tu, mɤ-kɯ-cʰi tu. \\
\textsc{lnk} \textsc{dem} again \textsc{ipfv}-\textsc{inv}-eat \textsc{lnk} \textsc{nmlz}:S/A-be.sweet exist:\textsc{fact} \textsc{neg}-\textsc{nmlz}:S/A-be.sweet exist:\textsc{fact} \\
\glt `When one eats them, some are sweet, some are not.' (08-rasti, 55)
\end{exe}

Table \ref{tab:lexicalized.S.nmlz} does not include the many names of profession / occupation built from the subject participles which are semantically transparent. We can distinguish two cases. 

First, labile verb derive participial forms such as \japhug{kɯ-lɤɣ}{shepherd} or \japhug{kɯ-mɯrkɯ}{thief} (from \japhug{lɤɣ}{graze} and \japhug{mɯrkɯ}{steal)}) without obligatory possessive prefix; the absence of these prefixes cannot be attributed to lexicalization, since these verbs can also be used intransitively (§ XXX). 

Second, plain transitive verbs have to undergo antipassive derivation (§\ref{sec:antipassive}) for their subject participles to be usable as names of profession. For instance, \japhug{kɯ-rɤ-rɤt}{writer} and \japhug{kɯ-rɤ-tʂɯβ}{tailor} are from the \forme{rɤ-} non-human antipassive forms of \japhug{rɤt}{write} and \japhug{tʂɯβ}{sew}, while  \japhug{kɯ-sɤ-sɯxɕɤt}{teacher} comes from the \forme{sɤ-} human antipassive of \japhug{sɯxɕɤt}{teach}. Without antipassive prefix, the subject participles of (non-labile) transitive verbs require either an overt object or a definite and anaphorically recoverable object, and are not appropriate as names of professions. For instance, in (\ref{ex:tArmi.WkWrAt}), the participle \forme{ɯ-kɯ-rɤt} `the one writing it' is used with \japhug{tɤ-rmi}{name} as its object.

\begin{exe}
\ex \label{ex:tArmi.WkWrAt}
 \gll  [tɤ-rmi ɯ-kɯ-rɤt] tɤ-pɤtso nɯ ɯ-rkɯ ʑo, [...] pjɤ-zɣɯt tɕe, \\
 \textsc{indef}.\textsc{poss}-name \textsc{3sg}.\textsc{poss}-\textsc{nmlz}:S/A-write  \textsc{indef}.\textsc{poss}-child \textsc{dem} \textsc{3sg}.\textsc{poss}-side \textsc{emph} { } \textsc{ifr}-reach \textsc{lnk} \\
\glt `It arrived near the boy who wrote the names (of the contestants).' (150826 shier shengxiao, 110)
\end{exe}

Moreover, I do not include among lexicalized participles cases like `shooting star' (\ref{ex:ZNgri.YWkWmArZaB}): although this expression is not compositional, the participle here is not frozen; the verb \japhug{mɤrʑaβ}{marry} can also be occur in finite forms with the noun \japhug{ʑŋgri}{star} in the meaning `appear, fall (of a shooting star)' as in (\ref{ex:ZNgri.nWmArZaB}).

\begin{exe}
\ex \label{ex:ZNgri.YWkWmArZaB}
 \gll ʑŋgri ɲɯ-kɯ-mɤrʑaβ \\
 star \textsc{ipfv}-\textsc{nmlz}:S/A-marry \\
 \glt `Shooting star' (`the wedding star')
 \ex \label{ex:ZNgri.nWmArZaB}
 \gll ʑŋgri nɯ-mɤrʑaβ ɯ-raŋ tɕe, tɯ-kɤrme cʰɯ́-wɣ-rɤɕi tɕe, cʰɯ-rɲɟi ŋu\\
 star \textsc{pfv}-marry \textsc{3sg}.\textsc{poss}-time \textsc{lnk} \textsc{genr}.\textsc{poss}-hair \textsc{ipfv}:\textsc{downstream}-\textsc{inv}-pull \textsc{lnk} \textsc{ipfv}-be.long be:\textsc{fact} \\
 \glt `When a shooting star falls,  if one pulls one's hair, it become longer.' (29-mWBZi, 101)
\end{exe}

\begin{table}[H]
\caption{Lexicalized subject participles} \label{tab:lexicalized.S.nmlz} \centering
\begin{tabular}{llll}
\lsptoprule
Noun & Base verb \\
\midrule
\japhug{kɯβʁa}{noble} & \japhug{βʁa}{prevail, win}  \\
\japhug{kɯspoʁ}{hole} & \japhug{spoʁ}{have a hole}  \\
 \japhug{kɯcʰi}{candy} & \japhug{cʰi}{be sweet} \\
 \japhug{kɯmŋɤm}{ailment} & \japhug{mŋɤm}{hurt, feel pain} \\
 \japhug{kɯŋu}{right thing} & \japhug{ŋu}{be} \\
 \japhug{kɯmaʁ}{bad thing} & \japhug{maʁ}{not be} \\
\lspbottomrule
\end{tabular}
\end{table}


In the case of  \japhug{kɯŋu}{right thing}  and  \japhug{kɯmaʁ}{bad thing}, lexicalization is very advanced, and the meaning of the noun is very distinct from the corresponding participles \japhug{kɯ-ŋu}{the one that is}  and  \japhug{kɯ-maʁ}{the one that is not} respectively. Examples such as (\ref{ex:kWNu.mAtWnAme}) and (\ref{ex:kWNu.mAtWnAme}) illustrate their use in collocation with verbs like \japhug{nɤma}{work, make} and \japhug{fse}{be like}.

\begin{exe}
\ex \label{ex:kWNu.mAtWnAme}
 \gll  mɤ-ti-a ma kɯŋu mɤ-tɯ-nɤme \\
\textsc{neg}-say:\textsc{fact}-\textsc{1sg} \textsc{lnk} right.thing \textsc{neg}-2-make[III]:\textsc{fact} \\
\glt `I won't say it, because you will not do the right thing.' (2005 Kunbzang, 397)
\end{exe}

\begin{exe}
\ex \label{ex:kWNu.mAfse}
 \gll  a-lɤ́-wɣ-ɕaβ-a tɕe tɕendɤre kɯŋu mɤ-fse \\
 \textsc{irr}-\textsc{pfv}-\textsc{inv}-catch.up-\textsc{1sg} \textsc{lnk} \textsc{lnk} right.thing \textsc{neg}-be.like:\textsc{fact} \\ 
\glt `If he catches up with me, (our enterprise) won't succeed.' (25-kAmYW-XpAltCin, 37)
\end{exe}

The participle \japhug{kɯ-maʁ}{the one that is not} has been independently grammaticalized as an identity pronoun/determined \japhug{kɯmaʁ}{other} (see §\ref{sec:other.pro} and §\ref{sec:identity.modifier}).

From the nouns \japhug{kɯŋu}{right thing}  and  \japhug{kɯmaʁ}{bad thing}, the intransitive verbs \japhug{rɯkɯŋu}{do the right thing, take good care of one's family} and \japhug{rɯkɯŋu}{do bad things, happen bad things, be clumsy} and the transitive verb \japhug{nɯkɯmaʁ}{make a mistake} have been derived by denominal derivation with \forme{rɯ-} and \forme{nɯ-} (§ XXX).

The subject participle \forme{kɯ-mpɕɤr} `the beautiful one' of the verb \japhug{mpɕɤr}{be beautiful} has a derived denominal transitive verb \japhug{nɯkɯmpɕɤr}{wear (on important occasions)} with highly derived semantics, reflecting the lexicalized use of the participle in the meaning `decoration' as in (\ref{ex:WkWmpCAr.tAkABzu}).

 \begin{exe}
\ex \label{ex:WkWmpCAr.tAkABzu}
 \gll tɕe li ɯ-kɯ-mpɕɤr kɯ-fse tɤ-kɤ-βzu ɲɯ-ŋu tɕe    \\
\textsc{lnk} again \textsc{3sg}.\textsc{poss}-\textsc{nmlz}:S/A-be.beautiful \textsc{nmlz}:S/A-be.like \textsc{pfv}-\textsc{nmlz}:P-make \textsc{sens}-be \textsc{lnk} \\
 \glt `(The seal on breads) is used for decoration.' (160706 WzbroN, 6)
\end{exe}

Several names of diseases only exist as intransitive verbs, and the disease itself or the person suffering from the disease can only be referred to by using a participial or infinitive form. In particular, the word \japhug{kɤ-kɯ-nɤndza}{leper} is the perfective subject participle of \japhug{nɤndza}{have leprosy}; this word has some degree of lexicalization (in particular, it is a common insult), but it behaves like a participle grammatically; in particular, it can undergo totalitative reduplication (§ XXX), as in (\ref{ex:kWkAkWnAndza}).

\begin{exe}
\ex \label{ex:kWkAkWnAndza}
 \gll nɯnɯ kɯ, nɯnɯtɕu kɯ\redp{}kɯ-rɤʑi nɯ to-ɣɤ-mna. to-ɣɤ-mna ɯ-qʰu tɕe tɕendɤre <quanxian> tɕe kɯ\redp{}kɤ-kɯ-nɤndza nɯ ɲɤ-ɣɤ-me \\
 \textsc{dem} \textsc{erg} \textsc{dem}:\textsc{loc} \textsc{total}\redp{}\textsc{nmlz}:S/A-stay \textsc{dem} \textsc{ifr}-\textsc{caus}-recover  \textsc{ifr}-\textsc{caus}-recover  \textsc{3sg}.\textsc{poss}-after lnk lnk all.the.district loc \textsc{total}\redp{}\textsc{pfv}-\textsc{nmlz}:S/A-have.leprosy \textsc{dem} \textsc{ifr}-\textsc{caus}-not.exist \\
\glt `He healed all those who were staying there (in the leper house). After he healed them, he had eradicated leprosy (removed all lepers) from our district.' (25-khArWm, 82)
\end{exe}

Other disease names such as \japhug{tɤkɤzbɣaʁ}{migraine} (as in \ref{ex:tAkAzbGaR}), although clearly the perfective participle or infinitive of a verb root \forme{*azbɣaʁ}, is hardly ever attested in finite form.

\begin{exe}
\ex \label{ex:tAkAzbGaR}
 \gll tɤkɤzbɣaʁ nɯ tɤ-mŋɤm qʰe, tɕe nɯ ɯ-qʰu nɤ, ŋgɯsqɤ-rʑaʁ ʑo mɯ-tu-mna \\
 migraine \textsc{dem} \textsc{pfv}-hurt \textsc{lnk} \textsc{lnk} \textsc{dem} \textsc{3sg}.\textsc{poss}-after \textsc{lnk} nine.or.ten-night \textsc{emph} \textsc{neg}-\textsc{ipfv}-recover \\
\glt `After the migraine starts, it does not recede until nine or ten days. (conversation taRrdo 2003, 9)
\end{exe}

In addition, we find nouns in \forme{kɯ-} that can be suspected to be former lexicalized participles, such as \japhug{kɯjŋu}{oath}, which appears to contain the root of the verb   \japhug{ŋu}{be}, though the segment \forme{-j-} cannot be accounted for at the present moment,\footnote{In any case, the Tangut cognate \tangut{𗡔}{4600}{ŋwụ}{1.58}  oath' shows that this derivation is very ancient and reflects a non-productive morphological process. } and \japhug{kɯmtɕʰɯ}{toy}, whose verbal root cannot be identified. The name \japhug{kɯsɤɣru}{mirror} (an archaic word in the process of being replaced by the Tibetan \japhug{χɕɤlzgoŋ}{mirror}) could also be a frozen subject participle  of the verb \japhug{ru}{look at}, but the nature of the prefix \forme{sɤɣ-} is unclear: it could be deexperiencer (§ XXX). Alternatively, \forme{sɤɣ-} could be analyzed as a frozen oblique participle (§\ref{sec:lexicalized.oblique.participle}), but in this view the prefix \forme{kɯ-} would not be identifiable.

Lexicalized subject participles appearing in compounds are also found. Several cases must be distinguished. First, we find subject participles of transitive verbs as second member of a compound, with their object as first member, as kind a lexicalized headless relative clause, like \japhug{qalekɯtsʰi}{species of kite}, which combines  \japhug{qale}{wind} and the participle \forme{ɯ-kɯ-tsʰi}  of the transitive verb \japhug{tsʰi}{block}, literally `blocking the wind'   (§\ref{sec:tatpurusha.n.n}), a designation referring to this bird's ability to apparently remain unmoving in the sky, as described in (\ref{ex:kAnWqambWmbjom.mWjCe}).

\begin{exe}
\ex \label{ex:kAnWqambWmbjom.mWjCe}
 \gll  kɤ-nɯqambɯmbjom mɯ́j-ɕe kɯ nɯnɯre ɯ-stu ri ku-rɤʑi tɕe, [...] ɯ-ʁar nɯ tu-sɤlqɤlqɤt nɤ tu-sɤlqɤlqɤt ŋgrɤl  \\
 \textsc{inf}-fly \textsc{neg}:\textsc{sens}-go \textsc{erg} there \textsc{3sg}.\textsc{poss}-place \textsc{loc} \textsc{ipfv}-stay \textsc{lnk} { } \textsc{3sg}.\textsc{poss}-wing \textsc{dem} \textsc{ipfv}-flap.slightly \textsc{lnk}  \textsc{ipfv}-flap.slightly be.usually.the.case:\textsc{fact} \\
 \glt `It does not move (flying) but remain there (in the sky) at his place, slightly flapping its wings.' (23-RmWrcWftsa, 40)
\end{exe}

A second type involves two participles in apposition, as \japhug{kɯrŋukɯɣndʑɯr}{harvestman}, build from the subject participles of \japhug{rŋu}{parch} and  \japhug{ɣndʑɯr}{grind} (§\ref{sec:appositive.n.n}). Both verbs being transitive, the absence of a possessive prefix \forme{ɯ-} is an additional clue that the form is fully lexicalized.

Third, there are compounds with the subject participle of transitive or intransitive verbs as first element (see also §\ref{sec.v.n.compounds}), for instance \japhug{kɯqurʑŋgri}{evening star} from \forme{ɯ-kɯ-qur} `the one helping him' (\japhug{qur}{help}) and \japhug{ʑŋgri}{star} literally `the star of the helper', for reasons explained in the following excerpt (\ref{ex:kWqur.ZNgri}).

\begin{exe}
\ex \label{ex:kWqur.ZNgri}
\gll ɯnɯnɯ kɯɕɯŋgɯ tɕe kɯ-qur ju-kɯ-ɕe tɕe nɯnɯ, mɯ-nɯ-ɬoʁ mɤɕtʂa nɯ tu-kɯ-nɯna mɯ-pjɤ-jɤɣ ɲɯ-ŋu tɕe,  tɕe núndʐa kɯqurʑŋgri tu-sɤrmi-nɯ \\
\textsc{dem} before \textsc{lnk} \textsc{nmlz}:S/A-help \textsc{ipfv}-\textsc{genr}:S/P-go \textsc{lnk} \textsc{dem} \textsc{neg}-\textsc{pfv}:\textsc{west}-come.out until \textsc{dem} \textsc{ipfv}-\textsc{genr}:S/P-rest \textsc{neg}-\textsc{ifr}.\textsc{ipfv}-be.possible \textsc{sens}-be \textsc{lnk} \textsc{lnk} for.this.reason evening.star \textsc{ipfv}-call-\textsc{pl} \\
\glt `Long ago, when one would go helping, one was not supposed to rest until it comes out, and for this reason it was called `star of the helper'.' (29-mWBZi, 62)
\end{exe}


Nominalizations with the \forme{x-/ɣ-} prefix (§\ref{sec:G.nmlz}) are ancient lexicalized subject participles that have undergone a syllable reduction rule (§ XXX) and have become completely separated from their base verb synchronically.

\subsection{Object participles} \label{sec:object.participle}
The object participle is a nominalized form which refers to an entity corresponding to the object (\ref{sec:absolutive.P}) or semi-object (§\ref{sec:semi.object}) of the base verb. All transitive and semi-transitive verbs (except for \japhug{kɤtɯpa}{tell}, § XXX) can build an object participle by adding the prefix \forme{kɤ-} (for instance \forme{kɤ-ndza} from the verb \japhug{ndza}{eat} in \ref{ex:kAndza}). This form is homophonous with, and historically related to the velar infinitive (§\ref{sec:velar.inf}, §\ref{sec:velar.nmlz.history}).

 \begin{exe} 
\ex \label{ex:kAndza}
\gll kɤ-ndza \\
   \textsc{nmlz}:P-eat \\
 \glt  `The one that is eaten.' (many attestations)
 \end{exe}

In the case of secundative verbs (§ XXX), the object participle can either refer to the recipient or the theme, as in (\ref{ex:nWkAmbi}); this question is discussed in more detail in §\ref{sec:object.participle.relatives}.

  \begin{exe} 
\ex \label{ex:nWkAmbi}
\gll nɯ-kɤ-mbi \\
   \textsc{pfv}-\textsc{nmlz}:P-give \\
 \glt  `The one that he has given it to.'
 \glt `The one that has been given to him.'  (many attestations)
 \end{exe}

   
 In this section, I first describe the morphological properties of object participles (compatibility with possessive prefixes §\ref{sec:object.participle.possessive} and other prefixes §\ref{sec:object.participle.other.prefixes}). Then, I discuss several cases of ambiguity between object participles and other \forme{kɤ-} prefixed forms in §\ref{sec:object.participle.ambiguity} (see also §\ref{sec:infinitives.participles}). The uses of object participles to build relative clauses and complement clauses are described in  §\ref{sec:object.participle.relatives} and §\ref{sec:object.participles.complement}. Finally, I present a few cases of lexicalized object participles in §\ref{sec:lexicalized.object.participle}.
 
\subsubsection{Possessive prefixes on object participles}  \label{sec:object.participle.possessive} 
Unlike subject participles, object participles never require a possessive prefix. An optional possessive prefix coreferent with the transitive subject, as in (\ref{ex:akAsWz}), can however be added.
  
  \begin{exe}
\ex \label{ex:akAsWz}
\gll a-kɤ-sɯz    \\
   \textsc{1sg-nmlz}:P-know \\
 \glt  `The one that I know.' (many attestations)
 \end{exe}

In the case of semi-transitive verbs, the possessive prefix is also coreferent with the subject, as in the form \forme{ɯ-kɤ-rga} `the one that he likes' in (\ref{ex:stu.WkArga}), build in the same way as the object participle of the transitive (tropative § XXX) verb \japhug{nɤmɯm}{find tasty}.

\begin{exe}
\ex \label{ex:stu.WkArga}
\gll ri nɯnɯ stu ɯ-kɤ-rga, ɯ-kɤ-nɤ-mɯm pjɤ-ɕti. \\
\textsc{lnk} \textsc{dem}  most \textsc{3sg}.\textsc{poss}-\textsc{nmlz}:P-like \textsc{3sg}.\textsc{poss}-\textsc{nmlz}:P-\textsc{trop}-be.tasty \textsc{ifr}.\textsc{ipfv}-be.\textsc{affirm} \\
\glt `But it was what he liked most, what he found most tasty.' (160703 poucet3, 74)
\end{exe}

In addition to semi-transitive verbs, the complement-taking verb \japhug{cʰa}{can} has object participles taking possessive prefixes meaning `the one that $X$ can $Y$', $X$ being the subject (marked by the possessive prefix), and $Y$ the verb in the complement clause, which can be overt or not as in (\ref{ex:nWmAkAcha}), where \forme{nɯ-mɤ-kɤ-cʰa} stands for \forme{kɤ-ndo nɯ-mɤ-kɤ-cha} `the one(s) that they are able to catch'.
 
\begin{exe}
\ex  \label{ex:nWmAkAcha}
\gll tɕe nɯ-mɤ-kɤ-cʰa nɯ kʰɯna χsɯm pɯ-tu qʰe, nɯra kɯ rcanɯ ɕlaʁ ʑo ku-ndo-nɯ ɲɯ-ɕti. \\
\textsc{lnk} \textsc{3pl}.\textsc{poss}-\textsc{neg}-\textsc{inf}-can \textsc{dem} dog three \textsc{pst}.\textsc{ipfv}-exist \textsc{lnk} \textsc{dem}:\textsc{pl} \textsc{erg} unexpectedly \textsc{ideo}.I:immediately \textsc{emph} \textsc{ipfv}-catch-\textsc{pl} \textsc{sens}-be.\textsc{affirm} \\
\glt `The (rats) that they (the people) had been unable to (catch), there were three dogs, these (dogs) caught them at once.' (150831 BZW kAnArRaR, 48)
\end{exe}

\subsubsection{Associated motion, polarity and orientation prefixes on object participles}  \label{sec:object.participle.other.prefixes}
Object participles, like subject participles, are compatible with polarity (\ref{ex:amAkAsWz}), associated motion (\ref{ex:WCWkAnAma}) and orientation prefixes (\ref{ex:WCWkAnAma}).

\begin{exe}
\ex  \label{ex:amAkAsWz}
\gll tɕe aʑo a-mɤ-kɤ-sɯz tɤjmɤɣ nɯ kɤ-ndza mɤ-naz-a \\
\textsc{lnk} \textsc{1sg} \textsc{1sg}.\textsc{poss}-\textsc{neg}-\textsc{nmlz}:P-know mushroom \textsc{dem} \textsc{inf}-eat \textsc{neg}-dare:\textsc{fact}-\textsc{1sg}  \\
\glt `I do not dare to eat the mushrooms that I do not know.' (23-mbrAZim, 113)
\end{exe}

\begin{exe}
\ex  \label{ex:WCWkAnAma}
\gll ɯ-pɕi tɕe ɯ-ɕɯ-kɤ-nɤma ci pjɤ-tu tɕe, \\
\textsc{3sg}.\textsc{poss}-outside \textsc{lnk} \textsc{3sg}.\textsc{poss}-\textsc{transloc}-\textsc{nmlz}:P-work \textsc{indef} \textsc{ifr}.\textsc{ipfv}-exist \textsc{lnk} \\
\glt  `(The mouse) had something to do outside.' (140518 mao he laoshu-zh, 88)
\end{exe}

Associated motion and polarity prefixes on object participles co-occur with possessive prefixes, as shown by  (\ref{ex:amAkAsWz}) and  (\ref{ex:WCWkAnAma})  above, but orientation prefixes (whether perfective or imperfective) do not. This is an important difference between subject and object participles (§\ref{sec:subject.participle.other.prefixes}). Object participles only have at most two prefixes.

\begin{exe}
\ex  \label{ex:pjWKAnWji}
\gll tɕe pɤjka wuma nɯnɯ tɕe, pjɯ-kɤ-nɯ-ji ŋu tɕe, \\
\textsc{lnk} gourd really \textsc{dem} \textsc{lnk} \textsc{ipfv}-\textsc{nmlz}:P-\textsc{auto}-plant be:\textsc{fact} \textsc{lnk} \\
\glt `The gourd proper is cultivated (it does not grow on its own).' (16-CWrNgo, 63)
\end{exe}

Finite relative clauses, instead of object participles, can be used to specify both TAM and the subject (§ XXX).

Unlike subject participle, object participles are attested with the progressive \forme{asɯ-} prefix, as in (\ref{ex:pWkASWndo}). It is the only non-finite form compatible with this prefix.  

\begin{exe}
\ex  \label{ex:pWkASWndo}
\gll  tɕʰeme nɯ kɯ iɕqʰa, ɯ-jaʁ <meihua>, mɯntoʁ pɯ-kɤ-ɤsɯ-ndo nɯ pjɤ-ɣɤrɤt.  \\
girl \textsc{dem} \textsc{erg} the.aforementioned \textsc{3sg}.\textsc{poss}-hand plum.blossom flower \textsc{pst}.\textsc{ipfv}-\textsc{nmlz}:P-\textsc{prog}-take \textsc{dem} \textsc{ifr}-throw \\
\glt `The girl threw down the plum blossom, the flower that she was holding in her hand.' (150907 yingning-zh, 30)
\end{exe}

These forms are rare and difficult to identify, as they are always ambiguous with object participles or infinitive of causativized verbs. In the case of (\ref{ex:pWkASWndo}), the context makes it clear that interpretation as the participle of a progressive form is the only possible, as the same verb with the progressive appears a few sentences before in (\ref{ex:meihua.ci.pjAkAsWndoci}).

\begin{exe}
\ex  \label{ex:meihua.ci.pjAkAsWndoci}
\gll   ɯ-jaʁ nɯtɕu, iɕqʰa, <meihua> ci pjɤ-k-ɤsɯ-ndo-ci, \\
\textsc{3sg}.\textsc{poss}-hand \textsc{dem}:\textsc{loc} \textsc{filler} plum.blossom \textsc{indef} \textsc{ifr}.\textsc{ipfv}-\textsc{evd}-\textsc{prog}-take-\textsc{evd} \\
 \glt `She was holding a plum blossom in her hand.' (150907 yingning-zh, 20)
\end{exe}

\subsubsection{Ambiguity} \label{sec:object.participle.ambiguity}
There is rampant ambiguity between object participles, \forme{kɤ-} infinitives and subject participles of passive verbs.  The question of the ambiguity between object participles and \forme{kɤ-} infinitives is discussed in §\ref{sec:velar.inf.ambiguity}.

The passive \forme{a-} merges with the subject participle as \ipa{kɤ}, homophonous with the infinitive and the object participle. Potentially ambiguous examples are very common. For instance, in (\ref{ex:kArku.passive}), the form \ipa{kɤrku} could be argued to be an object participle \forme{kɤ-rku} or a passive subject participle \forme{kɯ-ɤ-rku}; the second option is chosen here due to the semantics, which fits the passive \japhug{arku}{be put in, be located in} better (as this passive verb is in the process of becoming a locative existential verb, § XXX). In the absence of any argument in favour of the passive analysis, the ambiguous \ipa{kɤ-} forms are analyzed as object participles by default.

\begin{exe}
\ex \label{ex:kArku.passive}
 \gll  sɤtɕʰa ɯ-ŋgɯ kɯ-ɤ-rku <yangyu> cʰo lɤpɯɣ nɯra tu-ndze ŋgrɤl. \\
 earth \textsc{3sg}.\textsc{poss}-inside \textsc{nmlz}:S/A-\textsc{pass}-put.in potato \textsc{comit} radish \textsc{dem}:\textsc{pl} \textsc{ipfv}-eat[III] be.usually.the.case:\textsc{fact} \\
 \glt `It eats the radish and the potatoes that are in the ground.' (25-akWzgumba, 22)
\end{exe}

Due to the fact that passive verbs in Japhug are barely attested in perfective forms (§ XXX), participles with perfective prefixes can be considered to be object participles, especially in cases like (\ref{ex:YAXtAr.nWkAXtAr}), where the participle \forme{nɯ-kɤ-χtɤr}  `(those) that have been scattered' occurs in a sentence following the transitive form \forme{ɲɤ-χtɤr} `it scattered, it smashed'

\begin{exe}
\ex \label{ex:YAXtAr.nWkAXtAr}
 \gll    to-ɣi tɕe nɯ-ʑmbrɯ ɲɤ-χtɤr ʑo ɲɯ-ŋu tɕe, ɯ-zda ra nɯ-pʰe, nɤki,  ``nɯnɯ ʑmbrɯ nɯ-kɤ-χtɤr nɯ ɯ-taʁ kɤ-ɴqoʁ-nɯ ra'' to-ti tɕe,  \\
 \textsc{ifr}:\textsc{up}-come \textsc{lnk} \textsc{3pl}.\textsc{poss}-ship \textsc{ifr}-scatter \textsc{emph} \textsc{sens}-be \textsc{lnk} \textsc{3sg}.\textsc{poss}-companion pl \textsc{3pl}.\textsc{poss}-\textsc{dat} \textsc{filler} dem ship \textsc{pfv}-\textsc{nmlz}:P-scatter \textsc{dem} \textsc{3sg}.\textsc{poss}-on \textsc{imp}-hang-\textsc{pl} have.to:\textsc{fact} \textsc{ifr}-say \textsc{lnk} \\
 \glt `The (monster) came up and smashed their ship, and (Norbzang) said to his companions: ``Grab the (pieces of the) ship that have been scattered''.' (2012 Norbzang, 31-32)
\end{exe}

The same analysis as object participles, rather than passive subject participles is applied to examples of perfective \forme{kɤ-} forms also when the transitive verb is not found in finite form in a neighbouring sentence, such as (\ref{ex:pWkAprAt}).

\begin{exe}
\ex \label{ex:pWkAprAt}
 \gll fsapaʁ ɯ-ŋgo rcanɯ, pɯ-kɤ-prɤt ʑo tɤ-fse ɲɯ-ŋu. \\
 animals \textsc{3sg}.\textsc{poss}-disease unexpectedly \textsc{pfv}-\textsc{nmlz}:P-break \textsc{emph} \textsc{pfv}-be.like \textsc{sens}-be \\
 \glt  `It was like the disease of the cattle had been (suddenly) stopped.' (2003 kAndZislama, 190)
\end{exe}

A more marginal case of homophony occurs between object participles  and velar infinitives on the one hand, and several finite forms taking the series A orientation prefix \forme{kɤ-} on the other hand (§\ref{sec:infinitives.participles}).

\subsubsection{Object relative clauses} \label{sec:object.participle.relatives}
Object participles can be used to build object relative clauses, but compete in this function with finite relatives (§ XXX). They differ in this regard from subject relatives, which are the only available construction to relativize transitive and intransitive subjects.

In object participial relatives, when the relativized element is overt, it is generally located before the participle, as in (\ref{ex:thWkAraGdWt}). Prenominal object participial relatives can be elicited, but are rarer in the corpus (\ref{ex:akAsWz.Cku} is such an example; see however the discussion about genitival relatives and example \ref{ex:tAkAsWBzu.GW.tWxtsa} below).

\begin{exe}
\ex \label{ex:thWkAraGdWt}
\gll  nɯŋa ɯ-ndʐi tʰɯ-kɤ-rɤɣdɯt, tʰɯ-kɤ-tʂɯβ nɯ ɯ-ŋgɯ nɯtɕu ko-ɕe  \\
cow \textsc{3sg.poss}-skin \textsc{pfv-nmlz:P}-skin \textsc{pfv-nmlz:P}-sew \textsc{dem} \textsc{3sg.poss}-inside \textsc{dem}:\textsc{loc} \textsc{evd:east}-go \\
\glt  `He went into the cow hide that had been skinned and sewed.'    (02-deluge2012, 32)
\end{exe}  

\begin{exe}
\ex \label{ex:akAsWz.Cku}
\gll   tɕe [aʑo a-kɤ-sɯz] ɕku nɯ nɯra ŋu \\
\textsc{lnk} \textsc{1sg} \textsc{1sg}.\textsc{poss}-\textsc{nmlz}:P-know allium dem \textsc{dem}:\textsc{pl} be:\textsc{fact} \\
\glt `These are the (plants belonging to the gender) \textit{allium} that I know about.' (07-Cku, 165)
\end{exe}  

When the relative only consists of the relativized element followed by the object participle as in (\ref{ex:thWkAraGdWt}), it is not possible to determine whether the relative is postnominal or hea-internal (§ XXX). In the case of longer relatives, in particular containing locative or instrumental adjuncts, it is possible to discriminate between the two types using the relative position of the relativized element and the adjunct. In (\ref{ex:qaR.thWkAsWBzu}) for instance, the head noun \japhug{qaʁ}{hoe} occurs between the instrumental adjunct \forme{qaʁ kɯ}  and the participle: this is an uncontroversial example of head-internal relative (§ XXX). There are no clear examples of postnominal object participial relatives in the corpus.

\begin{exe}
\ex \label{ex:qaR.thWkAsWBzu}
\gll   [rŋɯl kɯ qaʁ tʰɯ-kɤ-sɯ-βzu] nɯra ko-sɯ-ɤʑirja-nɯ. \\
silver \textsc{erg} hoe \textsc{pfv}-\textsc{nmlz}:P-\textsc{caus}-make \textsc{dem}:\textsc{pl} \textsc{ifr}-\textsc{caus}-be.align-\textsc{pl} \\
\glt `They aligned the hoe that had been made from silver.' (28-qAjdoskAt, 102)
\end{exe}  

As was described in §\ref{sec:object.participle.other.prefixes}, object participles, unlike subject and oblique participles, cannot combine possessive and orientation prefixes.

Object participles with orientation prefixes are used in relative clauses with indefinite subjects, or with definite third person subjects as in ( \ref{ex:qajGi.nWkAmbi}).  
 
\begin{exe}
\ex \label{ex:qajGi.nWkAmbi}
\gll     [ɬamu kɯ qajɣi nɯ-kɤ-mbi] nɯ tu-ndze pjɤ-ŋu \\
p.n. \textsc{erg} bread \textsc{pfv}-\textsc{nmlz}:S/A-give \textsc{dem} \textsc{ipfv}-eat[III] \textsc{ifr}.\textsc{ipfv}-be \\
\glt `(As) he was eating the (pieces of) bread that Lhamo had given him.' (2002 qajdoskAt, 111)
\end{exe}  

The only example of first or second person that could be interpreted as subject in a perfective object participle relative in the corpus is (\ref{ex:aZo.kW.pWkAsWrAt}), but in this example (translated from Chinese), the referent of the first person is a pen that has been used to write a poem; the ergative postpositional phrase \forme{aʑo kɯ} here can be either analyzed as an instrument (`the poem that has been written using me') or as a causee  (§\ref{sec:causee.kW}, `the poem that he has made me write'), as shown by the presence of the causative \forme{sɯ-} prefix, not a subject.

\begin{exe}
\ex \label{ex:aZo.kW.pWkAsWrAt}
\gll   [aʑo kɯ pɯ-kɤ-sɯ-rɤt] nɯnɯ pjɯ-ndɯn ɲɯ-ŋu nétɕi \\
\textsc{1sg} \textsc{erg} \textsc{pfv}-\textsc{nmlz}:P-\textsc{caus}-write \textsc{dem} \textsc{ipfv}-read \textsc{sens}-be \textsc{sfp} \\
\glt (The pen said: the poet is reading the poem) that has been written using me.' (150818 bi he moshuihu-zh, 143)
\end{exe}  

When the subject is first or second person, an object participle with a possessive prefix  is used instead. In (\ref{ex:iZo.jikArku}) for instance, we find \forme{ji-kɤ-rku} `the thing that we give' (see §\ref{sec:z.nmlz} concerning the meaning of this verb) and \forme{nɤ-kɤ-sɯso} `the thing that you think / that you want' with a first plural and a second singular subject respectively.

\begin{exe}
\ex \label{ex:iZo.jikArku}
\gll   nɯ ma iʑo ji-kɤ-rku me,  atu spɣi tɤ-ɕe qʰe, laχtɕʰa ŋotɕu nɤ-kɤ-sɯso ʑo nɯnɯ, 
nɤ-mɲaʁ, nɤ-rna, nɤ-ɕna cʰo ra kɯ\redp{}kɯ-spoʁ nɯ ɯ-ŋgɯ tɕe a-kɤ-tɯ-rke qʰe, \\
\textsc{dem} apart.from \textsc{1pl} \textsc{1pl}.\textsc{poss}-\textsc{nmlz}:P-put.in not.exist:\textsc{fact} up.there granary imp:up-go lnk thing where \textsc{2sg}.\textsc{poss}-\textsc{nmlz}:P-think \textsc{emph} \textsc{dem} \textsc{2sg}.\textsc{poss}-eye  \textsc{2sg}.\textsc{poss}-ear  \textsc{2sg}.\textsc{poss}-nose \textsc{comit} \textsc{pl}  \textsc{total}\redp{}\textsc{nmlz}:S/A-have.a.hole \textsc{dem} \textsc{3sg}.\textsc{poss}-inside \textsc{loc} \textsc{irr}-\textsc{pfv}-2-put.in[III] \textsc{lnk} \\
\glt `We don't have anything else to give you as a departing present, go up there in the granary, and whatever you want, put it in all the holes (in your body), you eyes, you ears, you nose etc.' (31-deluge, 136)
\end{exe}  

When the subject is a definite third person, it is also possible to have a third person possessive prefix on the object participle, as in (\ref{ex:WkWnWmbrApW}) (or \ref{ex:WCWkAnAma} above).

\begin{exe}
\ex \label{ex:WkWnWmbrApW}
\gll  lɤ-fsoʁ ɯ-jɯja nɯ pjɯ-ru tɕe [ɯ-kɤ-nɯmbrɤpɯ] nɯ kʰu pɯ-ɕti ɲɯ-ŋu,  \\
\textsc{pfv}-be.clear    \textsc{3sg}-along  \textsc{dem} \textsc{ipfv:down}-look \textsc{lnk} \textsc{3sg-nmlz:P}-ride \textsc{dem} tiger \textsc{pst.ipfv}-be.\textsc{affirm}  \textsc{sens}-be \\
\glt `As the day broke, looking down, he (progressively realized that) what he was riding was a tiger.' (2005 khu, 20)
\end{exe}

Unlike in Tshobdun (\citealt[10]{jacksonlin07}), in Japhug object participial relatives with possessive prefixes are not restricted to generic state of affairs, but can refer to particular situations as in examples such as (\ref{ex:WkWnWmbrApW}) and (\ref{ex:nAkAti.nWra}).

\begin{exe}
\ex \label{ex:nAkAti.nWra}
\gll a-ɬaʁ, tɕe nɤ-kɯ-mŋɤm tɕʰi ɲɯ-fse ma [alo qʰaqʰu nɤ-kɤ-ti] nɯra tɤ-stu-t-a \\
\textsc{1sg}.\textsc{poss}-aunt \textsc{lnk} \textsc{2sg}.\textsc{poss}-\textsc{nmlz}:S/A-hurt what \textsc{sens}-be.like \textsc{lnk} upstream behind.the.house \textsc{2sg}.\textsc{poss}-\textsc{nmlz}:P-say \textsc{dem}:\textsc{pl} \textsc{pfv}-do.like-\textsc{pst}:\textsc{tr}-\textsc{1sg} \\
\glt `Stepmother, how do you feel, I did the things you said (about creating a lake) up there behind the house.' (28-smAnmi, 361)
\end{exe}

As mentioned above (example \ref{ex:nWkAmbi}), the object participles of secundative verbs can either refer to their object proper (the recipient, § XXX) or to the theme, which is not indexed on the verb but occurs in absolutive form (§\ref{sec:theme.ditransitive}). In fact, in the corpus examples of theme relativization with the object participle are quite common (as \ref{ex:qajGi.nWkAmbi} above and \ref{ex:WkAmbi.maNe} and \ref{ex:pWkAsWxCAt.nWra} below), but recipient relativization is quite rare (\ref{ex:xCiri.nWkAsWxCAt}). Examples can however be elicited without difficulty.

\begin{exe}
\ex \label{ex:WkAmbi.maNe}
\gll nɯ ma ɯ-kɤ-mbi maŋe tɕe, ``a-me ta-mbi ra" to-ti tɕe, \\
\textsc{dem} apart.from \textsc{3sg}.\textsc{poss}-\textsc{nmlz}:P-give not.exist:\textsc{sens} \textsc{lnk} \textsc{1sg}.\textsc{poss}-daughter 1\fl{}2-give:\textsc{fact}  have.to:\textsc{fact} \textsc{ifr}-say \textsc{lnk} \\
\glt `He had nothing else to give him, and said `I give you my daughter'.' (2011-04-smanmi, 171)
\end{exe}

\begin{exe}
\ex \label{ex:pWkAsWxCAt.nWra}
\gll   tɕendɤre [tɯmɯkɤrŋi kɯ pɯ-kɤ-sɯxɕɤt] ra ɲɤ-nɤxtʂɯn tɕe tɕe nɯɕɯmɯma ʑo pjɤ-nɯ-ɕe. \\
\textsc{lnk} heaven \textsc{erg} \textsc{pfv}-\textsc{nmlz}:P-teach \textsc{pl} \textsc{ifr}-be.grateful \textsc{lnk} \textsc{lnk} immediately \textsc{emph} \textsc{ifr}:\textsc{down}-\textsc{vert}-go \\
\glt `(Pu'an) was thankful for the things that the god of heaven had taught him and went back (to earth) immediately.' (150827 taisui-zh, 135)
\end{exe}

\begin{exe}
\ex \label{ex:xCiri.nWkAsWxCAt}
\gll    iɕqʰa, [kɤntɕʰɯ-xpa ʑo, nɤki, xɕiri nɯ-kɤ-sɯxɕɤt] nɯ pjɤ-sat, \\
 the.aforementioned several-year \textsc{emph} \textsc{filler} weasel \textsc{pfv}-\textsc{nmlz}:P-teach \textsc{dem} \textsc{ifr}-kill \\
\glt `He killed the weasel that he had trained for several years.'  (140518 xuezhe he huangshulang-zh, 28)
\end{exe}

With indirective verbs, the object participle can only refer to the theme, as in (\ref{ex:nAkAthu.WGAZu}), for these verbs the recipient must be relativized with the oblique participle  (§\ref{sec:other.oblique.participle.relatives}).

\begin{exe}
\ex \label{ex:nAkAthu.WGAZu}
\gll nɤ-kɤ-tʰu ɯ-ɣɤʑu nɤ, tɤ-tʰe jɤɣ \\
\textsc{2sg}.\textsc{poss}-\textsc{nmlz}:P-ask \textsc{qu}-exist:\textsc{sens} \textsc{lnk} \textsc{imp}-ask[III] be.allowed:\textsc{fact} \\
\glt `If you have and questions, you can ask them.' (conversation 14-11-08)
\end{exe}

The semi-object of semi-transitive verbs (§\ref{sec:semi.object}) can also be relativized with a object participial relative, as \forme{ji-kɤ-rga} `the one that we like' in (\ref{ex:stu.jikArga}).

\begin{exe}
\ex \label{ex:stu.jikArga}
\gll  iɕqʰa <macha> kɤ-ti nɯ [iʑora stu ji-kɤ-rga] ŋu \\
the.aforementioned macha.tea \textsc{nmlz}:P-say \textsc{dem} \textsc{1pl} most \textsc{1pl}.\textsc{poss}-\textsc{nmlz}:P-like be:\textsc{fact} \\
\glt `The (type of tea) called `macha' is what we like most.' (30-macha, 1)
\end{exe}

Secundative verbs undergoing antipassivization become semi-transitive verbs (§ XXX) with the theme remaining the semi-object. Like other semi-transitive verbs, these antipassive verbs can build an object participle, which can then be used to relativize the theme, as in (\ref{ex:nAkArAmbi}).

\begin{exe}
\ex \label{ex:nAkArAmbi}
\gll  nɤ-kɤ-rɤ-mbi nɯ tɕʰi pɯ-ŋu? \\
\textsc{2sg}.\textsc{poss}-\textsc{nmlz}:P-\textsc{apass}-give \textsc{dem} what \textsc{pst}.\textsc{ipfv}-be \\
\glt `What was it that you gave (to people)?' (elicited)
\end{exe}

Object participles also occur in genitival relatives (postnominal relative with the genitive postposition \forme{ɣɯ} occurring between the relative clause and the head noun, built like adnominal complement clauses, § XXX), as in example (\ref{ex:tAkAsWBzu.GW.tWxtsa}). This type of examples is frequently found in texts translated from Chinese, but unattested in the rest of the corpus for object relativization, and is a clear case of calque (§ XXX); although speakers do not find these examples to be ungrammatical, they cannot be considered to representative of the normal grammar of the language.

\begin{exe}
\ex \label{ex:tAkAsWBzu.GW.tWxtsa}
\gll  tɕe [<shuijing> kɯ tɤ-kɤ-sɯ-βzu] ɣɯ tɯ-xtsa nɯra jo-ɣɯt. \\
\textsc{lnk} crystal \textsc{erg} \textsc{pfv}-\textsc{nmlz}:P-\textsc{caus}-make \textsc{gen} \textsc{indef}.\textsc{poss}-shoe \textsc{dem}:\textsc{pl} \textsc{ifr}-bring \\
\glt `(The bird) brought shoes made of crystal.' (140504 huiguniang-zh, 162)
\end{exe}

\subsubsection{Other relative clauses}  \label{sec:object.participle.other.relative}
There is one case where an object participle to relativize a locative adjunct. The object participle of the perception verbs \japhug{mto}{see} and \japhug{mtsʰɤm}{hear} can be used to make headless locative relative clauses meaning `(a place) where $X$ can see/hear $Y$, in particular when occurring as the goal of a motion verb as in (\ref{ex:tCirna.mAkAmtshAm}) or (\ref{ex:WmAkAmto}). Note the optionality of the ergative on the nouns \japhug{tɕi-rna}{our ears} and \japhug{tɕi-mɲaʁ}{our eyes} in (\ref{ex:tCirna.mAkAmtshAm}). 

\begin{exe}
\ex \label{ex:tCirna.mAkAmtshAm}
\gll [tɕi-rna mɤ-kɤ-mtsʰɤm], [tɕi-mɲaʁ mɤ-kɤ-mto] a-jɤ-ɕe-ndʑi ra \\
\textsc{1du}.\textsc{poss}-ear \textsc{neg}-\textsc{nmlz}:P-hear \textsc{1du}.\textsc{poss}-eye \textsc{neg}-\textsc{nmlz}:P-see \textsc{irr}-\textsc{pfv}-go-\textsc{du} have.to:\textsc{fact} \\
\glt  `May they go away (to a place) where our ears cannot hear them, where our eyes cannot see them.' (2003-kWBRa, 23)
\end{exe}

\begin{exe}
\ex \label{ex:WmAkAmto}
\gll [iɕqʰa qaɕpa kɯ ɯ-mɤ-kɤ-mto] ʑo jo-ɕe  \\
the.aforementioned frog \textsc{erg} \textsc{3sg}.\textsc{poss}-\textsc{neg}-\textsc{nmlz}:P-see \textsc{emph} \textsc{ifr}-go   \\
\glt `She went to (a place) where the frog would not find her.' (150818 muzhi guniang-zh, 149)
\end{exe}

It is not possible in (\ref{ex:tCirna.mAkAmtshAm}) or (\ref{ex:WmAkAmto}) to replace the object participle by an oblique participle \forme{sɤ-}.

\subsubsection{Complement strategies} \label{sec:object.participles.complement}
While \forme{kɤ-} prefixed non-finite verb forms are very common in complement clauses, the near-totality of these forms are infinitives rather than object participles (§\ref{sec:inf.complementation}), since there are no restrictions on intransitive verbs (§\ref{sec:velar.inf}).

The only complement strategy where an object participle, rather than an infinitive, has to be posited occurs in the purposive clause of  motion verbs when the verb of the purposive clause is transitive and coreference occurs between its object (rather than subject) and the subject of the matrix motion verb, as \forme{kɤ-nɤkʰu} in (\ref{ex:kAnAkhu.juGi}).

\begin{exe}
\ex \label{ex:kAnAkhu.juGi}
\gll <xingqi> raŋri ʑo tɕe nɯnɯ sɤβʑɯ ɣɯ ɯ-kʰa nɯtɕu kɤ-nɤkʰu ju-ɣi pjɤ-ŋu  \\
week each \textsc{emph} \textsc{lnk} \textsc{dem} mouse \textsc{gen} \textsc{3sg.poss}-house \textsc{dem:loc} \textsc{nmlz:P}-invite \textsc{ipfv}-come \textsc{ipfv.ifr}-be \\
\glt `He would come to the mouse's house as a guest.' (150818 muzhi guniang-zh, 299).
\end{exe}

In purposive clauses, the rule is thus that the subject participle is used when there is subject-subject coreference (§\ref{sec:subject.participle.complementation}), and the object participle in cases of object-subject coreference  (\citealt[248]{jacques16complementation}).
 
 Coreference between the P of the complement clause and the S of the matrix clause is possible only if the P has control over the action, something that is possible for only a few transitive verbs such as \japhug{nɤkʰu}{invite} (since the guest has the choice of accepting or refusing the invitation), explaining the rarity of this construction. 
 
 A possessive prefix coreferent with the transitive subject of \forme{nɤkʰu} can be optionally added on this object participle, as in (\ref{ex:akAnAkhu}).

\begin{exe}
\ex \label{ex:akAnAkhu}
\gll a-kɤ-nɤkʰu jɤ-ɣe  \\
 \textsc{1sg.poss-nmlz:P}-invite \textsc{pfv}-come[II] \\
\glt `He came to my house as a guest (following my invitation).' (elicited)
\end{exe}
 
\subsubsection{Lexicalized object participles} \label{sec:lexicalized.object.participle}
While some object participles are commonly used as headless relative clauses, few can be considered to be fully lexicalized. 

The verbs related to food ingestion such as \japhug{ndza}{eat}, \japhug{tsʰi}{drink}, \japhug{ndzɤtsʰi}{eat and drink}, \japhug{moʁ}{eat powdery food} have object participles such as \japhug{kɤ-ndza}{food}, \japhug{kɤ-tsʰi}{drink (n), beverage}, \japhug{kɤ-ndzɤtsʰi}{food and drink} and \japhug{kɤmoʁ}{dry tsampa}, which commonly occur in enumerations (§\ref{sec:noun.enumeration}) with nouns not derived from verbs, as in (\ref{ex:WkAndza.WkAtshi}).  

\begin{exe}
\ex \label{ex:WkAndza.WkAtshi}
\gll  ɯ-kɤ-ndza ɯ-kɤ-tsʰi ɯ-tɯkrimgo ra to-ɣɯt qʰe, tɕendɤre, nɯra ɲɤ́-wɣ-mbi qʰe, \\
\textsc{3sg}.\textsc{poss}-\textsc{nmlz}:P-eat \textsc{3sg}.\textsc{poss}-\textsc{nmlz}:P-drink \textsc{3sg}.\textsc{poss}-butter.bread \textsc{pl} \textsc{ifr}:\textsc{up}-bring \textsc{lnk} \textsc{lnk} \textsc{dem}:\textsc{pl} \textsc{ifr}-\textsc{inv}-give \textsc{lnk} \\
 \glt  `She brought food, drinks and butter bread for her and gave them to her.' (2003-kWBRa, 70)
\end{exe}

In these enumerations, sometimes only the first element takes a possessive prefix, as in (\ref{ex:ndZikAndza.kAtshi}), where we find \forme{ndʑi-kɤ-ndza kɤ-tsʰi} instead of the equally possible \forme{ndʑi-kɤ-ndza ndʑi-kɤ-tsʰi} (however, if the first \forme{kɤ-} participle in the enumeration has no possessive prefix, the following participle cannot take one).

\begin{exe}
\ex \label{ex:ndZikAndza.kAtshi}
\gll  ndʑi-kɤ-ndza kɤ-tsʰi mɤ-mbrɤt, ndʑi-kɯ-ndzɤtsʰi a-pɯ-me smɯlɤm \\
\textsc{2du}.\textsc{poss}-\textsc{nmlz}:P-eat \textsc{nmlz}:P-drink \textsc{neg}-\textsc{acaus}:cut  \textsc{2du}.\textsc{poss}-\textsc{nmlz}:S/A-eat.and.drink \textsc{irr}-\textsc{ipfv}-not.exist prayer \\
\glt `May you never lack food or drink, may there nobody (coming to) eat you.' (2003kAndZWslama, 218)
\end{exe}

In these examples, the possessive prefix always refer to the person or animal ingesting the food (not the person giving the food), and although these forms are very common, since their semantics is completely predictable from the base verb, and since the possessive prefix behaves like that of a normal oblique participle, there is no specific reason to consider that they have become nouns and constitute lexical entries distinct from the verb (except in the case of \japhug{kɤmoʁ}{dry tsampa}, whose meaning has become more specific).

The forms \forme{kɤ-pa} and \forme{kɤ-stu} from the verbs from the verbs \japhug{pa}{do} and \japhug{stu}{do like} respectively, both meaning `manner, method (to solve a problem)' (like Chinese \ch{办法}{bànfǎ}{method}) are other potential candidates to be analyzed as lexicalized object participles (or infinitives).  They are particularly commonly used with existential verbs to mean `$X$ has (no/a) way to do it' ($X$ being referred to by the possessive prefix on \forme{kɤ-pa} or \forme{kɤ-stu}), as in (\ref{ex:akApa.maNe}).

\begin{exe}
\ex \label{ex:akApa.maNe}
\gll a-kɤpa maŋe \\
\textsc{1sg}.\textsc{poss}-method not.exist:\textsc{sens} \\
\glt `I have no way to do it.' (many attestations)
\end{exe}
 
However, collocation in texts of \forme{kɤ-pa} and \forme{kɤ-stu} with the finite forms of the verbs \japhug{pa}{do} and \japhug{stu}{do like}, as in (\ref{ex:nAkApa.tWtu}) and, suggest that these forms are still synchronically linked with these verbs, and that it may be more economical to analyze them as participles rather than derived nouns.
 
 \begin{exe}
\ex \label{ex:nAkApa.tWtu}
\gll  nɤ-kɤ-pa tɯ\redp{}tu nɤ,  tɤ-pe ma mtsʰoʁlaŋ tɤ-ɣe  \\
\textsc{2sg}.\textsc{poss}-\textsc{nmlz}:P-do \textsc{cond}\redp{}exist:\textsc{fact} \textsc{lnk} \textsc{imp}-do[III] \textsc{lnk} water.monster \textsc{pfv}:\textsc{up}-come[II] \\
\glt `If you have some way (to protect us), use it, because the water monster has come.' (Norbzang 2012, 27-28)
 \end{exe}
 
  \begin{exe}
\ex \label{ex:nAkAstu.WGAZu}
\gll   nɯnɯ ɯ-taʁ nɯtɕu nɤ-kɤ-stu ɯ-ɣɤʑu tɕe a-tɤ-tɯ-ste ma tɕe \\
\textsc{dem} \textsc{3sg}.\textsc{poss}-on \textsc{dem}:\textsc{loc} \textsc{2sg}.\textsc{poss}-\textsc{nmlz}:P-do.like \textsc{qu}-exist:\textsc{sens} \textsc{lnk} \textsc{irr}-\textsc{pfv}-2-do.like[III] \textsc{lnk} \textsc{lnk} \\
\glt `If you have a way to deal with him, use it.'   (25-kAmYW-XpAltCin, 37)
 \end{exe}
 
The object participle \forme{kɤ-ti} from the verb \japhug{ti}{say}, although transparently derived, has an unpredictable meaning in the existential construction. With a negative existential verb, in addition to the expected meaning `have nothing to say', it can be interpreted as `be unable to say for sure', as in (\ref{ex:akAti.maNe}).

  \begin{exe}
\ex \label{ex:akAti.maNe}
\gll    a-kɤ-ti ci maŋe \\
 \textsc{1sg}.\textsc{poss}-\textsc{nmlz}:P-say \textsc{indef} not.exist:\textsc{sens} \\
 \glt `I cannot say for sure.' (many examples)
  \end{exe}
  
 %kɤpupu
 
In addition, there are highly lexicalized oblique participles occurring as member of compounds; these cases are generally ambiguous, and alternatively analyzable as  lexicalized velar infinitives (§\ref{sec:lexicalized.velar.inf}). The incorporating verb \japhug{kɤtɯpa}{tell} is an interesting case: it combines the form \forme{kɤ-ti} (either the participle `what one says' or the infinitive `to say') in \textit{status constructus} \forme{kɤtɯ-} (the alternative form \forme{kɤtipa} is also attested) with the auxiliary \japhug{pa}{do} (§ XXX).
 
\subsection{Oblique participles} \label{sec:oblique.participle}
The \forme{sɤ}-prefix (and its allomorphs \forme{sɤɣ}-, \forme{sɤz}- and \forme{z}-) is used for non-core argument nominalization, in particular recipient of indirective verbs (§\ref{sec:gen.beneficiary}, §\ref{sec:dative}), instruments (\ref{sec:instr.kW}), place and time adjuncts, as in (\ref{ex:come}). It takes a possessive prefix which can be coreferent with any core argument (subject or object).

\begin{exe}
\ex \label{ex:come}
\gll ɯ-sɤ-ɣi \\
 \textsc{3sg-nmlz:obl}-come \\
\glt  `The place/moment from where/when he/it comes.' (elicited)
\end{exe}
 
 
 \subsubsection{Allomorphy} \label{sec:oblique.participle.allomorphy}
The base form of the oblique participle is \forme{sɤ-}, but three additional allomorphs are also found: \forme{sɤz-}, \forme{z-} and \forme{sɤɣ-}.

The allomorph \forme{sɤɣ-} or \forme{sɤx-} (depending on the voicing of the next consonant) is attested with intransitive (or labile) monosyllabic verbs with an onset without velar/uvular consonant, and without consonant cluster involving a preinitial (see § XXX concerning this context). This allomorph is to some extent lexicalized, and is not found with all verbs fulfilling these criteria. Table (\ref{tab:sAG.participle}) present some of the most common examples of \forme{sɤɣ-} participles in Kamnyu Japhug.

\begin{table}
\caption{Examples of oblique participles in \forme{sɤɣ-}} \label{tab:sAG.participle}
\begin{tabular}{llllll}
\lsptoprule
Base verb & Oblique participle \\
\midrule
\japhug{pʰɤn}{be efficient} & \forme{ɯ-sɤx-pʰɤn}  `advantage' \\
\japhug{me}{not exist} & \forme{ɯ-sɤɣ-me}  `place where there is no X' \\
\japhug{ɬoʁ}{come out} & \forme{ɯ-sɤɣ-ɬoʁ} `place where X grows, \\
\japhug{lɤɣ}{graze} & \forme{ɯ-sɤɣ-lɤɣ} `pasture' \\
& place from which X comes out' \\
\japhug{ndzoʁ}{be attached} & \forme{ɯ-sɤɣ-ndzoʁ} `place where X is attached' \\
\japhug{zo}{land (of bird)} & \forme{ɯ-sɤɣ-zo} `place where X lands' \\
\japhug{ɕe}{go} & \forme{ɯ-sɤx-ɕe} `direction, place towards which X goes' \\
\lspbottomrule
\end{tabular}
\end{table} 

Some of the verbs taking the \forme{sɤɣ-} allomorph do also occur with \forme{sɤ-}. For instance, \japhug{me}{not exist} is attested with both \forme{ɯ-sɤɣ-me} as in (\ref{ex:WsAGme}) and \forme{ɯ-sɤ-me} in (\ref{ex:WsAme}). However, most verbs in Table \ref{tab:sAG.participle} are only compatible with the \forme{sɤɣ-} allomorph.

\begin{exe}
\ex \label{ex:WsAGme}
\gll ʑmbɯlɯm ɯ-sɤɣ-ɬoʁ nɯra tu-ɬoʁ ŋu. ʑmbɯlɯm ɯ-sɤɣ-me ra kɯnɤ tu-ɬoʁ ɕti. \\
species.of.mushroom \textsc{3sg}.\textsc{poss}-\textsc{nmlz}:\textsc{obl}-come.out \textsc{dem}:\textsc{pl} \textsc{ipfv}-come.out be:\textsc{fact} species.of.mushroom \textsc{3sg}.\textsc{poss}-\textsc{nmlz}:\textsc{obl}-not.exist \ \textsc{pl} also \textsc{ipfv}-come.out be.\textsc{affirm}:\textsc{fact} \\
\glt `It grows in the places where the \textit{youlaku} mushroom grows, and also in the places where there are no \textit{youlaku}.'  (22-BlamajmAG, 22)
\end{exe}

 \begin{exe}
\ex \label{ex:WsAme}
\gll tɤjmɤɣ ɯ-sɤ-tu ɯ-sɤ-me ɣɤʑu. \\
mushroom \textsc{3sg}.\textsc{poss}-\textsc{nmlz}:\textsc{obl}-exist  \textsc{3sg}.\textsc{poss}-\textsc{nmlz}:\textsc{obl}-not.exist  exist:\textsc{sens} \\
\glt `There are places where there are mushrooms, and other places where there aren't.' (20-grWBgrWB, 46)
\end{exe}

The allomorphs \forme{sɤz-} and \forme{z-} occur in the same context, with non-monosyllabic verb stems, where the first syllable (either a productive or a frozen prefix) is sonorant-initial. These two allomorphs are completely interchangeable, without restriction on particular verbs or the function of the the relativized element (instrument, locative or temporal adjunct). For instance, the locative participle of \japhug{rɤʑi}{stay} is attested as both \japhug{ɯ-sɤz-rɤʑi} and \japhug{ɯ-z-rɤʑi} `the place when he/it stays' in the corpus, as shown by examples (\ref{ex:WsAzrAZi}) and (\ref{ex:WzrAZi}), a few sentences away from each other in the same story.

\begin{exe}
\ex \label{ex:WsAzrAZi}
\gll  tɕeri nɯnɯ sɤtɕʰa nɯ li iɕqʰa qapribɯxsi ɯ-sɤz-rɤʑi pjɤ-ɕti. \\
but \textsc{dem} place \textsc{dem} again the.aforementioned python \textsc{3sg}.\textsc{poss}-\textsc{nmlz}:\textsc{obl}-stay \textsc{ifr}.\textsc{ipfv}-be.\textsc{affirm} \\
\glt `But that place was the abode of a python.' (140511 xinbada-zh, 92)
\end{exe}

\begin{exe}
\ex \label{ex:WzrAZi}
\gll tɕe <xinbaba> rcanɯ, maka nɯtɕu ɯ-z-rɤʑi ɯ-tɯ-sɤɣ-mu pjɤ-sɤre ʑo tɕe,\\
\textsc{lnk} Sinbad \textsc{unexpected} at.all \textsc{dem}:\textsc{loc} \textsc{3sg}.\textsc{poss}-\textsc{nmlz}:\textsc{obl}-stay \textsc{3sg}.\textsc{poss}-\textsc{nmlz}:\textsc{degree}-\textsc{deexp}-be.afraid \textsc{ifr}.\textsc{ipfv}-be.ridiculous \textsc{emph} \textsc{lnk}\\
\glt `Sinbad, the place where he stayed was extremely terrifying.' (140511 xinbada-zh, 99)
\end{exe}

The \forme{sɤ-} allomorph, rather than \forme{sɤz-} or \forme{z-}, is however found when preceding the vertitive (§ XXX) and autobenefactive (§ XXX) prefixes, as in (\ref{ex:sAnWlhoR.kWme}).

\begin{exe}
\ex \label{ex:sAnWlhoR.kWme}
\gll  qʰe tú-wɣ-cɯ mɤ-kɯ-khɯ, sɤ-nɯ-ɬoʁ ri kɯ-me ta-βzu. \\
\textsc{lnk} \textsc{ipfv}-\textsc{inv}-open \textsc{neg}-\textsc{nmlz}:S/A-be.possible \textsc{nmlz}:\textsc{obl}-\textsc{auto}-come.out also \textsc{nmlz}:S/A-not.exist \textsc{pfv}:3\fl{}3'-make \\
\glt `(He put tape on the drawers so that) there were not possible to open, and there was no way to come out of it (to prevent the rats inside from escaping).' (150831 BZW kAnArRaR)
\end{exe}

The allomorph \forme{sɤɣ-} is also (though more rarely) attested with the \forme{nɯ-} autobenefactive of verbs that take \forme{sɤɣ-} in their simplex form. For instance, next to \forme{sɤ-nɯ-ɬoʁ}, the oblique participle \forme{sɤɣ-nɯ-ɬoʁ} is found in (\ref{ex:WsAGnWlhoR}) (without autobenefactive prefix the oblique participle is \forme{ɯ-sɤɣ-ɬoʁ}, see Table \ref{tab:sAG.participle}).

\begin{exe}
\ex \label{ex:WsAGnWlhoR}
\gll ɯ-sɤɣ-nɯ-ɬoʁ ɣɯ ɯ-kɯ-spoʁ pjɤ-ŋu.  \\
\textsc{3sg}.\textsc{poss}-\textsc{nmlz}:\textsc{obl}-\textsc{auto}-come.out \textsc{gen} \textsc{3sg}.\textsc{poss}-\textsc{nmlz}:S/A-have.a.hole \textsc{ifr}.\textsc{ipfv}-be \\
\glt `It was the hole from which it (the animal) came out (of the cave).  (140511 xinbada-zh, 83)
\end{exe} 

The \forme{sɤz-} allomorph is not completely impossible with the autobenefactive \forme{nɯ-} prefix, but only one example, \forme{nɯ-sɤz-nɯ-ɴɢɤt} `the place where they (had) parted ways' (\ref{ex:nWsAznWNGAt}), is found in the whole corpus (and the form \forme{sɤ-nɯ-ɴɢɤt} is also attested, see example \ref{ex:YWsAnWNGAt} in §\ref{sec:locative.participle.relatives}).

\begin{exe}
\ex \label{ex:nWsAznWNGAt}
\gll  nɯ-sɤz-nɯ-ɴɢɤt ɣɯ iɕqʰa, tʂɤsɤɴɢɤt nɯtɕu jɤ-azɣɯt-nɯ tɕe, \\
\textsc{3pl}.\textsc{poss}-\textsc{nmlz}:\textsc{obl}-\textsc{auto}-\textsc{acaus}:separate \textsc{gen} \textsc{filler}  crossroads \textsc{dem}:\textsc{loc} \textsc{pfv}-reach-\textsc{pl} \textsc{lnk} \\
\glt `They arrived at the crossroads where they had parted ways.' (140508 benling gaoqiang de si xiongdi, 109)
\end{exe} 

\subsubsection{Transitivity} \label{sec:oblique.participle.transitivity}
Like subject and object participles, oblique participle keep the verb transitivity, and transitive verbs can take an overt object as \forme{qaj ɯ-sɤ-ji} `place for planting wheat' in (\ref{ex:qaj.WsAji}).

\begin{exe}
\ex \label{ex:qaj.WsAji}
\gll  qajsta nɯnɯ kɯɕɯŋgɯ qaj ɯ-sɤ-ji pjɤ-pe tɕe tɕe núndʐa qajsta tu-ti-nɯ ɲɯ-ŋu \\
pl.n. \textsc{dem} in.former.times wheat \textsc{3sg}.\textsc{poss}-\textsc{nmlz}:\textsc{obl}-plant \textsc{ifr}.\textsc{ipfv}-be.good \textsc{lnk} \textsc{lnk} for.this.reason pl.n. \textsc{ipfv}-say-\textsc{pl} \textsc{sens}-be \\
\glt `Qaysta, in former time it was a wheat field which was good, and for this reason, it is called `Qaysta' `the place of the wheat'.' (140522 kAmYW tWji2, 104)
\end{exe}

An antipassive form (\ref{sec:antipassive}) is necessary if there is no definite object. For instance in (\ref{ex:sAZrAji}), \forme{sɤz-rɤ-ji} `place for planting, field' is based on the \forme{rɤ-} antipassive of \japhug{ji}{plant}; this form, unlike \forme{ɯ-sɤ-ji} in (\ref{ex:qaj.WsAji}), is used without (and cannot occur with) any noun specifying the crop planted in the field.

\begin{exe}
\ex \label{ex:sAZrAji}
\gll  tɕe tɯmgri sɤz-rɤ-ji nɯ koŋla ɲɤ-ɣɤ-me-nɯ ma kʰa ʁɟa ʑo to-βzu-nɯ. \\
lnk pl.n. \textsc{nmlz}:\textsc{obl}-\textsc{apass}-plant \textsc{dem} completely \textsc{ifr}-\textsc{caus}-not.exist-\textsc{pl} \textsc{lnk} house completely \textsc{emph} \textsc{ifr}-make-\textsc{pl} \\
\glt `They removed all the fields in Temgri, and built houses there (instead).' (140522 kAmYW tWji2, 18)
\end{exe}

 \subsubsection{Possessive prefixes} \label{sec:oblique.participle.possessive}
Possessive prefixes on oblique participles are optional, though their presence is preferred in careful speech.

With intransitive verbs, the possessive prefix refers to the subject, as \forme{a-sɤz-nɤri} `the place where I stay' in (\ref{ex:asAzrAZi.ri.GAZu}).

\begin{exe}
\ex \label{ex:asAzrAZi.ri.GAZu}
\gll a-kɤ-ndza ri ɣɤʑu, a-sɤz-rɤʑi ri ɣɤʑu qʰe \\
\textsc{1sg}.\textsc{poss}-\textsc{nmlz}:P-eat also exist:\textsc{sens} \textsc{1sg}.\textsc{poss}-\textsc{nmlz}:oblique-stay also exist:\textsc{sens} \textsc{lnk} \\
\glt `(There), I have food to eat and a place to stay.' (150831 renshen wawa, 24)
\end{exe}

With transitive verbs, the possessive prefix can be coreferent with the object. For instance, in (\ref{ex:nWsthamtCAt.GW.nWsAtsxWB}), the plural \forme{nɯ-} on \forme{nɯ-sɤ-tʂɯβ} refers to the many types of clothes and shoes mentioned just before in the same text.

 \begin{exe}
\ex \label{ex:nWsthamtCAt.GW.nWsAtsxWB}
\gll nɯstʰamtɕɤt ɣɯ nɯ-sɤ-tʂɯβ nɯ tɯ-ŋgru tu-sɯ-βzu-nɯ.   \\
so.many \textsc{gen} \textsc{3pl}.\textsc{poss}-\textsc{nmlz}:oblique-sew \textsc{dem} \textsc{indef}.\textsc{poss}-sinew \textsc{ipfv}-\textsc{caus}-make-\textsc{pl} \\
\glt `People use sinew to sew that many (types of clothes and shoes).' (150906 tWNgru, 17)
\end{exe} 

However, it is also possible for the possessive prefix to be coreferent with the subject. This is particularly common when the subject is first or second person, and no overt object is present, as \forme{nɤ-sɤ-ta} `the place where you put it' in (\ref{ex:nAsAta.me}).
 
 \begin{exe}
\ex \label{ex:nAsAta.me}
\gll   kɯɕte nɯtɕu nɤ-sɤ-ta me ɯ́-ŋu \\
other \textsc{dem}:\textsc{loc} \textsc{2sg}.\textsc{poss}-\textsc{nmlz}:\textsc{obl}-put not.exist:\textsc{fact} \textsc{qu}-be:\textsc{fact} \\
\glt `Isn't there any other place where you put (the food)?' (meimei de gushi, 72)
 \end{exe} 

There is no person hierarchy in slot accessibility to the possessive prefix however; in (\ref{ex:nWsAntChoz}), the possessive prefix on \forme{nɯ-sɤ-ntɕʰoz} marks the subject, although the object is first person plural. 

 \begin{exe}
\ex \label{ex:nWsAntChoz}
\gll  tɕe iʑora ɣɯ nɯ-sɤ-ntɕʰoz a-pɯ-tu tɕe ɲɯ-tʂɯn \\
\textsc{lnk} \textsc{1pl} \textsc{gen} \textsc{3pl}.\textsc{poss}-\textsc{nmlz}:\textsc{obl}-use \textsc{irr}-\textsc{ipfv}-exist lnk \textsc{sens}-be.grateful \\
\glt `We are glad that (some) of us have an opportunity to be useful to them.' (conversation, 140510)
 \end{exe}
   
When the object is overt (§\ref{sec:oblique.participle.transitivity}), it is rare to put a first or second person possessive prefix coreferent with the subject on the participle. Rather, a possessive prefix occurs on the object, as in (\ref{ex:jikhWtsa.sArku}) and (\ref{ex:ambrAz.sArku}), as if \forme{kʰɯtsa sɤ-rku} `place where one puts the bowls' and \forme{mbrɤz sɤ-rku} `rice container' were compounds.

\begin{exe}
\ex \label{ex:jikhWtsa.sArku}
\gll    tɕe tɕe ji-kʰɯtsa sɤ-rku ɣɯ ɯ-ŋgɯ nɯ, <chouchou> ɯ-ŋgɯ pɯ-nnɯ-ŋu, nɯ kɯmaʁ nɯra pɯ-nnɯ-ŋu kɯnɤ maka, laχtɕʰa kɤ-rku me, kɤ-ndza kɤ-rku me, nɯra tu-ndze nɤ tu-ndze, \\
\textsc{lnk} \textsc{lnk} \textsc{1pl}.\textsc{poss}-bowl \textsc{nmlz}:\textsc{obl}-put.in \textsc{gen} \textsc{3sg}.\textsc{poss}-inside \textsc{dem} drawer \textsc{3sg}.\textsc{poss}-inside \textsc{pst}.\textsc{ipfv}-\textsc{auto}-be dem other \textsc{dem}:\textsc{pl} \textsc{pst}.\textsc{ipfv}-\textsc{auto}-be  also at.all thing \textsc{nmlz}:P-put.in whether \textsc{nmlz}:P-eat \textsc{nmlz}:P-put.in whether \textsc{dem} \textsc{ipfv}-eat[III] \textsc{lnk}  \textsc{ipfv}-eat[III] \\
\glt `Whether it was in the cupboard where we put (our) bowls, in the drawers or elsewhere, whether it was things put in there or food, (the mice) ate/gnawed it again and again.' (150831 BZW kAnArRaR, 5)
\end{exe} 

\begin{exe}
\ex \label{ex:ambrAz.sArku}
\gll  nɯ sɤznɤ a-mbrɤz sɤ-rku a-pɯ-ŋu ɲɯ-ra \\
\textsc{dem} \textsc{comp} \textsc{1sg}.\textsc{poss}-rice \textsc{nmlz}:\textsc{obl}-put.in \textsc{irr}-\textsc{ipfv}-be \textsc{sens}-have.to \\
\glt  `Why don't I use (this basin) as a rice container?' (150831 jubaopen, 27)
\end{exe} 

Using the possessive on the verb is never preferred, but appears to be grammatical in elicitation in negative existential constructions, thus next to (\ref{ex:jikhWtsa.sAta}),  (\ref{ex:khWtsa.jisAta}) is also possible.
\begin{exe}
\ex
\begin{xlist}
\ex \label{ex:jikhWtsa.sAta}
\gll iʑo ji-khɯtsa sɤ-ta me. \\
\textsc{1pl} \textsc{1pl}.\textsc{poss}-bowl  \textsc{nmlz}:\textsc{obl}-put not.exist:\textsc{fact} \\ 
\ex \label{ex:khWtsa.jisAta}
\gll iʑo kʰɯtsa ji-sɤ-ta me  \\
\textsc{1pl} bowl \textsc{1pl}.\textsc{poss}-\textsc{nmlz}:\textsc{obl}-put not.exist:\textsc{fact} \\ 
\glt `We  don't have any place to put the bowls.' (elicited)
\end{xlist}
\end{exe} 


If the object is an IPN, it can be alienabilized (§\ref{sec:alienabilization}). For instance, the \textsc{1pl} possessive form of \forme{tɯ-ŋga sɤ-χtɕi} `washing machine' can be either (\ref{ex:jitWNga.sAXtCi}) with alienabilization or (\ref{ex:jiNga.sAXtCi}) without it.

\begin{exe}
\ex
\begin{xlist}
\ex \label{ex:jitWNga.sAXtCi}
\gll ji-tɯ-ŋga sɤ-χtɕi \\
\textsc{1pl}.\textsc{poss}-\textsc{indef}.\textsc{poss}-clothes \textsc{nmlz}:\textsc{obl}-wash \\
\ex \label{ex:jiNga.sAXtCi}
\gll ji-ŋga sɤ-χtɕi \\
\textsc{1pl}.\textsc{poss}-clothes \textsc{nmlz}:\textsc{obl}-wash \\
\end{xlist}
\glt `Our washing machine' (\ref{ex:jitWNga.sAXtCi} heard in context, \ref{ex:jiNga.sAXtCi} elicited)
\end{exe}

\subsubsection{Polarity and orientation prefixes} \label{sec:oblique.participle.orientation}
Unlike subject and object participles, the only prefixes (other than possessive prefixes) that oblique participles can take are the polarity prefixes and series B orientation prefixes.

It is thus not possible to have perfective or past imperfective oblique participles, and alternative strategies are used to express the corresponding meanings. For instance, from the verb \japhug{sqa}{cook}, the form $\dagger$\forme{ɯ-pɯ-sɤ-sqa} (intended meaning: `the thing that has been used to cook') is incorrect, and the solution to circumvent this morphological constraint is to combine the plain oblique participle \forme{ɯ-sɤ-sqa} with \forme{pɯ-kɯ-ŋu} (the past imperfective subject participle of \japhug{ŋu}{be}) and with the phrase \forme{nɯ ɕɯŋgɯ} `before that', as in (\ref{ex:WsAsqa.pWkWNu}).

\begin{exe}
\ex \label{ex:WsAsqa.pWkWNu}
\gll  nɯ ɕɯŋgɯ ɯ-sɤ-sqa pɯ-kɯ-ŋu ɯ-ŋgɯ (tu-rku-nɯ) \\
\textsc{dem} before \textsc{3sg}.\textsc{poss}-\textsc{nmlz}:\textsc{obl}-cook \textsc{pst}.\textsc{ipfv}-\textsc{nmlz}:S/A-be \textsc{3sg}.\textsc{poss}-inside \textsc{ipfv}-put.in-\textsc{pl} \\
\glt `(They put it) in the (pan) that had been used before to cook (the barley grains).' (31-cha, 64)
\end{exe}

Negative forms of the oblique participle are not very common, but examples are found in the corpus (as in \ref{ex:WmAsApe}) and there is no difficulty to elicit them.

\begin{exe}
\ex \label{ex:WmAsApe}
\gll  qaʑmbri nɯ, nɤkinɯ, ɯ-sɤ-pe ra me, ɯ-mɤ-sɤ-pe ra me, \\
vine \textsc{dem} \textsc{filler} \textsc{3sg}.\textsc{poss}-\textsc{nmlz}:\textsc{obl}-be.good \textsc{pl} not.exist:\textsc{fact}  \textsc{3sg}.\textsc{poss}-\textsc{neg}-\textsc{nmlz}:\textsc{obl}-be.good \textsc{pl} not.exist:\textsc{fact} \\
\glt `The vine is neither an advantage nor a harm (to the plants on which it grows).' (06-qaZmbri, 17)
\end{exe}


\subsubsection{Locative relative clauses} \label{sec:locative.participle.relatives}
The oblique participle can be used to build many different types of relative clauses, with various non-core arguments and adjuncts as relativized elements, including locative, temporal, instrumental adjuncts and dative arguments. The most common ones are the locative relative clauses.

With  motion verbs like \japhug{ɕe}{go} or verbs of manipulation, the relativized element can be either the goal (place towards which the motion is conducted, as in \ref{ex:ndZisAxCe}), or the path through which the motion event takes place: in (\ref{ex:tsxu.kusAxCe}) for instance, the head \forme{tʂu} is a locative adjunct (`the road through which one goes to X') distinct from the goal (the placename \forme{prɤɕta}).

\begin{exe}
\ex \label{ex:ndZisAxCe}
\gll ndʑi-sɤx-ɕe nɯtɕu jo-zɣɯt tɕe \\
\textsc{3du}.\textsc{poss}-\textsc{nmlz}:\textsc{obl}-go \textsc{dem}:\textsc{loc} \textsc{ifr}-reach \textsc{lnk} \\
\glt  `(The ox) arrived at the place towards which the two of them were going.' (150826 shier shengxiao-zh, 86)
\end{exe}

\begin{exe}
\ex \label{ex:tsxu.kusAxCe}
\gll   [prɤɕta tʂu ku-sɤx-ɕe] nɯre ri tɯ-ji ci tu tɕe, nɯ cɤŋgɤɣ rmi. \\
pl.n. path \textsc{ipfv}:east-\textsc{nmlz}:\textsc{obl}-go \textsc{dem}:\textsc{loc} \textsc{loc} \textsc{indef}.\textsc{poss}-field \textsc{indef} exist:\textsc{fact} \textsc{lnk} \textsc{dem} pl.n. be.called:\textsc{fact} \\
\glt  `On the road (one has to go through to reach) Prashta there is a field, it is called Kyangag.' (140522 kAmYW tWji, 116-117)
\end{exe}

In (\ref{ex:WsAznACWCe}), the oblique participle designates the areas in which the plant grows, without a specific goal.

\begin{exe}
\ex \label{ex:WsAznACWCe}
\gll  pɤjka wuma ʑo a-pɯ-pe, tɕe ɯ-sɤx-ɕe nɯra a-pɯ-dɤn ɯ-sɤz-nɤɕɯɕe a-pɯ-dɤn tɕe, pɤjka tɯ-pʰɯ ɯ-taʁ nɯtɕu kɯβdɤsqi jamar, ɯ-mat ku-tsʰoʁ ɲɯ-cʰa \\
pumpkin really \textsc{emph} \textsc{irr}-\textsc{ipfv}-be.good \textsc{lnk} \textsc{3sg}.\textsc{poss}-\textsc{nmlz}:\textsc{obl}-go \textsc{dem}:\textsc{pl} \textsc{irr}-\textsc{ipfv}-be.many \textsc{3sg}.\textsc{poss}-\textsc{nmlz}:\textsc{obl}-go.around \textsc{irr}-\textsc{ipfv}-be.many  \textsc{lnk} pumpkin \textsc{one}-tree \textsc{3sg}.\textsc{poss}-on \textsc{dem}:\textsc{loc} fourty about \textsc{3sg}.\textsc{poss}-fruit \textsc{ipfv}-attach \textsc{sens}-can \\
\glt `When the pumpkin (grows) well, and when there are a lot of (places on) which it can spread, one plant can have about fourty pumpkin.' (16-CWrNgo, 96)
\end{exe}

The relativized locative adjunct can also be the place of origin rather than the goal in the case of the verb \japhug{ɣi}{come} as in (\ref{ex:jisAGi}).

\begin{exe}
\ex \label{ex:jisAGi}
\gll  iʑora nɯ ji-sɤ-ɣi nɯtɕu pjɤ-ŋu tɕe. \\
\textsc{1pl} \textsc{dem} \textsc{1pl}.\textsc{poss}-\textsc{nmlz}:\textsc{obl}-come \textsc{dem}:\textsc{loc} \textsc{ifr}.\textsc{ipfv}-be \textsc{lnk} \\
\glt  `The place from where we come was there.' (2010-06, 3)
\end{exe}

With stative verbs or dynamic verbs implying no motion, the oblique participle has a static locative meaning, as in (\ref{ex:WsAdAn}) and (\ref{ex:tWsAmdzW}).

\begin{exe}
\ex \label{ex:WsAdAn}
\gll  ma ɯ-sɤ-dɤn nɯ tɕɤtu rɯŋgu ŋu. \\
\textsc{lnk} \textsc{2sg}.\textsc{poss}-\textsc{nmlz}:oblique-be.many \textsc{dem} up.there pasture be:\textsc{fact} \\
\glt `The (place) where they are (most) numerous is up there on the pastures. (17-xCAj, 85)
\end{exe}

\begin{exe}
\ex \label{ex:tWsAmdzW}
\gll tɤ-tɕɯ tɕʰeme tɯ-sɤ-ɤmdzɯ ʑaka tu. \\
\textsc{indef}.\textsc{poss}-son girl \textsc{indef}.\textsc{poss}-\textsc{nmlz}:\textsc{obl}-sit each exist:\textsc{fact} \\ 
\glt `Gents and ladies each have their (own specific) sitting place.' (31-khAjmu, 10)
\end{exe}


Participial locative relative clauses are used to describe non-transient properties of places: directions, locations or places of origin that are unchanging characteristics of things or persons (\ref{ex:tsxu.kusAxCe} and \ref{ex:jisAGi}), places where some state of affair generally occurs due to a natural law (\ref{ex:WsAznACWCe} and \ref{ex:WsAdAn}) or places used for a specific purpose, as in (\ref{ex:tWsAmdzW}), and even more clearly in (\ref{ex:WsAta.Wrkoz}).

\begin{exe}
\ex \label{ex:WsAta.Wrkoz}
\gll tɕe saŋdi nɯ tɕe, nɯnɯ si ɯ-sɤ-ta ɯ-rkoz ʑo pjɤ-ŋu. \\
  \textsc{lnk} lower.side.of.the.hearth \textsc{dem} \textsc{lnk} \textsc{dem} firewood \textsc{3sg}.\textsc{poss}-\textsc{nmlz}:\textsc{obl}-put \textsc{3sg}.\textsc{poss}-special \textsc{emph} \textsc{ifr}.\textsc{ipfv}-be \\
\glt `The lower side of the hearth was specifically where (people) put firewood.' (2011-11, 33)
\end{exe}


For transient properties of places, finite relatives are used instead (§ XXX). In example (\ref{ex:nWNga.sAta}) translated from Chinese,\footnote{The original text has \ch{……走到仙女们放衣服的地方}{zǒu dào xiānnǚmen fàng yīfú de dìfāng}{...went to the place where the sky goddesses put/had put their clothes}; both interpretations are possible. }  Tshendzin hesitates between a participial relative (implying that there was a specific place where the sky goddesses put their clothes each time they came to earth) and a finite relative (suggesting that they put their clothes in some unspecific place, perhaps in a casual way as the autobenefactive \forme{-nɯ-} could indicate, § XXX).

\begin{exe}
\ex \label{ex:nWNga.sAta}
\gll tɕʰemɤpɯ nɯra, [nɯ-ŋga sɤ-ta] nɯtɕu ko-ɕe, [nɯ-ŋga na-nɯ-ta-nɯ] nɯtɕu ko-ɕe matɕi, \\
girl \textsc{dem}:\textsc{pl} \textsc{3pl}.\textsc{poss}-clothes \textsc{nmlz}:\textsc{obl}-put \textsc{dem}:\textsc{loc} \textsc{ifr}:\textsc{east}-go \textsc{3pl}.\textsc{poss}-clothes \textsc{pfv}:3\fl{}3'-\textsc{auto}-put-\textsc{pl} \textsc{dem}:\textsc{loc} \textsc{ifr}:\textsc{east}-go \textsc{lnk} \\
\glt `He went to the place where the girls put their clothes, where they had put their clothes.' (150828 niulang-zh, 59)
\end{exe}

Locative participial relative clauses with overt head can be prenominal, in particular with genitival relatives as in (\ref{ex:sAnWCe.GW.Wtsxu}). 

\begin{exe}
\ex \label{ex:sAnWCe.GW.Wtsxu}
\gll  pjɤ-nɯkɯlu-nɯ ma sɤ-nɯ-ɕe ɣɯ ɯ-tʂu nɯ mɯ-ɲɤ-nɯ-mto-nɯ. \\
\textsc{ifr}-be.lost-\textsc{pl}  \textsc{lnk} \textsc{nmlz}:\textsc{obl}-\textsc{vert}-go \textsc{gen} \textsc{3sg}.\textsc{poss}-path \textsc{dem} \textsc{neg}-\textsc{ifr}-\textsc{auto}-see-\textsc{pl} \\
\glt `They were lost, and could not find the way back home.' (160630 poucet1, 55)
\end{exe}

However, head-internal relatives are also attested: in (\ref{ex:tsxu.kusAxCe}) above and (\ref{ex:YWsAnWNGAt}), the head of the participial relatives, the noun \japhug{tʂu}{road}, occurs between the verb in oblique participle form and a placename marking the goal or point of origin. Note in addition that (\ref{ex:YWsAnWNGAt}) illustrates two locative oblique participial relative clauses embedded within another participial relative.

\begin{exe}
\ex \label{ex:YWsAnWNGAt}
\gll   [[rpɤŋgɯ tʂu lu-sɤx-ɕe] cʰo [prɤscʰɯ tʂu lu-sɤ-ɣi] ɲɯ-sɤ-nɯ-ɴɢɤt] nɯtɕu, \\
pl.n.  path \textsc{ipfv}:\textsc{upstream}-\textsc{nmlz}:\textsc{obl}-go \textsc{comit} pl.n.  path \textsc{ipfv}:\textsc{upstream}-\textsc{nmlz}:\textsc{obl}-come \textsc{ipfv}:\textsc{west}-\textsc{nmlz}:\textsc{obl}-\textsc{auto}-\textsc{acaus}:separate \textsc{dem}:\textsc{loc} \\
\glt `At the place where the road towards Rpangu and the road from Praskhyu separate' (140522 kAmYW tWji2, 125)
\end{exe}

\subsubsection{Instrumental relative clauses} \label{sec:instrumental.participle.relatives}
Another very productive type of oblique participial relatives are the instrumental relative clauses. Although instruments, like transitive subjects, receive ergative case (§\ref{sec:instr.kW}), they cannot be relativized with subject participial relatives, and an oblique participle is used instead.

There is often ambiguity between instrument relativization and locative adjunct relativization; for instance, while the participle \forme{ɯ-z-rɤ-rɤt} can mean `pen (the tool used to write)' as in (\ref{ex:WzArAt}), this form can also designate the paper on which one writes or even one's office.

\begin{exe}
\ex \label{ex:WzArAt}
\gll ɯ-slamaχti nɯ ɣɯ, [ɯ-z-rɤ-rɤt] ci to-nɯ-ndo tɕe jo-nɯ-tsɯm ɲɯ-ŋu tɕe, \\
\textsc{3sg}.\textsc{poss}-classmate dem gen \textsc{3sg}.\textsc{poss}-\textsc{nmlz}:\textsc{obl}-\textsc{apass}-write \textsc{indef} \textsc{ifr}-\textsc{auto}-take \textsc{lnk} \textsc{ifr}-\textsc{vert}-take.away \textsc{sens}-be \textsc{lnk} \\
\glt `He took away the pen of a classmate.' (2014-tou dongxi de xiaohai, 5)
\end{exe}

Instrumental relative clauses built with oblique participles can occur as objects of the causativized verb \japhug{sɯ-βzu}{cause to make; use X to make}, as in (\ref{ex:tWthW.sAXtCi}).

\begin{exe}
\ex \label{ex:tWthW.sAXtCi}
\gll nɯnɯ [tɯtʰɯ sɤ-χtɕi], [tɯ-ŋga sɤ-pɕiz] nɯra tu-sɯ-βzu-nɯ pɯ-ŋgrɤl. \\
\textsc{dem} pan \textsc{nmlz}:\textsc{obl}-wash \textsc{indef}.\textsc{poss}-clothes \textsc{nmlz}:\textsc{obl}-wipe \textsc{dem}:\textsc{pl} \textsc{ipfv}-\textsc{caus}-make-\textsc{pl} \textsc{pst}.\textsc{ipfv}-be.usually.the.case \\
\glt `People used to employ (Usnea) as tools to wash pans or wipe clothes.' (20-sWrna, 151)
\end{exe} 

In this construction, the material used to make the tool is marked with the ergative (§\ref{sec:instr.kW}), as in (\ref{ex:WsACmi}).

\begin{exe}
\ex \label{ex:WsACmi}
\gll ɯnɯnɯ kɯ [ɯ-sɤ-ɕmi] tu-sɯ-βzu-nɯ pɯ-ŋgrɤl \\
\textsc{dem} \textsc{erg} \textsc{3sg}.\textsc{poss}-\textsc{nmlz}:\textsc{obl}-mix \textsc{ipfv}-\textsc{caus}-make-\textsc{pl} \textsc{pst}.\textsc{ipfv}-be.usually.the.case \\
\glt `People used to employ it (a boat oar) to mix it (the alcohol).' (31-cha, 45)
\end{exe}  

Oblique participles are used to make instrumental relative clauses, used like nouns of instruments, from both transitive and intransitive verbs. If the base verb is transitive, the participle retains its transitivity: thus in (\ref{ex:WsACmi}), the absence of object in the one-word relative clause \forme{ɯ-sɤ-ɕmi} `the tool used to mix it' is the result of zero-anaphora, and implies a definite object. Instrumental relative clauses with a transitive verb more often have an overt object, as in (\ref{ex:tWthW.sAXtCi}). For indefinite objects, antipassivization is necessary, as in \forme{ɯ-z-rɤ-rɤt} `the tool used to write' in (\ref{ex:WzArAt}) above (see also  §\ref{sec:oblique.participle.transitivity}). 

Instrumental participial relative clauses can take the generic IPN \japhug{ɯ-spa}{its material} as overt head, to disambiguate with other types of relatives, in particular locative ones, built with an oblique participle. For instance, in (\ref{ex:tusANke.Wspa}), the focus is on the use of the path (a path specially made in order to be able to walk inside the field), rather than simply on the location (`the place where one walks').

\begin{exe}
\ex \label{ex:tusANke.Wspa}
\gll tɕe tɯ-ji ɯ-χcɤl tu-kɯ-ŋke mɤ-khɯ ma tɤ-rɤku tu tɕe tɕe, nɯ ɣɯ [tu-sɤ-ŋke ɯ-spa], ɯ-tʂu <zhuanmen> ɯ-rkoz ɲɯ́-wɣ-βzu ŋgrɤl tɕe ɯnɯnɯ tʂu nɯ ftɕɤru tu-kɯ-ti ŋu \\
\textsc{lnk} \textsc{indef}.\textsc{poss}-field \textsc{3sg}.\textsc{poss}-middle \textsc{ipfv}-\textsc{genr}:S/P-walk \textsc{neg}-be.possible:\textsc{fact} \textsc{lnk} \textsc{indef}.\textsc{poss}-crops exist:\textsc{fact} \textsc{lnk} \textsc{lnk} \textsc{dem} \textsc{gen} \textsc{ipfv}-\textsc{nmlz}:\textsc{obl}-walk \textsc{3sg}.\textsc{poss}-material \textsc{3sg}.\textsc{poss}-path specially \textsc{3sg}.\textsc{poss}-special \textsc{ipfv}-\textsc{inv}-make be.usually.the.case:\textsc{fact} \textsc{lnk} \textsc{dem} path \textsc{dem} summer.path \textsc{ipfv}-\textsc{genr}-say be:\textsc{fact} \\
\glt `One cannot walk in the middle of the fields, because there are crops, so as a way to walk into it, one makes a specially path, and that path is call `summer path'.' (definition, 15-06-05)
\end{exe} 
 
\subsubsection{Other oblique relative clauses} \label{sec:other.oblique.participle.relatives}
In addition to goals, locative adjuncts and instruments, oblique participles are used to relativize various other types of arguments and adjuncts, though those cases are considerably less common in the corpus.

Dative arguments (in \forme{ɯ-ɕki} or \forme{ɯ-pʰe}, §\ref{sec:dative}) are relativized with an oblique participle, as is shown by (\ref{ex:WsAfCAt}),  where the verb \japhug{fɕɤt}{tell} also occurs as main verb of the second clause with an overt recipient marked with the dative.

\begin{exe}
\ex \label{ex:WsAfCAt}
\gll [ɯ-sɤ-fɕɤt] pjɤ-me qʰe tɕe tɤ-pɤtso ɯ-ɕki nɯ tɕu nɯra tɕʰi pɯ-kɯ-fse nɯra pjɤ-fɕɤt. \\
\textsc{3sg-nmlz:obl}-tell \textsc{ipfv.ifr}-not.exist \textsc{lnk} \textsc{lnk} \textsc{indef.poss}-child \textsc{3sg-dat} \textsc{dem} \textsc{loc} \textsc{dem:pl} what \textsc{pst-nmlz:S}-be.like  \textsc{dem:pl} \textsc{ifr}-tell \\
\glt `She had no one (else) to tell it to, so she told the boy everything that had happened.' (140515 congming de wusui xiaohai-zh, 77)
\end{exe} 

Likewise, comitative arguments, marked with the postposition \forme{cʰo} and selected by a handful of verbs (\japhug{naχtɕɯɣ}{be the same} and  \japhug{amɯmi}{be in good terms with}, §\ref{sec:comitative}), are relativized with an oblique participle, as in (\ref{ex:WsAmWmi}).

\begin{exe}
   \ex \label{ex:WsAmWmi}
 \gll  tɕe ɯʑo [ɯ-sɤ-ɤmɯmi] nɯ dɤn ma ca kɯ-fse qaʑo kɯ-fse, tsʰɤt kɯ-fse,  ɯʑo cʰo kɯ-naχtɕɯɣ sɯjno, xɕaj ma mɤ-kɯ-ndza nɯ ra cʰo nɯ amɯmi-nɯ tɕe, \\
\textsc{lnk} it \textsc{3sg-nmlz:oblique}-be.in.good.terms \textsc{dem} be.many:\textsc{fact} because musk.deer \textsc{nmlz:S}-be.like sheep \textsc{nmlz:S}-be.like goat  \textsc{nmlz:S}-be.like it with  \textsc{nmlz:S}-be.identical herbs grass apart.from \textsc{neg-nmlz:A}-eat \textsc{dem} \textsc{pl} with \textsc{dem} be.in.good.term:\textsc{fact}-\textsc{pl} \textsc{lnk} \\
\glt `The (animals) that are in good terms with the rabbit are many, it is in good terms with those that only eat grass, like musk deer, sheep or goats.' (04 qala1, 33-4)
\end{exe}

Time adjuncts are also possibly relativized using oblique participles, as \forme{ɯ-sɤ-ji}, which means `the period when it is planted' in (\ref{ex:WsAji}). However, finite relative clauses are the preferred way of relativizing time adjuncts (§ XXX). As in the case of locative relative clauses (§\ref{sec:locative.participle.relatives}), participial relative clauses are only used to refer to specific dates and time periods that are intrinsic properties of the event.

\begin{exe}
   \ex \label{ex:WsAji}
   \gll    tɕe nɯnɯ ʑaka [ɯ-sɤ-ji] ɲɯ-ŋu tɕe \\
   \textsc{lnk} \textsc{dem} each \textsc{3sg-nmlz:oblique}-plant \textsc{sens}-be \textsc{lnk}\\
\glt `These are the (periods) when people plant each of these (crops).' (15 tChWma, 19)
\end{exe}

Temporal relative clauses are generally headless, but (\ref{ex:WsAsna}) shows an example of head-internal (or postnominal) relative clause, with \japhug{skɤrma}{minute, date} as its head noun. In addition, this participial relative has here a superlative interpretation, like the subject participle in (\ref{ex:thamtCAt.GW.nWkWmpCAr}) above (§\ref{sec:subject.participle.possessive})

\begin{exe}
   \ex \label{ex:WsAsna}
   \gll    lɤsɤr χsɯm ɯ-raŋ tɕe, tɯxpalɤskɤr ɣɯ, nɤkinɯ, [skɤrma ɯ-sɤ-sna] ŋu tu-kɯ-ti ŋu. \\
   new.year three \textsc{3sg}.\textsc{poss}-time \textsc{loc} whole.year \textsc{gen} \textsc{filler} date \textsc{3sg}.\textsc{poss}-\textsc{nmlz}:\textsc{obl}-be.good be:\textsc{fact} \textsc{ipfv}-\textsc{genr}-say be:\textsc{fact} \\
\glt `We say that the third day of the year is the (most) auspicious day in the whole year.' (2010-10, 143)
\end{exe}

A further derived meaning of the oblique participle is that of `opportunity to do X', as in (\ref{ex:sAznWNgra.GAZu}) and (\ref{ex:nWsAntChoz}) above.

\begin{exe}
\ex \label{ex:sAznWNgra.GAZu}
\gll tɕe sɤz-nɯŋgra ɣɤʑu ri, li sɤɣʑɯr \\
\textsc{lnk} \textsc{nmlz}:\textsc{obl}-earn.wages exist:\textsc{sens} \textsc{lnk} again be.dangerous:\textsc{fact} \\
\glt `Although it (provides) an opportunity to earn wages, it is also dangerous.' (conversation 140510)
\end{exe}

Even in context, the exact meaning of a particular oblique relative clause may allow some leeway in interpretation. For instance, the participle \forme{ji-sɤx-ɕe} in the negative existential construction in (\ref{ex:jisAxCe.maNe}) could be understood as a locative relative clause `(we had no) place to go' but also alternatively as `(we had no) opportunity to go (anywhere)' in this particular context.

\begin{exe}
\ex \label{ex:jisAxCe.maNe}
\gll  iʑora tɕe kʰa ɯ-ŋgɯ kɤ-kɤ-ja ʑo ɲɯ-fse-j ku-rɤʑit-i ma [ji-sɤx-ɕe] maŋe \\
\textsc{1pl} \textsc{lnk} house \textsc{3sg}.\textsc{poss}-inside \textsc{pfv}-\textsc{nmlz}:P-close \textsc{emph} \textsc{sens}-be.like-\textsc{1pl} \textsc{ipfv}-stay-\textsc{1pl} \textsc{lnk} \textsc{1pl}.\textsc{poss}-\textsc{nmlz}:\textsc{obl}-go not.exist:\textsc{sens} \\
\glt  `We were like locked in the house, with nowhere to go.' (140501 tshering skyid, 116)
\end{exe}

\subsubsection{Ambiguity} \label{sec:oblique.participle.ambiguity}
The various allomorphs of the oblique participle do resemble other prefixes found in Japhug. The \forme{sɤ-} and \forme{sɤɣ-} allomorphs are also found with the deexperiencer derivation (§ XXX), and \forme{sɤ-} is also similar to the human antipassive, or the sigmatic causative of \forme{a-} initial verbs (§ XXX). Finally, the \forme{z-} allomorph of the oblique participle can resemble the causative (§ XXX) or one allomorph of the translocative prefix (§\ref{sec:translocative.morpho}).

However, unlike subject (§\ref{sec:subject.participle.ambiguities}) and object (§\ref{sec:object.participle.ambiguity}) participles, these surface ambiguities are only very superficial, as the forms with which the oblique participle could potentially be confused are all finite, and hardly ever occur in the same syntactic context as the oblique participle (and except for their bare infinitive form, in the case of transitive verb, never occur with a possessive prefix). 

In the case of the \forme{z-} allomorph of translocative prefix, note that it only occurs before a few orientation prefixes (\forme{ɲɯ-}, \forme{ɲɤ-}, \forme{ju-}, \forme{jɤ-}, \forme{jo-}, \forme{ja-}), whereas the oblique participle \forme{z-} can only follow an orientation prefix, as in \forme{ɯ-cʰɯ-z-raʁrɯz} in (\ref{ex:WchWzraRWz}), so that the two forms can never be confused.

\begin{exe}
\ex \label{ex:WchWzraRWz}
\gll  nɯnɯ ɣɯ ɯ-cʰɯ-z-raʁrɯz nɯ ɯ-sɤ-pɕiz rmi \\ 
\textsc{dem} \textsc{gen} \textsc{3sg}.\textsc{poss}-\textsc{ipfv}:\textsc{downstream}-\textsc{nmlz}:\textsc{obl}-sweep \textsc{dem}  \textsc{3sg}.\textsc{poss}-\textsc{nmlz}:\textsc{obl}-wipe be.called:fact \\
\glt `(The tool used to) sweep (the flour) is called a `wiper'.' (06-BGa, 216)
\end{exe}

\subsubsection{Lexicalized oblique participles} \label{sec:lexicalized.oblique.participle}
Nouns of instruments and of location, including placenames, are often made from oblique participles. 

The noun \japhug{sɤcɯ}{key}, although transparently originating from the instrumental use of the oblique participle of \japhug{cɯ}{open}, is lexicalized as shown by the fact that it cannot take orientation prefixes, and that it occurs in collocation with the auxiliary \forme{lɤt} to mean `lock (the door)' as in (\ref{ex:sAcW.malAt}).

\begin{exe}
\ex \label{ex:sAcW.malAt}
\gll   ɯ-ŋgɯ lɤ-ɣi jɤɣ ma sɤcɯ mɤ-a-lɤt \\
\textsc{3sg}.poss-inside \textsc{imp}:\textsc{upstream}-come be.allowed:\textsc{fact} \textsc{lnk} key \textsc{neg}-\textsc{pass}-throw \\
\glt `Come in, the door is not locked.' (140428 xiaohongmao, 78)
\end{exe}

Place names built from oblique participle include \forme{Znɤrɣɤma}, from the locative participle \forme{z-nɤrɣɤma} of the verb \japhug{nɤrɣɤma}{pray for rain} (see § XXX on the etymology of this verb), as it was the place where people use to perform this activity in Kamnyu in the traditional society, as explained in § XXX.

Another example is the uninhabited place called \forme{kɯlɤɣsɤmdzɯ}, a transparent combination  \japhug{kɯ-lɤɣ}{shepherd} (§\ref{sec:lexicalized.subject.participle}) and \japhug{ɯ-sɤ-ɤmdzɯ}{sitting place} reflecting the use of this place (as described in \ref{ex:kWlAG.sAmdzW}).

\begin{exe}
\ex \label{ex:kWlAG.sAmdzW}
\gll    kɯ-xtɕɯ\redp{}xtɕi ci ʑo antɤm, tɕe nɯ kɯ-lɤɣ ra nɯtɕu ku-rɤʑi-nɯ pjɤ-ŋgrɤl \\
\textsc{inf}:\textsc{stat}-\textsc{emph}\redp{}be.small a.little \textsc{emph} be.flat:\textsc{fact}  \textsc{lnk} \textsc{dem} \textsc{nmlz}:S/A-graze \textsc{pl} \textsc{dem}:\textsc{loc} \textsc{ipfv}-stay-\textsc{pl} \textsc{ifr}.\textsc{ipfv}-be.usually.the.case \\
\glt `(The place called \forme{kɯlɤɣsɤmdzɯ}) is a bit flat, and shepherd used to stay there.' (140522 Kamnyu zgo, 281)
 \end{exe}
 
There are also case of nouns of instruments in \forme{sɤ-} whose base verb is not identifiable. For instance, the noun \japhug{sɤɕtɕɯɣ}{strap to carry children on the back}, which is glossed using an oblique participle as in (\ref{ex:tApAtso.WsAzbWwa}), is most certainly a frozen oblique participle, but there is no verb \forme{*ɕtɕɯɣ} in Japhug.

\begin{exe}
\ex \label{ex:tApAtso.WsAzbWwa}
\gll   tɤ-pɤtso ɯ-sɤz-bɯwa \\
\textsc{indef}.\textsc{poss}-child \textsc{3sg}.\textsc{poss}-\textsc{nmlz}:\textsc{obl}-carry.on.the.back \\
\glt `Something used to carry children on the back' (definition given for the noun \forme{sɤɕtɕɯɣ})
\end{exe}

An even more pronounced type of lexicalization occurs when an oblique participle becomes member of a compound with \textit{status contructus} vowel alternation. In these cases, it is probable that the compound is genitival, rather than a reduced relative clause.

As first element of compound, we find the oblique participle \forme{ɯ-sɤ-qru} (from the verb \japhug{qru}{greet, welcome, receive}) in \textit{status constructus} combined with the noun \japhug{cʰa}{alcohol} into 
\japhug{sɤqrɤcʰa}{alcohol to treat the guests} (§\ref{sec:tatpurusha.n.n}). There are also cases of undetectable \textit{status constructus}, as in \japhug{sɤrŋgɯŋga}{bed cover} from the oblique participle of \japhug{rŋgɯ}{lie down} and the IPN \japhug{tɯ-ŋga}{clothes}: since the first element of this compound \forme{sɤ-rŋgɯ} ends in \forme{-ɯ}, it would not have a distinct bound form.

Oblique participles are also attested as second element of compounds from both transitive and intransitive verbs. 

As an example of lexicalized participle from an intransitive verb, the compound \japhug{tʂɤsɤɴɢɤt}{crossroad} combines the participle \japhug{ɯ-sɤ-ɴɢɤt}{place where X part ways} from the anticausative verb \japhug{ɴɢɤt}{part ways, part company}) with the \textit{status constructus} of the noun \japhug{tʂu}{path}, from an earlier locative participial relative \forme{*tʂu ɯ-sɤ-ɴɢɤt} `the place where roads separate'. 

With a transitive verb, we find the noun \japhug{βɣɤsɤprɤt}{watermill valve}, from an earlier instrumental relative \forme{*βɣa ɯ-sɤ-prɤt} `the tool used to stop (water) in the mill' from the \textit{status constructus} \forme{βɣɤ-} of \japhug{βɣa}{watermill} combined with the oblique participle \forme{ɯ-sɤ-prɤt} of the verb \japhug{prɤt}{break, stop}. With undetectable \textit{status constructus}, we have for instance \japhug{pʰɯsɤti}{belows} from the  onomatopoeia \forme{pʰɯ} and the oblique participle of \japhug{ti}{saying}, literally `the tool used to make `pff' sound'. In these cases, the noun occurring as first element of the compound corresponds to the object of the verb.


\section{Infinitives}

 
\subsection{Velar infinitives} \label{sec:velar.inf}
The most common infinitives in Japhug are the velar infinitives, build from the stem I of the verb and prefixed either with \forme{kɤ-} or or \forme{kɯ-}; they are homophonous with participles and not always easily distinguishable from them (§\ref{sec:infinitives.participles}). They are in particular the preferred citation form of the verbs (§\ref{sec:inf.citation}), though not with all speakers.
 
The \forme{kɯ-} infinitives are found with stative verbs (including adjectives and existential verbs), impersonal modal verbs and some anticausative verbs (§ XXX); other verbs take the \forme{kɤ-} infinitives. 

Stative verbs in \forme{a-} have regular fusion of \forme{kɯ-} and \forme{a-} as \ipa{kɤ-}, and thus superficially appear to have \forme{kɤ-} infinitives (for instance, the infinitive of \japhug{arŋi}{be green} is \forme{kɯ-ɤrŋi} \ipa{kɤrŋi}).

\subsubsection{Infinitives vs participles} \label{sec:infinitives.participles}
It is not immediately obvious that a category of `velar infinitives' needs to be distinguished from participles in Japhug, as both are non-finite verbal categories prefixed in \forme{kɤ-} or \forme{kɯ-} (§\ref{sec:object.participle.ambiguity} and §\ref{sec:subject.participle.ambiguities}). 

The necessity to set \forme{kɤ-} infinitives apart from object participles stems from the fact that the latter can only be built from transitive or semi-transitive verbs, while the former also occurs with strictly intransitive verbs. Thus, if one were to argue that all \forme{kɤ-}prefixed non-finite forms of transitive verbs are object participles, including in the case of complement clauses (for instance \forme{kɤ-ndza} in \ref{ex:kAndza.mWpWrYota}), one would not be able to account for the \forme{kɤ-}prefixed forms of intransitive verbs occurring in the same context such as \forme{kɤ-ɕe} in (\ref{ex:kACe.mWpWrYota}) -- even though \japhug{ɕe}{go} could be considered to be an indirect transitive verb (since it can take a goal, which can be relativized with a finite relative clause, § XXX), it is not possible to build a participial relative clause by prefixing \forme{kɤ-} on this verb (the oblique participle \forme{sɤ-} must be used instead, §\ref{sec:oblique.participle}).

\begin{exe}
\ex \label{ex:kAndza.mWpWrYota}
\gll aʑo kɤ-ndza mɯ-pɯ-rɲo-t-a \\
\textsc{1sg} ???-eat \textsc{neg}-\textsc{pfv}-experience-\textsc{pst}:\textsc{tr}-\textsc{1sg} \\
\glt `I never ate that.' (many attestations)
\end{exe}

\begin{exe}
\ex \label{ex:kACe.mWpWrYota}
\gll  aj kɤ-ɕe mɯ-pɯ-rɲo-t-a \\
\textsc{1sg} \textsc{inf}-go \textsc{neg}-\textsc{pfv}-experience-\textsc{pst}:\textsc{tr}-\textsc{1sg}  \\
\glt `I never went there.' (150820 ZNGWloR, 4)
\end{exe}

I therefore adopt the following criteria to discriminate between a \forme{kɤ-} infinitive and an object participle:  \forme{kɤ-} non-finite forms of (non-semi-transitive) intransitive verbs are infinitives; \forme{kɤ-} non-finite forms of transitive and semi-transitive verbs occurring in the same contexts as the infinitives of intransitive verbs are infinitives.

By systematically applying these criteria, we can identify three contexts where infinitives are attested: citation form (§\ref{sec:inf.citation}), complementation (§\ref{sec:inf.complementation}; there are however a few cases of object participles used in complement clauses, §\ref{sec:object.participles.complement}), and manner converbs (§\ref{sec:inf.converb}). 

Distinguishing between subject participles and \forme{kɯ-} infinitives in Japhug is less straightforward, unlike in other Gyalrong languages such as Tshobdun for instance, \citealt{jackson14morpho}, since even stative verbs take the \forme{kɤ-} infinitive in complement clauses (§\ref{sec:inf.complementation}). The only clear contexts where \forme{kɯ-} infinitives do occur is that of citation forms (§\ref{sec:inf.citation}) and complement clauses containing impersonal modal verbs (§\ref{sec:inf.complementation}). Converbs in \forme{kɯ-} are analyzed as infinitives rather than subject participle because they are only attested with stative verbs or other verbs taking the \forme{kɯ-} infinitives (§\ref{sec:inf.converb}).
 
\subsubsection{Associated motion, polarity and orientation prefixes on infinitives}  \label{sec:infinitives.other.prefixes}
With the exception of the construction in §\ref{sec:inf.exist} and some converbial uses (§\ref{sec:inf.converb}), \forme{kɤ-} infinitives do not take possessive prefixes. However, like participles, they are compatible with associated motion (\ref{ex:CWkAXtW}), negative prefixes (\ref{ex:mAkACe.mAkhW}) and imperfective orientation prefixes (\ref{mWpjWkAlhoR.ftCaka}), with combinations of two prefixes.

\begin{exe}
\ex \label{ex:CWkAXtW}
\gll a-mgɯr ɲɯ-mŋɤm tɕe ɕɯ-kɤ-χtɯ mɯ́j-cʰa-a \\
\textsc{1sg}.\textsc{poss}-back \textsc{sens}-hurt \textsc{lnk} \textsc{transloc}-\textsc{inf}-buy \textsc{neg}:\textsc{sens}-can-\textsc{1sg} \\
\glt `My back hurts and I cannot go to  buy (apples).' (conversation, 30-04-2018)
\end{exe}
 
\begin{exe}
\ex \label{ex:mAkACe.mAkhW}
\gll   rɟɤlpu fka ɕti tɕe, mɤ-kɤ-ɕe mɤ-kʰɯ \\
king order be.\textsc{affirm}:\textsc{fact} \textsc{lnk} \textsc{neg}-\textsc{inf}-go \textsc{neg}-be.possible:\textsc{fact} \\
\glt `This is the king's order, (I) have no choice but to go.' (Norbzang 2005, 12)
\end{exe}
 
\begin{exe}
\ex \label{mWpjWkAlhoR.ftCaka}
 \gll  tɤ-se mɯ-pjɯ-kɤ-ɬoʁ ftɕaka tu-βze-a tu-mdzoz-a pɯ-ŋu ma, \\
 \textsc{indef}.\textsc{poss}-blood \textsc{neg}-\textsc{ipfv}-\textsc{inf}-come.out manner \textsc{ipfv}-make[III]-\textsc{1sg} \textsc{ipfv}-avoid-\textsc{1sg} \textsc{pst}.\textsc{ipfv}-be \textsc{lnk} \\
\glt `I avoided by all means to let the blood come out.' (24-pGArtsAG, 57)
 \end{exe}
 
Negative infinitives take the allomorph \forme{mɤ-} (as in \ref{ex:mAkACe.mAkhW}), unless an imperfective orientation prefix is present, in which case the negative is \forme{mɯ-} as in (\ref{mWpjWkAlhoR.ftCaka}).

Impersonal and stative infinitives in \forme{kɯ-} are only attested with the negative prefix \forme{mɤ-}.

\subsubsection{Ambiguity}  \label{sec:velar.inf.ambiguity}
Aside from the homophony between infinitives and participles discussed in §\ref{sec:infinitives.participles}, another type of ambiguity occurs with verbs selecting the orientation prefix `east', whose A form is \forme{kɤ-} (§ XXX). Intransitive verbs in imperative singular and perfective third singular forms (for instance \forme{kɤ-rŋgɯ} `he laid down') and transitive verbs without stem alternation in imperative singular (\forme{kɤ-tsʰi} `drink!', § XXX) have forms that are homophonous with the corresponding infinitives ( \forme{kɤ-rŋgɯ} `to lie down', \forme{kɤ-tsʰi}  to drink'), but these cases are never really ambiguous, as it is trivial to distinguish between a finite verb form and a non-finite one, for instance by changing from singular to dual or plural. 

For instance, in a particular context, if a form such as \forme{kɤ-rŋgɯ} can be changed to the corresponding plural \forme{kɤ-rŋgɯ-nɯ} (which can be either perfective \textsc{pfv}-lie.down-\textsc{pl} `they laid down' or imperative `lie down!'), it is possible to conclude that this \forme{kɤ-rŋgɯ} is necessarily finite (since non-finite verb forms in Japhug never take indexation affixes, § XXX) and cannot be an object participle or an infinitive.

\subsubsection{Citation form} \label{sec:inf.citation}
The infinitive is the preferred form to refer to a verb in metalinguistic discourse as a citation form. In this context stative and impersonal verbs consistently take the \forme{kɯ-} prefix as in (\ref{ex:mAkWBdi}), and the rest of verbs the \forme{kɤ-} prefix, as in (\ref{ex:kAnARarphAB}) and (\ref{ex:WkAlAjme.pjWkACthWz}). Even in citation form, the infinitive verb can take orientation (\ref{ex:WkAlAjme.pjWkACthWz}) and polarity prefixes (\ref{ex:mAkWBdi}).  

\begin{exe}
\ex  \label{ex:mAkWBdi}
 \gll ɯnɯnɯ tɕe tɕe [ɯ-tɯ-tʂɯβ mɤ-kɯ-βdi] tu-kɯ-ti ŋu \\ 
 \textsc{dem} \textsc{lnk} \textsc{lnk} \textsc{3sg.poss-nmlz:action}-sew \textsc{neg-inf:stat}-be.good  \textsc{ipfv-genr}:A-say be:\textsc{fact}  \\
\glt `People call this `badly sewn'.'  (12-kAtsxWb-zh, 12)
\end{exe}

\begin{exe}
\ex \label{ex:kAnARarphAB}
 \gll  pjɯ-sɯ-ʁndi tɕe pjɯ-sɯ-sat tɕe nɯ kóʁmɯz nɤ cʰɯ-nɯtsɯm ɲɯ-ra tɕe nɯnɯ [kɤ-nɤʁarphɤβ] tu-kɯ-ti ŋu  \\
 \textsc{ipfv-caus}-hit[III]  \textsc{lnk} \textsc{ipfv-caus}-kill \textsc{lnk} \textsc{dem} only.after \textsc{lnk} \textsc{ipfv:downstream}-take.away \textsc{sens}-have.to \textsc{lnk} \textsc{dem} \textsc{inf}-strike.with.wings \textsc{ipfv-genr}:A-say be:\textsc{fact}  \\
 \glt `It strikes it and kills it (with its wings) and only then takes it away. This is called \japhug{kɤ-nɤʁarphɤβ}{strike with one's wings}.' (150819 RarphAB-zh, 11)
\end{exe}

The infinitive is commonly used in metalinguistic discussions about collocations, and in those cases can appear together with intransitive subjects (\ref{ex:mAkWBdi}) or objects (\ref{ex:WkAlAjme.pjWkACthWz}). In the latter, the focus is on the noun \japhug{ɯ-kɤlɤjme}{head upside down} (§\ref{sec:dvandva.coll}), the verb \japhug{ɕtʰɯz}{turn towards} being present only because it is selected by \forme{ɯ-kɤlɤjme}.

\begin{exe}
\ex \label{ex:WkAlAjme.pjWkACthWz}
 \gll  ɯ-mŋu nɯ pa pjɯ́-wɣ-ɕtʰɯz tɕe nɯ [ɯ-kɤlɤjme pjɯ-kɤ-ɕtʰɯz] tu-kɯ-ti ŋu. \\
\textsc{3sg}.\textsc{poss}-mouth \textsc{dem} down \textsc{ipfv}:\textsc{down}-\textsc{inv}-turn.towards \textsc{lnk} \textsc{dem}  \textsc{3sg.poss}-head.upside.down   \textsc{ipfv}:\textsc{down}-\textsc{inf}-turn.toward \textsc{ipfv-genr}:A-say be:\textsc{fact}  \\
\glt `One turns the mouth (of the container) downwards, it is called `to turn upside down'.' (30-macha, 68)
\end{exe}

The infinitive is not the only possible choice as citation form; some speakers sometimes cite a generic form (especially with the imperfective, as \forme{pjɯ́-wɣ-ɕtʰɯz} in \ref{ex:WkAlAjme.pjWkACthWz}) or other finite forms (even imperatives).

In the topical position, the infinitive of stative verbs can be neutralized to the \forme{kɤ-} form, as in (\ref{ex:kArZi}) with an adjectival stative verb and (\ref{ex:kAtu.nW.tu}) with the existential verb \japhug{tu}{exist}.

\begin{exe}
\ex \label{ex:kArZi}
 \gll kɤ-rʑi ri pjɤ-rʑi, 	  \\
 \textsc{inf}-be.heavy also \textsc{ifr.ipfv}-be.heavy \\
 \glt `As for being heavy, (the old man) was heavy.'  (140511 xinbada-zh, 138)
\end{exe}

\begin{exe}
\ex \label{ex:kAtu.nW.tu}
 \gll ɯʑo rkɯn, ri kɤ-tu nɯ tu  \\
 \textsc{3sg} be.rare:\textsc{fact} \textsc{lnk} \textsc{inf}-exist \textsc{dem} exist:\textsc{fact} \\
 \glt `It is rare, but as for existing, it does exist.' (140511 qamtsWrmdzu, 17)
\end{exe}

\subsubsection{Complementation}    \label{sec:inf.complementation}
The most common function of the velar infinitive is to build complement clauses. Apart from a handful of well-identified cases (§\ref{sec:subject.participle.complementation}), all \forme{kɤ-} prefixed verb forms in complement clauses are infinitive rather than object participles (using the criteria in §\ref{sec:infinitives.participles}). 

Velar infinitives occur with a great variety of auxiliaries (§ XXX), transitive and intransitive lexical verbs (§ XXX), and also a few nouns (§ XXX). Few verbs however require the infinite in complement clauses. Some complement-taking verbs like \japhug{cʰa}{can} occur with either infinitival complement clauses as in (\ref{ex:kAnWCe.YWCti}) or finite complement clauses (§ XXX), and other verbs like \japhug{rɲo}{experience} are compatible with both velar infinitives and bare infinitives (§\ref{sec:dental.inf}). The constraints on co-reference between the subject of the matrix verb and the participants of the complement clause is treated in \ref{sec:velar.inf.coreference}.

\begin{exe}
\ex \label{ex:kAnWCe.YWCti}
 \gll  ɯ-kɤχcɤl ɯ-ʁrɯ nɯ a-nɯ-pʰɯt tɕe kɤ-nɯ-ɕe cʰa ɲɯ-ɕti \\
 \textsc{3sg}.\textsc{poss}-top.of.the.head \textsc{3sg}.\textsc{poss}-horn \textsc{dem} \textsc{irr}-\textsc{pfv}-take.out \textsc{lnk} \textsc{inf}-\textsc{vert}-go can:\textsc{fact} \textsc{sens}-be.\textsc{affirm} \\
\glt `If one takes out the horn on his head, he will be able to go back (to heavens).' (divination 2005, 91)
\end{exe}
 
 Complex velar infinitive forms with polarity, orientation and associated motion prefixes are attested in complement clauses, as in (\ref{ex:CWkACar.pWrYota}) (see also for instance \ref{ex:CWkAXtW} and \ref{ex:mAkACe.mAkhW} in §\ref{sec:infinitives.other.prefixes}).
 
\begin{exe}
\ex \label{ex:CWkACar.pWrYota}
 \gll aʑo ɕɯ-kɤ-ɕar pɯ-rɲo-t-a. \\
 \textsc{1sg} \textsc{transloc}-\textsc{inf}-search \textsc{pfv}-experience-\textsc{tr}:\textsc{pst}-\textsc{1sg} \\
 \glt `I did go to search (for Amanita caesarea).'  (22-BlamajmAG, 32)
\end{exe}

Stative verbs, when occurring in a complement clause, generally take the \forme{kɤ-} infinitive, as in example  (\ref{ex:rYo}) and (\ref{ex:kAscit}). The main verb of the complement clauses in these examples have the \forme{kɤ-} infinitive, even though both \japhug{tu}{exist} and \japhug{scit}{be happy} are stative verbs and have a citation form with the \forme{kɯ-} prefix.

\begin{exe}
\ex \label{ex:rYo}
\gll   a-rŋɯl kɤ-tu pɯ-rɲo-t-a \\
\textsc{1sg.poss}-money \textsc{inf}-exist \textsc{pst:ipfv}-experience-\textsc{pst:tr-1sg} \\
\glt `I used to have money'. (elicited)
\end{exe}

\begin{exe}
\ex \label{ex:kAscit}
 \gll  kɤ-scit pjɤ-ŋgrɯ ɲɯ-ŋu  \\
 \textsc{inf}-be.happy \textsc{ifr}-succeed \textsc{sens}-be \\
 \glt `She succeeded in being happy.' (150818 muzhi guniang, 6)
 \end{exe} 
 
The conversion to \forme{kɤ-} infinitive only applies to stative verbs, not to  impersonal modal verbs such as \japhug{ra}{have to, need}. When the latter occur in a complement clause, as in example (\ref{ex:kAndza.kWra}), they always have the \forme{kɯ-} prefix.

\begin{exe}
\ex \label{ex:kAndza.kWra}
\gll  smɤn kɤ-ndza kɯ-ra pɯ-rɲo-t-a  \\ 
medicine \textsc{inf}-eat \textsc{inf:impers}-have.to  \textsc{pst:ipfv}-experience-\textsc{pst:tr-1sg} \\
\glt `I used to have to take medicine.' (elicited)
\end{exe} 
 
The velar infinitive is also found in some adnominal complement clauses (criteria for distinguishing between adnominal complements and relative clauses is presented in § XXX), for instance in the collocation of  the nouns \japhug{ftɕaka}{manner} or \japhug{kowa}{manner} with the transitive verb \japhug{βzu}{make}, as (\ref{ex:kAndza.kowa.tWwGBzu}) below (see also \ref{ex:kongzhi} and \ref{mWpjWkAlhoR.ftCaka} above). 

In infinitive complement clauses, the complement verb lacks person/number indexation. However, in the case of the verbs that require coreference between the core arguments of the matrix clause and those in the complement clause, the person and number of the subject (and sometimes also the object) of the complement clauses are reflected on the indexation of the verb in the matrix clause (§ XXX). For instance, in (\ref{ex:kAti.mWjspea}), both the infinitive \forme{kɤ-ti} `to say' and the matrix verb \forme{mɯ́j-spe-a} `I am not able' share the same \textsc{1sg} subject and \textsc{3sg} object (\forme{ɯ-mdoʁ} `its colour').  

\begin{exe}
\ex \label{ex:kAti.mWjspea}
\gll nɯ ɯ-mdoʁ nɯ aj [kɤ-ti] mɯ́j-spe-a \\
\textsc{dem} \textsc{3sq}.\textsc{poss}-colour \textsc{dem} \textsc{1sg} \textsc{inf}-say \textsc{neg}:\textsc{sens}-be.able[III]-\textsc{1sg} \\
\glt `I don't know how to say (describe) its colour.' (06-qaZmbri, 05)
  \end{exe} 
  
  This is also the case with noun+verb collocations taking complement clauses: in (\ref{ex:kAndza.kowa.tWwGBzu}), the verb form \forme{tɯ́-wɣ-βzu} reflects the \textsc{3pl}\fl{}\textsc{2sg} configuration of the infinitive \forme{kɤ-ndza} `to eat' in the complement clause. 
  
  \begin{exe}
\ex \label{ex:kAndza.kowa.tWwGBzu}
\gll   a-rɟit ra nɯ-ɣi-nɯ ɕti tɕetʰa, kɤ-ndza kowa tɯ́-wɣ-βzu ɕti \\
 \textsc{1sg}.\textsc{poss}-children \textsc{pl} \textsc{vert}-come:\textsc{fact}-\textsc{pl} be.\textsc{affirm}:\textsc{fact} in.a.moment \textsc{inf}-eat manner 2-\textsc{inv}-make:\textsc{fact}  be.\textsc{affirm}:\textsc{fact} \\
 \glt `My children are coming back soon and will try to eat you.' (Norbzang2012, 300-301)
 \end{exe} 
 
With velar infinitive complements, some auxiliary verbs take the orientation prefix selected by the verb in the complement clause, a phenomenon studied in §XXX.
 
\subsubsection{Coreference restrictions} \label{sec:velar.inf.coreference}
Coreference restrictions between the arguments of complement clauses with \forme{kɤ-} infinitives and their matrix clauses differ from verb to verb, and three cases can be distinguished.

First, in the case of impersonal verbs such as \japhug{ra}{have to, need} (see §XXX), there is no argument coreference between the matrix clause and the complement clause.

Second, with a few transitive complement-taking verbs such as the transitive \japhug{spa}{be able} (§XXX) and the intransitive \japhug{nɤz}{dare} (§XXX), coreference between the subject of the matrix clause and that of the complement clause is required.

Third, for most verbs taking infinitives (like the semi-transitive \japhug{rga}{like} or the transitive \japhug{rɲo}{experience}),  the subject of the matrix clauses can be coreferential to either the subject of an intransitive verb (\ref{ex:kAnWrAGo.rganW}), the subject of a transitive verb (\ref{ex:kAnArtoXpjAt.pWrgaa}) and also the object (\ref{ex:YWrganW} and \ref{ex:kAmtsWG.P}).

 \begin{exe}
   \ex   \label{ex:kAnWrAGo.rganW} 
\gll tsuku tɕe [kɤ-nɯrɤɣo] wuma ʑo rga-nɯ tɕe \\
some \textsc{lnk} \textsc{inf}-sing really \textsc{emph} like:\textsc{fact-pl}  \textsc{lnk} \\
\glt `Some people like to sing.' (26-kWrNukWGndZWr, 104)  (S=S)
\end{exe}  
 
   \begin{exe}
   \ex   \label{ex:kAnArtoXpjAt.pWrgaa} 
\gll aʑo qajɯ nɯra kɤ-nɤrtoχpjɤt pɯ-rga-a tɕe  	\\
  	\textsc{1sg} bugs \textsc{dem} \textsc{pl} \textsc{inf}-observe \textsc{pst.ipfv}-like-\textsc{1sg} \textsc{lnk}  \\
 \glt `I liked to observe bugs.' (26-quspunmbro, 15) (A=S)
     \end{exe}  
 
  \begin{exe}
   \ex   \label{ex:YWrganW} 
\gll maka tu-kɤ-nɤjoʁjoʁ, tu-kɤ-fstɤt nɯ ɲɯ-rga-nɯ  \\
at.all \textsc{ipfv-inf}-flatter \textsc{ipfv-inf}-praise \textsc{dem} \textsc{ipfv}-like-\textsc{pl} \\
\glt `They like to be flattered or praised.' (140427 yuanhou-zh, 53) (P=S)
    \end{exe}  
    
In (\ref{ex:kAmtsWG.P}), the personal name \ipa{χpɤltɕɯn} takes the ergative due to the fact that it is the transitive subject of \japhug{rɲo}{experience}; however, in this example, the infinitive \forme{kɤ-mtsɯɣ} `to bite' present in the first clause is elided in the second one. The complete clause without elision would \forme{χpɤltɕɯn kɯ \textbf{kɤ-mtsɯɣ} pjɤ-rɲo} `Dpalcan has been stung (by a wasp) before'. Note here that \ipa{χpɤltɕɯn} is the subject of the matrix and the object of the complement clause at the same time, with the ergative flagging of the matrix clause taking over the absolutive marking expected in the complement clause.

      \begin{exe}
   \ex   \label{ex:kAmtsWG.P} 
\gll aʑo [kɤ-mtsɯɣ] mɯ-pɯ-rɲo-t-a ri, χpɤltɕɯn kɯ pjɤ-rɲo  \\
\textsc{1sg} \textsc{inf}-bite \textsc{neg-pfv}-experience-\textsc{pst:tr-1sg} but p.n \textsc{erg}  \textsc{ifr}-experience \\
\glt `I have never been stung (by a wasp), but Dpalcan has.' (26-ndzWrnaR, 19) (P=A)
    \end{exe}  

The subject of the matrix clause can even be coreferential with the possessor of the intransitive subject, as in example (\ref{ex:kAmNAm}), where the non-overt subject should be \japhug{a-xtu}{my belly}, as in (\ref{ex:kAmNAm.pWrYota}) (see §XXX on this type of collocations).
 
 \begin{exe}
\ex \label{ex:kAmNAm}
\gll aʑo pɯ-xtɕɯ\redp{}xtɕi-a ʑo ri tɯxtɤŋɤm nɯ-atɯɣ-a tɕe, nɯ kɤ-mŋɤm pɯ-rɲo-t-a \\
\textsc{1sg} \textsc{pst:ipfv-emph}\redp{}be.small-\textsc{1sg} \textsc{emph} \textsc{loc} dysentery \textsc{pfv}-meet-\textsc{1sg} \textsc{lnk} \textsc{dem} \textsc{inf}-hurt \textsc{pfv}-experience-\textsc{1sg} \\
\glt `When I was very small, I had dysentery, (my belly) ached.'  (24-pGArtsAG, 121)
\end{exe}

\begin{exe} 
\ex \label{ex:kAmNAm.pWrYota}
\gll a-xtu kɤ-mŋɤm pɯ-rɲo-t-a. \\
\textsc{1sg}.\textsc{poss}-belly \textsc{inf}-hurt \textsc{pfv}-experience-\textsc{pst}:\textsc{tr}-\textsc{1sg} \\
\glt `I have had belly ache.' (elicited)
\end{exe} 

Other types of complement clauses differ from velar infinitive clauses by their constraints on coreference (see §\ref{sec:bare.inf.coreference}).

\subsubsection{Doubly prefixed velar infinitives with negative existential verb} \label{sec:inf.exist}
The infinitive in \forme{kɤ-} can take two prefixes in a construction combining the negative existential verb \japhug{me}{not exist} with a verb in the infinitive prefixed with a series B orientation prefix and a possessive prefix coreferent with the subject,\footnote{The assertion in \citet[228]{jacques16complementation} that infinitives cannot take possessive prefixes is thus wrong.} meaning `have no way to X, be completely unable to X', as in (\ref{ex:ndZijukACe}) and (\ref{ex:WpjWkAnWZWB}). Note that since in both of these examples, the verbs are intransitive and lack an object participle, the \forme{kɤ-} form can only be analyzed as an infinitive here. This construction is also possible with transitive verbs, in which case the possessive prefix corresponds to the transitive subject.

\begin{exe}
\ex \label{ex:ndZijukACe}
\gll tɕe ndʑi-ju-kɤ-ɕe pjɤ-me \\
\textsc{lnk} \textsc{3du}.\textsc{poss}-\textsc{ipfv}-\textsc{inf}-go \textsc{ifr}.\textsc{ipfv}-not.exist \\
\glt `They could not go.' (150908 menglang-zh, 46)
\end{exe}

\begin{exe}
\ex \label{ex:WpjWkAnWZWB}
\gll tɯ-rʑaʁ nɯ ɯ-pjɯ-kɤ-nɯʑɯβ pjɤ-me matɕi, \\
one-night \textsc{dem} \textsc{3sg}.\textsc{poss}-\textsc{ipfv}-\textsc{inf}-sleep \textsc{ifr}.\textsc{ipfv}-not.exist \textsc{lnk} \\
\glt `He could not sleep the whole night, because...' (150831 BZW kAnArRaR, 12)
\end{exe}

Stative verbs and impersonal verb also present a \forme{kɤ-} infinitive in this construction, as illustrated by (\ref{ex:kongzhi}), where the stative verb \japhug{mbro}{be high} occurs with the prefix \forme{kɤ-}, arguably because it is turned into a dynamic verb `become higher' by the presence of the imperfective orientation prefix.

 \begin{exe}
\ex \label{ex:kongzhi}
\gll  a-<xuetang> ɯ-tɯ-mbro <kongzhi> tu-βze-a ŋu, mɯ-tu-kɤ-mbro ftɕaka tu-βze-a ŋu \\
\textsc{1sg.poss}-blood.sugar \textsc{3sg.poss-nmlz:action}-be.high control \textsc{ipfv}-do[III]-\textsc{1sg} be:\textsc{fact} \textsc{neg-ipfv-inf}-be.high manner \textsc{ipfv}-do[III]-\textsc{1sg} be:\textsc{fact} \\
\glt `I do whatever I can to prevent my blood sugar being too high'. (conversation, 15/12/05)
\end{exe}

A derived construction involves the causative \japhug{ɣɤme}{cause not to exist, suppress} with doubly prefixed infinitives to `make it impossible for X to Y' as in (\ref{ex:apjWkAnWZWB.naGAme}).

\begin{exe}
\ex \label{ex:apjWkAnWZWB.naGAme}
\gll a-pjɯ-kɤ-nɯʑɯβ na-ɣɤ-me \\
\textsc{1sg}.\textsc{poss}-\textsc{ipfv}-\textsc{inf}-sleep \textsc{pfv}:3\fl{}3'-\textsc{caus}-not.exist \\
\glt `He made me unable to sleep.' (elicited)
\end{exe}

%cʰa a-ku-kɤ-tsʰi na-ɣɤme
There is a variant of this construction with imperfective subject participles in \forme{kɯ-} instead of infinitives (§\ref{sec:subject.participle.complementation}).
 

\subsubsection{Converbial use}    \label{sec:inf.converb}
Velar infinitives in \forme{kɤ-} can also be used as converbs, with a variety of meanings.


In (\ref{ex:mAkAtWG.kuwWGsWjGatandZi}), the converbial clause with the negative infinitive \forme{mɤ-kɤ-ɤtɯɣ} `not meeting' refers not the manner in which the action of the main verb took place, but rather to the non-realization of another event. 

\begin{exe}
\ex \label{ex:mAkAtWG.kuwWGsWjGatandZi}
\gll [a-mu a-wi ni mɤ-kɤ-ɤtɯɣ] kú-wɣ-sɯ-jɣat-a-ndʑi. \\
\textsc{3sg}.\textsc{poss}-mother \textsc{3sg}.\textsc{poss}-grand.mother  du \textsc{neg}-\textsc{inf}-meet \textsc{ipfv}:\textsc{east}-\textsc{inv}-\textsc{caus}-go.back-\textsc{1sg}-\textsc{du} \\
\glt `(My uncles) forced me to go back (to school) without having met my mother and my grandmother.' (2010-Dpalcan-09, 54)
\end{exe}

Converbial infinitival clause can contain an object or a semi-object which does not belong to the main clause, as \forme{a-mu a-wi ni} `my mother and my grandmother' in (\ref{ex:mAkAtWG.kuwWGsWjGatandZi}). The dual on \forme{kú-wɣ-sɯ-jɣat-a-ndʑi} refers to the subject (`the uncles', mentioned in the previous clause), not the mother and the grandmother.

Infinitives in \forme{kɯ-} (impersonal or stative) also occur as converbs, though these forms could in principle also be analyzed as subject participles (\ref{sec:subject.participles}). For instance, in (\ref{ex:mAkWmbrAt.YWrAma}) \forme{mɤ-kɯ-mbrɤt} `without stop' is considered to be an impersonal infinitive serving as a manner converb, but it could be possible to propose an alternative analysis as a \forme{kɯ-} subject participle `the one which does not stop' used adverbially.  The analysis as infinitives however better accounts for the fact that only the verbs whose infinitive is in \forme{kɯ-} have converbial forms in \forme{kɯ-}.

\begin{exe}
\ex \label{ex:mAkWmbrAt.YWrAma}
 \gll nɯ maka mɤ-kɯ-mbrɤt ʑo ɲɯ-rɤma ɲɯ-ɕti tɕe,  \\
 \textsc{dem} at.all  \textsc{neg}-\textsc{inf:impers}:S/A-\textsc{acaus}:break \textsc{emph} \textsc{ipfv}-work \textsc{sens}-be.\textsc{affirm} \textsc{lnk} \\
 \glt `It works without stopping at all.' (26-GZo, 69)
\end{exe}

Adjectives are also used adverbially, and can even become degree adverbs, as \japhug{kɯ-xtɕɯ\redp{}xtɕi}{a little} from \japhug{xtɕi}{be small} in examples such as (\ref{ex:kWxtCWxtCi.wxti}).

\begin{exe}
\ex \label{ex:kWxtCWxtCi.wxti}
 \gll βʑɯ sɤz kɯ-xtɕɯ\redp{}xtɕi wxti. \\
 mouse \textsc{comp} \textsc{inf:stat}-\textsc{emph}\redp{}be.small be.big:\textsc{fact} \\
 \glt `It is a little bigger than a mouse.' (21-GzWLa, 4)
\end{exe}

The existential verbs also occur in converbial use. For instance, \japhug{tu}{exist} in infinitive form \forme{kɯ-tu} following a noun or a pronoun can mean `in the presence of...' as in (\ref{ex:Zara.kWtu.Zo}).

\begin{exe}
\ex \label{ex:Zara.kWtu.Zo}
\gll ʑara kɯ-tu ʑo to-sɤrɯru tɕe, \\
\textsc{3pl} \textsc{inf:stat}-exist \textsc{emph} \textsc{ifr}-compare \textsc{lnk} \\
\glt `He compared (his testimony with theirs) in their presence.' (150909 xifangping-zh, 155)
\end{exe}


 
\subsubsection{Lexicalized velar infinitives}    \label{sec:lexicalized.velar.inf}
Lexicalized velar infinitives in \forme{kɤ-} found in some compounds, though some cases are could alternatively be analyzed as lexicalized object participles, and are treated in §\ref{sec:lexicalized.object.participle}.

The delocutive expression \japhug{ŋɤtɕɯkɤti,kʰɯ}{obey to everything} provides an unambiguous example of lexicalized velar infinitive. The compound \forme{ŋɤtɕɯkɤti} combines the pronoun \japhug{ŋotɕu}{where} in \textit{status constructus} form \forme{ŋɤtɕɯ-} with the  infinitive \forme{kɤ-ti} of the verb \japhug{ti}{say}, and is exclusively used in collocation with \japhug{kʰɯ}{be possible, agree}, as in (\ref{ex:NAtCWkAti}).\footnote{The causative \japhug{ŋɤtɕɯkɤti,sɯkʰɯ}{cause to obey to everything} also exists.} 
This expression originates presumably from a phrase such as `agree (\forme{kʰɯ}) to whatever (\forme{ŋotɕu}) X says (\forme{kɤ-ti})'. However, it should be noted that the pronoun \japhug{tɕʰi}{what}, not \japhug{ŋotɕu}{where} is used in Japhug in the free-choice indefinite construction meaning `whatever' as in examples (\ref{ex:tChi.pWnWNWNu}) to (\ref{ex:tChi.kWstWstua}) in §\ref{sec:interrogative.indef}). The form \forme{kɤ-ti} in any case was originally the complement of the verb \japhug{kʰɯ}{be possible, agree}, which takes infinitival complements (§ XXX), and thus is not analyzable as a former object participle.

 \begin{exe}
\ex \label{ex:NAtCWkAti}
\gll  ɯ-tɕɯ kɯβde nɯra wuma ʑo ŋɤtɕɯkɤti pjɤ-kʰɯ-nɯ  \\
\textsc{3sg}.\textsc{poss}-son four \textsc{dem}:\textsc{pl} really \textsc{emph} obey.to.everything(1) \textsc{ifr.ipfv}-obey.to.everything(2)-\textsc{pl} \\
\glt `His four sons were very obedient.' (140508 benling gaoqiang de si xiongdi-zh, 15)
\end{exe} 

\subsection{Bare infinitives} \label{sec:bare.inf}
Bare infinitives are formed by combining the stem 1 of the verb with a possessive prefix coreferential with the object of the complement clause, as in example (\ref{ex:Wmto.mWpWrYota}). Bare infinitives are not attested with orientation, polarity or associated motion prefixes.

\begin{exe} 
\ex \label{ex:Wmto.mWpWrYota}
\gll nɤʑo kɯ-fse a-ŋkʰor nɯ ɯ-mto mɯ-pɯ-rɲo-t-a \\
you \textsc{nmlz:stat}-be.like \textsc{1sg.poss}-subject \textsc{top} \textsc{3sg.poss}-\textsc{bare.inf:}see \textsc{neg-pfv}-experience-\textsc{pst:tr-1sg} \\
\glt `I never saw anyone like you among my subjects.' (28-smAnmi,  393)
\end{exe} 

Intransitive verbs do not have bare infinitives. Complement-taking verbs selecting bare infinitives for transitive verbs either take dental \forme{tɯ-} infinitives (§\ref{sec:dental.inf}) or velar infinitives (§\ref{sec:velar.inf}) when occurring with intransitive verbs.

\subsubsection{Complement clauses} \label{sec:bare.inf.complement}
Bare infinitives only occur in complement clauses. Apart from the aspectual verb \japhug{rɲo}{experience, have already} mentioned above in (\ref{ex:Wmto.mWpWrYota}), bare infinitives are found with two categories of complement-taking verbs.

First, they are compatible with some phasal verbs (§XXX) such as \japhug{ʑa}{begin}, \japhug{sɤʑa}{begin}, \japhug{stʰɯt}{finish} and \japhug{jɤɣ}{finish} as in (\ref{ex:WtsxWB.thWjAG}).

\begin{exe} 
\ex \label{ex:WtsxWB.thWjAG}
\gll  tɕe nɯnɯ tɯ-ŋga nɯ ɯ-tʂɯβ tʰɯ-jɤɣ tɕe tɕe, \\
\textsc{lnk} \textsc{dem} \textsc{indef}.\textsc{poss}-clothes \textsc{dem} \textsc{3sg}.\textsc{poss}-\textsc{bare}.\textsc{inf}:sew \textsc{pfv}-finish \textsc{lnk} \\
\glt `When one has finished sewing the clothes,' (30-tWNga, 29)
\end{exe} 

Second, they are found with causative verbs derived from adjectives such as \forme{ɣɤ-βdi} `repair, cause to be good, do X well' (\forme{ɣɤ-} causative of \japhug{βdi}{be well, be good}, §XXX) as in (\ref{ex:WnApWpa.tuGABdinW}).

\begin{exe} 
\ex \label{ex:WnApWpa.tuGABdinW}
\gll lu-ji-nɯ qʰe ɯ-nɤpɯpa tu-ɣɤ-βdi-nɯ, ɯ-ɣli ra ku-sɤpe-nɯ qʰe, cʰɯ-do ɲɯ-cʰa. \\
\textsc{ipfv}-plant-\textsc{pl} \textsc{lnk} \textsc{3sg}.\textsc{poss}-\textsc{bare}.\textsc{inf}:take.care \textsc{ipfv}-\textsc{caus}-be.good \textsc{3sg}.\textsc{poss}-dung \textsc{pl} \textsc{ipfv}-do.well-\textsc{pl} \textsc{lnk} \textsc{ipfv}-be.fibrous \textsc{sens}-can \\
\glt `(Now people) plant (pumpkin also in higher areas), (if) they take good care of it and put enough fertilizers, it can become fibrous (so that it can be sowed for the next year).' (140522 kAmYW tWji, 47)
\end{exe} 

The same set of verbs take complement clauses with dental infinitives when the verb of the complement is intransitive (\ref{sec:dental.inf.complement}).

No verb requires a bare infinitive: all complement-taking verbs selecting it are either alternatively compatible with velar infinitives (§\ref{sec:inf.complementation}) or serial verb constructions (§XXX) when used with transitive verbs.


\subsubsection{Coreference restrictions} \label{sec:bare.inf.coreference}
Bare infinitives and velar infinitives strongly differ as to their coreference restrictions. When the verb \japhug{rɲo}{experience} occurs with velar infinitives, the subject of the matrix clause can be coreferential with either the subject, the object or even the possessor of the intransitive subject of the complement clause (§\ref{sec:velar.inf.coreference}).

The ambiguity between transitive subject or object coreference is particularly clear with the verb \japhug{nɤkʰu}{invite to one's home as a guest} (see examples \ref{ex:kAnAkhu1} and \ref{ex:kAnAkhu2}), as with this verb both arguments are equal in terms of volition and control.

\begin{exe}
\ex  \label{ex:kAnAkhu1}
\gll ɯʑo kɯ kɤ-nɤkʰu pɯ-rɲo-t-a  \\
\textsc{3sg} \textsc{erg} \textsc{inf}-invite \textsc{pfv}-experience-\textsc{pst:tr-1sg} \\
\glt `I have been to his house as a guest.'  (= `He has invited me to come to his house as a guest and I came.') (P=A)
\ex  \label{ex:kAnAkhu2}
\gll ɯʑo kɤ-nɤkʰu pɯ-rɲo-t-a  \\
\textsc{3sg}  \textsc{inf}-invite \textsc{pfv}-experience-\textsc{pst:tr-1sg} \\
\glt `He has been to my house as a guest.' (= `I have invited him to come to my house as a guest and he came.') (A=A)
\end{exe}

In the case of bare infinitives, on the other hand, the subjects of the matrix and complement clause must be identical, but the object of the matrix clause can however be neutralized to third person.

In example (\ref{ex:nAkhu1}), the shared subject (referring to the host) is \textsc{3sg}. The verb of the matrix clause takes the complement clause as a \textsc{3sg} object (hence the verb takes the 3$\rightarrow$3' form without \textsc{1sg} marking), while the verb of the complement clause takes a \textsc{1sg} object (referring to the guest), marked by the possessive prefix \forme{a-}.\footnote{In the English translation, the \textsc{1sg} is rendered as a subject, because translating  \forme{a-nɤkʰu pa-rɲo} as `He has invited me' would be inexact, as this English sentence does not imply that the \textsc{1sg} did attend the invitation. }

\begin{exe}
\ex  \label{ex:nAkhu1}
\gll a-nɤkʰu pa-rɲo \\
\textsc{1sg.poss-bare.inf:}invite \textsc{pfv:3$\rightarrow$3'}-experience \\
\glt `I have been to his house as a guest.' (= `He has invited me to come to his house as a guest and I came.')
\ex  \label{ex:nAkhu2}
\gll ɯʑo ɯ-nɤkʰu pɯ-rɲo-t-a  \\
\textsc{3sg}  \textsc{3sg.poss-bare.inf:}invite \textsc{pfv}-experience-\textsc{pst:tr-1sg} \\
\glt `He has been to my house as a guest.' (= `I have invited him to come to my house as a guest and he came.')
\end{exe}

This generalization is observed for all transitive verbs taking bare infinitive complement clauses. However, the intransitive impersonal verb \japhug{jɤɣ}{finish} takes the bare infinitive clause as intransitive subject, and remains in third person singular regardless of the subject and object of the complement clause, as in (\ref{ex:Wti.tojAG}), where although the subject of the complement clause is third person plural, no plural marker can appear on \forme{jɤɣ}.

\begin{exe}
\ex \label{ex:Wti.tojAG}
\gll nɯra ɯ-ti to-jɤɣ tɕe \\
\textsc{dem:pl} \textsc{3sg.poss-bare.inf}:say \textsc{ifr}-finish \textsc{lnk}\\
\glt `After having finished saying that, (they went to the park)' (140515 congming de wusui xiaohai-zh, 15)
\end{exe}

The bare infinitives also occur in adnominal complement clauses with the noun \japhug{ɯ-tsʰɯɣa}{shape, manner}, as in (\ref{ex:bare.inf.noun}).

\begin{exe}
\ex \label{ex:bare.inf.noun}
\gll ndʑi-mi ɯ-tsʰoʁ ɯ-tsʰɯɣa nɯra wuma ʑo naχtɕɯɣ-ndʑi.   \\
\textsc{3du.poss}-foot \textsc{3sg}-\textsc{bare.inf:}attach.to \textsc{3sg.poss}-form \textsc{dem:pl} very \textsc{emph} be.the.same:\textsc{fact}-\textsc{du}  \\
\glt `The way their feet (of fleas and crickets) touch the ground is very similar.' (26-mYaRmtsaR, 17)
\end{exe}


\subsection{Dental infinitives} \label{sec:dental.inf}
Dental infinitives (glossed as `second infinitives' \textsc{inf}:II) are built by prefixing \forme{tɯ-} with the verb stem. Dental infinitives occur with intransitive verbs, including dynamic (\ref{ex:tWrJaR.koZandZi}) and stative verbs (\ref{ex:tArNi.YoZa}), including semi-transitive verbs, but are not attested with transitive verbs, which take bare infinitives or velar infinitives instead.

\begin{exe} 
\ex \label{ex:tWrJaR.koZandZi}
\gll tɯ-rɟaʁ ko-ʑa-ndʑi \\
\textsc{inf:II}-dance \textsc{ifr}-start-\textsc{du} \\
\glt `They started to dance.' (140504 huiguniang, 147)
\end{exe} 

When the stem begins in \forme{a-} (§XXX), regular vowel fusion occurs, resulting in the surface form \ipa{tɤ-} as in (\ref{ex:tArNi.YoZa}).

\begin{exe} 
\ex \label{ex:tArNi.YoZa}
\gll si nɯ daltsɯtsa nɯ tɯ-ɤrŋi ɲo-ʑa tɕe \\
tree \textsc{dem} slowly \textsc{dem} \textsc{inf}:II-be.green \textsc{ifr}-start \textsc{lnk} \\
\glt `The tree slowly started to become green.' (divination 2003, 110)
\end{exe} 

However, a few morphologically transitive verbs used in complex predicates referring to weather phenomena, in particular \japhug{lɤt}{throw} and \japhug{βzu}{make, do} do take \forme{tɯ-} infinitives as in (\ref{ex:tWlAt.pjAZa}).  Note that in these complex predicates, the light verbs \japhug{lɤt}{throw} and \japhug{βzu}{make, do}, although transitively conjugated, cannot take an overt subject marked with the ergative, and only have one argument.
 
\begin{exe}
\ex  \label{ex:tWlAt.pjAZa}
\gll tɯ-mɯ kɯ-wxtɯ\redp{}wxti ʑo tɯ-lɤt pjɤ-ʑa \\
\textsc{indef.poss}-sky \textsc{nmlz:S/A-emph}\redp{}be.big \textsc{emph} \textsc{inf}-throw \textsc{ifr}-start \\
\glt `A big rain started.' (150819 haidenver-zh, 104)
\end{exe}

Dental infinitives are probably historically related to degree nominals (§\ref{sec:degree.nominals}) and action nominals (§\ref{sec:action.nominals}), which however do not display the same transitivity restrictions.

\subsubsection{Polarity prefixes}
Dental infinitives are only compatible with polarity prefixes (as in example \ref{ex:mAtWrga}), and cannot take orientation prefixes, associated motion or possessive prefixes.

\begin{exe}
\ex  \label{ex:mAtWrga}
\gll qaɟy ɯ-me nɯnɯ, tɕendɤre kʰro mɤ-tɯ-rga to-ʑa\\
fish \textsc{3sg.poss}-daughter \textsc{dem} \textsc{lnk} a.lot \textsc{neg-inf:II}-like \textsc{ifr}-start \\
\glt `He started not liking the mermaid that much.' (150819 haidenver-zh, 154)
\end{exe}


\subsubsection{Complement clauses} \label{sec:dental.inf.complement}
Dental \forme{tɯ-} infinitives are only attested in complement clauses, and are only found with the verbs that select a bare infinitive (§\ref{sec:bare.inf.complement}) when the verb in the complement clause is transitive. This complementary distribution suggests that bare infinitive and dental infinitives could be treated as two variants of the same grammatical category.

Dental infinitives are more often attested with phasal verbs, in particular \japhug{ʑa}{start} and \japhug{sɤʑa}{start} as in (\ref{ex:tWskAm.pjAsAZa}).

\begin{exe}
\ex \label{ex:tWskAm.pjAsAZa}
\gll mtsʰu nɯ cʰɯmcʰɯm ʑo, tɕe, tɯ-skɤm pjɤ-sɤʑa \\
lake \textsc{dem} \textsc{idph}:II:slowly.retreating \textsc{emph} \textsc{lnk} \textsc{inf}:II-be.dry \textsc{ifr}-start \\
\glt `The (water of the) lake started to retreat slowly.' (nyima wodzer 2003, 105)
\end{exe}

Non-phasal verbs selecting bare infinitive such as \japhug{rɲo}{experience} never occur with the dental infinitive in the corpus, but such forms can be elicited, as in (\ref{ex:tWGi.pWrYota}). In the corpus, intransitive complements of the verb \japhug{rɲo}{experience} rather velar infinitives; it is also possible  in (\ref{ex:tWGi.pWrYota}) to replace the dental infinitive \forme{tɯ-ɣi} with a velar infinitive \forme{kɤ-ɣi}.

\begin{exe}
\ex \label{ex:tWGi.pWrYota}
\gll  mbarkʰom tɯ-ɣi pɯ-rɲo-t-a  \\
pl.n. \textsc{inf}:II-come \textsc{pfv}-experience-\textsc{pst}:\textsc{tr}-\textsc{1sg} \\
\glt `I came to Mbarkham before.' (elicited)
\end{exe}

Like other intransitive verbs, antipassivized transitive verbs can take a dental infinitive (as in \ref{ex:tWrArAt.paZa}), unlike the base verb from which they are derived (as in \ref{ex:WrAt.paZa}, where a bare infinitive is used instead).

\begin{exe}
\ex \label{ex:tWrArAt.paZa}
\gll tɯ-rɤ-rɤt pa-ʑa \\
\textsc{inf}:II-\textsc{apass}-write \textsc{pfv}:3\fl{}3'-start \\
\glt `He started writing.' (elicited)
\end{exe}

\begin{exe}
\ex \label{ex:WrAt.paZa}
\gll tɤscoz ɯ-rɤt pa-ʑa \\
letter \textsc{3sg}.\textsc{poss}-\textsc{bare.inf:}write \textsc{pfv}:3\fl{}3'-start \\
\glt `He started writing the/a letter.' (elicited)
\end{exe}

%tɯ-tʂɯβ nɯ thɯ-jɤɣ ri tɕe tɕe chɯ́-wɣ-pɣaʁ.
\subsubsection{Coreference restrictions} \label{sec:dental.inf.coreference}
As in the case of bare infinitives (§\ref{sec:bare.inf.coreference}), there is obligatory coreference between the subject of the matrix verb and the intransitive subject of the \ipa{tɯ-} infinitival complement, except for the  impersonal verb \japhug{jɤɣ}{finish}. When the verb of the matrix clause is transitive and that of the complement clause intransitive, there is a conflict in case assignment on their common subject: the transitive verb requires the ergative on third person subjects, while the intransitive one precludes it. In most cases of this type, such as example (\ref{ex:tWNke}), the common subject is in the absolutive, following the intransitive verb of the complement clause. However in a few examples such as (\ref{ex:tWnWrAGo}), the shared argument takes the ergative following the transitive matrix verb (the phasal verb \japhug{ʑa}{begin} is morphologically transitive).  

\begin{exe}
\ex \label{ex:tWNke}
\gll [<xinbada> nɯ tɕe li tɯ-ŋke] to-ʑa \\
Sinbad \textsc{dem} \textsc{lnk} again  \textsc{inf}-walk \textsc{ifr}-begin \\
\glt `Sinbad started to walk again.' (140511 xinbada-zh, 217)
\end{exe}

\begin{exe}
\ex \label{ex:tWnWrAGo}
\gll pɣɤtɕɯ nɯ kɯ [nɯɕɯmɯma ʑo tɯ-nɯrɤɣo] cʰɤ-ʑa \\
bird \textsc{dem} \textsc{erg} immediately \textsc{emph} \textsc{inf}-sing \textsc{ifr}-begin \\
\glt `The bird immediately started to sing.' (140514 huishuohua de niao-zh, 221)
\end{exe}

\section{Degree nominals} \label{sec:degree.nominals}

\subsection{Degree construction}
\subsection{Complement clauses}
%ɯ-tɯ-sɯɕqraʁ ɯβrɤ-tɤ-ɣɤtɕhom-a ma

% tu-mbro nɯ, ɯ-tɯ-mbro nɯ tu-orɕo tsa tɕe tɕe,
 %paʁ nɯ ɯ-βri 
%nɤki ɯ-rme kɯ-me ʑo ɯ-tshɯɣa ɲɯ-fse ma
%ɯ-tɯ-tshu chondɤre ɯ-tɯ-rʁom ɯ-tshɯɣa nɯ ʑo ɲɯ-fse
\section{Action nominals} \label{sec:action.nominals}
There are two types of action nominals in Japhug: action nominals in \forme{tɯ-}, a very productive formation which is the main topic of this section, and the bare action nominals, treated in §\ref{sec:bare.action.nominals}.

Action nominals in \forme{tɯ-} can be built from both intransitive and transitive verbs. They differ from both participles and infinitives in that the argument structure of the verb is lost and the transitivity contrast neutralized, and cannot take objects or oblique arguments other than possessors like normal alienably possessed nouns.


The action nominal has three potential meanings.  First, it can refer to the action itself, for instance \japhug{tɯji}{planting and sowing} from the verb \japhug{ji}{plant}, not to be confused with the related IPN \japhug{tɯ-ji}{field}), \japhug{tɯɣɟaβ}{action of churning} from \japhug{ɣɟaβ}{churn}, \japhug{tɯrɟaʁ}{dance (n)} from \japhug{rɟaʁ}{dance} (intransitive verb), \japhug{tɯsi}{death} from \japhug{si}{die} or \japhug{tɯmu}{fear} from \japhug{mu}{fear}.

Second, it can mean an object affected by, or resulting from the action. This is also the case with some intransitive verbs, for instance \forme{tɯqioʁ} from the intransitive \japhug{qioʁ}{vomit} which can either mean `the action of vomiting' or `vomitus' or \forme{tɯɕkʰo} from \japhug{ɕkʰo}{dry in the sun} which can either be `action of drying in the sun' or `grain that are dried in the sun' (see \ref{ex:tWCkho.chWBze} in §\ref{sec:action.nominal.Bzu}).

Third, it can also refer to the way an action is performed, for instance \forme{tɯ-rɤt} which means `style of writing, way of writing' as in (\ref{ex:er.WtWrAt}). This function is probably the direct historical origin of the degree nominals (§\ref{sec:degree.nominals}).

\begin{exe}
\ex \label{ex:er.WtWrAt}
\gll  tɕe <er> ɯ-tɯ-rɤt tsa ɲɯ-fse ri, nɯ sthɯci mɯ-ɲɯ-ŋgɤɣ \\
\textsc{lnk} two \textsc{3sg}.\textsc{poss}-\textsc{nmlz}:\textsc{action}-write a.little \textsc{sens}-be.like \textsc{lnk} \textsc{dem} so.much \textsc{neg}-\textsc{sens}-\textsc{acaus}:warp \\
\glt `(The constellation) looks a little bit like the way `two' is written, but not as curved.' (29-LAntshAm, 55)
\end{exe} 
 
Stative verbs can also have \forme{tɯ-} nominals, expressing abstract nouns, for instance \japhug{tɯtʂaŋ}{justice} from \japhug{tʂaŋ}{be fair}.
%tɯkon
%tɯkrɤz

 \subsection{Collocation with \japhug{βzu}{make}} \label{sec:action.nominal.Bzu}
The \forme{tɯ-} action nominals can be used in collocation with the verb \japhug{βzu}{make} to express habitual actions, especially actions occupying a considerable amount of time. For instance \forme{tɯ-taʁ cʰɯ-βze} `she was weaving' in (\ref{ex:tWtAR.chWBze}) and \forme{tɯ-ɕkʰo pjɯ́-wɣ-nɯ-βzu} `(when) we dry (grains in the field)' refer to actions taking place every day (and taking up most of the day) during a certain time period.

\begin{exe}
\ex \label{ex:tWtAR.chWBze}
\gll tɤ-tɕɯ nɯ lu-rɤ-ji,  tɕe tɕʰeme nɯ kɯ li tɯ-taʁ cʰɯ-βze tɕe, mɯntoʁ ra tu-tʂɯβ qʰe ku-nɯ-rɤʑi-ndʑi pjɤ-ŋu.  \\
\textsc{indef}.\textsc{poss}-boy \textsc{dem} \textsc{ipfv}-\textsc{apass}-plant \textsc{lnk} girl \textsc{dem} \textsc{erg} again \textsc{nmlz}:\textsc{action}-weave \textsc{ipfv}-make[III] \textsc{lnk} flower \textsc{pl} \textsc{ipfv}-sew \textsc{lnk} \textsc{ipfv}-\textsc{auto}-stay-\textsc{du} \textsc{ipfv}.\textsc{ifr}-be \\
\glt `The boy was working in the fields, the girl was weaving and doing embroidery, they were living like that.' (150828 donglang, 137)
\end{exe}
%tɕheme ra kɯ tɯ-nbraʁ tu-sɤftɕaka-nɯ.

As mentioned above, action nominals either refer to the action itself or an object affected by the action. While the collocation with \japhug{βzu}{make} probably derives from the first meaning of the action nominals, in some examples it cannot be excluded that the second meaning also intervenes. For instance, in (\ref{ex:tWCkho.chWBze}), \forme{tɯɕkʰo+βzu} can be understood as `to do the action of drying in the sun', but also as `to do the action related to grains that are dried in the sun', since the grains in question are directly referred to in the next clause as \forme{kɤ-ɣndʑɯr ɯ-spa} `(grains) to be ground'.

\begin{exe}
\ex \label{ex:tWCkho.chWBze}
\gll tɕendɤre tɯ-ɕkʰo pjɯ́-wɣ-nɯ-βzu qhe, nɯ kɤ-ɣndʑɯr ɯ-spa nɯ pjɯ́-wɣ-ɕkʰo tɕe nɯ-rom kóʁmɯz cʰɯ́-wɣ-ndʑɯr ra tɕe, ɯnɯnɯra ɣɯ-tu-mɯrki tu-ndze ŋu. \\
\textsc{lnk} \textsc{nmlz}:\textsc{action}-dry.in.the.sun \textsc{ipfv}-\textsc{inv}-auto-make \textsc{lnk} \textsc{dem} \textsc{nmlz}:P-grind \textsc{3sg}.\textsc{poss}-material \textsc{dem} \textsc{ipfv}-\textsc{inv}-dry.in.the.sun \textsc{lnk} \textsc{pfv}-be.dry only.after \textsc{ipfv}-\textsc{inv}-grind have.to:\textsc{fact} \textsc{lnk} \textsc{dem}:\textsc{pl} \textsc{cisloc}-\textsc{ipfv}-steal[III] \textsc{ipfv}-eat[III] be:\textsc{fact} \\
\glt `When we do drying in the sun, when we dry the grains that are to be ground (one grinds them only after they have dried), it comes, steals them and eats them.' (22-CAGpGa, 66-68)
\end{exe}

\subsection{Simultaneous} 
The simultaneous action nominal is built by prefixing an additional \forme{tɯ-} to the base form of the action nominal, resulting in a double \forme{tɯ-tɯ-} prefixed form. It is found in collocation with the verb \japhug{βzu}{make}, with both intransitive (\ref{ex:tWtWnWmdar.pjABzundZi}, \ref{ex:tWtWCe.jaBzundZi}) or transitive (\ref{ex:tWtWtsxaB.Zo.pjABzu}, \ref{ex:tWtWtsxWB.kABzuta}) verbs. In this construction orientation prefix on \japhug{βzu}{make} is the one that is lexically selected by the verb in simultaneous action nominal form, for instance the orientation `down' \forme{pjɤ-} of the verb \japhug{nɯmdar}{jump} in (\ref{ex:tWtWnWmdar.pjABzundZi}).

%tɯŋga tɯtɯtʂɯβ kɤβzuta

With an intransitive verb, the simultaneous constructions occur with a dual or plural subject, and means that several individuals referred do an action together at the same moment, as \forme{tɯ-tɯ-nɯmdar} `jumping together' in (\ref{ex:tWtWnWmdar.pjABzundZi}). The auxiliary \forme{βzu} can even take the indefinite orientation prefixes \forme{jɤ-} \forme{ja-} as in (\ref{ex:tWtWCe.jaBzundZi}) when occurring with a motion verb.

\begin{exe}
\ex \label{ex:tWtWnWmdar.pjABzundZi}
\gll  mtsʰu ɯ-ŋgɯ tɯ-tɯ-nɯmdar pjɤ-βzu-ndʑi tɕe, \\
 lake \textsc{3sg}.\textsc{poss}-inside \textsc{simult}-\textsc{nmlz}:\textsc{action}-jump \textsc{ifr}:\textsc{down}-make-\textsc{du} \textsc{lnk} \\
 \glt `They (the two of them) had jumped together at the same time into the lake.' (nyima wodzer 2003, 104)
\end{exe}

\begin{exe}
\ex \label{ex:tWtWCe.jaBzundZi}
\gll ʑɤni tɯ-tɯ-ɕe ja-βzu-ndʑi \\
\textsc{3du} \textsc{simult}-\textsc{nmlz}:\textsc{action}-go \textsc{pfv}:3\fl{}3'-make-\textsc{du} \\
\glt `They went at the same time.' (elicited)
 \end{exe}

 When used with transitive verbs, this construction is only found with dual or plural object, and implies that several entities were subjected to the action together at the same time, as \forme{tɯ-tɯ-tʂaβ} `causing to roll down together' in (\ref{ex:tWtWtsxaB.Zo.pjABzu}), or that several object end up being tied together as result of the action as with \forme{tɯ-tɯ-tʂɯβ} `sewing together' in (\ref{ex:tWtWtsxWB.kABzuta}).
 
\begin{exe}
\ex \label{ex:tWtWtsxaB.Zo.pjABzu}
\gll  pri cʰo jlɤkrɯ nɯra tɯ-tɯ-tʂaβ ʑo pjɤ-βzu  \\
 bear \textsc{comit} \textsc{dem}:\textsc{pl}  basket  \textsc{simult}-\textsc{nmlz}:\textsc{action}-cause.to.roll.down \textsc{emph} \textsc{ifr}:\textsc{down}-make \\
\glt `He made the basket with the bear (in it) roll down together.' (2011-13-qala, 53) 
\end{exe}

\begin{exe}
\ex \label{ex:tWtWtsxWB.kABzuta}
\gll   tɯ-ŋga tɯ-tɯ-tʂɯβ kɤ-βzut--a \\
\textsc{indef}.\textsc{poss}-clothes \textsc{simult}-\textsc{nmlz}:\textsc{action}-sew \textsc{pfv}-make:\textsc{pst}:\textsc{tr}-\textsc{1sg} \\
\glt `I sewed the clothes together.' (elicited)
  \end{exe}
  
\subsection{Lexicalized action nominals}  
  %tɯfɕɤl nɯtɯfɕɤl
  %tɯɲɟoʁ
%tɯpɣaʁ
%tɯpu
%tɯrma
%tɯsqa
%tɯtshi
% tɯqartsɯ kick (n) > sɯqartsɯ, base verb lost

  \subsection{Inalienably possessed bare action nominals} \label{sec:bare.action.nominals}
\section{Other deverbal nouns}


\subsection{Nominalization \forme{-z} suffix} \label{sec:z.nmlz}
Japhug has three inalienably possessed nouns derived from verbs by means of a nominalizing \forme{-z}, two of which take the indefinite possessive prefix \forme{tɤ-} (§\ref{sec:inalienably.possessed}).

The first one, \japhug{tɤ-rkuz}{parting present}, a biactantial IPN whose possessor corresponds to the recipient (§\ref{sec:biactantial.ipn}), comes from \japhug{rku}{put in}. The etymological relationship  between these two words is obvious when the use of  \forme{rku} in the sense of `give as a present to take away' (put in someone's luggage) is considered, as in (\ref{ex:kWki.nArkuz.Nu}).

\begin{exe}
\ex \label{ex:kWki.nArkuz.Nu}
\gll tɕe tó-wɣ-z-rɤŋgat tɕe, tɕendɤre nɯnɯ kɯ, iɕqʰa nɯ,  tɯ-ci tɯ-tɤ-ste to-rku. tɕe `kɯki nɤ-rkuz ŋu' to-ti. \\
\textsc{lnk} \textsc{ifr}-\textsc{inv}-\textsc{caus}-prepare.to.leave \textsc{lnk} \textsc{lnk} \textsc{dem} \textsc{erg} \textsc{filler} \textsc{dem} \textsc{indef}.\textsc{poss}-water \textsc{one}-\textsc{indef}.\textsc{poss}-bladder \textsc{ifr}-put.in \textsc{lnk} \textsc{dem}.\textsc{prox} \textsc{2sg}.\textsc{poss}-present be:\textsc{fact} \textsc{ifr}-say \\
\glt `He prepared his departure, and gave him a bladder full of water to take with him, and said `this is your departing present'.' (28-smAnmi.txt, 264-265)
\end{exe}


The verb \japhug{rku}{put in} can even occur with its derived noun \japhug{tɤ-rkuz}{parting present} in the \textit{figura etymologica} construction in (\ref{ex:arkuz.tarku}) (the verb \japhug{βzu}{make} can alternatively be used instead of \japhug{rku}{put in}).

\begin{exe}
\ex \label{ex:arkuz.tarku}
\gll a-me kɯ a-rkuz rŋɯl ta-rku \\
\textsc{1sg}.\textsc{poss}-daughter \textsc{erg} \textsc{1sg}.\textsc{poss}-present money \textsc{pfv}:3\fl{}3'-put.in \\
\glt `My daughter gave me some money (as present for my departure) (elicited)
\end{exe}

The second one, \japhug{ɯ-mɲoz}{preparation}, only occurs to express `preparation' (used in a collocation with the verb \japhug{βzu}{make} as in \ref{ex:WmYoz.tuwGBzu}), is derived from the transitive verb \japhug{mɲo}{prepare}, itself the irregular causative of \japhug{ɲo}{be prepared} (§ XXX). A \forme{-z}-less variant \japhug{ɯ-mɲo}{preparation} is also attested.

\begin{exe}
\ex \label{ex:WmYoz.tuwGBzu}
\gll  pjɯ-ɲɟo ɕɯŋgɯ tɕe ɯ-mɲoz tú-wɣ-βzu ra \\
\textsc{ipfv}-be.damaged before \textsc{lnk} \textsc{3sg}.\textsc{poss}-preparation \textsc{ipfv}-\textsc{inv}-make have.to:\textsc{fact} \\
\glt `One has to take preparations before it gets damaged.' (elicited)
\end{exe}

The third one, the noun \japhug{tɤ-scoz}{letter, writing} possibly originates from the verb \japhug{sco}{see off, accompany}, though it might be a loan from Situ (\citealt{jacques03s.houzhui}). 

The \forme{-z} nominalizing suffix, though rare in Japhug, is of Sino-Tibetan origin. In Situ, the corresponding nominalizing \forme{-s} suffix is much more common (\citealt{jacques03s.houzhui}), and Tibetan and Chinese have traces of a cognate suffix (\citealt{jacques16ssuffixes}).


\subsection{Nominalization \forme{ɣ-/x-} prefix} \label{sec:G.nmlz}
A handful of nouns, most of them IPNs, are derives from intransitive verbs by means of a velar prefix  \forme{ɣ}- or \forme{x}- (harmonizing in voicing with the initial consonant of the stem). These nouns are lexicalized ancient subject participles (§\ref{sec:subject.participles}) which underwent the same phonological change as that observed with the velar animal class prefix (§\ref{sec:velar.class.prefix}), that has a syllabic allomorph \forme{kɯ-} and reduced allomorphs \forme{ɣ}- or \forme{x}-. 

The reduced \forme{ɣ}- / \forme{x}- prefix only derives nouns from intransitive verbs with monosyllabic stems, without consonant clusters. 


\begin{table}[H]
\caption{Irregular subject nominalizations in \forme{ɣ}- and \forme{x}-} \label{tab:irregular.nmlz} \centering
\begin{tabular}{llll}
\lsptoprule
Noun & Base verb & Reference \\
\midrule
\japhug{ɣndʑɤβ}{disastrous fire} & \japhug{ndʑɤβ}{burn} \\
\japhug{ɯ-ɣɲaʁ}{disaster}& \japhug{ɲaʁ}{be black} \\
\japhug{ɯ-ɣɲɟɯ}{orifice} & \japhug{ɲɟɯ}{be opened} \\
\japhug{ɯ-xso}{empty, normal} &\japhug{so}{be empty} &  \ref{sec:property.nouns} \\
\japhug{ɯ-ɣrom}{dried thing} & \japhug{rom}{be dry} \\
\lspbottomrule
\end{tabular}
\end{table}

The noun \japhug{tɯ-xpa}{one year}, although derived from the verb \japhug{pa}{pass X years} and having an additional \forme{x-} element, does not belong to this category, see  §\ref{sec:num.prefix.paradigm.history} and §\ref{sec:CN.verbs}.


\section{Converbs}
\subsection{Gerund} \label{sec:gerund}
%nɤʑo laχtɕha kɤ-fkur tu-tɯ-ŋke ɲɯ-ŋu tɕe,
\subsection{Purposive} \label{sec:purposive.converb}


With intransitive and semi-transitive verbs, the possessive prefix is coreferent to the intransitive subject, as in (\ref{ex:aYWsAstWstu}).

\begin{exe}
\ex \label{ex:aYWsAstWstu}
\gll  tɕe nɯnɯ a-ɲɯ-sɤ-stɯ\redp{}stu nɯra tu-nɤme pjɤ-ŋu \\
\textsc{lnk} \textsc{dem} \textsc{1sg}-\textsc{ipfv}-\textsc{purp}-believe \textsc{dem}:\textsc{pl} \textsc{ipfv}-make[III] \textsc{ifr}.\textsc{ipfv}-be \\
\glt `He was doing these things so that I would believe (in his predictions).' (150904 yaoshu-zh, 104)
\end{exe}

\subsection{Immediate} \label{sec:immediate.converb}

\section{Historical perspectives} \label{sec:nmlz.historical.perspectives}

\subsection{Velar non-finite prefixes} \label{sec:velar.nmlz.history}

\subsection{Sigmatic non-finite prefixes} \label{sec:sigmatic.nmlz.history}