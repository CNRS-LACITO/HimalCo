\chapter{The noun phrase}

\section{Postpositions and relator nouns}

\subsection{Absolutive} \label{sec:absolutive}
\subsubsection{Intransitive subject}
\subsubsection{Object}
\subsubsection{Semi-object}
\subsubsection{Theme}
\subsubsection{Essive} \label{sec:essive.abs}
%tsuku kɯ paʁndza ɲɯ-nɯ-phɯt-nɯ ɲɯ-ŋu ri,
\subsubsection{Locative adjunct}

\subsection{Ergative} \label{sec:erg.kW}
\subsubsection{Transitive subject} \label{sec:A.kW}
\subsubsection{Instrumental} \label{sec:instr.kW}
\subsubsection{Comparee marker}
\subsubsection{Partitive}

\subsection{Genitive} \label{sec:genitive}
\subsubsection{Possession}
\subsubsection{Beneficiary}
\subsection{Locative} \label{sec:locative}
\subsection{Standard marker} \label{sec:comparative} %\japhug{sɤz}{compare with}
\subsection{Exceptive} \label{sec:exceptive} %\japhug{ma}{apart from}

The exceptive \japhug{ma}{apart from} and its variants are required in the restrictive focus construction (§ \ref{sec:restrictive.focus}).

\subsection{Terminative} \label{sec:terminative}  %\japhug{mɤɕtʂa}{until}

\section{Relator nouns}
\subsection{Dative} \label{sec:dative} 

\section{Noun modifiers and determiners}
This section discusses all nouns modifiers and determiners except relative clauses (§ XXX) and complement clauses (§ XXX). 

\subsection{Demonstratives} \label{sec:demonstrative.determiners}

\subsection{Quantifiers}
\subsubsection{Universal quantifiers} \label{sec:universal.quant}
\subsubsection{Mid-scalar quantifier} \label{sec:tsuku}
(\ref{sec:partitive.pronouns})

\subsection{Indefinite and definite markers} \label{sec:indefinite.markers}

\subsubsection{Indefinite article} \label{sec:indef.article}
The form \japhug{ci}{one} has among its many functions (in addition to pronoun, numeral and adverb, see § \ref{sec:other.pro}, § \ref{sec:partitive.pronouns}, § \ref{sec:identity.modifier}, § \ref{sec:one.to.ten} and § XXX) that of singular indefinite article, as in (\ref{ex:ci.indef}) and (\ref{ex:ci.chAGi}). It is typically used to introduce a new referent in a story.

\begin{exe}
\ex \label{ex:ci.indef}
\gll tɕʰeme kɯ-mpɕɯ\redp{}mpɕɤr ci ɲɤ-nɯ-ɬoʁ \\
girl \textsc{nmlz}:S/A-\textsc{emph}\redp{}beautiful \textsc{indef} \textsc{ifr}-\textsc{auto}-come.out \\
\glt `A very beautiful girl appeared (out of it).' (The flood, 39)
\end{exe}

\begin{exe}
\ex \label{ex:ci.chAGi}
\gll tɕɤlo tɕe tɤ-tɕɯ ci cʰɤ-ɣi qʰe, \\
upstream \textsc{lnk} \textsc{indef}.\textsc{poss}-son \textsc{indef} \textsc{ifr}:\textsc{downstream}-come \textsc{lnk} \\
\glt `A boy came from upstream.' (2003-kWBRa, 41)
\end{exe}

Although \forme{ci} can be used as a partitive pronoun `one of them' (§ \ref{sec:partitive.pronouns}), as a postnominal determiner it does not have partitive meaning. To express a meaning such as `one of the boys', a CN such as \japhug{tɯ-rdoʁ}{one piece} is used instead (§ \ref{sec:ICN}). 

Note that when used as a prenominal modifier, \forme{ci} has a completely different (definite) meaning `the other X' (§ \ref{sec:identity.modifier}). 

There are no dual or plural indefinite articles in Japhug. The plural marker \forme{ra} can occur after the indefinite \forme{ci}, but with a vague associative meaning `and other things' as in (\ref{ex:ci.ra}).

\begin{exe}
\ex \label{ex:ci.ra}
 \gll  ndʑi-tɕɯ ci, ndʑi-me ci ra to-tu. \\
 \textsc{3du}.\textsc{poss}-son \textsc{indef}  \textsc{3du}.\textsc{poss}-girl \textsc{indef} \textsc{pl} \textsc{ifr}-exist \\
 \glt  `They$_{du}$ had a boy and a girl (etc).' (150827 tianluo-zh, 155)
\end{exe}

\subsubsection{Indefinite pronoun as modifier} \label{sec:indefinite}
The indefinite pronoun \japhug{tʰɯci}{something} (§ \ref{sec:thWci}) has marginal uses as a prenominal indefinite modifier, as in  (\ref{ex:thWci.laXCi}), (\ref{ex:thWci.WjmNo}) and (\ref{ex:laXtCha.ci.nWnW}) below. 

\begin{exe}
\ex \label{ex:thWci.laXCi}
\gll   tʰɯci laχɕi ci ɕ-pɯ-nɯ-βzjoz-nɯ tɕe, jɤ-ɕe-nɯ ra \\
something trade \textsc{indef} \textsc{transloc-imp-auto}-learn-\textsc{pl} \textsc{lnk} \textsc{imp}-go-\textsc{pl} have.to:\textsc{fact} \\
\glt `Go and learn some trade!' (140508 benling gaoqiang de si xiongdi-zh, 29)
 \end{exe}
 
 This construction arose perhaps from the use of the pronoun \forme{tʰɯci} as head of a postnominal relative clause with the verb \japhug{fse}{be like}, as illustrated by examples like (\ref{ex:thWci.kAnWsaXCWB}) or (\ref{ex:thWci.akAspa}) in § \ref{sec:thWci}. Turning the verb \japhug{fse}{be like} to a finite form as in (\ref{ex:thWci.WjmNo}) could cause the indefinite \forme{tʰɯci}, head of the relative in (\ref{ex:thWci.kAnWsaXCWB}), to be reanalyzed as the prenominal modifier of the immediately adjacent noun in (\ref{ex:thWci.WjmNo}).

 \begin{exe}
\ex \label{ex:thWci.kAnWsaXCWB}
\gll nɯra [tʰɯci [kɤ-nɯsaχɕɯβ kɯ-fse]] pɯ-ŋu wo.  \\
\textsc{dem}:\textsc{pl} something \textsc{inf}-have.a.contest \textsc{nmlz}:S/A-be.like \textsc{pst}.\textsc{ipfv}-be \textsc{sfp} \\
\glt `It was like a kind of contest.' (160706 thotsi, 16)
 \end{exe}
 
\begin{exe}
\ex \label{ex:thWci.WjmNo}
\gll [tʰɯci ɯ-jmŋo] ci ʑo pɯ-fse ri \\
something \textsc{3sg}.\textsc{poss}-dream one \textsc{emph} \textsc{pst}.\textsc{ipfv}-be.like \textsc{lnk} \\
\glt `It looked like (he had had) some dream.' (Lobzang2005, 74)
 \end{exe}
 
 
\subsubsection{The marking of definiteness} \label{sec:definiteness}
Japhug has no dedicated definite determiner, but  \forme{nɯ} and \forme{nɯnɯ}  as demonstrative determiners (\ref{sec:demonstrative.determiners}) and as topic markers (\ref{sec:topic}) and the prenominal aforementioned topic marker \forme{iɕqʰa} (§ \ref{sec:iCqha}) are generally used with definite referents.  

Example (\ref{ex:ci.joGi}) illustrates a typical example with the determiner \forme{nɯ}; the indefinite article \forme{ci} (§ \ref{sec:indef.article}) occurs in the first introduction of a new referent in the story as in the first clause of example (\ref{ex:ci.joGi}), but on the following occurrence of the same noun \forme{nɯ} is found.

\begin{exe}
\ex \label{ex:ci.joGi}
 \gll  tɕe qajdo ci jo-ɣi tɕe, tɕe qajdo nɯ kɯ `mo laz tu, pʰo laz me' to-ti. \\
 \textsc{lnk} crow \textsc{indef} \textsc{ifr}-come \textsc{lnk} \textsc{lnk} crow \textsc{dem} \textsc{erg} girl karma exist:\textsc{fact} boy karma not.exist:\textsc{fact} \textsc{ifr}-say \\
 \glt `A crow came. The crow said: `The girl will have chance, the boy won't.'' (28-qAjdoskAt, 8)
\end{exe}

However, although nouns phrases followed by \forme{nɯ} and \forme{nɯnɯ} more often than not denote definite referents, these determiners cannot be analyzed as definite articles, as noun phrases with \forme{nɯ} or \forme{nɯnɯ} can in certain cases have indefinite referents. 

A very clear case of use of \forme{nɯ} with an indefinite referent occurs on nouns serving as heads of head-internal relative clauses. A well-attested typological generalization is that in this type of relative clauses, definiteness marking is agrammatical (see \citealt{basilico96internally} and § XXX). In Khroskyabs, \citet[636]{lai17khroskyabs} reports that the definiteness marker \forme{=tə} is indeed not accepted on the head noun of head-internal relatives. In Japhug however, \forme{nɯ} does occur in such a syntactic context. For instance, in (\ref{ex:tAnmaR.nW.kW}), the head \forme{tɤ-nmaʁ nɯ kɯ} is subject of the participle \japhug{ɲɯ-kɯ-nɯ-ɕar}{looking for}, and is embedded in the participial relative clause indicated in brackets -- the presence of the ergative \forme{kɯ} precludes to analyze it as a post-nominal relative (§ XXX). From the meaning of the sentence the head \japhug{tɤ-nmaʁ}{husband} is clearly indefinite non-specific non-generic  (see \citealt[286-291]{lehmann84relativsatz}). The fact that it takes the marker \forme{nɯ} shows that this marker, unlike Khroskyabs \forme{=tə}, is not primarily marking definiteness.

\begin{exe}
\ex \label{ex:tAnmaR.nW.kW}
 \gll tɕeri [tɤ-nmaʁ nɯ kɯ ɯ-rʑaʁ kɯ-ɤntɕʰɯ ɲɯ-kɯ-nɯ-ɕar], aʁɤndɯndɤt tɤndɤɣri tu-kɯ-βzu pjɤ-tu.  \\
but  \textsc{indef}.\textsc{poss}-husband \textsc{dem} \textsc{erg} \textsc{3sg}.\textsc{poss}-wife  \textsc{nmlz}:S/A-be.many \textsc{ipfv}-\textsc{nmlz}:S/A-\textsc{auto}-search everywhere  illegitimate.child  \textsc{ipfv}-\textsc{nmlz}:S/A-make \textsc{ifr}.\textsc{ipfv}-exist \\
\glt `However there were husbands who were looking for several women and had illegitimate children.' (140427 tAndAGri, 3)
\end{exe}

Other cases of indefinite noun phrase with \forme{nɯ} are observed with left-dislocated topics. In example (\ref{ex:RnWz.nWnW}), we find a type of tail-head linkeage  (§ XXX) where both the noun phrase \japhug{spjaŋkɯ ʁnɯz}{two wolves} and the verb \japhug{ɲɤ-k-ɤtɯɣ-ci}{he met} are repeated; in the second occurrence, the noun phrase is topicalized and is followed by the topic marker \forme{nɯnɯ}, with a slight pause of hesitation. The determiner \forme{nɯnɯ} in this clause, unlike \forme{nɯ} in (\ref{ex:ci.joGi}), does not mark definiteness: that clause cannot be understood as `He met the two wolves'.

\begin{exe} 
\ex \label{ex:RnWz.nWnW} 
 \gll spjaŋkɯ ʁnɯz ɲɤ-k-ɤtɯɣ-ci. spjaŋkɯ ʁnɯz nɯnɯ, tɕendɤre ɲɤ-k-ɤtɯɣ-ci tɕe iɕqʰa, kɯ-rɤ-ntɕʰa nɯ wuma ʑo ɲɤ-mu. \\ 
 wolf two \textsc{ifr}-\textsc{evd}-meet-\textsc{evd}  wolf two \textsc{dem} \textsc{lnk} \textsc{ifr}-\textsc{evd}-meet-\textsc{evd} \textsc{lnk} the.aforementioned \textsc{nmlz}:S/A-\textsc{a.pass}:\textsc{n.hum}-kill \textsc{dem} really \textsc{emph} \textsc{ifr}-be.afraid \\ 
 \glt `He$_i$ (the butcher) met two wolves. He$_i$ met two wolves, and the butcher$_i$ was very much afraid.' (150902 liaozhai lang-zh, 7-8)
\end{exe}

The determiners \forme{nɯ} or \forme{nɯnɯ} are not attested in the corpus with the indefinite singular article \forme{ci} if both have scope on the same noun. In all cases with \forme{ci} followed by \forme{nɯ} (other than the identity pronoun in § \ref{sec:other.pro}), or of \forme{nɯ} followed by \forme{ci} in the corpus, they belong to different constituents. For instance, in (\ref{ex:ci.YAZGAsAphAr}), \forme{ci} is in adverbial use (`a little, once', see § XXX) and does not belong to the preceding noun phrase.  

\begin{exe}
\ex \label{ex:ci.YAZGAsAphAr}
\gll [tɕʰeme nɯ] ci ɲɤ-ʑɣɤ-sɤpʰɤr qʰe  \\
girl \textsc{dem} one \textsc{ifr}-\textsc{refl}-shake \textsc{lnk} \\
\glt `The girl shook herself.' (02-deluge2012, 125)
\end{exe}

In (\ref{ex:laXtCha.ci.nWnW}) although \forme{nɯnɯ} follows \forme{ci}, it has scope over the both preceding phrases, which are left-dislocated and followed by a pause.

\begin{exe}
\ex \label{ex:laXtCha.ci.nWnW}
\gll  kɤ-xtɕɤr tɕe nɯnɯ tɕe tɕe iɕqʰa, [[tʰɯci tɯmbri tɤ-ri kɯ-fse kɯ] [laχtɕʰa ci] nɯnɯ], ci kú-wɣ-sɯ-pa tɕe, kú-wɣ-xtɕɤr, \\
\textsc{inf}-attach \textsc{lnk} \textsc{dem} \textsc{lnk} \textsc{lnk} the.aforementioned something rope \textsc{indef}.\textsc{poss}-thread \textsc{nmlz}:S/A-be.like \textsc{erg} thing \textsc{indef} \textsc{dem} one \textsc{ipfv}-\textsc{inv}-\textsc{caus}-do \textsc{lnk} \textsc{ipfv}-\textsc{inf}-attach \\
\glt ``To attach' (means), to put together, attach something with something like a rope or a thread.'  (150902 kAxtCAr, 2-3)
\end{exe}

The aforementioned topic marker \forme{iɕqʰa} (§ \ref{sec:iCqha}) is almost always used with definite referents when prenominal, as in (\ref{ex:RnWz.nWnW}) above, and is the closest candidate fro analysis as a definiteness marker in Japhug. It does occur with non-specific generic referents as in (\ref{ex:lWlAmu}), including some that are very clearly indefinite as in (\ref{ex:lApWG}); note the absence of postnominal determiner \forme{nɯ} (\ref{ex:lApWG}).

\begin{exe}
\ex \label{ex:lWlAmu}
 \gll iɕqʰa lɯlɤmu nɯ tʰɯ-rɤpɯ tɕe tɕe ɯ-sŋi tɕe kɤ-nɯ-rŋgɯ nɯ stʰɯci mɯ́j-tsu ma ɯ-pɯ ra χse ɲɯ-ra tɕe, \\
 the.aforementioned female.cat \textsc{dem} \textsc{ipfv}-bear.young \textsc{lnk} \textsc{lnk} \textsc{3sg}.\textsc{poss}-day \textsc{lnk} \textsc{inf}-\textsc{auto}-lie.down \textsc{dem} so.much \textsc{neg}:\textsc{sens}-have.time.to \\
 \glt `A/the female cat (unlike male cats), when it had had youngs, does not have time to sleep during the day, as it has to feed its youngs.' (21-lWLU, 
\end{exe}

\begin{exe}
\ex \label{ex:lApWG}
\gll  iɕqʰa lɤpɯɣ ɯ-rɣi ʑo fse. \\
the.aforementioned radish \textsc{3sg}.\textsc{poss}-seed \textsc{emph} be.like:\textsc{fact} \\
\glt `It looks like a radish seed.' (hist-26-qro-fourmi, 61)
\end{exe}

In  (\ref{ex:laXtCha.ci.nWnW}), \forme{iɕqʰa}  also precedes two phrases involving indefinite referents, but  there is a marked pause, and this is a case of \forme{iɕqʰa} in its function as speech filler (see § XXX).

\subsubsection{Absence of definiteness marking}
Like many languages (\citealt[130]{creissels06sgit1}), Japhug uses bare nouns without any definiteness marking. Bare nouns are most often non-referential, as \japhug{tɕʰeme}{girl} in (\ref{ex:tCheme.tWtAtu}).

\begin{exe}
\ex \label{ex:tCheme.tWtAtu}
\gll ʁnaʁna tɕʰeme tɯ\redp{}tɤ-tu nɤ, kɤndʑɯsqʰaj tu-kɤ-sɯ-βzu \\
both girl \textsc{cond}\redp{}\textsc{pfv}-exist \textsc{lnk} \textsc{coll}:sister \textsc{ipfv}-\textsc{inf}-\textsc{caus}-make \\
\glt `If both of them have girls, let them be sisters.' (zrAntCW, 4)
\end{exe}

Bare nouns are less common with referential nouns (except in answers to questions), but examples can be found, as \japhug{qacʰɣa}{fox} in (\ref{ex:qachGa.kW}).

\begin{exe}
\ex \label{ex:qachGa.kW}
\gll qacʰɣa 	kɯ maχtɕɯ tɤ-tɯt-a nɯ mɤ-tɯ-ste ti ɲɯ-ŋu \\
fox \textsc{erg} I.told.you.so \textsc{pfv}-say[II]-\textsc{1sg} \textsc{dem} \textsc{neg}-2-do.like[III]:\textsc{fact} say:\textsc{fact} \textsc{sens}-be \\
\glt `The fox says: `You do not do as I told you to." (2003qachGa, 44)
\end{exe}

Personal names generally occur as bare nouns, without any definiteness marker as in (\ref{ex:WrJAnpanma}), but there are no constraints against co-occurrence of personal names with the determiner \forme{nɯ} either (see § \ref{sec:personal.names.modifiers}).

\begin{exe}
\ex \label{ex:WrJAnpanma}
\gll  ɯrɟɤnpanma kɯ ʁlaŋsaŋtɕhin ɯ-ɕki  \\
 Padmasambhava \textsc{erg} Gesar \textsc{3sg}-\textsc{dat} \\
\glt `Padmasambhava (told) Gesar.' (Gesar, 2)
\end{exe}

 \subsection{Topic markers} \label{sec:topic}
 
 \subsubsection{Aforementioned topic} \label{sec:iCqha}
 The marker \japhug{iɕqʰa}{the aforementioned}  is used on referents that have been previously mentioned in the same story, usually only a few sentences back. It is strictly prenominal. 
 
Example (\ref{ex:iCqha.aforementioned}) illustrates the most typical use of this marker. Sentence (\ref{ex:kAtWm}) introduces a new reference, \japhug{kɤtɯm}{ball of thread} marked with the indefinite article \forme{ci} (§ \ref{sec:indef.article}). Three clauses later in (\ref{ex:iCqha.kAtWm}), the same referent occurs again with two topic markers, the postnominal \textit{nɯ} and the prenominal \textit{iɕqʰa}.
 
 
\begin{exe}
\ex \label{ex:iCqha.aforementioned}
\begin{xlist}
\ex \label{ex:kAtWm}
\gll `razri \textbf{kɤtɯm} \textbf{ci} ɲɯ-ra, taqaβ ci ɲɯ-ra' to-ti qʰe   \\
 thread ball \textsc{indef} \textsc{sens}-need needle \textsc{indef} \textsc{sens}-need \textsc{ifr}-say \textsc{lnk}  \\
\glt `He told (Rgyabza) `I need a ball of thread and a needle.''  
\ex  
\gll tɕendɤre ɲɤ-kʰo qʰe,  \\
\textsc{lnk} \textsc{ifr}-give \textsc{lnk}   \\
\glt `She gave it to him.'
\ex 
\gll  tɕe ɯ-ndzɤtsʰi ka-tsɯm-nɯ nɯtɕu qʰe tɕe,   \\
 \textsc{lnk} \textsc{3sg}.\textsc{poss}-meal \textsc{pfv}:3\fl{}3'-bring-\textsc{pl} \textsc{dem}:\textsc{loc}  \textsc{lnk} \textsc{lnk}    \\
\glt `When they brought his meal,'
\ex \label{ex:iCqha.kAtWm}
\gll   \textbf{iɕqʰa} \textbf{kɤtɯm} \textbf{nɯ} ɯʑo kɯ ko-ndo, \\
   the.aforementioned ball \textsc{dem} \textsc{3sg} \textsc{erg} \textsc{ifr}-take \\
\glt `he took the ball of thread, and...' (Gesar 270-272)
\end{xlist}
\end{exe}
 
 
A systematic study of the use of the topic marker \forme{iɕqʰa} in Japhug must overcome two inherent difficulties. First, this topic marker is homophonous with (and historically related to) the speech filler \forme{iɕqʰa} (§ XXX) and with the adverb \japhug{iɕqʰa}{just now}, which can also precede noun phrases. Listening to the sound files can help distinguishing between the three, as the speech filler is always followed by a pause (and optionally by the demonstrative \forme{nɯ}), but there are still ambiguous sentences (see below). Second, \forme{iɕqʰa} occurs on nouns designating entities that the speaker considers to have been previously referred to in the conversation, even if they are not present in the same recording. 

For instance in (\ref{ex:iCqha.pɣArnoR}) the noun \japhug{pɣɤrnoʁ}{a species of fungus} is used with \forme{iɕqʰa}, although this name does not occur before in the same text; it was however mentioned the day before in another recording.

\begin{exe}
\ex \label{ex:iCqha.pɣArnoR}
\gll nɯ zdɯmqe cʰo iɕqʰa, pɣɤrnoʁ nɯni ndʑi-tsʰɯɣa wuma ʑo naχtɕɯɣ. \\
\textsc{dem} fungi.sp. \textsc{comit} the.aforementioned fungi.sp. \textsc{dem}:\textsc{du} \textsc{3du}.\textsc{poss}-form really \textsc{emph} be:identical:\textsc{fact} \\
\glt `The \forme{zdɯmqe} and the \forme{pɣɤrnoʁ} are very similar.' (23-mbrAZim, 82)
\end{exe}

 
The topic marker \forme{iɕqʰa} transparently comes from the adverb \japhug{iɕqʰa}{just now} (§ XXX). The pivot constructions that allowed reanalysis from adverb to prenominal topic marker are very probably headless relatives (§ XXX) as in  (\ref{ex:iCqha.tAtWta}), or complement clauses as in (\ref{ex:iCqha.ZnWzmWnmuta}). 

\begin{exe}
\ex \label{ex:iCqha.tAtWta}
 \gll  [iɕqʰa tɤ-tɯt-a] nɯ tú-wɣ-stu qʰe, \\
 just.now \textsc{ifr}-say[II]-\textsc{1sg} \textsc{dem} \textsc{ipfv}-\textsc{inv}-do.like \textsc{lnk} \\
\glt `One does as I just said, and...' (2002tWsqar, 139)
\end{exe}

\begin{exe}
\ex \label{ex:iCqha.ZnWzmWnmuta}
 \gll iɕqʰa [ʑ-nɯ-z-mɯnmu-t-a] nɯ mɯ-pjɤ-pe rcama.  \\
the.aforementioned  \textsc{transloc}-\textsc{pfv}-\textsc{caus}-\textsc{move}-\textsc{pst}:\textsc{tr}-\textsc{1sg} \textsc{dem} neg-\textsc{ifr}.\textsc{ipfv}-be.good \textsc{fsp} \\
\glt `It was probably not a good thing that I had moved them (as I said above).' (150819 kumpGa, 45)
 \end{exe}
 
 These sentences are still synchronically ambiguous in Japhug; in  (\ref{ex:iCqha.ZnWzmWnmuta}) the context makes it clear that \forme{iɕqʰa} is the topic marker (since the fact of having moved (the eggs) had been told a few sentences back) and not an adverb `just now' with a temporal reference in the past, as the meaning would be `it was probably not a good thing that I had just moved them' (an impossible interpretation in this context, since this sentence is an explanation why several eggs had not given chicks, several days after they had been brought to another place). However, extracted from the context, both interpretation would be equally possible for (\ref{ex:iCqha.ZnWzmWnmuta}), and correspond to two distinct syntactic structures.

With postnominal (§ XXX) or left-headed head-internal relative clauses (§ XXX) as in (\ref{ex:tWrpa.thafse}), \forme{iɕqʰa} can also be ambiguous. Since the adverb \japhug{iɕqʰa}{just now} can occur both before the object (\ref{ex:tWrpa.thWfseta}) or before the verb (\ref{ex:tWrpa.thWfseta2}) in an independent clause, a relative such as (\ref{ex:tWrpa.thafse}) can be either interpreted `the axe (mentioned above) that he had whetted' (with the topic marker \forme{iɕqʰa} outside of the relative clause, having scope on its head) and `the axe that he had just whetted' with the adverb \japhug{iɕqʰa}{just now} inside the relative clause.

 \begin{exe}
\ex \label{ex:tWrpa.thafse}
 \gll  tɕendɤre <luban> kɯ iɕqʰa [tɯrpa tʰa-fse] nɯ to-ndo tɕe, \\
 \textsc{lnk} p.n. \textsc{erg} the.aforementioned axe \textsc{pfv}:3\fl{}3'-whet \textsc{dem} \textsc{ifr}-take \textsc{lnk} \\
 \glt `Luban took the axe that he had whetted.' (150902 luban-zh, 90)
 \end{exe}

  \begin{exe}
  \ex 
  \begin{xlist}
\ex \label{ex:tWrpa.thWfseta}
 \gll   iɕqʰa tɯrpa tʰɯ-fse-t-a \\
just.now axe \textsc{pfv}-whet-\textsc{pst}:\textsc{tr}-\textsc{1sg} \\
\ex \label{ex:tWrpa.thWfseta2}
 \gll   tɯrpa  iɕqʰa tʰɯ-fse-t-a \\
 axe just.now \textsc{pfv}-whet-\textsc{pst}:\textsc{tr}-\textsc{1sg} \\
 \glt `I just whetted a/the axe.' (elicited)
 \end{xlist}
 \end{exe}

The use of \forme{iɕqʰa} as a topic marker with nouns (as in \ref{ex:iCqha.kAtWm} above) probably took place by reanalysis of the adverb in headless or postnominal relatives, or in complment clauses as above, then generalized to all noun phrases even those without subordinate clause.

 \subsection{Focus markers} \label{sec:focus}
   \subsubsection{Unexpected focus} \label{sec:unexpected}
 \subsubsection{Additive and scalar focus marker \forme{kɯnɤ} } \label{sec:kWnA}
The additive and scalar focus marker \japhug{kɯnɤ}{also, even} follows the constituent over which it has scope, which can be noun phrases, postpositional phrases but also subordinate clauses (these are treated in § XXX). The stress is on the first syllable (\forme{kɯ́nɤ}) and the vowel on the second syllable is often elited (a pronunciation \forme{kɯn} is often heard).

The marker \forme{kɯnɤ} expresses both additive focus, as in (\ref{ex:aZo.kWNA.staRlupa}), and scalar focus, as in (\ref{ex:WNgWz.kWnA.tunAndWtnW}) in affirmative sentences. It is also compatible with negative verb forms, as in (\ref{ex:tWrdoR.kWnA}), expressing the meaning `not even' (see also \japhug{cinɤ}{(not) even one} in § \ref{sec:cinA}).

\begin{exe}
\ex \label{ex:aZo.kWNA.staRlupa}
\gll aʑo kɯnɤ staʁlupa ŋu-a tɕe \\
\textsc{1sg} also born.in.the.tiger.year be:\textsc{fact}-\textsc{1sg} \textsc{lnk} \\
\glt `Me too (like you), I am of the Tiger year.' (2011-05-nyima, 168)
\end{exe}

\begin{exe}
\ex \label{ex:WNgWz.kWnA.tunAndWtnW}
\gll ʑara ʑo ɯ-ŋgɯz kɯnɤ tu-nɤndɯt-nɯ tɕe nɯ kɯ-βʁa ɣɤʑu, kɯ-nŋo ɣɤʑu qʰe, \\
\textsc{3pl} \textsc{emph} \textsc{3sg}.\textsc{poss}-among:\textsc{loc} also \textsc{ipfv}-fight-\textsc{pl} \textsc{lnk} \textsc{dem} \textsc{nmlz}:S/A-win \textsc{sens}:exist \textsc{nmlz}:S/A-lose  \textsc{sens}:exist \textsc{lnk} \\
\glt `Even among themselves, they fight, and there are winners and losers.' (20-sWNgi, 62-63)
\end{exe}
 
\begin{exe}
\ex \label{ex:tWrdoR.kWnA}
\gll tɯ-sŋi mɯntoʁ tɯ-rdoʁ kɯnɤ ci ci tɕe mɯ́j-stʰɯt \\
one-day flower one-piece also once once \textsc{lnk} \textsc{neg}:\textsc{sens}-finish \\
\glt `Sometimes one cannot finish even one pattern (on the belt) in one day.' (2011-06-thaXtsa, 47)
\end{exe}

As an additive focus marker, \forme{kɯnɤ} can be repeated on all the nouns designating the members of a group sharing a particular property, in the construction $X$ \forme{kɯnɤ}, $Y$ \forme{kɯnɤ}  `both $X$ and $Y$', as in (\ref{ex:Dpalcan.kWnA}).

\begin{exe}
\ex \label{ex:Dpalcan.kWnA}
 \gll a-pɯ-ŋu tɕe, aʑo kɯnɤ taʁrdo rɟitpa a-pɯ-ŋu-a, χpɤltɕin kɯnɤ taʁrdo rɟitpa a-pɯ-ŋu, ... nɯ tɕi-rɟit nɯni tɕe taʁrdo rɟitpa ma nɯ ma kɯmaʁ rɟitpa nɯ kɤ-rtsi me.  \\
 \textsc{irr}-\textsc{ipfv}-be \textsc{lnk} \textsc{1sg} also pl.n. lineage  \textsc{irr}-\textsc{ipfv}-be-\textsc{1sg}  p.n. also pl.n. lineage  \textsc{irr}-\textsc{ipfv}-be { } \textsc{dem} \textsc{1du}.\textsc{poss}-offspring \textsc{dem}:\textsc{du} \textsc{lnk} pl.n. lineage apart.from \textsc{dem} apart.from other lineage \textsc{dem} \textsc{nmlz}:O-count not.exist:\textsc{fact} \\
 \glt `For instance suppose that both Dpalcan and I were from Tarrdo lineage, then our two children would only count as members of the Tarrdo lineage and no other lineage.' (140426 rJitpa, 13-15)
\end{exe}

The scope of  \forme{kɯnɤ} is generally exclusively on the constituent that it immediately follows, but there are cases where the scope is more extensive. In (\ref{ex:aZo.kWnA.akAsWso}), \forme{kɯnɤ} occurs between the pronoun \forme{aʑo} and the following participial verb form, which bears a \textsc{1sg} possessive prefix \forme{a-} coreferent with that pronoun (see also \ref{ex:aZWG.kWnA} below). The semantic scope of \forme{kɯnɤ} here is on the whole relative \forme{aʑo a-kɤ-sɯso} `(the things) that I want' rather than exclusively on the pronoun \forme{aʑo}.

\begin{exe}
\ex \label{ex:aZo.kWnA.akAsWso}
 \gll aʑo kɯnɤ a-kɤ-sɯso nɯ tɤ-stu-nɯ ra \\
 \textsc{1sg} also \textsc{1sg}.\textsc{poss}-\textsc{nmlz}:O-think \textsc{dem} \textsc{imp}-do.like-\textsc{pl} have.to:\textsc{fact} \\
 \glt `(I will do as you say, but) do also the things I want.' (2003kAndzwsqhaj2, 47)
\end{exe}

 

The focus marker \forme{kɯnɤ} is found with nouns or pronouns in core argument function, including S (\ref{ex:kWnA.nArca}), O (\ref{ex:nWXpWm.kWnA}), and semi-objects (\ref{ex:kWnA.mAsna}).  Examples with transitive subjects are presented below (\ref{ex:nWra.kWnA} and \ref{ex:Wzda.ra.kWnA}).

 \begin{exe}
\ex \label{ex:kWnA.nArca}
\gll aʑo kɯnɤ nɤ-rca ɣi-a ɕti  \\
\textsc{1sg} also \textsc{2sg}.\textsc{poss}-following come:\textsc{fact}-\textsc{1sg} be.\textsc{affirm}:\textsc{fact} \\
\glt `I am coming with you too.' (2011-05-nyima, 171)
 \end{exe}
 
   \begin{exe}
\ex \label{ex:nWXpWm.kWnA}
\gll    ma nɯ-χpɯm kɯnɤ kʰro mɤ-kɯ-fkaβ kɯ-fse ku-rɤʑi-nɯ  \\
lnk 3pl.poss-knee also much \textsc{neg}-\textsc{nmlz}:S/A-cover \textsc{nmlz}:S/A-be.like \textsc{ipfv}-stay-\textsc{pl} \\
\glt `(Gents) would (wear trousers that did) not cover much even their knees.'  (30-rkAsnom, 5) 
  \end{exe}
  
  \begin{exe}
 \ex \label{ex:kWnA.mAsna}
 \gll   ɯ-ru nɯra laʁdɯn ɯ-jɯ kɯnɤ mɤ-sna, ma mɤ-ngɯt. \\
 \textsc{3sg}.\textsc{poss}-trunk \textsc{dem}:\textsc{pl} tool \textsc{3sg}.\textsc{poss}-handle also \textsc{neg}-be.worth \textsc{lnk}  \textsc{neg}-be.strong:\textsc{fact} \\
 \glt `(The wood from) its trunk is not even good (enough to be used to make) tool handles, as it is not strong.' 
  \end{exe}

It also occurs with all types of oblique arguments and adjuncts, including genitive (\ref{ex:aZWG.kWnA}), dative (\forme{ɯ-ɕki} \ref{ex:nWCki.kWnA}),  locational adjuncts in \forme{tɕu} (\ref{ex:kutCu.kWnA}) or \forme{ri} (\ref{ex:ri.kWnA}), temporal adjuncts (\ref{ex:ftCAXcAl.kWnA}) or adjuncts expressing manner or cause (\ref{ex:nWtCu.kWnA}).  
  
   \begin{exe}
\ex \label{ex:aZWG.kWnA}
\gll aʑɯɣ kɯnɤ a-mpʰrɯmɯ a-pɯ-tɯ-sɯ-re ɯ-tɯ́-cʰa \\
\textsc{1sg}:\textsc{gen} also \textsc{1sg}.\textsc{poss}-divination \textsc{irr}-\textsc{pfv}-2-\textsc{caus}-look[III] \textsc{qu}-2-can:\textsc{fact} \\
\glt `Can you ask (the monk) to make a divination for me too?' (The divination, 31)
\end{exe}  
  
   \begin{exe}
\ex \label{ex:nWCki.kWnA}
\gll  tɯ-pi ɣɯ ɯ-nmaʁ ra nɯ-ɕki kɯnɤ `a-pi' tu-kɯ-ti ɕti ma nɯ ma kupa kɯ-fse ʑaka ɯ-rmi me. \\
\textsc{genr}.\textsc{poss}-elder.sibling \textsc{gen} \textsc{3sg}.\textsc{poss}-husband \textsc{pl} \textsc{3pl}.\textsc{poss}-\textsc{dat} also \textsc{1sg}.\textsc{poss}-elder.sibling \textsc{ipfv}-\textsc{genr}-say be.\textsc{affirm}:\textsc{fact} \textsc{lnk} \textsc{dem} apart.from Chinese \textsc{nmlz}:S/A-be.like each \textsc{3sg}.\textsc{poss}-name not.exist:\textsc{fact} \\
\glt  `One calls one's sister's husband (and others from his family) `my elder brother', there are no other special terms as in Chinese.' (140425 kWmdza05)
\end{exe}


  \begin{exe}
\ex \label{ex:kutCu.kWnA}
\gll  kutɕu kɯnɤ nɯ ɲɯ-fse, jɯfɕɯndʐi ra kɯ-xtɕɯ\redp{}xtɕi tɤ-ɣɤndʐo kɯ-fse ri, ɕɤxɕo tɕe kɯ-xtɕɯ\redp{}xtɕi ɲɯ-ʑi kɯ-fse \\
here also \textsc{dem} \textsc{sens}-be.like a.few.days.ago \textsc{nmlz}:S/A-\textsc{emph}\redp{}be.small \textsc{pfv}-be.cold \textsc{nmlz}:S/A-be.like \textsc{lnk} the.last.days \textsc{lnk} \textsc{nmlz}:S/A-\textsc{emph}\redp{}be.small \textsc{sens}-subside \textsc{nmlz}:S/A-be.like \\
\glt `It is like that here too, a few days ago the weather became a little cold, but the last days it has eased a bit.' (conversation, 141027)
  \end{exe}
  
    \begin{exe}
\ex \label{ex:ri.kWnA}
\gll   maldzɯ nɯ, nɯ ɯ-tʰɤcu tsa ri kɯnɤ ɣɤʑu. qarɣɤpɤt ɯ-rca ri kɯnɤ tu-ɬoʁ ɲɯ-ŋu. \\
plant.name \textsc{dem} \textsc{dem} \textsc{3sg}.\textsc{poss}-downstream a.little \textsc{loc} also exist:\textsc{sens} plant.name \textsc{3sg}.\textsc{poss}-among \textsc{loc} also \textsc{ipfv}-come.out \textsc{sens}-be \\
\glt `The \forme{maldzɯ} plant, it is also found in places of slightly lower altitude, but grows also in the same places as  \forme{qarɣɤpɤt} plants.' (18-qromJoR, 81-82)
    \end{exe}
    
\begin{exe}
\ex \label{ex:ftCAXcAl.kWnA}
\gll   kukutɕu ftɕɤχcɤl kɯnɤ <baonuanyi> tu-tɯ-ŋge pɯ-ɕti. \\
  here mid.summer also warm.clothes \textsc{ipfv}-2-wear[III] \textsc{pst}.\textsc{ipfv}-be.\textsc{affirm} \\
  \glt `Here you were wearing warm clothes even in mid summer.' (conversation, 141017)
    \end{exe}
    
    \begin{exe}
\ex \label{ex:nWtCu.kWnA}
\gll    tɕe nɯtɕu kɯnɤ ɯ-jaʁ ɯ-ntsi tɤɲi pjɯ-sɤtse, ɯ-jaʁ ɯ-ntsi kɯ tsʰitsuku ɲɯ-z-nɤme qʰe, \\
\textsc{lnk} \textsc{dem}:\textsc{loc} also \textsc{3sg}.\textsc{poss}-hand \textsc{3sg}.\textsc{poss}-one.of.a.pair erg various.things \textsc{ipfv}-\textsc{caus}-do[III] \textsc{lnk}  \\
\glt `Even like that (despite the pain in her legs), she props herself with a cane using one hand, and does all kinds of things with her other hand.' (14-tApitaRi, 52)
\end{exe}

Although \japhug{kɯnɤ}{also, even} can be combined with most postpositions and relator nouns as shown by the examples above, it is however incompatible with the ergative \forme{kɯ}. For instance, in  (\ref{ex:nWra.kWnA}), although the demonstrative pronoun \forme{nɯra} `they, those' in the second clause is the subject of the transitive verb \japhug{ndza}{eat}, it does not take the ergative \forme{kɯ} as would be expected (§ \ref{sec:A.kW}). The same applies to \forme{ɯ-zda ra} `his companions', subject of the transitive verb \forme{na-nɯ-ɕar-nɯ} `they looked for themselves' in (\ref{ex:Wzda.ra.kWnA}), 

  \begin{exe}
\ex \label{ex:nWra.kWnA}
\gll ɯ-pɯ nɯra li ju-ɣi-nɯ qʰe, nɯra kɯnɤ ɣɯ-tu-ndza-nɯ. \\
\textsc{3sg}.\textsc{poss}-young \textsc{dem}:\textsc{pl} again \textsc{ipfv}-come-\textsc{pl} \textsc{lnk} \textsc{dem}:\textsc{pl} also \textsc{cisloc}-\textsc{ipfv}-eat-\textsc{pl} \\
\glt `Its youngs also come and they too eat it.' (20-sWNgi, 59-60)
  \end{exe}
  
    \begin{exe}
\ex \label{ex:Wzda.ra.kWnA}
\gll   ɯ-zda ra kɯnɤ nɯ-rʑaβ tɯka na-nɯ-ɕar-nɯ ɲɯ-ŋu \\
\textsc{3sg}.\textsc{poss}-companion \textsc{pl} also \textsc{3sg}.\textsc{poss}-wife each \textsc{pfv}:3\fl{}3'-\textsc{auto}-search \textsc{sens}-be \\
\glt `His companions also took each a wife for himself (among the women of the island).' (2005Norbzang, 44)
    \end{exe}
    
The combinations $\dagger$\forme{kɯ kɯnɤ} or $\dagger$\forme{kɯnɤ kɯ} are unattested, and not accepted by native speakers. The contrast between absolutive and ergative noun phrases is therefore neutralized in additive or scalar focus with \forme{kɯnɤ}. ŋote that other focus markers, such as \forme{ri} and \forme{tɕi} (see \ref{ex:tCi.ndze} in § \ref{sec:ri.additive}) differ from \forme{kɯnɤ} in this regard.

Four distinct facts converge to suggest that the first syllable of \forme{kɯnɤ} is historically related to the ergative postposition \forme{kɯ}: (i) the incompatibility of co-occurrence of \forme{kɯnɤ} and \forme{kɯ}; (ii) the stress on the first syllable in \forme{kɯ́nɤ}; (iii) the similar \forme{-nɤ} element in the other scalar focus marker \japhug{cinɤ}{(not) even one} (§ \ref{sec:cinA}) (iv) the existence of the linker \forme{nɤ}, possibly of Tibetan origin (§ XXX). A detailed examination of this topic is however impossible on the basis Japhug-internal evidence, and will require extensive syntactic comparison between Gyalrong languages.

 \subsubsection{Correlative additive focus markers \forme{ri} and \forme{tɕi}} \label{sec:ri.additive} 
 The additive focus markers \forme{ri} and \forme{tɕi}  are used in enumerations, repeated after each noun referring to  members of a group, to focus on the fact that their referents share a common property (or properties that are semantically close enough), as in (\ref{ex:ri.kWsthWci.WWmpCar}) and (\ref{ex:tCi.tulhoR.cha}) (see additional examples in \citealt[313-314]{jacques14linking}).
 
 \begin{exe}
\ex \label{ex:ri.kWsthWci.WWmpCar}
 \gll  a-rʑaβ ri kɯstʰɯci ɲɯ-mpɕɤr, a-mbro ri kɯstʰɯci ɲɯ-ʑru, a-pɣɤtɕɯ ri kɯstʰɯci ɲɯ-mpɕɤr tɕe, \\
 \textsc{1sg}.\textsc{poss}-wife also so.much \textsc{sens}-be.beautiful  \textsc{1sg}.\textsc{poss}-horse also so.much \textsc{sens}-be.strong  \textsc{1sg}.\textsc{poss}-bird also so.much \textsc{sens}-be.beautiful \textsc{lnk} \\
 \glt `My wife is so beautiful, my horse so strong, my bird so beautiful.' (2003qachga, 116)
 \end{exe}
 
  \begin{exe}
\ex \label{ex:tCi.tulhoR.cha}
 \gll  ɴqiaβ tɕi tu-ɬoʁ cʰa, zrɯ tɕi tu-ɬoʁ cʰa, \\
 dark.side.of.the.mountain also \textsc{ipfv}-come.out can:\textsc{fact}   sunny.side.of.the.mountain also \textsc{ipfv}-come.out can:\textsc{fact}  \\
 \glt `It can grow in both the dark and the sunny sides of the mountains.' (17-thowum, 14)
  \end{exe}
  
The correlative focus markers \forme{ri} and \forme{tɕi} can occur after any noun phrase or postpositional phrase, including with the ergative  \forme{kɯ} as shown by (\ref{ex:tCi.ndze}), unlike the marker \japhug{kɯnɤ}{even, also} (see examples \ref{ex:nWra.kWnA} and \ref{ex:Wzda.ra.kWnA}, § \ref{sec:kWnA}).
  
  \begin{exe}
\ex \label{ex:tCi.ndze}
 \gll paʁ kɯ tɕi ndze, nɯŋa kɯ tɕi ndze, jla kɯ tɕi ndze.   \\
 pig \textsc{erg} also eat[III]:\textsc{fact}  cow \textsc{erg} also eat[III]:\textsc{fact}  hybrid.yak \textsc{erg} also eat[III]:\textsc{fact}  \\
 \glt `Pigs eat it, cows eat it, hybrid yaks eat it.' (18-NGolo, 171)
  \end{exe}

The focus markers \forme{ri} and \forme{tɕi} can have scope on only part of the noun/propositional phrase, and even on the relator nouns as in (\ref{ex:WNgW.tCi}).

   \begin{exe}
\ex \label{ex:WNgW.tCi}
 \gll   sɤtɕʰa ɯ-ŋgɯ tɕi ɣɤʑu, sɤtɕʰa ɯ-taʁ tɕi ʑo ɣɤʑu \\
 ground \textsc{3sg}.\textsc{poss}-inside also exist:\textsc{sens}  ground \textsc{3sg}.\textsc{poss}-inside also \textsc{emph} exist:\textsc{sens} \\
 \glt `It is found both inside the ground, and on the ground.' (25-GdAso, 17)
    \end{exe}
    
Alternatively, it is possible to enumerate distinct related properties of the same referent using \forme{ri}, but that marker still follows the noun phrase (correlative \forme{ri} can follow verbs, but only in a specific construction, see \ref{ex:ri.kWmWm.ri} below). In this case the referent cannot be elided, and must be repeated in both clauses, at least as a third person pronoun \forme{ɯʑo} as in (\ref{ex:WlWz.ri.pjAxtCi}).

  \begin{exe}
\ex \label{ex:WlWz.ri.pjAxtCi}
 \gll pʰaʁrgot nɯnɯ ɯʑo ri pjɤ-rʑi, ɯʑo ri pjɤ-tsʰu tɕe \\
 boar \textsc{dem} \textsc{3sg} also \textsc{ifr}.\textsc{ipfv}-be.heavy \textsc{3sg} also \textsc{ifr}.\textsc{ipfv}-be.fat \textsc{lnk} \\ 
\glt  `The boar, it was heavy and fat.' (140428 yonggan de xiaocaifeng-zh, 244)
 \end{exe}
 
The correlative construction can involve the possessor of an IPN, as in (\ref{ex:WlWz.ri.pjAxtCi}), where in the first clause the referent `the girl' is possessor of the intransitive subject (literally `her age was small', § XXX) and in second it corresponds to the intransitive subject, realized as a third person pronoun \forme{ɯʑo} `she'.

  \begin{exe}
\ex \label{ex:WlWz.ri.pjAxtCi}
 \gll tɕʰeme nɯ ɯ-lɯz ri pjɤ-xtɕi, ɯʑo ri pjɤ-mpɕɤr,  \\
 girl \textsc{dem} \textsc{3sg}.\textsc{poss}-age also \textsc{ifr}.\textsc{ipfv}-be.small \textsc{3sg} also \textsc{ifr}.\textsc{ipfv}-be.beautiful \\
\glt `The girl was young and beautiful.' (150909 hua pi-zh, 10)
 \end{exe}
 
 More complex correlations, involving different subjects and predicates related to another referent, are also possible as shown by example (\ref{ex:lWlu.kW}), where \forme{ri} occurs after the intransitive subject \japhug{tɯ-ci}{wtaer}, the transitive subject \japhug{lɯlu}{cat} with the ergative and the finite verb \japhug{tu-ɕe}{it goes up} (on which see below and refer to § XXX).
 
 \begin{exe}
\ex   \label{ex:lWlu.kW}
\gll <yancong> ku-kɯ-rɤloʁ tɕe ɯ-taʁ tɯ-ci ri mɯ́j-ɣi lɯlu kɯ ri mɯ-ɲɯ́-wɣ-ɕaβ qapri tu-ɕe ri mɯ́j-cʰa tɕe \\
 chimney \textsc{ipfv}-\textsc{genr}:S/P-make.a.nest \textsc{lnk} \textsc{3sg}.\textsc{poss}-on \textsc{indef}.\textsc{poss}-water also \textsc{neg}:\textsc{sens}-come cat \textsc{erg} also \textsc{neg}-\textsc{ipfv}-\textsc{inv}-catch snake \textsc{ipfv}:\textsc{up}-go also \textsc{neg}:\textsc{sens}-can \textsc{lnk} \\
 \glt `(The sparrows) make their nest in the chimney, (because) water cannot come up there, the cats cannot catch them, and the snakes cannot go up there.' (22-kumpGatCW, 69)
 \end{exe}
 
 The marker \forme{ri} is homophonous with the locative \forme{ri} (§ \ref{sec:locative}), and in cases with an enumeration of locative adjuncts, there can be ambiguity between the two. In (\ref{ex:Xcha.ri.ci}), \forme{ri} is analyzed as a locative because of the position of the determiner \forme{ci}: if \forme{ri} were the focus marker here, it should occur after the whole noun phrase, including \forme{ci}.  %à vérifier
 
 \begin{exe}
\ex \label{ex:Xcha.ri.ci}
\gll   χcʰa ri ci, ɯ-ʁe ri ci ɯ-jme cʰɯ-ɬoʁ ɲɯ-ŋu. \\
right \textsc{loc} one  \textsc{3sg}.\textsc{poss}-left \textsc{loc} one \textsc{3sg}.\textsc{poss}-tail \textsc{ipfv}:\textsc{downstream}-come.out \textsc{sens}-be \\
\glt `It has one tail on the right, and one on the left.' (26-qro, 116)
\end{exe}

The marker \forme{ri} can follow verbs only if combined with an existential verb, a copula or a modal auxiliary verb as main predicate (meaning `both $X$ and $Y$' with positive copulas, and `neither $X$ nor $Y$' with negative ones). In this type of construction, verbs are mostly in non-finite form, as in (\ref{ex:ri.kWmWm.ri}). Examples with finite verbs however do exist; this topic is treated in § XXX. %ɲɯ-ɣɤwu ri kɯ-maʁ, ɲɯ-nɤre ri kɯ-maʁ kɯ-fse ɲɤ-k-ɤβzu-ci  ; tu-rɯɕmi ri mɤ-kɯ-khɯ, chɯ-nɯrɤɣo ri mɤ-kɯ-khɯ ci ɲɤ-k-ɤβzu-ci. ; tu-ndzur ri pjɤ-maʁ, ku-omdzɯ ri pjɤ-maʁ.

 \begin{exe}
\ex \label{ex:ri.kWmWm.ri}
 \gll   nɯ pɯ́-wɣ-ta ri  kɯroz kɯ-mɯm ri maŋe, kɯroz mɤ-kɯ-ɣɤ-mɲɤt ri maŋe qʰe, \\
 \textsc{dem} \textsc{pfv}-\textsc{inv}-put \textsc{lnk} specially \textsc{nmlz}:S/A-be.tasty also not.exist:\textsc{sens} specially \textsc{neg}-\textsc{nmlz}:S/A-\textsc{facil}-be.spoiled also not.exist:\textsc{sens} \textsc{lnk} \\
 \glt `When if one puts (a seal on the bread), there is nothing especially tasty about it, and nothing special concerning the preservation (of the bread).' (160706 thotsi, 27)
  \end{exe}
  

  
 \subsubsection{Scalar focus marker \forme{cinɤ}} \label{sec:cinA} 
 
  \subsubsection{Restrictive focus} \label{sec:restrictive.focus} 
 Japhug does not have a restrictive focus marker `only', and the only way to express this meaning is to combine the exceptive \japhug{ma}{apart from} (and its reduplicated variant \forme{mɯma} § \ref{sec:exceptive}) with a negative predicate. This can be a verb with a negative prefix as in (\ref{ex:XsArZaR}), or a negative existential verb as in (\ref{ex:Wmi.Wntsi.ma.me}).
 
 \begin{exe}
\ex  \label{ex:XsArZaR}
\gll   χsɤ-rʑaʁ ma mɯ-pɯ-tsu-a ɲɤ-sɯso ri χsɯ-xpa pjɤ-tsu tɕe,  \\
three-day apart.from \textsc{neg}-\textsc{pfv}-pass-\textsc{1sg} \textsc{ifr}-think \textsc{lnk} three-year \textsc{ifr}-pass \textsc{lnk} \\
\glt `He thought that he had spent only three days, but three years had passed.' (2011-4-smanmi, 178)
  \end{exe}
  
  \begin{exe}
\ex  \label{ex:Wmi.Wntsi.ma.me}
\gll  rkoŋɟɤl nɯnɯ, ɯ-mi ɯ-ntsi nɯ ma me kʰi.   \\
one.legged.demon \textsc{dem} \textsc{3sg}.\textsc{poss}-leg \textsc{3sg}.\textsc{poss}-one.of.a.pair \textsc{dem} apart.from not.exist:\textsc{fact} \textsc{hearsay} \\
\glt  `It is said that one-legged demons only had one leg.' (140510 rkoNJAl, 4)
  \end{exe}
  
The restrictive focus construction implies the presence of a noun phrase with a numeral or a CN when the restriction bears on the quantity, but restriction can also be qualitative, without quantifier, as in (\ref{ex:karGi.Zo.kWfse.ma.me}).

\begin{exe}
\ex \label{ex:karGi.Zo.kWfse.ma.me}
 \gll   ɯ-mat nɯnɯ na-lɤt ɕɯmɯma nɤ kɯ-ndɯ\redp{}ndɯβ ʑo ma me, karɣi ʑo kɯ-fse ma me  \\
 \textsc{3sg}.\textsc{poss}-fruit \textsc{dem} \textsc{pfv}:3\fl{}3'-throw just \textsc{lnk}  \textsc{nmlz}:S/A-\textsc{emph}\redp{}small \textsc{emph} apart.from not.exist:\textsc{fact} turnip.seed \textsc{emph} \textsc{nmlz}:S/A-be.like apart.from not.exist:\textsc{fact} \\
 \glt  `When the fruit of (xanthoxyllum) has just come out, there is only something very small, only like a turnip seed.'  (07-tCGom, 7)
  \end{exe}
  
The restrictive focus construction can be combined with a scalar focus in \forme{kɯnɤ} (see §  \ref{sec:kWnA}), as in (\ref{ex:ma.kWme.kWnA}). In this example, \forme{kɯnɤ} has scope over the subordinate clause \forme{stɯsti ma kɯ-me}, which is ambiguous between a participial headless relative (§ XXX) `consisting of only a female all alone' and a manner infinitival clause (§ XXX; in this case the gloss of \forme{kɯ-me} would be \textsc{inf}:\textsc{stat}-not.exist) `even (when) there is only a female all alone'.

  \begin{exe}
\ex \label{ex:ma.kWme.kWnA}
\gll  mu ma, stɯsti ma kɯ-me kɯnɤ cʰɯ-rɤŋgɯm ɲɯ-ɕti. \\
female apart.from alone apart.from \textsc{nmlz}:S/A-not.exist also \textsc{ipfv}-lay.eggs \textsc{sens}-be.\textsc{affirm} \\
\glt `Even only a female (hen) alone does lay eggs.' (150819 kumpGa, 11)
\end{exe}
  
\subsection{Genitival phrases}
%
%tɕendɤre ɯ-jaʁ nɯtɕu ftsoʁ kɯngɯt ɯ-phɯ ɣɯ srɯnloʁ pjɤ-k-ɤrku-ci
%2003gesar, 239
%
%\subsection{Determiners} \label{sec:determiners}
%\japhug{ɕɯŋarɯra}{each better than the other}
% rɟɤlpu ɕɯŋarɯra kɯ ta-tʰu-nɯ ɕti ri, mɯ-tɤ-nɤla-j ɕti tɕe,
% 2003 qachga, 71
\subsection{Identity modifiers} \label{sec:identity.modifier}

%nɤki tɕheme nɯ ɯ-ɕki ɯ-kɯ-sɤja jo-ɕe, ci tɕheme kɯ-ŋɤn nɯ ɯ-ɕki.

%ci qhɤjmbaʁ nɯ kɯ-jaʁ kɯ-fse nɯnɯ 
%mtshalu ɯ-cu tɕe nɤki,
%tɯ-mgo zmɤrɤβ kú-wɣ-nɯ-lɤt sna.
%16-RlWmsWsi
%li ci /ɯt/ ɯ-tɯphu nɯ tɤpu qhɤjmbaʁ tu-ti-nɯ ŋu tɕe,
\subsection{Attributes}

\section{The structure of the noun phrase}

\section{Nominal predicates}
