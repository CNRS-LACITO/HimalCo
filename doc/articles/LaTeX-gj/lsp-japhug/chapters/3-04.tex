\chapter{The noun phrase}

\section{Postpositions and relator nouns}

\subsection{Absolutive} \label{sec:absolutive}
\subsubsection{Intransitive subject}
\subsubsection{Object}
\subsubsection{Semi-object}
\subsubsection{Theme}
\subsubsection{Essive}
%tsuku kɯ paʁndza ɲɯ-nɯ-phɯt-nɯ ɲɯ-ŋu ri,
\subsubsection{Locative adjunct}

\subsection{Ergative} \label{sec:erg.kW}
\subsubsection{Core argument}
\subsubsection{Instrumental}
\subsubsection{Comparee marker}
\subsubsection{Partitive}

\subsection{Genitive} \label{sec:genitive}
\subsubsection{Possession}
\subsubsection{Beneficiary}
\subsection{Locative}
\subsection{Standard marker} \label{sec:comparative} %\japhug{sɤz}{compare with}
\subsection{Exceptive} \label{sec:exceptive} %\japhug{ma}{apart from}

The exceptive \japhug{ma}{apart from} and its variants are required in the restrictive focus construction (§ \ref{sec:restrictive.focus}).

\subsection{Terminative} \label{sec:terminative}  %\japhug{mɤɕtʂa}{until}


\section{Noun modifiers and determiners}
This section discusses all nouns modifiers and determiners except relative clauses (§ XXX) and complement clauses (§ XXX). 

\subsection{Demonstratives} \label{sec:demonstrative.determiners}

\subsection{Quantifiers}
\subsubsection{Universal quantifiers} \label{sec:universal.quant}
\subsubsection{Mid-scalar quantifier} \label{sec:tsuku}
(\ref{sec:partitive.pronouns})

\subsection{Indefinite and definite markers} \label{sec:indefinite.markers}

\subsubsection{Indefinite article} \label{sec:indef.article}
The form \japhug{ci}{one} has among its many functions (in addition to pronoun, numeral and adverb, see § \ref{sec:other.pro}, § \ref{sec:partitive.pronouns}, § \ref{sec:identity.modifier}, § \ref{sec:one.to.ten} and § XXX) that of singular indefinite article, as in (\ref{ex:ci.indef}) and (\ref{ex:ci.chAGi}). It is typically used to introduce a new referent in a story.

\begin{exe}
\ex \label{ex:ci.indef}
\gll tɕʰeme kɯ-mpɕɯ\redp{}mpɕɤr ci ɲɤ-nɯ-ɬoʁ \\
girl \textsc{nmlz}:S/A-\textsc{emph}\redp{}beautiful \textsc{indef} \textsc{ifr}-\textsc{auto}-come.out \\
\glt `A very beautiful girl appeared (out of it).' (The flood, 39)
\end{exe}

\begin{exe}
\ex \label{ex:ci.chAGi}
\gll tɕɤlo tɕe tɤ-tɕɯ ci cʰɤ-ɣi qʰe, \\
upstream \textsc{lnk} \textsc{indef}.\textsc{poss}-son \textsc{indef} \textsc{ifr}:\textsc{downstream}-come \textsc{lnk} \\
\glt `A boy came from upstream.' (2003-kWBRa, 41)
\end{exe}

Although \forme{ci} can be used as a partitive pronoun `one of them' (§ \ref{sec:partitive.pronouns}), as a postnominal determiner it does not have partitive meaning. To express a meaning such as `one of the boys', a CN such as \japhug{tɯ-rdoʁ}{one piece} is used instead (§ \ref{sec:ICN}). 

Note that when used as a prenominal modifier, \forme{ci} has a completely different (definite) meaning `the other X' (§ \ref{sec:identity.modifier}). 

There are no dual or plural indefinite articles in Japhug. The plural marker \forme{ra} can occur after the indefinite \forme{ci}, but with a vague associative meaning `and other things' as in (\ref{ex:ci.ra}).

\begin{exe}
\ex \label{ex:ci.ra}
 \gll  ndʑi-tɕɯ ci, ndʑi-me ci ra to-tu. \\
 \textsc{3du}.\textsc{poss}-son \textsc{indef}  \textsc{3du}.\textsc{poss}-girl \textsc{indef} \textsc{pl} \textsc{ifr}-exist \\
 \glt  `They$_{du}$ had a boy and a girl (etc).' (150827 tianluo-zh, 155)
\end{exe}

\subsubsection{Indefinite pronoun as modifier} \label{sec:indefinite}
The indefinite pronoun \japhug{tʰɯci}{something} (§ \ref{sec:thWci}) has marginal uses as a prenominal indefinite modifier, as in  (\ref{ex:thWci.laXCi}) and (\ref{ex:laXtCha.ci.nWnW}) below. 

\begin{exe}
\ex \label{ex:thWci.laXCi}
\gll   tʰɯci laχɕi ci ɕ-pɯ-nɯ-βzjoz-nɯ tɕe, jɤ-ɕe-nɯ ra \\
something trade \textsc{indef} \textsc{transloc-imp-auto}-learn-\textsc{pl} \textsc{lnk} \textsc{imp}-go-\textsc{pl} have.to:\textsc{fact} \\
\glt `Go and learn some trade!' (140508 benling gaoqiang de si xiongdi-zh, 29)
 \end{exe}

\subsubsection{The marking of definiteness} \label{sec:definiteness}
Japhug has no dedicated definite determiner, but  \forme{nɯ} and \forme{nɯnɯ}  as demonstrative determiners (\ref{sec:demonstrative.determiners}) and as topic markers (\ref{sec:topic}) and the prenominal aforementioned topic marker \forme{iɕqʰa} (§ \ref{sec:iCqha}) are generally used with definite referents.  

Example (\ref{ex:ci.joGi}) illustrates a typical example with the determiner \forme{nɯ}; the indefinite article \forme{ci} (§ \ref{sec:indef.article}) occurs in the first introduction of a new referent in the story as in the first clause of example (\ref{ex:ci.joGi}), but on the following occurrence of the same noun \forme{nɯ} is found.

\begin{exe}
\ex \label{ex:ci.joGi}
 \gll  tɕe qajdo ci jo-ɣi tɕe, tɕe qajdo nɯ kɯ `mo laz tu, pʰo laz me' to-ti. \\
 \textsc{lnk} crow \textsc{indef} \textsc{ifr}-come \textsc{lnk} \textsc{lnk} crow \textsc{dem} \textsc{erg} girl karma exist:\textsc{fact} boy karma not.exist:\textsc{fact} \textsc{ifr}-say \\
 \glt `A crow came. The crow said: `The girl will have chance, the boy won't.'' (28-qAjdoskAt, 8)
\end{exe}

However, although nouns phrases followed by \forme{nɯ} and \forme{nɯnɯ} more often than not denote definite referents, these determiners cannot be analyzed as definite articles, as noun phrases with \forme{nɯ} or \forme{nɯnɯ} can in certain cases have indefinite referents. 

A very clear case of use of \forme{nɯ} with an indefinite referent occurs on nouns serving as heads of head-internal relative clauses. A well-attested typological generalization is that in this type of relative clauses, definiteness marking is agrammatical (see \citealt{basilico96internally} and § XXX). In Khroskyabs, \citet[636]{lai17khroskyabs} reports that the definiteness marker \forme{=tə} is indeed not accepted on the head noun of head-internal relatives. In Japhug however, \forme{nɯ} does occur in such a syntactic context. For instance, in (\ref{ex:tAnmaR.nW.kW}), the head \forme{tɤ-nmaʁ nɯ kɯ} is subject of the participle \japhug{ɲɯ-kɯ-nɯ-ɕar}{looking for}, and is embedded in the participial relative clause indicated in brackets -- the presence of the ergative \forme{kɯ} precludes to analyze it as a post-nominal relative (§ XXX). From the meaning of the sentence the head \japhug{tɤ-nmaʁ}{husband} is clearly indefinite non-specific non-generic  (see \citealt[286-291]{lehmann84relativsatz}). The fact that it takes the marker \forme{nɯ} shows that this marker, unlike Khroskyabs \forme{=tə}, is not primarily marking definiteness.

\begin{exe}
\ex \label{ex:tAnmaR.nW.kW}
 \gll tɕeri [tɤ-nmaʁ nɯ kɯ ɯ-rʑaʁ kɯ-ɤntɕʰɯ ɲɯ-kɯ-nɯ-ɕar], aʁɤndɯndɤt tɤndɤɣri tu-kɯ-βzu pjɤ-tu.  \\
but  \textsc{indef}.\textsc{poss}-husband \textsc{dem} \textsc{erg} \textsc{3sg}.\textsc{poss}-wife  \textsc{nmlz}:S/A-be.many \textsc{ipfv}-\textsc{nmlz}:S/A-\textsc{auto}-search everywhere  illegitimate.child  \textsc{ipfv}-\textsc{nmlz}:S/A-make \textsc{ifr}.\textsc{ipfv}-exist \\
\glt `However there were husbands who were looking for several women and had illegitimate children.' (140427 tAndAGri, 3)
\end{exe}

Other cases of indefinite noun phrase with \forme{nɯ} are observed with left-dislocated topics. In example (\ref{ex:RnWz.nWnW}), we find a type of tail-head linkeage  (§ XXX) where both the noun phrase \japhug{spjaŋkɯ ʁnɯz}{two wolves} and the verb \japhug{ɲɤ-k-ɤtɯɣ-ci}{he met} are repeated; in the second occurrence, the noun phrase is topicalized and is followed by the topic marker \forme{nɯnɯ}, with a slight pause of hesitation. The determiner \forme{nɯnɯ} in this clause, unlike \forme{nɯ} in (\ref{ex:ci.joGi}), does not mark definiteness: that clause cannot be understood as `He met the two wolves'.

\begin{exe} 
\ex \label{ex:RnWz.nWnW} 
 \gll spjaŋkɯ ʁnɯz ɲɤ-k-ɤtɯɣ-ci. spjaŋkɯ ʁnɯz nɯnɯ, tɕendɤre ɲɤ-k-ɤtɯɣ-ci tɕe iɕqʰa, kɯ-rɤ-ntɕʰa nɯ wuma ʑo ɲɤ-mu. \\ 
 wolf two \textsc{ifr}-\textsc{evd}-meet-\textsc{evd}  wolf two \textsc{dem} \textsc{lnk} \textsc{ifr}-\textsc{evd}-meet-\textsc{evd} \textsc{lnk} the.aforementioned \textsc{nmlz}:S/A-\textsc{a.pass}:\textsc{n.hum}-kill \textsc{dem} really \textsc{emph} \textsc{ifr}-be.afraid \\ 
 \glt `He$_i$ (the butcher) met two wolves. He$_i$ met two wolves, and the butcher$_i$ was very much afraid.' (150902 liaozhai lang-zh, 7-8)
\end{exe}

The determiners \forme{nɯ} or \forme{nɯnɯ} are not attested in the corpus with the indefinite singular article \forme{ci} if both have scope on the same noun. In all cases with \forme{ci} followed by \forme{nɯ} (other than the identity pronoun in § \ref{sec:other.pro}), or of \forme{nɯ} followed by \forme{ci} in the corpus, they belong to different constituents. For instance, in (\ref{ex:ci.YAZGAsAphAr}), \forme{ci} is in adverbial use (`a little, once', see § XXX) and does not belong to the preceding noun phrase.  

\begin{exe}
\ex \label{ex:ci.YAZGAsAphAr}
\gll [tɕʰeme nɯ] ci ɲɤ-ʑɣɤ-sɤpʰɤr qʰe  \\
girl \textsc{dem} one \textsc{ifr}-\textsc{refl}-shake \textsc{lnk} \\
\glt `The girl shook herself.' (02-deluge2012, 125)
\end{exe}

In (\ref{ex:laXtCha.ci.nWnW}) although \forme{nɯnɯ} follows \forme{ci}, it has scope over the both preceding phrases, which are left-dislocated and followed by a pause.

\begin{exe}
\ex \label{ex:laXtCha.ci.nWnW}
\gll  kɤ-xtɕɤr tɕe nɯnɯ tɕe tɕe iɕqʰa, [[tʰɯci tɯmbri tɤ-ri kɯ-fse kɯ] [laχtɕʰa ci] nɯnɯ], ci kú-wɣ-sɯ-pa tɕe, kú-wɣ-xtɕɤr, \\
\textsc{inf}-attach \textsc{lnk} \textsc{dem} \textsc{lnk} \textsc{lnk} the.aforementioned something rope \textsc{indef}.\textsc{poss}-thread \textsc{nmlz}:S/A-be.like \textsc{erg} thing \textsc{indef} \textsc{dem} one \textsc{ipfv}-\textsc{inv}-\textsc{caus}-do \textsc{lnk} \textsc{ipfv}-\textsc{inf}-attach \\
\glt ``To attach' (means), to put together, attach something with something like a rope or a thread.'  (150902 kAxtCAr, 2-3)
\end{exe}

The aforementioned topic marker \forme{iɕqʰa} (§ \ref{sec:iCqha}) is almost always used with definite referents when prenominal, as in (\ref{ex:RnWz.nWnW}) above, and is the closest candidate fro analysis as a definiteness marker in Japhug. It does occur with non-specific generic referents as in (\ref{ex:lWlAmu}), including some that are very clearly indefinite as in (\ref{ex:lApWG}); note the absence of postnominal determiner \forme{nɯ} (\ref{ex:lApWG}).

\begin{exe}
\ex \label{ex:lWlAmu}
 \gll iɕqʰa lɯlɤmu nɯ tʰɯ-rɤpɯ tɕe tɕe ɯ-sŋi tɕe kɤ-nɯ-rŋgɯ nɯ stʰɯci mɯ́j-tsu ma ɯ-pɯ ra χse ɲɯ-ra tɕe, \\
 the.aforementioned female.cat \textsc{dem} \textsc{ipfv}-bear.young \textsc{lnk} \textsc{lnk} \textsc{3sg}.\textsc{poss}-day \textsc{lnk} \textsc{inf}-\textsc{auto}-lie.down \textsc{dem} so.much \textsc{neg}:\textsc{sens}-have.time.to \\
 \glt `A/the female cat (unlike male cats), when it had had youngs, does not have time to sleep during the day, as it has to feed its youngs.' (21-lWLU, 
\end{exe}

\begin{exe}
\ex \label{ex:lApWG}
\gll  iɕqʰa lɤpɯɣ ɯ-rɣi ʑo fse. \\
the.aforementioned radish \textsc{3sg}.\textsc{poss}-seed \textsc{emph} be.like:\textsc{fact} \\
\glt `It looks like a radish seed.' (hist-26-qro-fourmi, 61)
\end{exe}

In  (\ref{ex:laXtCha.ci.nWnW}), \forme{iɕqʰa}  also precedes two phrases involving indefinite referents, but  there is a marked pause, and this is a case of \forme{iɕqʰa} in its function as speech filler (see § XXX).

\subsubsection{Absence of definiteness marking}
Like many languages (\citealt[130]{creissels06sgit1}), Japhug uses bare nouns without any definiteness marking. Bare nouns are most often non-referential, as \japhug{tɕʰeme}{girl} in (\ref{ex:tCheme.tWtAtu}).

\begin{exe}
\ex \label{ex:tCheme.tWtAtu}
\gll ʁnaʁna tɕʰeme tɯ\redp{}tɤ-tu nɤ, kɤndʑɯsqʰaj tu-kɤ-sɯ-βzu \\
both girl \textsc{cond}\redp{}\textsc{pfv}-exist \textsc{lnk} \textsc{coll}:sister \textsc{ipfv}-\textsc{inf}-\textsc{caus}-make \\
\glt `If both of them have girls, let them be sisters.' (zrAntCW, 4)
\end{exe}

Bare nouns are less common with referential nouns (except in answers to questions), but examples can be found, as \japhug{qacʰɣa}{fox} in (\ref{ex:qachGa.kW}).

\begin{exe}
\ex \label{ex:qachGa.kW}
\gll qacʰɣa 	kɯ maχtɕɯ tɤ-tɯt-a nɯ mɤ-tɯ-ste ti ɲɯ-ŋu \\
fox \textsc{erg} I.told.you.so \textsc{pfv}-say[II]-\textsc{1sg} \textsc{dem} \textsc{neg}-2-do.like[III]:\textsc{fact} say:\textsc{fact} \textsc{sens}-be \\
\glt `The fox says: `You do not do as I told you to." (2003qachGa, 44)
\end{exe}

Personal names generally occur as bare nouns, without any definiteness marker as in (\ref{ex:WrJAnpanma}), but there are no constraints against co-occurrence of personal names with the determiner \forme{nɯ} either (see § \ref{sec:personal.names.modifiers}).

\begin{exe}
\ex \label{ex:WrJAnpanma}
\gll  ɯrɟɤnpanma kɯ ʁlaŋsaŋtɕhin ɯ-ɕki  \\
 Padmasambhava \textsc{erg} Gesar \textsc{3sg}-\textsc{dat} \\
\glt `Padmasambhava (told) Gesar.' (Gesar, 2)
\end{exe}

 \subsection{Topic markers} \label{sec:topic}
 
 \subsubsection{Aforementioned topic} \label{sec:iCqha}
 The marker \japhug{iɕqʰa}{the aforementioned}  is used on referents that have been previously mentioned in the same story, usually only a few sentences back. It is strictly prenominal. 
 
Example (\ref{ex:iCqha.aforementioned}) illustrates the most typical use of this marker. Sentence (\ref{ex:kAtWm}) introduces a new reference, \japhug{kɤtɯm}{ball of thread} marked with the indefinite article \forme{ci} (§ \ref{sec:indef.article}). Three clauses later in (\ref{ex:iCqha.kAtWm}), the same referent occurs again with two topic markers, the postnominal \textit{nɯ} and the prenominal \textit{iɕqʰa}.
 
 
\begin{exe}
\ex \label{ex:iCqha.aforementioned}
\begin{xlist}
\ex \label{ex:kAtWm}
\gll `razri \textbf{kɤtɯm} \textbf{ci} ɲɯ-ra, taqaβ ci ɲɯ-ra' to-ti qʰe   \\
 thread ball \textsc{indef} \textsc{sens}-need needle \textsc{indef} \textsc{sens}-need \textsc{ifr}-say \textsc{lnk}  \\
\glt `He told (Rgyabza) `I need a ball of thread and a needle.''  
\ex  
\gll tɕendɤre ɲɤ-kʰo qʰe,  \\
\textsc{lnk} \textsc{ifr}-give \textsc{lnk}   \\
\glt `She gave it to him.'
\ex 
\gll  tɕe ɯ-ndzɤtsʰi ka-tsɯm-nɯ nɯtɕu qʰe tɕe,   \\
 \textsc{lnk} \textsc{3sg}.\textsc{poss}-meal \textsc{pfv}:3\fl{}3'-bring-\textsc{pl} \textsc{dem}:\textsc{loc}  \textsc{lnk} \textsc{lnk}    \\
\glt `When they brought his meal,'
\ex \label{ex:iCqha.kAtWm}
\gll   \textbf{iɕqʰa} \textbf{kɤtɯm} \textbf{nɯ} ɯʑo kɯ ko-ndo, \\
   the.aforementioned ball \textsc{dem} \textsc{3sg} \textsc{erg} \textsc{ifr}-take \\
\glt `he took the ball of thread, and...' (Gesar 270-272)
\end{xlist}
\end{exe}
 
 
A systematic study of the use of the topic marker \forme{iɕqʰa} in Japhug must overcome two inherent difficulties. First, this topic marker is homophonous with (and historically related to) the speech filler \forme{iɕqʰa} (§ XXX) and with the adverb \japhug{iɕqʰa}{just now}, which can also precede noun phrases. Listening to the sound files can help distinguishing between the three, as the speech filler is always followed by a pause (and optionally by the demonstrative \forme{nɯ}), but there are still ambiguous sentences (see below). Second, \forme{iɕqʰa} occurs on nouns designating entities that the speaker considers to have been previously referred to in the conversation, even if they are not present in the same recording. 

For instance in (\ref{ex:iCqha.pɣArnoR}) the noun \japhug{pɣɤrnoʁ}{a species of fungus} is used with \forme{iɕqʰa}, although this name does not occur before in the same text; it was however mentioned the day before in another recording.

\begin{exe}
\ex \label{ex:iCqha.pɣArnoR}
\gll nɯ zdɯmqe cʰo iɕqʰa, pɣɤrnoʁ nɯni ndʑi-tsʰɯɣa wuma ʑo naχtɕɯɣ. \\
\textsc{dem} fungi.sp. \textsc{comit} the.aforementioned fungi.sp. \textsc{dem}:\textsc{du} \textsc{3du}.\textsc{poss}-form really \textsc{emph} be:identical:\textsc{fact} \\
\glt `The \forme{zdɯmqe} and the \forme{pɣɤrnoʁ} are very similar.' (23-mbrAZim, 82)
\end{exe}

 
The topic marker \forme{iɕqʰa} transparently comes from the adverb \japhug{iɕqʰa}{just now} (§ XXX). The pivot constructions that allowed reanalysis from adverb to prenominal topic marker are very probably headless relatives (§ XXX) as in  (\ref{ex:iCqha.tAtWta}), or complement clauses as in (\ref{ex:iCqha.ZnWzmWnmuta}). 

\begin{exe}
\ex \label{ex:iCqha.tAtWta}
 \gll  [iɕqʰa tɤ-tɯt-a] nɯ tú-wɣ-stu qʰe, \\
 just.now \textsc{ifr}-say[II]-\textsc{1sg} \textsc{dem} \textsc{ipfv}-\textsc{inv}-do.like \textsc{lnk} \\
\glt `One does as I just said, and...' (2002tWsqar, 139)
\end{exe}

\begin{exe}
\ex \label{ex:iCqha.ZnWzmWnmuta}
 \gll iɕqʰa [ʑ-nɯ-z-mɯnmu-t-a] nɯ mɯ-pjɤ-pe rcama.  \\
the.aforementioned  \textsc{transloc}-\textsc{pfv}-\textsc{caus}-\textsc{move}-\textsc{pst}:\textsc{tr}-\textsc{1sg} \textsc{dem} neg-\textsc{ifr}.\textsc{ipfv}-be.good \textsc{fsp} \\
\glt `It was probably not a good thing that I had moved them (as I said above).' (150819 kumpGa, 45)
 \end{exe}
 
 These sentences are still synchronically ambiguous in Japhug; in  (\ref{ex:iCqha.ZnWzmWnmuta}) the context makes it clear that \forme{iɕqʰa} is the topic marker (since the fact of having moved (the eggs) had been told a few sentences back) and not an adverb `just now' with a temporal reference in the past, as the meaning would be `it was probably not a good thing that I had just moved them' (an impossible interpretation in this context, since this sentence is an explanation why several eggs had not given chicks, several days after they had been brought to another place). However, extracted from the context, both interpretation would be equally possible for (\ref{ex:iCqha.ZnWzmWnmuta}), and correspond to two distinct syntactic structures.

With postnominal (§ XXX) or left-headed head-internal relative clauses (§ XXX) as in (\ref{ex:tWrpa.thafse}), \forme{iɕqʰa} can also be ambiguous. Since the adverb \japhug{iɕqʰa}{just now} can occur both before the object (\ref{ex:tWrpa.thWfseta}) or before the verb (\ref{ex:tWrpa.thWfseta2}) in an independent clause, a relative such as (\ref{ex:tWrpa.thafse}) can be either interpreted `the axe (mentioned above) that he had whetted' (with the topic marker \forme{iɕqʰa} outside of the relative clause, having scope on its head) and `the axe that he had just whetted' with the adverb \japhug{iɕqʰa}{just now} inside the relative clause.

 \begin{exe}
\ex \label{ex:tWrpa.thafse}
 \gll  tɕendɤre <luban> kɯ iɕqʰa [tɯrpa tʰa-fse] nɯ to-ndo tɕe, \\
 \textsc{lnk} p.n. \textsc{erg} the.aforementioned axe \textsc{pfv}:3\fl{}3'-whet \textsc{dem} \textsc{ifr}-take \textsc{lnk} \\
 \glt `Luban took the axe that he had whetted.' (150902 luban-zh, 90)
 \end{exe}

  \begin{exe}
  \ex 
  \begin{xlist}
\ex \label{ex:tWrpa.thWfseta}
 \gll   iɕqʰa tɯrpa tʰɯ-fse-t-a \\
just.now axe \textsc{pfv}-whet-\textsc{pst}:\textsc{tr}-\textsc{1sg} \\
\ex \label{ex:tWrpa.thWfseta2}
 \gll   tɯrpa  iɕqʰa tʰɯ-fse-t-a \\
 axe just.now \textsc{pfv}-whet-\textsc{pst}:\textsc{tr}-\textsc{1sg} \\
 \glt `I just whetted a/the axe.' (elicited)
 \end{xlist}
 \end{exe}

The use of \forme{iɕqʰa} as a topic marker with nouns (as in \ref{ex:iCqha.kAtWm} above) probably took place by reanalysis of the adverb in headless or postnominal relatives, or in complment clauses as above, then generalized to all noun phrases even those without subordinate clause.

 \subsection{Focus markers} \label{sec:focus}
  \subsubsection{Unexpected focus} \label{sec:unexpected}
 \subsubsection{Additive focus} \label{sec:kWnA}
The additive focus marker \japhug{kɯnɤ}{also} always immediately follows the constituent over which it has scope, which can be noun phrases, postpositional phrases but also subordinate clause.
 
The marker \japhug{kɯnɤ}{also} however is incompatible with the ergative \forme{kɯ}, and the combinations $\dagger$\forme{kɯ kɯnɤ} or $\dagger$\forme{kɯnɤ kɯ} are not grammatical. The contrast between absolutive and ergative noun phrases is therefore neutralized in additive focus.

%It can also express scalar focus
 \subsubsection{Scalar focus} \label{sec:cinA} 
  \subsubsection{Restrictive focus} \label{sec:restrictive.focus} 
 Japhug does not have a restrictive focus marker `only', and the only way to express this meaning is to combine the exceptive \japhug{ma}{apart from} (and its reduplicated variant \forme{mɯma} § \ref{sec:exceptive}) with a negative predicate. This can be a verb with a negative prefix as in (\ref{ex:XsArZaR}), or a negative existential verb as in (\ref{ex:Wmi.Wntsi.ma.me}).
 
 \begin{exe}
\ex  \label{ex:XsArZaR}
\gll   χsɤ-rʑaʁ ma mɯ-pɯ-tsu-a ɲɤ-sɯso ri χsɯ-xpa pjɤ-tsu tɕe,  \\
three-day apart.from \textsc{neg}-\textsc{pfv}-pass-\textsc{1sg} \textsc{ifr}-think \textsc{lnk} three-year \textsc{ifr}-\textsc{pass} \\
\glt `He thought that he had spent only three days, but three years had passed.' ( 2011-4-smanmi, 178)
  \end{exe}
  
  \begin{exe}
\ex  \label{ex:Wmi.Wntsi.ma.me}
\gll  rkoŋɟɤl nɯnɯ, ɯ-mi ɯ-ntsi nɯ ma me kʰi.   \\
one.legged.demon \textsc{dem} \textsc{3sg}.\textsc{poss}-leg \textsc{3sg}.\textsc{poss}-one.of.a.pair apart.from not.exist:\textsc{fact} \textsc{hearsay} \\
\glt  `It is said that one-legged demons only had one leg.' (140510 rkoNJAl, 4)
  \end{exe}
  
  
The restrictive focus construction implies the presence of a noun phrase with a numeral or a CN when the restriction bears on the quantity, but restriction can also be qualitative, without quantifier, as in (\ref{ex:karGi.Zo.kWfse.ma.me}).

\begin{exe}
\ex \label{ex:karGi.Zo.kWfse.ma.me}
 \gll   ɯ-mat nɯnɯ na-lɤt ɕɯmɯma nɤ kɯ-ndɯ\redp{}ndɯβ ʑo ma me, karɣi ʑo kɯ-fse ma me  \\
 \textsc{3sg}.\textsc{poss}-fruit \textsc{dem} \textsc{pfv}:3\fl{}3'-throw just \textsc{lnk}  \textsc{nmlz}:S/A-\textsc{emph}\redp{}small \textsc{emph} apart.from not.exist:\textsc{fact} turnip.seed \textsc{emph} \textsc{nmlz}:S/A-be.like apart.from not.exist:\textsc{fact} \\
 \glt  `When the fruit of (xanthoxyllum) has just come out, there is only something very small, only like a turnip seed.'  (07-tCGom, 7)
  \end{exe}
  
\subsection{Genitival phrases}
%
%tɕendɤre ɯ-jaʁ nɯtɕu ftsoʁ kɯngɯt ɯ-phɯ ɣɯ srɯnloʁ pjɤ-k-ɤrku-ci
%2003gesar, 239
%
%\subsection{Determiners} \label{sec:determiners}
%\japhug{ɕɯŋarɯra}{each better than the other}
% rɟɤlpu ɕɯŋarɯra kɯ ta-tʰu-nɯ ɕti ri, mɯ-tɤ-nɤla-j ɕti tɕe,
% 2003 qachga, 71
\subsection{Identity modifiers} \label{sec:identity.modifier}

%nɤki tɕheme nɯ ɯ-ɕki ɯ-kɯ-sɤja jo-ɕe, ci tɕheme kɯ-ŋɤn nɯ ɯ-ɕki.

%ci qhɤjmbaʁ nɯ kɯ-jaʁ kɯ-fse nɯnɯ 
%mtshalu ɯ-cu tɕe nɤki,
%tɯ-mgo zmɤrɤβ kú-wɣ-nɯ-lɤt sna.
%16-RlWmsWsi
%li ci /ɯt/ ɯ-tɯphu nɯ tɤpu qhɤjmbaʁ tu-ti-nɯ ŋu tɕe,
\subsection{Attributes}

\section{The structure of the noun phrase}

\section{Nominal predicates}
