\chapter{The noun phrase} \label{chap:noun.phrase}

\section{Independent words vs clitics}
The present chapter deals with grammatical elements that are independent words rather than affixes, like those described in chapter \ref{chap:nominal.morphology}. Since some scholars such as 
\citet{jackson98morphology, jackson14morpho} treat the postpositions and the number modifiers as clitics rather than independent words as is done in the present work, a justification of my analysis is necessary.

The postpositions \japhug{kɯ}{ergative} and  \japhug{ɣɯ}{genitive} do have some clitic-like characteristics: they cannot be used without a preceding noun phrase (or subordinate clause in some cases, see § XXX), are unstressed, and in the case of the genitive have special irregular forms with pronouns (§ \ref{sec:pronouns.gen}).

However, a pause can occur between these postpositions (\ref{ex:kW.nAmWmnW}) and the noun phrase they follow. For instance, in example (\ref{ex:kW.nAmWmnW}), a two second pause (with an inspiration) is found between the phrase \forme{nɯŋa ra} and the following ergative \forme{kɯ}. 

\begin{exe}
\ex \label{ex:kW.nAmWmnW}
\gll tɕe tɯrtsi nɯ pjɯ́-wɣ-βzu tɕe, nɯŋa ra, kɯ nɤ-mɯm-nɯ cʰo wuma ʑo ɣɯ-ɕɯ-fka-nɯ \\
\textsc{lnk} cow.food \textsc{dem} \textsc{ipfv}-\textsc{inv}-make \textsc{lnk} cow \textsc{pl} \textsc{erg} \textsc{trop}-be.tasty:\textsc{fact}-\textsc{pl} \textsc{comit} really \textsc{emph} \textsc{inv}-\textsc{caus}-be.satiated:\textsc{fact}-\textsc{pl} \\
\glt `They make cow food with flour, the cows find it tasty, and it satisfies their hunger.' (140513 tWrtsi, 15)
\end{exe}


Such cases are by no means exceptional; at least 54+35 examples of ergative and genitive preceded by a pause are attested in the corpus (they can be found by searching \forme{kɯ} or \forme{ɣɯ}  preceded by a comma). Most of these cases are found in sentences where the speaker hesitates, and are especially common in texts translated from Chinese.

\section{Postpositions} \label{ex:postpositions}

\subsection{Absolutive} \label{sec:absolutive}
\subsubsection{Intransitive subject}
\subsubsection{Object}
\subsubsection{Semi-object}
\subsubsection{Theme}
\subsubsection{Essive} \label{sec:essive.abs}
%tsuku kɯ paʁndza ɲɯ-nɯ-phɯt-nɯ ɲɯ-ŋu ri,
\subsubsection{Goal} \label{absolutive.goal}
\subsubsection{Locative adjunct} \label{absolutive.locative}

\subsection{Ergative} \label{sec:erg.kW}
\subsubsection{Transitive subject} \label{sec:A.kW}
\subsubsection{Instrumental} \label{sec:instr.kW}

%manner
%kumpɣa cho khɯna ni li tɤ-mqe tɤ-ndɯt kɯ jo-ɣi-ndʑi tɕe,
\subsubsection{Causee} \label{sec:causee.kW}
\subsubsection{Comparee marker} \label{sec:comparee.kW}

%mahi nɯnɯ kɯ aʑo sɤz cha
\subsubsection{Partitive} \label{sec:partitive.kW}
\subsubsection{Oblique argument} \label{sec:oblique.kW}
 The transitive verb \japhug{kʰɤt}{do repeatedly, do a long time} and its causative form \japhug{sɯ-kʰɤt}{cause to do repeatedly, cause to do a long time} occur in a construction with instrumental-like noun phrases marked with the ergative \forme{kɯ}, indicating the action which is performed repeatedly or done over a long time. These noun phrases can include either an action nominal derived from a verb with the prefix \forme{tɯ-} (§ XXX) as in (\ref{ex:tWqioR.kW}), or an underived action noun, as in (\ref{ex:tama.kW.takhAt}) and (\ref{ex:khAcAl.kW.takhata}).  
 
  \begin{exe}
\ex \label{ex:tWqioR.kW}
\gll tɯ-qioʁ kɯ tó-wɣ-sɯ-kʰɤt ʑo tɕe, tɕe nóʁmɯz nɤ tɯɣ nɯnɯ ló-wɣ-sɯ-tɕɤt  \\
\textsc{nmlz:action}-vomit \textsc{erg} \textsc{ifr-inv-caus}-do.a.long.time \textsc{emph} \textsc{lnk} \textsc{lnk} only.then \textsc{lnk} poison \textsc{dem} \textsc{ifr-inv-caus}-take.out \\
\glt `(The medicine) caused (Gesar) to vomit a long time until he expelled the poison.' (Gesar, 266)
\end{exe}

  \begin{exe}
\ex \label{ex:tama.kW.takhAt}
\gll ta-ma kɯ ta-kʰɤt ʑo  \\
\textsc{indef.poss}-work \textsc{erg} \textsc{pfv}:3$\rightarrow$3'-do.a.long.time \textsc{emph} \\
\glt `He did a lot of work.' (elicited)
\end{exe}

Example (\ref{ex:khAcAl.kW.takhata}), with the verb \japhug{kʰɤt}{do repeatedly, do a long time}  taking \textsc{1sg}\fl{}3 indexation (§ XXX), shows that the ergative phrase cannot be analyzed as a transitive subject; moreover, the fact that adding the causative in this case would imply a real causative interpretation (`cause X to repeatedly') also indicates that this phrase is not an instrumental adjunct (see § \ref{sec:instr.kW}).

  \begin{exe}
\ex \label{ex:khAcAl.kW.takhata}
\gll kʰɤcɤl kɯ tɤ-kʰat-a ʑo \\
conversation \textsc{erg} \textsc{pfv}-do.a.long.time-\textsc{1sg} \textsc{emph} \\
\glt `I have a long conversation.' (elicited)
\end{exe}

No other verb takes this type of oblique ergative phrase.

\subsection{Genitive} \label{sec:genitive}
With the exception of particular forms for some pronouns (§ \ref{sec:pronouns.gen}), the genitive postposition has the invariant form \forme{ɣɯ} in Kamnyu Japhug. Like the ergative \forme{kɯ}, it is likely borrowed from the Amdo clitic \forme{-ɣə/-kə} (\citealt[62]{haller04themchen}). It is used in possessive contructions, but also expresses beneficiary and recipient.

\subsubsection{Possession} \label{sec:gen.possession}
The genitive \forme{ɣɯ} occurs in various type of possessive constructions, including genitival noun complements and possessive existential predicates (§ XXX).

Inside the noun phrase, the genitive occurs between possessor and possessum, and a possessive prefix is found on the possessum (§ \ref{ex:prefix.expression.of.possession}), as in (\ref{ex:GZAndza.GW.WjwaR}).  

\begin{exe}
\ex \label{ex:GZAndza.GW.WjwaR}
\gll ri ɣʑɤndza ɣɯ ɯ-jwaʁ nɯra mɤ-wxti ri, ɲaʁ ʑo qhe, \\
\textsc{lnk} Agastache.rugosa \textsc{gen} \textsc{3sg}.\textsc{poss}-leaf \textsc{dem}:\textsc{pl} \textsc{neg}-be.big:\textsc{fact} \textsc{lnk} be.black:\textsc{fact} \textsc{emph} \textsc{lnk} \\
\glt `The leaves of the Agastache rugosa are not large and quite dark in colour.' (11-qarGW, 137)
\end{exe}

Genitival phrases without possessive prefix on the possessum are rare but do exist, in particular when the possessum is a noun borrowed from Chinese and non-fully nativized like \ch{国语}{guóyǔ}{national language} in (\ref{ex:iZo.GW.guoyu}).  

\begin{exe}
\ex \label{ex:iZo.GW.guoyu}
\gll iʑo ɣɯ <guoyu> ɲɯ-ŋu tɕe, nɯnɯ kɤsɯfse ɣɯ ji-rju ɲɯ-ŋu tɕe, \\
\textsc{1pl} \textsc{gen} national.language \textsc{sens}-be \textsc{lnk} dem all \textsc{gen} \textsc{1pl}.\textsc{poss}-speech \textsc{sens}-be \textsc{lnk} \\
\glt `(Chinese) is our national language, it is everybody's language.' (150901 tshuBdWnskAt, 15-16)
\end{exe}

For singular noun possessors, the presence or not of a third person possessive prefix \forme{ɯ-} is not always easy to tell from recordings, as due to the external sandhi (§ XXX), \forme{ɣɯ ɯ-} merges as \ipa{ɣɯ} when no pause occurs between the two. In careful speech, the third person prefix is clearly audible.

Nominal modifiers can sometimes be marked like possessors, with the genitive and/or with a possessive prefix on the following head noun, see § \ref{sec:gen.other}. 

The genitive can also appear between a noun phrase and a relator noun, and even be followed by focus markers in this position, as in (\ref{ex:GW.kWnA.WrkW.ri}).

\begin{exe}
\ex \label{ex:GW.kWnA.WrkW.ri}
\gll   tɯ-ci kɯ-wxti ɣɯ kɯnɤ ɯ-rkɯ ri nɯra tu ŋgrɤl.  \\
\textsc{indef}.\textsc{poss}-water \textsc{nmlz}:S/A-be.big \textsc{gen} also \textsc{3sg}.\textsc{poss}-side \textsc{loc} \textsc{dem}:\textsc{pl} exist:\textsc{fact} be.usually.the.case:\textsc{fact} \\
\glt `(Dragonflies) are also found near rivers.' (26-quspunmbro, 7)
\end{exe}

In these constructions, the genitive is always optional, and the prefix on the possessum suffices to express possession, as in (\ref{ex:paXCi.WjwaR}) (see § \ref{ex:prefix.expression.of.possession}).

\begin{exe}
\ex \label{ex:paXCi.WjwaR}
\gll paχɕi ɯ-jwaʁ tsa fse ri, nɯ sɤznɤ artɯm,\\
apple \textsc{3sg}.\textsc{poss}-leaf a.little be.like:fact \textsc{lnk} \textsc{dem} \textsc{comp} be.round:\textsc{fact} \\
\glt `(Its leaves) are a little like the leaves of an apple tree, but more round.' (09-mi, 15)
\end{exe}

When the possessum is elided however, the genitive postposition becomes obligatory, as in (\ref{ex:baigua.GW.sAz}).

\begin{exe}
\ex \label{ex:baigua.GW.sAz}
\gll ɯ-rɣi nɯnɯ, nɤki, <beigua> ɣɯ sɤz ɲɯ-jaʁjɯ. \\
\textsc{3sg}.\textsc{poss}-seed \textsc{dem} \textsc{filler}  pumpkin \textsc{gen} \textsc{comp} \textsc{sens}-be.thick.and.strong \\
\glt `Its seeds are thicker than those of the pumpkin.' (16-CWrNgo, 130)
\end{exe}

While there are transitive and semi-transitive verbs expressing possession (§ XXX), the most common possessive construction involves an existential verb taking the possessum as subject, with the possessor marked by a possessive prefix on the possessum, and optionally with the genitive, as in (\ref{ex:phu.nW.GW.WRrW.GAZu}). 

\begin{exe}
\ex \label{ex:phu.nW.GW.WRrW.GAZu}
\gll qartsʰaz pʰu nɯ ɣɯ ɯ-ʁrɯ ɣɤʑu \\
deer male \textsc{dem} \textsc{gen} \textsc{3sg}.\textsc{poss}-horn exist:\textsc{sens} \\
\glt `The male deer has horns.' (27-qartshAz, 32)
\end{exe}

This construction is also used for abstract possession, as in (\ref{ex:aZWG.aBlu.tu}).

\begin{exe}
\ex \label{ex:aZWG.aBlu.tu}
\gll aʑɯɣ a-βlu tu \\
\textsc{1sg}:\textsc{gen} \textsc{1sg}.\textsc{poss}-stratagem exist:\textsc{fact} \\
\glt `I have an idea.' (140507 tangguowu, 29)
\end{exe}

The causative verbs \japhug{ɣɤtu}{cause to have} and \japhug{ɣɤme}{cause not to have, destroy} derived from \japhug{tu}{exist} and \japhug{me}{not exist} respectively (see § XXX) select an oblique argument with the genitive, as in (\ref{ex:WZo.GW.tuGAtea}). Although this argument could be considered to be a type of beneficiary (§ \ref{sec:other.uses.poss.prefixes}), we observe here stability in case marking of the possessor between the base construction and the derived causative one.

\begin{exe} 
\ex \label{ex:WZo.GW.tuGAtea} 
\gll ɯʑo kɯ maka kɤ-ntɕʰoz mɤ-kɯ-ɤrɕo kɯ-fse ʑo tɯrɟɯ laχtɕʰa ɯʑo ɣɯ tu-ɣɤte-a jɤɣ \\ 
\textsc{3sg}.\textsc{poss} \textsc{erg} at.all \textsc{inf}-use \textsc{neg}-\textsc{inf}:\textsc{stat}-be.finished \textsc{inf}:\textsc{stat}-be.like \textsc{emph} wealth thing \textsc{3sg}.\textsc{poss} \textsc{gen} \textsc{ipfv}-\textsc{caus}-exist[III]-\textsc{1sg} be.possible:\textsc{fact} \\ 
\glt `(If someone saves me), I will make him have more wealth and riches than he can ever use.' (140512 yufu yu mogui, 84) 
\end{exe} 
%ma aʑo a-kɤ-cha,  a-kɤ-cha kɯ-tu nɯra, a-kɤ-spa, tu-βze-a kɤ-cha nɯra lonba ʑo nɤʑɯɣ tɤ-ɣɤtu-t-a ɕti tɕe,

Not all combinations of existential verbs and genitival phrases are existential possessive constructions. For instance, in (\ref{ex:BZW.GW.WqiW}), the second clause could appear to contain a possessive construction meaning `the mouse only has half of it', but the context makes it clear that a different interpretation is necessary (see § XXX on this use of the existential verbs).

\begin{exe}
\ex \label{ex:BZW.GW.WqiW}
\gll qamtɕɯr nɯ ɯ-mtɕʰi nɯnɯ βʑɯ sɤznɤ mɤʑɯ ʑo amtɕoʁ tɕe nɯ βʑɯ ɣɯ ɯ-qiɯ kɯnɤ me \\
shrew \textsc{dem} \textsc{3sg}.\textsc{poss}-mouth \textsc{dem} mouse \textsc{comp} yet \textsc{emph} be.pointy \textsc{lnk} \textsc{dem} mouse \textsc{gen} \textsc{3sg}.\textsc{poss}-half even not.exist:\textsc{fact} \\
\glt `The shrew's mouth is even sharper than that of the mouse, and (its size) is not even half that of the mouse.' (27-spjaNkW, 204-205)
\end{exe}

 
\subsubsection{Recipient and beneficiary} \label{sec:gen.beneficiary}
 
The genitive is selected to mark the recipient by the indirective verb \japhug{kʰo}{give, pass over}, as in (\ref{ex:aZWG.nWkhAm}) and (\ref{ex:GW.anWtWkhAm}).   

\begin{exe}
\ex \label{ex:aZWG.nWkhAm}
 \gll ɕɯ ʑo stu kɯ-mɤku pɯ-tɯ-mto-t nɯnɯ, laχtɕha pɯ-nnɯ-ŋu, tɯrme pɯ-nnɯ-ŋu nɯ, aʑɯɣ nɯ-kʰɤm tɕe tɕendɤre, aʑo ɲɯ-ta-lɤt jɤɣ \\
 who \textsc{emph} most \textsc{nmlz}:S/A-be.first \textsc{pfv}-2-see-\textsc{pst}:\textsc{tr} \textsc{dem}  thing \textsc{pst}.\textsc{ipfv}-\textsc{auto}-be   person \textsc{pst}.\textsc{ipfv}-\textsc{auto}-be \textsc{dem} \textsc{1sg}:\textsc{gen} \textsc{imp}-give[III] \textsc{lnk} \textsc{lnk} \textsc{1sg} \textsc{ipfv}-1\fl{}2-release be.possible:\textsc{fact} \\
 \glt `Give me the first thing you see (when you go back home), be it a person or an object, and I will release you.' (140506 shizi he huichang de bailingniao-zh, 50-52)
\end{exe}

\begin{exe}
\ex \label{ex:GW.anWtWkhAm}
 \gll jɤ-tsɯm tɕe iɕqʰa nɯ kɯβʁa nɯ ɣɯ a-nɯ-tɯ-kʰɤm \\
 \textsc{imp}-take.away \textsc{lnk} the.aforementioned \textsc{dem} noble \textsc{dem} \textsc{gen} \textsc{irr}-\textsc{pfv}-2-give[III] \\
 \glt  Take it and give it to the nobleman.' (150831 renshen wawa-zh, 43)
\end{exe}
 
The recipient of the verb  \japhug{kʰo}{give, pass over} can alternatively also be marked by a possessive prefix on the IPN \japhug{tɯ-jaʁ} (with the meaning  hand over', \ref{sec:semi.grammaticalized.relator}) or, most commonly, with the dative \forme{ɯ-ɕki} or \forme{ɯ-pʰe} (§ \ref{sec:dative}).

The genitive is selected by a few intransitive modal verbs to indicate the experiencer/beneficiary, in particular  \japhug{ra}{need, have to}, \japhug{ʁzi}{be necessary}, as in (\ref{ex:aZWG.WCArW}) and (\ref{ex:aZWG.Rzi}).

\begin{exe}
\ex \label{ex:aZWG.WCArW}
 \gll aʑɯɣ ɯ-ɕɤrɯ ra \\
 \textsc{1sg:gen} \textsc{3sg.poss}-bone have.to:\textsc{fact} \\
\glt `I want its bones.' (07-deluge, 9)
\end{exe}

\begin{exe}
\ex \label{ex:aZWG.Rzi}
 \gll aʑɯɣ wuma ʑo ʁzi ɲɯ-ŋu, a-kɤ-ntɕʰoz sna ɲɯ-ŋu \\
  \textsc{1sg:gen} really \textsc{emph} be.necessary:\textsc{fact} \textsc{sens}-be \textsc{1sg}.\textsc{poss}-nmlz:P-use be.good:\textsc{fact}  \textsc{sens}-be \\
  \glt `It will be useful for me, it will have good use of it.'  (150902 hailibu-zh, 44-45)
\end{exe}

The experiencer/beneficiary can also be marked by possessive prefixes on the subject, without genitive, as in 
(\ref{ex:arNWl.mAra}) (see also § \ref{sec:other.uses.poss.prefixes} for additional examples).

\begin{exe}
\ex \label{ex:arNWl.mAra}
 \gll aʑo a-rŋɯl a-χsɤr ra mɤ-ra \\
 \textsc{1sg} \textsc{1sg}.\textsc{poss}-silver \textsc{1sg}.\textsc{poss}-gold \textsc{pl} \textsc{neg}-have.to:\textsc{fact} \\
 \glt `I don't  need silver or gold.' (2014-kWlAG, 367)
\end{exe}

The genitive also occurs with beneficiaries/maleficiaries as adjuncts, not selected by the main verb, with transitive verbs such as \japhug{nɤma}{do} (\ref{ex:tChi.tunAmea}) and \japhug{wum}{collect} (\ref{ex:WZAG.pjAmaR}) or stative intransitive verbs such as \japhug{pe}{be good} as in (\ref{ex:aZWG.mApe}).

\begin{exe}
\ex \label{ex:tChi.tunAmea}
\gll nɤʑɯɣ tɕʰi tu-nɤme-a ra, tɤ-ti  \\
\textsc{2sg}:\textsc{gen} what \textsc{ipfv}-do[III]-\textsc{1sg} have.to:\textsc{fact} \textsc{imp}-say \\
\glt `Tell me what I shall do for you.' (140511 alading-zh, 175)
\end{exe}

\begin{exe}
\ex \label{ex:aZWG.mApe}
\gll  ɯ-fso tʰɯ-wxti tɕe aʑɯɣ mɤ-pe \\ 
\textsc{3sg}.\textsc{poss}-tomorrow \textsc{pfv}-be.big \textsc{lnk} \textsc{1sg}:\textsc{gen} \textsc{neg}-be.good:\textsc{fact} \\
\glt `In the future, when he will have grown up, he will cause me trouble.' (`he will not be good to me', 2011-05-nyima, 22)
\end{exe}

The beneficiary adjunct is not necessarily contiguous with the verb on which it depends, as in (\ref{ex:iZora.GW.tChi.tufsej}) where the genitive phrase \forme{iʑora ɣɯ} `for us, on our behalf'  is separated from the verb \japhug{tʰu}{ask} by a lengthy complement comprising two clauses.

\begin{exe}
\ex \label{ex:iZora.GW.tChi.tufsej}
\gll  iʑora ɣɯ [tɕʰi tu-fse-j tɕe ji-tɯ-ci ɣɤʑu] tu-tɯ-tʰe ɯ-tɯ́-cʰa? \\
\textsc{1pl} \textsc{gen} what \textsc{ipfv}-be.like-\textsc{1pl} \textsc{lnk} \textsc{1pl}.\textsc{poss}-\textsc{indef}.\textsc{poss}-water exist:\textsc{sens} \textsc{ipfv}-2-ask[III] \textsc{qu}-2-can:\textsc{fact} \\
\glt `Can you ask on our behalf how we should do to have water?' (2005tamukatsa, 14)
\end{exe}

Beneficiary genitive phrases can occur as predicates with a copula as \japhug{ɯʑɤɣ}{\textsc{3sg}:\textsc{gen}} in (\ref{ex:WZAG.pjAmaR}).

 \begin{exe}
\ex \label{ex:WZAG.pjAmaR}
\gll   tʰoʁtɤm ka-wum tɕe, ɯʑɤɣ pjɤ-maʁ kɯ, tɕoχtsi rɟɤlpu ɣɯ ku-wum,  \\
taxes \textsc{pfv}:3\fl{}3'-collect \textsc{lnk} \textsc{3sg}:\textsc{gen} \textsc{ifr}.\textsc{ipfv}-not.be \textsc{erg} p.n. king \textsc{gen} \textsc{ipfv}-collect \\
\glt `The taxes that he had collected were not for himself, he was collecting them for the king of Cogtse.' (150901 NAjstsa, 28)
\end{exe}

In this use too, it is alternatively possible to indicate the beneficiary as a possessive prefix on the object, without genitive postposition, as in (\ref{ex:atWci.tArke}).

 \begin{exe}
\ex \label{ex:atWci.tArke}
\gll   χsɤr kʰɯtsa ɯ-ŋgɯ nɯtɕu a-tɯ-ci ci tɤ-rke ma wuma ɲɯ-ɕpaʁ-a \\
gold bowl \textsc{3sg}.\textsc{poss}-inside \textsc{dem}:\textsc{loc} \textsc{1sg}.\textsc{poss}-\textsc{indef}.\textsc{poss}-water a.little \textsc{imp}-put.in[III] \textsc{lnk} really \textsc{sens}-be.thirsty-\textsc{1sg} \\
\glt  `Please pour some water in the golden bowl for me, I am thirsty.' (140428 mu e guniang-zh, 47)
\end{exe}

The genitive is also attested with a noun-verb collocations (§ XXX), like \japhug{ɯ-kɤrnoʁ+mtɕɯr}{feel dizzy}, in which the possessor  of the noun is an experiencer as in (\ref{ex:fsapaR.GW.kWnA}). This example also illustrates the use of the genitive followed by a focus marker, as (\ref{ex:GW.kWnA.WrkW.ri}) above.

\begin{exe}
\ex \label{ex:fsapaR.GW.kWnA}
\gll tɕeri fsapaʁ ɣɯ kɯnɤ ɯ-kɤrnoʁ ɲɯ-mtɕɯr ɲɯ-ŋu \\
\textsc{lnk} animal \textsc{gen} also \textsc{3sg}.\textsc{poss}-head \textsc{sens}-turn \textsc{sens}-be \\
\glt `But animals too can feel dizzy.' (29-tAmtshAzkAkWndo, 71)
\end{exe}


\subsubsection{Other uses} \label{sec:gen.other}
The genitive \forme{ɣɯ} occurs with various types of noun complements which are semantically neither possessive or beneficiaries/recipients. 

Nouns used as prenominal modifiers are in rare cases followed by a genitive postposition before the head noun. If the head noun is an APN, the presence of a third singular possessive prefix \forme{ɯ-} is optional, as shown by examples such as (\ref{ex:χsAr.GW.khWtsa}) and (\ref{ex:ftsoR.kWngWt.WphW}). 

\begin{exe}
\ex \label{ex:χsAr.GW.khWtsa}
\gll  χsɤr ɣɯ, nɤkinɯ, kʰɯtsa ci to-nɯ-ndo. \\
gold \textsc{gen} \textsc{filler} bowl \textsc{indef} \textsc{ifr}-\textsc{auto}-take \\
\glt `He took a golden bowl' (140508 shier ge tiaowu de gongzhu-zh, 158)
\end{exe}

This type of construction is most common in texts translated from Chinese, but does also occur in more spontaneous material as in (\ref{ex:ftsoR.kWngWt.WphW}), with a complex modifier \forme{ftsoʁ kɯngɯt ɯ-pʰɯ} `the price of nine female hybrid yaks'.

\begin{exe}
\ex \label{ex:ftsoR.kWngWt.WphW}
\gll tɕendɤre ɯ-jaʁ nɯtɕu [ftsoʁ kɯngɯt ɯ-pʰɯ] ɣɯ srɯnloʁ pjɤ-k-ɤ-rku-ci \\
\textsc{lnk} \textsc{3sg}.\textsc{poss} \textsc{dem}:\textsc{loc} female.hybrid.yak nine \textsc{3sg}.\textsc{poss}-price \textsc{gen} ring \textsc{ifr}.\textsc{ipfv}-\textsc{evd}-pass-put.in-\textsc{evd} \\
\glt `She had a ring worth nine female hybrid yak in her hand.' (2003gesar, 239)
\end{exe}

In a construction with a prenominal modifier marker in the genitive, even when a possessive prefix is present on the head noun (in particular when it is an IPN), that prefix does not necessarily refer to the modifier. For instance, in (\ref{ex:tWpAlAskAr.GW.nWmgozmArAB}), the third plural possessive prefix \forme{nɯ-} on \forme{nɯ-mgozmɤrɤβ} `their vegetables' refers to the people eating the vegetable, not the the modifier \japhug{tɯxpalɤskɤr}{the whole year} (on whose formation see § \ref{sec:dvandva.coll}) which would require a third singular prefix instead (an option which is also attested with this noun). Alternatively, it is also possible to have an indefinite possessive prefix on the head noun if it is an IPN, as in (\ref{ex:tWxpa.GW.tWGli}). Note that both options are attested in the construction with a prenominal modifier without the genitive, as seen in § \ref{sec:possessive.prefixes.prenominal}.
 
 \begin{exe}
\ex \label{ex:tWpAlAskAr.GW.nWmgozmArAB}
\gll  tɕe nɯnɯ tɯxpalɤskɤr ɣɯ nɯ-mgozmɤrɤβ nɯ nɯ ma pjɤ-me.  \\
\textsc{lnk} \textsc{dem} whole.year \textsc{gen} \textsc{3pl}.\textsc{poss}-vegetable \textsc{dem} \textsc{dem} apart.from \textsc{ifr}.\textsc{ipfv}-not.exist \\
\glt `It was the only vegetable that they had the whole year.' 
\end{exe}

\begin{exe}
\ex \label{ex:tWxpa.GW.tWGli}
\gll  tɯ-xpa ɣɯ tɯ-ɣli nɯ cʰɯ́-wɣ-tɕɤt tú-wɣ-rmbɯ  \\
one-year \textsc{gen} \textsc{indef}.\textsc{poss}-manure \textsc{dem} \textsc{ipfv}:\textsc{downstream}-\textsc{inv}-take.out \textsc{ipfv}-\textsc{inv}-heap \\
\glt `People take out (from the stable) the whole year's manure and heap it up.' (2010-tArAku)
\end{exe}

Some apparently unclassifiable uses of the genitive can be accounted for to some extent by assuming the elision of a head noun.  For instance, in (\ref{ex:iZora.GW}), the phrase \forme{iʑora ɣɯ}, meaning `in our language', can be explained as coming from \forme{iʑora ɣɯ ji-skɤt} `our language' used as a absolutive locative phrase (§ \ref{absolutive.locative}) `in our language', with elision of the head noun. This example does not illustrate a distinct function of the genitive, it is simply a particular case of possessive.

\begin{exe}
\ex \label{ex:iZora.GW}
\gll  <longtoutan> nɯ kupa-skɤt ɕti. tɕe iʑora ɣɯ tɕʰi tu-kɯ-ti ŋu mɤ-xsi. \\
pl.n. \textsc{dem} Chinese-language be.\textsc{affirm}:\textsc{fact} \textsc{lnk} \textsc{1pl} \textsc{gen} what \textsc{ipfv}-\textsc{genr}-say be:\textsc{fact} \textsc{neg}-\textsc{genr}:know \\
\glt `Longtoutan is a Chinese word; I don't know how it is said in our (language).'  (150820 qaprANar, 32)
\end{exe}

The same is true of the use of the genitive with the verb \japhug{mŋɤm}{be paintful} and its causative \japhug{ɕɯmŋɤm}{cause to be paintful}, which take a body part (not the person or animal feeling pain) as their subject and object respectively. In (\ref{ex:aZWG.taCWmNAm}), the genitive first person \japhug{aʑɯɣ}{\textsc{1sg:gen}} is not an oblique argument or even a malefactive adjunct. Rather, its presence implies an elided noun \japhug{a-βri}{my body} (`he caused pain to my body'). It is however likely that sentences like this are the pivot constructions which made possible the reanalysis of possessive genitive phrases as benefactive/malefactive adjuncts.

\begin{exe}
\ex \label{ex:aZWG.taCWmNAm}
\gll aʑɯɣ ta-ɕɯ-mŋɤm, aʑɯɣ a-laχtɕʰa ra ja-nɯ-tsɯm-nɯ \\
\textsc{1sg}:\textsc{gen} \textsc{pfv}:3\fl{}3'-\textsc{caus}-be.paintful \textsc{1sg}:\textsc{gen} \textsc{1sg}.\textsc{poss}-thing \textsc{pl} \textsc{pfv}:3\fl{}3'-\textsc{vert}-take.away-\textsc{pl} \\
\glt `He hurt me and took away my things.' (140426 luozi he qiangdao, 35)
\end{exe}

The genitive can also occur between prenominal relatives (§ XXX) and their head noun. In this construction the head noun generally does not take a possessive prefix. This type of relative is particularly common in story translated from Chinese, where it calques the prenominal relatives in \zh{的} <de>, as in (\ref{ex:makWra.GW.sAtCha}). The same situation has been observed in Khroskyabs (\citealt[640-643]{lai17khroskyabs}).

\begin{exe}
\ex \label{ex:makWra.GW.sAtCha}
\gll [kɯ-ɣɤndʐo ri kɯ-me], [kɯ-sɤ-mtsɯr ri kɯ-me], [kɤ-nɯsɯmɯzdɯɣ ri mɤ-kɯ-ra] ɣɯ sɤtɕʰa nɯtɕu jo-ɕe-ndʑi ɲɯ-ŋu. \\
\textsc{nmlz}:S/A-be.cold also \textsc{nmlz}:S/A-not.exist \textsc{nmlz}:S/A-\textsc{deexp}-be.hungry also \textsc{nmlz}:S/A-not.exist inf-worry also \textsc{neg}-\textsc{nmlz}:S/A-have.to \textsc{gen} place \textsc{dem}:\textsc{loc} \textsc{ifr}-go-\textsc{du} \textsc{sens}-be \\
\glt  `The two of them went to a place where they was cold cold and hunger, and where one did not need to worry.' (140519 mai huochai de xiao nvhai-zh, 182-183)
\end{exe}

However, this type of relative is also attested, though rarer, in non-translated texts, for instance in (\ref{ex:tWxpa.tukWlhoR.GW.sWjno}) with intransitive subject relativization (see § XXX for additional examples).
%kɯki aʑo ɕkom tu-mtshi-a ki ɣɯ ɯ-χpi ci pjɯ-fɕat-a

\begin{exe}
\ex \label{ex:tWxpa.tukWlhoR.GW.sWjno}
\gll  tɕe [tɯ-xpa tu-kɯ-ɬoʁ] ɣɯ sɯjno nɯ ŋu tɕe, \\
\textsc{lnk} one-year \textsc{ipfv}-\textsc{nmlz}:S/A-come.out \textsc{gen} plant \textsc{dem} be:\textsc{fact} \textsc{lnk} \\
\glt  `It is an annual plant.' (18-NGolo, 105)
\end{exe}

Genitival prenominal relative clauses are to be distinguished from relatives as possessors, as in (\ref{ex:tWCGA.kWmNAm.GW.WrJAnNgo}), where the possessum  \japhug{ɯ-rɟɤŋgo}{its radiating pain} is not an argument of the relative \forme{tɯ-ɕɣa kɯ-mŋɤm} `a tooth that hurts'. 

\begin{exe}
\ex \label{ex:tWCGA.kWmNAm.GW.WrJAnNgo}
\gll tɯ-ɕɣa a-tɤ-mŋɤm tɕe tɕe tɤ-rca tɯ-ɣmba, tɯ-ku nɯra tu-mŋɤm ɲɯ-ŋu tɕe,  nɯnɯ ``[tɯ-ɕɣa kɯ-mŋɤm] ɣɯ ɯ-rɟɤŋgo ɣɤʑu" tu-kɯ-ti ŋu. \\
\textsc{genr}.\textsc{poss}-tooth \textsc{irr}-\textsc{pfv}-be.painful \textsc{lnk} \textsc{lnk} \textsc{indef}.\textsc{poss}-following \textsc{genr}.\textsc{poss}-cheek \textsc{genr}.\textsc{poss}-head \textsc{dem}:\textsc{pl} \textsc{ipfv}-be.painful \textsc{sens}-be \textsc{lnk} \textsc{dem} \textsc{indef}.\textsc{poss}-tooth \textsc{nmlz}:S/A-be.painful \textsc{gen} \textsc{3sg}.\textsc{poss}-radiating.pain exist:\textsc{sens} \textsc{ipfv}-\textsc{genr}-say be:\textsc{fact} \\
\glt `When one has a toothache, and that one feels pain in one's cheek or a headache, one says `the toothache has a radiating pain.'' (140516 WrJANgo, 3)
\end{exe}

Adnominal complement clauses (§ XXX) can also take a genitive marker, as in (\ref{ex:mWjnaXtChWG.GW.WtCha}).

\begin{exe}
\ex \label{ex:mWjnaXtChWG.GW.WtCha}
\gll [<donggua> cʰo <qiezi> ni tɕʰi ʑo mɯ́j-naχtɕɯɣ] ɣɯ ɯ-tɕʰa a-jɤ-tɯ-ɣɯt ra \\
gourd \textsc{comit} eggplant \textsc{du} what \textsc{emph} \textsc{neg}:\textsc{sens}-be.the.same \textsc{gen} \textsc{3sg}.\textsc{poss}-information \textsc{irr}-\textsc{pfv}-2-bring have.to:\textsc{fact} \\
\glt `(Go there and come back to) tell me in what way gourd and eggplant differ from each other.' (2010-02-yitian bi yitian-zh, 7)
\end{exe}

Some relative clauses can take possessors marked in the genitive, relating to an argument within the relative, as in  (\ref{ex:stu.WkAnWmga}) and (\ref{ex:slama.ra.GW}). It is debatable whether the genitival phrase belongs to the relative in this type of construction.

\begin{exe}
\ex \label{ex:stu.WkAnWmga}
 \gll tɕe paʁ ɣɯ [stu ɯ-kɤ-nɯmga], iʑora ji-kɤ-nɯmga nɯ ɯ-ɕa ŋu tɕe \\
 \textsc{lnk} pig \textsc{gen} most \textsc{3sg}.\textsc{poss}-\textsc{nmlz}:P-want.from \textsc{1pl} \textsc{1pl}.\textsc{poss}-\textsc{nmlz}:P-want.from \textsc{dem} \textsc{3sg}.\textsc{poss}-meat be:\textsc{fact} \textsc{lnk} \\
\glt  `What is most wanted from pigs, what we want from them is their meat.' (05-paR, 13)
\end{exe}

\begin{exe}
\ex \label{ex:slama.ra.GW}
\gll  slama ra ɣɯ [tʰɯtʰɤci kɯ-fse], nɯ kɤ-rɤ-βzjoz ra ɲɯ-stu mɯ́j-stu nɯ, nɯ-stu ɲɯ-nɤma-nɯ mɯ́j-nɤma-nɯ,  nɯnɯra nɯ-pʰama ra nɯ-ɕki kɯ-rɤ-fɕɤt ɲɯ-ra. \\
student \textsc{pl} \textsc{gen} something \textsc{nmlz}:S/A-be.like \textsc{dem}  \textsc{inf}-\textsc{antipass}-learn \textsc{pl} \textsc{sens}-be.assiduous \textsc{sens}-be.assiduous \textsc{dem} \textsc{3pl}.\textsc{poss}-truth \textsc{sens}-work-\textsc{pl} \textsc{neg:sens}-work-\textsc{pl} \textsc{dem}:\textsc{pl} \textsc{3pl}.\textsc{poss}-parent \textsc{pl} \textsc{3pl}.\textsc{poss}-\textsc{dat} \textsc{genr}:S/P-\textsc{antipass}-tell:\textsc{fact} \textsc{sens}-have.to \\
\glt `One had to tell the parents all kinds of things concerning the students, whether they try hard or not, whether they work seriously or not.' (150901 tshuBdWnskAt, 18-20)
\end{exe}


\subsection{Locative} \label{sec:locative}
\subsection{Comitative} \label{sec:comitative} 
Postpositional phrases with the comitative postposition \japhug{cʰo}{and, with} and its variants \forme{cʰondɤre} and \forme{cʰonɤ} (comprising the linkers \forme{nɤ} and \forme{ndɤre}, see § XXX) are selected as oblique arguments by a handful of verbs, including \japhug{naχtɕɯɣ}{be the same} (§ \ref{sec:identity.modifier}) and \japhug{amɯmi}{be in good terms with}, as shown in (\ref{ex:cho.kWnaXtCWG}).

\begin{exe}
\ex \label{ex:cho.kWnaXtCWG}
\gll [ɯʑo cʰo] kɯ-naχtɕɯɣ [sɯjno, xɕaj ma mɤ-kɯ-ndza nɯra cʰonɤ] amɯmi-nɯ tɕe, \\
\textsc{3sg} \textsc{comit} \textsc{nmlz}:S/A-be.the.same vegetables grass apart.from \textsc{neg}-\textsc{nmlz}:S/A-eat \textsc{dem}:\textsc{pl} \textsc{comit} be.in.good.terms:\textsc{fact}-\textsc{pl} \textsc{lnk} \\
\glt `(The rabbit) is in good terms with (the animals) which eat only grass and vegetables like him.' (04-qala2, 8)
\end{exe}

Postpositional phrases in \forme{cʰo} are oblique arguments in the sense that they are relativized using the oblique participle (§ XXX). However, verbs that select \forme{cʰo} phrases index not only the intransitive subject proper, but the sum of the subject and the \forme{cʰo} phrase, which can be in the dual as in (\ref{ex:cho.YWnaXtCWGndZi}) (the white birch and the red birch) or in the plural (\ref{ex:cho.kWnaXtCWG}) (the rabbit and the other animals). 

\begin{exe}
\ex \label{ex:cho.YWnaXtCWGndZi}
\gll tɕe ɯ-rqʰu nɯ ɣɯrni laʁma ɯ-ŋgɯ nɯ [sɤjku cʰo] ɲɯ-naχtɕɯɣ-ndʑi ri\\
\textsc{lnk} \textsc{3sg}.\textsc{poss}-bark \textsc{dem} be.red:\textsc{fact} apart.from.the.fact \textsc{3sg}.\textsc{poss}-inside \textsc{dem} birch \textsc{comit} \textsc{sens}-be.the.same-\textsc{du} \textsc{lnk} \\
\glt `Apart from the fact that its bark is red, it is identical in the inside with the birch.' (06-mbrAj, 13)
\end{exe}

The verb \japhug{naχtɕɯɣ}{be the same} with a \forme{cʰo} phrase can be used in an equative construction (see § XXX).

Apart from the function presented above, \japhug{cʰo}{and, with} is commonly used to link together two nouns inside a single noun phrase, as in (\ref{ex:awW.cho.aRi}). In this case too, the main verb of the clause indexes the whole noun phrase, comprising the sum of referents designated by the nouns linked by \forme{cʰo}.

\begin{exe}
\ex \label{ex:awW.cho.aRi}
\gll a-wɯ cʰo a-ʁi ni cʰɯ-ɣi-ndʑi ra ma ʑɤni-sti kɤ-rɤʑi mɤ-cʰa-ndʑi tɕe, \\
\textsc{1sg}.\textsc{poss}-grand.father \textsc{comit} \textsc{1sg}.\textsc{poss}-younger.sibling \textsc{du} \textsc{ipfv}:\textsc{downstream}-come-\textsc{du} have.to:\textsc{fact} \textsc{lnk} \textsc{3du}-alone \textsc{inf}-stay \textsc{neg}-can:\textsc{fact}-\textsc{du} \textsc{lnk} \\ 
\glt `My grandfather and my younger brother have to come, they cannot stay by themselves.' (2011-05-nyima, 209)
\end{exe}

The marker \forme{cʰo} can also link verb phrases and even entire clauses (see § XXX and \citealt[313]{jacques14linking}).

Given the apparently equal status of the two linked nouns in(\ref{ex:awW.cho.aRi}), in particular with regard to indexation, it is legitimate to wonder whether analyzing it as a postposition makes more sense than considering it to be a coordinator (§ \ref{sec:coordinator}). There are two arguments supporting the postposition analysis. First, \forme{cʰo} necessarily follows a noun phrase (or at the very least a demonstrative pronoun), but does not require to be followed by another noun as in (\ref{ex:cho.YWnaXtCWGndZi}) above. Second, phrases comprising \forme{cʰo} and the preceding noun are relativized using the oblique participle (see § XXX).

A \forme{cʰo} phrase can be followed by the associative plural marker \forme{ra} (§ \ref{sec:number.determiners}) as in (\ref{ex:cho.ra.kW}) to mean `et caetera', and the whole phrase can taken case marking such as ergative.

\begin{exe}
\ex \label{ex:cho.ra.kW}
\gll tɕeri ɯʑo ndɤre, qajdo cʰo ra kɯ ndɤ tú-wɣ-ndza ɕti \\
but \textsc{3sg} \textsc{advers} crow \textsc{comit} \textsc{pl} \textsc{erg} \textsc{advers} \textsc{ipfv}-\textsc{inv}-eat be.\textsc{affirm}:\textsc{fact} \\
\glt `But it is eaten by crows and other (animals).' (26-NalitCaRmbWm, 140)
\end{exe}

\subsection{Standard marker} \label{sec:comparative} 
Japhug has several postpositions that are mainly used to mark the standard in the comparative construction. The most common one is \japhug{sɤz}{compared with}, but the variants \forme{staʁ}, \forme{sɤznɤ}, \forme{staʁnɤ} and \forme{sɯstaʁ} are also attested. Their relative frequency appears to be speaker-dependent, and no meaningful difference could be detected between them. 

In the comparative construction (§ XXX), the comparee is the intransitive subject of the main verb (the parameter, generally an adjectival stative verb) and is indexed on the verb. The comparee is either in the absolutive or in the ergative (§ \ref{sec:comparee.kW}). The standard is necessarily marked by one of the postpositions listed above, and cannot be indexed on the main verb. Neither the standard not the comparee are required to be overt. An adjectival stative verb with a standard postpositional phrase as in  (\ref{ex:sAznA.YWwxti})  is a well-formed comparative construction. Examples like (\ref{ex:aZo.YWwxti}) with overt comparee and standard are rarer.

\begin{exe}
\ex \label{ex:sAznA.YWwxti}
\gll  qandʑɣi sɤznɤ ɲɯ-wxti, qaliaʁ sɤznɤ ɲɯ-xtɕi \\
falcon \textsc{comp} \textsc{sens}-be.big eagle \textsc{comp} \textsc{sens}-be.small \\
\glt `It is bigger than a falcon, and smaller than an eagle.' (2011-08-kuwu, 40-41)
\end{exe}

\begin{exe}
\ex \label{ex:aZo.YWwxti}
\gll ɯʑo nɯ aʑo sɤz tɯ-xpa wxti  \\
\textsc{3sg} \textsc{dem} \textsc{1sg} \textsc{comp} one-year be.big:\textsc{fact} \\
\glt `She is one year older than me.' (12-BzaNsa, 94)
\end{exe}

The standard marker \forme{sɤz} (and its variants) also occurs in a  construction expressing progressive increase throughout the time, where a time counted noun like \japhug{tɯ-sŋi}{one day} or \japhug{tɯ-xpa}{one year} is followed by the standard marker and then repeated, as \forme{tɯ-xpa sɤz tɯ-xpa} `more each year' in (\ref{ex:tWxpa.sAz.tWxpa}). This construction, although attested in non-translated texts, is more common in texts from Chinese, where it calques the construction \ch{一年比一年}{yīnián bǐ yīnián}{more each year}. The more idiomatic Japhug construction to express the same meaning is through partial reduplication of the first syllable of the main verb (see § XXX).
 
 \begin{exe}
 \ex \label{ex:tWxpa.sAz.tWxpa}
 \gll nɯ-jwaʁ nɯ, [...] tɯ-xpa sɤz tɯ-xpa lu-dɤn ŋu ma \\
 \textsc{3pl}.\textsc{poss}-leaf \textsc{dem} { } one-year \textsc{comp} one-year \textsc{ipfv}-be.many be:\textsc{fact} \textsc{lnk} \\
\glt  `There are more needles (leaves) each year.' (08-saCW, 17)
\end{exe}
 
 The standard markers can also be used with subordinate clauses (§ XXX). The standard marker with the distal demonstrative \forme{nɯ sɤznɤ} has the meaning `rather than that, could ... as well' as in (\ref{ex:nW.sAznA.arca}).  
 
 \begin{exe}
 \ex \label{ex:nW.sAznA.arca}
 \gll  nɤ-mu kɤ-fsraŋ mɤ-tɯ-cʰa tɕe, nɯ sɤznɤ, a-rca jɤ-ɣi tɕe, a-rca, nɤki, laχɕi pɯ-βzjoz \\
 \textsc{2sg}.\textsc{poss}-mother \textsc{inf}-protect \textsc{neg}-2-can:\textsc{fact} \textsc{lnk} \textsc{dem} \textsc{comp} \textsc{1sg}.\textsc{poss}-following \textsc{imp}-come \textsc{lnk} \textsc{1sg}.\textsc{poss}-following \textsc{filler} trade \textsc{imp}-learn \\
\glt `You cannot save your mother, rather than that, come with me to learn  some abilities.' (150826 baoliandeng-zh, 142-143)
\end{exe}

This phrase can also be used as a scalar marker `even' with scope on the following clause, as in (\ref{ex:nW.sAznA.chaa}). See § XXX for a more detailed discussion. % nɯ sɤznɤ ... ʁo alala ri
 

 \begin{exe}
 \ex \label{ex:nW.sAznA.chaa}
 \gll ki kɤ-rtsi kɯ-tu me nɤ, aʑo nɯ sɤznɤ, nɤkinɯ, kʰa kɯ-qanɯ\redp{}nɯ ɯ-ŋgɯ zɯ, nɤkinɯ, tɯ-ɕpɤβ kɯβde-rzɯɣ tɤ-kɤ-lɤt nɯnɯ ku-sɤlɤɣi-a cʰa-a ɕti nɤ! \\
 \textsc{dem}.\textsc{prox} \textsc{inf}-count \textsc{nmlz}:S/A-exist not.exist:\textsc{fact} \textsc{sfp} \textsc{1sg} \textsc{dem} \textsc{comp} \textsc{filler} house   \textsc{nmlz}:S/A-\textsc{emph}\redp{}be.dark \textsc{3sg}.\textsc{poss}-inside \textsc{loc}  \textsc{filler} \textsc{indef}.\textsc{poss}-corpse four-section \textsc{pfv}-\textsc{nmlz}:P-throw \textsc{dem} \textsc{ipfv}-combine-\textsc{1sg} can:\textsc{fact}-\textsc{1sg} be.\textsc{affirm}:\textsc{fact} \textsc{sfp}  \\
\glt  `(What you ask) is nothing, I am even able to put together a body cut into four sections in a dark house.' (140512 alibaba-zh, 170)
\end{exe}

\subsection{Exceptive} \label{sec:exceptive} %\japhug{ma}{apart from} laʁma mɯma
The exceptive postposition \japhug{ma}{apart from} and its reduplicated variant \forme{mɯma} are not selected by any verb, and only used in adjunct postpositional phrases as in (\ref{ex:kWm.ci.mWma}).

 \begin{exe}
 \ex \label{ex:kWm.ci.mWma}
 \gll kɯm ci mɯma nɯnɯ tɕe znde ʁɟa ʑo ɕti \\
 door one apart.from \textsc{dem} \textsc{lnk} wall completely \textsc{emph} be.\textsc{affirm}:\textsc{fact} \\
 \glt `Apart from one door, there are walls everywhere.' (2011-11-kha, 40)
\end{exe}

The exceptive \japhug{ma}{apart from} is used in particular in restrictive focus constructions (§ \ref{sec:restrictive.focus}).

When the scope of the restrictive construction is on an entire clause rather than a single noun phrase, the clause is followed by the linker \forme{ma} (homophonous with the exceptive) and an exceptive phrase limited to the demonstrative pronoun \forme{nɯ} (here in resumptive use, coreferent with the entire preceding clause) and the postposition \forme{ma}, as in (\ref{ex:ra.ma.nW.ma}). The first \forme{ma} in this construction is not to be analyzed as the postposition: while it is possible to reduplicated the second one as in \forme{ma nɯ mɯma} (example \ref{ex:mAspea.ma.nW.ma}), reduplication of the first \forme{ma} is not attested.

 \begin{exe}
 \ex \label{ex:ra.ma.nW.ma}
 \gll [aʑɯɣ ɯ-ɕa ra] ma nɯ ma kɯ-ra me \\
 \textsc{1sg}.\textsc{gen} \textsc{3sg}.\textsc{poss}-meat have.to:fact \textsc{lnk} \textsc{dem} apart.from \textsc{nmlz}:S/A-have.to not.exist:\textsc{fact} \\
 \glt `I want its meat, and nothing else.' (02-deluge2012, 14)
\end{exe}
 \begin{exe}
 \ex \label{ex:mAspea.ma.nW.ma}
 \gll tɤ-pɤtso kɯ-ɣɤwu ʑo kɤ-nɯɕpɯz mɤ-spe-a ma nɯ mɯma spe-a \\
 \textsc{indef}.\textsc{poss}-child \textsc{nmlz}:S/A-cry \textsc{emph} \textsc{inf}-imitate \textsc{neg}-be.able[III]:\textsc{fact}-\textsc{1sg} \textsc{lnk} \textsc{dem} apart.from be.able[III]:\textsc{fact}-\textsc{1sg}  \\
\glt  `I cannot imitate a baby crying, but apart form that I can imitate (all animal sounds).' (27-kikakCi, 143)
\end{exe}
\subsection{Terminative} \label{sec:terminative}  %\japhug{mɤɕtʂa}{until}


\section{Relator nouns}  \label{sec:relator.nouns}  
\subsection{Dative} \label{sec:dative} 

The semi-transitive verb \japhug{ru}{look at} can mark its goal with the dative, as in (\ref{ex:WCki.Cturu}); this is however optional, as this verbs also takes goals in the absolutive (§ \ref{absolutive.goal}) or locative (§ \ref{sec:locative}).

\begin{exe}
\ex \label{ex:WCki.Cturu}
\gll tɕe tɤŋe nɯ nɯɣ-me tɕe, tɕe tɤŋe ɯ-ɕki ʁɟa ʑo ɕ-tu-ru tɕe, tɯʑo tɯ-ɕki maka ʑo mɤ-ru \\
lnk sun dem \textsc{appl}-be.afraid[III]:\textsc{fact}-\textsc{1sg} \textsc{lnk} \textsc{lnk} sun \textsc{3sg}.\textsc{poss}-\textsc{dat} completely \textsc{emph} \textsc{transloc}-\textsc{ipfv}:\textsc{up}-look.at \textsc{lnk} \textsc{genr} \textsc{genr}.\textsc{poss}-\textsc{dat} at.all \textsc{emph} \textsc{neg}-look.at:\textsc{fact} \\
\glt `(If the yeti catches you), it is afraid of the sun, it looks at the sun the whole time, and does not look at you.' (140510 mYWrgAt, 13)
\end{exe}

\subsection{Secutive} \label{sec:secutive} 
The secutive relator noun \japhug{ɯ-rca}{following} is used with verbs of motion such as \japhug{gi}{come} to express the meaning `follow', `come/go with' as in (\ref{ex:nArca.Gia}).

\begin{exe}
\ex \label{ex:nArca.Gia}
 \gll  aʑo kɯnɤ nɤ-rca ɣi-a ɕti \\
 \textsc{1sg} also \textsc{2sg}.\textsc{poss}-following come:\textsc{fact}-\textsc{1sg} be.\textsc{affirm}:\textsc{fact} \\
\glt `I am coming/going with you.' (2011-05-nyima, 171)
\end{exe}

The secutive can have a meaning similar to that of the comitative adverb (§ \ref{sec:comitative.adverb}) `together with X', as in (\ref{ex:WBGi.Wrca}).

\begin{exe}
\ex \label{ex:WBGi.Wrca}
 \gll  pɤnmawombɤr ɣɯ ɯ-ɕɤrɯ ɯ-βɣi ɯ-rca tsʰɯntsʰɯn ʑo ta-wum-nɯ ɲɯ-ŋu \\ 
p.n. \textsc{gen} \textsc{3sg}.\textsc{poss}-bone \textsc{3sg}.\textsc{poss}-ash 3sg.poss-following \textsc{idph}:II:neat \textsc{emph} \textsc{pfv}:3\fl{}3'-collect-\textsc{pl} \textsc{sens}-be  \\
\glt `They collected all of Padma 'Od-'bar's bones together with his ashes.' (Norbzang2005, 410)
\end{exe}

The secutive phrase can follow  (\ref{ex:WBGi.Wrca}), or precede (\ref{ex:tWjAGAt.Wrca}) the noun phrase it accompanies.

\begin{exe}
\ex \label{ex:tWjAGAt.Wrca}
 \gll   tɯ-jɤɣɤt ɯ-rca tɤ-se cʰɯ-nɯ-ɬoʁ \\
 \textsc{indef}.\textsc{poss}-feces \textsc{3sg}.\textsc{poss}-following \textsc{indef}.\textsc{poss}-blood \textsc{ipfv}:\textsc{downstream}-\textsc{auto}-come.out \\
 \glt `(In the case of this disease), blood comes out together with the feces.' 
 \end{exe}

The secutive with third person singular possessive \forme{ɯ-rca} is also used as a linker meaning `in addition' (see § XXX). With the indefinite possessive prefix \forme{tɤ-rca} and  \forme{tɯ-tɯ-rca}, the secutive appears in adverbial function with the meaning `together' (§ XXX), though in examples such as  (\ref{ex:tArAku.tArca}) the form  \forme{tɤ-rca} retains its nominal status.


\begin{exe}
\ex \label{ex:tArAku.tArca}
 \gll ɕoʁ nɯnɯ tɤ-rɤku tɤ-rca ŋu, sɯjno maʁ. \\
 buckwheat \textsc{dem} \textsc{indef}.\textsc{poss}-crops \textsc{indef}.\textsc{poss}-following be:fact grass not.be:fact \\
 \glt `Buckwheat (belongs) with the crops, it is not a (type of) grass.' (13-NanWkWmtsWG, 68)
\end{exe}

In addition, the unexpected focus marker \forme{rcanɯ} (§ \ref{sec:unexpected}) and the dubitative sentence final particle \forme{rca} (§ XXX) are historically related to the secutive.

\subsection{Deputative} \label{sec:deputative} 
The IPN \forme{ɯ-tsʰɤt} has two meanings. First, it can serve as a deputative relator noun `instead of, on behalf of' as in (\ref{ex:nWtAsno.WtshAt}) and (\ref{ex:nWsi.WtshAt}). No verb selects this relator noun. 

The deputative adjunct can correspond to the intransitive subject (as in \ref{ex:nWtAsno.WtshAt}, with the verb \japhug{tu}{exist}), the transitive subject (as in \ref{ex:aZo.nAtshAt}, with \japhug{ɣɯjtsi}{support}) or the object.

\begin{exe}
\ex \label{ex:nWtAsno.WtshAt}
\gll nɯʑora ɣɯ nɯ-tɤ-sno kɯ-fse ɯ-tsʰɤt nɯ, tɕiʑo ɣɯ, tɕi-xɕɤndʑu χsɯ-ldʑa pɯ-tu tɕe, nɯnɯ lɤ-nɯ-βlɯ-tɕi ɕti wo \\
\textsc{2pl} \textsc{gen} \textsc{2pl}.\textsc{poss}-\textsc{indef}.\textsc{poss}-saddle \textsc{nmlz}:S/A-be.like \textsc{3sg}.\textsc{poss}-instead.of \textsc{dem} \textsc{1du} \textsc{gen} \textsc{1du}.\textsc{poss}-twig three-long.object \textsc{pst}.\textsc{ipfv}-exist \textsc{lnk} \textsc{dem} \textsc{pfv}-\textsc{auto}-burn-\textsc{1du} be.\textsc{affirm}:\textsc{fact} \textsc{sfp} \\
\glt `Instead of a saddle like yours, we had three twigs, this is what we burned.' (Kubzang2003, 203)
\end{exe}

\begin{exe}
\ex \label{ex:aZo.nAtshAt}
\gll aʑo nɤ-tsʰɤt, nɤki, si nɯ tu-ɣɯjtsi-a jɤɣ \\
\textsc{1sg} \textsc{2sg}.\textsc{poss}-instead.of \textsc{filler} tree \textsc{dem} \textsc{ipfv}-support-\textsc{1sg} be.possible:\textsc{fact} \\
\glt `I can support the tree for you/instead of you (while you fetch it).' (150830 afanti, 136)
\end{exe}

 The noun phrase headed by \forme{ɯ-tsʰɤt} can be either an adjunct as in (\ref{ex:nWtAsno.WtshAt}) and (\ref{ex:aZo.nAtshAt}), the object of the verb \japhug{βzu}{make}, or a nominal predicate with a copula as in (\ref{ex:nWsi.WtshAt}) and (\ref{ex:aZo.atshAt}).  In the latter case, to express the meaning `do to $X$ instead of to $Y$', a biclausal construction `do to $X$, ($X$) is instead of $Y$' is used as in (\ref{ex:aZo.atshAt}).

\begin{exe}
\ex \label{ex:nWsi.WtshAt}
\gll si maŋe tɕe tɕe nɯnɯtɕu tɕe, nɯ-si ɯ-tsʰɤt ɲɯ-ŋu  \\
tree not.exist:\textsc{sens} \textsc{lnk} \textsc{lnk} \textsc{dem}:\textsc{loc} \textsc{lnk} \textsc{3pl}.\textsc{poss}-wood \textsc{3sg}.\textsc{poss}-instead.of \textsc{sens}-be \\
\glt `There no trees, there (dung) is used to replace the firewood.' (05-tamar, 10-11)
\end{exe}


\begin{exe}
\ex \label{ex:aZo.atshAt}
\gll nɯ tɤ-nɯ-ndɤm tɕe aʑo a-tsʰɤt ŋu tɕe \\
\textsc{dem} \textsc{imp}-\textsc{auto}-take[III] \textsc{lnk} \textsc{1sg} \textsc{1sg}.\textsc{poss}-instead.of be:\textsc{fact} \textsc{lnk} \\
\glt `Take these instead of me (as a compensation).' (2003kAndzwsqhaj2, 141)
\end{exe}

The examples (\ref{ex:aZo.nAtshAt}) and (\ref{ex:aZo.atshAt}) also show that the relator noun \japhug{ɯ-tsʰɤt}{instead of} can occur with a first or second person possessive prefix.

Second, \forme{ɯ-tsʰɤt} also means `with proper measure', mainly occurring in adverbial function as in (\ref{ex:WtshAt.tsa}) or in collocation with the verb \japhug{βzu}{make} in the sense `do with proper measure' as in (\ref{ex:WtshAt.tusWBzunW}). 

\begin{exe}
\ex \label{ex:WtshAt.tsa}
\gll rkaŋraŋ ɯ-tsʰɤt tsa ɲɯ-kɯ-nɤɕtʂaʁli-a-nɯ raʁmaʁ ma  \\
p.n. \textsc{3sg}.\textsc{poss}-proper.measure a.little \textsc{ipfv}-2\fl{}1-torture-\textsc{1sg}-\textsc{pl} \textsc{sfp} \textsc{lnk}  \\
\glt `Rkangrang, your torturing of me should have a limit.' 
\end{exe}

\begin{exe}
\ex \label{ex:WtshAt.tusWBzunW}
\gll ɯ-tsʰɤt tu-sɯ-βzu-nɯ mɯ́j-kʰɯ ma, nɯ-kɤ-kʰo nɯ mɯ-tʰa-ɕkɯt mɤɕtʂa tu-ndze ɲɯ-ɕti. \\
\textsc{3sg}.\textsc{poss}-proper.measure \textsc{ipfv}-\textsc{caus}-make-\textsc{pl} \textsc{neg}:\textsc{sens}-be.possible \textsc{pfv}-\textsc{nmlz}:P-give \textsc{dem} \textsc{neg}-\textsc{pfv}:3\fl{}3'-eat.completely until \textsc{ipfv}-eat[III] \textsc{sens}-be.\textsc{affirm} \\
\glt `They cannot make (the monkey eat) with measure, as it continues eating the (food) that is given to it until there is none.' (19-GzW, 60)
\end{exe}

In some contexts as in (\ref{ex:nWnW.WtshAt}), \japhug{ɯ-tsʰɤt}{proper measure} in adverbial used is better translated as `depending on the circumstances'.\footnote{This example is taken from a text describing goats and sheep; goats are called \forme{tsʰɤt} in Japhug, but it is clear from the context that \forme{ɯ-tsʰɤt} cannot be the possessed form of this noun. }

\begin{exe}
\ex \label{ex:nWnW.WtshAt}
\gll tɕe nɯnɯ ɯ-tsʰɤt nɯnɯ ɯ-pɯ ci ci ʁnɯz tu, ci ci tɯ-rdoʁ ma me tɕe núndʐa ɲɯ-ŋu. \\
\textsc{lnk} \textsc{dem} \textsc{3sg}.\textsc{poss}-proper.measure \textsc{dem}  \textsc{3sg}.\textsc{poss}-young once once two exist:\textsc{fact} once once one-piece apart.from not.exist:\textsc{fact} \textsc{lnk} for.this.reason \textsc{sens}-be \\
\glt `This is why, depending on the circumstances, sometimes (the goat) has two youngs, sometimes only one.' (05-qaZo, 28)
\end{exe}

The IPN  \forme{ɯ-tsʰɤt} (at least in the meaning `proper measure') is borrowed from \tibet{ཚད་}{tsʰad}{measure, limit}. It occurs as second element in the compound \japhug{xtɤtsʰɤt}{restraint of one's appetite}(with the \textit{status constructus} \forme{xtɤ-} of \japhug{tɯ-xtu}{belly}).

\subsection{Relator nouns of location} \label{sec:relator.location}
% tɕe phaʁrgot ri li ɯʑo ɯ-stu ʑo ɲɤ-ɣe qhe,  direction
\subsection{Semi-grammaticalized relator nouns} \label{sec:semi.grammaticalized.relator} 
%nɤki tɤtʂu nɯ a-jaʁ tɤ-khɤm! hand over
%kɯki mbro ki nɤ-jaʁ ɲɯ-kho-j hist-X1-qachGa,62

\section{Noun modifiers and determiners}
This section discusses all nouns modifiers and determiners except relative clauses (§ XXX) and complement clauses (§ XXX). 
 
\subsection{Number}  \label{sec:number.determiners}
Japhug has two number markers, the dual \forme{ni} and the plural \forme{ra}. These clitics are not obligatory for non-singular arguments (even in the case of human referents), and do not necessary trigger plural or dual agreement on the verb. 

\subsubsection{Dual} \label{sec:dual.determiners}
The dual \forme{ni} is historically related to the numeral \forme{ʁnɯz} (§ \ref{sec:one.to.ten}), but their relationship is synchronically opaque. It combines with the proximal and distal demonstratives \forme{ki} and \forme{nɯ} respectively to form the dual demonstratives \forme{kɯni} and \forme{nɯni} (§ \ref{sec:demonstrative.pronouns}, § \ref{sec:demonstrative.determiners}).

There is no semantic restriction on the use of \forme{ni}, it most often occurs with human referents (\ref{ex:awW.cho.aRi.ni}, \ref{ex:Wmu.Wwa.ni}, \ref{ex:tCiZo.ni}, \ref{ex:ni.ndZisroR}), but is also commonly attested with animals (\ref{ex:ʁnWz.ni}) inanimate objects (\ref{ex:ni.RnaRna}), and placenames (\ref{ex:rgWnba.ni}).

\begin{exe}
\ex \label{ex:rgWnba.ni}
\gll prɤɕta cʰo rgɯnba ni ndʑi-pɤrtʰɤβ ri ŋu \\
pl.n. \textsc{comit} monastery \textsc{du} \textsc{3du}.\textsc{poss}-between \textsc{loc} be:\textsc{fact} \\
\glt `It is between Prashta and the monastery.' (140522 Kamnyu zgo, 115)
\end{exe}

The dual can follow the numeral \japhug{ʁnɯz}{two}, as in (\ref{ex:ʁnWz.ni}). This combination is however very rare (only 13 examples in the corpus out of hundreds of dual \forme{ni}). The opposite order (dual followed by numeral) is not grammatical.

\begin{exe}
\ex \label{ex:ʁnWz.ni}
\gll mbɣɤru nɯ jla ʁnɯz ni ndʑi-tʰɤβ ri ɲɯ-ɕe tɕe \\
plough.beam \textsc{dem} hybrid.yak two \textsc{3du}.\textsc{poss}-between \textsc{loc} \textsc{ipfv}:\textsc{west}-go \textsc{lnk} \textsc{lnk} \\
\glt `The beam of the plough goes between the two hybrid yaks.' 
\end{exe}


The adverb \japhug{ʁnaʁna}{both} (§ XXX) commonly co-occurs with dual, as in (\ref{ex:ni.RnaRna}).
%tɤ-pi ʁnaʁna ʑo pɯ́-wɣ-sat-ndʑi ɲɯ-ŋu. 

\begin{exe}
\ex \label{ex:ni.RnaRna}
\gll zaŋ cʰo raʁ ni ʁnaʁna ʑo ʁja ku-te ɲɯ-ŋu \\
copper \textsc{comit} brass \textsc{du} both \textsc{emph} verdigris \textsc{ipfv}-put[III] \textsc{sens}-be \\
\glt `Both copper and brass can get verdigris.' (30-Com, 101)
\end{exe}

The marker \forme{ni} can appear with a noun phrase comprising two nouns (each with singular referents) linked by the comitative \forme{cʰo} (§ \ref{sec:comitative}).

\begin{exe}
\ex \label{ex:awW.cho.aRi.ni}
\gll  tɕe a-wɯ cʰo a-ʁi ni pjɯ-tɯ-sat mɤ-jɤɣ \\
\textsc{lnk} \textsc{1sg}.\textsc{poss}-grandfather \textsc{comit} \textsc{1sg}.\textsc{poss}-younger.sibling \textsc{du} \textsc{ipfv}-2-kill \textsc{neg}-be.possible:\textsc{fact} \\
\glt `You cannot kill my grandfather and my younger brother.' (2011-05-nyima, 133)
\end{exe}

The dual can also be used with noun dyads (§ \ref{sec:dyads}), as in (\ref{ex:Wmu.Wwa.ni}). 

\begin{exe}
\ex \label{ex:Wmu.Wwa.ni}
\gll   ɯ-mu ɯ-wa ni kɯ ɲɯ-z-nɤja-ndʑi qʰe \\
\textsc{3sg}.\textsc{poss}-mother \textsc{3sg}.\textsc{poss}-father \textsc{du} \textsc{erg} \textsc{ipfv}-\textsc{caus}-be.a.pity-\textsc{du} \textsc{lnk} \\
\glt `Her parents would not be parted from her.' (14-tApitaRi, 305)
\end{exe}

The third person dual pronoun \forme{ʑɤni} is build by combining the pronominal root \forme{-ʑo-} with the dual \forme{ni} (§ \ref{sec:pers.pronouns}), and is not attested in combination with the dual. The first and second dual pronouns \forme{tɕiʑo} and \forme{ndʑiʑo}, do occur with the dual marker as in (\ref{ex:tCiZo.ni}), though examples are very rare.

\begin{exe}
\ex \label{ex:tCiZo.ni}
\gll  tɕiʑo ni wuma ʑo pɯ-amɯmi-tɕi tɕe \\
\textsc{1du} \textsc{du} really \textsc{emph} \textsc{pst}.\textsc{ipfv}-be.in.good.terms-\textsc{1du} \textsc{lnk} \\
\glt `We were in harmony together.' (140512 fushang he yaomo-zh, 85)
\end{exe}

Noun phrases with the dual \forme{ni} are always correlated with a dual prefix on the following noun in possessive constructions or with relator nouns, as in (\ref{ex:rgWnba.ni}), (\ref{ex:ʁnWz.ni}) and (\ref{ex:ni.ndZisroR}). Not a single example of a noun phrase in \forme{ni} followed by a noun with singular of plural possessive prefix is found in the corpus.

\begin{exe}
\ex \label{ex:ni.ndZisroR} 
\gll ɯ-pi ni ndʑi-sroʁ ko-ri tɕe \\
\textsc{3sg}.\textsc{poss}-elder.sibling \textsc{du} \textsc{3du}.\textsc{poss}-life \textsc{ifr}-save \textsc{lnk} \\
\glt `He saved the life of his two brothers.' (qachGa 2012, 139)
\end{exe}

The marker \forme{ni} is not obligatory with dual referents, in particular when the numeral \japhug{ʁnɯz}{two} is present. An overt noun phrase without dual marking can trigger indexation on the verb, especially with collectives expressing a pair of individuals as \japhug{ʁzɤmi}{husband and wife} in (\ref{ex:RjWmbrWg.RzAmi}), but also with other types of noun phrases as in (\ref{ex:nW.talWlAtndZi}).

\begin{exe}
\ex \label{ex:RjWmbrWg.RzAmi}
\gll  kɯɕɯŋgɯ tɕe tɕe atu <qinghai> ʑɴɢɯloʁ nɯtɕu tɕe, ʁjɯmbrɯɣ ʁzɤmi ci pjɤ-tu-ndʑi tɕe,  \\
in.former.times \textsc{lnk}  \textsc{lnk} up.there p.n. p.n. \textsc{dem}:\textsc{loc} \textsc{lnk} dragon husband.and.wife one \textsc{ifr}.\textsc{ipfv}-exist-\textsc{du} \textsc{lnk} \\
\glt `In former times, in Qinghai, in the Mgolog area, there was a couple of dragons.' (150820 qaprANar, 44)
\end{exe}

\begin{exe}
\ex \label{ex:nW.talWlAtndZi}
\gll  ʁdɯxpanaχpu ɯ-tɕɯ cʰo aʑo a-tɕɯ nɯ tɤ-alɯlɤt-ndʑi tɕe, \\
p.n. \textsc{3sg}.\textsc{poss}-son \textsc{comit} \textsc{1sg} \textsc{1sg}.\textsc{poss}-son \textsc{dem} \textsc{pfv}-fight-\textsc{du} \textsc{lnk} \\
\glt `The son of Gdugpa Nagpo and my son were fighting.' (28-smAnmi, 280)
\end{exe}

Such examples are however surprisingly rare in the corpus; dual indexation is most often correlated with a dual marker on the corresponding noun phrase, if overt.

The numeral \japhug{ʁnɯz}{two} without the dual also triggers dual indexation, as in (\ref{ex:RnWz.tundZi}).

\begin{exe}
\ex \label{ex:RnWz.tundZi}
\gll   sɯŋgɯ zɯ tɯrme wuma ʑo kɯ-wxti ʁnɯz tu-ndʑi tɕe\\
forest \textsc{loc} person really \textsc{emph} \textsc{nmlz}:S/A-be.big two exist:\textsc{fact}-\textsc{du} \textsc{lnk}\\
\glt `In the forest, there are two giants.'  (140428 yonggan de xiaocaifeng, 172)
\end{exe}

Dual marking on a noun phrase is not necessarily correlated with dual indexation on the verb, especially, but not exclusively, with inanimate referents, as in (\ref{ex:ni.tomto}). This question is studied in more detail in § XXX.

\begin{exe}
\ex \label{ex:ni.tomto}
\gll  ɯ-mɲaʁ χcʰoʁe ni to-mto. \\
\textsc{3sg}.\textsc{poss}-eye left.and.right \textsc{du} \textsc{pfv}-have.sight \\
\glt `His left and right eyes recovered sight.' (140517 mogui de jing, 105)
\end{exe}

However, a noun phrase with \forme{ni} is never correlated with a plural indexation marker on the verb. Apparent exceptions are either speech errors (a topic treated in § XXX), or cases of ambiguous indexation, as in (\ref{ex:paznAkharnW}).

\begin{exe}
\ex \label{ex:paznAkharnW}
 \gll  nɤ-pi ni kɯ nɤʑo nɯɣi kɤ-sɯso kɯ ʁmaʁ χsɯ-tɤxɯr kɯ pa-z-nɤkʰar-nɯ ɕti tɕe, \\
 \textsc{2sg}.\textsc{poss}-elder.sibling \textsc{du} \textsc{erg} \textsc{2sg} come.back:\textsc{fact} \textsc{inf}-think \textsc{erg} soldier three-round \textsc{erg} \textsc{pfv}:3\fl{}3'-\textsc{caus}-surround-\textsc{pl} be.\textsc{affirm}:\textsc{fact} \textsc{lnk} \\
 \glt `Your two elder brothers, thinking that you are coming back, had (the palace) guarded on all sides by three rows of soldiers.' (qachGa2012, 157)
\end{exe}

Example (\ref{ex:paznAkharnW}) is not completely straightforward, and deserves a detailed comment. The form \forme{paznɤkʰarnɯ} can be parsed as either \forme{pɯ-az-nɤkʰar-nɯ} \textsc{pst}.\textsc{ipfv}-\textsc{prog}-surround-\textsc{pl} `They were guarding it' with vowel fusion (§ XXX) or \forme{pa-z-nɤkʰar-nɯ} \textsc{pfv}:3\fl{}3'-\textsc{caus}-surround-\textsc{pl} `(He/they) had them guard it'. Context makes it clear here that the second option is the correct one, in particular because in the same passage in another version of the same story, we find the verb \forme{pa-sɯ-lɤt} \textsc{pfv}:3\fl{}3'-\textsc{caus}-throw `he had (them) make' with the perfective 3\fl{}3' form of a causative verb (\citealt[242]{jacques16complementation}, § XXX). Moreover, while the phrase \forme{nɤ-pi ni kɯ}  `your two elder brothers' could in principle belong to the infinitival clause in \forme{kɤ-sɯso}\footnote{Incidentally, note that this infinitival clause contains another complement in Hybrid Reported Speech, see § XXX.}, it is clear from the context and the explanations provided by native speakers that \forme{nɤ-pi ni kɯ} is the causer, and \forme{ʁmaʁ χsɯ-tɤxɯr kɯ} `three rows of soldiers' is the causee (also marked by the ergative, see § \ref{sec:causee.kW}). 

We thus observe plural indexation \forme{-nɯ} on the main verb \forme{pa-z-nɤkʰar-nɯ}, while the subject \forme{nɤ-pi ni kɯ}  has a dual marker. However, this is neither a counterexample to the number indexation rule stated above, nor a speech error: rather, it is a consequence of the fact that causees rather than causers can trigger number indexation on the verb in specific cases (see § XXX).

\subsubsection{Plural} \label{sec:plural.determiners}
The plural marker \forme{ra}, like the dual, follows the noun and most of its modifiers, and fuses with the demonstratives \forme{ki} and \forme{nɯ} respectively to build the plural demonstratives \forme{kɯra} and \forme{nɯra} (§ \ref{sec:demonstrative.pronouns}, § \ref{sec:demonstrative.determiners}). The etymology of the plural marker \forme{ra} is unknown, but a potential cognate exists in Pumi (\forme{=ɹə}, (\citealt[135]{daudey14grammar}; Japhug \forme{-a} regularly corresponds to Pumi \forme{-ə} in the native vocabulary). It should not be confused with the auxiliary verb \japhug{ra}{have to, need} (§ XXX), though there are cases where some ambiguity may occur (§ XXX).

Like the dual \forme{ni}, the plural \forme{ra} is compatible with both animate and inanimate referents, as in (\ref{ex:si.ra.cho}) and (\ref{ex:rdAstaR.ra}). It can be a plain marker of plurality as in (\ref{ex:si.ra.cho}).

\begin{exe}
\ex \label{ex:si.ra.cho}
\gll kɯmaʁ si ra cʰo nɯ-mdoʁ mɤ-naχtɕɯɣ \\
other tree \textsc{pl} \textsc{comit} \textsc{3pl}.\textsc{poss}-colour \textsc{neg}-be.the.same:\textsc{fact} \\
\glt `Its colour is different from that of the other trees.' (11-qrontshom, 56)
\end{exe} 

The marker \forme{ra} is also often an associative plural, understandable as `and other things', as in (\ref{ex:rdAstaR.ra}).

\begin{exe}
\ex \label{ex:rdAstaR.ra}
\gll rdɤstaʁ ra pjɯ-tʂaβ-nɯ qʰe tɯrme tu-xtsɯɣ ɲɯ-ŋu \\
stone \textsc{pl} \textsc{ipfv}-cause.to.fall-\textsc{pl} \textsc{lnk} people \textsc{ipfv}-hit \textsc{sens}-be \\
\glt `(Goats and sheep, as they climb high) cause stones (and other things) to fall and these hit people.' (tshAt-qaZo-kAlAG, 4)
\end{exe} 

The plural can follow numerals (even without head noun) to express an approximative number, as in (\ref{ex:XsWm.kWBde}).\footnote{Note that in (\ref{ex:XsWm.kWBde}) \forme{ci ci} is the expression for `sometimes', not used as a numeral, see § XXX.} 

\begin{exe}
\ex \label{ex:XsWm.kWBde}
\gll ci ci χsɯm kɯβde ra ɲɯ-lɤt ɲɯ-ŋgrɤl. tsuku tɕe ʁnɯz jamar ma mɯ́j-lɤt,\\
one one three four \textsc{pl} \textsc{sens}-throw \textsc{sens}-be.usually.the.case. some \textsc{lnk} two about apart.from \textsc{neg}:\textsc{sens}-throw \\
\glt  `Sometimes (dogs) have three or four (litters), some only have two.' (05-khWna, 22)
\end{exe} 

The plural marker \forme{ra} can also indicate approximate location, with or without locative markers. In (\ref{ex:kha.ra}), we find approximate location \forme{ra} in \forme{kʰa ra} `(everywhere) in the house, around the house' and \forme{tɯ-ji ɯ-ngɯ ra} `in the fields', and in (\ref{ex:nWrNa.ra}) with body parts.

This use of \forme{ra} can convey a meaning of distributed location, and is often combined with the adverb \japhug{aʁɤndɯndɤt}{everywhere} (§ \ref{sec:aRandWndAt}). It is reminiscent of plural markers in Kirghiz and Old Japanese, which combine collective, hypocoristic and approximate locative meanings (see \citealt[195]{antonov07ra}).

\begin{exe}
\ex \label{ex:kha.ra}
\gll βʑɯ nɯ wuma ʑo ŋɤn tɕe, tɕendɤre aʁɤndɯndɤt ʑo kʰa ra cʰɯ-rɤpɯ. tɯ-ji ɯ-ngɯ ra cʰɯ-rɤpɯ, \\
mouse \textsc{dem} really \textsc{emph} be.evil:\textsc{fact} \textsc{lnk} \textsc{lnk} everywhere \textsc{emph} house \textsc{pl} \textsc{ipfv}-bear.young \textsc{indef}.\textsc{poss}-field \textsc{3sg}.\textsc{poss}-inside \textsc{pl}  \textsc{ipfv}-bear.young \\
\glt `The mouse is fierce, it has youngs everywhere in the house, and has youngs in the fields.' (27-spjaNkW, 166)
\end{exe} 

\begin{exe}
\ex \label{ex:nWrNa.ra}
\gll nɯ-βri ra ɲɯ-ɬoʁ, nɯ-mke nɯra ɲɯ-ɬoʁ nɯ-rŋa ra brɤβbrɤβ ʑo ɲɯ-ɬoʁ ɲɯ-ŋu. \\
\textsc{3pl}.\textsc{poss}-body \textsc{pl} \textsc{ipfv}-come.out \textsc{3pl}.\textsc{poss}-neck \textsc{dem:pl} \textsc{ipfv}-come.out \textsc{3pl}.\textsc{poss}-face \textsc{pl} \textsc{idph}:II:covered.by.tiny.bumps \textsc{emph} \textsc{ipfv}-come.out  \textsc{sens}-be \\
\glt `(People who suffer from this disease have little blisters) appearing on their body, on their neck and all over their face.' (27-kharwut, 58)
\end{exe} 

The marker \forme{ra} even occurs with referents which are clearly singular, not only in the approximative location function, but also in examples such as (\ref{ex:tAwi.ra}) where the reason for the presence of \forme{ra} is less immediately obvious. In (\ref{ex:tAwi.ra}), a sentence taken from the translation of Rotkäppchen into Japhug (from Chinese, though here the presence of \forme{ra} cannot be due to calque), the function of the plural on the phrase \forme{tɤ-wi ra} `the grandmother' is more subtle: it conveys the idea idea that the impersonation takes on several aspects of the grandmother, not only her physical appearance, but also her voice, as implied by the second clause. 

\begin{exe}
\ex \label{ex:tAwi.ra}
\gll  qapar nɯ kɯ li, [...] tɤ-wi ra to-nɯɕpɯz tɕe, tɕe ɯ-skɤt ra cʰɤ-sɯ-ɤmtɕoʁ ʑo tɕe nɯra to-ti. \\
dhole \textsc{dem} \textsc{erg} again { } \textsc{indef}.\textsc{poss}-grandmother \textsc{pl} \textsc{ifr}-impersonate \textsc{lnk} \textsc{lnk} \textsc{3sg}.\textsc{poss}-voice \textsc{pl} \textsc{ifr}-\textsc{caus}-be.sharp \textsc{emph} \textsc{lnk} \textsc{dem}:\textsc{pl} \textsc{ifr}-say \\
\glt `The wolf was pretending to be the grandmother, and said these (words) with a sharp voice.' (140428 xiaohongmao-zh, 95-96)
\end{exe} 

Just like noun phrases with dual \forme{ni} correlate with dual possessive prefixe (see \ref{ex:ni.ndZisroR} in § \ref{sec:dual.determiners}), those with plural \forme{ra} can only be coreferent with a plural possessive prefix, as \forme{nɯ-} in (\ref{ex:si.ra.nWmat}).

\begin{exe}
\ex \label{ex:si.ra.nWmat}
 \gll  sɯku tɕe tʰɣe kɯ-fse, kɯmaʁ si ra nɯ-mat nɯra ɕ-pjɯ-nɯ-pʰɯt tɕe tu-ndze ɲɯ-ŋu.\\
tree \textsc{lnk} acorn \textsc{nmlz}:S/A-be.like other tree \textsc{pl} \textsc{3pl}.\textsc{poss}-fruit \textsc{dem}:\textsc{pl} \textsc{transloc}-\textsc{ipfv}:\textsc{down}-\textsc{auto}-pluck \textsc{lnk} \textsc{ipfv}-eat[III] \textsc{sens}-be \\
\glt `On the trees, (the bear) plucks acorn or fruits from other trees to eat.' (21-pri, 44)
\end{exe}

Apparent counterexamples such as (\ref{ex:WtaR.ra.Wmat}), where \forme{ra} is followed by a noun with the singular possessive prefix \forme{ɯ-}, occur when the preceding noun phrase is not the possessor of the following noun. For instance, in (\ref{ex:WtaR.ra.Wmat}) \forme{ra} has the vague locative function, and the phrase \forme{tɯ-ŋga ɯ-taʁ ra} `on the clothes' is not the possessor of \japhug{ɯ-mat}{its fruits}, it is a locative adjunct.

\begin{exe}
\ex \label{ex:WtaR.ra.Wmat}
 \gll tɯ-ŋga ɯ-taʁ ra ɯ-mat bɤbɤβ ʑo ku-ndzoʁ. \\
 \textsc{indef}.\textsc{poss}-clothes \textsc{3sg}.\textsc{poss}-on \textsc{pl} \textsc{3sg}.\textsc{poss}-fruit \textsc{idph}:II:in.clusters \textsc{emph} \textsc{ipfv}-\textsc{anticaus}:attach \\
\glt `Its seeds attach on clothes in clusters.' (18-qromJoR, 169)
\end{exe}

The plural \forme{ra} very commonly occurs with headless relatives, with or without a demonstrative, as in (\ref{ex:nW.tCaGi}), where we find both relatives followed by \forme{nɯnɯra} and another one followed by \forme{ra}.

\begin{exe}
\ex \label{ex:nW.tCaGi}
\gll [kɤ-ti mɤ-kɯ-pe kɯ-fse tu-kɯ-ti] nɯnɯra tɕe, [[kɤ-nɯtsɯ kɯ-ra] ra kɯnɤ tu-kɯ-ti] nɯnɯra, 
tɯrme ra kɯnɤ, tɕaɣi tu-sɤrmi-nɯ ŋgrɤl.  \\
\textsc{inf}-say \textsc{neg}-\textsc{nmlz}.S/A-be.good \textsc{nmlz}.S/A-be.like \textsc{ipfv}-\textsc{nmlz}.S/A-say \textsc{dem}:\textsc{pl} \textsc{lnk} \textsc{inf}-hide \textsc{nmlz}.S/A-have.to \textsc{pl} also \textsc{ipfv}-\textsc{nmlz}.S/A-say \textsc{dem}:\textsc{pl} people \textsc{pl} also  parrot \textsc{ipfv}-call-\textsc{pl} be.usually.the.case:\textsc{fact} \\
\glt `Those who say things that one should not say, who say even what should be concealed, even (if they are) people, they call them `parrots'. (24-qro, 125)
\end{exe} 

%ɯʑo sɤz pɣɤtɕɯ kɯ-xtɕi nɯra tu-ndze ɲɯ-ŋu. tɕe nɯnɯ tu-ti-nɯ ɲɯ-ŋu tɕe ɯ-mɤ-ŋu ma,
%ta-ndza ra pɯ́-wɣ-mto me
 
The plural \forme{ra} also occurs between auxiliaries and the preceding complement clause with a verb in finite (\ref{ex:GWkWnWru.ra}) or non-finite (\ref{ex:kAnAjaR.ra}) form, with a vague implication that additional related actions are concerned.

\begin{exe}
\ex \label{ex:GWkWnWru.ra}
 \gll li tɯ-ji ɯ-ŋgɯ ra ɣɯ-ku-nɯru ra ŋgrɤl. \\
 again \textsc{indef}.\textsc{poss}-field \textsc{3sg}.\textsc{poss}-inside \textsc{pl} \textsc{cisloc}-\textsc{ipfv}-eat.crops \textsc{pl} be.usually.the.case:\textsc{fact} \\
\glt `It also (usually) comes to eat crops in the fields.' (24-ZmbrWpGa, 37)
\end{exe}

\begin{exe}
\ex \label{ex:kAnAjaR.ra}
 \gll  ɣɤmdzu tɕe nɯnɯ kɤ-nɤjaʁ ra mɤ-sɤ-nɤz tɕe \\
be.thorny:\textsc{fact} \textsc{lnk} \textsc{dem} \textsc{inf}-touch \textsc{pl} \textsc{neg}-\textsc{deexp}-dare:\textsc{fact} \textsc{lnk} \\
\glt `It is thorny and one does not dare to touch it with the hand.' (11-qrontshom, 91)
\end{exe}

The marker \forme{ra} following a locative noun or adverb can have the meaning `the people/things from X', as in (\ref{ex:alo.ra}), without the need to add a demonstrative.

\begin{exe}
\ex \label{ex:alo.ra}
 \gll alo ra ɲɯ-mbɣom-nɯ qʰe \\
 upstream \textsc{pl} \textsc{sens}-be.in.a.hurry-\textsc{pl} \textsc{lnk} \\
 \glt `Those in the village, they (do things) in hurry.' (conversation140510 tshering, 175)
\end{exe}

\subsection{Demonstratives} \label{sec:demonstrative.determiners}
Japhug demonstrative determiners are formally identical  to the demonstrative pronouns (§ \ref{sec:demonstrative.pronouns}). They distinguish between proximal and distal demonstratives with different roots, and fuse with the dual and plural markers studied in § \ref{sec:number.determiners}; the proximal \forme{ki} undergoes change to \forme{kɯ-} in those fused forms.

As with the demonstrative pronouns, there are three sets of demonstratives, the base form, the reduplicated one (obtained by reduplicating the first syllable), and the emphatic one, with added \forme{ɯ-} prefix. Note that the latter two sets are not attested in the dual for determiners in the corpus, but the forms exist and are easily deducible from the corresponding plural ones. In addition, there is a medial demonstrative \forme{nɤki} which occurs in prenominal position.

\begin{table}
\caption{Demonstrative determiners}\label{tab:dem.determiners}
\begin{tabular}{ll|l|ll} 
\lsptoprule
&Base form & Reduplicated & Emphatic \\
\midrule
\textsc{prox.sg} & \forme{ki} & \forme{kɯki} &  \forme{ɯkɯki}  \\
\textsc{dist.sg} & \forme{nɯ} &  \forme{nɯnɯ} & \forme{ɯnɯnɯ} \\
\midrule
\textsc{prox.pl} & \forme{kɯni} & X &  X \\
\textsc{dist.pl} & \forme{nɯni} &  X & X \\
\midrule
\textsc{prox.pl} & \forme{kɯra} & \forme{kɯkɯra} &  \forme{ɯkɯkɯra}  \\
\textsc{dist.pl} & \forme{nɯra} &  \forme{nɯnɯra} & \forme{ɯnɯnɯra} \\
\midrule
\textsc{medial} &  \forme{nɤki} \\
\lspbottomrule
\end{tabular}
\end{table}

In Japhug, as in other Gyalrong languages, demonstrative determiners can be either/both pre- and postnominal, as shown by an example such as (\ref{ex:ki.srWnloRpW.ki}) with the proximal \forme{ki} both before and after the noun \japhug{srɯnloʁpɯ}{little ring}.

\begin{exe}
\ex \label{ex:ki.srWnloRpW.ki}
 \gll aʑo ɣɯ-ɕaβ-a tɤ-ŋu tɕe, ki srɯnloʁ-pɯ ki ɲɯ-ɕtʰɯz-a tɕe,  \\
 \textsc{1sg} \textsc{inv}-catch.up:\textsc{fact}-\textsc{1sg} \textsc{pfv}-be \textsc{lnk} \textsc{dem}.\textsc{prox} ring-\textsc{dim} \textsc{dem}.\textsc{prox} \textsc{ipfv}:\textsc{west}-turn.toward-\textsc{1sg} \\
\glt `When (the râkshasas) will be about to catch up with me, I will  turn this little ring towards west (in their direction).' (28-smAnmi, 222)
\end{exe}

All possible combinations of base demonstratives (B) and reduplicated demonstratives (R) are attested as pre- or postnominal determiners:

\begin{itemize}
\item BNB: \forme{ki} N \forme{ki}, \forme{nɯ} N \forme{nɯ} (\ref{ex:ki.srWnloRpW.ki})
\item RNB: \forme{kɯki} N \forme{ki}, \forme{nɯnɯ} N \forme{nɯ} (\ref{ex:kWki.tAYi.ki})
\item BNR: \forme{ki} N \forme{kɯki}, \forme{nɯ} N \forme{nɯnɯ} (\ref{ex:ki.rgAtpu.kWki})
\item RNR: \forme{kɯki} N \forme{kɯki}, \forme{nɯnɯ} N \forme{nɯnɯ} (\ref{ex:kWki.qingjiao.kWki})
\end{itemize}  

The types BNB and RNB, with the postnominal determiner as a base demonstrative, are by far the most common ones in the corpus.

\begin{exe}
\ex \label{ex:kWki.tAYi.ki}
 \gll  aʑo kɯki tɤɲi ki lu-nɤkʰɯkʰrɯt-a tɕe \\
 \textsc{1sg} \textsc{dem}.\textsc{prox} staff \textsc{dem}.\textsc{prox} \textsc{ipfv}:\textsc{upstream}-drag-\textsc{1sg} \textsc{lnk} \\
 \glt `I will drag along this staff (on the ground).' (Kunbzang2003, 225)
\end{exe}
 

\begin{exe}
\ex \label{ex:kWki.qingjiao.kWki}
 \gll iɕqʰa kɯki <qingjiao> kɯki tɕe, ɯ-qa kɯ-wɣrum ɲɯ-ŋu. \\
 the.aforementioned \textsc{dem}.\textsc{prox} plant.name \textsc{dem}.\textsc{prox} \textsc{lnk} \textsc{3sg}.\textsc{poss}-root \textsc{nmlz}:S/A-be.white \textsc{sens}-be \\
 \glt `This (plant that is called) \textit{qingjiao} (in Chinese), its root is white (unlike the other \textit{qingjiao} whose root is red).' (17-ndZWnW, 81)
\end{exe}

\begin{exe}
\ex \label{ex:ki.rgAtpu.kWki}
 \gll ki rgɤtpu kɯki kɯ, iɕqʰa, qaʑo nɯ to-mtsʰi qʰe, li tʂu kɯ-wxti nɯtɕu jo-ɕe tɕe, \\
\textsc{dem}.\textsc{prox} old.man \textsc{dem}.\textsc{prox} \textsc{erg} the.aforementioned sheep \textsc{dem} \textsc{ifr}-lead \textsc{lnk} again road \textsc{nmlz}:S/A-be.big \textsc{dem}:\textsc{loc} \textsc{ifr}-go \textsc{lnk} \\
\glt `The old man, leading the sheep, went to the big road.' (150822 laoye zuoshi zongshi duide-zh, 101
\end{exe}

The emphatic form is only used prenominally as in (\ref{ex:WkWki.arZaB.kWki}) to differentiate in case of confusion -- in this case, because the story is about two persons designated by the term \japhug{tɤ-rʑaβ}{wife}, even if they have different possessors (\textsc{3sg} vs \textsc{1sg}).

\begin{exe}
\ex \label{ex:WkWki.arZaB.kWki}
 \gll   nɯ ɯ-rʑaβ nɯ kɯ, ɯkɯki a-rʑaβ kɯki, kɯki ɕkom ki na-sɯ-ɤβzu tɕe, \\
 \textsc{dem} \textsc{3sg}.\textsc{poss}-wife \textsc{dem} \textsc{erg} \textsc{dem}.\textsc{prox}.\textsc{emph} \textsc{1sg}.\textsc{poss}-wide \textsc{dem}.\textsc{prox} \textsc{dem}.\textsc{prox} muntjac \textsc{dem}.\textsc{prox}  \textsc{pfv}:3\fl{}3'-\textsc{caus}-become \textsc{lnk} \\
\glt `His wife turned this wife of mine into this muntjac.' (140512 fushang he yaomo-zh, 187)
\end{exe}

When the postnominal demonstrative is in plural or dual form, the prenominal one is generally unmarked for number, as in (\ref{ex:kWki.tCheme.kWra}).

\begin{exe}
\ex \label{ex:kWki.tCheme.kWra}
 \gll kɯki tɕʰeme kɯra nɯ-rca aʑo tu-ɕe-a ɲɯ-ntsʰi ma mɯ́j-pe \\
 \textsc{dem} girl \textsc{dem}:\textsc{pl} \textsc{3pl}.\textsc{poss}-following \textsc{1sg} \textsc{ipfv}:\textsc{up}-go-\textsc{1sg} \textsc{sens}-have.better apart.from \textsc{neg}:\textsc{sens}-be.good \\
 \glt `I have no other choice but to go (to heaven) with these girls.' (31-deluge, 61)
 \end{exe}

However, there are also a few examples with plural marking on both pre- and postnominal demonstratives, as in (\ref{ex:nWnWra.pGa.nWra}), a remarkable phenomenon given the fact that the number markers are strictly postnominal. Plural marking on the prenominal demonstrative with a singular postnominal demonstrative is not attested.
 
\begin{exe}
\ex \label{ex:nWnWra.pGa.nWra}
 \gll nɯnɯra pɣa nɯra lonba ʑo ɲɤ-me-nɯ tɕe, ʁʑɯnɯ sqaptɯɣ ɲɤ-k-ɤpa-nɯ-ci. \\
 \textsc{dem.pl} bird  \textsc{dem.pl}  all \textsc{emph} \textsc{ifr}-not.exist \textsc{lnk} young.man eleven \textsc{ifr}-\textsc{evd}-become-\textsc{pl}-\textsc{evd} \\
 \glt `All those birds disappeared, and became eleven young men.' (140520 ye tiane-zh, 121)
\end{exe}

Proximal prenominal demonstratives can be combined with the postnominal \forme{nɯ}, as in (\ref{ex:kWki.Xpi.nW}), where the latter one is used as a topic marker. The opposite combination, a distal prenominal demonstrative with proximal postnominal one, is not attested in the corpus and presumably agrammatical.

\begin{exe}
\ex \label{ex:kWki.Xpi.nW}
 \gll kɯki χpi nɯ pɯpɯŋu nɤ,  \\
 \textsc{dem}.\textsc{prox} story \textsc{dem} \textsc{top} \textsc{lnk} \\
 \glt `As far as this story goes,' (11 examples in the corpus)
\end{exe}

The medial demonstrative \forme{nɤki}, used to designate referents closer to the addressee than the speaker, is found as a pronoun (§ \ref{sec:medial.dem.pro}), but also occurs as a prenominal determiner, with or without postnominal demonstrative (either proximal or distal), as in (\ref{ex:nAki.nAtAYi}) and (\ref{ex:nAki.nAtAri}). It is frequently used with a noun taking a second person possessive prefix -- note that the first syllable \forme{nɤ-} of the demonstrative \forme{nɤki} itself probably originates from the second singular possessive, as proposed in § \ref{sec:medial.dem.pro}.

\begin{exe}
\ex \label{ex:nAki.nAtAYi}
 \gll nɤki nɯ-tɤɲi ɯ-taʁ kɤ-rɤt nɯ ɯβrɤ-kɯ-z-nɤmɲo-a-nɯ \\
 \textsc{dem}:\textsc{medial} \textsc{2pl}.\textsc{poss}-staff \textsc{3sg}.\textsc{poss}-on \textsc{nmlz}:P-write \textsc{dem} \textsc{pot}-2\fl{}1-\textsc{caus}-watch-\textsc{1sg}-\textsc{pl} \\
 \glt `Would you show me what is written on that staff of yours?' (2003sras, 61)
\end{exe}

\begin{exe}
\ex \label{ex:nAki.nAtAri}
 \gll nɤki nɤ-tɤ-ri nɯ ŋotɕu pɯ-tu \\
 \textsc{dem}:\textsc{medial} \textsc{2sg}.\textsc{poss}-\textsc{indef}.\textsc{poss}-thread \textsc{dem} where \textsc{pst}.\textsc{ipfv}-exist \\
\glt `That thread of yours, where is it from?' (Norbzang2005, 180)
\end{exe}

The relative position of prenominal demonstrative and other pronominal elements is not free. The aforementioned topic marker \forme{iɕqʰa} strictly occurs before prenominal demonstratives (as in \ref{ex:kWki.qingjiao.kWki} and \ref{ex:kWki.XsAr.pGAtCW} respectively), while nominal modifiers such as \japhug{χsɤr}{gold} in \ref{ex:kWki.XsAr.pGAtCW} appear closer to the noun. Pronouns coreferent with a possessive prefix on the head noun, however, can be placed either after (\ref{ex:nW.aZo.aCArW.nW}) or before (\ref{ex:aZo.ki.aku.ki}) prenominal demonstratives.

\begin{exe}
\ex \label{ex:kWki.XsAr.pGAtCW}
 \gll kɯki χsɤr pɣɤtɕɯ ki nɤ-jaʁ ɲɯ-kham-a ŋu \\
\textsc{dem}.\textsc{prox} gold bird \textsc{dem}.\textsc{prox} \textsc{1sg}.\textsc{poss}-hand \textsc{ipfv}-give[III]-\textsc{1sg} be:\textsc{fact} \\
\glt `(If you succeed) I will give you this golden bird.' (2012qachGa, 46)
\end{exe}

\begin{exe}
\ex \label{ex:nW.aZo.aCArW.nW}
 \gll nɯtɕu a-tɯrsa ŋu, tɕe nɤʑo kɯ [nɯ aʑo a-ɕɤrɯ nɯnɯra] a-tɤ-tɯ-tɕɤt tɕe, \\
 \textsc{dem}:\textsc{loc} \textsc{1sg}.\textsc{poss}-tomb be:\textsc{fact} \textsc{lnk} \textsc{2sg} \textsc{erg} \textsc{dem} \textsc{1sg} \textsc{1sg}.\textsc{poss}-bone \textsc{dem}:\textsc{pl} \textsc{irr}-\textsc{pfv}-2-take.out \textsc{lnk} \\
\glt `My tomb is there, if you take out my bones (from it),' (150907 niexiaoqian-zh, 109)
\end{exe}

\begin{exe}
\ex \label{ex:aZo.ki.aku.ki}
 \gll  kɯki, aʑo [ki a-ku ki] pɯ-pʰɯt ra \\
 \textsc{dem}.\textsc{prox} \textsc{1sg}  \textsc{dem}.\textsc{prox} \textsc{1sg}.\textsc{poss}-head  \textsc{dem}.\textsc{prox} \textsc{imp}-cut have.to:\textsc{fact} \\
 \glt `Please behead me!' (140507 jinniao-zh, 292)
\end{exe}

The principles governing the presence and absence of the demonstrative determiners, and the choice of the various patterns described above, is particularly complex to describe and will be a topic for future research, when a larger corpus of texts will become available. While the proximal demonstratives always have some deictic function (although it may not be always appropriate to translate them with a demonstrative in other languages such as English), the distal demonstratives clearly contribute to marking topic (§ \ref{sec:topic}) and definiteness (§ \ref{sec:definiteness}), and disentangling these various functions is a complex matter.

\subsection{Quantifiers} \label{sec:quantifiers.determiners}
\subsubsection{Universal quantifiers} \label{sec:universal.quant}
\subsubsection{Mid-scalar quantifier} \label{sec:tsuku}
(\ref{sec:partitive.pronouns})

\subsection{Indefinite and definite markers} \label{sec:indefinite.markers}

\subsubsection{Indefinite article} \label{sec:indef.article}
The form \japhug{ci}{one} has among its many functions (in addition to pronoun, numeral and adverb, see § \ref{sec:other.pro}, § \ref{sec:partitive.pronouns}, § \ref{sec:identity.modifier}, § \ref{sec:one.to.ten} and § XXX) that of singular indefinite article, as in (\ref{ex:ci.indef}) and (\ref{ex:ci.chAGi}). It is typically used to introduce a new referent in a story.

\begin{exe}
\ex \label{ex:ci.indef}
\gll tɕʰeme kɯ-mpɕɯ\redp{}mpɕɤr ci ɲɤ-nɯ-ɬoʁ \\
girl \textsc{nmlz}:S/A-\textsc{emph}\redp{}beautiful \textsc{indef} \textsc{ifr}-\textsc{auto}-come.out \\
\glt `A very beautiful girl appeared (out of it).' (The flood, 39)
\end{exe}

\begin{exe}
\ex \label{ex:ci.chAGi}
\gll tɕɤlo tɕe tɤ-tɕɯ ci cʰɤ-ɣi qʰe, \\
upstream \textsc{lnk} \textsc{indef}.\textsc{poss}-son \textsc{indef} \textsc{ifr}:\textsc{downstream}-come \textsc{lnk} \\
\glt `A boy came from upstream.' (2003-kWBRa, 41)
\end{exe}

Although \forme{ci} can be used as a partitive pronoun `one of them' (§ \ref{sec:partitive.pronouns}), as a postnominal determiner it does not have partitive meaning. To express a meaning such as `one of the boys', a CN such as \japhug{tɯ-rdoʁ}{one piece} is used instead (§ \ref{sec:ICN}). 

Note that when used as a prenominal modifier, \forme{ci} has a completely different (definite) meaning `the other X' (§ \ref{sec:identity.modifier}). 

There are no dual or plural indefinite articles in Japhug. The plural marker \forme{ra} can occur after the indefinite \forme{ci}, but with a vague associative meaning `and other things' as in (\ref{ex:ci.ra}).

\begin{exe}
\ex \label{ex:ci.ra}
 \gll  ndʑi-tɕɯ ci, ndʑi-me ci ra to-tu. \\
 \textsc{3du}.\textsc{poss}-son \textsc{indef}  \textsc{3du}.\textsc{poss}-girl \textsc{indef} \textsc{pl} \textsc{ifr}-exist \\
 \glt  `They$_{du}$ had a boy and a girl (etc).' (150827 tianluo-zh, 155)
\end{exe}

\subsubsection{Indefinite pronoun as modifier} \label{sec:indefinite}
The indefinite pronoun \japhug{tʰɯci}{something} (§ \ref{sec:thWci}) has marginal uses as a prenominal indefinite modifier, as in  (\ref{ex:thWci.laXCi}), (\ref{ex:thWci.WjmNo}) and (\ref{ex:laXtCha.ci.nWnW}) below. 

\begin{exe}
\ex \label{ex:thWci.laXCi}
\gll   tʰɯci laχɕi ci ɕ-pɯ-nɯ-βzjoz-nɯ tɕe, jɤ-ɕe-nɯ ra \\
something trade \textsc{indef} \textsc{transloc-imp-auto}-learn-\textsc{pl} \textsc{lnk} \textsc{imp}-go-\textsc{pl} have.to:\textsc{fact} \\
\glt `Go and learn some trade!' (140508 benling gaoqiang de si xiongdi-zh, 29)
 \end{exe}
 
 This construction arose perhaps from the use of the pronoun \forme{tʰɯci} as head of a postnominal relative clause with the verb \japhug{fse}{be like}, as illustrated by examples like (\ref{ex:thWci.kAnWsaXCWB}) or (\ref{ex:thWci.akAspa}) in § \ref{sec:thWci}. Turning the verb \japhug{fse}{be like} to a finite form as in (\ref{ex:thWci.WjmNo}) could cause the indefinite \forme{tʰɯci}, head of the relative in (\ref{ex:thWci.kAnWsaXCWB}), to be reanalyzed as the prenominal modifier of the immediately adjacent noun in (\ref{ex:thWci.WjmNo}).

 \begin{exe}
\ex \label{ex:thWci.kAnWsaXCWB}
\gll nɯra [tʰɯci [kɤ-nɯsaχɕɯβ kɯ-fse]] pɯ-ŋu wo.  \\
\textsc{dem}:\textsc{pl} something \textsc{inf}-have.a.contest \textsc{nmlz}:S/A-be.like \textsc{pst}.\textsc{ipfv}-be \textsc{sfp} \\
\glt `It was like a kind of contest.' (160706 thotsi, 16)
 \end{exe}
 
\begin{exe}
\ex \label{ex:thWci.WjmNo}
\gll [tʰɯci ɯ-jmŋo] ci ʑo pɯ-fse ri \\
something \textsc{3sg}.\textsc{poss}-dream one \textsc{emph} \textsc{pst}.\textsc{ipfv}-be.like \textsc{lnk} \\
\glt `It looked like (he had had) some dream.' (Lobzang2005, 74)
 \end{exe}
 
 
\subsubsection{The marking of definiteness} \label{sec:definiteness}
Japhug has no dedicated definite determiner, but  \forme{nɯ} and \forme{nɯnɯ}  as demonstrative determiners (\ref{sec:demonstrative.determiners}) and as topic markers (\ref{sec:topic}) and the prenominal aforementioned topic marker \forme{iɕqʰa} (§ \ref{sec:iCqha}) are generally used with definite referents.  

Example (\ref{ex:ci.joGi}) illustrates a typical example with the determiner \forme{nɯ}; the indefinite article \forme{ci} (§ \ref{sec:indef.article}) occurs in the first introduction of a new referent in the story as in the first clause of example (\ref{ex:ci.joGi}), but on the following occurrence of the same noun \forme{nɯ} is found.

\begin{exe}
\ex \label{ex:ci.joGi}
 \gll  tɕe qajdo ci jo-ɣi tɕe, tɕe qajdo nɯ kɯ `mo laz tu, pʰo laz me' to-ti. \\
 \textsc{lnk} crow \textsc{indef} \textsc{ifr}-come \textsc{lnk} \textsc{lnk} crow \textsc{dem} \textsc{erg} girl karma exist:\textsc{fact} boy karma not.exist:\textsc{fact} \textsc{ifr}-say \\
 \glt `A crow came. The crow said: `The girl will have chance, the boy won't.'' (28-qAjdoskAt, 8)
\end{exe}

However, although nouns phrases followed by \forme{nɯ} and \forme{nɯnɯ} more often than not denote definite referents, these determiners cannot be analyzed as definite articles, as noun phrases with \forme{nɯ} or \forme{nɯnɯ} can in certain cases have indefinite referents. 

A very clear case of use of \forme{nɯ} with an indefinite referent occurs on nouns serving as heads of head-internal relative clauses. A well-attested typological generalization is that in this type of relative clauses, definiteness marking is agrammatical (see \citealt{basilico96internally} and § XXX). In Khroskyabs, \citet[636]{lai17khroskyabs} reports that the definiteness marker \forme{=tə} is indeed not accepted on the head noun of head-internal relatives. In Japhug however, \forme{nɯ} does occur in such a syntactic context. For instance, in (\ref{ex:tAnmaR.nW.kW}), the head \forme{tɤ-nmaʁ nɯ kɯ} is subject of the participle \japhug{ɲɯ-kɯ-nɯ-ɕar}{looking for}, and is embedded in the participial relative clause indicated in brackets -- the presence of the ergative \forme{kɯ} precludes to analyze it as a post-nominal relative (§ XXX). From the meaning of the sentence the head \japhug{tɤ-nmaʁ}{husband} is clearly indefinite non-specific non-generic  (see \citealt[286-291]{lehmann84relativsatz}). The fact that it takes the marker \forme{nɯ} shows that this marker, unlike Khroskyabs \forme{=tə}, is not primarily marking definiteness.

\begin{exe}
\ex \label{ex:tAnmaR.nW.kW}
 \gll tɕeri [tɤ-nmaʁ nɯ kɯ ɯ-rʑaʁ kɯ-ɤntɕʰɯ ɲɯ-kɯ-nɯ-ɕar], aʁɤndɯndɤt tɤndɤɣri tu-kɯ-βzu pjɤ-tu.  \\
but  \textsc{indef}.\textsc{poss}-husband \textsc{dem} \textsc{erg} \textsc{3sg}.\textsc{poss}-wife  \textsc{nmlz}:S/A-be.many \textsc{ipfv}-\textsc{nmlz}:S/A-\textsc{auto}-search everywhere  illegitimate.child  \textsc{ipfv}-\textsc{nmlz}:S/A-make \textsc{ifr}.\textsc{ipfv}-exist \\
\glt `However there were husbands who were looking for several women and had illegitimate children.' (140427 tAndAGri, 3)
\end{exe}

Other cases of indefinite noun phrase with \forme{nɯ} are observed with left-dislocated topics. In example (\ref{ex:RnWz.nWnW}), we find a type of tail-head linkeage  (§ XXX) where both the noun phrase \japhug{spjaŋkɯ ʁnɯz}{two wolves} and the verb \japhug{ɲɤ-k-ɤtɯɣ-ci}{he met} are repeated; in the second occurrence, the noun phrase is topicalized and is followed by the topic marker \forme{nɯnɯ}, with a slight pause of hesitation. The determiner \forme{nɯnɯ} in this clause, unlike \forme{nɯ} in (\ref{ex:ci.joGi}), does not mark definiteness: that clause cannot be understood as `He met the two wolves'.

\begin{exe} 
\ex \label{ex:RnWz.nWnW} 
 \gll spjaŋkɯ ʁnɯz ɲɤ-k-ɤtɯɣ-ci. spjaŋkɯ ʁnɯz nɯnɯ, tɕendɤre ɲɤ-k-ɤtɯɣ-ci tɕe iɕqʰa, kɯ-rɤ-ntɕʰa nɯ wuma ʑo ɲɤ-mu. \\ 
 wolf two \textsc{ifr}-\textsc{evd}-meet-\textsc{evd}  wolf two \textsc{dem} \textsc{lnk} \textsc{ifr}-\textsc{evd}-meet-\textsc{evd} \textsc{lnk} the.aforementioned \textsc{nmlz}:S/A-\textsc{a.pass}:\textsc{n.hum}-kill \textsc{dem} really \textsc{emph} \textsc{ifr}-be.afraid \\ 
 \glt `He$_i$ (the butcher) met two wolves. He$_i$ met two wolves, and the butcher$_i$ was very much afraid.' (150902 liaozhai lang-zh, 7-8)
\end{exe}

The determiners \forme{nɯ} or \forme{nɯnɯ} are not attested in the corpus with the indefinite singular article \forme{ci} if both have scope on the same noun. In all cases with \forme{ci} followed by \forme{nɯ} (other than the identity pronoun in § \ref{sec:other.pro}), or of \forme{nɯ} followed by \forme{ci} in the corpus, they belong to different constituents. For instance, in (\ref{ex:ci.YAZGAsAphAr}), \forme{ci} is in adverbial use (`a little, once', see § XXX) and does not belong to the preceding noun phrase.  

\begin{exe}
\ex \label{ex:ci.YAZGAsAphAr}
\gll [tɕʰeme nɯ] ci ɲɤ-ʑɣɤ-sɤpʰɤr qʰe  \\
girl \textsc{dem} one \textsc{ifr}-\textsc{refl}-shake \textsc{lnk} \\
\glt `The girl shook herself.' (02-deluge2012, 125)
\end{exe}

In (\ref{ex:laXtCha.ci.nWnW}) although \forme{nɯnɯ} follows \forme{ci}, it has scope over the both preceding phrases, which are left-dislocated and followed by a pause.

\begin{exe}
\ex \label{ex:laXtCha.ci.nWnW}
\gll  kɤ-xtɕɤr tɕe nɯnɯ tɕe tɕe iɕqʰa, [[tʰɯci tɯmbri tɤ-ri kɯ-fse kɯ] [laχtɕʰa ci] nɯnɯ], ci kú-wɣ-sɯ-pa tɕe, kú-wɣ-xtɕɤr, \\
\textsc{inf}-attach \textsc{lnk} \textsc{dem} \textsc{lnk} \textsc{lnk} the.aforementioned something rope \textsc{indef}.\textsc{poss}-thread \textsc{nmlz}:S/A-be.like \textsc{erg} thing \textsc{indef} \textsc{dem} one \textsc{ipfv}-\textsc{inv}-\textsc{caus}-do \textsc{lnk} \textsc{ipfv}-\textsc{inf}-attach \\
\glt ``To attach' (means), to put together, attach something with something like a rope or a thread.'  (150902 kAxtCAr, 2-3)
\end{exe}

The aforementioned topic marker \forme{iɕqʰa} (§ \ref{sec:iCqha}) is almost always used with definite referents when prenominal, as in (\ref{ex:RnWz.nWnW}) above, and is the closest candidate to be analyzed as a definiteness marker in Japhug. It does occur with non-specific generic referents as in (\ref{ex:lWlAmu}), including some that are very clearly indefinite as in (\ref{ex:lApWG}); note the absence of postnominal determiner \forme{nɯ} (\ref{ex:lApWG}).

\begin{exe}
\ex \label{ex:lWlAmu}
 \gll iɕqʰa lɯlɤmu nɯ tʰɯ-rɤpɯ tɕe tɕe ɯ-sŋi tɕe kɤ-nɯ-rŋgɯ nɯ stʰɯci mɯ́j-tsu ma ɯ-pɯ ra χse ɲɯ-ra tɕe, \\
 the.aforementioned female.cat \textsc{dem} \textsc{ipfv}-bear.young \textsc{lnk} \textsc{lnk} \textsc{3sg}.\textsc{poss}-day \textsc{lnk} \textsc{inf}-\textsc{auto}-lie.down \textsc{dem} so.much \textsc{neg}:\textsc{sens}-have.time.to \\
 \glt `A/the female cat (unlike male cats), when it had had youngs, does not have time to sleep during the day, as it has to feed its youngs.' (21-lWLU, 
\end{exe}

\begin{exe}
\ex \label{ex:lApWG}
\gll  iɕqʰa lɤpɯɣ ɯ-rɣi ʑo fse. \\
the.aforementioned radish \textsc{3sg}.\textsc{poss}-seed \textsc{emph} be.like:\textsc{fact} \\
\glt `It looks like a radish seed.' (hist-26-qro-fourmi, 61)
\end{exe}

In  (\ref{ex:laXtCha.ci.nWnW}), \forme{iɕqʰa}  also precedes two phrases involving indefinite referents, but  there is a marked pause, and this is a case of \forme{iɕqʰa} in its function as speech filler (see § XXX).

\subsubsection{Absence of definiteness marking}
Like many languages (\citealt[130]{creissels06sgit1}), Japhug uses bare nouns without any definiteness marking. Bare nouns are most often non-referential, as \japhug{tɕʰeme}{girl} in (\ref{ex:tCheme.tWtAtu}).

\begin{exe}
\ex \label{ex:tCheme.tWtAtu}
\gll ʁnaʁna tɕʰeme tɯ\redp{}tɤ-tu nɤ, kɤndʑɯsqʰaj tu-kɤ-sɯ-βzu \\
both girl \textsc{cond}\redp{}\textsc{pfv}-exist \textsc{lnk} \textsc{coll}:sister \textsc{ipfv}-\textsc{inf}-\textsc{caus}-make \\
\glt `If both of them have girls, let them be sisters.' (zrAntCW, 4)
\end{exe}

Bare nouns are less common with referential nouns (except in answers to questions), but examples can be found, as \japhug{qacʰɣa}{fox} in (\ref{ex:qachGa.kW}).

\begin{exe}
\ex \label{ex:qachGa.kW}
\gll qacʰɣa 	kɯ maχtɕɯ tɤ-tɯt-a nɯ mɤ-tɯ-ste ti ɲɯ-ŋu \\
fox \textsc{erg} I.told.you.so \textsc{pfv}-say[II]-\textsc{1sg} \textsc{dem} \textsc{neg}-2-do.like[III]:\textsc{fact} say:\textsc{fact} \textsc{sens}-be \\
\glt `The fox says: `You do not do as I told you to." (2003qachGa, 44)
\end{exe}

Personal names generally occur as bare nouns, without any definiteness marker as in (\ref{ex:WrJAnpanma}), but there are no constraints against co-occurrence of personal names with the determiner \forme{nɯ} either (see § \ref{sec:personal.names.modifiers}).

\begin{exe}
\ex \label{ex:WrJAnpanma}
\gll  ɯrɟɤnpanma kɯ ʁlaŋsaŋtɕhin ɯ-ɕki  \\
 Padmasambhava \textsc{erg} Gesar \textsc{3sg}-\textsc{dat} \\
\glt `Padmasambhava (told) Gesar.' (Gesar, 2)
\end{exe}

 \subsection{Topic markers} \label{sec:topic}
 
  \subsubsection{Delimitative topic} \label{sec:delimitative}
The delimitative topic marker \forme{pɯ\redp{}pɯ-ŋu nɤ} `as for..., concerning...' is transparently derived from the past imperfective of the verb `be' in conditional form `if it was...' (with reduplication of the first syllable, see § XXX), as other copulas such as affirmative \japhug{ɕti}{be} and \japhug{maʁ}{not be} in (\ref{ex:pWpWmaʁ}).

\begin{exe}
\ex \label{ex:pWpWmaʁ}
\gll nɯnɯ koŋla ʑo tɤɕime pɯ\redp{}pɯ-maʁ nɤ \\
\textsc{dem} really \textsc{emph} princess \textsc{cond}\redp{}\textsc{pst.ipfv}-not.be lnk \\
\glt `If she was not really a princess,' (140519 wandou gongzhu, 71)
\end{exe}

The delimitative construction generally has scope over a noun phrase, which can have an additional demonstrative \forme{nɯ} as topicalizer as in (\ref{ex:nW.pWpWNunA}) (see § \ref{sec:nW.topic}).

\begin{exe}
\ex \label{ex:nW.pWpWNunA}
\gll a-mu nɯ pɯ\redp{}pɯ-ŋu nɤ, qhlɯ ʁdɯxpakɤrpu ɣɯ ɯ-me stu kɯ-xtɕi nɯ a-mu ɲɯ-pe, \\
\textsc{1sg}.\textsc{poss}-mother \textsc{dem} \textsc{cond}\redp{}\textsc{pst.ipfv}-be \textsc{lnk} nâga p.n \textsc{gen} \textsc{3sg}.\textsc{poss}-daughter most \textsc{nmlz}:S/A-be.small \textsc{dem} \textsc{1sg}.\textsc{poss}-mother \textsc{sens}-be.good \\
\glt `As for my mother, the daughter of the Nâga Gdugpa dkarpo is good to be my mother.' (Gesar, 5)
\end{exe}

In this construction, the verb is in the process of becoming grammaticalized as a topic particle. It is possible to find examples where the verb still takes person indexation in the delimitative construction when the topicalized element is a first or second person pronoun, as in (\ref{ex:pWpWNuanA}). 

\begin{exe}
\ex \label{ex:pWpWNuanA}
\gll aʑo pɯ\redp{}pɯ-ŋu-a nɤ, kɤndʑɯʁi kɯmŋu tu-j, \\
\textsc{1sg} \textsc{cond}\redp{}\textsc{pst.ipfv}-be-\textsc{1sg} \textsc{lnk} siblings five exist:\textsc{fact}-\textsc{1sg} \\
\glt `Concerning me, we are five brothers and sisters.' (hist140501 tshering skyid, 1)
\end{exe}

However, there are also examples with first or second person pronoun without indexation on the delimitative marker, as in (\ref{ex:pWpWNunA}), (\ref{ex:pWpWNunA2}) and (\ref{ex:pWpWNunA3}), where a first person singular form \forme{pɯ\redp{}pɯ-ŋu-a nɤ} or second person \forme{pɯ\redp{}pɯ-tɯ-ŋu nɤ} would have been expected. Such examples show that \forme{pɯpɯŋunɤ} has ceased to be analyzed as a verb form at least in these cases. Moreover, third person plural and dual indexation is hardly ever found in the delimitative construction.

\begin{exe}
\ex \label{ex:pWpWNunA}
\gll nɤʑo pɯpɯŋunɤ, ɬɤndʐi ra ɣɯ nɯ-kɯ-βʁa, nɯ-rɟɤlpu tɯ-ŋu \\
\textsc{2sg} as.for demon \textsc{pl} \textsc{gen} \textsc{3pl.poss}-\textsc{nmlz}:S/A-be.victorious \textsc{3pl.poss}-king 2-be:\textsc{fact} \\
\glt `You, you are the king of the demons.' (hist140512 fushang he yaomo-zh, 61)
\end{exe}

\begin{exe}
\ex \label{ex:pWpWNunA2}
\gll  aʑo kɯ-fse pɯpɯŋunɤ, ɕɯŋgɯ sɤ-xtɕɯ\redp{}xtɕi nɯtɕu, χpɯn lɤ-kɤ-ta, \\
\textsc{1sg} \textsc{nmlz}:S/A-be.like as.for  before \textsc{conv}-\redp{}be.small \textsc{dem}:\textsc{loc} monk \textsc{pfv}:\textsc{upstream}-\textsc{nmlz}:P-put \\
\glt `For instance me, (I was) sent to become monk early in my childhood.' (160721 XpWN, 7)
  \end{exe}

\begin{exe}
\ex \label{ex:pWpWNunA3}
\gll aʑo pɯpɯŋunɤ, nɯnɯ [...] aʑo ɣɯ a-ndʐa nɯ tu-o<nɯ>lɯlat-a pɯ-ŋu tɕe, \\
\textsc{1sg} as.for \textsc{dem} { } \textsc{1sg} \textsc{gen} \textsc{1sg}.\textsc{poss}-reason \textsc{dem} \textsc{ipfv}-<\textsc{auto}>fight-\textsc{1sg} \textsc{pst}.\textsc{ipfv}-be \textsc{lnk} \\
\glt  As for me, I was fighting for my own sake.' (140512 abide he mogui-zh, 92)
 \end{exe}
 
The delimitative topic  construction is appropriate to introduce the main topic of a following discourse (as in \ref{ex:pWpWNuanA} and \ref{ex:pWpWNunA2}), but can be used for contrastive topics, as in example (\ref{ex:pWpWNunA3}) where the speaker expresses a contrast between his and the addresses action (`you, you were fighting for the sake of other people').


 \subsubsection{Aforementioned topic} \label{sec:iCqha}
 The marker \japhug{iɕqʰa}{the aforementioned}  is used on referents that have been previously mentioned in the same story, usually only a few sentences back. It is strictly prenominal. 
 
Example (\ref{ex:iCqha.aforementioned}) illustrates the most typical use of this marker. Sentence (\ref{ex:kAtWm}) introduces a new reference, \japhug{kɤtɯm}{ball of thread} marked with the indefinite article \forme{ci} (§ \ref{sec:indef.article}). Three clauses later in (\ref{ex:iCqha.kAtWm}), the same referent occurs again with two topic markers, the postnominal \textit{nɯ} and the prenominal \textit{iɕqʰa}.
 
 
\begin{exe}
\ex \label{ex:iCqha.aforementioned}
\begin{xlist}
\ex \label{ex:kAtWm}
\gll `razri \textbf{kɤtɯm} \textbf{ci} ɲɯ-ra, taqaβ ci ɲɯ-ra' to-ti qʰe   \\
 thread ball \textsc{indef} \textsc{sens}-need needle \textsc{indef} \textsc{sens}-need \textsc{ifr}-say \textsc{lnk}  \\
\glt `He told (Rgyabza) `I need a ball of thread and a needle.''  
\ex  
\gll tɕendɤre ɲɤ-kʰo qʰe,  \\
\textsc{lnk} \textsc{ifr}-give \textsc{lnk}   \\
\glt `She gave it to him.'
\ex 
\gll  tɕe ɯ-ndzɤtsʰi ka-tsɯm-nɯ nɯtɕu qʰe tɕe,   \\
 \textsc{lnk} \textsc{3sg}.\textsc{poss}-meal \textsc{pfv}:3\fl{}3'-bring-\textsc{pl} \textsc{dem}:\textsc{loc}  \textsc{lnk} \textsc{lnk}    \\
\glt `When they brought his meal,'
\ex \label{ex:iCqha.kAtWm}
\gll   \textbf{iɕqʰa} \textbf{kɤtɯm} \textbf{nɯ} ɯʑo kɯ ko-ndo, \\
   the.aforementioned ball \textsc{dem} \textsc{3sg} \textsc{erg} \textsc{ifr}-take \\
\glt `he took the ball of thread, and...' (Gesar 270-272)
\end{xlist}
\end{exe}
 
A systematic study of the use of the topic marker \forme{iɕqʰa} in Japhug must overcome two inherent difficulties. First, this topic marker is homophonous with (and historically related to) the speech filler \forme{iɕqʰa} (§ XXX) and with the adverb \japhug{iɕqʰa}{just now}, which can also precede noun phrases. Listening to the sound files can help distinguishing between the three, as the speech filler is always followed by a pause (and optionally by the demonstrative \forme{nɯ}), but there are still ambiguous sentences (see below). Second, \forme{iɕqʰa} occurs on nouns designating entities that the speaker considers to have been previously referred to in the conversation, even if they are not present in the same recording. 

For instance in (\ref{ex:iCqha.pɣArnoR}) the noun \japhug{pɣɤrnoʁ}{a species of fungus} is used with \forme{iɕqʰa}, although this name does not occur before in the same text; it was however mentioned the day before in another recording.

\begin{exe}
\ex \label{ex:iCqha.pɣArnoR}
\gll nɯ zdɯmqe cʰo iɕqʰa, pɣɤrnoʁ nɯni ndʑi-tsʰɯɣa wuma ʑo naχtɕɯɣ. \\
\textsc{dem} fungi.sp. \textsc{comit} the.aforementioned fungi.sp. \textsc{dem}:\textsc{du} \textsc{3du}.\textsc{poss}-form really \textsc{emph} be:identical:\textsc{fact} \\
\glt `The \forme{zdɯmqe} and the \forme{pɣɤrnoʁ} are very similar.' (23-mbrAZim, 82)
\end{exe}

 
The topic marker \forme{iɕqʰa} transparently comes from the adverb \japhug{iɕqʰa}{just now} (§ XXX). The pivot constructions that allowed reanalysis from adverb to prenominal topic marker are very probably headless relatives (§ XXX) as in  (\ref{ex:iCqha.tAtWta}), or complement clauses as in (\ref{ex:iCqha.ZnWzmWnmuta}). 

\begin{exe}
\ex \label{ex:iCqha.tAtWta}
 \gll  [iɕqʰa tɤ-tɯt-a] nɯ tú-wɣ-stu qʰe, \\
 just.now \textsc{ifr}-say[II]-\textsc{1sg} \textsc{dem} \textsc{ipfv}-\textsc{inv}-do.like \textsc{lnk} \\
\glt `One does as I just said, and...' (2002tWsqar, 139)
\end{exe}

\begin{exe}
\ex \label{ex:iCqha.ZnWzmWnmuta}
 \gll iɕqʰa [ʑ-nɯ-z-mɯnmu-t-a] nɯ mɯ-pjɤ-pe rcama.  \\
the.aforementioned  \textsc{transloc}-\textsc{pfv}-\textsc{caus}-move-\textsc{pst}:\textsc{tr}-\textsc{1sg} \textsc{dem} \textsc{neg}-\textsc{ifr}.\textsc{ipfv}-be.good \textsc{fsp} \\
\glt `It was probably not a good thing that I had moved them (as I said above).' (150819 kumpGa, 45)
 \end{exe}
 
 These sentences are still synchronically ambiguous in Japhug; in  (\ref{ex:iCqha.ZnWzmWnmuta}) the context makes it clear that \forme{iɕqʰa} is the topic marker (since the fact of having moved (the eggs) had been told a few sentences back) and not an adverb `just now' with a temporal reference in the past, as the meaning would be `it was probably not a good thing that I had just moved them' (an impossible interpretation in this context, since this sentence is an explanation why several eggs had not given chicks, several days after they had been brought to another place). However, extracted from the context, both interpretation would be equally possible for (\ref{ex:iCqha.ZnWzmWnmuta}), and correspond to two distinct syntactic structures.

With postnominal (§ XXX) or left-headed head-internal relative clauses (§ XXX) as in (\ref{ex:tWrpa.thafse}), \forme{iɕqʰa} can also be ambiguous. Since the adverb \japhug{iɕqʰa}{just now} can occur both before the object (\ref{ex:tWrpa.thWfseta}) or before the verb (\ref{ex:tWrpa.thWfseta2}) in an independent clause, a relative such as (\ref{ex:tWrpa.thafse}) can be either interpreted `the axe (mentioned above) that he had whetted' (with the topic marker \forme{iɕqʰa} outside of the relative clause, having scope on its head) and `the axe that he had just whetted' with the adverb \japhug{iɕqʰa}{just now} inside the relative clause.

 \begin{exe}
\ex \label{ex:tWrpa.thafse}
 \gll  tɕendɤre <luban> kɯ iɕqʰa [tɯrpa tʰa-fse] nɯ to-ndo tɕe, \\
 \textsc{lnk} p.n. \textsc{erg} the.aforementioned axe \textsc{pfv}:3\fl{}3'-whet \textsc{dem} \textsc{ifr}-take \textsc{lnk} \\
 \glt `Luban took the axe that he had whetted.' (150902 luban-zh, 90)
 \end{exe}

  \begin{exe}
  \ex 
  \begin{xlist}
\ex \label{ex:tWrpa.thWfseta}
 \gll   iɕqʰa tɯrpa tʰɯ-fse-t-a \\
just.now axe \textsc{pfv}-whet-\textsc{pst}:\textsc{tr}-\textsc{1sg} \\
\ex \label{ex:tWrpa.thWfseta2}
 \gll   tɯrpa  iɕqʰa tʰɯ-fse-t-a \\
 axe just.now \textsc{pfv}-whet-\textsc{pst}:\textsc{tr}-\textsc{1sg} \\
 \glt `I just whetted a/the axe.' (elicited)
 \end{xlist}
 \end{exe}

The use of \forme{iɕqʰa} as a topic marker with nouns (as in \ref{ex:iCqha.kAtWm} above) probably took place by reanalysis of the adverb in headless or postnominal relatives, or in complment clauses as above, then generalized to all noun phrases even those without subordinate clause.

\subsubsection{The demonstrative as a topic marker} \label{sec:nW.topic}
The postnominal determiner \forme{nɯ} and its reduplicated form \forme{nɯnɯ} is one of the most common words in Japhug, and has a considerable number of functions. It is used as a demonstrative (\ref{sec:demonstrative.determiners}), contributes to expressing definiteness (\ref{sec:definiteness}) and could be argued to be a subordinator (an analysis not adopted in the present work, see § XXX).

In addition, it is commonly used to mark topic: left-dislocated noun phrases generally (though not compulsorily) take this determiner. For instance, in texts presenting animals or plants, their name on first occurrence is left dislocated and followed by the determiner \forme{nɯ}, as in (\ref{ex:qawWz.nW}).

\begin{exe}
\ex \label{ex:qawWz.nW}
\gll  qawɯz nɯ, (qawɯz nɯ pɯ-tɯ-mto-t, ɣe?)  qawɯz nɯnɯ, nɤki, kɯɕɯŋgɯ tɕe, \\
Edelweiss \textsc{dem} Edelweiss \textsc{dem} \textsc{pfv}-2-see-\textsc{pst}:\textsc{tr} \textsc{sfp} Edelweiss \textsc{dem} \textsc{filler} before \textsc{lnk} \\
\glt `The edelweiss, (you saw Edelweiss before, right?)... The edelweiss, in former times,' (15-babW, 177)
\end{exe}

In its function as a topicalizer, the determiner \forme{nɯ} can follow a noun with postnominal demonstratives, as in (\ref{ex:kWki.nW}).

\begin{exe}
\ex \label{ex:kWki.nW}
\gll tɕeri kɯki mɯntoʁ kɯki nɯ pɯpɯŋunɤ, wuma ʑo kɯ-ʑru, kɯ-pe, \\
\textsc{lnk} \textsc{dem}.\textsc{prox} flower \textsc{dem}.\textsc{prox} \textsc{dem} as.far really \textsc{emph} \textsc{nmlz}:S/A-be.strong \textsc{nmlz}:S/A-be.good \\ 
\glt `But concerning this flower, so precious and nice' (150820 meili de meiguihua, 58)
\end{exe}

 \subsection{Focus markers} \label{sec:focus}
   \subsubsection{Unexpected focus} \label{sec:unexpected}
 \subsubsection{Additive and scalar focus marker \forme{kɯnɤ} } \label{sec:kWnA}
The additive and scalar focus marker \japhug{kɯnɤ}{also, even} follows the constituent over which it has scope, which can be noun phrases, postpositional phrases but also subordinate clauses (these are treated in § XXX). The stress is on the first syllable (\forme{kɯ́nɤ}) and the vowel on the second syllable is often elited (a pronunciation \forme{kɯn} is often heard).

The marker \forme{kɯnɤ} expresses both additive focus, as in (\ref{ex:aZo.kWNA.staRlupa}), and scalar focus, as in (\ref{ex:WNgWz.kWnA.tunAndWtnW}) in affirmative sentences. It is also compatible with negative verb forms, as in (\ref{ex:tWrdoR.kWnA}), expressing the meaning `not even' (see also \japhug{cinɤ}{(not) even one} in § \ref{sec:cinA}).

\begin{exe}
\ex \label{ex:aZo.kWNA.staRlupa}
\gll aʑo kɯnɤ staʁlupa ŋu-a tɕe \\
\textsc{1sg} also born.in.the.tiger.year be:\textsc{fact}-\textsc{1sg} \textsc{lnk} \\
\glt `Me too (like you), I am of the Tiger year.' (2011-05-nyima, 168)
\end{exe}

\begin{exe}
\ex \label{ex:WNgWz.kWnA.tunAndWtnW}
\gll ʑara ʑo ɯ-ŋgɯz kɯnɤ tu-nɤndɯt-nɯ tɕe nɯ kɯ-βʁa ɣɤʑu, kɯ-nŋo ɣɤʑu qʰe, \\
\textsc{3pl} \textsc{emph} \textsc{3sg}.\textsc{poss}-among:\textsc{loc} also \textsc{ipfv}-fight-\textsc{pl} \textsc{lnk} \textsc{dem} \textsc{nmlz}:S/A-win \textsc{sens}:exist \textsc{nmlz}:S/A-lose  \textsc{sens}:exist \textsc{lnk} \\
\glt `Even among themselves, they fight, and there are winners and losers.' (20-sWNgi, 62-63)
\end{exe}
 
\begin{exe}
\ex \label{ex:tWrdoR.kWnA}
\gll tɯ-sŋi mɯntoʁ tɯ-rdoʁ kɯnɤ ci ci tɕe mɯ́j-stʰɯt \\
one-day flower one-piece also once once \textsc{lnk} \textsc{neg}:\textsc{sens}-finish \\
\glt `Sometimes one cannot finish even one pattern (on the belt) in one day.' (2011-06-thaXtsa, 47)
\end{exe}

As an additive focus marker, \forme{kɯnɤ} can be repeated on all the nouns designating the members of a group sharing a particular property, in the construction $X$ \forme{kɯnɤ}, $Y$ \forme{kɯnɤ}  `both $X$ and $Y$', as in (\ref{ex:Dpalcan.kWnA}).

\begin{exe}
\ex \label{ex:Dpalcan.kWnA}
 \gll a-pɯ-ŋu tɕe, aʑo kɯnɤ taʁrdo rɟitpa a-pɯ-ŋu-a, χpɤltɕin kɯnɤ taʁrdo rɟitpa a-pɯ-ŋu, ... nɯ tɕi-rɟit nɯni tɕe taʁrdo rɟitpa ma nɯ ma kɯmaʁ rɟitpa nɯ kɤ-rtsi me.  \\
 \textsc{irr}-\textsc{ipfv}-be \textsc{lnk} \textsc{1sg} also pl.n. lineage  \textsc{irr}-\textsc{ipfv}-be-\textsc{1sg}  p.n. also pl.n. lineage  \textsc{irr}-\textsc{ipfv}-be { } \textsc{dem} \textsc{1du}.\textsc{poss}-offspring \textsc{dem}:\textsc{du} \textsc{lnk} pl.n. lineage \textsc{lnk} \textsc{dem} apart.from other lineage \textsc{dem} \textsc{nmlz}:O-count not.exist:\textsc{fact} \\
 \glt `For instance suppose that both Dpalcan and I were from Taqrdo lineage, then our two children would only count as members of the Taqrdo lineage and no other lineage.' (140426 rJitpa, 13-15)
\end{exe}

The scope of  \forme{kɯnɤ} is generally exclusively on the constituent that it immediately follows, but there are cases where the scope is more extensive. In (\ref{ex:aZo.kWnA.akAsWso}), \forme{kɯnɤ} occurs between the pronoun \forme{aʑo} and the following participial verb form, which bears a \textsc{1sg} possessive prefix \forme{a-} coreferent with that pronoun (see also \ref{ex:aZWG.kWnA} below). The semantic scope of \forme{kɯnɤ} here is on the whole relative \forme{aʑo a-kɤ-sɯso} `(the things) that I want' rather than exclusively on the pronoun \forme{aʑo}.

\begin{exe}
\ex \label{ex:aZo.kWnA.akAsWso}
 \gll aʑo kɯnɤ a-kɤ-sɯso nɯ tɤ-stu-nɯ ra \\
 \textsc{1sg} also \textsc{1sg}.\textsc{poss}-\textsc{nmlz}:O-think \textsc{dem} \textsc{imp}-do.like-\textsc{pl} have.to:\textsc{fact} \\
 \glt `(I will do as you say, but) do also the things I want.' (2003kAndzwsqhaj2, 47)
\end{exe}

The focus marker \forme{kɯnɤ} is found with nouns or pronouns in core argument function, including S (\ref{ex:kWnA.nArca}), O (\ref{ex:nWXpWm.kWnA}), and semi-objects (\ref{ex:kWnA.mAsna}).  Examples with transitive subjects are presented below (\ref{ex:nWra.kWnA} and \ref{ex:Wzda.ra.kWnA}).

 \begin{exe}
\ex \label{ex:kWnA.nArca}
\gll aʑo kɯnɤ nɤ-rca ɣi-a ɕti  \\
\textsc{1sg} also \textsc{2sg}.\textsc{poss}-following come:\textsc{fact}-\textsc{1sg} be.\textsc{affirm}:\textsc{fact} \\
\glt `I am coming with you too.' (2011-05-nyima, 171)
 \end{exe}
 
   \begin{exe}
\ex \label{ex:nWXpWm.kWnA}
\gll    ma nɯ-χpɯm kɯnɤ kʰro mɤ-kɯ-fkaβ kɯ-fse ku-rɤʑi-nɯ  \\
lnk 3pl.poss-knee also much \textsc{neg}-\textsc{nmlz}:S/A-cover \textsc{nmlz}:S/A-be.like \textsc{ipfv}-stay-\textsc{pl} \\
\glt `(Gents) would (wear trousers that did) not cover much even their knees.'  (30-rkAsnom, 5) 
  \end{exe}
  
  \begin{exe}
 \ex \label{ex:kWnA.mAsna}
 \gll   ɯ-ru nɯra laʁdɯn ɯ-jɯ kɯnɤ mɤ-sna, ma mɤ-ngɯt. \\
 \textsc{3sg}.\textsc{poss}-trunk \textsc{dem}:\textsc{pl} tool \textsc{3sg}.\textsc{poss}-handle also \textsc{neg}-be.worth \textsc{lnk}  \textsc{neg}-be.strong:\textsc{fact} \\
 \glt `(The wood from) its trunk is not even good (enough to be used to make) tool handles, as it is not strong.'  (17-xCAj, 79)
  \end{exe}

It also occurs with all types of oblique arguments and adjuncts, including genitive (\ref{ex:aZWG.kWnA}), dative (\forme{ɯ-ɕki} \ref{ex:nWCki.kWnA}),  locational adjuncts in \forme{tɕu} (\ref{ex:kutCu.kWnA}) or \forme{ri} (\ref{ex:ri.kWnA}), temporal adjuncts (\ref{ex:ftCAXcAl.kWnA}) or adjuncts expressing manner or cause (\ref{ex:nWtCu.kWnA}).  
  
   \begin{exe}
\ex \label{ex:aZWG.kWnA}
\gll aʑɯɣ kɯnɤ a-mpʰrɯmɯ a-pɯ-tɯ-sɯ-re ɯ-tɯ́-cʰa \\
\textsc{1sg}:\textsc{gen} also \textsc{1sg}.\textsc{poss}-divination \textsc{irr}-\textsc{pfv}-2-\textsc{caus}-look[III] \textsc{qu}-2-can:\textsc{fact} \\
\glt `Can you ask (the monk) to make a divination for me too?' (The divination, 31)
\end{exe}  
  
   \begin{exe}
\ex \label{ex:nWCki.kWnA}
\gll  tɯ-pi ɣɯ ɯ-nmaʁ ra nɯ-ɕki kɯnɤ `a-pi' tu-kɯ-ti ɕti ma nɯ ma kupa kɯ-fse ʑaka ɯ-rmi me. \\
\textsc{genr}.\textsc{poss}-elder.sibling \textsc{gen} \textsc{3sg}.\textsc{poss}-husband \textsc{pl} \textsc{3pl}.\textsc{poss}-\textsc{dat} also \textsc{1sg}.\textsc{poss}-elder.sibling \textsc{ipfv}-\textsc{genr}-say be.\textsc{affirm}:\textsc{fact} \textsc{lnk} \textsc{dem} apart.from Chinese \textsc{nmlz}:S/A-be.like each \textsc{3sg}.\textsc{poss}-name not.exist:\textsc{fact} \\
\glt  `One calls one's sister's husband (and others from his family) `my elder brother', there are no other special terms as in Chinese.' (140425 kWmdza05)
\end{exe}


  \begin{exe}
\ex \label{ex:kutCu.kWnA}
\gll  kutɕu kɯnɤ nɯ ɲɯ-fse, jɯfɕɯndʐi ra kɯ-xtɕɯ\redp{}xtɕi tɤ-ɣɤndʐo kɯ-fse ri, ɕɤxɕo tɕe kɯ-xtɕɯ\redp{}xtɕi ɲɯ-ʑi kɯ-fse \\
here also \textsc{dem} \textsc{sens}-be.like a.few.days.ago \textsc{nmlz}:S/A-\textsc{emph}\redp{}be.small \textsc{pfv}-be.cold \textsc{nmlz}:S/A-be.like \textsc{lnk} the.last.days \textsc{lnk} \textsc{nmlz}:S/A-\textsc{emph}\redp{}be.small \textsc{sens}-subside \textsc{nmlz}:S/A-be.like \\
\glt `It is like that here too, a few days ago the weather became a little cold, but the last days it has eased a bit.' (conversation, 141027)
  \end{exe}
  
    \begin{exe}
\ex \label{ex:ri.kWnA}
\gll   maldzɯ nɯ, nɯ ɯ-tʰɤcu tsa ri kɯnɤ ɣɤʑu. qarɣɤpɤt ɯ-rca ri kɯnɤ tu-ɬoʁ ɲɯ-ŋu. \\
plant.name \textsc{dem} \textsc{dem} \textsc{3sg}.\textsc{poss}-downstream a.little \textsc{loc} also exist:\textsc{sens} plant.name \textsc{3sg}.\textsc{poss}-among \textsc{loc} also \textsc{ipfv}-come.out \textsc{sens}-be \\
\glt `The \forme{maldzɯ} plant, it is also found in places of slightly lower altitude, but grows also in the same places as  \forme{qarɣɤpɤt} plants.' (18-qromJoR, 81-82)
    \end{exe}
    
\begin{exe}
\ex \label{ex:ftCAXcAl.kWnA}
\gll   kukutɕu ftɕɤχcɤl kɯnɤ <baonuanyi> tu-tɯ-ŋge pɯ-ɕti. \\
  here mid.summer also warm.clothes \textsc{ipfv}-2-wear[III] \textsc{pst}.\textsc{ipfv}-be.\textsc{affirm} \\
  \glt `Here you were wearing warm clothes even in mid summer.' (conversation, 141017)
    \end{exe}
    
    \begin{exe}
\ex \label{ex:nWtCu.kWnA}
\gll    tɕe nɯtɕu kɯnɤ ɯ-jaʁ ɯ-ntsi tɤɲi pjɯ-sɤtse, ɯ-jaʁ ɯ-ntsi kɯ tsʰitsuku ɲɯ-z-nɤme qʰe, \\
\textsc{lnk} \textsc{dem}:\textsc{loc} also \textsc{3sg}.\textsc{poss}-hand \textsc{3sg}.\textsc{poss}-one.of.a.pair erg various.things \textsc{ipfv}-\textsc{caus}-do[III] \textsc{lnk}  \\
\glt `Even like that (despite the pain in her legs), she props herself with a cane using one hand, and does all kinds of things with her other hand.' (14-tApitaRi, 52)
\end{exe}

Although \japhug{kɯnɤ}{also, even} can be combined with most postpositions and relator nouns as shown by the examples above, it is however incompatible with the ergative \forme{kɯ}. For instance, in  (\ref{ex:nWra.kWnA}), although the demonstrative pronoun \forme{nɯra} `they, those' in the second clause is the subject of the transitive verb \japhug{ndza}{eat}, it does not take the ergative \forme{kɯ} as would be expected (§ \ref{sec:A.kW}). The same applies to \forme{ɯ-zda ra} `his companions', subject of the transitive verb \forme{na-nɯ-ɕar-nɯ} `they looked for themselves' in (\ref{ex:Wzda.ra.kWnA}), 

  \begin{exe}
\ex \label{ex:nWra.kWnA}
\gll ɯ-pɯ nɯra li ju-ɣi-nɯ qʰe, nɯra kɯnɤ ɣɯ-tu-ndza-nɯ. \\
\textsc{3sg}.\textsc{poss}-young \textsc{dem}:\textsc{pl} again \textsc{ipfv}-come-\textsc{pl} \textsc{lnk} \textsc{dem}:\textsc{pl} also \textsc{cisloc}-\textsc{ipfv}-eat-\textsc{pl} \\
\glt `Its youngs also come and they too eat it.' (20-sWNgi, 59-60)
  \end{exe}
  
    \begin{exe}
\ex \label{ex:Wzda.ra.kWnA}
\gll   ɯ-zda ra kɯnɤ nɯ-rʑaβ tɯka na-nɯ-ɕar-nɯ ɲɯ-ŋu \\
\textsc{3sg}.\textsc{poss}-companion \textsc{pl} also \textsc{3sg}.\textsc{poss}-wife each \textsc{pfv}:3\fl{}3'-\textsc{auto}-search \textsc{sens}-be \\
\glt `His companions also took each a wife for himself (among the women of the island).' (2005Norbzang, 44)
    \end{exe}
    
The combinations $\dagger$\forme{kɯ kɯnɤ} or $\dagger$\forme{kɯnɤ kɯ} are unattested, and not accepted by native speakers. The contrast between absolutive and ergative noun phrases is therefore neutralized in additive or scalar focus with \forme{kɯnɤ}. Note that other focus markers, such as \forme{ri} and \forme{tɕi} (see \ref{ex:tCi.ndze} in § \ref{sec:ri.additive}) differ from \forme{kɯnɤ} in this regard.

Four distinct facts converge to suggest that the first syllable of \forme{kɯnɤ} is historically related to the ergative postposition \forme{kɯ}: (i) the incompatibility of co-occurrence of \forme{kɯnɤ} and \forme{kɯ}; (ii) the stress on the first syllable in \forme{kɯ́nɤ}; (iii) the similar \forme{-nɤ} element in the other scalar focus marker \japhug{cinɤ}{(not) even one} (§ \ref{sec:cinA}) (iv) the existence of the linker \forme{nɤ}, possibly of Tibetan origin (§ XXX). A detailed examination of this topic is however impossible on the basis Japhug-internal evidence, and will require extensive syntactic comparison between Gyalrong languages.

 \subsubsection{Correlative additive focus markers \forme{ri} and \forme{tɕi}} \label{sec:ri.additive} 
 The additive focus markers \forme{ri} and \forme{tɕi}  are used in enumerations, repeated after each noun referring to  members of a group, to focus on the fact that their referents share a common property (or properties that are semantically close enough), as in (\ref{ex:ri.kWsthWci.WWmpCar}) and (\ref{ex:tCi.tulhoR.cha}) (see additional examples in \citealt[313-314]{jacques14linking}).
 
 \begin{exe}
\ex \label{ex:ri.kWsthWci.WWmpCar}
 \gll  a-rʑaβ ri kɯstʰɯci ɲɯ-mpɕɤr, a-mbro ri kɯstʰɯci ɲɯ-ʑru, a-pɣɤtɕɯ ri kɯstʰɯci ɲɯ-mpɕɤr tɕe, \\
 \textsc{1sg}.\textsc{poss}-wife also so.much \textsc{sens}-be.beautiful  \textsc{1sg}.\textsc{poss}-horse also so.much \textsc{sens}-be.strong  \textsc{1sg}.\textsc{poss}-bird also so.much \textsc{sens}-be.beautiful \textsc{lnk} \\
 \glt `My wife is so beautiful, my horse so strong, my bird so beautiful.' (2003qachga, 116)
 \end{exe}
 
  \begin{exe}
\ex \label{ex:tCi.tulhoR.cha}
 \gll  ɴqiaβ tɕi tu-ɬoʁ cʰa, zrɯ tɕi tu-ɬoʁ cʰa, \\
 dark.side.of.the.mountain also \textsc{ipfv}-come.out can:\textsc{fact}   sunny.side.of.the.mountain also \textsc{ipfv}-come.out can:\textsc{fact}  \\
 \glt `It can grow in both the dark and the sunny sides of the mountains.' (17-thowum, 14)
  \end{exe}
  
The correlative focus markers \forme{ri} and \forme{tɕi} can occur after any noun phrase or postpositional phrase, including with the ergative  \forme{kɯ} as shown by (\ref{ex:tCi.ndze}), unlike the marker \japhug{kɯnɤ}{even, also} (see examples \ref{ex:nWra.kWnA} and \ref{ex:Wzda.ra.kWnA}, § \ref{sec:kWnA}).
  
  \begin{exe}
\ex \label{ex:tCi.ndze}
 \gll paʁ kɯ tɕi ndze, nɯŋa kɯ tɕi ndze, jla kɯ tɕi ndze.   \\
 pig \textsc{erg} also eat[III]:\textsc{fact}  cow \textsc{erg} also eat[III]:\textsc{fact}  hybrid.yak \textsc{erg} also eat[III]:\textsc{fact}  \\
 \glt `Pigs eat it, cows eat it, hybrid yaks eat it.' (18-NGolo, 171)
  \end{exe}

The focus markers \forme{ri} and \forme{tɕi} can have scope on only part of the noun/propositional phrase, and even on the relator nouns as in (\ref{ex:WNgW.tCi}).

   \begin{exe}
\ex \label{ex:WNgW.tCi}
 \gll   sɤtɕʰa ɯ-ŋgɯ tɕi ɣɤʑu, sɤtɕʰa ɯ-taʁ tɕi ʑo ɣɤʑu \\
 ground \textsc{3sg}.\textsc{poss}-inside also exist:\textsc{sens}  ground \textsc{3sg}.\textsc{poss}-inside also \textsc{emph} exist:\textsc{sens} \\
 \glt `It is found both inside the ground, and on the ground.' (25-GdAso, 17)
    \end{exe}
    
Alternatively, it is possible to enumerate distinct related properties of the same referent using \forme{ri} (this usage is not found with \forme{tɕi}), but that marker still follows the noun phrase (correlative \forme{ri} can follow verbs, but only in a specific construction, see \ref{ex:ri.kWmWm.ri} below). In this case the referent cannot be elided, and must be repeated in both clauses, at least as a third person pronoun \forme{ɯʑo} as in (\ref{ex:WlWz.ri.pjAxtCi}). 

  \begin{exe}
\ex \label{ex:WlWz.ri.pjArZi}
 \gll pʰaʁrgot nɯnɯ ɯʑo ri pjɤ-rʑi, ɯʑo ri pjɤ-tsʰu tɕe \\
 boar \textsc{dem} \textsc{3sg} also \textsc{ifr}.\textsc{ipfv}-be.heavy \textsc{3sg} also \textsc{ifr}.\textsc{ipfv}-be.fat \textsc{lnk} \\ 
\glt  `The boar, it was heavy and fat.' (140428 yonggan de xiaocaifeng-zh, 244)
 \end{exe}

A variant of this construction is found with internally-headed relative clauses in apposition, taking the third person pronoun \forme{ɯʑo} as head, as in (\ref{ex:WZo.ri.kWwxti}).

\begin{exe}
\ex \label{ex:WZo.ri.kWwxti}
\gll  [ɯʑo ri kɯ-wxti], [ɯʑo ri kɯ-sɤjlɯ\redp{}jloʁ] ci pjɤ-ŋu. \\
\textsc{3sg} also \textsc{nmlz}:S/A-be.big \textsc{3sg} also \textsc{nmlz}:S/A-\textsc{emph}\redp{}be.big \textsc{indef} \textsc{ifr}.\textsc{ipfv}-be \\
\glt `(The toad) was a big and disgusting (creature).' (150818 muzhi guniang, 86)
\end{exe}

 
The correlative construction can involve the possessor of an IPN, as in (\ref{ex:WlWz.ri.pjAxtCi}), where in the first clause the referent `the girl' is possessor of the intransitive subject (literally `her age was small', § XXX) and in second it corresponds to the intransitive subject, realized as a third person pronoun \forme{ɯʑo} `she'.

  \begin{exe}
\ex \label{ex:WlWz.ri.pjAxtCi}
 \gll tɕʰeme nɯ ɯ-lɯz ri pjɤ-xtɕi, ɯʑo ri pjɤ-mpɕɤr,  \\
 girl \textsc{dem} \textsc{3sg}.\textsc{poss}-age also \textsc{ifr}.\textsc{ipfv}-be.small \textsc{3sg} also \textsc{ifr}.\textsc{ipfv}-be.beautiful \\
\glt `The girl was young and beautiful.' (150909 hua pi-zh, 10)
 \end{exe}
 
 More complex correlations, involving different subjects and predicates related to another referent, are also possible as shown by example (\ref{ex:lWlu.kW}), where \forme{ri} occurs after the intransitive subject \japhug{tɯ-ci}{water}, after the transitive subject \japhug{lɯlu}{cat} with the ergative and after the finite verb \japhug{tu-ɕe}{it goes up} (on which see below and refer to § XXX).
 
 \begin{exe}
\ex   \label{ex:lWlu.kW}
\gll <yancong> ku-kɯ-rɤloʁ tɕe ɯ-taʁ tɯ-ci ri mɯ́j-ɣi lɯlu kɯ ri mɯ-ɲɯ́-wɣ-ɕaβ qapri tu-ɕe ri mɯ́j-cʰa tɕe \\
 chimney \textsc{ipfv}-\textsc{genr}:S/P-make.a.nest \textsc{lnk} \textsc{3sg}.\textsc{poss}-on \textsc{indef}.\textsc{poss}-water also \textsc{neg}:\textsc{sens}-come cat \textsc{erg} also \textsc{neg}-\textsc{ipfv}-\textsc{inv}-catch snake \textsc{ipfv}:\textsc{up}-go also \textsc{neg}:\textsc{sens}-can \textsc{lnk} \\
 \glt `(The sparrows) make their nest in the chimney, (because) water cannot come up there, the cats cannot catch them, and the snakes cannot go up there.' (22-kumpGatCW, 69)
 \end{exe}
 
 The marker \forme{ri} is homophonous with the locative \forme{ri} (§ \ref{sec:locative}), and in cases with an enumeration of locative adjuncts, there can be ambiguity between the two. In (\ref{ex:Xcha.ri.ci}), \forme{ri} is analyzed as a locative because of the position of the determiner \forme{ci}, and also because it can be replaced with other locative postpositions.
 
 \begin{exe}
\ex \label{ex:Xcha.ri.ci}
\gll   χcʰa ri ci, ɯ-ʁe ri ci ɯ-jme cʰɯ-ɬoʁ ɲɯ-ŋu. \\
right \textsc{loc} one  \textsc{3sg}.\textsc{poss}-left \textsc{loc} one \textsc{3sg}.\textsc{poss}-tail \textsc{ipfv}:\textsc{downstream}-come.out \textsc{sens}-be \\
\glt `It has one tail on the right, and one on the left.' (26-qro, 116)
\end{exe}

The marker \forme{ri} can follow verbs only if combined with an existential verb, a copula or a modal auxiliary verb as main predicate (meaning `both $X$ and $Y$' with positive copulas, and `neither $X$ nor $Y$' with negative ones). In this type of construction, verbs are mostly in non-finite form, as in (\ref{ex:ri.kWmWm.ri}). Examples with finite verbs however do exist; this topic is treated in § XXX. %ɲɯ-ɣɤwu ri kɯ-maʁ, ɲɯ-nɤre ri kɯ-maʁ kɯ-fse ɲɤ-k-ɤβzu-ci  ; tu-rɯɕmi ri mɤ-kɯ-khɯ, chɯ-nɯrɤɣo ri mɤ-kɯ-khɯ ci ɲɤ-k-ɤβzu-ci. ; tu-ndzur ri pjɤ-maʁ, ku-omdzɯ ri pjɤ-maʁ.

 \begin{exe}
\ex \label{ex:ri.kWmWm.ri}
 \gll   nɯ pɯ́-wɣ-ta ri  kɯroz kɯ-mɯm ri maŋe, kɯroz mɤ-kɯ-ɣɤ-mɲɤt ri maŋe qʰe, \\
 \textsc{dem} \textsc{pfv}-\textsc{inv}-put \textsc{lnk} specially \textsc{nmlz}:S/A-be.tasty also not.exist:\textsc{sens} specially \textsc{neg}-\textsc{nmlz}:S/A-\textsc{facil}-be.spoiled also not.exist:\textsc{sens} \textsc{lnk} \\
 \glt `When if one puts (a seal on the bread), there is nothing especially tasty about it, and nothing special concerning the preservation (of the bread).' (160706 thotsi, 27)
  \end{exe}
  

  
 \subsubsection{Scalar focus marker \forme{cinɤ}} \label{sec:cinA} 
 The focus marker \japhug{cinɤ}{(not) even one} exclusively occurs with a negative verb. Like \japhug{kɯnɤ}{also, even}, this marker has stress on the first syllable \forme{cínɤ}, which is obviously related to the numeral \japhug{ci}{one} (§ \ref{sec:one.to.ten}, § \ref{sec:indef.article}).
 
 The marker \forme{cinɤ} has scope over the constituent that immediately precedes it, generally a noun phrase including or consisting of a CN, as in (\ref{ex:tWrdoR.cinA3}), but also object and subject participial relative clauses as in (\ref{ex:zrWG.kAmto.cinA}), (\ref{ex:WrNa.WkWru.cinA}) and (\ref{ex:lukWpGaR.nW.cinA}).
 
 \begin{exe}
\ex \label{ex:tWrdoR.cinA3}
\gll tsuku kɯ qʰe tɯ-rdoʁ cinɤ mɤ-kɯ-mto tu. \\
some erg lnk one-piece even neg-nmlz:S/A-see exist:fact \\
\glt `There are some people who (cannot) even find a single one.' (20-grWBgrWB, 36)
 \end{exe} 

 \begin{exe}
\ex \label{ex:zrWG.kAmto.cinA}
\gll  ma tɕe jinde nɯ zrɯɣ kɤ-mto cinɤ maŋe. \\
\textsc{lnk} \textsc{lnk} nowadays \textsc{dem} louse \textsc{nmlz:P}-see even not.exist:\textsc{sens} \\
\glt `Nowadays there isn't even a single louse to be seen/one cannot even see a single louse.' (21-mdzadi, 77)
\end{exe} 

\begin{exe}
\ex \label{ex:WrNa.WkWru.cinA}
\gll ɯ-rŋa ɯ-kɯ-ru cinɤ ʑo pjɤ-me \\
3sg.poss-face 3sg.poss-nmlz:S/A-look even \textsc{emph} \textsc{ipfv}.\textsc{ifr}-not.exist \\
\glt `Not even one (of the thieves) looked at it/The (thieves) did not even so much as looked at it.' (140426 luozi he qiangdao)
\end{exe}

\begin{exe}
\ex \label{ex:lukWpGaR.nW.cinA}
\gll tɕe ɯ-ɲɯ-kɯ-ɣɤ-rkɯn nɯ ɲɯ-dɤn ma lu-kɯ-pɣaʁ nɯ tɯ-rdoʁ cinɤ ʑo maŋe \\
\textsc{lnk} \textsc{3sg}.\textsc{poss}-\textsc{ipfv}-\textsc{nmlz}:S/A-\textsc{caus}-be.few \textsc{dem} \textsc{sens}-be.many \textsc{lnk} \textsc{ipfv}:\textsc{upstream}-\textsc{nmlz}:S/A-plough \textsc{dem} one-piece even \textsc{emph} not.exist:\textsc{sens} \\
\glt `A lot of people diminish their fields, and not a single of them opens new fields.' (150903 friche, 6)
\end{exe}

In the case of relative clauses before \forme{cinɤ}, there is some ambiguity as to whether the scope of the focus marker is on the head of the relative or on the main verb of the relative clause, hence the two proposed translations above for (\ref{ex:zrWG.kAmto.cinA}) and (\ref{ex:WrNa.WkWru.cinA}).

It is not possible to use \forme{cinɤ} with scope over transitive subjects, followed by the ergative.

The form \forme{cinɤ} also occurs in the expression \forme{ŋu cinɤ maʁ kɯ} `in any case it is not', as in (\ref{ex:Nu.cinA.maR.kW}), literally `It is not even the case that...' ; in this construction, only the first verb \japhug{ŋu}{be} receives person indexation, as shown by (\ref{ex:Nua.cinA.maR.kW}). In addition to \japhug{ŋu}{be}, a few other verbs such as \japhug{fse}{be like} can occur with \forme{ci nɤ maʁ kɯ} `anyway X does not' .

 \begin{exe}
\ex \label{ex:Nu.cinA.maR.kW}
\gll qajdo kɯ tɕʰi mɤ-nɯ-ti ɕti nɤ, a-tɤ-nɯ-ti ma ŋu cinɤ maʁ kɯ, nɯ sɤznɤ kɯ-scɯ-scit rɤʑi-tɕi \\
crow \textsc{erg} what \textsc{neg}-\textsc{auto}-say:\textsc{fact} be.\textsc{affirm}:\textsc{fact} \textsc{lnk} \textsc{irr}-\textsc{pfv}-\textsc{auto}-say \textsc{lnk} be:\textsc{fact} even not.be:\textsc{fact} \textsc{sfp} \textsc{dem} \textsc{comp} \textsc{nmlz}:S/A-\textsc{emph}\redp{}happy stay:\textsc{fact}-\textsc{1du} \\
\glt `What would not a crow say (a crow tells only lies), let it say as it wants, in any case it is not (true), let us rather live (together) happily.' (28-qAjdoskAt, 28)
\end{exe} 

 \begin{exe}
\ex \label{ex:Nua.cinA.maR.kW}
\gll  kɯ-mɯrkɯ ŋu-a cinɤ maʁ kɯ  \\
\textsc{nmlz}:S/A-steal be:\textsc{fact}-\textsc{1sg} even not.be \textsc{sfp} \\
\glt `Anyway it is not me who is the thief.' (elicited)
\end{exe}

\subsubsection{Restrictive focus} \label{sec:restrictive.focus} 
 The most common way to express restrictive focus in Japhug is to combine the exceptive \japhug{ma}{apart from} (and its reduplicated variant \forme{mɯma} § \ref{sec:exceptive}) with a negative predicate. This can be a verb with a negative prefix as in (\ref{ex:XsArZaR}), or a negative existential verb as in (\ref{ex:Wmi.Wntsi.ma.me}).
 
 \begin{exe}
\ex  \label{ex:XsArZaR}
\gll   χsɤ-rʑaʁ ma mɯ-pɯ-tsu-a ɲɤ-sɯso ri χsɯ-xpa pjɤ-tsu tɕe,  \\
three-day apart.from \textsc{neg}-\textsc{pfv}-pass-\textsc{1sg} \textsc{ifr}-think \textsc{lnk} three-year \textsc{ifr}-pass \textsc{lnk} \\
\glt `He thought that he had spent only three days, but three years had passed.' (2011-4-smanmi, 178)
  \end{exe}
  
  \begin{exe}
\ex  \label{ex:Wmi.Wntsi.ma.me}
\gll  rkoŋɟɤl nɯnɯ, ɯ-mi ɯ-ntsi nɯ ma me kʰi.   \\
one.legged.demon \textsc{dem} \textsc{3sg}.\textsc{poss}-leg \textsc{3sg}.\textsc{poss}-one.of.a.pair \textsc{dem} apart.from not.exist:\textsc{fact} \textsc{hearsay} \\
\glt  `It is said that one-legged demons only had one leg.' (140510 rkoNJAl, 4)
  \end{exe}
  
The restrictive focus construction implies the presence of a noun phrase with a numeral or a CN when the restriction bears on the quantity, but restriction can also be qualitative, without quantifier, as in (\ref{ex:karGi.Zo.kWfse.ma.me}).

\begin{exe}
\ex \label{ex:karGi.Zo.kWfse.ma.me}
 \gll   ɯ-mat nɯnɯ na-lɤt ɕɯmɯma nɤ kɯ-ndɯ\redp{}ndɯβ ʑo ma me, karɣi ʑo kɯ-fse ma me  \\
 \textsc{3sg}.\textsc{poss}-fruit \textsc{dem} \textsc{pfv}:3\fl{}3'-throw just \textsc{lnk}  \textsc{nmlz}:S/A-\textsc{emph}\redp{}small \textsc{emph} apart.from not.exist:\textsc{fact} turnip.seed \textsc{emph} \textsc{nmlz}:S/A-be.like apart.from not.exist:\textsc{fact} \\
 \glt  `When the fruit of (xanthoxyllum) has just come out, there is only something very small, only like a turnip seed.'  (07-tCGom, 7)
  \end{exe}
  
The restrictive focus construction can be combined with a scalar focus in \forme{kɯnɤ} (see §  \ref{sec:kWnA}), as in (\ref{ex:ma.kWme.kWnA}). In this example, \forme{kɯnɤ} has scope over the subordinate clause \forme{stɯsti ma kɯ-me}, which is ambiguous between a participial headless relative (§ XXX) `consisting of only a female all alone' and a manner infinitival clause (§ XXX; in this case the gloss of \forme{kɯ-me} would be \textsc{inf}:\textsc{stat}-not.exist) `even (when) there is only a female all alone'.

  \begin{exe}
\ex \label{ex:ma.kWme.kWnA}
\gll  mu ma, stɯsti ma kɯ-me kɯnɤ cʰɯ-rɤŋgɯm ɲɯ-ɕti. \\
female apart.from alone apart.from \textsc{nmlz}:S/A-not.exist also \textsc{ipfv}-lay.eggs \textsc{sens}-be.\textsc{affirm} \\
\glt `Even only a female (hen) alone does lay eggs.' (150819 kumpGa, 11)
\end{exe}
   
A second possibility to express restrictive focus is the use of the adverb \japhug{ʁɟa}{completely, all} (§ XXX) with scope on a  noun phrase rather than the whole clause as in (\ref{ex:RJa.tunWndze}).\footnote{The form \forme{ʁɟa} possibly originates from the first syllable of Tibetan \tibet{གཡའ་མ་}{gja.ma}{stone slab}, through a meaning `bare rock'.}  

\begin{exe}
\ex \label{ex:stAmku.RJa}
\gll alo mbroχpa ra tɕe tɕe nɤki qra cʰo qambrɯ ra ɣɯ nɯ-ɣli nɯnɯ
tɕe nɯ tu-wum-nɯ, tu-sɯɣ-rom-nɯ mbroχpa sɤtɕʰa tɕe stɤmku ʁɟa ɲɯ-ɕti ma si maŋe tɕe tɕe    \\
upstream nomad \textsc{pl} \textsc{lnk} \textsc{lnk} \textsc{filler} female.yak \textsc{comit} male.yak \textsc{pl} \textsc{gen} \textsc{3pl}.\textsc{poss}-dung \textsc{dem} \textsc{lnk} \textsc{dem} \textsc{ipfv}-gather-\textsc{pl} \textsc{ipfv}-\textsc{caus}-be.dry-\textsc{pl} nomad place \textsc{lnk} grassland completely \textsc{sens}-be.\textsc{affirm} \textsc{lnk} tree not.exist:\textsc{sens} \textsc{lnk} \textsc{lnk}  \\
\glt `Upstream, in the nomad areas, they gather and dry yak dung, as in nomad places there is only grassland, there no trees.' (05-tamar, 7-10)
\end{exe}

The adverb \forme{ʁɟa} (here used rather as a noun modifier) is related to the denominal verb \japhug{aʁɟa}{be bald, be bare} (see § XXX on the \forme{a-} derivation), which can be applied to nouns such as \japhug{stɤmku}{grassland} and \japhug{zgo}{mountain}.
 
\begin{exe}
\ex \label{ex:RJa.tunWndze}
 \gll qajɯ ʁɟa tu-nɯ-ndze, ma nɯ ma tɤ-rɤku kɯ-fse ra ndze mɤ-ŋgrɤl. \\
 bug completely \textsc{ipfv}-\textsc{auto}-eat[III] \textsc{lnk} \textsc{dem} apart.from \textsc{indef}.\textsc{poss}-harvest \textsc{nmlz}:S/A-be.like \textsc{pl} eat[III]:\textsc{fact} \textsc{neg}-be.usually.the.case:\textsc{fact} \\ 
\glt `It only eats insects, it does not eat cultivated plants.' (140511 qamtsWrmdzu, 16)
\end{exe}

While in (\ref{ex:RJa.tunWndze})  and (\ref{ex:stAmku.RJa}) it remains ambiguous whether \forme{ʁɟa} forms a syntactic constituent with the previous nouns or the following verb, in (\ref{ex:RJa.kW}) the presence of the ergative makes it clear that \forme{ʁɟa} is not a clausal adverb, and belongs to the postpositional phrase headed by \forme{kɯ}.

\begin{exe}
\ex \label{ex:RJa.kW}
 \gll [tɤ-lu cʰo tɯkrimgo ʁɟa kɯ] cʰɯ-z-ɣɤ-wxti-nɯ. \\
 \textsc{indef}.\textsc{poss}-milk \textsc{comit} doughnut completely \textsc{erg} \textsc{ipfv}-\textsc{caus}-\textsc{caus}-be.big-\textsc{pl} \\
\glt `They (used to) raise up (the babies) by feeding them milk and doughnuts only.' (140426 tApAtso kAnWBdaR, 102)
\end{exe}

The same applies to (\ref{ex:Wru.RJa.nW}), where the presence of the demonstrative \forme{nɯ} after \forme{ʁɟa} shows that it belongs to the same noun phrase.

\begin{exe}
\ex \label{ex:Wru.RJa.nW}
 \gll ɯ-rdoʁ nɯ-me tɕe, [ɯ-ru ʁɟa nɯ], pɯ-kɤ-tɤβ nɯnɯ, taʁndzɤr ɯ-ŋgɯ tú-wɣ-rku tɕe, \\
 \textsc{3sg}.\textsc{poss}-grain \textsc{pfv}-not.exist \textsc{lnk} \textsc{3sg}.\textsc{poss}-stalk completely \textsc{dem} \textsc{pfv}-\textsc{nmlz}:P-thresh \textsc{dem} feeding.emmer \textsc{3sg}.\textsc{poss}-inside \textsc{ipfv}-\textsc{inv}-put.in \textsc{lnk} \\
 \glt `When all the grains have been removed, the bare stalks, the one that have been threshed, one puts them in a feeding emmer.' (140513 tWrtsi, 5)
\end{exe}

A reduplicated emphatic form \forme{ʁɟɯ\redp{}ʁɟa} is also found as in (\ref{ex:RJWRJa.kW})

\begin{exe}
\ex \label{ex:RJWRJa.kW}
 \gll χtɕɤnzɤn ʁɟɯ\redp{}ʁɟa kɯ ʑo pɯ́-wɣ-nɤjo ɕti ɲɯ-ŋu.  \\
beast \textsc{emph}\redp{}completely \textsc{erg} \textsc{emph} \textsc{pst}.\textsc{ipfv}-\textsc{inv}-wait be.\textsc{affirm}:\textsc{fact} \textsc{sens}-be \\
\glt `It was all wild beasts waiting for him (there).' (Norbzang2005, 308)
 \end{exe}
 
A third option to express restrictive focus is the IPN \forme{ɯ-jlu}, which is used in the meaning `uncooked' as a property IPN (§ \ref{sec:property.nouns}), but has become grammaticalized as a restrictive marker `exclusively, without anything else' (presumably from an intermediate meaning `plain'), as in (\ref{ex:Wjlu.Zo}).

\begin{exe}
\ex \label{ex:Wjlu.Zo}
 \gll srɤz nɯ kɯ tɕʰoz ɯ-jlu ʑo pjɯ-nɯjɤntɤn pɯ-ɕti ma jɯm nɯ mɯ-pjɤ-ɕar ɲɯ-ŋu, \\
prince \textsc{dem} \textsc{erg}  religion \textsc{3sg}.\textsc{poss}-exclusively \textsc{emph} \textsc{ipfv}-be.assiduous.in  \textsc{pst}.\textsc{ipfv}-be.\textsc{affirm} \textsc{lnk} wife \textsc{dem} \textsc{neg}-\textsc{ifr}.\textsc{ipfv}-look.for \textsc{sens}-be \\
 \glt `The prince was focused exclusively in the study of religion, and was not looking for a wife.' (sras2003, 3)
 \end{exe}

\subsection{Identity modifiers} \label{sec:identity.modifier}
There is no specific identity modifier `the same' in Japhug. The only way to express this meaning is to use the S-participle of the verb \japhug{naχtɕɯɣ}{be the same} (a denumeral verb of Tibetan origin, § \ref{sec:tibetan.numerals}, see also § \ref{sec:comitative} on the syntax of this stative verb and § XXX on its derivation) in a relative clause, as in (\ref{ex:tArmi.kWnaXtCWG}) (a possessor relative, § XXX). This participle is also used adverbially (see § XXX).

\begin{exe}
\ex \label{ex:tArmi.kWnaXtCWG}
\gll tɤ-rmi kɯ-naχtɕɯɣ pjɤ-dɤn wo kɤmɲɯ, nɤki kɯrɯ ra tɕe. \\
\textsc{indef}.\textsc{poss}-name \textsc{nmlz}:S/A-be.the.same \textsc{ifr}.\textsc{ipfv}-be.many \textsc{sfp} pl.n. \textsc{filler} Tibetan \textsc{pl} \textsc{lnk} \\
\glt `There were many people who had identical names, in Kamnyu, among the Tibetans.' (140522 tshupa, 161)
\end{exe}


There are two prenominal modifiers expressing non-identity in Japhug: \japhug{kɯmaʁ}{other} and the numeral \japhug{ci}{one}, which in prenominal position means `the other one' (in postnominal position, it is used as an indefinite article, see § \ref{sec:indef.article}). Both of these words can also be used as pronouns, though \forme{ci} requires to be combined with the demonstrative \forme{nɯ} in this usage (see § \ref{sec:other.pro}).

The modifier \forme{kɯmaʁ} is prenominal in its meaning `other', as in (\ref{ex:kWmaR.tWrme}). 

\begin{exe}
\ex \label{ex:kWmaR.tWrme}
\gll tɯ-zda nɯ ma kɯmaʁ tɯrme a-pɯ-me tɕe, kʰa ra aʁɤndɯndɤt ɲɯ-ɤ<nɯ>ɣro ɲɯ-ŋu ɲɯ-ti. \\
\textsc{genr}.\textsc{poss}-companion \textsc{dem} apart.from other person \textsc{irr}-\textsc{ipfv}-not.exist \textsc{lnk} house \textsc{pl} everywhere \textsc{ipfv}-<\textsc{auto}>play \textsc{sens}-be \textsc{sens}-say \\
\glt `(Our neighbour) says that if there are no other persons apart from family members, (the monkey) would play everywhere in the house.' (19-GzW2, 10)
\end{exe}

There are apparent examples of \japhug{kɯmaʁ}{other} in postnominal position, as in (\ref{ex:kWmaR.taXtW}) and (\ref{ex:kWmaR.tanWsWBzu}), but in such sentences \forme{kɯmaʁ} is a preverbal adverb, not a noun modifier, with a slightly different meaning `anew'. In (\ref{ex:kWmaR.taXtW}), the usage of \forme{kɯmaʁ} is very similar to its Chinese equivalent \ch{另外}{lìngwài}{other} in the corresponding Chinese sentence \zh{阿兰另外给我买了一部手机}, where the preverbal position of \ch{另外}{lìngwài}{other} clearly shows that it is not a noun modifier. 

\begin{exe}
\ex \label{ex:kWmaR.taXtW}
\gll <alan> kɯ a-<dianhua> kɯmaʁ ta-χtɯ \\
p.n. \textsc{erg} \textsc{1sg}.\textsc{poss}-phone other \textsc{pfv}:3\fl{}3'-buy \\
\glt `Alan bought me a new phone.' (conversation, 17-03-27)
\end{exe}

\begin{exe}
\ex \label{ex:kWmaR.tanWsWBzu}
\gll a-ʁi kɯ kʰa kɯmaʁ ta-nɯ-sɯ-βzu qʰe, \\
\textsc{1sg}.\textsc{poss}-younger.sibling \textsc{erg} house other \textsc{pfv}:3\fl{}3'-\textsc{auto}-\textsc{caus}-make \textsc{lnk} \\
\glt `My brother made himself a new house.' (14-tApitaRi, 304)
\end{exe}

The identity determiner \japhug{kɯmaʁ}{other} is grammaticalized from the S-participle of the verb \japhug{maʁ}{not be}, \forme{kɯ-maʁ} `who/which is not X', which is still widely used, as in (\ref{ex:tChWrtsAm.kWmaR}) and (\ref{ex:sthWci.kWmaR}).


%\begin{exe}
%\ex \label{ex:Wstu.kWmaR}
%\gll ɯ-stu kɯ-maʁ me, kɯki mɤ-kɯ-pe me \\
%\textsc{3sg}.\textsc{poss}-truth \textsc{nmlz}:S/A-not.be not.exist:\textsc{fact} dem.\textsc{prox} \textsc{neg}-\textsc{nmlz}:S/A-be.good not.exist:\textsc{fact} \\
%\glt ` (28-smAnmi, 16)
%\end{exe}

\begin{exe}
\ex \label{ex:tChWrtsAm.kWmaR}
\gll mɤʑɯ [tɕʰɯrtsɤm kɯ-maʁ] nɯnɯ tɕe, tú-wɣ-χtɕi ma nɯ ma kɤ-sqa (mɤ-ra) \\
yet type.of.tsampa \textsc{nmlz}:S/A-not.be \textsc{dem} \textsc{lnk} \textsc{ipfv}-\textsc{inv}-wash \textsc{lnk} \textsc{dem} apart.from \textsc{inf}-boil \textsc{neg}-have.to:\textsc{fact} \\
\glt `The tsampa that is not `chu.rtsam', one needs to wash it, but not to boil it.' (2002tWsqar, 112)
\end{exe}

\begin{exe}
\ex \label{ex:sthWci.kWmaR}
\gll  [ɯ-rkɯ wuma ʑo stʰɯci kɯ-maʁ] nɯtɕu tɤ-ri ci kú-wɣ-lɤt \\
\textsc{3sg}.\textsc{poss}-side really \textsc{emph} so.much \textsc{nmlz}:S/A-not.be \textsc{dem}:\textsc{loc} \textsc{indef}.\textsc{poss}-thread once \textsc{ipfv}-\textsc{inv}-throw \\
\glt `One sews a thread at a place which is not too much on the border (of the patch)'. (12-kAtsxWb, 16)
\end{exe}

The modifier \forme{ci} differs from \forme{kɯmaʁ} in that it is necessarily definite, meaning `the other one', as in (\ref{ex:ci.rWdaR}), where it refers to an animal that it chased by lions, which was previously mentioned in the text.

\begin{exe}
\ex \label{ex:ci.rWdaR}
\gll ʑɯrɯʑɤri qʰe ci rɯdaʁ nɯ dɯxpa ma nɯ-kɤ-ndza ɯ-spa ɲɯ-ɕti qʰe, qʰe pjɯ-ndʐaβ qʰe mɯ-ɲɯ-cʰa qʰe, \\
progressively \textsc{lnk} one animal \textsc{dem} poor.of \textsc{lnk} \textsc{3pl}.\textsc{poss}-\textsc{nmlz}:P-eat \textsc{3sg}.\textsc{poss}-material \textsc{sens}-be.affirm \textsc{lnk} \textsc{lnk} \textsc{ipfv}-\textsc{anticaus}:make.fall \textsc{lnk} \textsc{neg}-\textsc{ipfv}-can \textsc{lnk} \\
\glt `The other animal, poor of him, it is their prey, progressively it falls down and cannot stand it anymore.' (20-sWNgi, 43)
\end{exe}

Interestingly, the determiner \forme{ci} does not have scope over other noun modifiers. For instance, in (\ref{ex:ci.tCheme.kWNAn}), the noun \japhug{tɕʰeme}{woman} occurs with an attributive adjective in participial form \forme{kɯ-ŋɤn} `who is evil' (a relative clause, see § \ref{sec:attributes}), but the meaning is not `the other evil woman' as could have been expected (since the woman who is the subject of the sentence is, by contrast, a kind person), and rather must be `the other woman, the evil one'. There is no pause in the recording that could lead us to suppose that \forme{kɯ-ŋɤn} here is an apposition -- it is rather a postnominal relative.

\begin{exe}
\ex \label{ex:ci.tCheme.kWNAn}
\gll nɤki, tɕʰeme nɯ ɯ-ɕki ɯ-kɯ-sɤja jo-ɕe, ci tɕʰeme kɯ-ŋɤn nɯ ɯ-ɕki. \\
\textsc{filler} women \textsc{dem} \textsc{3sg}-\textsc{dat} \textsc{3sg}.\textsc{poss}-\textsc{nmlz}:S/A-give.back \textsc{ifr}-go one woman \textsc{nmlz}:S/A-be.evil \textsc{dem} \textsc{3sg}-\textsc{dat} \\
\glt `She went to give it back to the woman, the other one, the evil woman.' (140515 jiesu de laoren, 90)
\end{exe}

%a-ʁi kɯnɤ tɯrme kha kɤ-sɯxɕe mɯ́j-khɯ qhe,

%ci qhɤjmbaʁ nɯ kɯ-jaʁ kɯ-fse nɯnɯ 
%mtshalu ɯ-cu tɕe nɤki,
%tɯ-mgo zmɤrɤβ kú-wɣ-nɯ-lɤt sna.
%16-RlWmsWsi
%li ci /ɯt/ ɯ-tɯphu nɯ tɤpu qhɤjmbaʁ tu-ti-nɯ ŋu tɕe,
\subsection{Attributes} \label{sec:attributes}
%\ref{sec:property.nouns}
\subsubsection{Attributive postnominal modifiers} \label{ex:attributive.postnominal}
In addition to the postnominal markers studied above (numeral and number § \ref{sec:number.determiners}, demonstratives § \ref{sec:demonstrative.determiners}, quantifiers § \ref{sec:quantifiers.determiners}, definiteness markers § \ref{sec:indef.article}, topic and focus markers), there are a certain number of nouns that can serve a post-nominal modifiers.

An entire class of such nouns consists of the privative nouns in \forme{-lu} `...less', described in § \ref{sec:privative}.

The word \japhug{wuma}{real, really} from Tibetan \tibet{ངོ་མ་}{ŋo.ma}{real, true} is generally used adverbially as an intensifier, in particular with stative verbs (§ XXX), but also occurs as a postnominal modifier meaning ` real', its original meaning, as in (\ref{ex:lhAndzxi.wuma}).

\begin{exe}
\ex \label{ex:lhAndzxi.wuma}
\gll ɬɤndʐi wuma nɯ nɤʑo ɲɯ-tɯ-ŋu ma aʑo ɬɤndʐi ɲɯ-maʁ-a \\
demon real \textsc{dem} \textsc{2sg} \textsc{sens}-2-be \textsc{lnk} \textsc{1sg} demon \textsc{sens}-not.be-\textsc{1sg} \\
\glt `You are the real demon, not me.' (2002lhandzi, 12)
\end{exe}
%ɯ-qa nɯ qarŋe, tɤtsoʁ wuma nɯ. 

%\japhug{ɕɯŋarɯra}{each better than the other} XXXX
% rɟɤlpu ɕɯŋarɯra kɯ ta-tʰu-nɯ ɕti ri, mɯ-tɤ-nɤla-j ɕti tɕe,
% 2003 qachga, 71

\subsubsection{Attributive prenominal modifiers}

\subsubsection{Participial relatives}
%Postnominal or head-internal? definiteness wuma ʑo ... kɯ-

 %rɟɤlpu nɤrɯβzaŋ nɯ kɯ nɤki stu kɯ-mna tɕheme nɯ ɲɤ-nɯ-ɕar ɲɯ-ŋu
 
 
 %ɯʑo sɤz rɯdaʁ kɯ-xtɕi nɯra tu-ndze
 \section{Noun coordination}
\subsection{Coordination or dependency} \label{sec:coordinator}
The closest thing to a noun coordinator in Japhug is the comitative marker \forme{cʰo}, which is argued to be a postposition in § \ref{sec:comitative}.

\subsection{Bare coordination}

\subsubsection{Enumeration} \label{sec:noun.enumeration}
%mbro, jla, nɯŋa, mbala, tshɤt, qaʑo, paʁ, nɯra nɯtɕu ʁɟa z-ɲɯ́-wɣ-lɤɣ pɯ-ŋu.
%qhe tshɤt qaʑo ra ɣɯ nɯ-ndza nɯra ɲɯ-sna. 

\subsubsection{Noun dyads} \label{sec:dyads}
Noun dyads are a pair of nouns occurring in a fixed order, without intervening linker or postposition, and sharing their number and case markers. A good example is provided by the expression `parents' comprising the kinship terms \japhug{tɤ-mu}{mother} and \japhug{tɤ-wa}{father}, as in (\ref{ex:amu.awa.ni.GW}). Note that while number and case markers are shared by both nouns, each of them takes its own possessive prefix, and both prefixes are coreferent. 

\begin{exe}
\ex \label{ex:amu.awa.ni.GW}
 \gll nɯ a-mu a-wa ni ɣɯ ŋu \\
 \textsc{dem}  \textsc{1sg}.\textsc{poss}-mother \textsc{1sg}.\textsc{poss}-father \textsc{du} \textsc{gen} be:\textsc{fact} \\
 \glt `This is for my parents.' (meimei de gushi)
\end{exe}

The dyad for `parents' has a honorific variant, originally used for noblemen in the traditional society. It comprises the terms \japhug{tɤ-pa}{father} and \japhug{tɤ-ma}{mother}, which are borrowed from Tibetan \tibet{ཨ་ཕ་}{ʔa.pʰa}{father}  and \tibet{ཨ་མ་}{ʔa.ma}{mother} respectively. Interesting, the honorific expression follows the `father-mother' order (as in example \ref{ex:apa.ama}), while the native one puts `mother' in the first place.

\begin{exe}
\ex \label{ex:apa.ama}
 \gll nɯ kɯ-fse a-pa a-ma ni kɯ ɲɯ-ti-ndʑi tɕe \\
 \textsc{dem} \textsc{nmlz}:S/A-be.like \textsc{1sg}.\textsc{poss}-father  \textsc{1sg}.\textsc{poss}-mother \textsc{du} \textsc{erg} \textsc{sens}-say-\textsc{du} \textsc{lnk} \\
 \glt `My parents say this.' (2003nyima2, 94)
\end{exe}

Other common dyads include \forme{rgɤtpu rgɤnmɯ} `old man(men) and woman(women)', \forme{tɤ-tɕɯ tɕʰeme} `boy(s) and girl(s)' with APNs. They are most commonly used as collectives with indefinite referents as in (\ref{ex:tAtCW.tCheme.tWsAmdzW}), but are also attested with definite ones, as in (\ref{ex:rgAtpu.rgAnmW}).

\begin{exe}
\ex \label{ex:tAtCW.tCheme.tWsAmdzW}
 \gll  tɤ-tɕɯ tɕʰeme tɯ-sɤ-ɤmdzɯ ʑaka tu \\
 \textsc{indef}.\textsc{poss}-son girl \textsc{genr}.\textsc{poss}-\textsc{nmlz}:\textsc{oblique}-sit each \textsc{exist}:fact \\
\glt `Gents and ladies each have (different) seating places.' (31-khAjmu, 10)
\end{exe}

\begin{exe}
\ex \label{ex:rgAtpu.rgAnmW}
 \gll rgɤtpu rgɤnmɯ ni kɯ kɯki tɤ-pɤtso χsɯm ki kɤsɯfse ʑo cʰɤ-ɣɤ-wxti-ndʑi. \\
 old.man old.woman \textsc{du} \textsc{erg} \textsc{dem}.\textsc{prox} \textsc{indef}.\textsc{poss}-child three \textsc{dem.prox} all \textsc{emph} \textsc{ifr}-\textsc{caus}-be.big-\textsc{du} \\
\glt `The old man and the old woman raised all these three children.' (140514 huishuohua de niao, 60)
\end{exe}

\section{Apposition}

\section{The structure of the noun phrase}


\section{Nominal predicates}
