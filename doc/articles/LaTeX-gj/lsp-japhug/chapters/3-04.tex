\chapter{The noun phrase}

\section{Postpositions and relator nouns}

\subsection{Absolutive}
\subsubsection{Core argument}
\subsubsection{Essive}
%tsuku kɯ paʁndza ɲɯ-nɯ-phɯt-nɯ ɲɯ-ŋu ri,
\subsubsection{Locative adjunct}

\subsection{Ergative}
\subsubsection{Core argument}
\subsubsection{Instrumental}
\subsubsection{Comparee marker}

\subsection{Genitive}
\subsection{Locative}
\section{Noun modifiers and determiners}
This section discusses all nouns modifiers and determiners except relative clauses (§ XXX) and complement clauses (§ XXX). 

\subsection{Demonstratives}

\subsection{Quantifiers}
\subsubsection{Universal quantifiers} \label{sec:universal.quant}
\subsubsection{Mid-scalar quantifier} \label{sec:tsuku}
(\ref{sec:partitive.pronouns})
\subsection{Indefinite and definite markers} \label{sec:indefinite.markers}
Japhug has no definite article. The demonstrative \forme{nɯ} and topic markers such as \forme{iɕqʰa} (§ \ref{sec:iCqha})

Like many languages (\citealt[130]{creissels06sgit1}), Japhug  uses bare nouns without any definiteness marking. Bare nouns are most often non-referential, as \japhug{tɕʰeme}{girl} in (\ref{ex:tCheme.tWtAtu}).

\begin{exe}
\ex \label{ex:tCheme.tWtAtu}
\gll ʁnaʁna tɕʰeme tɯ\redp{}tɤ-tu nɤ, kɤndʑɯsqʰaj 	tu-kɤ-sɯ-βzu \\
both girl \textsc{cond}\redp{}\textsc{pfv}-exist \textsc{lnk} \textsc{coll}:sister \textsc{ipfv}-\textsc{inf}-\textsc{caus}-make \\
\glt `If both of them have girls, let them be sisters.' (zrAntCW, 4)
\end{exe}

Bare nouns are less common with referential nouns (except in answers to questions), but examples can be found, as \japhug{qacʰɣa}{fox} in (\ref{ex:qachGa.kW}).

\begin{exe}
\ex \label{ex:qachGa.kW}
\gll qacʰɣa 	kɯ 	maχtɕɯ tɤ-tɯt-a nɯ mɤ-tɯ-ste 	ti 	ɲɯ-ŋu  \\
fox \textsc{erg} I.told.you.so \textsc{pfv}-say[II]-\textsc{1sg} \textsc{dem} \textsc{neg}-2-do.like[III]:\textsc{fact} say:\textsc{fact} \textsc{sens}-be \\
\glt `The fox says: `You do not do as I told you to." (2003qachGa, 44)
\end{exe}

Personal names generally occur as bare nouns, without any definiteness marker as in (\ref{ex:WrJAnpanma}), but there are no constraints against co-occurrence of personal names with the determiner \forme{nɯ} either (see § \ref{sec:personal.names.modifiers}).

\begin{exe}
\ex \label{ex:WrJAnpanma}
\gll  ɯrɟɤnpanma kɯ ʁlaŋsaŋtɕhin ɯ-ɕki  \\
 Padmasambhava \textsc{erg} Gesar \textsc{3sg}-\textsc{dat} \\
\glt `Padmasambhava (told) Gesar.' (Gesar, 2)
\end{exe}

The form \japhug{ci}{one} has among its many functions (in addition to pronoun, numeral and adverb, see § \ref{sec:other.pro}, § \ref{sec:partitive.pronouns}, § \ref{sec:identity.modifier}, § \ref{sec:one.to.ten} and § XXX) that of indefinite article, as in (\ref{ex:ci.indef}) and (\ref{ex:ci.chAGi}). It is typically used to introduce a new referent in a story.

\begin{exe}
\ex \label{ex:ci.indef}
\gll tɕʰeme kɯ-mpɕɯ\redp{}mpɕɤr 	ci 	ɲɤ-nɯ-ɬoʁ \\
girl \textsc{nmlz}:S/A-\textsc{emph}\redp{}beautiful \textsc{indef} \textsc{ifr}-\textsc{auto}-come.out \\
\glt `A very beautiful girl appeared (out of it).' (The flood, 39)
\end{exe}

\begin{exe}
\ex \label{ex:ci.chAGi}
\gll tɕɤlo tɕe tɤ-tɕɯ ci cʰɤ-ɣi qʰe, \\
upstream \textsc{lnk} \textsc{indef}.\textsc{poss}-son \textsc{indef} \textsc{ifr}:\textsc{downstream}-come \textsc{lnk} \\
\glt `A boy came from upstream.' (2003-kWBRa, 41)
\end{exe}

Although \forme{ci} can be used as a partitive pronoun `one of them' (§ \ref{sec:partitive.pronouns}), as a postnominal determiner it does not have partitive meaning. To express a meaning such as `one of the boys', a CN such as \japhug{tɯ-rdoʁ}{one piece} is used instead (§ \ref{sec:ICN}). 

Note that when used as a prenominal modifier, \forme{ci} has a completely different (definite) meaning `the other X' (§ \ref{sec:identity.modifier}).

 \subsection{Topic and focus markers} \label{sec:topic}
 
 \subsubsection{Aforementioned topic} \label{sec:iCqha}
 The marker \japhug{iɕqʰa}{the aforementioned} (from the adverb \japhug{iɕqʰa}{just before}, see § XXX)  is used on referents that have been previously mentioned in the same story, usually only a few sentences back.
 
 
%  \begin{exe}
%\ex \label{ex:indef}
%\gll \textbf{``razri} 	\textbf{kɤtɯm} 	\textbf{ci} 	ɲɯ-ra, 	taqaβ 	ci 	ɲɯ-ra" to-ti qhe   \\
% thread ball \textsc{indef} \const{}-need needle \textsc{indef} \const{}-need \evd{}-say \coord{}  \\
%\glt He told (Rgyabza) ``I need a ball of thread and a needle''.
%\ex \label{ex:icqha}
%\gll tɕendɤre 	ɲo-kho 	qhe, 	tɕe 	ɯ-ndzɤtshi 	kɤ-tsɯm 	nɯ 	tɕu 	qhe 	tɕe, \textbf{iɕqʰa} 	\textbf{kɤtɯm} 	\textbf{nɯ }	ɯʑo 	kɯ 	ko-ndo, 	taqaβ-rna 	nɯ 	ɲɤ-rku qhe,  \\
% \coord{} \evd{}-give \coord{} \coord{} \textsc{3sg}.\poss{}-meal \inftv{}-bring \textsc{compl}  \loc{} \coord{} \coord{} the.aforementioned ball \topic{} he \erg{} \evd{}-take needle-ear \topic{} \evd{}put.in \coord{} \\
%\glt She gave it to him. While (people) brought his meal, he took the ball of thread and put it into the ear of the needle. (Gesar 270-2)
%\end{exe}
%The referent ``ball of thread'', first introduced in sentence \ref{ex:indef}, appears again two sentences later with both the topic markers nɯ and \textit{iɕqʰa}. 
%
%
%There seems to be a limit to the number of sentences that can separate a noun phrase in iɕqʰa from its preceding occurrence (probably no more than five-six), but this topic deserves of systematic study based on all available stories.
 
\subsection{Genitival phrases}
%
%tɕendɤre ɯ-jaʁ nɯtɕu ftsoʁ kɯngɯt ɯ-phɯ ɣɯ srɯnloʁ pjɤ-k-ɤrku-ci
%2003gesar, 239
%
%\subsection{Determiners} \label{sec:determiners}
%\japhug{ɕɯŋarɯra}{each better than the other}
% rɟɤlpu ɕɯŋarɯra kɯ ta-tʰu-nɯ ɕti ri, mɯ-tɤ-nɤla-j ɕti tɕe,
% 2003 qachga, 71
\subsection{Identity modifiers} \label{sec:identity.modifier}

%nɤki tɕheme nɯ ɯ-ɕki ɯ-kɯ-sɤja jo-ɕe, ci tɕheme kɯ-ŋɤn nɯ ɯ-ɕki.
\subsection{Attributes}

\section{The structure of the noun phrase}

\section{Nominal predicates}
