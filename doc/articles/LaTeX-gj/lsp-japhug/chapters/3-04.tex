 \chapter{Postpositions and relator nouns} \label{chap:postpositions.relators}

 

\section{Absolutive} \label{sec:absolutive}
\subsection{Intransitive subject}
\subsection{Object}
\subsection{Semi-object}
\subsection{Theme}
\subsection{Essive} \label{sec:essive.abs}
%tsuku kɯ paʁndza ɲɯ-nɯ-phɯt-nɯ ɲɯ-ŋu ri,
\subsection{Goal} \label{absolutive.goal}  
\subsection{Locative adjunct} \label{absolutive.locative}

\section{Postpositions} \label{ex:postpositions}

\subsection{Independent words vs clitics}  \label{ex:word.vs.clitic.postp}  
Since other Gyalrongologists, in particular \citet{jackson98morphology, jackson14morpho} treat the postpositions in related languages as clitics rather than as independent words as is done in the present work, a justification of the present analysis is necessary.

In Japhug, the postpositions \japhug{kɯ}{ergative} (§ \ref{sec:erg.kW}) and  \japhug{ɣɯ}{genitive} (§ \ref{sec:genitive}) do have some clitic-like characteristics: they cannot be used without a preceding noun phrase (or a subordinate clause in some cases, see § XXX), are unstressed, and in the case of the genitive have special irregular forms with pronouns (§ \ref{sec:pronouns.gen}).

However, a pause can occur between these postpositions (\ref{ex:kW.nAmWmnW}) and the noun phrase they follow. For instance, in example (\ref{ex:kW.nAmWmnW}), a two second pause (with an inspiration) is found between the phrase \forme{nɯŋa ra} and the following ergative \forme{kɯ}. 

\begin{exe}
\ex \label{ex:kW.nAmWmnW}
\gll tɕe tɯrtsi nɯ pjɯ́-wɣ-βzu tɕe, nɯŋa ra, kɯ nɤ-mɯm-nɯ cʰo wuma ʑo ɣɯ-ɕɯ-fka-nɯ \\
\textsc{lnk} cow.food \textsc{dem} \textsc{ipfv}-\textsc{inv}-make \textsc{lnk} cow \textsc{pl} \textsc{erg} \textsc{trop}-be.tasty:\textsc{fact}-\textsc{pl} \textsc{comit} really \textsc{emph} \textsc{inv}-\textsc{caus}-be.satiated:\textsc{fact}-\textsc{pl} \\
\glt `They make cow food with flour, the cows find it tasty, and it satisfies their hunger.' (140513 tWrtsi, 15)
\end{exe}

Such cases are by no means exceptional; at least 54+35 examples of ergative and genitive preceded by a pause are attested in the corpus (they can be found by searching \forme{kɯ} or \forme{ɣɯ}  preceded by a comma). Most of these cases are found in sentences where the speaker hesitates, and are especially common in texts translated from Chinese.

The same is true of all postpositions studied in this section. Examples of pause between the noun phrase and the following postposition can be found for most of them, for instance (\ref{ex:nWnWtCu.zW}) for the locative \forme{zɯ} (§ \ref{sec:core.locative}).

\begin{exe}
\ex \label{ex:nWnWtCu.zW}
\gll  <bageda> kɤ-ti nɯnɯtɕu, zɯ, nɤkinɯ, \\
pl.n. \textsc{nmlz}:P-say \textsc{dem}:\textsc{loc} \textsc{loc} \textsc{filler} \\
\glt `In the (place) called Bagdad...' (140515 facaimeng, 2)
\end{exe}

\subsection{Ergative} \label{sec:erg.kW}
\subsubsection{Transitive subject} \label{sec:A.kW}
\subsubsection{Instrumental} \label{sec:instr.kW}

%manner
%kumpɣa cho khɯna ni li tɤ-mqe tɤ-ndɯt kɯ jo-ɣi-ndʑi tɕe,
\subsubsection{Causee} \label{sec:causee.kW}
\subsubsection{Comparee marker} \label{sec:comparee.kW}

%mahi nɯnɯ kɯ aʑo sɤz cha
\subsubsection{Partitive} \label{sec:partitive.kW}
\subsubsection{Oblique argument} \label{sec:oblique.kW}
 The transitive verb \japhug{kʰɤt}{do repeatedly, do a long time} and its causative form \japhug{sɯ-kʰɤt}{cause to do repeatedly, cause to do a long time} occur in a construction with instrumental-like noun phrases marked with the ergative \forme{kɯ}, indicating the action which is performed repeatedly or done over a long time. These noun phrases can include either an action nominal derived from a verb with the prefix \forme{tɯ-} (§ XXX) as in (\ref{ex:tWqioR.kW}), or an underived action noun, as in (\ref{ex:tama.kW.takhAt}) and (\ref{ex:khAcAl.kW.takhata}).  
 
  \begin{exe}
\ex \label{ex:tWqioR.kW}
\gll tɯ-qioʁ kɯ tó-wɣ-sɯ-kʰɤt ʑo tɕe, tɕe nóʁmɯz nɤ tɯɣ nɯnɯ ló-wɣ-sɯ-tɕɤt  \\
\textsc{nmlz:action}-vomit \textsc{erg} \textsc{ifr-inv-caus}-do.a.long.time \textsc{emph} \textsc{lnk} \textsc{lnk} only.then \textsc{lnk} poison \textsc{dem} \textsc{ifr-inv-caus}-take.out \\
\glt `(The medicine) caused (Gesar) to vomit a long time until he expelled the poison.' (Gesar, 266)
\end{exe}

  \begin{exe}
\ex \label{ex:tama.kW.takhAt}
\gll ta-ma kɯ ta-kʰɤt ʑo  \\
\textsc{indef.poss}-work \textsc{erg} \textsc{pfv}:3$\rightarrow$3'-do.a.long.time \textsc{emph} \\
\glt `He did a lot of work.' (elicited)
\end{exe}

Example (\ref{ex:khAcAl.kW.takhata}), with the verb \japhug{kʰɤt}{do repeatedly, do a long time}  taking \textsc{1sg}\fl{}3 indexation (§ XXX), shows that the ergative phrase cannot be analyzed as a transitive subject; moreover, the fact that adding the causative in this case would imply a real causative interpretation (`cause X to repeatedly') also indicates that this phrase is not an instrumental adjunct (see § \ref{sec:instr.kW}).

  \begin{exe}
\ex \label{ex:khAcAl.kW.takhata}
\gll kʰɤcɤl kɯ tɤ-kʰat-a ʑo \\
conversation \textsc{erg} \textsc{pfv}-do.a.long.time-\textsc{1sg} \textsc{emph} \\
\glt `I have a long conversation.' (elicited)
\end{exe}

No other verb takes this type of oblique ergative phrase.

\subsection{Genitive} \label{sec:genitive}
With the exception of particular forms for some pronouns (§ \ref{sec:pronouns.gen}), the genitive postposition has the invariant form \forme{ɣɯ} in Kamnyu Japhug. Like the ergative \forme{kɯ}, it is likely borrowed from the Amdo clitic \forme{-ɣə/-kə} (\citealt[62]{haller04themchen}). It is used in possessive contructions, but also expresses beneficiary and recipient.

\subsubsection{Possession} \label{sec:gen.possession}
The genitive \forme{ɣɯ} occurs in various type of possessive constructions, including genitival noun complements and possessive existential predicates (§ XXX).

Inside the noun phrase, the genitive occurs between possessor and possessum, and a possessive prefix is found on the possessum (§ \ref{ex:prefix.expression.of.possession}), as in (\ref{ex:GZAndza.GW.WjwaR}).  

\begin{exe}
\ex \label{ex:GZAndza.GW.WjwaR}
\gll ri ɣʑɤndza ɣɯ ɯ-jwaʁ nɯra mɤ-wxti ri, ɲaʁ ʑo qhe, \\
\textsc{lnk} Agastache.rugosa \textsc{gen} \textsc{3sg}.\textsc{poss}-leaf \textsc{dem}:\textsc{pl} \textsc{neg}-be.big:\textsc{fact} \textsc{lnk} be.black:\textsc{fact} \textsc{emph} \textsc{lnk} \\
\glt `The leaves of the Agastache rugosa are not large and quite dark in colour.' (11-qarGW, 137)
\end{exe}

Genitival phrases without possessive prefix on the possessum are rare but do exist, in particular when the possessum is a noun borrowed from Chinese and non-fully nativized like \ch{国语}{guóyǔ}{national language} in (\ref{ex:iZo.GW.guoyu}).  

\begin{exe}
\ex \label{ex:iZo.GW.guoyu}
\gll iʑo ɣɯ <guoyu> ɲɯ-ŋu tɕe, nɯnɯ kɤsɯfse ɣɯ ji-rju ɲɯ-ŋu tɕe, \\
\textsc{1pl} \textsc{gen} national.language \textsc{sens}-be \textsc{lnk} dem all \textsc{gen} \textsc{1pl}.\textsc{poss}-speech \textsc{sens}-be \textsc{lnk} \\
\glt `(Chinese) is our national language, it is everybody's language.' (150901 tshuBdWnskAt, 15-16)
\end{exe}

For singular noun possessors, the presence or not of a third person possessive prefix \forme{ɯ-} is not always easy to tell from recordings, as due to the external sandhi (§ XXX), \forme{ɣɯ ɯ-} merges as \ipa{ɣɯ} when no pause occurs between the two. In careful speech, the third person prefix is clearly audible.

Nominal modifiers can sometimes be marked like possessors, with the genitive and/or with a possessive prefix on the following head noun, see § \ref{sec:gen.other}. 

The genitive can also appear between a noun phrase and a relator noun (§ \ref{sec:relator.nouns}), and even be followed by focus markers in this position, as in (\ref{ex:GW.kWnA.WrkW.ri}).

\begin{exe}
\ex \label{ex:GW.kWnA.WrkW.ri}
\gll   tɯ-ci kɯ-wxti ɣɯ kɯnɤ ɯ-rkɯ ri nɯra tu ŋgrɤl.  \\
\textsc{indef}.\textsc{poss}-water \textsc{nmlz}:S/A-be.big \textsc{gen} also \textsc{3sg}.\textsc{poss}-side \textsc{loc} \textsc{dem}:\textsc{pl} exist:\textsc{fact} be.usually.the.case:\textsc{fact} \\
\glt `(Dragonflies) are also found near rivers.' (26-quspunmbro, 7)
\end{exe}

In these constructions, the genitive is always optional, and the prefix on the possessum suffices to express possession, as in (\ref{ex:paXCi.WjwaR}) (see § \ref{ex:prefix.expression.of.possession}).

\begin{exe}
\ex \label{ex:paXCi.WjwaR}
\gll paχɕi ɯ-jwaʁ tsa fse ri, nɯ sɤznɤ artɯm,\\
apple \textsc{3sg}.\textsc{poss}-leaf a.little be.like:fact \textsc{lnk} \textsc{dem} \textsc{comp} be.round:\textsc{fact} \\
\glt `(Its leaves) are a little like the leaves of an apple tree, but more round.' (09-mi, 15)
\end{exe}

When the possessum is elided however, the genitive postposition becomes obligatory, as in (\ref{ex:baigua.GW.sAz}).

\begin{exe}
\ex \label{ex:baigua.GW.sAz}
\gll ɯ-rɣi nɯnɯ, nɤki, <beigua> ɣɯ sɤz ɲɯ-jaʁjɯ. \\
\textsc{3sg}.\textsc{poss}-seed \textsc{dem} \textsc{filler}  pumpkin \textsc{gen} \textsc{comp} \textsc{sens}-be.thick.and.strong \\
\glt `Its seeds are thicker than those of the pumpkin.' (16-CWrNgo, 130)
\end{exe}

While there are transitive and semi-transitive verbs expressing possession (§ XXX), the most common possessive construction involves an existential verb taking the possessum as subject, with the possessor marked by a possessive prefix on the possessum, and optionally with the genitive, as in (\ref{ex:phu.nW.GW.WRrW.GAZu}). 

\begin{exe}
\ex \label{ex:phu.nW.GW.WRrW.GAZu}
\gll qartsʰaz pʰu nɯ ɣɯ ɯ-ʁrɯ ɣɤʑu \\
deer male \textsc{dem} \textsc{gen} \textsc{3sg}.\textsc{poss}-horn exist:\textsc{sens} \\
\glt `The male deer has horns.' (27-qartshAz, 32)
\end{exe}

This construction is also used for abstract possession, as in (\ref{ex:aZWG.aBlu.tu}).

\begin{exe}
\ex \label{ex:aZWG.aBlu.tu}
\gll aʑɯɣ a-βlu tu \\
\textsc{1sg}:\textsc{gen} \textsc{1sg}.\textsc{poss}-stratagem exist:\textsc{fact} \\
\glt `I have an idea.' (140507 tangguowu, 29)
\end{exe}

The causative verbs \japhug{ɣɤtu}{cause to have} and \japhug{ɣɤme}{cause not to have, destroy} derived from \japhug{tu}{exist} and \japhug{me}{not exist} respectively (see § XXX) select an oblique argument with the genitive, as in (\ref{ex:WZo.GW.tuGAtea}). Although this argument could be considered to be a type of beneficiary (§ \ref{sec:other.uses.poss.prefixes}), we observe here stability in case marking of the possessor between the base construction and the derived causative one.

\begin{exe} 
\ex \label{ex:WZo.GW.tuGAtea} 
\gll ɯʑo kɯ maka kɤ-ntɕʰoz mɤ-kɯ-ɤrɕo kɯ-fse ʑo tɯrɟɯ laχtɕʰa ɯʑo ɣɯ tu-ɣɤte-a jɤɣ \\ 
\textsc{3sg}.\textsc{poss} \textsc{erg} at.all \textsc{inf}-use \textsc{neg}-\textsc{inf}:\textsc{stat}-be.finished \textsc{inf}:\textsc{stat}-be.like \textsc{emph} wealth thing \textsc{3sg}.\textsc{poss} \textsc{gen} \textsc{ipfv}-\textsc{caus}-exist[III]-\textsc{1sg} be.possible:\textsc{fact} \\ 
\glt `(If someone saves me), I will make him have more wealth and riches than he can ever use.' (140512 yufu yu mogui, 84) 
\end{exe} 
%ma aʑo a-kɤ-cha,  a-kɤ-cha kɯ-tu nɯra, a-kɤ-spa, tu-βze-a kɤ-cha nɯra lonba ʑo nɤʑɯɣ tɤ-ɣɤtu-t-a ɕti tɕe,

Not all combinations of existential verbs and genitival phrases are existential possessive constructions. For instance, in (\ref{ex:BZW.GW.WqiW}), the second clause could appear to contain a possessive construction meaning `the mouse only has half of it', but the context makes it clear that a different interpretation is necessary (see § XXX on this use of the existential verbs).

\begin{exe}
\ex \label{ex:BZW.GW.WqiW}
\gll qamtɕɯr nɯ ɯ-mtɕʰi nɯnɯ βʑɯ sɤznɤ mɤʑɯ ʑo amtɕoʁ tɕe nɯ βʑɯ ɣɯ ɯ-qiɯ kɯnɤ me \\
shrew \textsc{dem} \textsc{3sg}.\textsc{poss}-mouth \textsc{dem} mouse \textsc{comp} yet \textsc{emph} be.pointy \textsc{lnk} \textsc{dem} mouse \textsc{gen} \textsc{3sg}.\textsc{poss}-half even not.exist:\textsc{fact} \\
\glt `The shrew's mouth is even sharper than that of the mouse, and (its size) is not even half that of the mouse.' (27-spjaNkW, 204-205)
\end{exe}



 
\subsubsection{Recipient and beneficiary} \label{sec:gen.beneficiary}
 
The genitive can be used to mark the recipient by the indirective verb \japhug{kʰo}{give, pass over}, as in (\ref{ex:aZWG.nWkhAm}) and (\ref{ex:GW.anWtWkhAm}).   

\begin{exe}
\ex \label{ex:aZWG.nWkhAm}
 \gll ɕɯ ʑo stu kɯ-mɤku pɯ-tɯ-mto-t nɯnɯ, laχtɕha pɯ-nnɯ-ŋu, tɯrme pɯ-nnɯ-ŋu nɯ, aʑɯɣ nɯ-kʰɤm tɕe tɕendɤre, aʑo ɲɯ-ta-lɤt jɤɣ \\
 who \textsc{emph} most \textsc{nmlz}:S/A-be.first \textsc{pfv}-2-see-\textsc{pst}:\textsc{tr} \textsc{dem}  thing \textsc{pst}.\textsc{ipfv}-\textsc{auto}-be   person \textsc{pst}.\textsc{ipfv}-\textsc{auto}-be \textsc{dem} \textsc{1sg}:\textsc{gen} \textsc{imp}-give[III] \textsc{lnk} \textsc{lnk} \textsc{1sg} \textsc{ipfv}-1\fl{}2-release be.possible:\textsc{fact} \\
 \glt `Give me the first thing you see (when you go back home), be it a person or an object, and I will release you.' (140506 shizi he huichang de bailingniao-zh, 50-52)
\end{exe}

\begin{exe}
\ex \label{ex:GW.anWtWkhAm}
 \gll jɤ-tsɯm tɕe iɕqʰa nɯ kɯβʁa nɯ ɣɯ a-nɯ-tɯ-kʰɤm \\
 \textsc{imp}-take.away \textsc{lnk} the.aforementioned \textsc{dem} noble \textsc{dem} \textsc{gen} \textsc{irr}-\textsc{pfv}-2-give[III] \\
 \glt  Take it and give it to the nobleman.' (150831 renshen wawa-zh, 43)
\end{exe}
 
The recipient of the verb  \japhug{kʰo}{give, pass over} can alternatively also be marked by a possessive prefix on the IPN \japhug{tɯ-jaʁ} (with the meaning  hand over', \ref{sec:semi.grammaticalized.relator}) or, with the dative relator nouns \forme{ɯ-ɕki} or \forme{ɯ-pʰe} (§ \ref{sec:dative}).

The genitive is selected by a few intransitive modal verbs to indicate the experiencer/beneficiary, in particular  \japhug{ra}{need, have to}, \japhug{ʁzi}{be necessary}, as in (\ref{ex:aZWG.WCArW}) and (\ref{ex:aZWG.Rzi}).

\begin{exe}
\ex \label{ex:aZWG.WCArW}
 \gll aʑɯɣ ɯ-ɕɤrɯ ra \\
 \textsc{1sg:gen} \textsc{3sg.poss}-bone have.to:\textsc{fact} \\
\glt `I want its bones.' (07-deluge, 9)
\end{exe}

\begin{exe}
\ex \label{ex:aZWG.Rzi}
 \gll aʑɯɣ wuma ʑo ʁzi ɲɯ-ŋu, a-kɤ-ntɕʰoz sna ɲɯ-ŋu \\
  \textsc{1sg:gen} really \textsc{emph} be.necessary:\textsc{fact} \textsc{sens}-be \textsc{1sg}.\textsc{poss}-nmlz:P-use be.good:\textsc{fact}  \textsc{sens}-be \\
  \glt `It will be useful for me, it will have good use of it.'  (150902 hailibu-zh, 44-45)
\end{exe}

The experiencer/beneficiary can also be marked by possessive prefixes on the subject, without genitive, as in 
(\ref{ex:arNWl.mAra}) (see also § \ref{sec:other.uses.poss.prefixes} for additional examples).

\begin{exe}
\ex \label{ex:arNWl.mAra}
 \gll aʑo a-rŋɯl a-χsɤr ra mɤ-ra \\
 \textsc{1sg} \textsc{1sg}.\textsc{poss}-silver \textsc{1sg}.\textsc{poss}-gold \textsc{pl} \textsc{neg}-have.to:\textsc{fact} \\
 \glt `I don't  need silver or gold.' (2014-kWlAG, 367)
\end{exe}

The genitive also occurs with beneficiaries/maleficiaries as adjuncts, not selected by the main verb, with transitive verbs such as \japhug{nɤma}{do} (\ref{ex:tChi.tunAmea}) and \japhug{wum}{collect} (\ref{ex:WZAG.pjAmaR}) or stative intransitive verbs such as \japhug{pe}{be good} as in (\ref{ex:aZWG.mApe}).

\begin{exe}
\ex \label{ex:tChi.tunAmea}
\gll nɤʑɯɣ tɕʰi tu-nɤme-a ra, tɤ-ti  \\
\textsc{2sg}:\textsc{gen} what \textsc{ipfv}-do[III]-\textsc{1sg} have.to:\textsc{fact} \textsc{imp}-say \\
\glt `Tell me what I shall do for you.' (140511 alading-zh, 175)
\end{exe}

\begin{exe}
\ex \label{ex:aZWG.mApe}
\gll  ɯ-fso tʰɯ-wxti tɕe aʑɯɣ mɤ-pe \\ 
\textsc{3sg}.\textsc{poss}-tomorrow \textsc{pfv}-be.big \textsc{lnk} \textsc{1sg}:\textsc{gen} \textsc{neg}-be.good:\textsc{fact} \\
\glt `In the future, when he will have grown up, he will cause me trouble.' (`he will not be good to me', 2011-05-nyima, 22)
\end{exe}

The beneficiary adjunct is not necessarily contiguous with the verb on which it depends, as in (\ref{ex:iZora.GW.tChi.tufsej}) where the genitive phrase \forme{iʑora ɣɯ} `for us, on our behalf'  is separated from the verb \japhug{tʰu}{ask} by a lengthy complement comprising two clauses.

\begin{exe}
\ex \label{ex:iZora.GW.tChi.tufsej}
\gll  iʑora ɣɯ [tɕʰi tu-fse-j tɕe ji-tɯ-ci ɣɤʑu] tu-tɯ-tʰe ɯ-tɯ́-cʰa? \\
\textsc{1pl} \textsc{gen} what \textsc{ipfv}-be.like-\textsc{1pl} \textsc{lnk} \textsc{1pl}.\textsc{poss}-\textsc{indef}.\textsc{poss}-water exist:\textsc{sens} \textsc{ipfv}-2-ask[III] \textsc{qu}-2-can:\textsc{fact} \\
\glt `Can you ask on our behalf how we should do to have water?' (2005tamukatsa, 14)
\end{exe}

Beneficiary genitive phrases can occur as predicates with a copula as \japhug{ɯʑɤɣ}{\textsc{3sg}:\textsc{gen}} in (\ref{ex:WZAG.pjAmaR}).

 \begin{exe}
\ex \label{ex:WZAG.pjAmaR}
\gll   tʰoʁtɤm ka-wum tɕe, ɯʑɤɣ pjɤ-maʁ kɯ, tɕoχtsi rɟɤlpu ɣɯ ku-wum,  \\
taxes \textsc{pfv}:3\fl{}3'-collect \textsc{lnk} \textsc{3sg}:\textsc{gen} \textsc{ifr}.\textsc{ipfv}-not.be \textsc{erg} p.n. king \textsc{gen} \textsc{ipfv}-collect \\
\glt `The taxes that he had collected were not for himself, he was collecting them for the king of Cogtse.' (150901 NAjstsa, 28)
\end{exe}

In this use too, it is alternatively possible to indicate the beneficiary as a possessive prefix on the object, without genitive postposition, as in (\ref{ex:atWci.tArke}).

 \begin{exe}
\ex \label{ex:atWci.tArke}
\gll   χsɤr kʰɯtsa ɯ-ŋgɯ nɯtɕu a-tɯ-ci ci tɤ-rke ma wuma ɲɯ-ɕpaʁ-a \\
gold bowl \textsc{3sg}.\textsc{poss}-inside \textsc{dem}:\textsc{loc} \textsc{1sg}.\textsc{poss}-\textsc{indef}.\textsc{poss}-water a.little \textsc{imp}-put.in[III] \textsc{lnk} really \textsc{sens}-be.thirsty-\textsc{1sg} \\
\glt  `Please pour some water in the golden bowl for me, I am thirsty.' (140428 mu e guniang-zh, 47)
\end{exe}

The genitive is also attested with a noun-verb collocations (§ XXX), like \japhug{ɯ-kɤrnoʁ+mtɕɯr}{feel dizzy}, in which the possessor  of the noun is an experiencer as in (\ref{ex:fsapaR.GW.kWnA}). This example also illustrates the use of the genitive followed by a focus marker, as (\ref{ex:GW.kWnA.WrkW.ri}) above.

\begin{exe}
\ex \label{ex:fsapaR.GW.kWnA}
\gll tɕeri fsapaʁ ɣɯ kɯnɤ ɯ-kɤrnoʁ ɲɯ-mtɕɯr ɲɯ-ŋu \\
\textsc{lnk} animal \textsc{gen} also \textsc{3sg}.\textsc{poss}-head \textsc{sens}-turn \textsc{sens}-be \\
\glt `But animals too can feel dizzy.' (29-tAmtshAzkAkWndo, 71)
\end{exe}


\subsubsection{Other uses} \label{sec:gen.other}
The genitive \forme{ɣɯ} occurs with various types of noun complements which are semantically neither possessive or beneficiaries/recipients. 

Nouns used as prenominal modifiers are in rare cases followed by a genitive postposition before the head noun. If the head noun is an APN, the presence of a third singular possessive prefix \forme{ɯ-} is optional, as shown by examples such as (\ref{ex:χsAr.GW.khWtsa}) and (\ref{ex:ftsoR.kWngWt.WphW}). 

\begin{exe}
\ex \label{ex:χsAr.GW.khWtsa}
\gll  χsɤr ɣɯ, nɤkinɯ, kʰɯtsa ci to-nɯ-ndo. \\
gold \textsc{gen} \textsc{filler} bowl \textsc{indef} \textsc{ifr}-\textsc{auto}-take \\
\glt `He took a golden bowl' (140508 shier ge tiaowu de gongzhu-zh, 158)
\end{exe}

This type of construction is most common in texts translated from Chinese, but does also occur in more spontaneous material as in (\ref{ex:ftsoR.kWngWt.WphW}), with a complex modifier \forme{ftsoʁ kɯngɯt ɯ-pʰɯ} `the price of nine female hybrid yaks'.

\begin{exe}
\ex \label{ex:ftsoR.kWngWt.WphW}
\gll tɕendɤre ɯ-jaʁ nɯtɕu [ftsoʁ kɯngɯt ɯ-pʰɯ] ɣɯ srɯnloʁ pjɤ-k-ɤ-rku-ci \\
\textsc{lnk} \textsc{3sg}.\textsc{poss} \textsc{dem}:\textsc{loc} female.hybrid.yak nine \textsc{3sg}.\textsc{poss}-price \textsc{gen} ring \textsc{ifr}.\textsc{ipfv}-\textsc{evd}-pass-put.in-\textsc{evd} \\
\glt `She had a ring worth nine female hybrid yak in her hand.' (2003gesar, 239)
\end{exe}

In a construction with a prenominal modifier marker in the genitive, even when a possessive prefix is present on the head noun (in particular when it is an IPN), that prefix does not necessarily refer to the modifier. For instance, in (\ref{ex:tWpAlAskAr.GW.nWmgozmArAB}), the third plural possessive prefix \forme{nɯ-} on \forme{nɯ-mgozmɤrɤβ} `their vegetables' refers to the people eating the vegetable, not the the modifier \japhug{tɯxpalɤskɤr}{the whole year} (on whose formation see § \ref{sec:dvandva.coll}) which would require a third singular prefix instead (an option which is also attested with this noun). Alternatively, it is also possible to have an indefinite possessive prefix on the head noun if it is an IPN, as in (\ref{ex:tWxpa.GW.tWGli}). Note that both options are attested in the construction with a prenominal modifier without the genitive, as seen in § \ref{sec:possessive.prefixes.prenominal}.
 
 \begin{exe}
\ex \label{ex:tWpAlAskAr.GW.nWmgozmArAB}
\gll  tɕe nɯnɯ tɯxpalɤskɤr ɣɯ nɯ-mgozmɤrɤβ nɯ nɯ ma pjɤ-me.  \\
\textsc{lnk} \textsc{dem} whole.year \textsc{gen} \textsc{3pl}.\textsc{poss}-vegetable \textsc{dem} \textsc{dem} apart.from \textsc{ifr}.\textsc{ipfv}-not.exist \\
\glt `It was the only vegetable that they had the whole year.' 
\end{exe}

\begin{exe}
\ex \label{ex:tWxpa.GW.tWGli}
\gll  tɯ-xpa ɣɯ tɯ-ɣli nɯ cʰɯ́-wɣ-tɕɤt tú-wɣ-rmbɯ  \\
one-year \textsc{gen} \textsc{indef}.\textsc{poss}-manure \textsc{dem} \textsc{ipfv}:\textsc{downstream}-\textsc{inv}-take.out \textsc{ipfv}-\textsc{inv}-heap \\
\glt `People take out (from the stable) the whole year's manure and heap it up.' (2010-tArAku)
\end{exe}

Some apparently unclassifiable uses of the genitive can be accounted for to some extent by assuming the elision of a head noun.  For instance, in (\ref{ex:iZora.GW}), the phrase \forme{iʑora ɣɯ}, meaning `in our language', can be explained as coming from \forme{iʑora ɣɯ ji-skɤt} `our language' used as a absolutive locative phrase (§ \ref{absolutive.locative}) `in our language', with elision of the head noun. This example does not illustrate a distinct function of the genitive, it is simply a particular case of possessive.

\begin{exe}
\ex \label{ex:iZora.GW}
\gll  <longtoutan> nɯ kupa-skɤt ɕti. tɕe iʑora ɣɯ tɕʰi tu-kɯ-ti ŋu mɤ-xsi. \\
pl.n. \textsc{dem} Chinese-language be.\textsc{affirm}:\textsc{fact} \textsc{lnk} \textsc{1pl} \textsc{gen} what \textsc{ipfv}-\textsc{genr}-say be:\textsc{fact} \textsc{neg}-\textsc{genr}:know \\
\glt `Longtoutan is a Chinese word; I don't know how it is said in our (language).'  (150820 qaprANar, 32)
\end{exe}

The same is true of the use of the genitive with the verb \japhug{mŋɤm}{be paintful} and its causative \japhug{ɕɯmŋɤm}{cause to be paintful}, which take a body part (not the person or animal feeling pain) as their subject and object respectively. In (\ref{ex:aZWG.taCWmNAm}), the genitive first person \japhug{aʑɯɣ}{\textsc{1sg:gen}} is not an oblique argument or even a malefactive adjunct. Rather, its presence implies an elided noun \japhug{a-βri}{my body} (`he caused pain to my body'). It is however likely that sentences like this are the pivot constructions which made possible the reanalysis of possessive genitive phrases as benefactive/malefactive adjuncts.

\begin{exe}
\ex \label{ex:aZWG.taCWmNAm}
\gll aʑɯɣ ta-ɕɯ-mŋɤm, aʑɯɣ a-laχtɕʰa ra ja-nɯ-tsɯm-nɯ \\
\textsc{1sg}:\textsc{gen} \textsc{pfv}:3\fl{}3'-\textsc{caus}-be.paintful \textsc{1sg}:\textsc{gen} \textsc{1sg}.\textsc{poss}-thing \textsc{pl} \textsc{pfv}:3\fl{}3'-\textsc{vert}-take.away-\textsc{pl} \\
\glt `He hurt me and took away my things.' (140426 luozi he qiangdao, 35)
\end{exe}

The genitive can also occur between prenominal relatives (§ XXX) and their head noun. In this construction the head noun generally does not take a possessive prefix. This type of relative is particularly common in story translated from Chinese, where it calques the prenominal relatives in \zh{的} <de>, as in (\ref{ex:makWra.GW.sAtCha}). The same situation has been observed in Khroskyabs (\citealt[640-643]{lai17khroskyabs}).

\begin{exe}
\ex \label{ex:makWra.GW.sAtCha}
\gll [kɯ-ɣɤndʐo ri kɯ-me], [kɯ-sɤ-mtsɯr ri kɯ-me], [kɤ-nɯsɯmɯzdɯɣ ri mɤ-kɯ-ra] ɣɯ sɤtɕʰa nɯtɕu jo-ɕe-ndʑi ɲɯ-ŋu. \\
\textsc{nmlz}:S/A-be.cold also \textsc{nmlz}:S/A-not.exist \textsc{nmlz}:S/A-\textsc{deexp}-be.hungry also \textsc{nmlz}:S/A-not.exist inf-worry also \textsc{neg}-\textsc{nmlz}:S/A-have.to \textsc{gen} place \textsc{dem}:\textsc{loc} \textsc{ifr}-go-\textsc{du} \textsc{sens}-be \\
\glt  `The two of them went to a place where they was cold cold and hunger, and where one did not need to worry.' (140519 mai huochai de xiao nvhai-zh, 182-183)
\end{exe}

However, this type of relative is also attested, though rarer, in non-translated texts, for instance in (\ref{ex:tWxpa.tukWlhoR.GW.sWjno}) with intransitive subject relativization (see § XXX for additional examples).
%kɯki aʑo ɕkom tu-mtshi-a ki ɣɯ ɯ-χpi ci pjɯ-fɕat-a

\begin{exe}
\ex \label{ex:tWxpa.tukWlhoR.GW.sWjno}
\gll  tɕe [tɯ-xpa tu-kɯ-ɬoʁ] ɣɯ sɯjno nɯ ŋu tɕe, \\
\textsc{lnk} one-year \textsc{ipfv}-\textsc{nmlz}:S/A-come.out \textsc{gen} plant \textsc{dem} be:\textsc{fact} \textsc{lnk} \\
\glt  `It is an annual plant.' (18-NGolo, 105)
\end{exe}

Genitival prenominal relative clauses are to be distinguished from relatives as possessors, as in (\ref{ex:tWCGA.kWmNAm.GW.WrJAnNgo}), where the possessum  \japhug{ɯ-rɟɤŋgo}{its radiating pain} is not an argument of the relative \forme{tɯ-ɕɣa kɯ-mŋɤm} `a tooth that hurts'. 

\begin{exe}
\ex \label{ex:tWCGA.kWmNAm.GW.WrJAnNgo}
\gll tɯ-ɕɣa a-tɤ-mŋɤm tɕe tɕe tɤ-rca tɯ-ɣmba, tɯ-ku nɯra tu-mŋɤm ɲɯ-ŋu tɕe,  nɯnɯ ``[tɯ-ɕɣa kɯ-mŋɤm] ɣɯ ɯ-rɟɤŋgo ɣɤʑu" tu-kɯ-ti ŋu. \\
\textsc{genr}.\textsc{poss}-tooth \textsc{irr}-\textsc{pfv}-be.painful \textsc{lnk} \textsc{lnk} \textsc{indef}.\textsc{poss}-following \textsc{genr}.\textsc{poss}-cheek \textsc{genr}.\textsc{poss}-head \textsc{dem}:\textsc{pl} \textsc{ipfv}-be.painful \textsc{sens}-be \textsc{lnk} \textsc{dem} \textsc{indef}.\textsc{poss}-tooth \textsc{nmlz}:S/A-be.painful \textsc{gen} \textsc{3sg}.\textsc{poss}-radiating.pain exist:\textsc{sens} \textsc{ipfv}-\textsc{genr}-say be:\textsc{fact} \\
\glt `When one has a toothache, and that one feels pain in one's cheek or a headache, one says `the toothache has a radiating pain.'' (140516 WrJANgo, 3)
\end{exe}

Adnominal complement clauses (§ XXX) can also take a genitive marker, as in (\ref{ex:mWjnaXtChWG.GW.WtCha}).

\begin{exe}
\ex \label{ex:mWjnaXtChWG.GW.WtCha}
\gll [<donggua> cʰo <qiezi> ni tɕʰi ʑo mɯ́j-naχtɕɯɣ] ɣɯ ɯ-tɕʰa a-jɤ-tɯ-ɣɯt ra \\
gourd \textsc{comit} eggplant \textsc{du} what \textsc{emph} \textsc{neg}:\textsc{sens}-be.the.same \textsc{gen} \textsc{3sg}.\textsc{poss}-information \textsc{irr}-\textsc{pfv}-2-bring have.to:\textsc{fact} \\
\glt `(Go there and come back to) tell me in what way gourd and eggplant differ from each other.' (2010-02-yitian bi yitian-zh, 7)
\end{exe}

Some relative clauses can take possessors marked in the genitive, relating to an argument within the relative, as in  (\ref{ex:stu.WkAnWmga}) and (\ref{ex:slama.ra.GW}). It is debatable whether the genitival phrase belongs to the relative in this type of construction.

\begin{exe}
\ex \label{ex:stu.WkAnWmga}
 \gll tɕe paʁ ɣɯ [stu ɯ-kɤ-nɯmga], iʑora ji-kɤ-nɯmga nɯ ɯ-ɕa ŋu tɕe \\
 \textsc{lnk} pig \textsc{gen} most \textsc{3sg}.\textsc{poss}-\textsc{nmlz}:P-want.from \textsc{1pl} \textsc{1pl}.\textsc{poss}-\textsc{nmlz}:P-want.from \textsc{dem} \textsc{3sg}.\textsc{poss}-meat be:\textsc{fact} \textsc{lnk} \\
\glt  `What is most wanted from pigs, what we want from them is their meat.' (05-paR, 13)
\end{exe}

\begin{exe}
\ex \label{ex:slama.ra.GW}
\gll  slama ra ɣɯ [tʰɯtʰɤci kɯ-fse], nɯ kɤ-rɤ-βzjoz ra ɲɯ-stu mɯ́j-stu nɯ, nɯ-stu ɲɯ-nɤma-nɯ mɯ́j-nɤma-nɯ,  nɯnɯra nɯ-pʰama ra nɯ-ɕki kɯ-rɤ-fɕɤt ɲɯ-ra. \\
student \textsc{pl} \textsc{gen} something \textsc{nmlz}:S/A-be.like \textsc{dem}  \textsc{inf}-\textsc{antipass}-learn \textsc{pl} \textsc{sens}-be.assiduous \textsc{sens}-be.assiduous \textsc{dem} \textsc{3pl}.\textsc{poss}-truth \textsc{sens}-work-\textsc{pl} \textsc{neg:sens}-work-\textsc{pl} \textsc{dem}:\textsc{pl} \textsc{3pl}.\textsc{poss}-parent \textsc{pl} \textsc{3pl}.\textsc{poss}-\textsc{dat} \textsc{genr}:S/P-\textsc{antipass}-tell:\textsc{fact} \textsc{sens}-have.to \\
\glt `One had to tell the parents all kinds of things concerning the students, whether they try hard or not, whether they work seriously or not.' (150901 tshuBdWnskAt, 18-20)
\end{exe}


\subsection{Locative} \label{sec:locative}
 

\subsubsection{Core locative postpositions} \label{sec:core.locative}
There are three locative postpositions in Japhug, \forme{zɯ}, \forme{tɕu} and \forme{ri}, the latter being homophonous with the correlative additive focus \forme{ri} (§ \ref{sec:ri.additive}). The exact conditions of their uses is still an unsolved problem of Japhug grammar. They appear to be always optional (goals and locative adjuncts can always be in absolutive form, see § \ref{absolutive.goal}  and \ref{absolutive.locative}) and seem to be interchangeable, as is illustrated in this section.

All three postpositions can be used to express static location, motion into, motion or from a place. Location or motion (into/from/on) a surface, (into/from/in) a container or with/without contact does not seem to be relevant factors for the selection of the locative postpositions.

The locative \forme{tɕu}  is most often used in combination with a demonstrative \forme{nɯ} as in (\ref{ex:co.nWtCu}), a form identical to the locative of the demonstrative pronoun (\japhug{nɯtɕu}{there}, see § \ref{sec:locative.pronoun}). Without demonstrative, \forme{tɕu} is also found as in (\ref{ex:tsxu.tCu}) and (\ref{ex:khAxtAndo.tCu}).

\begin{exe}
\ex \label{ex:co.nWtCu}
\gll japa tɕe alo <ercha> nɯtɕu, nɤkinɯ, iɕqʰa tsʰapa co nɯtɕu tɯɲɤt cʰɤ-ɣi. \\
last.year \textsc{lnk} upstream pl.n. \textsc{dem}:\textsc{loc} \textsc{filler}   \textsc{filler}  pl.n. valley \textsc{dem}:\textsc{loc} rock.slide \textsc{ifr}:\textsc{downstream}-come \\
\glt `Last year, at Ercha, at the valley of Tshapa, there was a rock slide.' (160715 nWNa, 1)
\end{exe}

\begin{exe}
\ex \label{ex:tsxu.tCu}
\gll jo-nɯ-ɕe tɕe, tʂu tɕu ɲɤ-mtsɯr, \\
\textsc{ifr}-\textsc{auto}-go \textsc{lnk} road \textsc{loc} \textsc{ifr}-be.hungry \\
\glt `He went away, and on the road he felt hungry.' (2002qajdoskAt, 109)
\end{exe}

\begin{exe}
\ex \label{ex:khAxtAndo.tCu}
\gll  kʰɤxtɤndo tɕu ko-zo \\
side.of.the.top.terrace \textsc{loc} \textsc{ifr}-land \\
\glt `(The raven) landed on the side of the top terrace.' (2002qajdoskAt, 24)
\end{exe}

The above examples show \forme{tɕu} used for location without motion (\ref{ex:co.nWtCu}), motion via a place (\ref{ex:tsxu.tCu}), motion onto a place resulting in contact with the surface (\ref{ex:khAxtAndo.tCu}), and (\ref{ex:WkWm.nWtCu}) illustrates \forme{tɕu} expressing motion from the inside.

\begin{exe}
\ex \label{ex:WkWm.nWtCu}
\gll nɯnɯ kɯβʁa ra ɣɯ nɯ-kʰa ɯ-kɯm nɯtɕu cʰɤ-nɯ-ɬoʁ tɕe, \\
\textsc{dem} nobleman \textsc{pl} \textsc{gen} \textsc{3pl}.\textsc{poss}-house \textsc{3sg}.\textsc{poss}-door \textsc{dem}:\textsc{loc} \textsc{ifr}:\textsc{downstream}-\textsc{auto}-come.out \textsc{lnk} \\
\glt  `He went out from the door of the nobleman's house.' (140513 mutong de disheng-zh, 166)
\end{exe}

The same diversity of uses is found with the locative \forme{zɯ}; (\ref{ex:nAmkha.zW.pjAnWlhoRnW}) shows \forme{zɯ} expressing static location (`in their hands') and motion from a place (`from the sky').  Example (\ref{ex:Wjme.zW.kAzo}) illustrates \forme{zɯ} used with the verb \japhug{zo}{land}, which is attested with \forme{tɕu} in (\ref{ex:khAxtAndo.tCu}). Note that in another version of the same story by the same speaker, the relator noun \japhug{ɯ-taʁ}{on} (§ \ref{sec:relator.nouns.3d}) is found instead of a locative postposition (see \ref{ex:ambro} in § \ref{ex:prefix.expression.of.possession}).

\begin{exe}
\ex \label{ex:nAmkha.zW.pjAnWlhoRnW}
\gll tɯmɯkɤrŋi ɯ-me kɯɕnɯz nɯ nɯ-jaʁ zɯ, nɤkinɯ, <shanzi> kɯ-mpɕɯ\redp{}mpɕɤr ʑo, qale ɯ-sɤ-lɤt nɯ pjɤ-k-ɤsɯ-ndo-nɯ-ci tɕe,  tɯmɯ nɤmkʰa zɯ pjɤ-nɯ-ɬoʁ-nɯ. \\
heaven \textsc{3sg}.\textsc{poss}-daughter seven \textsc{dem} \textsc{3pl}.\textsc{poss}-hand \textsc{loc} \textsc{filler} fan \textsc{nmlz}:S/A-\textsc{emph}\redp{}be.beautiful \textsc{emph} wind \textsc{3sg}.\textsc{poss}-\textsc{nmlz}:\textsc{oblique}-throw \textsc{dem} \textsc{ifr}.\textsc{ipfv}-\textsc{evd}-\textsc{prog}-hold-\textsc{pl}-\textsc{evd} \textsc{lnk} sky sky \textsc{loc} \textsc{ifr}:\textsc{down}-\textsc{auto}-come.out-\textsc{pl} \\
\glt `The seven daughters of heaven, holding beautiful fans in their hands, came down from the sky.' (150828 niulang-zh, 48)
\end{exe}

\begin{exe}
\ex \label{ex:Wjme.zW.kAzo}
\gll   a-mbro ɯ-jme zɯ kɤ-zo, \\
\textsc{1sg}.\textsc{poss}-horse \textsc{3sg}.\textsc{poss}-tail \textsc{loc} \textsc{imp}-land \\ 
\glt `Land on my horse's tail.' (2014-kWlAG, 562)
\end{exe}

The postposition \forme{zɯ} is related to the suffix \forme{-s} in Situ, which expresses motion from an origin or towards a goal (\citealt[330-331]{linxr93jiarongen}). Situ is certainly most archaic here (as it also preserves a locative \forme{-j} suffix of which only lexicalized traces remain in Japhug, § \ref{sec:locative.j}), and the Japhug form has to be explained as degrammaticalization  from suffix to clitic to independent word (see \ref{ex:nWnWtCu.zW} in § \ref{ex:word.vs.clitic.postp} for evidence that \forme{zɯ} is not a clitic). Japhug-internal evidence for the degrammaticalization is the otherwise unexplainable voicing to \forme{z}, a process that applied to all fricative codas (§ XXX), and the fact that some frozen forms preserve a \forme{-z} suffix, in particular the approximate locative \forme{cʰiz} (§ \ref{sec:approximate.locative}) and the relator \japhug{ɯ-ŋgɯz}{among} (see \ref{ex:kAndZWRi.nWNgWz}  and \ref{ex:arNi.WNgWz}  in § \ref{sec:other.locative.relator}). For further discussion on degrammaticalization in Japhug, see § XXX.


 The locative \forme{ri} also occurs in all the meanings attested above for \forme{tɕu} and \forme{zɯ}, though due to homophony with the additive correlative \forme{ri} (§ \ref{sec:ri.additive}), some examples are ambiguous. Examples (\ref{ex:khAXtu.ri.pWrAZia}) and (\ref{ex:kha.ri.kari}) show the locative \forme{ri} marking static location and motion towards a place respectively;  this postposition is however significantly more commonly used to mark static location than motion.

\begin{exe}
\ex \label{ex:khAXtu.ri.pWrAZia}
\gll   kʰɤxtu ri pɯ-rɤʑi-a tɕe tɤ-mtʰɯm tu-ndze-a pɯ-ŋu ri, \\
terrace \textsc{loc} \textsc{pst}.\textsc{ipfv}-stay-\textsc{1sg} \textsc{lnk} \textsc{indef}.\textsc{poss}-meat \textsc{ipfv}-eat[III]-\textsc{1sg} \textsc{pst}.\textsc{ipfv}-be \textsc{lnk} \\
\glt `I was on the terrace eating meat.' (150909 qandZGi, 5)
 \end{exe}
 
 \begin{exe}
\ex \label{ex:kha.ri.kari}
\gll   tɕe nɯ nɯ-kʰa ri ɯʑo kɤ-ari ɯ-qʰu tɕe, \\
\textsc{lnk} \textsc{dem} \textsc{3pl}.\textsc{poss}-house \textsc{loc} \textsc{3sg} \textsc{pfv}:\textsc{east}-go[II] \textsc{3sg}.\textsc{poss}-after \textsc{lnk} \\
\glt `After he went to their (his wife's family's) house...' (14-tApitaRi, 210)
  \end{exe}
  
The same interchangeability and optionality of the locative postpositions is observed when these are combined with relator nouns (§ \ref{sec:relator.postposition.location}). Although the three postpositions are almost identical in their range of uses, there are nevertheless differences in their compatibilities. With the locative demonstrative pronouns \japhug{kɯre}{here}, \japhug{nɯre}{there} and related forms (§ \ref{sec:locative.pronoun}), only \forme{ri} can be added, as in XXXXX.

All three postpositions are also used with time adjuncts, but do present some noticeable differences in usage. They can be interchangeably used with temporal relator nouns such as \japhug{ɯ-raŋ}{during, the time when} (§ \ref{sec:relator.temporal}), but in other contexts 

With counted nouns expressing time (§ \ref{sec:CN.time}), there is not a single example with \forme{zɯ} in the corpus. The postposition \forme{ri} is used to express a point in time  (as in \ref{ex:tWxsoz.ri}), while \forme{tɕu} mostly means `within (the time period)' as in (\ref{ex:tWxpa.nWtCu}).  

\begin{exe}
\ex \label{ex:tWxsoz.ri}
\gll tɯ-xsoz ri tɕe jo-ɣi ri, ɯ-βri tɤʑri ɣɤʑu, ɲɯ-ɤci. \\
one-morning \textsc{loc} \textsc{lnk} \textsc{ifr}-come \textsc{lnk} \textsc{3sg}.\textsc{poss}-body dew exist:\textsc{sens} \textsc{sens}-be.wet \\
\glt `One morning, (our hen) came (back from the forest, where it was laying eggs) and its body was wet from the dew.' (150819 kumpGa, 21-22)
\end{exe}

\begin{exe}
\ex \label{ex:tWxpa.nWtCu}
\gll  tɯ-xpa nɯtɕu [...] tɯ-tɯpʰu, ɣʑo nɯra kɯ, ɣʑɤzga nɯnɯ, nɯnɯ sqɯ-tɯrpa ɯ-ro nɯ ku-sɤjtɯ-nɯ ɲɯ-cʰa-nɯ. \\
one-year \textsc{dem}:\textsc{loc} { } one-hive bee \textsc{dem}:\textsc{pl} \textsc{erg} honey \textsc{dem} \textsc{dem} ten-pound \textsc{3sg}.\textsc{poss}-excess \textsc{dem} \textsc{ipfv}-gather-\textsc{pl} \textsc{sens}-can-\textsc{pl} \\
\glt  `In one year, one hive, the bees, the honey, they can gather more than ten pounds of it.' (26-GZo, 34-36)
\end{exe}

The form \forme{tɕe}, which  is mainly analyzable as a linker (§ XXX) and in some cases a topic marker (§ \ref{sec:tCe.topic}) occurs with locative and temporal adjuncts and could be analyzed as a postposition in these usages (see also § \ref{sec:locative.j}). With the temporal counted nouns, like \forme{ri} it expresses a specific point in time as in (\ref{ex:tWsNi.tCe}) rather than a duration.

 \begin{exe}
\ex \label{ex:tWsNi.tCe}
\gll   tɕe tɯ-sŋi tɕe ɲɤ-k-ɤtɯɣ-ci tɕe, \\
\textsc{lnk} one-day \textsc{lnk} \textsc{ifr}-\textsc{evd}-meet-\textsc{evd} \textsc{lnk} \\
\glt  `One day, (the bear finally) met (the rabbit).' (2011-13-qala, 22)
\end{exe}

%a-ʁi kɯ jɤ-ari nɯtɕu tɕe, nɤkinɯ, qapi jo-rɤmbɯmbri qhe,
%tɕendɤre nɯ jo-nɯɴqhu-j qhe, tɕe kha jɤ-zɣɯt-i

\subsubsection{Approximate locative} \label{sec:approximate.locative}

%pa jɯl pɕoʁ nɤki, tɯji ɯ-rkɯ nɯra maka me.
%rɯŋgu kɯnɤ lɤchu koŋla ʑo kɯ-ɣɤndʐo mɤɕtʂa thi nɯra me
%15-babW, 116-7 (6:20)
\subsubsection{Traces of the locative suffix \forme{*-j}} \label{sec:locative.j}
Situ has a locative suffix \forme{-j}, also used in the possessive construction (\citealt[325-330]{linxr93jiarongen}), which has disappeared in Japhug, though a few traces remain.


The form \forme{tɕe}, which is mainly used as a linker (§ XXX) and also occurs as a topic marker (§ \ref{sec:tCe.topic}), probably originates from the combination of the locative postposition \forme{tɕu} and the locative suffix \forme{*-j}, with vowel merger at a stage preceding the sound change \forme{*o} $\rightarrow$ \forme{u}  (\forme{*tɕo-j} $\rightarrow$ \forme{tɕe}; see § XXX for a discussion of these sound changes).

The former locative meaning of \forme{tɕe} is still indirectly visible in constructions such as \forme{ɯ-sɯm tɕe} `in his opinion, in his mind' as in (\ref{ex:asWm.tCe}) and (\ref{ex:WsWm.tCe}), where the \forme{tɕe} is neither a linker nor a topic marker (it cannot be replaced here by \forme{nɯ} § \ref{sec:nW.topic}, for instance). 

\begin{exe}
\ex \label{ex:asWm.tCe}
\gll aʑo a-sɯm tɕe, nɯ-ʁrɯ ʑo ɣɤʑu ɕti tɕe \\
\textsc{1sg} \textsc{1sg}.\textsc{poss}-mind \textsc{lnk} \textsc{3pl}.\textsc{poss}-horn \textsc{emph} exist:\textsc{sens} be.\textsc{affirm}:\textsc{fact} \textsc{lnk} \\
\glt `In my opinion, (since) they have horns, (they should be able to fight the predators off).' (20-RmbroN, 64)
\end{exe}

In (\ref{ex:WsWm.tCe}), the second \forme{tɕe} may be analyzed as a topic marker (§ \ref{sec:tCe.topic}).

\begin{exe}
\ex \label{ex:WsWm.tCe}
\gll 
tɕe ɯʑo ɯ-sɯm tɕe tɕe tu-tsɯm tɕe tɕendɤre iʑora ji-sɤtɕha ra lonba ʑo tɯ-ci ɲɯ-sɯ-ɤβze to-ʁmɯɣ. \\
\textsc{lnk} \textsc{3sg} \textsc{3sg}.\textsc{poss}-mind \textsc{lnk} \textsc{lnk} \textsc{ipfv}:\textsc{up}-take \textsc{lnk} \textsc{lnk} \textsc{1pl} \textsc{1pl}.\textsc{poss}-place \textsc{pl} all \textsc{emph} \textsc{indef}.\textsc{poss}-water \textsc{ipfv}-\textsc{caus}-become[III] \textsc{ifr}-have.the.intention \\
\glt `In his mind, (the snake) wanted to take the water upwards and transform our whole area into water.' (150820 qaprANar, 20)
\end{exe} 

Another trace of the locative suffix \forme{*-j} is found in the linker \japhug{qʰe}{then} (§ XXX) and the time ordinal \japhug{qʰuj}{this afternoon} (\ref{sec:time.ordinals}), combining the relator noun \japhug{ɯ-qʰu}{after} (§ \ref{sec:relator.temporal}) with the coda \forme{*-j}, in the former with vowel fusion (an earlier lexicalization), and the latter without fusion. The form  \japhug{qʰuj}{this afternoon}  shows that the suffix \forme{*-j} was still productive in Japhug after the sound change \forme{*o} $\rightarrow$ \forme{u} took place.



\subsection{Comitative} \label{sec:comitative} 
Postpositional phrases with the comitative postposition \japhug{cʰo}{and, with} and its variants \forme{cʰondɤre} and \forme{cʰonɤ} (comprising the linkers \forme{nɤ} and \forme{ndɤre}, see § XXX) are selected as oblique arguments by a handful of verbs, including \japhug{naχtɕɯɣ}{be the same} (§ \ref{sec:identity.modifier}) and \japhug{amɯmi}{be in good terms with}, as shown in (\ref{ex:cho.kWnaXtCWG}).

\begin{exe}
\ex \label{ex:cho.kWnaXtCWG}
\gll [ɯʑo cʰo] kɯ-naχtɕɯɣ [sɯjno, xɕaj ma mɤ-kɯ-ndza nɯra cʰonɤ] amɯmi-nɯ tɕe, \\
\textsc{3sg} \textsc{comit} \textsc{nmlz}:S/A-be.the.same vegetables grass apart.from \textsc{neg}-\textsc{nmlz}:S/A-eat \textsc{dem}:\textsc{pl} \textsc{comit} be.in.good.terms:\textsc{fact}-\textsc{pl} \textsc{lnk} \\
\glt `(The rabbit) is in good terms with (the animals) which eat only grass and vegetables like him.' (04-qala2, 8)
\end{exe}

Postpositional phrases in \forme{cʰo} are oblique arguments in the sense that they are relativized using the oblique participle (§ XXX). However, verbs that select \forme{cʰo} phrases index not only the intransitive subject proper, but the sum of the subject and the \forme{cʰo} phrase, which can be in the dual as in (\ref{ex:cho.YWnaXtCWGndZi}) (the white birch and the red birch) or in the plural (\ref{ex:cho.kWnaXtCWG}) (the rabbit and the other animals). 

\begin{exe}
\ex \label{ex:cho.YWnaXtCWGndZi}
\gll tɕe ɯ-rqʰu nɯ ɣɯrni laʁma ɯ-ŋgɯ nɯ [sɤjku cʰo] ɲɯ-naχtɕɯɣ-ndʑi ri\\
\textsc{lnk} \textsc{3sg}.\textsc{poss}-bark \textsc{dem} be.red:\textsc{fact} apart.from.the.fact \textsc{3sg}.\textsc{poss}-inside \textsc{dem} birch \textsc{comit} \textsc{sens}-be.the.same-\textsc{du} \textsc{lnk} \\
\glt `Apart from the fact that its bark is red, it is identical in the inside with the birch.' (06-mbrAj, 13)
\end{exe}

The verb \japhug{naχtɕɯɣ}{be the same} with a \forme{cʰo} phrase can be used in an equative construction (see § XXX).

 Apart from the function presented above, \japhug{cʰo}{and, with} is commonly used to link together two nouns inside a single noun phrase, as in (\ref{ex:awW.cho.aRi}). In this case too, the main verb of the clause indexes the whole noun phrase, comprising the sum of referents designated by the nouns linked by \forme{cʰo}.

\begin{exe}
\ex \label{ex:awW.cho.aRi}
\gll a-wɯ cʰo a-ʁi ni cʰɯ-ɣi-ndʑi ra ma ʑɤni-sti kɤ-rɤʑi mɤ-cʰa-ndʑi tɕe, \\
\textsc{1sg}.\textsc{poss}-grand.father \textsc{comit} \textsc{1sg}.\textsc{poss}-younger.sibling \textsc{du} \textsc{ipfv}:\textsc{downstream}-come-\textsc{du} have.to:\textsc{fact} \textsc{lnk} \textsc{3du}-alone \textsc{inf}-stay \textsc{neg}-can:\textsc{fact}-\textsc{du} \textsc{lnk} \\ 
\glt `My grandfather and my younger brother have to come, they cannot stay by themselves.' (2011-05-nyima, 209)
\end{exe}

The marker \forme{cʰo} can also link verb phrases and even entire clauses (see § XXX and \citealt[313]{jacques14linking}).

Given the apparently equal status of the two linked nouns in (\ref{ex:awW.cho.aRi}), in particular with regard to indexation, it is legitimate to wonder whether analyzing it as a postposition makes more sense than considering it to be a coordinator (§ \ref{sec:coordinator}). There are two arguments supporting the postposition analysis. First, \forme{cʰo} necessarily follows a noun phrase (or at the very least a demonstrative pronoun), but does not require to be followed by another noun as in (\ref{ex:cho.YWnaXtCWGndZi}) above. Second, phrases comprising \forme{cʰo} and the preceding noun are relativized using the oblique participle (see § XXX).

A \forme{cʰo} phrase can be followed by the associative plural marker \forme{ra} (§ \ref{sec:number.determiners}) as in (\ref{ex:cho.ra.kW}) to mean `et caetera', and the whole phrase can taken case marking such as ergative.

\begin{exe}
\ex \label{ex:cho.ra.kW}
\gll tɕeri ɯʑo ndɤre, qajdo cʰo ra kɯ ndɤ tú-wɣ-ndza ɕti \\
but \textsc{3sg} \textsc{advers} crow \textsc{comit} \textsc{pl} \textsc{erg} \textsc{advers} \textsc{ipfv}-\textsc{inv}-eat be.\textsc{affirm}:\textsc{fact} \\
\glt `But it is eaten by crows and other (animals).' (26-NalitCaRmbWm, 140)
\end{exe}

A postpositional comitative phrase can also serve as a dual or plural possessor, as if from a complex noun phrase `X \forme{cʰo} Y' with elided Y element, as in (\ref{ex:cho.ndZime}).\footnote{Note that (\ref{ex:cho.ndZime}) does not mean `There is his wife and their daughter' (dual indexation would be expected on the verb).}

\begin{exe}
\ex \label{ex:cho.ndZime}
\gll tɕe ɯ-rʑaβ cʰo ndʑi-me ci tu tɕe, \\
\textsc{lnk} \textsc{3sg}.\textsc{poss}-wife \textsc{comit} \textsc{3du}.\textsc{poss}-daughter one exist:\textsc{fact} \textsc{lnk} \\
\glt `He and his wife have a daughter.' (14-tApitaRi, 313)
\end{exe}

\subsection{Standard marker} \label{sec:comparative} 
Japhug has several postpositions that are mainly used to mark the standard in the comparative construction. The most common one is \japhug{sɤz}{compared with}, but the variants \forme{staʁ}, \forme{sɤznɤ}, \forme{staʁnɤ}, \forme{sɯstaʁ} and \forme{sɯχta} are also attested. Their relative frequency appears to be speaker-dependent, and no meaningful difference could be detected between them. 

In the comparative construction (§ XXX), the comparee is the intransitive subject of the main verb (the parameter, generally an adjectival stative verb) and is indexed on the verb. The comparee is either in the absolutive or in the ergative (§ \ref{sec:comparee.kW}). The standard is necessarily marked by one of the postpositions listed above, and cannot be indexed on the main verb. Neither the standard not the comparee are required to be overt. An adjectival stative verb with a standard postpositional phrase as in  (\ref{ex:sAznA.YWwxti})  is a well-formed comparative construction. Examples like (\ref{ex:aZo.YWwxti}) with overt comparee and standard are rarer.

\begin{exe}
\ex \label{ex:sAznA.YWwxti}
\gll  qandʑɣi sɤznɤ ɲɯ-wxti, qaliaʁ sɤznɤ ɲɯ-xtɕi \\
falcon \textsc{comp} \textsc{sens}-be.big eagle \textsc{comp} \textsc{sens}-be.small \\
\glt `It is bigger than a falcon, and smaller than an eagle.' (2011-08-kuwu, 40-41)
\end{exe}

\begin{exe}
\ex \label{ex:aZo.YWwxti}
\gll ɯʑo nɯ aʑo sɤz tɯ-xpa wxti  \\
\textsc{3sg} \textsc{dem} \textsc{1sg} \textsc{comp} one-year be.big:\textsc{fact} \\
\glt `She is one year older than me.' (12-BzaNsa, 94)
\end{exe}

The standard marker \forme{sɤz} (and its variants) also occurs in a  construction expressing progressive increase throughout the time, where a time counted noun like \japhug{tɯ-sŋi}{one day} or \japhug{tɯ-xpa}{one year} is followed by the standard marker and then repeated, as \forme{tɯ-xpa sɤz tɯ-xpa} `more each year' in (\ref{ex:tWxpa.sAz.tWxpa}). This construction, although attested in non-translated texts, is more common in texts from Chinese, where it calques the construction \ch{一年比一年}{yīnián bǐ yīnián}{more each year}. The more idiomatic Japhug construction to express the same meaning is through partial reduplication of the first syllable of the main verb (see § XXX).
 
 \begin{exe}
 \ex \label{ex:tWxpa.sAz.tWxpa}
 \gll nɯ-jwaʁ nɯ, [...] tɯ-xpa sɤz tɯ-xpa lu-dɤn ŋu ma \\
 \textsc{3pl}.\textsc{poss}-leaf \textsc{dem} { } one-year \textsc{comp} one-year \textsc{ipfv}-be.many be:\textsc{fact} \textsc{lnk} \\
\glt  `There are more needles (leaves) each year.' (08-saCW, 17)
\end{exe}
 
 The standard markers can also be used with subordinate clauses (§ XXX). The standard marker with the distal demonstrative \forme{nɯ sɤznɤ} has the meaning `rather than that, could ... as well' as in (\ref{ex:nW.sAznA.arca}).  
 
 \begin{exe}
 \ex \label{ex:nW.sAznA.arca}
 \gll  nɤ-mu kɤ-fsraŋ mɤ-tɯ-cʰa tɕe, nɯ sɤznɤ, a-rca jɤ-ɣi tɕe, a-rca, nɤki, laχɕi pɯ-βzjoz \\
 \textsc{2sg}.\textsc{poss}-mother \textsc{inf}-protect \textsc{neg}-2-can:\textsc{fact} \textsc{lnk} \textsc{dem} \textsc{comp} \textsc{1sg}.\textsc{poss}-following \textsc{imp}-come \textsc{lnk} \textsc{1sg}.\textsc{poss}-following \textsc{filler} trade \textsc{imp}-learn \\
\glt `You cannot save your mother, rather than that, come with me to learn  some abilities.' (150826 baoliandeng-zh, 142-143)
\end{exe}

This phrase can also be used as a scalar marker `even' with scope on the following clause, as in (\ref{ex:nW.sAznA.chaa}). See § XXX for a more detailed discussion. % nɯ sɤznɤ ... ʁo alala ri
 

 \begin{exe}
 \ex \label{ex:nW.sAznA.chaa}
 \gll ki kɤ-rtsi kɯ-tu me nɤ, aʑo nɯ sɤznɤ, nɤkinɯ, kʰa kɯ-qanɯ\redp{}nɯ ɯ-ŋgɯ zɯ, nɤkinɯ, tɯ-ɕpɤβ kɯβde-rzɯɣ tɤ-kɤ-lɤt nɯnɯ ku-sɤlɤɣi-a cʰa-a ɕti nɤ! \\
 \textsc{dem}.\textsc{prox} \textsc{inf}-count \textsc{nmlz}:S/A-exist not.exist:\textsc{fact} \textsc{sfp} \textsc{1sg} \textsc{dem} \textsc{comp} \textsc{filler} house   \textsc{nmlz}:S/A-\textsc{emph}\redp{}be.dark \textsc{3sg}.\textsc{poss}-inside \textsc{loc}  \textsc{filler} \textsc{indef}.\textsc{poss}-corpse four-section \textsc{pfv}-\textsc{nmlz}:P-throw \textsc{dem} \textsc{ipfv}-combine-\textsc{1sg} can:\textsc{fact}-\textsc{1sg} be.\textsc{affirm}:\textsc{fact} \textsc{sfp}  \\
\glt  `(What you ask) is nothing, I am even able to put together a body cut into four sections in a dark house.' (140512 alibaba-zh, 170)
\end{exe}

\subsection{Exceptive} \label{sec:exceptive} %\japhug{ma}{apart from} laʁma mɯma
The exceptive postposition \japhug{ma}{apart from} and its reduplicated variant \forme{mɯma} are not selected by any verb, and only used in adjunct postpositional phrases as in (\ref{ex:kWm.ci.mWma}).

 \begin{exe}
 \ex \label{ex:kWm.ci.mWma}
 \gll kɯm ci mɯma nɯnɯ tɕe znde ʁɟa ʑo ɕti \\
 door one apart.from \textsc{dem} \textsc{lnk} wall completely \textsc{emph} be.\textsc{affirm}:\textsc{fact} \\
 \glt `Apart from one door, there are walls everywhere.' (2011-11-kha, 40)
\end{exe}

The exceptive \japhug{ma}{apart from} is used in particular in restrictive focus constructions (§ \ref{sec:restrictive.focus}).

When the scope of the restrictive construction is on an entire clause rather than a single noun phrase, the clause is followed by the linker \forme{ma} (homophonous with the exceptive) and an exceptive phrase limited to the demonstrative pronoun \forme{nɯ} (here in resumptive use, coreferent with the entire preceding clause) and the postposition \forme{ma}, as in (\ref{ex:ra.ma.nW.ma}). The first \forme{ma} in this construction is not to be analyzed as the postposition: while it is possible to reduplicated the second one as in \forme{ma nɯ mɯma} (example \ref{ex:mAspea.ma.nW.ma}), reduplication of the first \forme{ma} is not attested.

 \begin{exe}
 \ex \label{ex:ra.ma.nW.ma}
 \gll [aʑɯɣ ɯ-ɕa ra] ma nɯ ma kɯ-ra me \\
 \textsc{1sg}.\textsc{gen} \textsc{3sg}.\textsc{poss}-meat have.to:fact \textsc{lnk} \textsc{dem} apart.from \textsc{nmlz}:S/A-have.to not.exist:\textsc{fact} \\
 \glt `I want its meat, and nothing else.' (02-deluge2012, 14)
\end{exe}

 \begin{exe}
 \ex \label{ex:mAspea.ma.nW.ma}
 \gll tɤ-pɤtso kɯ-ɣɤwu ʑo kɤ-nɯɕpɯz mɤ-spe-a ma nɯ mɯma spe-a \\
 \textsc{indef}.\textsc{poss}-child \textsc{nmlz}:S/A-cry \textsc{emph} \textsc{inf}-imitate \textsc{neg}-be.able[III]:\textsc{fact}-\textsc{1sg} \textsc{lnk} \textsc{dem} apart.from be.able[III]:\textsc{fact}-\textsc{1sg}  \\
\glt  `I cannot imitate a baby crying, but apart form that I can imitate (all animal sounds).' (27-kikakCi, 143)
\end{exe}

\subsection{Terminative} \label{sec:terminative}  

The postposition \japhug{mɤɕtʂa}{until} is used after noun phrases to indicate temporal (\ref{ex:RnWpArme.mACtsxa}) or locative (\ref{ex:akW.mACtsxa}) limit. It can be used in opposition with the egressive postpositions (see \ref{ex:tWmAmke.mACtsxa} in § \ref{sec:egressive}) or with \japhug{kóʁmɯz}{only after} (example \ref{ex:sqamNuxpa.koRmWz} in § \ref{sec:temporal.postpositions}).

\begin{exe}
\ex \label{ex:RnWpArme.mACtsxa}
 \gll tɤ-pɤtso kɯ-dɤn nɯra tɕe, tɯ-pɤrme, ʁnɯ-pɤrme jamar mɤɕtʂa tɯ-nɯ ku-tsʰi-nɯ. \\
 \textsc{indef}.\textsc{poss}-child \textsc{nmlz}:S/A-be.many \textsc{dem}:\textsc{pl} \textsc{lnk} one-year.old two-year.old about until \textsc{indef}.\textsc{poss}-breast \textsc{ipfv}-drink-\textsc{pl} \\
 \glt `In (families where) children are many, (mothers) breastfeed (the children) until (they are) one or two years old.' (140426 tApAtso kAnWBdaR, 13)
\end{exe}

\begin{exe}
\ex \label{ex:akW.mACtsxa}
 \gll akɯ mɤɕtʂa ɣɯ-ku-ta-lɤt \\ 
east until \textsc{cisloc}-\textsc{ipfv}:\textsc{east}-1\fl{}2-release \\
\glt  `I come with you (see you off) until the (land of the) east.' (28-smAnmi, 220)
\end{exe}

In combination with the demonstrative \japhug{nɯ}{that}, \japhug{mɤɕtʂa}{until} means `otherwise', as in (\ref{ex:nW.mACtsxa}) (see also § XXX).

\begin{exe}
\ex \label{ex:nW.mACtsxa}
 \gll   kɤ-sɤŋo ʁɟa qʰe,  nɯ-mtɕʰi kɤ-χpjɤt ʁɟa kɯ kú-wɣ-spa ɕti.  nɯ mɤɕtʂa mɤ-kʰɯ. \\
 \textsc{inf}-hear completely \textsc{lnk} \textsc{3pl}.\textsc{poss}-mouth \textsc{inf}-observe completely \textsc{erg} \textsc{ipfv}-\textsc{inv}-be.able be.\textsc{affirm}:\textsc{fact} \textsc{dem} until \textsc{neg}-be.possible:\textsc{fact} \\
\glt `(In order to learn the Tshobdun language, since it has no writing system), one has no choice but to listen and observe people's mouth to learn it, otherwise it is not possible.' (150901 tshuBdWnskAt, 41-44)
\end{exe}

The postposition \japhug{mɤɕtʂa}{until} also occurs with subordinate temporal clauses, as shown in § XXX.

\subsection{Egressive} \label{sec:egressive}  
There are six egressive postpositions in Japhug, which are built by combining the root \forme{ɕaŋ-} with either the root of locative relator nouns (§ \ref{sec:relator.location}) or locational adverbs (§ XXX) as shown in Table \ref{tab:egressive}. There is a one-to-one relationship between the orientations of these postpositions and the six definite orientations found in verb morphology (§ XXX).

The egressive postpositions are mainly used with noun phrases of location expressing a length (\ref{ex:CaNtaR.mAzri}) or a reference point marking a limit (\ref{ex:praRwW.CaNdi}). However, \japhug{ɕaŋtaʁ}{up from} and \japhug{ɕaŋpa}{down from} can also follow noun phrases referring to time reference or durations, as in (\ref{ex:kWmNArZaR}) or more generally any quantity (\ref{ex:XsWm.CaNtaR}). No examples of these postpositions following finite subordinate clauses have been found.

\begin{exe}
\ex \label{ex:CaNtaR.mAzri}
 \gll  tɯ-tɣa ɕaŋtaʁ mɤ-zri. \\
 one-span up.from \textsc{neg}-be.long:\textsc{fact} \\
 \glt `It is at most one handspan long.' (28-tshAwAre, 51)
 \end{exe}
 
\begin{exe}
\ex \label{ex:praRwW.CaNdi}
 \gll ma kɯtɕɯmke nɯnɯtɕu, akɯ ku-ru tɕe, praʁwɯ ɕaŋdi sɤ-mto, 
andi tɕe tɕe, ɕɯfco ɕaŋkɯ nɯ sɤ-mto tɕe, \\
\textsc{lnk} pl.n. \textsc{dem}:\textsc{loc} east \textsc{ipfv}:\textsc{east}-look.at \textsc{lnk} pl.n. west.from \textsc{deexp}-see:\textsc{fact} west \textsc{lnk} \textsc{lnk} pl.n. east.from \textsc{dem} \textsc{deexp}-see:\textsc{fact} \textsc{lnk} \\
\glt `In Kuchumke, looking towards the east, (the areas) to the west of Praqwu are visible, and in the west, (the areas) to the east of Shyufkyo are visible.' (150904 tshAcim, 30)
\end{exe}

\begin{exe}
\ex \label{ex:kWmNArZaR}
 \gll kɯmŋɤsqɤ-rʑaʁ ɕaŋtaʁ cʰɯ-mdɯ-nɯ mɤ-ŋgrɤl tu-ti-nɯ ɲɯ-ŋu.  \\
 fifty-day up.from \textsc{ipfv}-live.up.to-\textsc{pl} \textsc{neg}-be.usually.the.case:\textsc{fact} \textsc{ipfv}-say-\textsc{pl} \textsc{sens}-be \\
\glt `They cannot live more than fifty days, it is said.' (26-GZo, 41)
\end{exe}

\begin{exe}
\ex \label{ex:XsWm.CaNtaR}
 \gll ɯ-pɯ nɯnɯ χsɯm ɕaŋtaʁ tu mɯ́j-ŋgrɤl \\
\textsc{3sg}.\textsc{poss}-young \textsc{dem} three up.from exist \textsc{neg}:\textsc{sens}-be.usually.the.case \\
\glt  `It does not usually have more than three offsprings.' (2011-08-kuwu, 14)
\end{exe}

The postposition \japhug{ɕaŋtaʁ}{up from}  is by far more common than all the other ones, and is often combined with a negative predicate to mean `at most', as shown by (\ref{ex:CaNtaR.mAzri}), (\ref{ex:kWmNArZaR}) and (\ref{ex:XsWm.CaNtaR}) above and (\ref{ex:tWfsu.CaNtaR}) below. In addition to the locational, temporal and quantitative meanings presented above, it can be used in a more abstract and qualitative sense, as in (\ref{ex:BGAru.ci.CaNtaR}).

\begin{exe}
\ex \label{ex:tWfsu.CaNtaR}
 \gll tɯrme tɯ-fsu ɕaŋtaʁ tu-mbro mɤ-cʰa. \\
people \textsc{genr}.\textsc{poss}-same.size up.from \textsc{ipfv}:\textsc{up}-be.high \textsc{neg}-can:\textsc{fact} \\
\glt `It grows at most as tall as a person.'(11-qarGW, 29)
\end{exe}

 \begin{exe}
\ex \label{ex:BGAru.ci.CaNtaR}
 \gll  wortɕʰi ʑo βɣɤru ci ɕaŋtaʁ ʑo tɤ-sɯ-ɤwɯwum-nɯ tɕe, \\
 please \textsc{emph} miller \textsc{indef} up.from \textsc{emph} \textsc{ifr}-\textsc{caus}-\textsc{recip}:gather-\textsc{pl} \textsc{lnk} \\
\glt `Please gather (everybody) up from the miller (the lowliest of servants).' (2003kandZislama, 174)
\end{exe}

\begin{table}
\caption{Egressive postpositions} \label{tab:egressive} \centering
\begin{tabular}{llllll}
\lsptoprule
Posposition & Relator noun & Locational adverb\\
\midrule
\japhug{ɕaŋtaʁ}{up from} & \japhug{ɯ-taʁ}{up, top}& \\
\japhug{ɕaŋpa}{down from} & \japhug{ɯ-pa}{down, bottom}& \\
\japhug{ɕaŋlo}{upstream from} & & \japhug{alo}{upstream} \\
\japhug{ɕaŋtʰi}{downstream from} & & \japhug{atʰi}{upstream} \\
\japhug{ɕaŋkɯ}{east from} & & \japhug{akɯ}{east} \\
\japhug{ɕaŋdi}{west from} & & \japhug{andi}{west} \\
\lspbottomrule
\end{tabular}
\end{table}

The egressive postpositions can be used in contrast with the terminative \japhug{mɤɕtʂa}{until}, as in (\ref{ex:tWmAmke.mACtsxa}).

\begin{exe}
\ex \label{ex:tWmAmke.mACtsxa}
 \gll tɯ-mke ɕaŋpa tɕe tɕe ki tɯ-mɤmke mɤɕtʂa kɯ-zɣɯt kɯ-rɲɟi pjɯ-ŋu ra.  \\
\textsc{indef}.\textsc{poss}-neck down.from \textsc{lnk} \textsc{lnk} \textsc{dem}.\textsc{prox} \textsc{indef}.\textsc{poss}-ankle until \textsc{nmlz}:S/A-reach  \textsc{nmlz}:S/A-be.long \textsc{ipfv}-be have.to:\textsc{fact} \\  
\glt  `(Tibetan clothes) have to be long (enough) so as to reach the ankle down from the neck.' 
\end{exe}

As other postpositional phrases, egressive phrases followed by demonstratives (§ \ref{sec:demonstrative.determiners}) mean `the person(s)/thing(s) from X', as in (\ref{ex:TshuBdWn.CaNlo}).

 \begin{exe}
\ex \label{ex:TshuBdWn.CaNlo}
 \gll iʑora kɯ, nɤki, tsʰuβdɯn ɕaŋlo nɯra `stɤtpa-pɯ' tu-ti-j ŋu. \\
 \textsc{1pl} \textsc{erg} \textsc{filler} pl.n. upstream.from \textsc{dem}:\textsc{pl} pl.n.-person \textsc{ipfv}-say-\textsc{1pl} be:\textsc{fact} \\
\glt `We call the people (who live) in Tshobdun and further upstream `Stotpa'.' (23-tCAphW, 14)
\end{exe}

The postposition \japhug{ɕaŋlo}{upstream from} is homophonous with, and historically related to, the noun \japhug{ɕaŋlo}{seating place for old people and ladies in the kitchen} (§ XXX).
 
 \subsection{Other temporal postpositions} \label{sec:temporal.postpositions}
%    \japhug{ʁaz}{xxxxx}
Apart from the locative, terminative and egressive postpositions, a certain number of specifically temporal postpositions are found in Japhug, including \japhug{ɕɯŋgɯ}{before},  \japhug{ɕɯmɯma}{immediately after}, \japhug{kóʁmɯz}{only after}, \japhug{pɕɯntɕɤt}{since}, \japhug{jɤz}{when} and \japhug{ɕaŋpɕi}{from ... on}. All can be used with noun phrases and subordinate clauses; the latter use is studied in the section on temporal clauses (§ XXX).

The postposition  \japhug{ɕɯŋgɯ}{before} is an ancient compound containing as first element the \textit{status constructus} of a root cognate to Tangut \tangut{𗪘}{2104}{śji}{1.10} `formerly, before' and the relator noun \japhug{ɯ-ŋgɯ}{inside} (§ \ref{sec:relator.location}) as second element. It contrasts with  \japhug{ɯ-qʰu}{after} (\ref{sec:relator.temporal}), and can follow a noun phrase referring to a point in time, as in the common expression \japhug{saχsɯ ɕɯŋgɯ}{before lunch} (with the noun  \japhug{saχsɯ}{lunch}), or a duration, as in (\ref{ex:sqamNusNi.CWNgW}) and (\ref{ex:XsArZaR.CWNgW}). The latter example shows that \japhug{ɕɯŋgɯ}{before} can be used to refer to events occurring \textit{after} the current temporal point of reference (in \ref{ex:XsArZaR.CWNgW}, before three days from the present in the story).

\begin{exe}
\ex \label{ex:sqamNusNi.CWNgW}
 \gll tɤte sqamŋu-sŋi ɕɯŋgɯ nɯtɕu tɕe, nɤki, nɯ-rlaʁ tɕe nɯ-me pɯ-ŋu ɲɯ-ŋu, tɕʰeme nɯ, \\
 that.is fifteen-day before \textsc{dem}.\textsc{loc} \textsc{lnk} \textsc{filler} \textsc{pfv}-disappear \textsc{lnk} \textsc{pfv}-not.exist \textsc{pst}.\textsc{ipfv}-be \textsc{sens}-be girl \textsc{dem} \\
 \glt `That is, fifteen days before, she had disappeared, that girl.' (tWxtsa, 15)
\end{exe}


\begin{exe}
\ex \label{ex:XsArZaR.CWNgW}
 \gll χsɯ-sŋi χsɤ-rʑaʁ mɤɕtʂa a-mɤ-tɤ-tɯ-rɤru ra ma tɕe mɤ-pʰɤn nɯra to-ti. matɕi tɕetʰa χsɯ-sŋi χsɤ-rʑaʁ ɕɯŋgɯ ɯʑo ɲɯ-pʰɣo pjɤ-ra lo \\
 three-day  three-night until \textsc{irr}-\textsc{neg}-\textsc{pfv}-2-get.up have.to:\textsc{fact} \textsc{lnk} \textsc{lnk}  \textsc{neg}-be.efficient:\textsc{fact} \textsc{dem}:\textsc{pl} \textsc{ifr}-say because later   three-day  three-night before \textsc{3sg} \textsc{ipfv}-flee \textsc{ifr}.\textsc{ipfv}-have.to \textsc{sfp} \\
 \glt `(The rabbit) said `Don't get up until three days and three nights (have passed)', because (the rabbit was buying time) and had to flee before (the end of) these three days and nights.' (140427 qala cho kWrtsag, 30-31)
\end{exe}

The postposition   \japhug{ɕɯŋgɯ}{before} however most commonly occurs with subordinate clauses, and requires a finite verb in the imperfective form (§ XXX). It is attested following personal pronouns, as in (\ref{ex:aʑo.CWNgW}),  \japhug{ɕɯŋgɯ}{before} refers to an action concerning the referent of the pronoun, whose nature can be determined from the context, as if the main verb of a subordinate clause had been elided.

\begin{exe}
\ex \label{ex:aʑo.CWNgW}
 \gll  aʑo ɕɯŋgɯ a-pi ra atu rɤʑi-nɯ tɕe, nɯnɯra ɣɯ nɯ-rmi tɤ-z-mɤke qʰe, \\
 \textsc{1sg} before \textsc{1sg}.\textsc{poss}-elder.sibling \textsc{pl} up.there stay:\textsc{fact}-\textsc{pl} \textsc{lnk} \textsc{dem}:\textsc{pl} \textsc{gen} \textsc{3pl}.\textsc{poss}-name \textsc{imp}-\textsc{caus}-be.first[III] \textsc{lnk} \\
 \glt `Before (you choose a name for) me, my elder brothers up there, (choose) their names first.' (Gesar, 124)
\end{exe}


The common adverb \japhug{kɯɕɯŋgɯ}{in former times} comes from the combination of the \textit{status constructus} of the proximal demonstrative \japhug{ki}{this} (§ \ref{sec:demonstrative.pronouns}) with the postposition \japhug{ɕɯŋgɯ}{before}. The phrases \forme{ki ɕɯŋgɯ} `before this' \forme{nɯ ɕɯŋgɯ} `before that' with the demonstratives \japhug{ki}{this} and  \japhug{nɯ}{that} are also attested.

While the neutral antonym of \japhug{ɕɯŋgɯ}{before} is the relator noun \japhug{ɯ-qʰu}{after} (§ \ref{sec:relator.temporal}), subsequent temporality can also be expressed by \japhug{ɕɯmɯma}{immediately after} and \japhug{kóʁmɯz}{only after} (one of the rare uninflected words with a non-final stress, § XXX). These postpositions are mainly attested following the demonstrative pronoun \forme{nɯ}, and often used adverbially as \japhug{nɯ ɕɯmɯma}{immediately} and \japhug{nɯ kóʁmɯz nɤ}{only then}, but are also attested with temporal clauses (§ XXX) and with temporal counted nouns or adverbs as in (\ref{ex:sqamNuxpa.koRmWz}), which also illustrates the opposition between \japhug{kóʁmɯz}{only after} and the terminate postposition \japhug{mɤɕtʂa}{until} (§ \ref{sec:terminative}. 

\begin{exe}
\ex \label{ex:sqamNuxpa.koRmWz}
\gll ɯ-xso ʑɴɢɯloʁ nɯnɯ, tɕe sqamŋu-xpa mɤɕtʂa ɯ-mat ku-tsʰoʁ mɯ́j-cʰa. pɯ́-wɣ-ji ɕɯmɯma, ɯ-rɣi pjɯ́-wɣ-ji ɯ-qʰu,  sqamŋu-xpa kóʁmɯz nɤ ɯ-mat ku-tsʰoʁ ŋu tu-ti-nɯ ŋgrɤl tɕe,  \\
\textsc{3sg}.\textsc{poss}-normal walnut  \textsc{dem} \textsc{lnk} fifteen-year until \textsc{3sg}.\textsc{poss}-fruit \textsc{ipfv}-attach \textsc{neg}:\textsc{sens}-can \textsc{pfv}-\textsc{inv}-plant just.after \textsc{3sg}.\textsc{poss}-seed \textsc{ipfv}-\textsc{inv}-plant \textsc{3sg}.\textsc{poss}-after fifteen-year only.after \textsc{lnk} \textsc{3sg}.\textsc{poss}-fruit \textsc{ipfv}-attach be:\textsc{fact} \textsc{ipfv}-say-\textsc{pl} be.usually.the.case:\textsc{fact} \textsc{lnk} \\
\glt `Usually, the walnut tree cannot have walnuts until fifteen years (have passed). Just after one has planted it, after one plants its seeds, it is only fifteen years later that it bears nuts, they say.' (12-ndZiNgri, 166-167)
\end{exe}

There is an adverb \japhug{nóʁmɯz}{only then} (found for instance in \ref{ex:tWqioR.kW}, § \ref{sec:oblique.kW}) whose meaning is identical to \japhug{nɯ kóʁmɯz nɤ}{only then}, and apparently results from the fusion of the demonstrative \forme{nɯ} with the root \forme{-oʁmɯz}. The origin of the \forme{k-} element in \japhug{kóʁmɯz}{only after} is unclear; it could be the fused form of the proximal demonstrative \forme{ki} (§ \ref{sec:anaphoric.demonstrative.pro}).


The postposition \japhug{pɕɯntɕɤt}{since} (from \tibet{ཕྱིན་ཆད་}{pʰʲin.tɕʰad}{thereafter}) can follow a date (\ref{ex:69nian.pCintCAt}) or a subordinate clause (see § XXX), and is most often used with the relator noun \japhug{ɯ-qʰu}{after} to mean `from that time on' as in (\ref{ex:nW.Wqhu.pCintCAt}). Another postposition,  \japhug{ɕaŋpɕi}{from ... on} (combining the \forme{ɕaŋ-} element found in egressive postpositions § \ref{sec:egressive}  with \forme{-pɕi} from Tibetan \tibet{ཕྱི་}{pʰʲi}{later}) can be used like \japhug{pɕɯntɕɤt}{since} following  \japhug{ɯ-qʰu}{after} as in (\ref{ex:Wqhu.CaNpCi}), or after temporal clauses (§ XXX). The two postpositions \japhug{pɕɯntɕɤt}{since} and \japhug{ɕaŋpɕi}{from ... on} have the same meaning, but the latter is considered by Tshendzin to be an influence from the xtokavian dialects (§ XXX).

 \begin{exe}
\ex \label{ex:69nian.pCintCAt}
 \gll tɕe tʰam kɯβdesqi ɯ-ro to-pa ma <liu.jiu.nian> pɕɯntɕɤt \\
 \textsc{lnk} now fourty \textsc{3sg}.\textsc{poss}-excess \textsc{ifr}-pass.X.years \textsc{lnk}  1969 since \\
 \glt `Now it has been fourty years (we have known each other), since 1969.' (12-BzaNsa, 13)
 \end{exe}
 
  \begin{exe}
\ex \label{ex:nW.Wqhu.pCintCAt}
 \gll  tɕiʑo pɯ-ari-tɕi ŋu, mtsʰukʰa pɯ-ftɕɤt-tɕi ŋu, nɯ ɯ-qʰu pɕɯntɕɤt tɕe,  nɯ-ʁgra nɯ nɯ-me ŋu \\
 \textsc{1du} \textsc{pfv}:\textsc{down}-go[II]-\textsc{1du} be:\textsc{fact} lake pfv-subdue-1du be:\textsc{fact} \textsc{dem} \textsc{3sg}.\textsc{poss}-after since \textsc{lnk} \textsc{2pl}.\textsc{poss}-enemy \textsc{dem} \textsc{pfv}-not.exist be:\textsc{fact} \\
 \glt `We went down (into the lake), subdued the (demons in) the lake, and from that time on, your enemy is no more.' (Nyima.'Odzer2003.2, 109-110)
 \end{exe}
 
 \begin{exe}
\ex \label{ex:Wqhu.CaNpCi}
\gll  tɕe tɤ-mu nɯ, nɯ ɯ-qʰu ɕaŋpɕi ʑo kɤ-rɯndzɤqʰɤjɯ ta-znɯna \\
\textsc{lnk} \textsc{indef}.\textsc{poss}-mother \textsc{dem} \textsc{dem} \textsc{3sg}.\textsc{poss}-after from.that.time.on \textsc{emph} \textsc{inf}-eat.without.sharing \textsc{pfv}:3\fl{}3'-stop \\
\glt `From that (time) on, the mother stopped to eat on her own without sharing.' (tWJo2005, 53) 
\end{exe}

The postposition \japhug{jɤz}{when} and its variant \japhug{jɤznɤ}{when} follows either subordinate clauses (§ XXX), temporal adverbs (\ref{ex:jWfCWr.jAznA}) or the noun \japhug{ɯ-ŋgu}{beginning} (from \tibet{འགོ་}{go}{head, beginning}) as in (\ref{ex:jWfCWr.jAznA}); it is not attested with other noun phrases.

\begin{exe}
\ex \label{ex:jWfCWr.jAznA}
\gll  jɯfɕɯr nɯtɕu iɕqha tɤtɕɯpɯ nɯnɯra, jɯfɕɯr jɤznɤ tu-ndze-a tɕe pɯ-apa wo ri \\ 
yesterday \textsc{dem}:\textsc{loc} the.aforementioned boy \textsc{dem}:\textsc{pl} yesterday when \textsc{ipfv}-eat[III]-\textsc{1sg}  \textsc{lnk} \textsc{pst}.\textsc{ipfv}-be.correct \textsc{sfp} \textsc{lnk} \\
\glt  `I should have eaten these boys yesterday.' (160705 poucet5-v2, 36)
\end{exe}

 \begin{exe}
\ex \label{ex:WNgu.jAznA}
\gll  tɕe ɯ-ŋgu jɤznɤ tɕe, ɯ-mɤlɤjaʁ nɯra mɯ-cʰɯ-pʰaʁ-nɯ tɕe,  tɕe nɯ ɣɯ ɯ-ndʐi kɯ-fsɯ\redp{}fse nɯ pjɯ-qaʁ-nɯ tɕe tɕe \\
\textsc{lnk} \textsc{3sg}.\textsc{poss}-beginning when \textsc{lnk} \textsc{3sg}.\textsc{poss}-limb \textsc{dem}:\textsc{pl} \textsc{neg}-\textsc{ipfv}-cut-\textsc{pl} \textsc{lnk} \textsc{lnk} \textsc{dem} \textsc{gen} \textsc{3sg}.\textsc{poss}-skin \textsc{nmlz}:S/A-\textsc{emph}\redp{}be.like \textsc{dem} \textsc{ipfv}-remove.skin-\textsc{pl} \textsc{lnk} \textsc{lnk} \\
\glt `In the beginning, they don't cut off the limbs (from the cattle's body), and take out the skin (in such a way as to preserve its shape) exactly like (that of the living animal).' (06-BGa, 94)
\end{exe}


 
\section{Relator nouns}  \label{sec:relator.nouns}  
Relator nouns are IPNs (§ \ref{sec:inalienably.possessed}) used to mark the grammatical relations of oblique arguments or adjuncts. They differ from postpositions by at least three properties. 

First, they have an obligatory possessive prefix, which is coreferent with the preceding noun phrase or clause when one is present (generally the third person singular \forme{ɯ-} prefix). 

Second, unlike postpositions, which require at the very least a demonstrative (§ \ref{ex:postpositions}), they can occur without a preceding noun phrase or clause if the referent indicated by the possessive prefix is definite (including first or second person, as in \ref{ex:aCki.jAGe}) or generic (see example \ref{ex:tWCki} in § \ref{sec:indef.genr.poss}). 

\begin{exe}
\ex \label{ex:aCki.jAGe}
\gll jɯfɕɯr <gongxun> a-ɕki jɤ-ɣe. \\
yesterday p.n. \textsc{1sg}.\textsc{poss}-\textsc{dat} \textsc{pfv}-come[II] \\
\glt `Yesterday Gong Xun came to (see) me.' (160320, conversation)
\end{exe}

Third, the genitive \forme{ɣɯ} (§ \ref{sec:genitive}) can optionally occur between the preceding noun phrase and the relator (as in \ref{ex:GW.WCki.nWtCu} below and \ref{ex:GW.kWnA.WrkW.ri}), since relator noun phrases are a subtype of the possessive construction (§ \ref{sec:gen.possession}).

\begin{exe}
\ex \label{ex:GW.WCki.nWtCu}
\gll tɕe smɤt tɯmda rɟɤlpu ɣɯ ɯ-ɕki nɯtɕu, nɤkinɯ, ɯ-rʑaβ ɯ-kɯ-tʰu cʰɤ-ɕe tɕe, \\
\textsc{lnk} pl.n. pl.n. king \textsc{gen} \textsc{3sg}.\textsc{poss}-\textsc{dat} \textsc{dem}:\textsc{loc} \textsc{filler} \textsc{3sg}.\textsc{poss}-wife \textsc{3sg}.\textsc{poss}-\textsc{nmlz}:S/A-ask \textsc{ifr}:\textsc{downstream}-go \textsc{lnk} \\
\glt `He went to ask the king of the lower valley for (one his daughter to take as) a wife.' (2014-kWlAG, 16)
\end{exe}

In addition, most relator nouns still preserve non-grammaticalized uses revealing their diachronic source, and some of them can be followed by the locative postpositions (§ \ref{sec:core.locative}) or by other relator nouns.

\subsection{Dative} \label{sec:dative} 
Two dative markers are attested in Japhug, \forme{ɯ-ɕki} and \forme{ɯ-pʰe}; some speakers like Tshendzin prefer the former (as in \ref{ex:aCki.jAGe}, \ref{ex:WCki.zW}, \ref{ex:WCki.toti}), but most speakers I have recorded favor the latter (for instance, Kunbzang Mtsho who tells the story from which \ref{ex:Wphe.toti} is taken).

The dative can be followed by the locative postpositions \forme{zɯ} and \forme{tɕu}, as in (\ref{ex:WCki.zW}), (\ref{ex:nWCki.zYArNo}) and (\ref{ex:slANe.ZNgri.ra.nWphe}).


\begin{exe}
\ex \label{ex:WCki.zW}
\gll nɯ mbro tɯ-skɤt kɯ-tso nɯnɯ ɯ-ɕki zɯ .... to-ti \\
\textsc{dem} horse \textsc{indef}.\textsc{poss}-speech \textsc{nmlz}:S/A-understand \textsc{dem} \textsc{3sg}.\textsc{poss}-\textsc{dat} \textsc{loc} { } \textsc{ifr}-say \\
\glt `She said ... to the horse who could understand speech' (2003kAndzWsqhaj, 25)
\end{exe}

The dative is used to mark the recipient or addressee. It occurs with indirective verbs of speech such as \japhug{ti}{say} (\ref{ex:WCki.zW}, \ref{ex:WCki.toti} and \ref{ex:Wphe.toti}), \japhug{fɕɤt}{tell} and \japhug{tʰu}{ask} (\ref{ex:nWCki.tAthe}), and also with some intransitive verbs of speech such as \japhug{rɯɕmi}{speak} (\ref{ex:WCki.torWCmi}).

\begin{exe}
\ex \label{ex:WCki.toti}
\gll iɕqʰa srɯnmɯ nɯ kɯ, [...] smɤnmimitoʁ kuɕana ɯ-ɕki `nɤʑo tɕʰi ɯ-rɯɣ tɯ-ŋu' to-ti ri, \\
the.aforementioned râkshasî \textsc{dem} \textsc{erg} { } p.n. p.n. \textsc{3sg}.\textsc{poss}-\textsc{dat} \textsc{2sg} what \textsc{3sg}.\textsc{poss}-race 2-be:\textsc{fact} \textsc{ifr}-say \textsc{lnk} \\
\glt `The râkshasî asked Smanmi Metog Koshana, `What type of being are you?' (28-smAnmi, 378)
\end{exe}

\begin{exe}
\ex \label{ex:Wphe.toti}
\gll tɕe tɤ-tɕɯ nɯ kɯ ɯ-wa ɯ-pʰe nɯra pɯ-kɯ-fse nɯra to-ti ɲɯ-ŋu \\
\textsc{lnk} \textsc{indef}.\textsc{poss}-son \textsc{dem} \textsc{erg} \textsc{3sg}.\textsc{poss}-father \textsc{3sg}.\textsc{poss}-\textsc{dat} \textsc{dem}.\textsc{pl} \textsc{pfv}-\textsc{nmlz}:S/A-be.like \textsc{dem}.\textsc{pl}  \textsc{ifr}-say \textsc{sens}-be \\
\glt `The boy told his father the things that had happened.' (qachGa2012, 175)
\end{exe}

\begin{exe}
\ex \label{ex:nWCki.tAthe}
\gll nɤʑo ɯ-mɤ-ɲɯ-tɯ-stu nɤ, ʑara nɯ-ɕki tɤ-tʰe jɤɣ \\
\textsc{2sg} \textsc{qu}-\textsc{neg}-\textsc{sens}-2-believe \textsc{lnk} \textsc{3pl} \textsc{3pl}.\textsc{poss}-\textsc{dat} \textsc{imp}-ask[III] be.possible:\textsc{fact} \\
\glt `If you don't believe it, ask them!' (140508 shier ge tiaowu de gongzhu-zh, 190)
\end{exe}

\begin{exe}
\ex \label{ex:WCki.torWCmi}
\gll nɯ tɤ-pɤtso nɯ ɯ-ɕki to-rɯɕmi. \\
\textsc{dem} \textsc{indef}.\textsc{poss}-child \textsc{dem} \textsc{3sg}.\textsc{poss}-\textsc{dat} \textsc{ifr}-speak \\
\glt  `It spoke to the child.' (150831 renshen wawa-zh, 36)
\end{exe}

It also occurs with verbs of giving to mark the recipient as in (\ref{ex:Wphe.tokho}) and (\ref{ex:WCki.YAkho}) with the verb \japhug{kʰo}{give, pass over}, but also the source as in (\ref{ex:nWCki.zYArNo}) with verbs such as \japhug{rŋo}{borrow from} and \japhug{sɤmbi}{ask for}.

\begin{exe}
\ex \label{ex:Wphe.tokho}
\gll laʁjɯɣ nɯ ɯ-taʁ nɯ ɯ-pʰe to-kʰo tɕe,  \\
staff \textsc{dem} \textsc{3sg}.\textsc{poss}-on \textsc{dem} \textsc{3sg}.\textsc{poss}-\textsc{dat} \textsc{ifr}:\textsc{up}-give \textsc{lnk} \\
\glt `He gave the staff to the (thief) who was on (the tiger).' (khu2012, 15)
\end{exe}

\begin{exe}
\ex \label{ex:WCki.YAkho}
\gll  ɯ-nmaʁ ɯ-ɕki ɲɤ-kʰo tɕe,  \\
\textsc{3sg}.\textsc{poss}-husband \textsc{3sg}.\textsc{poss}-\textsc{dat} \textsc{ifr}-give \textsc{lnk} \\
\glt `She gave it to her husband.' (qajdoskAt, 71)
\end{exe}

\begin{exe}
\ex \label{ex:nWCki.zYArNo}
\gll 
kɯ-rɤrma ra nɯ-ɕki nɯtɕu, kuxtɕo ci z-ɲɤ-rŋo, \\
\textsc{nmlz}:S/A-work \textsc{pl} \textsc{3pl}.\textsc{poss}-\textsc{dat} \textsc{dem}:\textsc{loc} basket \textsc{indef} \textsc{transloc}-\textsc{ifr}-borrow \\
\glt `(The snow leopard) borrowed a basket from the workers.' (qala2002, 43)
\end{exe}

With the verb \japhug{kʰo}{give, pass over} the recipient is more often encoded with the genitive or a possessive prefix on the theme (§ \ref{sec:gen.beneficiary}) or with the semi-grammaticalized noun \japhug{tɯ-jaʁ}{hand} (§ \ref{sec:semi.grammaticalized.relator}).

 
The semi-transitive verb \japhug{ru}{look at} can mark its goal with the dative, as in (\ref{ex:WCki.Cturu}); this is however optional, as this verbs also takes goals in the absolutive (§ \ref{absolutive.goal}) or locative (§ \ref{sec:locative}).

\begin{exe}
\ex \label{ex:WCki.Cturu}
\gll tɕe tɤŋe nɯ nɯɣ-me tɕe, tɕe tɤŋe ɯ-ɕki ʁɟa ʑo ɕ-tu-ru tɕe, tɯʑo tɯ-ɕki maka ʑo mɤ-ru \\
\textsc{lnk} sun \textsc{dem} \textsc{appl}-be.afraid[III]:\textsc{fact}-\textsc{1sg} \textsc{lnk} \textsc{lnk} sun \textsc{3sg}.\textsc{poss}-\textsc{dat} completely \textsc{emph} \textsc{transloc}-\textsc{ipfv}:\textsc{up}-look.at \textsc{lnk} \textsc{genr} \textsc{genr}.\textsc{poss}-\textsc{dat} at.all \textsc{emph} \textsc{neg}-look.at:\textsc{fact} \\
\glt `(If the yeti catches you), it is afraid of the sun, it looks at the sun the whole time, and does not look at you.' (140510 mYWrgAt, 13)
\end{exe}

The dative \forme{ɯ-ɕki} derives from a relator noun meaning `side', `near' or `at X's place' (with or without motion). These locative meanings are still marginally present in Japhug in examples like (\ref{ex:aCki.jAGe}) above and (\ref{ex:WCki.loc}), (\ref{ex:slANe.ZNgri.ra.nWphe}) and (\ref{ex:WCki.kunWrAZi}) below.

\begin{exe}
\ex \label{ex:WCki.loc}
\gll  ɯ-rte nɯ ɯ-rna ɯ-ɕki pɯ-kɯ-ɴqoʁ nɯnɯ pjɤ-mɟa tɕe ɯ-ku ɯ-taʁ to-ta \\
\textsc{3sg.poss}-hat \textsc{dem} \textsc{3sg.poss}-ear \textsc{3sg}-\textsc{dat} \textsc{pfv:down-nmlz}:S/A-hang \textsc{dem} \textsc{ifr:down}-take \textsc{lnk} \textsc{3sg.poss}-head \textsc{3sg}-on \textsc{ifr}-put \\
\glt `He took the hard that was hanging on his ear and put it on his head.' (140505 liuhaohan zoubian tianxia, 164)
\end{exe}

\begin{exe}
\ex \label{ex:slANe.ZNgri.ra.nWphe}
\gll   tɤŋe cʰo slɤŋe ʑŋgri ra nɯ-pʰe nɯtɕu kɤ-nɤɕqa a-pɯ-tɯ-cʰa ra ma, \\
sun \textsc{comit} moon star \textsc{pl} \textsc{3pl}.\textsc{poss}-\textsc{dat} \textsc{dem}:\textsc{loc} \textsc{inf}-bear \textsc{irr}-\textsc{pfv}-2-can have.to:\textsc{fact} \textsc{lnk} \\
\glt `(When you are) by the sun, the moon and the stars, you will have to bear (the heat and the cold), otherwise...' (2003kandZislama, 53)
\end{exe}
 
\begin{exe}
\ex \label{ex:WCki.kunWrAZi}
\gll  li tɕɤtu tɤ-ɣe qʰe, ɯ-wa ɯ-ɕki ku-nɯ-rɤʑi, tɕɤki pɯ-ari qʰe, ɯ-wɯ ɯ-wi ni ndʑi-ɕki ju-nɯ-ɕe qʰe, \\
again up.there \textsc{pfv}:\textsc{up}-go[II] \textsc{lnk} \textsc{3sg}.\textsc{poss}-father \textsc{3sg}.\textsc{poss}-\textsc{dat} \textsc{ipfv}-\textsc{auto}-stay, down.there \textsc{pfv}-go[II] \textsc{lnk} \textsc{3sg}.\textsc{poss}-grand.father \textsc{3sg}.\textsc{poss}-grand.mother \textsc{du} \textsc{3du}.\textsc{poss}-\textsc{dat} \textsc{ipfv}-\textsc{auto}-go \textsc{lnk}  \\
\glt `When she comes up there, she stays at her father's house, and when she goes down there, she goes to her grandparent's place.' (14-tApitaRi, 305)
\end{exe}

\subsection{Secutive} \label{sec:secutive} 
The secutive relator noun \japhug{ɯ-rca}{following} is used with verbs of motion such as \japhug{gi}{come} to express the meaning `follow', `come/go with' as in (\ref{ex:nArca.Gia}).

\begin{exe}
\ex \label{ex:nArca.Gia}
 \gll  aʑo kɯnɤ nɤ-rca ɣi-a ɕti \\
 \textsc{1sg} also \textsc{2sg}.\textsc{poss}-following come:\textsc{fact}-\textsc{1sg} be.\textsc{affirm}:\textsc{fact} \\
\glt `I am coming/going with you.' (2011-05-nyima, 171)
\end{exe}

The secutive can have a meaning similar to that of the comitative adverb (§ \ref{sec:comitative.adverb}) `together with X', as in (\ref{ex:WBGi.Wrca}).

\begin{exe}
\ex \label{ex:WBGi.Wrca}
 \gll  pɤnmawombɤr ɣɯ ɯ-ɕɤrɯ ɯ-βɣi ɯ-rca tsʰɯntsʰɯn ʑo ta-wum-nɯ ɲɯ-ŋu \\ 
p.n. \textsc{gen} \textsc{3sg}.\textsc{poss}-bone \textsc{3sg}.\textsc{poss}-ash \textsc{3sg}.\textsc{poss}-following \textsc{idph}:II:neat \textsc{emph} \textsc{pfv}:3\fl{}3'-collect-\textsc{pl} \textsc{sens}-be  \\
\glt `They collected all of Padma 'Od-'bar's bones together with his ashes.' (Norbzang2005, 410)
\end{exe}

The secutive phrase can follow  (\ref{ex:WBGi.Wrca}), or precede (\ref{ex:tWjAGAt.Wrca}) the noun phrase it accompanies.

\begin{exe}
\ex \label{ex:tWjAGAt.Wrca}
 \gll   tɯ-jɤɣɤt ɯ-rca tɤ-se cʰɯ-nɯ-ɬoʁ \\
 \textsc{indef}.\textsc{poss}-feces \textsc{3sg}.\textsc{poss}-following \textsc{indef}.\textsc{poss}-blood \textsc{ipfv}:\textsc{downstream}-\textsc{auto}-come.out \\
 \glt `(In the case of this disease), blood comes out together with the feces.' 
 \end{exe}

The secutive with third person singular possessive \forme{ɯ-rca} is also used as a linker meaning `in addition' (see § XXX). With the indefinite possessive prefix \forme{tɤ-rca} and  \forme{tɯ-tɯ-rca}, the secutive appears in adverbial function with the meaning `together' (§ XXX), though in examples such as  (\ref{ex:tArAku.tArca}) the form  \forme{tɤ-rca} retains its nominal status.


\begin{exe}
\ex \label{ex:tArAku.tArca}
 \gll ɕoʁ nɯnɯ tɤ-rɤku tɤ-rca ŋu, sɯjno maʁ. \\
 buckwheat \textsc{dem} \textsc{indef}.\textsc{poss}-crops \textsc{indef}.\textsc{poss}-following be:fact grass not.be:fact \\
 \glt `Buckwheat (belongs) with the crops, it is not a (type of) grass.' (13-NanWkWmtsWG, 68)
\end{exe}

In addition, the unexpected focus marker \forme{rcanɯ} (§ \ref{sec:unexpected}) and the dubitative sentence final particle \forme{rca} (§ XXX) are historically related to the secutive.

\subsection{Deputative} \label{sec:deputative} 
The IPN \forme{ɯ-tsʰɤt} has two meanings. First, it can serve as a deputative relator noun `instead of, on behalf of' as in (\ref{ex:nWtAsno.WtshAt}) and (\ref{ex:nWsi.WtshAt}). No verb selects this relator noun. 

The deputative adjunct can correspond to the intransitive subject (as in \ref{ex:nWtAsno.WtshAt}, with the verb \japhug{tu}{exist}), the transitive subject (as in \ref{ex:aZo.nAtshAt}, with \japhug{ɣɯjtsi}{support}) or the object.

\begin{exe}
\ex \label{ex:nWtAsno.WtshAt}
\gll nɯʑora ɣɯ nɯ-tɤ-sno kɯ-fse ɯ-tsʰɤt nɯ, tɕiʑo ɣɯ, tɕi-xɕɤndʑu χsɯ-ldʑa pɯ-tu tɕe, nɯnɯ lɤ-nɯ-βlɯ-tɕi ɕti wo \\
\textsc{2pl} \textsc{gen} \textsc{2pl}.\textsc{poss}-\textsc{indef}.\textsc{poss}-saddle \textsc{nmlz}:S/A-be.like \textsc{3sg}.\textsc{poss}-instead.of \textsc{dem} \textsc{1du} \textsc{gen} \textsc{1du}.\textsc{poss}-twig three-long.object \textsc{pst}.\textsc{ipfv}-exist \textsc{lnk} \textsc{dem} \textsc{pfv}-\textsc{auto}-burn-\textsc{1du} be.\textsc{affirm}:\textsc{fact} \textsc{sfp} \\
\glt `Instead of a saddle like yours, we had three twigs, this is what we burned.' (Kubzang2003, 203)
\end{exe}

\begin{exe}
\ex \label{ex:aZo.nAtshAt}
\gll aʑo nɤ-tsʰɤt, nɤki, si nɯ tu-ɣɯjtsi-a jɤɣ \\
\textsc{1sg} \textsc{2sg}.\textsc{poss}-instead.of \textsc{filler} tree \textsc{dem} \textsc{ipfv}-support-\textsc{1sg} be.possible:\textsc{fact} \\
\glt `I can support the tree for you/instead of you (while you fetch it).' (150830 afanti, 136)
\end{exe}

 The noun phrase headed by \forme{ɯ-tsʰɤt} can be either an adjunct as in (\ref{ex:nWtAsno.WtshAt}) and (\ref{ex:aZo.nAtshAt}), the object of the verb \japhug{βzu}{make}, or a nominal predicate with a copula as in (\ref{ex:nWsi.WtshAt}) and (\ref{ex:aZo.atshAt}).  In the latter case, to express the meaning `do to $X$ instead of to $Y$', a biclausal construction `do to $X$, ($X$) is instead of $Y$' is used as in (\ref{ex:aZo.atshAt}).

\begin{exe}
\ex \label{ex:nWsi.WtshAt}
\gll si maŋe tɕe tɕe nɯnɯtɕu tɕe, nɯ-si ɯ-tsʰɤt ɲɯ-ŋu  \\
tree not.exist:\textsc{sens} \textsc{lnk} \textsc{lnk} \textsc{dem}:\textsc{loc} \textsc{lnk} \textsc{3pl}.\textsc{poss}-wood \textsc{3sg}.\textsc{poss}-instead.of \textsc{sens}-be \\
\glt `There no trees, there (dung) is used to replace the firewood.' (05-tamar, 10-11)
\end{exe}


\begin{exe}
\ex \label{ex:aZo.atshAt}
\gll nɯ tɤ-nɯ-ndɤm tɕe aʑo a-tsʰɤt ŋu tɕe \\
\textsc{dem} \textsc{imp}-\textsc{auto}-take[III] \textsc{lnk} \textsc{1sg} \textsc{1sg}.\textsc{poss}-instead.of be:\textsc{fact} \textsc{lnk} \\
\glt `Take these instead of me (as a compensation).' (2003kAndzwsqhaj2, 141)
\end{exe}

The examples (\ref{ex:aZo.nAtshAt}) and (\ref{ex:aZo.atshAt}) also show that the relator noun \japhug{ɯ-tsʰɤt}{instead of} can occur with a first or second person possessive prefix.

Second, \forme{ɯ-tsʰɤt} also means `with proper measure', mainly occurring in adverbial function as in (\ref{ex:WtshAt.tsa}) or in collocation with the verb \japhug{βzu}{make} in the sense `do with proper measure' as in (\ref{ex:WtshAt.tusWBzunW}). 

\begin{exe}
\ex \label{ex:WtshAt.tsa}
\gll rkaŋraŋ ɯ-tsʰɤt tsa ɲɯ-kɯ-nɤɕtʂaʁli-a-nɯ raʁmaʁ ma  \\
p.n. \textsc{3sg}.\textsc{poss}-proper.measure a.little \textsc{ipfv}-2\fl{}1-torture-\textsc{1sg}-\textsc{pl} \textsc{sfp} \textsc{lnk}  \\
\glt `Rkangrang, your torturing of me should have a limit.' 
\end{exe}

\begin{exe}
\ex \label{ex:WtshAt.tusWBzunW}
\gll ɯ-tsʰɤt tu-sɯ-βzu-nɯ mɯ́j-kʰɯ ma, nɯ-kɤ-kʰo nɯ mɯ-tʰa-ɕkɯt mɤɕtʂa tu-ndze ɲɯ-ɕti. \\
\textsc{3sg}.\textsc{poss}-proper.measure \textsc{ipfv}-\textsc{caus}-make-\textsc{pl} \textsc{neg}:\textsc{sens}-be.possible \textsc{pfv}-\textsc{nmlz}:P-give \textsc{dem} \textsc{neg}-\textsc{pfv}:3\fl{}3'-eat.completely until \textsc{ipfv}-eat[III] \textsc{sens}-be.\textsc{affirm} \\
\glt `They cannot make (the monkey eat) with measure, as it continues eating the (food) that is given to it until there is none.' (19-GzW, 60)
\end{exe}

In some contexts as in (\ref{ex:nWnW.WtshAt}), \japhug{ɯ-tsʰɤt}{proper measure} in adverbial used is better translated as `depending on the circumstances'.\footnote{This example is taken from a text describing goats and sheep; goats are called \forme{tsʰɤt} in Japhug, but it is clear from the context that \forme{ɯ-tsʰɤt} cannot be the possessed form of this noun. }

\begin{exe}
\ex \label{ex:nWnW.WtshAt}
\gll tɕe nɯnɯ ɯ-tsʰɤt nɯnɯ ɯ-pɯ ci ci ʁnɯz tu, ci ci tɯ-rdoʁ ma me tɕe núndʐa ɲɯ-ŋu. \\
\textsc{lnk} \textsc{dem} \textsc{3sg}.\textsc{poss}-proper.measure \textsc{dem}  \textsc{3sg}.\textsc{poss}-young once once two exist:\textsc{fact} once once one-piece apart.from not.exist:\textsc{fact} \textsc{lnk} for.this.reason \textsc{sens}-be \\
\glt `This is why, depending on the circumstances, sometimes (the goat) has two youngs, sometimes only one.' (05-qaZo, 28)
\end{exe}

The IPN  \forme{ɯ-tsʰɤt} (at least in the meaning `proper measure') is borrowed from \tibet{ཚད་}{tsʰad}{measure, limit}. It occurs as second element in the compound \japhug{xtɤtsʰɤt}{restraint of one's appetite}(with the \textit{status constructus} \forme{xtɤ-} of \japhug{tɯ-xtu}{belly}).

\subsection{Locative relator nouns} \label{sec:relator.location}
The dearth of specific locative postpositions (§ \ref{sec:locative}) other than the egressive ones (§ \ref{sec:egressive}) in Japhug is compensated by the existence of many relator nouns expressing various types of location, as shown in Table \ref{tab:relator.location}. 

As other relator nouns, they can follow noun phrases (including headless relative clauses), with the possessive prefix agreeing in person and number with the preceding constituant, but can also occur on their own if the referent is definite, as \japhug{ɯ-taʁ}{on} in (\ref{ex:WtaR.zW.kAmdzW}) and (\ref{ex:WtaR.nWtCu.YAXtAr}) below.

\begin{exe}
\ex \label{ex:WtaR.zW.kAmdzW}
\gll  ɯ-taʁ zɯ kɤ-amdzɯ nɤ tɤ-lu pa-tɕɤt ɲɯ-ŋu\\
\textsc{3sg}.\textsc{poss}-on \textsc{loc} \textsc{pfv}-sit \textsc{lnk} \textsc{indef}.\textsc{poss}-milk \textsc{pfv}:3\fl{}3'-take.out \textsc{sens}-be\\
\glt `She sat on him and milked.' (Kunbzang2005, 81)
\end{exe}

 

\subsubsection{The tridimensional system} \label{sec:relator.nouns.3d}

The first six nouns in Table \ref{tab:relator.location} are derived from location adverbs (§ XXX). The vertical dimension relator nouns \japhug{ɯ-taʁ}{on} and \japhug{ɯ-pa}{below} are simply built by adding a possessive prefix to the root of the adverb, while the other ones combine the status constructus of the adverbial root (\forme{lɤ-}, \forme{tʰɤ-}, \forme{kɤ-}, \forme{ndɤ-} from \forme{lo}, \forme{tʰi}, \forme{kɯ} and \forme{ndi} respectively) with a suffix \forme{-cu}. This sexpartite system comprising three pairs of elements along three dimensions (vertical, fluvial, solar) is the same as found in verbal morphological (§ XXX) and also egressive postpositions (§ \ref{sec:egressive}).


\begin{table}
\caption{Locative relator nouns in Japhug} \label{tab:relator.location}
\begin{tabular}{lllll}
\lsptoprule
& Lexical origin \\
\midrule
\japhug{ɯ-taʁ}{on, above} & \japhug{taʁ}{up}\\
\japhug{ɯ-pa}{below, under} & \japhug{pa}{down}\\
\japhug{ɯ-lɤcu}{upstream of} & \japhug{lo}{upstream}\\
\japhug{ɯ-tʰɤcu}{downstream of} & \japhug{thi}{downstream}\\
\japhug{ɯ-kɤcu}{east of} & \japhug{kɯ}{east}\\
\japhug{ɯ-ndɤcu}{west of} & \japhug{ndi}{west}\\
\midrule
\japhug{ɯ-ku}{top of} & \japhug{tɯ-ku}{head}\\
\japhug{ɯ-qa}{bottom of} & \japhug{tɤ-qa}{paw,  root} \\
\japhug{ɯ-ʁɤri}{before, in front of} \\
\japhug{ɯ-qʰu}{after, behind} \\
\japhug{ɯ-ŋgɯ}{inside} \\
\japhug{ɯ-pɕi}{outside} &  \tibet{ཕྱི་}{pʰʲi}{outside}\\
\japhug{ɯ-rkɯ}{side} \\
\japhug{ɯ-χcɤl}{middle, center} & \tibet{དཀྱིལ་}{dkʲil}{middle}\\
\japhug{ɯ-pɤrtʰɤβ}{between} & \tibet{བར་}{bar}{middle, between}\\
\japhug{ɯ-tʰɤβ}{between} \\
\japhug{ɯ-mŋu}{opening, edge, border} \\
\japhug{ɯ-ndo}{edge, border} \\
\lspbottomrule
\end{tabular}
\end{table}


\subsubsection{Other locative relator nouns} \label{sec:other.locative.relator}

Outside of the tridimensional system, other locative relator nouns also occur in antithetic pairs, in particular \japhug{ɯ-ʁɤri}{before, in front of}  vs \japhug{ɯ-qʰu}{after, behind}  (the latter also occurs as a temporal relator noun, cf § \ref{sec:relator.temporal}), \japhug{ɯ-ŋgɯ}{inside} vs \japhug{ɯ-pɕi}{outside}  and  \japhug{ɯ-mŋu}{opening, edge, border}  vs \japhug{ɯ-ndo}{edge, border} or \japhug{ɯ-qa}{bottom of}.

A few of these relator nouns are borrowed from Tibetan, as indicated in Table \ref{tab:relator.location}), but most of them are native Gyalrong words. In particular \japhug{ɯ-ʁɤri}{before, in front of} is one of the very rare cases of a disyllabic words that has a cognate in Tangut sharing both syllables (\tangut{𘁞𗙷}{5416-567}{ɣwə-rjir}{2.25-2.74} `before').\footnote{The syllable \forme{ʁɤ-} = \tangut{𘁞}{5416}{ɣwə}{2.25} may be a fossilized allomorph of \japhug{tɯ-ku}{head} with a uvular as in the Stau cognate \stau{ʁə}{head}.} The relator noun \japhug{ɯ-ku}{top of} (as in \ref{ex:Wku.ri}) is transparently grammaticalized from the body part \japhug{tɯ-ku}{head}.

\begin{exe}
\ex \label{ex:Wku.ri}
 \gll si ɣɯ ɯ-mat kɯ\redp{}kɯ-tu nɯ ɯ-ku ri ɕ-ku-zo ɲɯ-ŋu tɕe. \\
 tree \textsc{gen} \textsc{3sg}.\textsc{poss}-fruit \textsc{total}\redp{}\textsc{nmlz}:S/A-exist \textsc{dem} \textsc{3sg}.\textsc{poss}-top \textsc{loc} \textsc{transloc}-\textsc{ipfv}-land \textsc{sens}-be \textsc{lnk} \\
 \glt `It lands on the top of all trees that have fruits.' (24-ZmbrWpGa, 43)
\end{exe}

Some relator nouns other than \japhug{ɯ-ku}{top of} have non-grammaticalized uses; for instance \japhug{ɯ-ŋgɯ}{inside} still occurs as a noun meaning  `internal part, inside' in (\ref{ex:WNgW.nW.so}).

\begin{exe}
\ex \label{ex:WNgW.nW.so}
 \gll tɕe nɯnɯ kɯ-spoʁ ŋu, ɯ-ŋgɯ nɯ so tɕe  \\
 \textsc{lnk} \textsc{dem} \textsc{nmlz}:S/A-have.a.hole be:\textsc{fact} \textsc{3sg}.\textsc{poss}-inside \textsc{dem} be.hollow:\textsc{fact} \textsc{lnk} \\
 \glt `Its (inside) has a hole, its inside is hollow.' (12-Zmbroko, 23)
\end{exe}

The meaning of the relator nouns \forme{ɯ-mŋu} and \forme{ɯ-ndo} requires a specific description. These nouns are not antithetic to another in all cases. The basic (non-grammaticalized) meaning of  \forme{ɯ-mŋu}  is the border of the opening or mouth of a container / bag (\ref{ex:khWtsa.WmNu}), or the shoreline (of a lake), as in (\ref{ex:mtshu.WmNu}). In this use, it is opposed to \japhug{ɯ-qa}{bottom of}, as in (\ref{ex:mtshu.Wqa}).

\begin{exe}
\ex \label{ex:khWtsa.WmNu}
\gll  tɕendɤre nɯnɯ kʰɯtsa ɯ-mŋu jamar kɯ-wxti tu,\\
\textsc{lnk} \textsc{dem} bowl \textsc{3sg}.\textsc{poss}-border about \textsc{nmlz}:S/A-be.big exist:\textsc{fact}\\
\glt  `Some are about as big as the mouth of a bowl.' (22-BlamajmAG, 130)
\end{exe}

\begin{exe}
\ex \label{ex:mtshu.WmNu}
\gll   tɕendre pɣɤtɕɯ nɯ mtsʰu ɯ-mŋu nɯtɕu `ʂɯt' to-ti to-nɯ-ɬoʁ.  \\
\textsc{lnk} bird \textsc{dem} lake \textsc{3sg}.\textsc{poss}-border \textsc{dem}:\textsc{loc} \textsc{idph}.I:sound \textsc{ifr}-say \textsc{ifr}:\textsc{up}-\textsc{auto}-come.out \\
\glt  `The bird came out of the shore of the lake with a noise.' (2014-kWlAG, 556)
\end{exe}

\begin{exe}
\ex \label{ex:mtshu.Wqa}
\gll  mtsʰu ɯ-qa zɯ nɤrwɯ mɤ-kɯ-naχtɕɯɣ ci ɣɤʑu tɕe, \\
lake \textsc{3sg}.\textsc{poss}-bottom \textsc{loc} jewel \textsc{neg}-\textsc{nmlz}:S/A-be.similar \textsc{indef} exist:\textsc{sens} \textsc{lnk} \\
\glt `At the bottom of the sea, there is a jewel unlike any other.' (2012Kubzang, 
\end{exe}

The basic meaning of \forme{ɯ-ndo} includes `extremity' (for instance, of a limb as in \ref{ex:WRar.Wndo}) and also `end' in both the locative and temporal sense (\ref{ex:Wndo.tCe}).

\begin{exe}
\ex \label{ex:WRar.Wndo}
\gll ɯ-ʁar ɯ-ndo nɯra hanɯni ɲɯ-ɲaʁ. \\ 
\textsc{3sg}.\textsc{poss}-wing \textsc{3sg}.\textsc{poss}-border \textsc{dem}:\textsc{pl} a.little \textsc{sens}-be.black \\
\glt  `The extremities of its wings are a bit black.' (23-scuz, 133)
\end{exe}

\begin{exe}
\ex \label{ex:Wndo.tCe}
\gll  ɯ-ndo tɕe maka kɯ-tu ɲɯ-me ɲɯ-ŋu tɕe,  \\
\textsc{3sg}.\textsc{poss}-border \textsc{lnk} at.all \textsc{nmlz}:S/A-exist \textsc{ipfv}-not.exist \textsc{sens}-be \textsc{lnk} \\
\glt `In the end, nothing is left.' (04-xiaocunzhuang-zh, 63)
\end{exe}

In the case of clothes, \forme{ɯ-ndo} refers to the lower opening (towards the feet), as opposed to the collar, as in (\ref{ex:Wndo.ri.kulAtnW}).

\begin{exe}
\ex \label{ex:Wndo.ri.kulAtnW}
\gll tɕe nɯ ɯ-ndʐi nɯnɯ pjɯ-χtsɤβ-nɯ tɕe tɕe tɯ-ŋga ɯ-ndo ri ku-lɤt-nɯ, tɯ-ŋga ɯ-kuŋa, ɯ-pɤloʁ ɯ-ku, ɯ-ndo nɯra ku-lɤt-nɯ,  tu-sɯ-fskɤr-nɯ ŋu.   \\
\textsc{lnk} \textsc{dem} \textsc{3sg}.\textsc{poss}-skin \textsc{dem} \textsc{ipfv}-tan-\textsc{pl} \textsc{lnk} \textsc{lnk} \textsc{indef}.\textsc{poss}-clothes \textsc{3sg}.\textsc{poss}-border \textsc{loc} \textsc{ipfv}-throw-\textsc{pl} \textsc{indef}.\textsc{poss}-clothes \textsc{3sg}.\textsc{poss}-collar \textsc{3sg}.\textsc{poss}-sleeves \textsc{3sg}.\textsc{poss}-top \textsc{3sg}.\textsc{poss}-border  \textsc{dem}:\textsc{pl}  \textsc{ipfv}-throw-\textsc{pl} \textsc{ipfv}-\textsc{caus}-surround-\textsc{pl} be:\textsc{fact} \\
\glt `They tan its hide (of the otter) and put it on the lower opening of the clothes, on the collar of clothes, the cuffs of the sleeves and the lower opening, and make it around (these openings).' (28-qapar, 96)
\end{exe} 


A contrast between \forme{ɯ-mŋu} and \forme{ɯ-ndo} occurs in their uses as relator nouns, indicating opposite extremities or sides. In the case of fields (as in \ref{ex:tWji.WmNu.Wndo}), \forme{ɯ-mŋu} designates the higher side of the field (towards the mountain), while \forme{ɯ-ndo} refers to the side closer to the river (all arable lands in Gyalrong area lie in narrow valleys).

\begin{exe}
\ex \label{ex:tWji.WmNu.Wndo}
\gll  qaʑmbri nɯ, tɯ-ji ɯ-ndo, tɯ-ji ɯ-mŋu nɯra aʁɤndɯndɤt ʑo tu-ɬoʁ ɕti \\
vine \textsc{dem} \textsc{indef}.\textsc{poss}-field \textsc{3sg}.\textsc{poss}-border \textsc{indef}.\textsc{poss}-field \textsc{3sg}.\textsc{poss}-border \textsc{dem}:\textsc{pl} everywhere \textsc{emph} \textsc{ipfv}:\textsc{up}-come.out be.\textsc{affirm}:\textsc{fact} \\
\glt `The vine, it grows everywhere, on both sides of the fields.' (06-qaZmbri, 12)
\end{exe}


The relator noun \forme{ɯ-mŋu}  can also designates the top extremity of stairs, as in (\ref{ex:rJAskAt.WmNu}); for the lower side, either \japhug{ɯ-qa}{bottom} or \forme{ɯ-ndo} can be used (the former more commonly).

\begin{exe}
\ex \label{ex:rJAskAt.WmNu}
\gll  cɯŋglɯɣ nɯ rɟɤskɤt ɯ-mŋu zɯ na-ta ɲɯ-ŋu \\
 pestle \textsc{dem} stairs \textsc{3sg}.\textsc{poss}-border \textsc{loc} \textsc{pfv}:3\fl{}3'-put \textsc{sens}-be \\
 \glt `He put the pestle on the top of the stairs.' (tWJo2005, 49)
\end{exe} 

The superlative derivation (§ \ref{sec:superlative.XCWX}) can be applied to most locative relator nouns, as \forme{ɯ-mŋuɕɯmŋu} from  \forme{ɯ-mŋu} in (\ref{ex:WmNuCWmNu}), where it means that the liquid completely fills the bowl to the point of touching the border of its mouth, flowing out at the slightest motion.

\begin{exe}
\ex \label{ex:WmNuCWmNu}
\gll  tʂʰa tɤ́-wɣ-rku tɕe kʰɯtsa ɯ-mŋuɕɯmŋu stʰɯci tu-zɣɯt mɤ-ra ma kɤ-ndo tɕe sɤ-ɕke \\
tea \textsc{pfv}-\textsc{inv}-put.in \textsc{lnk} bowl \textsc{3sg}.\textsc{poss}-border:\textsc{superlative} so.much \textsc{ipfv}:\textsc{up}-reach \textsc{neg}-have.to:\textsc{fact} \textsc{lnk} \textsc{inf}-take \textsc{lnk} \textsc{deexp}-burn:\textsc{fact} \\
\glt `When one pours tea, it should not reach the limit of the mouth of the bowl, otherwise it will be burning when one holds it.' (elicited)
\end{exe} 

In addition to the relator nouns described above, the IPN \japhug{ɯ-stu}{straight ahead} is mainly used as an adverb, but can also serve as a relator noun to indicate the goal of a motion verb as in (\ref{ex:Wstu.Zo.YAGi}).

\begin{exe}
\ex \label{ex:Wstu.Zo.YAGi}
\gll  tɕe pʰaʁrgot ri li ɯʑo ɯ-stu ʑo ɲɤ-ɣi qʰe,  ɕɤmɯɣdɯ kɤ-lɤt mɯ-pjɤ-nɤz qʰe pʰaʁrgot jo-nɯ-ɕe. \\
\textsc{lnk} boar also again \textsc{3sg} \textsc{3sg}.\textsc{poss}-straight \textsc{emph} \textsc{ifr}:\textsc{west}-come \textsc{lnk} gun \textsc{inf}-throw \textsc{neg}-\textsc{ifr}.\textsc{ipfv}-dare \textsc{lnk} boar \textsc{ifr}-\textsc{auto}-go \\
\glt `The boar came directly at him, but he did not dare to shoot and the boar went away.' (150829 phaRrgot, 8)
\end{exe} 
 
 
 
\subsubsection{Locative relator nouns and locative postpositions} \label{sec:relator.postposition.location}
 All locative relator nouns can be also used with the locative postpositions \forme{zɯ}, \forme{ri}, \forme{tɕu} and the fused forms of the latter two with the demonstratives \forme{nɯre} and \forme{nɯtɕu} (§ \ref{sec:locative}). Example (\ref{ex:WtaR.ri.Wpa.ri}) illustrates the use of \japhug{ɯ-taʁ}{on, above} and \japhug{ɯ-pa}{below, under} with the locative \forme{ri} expressing both static position (with the existential verb \japhug{tu}{exist}) and motion towards (with the verb \japhug{lɤt}{throw}, here specifically meaning `direct water').

\begin{exe}
\ex \label{ex:WtaR.ri.Wpa.ri}
\gll ɯ-tʰɤcu maŋtʰi qʰajŋgɯ nɯ kɯ, nɤki, βɣa ɯ-pa ri tɕʰɯŋkʰɤr tu tɕe, tɕʰɯŋkʰɤr ɯ-taʁ ri cʰɯ-lɤt tɕe, tɕʰɯŋkʰɤr ɯ-taʁ nɯre ri βɣɤrnɤjwaʁ kɤ-ti tu tɕe \\
\textsc{3sg}.\textsc{poss}-downstream downstream water.trough \textsc{dem} \textsc{erg} \textsc{filler} mill \textsc{3sg}.\textsc{poss}-under \textsc{loc} water.wheel exist:\textsc{fact} \textsc{lnk} water.wheel \textsc{3sg}.\textsc{poss}-on \textsc{loc} \textsc{ipfv}:\textsc{downstream}-throw \textsc{lnk} \textsc{lnk} water.wheel \textsc{dem}:\textsc{loc} \textsc{loc} blades \textsc{nmlz}:P-say exist:\textsc{fact} \textsc{lnk} \\
\glt `The inferior water trough -- under the mill there is a water wheel -- (the water trough) directs (the water) onto that water wheel -- on the water wheel there are things called `blades'.' (06-BGa, 27-31)
\end{exe}

Example (\ref{ex:Wpa.ri.tulhoR}) illustrates \japhug{ɯ-pa}{below, under} with the locative \forme{ri} expressing motion from a place.

\begin{exe}
\ex \label{ex:Wpa.ri.tulhoR}
\gll ɯ-tʰoʁ ɯ-pa ri tu-ɬoʁ nɯra mɯ́j-cʰa \\
\textsc{3sg}.\textsc{poss}-earth \textsc{3sg}.\textsc{poss}-below \textsc{loc} \textsc{ipfv}:\textsc{up}-come.out \textsc{dem}:\textsc{pl} \textsc{neg}:\textsc{sens}-can \\
\glt `(Its shoots) cannot come out from under the ground.'  (15-babW, 45)
\end{exe}

Examples (\ref{ex:khri.WtaR.zW}) and (\ref{ex:WtaR.zW.kAmdzW}) above show the combination of \japhug{ɯ-taʁ}{on, above} with the locative \forme{zɯ}, also for static position and motion.

\begin{exe}
\ex \label{ex:khri.WtaR.zW}
\gll χsɤr kʰri ɯ-taʁ zɯ pjɤ-rɤʑi tɕe ɯ-tɯ-ɣɤχsrɯ pjɤ-saχaʁ ʑo. \\
gold bed \textsc{3sg}.\textsc{poss}-on \textsc{loc} \textsc{ifr}.\textsc{ipfv}-stay \textsc{lnk} \textsc{3sg}.\textsc{poss}-\textsc{nmlz}:\textsc{degree}-be.handsome \textsc{ifr}.\textsc{ipfv}-be.extremely \textsc{emph} \\
\glt `He was sitting on the golden bed, very handsome.' (2014-kWlAG, 409)
\end{exe}

The uses are attested with the locative \forme{tɕu}, as shown by (\ref{ex:WtaR.nWtCu.pjArAZi}) and (\ref{ex:WtaR.nWtCu.YAXtAr}). No clear criterion accounting for the presence or absence of these locative postpositions in combination with the relator nouns has been found.

\begin{exe}
\ex \label{ex:WtaR.nWtCu.pjArAZi}
\gll  rɟɤmtsʰu ɣɯ ɯ-rkɯ qambɯt ɯ-taʁ nɯtɕu pjɤ-rɤʑi. \\
ocean \textsc{gen} \textsc{3sg}.\textsc{poss}-side sand \textsc{3sg}.\textsc{poss}-on \textsc{dem}:\textsc{loc} \textsc{ifr}.\textsc{ipfv}-stay \\
\glt `He stayed on the beach.' (140511 xinbada-zh, 167)
\end{exe}

\begin{exe}
\ex \label{ex:WtaR.nWtCu.YAXtAr}
\gll tɕe ɯ-taʁ nɯtɕu pɣɤmuj tɯ-spra nɯ ɲɤ-χtɤr. \\
\textsc{lnk} \textsc{3sg}.\textsc{poss}-on \textsc{dem}:\textsc{loc} feather one-handful \textsc{ifr}-scatter \\
\glt `He scattered a handful of feathers on it.' (28-smAnmi, 327)
\end{exe}

There is one case of a fossilized \forme{zɯ} locative (§ \ref{sec:core.locative}) with the relator noun \japhug{ɯ-ŋgɯ}{inside}, the form \japhug{ɯ-ŋgɯz}{inside, among}, which is used in particular to single out an element for a group (as in \ref{ex:kAndZWRi.nWNgWz}) or to describe an intermediate colour (with stative verbs, as in example \ref{ex:arNi.WNgWz}).

\begin{exe}
\ex \label{ex:kAndZWRi.nWNgWz} 
\gll kɤndʑɯʁi nɯ-ŋgɯz stu kɯ-xtɕi nɯnɯ kɯ ... nɯra ntsɯ tu-ti pjɤ-ŋu. \\
\textsc{coll}:siblings \textsc{3pl}.\textsc{poss}-among:\textsc{loc} most \textsc{nmlz}:S/A-be.small \textsc{dem} \textsc{erg} { } \textsc{dem}:\textsc{pl} always \textsc{ipfv}-say \textsc{ifr}.\textsc{ipfv}-be \\
\glt `The youngest among the sisters was always saying ...' (150828 donglang, 26)
  \end{exe}
  
  \begin{exe}
\ex \label{ex:arNi.WNgWz} 
\gll nɯ ɯ-mdoʁ nɯ aj kɤ-ti mɯ́j-spe-a ma arŋi ɯ-ŋgɯz kɯnɤ pɣi kɯ-fse   \\
\textsc{dem} \textsc{3sg}.\textsc{poss}-colour \textsc{dem} \textsc{1sg} \textsc{inf}-say \textsc{neg}:\textsc{sens}-be.able[III]-\textsc{1sg} \textsc{lnk} be.green:\textsc{fact} \textsc{3sg}.\textsc{poss}-inside:\textsc{loc} also be.grey:\textsc{fact} \textsc{nmlz}:S/A-be.like \\
\glt `I cannot say its colour, it is somewhere between green and grey.' (06-qaZmbri, 56)
    \end{exe}
    
 
 \subsection{Temporal relator nouns} \label{sec:relator.temporal}
Many temporal postpositions are found in Japhug (§ \ref{sec:egressive}, § \ref{sec:terminative} , § \ref{sec:temporal.postpositions}), and temporal relator nouns are relatively fewer. The relator  \japhug{ɯ-raŋ}{during, the time when} (from Tibetan \tibet{རིང་}{riŋ}{long, during, when}) is very commonly used with subordinate clauses (§ XXX), but can also follow temporal adverbs and nouns, as in (§ \ref{ex:XCitka.WraN}).
 
   \begin{exe}
\ex \label{ex:XCitka.WraN} 
\gll χɕitka ɯ-raŋ tɕe, tɕendɤre pɣa ra, nɯ-kɤ-ndza maka cʰɯ-me ɲɯ-ŋu tɕe, tɕendɤre ɯ-tɯ-mtsɯr pjɤ-saχaʁ ʑo ɲɯ-ŋu,  \\
spring \textsc{3sg}.\textsc{poss}-during \textsc{lnk} \textsc{lnk} bird \textsc{pl} \textsc{3sg}.\textsc{poss}-\textsc{nmlz}:P-eat at.all \textsc{ipfv}-not.exist \textsc{sens}-be \textsc{lnk} \textsc{lnk} \textsc{3sg}.\textsc{poss}-\textsc{nmlz}:\textsc{degree}-be.hungry \textsc{ifr}.\textsc{ipfv}-be.extremely \textsc{emph} \textsc{sens}-be \\
\glt `In spring, the birds' food has gone out and (that crow) was extremely hungry.' (kWjujmAlu, 6)
\end{exe}

The IPN \forme{ɯ-rɤɣ}, whose basic meaning is `specific (and predictable) time' as (\ref{ex:WrAG.GAZu}), can be used as relator noun as in (\ref{ex:nW.WrAG.tCe.li}) to mean `the exact time when'.

\begin{exe}
\ex \label{ex:WrAG.GAZu}
\gll ɯ-rɤɣ ɣɤʑu ma nɤkinɯ sɲikuku ʑo ɲɯ-maʁ \\
\textsc{3sg}.\textsc{poss}-specific.time exist:\textsc{sens} \textsc{lnk} filler every.day \textsc{emph} \textsc{sens}-not.be \\
\glt `(The rut) occurs at a specific time, not every day.' (27-qartshAz, 155)
\end{exe}

\begin{exe}
\ex \label{ex:nW.WrAG.tCe.li}
\gll tɕendɤre ɯ-fsaqʰe tɕe nɯ ɯ-rɤɣ tɕe li lo-ɣi \\
\textsc{lnk} \textsc{3sg}.\textsc{poss}-next.year \textsc{lnk} \textsc{3sg}.\textsc{poss}-specific.time lnk again \textsc{ifr}:\textsc{upstream}-come \\
\glt `It came back at exactly the same time the next year.' (22-qomndroN, 46)
\end{exe}


There are two relator nouns that can be used as antonyms of the preposition \japhug{ɕɯŋgɯ}{before} (§ \ref{sec:temporal.postpositions}),  \japhug{ɯ-mpʰru}{after, following} (from \tibet{འཕྲོ་}{ⁿpʰro}{remnant}, in particular in expressions such as \tibet{དེའི་འཕྲོར་}{deɦi.ⁿpʰror}{next, after that}) and \japhug{ɯ-qʰu}{after}, which is also used for spatial relations (§ \ref{sec:other.locative.relator}). These two nouns are not synonymous. The former must be used in the expression \japhug{ci ɯ-mpʰru ci}{one after the other} (\ref{ex:ci.Wmphru.ci}), and means `next' as in (\ref{ex:nW.Wmphru.tCe}), and can follow a subordinate clause, meaning `just after' (§ XXX), though most commonly a demonstrative pronoun such as \forme{nɯ} or \forme{nɯnɯ} referring to the previous clause is used instead, as in (\ref{ex:nWnW.Wmphru.tWsla}).

\begin{exe}
\ex \label{ex:ci.Wmphru.ci}
 \gll ci ɯ-mpʰru ci ʑo ko-nɯpoʁ qʰe lo-sɯ-ɣe. \\
 one \textsc{3sg}.\textsc{poss}-after one \textsc{emph} \textsc{ifr}-kiss \textsc{lnk} \textsc{ifr}:\textsc{upstream}-\textsc{caus}-come  \\
\glt `(The mother) kissed (her children) one after the one and had them come (inside the house).' (160701 poucet2, 38)
\end{exe}

\begin{exe}
\ex \label{ex:nW.Wmphru.tCe}
 \gll  tɕe nɯ ɯ-mpʰru tɕe tɕʰi to-ti? \\
 \textsc{lnk} \textsc{dem} \textsc{3sg}.\textsc{poss}-after \textsc{lnk} what \textsc{ifr}-say \\
\glt `What does it say next?' (140522 Kamnyu zgo)
\end{exe}

\begin{exe}
\ex \label{ex:nWnW.Wmphru.tWsla}
 \gll tɕe tɯ-rdoʁ pjɤ-sat tɕe tɕeri nɯnɯ ɯ-mpʰru tɯ-sla jamar nɯnɯ tɯ-ci ɯ-taʁ nɯtɕu ɯ-zda nɯ lo-ɕe nɤ cʰɤ-ɣi, lo-ɕe nɤ cʰɤ-ɣi ɲɤ-ɕar \\
\textsc{lnk} one-piece \textsc{ifr}-kill \textsc{lnk} \textsc{lnk} \textsc{dem} \textsc{3sg}.\textsc{poss}-after one-month about \textsc{dem} \textsc{indef}.\textsc{poss}-water \textsc{3sg}.\textsc{poss}-on \textsc{dem}:\textsc{loc} \textsc{3sg}.\textsc{poss}-mater \textsc{dem} \textsc{ifr}:\textsc{upstream}-go \textsc{lnk} \textsc{ifr}:\textsc{downstream}-come  \textsc{ifr}:\textsc{upstream}-go \textsc{lnk} \textsc{ifr}:\textsc{downstream}-come \textsc{ifr}-search \\
\glt `(Someone) killed one of them, and after that, (the other one) flew along the river searching for its mate for about one month.' (22-qomndroN, 44)
\end{exe}

The relator \japhug{ɯ-qʰu}{after} occurs more often after temporal clauses (§ XXX), though it can also be used after noun phrases as in the expression \japhug{saχsɯ ɯ-qʰu}{after lunch, afternoon}, or simply following a demonstrative as in \japhug{nɯ ɯ-qʰu}{after that}.
%tɕendɤre χsɯ-xpa ɯ-qhu tɕe tɕendɤre,
%nɯnɯ nɤki, ɯ-sloχpɯn ɣɯ ɯ-laχɕi nɯ lonba ko-spa.
%150902 luban, 149

\subsection{Semi-grammaticalized relator nouns} \label{sec:semi.grammaticalized.relator} 
The noun \japhug{tɯ-jaʁ}{hand} occurs with several verbs in fixed collocation, the recipient of the action being indexed by the possessive prefix on this noun.

It is found with \japhug{kʰo}{give} to express the meaning `hand over to' as in (\ref{ex:ajaR.tAkhAm}) and (\ref{ex:nAjaR.YWkhoj}).

\begin{exe}
\ex \label{ex:ajaR.tAkhAm}
\gll nɤki tɤtʂu nɯ a-jaʁ tɤ-kʰɤm! \\
\textsc{dem}:\textsc{medial} lamp \textsc{dem}, \textsc{1sg}.\textsc{poss}-hand \textsc{imp}:\textsc{up}-give[III] \\
\glt `Hand over to me (up here) that lamp.' (140511 alading-zh, 122)
\end{exe}

\begin{exe}
\ex \label{ex:nAjaR.YWkhoj}
\gll kɯki mbro ki nɤ-jaʁ ɲɯ-kʰo-j ŋu \\
\textsc{dem}.\textsc{prox} horse \textsc{dem}.\textsc{prox} \textsc{2sg}.\textsc{poss}-hand \textsc{ipfv}-give-\textsc{1pl} be:\textsc{fact} \\
\glt  `(If you succeed), we will give you this horse.'  (X1-qachGa, 62)
\end{exe}

The collocation of \forme{tɯ-jaʁ} with the intransitive verb \japhug{zɣɯt}{reach, arrive} means `receive' or `obtain', as in (\ref{ex:ajaR.anWzGWt}). With the causative form \forme{sɤzɣɯt}, the collocation means `get' (with volition and controlability) as in (\ref{ex:WjaR.junWsAzGWt}) -- the recipient marked by the possessive prefix on  \forme{tɯ-jaʁ} is the same referent as the transitive subject of the main verb.

\begin{exe}
\ex \label{ex:ajaR.anWzGWt}
\gll iɕqʰa tɤtʂu nɯ a-jaʁ a-nɯ-zɣɯt ra \\
the.aforementioned lamp \textsc{dem} \textsc{1sg}.\textsc{poss}-hand \textsc{irr}-\textsc{pfv}-reach have.to:\textsc{fact} \\
\glt   `I have to obtain this lamp.' (140511 alading-zh, 212)
\end{exe}

\begin{exe}
\ex \label{ex:WjaR.junWsAzGWt}
\gll tɕʰi ra na-sɯso ʑo nɯ, ɯ-jaʁ ju-nɯ-sɯ-ɤzɣɯt pjɤ-cʰa.  \\
what \textsc{pl} \textsc{pfv}:3\fl{}3'-think \textsc{emph} \textsc{dem} \textsc{3sg}.\textsc{poss}-hand \textsc{ipfv}-\textsc{auto}-\textsc{caus}-reach \textsc{ifr}.\textsc{ipfv}-can \\
\glt  `He was able to get whatever he wanted.' (140508 benling gaoqiang de si xiongdi, 47)
\end{exe}

The noun \japhug{tɯ-jaʁ}{hand} the verb \japhug{ɣi}{come} also means `obtain' or `find', as in (\ref{ex:tWjaR.mWjGi}).

\begin{exe}
\ex \label{ex:tWjaR.mWjGi}
\gll jinde tɕe ɯ-kɯ-sat koŋla maŋe tɕe, nɯ qarma ɯ-muj kɯnɤ tɯ-jaʁ mɯ́j-ɣi wo \\
nowadays \textsc{lnk} \textsc{3sg}.\textsc{poss}-\textsc{nmlz}:S/A-kill completely not.exist:\textsc{sens} \textsc{lnk} \textsc{dem} crossoptilon \textsc{3sg}.\textsc{poss}-feather also \textsc{genr}.\textsc{poss}-hand \textsc{neg}:\textsc{sens}-come \textsc{sfp} \\
\glt `Nowadays nobody kills crossoptilons, one cannot even get their feathers (to use as ornaments).' (23-qapGAmtWmtW, 170)
\end{exe}

The noun \japhug{tɯ-jaʁ}{hand} does not occur in metaphoric use outside of these collocations; it cannot be used in particular to mark any adjunct. Other noun-verb collocations where an experiencer or a recipient is marked as possessor of the noun are described in § XXX.
