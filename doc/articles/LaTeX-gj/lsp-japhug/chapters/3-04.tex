\chapter{The noun phrase} \label{chap:noun.phrase}

\section{Independent words vs clitics}
The present chapter deals with grammatical elements that are independent words rather than affixes, like those described in chapter \ref{chap:nominal.morphology}. Since some scholars such as 
\citet{jackson98morphology, jackson14morpho} treat the postpositions and the number modifiers as clitics rather than independent words as is done in the present work, a justification of my analysis is necessary.

The postpositions \japhug{kɯ}{ergative} and  \japhug{ɣɯ}{genitive} do have some clitic-like characteristics: they cannot be used without a preceding noun phrase (or subordinate clause in some cases, see § XXX), are unstressed, and in the case of the genitive have special irregular forms with pronouns (§ \ref{sec:pronouns.gen}).

However, a pause can occur between these postpositions (\ref{ex:kW.nAmWmnW}) and the noun phrase they follow. For instance, in example (\ref{ex:kW.nAmWmnW}), a two second pause (with an inspiration) is found between the phrase \forme{nɯŋa ra} and the following ergative \forme{kɯ}. 

\begin{exe}
\ex \label{ex:kW.nAmWmnW}
\gll tɕe tɯrtsi nɯ pjɯ́-wɣ-βzu tɕe, nɯŋa ra, kɯ nɤ-mɯm-nɯ cʰo wuma ʑo ɣɯ-ɕɯ-fka-nɯ \\
\textsc{lnk} cow.food \textsc{dem} \textsc{ipfv}-\textsc{inv}-make \textsc{lnk} cow \textsc{pl} \textsc{erg} \textsc{trop}-be.tasty:\textsc{fact}-\textsc{pl} \textsc{comit} really \textsc{emph} \textsc{inv}-\textsc{caus}-be.satiated:\textsc{fact}-\textsc{pl} \\
\glt `They make cow food with flour, the cows find it tasty, and it satisfies their hunger.' (140513 tWrtsi, 15)
\end{exe}


Such cases are by no means exceptional; at least 54+35 examples of ergative and genitive preceded by a pause are attested in the corpus (they can be found by searching \forme{kɯ} or \forme{ɣɯ}  preceded by a comma). Most of these cases are found in sentences where the speaker hesitates, and are especially common in texts translated from Chinese.

\section{Postpositions} 

\subsection{Absolutive} \label{sec:absolutive}
\subsubsection{Intransitive subject}
\subsubsection{Object}
\subsubsection{Semi-object}
\subsubsection{Theme}
\subsubsection{Essive} \label{sec:essive.abs}
%tsuku kɯ paʁndza ɲɯ-nɯ-phɯt-nɯ ɲɯ-ŋu ri,
\subsubsection{Locative adjunct}

\subsection{Ergative} \label{sec:erg.kW}
\subsubsection{Transitive subject} \label{sec:A.kW}
\subsubsection{Instrumental} \label{sec:instr.kW}
\subsubsection{Comparee marker}
\subsubsection{Partitive}

\subsection{Genitive} \label{sec:genitive}
With the exception of particular forms for some pronouns (§ \ref{sec:pronouns.gen}), the genitive postposition has the invariant form \forme{ɣɯ} in Kamnyu Japhug. Like the ergative \forme{kɯ}, it is likely borrowed from the Amdo clitic \forme{-ɣə/-kə} (\citealt[62]{haller04themchen}). It is used in possessive contructions, but also expresses beneficiary and recipient.

\subsubsection{Possession} \label{sec:gen.possession}
The genitive \forme{ɣɯ} occurs in various type of possessive constructions, including genitival noun complements and possessive existential predicates (§ XXX).

Inside the noun phrase, the genitive occurs between possessor and possessum, and a possessive prefix is found on the possessum (§ \ref{ex:prefix.expression.of.possession}), as in (\ref{ex:GZAndza.GW.WjwaR}). Genitival phrases without possessive prefix on the possessum are rare but do exist, however they used to express functions other than possession, see § \ref{sec:gen.other}. 

\begin{exe}
\ex \label{ex:GZAndza.GW.WjwaR}
\gll ri ɣʑɤndza ɣɯ ɯ-jwaʁ nɯra mɤ-wxti ri, ɲaʁ ʑo qhe, \\
\textsc{lnk} Agastache.rugosa \textsc{gen} \textsc{3sg}.\textsc{poss}-leaf \textsc{dem}:\textsc{pl} \textsc{neg}-be.big:\textsc{fact} \textsc{lnk} be.black:\textsc{fact} \textsc{emph} \textsc{lnk} \\
\glt `The leaves of the Agastache rugosa are not large and quite dark in colour.' (11-qarGW, 137)
\end{exe}

The genitive can also appear between a noun phrase and a relator noun, and even be followed by focus markers in this position, as in (\ref{ex:GW.kWnA.WrkW.ri}).

\begin{exe}
\ex \label{ex:GW.kWnA.WrkW.ri}
\gll   tɯ-ci kɯ-wxti ɣɯ kɯnɤ ɯ-rkɯ ri nɯra tu ŋgrɤl.  \\
\textsc{indef}.\textsc{poss}-water \textsc{nmlz}:S/A-be.big \textsc{gen} also \textsc{3sg}.\textsc{poss}-side \textsc{loc} \textsc{dem}:\textsc{pl} exist:\textsc{fact} be.usually.the.case:\textsc{fact} \\
\glt `(Dragonflies) are also found near rivers.' (26-quspunmbro, 7)
\end{exe}

In these constructions, the genitive is always optional, and the prefix on the possessum suffices to express possession, as in (\ref{ex:paXCi.WjwaR}) (see § \ref{ex:prefix.expression.of.possession}).

\begin{exe}
\ex \label{ex:paXCi.WjwaR}
\gll paχɕi ɯ-jwaʁ tsa fse ri, nɯ sɤznɤ artɯm,\\
apple \textsc{3sg}.\textsc{poss}-leaf a.little be.like:fact \textsc{lnk} \textsc{dem} \textsc{comp} be.round:\textsc{fact} \\
\glt `(Its leaves) are a little like the leaves of an apple tree, but more round.' (09-mi, 15)
\end{exe}

When the possessum is elided however, the genitive postposition becomes obligatory, as in (\ref{ex:baigua.GW.sAz}).

\begin{exe}
\ex \label{ex:baigua.GW.sAz}
\gll ɯ-rɣi nɯnɯ, nɤki, <beigua> ɣɯ sɤz ɲɯ-jaʁjɯ. \\
\textsc{3sg}.\textsc{poss}-seed \textsc{dem} \textsc{filler}  pumpkin \textsc{gen} \textsc{comp} \textsc{sens}-be.thick.and.strong \\
\glt `Its seeds are thicker than those of the pumpkin.' (16-CWrNgo, 130)
\end{exe}

While there are transitive and semi-transitive verbs expressing possession (§ XXX), the most common possessive construction involves an existential verb taking the possessum as subject, with the possessor marked by a possessive prefix on the possessum, and optionally with the genitive, as in (\ref{ex:phu.nW.GW.WRrW.GAZu}).

\begin{exe}
\ex \label{ex:phu.nW.GW.WRrW.GAZu}
\gll qartsʰaz pʰu nɯ ɣɯ ɯ-ʁrɯ ɣɤʑu \\
deer male \textsc{dem} \textsc{gen} \textsc{3sg}.\textsc{poss}-horn exist:\textsc{sens} \\
\glt `The male deer has horns.' (27-qartshAz, 32)
\end{exe}

The causative verbs \japhug{ɣɤtu}{cause to have} and \japhug{ɣɤme}{cause not to have, destroy} derived from \japhug{tu}{exist} and \japhug{me}{not exist} respectively (see § XXX) select an oblique argument with the genitive, as in (\ref{ex:WZo.GW.tuGAtea}). Although this argument could be considered to be a type of beneficiary (§ \ref{sec:other.uses.poss.prefixes}), we observe here stability in case marking of the possessor between the base construction and the derived causative one.

\begin{exe} 
\ex \label{ex:WZo.GW.tuGAtea} 
\gll ɯʑo kɯ maka kɤ-ntɕʰoz mɤ-kɯ-ɤrɕo kɯ-fse ʑo tɯrɟɯ laχtɕʰa ɯʑo ɣɯ tu-ɣɤte-a jɤɣ \\ 
\textsc{3sg}.\textsc{poss} \textsc{erg} at.all \textsc{inf}-use \textsc{neg}-\textsc{inf}:\textsc{stat}-be.finished \textsc{inf}:\textsc{stat}-be.like \textsc{emph} wealth thing \textsc{3sg}.\textsc{poss} \textsc{gen} \textsc{ipfv}-\textsc{caus}-exist[III]-\textsc{1sg} be.possible:\textsc{fact} \\ 
\glt `(If someone saves me), I will make him have more wealth and riches than he can ever use.' (140512 yufu yu mogui, 84) 
\end{exe} 
%ma aʑo a-kɤ-cha,  a-kɤ-cha kɯ-tu nɯra, a-kɤ-spa, tu-βze-a kɤ-cha nɯra lonba ʑo nɤʑɯɣ tɤ-ɣɤtu-t-a ɕti tɕe,

Not all combinations of existential verbs and genitival phrases are existential possessive constructions. For instance, in (\ref{ex:BZW.GW.WqiW}), the second clause could appear to contain a possessive construction meaning `the mouse only has half of it', but the context makes it clear that a different interpretation is necessary (see § XXX on this use of the existential verbs).

\begin{exe}
\ex \label{ex:BZW.GW.WqiW}
\gll qamtɕɯr nɯ ɯ-mtɕʰi nɯnɯ βʑɯ sɤznɤ mɤʑɯ ʑo amtɕoʁ tɕe nɯ βʑɯ ɣɯ ɯ-qiɯ kɯnɤ me \\
shrew \textsc{dem} \textsc{3sg}.\textsc{poss}-mouth \textsc{dem} mouse \textsc{comp} yet \textsc{emph} be.pointy \textsc{lnk} \textsc{dem} mouse \textsc{gen} \textsc{3sg}.\textsc{poss}-half even not.exist:\textsc{fact} \\
\glt `The shrew's mouth is even sharper than that of the mouse, and (its size) is not even half that of the mouse.' (27-spjaNkW, 204-205)
\end{exe}

 
\subsubsection{Recipient and beneficiary} \label{sec:gen.beneficiary}
§ \ref{sec:other.uses.poss.prefixes}
The genitive is selected to mark the recipient by the indirective verb \japhug{kʰo}{give, pass over}, as in (\ref{ex:aZWG.nWkhAm}) and (\ref{ex:GW.anWtWkhAm}).   

\begin{exe}
\ex \label{ex:aZWG.nWkhAm}
 \gll ɕɯ ʑo stu kɯ-mɤku pɯ-tɯ-mto-t nɯnɯ, laχtɕha pɯ-nnɯ-ŋu, tɯrme pɯ-nnɯ-ŋu nɯ, aʑɯɣ nɯ-kʰɤm tɕe tɕendɤre, aʑo ɲɯ-ta-lɤt jɤɣ \\
 who \textsc{emph} most \textsc{nmlz}:S/A-be.first \textsc{pfv}-2-see-\textsc{pst}:\textsc{tr} \textsc{dem}  thing \textsc{pst}.\textsc{ipfv}-\textsc{auto}-be   person \textsc{pst}.\textsc{ipfv}-\textsc{auto}-be \textsc{dem} \textsc{1sg}:\textsc{gen} \textsc{imp}-give[III] \textsc{lnk} \textsc{lnk} \textsc{1sg} \textsc{ipfv}-1\fl{}2-release be.possible:\textsc{fact} \\
 \glt `Give me the first thing you see (when you go back home), be it a person or an object, and I will release you.' (140506 shizi he huichang de bailingniao-zh, 50-52)
\end{exe}

\begin{exe}
\ex \label{ex:GW.anWtWkhAm}
 \gll jɤ-tsɯm tɕe iɕqʰa nɯ kɯβʁa nɯ ɣɯ a-nɯ-tɯ-kʰɤm \\
 \textsc{imp}-take.away \textsc{lnk} the.aforementioned \textsc{dem} noble \textsc{dem} \textsc{gen} \textsc{irr}-\textsc{pfv}-2-give[III] \\
 \glt  Take it and give it to the nobleman.' (150831 renshen wawa, 43)
\end{exe}
 
The recipient of the verb  \japhug{kʰo}{give, pass over} can alternatively also be marked by a possessive prefix on the IPN \japhug{tɯ-jaʁ} (with the meaning  hand over', \ref{sec:semi.grammaticalized.relator}) or, most commonly, with the dative \forme{ɯ-ɕki} or \fomrme{ɯ-pʰe} (§ \ref{sec:dative}).

The genitive also occurs with beneficiaries as adjuncts, not selected by the main verb, as in (\ref{ex:tChi.tunAmea}), and also as predicates with a copula as in (\ref{ex:WZAG.pjAmaR}).

\begin{exe}
\ex \label{ex:tChi.tunAmea}
\gll nɤʑɯɣ tɕʰi tu-nɤme-a ra, tɤ-ti  \\
\textsc{2sg}:\textsc{gen} what \textsc{ipfv}-do[III]-\textsc{1sg} have.to:\textsc{fact} \textsc{imp}-say \\
\glt `Tell me what I shall do for you.'
\end{exe}

 \begin{exe}
\ex \label{ex:WZAG.pjAmaR}
\gll   tʰoʁtɤm ka-wum tɕe, ɯʑɤɣ pjɤ-maʁ kɯ, tɕoχtsi rɟɤlpu ɣɯ ku-wum,  \\
taxes \textsc{pfv}:3\fl{}3'-collect \textsc{lnk} \textsc{3sg}:\textsc{gen} \textsc{ifr}.\textsc{ipfv}-not.be \textsc{erg} p.n. king \textsc{gen} \textsc{ipfv}-collect \\
\glt `The taxes that he had collected were not for himself, he was collecting them for the king of Cogtse.' (150901 NAjstsa, 28)
\end{exe}

The genitive is also attested with a noun-verb collocations (§ XXX), like \japhug{ɯ-kɤrnoʁ+mtɕɯr}{feel dizzy}, in which the possessor  of the noun is an experiencer as in (\ref{ex:fsapaR.GW.kWnA}). This example also illustrates the use of the genitive followed by a focus marker, as (\ref{ex:GW.kWnA.WrkW.ri}) above.

\begin{exe}
\ex \label{ex:fsapaR.GW.kWnA}
\gll tɕeri fsapaʁ ɣɯ kɯnɤ ɯ-kɤrnoʁ ɲɯ-mtɕɯr ɲɯ-ŋu \\
\textsc{lnk} animal \textsc{gen} also \textsc{3sg}.\textsc{poss}-head \textsc{sens}-turn \textsc{sens}-be \\
\glt `But animals too can feel dizzy.' (29-tAmtshAzkAkWndo, 71)
\end{exe}

\subsubsection{Other uses} \label{sec:gen.other}
%slama ra ɣɯ thɯthɤci kɯ-fse, kɤ-rɤβzjoz ra ɲɯ-stu mɯ́j-stu-nɯ,
%nɯ-stu ɲɯ-nɤma-nɯ mɯ́j-nɤma-nɯ, 
%nɯnɯra nɯ-phama ra nɯ-ɕki kɯ-rɤfɕɤt ɲɯ-ra.
%(150901 tshuBdWnskAt, 18-20)

%@longtoutan nɯ kupa-skɤt ɕti.
%tɕe iʑora ɣɯ tɕhi tu-kɯ-ti ŋu mɤxsi.


%
%tɕendɤre ɯ-jaʁ nɯtɕu ftsoʁ kɯngɯt ɯ-phɯ ɣɯ srɯnloʁ pjɤ-k-ɤrku-ci
%2003gesar, 239

 

%tɕe paʁ ɣɯ stu ɯ-kɤ-nɯmga,
%nɯra ɣɯ ɯ-kɯ-ntɕhoz me
\subsection{Locative} \label{sec:locative}
\subsection{Comitative} \label{sec:comitative} 
Postpositional phrases with the comitative postposition \japhug{cʰo}{and, with} and its variants \forme{cʰondɤre} and \forme{cʰonɤ} (comprising the linkers \forme{nɤ} and \forme{ndɤre}, see § XXX) are selected as oblique arguments by a handful of verbs, including \japhug{naχtɕɯɣ}{be the same} (§ \ref{sec:identity.modifier}) and \japhug{amɯmi}{be in good terms with}, as shown in (\ref{ex:cho.kWnaXtCWG}).

\begin{exe}
\ex \label{ex:cho.kWnaXtCWG}
\gll [ɯʑo cʰo] kɯ-naχtɕɯɣ [sɯjno, xɕaj ma mɤ-kɯ-ndza nɯra cʰonɤ] amɯmi-nɯ tɕe, \\
\textsc{3sg} \textsc{comit} \textsc{nmlz}:S/A-be.the.same vegetables grass apart.from \textsc{neg}-\textsc{nmlz}:S/A-eat \textsc{dem}:\textsc{pl} \textsc{comit} be.in.good.terms:\textsc{fact}-\textsc{pl} \textsc{lnk} \\
\glt `(The rabbit) is in good terms with (the animals) which eat only grass and vegetables like him.' (04-qala2, 8)
\end{exe}

Postpositional phrases in \forme{cʰo} are oblique arguments in the sense that they are relativized using the oblique participle (§ XXX). However, verbs that select \forme{cʰo} phrases index not only the intransitive subject proper, but the sum of the subject and the \forme{cʰo} phrase, which can be in the dual as in (\ref{ex:cho.YWnaXtCWGndZi}) (the white birch and the red birch) or in the plural (\ref{ex:cho.kWnaXtCWG}) (the rabbit and the other animals). 

\begin{exe}
\ex \label{ex:cho.YWnaXtCWGndZi}
\gll tɕe ɯ-rqʰu nɯ ɣɯrni laʁma ɯ-ŋgɯ nɯ [sɤjku cʰo] ɲɯ-naχtɕɯɣ-ndʑi ri\\
\textsc{lnk} \textsc{3sg}.\textsc{poss}-bark \textsc{dem} be.red:\textsc{fact} apart.from.the.fact \textsc{3sg}.\textsc{poss}-inside \textsc{dem} birch \textsc{comit} \textsc{sens}-be.the.same-\textsc{du} \textsc{lnk} \\
\glt `Apart from the fact that its bark is red, it is identical in the inside with the birch.' (06-mbrAj, 13)
\end{exe}

The verb \japhug{naχtɕɯɣ}{be the same} with a \forme{cʰo} phrase can be used in an equative construction (see § XXX).

Apart from the function presented above, \japhug{cʰo}{and, with} is commonly used to link together two nouns inside a single noun phrase, as in (\ref{ex:awW.cho.aRi}). In this case too, the main verb of the clause indexes the whole noun phrase, comprising the sum of referents designated by the nouns linked by \forme{cʰo}.

\begin{exe}
\ex \label{ex:awW.cho.aRi}
\gll a-wɯ cʰo a-ʁi ni cʰɯ-ɣi-ndʑi ra ma ʑɤni-sti kɤ-rɤʑi mɤ-cʰa-ndʑi tɕe, \\
\textsc{1sg}.\textsc{poss}-grand.father \textsc{comit} \textsc{1sg}.\textsc{poss}-younger.sibling \textsc{du} \textsc{ipfv}:\textsc{downstream}-come-\textsc{du} have.to:\textsc{fact} \textsc{lnk} \textsc{3du}-alone \textsc{inf}-stay \textsc{neg}-can:\textsc{fact}-\textsc{du} \textsc{lnk} \\ 
\glt `My grandfather and my younger brother have to come, they cannot stay by themselves.' (2011-05-nyima, 209)
\end{exe}

The marker \forme{cʰo} can also link verb phrases and even entire clauses (see § XXX and \citealt[313]{jacques14linking}).

Given the apparently equal status of the two linked nouns in(\ref{ex:awW.cho.aRi}), in particular with regard to indexation, it is legitimate to wonder whether analyzing it as a postposition makes more sense than considering it to be a conjunction. There are two arguments supporting the postposition analysis. First, \forme{cʰo} necessarily follows a noun phrase (or at the very least a demonstrative pronoun), but does not require to be followed by another noun as in (\ref{ex:cho.YWnaXtCWGndZi}) above. Second, phrases comprising \forme{cʰo} and the preceding noun are relativized using the oblique participle (see § XXX).

A \forme{cʰo} phrase can be followed by the associative plural marker \forme{ra} (§ \ref{sec:number.determiners}) as in (\ref{ex:cho.ra.kW}) to mean `et caetera', and the whole phrase can taken case marking such as ergative.

\begin{exe}
\ex \label{ex:cho.ra.kW}
\gll tɕeri ɯʑo ndɤre, qajdo cʰo ra kɯ ndɤ tú-wɣ-ndza ɕti \\
but \textsc{3sg} \textsc{advers} crow \textsc{comit} \textsc{pl} \textsc{erg} \textsc{advers} \textsc{ipfv}-\textsc{inv}-eat be.\textsc{affirm}:\textsc{fact} \\
\glt `But it is eaten by crows and other (animals).' (26-NalitCaRmbWm, 140)
\end{exe}

\subsection{Standard marker} \label{sec:comparative} %\japhug{sɤz}{compared with} staʁ sɤznɤ staʁnɤ
\subsection{Exceptive} \label{sec:exceptive} %\japhug{ma}{apart from} laʁma mɯma

The exceptive \japhug{ma}{apart from} and its variants are required in the restrictive focus construction (§ \ref{sec:restrictive.focus}).

\subsection{Terminative} \label{sec:terminative}  %\japhug{mɤɕtʂa}{until}


\section{Relator nouns}
\subsection{Dative} \label{sec:dative} 
\subsection{Deputative} \label{sec:deputative} 
The IPN \forme{ɯ-tsʰɤt} has two meanings. First, it can serve as a deputative relator noun `instead of, on behalf of' as in (\ref{ex:nWtAsno.WtshAt}) and (\ref{ex:nWsi.WtshAt}). No verb selects this relator noun. 

The deputative adjunct can correspond to the intransitive subject (as in \ref{ex:nWtAsno.WtshAt}, with the verb \japhug{tu}{exist}), the transitive subject (as in \ref{ex:aZo.nAtshAt}, with \japhug{ɣɯjtsi}{support}) or the object.

\begin{exe}
\ex \label{ex:nWtAsno.WtshAt}
\gll nɯʑora ɣɯ nɯ-tɤ-sno kɯ-fse ɯ-tsʰɤt nɯ, tɕiʑo ɣɯ, tɕi-xɕɤndʑu χsɯ-ldʑa pɯ-tu tɕe, nɯnɯ lɤ-nɯ-βlɯ-tɕi ɕti wo \\
\textsc{2pl} \textsc{gen} \textsc{2pl}.\textsc{poss}-\textsc{indef}.\textsc{poss}-saddle \textsc{nmlz}:S/A-be.like \textsc{3sg}.\textsc{poss}-instead.of \textsc{dem} \textsc{1du} \textsc{gen} \textsc{1du}.\textsc{poss}-twig three-long.object \textsc{pst}.\textsc{ipfv}-exist \textsc{lnk} \textsc{dem} \textsc{pfv}-\textsc{auto}-burn-\textsc{1du} be.\textsc{affirm}:\textsc{fact} \textsc{sfp} \\
\glt `Instead of saddle like yours, we had three twigs, this is what we burned.' (Kubzang2003, 203)
\end{exe}

\begin{exe}
\ex \label{ex:aZo.nAtshAt}
\gll aʑo nɤ-tsʰɤt, nɤki, si nɯ tu-ɣɯjtsi-a jɤɣ \\
\textsc{1sg} \textsc{2sg}.\textsc{poss}-instead.of \textsc{filler} tree \textsc{dem} \textsc{ipfv}-support-\textsc{1sg} be.possible:\textsc{fact} \\
\glt `I can support the tree for you/instead of you (while you fetch it).' (150830 afanti, 136)
\end{exe}

 The noun phrase headed by \forme{ɯ-tsʰɤt} can be either an adjunct as in (\ref{ex:nWtAsno.WtshAt}) and (\ref{ex:aZo.nAtshAt}), the object of the verb \japhug{βzu}{make}, or a nominal predicate with a copula as in (\ref{ex:nWsi.WtshAt}) and (\ref{ex:aZo.atshAt}).  In the latter case, to express the meaning `do to $X$ instead of to $Y$', a biclausal construction `do to $X$, ($X$) is instead of $Y$' is used as in (\ref{ex:aZo.atshAt}).

\begin{exe}
\ex \label{ex:nWsi.WtshAt}
\gll alo mbroχpa ra tɕe tɕe nɤki qra cʰo qambrɯ ra ɣɯ nɯ-ɣli nɯnɯ
tɕe nɯ tu-wum-nɯ, tu-sɯɣ-rom-nɯ mbroχpa sɤtɕʰa tɕe stɤmku ʁɟa ɲɯ-ɕti ma si maŋe tɕe tɕe nɯnɯtɕu tɕe, nɯ-si ɯ-tsʰɤt ɲɯ-ŋu  \\
upstream nomad \textsc{pl} \textsc{lnk} \textsc{lnk} \textsc{filler} female.yak \textsc{comit} male.yak \textsc{pl} \textsc{gen} \textsc{3pl}.\textsc{poss}-dung \textsc{dem} \textsc{lnk} \textsc{dem} \textsc{ipfv}-gather-\textsc{pl} \textsc{ipfv}-\textsc{caus}-be.dry-\textsc{pl} nomad place \textsc{lnk} grassland completely \textsc{sens}-be.\textsc{affirm} \textsc{lnk} tree not.exist:\textsc{sens} \textsc{lnk} \textsc{lnk} \textsc{dem}:\textsc{loc} \textsc{lnk} \textsc{3pl}.\textsc{poss}-wood \textsc{3sg}.\textsc{poss}-instead.of \textsc{sens}-be \\
\glt `Upstream, in the nomad areas, they gather and dry yak dung, as in nomad places there is only are no trees, there (dung) is used to replace the firewood.' (05-tamar, 7-11)
\end{exe}


\begin{exe}
\ex \label{ex:aZo.atshAt}
\gll nɯ tɤ-nɯ-ndɤm tɕe aʑo a-tsʰɤt ŋu tɕe \\
\textsc{dem} \textsc{imp}-\textsc{auto}-take[III] \textsc{lnk} \textsc{1sg} \textsc{1sg}.\textsc{poss}-instead.of be:\textsc{fact} \textsc{lnk} \\
\glt `Take these instead of me (as a compensation).' (2003kAndzwsqhaj2, 141)
\end{exe}

The examples (\ref{ex:aZo.nAtshAt}) and (\ref{ex:aZo.atshAt}) also show that the relator noun \japhug{ɯ-tsʰɤt}{instead of} can occur with a first or second person possessive prefix.

Second, \forme{ɯ-tsʰɤt} also means `with proper measure', mainly occurring in adverbial function as in (\ref{ex:WtshAt.tsa}) or in collocation with the verb \japhug{βzu}{make} in the sense `do with proper measure' as in (\ref{ex:WtshAt.tusWBzunW}). 

\begin{exe}
\ex \label{ex:WtshAt.tsa}
\gll rkaŋraŋ ɯ-tsʰɤt tsa ɲɯ-kɯ-nɤɕtʂaʁli-a-nɯ raʁmaʁ ma  \\
p.n. \textsc{3sg}.\textsc{poss}-proper.measure a.little \textsc{ipfv}-2\fl{}1-torture-\textsc{1sg}-\textsc{pl} \textsc{sfp} \textsc{lnk}  \\
\glt `Rkangrang, your torturing of me should have a limit.' 
\end{exe}

\begin{exe}
\ex \label{ex:WtshAt.tusWBzunW}
\gll ɯ-tsʰɤt tu-sɯ-βzu-nɯ mɯ́j-kʰɯ ma, nɯ-kɤ-kʰo nɯ mɯ-tʰa-ɕkɯt mɤɕtʂa tu-ndze ɲɯ-ɕti. \\
\textsc{3sg}.\textsc{poss}-proper.measure \textsc{ipfv}-\textsc{caus}-make-\textsc{pl} \textsc{neg}:\textsc{sens}-be.possible \textsc{pfv}-\textsc{nmlz}:P-give \textsc{dem} \textsc{neg}-\textsc{pfv}:3\fl{}3'-eat.completely until \textsc{ipfv}-eat[III] \textsc{sens}-be.\textsc{affirm} \\
\glt `They cannot make (the monkey eat) with measure, as it continues eating the (food) that is given to it until there is none.' (19-GzW, 60)
\end{exe}

In some contexts as in (\ref{ex:nWnW.WtshAt}), \forme{ɯ-tsʰɤt}{proper measure} in adverbial used is better translated as `depending on the circumstances'.\footnote{This example is taken from a text describing goats and sheep; goats are called \forme{tsʰɤt} in Japhug, but it is clear from the context that \forme{ɯ-tsʰɤt} cannot be the possessed form of this noun. }

\begin{exe}
\ex \label{ex:nWnW.WtshAt}
\gll tɕe nɯnɯ ɯ-tsʰɤt nɯnɯ ɯ-pɯ ci ci ʁnɯz tu, ci ci tɯ-rdoʁ ma me tɕe núndʐa ɲɯ-ŋu. \\
\textsc{lnk} \textsc{dem} \textsc{3sg}.\textsc{poss}-proper.measure \textsc{dem}  \textsc{3sg}.\textsc{poss}-young once once two exist:\textsc{fact} once once one-piece apart.from not.exist:\textsc{fact} \textsc{lnk} for.this.reason \textsc{sens}-be \\
\glt `This is why, depending on the circumstances, sometimes (the goat) has two youngs, sometimes only one.' (05-qaZo, 28)
\end{exe}

The IPN  \forme{ɯ-tsʰɤt} (at least in the meaning `proper measure') is borrowed from \tibet{ཚད་}{tsʰad}{measure, limit}. It occurs as second element in the compound \japhug{xtɤtsʰɤt}{restraint of one's appetite}(with the \textit{status constructus} \forme{xtɤ-} of \japhug{tɯ-xtu}{belly}).

\subsection{Relator nouns of location}

\subsection{Semi-grammaticalized relator nouns} \label{sec:semi.grammaticalized.relator} 
%nɤki tɤtʂu nɯ a-jaʁ tɤ-khɤm! hand over
%kɯki mbro ki nɤ-jaʁ ɲɯ-kho-j hist-X1-qachGa,62

\section{Noun modifiers and determiners}
This section discusses all nouns modifiers and determiners except relative clauses (§ XXX) and complement clauses (§ XXX). 
 
\subsection{Number}  \label{sec:number.determiners}

\subsection{Demonstratives} \label{sec:demonstrative.determiners}

\subsection{Quantifiers}
\subsubsection{Universal quantifiers} \label{sec:universal.quant}
\subsubsection{Mid-scalar quantifier} \label{sec:tsuku}
(\ref{sec:partitive.pronouns})

\subsection{Indefinite and definite markers} \label{sec:indefinite.markers}

\subsubsection{Indefinite article} \label{sec:indef.article}
The form \japhug{ci}{one} has among its many functions (in addition to pronoun, numeral and adverb, see § \ref{sec:other.pro}, § \ref{sec:partitive.pronouns}, § \ref{sec:identity.modifier}, § \ref{sec:one.to.ten} and § XXX) that of singular indefinite article, as in (\ref{ex:ci.indef}) and (\ref{ex:ci.chAGi}). It is typically used to introduce a new referent in a story.

\begin{exe}
\ex \label{ex:ci.indef}
\gll tɕʰeme kɯ-mpɕɯ\redp{}mpɕɤr ci ɲɤ-nɯ-ɬoʁ \\
girl \textsc{nmlz}:S/A-\textsc{emph}\redp{}beautiful \textsc{indef} \textsc{ifr}-\textsc{auto}-come.out \\
\glt `A very beautiful girl appeared (out of it).' (The flood, 39)
\end{exe}

\begin{exe}
\ex \label{ex:ci.chAGi}
\gll tɕɤlo tɕe tɤ-tɕɯ ci cʰɤ-ɣi qʰe, \\
upstream \textsc{lnk} \textsc{indef}.\textsc{poss}-son \textsc{indef} \textsc{ifr}:\textsc{downstream}-come \textsc{lnk} \\
\glt `A boy came from upstream.' (2003-kWBRa, 41)
\end{exe}

Although \forme{ci} can be used as a partitive pronoun `one of them' (§ \ref{sec:partitive.pronouns}), as a postnominal determiner it does not have partitive meaning. To express a meaning such as `one of the boys', a CN such as \japhug{tɯ-rdoʁ}{one piece} is used instead (§ \ref{sec:ICN}). 

Note that when used as a prenominal modifier, \forme{ci} has a completely different (definite) meaning `the other X' (§ \ref{sec:identity.modifier}). 

There are no dual or plural indefinite articles in Japhug. The plural marker \forme{ra} can occur after the indefinite \forme{ci}, but with a vague associative meaning `and other things' as in (\ref{ex:ci.ra}).

\begin{exe}
\ex \label{ex:ci.ra}
 \gll  ndʑi-tɕɯ ci, ndʑi-me ci ra to-tu. \\
 \textsc{3du}.\textsc{poss}-son \textsc{indef}  \textsc{3du}.\textsc{poss}-girl \textsc{indef} \textsc{pl} \textsc{ifr}-exist \\
 \glt  `They$_{du}$ had a boy and a girl (etc).' (150827 tianluo-zh, 155)
\end{exe}

\subsubsection{Indefinite pronoun as modifier} \label{sec:indefinite}
The indefinite pronoun \japhug{tʰɯci}{something} (§ \ref{sec:thWci}) has marginal uses as a prenominal indefinite modifier, as in  (\ref{ex:thWci.laXCi}), (\ref{ex:thWci.WjmNo}) and (\ref{ex:laXtCha.ci.nWnW}) below. 

\begin{exe}
\ex \label{ex:thWci.laXCi}
\gll   tʰɯci laχɕi ci ɕ-pɯ-nɯ-βzjoz-nɯ tɕe, jɤ-ɕe-nɯ ra \\
something trade \textsc{indef} \textsc{transloc-imp-auto}-learn-\textsc{pl} \textsc{lnk} \textsc{imp}-go-\textsc{pl} have.to:\textsc{fact} \\
\glt `Go and learn some trade!' (140508 benling gaoqiang de si xiongdi-zh, 29)
 \end{exe}
 
 This construction arose perhaps from the use of the pronoun \forme{tʰɯci} as head of a postnominal relative clause with the verb \japhug{fse}{be like}, as illustrated by examples like (\ref{ex:thWci.kAnWsaXCWB}) or (\ref{ex:thWci.akAspa}) in § \ref{sec:thWci}. Turning the verb \japhug{fse}{be like} to a finite form as in (\ref{ex:thWci.WjmNo}) could cause the indefinite \forme{tʰɯci}, head of the relative in (\ref{ex:thWci.kAnWsaXCWB}), to be reanalyzed as the prenominal modifier of the immediately adjacent noun in (\ref{ex:thWci.WjmNo}).

 \begin{exe}
\ex \label{ex:thWci.kAnWsaXCWB}
\gll nɯra [tʰɯci [kɤ-nɯsaχɕɯβ kɯ-fse]] pɯ-ŋu wo.  \\
\textsc{dem}:\textsc{pl} something \textsc{inf}-have.a.contest \textsc{nmlz}:S/A-be.like \textsc{pst}.\textsc{ipfv}-be \textsc{sfp} \\
\glt `It was like a kind of contest.' (160706 thotsi, 16)
 \end{exe}
 
\begin{exe}
\ex \label{ex:thWci.WjmNo}
\gll [tʰɯci ɯ-jmŋo] ci ʑo pɯ-fse ri \\
something \textsc{3sg}.\textsc{poss}-dream one \textsc{emph} \textsc{pst}.\textsc{ipfv}-be.like \textsc{lnk} \\
\glt `It looked like (he had had) some dream.' (Lobzang2005, 74)
 \end{exe}
 
 
\subsubsection{The marking of definiteness} \label{sec:definiteness}
Japhug has no dedicated definite determiner, but  \forme{nɯ} and \forme{nɯnɯ}  as demonstrative determiners (\ref{sec:demonstrative.determiners}) and as topic markers (\ref{sec:topic}) and the prenominal aforementioned topic marker \forme{iɕqʰa} (§ \ref{sec:iCqha}) are generally used with definite referents.  

Example (\ref{ex:ci.joGi}) illustrates a typical example with the determiner \forme{nɯ}; the indefinite article \forme{ci} (§ \ref{sec:indef.article}) occurs in the first introduction of a new referent in the story as in the first clause of example (\ref{ex:ci.joGi}), but on the following occurrence of the same noun \forme{nɯ} is found.

\begin{exe}
\ex \label{ex:ci.joGi}
 \gll  tɕe qajdo ci jo-ɣi tɕe, tɕe qajdo nɯ kɯ `mo laz tu, pʰo laz me' to-ti. \\
 \textsc{lnk} crow \textsc{indef} \textsc{ifr}-come \textsc{lnk} \textsc{lnk} crow \textsc{dem} \textsc{erg} girl karma exist:\textsc{fact} boy karma not.exist:\textsc{fact} \textsc{ifr}-say \\
 \glt `A crow came. The crow said: `The girl will have chance, the boy won't.'' (28-qAjdoskAt, 8)
\end{exe}

However, although nouns phrases followed by \forme{nɯ} and \forme{nɯnɯ} more often than not denote definite referents, these determiners cannot be analyzed as definite articles, as noun phrases with \forme{nɯ} or \forme{nɯnɯ} can in certain cases have indefinite referents. 

A very clear case of use of \forme{nɯ} with an indefinite referent occurs on nouns serving as heads of head-internal relative clauses. A well-attested typological generalization is that in this type of relative clauses, definiteness marking is agrammatical (see \citealt{basilico96internally} and § XXX). In Khroskyabs, \citet[636]{lai17khroskyabs} reports that the definiteness marker \forme{=tə} is indeed not accepted on the head noun of head-internal relatives. In Japhug however, \forme{nɯ} does occur in such a syntactic context. For instance, in (\ref{ex:tAnmaR.nW.kW}), the head \forme{tɤ-nmaʁ nɯ kɯ} is subject of the participle \japhug{ɲɯ-kɯ-nɯ-ɕar}{looking for}, and is embedded in the participial relative clause indicated in brackets -- the presence of the ergative \forme{kɯ} precludes to analyze it as a post-nominal relative (§ XXX). From the meaning of the sentence the head \japhug{tɤ-nmaʁ}{husband} is clearly indefinite non-specific non-generic  (see \citealt[286-291]{lehmann84relativsatz}). The fact that it takes the marker \forme{nɯ} shows that this marker, unlike Khroskyabs \forme{=tə}, is not primarily marking definiteness.

\begin{exe}
\ex \label{ex:tAnmaR.nW.kW}
 \gll tɕeri [tɤ-nmaʁ nɯ kɯ ɯ-rʑaʁ kɯ-ɤntɕʰɯ ɲɯ-kɯ-nɯ-ɕar], aʁɤndɯndɤt tɤndɤɣri tu-kɯ-βzu pjɤ-tu.  \\
but  \textsc{indef}.\textsc{poss}-husband \textsc{dem} \textsc{erg} \textsc{3sg}.\textsc{poss}-wife  \textsc{nmlz}:S/A-be.many \textsc{ipfv}-\textsc{nmlz}:S/A-\textsc{auto}-search everywhere  illegitimate.child  \textsc{ipfv}-\textsc{nmlz}:S/A-make \textsc{ifr}.\textsc{ipfv}-exist \\
\glt `However there were husbands who were looking for several women and had illegitimate children.' (140427 tAndAGri, 3)
\end{exe}

Other cases of indefinite noun phrase with \forme{nɯ} are observed with left-dislocated topics. In example (\ref{ex:RnWz.nWnW}), we find a type of tail-head linkeage  (§ XXX) where both the noun phrase \japhug{spjaŋkɯ ʁnɯz}{two wolves} and the verb \japhug{ɲɤ-k-ɤtɯɣ-ci}{he met} are repeated; in the second occurrence, the noun phrase is topicalized and is followed by the topic marker \forme{nɯnɯ}, with a slight pause of hesitation. The determiner \forme{nɯnɯ} in this clause, unlike \forme{nɯ} in (\ref{ex:ci.joGi}), does not mark definiteness: that clause cannot be understood as `He met the two wolves'.

\begin{exe} 
\ex \label{ex:RnWz.nWnW} 
 \gll spjaŋkɯ ʁnɯz ɲɤ-k-ɤtɯɣ-ci. spjaŋkɯ ʁnɯz nɯnɯ, tɕendɤre ɲɤ-k-ɤtɯɣ-ci tɕe iɕqʰa, kɯ-rɤ-ntɕʰa nɯ wuma ʑo ɲɤ-mu. \\ 
 wolf two \textsc{ifr}-\textsc{evd}-meet-\textsc{evd}  wolf two \textsc{dem} \textsc{lnk} \textsc{ifr}-\textsc{evd}-meet-\textsc{evd} \textsc{lnk} the.aforementioned \textsc{nmlz}:S/A-\textsc{a.pass}:\textsc{n.hum}-kill \textsc{dem} really \textsc{emph} \textsc{ifr}-be.afraid \\ 
 \glt `He$_i$ (the butcher) met two wolves. He$_i$ met two wolves, and the butcher$_i$ was very much afraid.' (150902 liaozhai lang-zh, 7-8)
\end{exe}

The determiners \forme{nɯ} or \forme{nɯnɯ} are not attested in the corpus with the indefinite singular article \forme{ci} if both have scope on the same noun. In all cases with \forme{ci} followed by \forme{nɯ} (other than the identity pronoun in § \ref{sec:other.pro}), or of \forme{nɯ} followed by \forme{ci} in the corpus, they belong to different constituents. For instance, in (\ref{ex:ci.YAZGAsAphAr}), \forme{ci} is in adverbial use (`a little, once', see § XXX) and does not belong to the preceding noun phrase.  

\begin{exe}
\ex \label{ex:ci.YAZGAsAphAr}
\gll [tɕʰeme nɯ] ci ɲɤ-ʑɣɤ-sɤpʰɤr qʰe  \\
girl \textsc{dem} one \textsc{ifr}-\textsc{refl}-shake \textsc{lnk} \\
\glt `The girl shook herself.' (02-deluge2012, 125)
\end{exe}

In (\ref{ex:laXtCha.ci.nWnW}) although \forme{nɯnɯ} follows \forme{ci}, it has scope over the both preceding phrases, which are left-dislocated and followed by a pause.

\begin{exe}
\ex \label{ex:laXtCha.ci.nWnW}
\gll  kɤ-xtɕɤr tɕe nɯnɯ tɕe tɕe iɕqʰa, [[tʰɯci tɯmbri tɤ-ri kɯ-fse kɯ] [laχtɕʰa ci] nɯnɯ], ci kú-wɣ-sɯ-pa tɕe, kú-wɣ-xtɕɤr, \\
\textsc{inf}-attach \textsc{lnk} \textsc{dem} \textsc{lnk} \textsc{lnk} the.aforementioned something rope \textsc{indef}.\textsc{poss}-thread \textsc{nmlz}:S/A-be.like \textsc{erg} thing \textsc{indef} \textsc{dem} one \textsc{ipfv}-\textsc{inv}-\textsc{caus}-do \textsc{lnk} \textsc{ipfv}-\textsc{inf}-attach \\
\glt ``To attach' (means), to put together, attach something with something like a rope or a thread.'  (150902 kAxtCAr, 2-3)
\end{exe}

The aforementioned topic marker \forme{iɕqʰa} (§ \ref{sec:iCqha}) is almost always used with definite referents when prenominal, as in (\ref{ex:RnWz.nWnW}) above, and is the closest candidate fro analysis as a definiteness marker in Japhug. It does occur with non-specific generic referents as in (\ref{ex:lWlAmu}), including some that are very clearly indefinite as in (\ref{ex:lApWG}); note the absence of postnominal determiner \forme{nɯ} (\ref{ex:lApWG}).

\begin{exe}
\ex \label{ex:lWlAmu}
 \gll iɕqʰa lɯlɤmu nɯ tʰɯ-rɤpɯ tɕe tɕe ɯ-sŋi tɕe kɤ-nɯ-rŋgɯ nɯ stʰɯci mɯ́j-tsu ma ɯ-pɯ ra χse ɲɯ-ra tɕe, \\
 the.aforementioned female.cat \textsc{dem} \textsc{ipfv}-bear.young \textsc{lnk} \textsc{lnk} \textsc{3sg}.\textsc{poss}-day \textsc{lnk} \textsc{inf}-\textsc{auto}-lie.down \textsc{dem} so.much \textsc{neg}:\textsc{sens}-have.time.to \\
 \glt `A/the female cat (unlike male cats), when it had had youngs, does not have time to sleep during the day, as it has to feed its youngs.' (21-lWLU, 
\end{exe}

\begin{exe}
\ex \label{ex:lApWG}
\gll  iɕqʰa lɤpɯɣ ɯ-rɣi ʑo fse. \\
the.aforementioned radish \textsc{3sg}.\textsc{poss}-seed \textsc{emph} be.like:\textsc{fact} \\
\glt `It looks like a radish seed.' (hist-26-qro-fourmi, 61)
\end{exe}

In  (\ref{ex:laXtCha.ci.nWnW}), \forme{iɕqʰa}  also precedes two phrases involving indefinite referents, but  there is a marked pause, and this is a case of \forme{iɕqʰa} in its function as speech filler (see § XXX).

\subsubsection{Absence of definiteness marking}
Like many languages (\citealt[130]{creissels06sgit1}), Japhug uses bare nouns without any definiteness marking. Bare nouns are most often non-referential, as \japhug{tɕʰeme}{girl} in (\ref{ex:tCheme.tWtAtu}).

\begin{exe}
\ex \label{ex:tCheme.tWtAtu}
\gll ʁnaʁna tɕʰeme tɯ\redp{}tɤ-tu nɤ, kɤndʑɯsqʰaj tu-kɤ-sɯ-βzu \\
both girl \textsc{cond}\redp{}\textsc{pfv}-exist \textsc{lnk} \textsc{coll}:sister \textsc{ipfv}-\textsc{inf}-\textsc{caus}-make \\
\glt `If both of them have girls, let them be sisters.' (zrAntCW, 4)
\end{exe}

Bare nouns are less common with referential nouns (except in answers to questions), but examples can be found, as \japhug{qacʰɣa}{fox} in (\ref{ex:qachGa.kW}).

\begin{exe}
\ex \label{ex:qachGa.kW}
\gll qacʰɣa 	kɯ maχtɕɯ tɤ-tɯt-a nɯ mɤ-tɯ-ste ti ɲɯ-ŋu \\
fox \textsc{erg} I.told.you.so \textsc{pfv}-say[II]-\textsc{1sg} \textsc{dem} \textsc{neg}-2-do.like[III]:\textsc{fact} say:\textsc{fact} \textsc{sens}-be \\
\glt `The fox says: `You do not do as I told you to." (2003qachGa, 44)
\end{exe}

Personal names generally occur as bare nouns, without any definiteness marker as in (\ref{ex:WrJAnpanma}), but there are no constraints against co-occurrence of personal names with the determiner \forme{nɯ} either (see § \ref{sec:personal.names.modifiers}).

\begin{exe}
\ex \label{ex:WrJAnpanma}
\gll  ɯrɟɤnpanma kɯ ʁlaŋsaŋtɕhin ɯ-ɕki  \\
 Padmasambhava \textsc{erg} Gesar \textsc{3sg}-\textsc{dat} \\
\glt `Padmasambhava (told) Gesar.' (Gesar, 2)
\end{exe}

 \subsection{Topic markers} \label{sec:topic}
 
 \subsubsection{Aforementioned topic} \label{sec:iCqha}
 The marker \japhug{iɕqʰa}{the aforementioned}  is used on referents that have been previously mentioned in the same story, usually only a few sentences back. It is strictly prenominal. 
 
Example (\ref{ex:iCqha.aforementioned}) illustrates the most typical use of this marker. Sentence (\ref{ex:kAtWm}) introduces a new reference, \japhug{kɤtɯm}{ball of thread} marked with the indefinite article \forme{ci} (§ \ref{sec:indef.article}). Three clauses later in (\ref{ex:iCqha.kAtWm}), the same referent occurs again with two topic markers, the postnominal \textit{nɯ} and the prenominal \textit{iɕqʰa}.
 
 
\begin{exe}
\ex \label{ex:iCqha.aforementioned}
\begin{xlist}
\ex \label{ex:kAtWm}
\gll `razri \textbf{kɤtɯm} \textbf{ci} ɲɯ-ra, taqaβ ci ɲɯ-ra' to-ti qʰe   \\
 thread ball \textsc{indef} \textsc{sens}-need needle \textsc{indef} \textsc{sens}-need \textsc{ifr}-say \textsc{lnk}  \\
\glt `He told (Rgyabza) `I need a ball of thread and a needle.''  
\ex  
\gll tɕendɤre ɲɤ-kʰo qʰe,  \\
\textsc{lnk} \textsc{ifr}-give \textsc{lnk}   \\
\glt `She gave it to him.'
\ex 
\gll  tɕe ɯ-ndzɤtsʰi ka-tsɯm-nɯ nɯtɕu qʰe tɕe,   \\
 \textsc{lnk} \textsc{3sg}.\textsc{poss}-meal \textsc{pfv}:3\fl{}3'-bring-\textsc{pl} \textsc{dem}:\textsc{loc}  \textsc{lnk} \textsc{lnk}    \\
\glt `When they brought his meal,'
\ex \label{ex:iCqha.kAtWm}
\gll   \textbf{iɕqʰa} \textbf{kɤtɯm} \textbf{nɯ} ɯʑo kɯ ko-ndo, \\
   the.aforementioned ball \textsc{dem} \textsc{3sg} \textsc{erg} \textsc{ifr}-take \\
\glt `he took the ball of thread, and...' (Gesar 270-272)
\end{xlist}
\end{exe}
 
 
A systematic study of the use of the topic marker \forme{iɕqʰa} in Japhug must overcome two inherent difficulties. First, this topic marker is homophonous with (and historically related to) the speech filler \forme{iɕqʰa} (§ XXX) and with the adverb \japhug{iɕqʰa}{just now}, which can also precede noun phrases. Listening to the sound files can help distinguishing between the three, as the speech filler is always followed by a pause (and optionally by the demonstrative \forme{nɯ}), but there are still ambiguous sentences (see below). Second, \forme{iɕqʰa} occurs on nouns designating entities that the speaker considers to have been previously referred to in the conversation, even if they are not present in the same recording. 

For instance in (\ref{ex:iCqha.pɣArnoR}) the noun \japhug{pɣɤrnoʁ}{a species of fungus} is used with \forme{iɕqʰa}, although this name does not occur before in the same text; it was however mentioned the day before in another recording.

\begin{exe}
\ex \label{ex:iCqha.pɣArnoR}
\gll nɯ zdɯmqe cʰo iɕqʰa, pɣɤrnoʁ nɯni ndʑi-tsʰɯɣa wuma ʑo naχtɕɯɣ. \\
\textsc{dem} fungi.sp. \textsc{comit} the.aforementioned fungi.sp. \textsc{dem}:\textsc{du} \textsc{3du}.\textsc{poss}-form really \textsc{emph} be:identical:\textsc{fact} \\
\glt `The \forme{zdɯmqe} and the \forme{pɣɤrnoʁ} are very similar.' (23-mbrAZim, 82)
\end{exe}

 
The topic marker \forme{iɕqʰa} transparently comes from the adverb \japhug{iɕqʰa}{just now} (§ XXX). The pivot constructions that allowed reanalysis from adverb to prenominal topic marker are very probably headless relatives (§ XXX) as in  (\ref{ex:iCqha.tAtWta}), or complement clauses as in (\ref{ex:iCqha.ZnWzmWnmuta}). 

\begin{exe}
\ex \label{ex:iCqha.tAtWta}
 \gll  [iɕqʰa tɤ-tɯt-a] nɯ tú-wɣ-stu qʰe, \\
 just.now \textsc{ifr}-say[II]-\textsc{1sg} \textsc{dem} \textsc{ipfv}-\textsc{inv}-do.like \textsc{lnk} \\
\glt `One does as I just said, and...' (2002tWsqar, 139)
\end{exe}

\begin{exe}
\ex \label{ex:iCqha.ZnWzmWnmuta}
 \gll iɕqʰa [ʑ-nɯ-z-mɯnmu-t-a] nɯ mɯ-pjɤ-pe rcama.  \\
the.aforementioned  \textsc{transloc}-\textsc{pfv}-\textsc{caus}-\textsc{move}-\textsc{pst}:\textsc{tr}-\textsc{1sg} \textsc{dem} neg-\textsc{ifr}.\textsc{ipfv}-be.good \textsc{fsp} \\
\glt `It was probably not a good thing that I had moved them (as I said above).' (150819 kumpGa, 45)
 \end{exe}
 
 These sentences are still synchronically ambiguous in Japhug; in  (\ref{ex:iCqha.ZnWzmWnmuta}) the context makes it clear that \forme{iɕqʰa} is the topic marker (since the fact of having moved (the eggs) had been told a few sentences back) and not an adverb `just now' with a temporal reference in the past, as the meaning would be `it was probably not a good thing that I had just moved them' (an impossible interpretation in this context, since this sentence is an explanation why several eggs had not given chicks, several days after they had been brought to another place). However, extracted from the context, both interpretation would be equally possible for (\ref{ex:iCqha.ZnWzmWnmuta}), and correspond to two distinct syntactic structures.

With postnominal (§ XXX) or left-headed head-internal relative clauses (§ XXX) as in (\ref{ex:tWrpa.thafse}), \forme{iɕqʰa} can also be ambiguous. Since the adverb \japhug{iɕqʰa}{just now} can occur both before the object (\ref{ex:tWrpa.thWfseta}) or before the verb (\ref{ex:tWrpa.thWfseta2}) in an independent clause, a relative such as (\ref{ex:tWrpa.thafse}) can be either interpreted `the axe (mentioned above) that he had whetted' (with the topic marker \forme{iɕqʰa} outside of the relative clause, having scope on its head) and `the axe that he had just whetted' with the adverb \japhug{iɕqʰa}{just now} inside the relative clause.

 \begin{exe}
\ex \label{ex:tWrpa.thafse}
 \gll  tɕendɤre <luban> kɯ iɕqʰa [tɯrpa tʰa-fse] nɯ to-ndo tɕe, \\
 \textsc{lnk} p.n. \textsc{erg} the.aforementioned axe \textsc{pfv}:3\fl{}3'-whet \textsc{dem} \textsc{ifr}-take \textsc{lnk} \\
 \glt `Luban took the axe that he had whetted.' (150902 luban-zh, 90)
 \end{exe}

  \begin{exe}
  \ex 
  \begin{xlist}
\ex \label{ex:tWrpa.thWfseta}
 \gll   iɕqʰa tɯrpa tʰɯ-fse-t-a \\
just.now axe \textsc{pfv}-whet-\textsc{pst}:\textsc{tr}-\textsc{1sg} \\
\ex \label{ex:tWrpa.thWfseta2}
 \gll   tɯrpa  iɕqʰa tʰɯ-fse-t-a \\
 axe just.now \textsc{pfv}-whet-\textsc{pst}:\textsc{tr}-\textsc{1sg} \\
 \glt `I just whetted a/the axe.' (elicited)
 \end{xlist}
 \end{exe}

The use of \forme{iɕqʰa} as a topic marker with nouns (as in \ref{ex:iCqha.kAtWm} above) probably took place by reanalysis of the adverb in headless or postnominal relatives, or in complment clauses as above, then generalized to all noun phrases even those without subordinate clause.

 \subsection{Focus markers} \label{sec:focus}
   \subsubsection{Unexpected focus} \label{sec:unexpected}
 \subsubsection{Additive and scalar focus marker \forme{kɯnɤ} } \label{sec:kWnA}
The additive and scalar focus marker \japhug{kɯnɤ}{also, even} follows the constituent over which it has scope, which can be noun phrases, postpositional phrases but also subordinate clauses (these are treated in § XXX). The stress is on the first syllable (\forme{kɯ́nɤ}) and the vowel on the second syllable is often elited (a pronunciation \forme{kɯn} is often heard).

The marker \forme{kɯnɤ} expresses both additive focus, as in (\ref{ex:aZo.kWNA.staRlupa}), and scalar focus, as in (\ref{ex:WNgWz.kWnA.tunAndWtnW}) in affirmative sentences. It is also compatible with negative verb forms, as in (\ref{ex:tWrdoR.kWnA}), expressing the meaning `not even' (see also \japhug{cinɤ}{(not) even one} in § \ref{sec:cinA}).

\begin{exe}
\ex \label{ex:aZo.kWNA.staRlupa}
\gll aʑo kɯnɤ staʁlupa ŋu-a tɕe \\
\textsc{1sg} also born.in.the.tiger.year be:\textsc{fact}-\textsc{1sg} \textsc{lnk} \\
\glt `Me too (like you), I am of the Tiger year.' (2011-05-nyima, 168)
\end{exe}

\begin{exe}
\ex \label{ex:WNgWz.kWnA.tunAndWtnW}
\gll ʑara ʑo ɯ-ŋgɯz kɯnɤ tu-nɤndɯt-nɯ tɕe nɯ kɯ-βʁa ɣɤʑu, kɯ-nŋo ɣɤʑu qʰe, \\
\textsc{3pl} \textsc{emph} \textsc{3sg}.\textsc{poss}-among:\textsc{loc} also \textsc{ipfv}-fight-\textsc{pl} \textsc{lnk} \textsc{dem} \textsc{nmlz}:S/A-win \textsc{sens}:exist \textsc{nmlz}:S/A-lose  \textsc{sens}:exist \textsc{lnk} \\
\glt `Even among themselves, they fight, and there are winners and losers.' (20-sWNgi, 62-63)
\end{exe}
 
\begin{exe}
\ex \label{ex:tWrdoR.kWnA}
\gll tɯ-sŋi mɯntoʁ tɯ-rdoʁ kɯnɤ ci ci tɕe mɯ́j-stʰɯt \\
one-day flower one-piece also once once \textsc{lnk} \textsc{neg}:\textsc{sens}-finish \\
\glt `Sometimes one cannot finish even one pattern (on the belt) in one day.' (2011-06-thaXtsa, 47)
\end{exe}

As an additive focus marker, \forme{kɯnɤ} can be repeated on all the nouns designating the members of a group sharing a particular property, in the construction $X$ \forme{kɯnɤ}, $Y$ \forme{kɯnɤ}  `both $X$ and $Y$', as in (\ref{ex:Dpalcan.kWnA}).

\begin{exe}
\ex \label{ex:Dpalcan.kWnA}
 \gll a-pɯ-ŋu tɕe, aʑo kɯnɤ taʁrdo rɟitpa a-pɯ-ŋu-a, χpɤltɕin kɯnɤ taʁrdo rɟitpa a-pɯ-ŋu, ... nɯ tɕi-rɟit nɯni tɕe taʁrdo rɟitpa ma nɯ ma kɯmaʁ rɟitpa nɯ kɤ-rtsi me.  \\
 \textsc{irr}-\textsc{ipfv}-be \textsc{lnk} \textsc{1sg} also pl.n. lineage  \textsc{irr}-\textsc{ipfv}-be-\textsc{1sg}  p.n. also pl.n. lineage  \textsc{irr}-\textsc{ipfv}-be { } \textsc{dem} \textsc{1du}.\textsc{poss}-offspring \textsc{dem}:\textsc{du} \textsc{lnk} pl.n. lineage apart.from \textsc{dem} apart.from other lineage \textsc{dem} \textsc{nmlz}:O-count not.exist:\textsc{fact} \\
 \glt `For instance suppose that both Dpalcan and I were from Taqrdo lineage, then our two children would only count as members of the Taqrdo lineage and no other lineage.' (140426 rJitpa, 13-15)
\end{exe}

The scope of  \forme{kɯnɤ} is generally exclusively on the constituent that it immediately follows, but there are cases where the scope is more extensive. In (\ref{ex:aZo.kWnA.akAsWso}), \forme{kɯnɤ} occurs between the pronoun \forme{aʑo} and the following participial verb form, which bears a \textsc{1sg} possessive prefix \forme{a-} coreferent with that pronoun (see also \ref{ex:aZWG.kWnA} below). The semantic scope of \forme{kɯnɤ} here is on the whole relative \forme{aʑo a-kɤ-sɯso} `(the things) that I want' rather than exclusively on the pronoun \forme{aʑo}.

\begin{exe}
\ex \label{ex:aZo.kWnA.akAsWso}
 \gll aʑo kɯnɤ a-kɤ-sɯso nɯ tɤ-stu-nɯ ra \\
 \textsc{1sg} also \textsc{1sg}.\textsc{poss}-\textsc{nmlz}:O-think \textsc{dem} \textsc{imp}-do.like-\textsc{pl} have.to:\textsc{fact} \\
 \glt `(I will do as you say, but) do also the things I want.' (2003kAndzwsqhaj2, 47)
\end{exe}

The focus marker \forme{kɯnɤ} is found with nouns or pronouns in core argument function, including S (\ref{ex:kWnA.nArca}), O (\ref{ex:nWXpWm.kWnA}), and semi-objects (\ref{ex:kWnA.mAsna}).  Examples with transitive subjects are presented below (\ref{ex:nWra.kWnA} and \ref{ex:Wzda.ra.kWnA}).

 \begin{exe}
\ex \label{ex:kWnA.nArca}
\gll aʑo kɯnɤ nɤ-rca ɣi-a ɕti  \\
\textsc{1sg} also \textsc{2sg}.\textsc{poss}-following come:\textsc{fact}-\textsc{1sg} be.\textsc{affirm}:\textsc{fact} \\
\glt `I am coming with you too.' (2011-05-nyima, 171)
 \end{exe}
 
   \begin{exe}
\ex \label{ex:nWXpWm.kWnA}
\gll    ma nɯ-χpɯm kɯnɤ kʰro mɤ-kɯ-fkaβ kɯ-fse ku-rɤʑi-nɯ  \\
lnk 3pl.poss-knee also much \textsc{neg}-\textsc{nmlz}:S/A-cover \textsc{nmlz}:S/A-be.like \textsc{ipfv}-stay-\textsc{pl} \\
\glt `(Gents) would (wear trousers that did) not cover much even their knees.'  (30-rkAsnom, 5) 
  \end{exe}
  
  \begin{exe}
 \ex \label{ex:kWnA.mAsna}
 \gll   ɯ-ru nɯra laʁdɯn ɯ-jɯ kɯnɤ mɤ-sna, ma mɤ-ngɯt. \\
 \textsc{3sg}.\textsc{poss}-trunk \textsc{dem}:\textsc{pl} tool \textsc{3sg}.\textsc{poss}-handle also \textsc{neg}-be.worth \textsc{lnk}  \textsc{neg}-be.strong:\textsc{fact} \\
 \glt `(The wood from) its trunk is not even good (enough to be used to make) tool handles, as it is not strong.' 
  \end{exe}

It also occurs with all types of oblique arguments and adjuncts, including genitive (\ref{ex:aZWG.kWnA}), dative (\forme{ɯ-ɕki} \ref{ex:nWCki.kWnA}),  locational adjuncts in \forme{tɕu} (\ref{ex:kutCu.kWnA}) or \forme{ri} (\ref{ex:ri.kWnA}), temporal adjuncts (\ref{ex:ftCAXcAl.kWnA}) or adjuncts expressing manner or cause (\ref{ex:nWtCu.kWnA}).  
  
   \begin{exe}
\ex \label{ex:aZWG.kWnA}
\gll aʑɯɣ kɯnɤ a-mpʰrɯmɯ a-pɯ-tɯ-sɯ-re ɯ-tɯ́-cʰa \\
\textsc{1sg}:\textsc{gen} also \textsc{1sg}.\textsc{poss}-divination \textsc{irr}-\textsc{pfv}-2-\textsc{caus}-look[III] \textsc{qu}-2-can:\textsc{fact} \\
\glt `Can you ask (the monk) to make a divination for me too?' (The divination, 31)
\end{exe}  
  
   \begin{exe}
\ex \label{ex:nWCki.kWnA}
\gll  tɯ-pi ɣɯ ɯ-nmaʁ ra nɯ-ɕki kɯnɤ `a-pi' tu-kɯ-ti ɕti ma nɯ ma kupa kɯ-fse ʑaka ɯ-rmi me. \\
\textsc{genr}.\textsc{poss}-elder.sibling \textsc{gen} \textsc{3sg}.\textsc{poss}-husband \textsc{pl} \textsc{3pl}.\textsc{poss}-\textsc{dat} also \textsc{1sg}.\textsc{poss}-elder.sibling \textsc{ipfv}-\textsc{genr}-say be.\textsc{affirm}:\textsc{fact} \textsc{lnk} \textsc{dem} apart.from Chinese \textsc{nmlz}:S/A-be.like each \textsc{3sg}.\textsc{poss}-name not.exist:\textsc{fact} \\
\glt  `One calls one's sister's husband (and others from his family) `my elder brother', there are no other special terms as in Chinese.' (140425 kWmdza05)
\end{exe}


  \begin{exe}
\ex \label{ex:kutCu.kWnA}
\gll  kutɕu kɯnɤ nɯ ɲɯ-fse, jɯfɕɯndʐi ra kɯ-xtɕɯ\redp{}xtɕi tɤ-ɣɤndʐo kɯ-fse ri, ɕɤxɕo tɕe kɯ-xtɕɯ\redp{}xtɕi ɲɯ-ʑi kɯ-fse \\
here also \textsc{dem} \textsc{sens}-be.like a.few.days.ago \textsc{nmlz}:S/A-\textsc{emph}\redp{}be.small \textsc{pfv}-be.cold \textsc{nmlz}:S/A-be.like \textsc{lnk} the.last.days \textsc{lnk} \textsc{nmlz}:S/A-\textsc{emph}\redp{}be.small \textsc{sens}-subside \textsc{nmlz}:S/A-be.like \\
\glt `It is like that here too, a few days ago the weather became a little cold, but the last days it has eased a bit.' (conversation, 141027)
  \end{exe}
  
    \begin{exe}
\ex \label{ex:ri.kWnA}
\gll   maldzɯ nɯ, nɯ ɯ-tʰɤcu tsa ri kɯnɤ ɣɤʑu. qarɣɤpɤt ɯ-rca ri kɯnɤ tu-ɬoʁ ɲɯ-ŋu. \\
plant.name \textsc{dem} \textsc{dem} \textsc{3sg}.\textsc{poss}-downstream a.little \textsc{loc} also exist:\textsc{sens} plant.name \textsc{3sg}.\textsc{poss}-among \textsc{loc} also \textsc{ipfv}-come.out \textsc{sens}-be \\
\glt `The \forme{maldzɯ} plant, it is also found in places of slightly lower altitude, but grows also in the same places as  \forme{qarɣɤpɤt} plants.' (18-qromJoR, 81-82)
    \end{exe}
    
\begin{exe}
\ex \label{ex:ftCAXcAl.kWnA}
\gll   kukutɕu ftɕɤχcɤl kɯnɤ <baonuanyi> tu-tɯ-ŋge pɯ-ɕti. \\
  here mid.summer also warm.clothes \textsc{ipfv}-2-wear[III] \textsc{pst}.\textsc{ipfv}-be.\textsc{affirm} \\
  \glt `Here you were wearing warm clothes even in mid summer.' (conversation, 141017)
    \end{exe}
    
    \begin{exe}
\ex \label{ex:nWtCu.kWnA}
\gll    tɕe nɯtɕu kɯnɤ ɯ-jaʁ ɯ-ntsi tɤɲi pjɯ-sɤtse, ɯ-jaʁ ɯ-ntsi kɯ tsʰitsuku ɲɯ-z-nɤme qʰe, \\
\textsc{lnk} \textsc{dem}:\textsc{loc} also \textsc{3sg}.\textsc{poss}-hand \textsc{3sg}.\textsc{poss}-one.of.a.pair erg various.things \textsc{ipfv}-\textsc{caus}-do[III] \textsc{lnk}  \\
\glt `Even like that (despite the pain in her legs), she props herself with a cane using one hand, and does all kinds of things with her other hand.' (14-tApitaRi, 52)
\end{exe}

Although \japhug{kɯnɤ}{also, even} can be combined with most postpositions and relator nouns as shown by the examples above, it is however incompatible with the ergative \forme{kɯ}. For instance, in  (\ref{ex:nWra.kWnA}), although the demonstrative pronoun \forme{nɯra} `they, those' in the second clause is the subject of the transitive verb \japhug{ndza}{eat}, it does not take the ergative \forme{kɯ} as would be expected (§ \ref{sec:A.kW}). The same applies to \forme{ɯ-zda ra} `his companions', subject of the transitive verb \forme{na-nɯ-ɕar-nɯ} `they looked for themselves' in (\ref{ex:Wzda.ra.kWnA}), 

  \begin{exe}
\ex \label{ex:nWra.kWnA}
\gll ɯ-pɯ nɯra li ju-ɣi-nɯ qʰe, nɯra kɯnɤ ɣɯ-tu-ndza-nɯ. \\
\textsc{3sg}.\textsc{poss}-young \textsc{dem}:\textsc{pl} again \textsc{ipfv}-come-\textsc{pl} \textsc{lnk} \textsc{dem}:\textsc{pl} also \textsc{cisloc}-\textsc{ipfv}-eat-\textsc{pl} \\
\glt `Its youngs also come and they too eat it.' (20-sWNgi, 59-60)
  \end{exe}
  
    \begin{exe}
\ex \label{ex:Wzda.ra.kWnA}
\gll   ɯ-zda ra kɯnɤ nɯ-rʑaβ tɯka na-nɯ-ɕar-nɯ ɲɯ-ŋu \\
\textsc{3sg}.\textsc{poss}-companion \textsc{pl} also \textsc{3sg}.\textsc{poss}-wife each \textsc{pfv}:3\fl{}3'-\textsc{auto}-search \textsc{sens}-be \\
\glt `His companions also took each a wife for himself (among the women of the island).' (2005Norbzang, 44)
    \end{exe}
    
The combinations $\dagger$\forme{kɯ kɯnɤ} or $\dagger$\forme{kɯnɤ kɯ} are unattested, and not accepted by native speakers. The contrast between absolutive and ergative noun phrases is therefore neutralized in additive or scalar focus with \forme{kɯnɤ}. Note that other focus markers, such as \forme{ri} and \forme{tɕi} (see \ref{ex:tCi.ndze} in § \ref{sec:ri.additive}) differ from \forme{kɯnɤ} in this regard.

Four distinct facts converge to suggest that the first syllable of \forme{kɯnɤ} is historically related to the ergative postposition \forme{kɯ}: (i) the incompatibility of co-occurrence of \forme{kɯnɤ} and \forme{kɯ}; (ii) the stress on the first syllable in \forme{kɯ́nɤ}; (iii) the similar \forme{-nɤ} element in the other scalar focus marker \japhug{cinɤ}{(not) even one} (§ \ref{sec:cinA}) (iv) the existence of the linker \forme{nɤ}, possibly of Tibetan origin (§ XXX). A detailed examination of this topic is however impossible on the basis Japhug-internal evidence, and will require extensive syntactic comparison between Gyalrong languages.

 \subsubsection{Correlative additive focus markers \forme{ri} and \forme{tɕi}} \label{sec:ri.additive} 
 The additive focus markers \forme{ri} and \forme{tɕi}  are used in enumerations, repeated after each noun referring to  members of a group, to focus on the fact that their referents share a common property (or properties that are semantically close enough), as in (\ref{ex:ri.kWsthWci.WWmpCar}) and (\ref{ex:tCi.tulhoR.cha}) (see additional examples in \citealt[313-314]{jacques14linking}).
 
 \begin{exe}
\ex \label{ex:ri.kWsthWci.WWmpCar}
 \gll  a-rʑaβ ri kɯstʰɯci ɲɯ-mpɕɤr, a-mbro ri kɯstʰɯci ɲɯ-ʑru, a-pɣɤtɕɯ ri kɯstʰɯci ɲɯ-mpɕɤr tɕe, \\
 \textsc{1sg}.\textsc{poss}-wife also so.much \textsc{sens}-be.beautiful  \textsc{1sg}.\textsc{poss}-horse also so.much \textsc{sens}-be.strong  \textsc{1sg}.\textsc{poss}-bird also so.much \textsc{sens}-be.beautiful \textsc{lnk} \\
 \glt `My wife is so beautiful, my horse so strong, my bird so beautiful.' (2003qachga, 116)
 \end{exe}
 
  \begin{exe}
\ex \label{ex:tCi.tulhoR.cha}
 \gll  ɴqiaβ tɕi tu-ɬoʁ cʰa, zrɯ tɕi tu-ɬoʁ cʰa, \\
 dark.side.of.the.mountain also \textsc{ipfv}-come.out can:\textsc{fact}   sunny.side.of.the.mountain also \textsc{ipfv}-come.out can:\textsc{fact}  \\
 \glt `It can grow in both the dark and the sunny sides of the mountains.' (17-thowum, 14)
  \end{exe}
  
The correlative focus markers \forme{ri} and \forme{tɕi} can occur after any noun phrase or postpositional phrase, including with the ergative  \forme{kɯ} as shown by (\ref{ex:tCi.ndze}), unlike the marker \japhug{kɯnɤ}{even, also} (see examples \ref{ex:nWra.kWnA} and \ref{ex:Wzda.ra.kWnA}, § \ref{sec:kWnA}).
  
  \begin{exe}
\ex \label{ex:tCi.ndze}
 \gll paʁ kɯ tɕi ndze, nɯŋa kɯ tɕi ndze, jla kɯ tɕi ndze.   \\
 pig \textsc{erg} also eat[III]:\textsc{fact}  cow \textsc{erg} also eat[III]:\textsc{fact}  hybrid.yak \textsc{erg} also eat[III]:\textsc{fact}  \\
 \glt `Pigs eat it, cows eat it, hybrid yaks eat it.' (18-NGolo, 171)
  \end{exe}

The focus markers \forme{ri} and \forme{tɕi} can have scope on only part of the noun/propositional phrase, and even on the relator nouns as in (\ref{ex:WNgW.tCi}).

   \begin{exe}
\ex \label{ex:WNgW.tCi}
 \gll   sɤtɕʰa ɯ-ŋgɯ tɕi ɣɤʑu, sɤtɕʰa ɯ-taʁ tɕi ʑo ɣɤʑu \\
 ground \textsc{3sg}.\textsc{poss}-inside also exist:\textsc{sens}  ground \textsc{3sg}.\textsc{poss}-inside also \textsc{emph} exist:\textsc{sens} \\
 \glt `It is found both inside the ground, and on the ground.' (25-GdAso, 17)
    \end{exe}
    
Alternatively, it is possible to enumerate distinct related properties of the same referent using \forme{ri} (this usage is not found with \forme{tɕi}), but that marker still follows the noun phrase (correlative \forme{ri} can follow verbs, but only in a specific construction, see \ref{ex:ri.kWmWm.ri} below). In this case the referent cannot be elided, and must be repeated in both clauses, at least as a third person pronoun \forme{ɯʑo} as in (\ref{ex:WlWz.ri.pjAxtCi}). 

  \begin{exe}
\ex \label{ex:WlWz.ri.pjArZi}
 \gll pʰaʁrgot nɯnɯ ɯʑo ri pjɤ-rʑi, ɯʑo ri pjɤ-tsʰu tɕe \\
 boar \textsc{dem} \textsc{3sg} also \textsc{ifr}.\textsc{ipfv}-be.heavy \textsc{3sg} also \textsc{ifr}.\textsc{ipfv}-be.fat \textsc{lnk} \\ 
\glt  `The boar, it was heavy and fat.' (140428 yonggan de xiaocaifeng-zh, 244)
 \end{exe}

A variant of this construction is found with internally-headed relative clauses in apposition, taking the third person pronoun \forme{ɯʑo} as head, as in (\ref{ex:WZo.ri.kWwxti}).

\begin{exe}
\ex \label{ex:WZo.ri.kWwxti}
\gll  [ɯʑo ri kɯ-wxti], [ɯʑo ri kɯ-sɤjlɯ\redp{}jloʁ] ci pjɤ-ŋu. \\
\textsc{3sg} also \textsc{nmlz}:S/A-be.big \textsc{3sg} also \textsc{nmlz}:S/A-\textsc{emph}\redp{}be.big \textsc{indef} \textsc{ifr}.\textsc{ipfv}-be \\
\glt `(The toad) was a big and disgusting (creature).' (150818 muzhi guniang, 86)
\end{exe}

 
The correlative construction can involve the possessor of an IPN, as in (\ref{ex:WlWz.ri.pjAxtCi}), where in the first clause the referent `the girl' is possessor of the intransitive subject (literally `her age was small', § XXX) and in second it corresponds to the intransitive subject, realized as a third person pronoun \forme{ɯʑo} `she'.

  \begin{exe}
\ex \label{ex:WlWz.ri.pjAxtCi}
 \gll tɕʰeme nɯ ɯ-lɯz ri pjɤ-xtɕi, ɯʑo ri pjɤ-mpɕɤr,  \\
 girl \textsc{dem} \textsc{3sg}.\textsc{poss}-age also \textsc{ifr}.\textsc{ipfv}-be.small \textsc{3sg} also \textsc{ifr}.\textsc{ipfv}-be.beautiful \\
\glt `The girl was young and beautiful.' (150909 hua pi-zh, 10)
 \end{exe}
 
 More complex correlations, involving different subjects and predicates related to another referent, are also possible as shown by example (\ref{ex:lWlu.kW}), where \forme{ri} occurs after the intransitive subject \japhug{tɯ-ci}{water}, after the transitive subject \japhug{lɯlu}{cat} with the ergative and after the finite verb \japhug{tu-ɕe}{it goes up} (on which see below and refer to § XXX).
 
 \begin{exe}
\ex   \label{ex:lWlu.kW}
\gll <yancong> ku-kɯ-rɤloʁ tɕe ɯ-taʁ tɯ-ci ri mɯ́j-ɣi lɯlu kɯ ri mɯ-ɲɯ́-wɣ-ɕaβ qapri tu-ɕe ri mɯ́j-cʰa tɕe \\
 chimney \textsc{ipfv}-\textsc{genr}:S/P-make.a.nest \textsc{lnk} \textsc{3sg}.\textsc{poss}-on \textsc{indef}.\textsc{poss}-water also \textsc{neg}:\textsc{sens}-come cat \textsc{erg} also \textsc{neg}-\textsc{ipfv}-\textsc{inv}-catch snake \textsc{ipfv}:\textsc{up}-go also \textsc{neg}:\textsc{sens}-can \textsc{lnk} \\
 \glt `(The sparrows) make their nest in the chimney, (because) water cannot come up there, the cats cannot catch them, and the snakes cannot go up there.' (22-kumpGatCW, 69)
 \end{exe}
 
 The marker \forme{ri} is homophonous with the locative \forme{ri} (§ \ref{sec:locative}), and in cases with an enumeration of locative adjuncts, there can be ambiguity between the two. In (\ref{ex:Xcha.ri.ci}), \forme{ri} is analyzed as a locative because of the position of the determiner \forme{ci}, and also because it can be replaced with other locative postpositions.
 
 \begin{exe}
\ex \label{ex:Xcha.ri.ci}
\gll   χcʰa ri ci, ɯ-ʁe ri ci ɯ-jme cʰɯ-ɬoʁ ɲɯ-ŋu. \\
right \textsc{loc} one  \textsc{3sg}.\textsc{poss}-left \textsc{loc} one \textsc{3sg}.\textsc{poss}-tail \textsc{ipfv}:\textsc{downstream}-come.out \textsc{sens}-be \\
\glt `It has one tail on the right, and one on the left.' (26-qro, 116)
\end{exe}

The marker \forme{ri} can follow verbs only if combined with an existential verb, a copula or a modal auxiliary verb as main predicate (meaning `both $X$ and $Y$' with positive copulas, and `neither $X$ nor $Y$' with negative ones). In this type of construction, verbs are mostly in non-finite form, as in (\ref{ex:ri.kWmWm.ri}). Examples with finite verbs however do exist; this topic is treated in § XXX. %ɲɯ-ɣɤwu ri kɯ-maʁ, ɲɯ-nɤre ri kɯ-maʁ kɯ-fse ɲɤ-k-ɤβzu-ci  ; tu-rɯɕmi ri mɤ-kɯ-khɯ, chɯ-nɯrɤɣo ri mɤ-kɯ-khɯ ci ɲɤ-k-ɤβzu-ci. ; tu-ndzur ri pjɤ-maʁ, ku-omdzɯ ri pjɤ-maʁ.

 \begin{exe}
\ex \label{ex:ri.kWmWm.ri}
 \gll   nɯ pɯ́-wɣ-ta ri  kɯroz kɯ-mɯm ri maŋe, kɯroz mɤ-kɯ-ɣɤ-mɲɤt ri maŋe qʰe, \\
 \textsc{dem} \textsc{pfv}-\textsc{inv}-put \textsc{lnk} specially \textsc{nmlz}:S/A-be.tasty also not.exist:\textsc{sens} specially \textsc{neg}-\textsc{nmlz}:S/A-\textsc{facil}-be.spoiled also not.exist:\textsc{sens} \textsc{lnk} \\
 \glt `When if one puts (a seal on the bread), there is nothing especially tasty about it, and nothing special concerning the preservation (of the bread).' (160706 thotsi, 27)
  \end{exe}
  

  
 \subsubsection{Scalar focus marker \forme{cinɤ}} \label{sec:cinA} 
 The focus marker \japhug{cinɤ}{(not) even one} exclusively occurs with a negative verb. Like \japhug{kɯnɤ}{also, even}, this marker has stress on the first syllable \forme{cínɤ}, which is obviously related to the numeral \japhug{ci}{one} (§ \ref{sec:one.to.ten}, § \ref{sec:indef.article}).
 
 The marker \forme{cinɤ} has scope over the constituent that immediately precedes it, generally a noun phrase including or consisting of a CN, as in (\ref{ex:tWrdoR.cinA3}), but also object and subject participial relative clauses as in (\ref{ex:zrWG.kAmto.cinA}), (\ref{ex:WrNa.WkWru.cinA}) and (\ref{ex:lukWpGaR.nW.cinA}).
 
 \begin{exe}
\ex \label{ex:tWrdoR.cinA3}
\gll tsuku kɯ qʰe tɯ-rdoʁ cinɤ mɤ-kɯ-mto tu. \\
some erg lnk one-piece even neg-nmlz:S/A-see exist:fact \\
\glt `There are some people who (cannot) even find a single one.' (20-grWBgrWB, 36)
 \end{exe} 

 \begin{exe}
\ex \label{ex:zrWG.kAmto.cinA}
\gll  ma tɕe jinde nɯ zrɯɣ kɤ-mto cinɤ maŋe. \\
\textsc{lnk} \textsc{lnk} nowadays \textsc{dem} louse \textsc{nmlz:P}-see even not.exist:\textsc{sens} \\
\glt `Nowadays there isn't even a single louse to be seen/one cannot even see a single louse.' (21-mdzadi, 77)
\end{exe} 

\begin{exe}
\ex \label{ex:WrNa.WkWru.cinA}
\gll ɯ-rŋa ɯ-kɯ-ru cinɤ ʑo pjɤ-me \\
3sg.poss-face 3sg.poss-nmlz:S/A-look even \textsc{emph} \textsc{ipfv}.\textsc{ifr}-not.exist \\
\glt `Not even one (of the thieves) looked at it/The (thieves) did not even so much as looked at it.' (140426 luozi he qiangdao)
\end{exe}

\begin{exe}
\ex \label{ex:lukWpGaR.nW.cinA}
\gll tɕe ɯ-ɲɯ-kɯ-ɣɤ-rkɯn nɯ ɲɯ-dɤn ma lu-kɯ-pɣaʁ nɯ tɯ-rdoʁ cinɤ ʑo maŋe \\
\textsc{lnk} \textsc{3sg}.\textsc{poss}-\textsc{ipfv}-\textsc{nmlz}:S/A-\textsc{caus}-be.few \textsc{dem} \textsc{sens}-be.many \textsc{lnk} \textsc{ipfv}:\textsc{upstream}-\textsc{nmlz}:S/A-plough \textsc{dem} one-piece even \textsc{emph} not.exist:\textsc{sens} \\
\glt `A lot of people diminish their fields, and not a single of them opens new fields.' (150903 friche, 6)
\end{exe}

In the case of relative clauses before \forme{cinɤ}, there is some ambiguity as to whether the scope of the focus marker is on the head of the relative or on the main verb of the relative clause, hence the two proposed translations above for (\ref{ex:zrWG.kAmto.cinA}) and (\ref{ex:WrNa.WkWru.cinA}).

It is not possible to use \forme{cinɤ} with scope over transitive subjects, followed by the ergative.

The form \forme{cinɤ} also occurs in the expression \forme{ŋu cinɤ maʁ kɯ} `in any case it is not', as in (\ref{ex:Nu.cinA.maR.kW}), literally `It is not even the case that...' ; in this construction, only the first verb \japhug{ŋu}{be} receives person indexation, as shown by (\ref{ex:Nua.cinA.maR.kW}). In addition to \japhug{ŋu}{be}, a few other verbs such as \japhug{fse}{be like} can occur with \forme{ci nɤ maʁ kɯ} `anyway X does not' .

 \begin{exe}
\ex \label{ex:Nu.cinA.maR.kW}
\gll qajdo kɯ tɕʰi mɤ-nɯ-ti ɕti nɤ, a-tɤ-nɯ-ti ma ŋu cinɤ maʁ kɯ, nɯ sɤznɤ kɯ-scɯ-scit rɤʑi-tɕi \\
crow \textsc{erg} what \textsc{neg}-\textsc{auto}-say:\textsc{fact} be.\textsc{affirm}:\textsc{fact} \textsc{lnk} \textsc{irr}-\textsc{pfv}-\textsc{auto}-say \textsc{lnk} be:\textsc{fact} even not.be:\textsc{fact} \textsc{sfp} \textsc{dem} \textsc{comp} \textsc{nmlz}:S/A-\textsc{emph}\redp{}happy stay:\textsc{fact}-\textsc{1du} \\
\glt `What would not a crow say (a crow tells only lies), let it say as it wants, in any case it is not (true), let us rather live (together) happily.' (28-qAjdoskAt, 28)
\end{exe} 

 \begin{exe}
\ex \label{ex:Nua.cinA.maR.kW}
\gll  kɯ-mɯrkɯ ŋu-a cinɤ maʁ kɯ  \\
\textsc{nmlz}:S/A-steal be:\textsc{fact}-\textsc{1sg} even not.be \textsc{sfp} \\
\glt `Anyway it is not me who is the thief.' (elicited)
\end{exe}

\subsubsection{Restrictive focus} \label{sec:restrictive.focus} 
 Japhug does not have a restrictive focus marker `only', and the only way to express this meaning is to combine the exceptive \japhug{ma}{apart from} (and its reduplicated variant \forme{mɯma} § \ref{sec:exceptive}) with a negative predicate. This can be a verb with a negative prefix as in (\ref{ex:XsArZaR}), or a negative existential verb as in (\ref{ex:Wmi.Wntsi.ma.me}).
 
 \begin{exe}
\ex  \label{ex:XsArZaR}
\gll   χsɤ-rʑaʁ ma mɯ-pɯ-tsu-a ɲɤ-sɯso ri χsɯ-xpa pjɤ-tsu tɕe,  \\
three-day apart.from \textsc{neg}-\textsc{pfv}-pass-\textsc{1sg} \textsc{ifr}-think \textsc{lnk} three-year \textsc{ifr}-pass \textsc{lnk} \\
\glt `He thought that he had spent only three days, but three years had passed.' (2011-4-smanmi, 178)
  \end{exe}
  
  \begin{exe}
\ex  \label{ex:Wmi.Wntsi.ma.me}
\gll  rkoŋɟɤl nɯnɯ, ɯ-mi ɯ-ntsi nɯ ma me kʰi.   \\
one.legged.demon \textsc{dem} \textsc{3sg}.\textsc{poss}-leg \textsc{3sg}.\textsc{poss}-one.of.a.pair \textsc{dem} apart.from not.exist:\textsc{fact} \textsc{hearsay} \\
\glt  `It is said that one-legged demons only had one leg.' (140510 rkoNJAl, 4)
  \end{exe}
  
The restrictive focus construction implies the presence of a noun phrase with a numeral or a CN when the restriction bears on the quantity, but restriction can also be qualitative, without quantifier, as in (\ref{ex:karGi.Zo.kWfse.ma.me}).

\begin{exe}
\ex \label{ex:karGi.Zo.kWfse.ma.me}
 \gll   ɯ-mat nɯnɯ na-lɤt ɕɯmɯma nɤ kɯ-ndɯ\redp{}ndɯβ ʑo ma me, karɣi ʑo kɯ-fse ma me  \\
 \textsc{3sg}.\textsc{poss}-fruit \textsc{dem} \textsc{pfv}:3\fl{}3'-throw just \textsc{lnk}  \textsc{nmlz}:S/A-\textsc{emph}\redp{}small \textsc{emph} apart.from not.exist:\textsc{fact} turnip.seed \textsc{emph} \textsc{nmlz}:S/A-be.like apart.from not.exist:\textsc{fact} \\
 \glt  `When the fruit of (xanthoxyllum) has just come out, there is only something very small, only like a turnip seed.'  (07-tCGom, 7)
  \end{exe}
  
The restrictive focus construction can be combined with a scalar focus in \forme{kɯnɤ} (see §  \ref{sec:kWnA}), as in (\ref{ex:ma.kWme.kWnA}). In this example, \forme{kɯnɤ} has scope over the subordinate clause \forme{stɯsti ma kɯ-me}, which is ambiguous between a participial headless relative (§ XXX) `consisting of only a female all alone' and a manner infinitival clause (§ XXX; in this case the gloss of \forme{kɯ-me} would be \textsc{inf}:\textsc{stat}-not.exist) `even (when) there is only a female all alone'.

  \begin{exe}
\ex \label{ex:ma.kWme.kWnA}
\gll  mu ma, stɯsti ma kɯ-me kɯnɤ cʰɯ-rɤŋgɯm ɲɯ-ɕti. \\
female apart.from alone apart.from \textsc{nmlz}:S/A-not.exist also \textsc{ipfv}-lay.eggs \textsc{sens}-be.\textsc{affirm} \\
\glt `Even only a female (hen) alone does lay eggs.' (150819 kumpGa, 11)
\end{exe}
  
 
 
 
%\japhug{ɕɯŋarɯra}{each better than the other}
% rɟɤlpu ɕɯŋarɯra kɯ ta-tʰu-nɯ ɕti ri, mɯ-tɤ-nɤla-j ɕti tɕe,
% 2003 qachga, 71

\subsection{Identity modifiers} \label{sec:identity.modifier}
There is no specific identity modifier `the same' in Japhug. The only way to express this meaning is to use the S-participle of the verb \japhug{naχtɕɯɣ}{be the same} (a denumeral verb of Tibetan origin, § \ref{sec:tibetan.numerals}, see also § \ref{sec:comitative} on the syntax of this stative verb and § XXX on its derivation) in a relative clause, as in (\ref{ex:tArmi.kWnaXtCWG}) (a possessor relative, § XXX). This participle is also used adverbially (see § XXX).

\begin{exe}
\ex \label{ex:tArmi.kWnaXtCWG}
\gll tɤ-rmi kɯ-naχtɕɯɣ pjɤ-dɤn wo kɤmɲɯ, nɤki kɯrɯ ra tɕe. \\
\textsc{indef}.\textsc{poss}-name \textsc{nmlz}:S/A-be.the.same \textsc{ifr}.\textsc{ipfv}-be.many \textsc{sfp} pl.n. \textsc{filler} Tibetan \textsc{pl} \textsc{lnk} \\
\glt `There were many people who had identical names, in Kamnyu, among the Tibetans.' (140522 tshupa, 161)
\end{exe}


There are two prenominal modifiers expressing non-identity in Japhug: \japhug{kɯmaʁ}{other} and the numeral \japhug{ci}{one}, which in prenominal position means `the other one' (in postnominal position, it is used as an indefinite article, see § \ref{sec:indef.article}). Both of these words can also be used as pronouns, though \forme{ci} requires to be combined with the demonstrative \forme{nɯ} in this usage (see § \ref{sec:other.pro}).

The modifier \forme{kɯmaʁ} is prenominal in its meaning `other', as in (\ref{ex:kWmaR.tWrme}). 

\begin{exe}
\ex \label{ex:kWmaR.tWrme}
\gll tɯ-zda nɯ ma kɯmaʁ tɯrme a-pɯ-me tɕe, kʰa ra aʁɤndɯndɤt ɲɯ-ɤ<nɯ>ɣro ɲɯ-ŋu ɲɯ-ti. \\
\textsc{genr}.\textsc{poss}-companion \textsc{dem} apart.from other person \textsc{irr}-\textsc{ipfv}-not.exist \textsc{lnk} house \textsc{pl} everywhere \textsc{ipfv}-<\textsc{auto}>play \textsc{sens}-be \textsc{sens}-say \\
\glt `(Our neighbour) says that if there are no other persons apart from family members, (the monkey) would play everywhere in the house.' (19-GzW2, 10)
\end{exe}

There are apparent examples of \japhug{kɯmaʁ}{other} in postnominal position, as in (\ref{ex:kWmaR.taXtW}) and (\ref{ex:kWmaR.tanWsWBzu}), but in such sentences \forme{kɯmaʁ} is a preverbal adverb, not a noun modifier, with a slightly different meaning `anew'. In (\ref{ex:kWmaR.taXtW}), the usage of \forme{kɯmaʁ} is very similar to its Chinese equivalent \ch{另外}{lìngwài}{other} in the corresponding Chinese sentence \zh{阿兰另外给我买了一部手机}, where the preverbal position of \ch{另外}{lìngwài}{other} clearly shows that it is not a noun modifier. 

\begin{exe}
\ex \label{ex:kWmaR.taXtW}
\gll <alan> kɯ a-<dianhua> kɯmaʁ ta-χtɯ \\
p.n. \textsc{erg} \textsc{1sg}.\textsc{poss}-phone other \textsc{pfv}:3\fl{}3'-buy \\
\glt `Alan bought me a new phone.' (conversation, 17-03-27)
\end{exe}

\begin{exe}
\ex \label{ex:kWmaR.tanWsWBzu}
\gll a-ʁi kɯ kʰa kɯmaʁ ta-nɯ-sɯ-βzu qʰe, \\
\textsc{1sg}.\textsc{poss}-younger.sibling \textsc{erg} house other \textsc{pfv}:3\fl{}3'-\textsc{auto}-\textsc{caus}-make \textsc{lnk} \\
\glt `My brother made himself a new house.' (14-tApitaRi, 304)
\end{exe}

The identity determiner \japhug{kɯmaʁ}{other} is grammaticalized from the S-participle of the verb \japhug{maʁ}{not be}, \forme{kɯ-maʁ} `who/which is not X', which is still widely used, as in (\ref{ex:tChWrtsAm.kWmaR}) and (\ref{ex:sthWci.kWmaR}).


%\begin{exe}
%\ex \label{ex:Wstu.kWmaR}
%\gll ɯ-stu kɯ-maʁ me, kɯki mɤ-kɯ-pe me \\
%\textsc{3sg}.\textsc{poss}-truth \textsc{nmlz}:S/A-not.be not.exist:\textsc{fact} dem.\textsc{prox} \textsc{neg}-\textsc{nmlz}:S/A-be.good not.exist:\textsc{fact} \\
%\glt ` (28-smAnmi, 16)
%\end{exe}

\begin{exe}
\ex \label{ex:tChWrtsAm.kWmaR}
\gll mɤʑɯ [tɕʰɯrtsɤm kɯ-maʁ] nɯnɯ tɕe, tú-wɣ-χtɕi ma nɯ ma kɤ-sqa (mɤ-ra) \\
yet type.of.tsampa \textsc{nmlz}:S/A-not.be \textsc{lnk} ipfv-inv-wash apart.from dem apart.from inf-boil neg-have.to:fact \\
\glt `The tsampa that is not `chu.rtsam', one needs to wash it, but not to boil it.' (2002tWsqar, 112)
\end{exe}

\begin{exe}
\ex \label{ex:sthWci.kWmaR}
\gll  [ɯ-rkɯ wuma ʑo stʰɯci kɯ-maʁ] nɯtɕu tɤ-ri ci kú-wɣ-lɤt \\
\textsc{3sg}.\textsc{poss}-side really \textsc{emph} so.much \textsc{nmlz}:S/A-not.be \textsc{dem}:\textsc{loc} \textsc{indef}.\textsc{poss}-thread once \textsc{ipfv}-\textsc{inv}-throw \\
\glt `One sews a thread at a place which is not too much on the border (of the patch)'. (12-kAtsxWb, 16)
\end{exe}

The modifier \forme{ci} differs from \forme{kɯmaʁ} in that it is necessarily definite, meaning `the other one', as in (\ref{ex:ci.rWdaR}), where it refers to an animal that it chased by lions, which was previously mentioned in the text.

\begin{exe}
\ex \label{ex:ci.rWdaR}
\gll ʑɯrɯʑɤri qʰe ci rɯdaʁ nɯ dɯxpa ma nɯ-kɤ-ndza ɯ-spa ɲɯ-ɕti qʰe, qʰe pjɯ-ndʐaβ qʰe mɯ-ɲɯ-cʰa qʰe, \\
progressively \textsc{lnk} one animal \textsc{dem} poor.of \textsc{lnk} \textsc{3pl}.\textsc{poss}-\textsc{nmlz}:P-eat \textsc{3sg}.\textsc{poss}-material \textsc{sens}-be.affirm \textsc{lnk} \textsc{lnk} \textsc{ipfv}-\textsc{anticaus}:make.fall \textsc{lnk} \textsc{neg}-\textsc{ipfv}-can \textsc{lnk} \\
\glt `The other animal, poor of him, it is their prey, progressively it falls down and cannot stand it anymore.' (20-sWNgi, 43)
\end{exe}

Interestingly, the determiner \forme{ci} does not have scope over other noun modifiers. For instance, in (\ref{ex:ci.tCheme.kWNAn}), the noun \japhug{tɕʰeme}{woman} occurs with an attributive adjective in participial form \forme{kɯ-ŋɤn} `who is evil' (a relative clause, see § \ref{sec:attributes}), but the meaning is not `the other evil woman' as could have been expected (since the woman who is the subject of the sentence is, by contrast, a kind person), and rather must be `the other woman, the evil one'. There is no pause in the recording that could lead us to suppose that \forme{kɯ-ŋɤn} here is an apposition -- it is rather a postnominal relative.

\begin{exe}
\ex \label{ex:ci.tCheme.kWNAn}
\gll nɤki, tɕʰeme nɯ ɯ-ɕki ɯ-kɯ-sɤja jo-ɕe, ci tɕʰeme kɯ-ŋɤn nɯ ɯ-ɕki. \\
\textsc{filler} women \textsc{dem} \textsc{3sg}-\textsc{dat} \textsc{3sg}.\textsc{poss}-\textsc{nmlz}:S/A-give.back \textsc{ifr}-go one woman \textsc{nmlz}:S/A-be.evil \textsc{dem} \textsc{3sg}-\textsc{dat} \\
\glt `She went to give it back to the woman, the other one, the evil woman.' (140515 jiesu de laoren, 90)
\end{exe}

%a-ʁi kɯnɤ tɯrme kha kɤ-sɯxɕe mɯ́j-khɯ qhe,

%ci qhɤjmbaʁ nɯ kɯ-jaʁ kɯ-fse nɯnɯ 
%mtshalu ɯ-cu tɕe nɤki,
%tɯ-mgo zmɤrɤβ kú-wɣ-nɯ-lɤt sna.
%16-RlWmsWsi
%li ci /ɯt/ ɯ-tɯphu nɯ tɤpu qhɤjmbaʁ tu-ti-nɯ ŋu tɕe,
\subsection{Attributes} \label{sec:attributes}

\subsubsection{Nominal modifiers}
\subsubsection{Participial relatives}
%Postnominal or head-internal? definiteness wuma ʑo ... kɯ-

\section{The structure of the noun phrase}

\section{Nominal predicates}
