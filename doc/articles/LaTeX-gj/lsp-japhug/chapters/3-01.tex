\chapter{Nominal morphology}
This chapter does not treat of grammatical categories expressed by independent words or clitics, such as number and grammatical relations (discussed in XXX and XXX), and focuses on possessive prefixes, compounding and  noun derivations. Nominalization (including lexicalized deverbal nouns) and denominal verbalization are treated in chapters XXX and XXX respectively.

\section{Possessive prefixes}  \label{sec:possessive.prefixes}
Japhug nouns are divided into two main categories, inalienably possessed nouns, which require a possessive prefix (\ref{sec:inalienably.possessed}) and common nouns which can occur with or without possessive prefix. 

\subsection{Possessive paradigm} \label{sec:possessive.paradigm}
The paradigm of possessive prefixes in Japhug is indicated in Table \ref{tab:possessive.prefixes}. It presents obvious commonalities with the personal pronouns (section \ref{sec:pers.pronouns}) and the indexation suffixes (section XXX), a question studied in more detail in XXX. 

\begin{table}[h] \centering
\caption{Possessive prefixes }\label{tab:possessive.prefixes}
\begin{tabular}{lllllllll} \lsptoprule
 Prefix & Person \\
\midrule
\forme{a-}  &		1\sg{} \\
\forme{nɤ-}  &			2\sg{} \\
\forme{ɯ-}  &			3\sg{} \\
\midrule
\forme{tɕi-}  &			1\du{} \\
\forme{ndʑi-}  &		2/3\du{} \\	
\midrule
\forme{i-}  &			1\pl{} \\
\forme{nɯ-}  &			2/3\pl{} \\
\midrule
\forme{tɯ-/tɤ-}  &			indefinite \\
\forme{tɯ-}  &			generic \\
\lspbottomrule
\end{tabular}
\end{table}

In the possessive paradigm, the contrast between second and third person is neutralized in the dual and plural, while it is preserved in pronouns and person indexation.

Unlike languages like Situ which have two series of possessive pronouns with the same initial consonant but distinct vocalism (\citealt[168-169]{linxr93jiarong}),\footnote{\citet[118-119]{prins16kyomkyo} analyzes the vowel as part of the nominal root.} Japhug preserves the vowel contrast \ipa{ɯ} vs \ipa{ɤ} only with the indefinite possessor form of inalienably possessed nouns; with definite possessors, the contrast is neutralized.

Possession cannot be expressed without a possessive prefix on the possessee. Possessive prefixes can be used on any noun, including recent borrowings from Chinese (or quasi-code switching), as \zh{老家} \forme{lǎojiā} `place of origin; old house' in (\ref{ex:aʑo.GW.alaojia}).

\begin{exe}
\ex \label{ex:aʑo.GW.alaojia}
\gll
aʑo ɣɯ a-<laojia> ɣɯ ɯ-lɤcu nɯre ri ku-rɤʑi-nɯ ŋu \\
\textsc{1sg} \textsc{gen} \textsc{1sg.poss}-old.house \textsc{gen} \textsc{3sg.poss}-upstream there \textsc{loc} \textsc{ipfv}-stay-\textsc{pl} be:\textsc{fact} \\
\glt `They live in a place upstream from my old house.' (14-tApitaRi, 238)
\end{exe}

In the case of first or second person possessors, it is possible to have simply a possessive prefix on the noun, (\forme{a-ɣɲi}, \forme{a-mbro} and \forme{a-ʁgra} in \ref{ex:ambro}), a personal pronoun and a possessive prefix (same person and number, as in \ref{ex:aʑo.ambro}) or even a pronoun, the genitive clitic \forme{ɣɯ} and a possessive prefix as in (\ref{ex:aʑo.GW.alaojia}).

 \begin{exe}
\ex \label{ex:ambro} 
\gll a-ɣɲi ci tɯ\redp{}tɯ-ŋu nɤ, a-mbro ɯ-lwa ɯ-taʁ kɤ-zo, a-ʁgra ci tɯ\redp{}tɯ-ŋu nɤ, a-mbro ɯ-jme ɯ-taʁ kɤ-zo \\
\textsc{1sg.poss}-friend \textsc{indef} \textsc{cond}\redp{}2-be:\textsc{fact} \textsc{lnk} \textsc{1sg.poss}-horse \textsc{3sg.poss}-mane \textsc{3sg}-on \textsc{imp}-land \textsc{1sg.poss}-enemy \textsc{indef} \textsc{cond}\redp{}2-be:\textsc{fact} \textsc{lnk} \textsc{1sg.poss}-horse \textsc{3sg.poss}-tail  \textsc{3sg}-on \textsc{imp}-land  \\
\glt `If you are my friend, land on my horse's mane, if you are my enemy, land on my horse's tail.' (2002qaCpa, 196)
\end{exe}
\begin{exe}
\ex \label{ex:aʑo.ambro}
\gll aʑo a-mbro nɤrwɯrɯnbotɕʰi ŋu, tɯ-sŋi χpaχtsʰɤt ci ɲɯ́-wɣ-tsɯm-a cʰa \\
\textsc{1sg} \textsc{1sg.poss}-horse p.n. be:\textsc{fact} one-day yojana \textsc{indef} i\textsc{pfv:west}-\textsc{inv}-take.away-\textsc{1sg} can:\textsc{fact} \\
\glt `My horse is Norbu Rinpoche, he can make me cross one yojana per day.' (2003smanmi2, 54)
\end{exe}

Possessive prefixes are also used to express beneficiaries as in (\ref{ex:ambro.tARndo.kWtso}), a topic discussed in section XXX.

 \begin{exe}
\ex \label{ex:ambro.tARndo.kWtso}
\gll a-mbro taʁndo kɯ-tso ci tɕi ra \\
\textsc{1sg.poss}-horse speech \textsc{nmls}:S/A-understand one also have.to:\textsc{fact} \\
\glt `I also need a horse who understands speech.' (2003kAndzwsqhaj2, 52)
\end{exe}

%aʑɯɣ a-kɯ-ra, aʑɯɣ kɯ-ra

\subsection{Inalienably possessed nouns} \label{sec:inalienably.possessed}

\subsection{Indefinite vs generic possessor}

\subsection{How did words like `sky', `earth' and `water' become inalienably possessed in Japhug?}

\section{Status constructus}

\subsection{Second member of compounds}
Irregular forms for second members of compounds are very rare.  The noun \japhug{ftɕɤru}{path in the middle of the fields} is a compound of \japhug{ftɕar}{summer} and \japhug{tʂu}{path} (such paths are made during summer to allow workers to work in the field without damaging the crops). The first element \forme{ftɕɤ-} is the \textit{status constructus} of \forme{ftɕar} (with loss of final consonant) and the form \forme{-ru} for the second member of the compound is a clue that \forme{tʂu} comes from earlier \forme{*t-ro} with a dental stop+\ipa{r} cluster changing to a retroflex affricate (see section XXX) -- the \forme{*t-} element being prefixal (perhaps a fossilized indefinite possessor prefix).

\section{Compounding}
\subsection{Noun-Noun compounds}
\subsection{Verb-Verb compounds}
\subsection{Adverb-Verb compounds}
\subsection{Noun-Verb compounds}
\subsection{Verb-Noun compounds}
Verb-Noun compounds are extremely rare in Japhug, as they are in general in Trans-Himalayan languages other than Chinese.  A possible example is \japhug{ndzɤpri}{brown bear}, compound of \japhug{pri}{bear} and \japhug{ndza}{eat} -- as shown by (\ref{ex:ndzApri}) from a text about bears, it is considered by some native speakers of Japhug as a man eater, though this explanation could be folk-etymology.

\begin{exe}
\ex \label{ex:ndzApri}
\gll tɕe ndzɤpri kɤ-ti nɯ tɕe tɯrme tu-kɯ-ndza ɲɯ-ŋgrɤl \\
\textsc{lnk} brown.bear \textsc{nmlz}:S-say \textsc{dem} \textsc{lnk} people \textsc{ipfv}-\textsc{genr}:S/P-eat \textsc{sens}-be.usually.the.case \\
\glt `The brown bear, it eats people.' (21-pri, 94)
\end{exe} 


We find several examples of nominal compounds whose structure is \forme{tɤ-}+Verb+Noun, where the verb is an adjectival stative verb. This category includes \japhug{tɤqiaβjmɤɣ}{lactarius sp.}, literally `bitter mushroom', from the noun \japhug{tɤjmɤɣ}{mushroom} and the verb \japhug{qiaβ}{be bitter}, and \japhug{tɤmbextsa}{type of shoes} from \japhug{tɯ-xtsa}{shoe} and \japhug{mbe}{be old}. These should not be analyzed as Verb-Noun compounds however, as the first element originates from a nominalized form of the verb (such as \japhug{tɤ-mbe}{old thing}, see section XXX on this derivation): they really are a subtype of Noun-Noun compounds.

\section{Noun class prefixes}
\section{Nominal derivations}
\subsection{Superlative}
While there is no adjectival superlative derivation in Japhug (superlative constructions are synthetic, see section XXX), we find nevertheless a derivation applied to locative nouns, expressing the most extreme location. As shown in Table \ref{tab:superlative.n}, it is built by adding an element \forme{-ɕɯ-} followed by a complete copy of the root of the noun; the resulting noun is still an inalienably possessed locative noun. ɛxample (\ref{ex:WqaCWqa}) illustrate the use of one of these forms.

\begin{table}
\caption{Superlative noun derivation} \label{tab:superlative.n}
\begin{tabular}{l|lll}
 \lsptoprule
\japhug{tɯ-ku}{head, top} & \japhug{ɯ-kuɕɯku}{the highest place} \\
\japhug{tɯ-qa}{root, paw, bottom} & \japhug{ɯ-qaɕɯqa}{the deepest place} \\
\japhug{tɯ-rkɯ}{side} & \japhug{ɯ-rkɯɕɯrkɯ}{the place most on the side} \\
 \lspbottomrule
\end{tabular}
\end{table}

\begin{exe}
\ex \label{ex:WqaCWqa}
\gll rɟɤmtsʰu ɯ-qaɕɯqa pjɯ-ɕe tɕe, nɯnɯ ɯ-kɤ-nɤ-mɯm nɯra ɕ-tu-nɯ-tɕɤt ɲɯ-ŋu. \\
ocean \textsc{3sg.poss}-bottom:\textsc{super} \textsc{ipfv}:\textsc{down}-go  \textsc{lnk} \textsc{dem} \textsc{3sg.poss}-\textsc{nmlz}:P-\textsc{trop}-be.tasty \textsc{dem:pl} \textsc{transloc-ipfv}-\textsc{auto}-take.out \textsc{sens}-be \\
\glt `(The sperm whale) goes to the lowest depths of the ocean and catches the things it likes to eat.' (160703 jingyu, 24)
\end{exe}
\section{Denominal adverbs}
\subsection{Comitative adverbs}
\subsection{Time adverbs}
