\chapter{Nominal morphology} \label{chap:nominal.morphology}
This chapter does not discuss grammatical categories expressed by independent words or clitics, such as postpositions (treated in chapter \ref{chap:postpositions.relators}), number and other noun modifiers (§ \ref{chap:noun.phrase}), and focuses on possessive prefixes, compounding and  noun derivations. Nominalization (including lexicalized deverbal nouns) and denominal verbalization are treated in chapters XXX and XXX respectively. 

The morphology of counted nouns (quantifiers, time nominals) is discussed in § \ref{sec:counted.nouns}, and that of nouns of location in chapter XXX.

\section{Possessive prefixes}  \label{sec:possessive.prefixes}
 Nouns in Japhug can be divided into four main subclasses, inalienably possessed nouns (IPN), alienably possessed nouns (APN), unpossessible nouns (UN) and counted nouns (CN), depending on the type of prefixes they can take.  The present section focuses on the first two, the ones that are compatible with possessive prefixes. UNs are treated in § \ref{sec:unpossessible.nouns}, and CNs in § \ref{sec:counted.nouns}.

\subsection{Possessive paradigm} \label{sec:possessive.paradigm}
The paradigm of possessive prefixes in Japhug is indicated in Table \ref{tab:possessive.prefixes}. It presents obvious commonalities with the personal pronouns (section \ref{sec:pers.pronouns}) and the indexation suffixes (section XXX), a question studied in more detail in XXX. 

\begin{table}[h] \centering
\caption{Possessive prefixes }\label{tab:possessive.prefixes}
\begin{tabular}{lllllllll} \lsptoprule
 Prefix & Person \\
\midrule
\forme{a-}  &		1\sg{} \\
\forme{nɤ-}  &			2\sg{} \\
\forme{ɯ-}  &			3\sg{} \\
\midrule
\forme{tɕi-}  &			1\du{} \\
\forme{ndʑi-}  &		2/3\du{} \\	
\midrule
\forme{i-}  &			1\pl{} \\
\forme{nɯ-}  &			2/3\pl{} \\
\midrule
\forme{tɯ-/tɤ-/ta-}  &			indefinite \\
\forme{tɯ-}  &			generic \\
\lspbottomrule
\end{tabular}
\end{table}

In the possessive paradigm, the contrast between second and third person is neutralized in the dual and plural, while it is preserved in pronouns and person indexation.

Unlike languages like Situ which have two series of possessive pronouns with the same initial consonant but different vowels (\citealt[168-169]{linxr93jiarongen}),\footnote{\citet[118-119]{prins16kyomkyo} analyzes the vowel as part of the nominal root.} Japhug preserves the vowel contrast \ipa{ɯ} vs \ipa{ɤ} only with the indefinite possessor form of inalienably possessed nouns; with definite possessors, the contrast is neutralized.

Stacking of possessive prefixes is not allowed in Japhug, with the exception of the combination of a definite possessor prefix with an indefinite possessor prefix \forme{tɯ-} or \forme{tɤ-} to turn an inalienably possessed noun into an alienably possessed one (see § \ref{sec:alienabilization}).

Unlike verbs (see § XXX), nouns whose stem begins in \forme{a-} are extremely few in Japhug. Nevertheless, as is the case with verbs, the vowel \forme{a-} merges with any prefixed element, so that nouns of this type do not have regular possessive forms. The only noun in \forme{a-} to commonly receive possessive prefixes is \japhug{araʁ}{liquor} (a loanword from \tibet{ཨ་རག་}{ʔa.rag}{liquor}; its possessive forms are highly anomalous: \textsc{1sg} \forme{aʑɤ-raʁ}, \textsc{2sg} \forme{nɤʑɤ-nɤ-raʁ} and \textsc{2pl} \forme{nɯʑɤ-nɯ-raʁ} (as in \ref{ex:nWZAnWraR}), combining the pronoun in \textit{status constructus} followed by the possessive prefix, which takes over the initial \forme{a-}.

\begin{exe}
\ex \label{ex:nWZAnWraR}
\gll nɯʑɤ-nɯ-raʁ ɯ́-ra \\
\textsc{2pl}-\textsc{2pl.poss}-liquor \textsc{qu}-have.to:\textsc{fact} \\
\glt `Do you need liquor?' (elicited)
\end{exe}

\subsubsection{The expression of possession} \label{ex:prefix.expression.of.possession}
Possession cannot be expressed without a possessive prefix on the possessee. Possessive prefixes can be used on nearly any noun (except the unpossessed nouns, see § \ref{sec:unpossessible.nouns}), including recent borrowings from Chinese (or quasi-code switching), as \zh{老家} \forme{lǎojiā} `place of origin; old house' in (\ref{ex:aZo.GW.alaojia}). They also occur on several non-finite verbal forms, including participles (see § \ref{sec:subject.participle.possessive} and § \ref{sec:object.participle.possessive}), bare infinitives (§ \ref{sec:bare.inf}) and degree nominals (§ \ref{sec:degree.nominals}).

\begin{exe}
\ex \label{ex:aZo.GW.alaojia}
\gll
aʑo ɣɯ a-<laojia> ɣɯ ɯ-lɤcu nɯre ri ku-rɤʑi-nɯ ŋu \\
\textsc{1sg} \textsc{gen} \textsc{1sg.poss}-old.house \textsc{gen} \textsc{3sg.poss}-upstream there \textsc{loc} \textsc{ipfv}-stay-\textsc{pl} be:\textsc{fact} \\
\glt `They live in a place upstream from my old house.' (14-tApitaRi, 238)
\end{exe}

In the case of first or second person possessors, it is possible to have simply a possessive prefix on the noun, (\japhug{a-ɣɲi}{my friend}, \japhug{a-mbro}{my horse} and \japhug{a-ʁgra}{my enemy} in \ref{ex:ambro}), a personal pronoun and a possessive prefix (same person and number, as in \ref{ex:aZo.ambro}) or even a pronoun, the genitive clitic \forme{ɣɯ} and a possessive prefix as in (\ref{ex:aZo.GW.alaojia}) (§ \ref{sec:gen.possession}).

 \begin{exe}
\ex \label{ex:ambro} 
\gll a-ɣɲi ci tɯ\redp{}tɯ-ŋu nɤ, a-mbro ɯ-lwa ɯ-taʁ kɤ-zo, a-ʁgra ci tɯ\redp{}tɯ-ŋu nɤ, a-mbro ɯ-jme ɯ-taʁ kɤ-zo \\
\textsc{1sg.poss}-friend \textsc{indef} \textsc{cond}\redp{}2-be:\textsc{fact} \textsc{lnk} \textsc{1sg.poss}-horse \textsc{3sg.poss}-mane \textsc{3sg}-on \textsc{imp}-land \textsc{1sg.poss}-enemy \textsc{indef} \textsc{cond}\redp{}2-be:\textsc{fact} \textsc{lnk} \textsc{1sg.poss}-horse \textsc{3sg.poss}-tail  \textsc{3sg}-on \textsc{imp}-land  \\
\glt `If you are my friend, land on my horse's mane, if you are my enemy, land on my horse's tail.' (2002qaCpa, 196)
\end{exe}

\begin{exe}
\ex \label{ex:aZo.ambro}
\gll aʑo a-mbro nɤrwɯrɯnbotɕʰi ŋu, tɯ-sŋi χpaχtsʰɤt ci ɲɯ́-wɣ-tsɯm-a cʰa \\
\textsc{1sg} \textsc{1sg.poss}-horse p.n. be:\textsc{fact} one-day yojana \textsc{indef} \textsc{ipfv:west}-\textsc{inv}-take.away-\textsc{1sg} can:\textsc{fact} \\
\glt `My horse is Norbu Rinpoche, he can make me cross one yojana per day.' (2003smanmi2, 54)
\end{exe}

It is possible to have a first singular possessive preceded by a first plural pronoun, as in (\ref{ex:iZo.amu}) (see § XXX for other examples of person mismatch involving \textsc{1pl} pronouns).

\begin{exe}
\ex \label{ex:iZo.amu}
\gll iʑo a-mu nɯ tʰamtʰam kɯrcɤsqaptɯɣ tʰɯ-azɣɯt ŋu. \\
\textsc{1pl} \textsc{1sg.poss}-mother \textsc{dem} now 81 \textsc{pfv}-reach  be:\textsc{fact} \\
\glt `My mother is now 81.' (2010-histoire09-2, 15)
\end{exe}

\subsubsection{Definiteness and obviation}
Nouns with a definite possessor in Japhug can be indefinite, unlike in most languages of Europe. They can occur with an indefinite determiner (example \ref{ex:ambro}  above). With a quantifier such as \japhug{tɯ-rdoʁ}{one piece} as in (\ref{ex:Wzda.tWrdoR}), a noun with a definite possessor is interpreted as referring to a certain number of persons out of a group (`one of his X').

 \begin{exe}
\ex \label{ex:Wzda.tWrdoR}
\gll tɤ-tɕɯ nɯ kɯ ɯ-zda tɯ-rdoʁ ɯ-pʰe to-ti, tɯ-rdoʁ nɯ kɯ li ci ɯ-pʰe tɕe ɲɤ-k-ɤ-sɯ-ɤmɯ-mtsʰɯ\redp{}mtsʰɤm-nɯ \\
\textsc{indef.poss}-son \textsc{dem} \textsc{erg} \textsc{3sg.poss}-companion one-\textsc{cl} \textsc{3sg-dat} \textsc{ifr}-say one-\textsc{cl}  \textsc{dem} \textsc{erg} again \textsc{indef} \textsc{3sg-dat} \textsc{lnk}   \textsc{ifr}-\textsc{evd}-\textsc{pass}-\textsc{caus}-\textsc{recip}-hear-\textsc{pl} \\
\glt `The boy told one of his companion, and that one another one, and they informed each other.' (2012Norbzang, 82)
\end{exe}

Unlike in Algonquian languages, but like in Mapudungun (\citealt{haude16symmetrical}), nouns with a third person possessor are not automatically obviative, and inverse marking on the verb is not required if a possessed noun is subject, and its possessor object of a transitive verb, as shown by example (\ref{ex:prox.naBde}) where the direct form \forme{na-βde} appears (see \citealt{jacques10inverse} for other examples, and § XXX on inverse marking). The inverse \forme{nɯ́-wɣ-βde} is also possible in exactly the same context -- example  (\ref{ex:obv.nWwGBde}) comes from the same text and refers to the same event.

\begin{exe}
\ex \label{ex:prox.naBde}
\gll ɯ-rʑaβ nɯ kɯ na-βde \\
\textsc{3sg.poss}-wife \textsc{dem} \textsc{erg} \textsc{pfv}:3\fl3'-throw.away \\
\glt `His wife left him.' (14-tApitaRi, 289)
\end{exe}

\begin{exe}
\ex \label{ex:obv.nWwGBde}
\gll
ɯ-rʑaβ cʰo ɯ-tɕɯ nɯ ʁnaʁna kɯ nɯ́-wɣ-βde qʰe \\
\textsc{3sg.poss}-wife \textsc{comit} \textsc{3sg.poss}-son \textsc{dem} both \textsc{erg} \textsc{pfv}-\textsc{inv}-throw.away \textsc{lnk} \\
\glt `His wife and his son left him.' (14-tApitaRi, 294)
\end{exe}

\subsubsection{Other uses of possessive prefixes} \label{sec:other.uses.poss.prefixes}
Possessive prefixes are also used to express beneficiaries, recipients and other oblique arguments, such as the `person needing' in the construction with the verb \japhug{ra}{need, have to},  as in (\ref{ex:ambro.tARndo.kWtso}).\footnote{See § \ref{sec:gen.beneficiary} for a more detailed account of the expression of beneficiaries in Japhug.}

 \begin{exe}
\ex \label{ex:ambro.tARndo.kWtso}
\gll a-mbro taʁndo kɯ-tso ci tɕi ra \\
\textsc{1sg.poss}-horse speech \textsc{nmls}:S/A-understand one also have.to:\textsc{fact} \\
\glt `I also need a horse who understands speech.' (2003kAndzwsqhaj2, 52)
\end{exe}

In the case of beneficiaries and recipients, if a genitive pronoun or genitive phrase is present, the presence of a possessive prefix is possible (\ref{ex:aZWG.akWra}) but not obligatory (\ref{ex:aZWG.kWra}), in particular in the case of possessed nouns that already have a definite possessor (\ref{ex:aZWG.Wlu}).

 \begin{exe}
\ex \label{ex:aZWG.akWra}
\gll aʑɯɣ a-kɯ-ra ci tu tɕe nɯ `ɣa' tɤ-ti ra \\
\textsc{1sg:gen} \textsc{1sg.poss}-\textsc{nmlz}:S/A-have.to \textsc{indef} exist:\textsc{fact} \textsc{lnk} \textsc{dem} yes \textsc{imp}-say have.to:\textsc{fact} \\
\glt `There is one thing I need, and you have to say `yes' to it.' (140429 qingwa wangzi, 47)
\end{exe}

 \begin{exe}
\ex \label{ex:aZWG.kWra}
\gll  aʑɯɣ kɯ-ra me \\
\textsc{1sg:gen} \textsc{nmlz}:S/A-have.to not.exist:\textsc{fact} \\
\glt `I don't need anything.' (Norbzang, 275)
\end{exe}

 \begin{exe}
\ex \label{ex:aZWG.Wlu}
\gll aʑɯɣ ɯ-lu ra \\
\textsc{1sg:gen} \textsc{3sg.poss}-milk have.to:\textsc{fact} \\
\glt `I want its milk.' (02-deluge2012, 12)
\end{exe}

We also find \textsc{3sg} possessive prefixes \forme{ɯ-} indexing not a possessor or a beneficiary/recipient, but anaphorically referring to a whole clause, as in (\ref{ex:Wcha.mWm}), where \forme{ɯ-cʰa} does not mean `its/his alcohol', but `the alcohol made in the fashion described in the previous clause'.

 \begin{exe}
\ex \label{ex:Wcha.mWm}
\gll  kɯɕɯŋgɯ tɕe iɕqʰa ʑmbrɯβɟaj nɯ kɯ nɯnɯ cʰa nɯ tú-wɣ-sɯ-ɕmi tɕe 
ɯ-cʰa mɯm tu-ti-nɯ pɯ-ŋgrɤl \\
in.former.times \textsc{lnk} the.aforementioned boat.oar \textsc{dem} \textsc{erg} \textsc{dem} alcohol \textsc{dem} \textsc{ipfv}-\textsc{inv}-\textsc{caus}-mix \textsc{lnk} \textsc{3sg}.\textsc{poss}-alcohol be.tasty:\textsc{fact} \textsc{ipfv}-say-\textsc{pl} \textsc{pst}.\textsc{ipfv}-be.usually.the.case \\
\glt `In former times, people used to mix the alcohol with boat oars, the alcohol (made this way) is tasty, they used to say.' (cha-31, 41-2)
\end{exe}


\subsubsection{The form of the \textsc{3sg} possessive prefix}
Japhug differs from other Gyalrong languages (Table \ref{tab:3sg.inv}, data from \citealt{jackson02rentongdengdi}, \citealt{gongxun14agreement}) in that the third person prefix is \textit{not} homophonous with the inverse prefix. 

\begin{table}
\caption{The form of the \textsc{3sg} possessive prefix in Gyalrong languages} \label{tab:3sg.inv} 
\begin{tabular}{lllll}
\toprule
& \textsc{3sg.poss} & inverse \\
\midrule
Japhug &  \forme{ɯ-} & \forme{ɣɯ-}/\forme{-wɣ-} \\
Tshobdun &  \forme{o-} & \forme{o-}  \\
Zbu &   \forme{wə-} & \forme{wə-} \\
Situ &    \forme{wə-} & \forme{wə-} \\
\bottomrule
\end{tabular}
\end{table}

Independently of the question of whether these two prefixes could be historically related, it is probable that Japhug is innovative here, as there is a trace of an allomorph \forme{-w-} similar to the inverse (see § XXX on the allomorphs of the inverse prefix) . The linker \japhug{núndʐa}{for this reason} originates from a phrase combining the demonstrative \japhug{nɯ}{this} (on which see § \ref{sec:anaphoric.demonstrative.pro} and § XXX) with the \textsc{3sg} possessed form of the noun \japhug{ɯ-ndʐa}{reason}; in this frozen compound, the third person possessive survives as colouring of the vowel of the demonstrative.

There is a possible trace of the expected allomorph \forme{ɣɯ-} (from proto-Gyalrong \forme{*wə-}) in the noun \japhug{ɣɯfsu}{friend}, etymologically `his equal'; the IPN \japhug{ɯ-fsu}{equal in size to}, which shares the same root, has a regular possessive prefix that is coreferent with the standard of comparison (as in \ref{ex:nWfsu} with the \textsc{3pl}; see § XXX on this construction). The APN \japhug{ɣɯfsu}{friend} is thus possibly a lexicalized equivalent of \japhug{ɯ-fsu}{equal in size to}, whose \textsc{3sg} prefix was frozen before the change from \forme{*ɣɯ-} to \forme{ɯ-} occurred.

\begin{exe}
\ex \label{ex:nWfsu}
\gll tɯrme kɯ-mbro ra nɯ-fsu jamar tu-zɣɯt ma mɤ-cha \\
person \textsc{nmlz}:S/A-be.high \textsc{pl}  \textsc{3pl.poss}-equal about \textsc{ipfv}-reach apart.from \textsc{neg}-can:\textsc{fact} \\
\glt `It can only grow about as high as a tall human.' (15-babW, 4)
\end{exe}


It remains unclear why the regular allomorph of the third person possessive is \forme{ɯ-} rather than expected \forme{ɣɯ-}. A possible explanation could be false segmentation, due to reanalysis with the genitive marker \forme{ɣɯ}, since the genitive can optionally occur between the possessor and the possessee, as in (\ref{ex:qaCpa.GW.WpW}). A pre-Japhug form such as \forme{*qaɕpa ɣɯ-pɯ} could have been misanalyzed as \forme{qaɕpa ɣɯ ɯ-pɯ} due to vowel fusion sandhi (section XXX), and a new allomorph \forme{ɯ-} extracted from such constructions.\footnote{The weakness of this hypothesis is that some Japhug dialects have \forme{kɯ} rather than \forme{ɣɯ} as their genitive marker.}

\begin{exe}
\ex \label{ex:qaCpa.GW.WpW}
 \gll nɯnɯ qaɕpa ɣɯ ɯ-pɯ ŋu tɕe, \\
 \textsc{dem} frog \textsc{gen} \textsc{3sg.poss}-young be:\textsc{fact} \textsc{lnk} \\
 \glt `It (the tadpole) is the young of the frog.' (hist-28-kWpAz, 220)
\end{exe} 


\subsection{Inalienably possessed nouns} \label{sec:inalienably.possessed}
IPNs differ from APNs in that they require the presence of a possessive prefix.  When the possessive prefix is definite, IPNs are not formally distinguishable from APNs; for instance, \japhug{a-pi}{my elder sibling} and \japhug{a-mbro}{my horse} both take the 1\sg{} \forme{a-} prefix and no direct clue indicates that the first noun is IPN and that the second one is APN.

The citation form however differs between IPN and APN: the former must take an indefinite possessor prefix (or in some cases a 3\sg{} \forme{ɯ-}), while the latter can occur without possessive prefix, as for instance \japhug{tɤ-pi}{elder sibling} (with the indefinite \forme{tɤ-}; the bare root $\dagger$\forme{pi} is not a correct form) vs \japhug{mbro}{horse} (without prefix).

IPN are divided into four classes depending on their citation form. There are three distinct forms for the indefinite possessor prefix (\forme{tɯ-}, \forme{tɤ-}, \forme{ta-}) whose distribution is not completely predictable on the basis of phonology or semantics (though some generalizations are provided below). In addition, some IPN always only take definite possessive prefixes. The contrast between these four classes is neutralized when the noun takes a definite possessor prefix (unlike in Situ, see \citealt[168-169]{linxr93jiarongen} and \citealt[118-119]{prins16kyomkyo}).

The most common allomorph of the indefinite possessor prefix is \forme{tɯ-}. IPN which select this allomorph, such as \japhug{tɯ-jaʁ}{hand}, have identical indefinite and generic possessor forms (see § \ref{sec:indef.genr.poss}).

The allomorph \forme{tɤ-} is also very common, in particular with kinship terms and some body parts (see \ref{sec:body.part} and \ref{sec:kinship}). The form \forme{ta-} is mainly a phonological variant of \forme{tɤ-}, occurring mainly with nouns whose stem begins with a uvular such as \japhug{ta-ʁrɯ}{horn} or \japhug{ta-ʁi}{younger sibling}. The contrast between \ipa{ɤ} and \ipa{a} in this prefix is very difficult to perceive before uvulars with some speakers (see § \ref{sec:A.vs.a.prefixes}), and the transcription adopted in this grammar (and the online corpus and dictionary) is based on the slow syllable-by-syllable pronunciation of these words by Tshendzin. Two IPNs, however, \japhug{ta-ma}{work} and \japhug{ta-mar}{butter}, have the \forme{ta-} allomorph with an initial \forme{m-}, probably originally due to vowel harmony (see § XXX).

The minimal pair between \japhug{tɤ-ma}{mother} and \japhug{ta-ma}{work} shows that this vowel contrast, however marginal, is distinctive, and that even if the two allomorphs \forme{tɤ-} and \forme{ta-} were originally phonologically conditioned, it is not the case any more in Kamnyu Japhug.

Some IPN never occur with indefinite possessor prefix, for instance \japhug{ɯ-tʰoʁ}{ground} is only attested with the 3\sg{} \forme{ɯ-} prefix (see \ref{sec:earth.IPN}). For some IPNs, the indefinite possessor form is difficult to elicit and in case of doubt the third singular form is given in the dictionary \citet{jacques16japhug} (for instance \japhug{ɯ-mdoʁ}{colour}). Future investigations may reveal an indefinite possessor form for some of these nouns.

When denominal verbs are derived from IPNs, the vocalism of the denominal prefix tends to be the same as that of the  indefinite possessor prefix (for instance \japhug{tɤ-βɟu}{mattress} \fl{} \japhug{nɤβɟu}{use as a mattress}, not $\dagger$\forme{nɯβɟu}), though there are exceptions (\japhug{tɯ-rpaʁ}{shoulder} \fl{} \japhug{mɤrpaʁ}{carry on the shoulder}), as discussed in § XXX.

By analogy with several non-finite verb forms, in particular the subject participle of transitive verbs and the bare infinitive, which index one argument (the object) by a possessive prefix (section XXX), the possessors of IPNs are considered to be \textit{core arguments}, while those of APNs are treated as \textit{adjuncts}. In other words, IPNs have a valency of 1 like intransitive verbs, while APNs have a valency of 0. The indefinite possessive prefix can be viewed as a valency-decreasing device, the nominal equivalent of passive and antipassive derivations, especially given its use in the alienabilization of IPNs (see \ref{sec:alienabilization}). Wider implications of the assumption that possessors of IPNs are core arguments are explored in the chapters on complementation (§  XXX).

\subsubsection{Conversion from non-IPN to IPN} \label{sec:apn.to.ipn}
Derivation from APNs to IPNs is not common in Japhug. A interesting case is that of \japhug{ɯ-ʁle}{reputation}, which originates from the APN \japhug{qale}{wind} with a reduced form \forme{ʁ-} of the class prefix \forme{qa-}, as some second members of compounds (see \ref{sec:second.member.alternation} and \ref{sec:class.prefixes}).

Conversion of CN (counted nouns, § XXX) to IPNs is a regular process, studied in more detail in § XXX.

\subsubsection{Body parts} \label{sec:body.part}
The great majority of body parts are IPN with the indefinite possessor \forme{tɯ-}. These include native words, but also borrowings from Tibetan such as \japhug{tɯ-qʰoχpa}{organs, state of mind} from Tibetan \tibet{ཁོག་པ་}{kʰog.pa}{insides} (see \ref{sec:uvular.harmony} on the phonology of this word).

Body parts IPN selecting the prefix \forme{tɤ-} are mainly liquids from the body such as \japhug{tɤ-se}{blood}, \japhug{tɤ-spɯ}{pus} and \japhug{tɤ-lu}{milk} (though some liquids also take the prefix \forme{tɯ-}, for instance \japhug{tɯ-ɕtʂi}{sweat}), hair (\japhug{tɤ-rme}{hair, fur}, \japhug{tɤ-kɤrme}{hair (head)}) and some body parts of animals (\japhug{tɤ-jme}{tail}, \japhug{tɤ-ŋkɯ}{pig skin}, \japhug{tɤ-rkʰom}{feather rachis}).

Parts of plants on the other hand mainly have the prefix \forme{tɤ-}, as \japhug{tɤ-jwaʁ}{leaf}, \japhug{tɤ-tsrɯ}{sprout}, \japhug{tɤ-zrɤm}{root} etc.

APNs are rare among body parts. Some nouns with the \forme{qa-} class prefix (see § \ref{sec:class.prefixes}) such as \japhug{qame}{mole} and \japhug{qambɣo}{earwax} referring to physical defects or excretions from the body are APNs. A similar situation is observed in Koyukon Athabaskan, where nouns `denoting certain temporary or abnormal parts of the body' are also APN (\citealt[660]{thompson96koyukon}), though in Koyukon this subclass is considerably larger than in Japhug.

The compound \japhug{tɯciste}{amniotic sac} from \japhug{tɯ-ci}{water} and \japhug{tɤ-ste}{bladder} has a \forme{tɯ-} which is originally an indefinite possessive prefix (see \ref{sec:earth.IPN}), but which has become frozen after being integrated into a compound (\ref{sec:frozen.indef}), as can be shown by (\ref{ex:WtWciste}). 

\begin{exe}
\ex \label{ex:WtWciste}
\gll ɯ-tɯciste cʰɤ-ndʑɣaʁ \\
\textsc{3sg.poss}-amniotic.sac \textsc{ifr}-\textsc{anticaus}:squeeze.out \\
\glt `Her waters have broken.' (elicited)
\end{exe}


\subsubsection{Kinship terms} \label{sec:kinship}
The great majority of kinship terms select the indefinite possessor prefix \forme{tɤ-} or \forme{ta-} (see § XXX for a description of the kinship system). The only kinship terms in \forme{tɯ-} are \japhug{tɯ-me}{daughter} (but this form is not attested in the text corpus) and \japhug{tɯlɤt}{second sibling}; however, the \forme{tɯ-} prefix in the latter word has become non-analyzable and this word has become an UN (see \ref{sec:unpossessible.nouns}). 

There are other UNs among kinship terms, including \japhug{woɬaʁ}{(bad) stepmother}, which derives from \japhug{tɤ-ɬaʁ}{mother's sister} by replacing the possessive prefix with an unidentified element \forme{wo-}, and the social relation collectives (\ref{sec:social.collective}). Being a UN, \japhug{woɬaʁ}{(bad) stepmother} cannot take possessive prefixes, and the forms of \japhug{tɤ-ɬaʁ}{mother's sister} are used instead (\forme{a-ɬaʁ} can mean `my (bad) stepmother').

Kinship terms do not commonly occur with the indefinite possessive prefix. For those denoting spouses, forms with the indefinite prefix are found in the expression `look for a wife/husband', as in (\ref{ex:tArZaB.WkWCar}).
 
\begin{exe}
\ex \label{ex:tArZaB.WkWCar}
 \gll `ŋoj tɯ-ɕe?' to-ti, `aʑo tɤ-rʑaβ ɯ-kɯ-ɕar ɕe-a' to-ti. tɕe `ndʑiʑo ŋoj tɯ-ɕe-ndʑi?' to-ti ri, `tɕiʑo tɤ-nmaʁ ɯ-kɯ-ɕar ɕe-tɕi' to-ti. \\
 where 2-go:\textsc{fact} \textsc{ifr}-say \textsc{1sg} \textsc{indef}.\textsc{poss}-wife \textsc{3sg}.\textsc{poss}-\textsc{nmlz}:S/A-search go:\textsc{fact}-\textsc{1sg} \textsc{ifr}-say \textsc{lnk} \textsc{2du} where 2-go:\textsc{fact}-\textsc{du} \textsc{ifr}-say \textsc{lnk} \textsc{1du} \textsc{indef}.\textsc{poss}-husband  \textsc{3sg}.\textsc{poss}-\textsc{nmlz}:S/A-search go:\textsc{fact}-\textsc{1du} \textsc{ifr}-say  \\
 \glt `She said: `Where are you going?'; He said: `I am looking for a wife. Where are you going?'; She said `We are looking for a husband.'' (2003-kWBRa, 42-45)
\end{exe}

Kinship terms also occur with the indefinite possessive prefix to discuss about kinship in abstract terms, as in (\ref{ex:tArpW.tAftsa}) (see also \ref{ex:tArpW} below). Note that in this example the verb is in generic A form, with the inverse prefix (see § XXX) implying that the nouns \japhug{tɤ-rpɯ}{mother's brother} and \japhug{tɤ-ftsa}{sister's child} are in generic use (see § XXX; for the absence of ergative marker in this sentence, see § XXX). The possessive prefix cannot be the generic possessor prefix \forme{tɯ-} (\japhug{tɯ-rpɯ}{one's mother's brother} and \japhug{tɯ-ftsa}{one's sister's child}), since only one argument in a given sentence can be generic (§ XXX) and even if this were possible, the meaning would be completely different (`One's uncle cannot marry one's nephew').

\begin{exe}
\ex \label{ex:tArpW.tAftsa}
\gll tɤ-rpɯ cʰo tɤ-ftsa ni ci kú-wɣ-pa mɤ-kɯ-kʰɯ ɲɯ-ŋu. \\
\textsc{indef.poss}-MB \textsc{comit} \textsc{indef.poss}-ZC \textsc{du} one \textsc{ipfv}-\textsc{inv}-make \textsc{neg}-\textsc{inf}:\textsc{stat}-be.possible  \textsc{sens}-be \\
\glt `Maternal uncles and sister's children cannot marry each other.  (140427 kWmdza stWnmW, 14)
\end{exe}

Some kinship terms have an extended meaning when they take the indefinite possessor prefix: they can alternatively be used to denote a class of humans based on gender and age. The noun \japhug{tɤ-tɕɯ}{son} also commonly means `boy' or even `male human' (regardless of age). The nouns \japhug{tɤ-wa}{father} and \japhug{tɤ-mu}{mother} can denote older people without reference to their children; translations such as `old man' and `old lady' are more appropriate in these cases, for instance in (\ref{ex:tAmu.ci}). The same applies to  \japhug{tɤ-wɯ}{grandfather} and \japhug{tɤ-wi}{grandmother}.


\begin{exe}
\ex \label{ex:tAmu.ci}
\gll praʁkʰaŋ  zɯ tɤ-mu ci ɯ-ku tɤ-kɯ-wɣrum ci zɯŋzɯŋ pjɤ-rɤʑi tɕe, \\
cave \textsc{loc} \textsc{indef.poss}-mother \textsc{indef} \textsc{3sg}.\textsc{poss}-head \textsc{pfv}-\textsc{nmlz}:S/A-be.white \textsc{idph}:II:white \textsc{ifr}.\textsc{ipfv}-stay \textsc{lnk} \\
\glt `In the cave, there was an old woman whose hair was completely white.' (2003sras, 69)
\end{exe}

\subsubsection{Relator nouns}
Relator nouns are a subset of IPNs which have been grammaticalized as quasi-adpositions and compensate for the relative dearth of postpositions in Japhug  (§ \ref{ex:postpositions}). Some are used to express basic grammatical relations (such as the dative, § \ref{sec:dative}), as well as most locative and temporal relations with noun phrases  and subordinate clauses (§ \ref{sec:relator.location}, § XXX).  A list of relator nouns and a detailed account of their functions is presented in § \ref{sec:relator.nouns}.

\subsubsection{Complement-taking nouns and relativizers}
IPNs can take nominalized or finite clauses as prenominal modifiers. When the head IPN is at the same time an argument or an adjunct inside its modifiying clause, that clause is considered to be a prenominal relative (§ XXX). The generic IPN \japhug{ɯ-spa}{its material} is in the process of becoming a relativizer when occurring with a prenominal relative, as discussed in § XXX. In other Gyalrongic languages, such as Khroskyabs (\citealt[519]{lai17khroskyabs}), former generic nouns have become fully grammaticalized as relativizers.

When the head noun is not a participant of the clause, the modifying clause is a complement clause (§ XXX, see \citealt[239-241]{jacques16complementation}). Adnominal complement clauses in Japhug generally occur with IPNs related to language such as \japhug{ɯ-tɕʰa}{information, news} and \japhug{ɯ-skɤt}{its language, its noise}, in particular with reported speech complements (§ XXX). 
%voir aussi \citealt[267]{post08nmlz.galo}, dans section sur la complémentation
\subsubsection{Property nouns} \label{sec:property.nouns}
Property nouns are a subclass of IPN that designate a entity that possesses a particular (mainly derogative) characteristic. They generally follow another noun as in (\ref{ex:penzi.WpW}) and (\ref{ex:kha.WNqra}), but not exclusively (\ref{ex:Wxso.tWrme}). In the /noun+property noun/ phrase, the latter is the syntactic head but semantically modifies the former (see § XXX on the various property modifiers in Japhug).  

\begin{exe}
\ex \label{ex:penzi.WpW}
 \gll <penzi> ɯ-pɯ, sɤlaŋpʰɤn ɯ-pɯ jamar ɲɯ-wxti cʰa  \\
 basin \textsc{3sg.poss}-little.one   basin \textsc{3sg.poss}-little.one  about \textsc{ipfv}-be.big can:\textsc{fact} \\
 \glt `It can grow about as big as a little basin.' (18-NGolo, 48)
\end{exe}

\begin{exe}
\ex \label{ex:kha.WNqra}
 \gll
kʰa ɯ-ɴqra tɕe znde ɯ-mbe ma tʰam kɯ-tu me. \\
house \textsc{3sg.poss}-broken.one \textsc{lnk} wall \textsc{3sg.poss}-old.one apart.from now \textsc{nmlz}:S/A-exist not.exist:\textsc{fact} \\ 
\glt `Now there is nothing (there), apart from a ruin and old walls. (140522 tshupa, 58)
\end{exe}

These phrases can be turned into compounds made of the first noun and a quasi-suffix corresponding to the property noun, as all diminutive and derogative suffixes described in § \ref{sec:diminutive} and \ref{sec:derogative} (Table \ref{tab:property.nouns}) have corresponding property nouns. In the case of \forme{sɤlaŋpʰɤn ɯ-pɯ}  from example (\ref{ex:penzi.WpW}) for instance, it is possible to say \japhug{sɤlaŋpʰɤn-pɯ}{little basin} as one word. In some cases the corresponding noun has \textit{status constructus} on the first element, as in \japhug{kʰɤɴqra}{ruin} from \japhug{kʰa}{house} and \japhug{ɯ-ɴqra}{broken one}, a form which occurs in (\ref{ex:khANqra}), in the same text as  (\ref{ex:kha.WNqra}) (referring to the same house). The opposite however is not always possible; for instance, lexicalized diminutives like \japhug{staχpɯ}{pea} from \japhug{stoʁ}{broad bean} cannot be turned into a phrase with \japhug{ɯ-pɯ}{little one} as second element.

\begin{exe}
\ex \label{ex:khANqra}
 \gll tɕe nɯ tɤtsoʁsta nɯnɯ kʰɤɴqra ɕti tʰam tɕe kɯ-rɤʑi me \\
\textsc{lnk} \textsc{dem} place.name \textsc{dem} ruin be:\textsc{affirm}:fact  now \textsc{lnk} \textsc{nmlz}:S/A-stay not.exist:\textsc{fact} \\
\glt `Now Tatsogsta (`the place of silverweed') is a ruin, nobody lives there.' (140522 tshupa, 56)
\end{exe}

\begin{table}
\caption{Property nouns and corresponding quasi-suffixes} \label{tab:property.nouns}
\begin{tabular}{l|ll}
\lsptoprule
Property Noun & Suffix& \\
\midrule
\japhug{ɯ-pɯ}{little one} & \forme{-pɯ} &diminutive \\
\japhug{ɯ-ɴqra}{broken one} &  \forme{-ɴqra} &derogative \\
\japhug{ɯ-do}{old one} &  \forme{-do} & \\
\japhug{tɤ-mbe}{old thing} &  \forme{-mbe} & \\
\japhug{ɯ-kʰe}{nasty} & \\
\japhug{ɯ-rqɯ}{cold thing} &  \forme{-rqɯ} & other \\
\japhug{ɯ-xso}{empty, normal} & \\
\japhug{ɯ-jlu}{uncooked} & \\
\japhug{ɯ-maŋ}{in big groups} & \\
\lspbottomrule
\end{tabular}
\end{table}

The property nouns \japhug{ɯ-do}{old one}  and \japhug{tɤ-mbe}{old thing} differ in that the former one is used for living things (including animals and plants), while the second occurs with inanimate objects. The quasi-suffix \forme{-rqɯ} is mainly used in \japhug{tɯ-cirqɯ}{cold water}.

The noun \japhug{ɯ-jlu}{uncooked} (used in particular with \japhug{stoʁ}{broad bean}) has become grammaticalized as a restrictive focus marker (§ \ref{sec:restrictive.focus}).

Property nouns are not commonly used with an indefinite possessor prefix; in attested examples, it is always \forme{tɤ-}. Their origins are diverse: \japhug{ɯ-pɯ}{little one} derives from \japhug{tɤ-pɯ}{offspring, young} (see \ref{sec:diminutive}), while \japhug{tɤ-mbe}{old thing}, \japhug{ɯ-kʰe}{nasty} and \japhug{ɯ-do}{old thing}  originate  from \japhug{mbe}{be old}, \japhug{kʰe}{be stupid} and \japhug{do}{be old (of plants)} by deverbal derivation (section XXX).  The property  noun \japhug{ɯ-maŋ}{in big groups} derives from \japhug{maŋ}{be many}, itself from Tibetan \tibet{མང་}{maŋ}{many}. Some \forme{tɤ-} prefixed nouns of verbal origin like \japhug{tɤkʰe}{idiot, fool} (from \japhug{kʰe}{be stupid}) may come from former property nouns.
 
The property noun \japhug{ɯ-xso}{empty, normal} is related to the verb \japhug{so}{be empty}; it originally comes from its subject participle (the regular form \japhug{kɯ-so}{empty} is still attested) with loss of vowel and fricativization of the velar participle prefix (see § XXX). It had no corresponding quasi-suffix, but does appear as second element in some compounds  (see for instance \ref{sec:collective}).

The most common meaning of \forme{ɯ-xso} is `normal, usual, common', a meaning already very different from the base verb. It occurs both before and after the noun with which it is linked (compare \ref{ex:Wxso.tWrme} and \ref{ex:tWrme.Wxso}). It is also used adverbially meaning `usually' (\ref{ex:Wxso.kurAZi}).

\begin{exe}
\ex \label{ex:Wxso.tWrme}
\gll ɯ-pa ɲɯ-kɯ-ɕe nɯ tɕe kɯmaʁ tɯrme, ɯ-xso tɯrme ra nɯ-tɕʰaʁra pjɤ-ŋu \\
\textsc{3sg.poss}-down \textsc{ipfv}:\textsc{west}-\textsc{nmlz}:S/A-go \textsc{dem} \textsc{lnk} other people \textsc{3sg.poss}-normal people \textsc{pl} \textsc{3pl.poss}-toilet \textsc{ifr.ipfv}-be \\
\glt `The toilets for other people, for normal people (not lamas), were on the (balcony) oriented towards west under it.' (08-kWqhi, 10)
\end{exe} 

\begin{exe}
\ex \label{ex:tWrme.Wxso}
\gll   nɤʑo tɯrme ɯ-xso tɯ-maʁ \\
2sg people  \textsc{3sg.poss}-normal  2-not.be:\textsc{fact} \\
\glt `You are not a normal human.' (150829 taishan zhi zhu-zh, 40)
\end{exe} 

\begin{exe}
\ex \label{ex:Wxso.kurAZi}
\gll  ɯ-xso ku-rɤʑi tɕe,  ɯ-βri nɯnɯ scoʁ-pɯ pɯ-kɤ-βʁum ʑo fse  \\
\textsc{3sg.poss}-normal \textsc{ipfv}-stay \textsc{lnk} \textsc{3sg.poss}-body \textsc{dem} ladle-\textsc{dim} \textsc{pfv}:\textsc{down}-\textsc{nmlz}:P-cover \textsc{emph} be.like:\textsc{fact} \\
\glt `(The ladybug) usually stays (in one place), its body looks like a little laddle put upside down.' (26-kWlAGpopo, 3)
\end{exe} 

The meaning `empty' is however also attested; in (\ref{ex:nWxso.chAnWlhoRnW}) it is used adverbially, and note that the possessive prefix is coreferent with the plural intransitive subject.

\begin{exe}
\ex \label{ex:nWxso.chAnWlhoRnW}
\gll toʁde tɕe tɕendɤre, nɯ-xso chɤ-nɯ-ɬoʁ-nɯ. \\
a.moment \textsc{lnk} \textsc{lnk} \textsc{3pl.poss}-empty \textsc{ifr}:\textsc{downstream}-\textsc{auto}-come.out-\textsc{pl} \\
\glt `A moment later, they came out empty-handed.' (140512 alibaba-zh, 34)
\end{exe} 

\subsubsection{Exclamative nouns}
A small class of IPN in Japhug can be used as exclamative verbless nominal predicates (§ XXX), sometimes with the sentence final particle \forme{nɯ} (§ XXX). This exclamative use of IPNs is found with degree nominals (see § XXX), and also with a few non-derived nouns, \japhug{tɯ-scawa}{poor of X} and \japhug{tɯ-kʰi}{how lucky of X}.

The IPN \japhug{tɯ-scawa}{poor of X} only occurs in the exclamative constructions, as in (\ref{ex:nWscawa}) and (\ref{ex:ascawa}). In example (\ref{ex:nWscawa}), the possessive prefix is coreferent with the entities that experience suffering (the pigs).

\begin{exe}
\ex \label{ex:nWscawa} 
\gll tsuku kɯ paʁndza ɲɯ-nɯ-pʰɯt-nɯ ɲɯ-ŋu ri, paʁ ra nɯ-scawa ma mɤ-mɯm ma ɯ-tɯ-qiaβ saχaʁ ʑo. \\
some \textsc{erg} hogwash \textsc{ipfv}-\textsc{auto}-pluck-\textsc{pl} \textsc{sens}-be \textsc{lnk} pig \textsc{pl} \textsc{3pl}.\textsc{poss}-poor.of \textsc{lnk} \textsc{neg}-be.tasty:\textsc{fact} \textsc{lnk} \textsc{3sg}.\textsc{poss}-\textsc{nmlz}:\textsc{degree}-be.bitter be.extremely:\textsc{fact} \textsc{emph} \\
\glt `Some people cut it (Sambucus) as hogwash, the poor pigs, it is so bitter.' (12-ndZiNgri, 30-31)
\end{exe}

The possessive prefix on this noun can also be coreferent not with the person suffering, but rather with another person who caused it, and expresses his apologies in this manner, as in (\ref{ex:ascawa}).

\begin{exe}
\ex \label{ex:ascawa}
\gll wo a-tɤɕime a-scawa, wo a-tɤɕime a-scawa \\
\textsc{interj} \textsc{1sg}.\textsc{poss}-lady \textsc{1sg}.\textsc{poss}-poor.of \textsc{interj} \textsc{1sg}.\textsc{poss}-lady \textsc{1sg}.\textsc{poss}-poor.of  \\ 
\glt `My lady, sorry (for what) I (have done to you).' (2014-kWlAG, 160)
\end{exe}

Alternatively, the indefinite possessive form \forme{tɯ-scawa} can occur, even if the person/entity experiencing misfortune is definite and known, as in (\ref{ex:tWscawa}).

\begin{exe}
\ex \label{ex:tWscawa}
\gll wo tɯ-scawa, ku-tɯ-tso mɯ-pɯ-ra \\
\textsc{interj} \textsc{indef}.\textsc{poss}-poor.of \textsc{ipfv}-2-understand \textsc{neg}-\textsc{pst}.\textsc{ipfv}-have.to \\
\glt `Alas and woe, you should not have known that.' (Norbzang2012, 166)
\end{exe}

The IPN \japhug{tɯ-kʰi}{how lucky of X} is another example of the nominal exclamative construction, as in (\ref{ex:nWkhi.Ge}), with the possessive prefix coreferent with the person experiencing good luck. This noun can also occur in the idiom \japhug{tɯ-kʰi+ŋgɯ}{be lucky}, as in (\ref{ex:akhi.YWNgW}).

\begin{exe}
\ex \label{ex:nWkhi.Ge}
\gll ɕɯ ɣɯ ŋu kɯ, nɯ-kʰi ɣe! \\
who \textsc{gen} be:\textsc{fact} \textsc{sfp} \textsc{3pl}.\textsc{poss}-how.lucky \textsc{sfp} \\
\glt `Whose are these, how lucky they are!' (Kubzang2003, 220)
\end{exe}

\begin{exe}
\ex \label{ex:akhi.YWNgW}
\gll a-kʰi ɲɯ-ŋgɯ \\
\textsc{1sg}.\textsc{poss}-lucky(1) \textsc{sens}-be.lucky(2) \\
\glt `I am lucky.' (140425 shizi puluomixiusi he daxiang-zh, 41)
\end{exe}

\subsubsection{Alienabilization of IPN} \label{sec:alienabilization}
 It is possible to turn an IPN  into an APN one by adding a definite possessive prefix before the indefinite one; this is the only case of possessive prefix stacking in Japhug. This process is very productive, and better illustrated by minimal pairs; the following examples involve the IPNs \japhug{tɯ-ci}{water}, \japhug{tɤ-lu}{milk} and \japhug{tɤ-muj}{feather}.
 
The noun \japhug{tɯ-ci}{water} with a definite possessor (\japhug{ɯ-ci}{its juice/water}) refers either to the juice of a plant, or to water in which a plant has been soaked as in (\ref{ex:Wci})

  \begin{exe}
\ex \label{ex:Wci}
 \gll  ɯʑo tɯ-ci kɯ-sɤ-ɕke ɯ-ŋgɯ pjɯ́-wɣ-ɣɤ-la, tɕe nɯ ɣɯ ɯ-ci ɯ-ŋgɯ nɯtɕu tɯ-mi pjɯ́-wɣ-ɣɤ-la tɕe nɯnɯ, χtɕoŋ nɯ ɲɯ-pʰɤn ɲɯ-ti-nɯ ri, \\
\textsc{3sg}  \textsc{indef.poss}-water \textsc{nmlz}:S/A-\textsc{deexp}-burn \textsc{3sg}-in \textsc{ipfv}-\textsc{inv}-\textsc{caus}-soak \textsc{lnk} \textsc{dem} \textsc{gen} \textsc{3sg.poss}-water \textsc{3sg}-in \textsc{dem:loc}  \textsc{genr.poss}-foot  \textsc{ipfv}-\textsc{inv}-\textsc{caus}-soak \textsc{lnk} \textsc{dem}, rheumatism \textsc{dem} \textsc{sens}-be.efficient \textsc{sens}-say-\textsc{pl} \textsc{lnk}  \\
 \glt `One puts it in hot water, and then one puts one's feet in that water, and it efficient against rheumatism, they say.' (20-sWrna, 144)
 \end{exe}

The alienabilized form \japhug{ɯ-tɯ-ci}{its water}, as in (\ref{ex:WtWci}), is used to talk about water given to an animal to drink, or water (artificially irrigated or not) absorbed by a plant.

 \begin{exe}
\ex \label{ex:WtWci}
 \gll  tɕeri ɯ-tɯ-ci wuma ʑo na-ʁzi tɕe, ɯ-tɯ-ci nɯ mɯ-pjɯ-mbrɤt ɲɯ-ra. \\
 but \textsc{3sg.poss}-\textsc{indef.poss}-water really \textsc{emph} \textsc{trop}-need:\textsc{fact} \textsc{lnk} \textsc{3sg.poss}-\textsc{indef.poss}-water dem \textsc{neg-ipfv-anticaus}:break \textsc{sens}-have.to \\
 \glt  `But it needs water a lot, it needs to have water continuously.'  (07-Zmbri, 11)
 \end{exe}
 
 When a definite possessor is present on the noun \japhug{tɤ-lu}{milk} in a form such as \japhug{ɯ-lu}{her milk}, that prefix refers to the animal producing the milk, as in (\ref{ex:Wlu}).
 
 \begin{exe}
\ex \label{ex:Wlu}
 \gll 
tɤ-pi kɯ-wxti nɯ kɯ nɯŋa ɣɯ ɯ-lu nɯ cʰondɤre  ɯ-ɕa nɯ to-nɯ-ndo. \\
\textsc{indef.poss}-elder.sibling \textsc{nmlz}:S/A-be.big \textsc{dem} \textsc{erg} cow \textsc{gen} \textsc{3sg.poss}-milk \textsc{dem} \textsc{comit} \textsc{3sg.poss}-meat \textsc{dem} \textsc{ifr}-\textsc{auto}-take \\
\glt `The elder brother took the cow's milk and meat.' (02-deluge2012, 19)
 \end{exe}

The form \japhug{ɯ-tɤ-lu}{his/its milk} with alienabilization is used on the other hand when indicating the person or animal drinking the milk, as in (\ref{ex:WtAlu}).

  \begin{exe}
\ex \label{ex:WtAlu}
 \gll tɕe ɯ-tɤ-lu pjɯ́-wɣ-rku tɕe nɯnɯ pjɯ-tsʰi qʰe, tɯ-sŋi tɕe tɯ-kʰɯtsa jamar tɯ-rdoʁ kɯ pjɯ-tsʰi ɲɯ-cʰa. \\
\textsc{lnk} \textsc{3sg.poss}-\textsc{indef.poss}-milk \textsc{ipfv}:\textsc{down}-\textsc{inv}-put.in  \textsc{lnk} \textsc{dem} \textsc{ipfv}-drink \textsc{lnk} one-day \textsc{lnk} one-bowl about one-\textsc{cl} \textsc{erg} \textsc{ipfv}-drink \textsc{sens}-can \\
\glt `People pour drink for it (the cat) to drink, and it drinks it, one (cat) can drink about a bowl of milk per day.' (21-lWLU, 45)
  \end{exe}

The IPN \japhug{tɤ-muj}{feather} takes as its possessor a bird (or a bird body part such as `wings'), as in (\ref{ex:Wmuj}).

    \begin{exe}
\ex \label{ex:Wmuj}
 \gll   jinde tɕe ɯ-kɯ-sat koŋla maŋe tɕe, nɯ qarma ɯ-muj kɯnɤ tɯ-jaʁ mɯ́j-ɣi wo \\
 nowadays \textsc{lnk} \textsc{3sg.poss}-\textsc{nmlz}:S/A-kill completely not.exist:C \textsc{lnk} \textsc{dem} crossoptilon \textsc{3sg.poss}-feather also \textsc{genr.poss}-hand \textsc{neg}:\textsc{sens}-come \textsc{sfp} \\
 \glt `Nowadays, nobody kills them, and one cannot get crossoptilon feathers.' (23-qapGAmtWmtW, 173)
  \end{exe}

Its alienabilized form, such as \japhug{ɯ-tɤ-muj}{his feather} in (\ref{ex:WtAmuj}), is used when the feather is detached from the body of the bird, and belongs to a human.
  
\begin{exe}
\ex \label{ex:WtAmuj}
 \gll tɤtɕɯpɯ kɯ-xtɕi nɯ ɣɯ ɯ-tɤ-muj nɯ li ɯ-tʰoʁ nɯtɕu pjɤ-nɯ-jɣɤt  \\
 boy:\textsc{dim} \textsc{nmlz}:S/A-be.small dem gen \textsc{3sg.poss}-\textsc{indef.poss}-feather \textsc{dem} again \textsc{3sg.poss}-ground \textsc{dem:loc} \textsc{ifr}:\textsc{down}-\textsc{auto}-go.back  \\
 \glt `The younger boy's feather fell back on the ground again.' (140510 sanpian yumao, 68)
\end{exe}
 
As the examples above show, alienabilized IPNs occur to refer to disconnected or severed body parts, for instance body parts removed from an animal that are used or owned by a human or another animal on which they do not grow. They are also used for bodily fluids that have left the body, or also clothes that are not worn but held in the hand. Similar phenomena are observed in other Gyalrong languages (see \citealt[140]{jackson98morphology} on Tshobdun).

The referent marked by the possessive prefix can be beneficiary as in (\ref{ex:WtAlu}) or possessor as in (\ref{ex:WtAmuj}).
 
Alienabilization is also observed with prenominal modifiers (§ \ref{sec:possessive.prefixes.prenominal}), in compounding, when the indefinite possessive prefix of an IPN is preserved in the final member of the compound (see § \ref{sec:possessive.prefix.second.compounds}), in comitative adverbs derived from IPNs (\ref{sec:comitative.adverb}) and in conversion from IPN to counted noun (§ \ref{sec:CN.IPN}). A related phenomenon is also the optional neutralization of possessive prefixes in relative clauses (§ XXX).
 
A lexicalized way of alienabilizing nouns is by compounding with a generic possessor. For instance, \japhug{pɣɤmuj}{feather} is an APN built from the status constructus of \japhug{pɣa}{bird} and the IPN \japhug{tɤ-muj}{feather}. Here the first element of the compound \forme{pɣɤ-} saturates the inalienable possessor without need to use the prefix \forme{tɤ-}.

\subsubsection{Frozen indefinite possessors} \label{sec:frozen.indef}
APNs with a disyllabic root whose first element is \forme{tɯ-} or \forme{tɤ-}, with the exception of loanwords such as \japhug{tɯrsa}{grave} (from \tibet{དུར་ས་}{dur.sa}{grave}), are mainly ancient IPNs whose indefinite possessive prefix \forme{tɯ-} has become frozen and reanalyzed as part of the root. Comparison with other Gyalrong languages can demonstrate that such reanalysis took place in Japhug.

For instance, the noun \japhug{tɯrme}{man} is APN in Japhug(as shown by example such as \ref{ex:atWrme}), but in Situ the 3\sg{} form of \forme{tə-rmî} `man' is \forme{wǝ-rmî}  (\citealt[183;197]{lin09phd}), showing that \forme{tə-} is the indefinite possessive prefix, cognate of Japhug \forme{tɯ-}. 

\begin{exe}
\ex \label{ex:atWrme}
\gll ci nɯ tɤ-tɕɯ nɯ χawo, nɯnɯ aʑo a-tɯrme nɯ a-pɯ-ŋu ndɤre, nɯ ɯ-tɯ-pe nɯ\\
one \textsc{dem} \textsc{indef}.\textsc{poss}-boy \textsc{dem} \textsc{interj} \textsc{dem} \textsc{1sg} \textsc{1sg.poss}-man \textsc{dem} \textsc{irr}-\textsc{ipfv}-be \textsc{lnk} \textsc{dem} \textsc{3sg}-\textsc{nmlz}:\textsc{degree}-be.good \textsc{fsp} \\
\glt `That boy, if only he could be my man, it would be so nice.' (2014-kWlAG, 418)
\end{exe}

This shift may be due to the fact that the 3\sg{} form is used in Situ in constructions where the non-possessed form is preferred in Japhug, such as in prenominal relatives (\citealt[190]{lin09phd}), and was therefore less prone to lexicalization. The stem \forme{-rme} of  \japhug{tɯrme}{man} is still attested as second element of compounds like \japhug{tɯ-pɤrme}{one year of life} (with \japhug{tɯ-xpa}{one year} as first element).  

The noun \japhug{tɤjmɤɣ}{mushroom} is APN, but the stem \forme{jmɤɣ-} appears as first element of compounds such as \japhug{jmɤɣni}{russula}, suggesting that it is a former IPN occurring without its indefinite possessor prefix in this compound (see \ref{sec:loss.possessive.prefix.compounds}). The status of \japhug{tɤjmɤɣ}{mushroom} as a former IPN is less surprising if one takes into account the likely etymological relationship with Chinese \zh{帽} \forme{mawH} `hat' (from \forme{*mˤuk-s}; etymology suggested by L. Sagart; see also the Tibetan cognate \tibet{རྨོག་}{rmog}{helmet}).  If the noun for `mushroom' in Japhug and other Gyalrongic languages comes from `hat' (cf Breton \forme{tog touseg} `toad hat' for `mushroom'), it is expected that it would become an IPN (like \japhug{tɤ-rte}{hat}), and for the indefinite possessor \forme{tɤ-} to become frozen after the noun ceases to be a term for head covers.
 
\subsubsection{Unusual IPNs in Japhug} \label{sec:earth.IPN}
While IPN membership of nouns such as body parts of kinship terms is expected from a crosslinguistic point of view, Japhug has IPNs for nouns denoting natural entities such as \japhug{tɯ-ci}{water}, \japhug{tɯ-mɯ}{sky} or \japhug{ɯ-tʰoʁ}{ground}, a highly unusual fact. There is no grand insight about Gyalrong Weltanschauung to be gained from this observation however; explanations should be sought in the etymology of these words, and solved on an item per item basis.  

The noun \japhug{tɯ-ci}{water} also means `juice' or `water in which X has been plunged into' with a definite possessor, as was seen in § \ref{sec:alienabilization}. Cognates are found in Core Gyalrong languages, but not in West Gyalrongic (Stau \forme{ɣrə} and Wobzi Khroskyabs \forme{jdə̂}, \citealt[610]{jacques17stau}) or elsewhere in the family, and it is therefore a good candidate for a Core Gyalrong lexical innovation. 

Japhug has a transitive verb \japhug{ci}{pour completely (of grains or liquids)}, from which a IPN \forme{*ɯ-ci} `(liquid/grain) that has been poured out' could have been regularly derived (see § XXX; similar to \japhug{ɯ-ndzɯ}{instruction, advice} from \japhug{ndzɯ}{educate} in example \ref{ex:WmW.mbWt} below). The meaning `water' would then be trivial narrowing of the meaning of this noun `water poured out', then replacing the older term for `water' still preserved in West Rgyalrongic. In this view, the status of \japhug{tɯ-ci}{water}  as an IPN is a consequence of its morphological derivation from a transitive verb (as there are no unprefixed deverbal nouns in Japhug, see § XXX). The stative verb \japhug{aci}{be wet} is then derived, after the semantic change, from the noun  \japhug{tɯ-ci}{water} by denominal derivation (section XXX) -- despite superficially looking like a passive of \japhug{ci}{pour completely (of grains or liquids)} (section XXX), it is only indirectly deriving from it. In addition to this verb, many derived forms have been created from the noun  \japhug{tɯ-ci}{water}, including the body part \japhug{tɯ-mci}{saliva} (\ref{ex:body.part.prefix}): these words show that the morphological formation by which they were constructed was still productive after the breakup of proto-Gyalrongic.


Concerning \japhug{tɯ-mɯ}{sky}, it superficially resembles a noun with non-analyzable \forme{tɯ-} prefixal element, but the status of this element as an indefinite possessor prefix can be ascertained with rare examples such as (\ref{ex:WmW.mbWt}).

\begin{exe}
\ex  \label{ex:WmW.mbWt}
\gll  ɯ-ndzɯ mɤ-kɯ-sɤŋo ɯ-mɯ mbɯt  \\
\textsc{3sg.poss}-instruction \textsc{neg}-\textsc{nmlz}-S/A-listen \textsc{3sg.poss}-sky \textsc{anticaus}:take.off:\textsc{fact} \\
\glt `Those who do not listen to advice from other people do not end well (their sky falls).' (elicited)
\end{exe}
There is no clear explanation of how this noun come have become IPN, but I propose here a tentative etymology. Cognates of \japhug{tɯ-mɯ}{sky} are attested elsewhere in Trans-Himalayan language, but mainly in languages that poorly preserve presyllables (for instance Yongning Na \forme{mv̩˥°}, \citealt[132]{michaud17yongning}). Yet, in Rawang, among the conservative languages that preserve presyllables, has a word \forme{dvmø̀}  `celestial being'  (\citealt[13]{lapolla01rawang}), with the same vowel correspondence to Japhug \ipa{-ɯ} as \forme{sharø}  `bone' with \japhug{ɕɤrɯ}{bone}. If  \forme{dvmø̀} is indeed cognate of Japhug \japhug{tɯ-mɯ}{sky},\footnote{Another potential cognate could be Rawang  \forme{muq} `sky, thunder', but the final glottal stop transcribed \forme{-q} is from a former \forme{*-k}, and this word is better compared to Situ \forme{ta-rmōk} `thunder' (\citealt[73]{zhang16bragdbar})} the word may originally have been disyllabic, and its first syllable reinterpreted as indefinite possessive; the form \japhug{ɯ-mɯ}{his sky} in (\ref{ex:WmW.mbWt}) would then be a backformation, an idea compatible with its very marginal character.

 
The Japhug noun \japhug{ɯ-tʰoʁ}{ground} cannot take any possessive prefix other than 3\sg{} \forme{ɯ-}, not even the indefinite possessor prefix. It has no known cognates in other Gyalrongic languages, but is a perfect match for a Tibetan word with the shape \forme{tʰog} (compare the other borrowed noun \japhug{tʰoʁ}{thunder} from Tibetan \tibet{ཐོག་}{tʰog}{thunder}). The etymology of this word requires a four steps scenario.

First, Japhug borrowed the Tibetan relator noun  \tibet{ཐོག་ཏུ་}{tʰog(tu)}{on} as \forme{ɯ-tʰoʁ} *`on' (not attested), adding a third person possessive prefix like all relator nouns (see § \ref{sec:relator.nouns} ). This relator noun was in competition with the existing native equivalent \japhug{ɯ-taʁ}{on}.\footnote{It is not surprising in Japhug to have several competing relator noun for the same functional slot; the same is true of the dative \forme{ɯ-ɕki} and \forme{ɯ-pʰe}, see section \ref{sec:dative}. }
  
  Second, it  became restricted to the collocation \forme{*sɤtɕʰa ɯ-tʰoʁ zɯ} `on the ground' (not attested), with the native locative \forme{zɯ} and the  noun of Tibetan origin \japhug{sɤtɕʰa}{earth}.
  
    Third, the collocation  \forme{*sɤtɕʰa ɯ-tʰoʁ zɯ}`on the ground', becoming tautological, was reduced to \japhug{ɯ-tʰoʁ zɯ}{on the ground} (attested).
 
 Fourth, the noun \japhug{ɯ-tʰoʁ}{ground} was created by backformation from the locative phrase \forme{ɯ-tʰoʁ zɯ} `on the ground'. The fact that the locative postposition \ipa{zɯ} is always optional (section \ref{sec:core.locative}) undoubtedly made this step easier.  Thus, Japhug attests an example of degrammation (see \citealt[135]{norde09degrammaticalization}) from a relator noun meaning `on' (with or without motion) to a common noun meaning `ground'. 

The three etymologies discussed above show that natural phenomena IPNs in Japhug were created by unrelated pathways.

\subsubsection{IPNs as adverbs}
Some IPNs can be used adverbially (§ XXX). In this function, the possessive prefix can be neutralized to indefinite possessor or third singular possessor, as \japhug{ɯ-stu}{truth, truly} in (\ref{ex:Wstu.Zo}), but in some cases it can also be coreferent with an argument, and different IPNs display different alignment patterns.

\begin{exe}
\ex \label{ex:Wstu.Zo}
\gll koŋla ɯ-stu ʑo a-pɯ-tɯ-rɤ-βzjoz, <zuoye> a-pɯ-tɯ-βze, \\
really \textsc{3sg}.\textsc{poss}-truth \textsc{emph} \textsc{irr}-\textsc{pfv}-2-\textsc{antipass}-study homework \textsc{irr}-\textsc{pfv}-2-do[III] \\
\glt `Study seriously, do your homework.' (conversation 140501, 86)
\end{exe}

The prefix of the IPN \japhug{ɯ-stu}{truth, truly} can be coreferent with the singular or the transitive subject (like the \textsc{2sg} prefix \forme{nɤ-} in \ref{ex:nAstu.Zo.WYWkWnWrgaa}), but never with the object, thus displaying an accusative alignment.

\begin{exe}
\ex \label{ex:nAstu.Zo.WYWkWnWrgaa}
\gll nɤʑo nɤ-stu ʑo ɯ-ɲɯ-kɯ-nɯ-rga-a nɤ, \\
\textsc{2sg} \textsc{2sg}.\textsc{poss}-really \textsc{emph} \textsc{qu}-\textsc{ipfv}-2\fl{}1-\textsc{appl}-like-\textsc{1sg} \textsc{lnk} \\
\glt `If you really love me" (150907 yingning-zh, 147)
\end{exe}

On the other hand, \japhug{ɯ-βra}{it is X's turn to...} presents an neutral alignement pattern: the possessive prefix can be coreferent with the intransitive subject, the object (as in \ref{ex:nAZo.nABra}), or the transitive subject (\ref{ex:nAZo.nABra.tAndze}).

%XXXXXX
%jisŋi tɕe aʑo a-βra ci zgoku tu-ɕe-a tɕe, ʑ-ɲɯ-nɤmɲole-

\begin{exe}
\ex \label{ex:nAZo.nABra}
\gll iɕqʰa nɤ-zda nɯ pɯ-sat-a ŋu tɕe, tham tɕe nɤʑo nɤ-βra pjɯ-ta-sat ra \\
just.before \textsc{2sg}.\textsc{poss}-companion \textsc{dem} \textsc{pfv}-kill-\textsc{1sg} be:\textsc{fact} \textsc{lnk} now \textsc{lnk} \textsc{2sg} \textsc{2sg}.\textsc{poss}-turn \textsc{ipfv}-1\fl{}2-kill have.to:\textsc{fact} \\
\glt `I just killed your companion, now it's your turn.' (elicited)
\end{exe}

\begin{exe}
\ex \label{ex:nAZo.nABra.tAndze}
\gll nɤj nɤ-βra tɤ-ndze  \\
\textsc{2sg} \textsc{2sg}.\textsc{poss}-turn \textsc{imp}-eat[III] \\
\glt `It is your turn to eat.' (elicited)
\end{exe}

Thus, the alignment pattern of each adverbial IPN must be specified.

\subsubsection{Biactantial IPNs} \label{sec:biactantial.ipn}
A few IPNs, such as words designating speech or presents, select more than one argument, and can be considered to be the nominal equivalent of ditransitive verbs (if APNs and IPNs are compared to intransitive and transitive verbs respectively, see § \ref{sec:inalienably.possessed}). Since only one argument however is marked by a possessive prefix on the noun (possessive prefix stacking is not possible except for alienabilization, see \ref{sec:alienabilization}) a choice has to be made as which of the two arguments, the speaker/giver or the addressee/recipient, is marked on the noun.

The possessive prefix of IPN \japhug{tɤ-pɤro}{present} always marks the giver; the recipient of the present (which is optional) receives genitive case. 

For instance, in (\ref{ex:apAro}) the recipient is marked by the genitive pronoun \forme{nɤʑɯɣ}, and the possessive prefix on the noun is \textsc{1sg}, the subject of the main verb. In (\ref{ex:nApAro}), the recipient is not overt, and the \textsc{2sg} prefix on the noun again corresponds to the transitive subject of \japhug{ɣɯt}{bring}.

\begin{exe}
\ex \label{ex:apAro}
\gll nɤʑɯɣ a-pɤro tɕʰi ju-ɣɯt-a ra? \\
\textsc{2sg:gen} \textsc{1sg}.\textsc{poss}-present what \textsc{ipfv}-bring-\textsc{1sg} have.to:\textsc{fact} \\
\glt `What present should I bring for you?' (140504 huiguniang-zh, 28)
\end{exe}

\begin{exe}
\ex \label{ex:nApAro}
\gll nɤʑo dɯxpa pɯ-tɯ-tu ma li nɤ-pɤro jɤ-tɯ-ɣɯt! \\
\textsc{2sg} hardship \textsc{pst}.\textsc{ipfv}-2-exist \textsc{lnk} again \textsc{2sg}.\textsc{poss}-present \textsc{pfv}-2-bring \\
\glt `Thank you, you brought another present (for me).' (elicited)
\end{exe}

Other biactantial nouns use possessive prefixes to indicate the recipient rather than the agent. From instance, the IPN \japhug{tɤ-rkuz}{parting present} (a rare example of \forme{-z} nominalization suffix in Japhug, see \ref{sec:z.nmlz}) always marks the recipient, as in (\ref{ex:nArkuz}), never the agent -- the form \japhug{nɤ-rkuz}{your parting present} with \textsc{2sg} possessive prefix can only mean `a parting present for you', not `the parting present you give to me/him'.

\begin{exe}
\ex \label{ex:nArkuz}
\gll  kɯki nɤ-rkuz ŋu \\
\textsc{dem}:\textsc{prox} \textsc{2sg}.\textsc{poss}-parting.present be:\textsc{fact} \\
\glt `This is a parting present for you.' (28-smAnmi, 266)
\end{exe}

In other cases, the alignment of possessive prefixes on an IPN depends on the  particular construction where it appears. For instance, the ICN \japhug{tɯ-tɕʰa}{information, news} marks the recipient when used with the verbs \japhug{kʰo}{give} or \japhug{tu}{exist}, but has neutral alignment in other contexts. In (\ref{ex:atCha.mWnakho}), the indirective verb \japhug{kʰo}{give} (§ XXX) does not index the recipient, whose only mark is the possessive prefix on \japhug{a-tɕʰa}{news for me}. Changing the prefix to the third singular \forme{ɯ-} to refer to the subject here would be agrammatical (see however § XXX on hybrid indirect speech).


\begin{exe}
\ex \label{ex:atCha.mWnakho}
\gll  a-tɕɯ kɯ a-tɕʰa mɯ-na-kʰo. \\
\textsc{1sg}.\textsc{poss}-son erg \textsc{1sg}.\textsc{poss}-news \textsc{neg}-\textsc{pfv}:3\fl{}3'-give \\
\glt `I have not heard from my son (My son did not give me any response).' 
\end{exe}

However, in other constructions, for instance with the verb \japhug{ɣɯt}{bring}, there are no such constraints on the use of possessive prefixes on \japhug{tɯ-tɕʰa}{information, news}: for instance, in (\ref{ex:WtCha.pjWGWta}), although the recipient is second person dual, \forme{ɯ-tɕʰa} takes the \textsc{3sg} prefix, coreferent with the preceding complement clause (using second person singular \forme{nɤ-tɕʰa} here would be agrammatical).

\begin{exe}
\ex \label{ex:WtCha.pjWGWta}
\gll `ma-nɯ-tɯ-ɣɤwu-ndʑi tɕe aʑo tu-ɕe-a tɕe atu ɕ-tu-tʰe-a tɕe,
ndʑi-pa ndʑi-ma ni ɯ-ɲɯ-nɯjʁo-ndʑi kɯ' ɯ-tɕʰa pjɯ-ɣɯt-a \\
\textsc{neg}-\textsc{imp}-2-cry-\textsc{du} \textsc{lnk} \textsc{1sg} \textsc{ipfv}:\textsc{up}-go-\textsc{1sg} \textsc{lnk} up \textsc{transloc}-\textsc{ipfv}-ask[III]-\textsc{1sg} \textsc{lnk} \textsc{2du}.\textsc{poss}-father \textsc{2du}.\textsc{poss}-mother \textsc{du} \textsc{qu}-\textsc{ipfv}-scold-\textsc{du} \textsc{qu} \textsc{3sg}.\textsc{poss}-information \textsc{ipfv}:\textsc{down}-bring-\textsc{1sg} \\
\glt `Don't cry, I will go up there, ask whether your parents will scold you' and come back to tell you.' (2003-kWBRa, 15)
\end{exe}

Alignment effects are also found with APN. For instance, the APN \japhug{skɯrma}{present} can also optionally take a possessive prefix, which is always coreferent with the recipient, not with the agent, as shown by (\ref{ex:askWrma}), where \forme{nɤ-mu ɣɯ ɯ-skɯrma} means `a present (sent) to your mother' and \forme{a-skɯrma} can only mean `a present sent to me' (from a text explaining the meaning difference between \forme{skɯrma}, \forme{tɤ-pɤro} and \forme{tɤ-rkuz}).\footnote{Japhug has three words that can be translated as `present': \forme{tɤ-pɤro} is used for presents one hands in to the recipient in person, \forme{tɤ-rkuz} is a parting present one gives before a person leaves a place, and \forme{skɯrma} is a present that one transmits to the recipient through the help of another person.}

\begin{exe}
\ex \label{ex:askWrma}
\gll nɤ-mu ɣɯ tʰɯci tu-rke-a tɕe ju-tɯ-tsɯm tɕe nɯnɯ nɤ-mu ɣɯ ɯ-skɯrma ju-sɯ-ɣɯt-a ŋu nɤ-mu kɯ a-tɤ-rke tɕe, a-jɤ-tɯ-ɣɯt tɕe, tɕe nɯnɯ li a-skɯrma jɤ-kɤ-sɯ-ɣɯt ŋu \\
\textsc{2sg}.\textsc{poss}-mother \textsc{gen} something \textsc{ipfv}-put.in[III]-\textsc{1sg} \textsc{lnk} \textsc{ipfv}-2-take.away \textsc{lnk} \textsc{dem} \textsc{2sg}.\textsc{poss}-mother \textsc{gen} \textsc{3sg}.\textsc{poss}-present  \textsc{ipfv}-\textsc{caus}-bring-\textsc{1sg} be:\textsc{fact} \textsc{2sg}.\textsc{poss}-mother  \textsc{erg} \textsc{irr}-\textsc{pfv}-put.in[III] \textsc{lnk} \textsc{irr}-\textsc{pfv}-2-bring \textsc{lnk} \textsc{lnk} \textsc{dem} again \textsc{1sg}.\textsc{poss}-present \textsc{pfv}-\textsc{nmlz}:P-\textsc{caus}-bring be:\textsc{fact} \\
\glt `When I prepare something for your mother and you take it to her, (I can say) `I sent a present to you mother', if your mother prepares something and you bring it (to me), it is `a present sent to me.'' (def-skWrma, 18-19)
\end{exe}
 
\subsection{Indefinite vs generic possessor} \label{sec:indef.genr.poss}
The generic possessive prefix \forme{tɯ-} is formally identical to the indefinite possessor prefix of some IPNs, but must be strictly distinguished from it. Four criteria can be used to determine if a \forme{tɯ-} prefix is generic, rather than indefinite.

First, the generic possessor prefix appears on APNs, as in example (\ref{ex:tWlaXtCha}) with \japhug{tɯ-kʰa}{one's house} and \japhug{tɯ-laχtɕʰa}{one's things}, the generic forms of \japhug{kʰa}{house} and \japhug{laχtɕʰa}{thing}.

\begin{exe}
\ex \label{ex:tWlaXtCha}
\gll tɕe  	aʁɤndɯndɤt  	ʑo  	ku-zo  	qhe  	ɯ-qe  	ku-lɤt  	qʰe	wuma  ʑo  	tɯ-kʰa  	cʰo  	tɯ-laχtɕʰa  	ra  	sɯ-ɴqʰi.  \\
\textsc{lnk} everywhere \textsc{emph} \textsc{ipfv}-land \textsc{lnk} \textsc{3sg.poss}-feces \textsc{ipfv}-throw \textsc{lnk} really \textsc{emph} \textsc{genr.poss}-house \textsc{comit} \textsc{genr.poss}-thing \textsc{pl} \textsc{caus}-be.dirty:\textsc{fact} \\
\glt `(Flies) land everywhere, shit on it and make one's houses and things dirty.' (25 akWzgumba, 59)
\end{exe}

Second, the generic \forme{tɯ-} occurs on IPNs that normally select the \forme{tɤ-} indefinite possessive prefix, such as \japhug{tɯ-rɟit}{one's child} and \japhug{tɯ-rpɯ}{one's maternal uncle} in examples (\ref{ex:tWrJit}) and (\ref{ex:tWrpW}), by contrast with the citation forms  \japhug{tɤ-rɟit}{child} and \japhug{tɤ-rpɯ}{maternal uncle}.

\begin{exe}
\ex \label{ex:tWrJit}
\gll nɯ 	kɯ-fse 	tɕe 	tɯʑo 	tɯ-rɟit 	kɯnɤ 	ʑa 	mɤ-sci 	tu-ti-nɯ \\
\textsc{dem} \textsc{nmlz}:S-be.like \textsc{lnk} \textsc{genr} \textsc{genr.poss}-child also early \textsc{neg-fact}:be.born \textsc{ipfv}-say-\textsc{pl} \\
\glt `People say that in this way, one's child will be born late.' (27 qartshaz, 111)
\end{exe}

\begin{exe}
\ex \label{ex:tWrpW}
\gll  tɯ-rpɯ 	ɯ-rɟit 	ɯ-ɕki 	tɕe 	tɕe 	``a-rpɯ a-ɬaʁ" 	tu-kɯ-ti 	ŋu. \\
\textsc{genr.poss}-uncle \textsc{3sg.poss}-offspring \textsc{3sg-dat} \textsc{lnk} \textsc{lnk} \textsc{1sg.poss}-uncle \textsc{1sg.poss}-aunt \textsc{ipfv-genr}-say  be:\textsc{fact} \\
\glt `One has to say `my maternal uncle, my maternal aunt to one's maternal uncle's sons and daughters.' (140425 kWmdza01, 69)
\end{exe}

The use of the generic possessive \japhug{tɯ-rpɯ}{one's maternal uncle} in (\ref{ex:tWrpW}) can be contrasted with the indefinite possessed form with \forme{tɤ-} in example (\ref{ex:tArpW}).

\begin{exe}
\ex  \label{ex:tArpW}
\gll
nɤʑo 	tɤ-rpɯ 	ɯ-rɟit 	a-pɯ-tɯ-ŋu, 	tɕe 	tɕe 	aʑo 	kɯ 	`a-rpɯ' 	tu-ti-a 	kɯ-ra.  \\
\textsc{2sg} \textsc{indef.poss}-uncle \textsc{3sg.poss}-offspring \textsc{irr-ipfv}-2-be \textsc{lnk} \textsc{lnk} \textsc{1sg} \textsc{erg}  \textsc{1sg.poss}-uncle \textsc{ipfv}-say-\textsc{1sg} \textsc{nmlz:S/A}-have.to  \\
\glt `If you are the maternal uncle's son, (and I am the nephew) I have to say `my uncle' (to you).'  (hist140425 kWmdza, 114)
\end{exe}

Third, in the case of IPNs whose indefinite possessive is \forme{tɯ-}, such as \japhug{tɯ-mtɕʰi}{mouth}, the indefinite and generic forms are homophonous, but are nevertheless distinguishable. In the case of a generic form the generic pronoun \japhug{tɯʑo}{one} (\ref{sec:genr.pro}) can always be added as in (\ref{ex:tWZo.tWmtChi}). 

\begin{exe}
\ex  \label{ex:tWZo.tWmtChi}
\gll tɯʑo sɤz kɯ-mna, kɯ-ɤʑɯχtso ra a-pɯ-ŋu nɤ,  tɯʑo tɯ-mtɕʰi maŋtaʁ nɯtɕu ɲɯ-ɬoʁ ŋu. \\
\textsc{genr} \textsc{comp} \textsc{nmlz}:S/A-be.better \textsc{nmlz}:S/A-be.clean \textsc{pl} \textsc{irr}-\textsc{ipfv}-be \textsc{lnk}  \textsc{genr} \textsc{genr.poss}-mouth above \textsc{dem:pl} \textsc{ipfv}-come.out be:\textsc{fact} \\
\glt `If (one uses the bowl of) someone who is cleaner than oneself, the (pimple) will appear above one's mouth.' (25-khArWm, 11)
\end{exe}

Additionally, a generic noun such as \japhug{tɯrme}{person} can occur as possessor of a noun with a generic possessive prefix as in (\ref{ex:tWrme.tWCa}). This usage is similar to that found in other generic constructions (see § XXX).

\begin{exe}
\ex  \label{ex:tWrme.tWCa}
\gll tɯrme ɣɯ tɯ-ɕa ɯ-mdoʁ tsa asɯ-ndo kɯ-fse \\
people \textsc{gen} \textsc{genr.poss}-flesh \textsc{3sg.poss}-colour a.little \textsc{prog}-take:\textsc{fact} \textsc{nmlz}:S/A-be.like \\
\glt `It has a little the colour of human flesh.' (14-sWNgWJu, 97)
\end{exe}

Even when the generic pronoun or a generic noun is not present, it is possible to identify generic possessors, as they are coreferent with the generic argument indexed on the verb (by \forme{kɯ-} for S and P, and \forme{wɣɯ-} for A, see § XXX). For instance, in (\ref{ex:genr.tWmtChi}), we know that  the \forme{tɯ-} prefixes in \japhug{tɯ-mtɕʰi}{one's mouth} and  \japhug{tɯ-ɕɣa}{one's teeth} are generic and not indefinite possessor because they refer to the same generic human as the transitive subject of the verbs \japhug{pʰɯt}{take out} and \japhug{ndza}{eat} in the previous clause, marked by the inverse prefix.

\begin{exe}
\ex  \label{ex:genr.tWmtChi}
\gll
tɕe ɲɯ́-wɣ-pʰɯt tɕe tú-wɣ-ndza ŋgrɤl ri, wuma ʑo tɯ-mtɕʰi cʰo tɯ-ɕɣa ra ɲɯ-sɯɣ-ɲaʁ ŋu. \\
\textsc{lnk} \textsc{ipfv}-\textsc{inv}-take.out \textsc{lnk} \textsc{ipfv}-\textsc{inv}-eat be.usually.the.case:\textsc{fact} but really \textsc{emph}  \textsc{genr.poss}-mouth \textsc{comit} \textsc{genr.poss}-tooth \textsc{pl}  \textsc{ipfv}-\textsc{caus}-be.black be:\textsc{fact} \\
\glt `One can pluck it and eat it, but it causes one's mouth and teeth to become black.' (11-qarGW, 70) 
\end{exe}

Fourth, possessed case markers such as the dative \forme{ɯ-ɕki} (§ \ref{sec:dative} ) do not have indefinite possessive forms, and therefore if prefixed in \forme{tɯ-}, it will always mark a generic possessor, as in (\ref{ex:tWCki}) -- such forms are often preceded by the generic pronoun \japhug{tɯʑo}{one} anyway.

\begin{exe}
\ex  \label{ex:tWCki}
\gll ma tɯ-ɕki wuma ʑo ʑɣɤ-sɯ-ɤrmbat tɕe núndʐa kʰe tu-ti-nɯ ɲɯ-ŋu. \\
\textsc{lnk} \textsc{genr-dat} really \textsc{emph} \textsc{refl}-\textsc{caus}-be.near:\textsc{fact} \textsc{lnk} for.this.reason stupid:\textsc{fact} \textsc{ipfv}-say-\textsc{pl} \textsc{sens}-be \\
\glt `It easily comes near oneself, so people call it `stupid'.' (23-scuz, 62) 
\end{exe}

\subsubsection{The generic possessor as a first person marker} \label{sec:generic.tW.1sg}
As in the generic verbal forms (§ XXX), the generic possessive prefixes can be used as an indirect way to express first person singular or plural. In example (\ref{ex:tWrkW.mWjrAZinW}) the generic as first person and the first person are used in two contiguous clauses, both referring to the narrator.

\begin{exe}
\ex  \label{ex:tWrkW.mWjrAZinW}
\gll tɕe tɯ-mu tɯ-wa ra tɯ-rkɯ mɯ́j-rɤʑi-nɯ tɕe, aʑo a-wi ci pɯ-tu. \\
\textsc{lnk} \textsc{genr}.\textsc{poss}-mother  \textsc{genr}.\textsc{poss}-father \textsc{ra}  \textsc{neg}:\textsc{sens}-stay-\textsc{pl} \textsc{lnk} \textsc{1sg} \textsc{1sg}.\textsc{poss}-grandmother \textsc{indef} \textsc{pst}.\textsc{ipfv}-exist \\ 
\glt `My parents were not by my side, but I had a grandmother (to take care of me).' (2010-09, 13)
\end{exe}

\subsubsection{Comparative perspectives} \label{sec:indef.t.comparative}
Indefinite and generic possessive dental stop prefixes are found in all Gyalrong languages (\citealt{jackson98morphology}), but only indirect traces exist in Khroskyabs  (\citealt[155]{lai17khroskyabs}). 

Outside of Gyalrongic, potential cognates of these prefixes include the `relational prefix' \forme{tə-} in Ao (\citealt[84-85]{coupe07mongsen}, as first noticed by \citealt[141-2]{wolfenden29outlines}) and some \forme{d-} or \forme{g-} prefixes in body parts in Tibetan (see \citealt{jacques14snom}).

\subsection{Prenominal modifiers} \label{sec:possessive.prefixes.prenominal}
In noun clauses with prenominal modifiers, possessive prefixes on the head noun are generally \textsc{3sg}, and express a possessive relation between the nouns, as in (\ref{ex:XsAr.WloR}).  This type modifier also occur with the genitive, as is discussed in § \ref{sec:gen.other}.

\begin{exe}
\ex \label{ex:XsAr.WloR}
\gll kɯstʰɯci ʑo pɣɤtɕɯ kɯ-mpɕɤr nɯ χsɤr ɯ-loʁ ɯ-ŋgɯ tu-rke-a ndɤre, tɕʰi nɯ mɤ-nɯ-fse? \\
such \textsc{emph} bird \textsc{nmlz}:S/A-be.beautiful \textsc{dem} gold \textsc{3sg}.\textsc{poss}-nest \textsc{3sg}.\textsc{poss}-inside \textsc{ipfv}-put.in[III]-\textsc{1sg} \textsc{lnk} what \textsc{dem} \textsc{neg}-\textsc{auto}-be.like:\textsc{fact} \\
\glt `Such a beautiful bird, what wrong could there be to put it in the golden nest?' (2012-qachGa, 41)
\end{exe}

If the head noun is an IPN, it can also undergo alienabilization and the possessive prefix is changed to the indefinite possessor (§ \ref{sec:alienabilization}). Thus in (\ref{ex:XsAr.tAsno}) we find \forme{χsɤr tɤ-sno} `golden saddle' with the indefinite possessor prefix \forme{tɤ-}; \forme{χsɤr ɯ-sno} with the \textsc{3sg} possessive prefix is also attested in the same text.

\begin{exe}
\ex \label{ex:XsAr.tAsno}
\gll si tɤ-sno nɯ tʰa-sɤndu nɤ, χsɤr tɤ-sno tʰa-nɯ-ta ɲɯ-ŋu, \\
wood \textsc{indef}.\textsc{poss}-saddle \textsc{pfv}:3\fl{}3'-exchange \textsc{lnk} gold  \textsc{indef}.\textsc{poss}-saddle \textsc{pfv}:3\fl{}3'-put \textsc{sens}-be \\
\glt `He exchanged the wooden saddle for the golden one.' (2003qachga,111)
\end{exe}

When the whole modifier+head noun complex is possessed however, the possessor is rarely marked by a possessive prefix on the head noun; rather, the prefix occurs on the leftmost noun of the phrase, as in (\ref{ex:aXsAr.tArte}), where the \textsc{1sg} prefix \forme{a-} occurs on the modifier \japhug{χsɤr}{gold}, and alienabilization of the head noun \japhug{tɤ-rte}{hat} (compare with the form \japhug{a-rte}{my hat} when no prenominal modifier is present). %A noun phrase such as $\dagger$\forme{χsɤr a-rte} is incorrect.

\begin{exe}
\ex \label{ex:aXsAr.tArte}
\gll a-rte, a-χsɤr tɤ-rte ra kɯnɤ nɤʑɯɣ ɲɯ-kʰam-a jɤɣ \\
\textsc{1sg}.\textsc{poss}-hat \textsc{1sg}.\textsc{poss}-gold \textsc{indef}.\textsc{poss}-hat \textsc{pl} also \textsc{2sg:gen} \textsc{ipfv}-give[III]-\textsc{1sg} be.possible:\textsc{fact} \\
\glt `I will even give you my hat, my golden hat.' (140429 qingwa wangzi-zh, 54)
\end{exe}

When the head noun is an APN, it is not usual either to strand the modifier and the following noun by putting a possessive prefix on the latter. The possessor is normally indicated by a possessive prefix on the leftmost word. For instance in (\ref{ex:nAXsAr.khWtsa}) the \textsc{2sg} prefix \forme{nɤ-} occurs on the modifier \japhug{χsɤr}{gold}.

\begin{exe}
\ex \label{ex:nAXsAr.khWtsa}
\gll nɤ-χsɤr kʰɯtsa nɯra ku-kɯ-sɯ-ntɕʰoz-a \\
\textsc{2sg}.\textsc{poss}-gold bowl \textsc{dem}:\textsc{pl} \textsc{ipfv}-2\fl{}1-\textsc{caus}-use-\textsc{1sg} \\
\glt `Let me use your golden bowl.' (140429 qingwa wangzi-zh, 135)
\end{exe}

However, we do find cases with a stranded NP modifier when the possessor on the head noun is first or second person, as in (\ref{ex:XsAr.akWmtChW}), where the prenominal modifier \japhug{χsɤr}{gold} does appear before the possessive prefix \forme{a-}.\footnote{The alternative form \forme{a-χsɤr kɯmtɕʰɯ} is possible to express the same meaning, and does occur in the same text. } This construction, though rarer, is not considered to be incorrect by native speakers. Other examples are found with UPN modifiers (§ \ref{sec:place.names}).

\begin{exe}
\ex \label{ex:XsAr.akWmtChW}
\gll  nɯnɯ a-kɯmtɕʰɯ, χsɤr a-kɯmtɕʰɯ nɯnɯ kɤ-ɣɯt a-pɯ-tɯ-cʰa qʰe,  \\
\textsc{dem} \textsc{1sg}.\textsc{poss}-toy gold \textsc{1sg}.\textsc{poss}-toy \textsc{dem} \textsc{inf}-bring \textsc{irr}-\textsc{ipfv}-2-can \textsc{lnk} \\
\glt `If you can bring my toy, my golden toy back...' (140429 qingwa wangzi-zh, 52)
\end{exe}

Note that unlike the IPN and APN modifiers discussed above, \textit{pronominal} prenominal modifiers (§ XXX) do not take possessive prefixes that have scope on the head noun. For instance, with the modifier \japhug{kɯmaʁ}{other}, the possessive prefix must appear on the following noun, as second person \forme{nɤ-} on \japhug{slama}{student} in (\ref{ex:kWmaR.nAslama}).

\begin{exe}
\ex \label{ex:kWmaR.nAslama}
 \gll  kɯmaʁ nɤ-slama ci tu-tɯ-ndɤm ju-tɯ-ɣɯt ɯ́-ŋu \\
 other \textsc{2sg}.\textsc{poss}-student \textsc{indef} \textsc{ipfv}-2-take[III] \textsc{ipfv}-2-bring \textsc{qu}-be:\textsc{fact} \\
 \glt `So you are bringing other students of yours?' (conversation, 150418)
\end{exe}

 

\section{Unpossessible nouns} \label{sec:unpossessible.nouns}
In addition to IPNs and APNs seen in the previous sections, Japhug also has a category of unpossessible nouns (UN), which includes names of places and ethnic groups (as in Koyukon Athabaskan, \citealt[651]{thompson96koyukon}), colour terms of Tibetan origin and some derived nouns like the social relation collectives (\ref{sec:social.collective}). With the exception of colours terms (§ \ref{sec:tibetan.colours}), these nouns can be used as modifiers of other nouns and are one of the three classes of `property words' (corresponding to the adjectives of Standard Average European),  alongside adjectival stative verbs (see § XXX) and property nouns (\ref{sec:property.nouns}).

 
%schluecker17proper

\subsection{Place names}  \label{sec:place.names}
Place names (\japhug{mbarkʰom}{Mbarkham}, \japhug{kɤmɲɯ}{Kamnyu} etc) and names of ethnic groups (such as \japhug{kɯrɯ}{Tibetan, Gyalrong} or \japhug{kupa}{Chinese}), like personal names, cannot take possessive prefixes when used independently. They can only be used with independent pronouns as in (\ref{ex:jiphe.kAmYW}) or (\ref{ex:iZo.KamnYW}) below.

\begin{exe}
\ex \label{ex:jiphe.kAmYW}
\gll   iʑora ji-pʰe kɤmɲɯ nɯtɕu <xiaoxue> <yinianji> <ernianji> pɯ-ndɯn-a. \\
\textsc{1pl} \textsc{1pl}.\textsc{poss}-\textsc{dat} pl.n. \textsc{dem}:\textsc{loc} primary.school first.grade second.grade \textsc{pfv}-read-\textsc{1sg} \\
\glt `I studied the first and second grade of primary school at our place in Kamnyu.' (140501 tshering skyid, 12)
\end{exe}

These types of nouns can serve as strictly prenominal modifiers (as in \japhug{kɯrɯ sɤtɕʰa}{Tibetan areas}) and commonly occur as first member of nominal compounds (as in \japhug{kɯrɯɕɤmɯɣdɯ}{traditional gun}, with \japhug{ɕɤmɯɣdɯ}{gun} as second element, see \ref{sec:karmadharaya.n.n}). Although these nouns are  unpossessible by themselves, when used as first member of a compound, or even as prenominal modifier (on which see § \ref{sec:possessive.prefixes.prenominal}), they can take a possessive prefix which has scope on the head noun, as in examples (\ref{ex:jikWrWlAsAr}) and (\ref{ex:jikAmYWskAt}), where the \textsc{1pl} possessive prefix \forme{ji-} occurs prefixed on the name \japhug{kɯrɯ}{Tibetan} and on the place name  \japhug{kɤmɲɯ}{Kamnyu}.

\begin{exe}
\ex \label{ex:jikWrWlAsAr}
 \gll nɯtɕu tɕe iʑora ji-kɯrɯ-lɤsɤr ŋu \\
 \textsc{dem}:\textsc{loc} \textsc{lnk} \textsc{1pl} \textsc{1pl}.\textsc{poss}-Tibetan-new.year be:\textsc{fact} \\
 \glt `At that time, it is our Tibetan new year.' (conversation, 150102)
\end{exe}

\begin{exe}
\ex \label{ex:jikAmYWskAt}
 \gll nɤʑo ji-kɤmɲɯ-skɤt nɯnɯ <quanshijie> ʑo ju-tɯ-sɯ-ɤzɣɯt ŋu \\
 \textsc{2sg} \textsc{1pl}.\textsc{poss}-pl.n.-language \textsc{dem} whole.world \textsc{emph} \textsc{ipfv}-2-\textsc{caus}-reach be:\textsc{fact} \\
  \glt `You are spreading our Kamnyu language to the whole world.' (conversation, 150618)
\end{exe}

The pair of examples in (\ref{ex:nArmi}) illustrates the different behaviour of UPNs as first elements of compounds on the one hand, and as noun modifier on the other hand. In (\ref{ex:nAkWrWrmi}), \japhug{kɯrɯ-rmi}{Tibetan name}  constitutes a single compound APN (from \japhug{kɯrɯ}{Tibetan} and \japhug{tɤ-rmi}{name}; the phrase \ipa{kɯrɯ ɯ-rmi} is also possible).  As in the examples above, the possessive prefix occurs before \forme{kɯrɯ}. Stranding the modifier in a form such as $\dagger$\forme{kɯrɯ a-rmi} is considered to be agrammatical by native speakers.
 
In (\ref{ex:faguo.nArmi}) however, the modifier \forme{faguo} `France, French' (from Chinese) cannot be compounded with  \japhug{tɤ-rmi}{name} and cannot take possessive prefixes. This is a rare example where a noun modifier can be stranded from the stem of the head noun by a definite possessive prefix (§ \ref{sec:possessive.prefixes.prenominal}).

\begin{exe}
\ex \label{ex:nArmi}
\begin{xlist}
\ex  \label{ex:nAkWrWrmi}
\gll nɤ-kɯrɯ-rmi   \\
\textsc{2sg}:\textsc{poss}-Tibetan-name \\
\glt `Your Tibetan name.' 
\ex  \label{ex:faguo.nArmi}
\gll nɤʑɯɣ <faguo> nɤ-rmi   \\
\textsc{2sg}:\textsc{gen} France \textsc{2sg}.\textsc{poss}-name \\
\glt `Your French name.' 
\end{xlist}
\end{exe}
 
Place names followed by the plural \forme{ra} designate the people living in the place (\ref{ex:iZo.KamnYW}), even without \forme{-pɯ} suffixation (§ \ref{ex:inhabitant.pW}). Example (\ref{ex:iZo.KamnYW}) also shows that in this usage, it is possible to use a personal pronoun in apposition as in \forme{iʑo kɤmɲɯ ra}  `we Kamnyu people'.

\begin{exe}
\ex \label{ex:iZo.KamnYW}
 \gll iʑo kɤmɲɯ ra kɯ tɕʰɯχpri tu-ti-j ŋu. rcaqo ra cho mɤŋi ra kɯ tɕʰɯχpɯχpri tu-ti-nɯ ŋu \\
 \textsc{1pl} pl.n. \textsc{pl} \textsc{erg} salamander \textsc{ipfv}-say-\textsc{1pl} be:\textsc{fact} pl.n. \textsc{pl} \textsc{comit} pl.n. \textsc{pl} \textsc{erg}  salamander \textsc{ipfv}-say-\textsc{pl} be:\textsc{fact} \\
 \glt `We Kamnyu people call it \forme{tɕʰɯχpri}, and people from Rqakyo and Mangi call it \forme{tɕʰɯχpɯχpri}. ' (25-tChWXpri, 20)
\end{exe}

Place names can take some prenominal modifiers such as \japhug{pʰa}{whole} as in (\ref{ex:pha.RdWrJAt}), but no example of bare place names with prenominal demonstrative have been found. %XXXXX à revérifier

\begin{exe}
\ex \label{ex:pha.RdWrJAt}
 \gll pʰa ʁdɯrɟɤt nɯ ɯ-ŋgɯ tɕe rqaco cʰo katɕa nɯ stu ɣɤndʐo \\
 whole pl.n. \textsc{dem} \textsc{3sg}.\textsc{poss}-inside \textsc{lnk} pl.n. \textsc{comit} pl.n. \textsc{dem} most cold:\textsc{fact} \\
 \glt  `In the whole of Gdongbrgyad, Rqakyo and Kacha are the coldest.' (140522 RdWrJAt, 104)
\end{exe}

Place names and ethnic names can be used as core arguments, or nominal predicates with a copula, as in (\ref{ex:kupa.Nu}) and (\ref{ex:taRdo.Nu}).

\begin{exe}
\ex \label{ex:kupa.Nu}
\gll  a-wa nɯnɯ kupa ŋu \\
\textsc{1sg}.\textsc{poss}-father \textsc{dem} Chinese be:\textsc{fact}  \\
\glt `My father is Chinese.' 140501 tshering skyid, 4)
\end{exe}

\begin{exe}
\ex \label{ex:taRdo.Nu}
\gll  tɕe ʁnɯ-tɯpɯ nɯnɯ taʁrdo ŋu  \\
\textsc{lnk} two-household \textsc{dem} pl.n. be:\textsc{fact} \\
\glt `These two household are Taqrdo.' 
\end{exe}

Like other locative nouns (§ XXX), bare place names can be used without postposition to express motion (\ref{ex:mbarkhOm.thWwGGWta}) or static location (\ref{ex:kAmYW.GJW}), but are also found with locative postpositions, most often \forme{ri} as in (\ref{ex:tshuBdWn.ri}) but also \forme{tɕu} or \forme{zɯ} (as in \ref{ex:jiphe.kAmYW} above).

\begin{exe}
\ex \label{ex:mbarkhOm.thWwGGWta}
 \gll a-pi kɯ tɤ́-wɣ-ndo-a tɕe tɕe, mbarkʰom tʰɯ́-wɣ-ɣɯt-a, \\
 \textsc{1sg}.\textsc{poss}-elder.sibling  \textsc{erg} \textsc{pfv}-\textsc{inv}-take-\textsc{1sg} \textsc{lnk} \textsc{lnk} pl.n.   \textsc{pfv:downstream}-\textsc{inv}-bring-\textsc{1sg}  \\
\glt `My elder brother took me and brought me to Mbarkham.' (140501 tshering skyid, 28)
\end{exe}

\begin{exe}
\ex \label{ex:kAmYW.GJW}
 \gll  kɯɕɯŋgɯ tɕe kɤmɲɯ ɣɟɯ kɯɕnɯz pjɤ-tu. \\
 former.days \textsc{lnk} pl.n. watchtower  seven \textsc{ifr}.\textsc{ipfv}-exist \\
 \glt `In former times, there were seven watchtowers in Kamnyu.' (140522 GJW, 1)
\end{exe}

\begin{exe}
\ex \label{ex:tshuBdWn.ri}
 \gll  tɕe alo tsʰuβdɯn ri pɯ-rɤʑi-j tɕe \\
 \textsc{lnk} upstream pl.n. \textsc{loc} \textsc{pst}.\textsc{ipfv}-stay-\textsc{1sg} \textsc{lnk} \\
 \glt `We were living up there in Tshobdun.' (28-kWpAz, 178)
\end{exe}

\subsection{Colour nouns} \label{sec:tibetan.colours}
Colour names of Tibetan origin, such as \japhug{ldʑaŋkɯ}{blue/green} from \tibet{ལྗང་གུ་}{ldʑaŋ.gu}{green} or \japhug{ʁmɤrsmɯɣ}{dark red} from \tibet{དམར་སྨུག་}{dmar.smug}{dark red} designate objects or animals with a particular colour, occurring in the same context as free S-participles of adjectival stative verbs of colour (\ref{ex:ldZaNkW}). Although such nouns could be expected to occur as noun modifiers, good examples are not found in the corpus. The adjectival stative verbs in \forme{arɯ-} derived from them (for instance \japhug{arɯldʑaŋkɯ}{be green}, see § XXX) are as common as the colour nouns.
 
\begin{exe}
\ex \label{ex:ldZaNkW}
 \gll qambalɯla rcanɯ ɯ-mdoʁ ʑakastaka ʑo kɯ-ŋu tu. .... kɯ-qarŋe tu, ldʑaŋkɯ tu, kɯ-ɤrŋi tu, kɯ-ɲaʁ tu. \\
 butterfly \textsc{foc}:\textsc{unexp} \textsc{3sg}.\textsc{poss}-colour each \textsc{emph} \textsc{nmlz}:S/A-be exist:\textsc{fact} .... \textsc{nmlz}:S/A-be.yellow exist:\textsc{fact} blue/green exist:\textsc{fact}  \textsc{nmlz}:S/A-be.green exist:\textsc{fact} \textsc{nmlz}:S/A-be.black exist:\textsc{fact}\\
 \glt `There are butterflies with all kinds of colours, yellow, green, blue/green, black. (26-qambalWla, 6)
\end{exe}

\subsection{Other UPN}   \label{sec:other.upn}
UPN other than proper names and colour terms include nouns occurring as postnominal modifiers like \japhug{tɯlɤt}{second sibling}\footnote{In \japhug{tɯlɤt}{second sibling} the \forme{tɯ-} element in this word is originally an indefinite possessive prefix, but has become lexicalized.} as in (\ref{ex:tWlAt}) and  privative nouns in \forme{-lu} described in (\ref{sec:privative}). 

\begin{exe}
\ex \label{ex:tWlAt}
\gll  nɯ-me tɯlɤt nɯ ɲɤ-mbi-nɯ \\
\textsc{3pl.poss}-daughter second.sibling \textsc{dem} \textsc{ifr}-give-\textsc{pl} \\
\glt `They gave him their second daughter.' (2002qaCpa, 40)
\end{exe} 

Numerals under 99 are also unable to take possessive prefixes and serve as postnominal modifiers (§ \ref{sec:one.to.ten}), and can be considered to be a subclass of UPN.

\section{Personal names}  \label{sec:personal.names}
This section focuses on three topics: the absence of vocative forms, the Tibetan origin of personal names, and their use with pronouns and possessive prefixes. The question of the status of personal names in the empathy hierarchy is treated in § XXX in the section on inverse marking.

\subsection{Vocative} \label{sec:vocative}
Gyalrong languages other than Japhug have specific vocative forms for personal names and kinship terms. In Tshobdun, \citet[133]{jackson98morphology} and \citet[53]{jackson05yingao} reports that personal names in the vocative have stress retraction. The same is found in Khroskyabs (\citealt[153]{lai17khroskyabs}). In Situ, IPN nouns have their possessive prefixes replaced by \forme{a-} in vocative forms (\citealt[471]{nagano03cogtse}, \citealt[177]{prins16kyomkyo}). 

In Japhug, due to the almost entire loss of contrastive stress (§ XXX) and the fact that the \textsc{1sg} possessive prefix has the form \forme{a-} (§ \ref{sec:possessive.paradigm}) unlike in Tshobdun and Situ (where it is \forme{ŋa-/ŋə-}), there is no distinct vocative form for either personal names or kinship terms. 

A prefix \forme{a-} does occur in the familiar form of personal names (reminding of Lin's \citeyear[162]{linxr93jiarongen} description of this prefix as a \zh{爱称} `pet name' marker), but not exclusively in vocative use as in example (\ref{ex:kWlAGacAB.nW.kW}) where we see the name \forme{acɤβ} as transitive subject, familiar form of a Tibetan name with \forme{scɤβ} as second element (see section \ref{sec:names.tibet} below).

 \begin{exe}
\ex \label{ex:kWlAGacAB.nW.kW}
\gll kɯ-lɤɣ acɤβ nɯ kɯ, ɯ-pʰɯŋgɯ nɯtɕu qapɯtɯm ci na-rku ɲɯ-ŋu, \\
\textsc{nmlz}:S/A-graze Askyabs \textsc{dem} \textsc{erg} \textsc{3sg}.\textsc{poss}-fold.of.clothes \textsc{dem:loc} pebble.from.flint \textsc{indef} \textsc{pfv}:3\fl3'-put.in \textsc{sens}-be \\
\glt `The shepherd Askyabs put a pebble in the folds of his clothes (to avoid forgetting what he had to told the king).'  (Kunbzang 332)
\end{exe}

Since similar \forme{a-} prefixes exist in Tibetan and Chinese, and since personal names are exclusively from either of these languages (there are no clear remnants of native personal names in Japhug), it is likely that the familiar form of the names was also borrowed.

\subsection{Tibetan names} \label{sec:names.tibet}
Speakers of Japhug generally have Tibetan names (\forme{kɯrɯ ɯ-rmi} or \forme{kɯrɯ-rmi}), and in addition a Chinese official name which may or may not be related to the Tibetan one (see \ref{ex:nArmi} § \ref{sec:place.names} on the use of ethnic or countries names as prenominal modifiers with the IPN \japhug{tɤ-rmi}{name}). Buddhist or Bonpo monks are also given religious names (in Japhug \forme{χpɯn ɯ-rmi}, see \ref{ex:XpWn.Wrmi}). 
 
\begin{exe}
\ex \label{ex:XpWn.Wrmi}
\gll tɕe χpɯn ɯ-rmi nɯ, aʑo a-rmi nɯ stɤnbiɲima tɤ́-wɣ-sɤrmi-a-nɯ. \\
\textsc{lnk} monk \textsc{3sg}.\textsc{poss}-name \textsc{dem} \textsc{1sg} \textsc{1sg}.\textsc{poss}-name \textsc{dem} p.n. \textsc{pfv}-\textsc{inv}-give.name-\textsc{1sg}-\textsc{pl} \\
\glt `They gave me the name Bstanpa'i nyima as my monk name.' (160721 XpWN, 38)
\end{exe}

In one traditional story, we find an example of person names based on Japhug words as in (\ref{ex:zrAntCW}), but it looks so strange that the narrator felt it necessary to specify that these are people's names.

\begin{exe}
\ex  \label{ex:zrAntCW}
 \gll zrɤntɕɯ tɯrme ci pjɤ-tu, tɯpɕi kɯ-rmi ci pjɤ-tu, tɯrme nɯ-rmi ɲɯ-ŋu nɤ \\
mung.bean person \textsc{indef} \textsc{ifr}.\textsc{ipfv}-exist flax \textsc{nmlz}:S/A-call  \textsc{indef} \textsc{ifr}.\textsc{ipfv}-exist people  \textsc{3pl}.\textsc{poss}-name \textsc{sens}-be \textsc{sfp} \\
\glt `There was (a lady) was was called `Mung bean', and (another one) called `Flax', these are names of people.' (zrAntCW, 1)
\end{exe}

Names used by Japhug speakers are not markedly different from those found in other Tibetan areas. Lady names often include the suffixes \forme{ltɕɤm}, \forme{rcit} or \forme{mtsʰu}, (from \tibet{ལྕམ་}{ltɕam}{lady, sister}, \tibet{སྐྱིད་}{skʲid}{happy} and \tibet{མཚོ་}{mtsʰo}{lake}), and there are also non-gender specific suffixes like \forme{scɤβ} (from \tibet{སྐྱབས་}{skʲabs}{protector}, for instance \forme{tsʰɯraŋ scɤβ} from \tibet{ཚེ་རིང་སྐྱབས་}{tsʰe.riŋ.skʲabs}{p.n.}).). 

Many Tibetan names have alternative readings reflecting distinct reading traditions belonging to more than two distinct layers (see § XXX on the layers of Tibetan borrowings in Japhug). For instance, some people with the Tibetan name \tibet{འཕྲིན་ལས་}{ⁿpʰrin.las}{Karma} are called \forme{mpʰrɯlɤz} (with preservation of the coda), other \forme{mpʰrɯli} (with Amdo-type change to \forme{-i}). The names however tend to have non-Amdo phonological features even for people of younger generation. For instance, the name \tibet{ཀུན་དགའ་}{kun.dga}{Ânanda} is pronounced \forme{kɯnga} without assimilation of the dental nasal to a velar nasal, and  \tibet{ཀུན་བཟང་}{kun.bzaŋ}{Sarvabhadra} is \forme{kɯnɯβzaŋ} with an anaptyctic vowel (see § XXX for examples of this type in the borrowed vocabulary).  

%\tibet{བཀྲ་ཤིས་}{bkra.ɕis}{good fortune} appears as \forme{krɤɕiz} with preservation of the coda and part of the initial cluster, \forme{krɤɕi} with Amdo-like loss of final \forme{-s} and \forme{tʂaɕi} with cluster simplification. 

 

\subsection{APN or UPN?} \label{sec:personal.name.APN}
Personal names superficially look like UPN, as they do not usually occur with possessive prefixes, even when taking placenames as modifiers, as in \forme{taʁrdo χpɤltɕin} `Dpalcan from Taqrdo' (see \ref{ex:taRrdo.dpalcan} in § \ref{sec:personal.names.modifiers})

Personal names commonly occur preceded by kinship terms which, being IPNs (§ \ref{sec:kinship}), have a possessive prefix as in (\ref{ex:anmaR.dpalcan}). It is considered impolite to address someone from an older generation than oneself without adding a kinship term -- for instance, the author of this book, being much younger, has to call him \forme{a-βɣo χpɤltɕin} with the \textsc{1sg} form of \japhug{tɤ-βɣo}{father's brother}.

\begin{exe}
\ex \label{ex:anmaR.dpalcan}
\gll a-nmaʁ χpɤltɕin \\
\textsc{1sg}.\textsc{poss}-husband p.n. \\
\glt `My husband Dpalcan.' (heard in context)
\end{exe}

Although personal names rarely occur with possessive prefixes, there is no grammatical constraint against it. There is one such example in the whole corpus, in a conversation where a clarification was needed. Tshendzin asks about Dpalcan, younger brother of Tshering Sgrolma, but she does not understand at once, because Tshendzin's husband is also called Dpalcan; thus Tshendzin says (\ref{ex:nWχpAltCin}) with the possessed form \japhug{nɯ-χpɤltɕin}{your Dpalcan} to disambiguate between the two. 

\begin{exe}
\ex   
\begin{xlist}
\ex 
\gll χpɤltɕin kɯmaʁ kɯ-nɯhɯɲi mɯ-jo-ɕe ɯ́-ŋu. \\
p.n. other \textsc{nmlz}:S/A-do.work \textsc{neg}-\textsc{ifr}-go \textsc{qu}-be:\textsc{fact} \\
\\
\glt (Tshendzin): `Dpalcan did not go for another job, did he?'
\ex 
\gll ka? \\
\textsc{sfp} \\
\glt  (Tshering Sgrolma): `What?'
\ex  \label{ex:nWχpAltCin}
\gll χpɤltɕin, nɯʑo nɯ-χpɤltɕin nɯ \\
p.n. \textsc{2pl} \textsc{2pl}.\textsc{poss}-p.n. \textsc{dem} \\
\\
\glt (Tshendzin): `Dpalcan, your Dpalcan.'  
\end{xlist}
\glt  (140510 tshering)
\end{exe}

Given the existence of such forms, personal names are treated as a subclass of APN rather than as UPNs. Note that only plural forms (\forme{ji-χpɤltɕin} `our Dpalcan', \forme{ʑara nɯ-χpɤltɕin} `Their Dpalcan' etc) are possible; singular forms such as $\dagger$\forme{a-χpɤltɕin} are not grammatical.
 
\subsection{Personal names and modifiers} \label{sec:personal.names.modifiers}
Personal names are more often than not used without demonstratives and determiners (see § \ref{sec:indefinite.markers}), but examples are not difficult to find either (see \ref{ex:Yimawodzer.NW}).

\begin{exe}
\ex \label{ex:Yimawodzer.NW}
 \gll  ɲimawozɤr nɯ kɯ, srɯnmɯ nɯ pjɤ-ftɯl, \\
 p.n. \textsc{dem} \textsc{erg} râkshasî \textsc{dem} \textsc{ifr}-subdue \\
 \glt `Nyima 'Odzer subdued the râkshasî.' (2011-4-smanmi, 258)
\end{exe}

It is possible for personal names to take dual or plural markers, even without an associative plural meaning, to designate a group of people with the same name, as in (\ref{ex:Dpalcan.XsWm}).

\begin{exe}
\ex \label{ex:Dpalcan.XsWm}
 \gll a-pɯ-ŋu tɕe, χpɤltɕin ʁnɯz, nɯ maʁ nɤ χsɯm kɯ-fse kɯ-naχtɕɯɣ tɯtɯrca a-pɯ-rɤʑi-nɯ tɕe, `χpɤltɕin ni, χpɤltɕin ra" nɯra tu-kɯ-ti kʰɯ. \\
 \textsc{irr}-\textsc{ipfv}-be \textsc{lnk} p.n. two \textsc{dem} not.be:\textsc{fact} \textsc{lnk} three \textsc{nmzl}:S/A-be.like \textsc{nmzl}:S/A-be.identical together \textsc{irr}-\textsc{ipfv}-stay-\textsc{pl} \textsc{lnk} p.n. \textsc{du} p.n. \textsc{pl} \textsc{dem}:\textsc{pl} \textsc{ipfv}-\textsc{genr}-say be.possible:\textsc{fact} \\ 
 \glt `For instance, if two or three (people called) Dpalcan live together, one can say `the two Dpalcans', `the Dpalcans'. (elicited)
\end{exe}

To distinguish between persons with the same name (a common occurrence among speakers of Japhug, given the relatively limited inventory of Tibetan names available), house names (\forme{kʰa ɯ-rmi}) are generally added as prenominal modifiers, as in (\ref{ex:taRrdo.dpalcan}).

\begin{exe}
\ex \label{ex:taRrdo.dpalcan}
\gll χpɤltɕin ɯ-kʰa nɯ taʁrdo rmi tɕe taʁrdo χpɤltɕin tu-kɯ-ti. \\
p.n. \textsc{3sg}.\textsc{poss}-house \textsc{dem} pl.n. be.called:\textsc{fact} \textsc{lnk} pl.n. p.n. \textsc{ipfv}-\textsc{genr}-say \\
\glt `Dpalcan's house is called Taqrdo, so one (can) call him `Taqrdo Dpalcan'.' (elicited)
\end{exe}

If  two persons from the same household have the same name, locational modifiers (§ XXX) can be used instead, as illustrated in (\ref{ex:maNlo.dpalcan}).

\begin{exe}
\ex \label{ex:maNlo.dpalcan}
\gll  nɯ maʁ nɤ, ndʑi-kʰa ɯ-rmi kɯnɤ a-pɯ-naχtɕɯɣ tɕe, kʰa kundi, lotʰi kɯ-fse nɯra tɕe,
maŋlo χpɤltɕin, maŋtʰi χpɤltɕin, maŋkɯ χpɤltɕin, maŋndi χpɤltɕin, nɯra tu-kɯ-ti ŋgrɤl. \\
\textsc{dem} not.be:\textsc{fact} \textsc{lnk} \textsc{3du}.\textsc{poss}-house \textsc{3sg}.\textsc{poss}-name also \textsc{irr}-\textsc{ipfv}-be.identical \textsc{lnk} house east.west up.down.stream \textsc{nmlz}:S/A-be.like \textsc{dem}:\textsc{pl} \textsc{lnk} upstream p.n. downstream p.n. east p.n. west p.n. \textsc{dem}:\textsc{pl} \textsc{ipfv}-\textsc{genr}-say be.usually.the.case:\textsc{fact} \\
\glt `Otherwise, if their house name is also the same, using the east-west or the upstream-downstream dimensions, one can say `Dpalcan from upstream, downstream, east or west.' (elicited)
\end{exe}

Like other nouns, personal names can also occur as head of non-restrictive relatives, as in  (\ref{ex:mWtAkWrZaR}) and (\ref{ex:WnmaR.pWkWNu}), though such uses are rather uncommon. No examples of personal names as heads of head-internal relatives have been found.

  \begin{exe}
\ex \label{ex:mWtAkWrZaR}
\gll  tɕendɤre 	iɕqʰa 	ʁlaŋsaŋtɕʰin 	χsɯm 	ma 	mɯ-tɤ-kɯ-rʑaʁ 	nɯ, \\
\textsc{lnk} the.aforementioned Gesar three apart.from \textsc{neg}-\textsc{pfv}-\textsc{nmlz}:S/A-pass.days \textsc{dem} \\
\glt `Gesar, who was only three days old,'  (Gesar 81)
\end{exe}

\begin{exe}
\ex \label{ex:WnmaR.pWkWNu}
\gll nɯ ɕɯŋgɯ ɯ-nmaʁ pɯ-kɯ-ŋu tsʰɯraŋ nɯ pjɤ-mto  \\
\textsc{dem} before \textsc{3sg}.\textsc{poss}-husband \textsc{pst}-\textsc{nmlz}:S/A-be p.n. \textsc{dem} \textsc{ifr}-see \\
\glt `See saw Tshering, who had been her husband before.' (2002qajdoskAt, 
\end{exe}

\section{Status constructus} \label{sec:status.constructus}
The term \textit{status constructus} is used in Gyalrongic linguistics (\citealt{jacques12incorp}, \citealt[163-4]{lai17khroskyabs}) to refer to the non-autonomous form of (mainly nominal, but also verbal and adverbial) roots occurring as non-final element of compounds. The use of this term, adopted from Semitic linguistics,  differs from works such as \citet{creissels06hongrois} or \citet{creissels17construct} in which `construct form' refers to a specific form used that is obligatory on the head noun in specific noun-modifier constructions (including with a possessive marker). Given the fact in that Japhug and other Gyalrongic languages, nominal compounds generally follow modifier-head order (the opposite of Semitic), the form undergoing \textit{status constructus} alternation in Japhug is often the modifier noun,\footnote{In addition, in Japhug the possessed form of nouns do not present morphological alternations (\ref{sec:possessive.paradigm}) with only one exception (\ref{sec:apn.to.ipn}).} except in Noun-Verb compounds where the second element is an adjectival stative verb.

In this section, the various types of alternations attested for first or other non-final members of compounds are described, in particular vowel alternation (the most common type). Additionally, exceptional changes to the final members of compounds are discussed in \ref{sec:final.compounds}.

\subsection{Vowel alternations in non-final members of compounds} \label{sec:vowel.alternations.compounds}
Regular \textit{status constructus} is Japhug applies to open syllables, and involves shift of all vowels to either \ipa{ɯ} or \ipa{ɤ} back unrounded vowels, following the correspondences in Table \ref{tab:sc.regular}.

\begin{table}
\caption{Regular \textit{status constructus} in Japhug} \label{tab:sc.regular}
\begin{tabular}{lllll}
\lsptoprule
Base & SC & Example \\
\ipa{-a} &\ipa{-ɤ} & \japhug{βɣɤsni}{mill axle} from  \japhug{βɣa}{mill} + \japhug{tɯ-sni}{heart} \\
\ipa{-e} &\ipa{-ɤ} & \japhug{tɕʰemɤpɯ}{little girl} from  \japhug{tɕʰeme}{girl} + \japhug{ɯ-pɯ}{little one} \\
\ipa{-i} &\ipa{-ɯ} & \japhug{smɯɣot}{light of the fire} from  \japhug{smi}{fire}+ \japhug{ɣot}{light}  \\
\ipa{-o} &\ipa{-ɤ} &  \japhug{mbrɤsno}{horse saddle} from  \japhug{mbro}{horse} + \japhug{tɤ-sno}{saddle}\\
\ipa{-u} &\ipa{-ɤ} & \japhug{tɤ-kɤrme}{head hair} from  \japhug{tɯ-ku}{head} + \japhug{tɤ-rme}{hair} \\
\lspbottomrule
\end{tabular}
\end{table}

Alternatively, the vowel \ipa{i} alternates with \ipa{ɤ} in \textit{status constructus}, as in \japhug{qaprɤftsa}{centipede} from \japhug{qapri}{snake} and \japhug{tɤ-ftsa}{nephew} or 
\japhug{tɯ-mɤmɲaʁ}{astragalus} from \japhug{tɯ-mi}{leg, foot} and \japhug{tɯ-mɲaʁ}{eye}. %\japhug{ɯ-χtɯrca}{with the others} &&from  \japhug{tɯ-χti}{companion} + \japhug{tɤ-rca}{together with}

In very few case, \ipa{u} can also alternate with \ipa{ɯ}, as in \forme{ŋɤtɕɯ-} which occurs in the expression \japhug{ŋɤtɕɯkɤti,kʰɯ}{obey to everything} (\ref{sec:interrogative.indef}), the status constructus of \japhug{ŋotɕu}{where}.

Nouns ending in \ipa{-ɯ} never have a \textit{status constructus} form distinct from the base form, as for instance \japhug{tɯmɯpaʁ}{slug} from \japhug{tɯ-mɯ}{sky} and \japhug{paʁ}{pig}.

Vowel alternation in closed syllables is very rare, and affects only a few stems with \ipa{o} as the main vowel (Table \ref{tab:sc.irregular}). The \textit{status constructus}  \forme{ɕɤm-} of \japhug{ɕom}{iron} occurs in a few other nouns, but the form \forme{staʁ-} (with internal sandhi to \forme{staχ-}, cf \ref{sec:internal.sandhi.compounds}) from \japhug{stoʁ}{broad bean} is unique.

\begin{table}
\caption{Irregular \textit{status constructus} in closed syllable stems} \label{tab:sc.irregular}
\begin{tabular}{lllll}
\lsptoprule
Base & SC & Example \\
\ipa{-oʁ} &\ipa{-aʁ} & \japhug{staχpɯ}{pea} from  \japhug{stoʁ}{broad bean} + \japhug{ɯ-pɯ}{little one} \\
\ipa{-om} &\ipa{-ɤm} & \japhug{ɕɤmtsʰoʁ}{iron nail} from  \japhug{ɕom}{iron} + \japhug{tɤtsʰoʁ}{nail} \\
\lspbottomrule
\end{tabular}
\end{table}


\subsection{Other alternations} \label{sec.compounds.first.other.alternations}
Apart from the regular vowel changes described above, four types of alternations are observed in non-final member of compounds: internal sandhi, loss of coda, reduced forms and loss of possessive prefix.

\subsubsection{Internal sandhi in compounds} \label{sec:internal.sandhi.compounds}
First, the first element of a cluster undergoes internal sandhi (section XXX), with voicing and nasal assimilation as in Table \ref{tab:sandhi.compounds}. 

\begin{table}
\caption{Internal sandhi in compounds} \label{tab:sandhi.compounds} 
\begin{tabular}{lllll}
\lsptoprule
Type & Example \\
Nasal assimilation & \ipa{t} \fl{} \ipa{n} /\_[+nasal] & \japhug{tsʰɤnmu}{ewe} \\
&&from  \japhug{tsʰɤt}{goat} + \japhug{mu}{female} \\
Voicing assimilation & \ipa{ɣ} \fl{} \ipa{x} /\_[-voiced] & \japhug{zrɯxpɯ}{little louse} \\
&&from  \japhug{zrɯɣ}{louse} + \japhug{ɯ-pɯ}{little one} \\
  & \ipa{ʁ} \fl{} \ipa{χ} /\_[-voiced] & \japhug{tɯ-jaχpa}{palm} \\
&&from  \japhug{tɯ-jaʁ}{arm, hand} + \japhug{pa}{down} \\
  & \ipa{z} \fl{} \ipa{s} /\_[-voiced] & \japhug{mbrɤstshi}{rice soup} \\
&&from  \japhug{mbrɤz}{rice} + \japhug{tɯtsʰi}{rice soup} \\
\lspbottomrule
\end{tabular} 
\end{table}

There are cases of irregular internal sandhi only attested in lexicalized compounds. For instance \japhug{jaŋntsɤrpa}{one-handed axe} from \japhug{tɯ-jaʁ}{arm, hand}, \japhug{ɯ-ntsi}{one of a pair} and \japhug{tɯ-rpa}{axe}, showing a nasal assimilation rule \hbox{\ipa{ʁ} \fl{} \ipa{ŋ} /\_[+nasal]} which is not productive in the language (as shown by words such \japhug{tɯ-jaʁndzu}{finger}, also with \japhug{tɯ-jaʁ}{arm, hand} as first element).

\subsubsection{Loss of codas in compounds} \label{sec:loss.codas.compounds}
Loss of coda is not a regular process in first elements of compounds. The following list collects some of the most representative examples. Many examples are found in numerals (see § \ref{sec:approx.numerals} and § \ref{sec:numeral.prefixes}).

\begin{itemize}
\item Loss of \ipa{-β}: 

\japhug{ɴqiaβ}{dark side of the mountain}  + \japhug{zwɤr}{mugwort} \fl{}  \japhug{ɴqiazwɤr}{Artemisia sp.}  
\item Loss of  \ipa{-t}: 

\japhug{xtɯt}{be short} + \japhug{rɲɟi}{be long} \fl{} \japhug{xtɯrɲɟi}{length (n)}  

 \japhug{tsʰɤt}{goat} + \japhug{ta-ʁrɯ}{horn} \fl{} \japhug{tsʰɤʁrɯ}{goat horn}  
\item Loss of \ipa{-z}: 

\japhug{qartsʰaz}{deer}  + \japhug{tɯ-ndʐi}{skin} \fl{}  \japhug{qartsʰɤndʐi}{deer hide}  
\item Loss of \ipa{-r}:

 \japhug{zwɤr}{mugwort} + \japhug{wɣrum}{be white} \fl{} \japhug{zwɤɣrum}{Artemisia sp.}  

\japhug{ɕɤr}{night} + \japhug{ɯ-χcɤl}{middle} \fl{}  \japhug{ɕɤχcɤl}{middle of the night}  
\item Loss of \ipa{-ɣ}:

 \japhug{tɤjmɤɣ}{mushroom}  + \japhug{tɯ-sti}{alone}  \fl{}  \japhug{jmɤtɤsti}{species of mushroom}  
 
\japhug{tɯ-mtʰɤɣ}{waist}  + \japhug{rŋgɤβ}{attach} \fl{}  \japhug{tɯ-mtʰɤrɴɢɤβ}{part of the trouser where one can tuck things in} 
\item Loss of \ipa{-ʁ}: 

\japhug{ɕoʁ}{buckwheat} + \japhug{wɣrum}{be white} \fl{}  \japhug{ɕɤɣrum}{buckwheat sp}  

\japhug{paʁ}{pig} + \japhug{tɯ-qa}{root, paw, bottom} \fl{}  \japhug{pɤqa}{stuffed pig feet}  
\end{itemize}

With the exception of the loss of \forme{-t}, which is relatively common, the other cases are rare and cannot be predicted by any rule based on phonology (the presence of a cluster in the following element is irrelevant, for instance). Some of them occur with other alternations in the second syllable (cf \ref{sec:second.member.alternation}).

\subsubsection{Reduced forms} \label{sec:reduced.forms.compounds}   
A handful of nouns have reduced \textit{status constructus} forms when occurring as first member of compounds; the nouns \japhug{nɯŋa}{cow} and \japhug{kʰɯna}{dog} are treated below. 

The noun \japhug{nɯŋa}{cow} corresponds to the syllable \forme{ŋɤ-} in the compounds \japhug{ŋɤnɯ}{udder} (with \japhug{tɯ-nɯ}{teat} as second element), \japhug{ŋɤqe}{cow dung} (with \japhug{tɯ-qe}{shit, dung}) and \japhug{ŋɤlitɕaʁmbɯm}{dung beetle} (on whcih see \ref{sec:second.member.alternation}), which would be the regular \textit{status constructus} from a stem \forme{ŋa-}. The apparent `loss' of a \forme{nɯ-} element is due to the fact that the noun \japhug{nɯŋa}{cow} is itself an ancient compound comprising \japhug{tɯ-nɯ}{teat} as first element (`bovid with udders').

In the case of \japhug{kʰɯna}{dog}, we find the \textit{status constructus} \forme{kʰɯ-} in the compounds \japhug{kʰɯndʐi}{dog skin} (with \japhug{tɯ-ndʐi}{skin} as second element), \japhug{kʰɯdo}{old dog} (see \ref{sec:derogative}), \japhug{kʰɯtsʰoʁ}{hunting with dog} (probably a noun-verb compound with \japhug{tsʰoʁ}{attach}, see also the related incorporating verb in § XXX) and a few plant names such as \japhug{kʰɯlu}{Euphorbia helioscopia} (a \textit{bahuvrīhi} meaning `(the plant) having dog milk' -- referring to its toxic juice, see \ref{sec:bahuvrihi.n.n}) and \japhug{kʰɯrtsʰɤz}{unindentified plant} (`dog lung'; the second element is \japhug{tɯ-rtsʰɤz}{lung}). Unlike  \japhug{nɯŋa}{cow}, whose reduced \textit{status constructus} corresponds to the second syllable, the syllable \forme{kʰɯ-} corresponds to the first syllable of \japhug{kʰɯna}{dog}, which must also be an obscured compound. The etymology of  the element \forme{-na} is unclear.

\subsubsection{Loss of possessive prefix} \label{sec:loss.possessive.prefix.compounds}
Some IPNs or alienabilized former IPNs lose their possessive prefix (or frozen indefinite possessive \forme{tɯ-/tɤ-}, see \ref{sec:frozen.indef}), as for instance the noun  \japhug{jmɤrtaʁ}{weevil}, which comes from \japhug{tɤ-jme}{tail} and \japhug{artaʁ}{be forked} (`forked tail').\footnote{The verb \japhug{artaʁ}{be forked} itself is denominal from \japhug{tɤ-jwaʁ}{branch}.} Its first element \japhug{tɤ-jme}{tail} loses the prefix \forme{tɤ-} and undergoes regular vowel alternation.

Similar examples are particularly common with \japhug{tɯ-xtsa}{shoe}, as mainly parts of the shoes are referred to by APN compounds with \forme{xtsɤ-} as first element (\japhug{xtsɤɕna}{tip of the shoe}, \japhug{xtsɤrkɯ}{sides of the shoe} etc).

In some derivations that originate from compounds, such as the privative (\ref{sec:privative}) or the derogative  (\ref{sec:derogative}), the indefinite possessor prefix is also removed.

\subsection{Final member of compounds} \label{sec:final.compounds}
Morphological changes affecting the last members of compounds are less common that those on the first members. The only productive morphological alternation in this context is the loss of possessive prefix when the last member is an IPN.

\subsubsection{Loss of possessive prefix} \label{sec:possessive.prefix.second.compounds}
In compounds with an IPN as final element, the indefinite possessive prefix is lost as a rule, as in for example in the plant name \japhug{kʰɯnajme}{Setaria viridis} from \japhug{kʰɯna}{dog} and \japhug{tɤ-jme}{tail}.

Exceptions are very few. They include compounds whose second element is itself a compound, such as \japhug{lɤndʐitɤlɤtsʰaʁ}{Delphinium sp.} from \japhug{lɤndʐi}{ghost} and \japhug{tɤlɤtsʰaʁ}{milk filter}; the second element is from \japhug{tɤ-lu}{milk} in \textit{status constructus} and \japhug{tsʰaʁ}{sieve}. In \japhug{lɤndʐitɤlɤtsʰaʁ}{Delphinium sp.}, the indefinite possessive prefix \forme{tɤ-} has become frozen when the compound \japhug{tɤlɤtsʰaʁ}{milk filter} was formed, and is therefore not subject to deletion.

Another exceptional example is \japhug{ɯ-qataʁrɯ}{hoof} from \japhug{tɯ-qa}{root, paw, bottom} and \japhug{ta-ʁrɯ}{horn}, perhaps because the second element was perceived as being alienabilized, meaning `the horn-like thing on the foot'; in alienabilized possessive forms, definite possessive prefixes are stacked onto the indefinite possessive instead of replacing it,  see \ref{sec:alienabilization}).

\subsubsection{Alternations} \label{sec:second.member.alternation} 
Morphophonological alternations affecting last members of compounds are very rare in Japhug. 

Internal sandhi influencing the second member of a compound rather than the first occur when a root ending in \ipa{-ʁ} is followed by a cluster with a velar fricative as first element. Thus, the incorporating verb \japhug{amɲaχtsʰɯm}{be petty} is the denominal of a lost compound \forme{*mɲaχtsʰɯm} comprising \japhug{tɯ-mɲaʁ}{eye} as first element and \japhug{xtsʰɯm}{be thin}: the combination of \forme{-ʁ+xtsʰ-} yields \forme{-χtsʰ-}.

Several cases of alternations in the last member are found with animal nouns with the uvular class prefix \forme{qa-}, which has a variant \ipa{χ-/ʁ-} in this context in some compounds (see \ref{sec:uvular.animal} and \ref{sec:uvular.other}). 

Other alternations are restricted to specific lexical items, which are discussed below one by one (\japhug{rŋgɤβ}{attach}, \japhug{ɣɯrni}{be red}, \japhug{tʂu}{path} and \japhug{tɯ-ɣli}{excrement, dung}).

The IPN \japhug{tɯ-mtʰɤrɴɢɤβ}{part of the trouser where one can tuck things in} (a noun whose meaning is better explained by an example sentence like \ref{ex:WmthArNGAB}) is a compound of the noun \japhug{tɯ-mtʰɤɣ}{waist} with the transitive verb \japhug{rŋgɤβ}{attach}, which appears as a uvularized allomorph \forme{-rɴɢɤβ} not attested otherwise; it is unclear why uvularization took place in this word (dissimilation with the coda \ipa{-ɣ} of the previous root is unlikely).

\begin{exe}
\ex \label{ex:WmthArNGAB}
\gll tsʰi tɤ-mda tɕe nɯ ɯʑo ɯ-cʰɤmdɤru nɯ pjɯ-nɯ-rʁe tɕe pjɯ-nɯ-tsʰi, mɯ-na-tsʰi tɕe tɕe li tu-nɯ-χɕoʁ tɕe ɯ-mtʰɤrɴɢɤβ cʰɯ-nɯ-rʁe \\
drink:\textsc{fact} \textsc{pfv}-be.the.moment \textsc{lnk} \textsc{dem} \textsc{3sg} \textsc{3sg.poss}-drinking.straw \textsc{dem} \textsc{ipfv}:\textsc{down}-\textsc{auto}-insert \textsc{lnk} \textsc{ipfv}:\textsc{down}-\textsc{auto}-drink \textsc{neg}-\textsc{pfv}:3\fl3'-drink \textsc{lnk} \textsc{lnk} again \textsc{ipfv}:\textsc{up}-\textsc{auto}-take.out \textsc{lnk} \textsc{3sg.poss}-tuck \textsc{ipfv}:\textsc{downstream}-\textsc{auto}-insert \\
\glt `When it is time to drink, he inserts his straw (into the jar) and drinks from it, and when he does not drink any more, he takes it out and tucks it back into his trousers.' (30-tChorzi, 45)
\end{exe}

The noun \japhug{ftɕɤru}{path in the middle of the fields} is a compound of \japhug{ftɕar}{summer} and \japhug{tʂu}{path} (such paths are made during summer to allow workers to work in the field without damaging the crops). The first element \forme{ftɕɤ-} is the \textit{status constructus} of \forme{ftɕar} (with loss of final consonant) and the form \forme{-ru} for the second member of the compound is a clue that \forme{tʂu} comes from earlier \forme{*t-ro} with a dental stop+\ipa{r} cluster changing to a retroflex affricate (see § XXX and § \ref{sec:teens}) -- the \forme{*t-} element being prefixal (perhaps a fossilized indefinite possessor prefix).

The noun \japhug{jmɤɣni}{russula} clearly derives from \japhug{tɤjmɤɣ}{mushroom} and \japhug{ɣɯrni}{be red}, but while the loss of the \forme{tɤ-} prefix can be explained (see \ref{sec:frozen.indef}), the form of the second element (without \forme{r-} preinitial) is a mystery. The form \forme{-rni} (without \forme{ɣɯ-}, a prefix possibly of denominal origin, see § XXX) is found in \japhug{qrorni}{red ant} with \japhug{qro}{ant} as first element (a late innovation specific to the Kamnyu dialect, § XXX).
 
 The compound \japhug{ŋɤlitɕaʁmbɯm}{dung beetle}, with the irregular \textit{status constructus} \forme{ŋɤ-} (see \ref{sec:reduced.forms.compounds}) of the noun \japhug{nɯŋa}{cow}, contains a syllable  \forme{-li} clearly derived from the IPN \japhug{tɯ-ɣli}{excrement, dung}.\footnote{The second part of the noun \forme{-tɕaʁmbɯm} contains \japhug{aʁmbɯm}{be concave}. }  
 
The examples above show that most of the forms with irregular second member also present some irregularity in the first member of the compound. 

\section{Compound nouns}
Nominal compounds in Japhug can be build by compounding nouns, but also verbs and adverbs. In this section, compounds are first classified by the part of speech of their elements, and then by the semantic relationship between these elements.

\subsection{Noun-Noun compounds} \label{sec.n.n.compounds}
Noun-Noun compounds can be divided in three classes: \textit{tatpurusha} (determinative),   \textit{karmadhāraya} (attributive), appositive, \textit{bahuvrīhi} (possessive)  and \textit{dvandva}.

\subsubsection{Tatpurusha} \label{sec:tatpurusha.n.n}
\textit{Tatpurusha} or determinative compounds (corresponding to a genitive phrase followed by its head noun) are the most common type of compounds in Japhug. Almost all compounds of this type follow Modifier-Head order.

While genitive phrases are followed by a noun with a third person possessive prefix (see § XXX), in the corresponding compounds the possessive prefix is deleted (except the indefinite possessor prefix in exceptional examples, see \ref{sec:possessive.prefix.second.compounds}).

In this type of compounds, the first element is most commonly in \textit{status constructus} if from a word ending in open syllable, both for highly lexicalized compounds \japhug{qaɕpɤrnoʁ}{wild strawberry} (`frog's brain', from \japhug{qaɕpa}{frog} and \japhug{tɯ-rnoʁ}{brain}) and more transparent ones (\japhug{jlɤndʐi}{hybrid yak hide} from \japhug{jla}{hybrid yak}  and \japhug{tɯ-ndʐi}{skin}). 

Yet, \textit{tatpurusha}-s without \textit{status constructus} are also attested even among lexicalized terms such as plant names, for instance \japhug{qaprimdʑu}{Sagittaria trifolia} (from \japhug{qapri}{snake} and \japhug{tɯ-mdʑu}{tongue}). At least some of these compounds can be turned back into a noun phrase with a possessive prefix on the head noun, for instance \forme{qapri ɯ-mdʑu} (snake \textsc{3sg.poss}-tongue) `a snake's tongue'.

Among \textit{tatpurusha}-s are compounds with a participle as second element, including both S/A-participles in \forme{kɯ-} and oblique participles in \forme{sɤ-}. Common examples include \japhug{tʂɤsɤɴɢɤt}{crossroad}  from the \textit{status constructus} of \japhug{tʂu}{path} and the oblique participle \japhug{ɯ-sɤ-ɴɢɤt}{place where X part ways} from \japhug{ɴɢɤt}{part ways, part company}.\footnote{This intransitive verb itself is the anticausative of \japhug{qɤt}{separate}, § XXX. }

Compounds with oblique participles as first, rather than second, element, are also attested, for instance the obsolete noun \japhug{sɤqrɤcʰa}{alcohol to treat the guests}, from the oblique participle \forme{sɤ-qru} of the verb \japhug{qru}{greet, welcome, receive} and the noun \japhug{cʰa}{alcohol}.

In the case of subject participles, the compound does not derive from a genitival construction, though it is superficially similar to it. For instance \japhug{qalekɯtsʰi}{species of kite} comes from \japhug{qale}{wind} and the participle \japhug{ɯ-kɯ-tsʰi}{blocking (it)} of the transitive verb \japhug{tsʰi}{block}; the phrase \forme{qale ɯ-kɯ-tsʰi} (wind \textsc{3sg.poss}-\textsc{nmlz}:S/A-block) `blocking the wind' is more properly a headless participial relative (§ XXX), and is more similar to Object-Verb compounds (\ref{sec:object.verb.compounds}).

An example of Head-Modifier \textit{tatpurusha} in Japhug is provided by the ICN \japhug{tɯ-pɤrme}{one year of life}, which comes from \japhug{tɯ-xpa}{one year} and   \japhug{tɯrme}{man} (see § \ref{sec:frozen.indef} on the \forme{tɯ-} prefix). In this word, which originally means `man's year (of life)', the word `man' (originally a modifier) appears second element.

\subsubsection{Karmadhāraya}  \label{sec:karmadharaya.n.n}

Karmadhāraya are compounds built from nouns and their pre- or post nominal modifiers (§ XXX). These modifiers are distinct from genitival modifiers (§ XXX) and include in particular nouns referring to places, species, or people (\ref{sec:unpossessible.nouns}).  Both Modifier-Head and Head-Modifier orders are attested.


The ethnic names \japhug{kɯrɯ}{Tibetan} and \japhug{kupa}{Chinese} commonly occur as first-element modifiers in compounds such as \japhug{kupaŋga}{Chinese-style clothes} (with \japhug{tɯ-ŋga}{clothes}) or \japhug{kupastaχpɯ}{soja}(with \japhug{staχpɯ}{pea}, on which see \ref{sec:vowel.alternations.compounds}). 

Nouns denoting locations and places as first element of compounds include \japhug{sɯŋgɯ}{forest} in \japhug{sɯŋgɯrmɤβja}{lophophoprus} (with \japhug{rmɤβja}{peacock}) and \japhug{sɯŋgɯpɤjka}{type wild squash} (with \japhug{pɤjka}{squash}) -- although \forme{sɯŋgɯ} means `forest' (this word itself is a compound from \japhug{si}{tree} and \japhug{ɯ-ŋgɯ}{inside}, and literally means `among the trees), it is better translated as `wild' when occurring as prenominal modifier or first member of compounds. Compounds also exist with specific placenames such as \japhug{tɕʰɯtɕɯn}{Jinchuan}, for instance in \japhug{tɕʰɯtɕɯnpaχɕi}{pear} (with  \japhug{paχɕi}{apple} as second element), a noun which can undergo denominal derivation to  \japhug{nɯtɕʰɯtɕɯnpaχɕi}{pick pears} (§ XXX), showing that the place name modifier has been integrated.

The noun \japhug{qajɯ}{bug} occurs as first element of many compounds, such as \japhug{qajɯsmɤnba}{leech} (with \japhug{smɤnba}{doctor}). In this case meaning alone makes it clear that the  first element of the compound does not derive from a genitive phrase, as this compound means `bug acting as a doctor' or `doctor who is a bug', not `doctor treating bugs'.

Head-Modifier \textit{karmadhāraya}-s are the lexicalized versions of nouns followed by post-nominal modifiers (\ref{sec:unpossessible.nouns}). A good example of such compounds is provided by \japhug{ʑmbrɯkɤlu}{willow that does not grow high} from \japhug{ʑmbri}{willow} and the privative form \japhug{kɤlu}{headless} (\ref{sec:privative}) of \japhug{tɯ-ku}{head}), a name explained in (\ref{ex:kAlu}). 

\begin{exe}
\ex \label{ex:kAlu}  
\gll ɯ-taʁ ɯ-mnɯ kɯnɤ kɯ-zri tu-ɬoʁ mɯ́j-cʰa tɕe, nɯ-kɤ-ʁndzɤr ʑo ɲɯ-fse tɕe nɯ ʑmbrɯkɤlu tu-kɯ-ti ŋu. tɕe nɯ ɯ-ku kɯ-me kɤ-ti ɲɯ-ŋu.  kɤlu nɯ ɯ-ku kɯ-me kɤ-ti ɲɯ-ŋu.\\
\textsc{3sg}-on \textsc{3sg.poss}-new.twig also \textsc{nmlz}:S/A-be.long \textsc{ipfv}:\textsc{up}-come.out \textsc{neg:sens}-can \textsc{lnk} \textsc{pfv}-\textsc{nmlz}:P-cut \textsc{emph} \textsc{sens}-be.like \textsc{lnk} \textsc{dem} plant.name \textsc{ipfv}-\textsc{genr}-say be:\textsc{fact} \textsc{lnk}  \textsc{dem} \textsc{3sg.poss}-head \textsc{nmlz}:S/A-not.exist \textsc{inf}-say \textsc{sens}-be headless \textsc{dem} \textsc{3sg.poss}-head \textsc{nmlz}:S/A-not.exist \textsc{inf}-say \textsc{sens}-be\\
\glt `Its new twigs cannot grow very long, and look like they have been sawed short, therefore it is called `headless willow'. `Headless' means `without head'.'(07-Zmbri, 34-36)
\end{exe}

\subsubsection{Appositive} \label{sec:appositive.n.n}
Appositive compounds, traditionally analyzed as a subclass of \textit{karmadhāraya}, are rare in Japhug. We find one example both of whose members are participles: \japhug{kɯrŋukɯɣndʑɯr}{harvestman}, from the \forme{kɯ-} participles of the transitive verbs \japhug{rŋu}{parch} and  \japhug{ɣndʑɯr}{grind}. The two elements of the compound refers to the actions supposedly performed by that type of chelicerate (`the parcher-grinder') like the participial form of a bipartite verb (section XXX).


\subsubsection{Bahuvrīhi} \label{sec:bahuvrihi.n.n}
\textit{Bahuvrīhi}-s are considerably less common than \textit{tatpurusha}-s, and tend to be synchronically obscure. Head-Modifier order is found for instance in the \textit{bahuvrīhi} plant name  \japhug{kʰɯlu}{Euphorbia helioscopia} combining the reduced \textit{status constructus} of \japhug{kʰɯna}{dog} (\ref{sec:reduced.forms.compounds}) with \japhug{tɤ-lu}{milk}. It presumably means `having dog milk', a reference to a whitish toxic liquid that comes from it (\ref{ex:khWlu}).

\begin{exe}
\ex \label{ex:khWlu}  
\gll tɕe nɯ kʰɯlu nɯnɯ sɤndɤɣ. ɯ-lu tu tɕe, tɤ-lu kɯ-fse kɯ-wɣrɯ\redp{}wɣrum ŋu. koŋla ʑo, pjɯ́-wɣ-qlɯt tɕe, nɯre ɯ-lu tu. \\
\textsc{lnk} \textsc{dem} Euphorbia.helioscopia \textsc{dem} poisonous:\textsc{fact} \textsc{3sg.poss}-milk exist:\textsc{fact} \textsc{lnk}  \textsc{indef.poss}-milk \textsc{nmlz}:S/A-be.like \textsc{nmlz}:S/A-\textsc{emph}\redp{}be.white be:\textsc{fact} completely \textsc{emph} \textsc{ipfv}-\textsc{inv}-break there \textsc{3sg.poss}-milk exist:\textsc{fact} \\
\glt `The Euphorbia is toxic, it has a juice white like milk, when it is broken, there is milk in (the stalk). (19-khWlu, 20-22)
\end{exe}
Among \textit{bahuvrīhi}-s are found compounds containing a numeral (\textit{dvigu}). The numeral \japhug{kɯngɯt}{nine} in \japhug{kɯngɯttɤrtsɤɣ}{Leonurus} (with \japhug{tɤ-rtsɤɣ}{stairs}, `(plant) having nine stairs') is prenominal, unlike normal order (see § XXX) but similar to quantifiers (see § XXX). The same Numeral-Noun order is found in the more complex compound \japhug{kɯngɯttɤrqʰɤɴɢaʁ}{unidentified plant} discussed in \ref{sec.n.v.compounds}.

\subsubsection{Dvandva} \label{sec:dvandva.n.n}
\textit{Dvandva} compounds are less common, and formally indistinguishable from the previous classes, but semantically intrinsically collectives (see § \ref{sec:collective} for collective derivations, some of which also combine several nominal roots). Examples include \japhug{cʰɤmtʰɯm}{food and drinks} from \japhug{cʰa}{alcohol} and \japhug{tɤ-mtʰɯm}{meat}, \japhug{sŋiɕɤr}{night and day} from \japhug{tɯ-sŋi}{one day} and \japhug{ɕɤr}{night} and \japhug{χcʰoʁe}{right and left} from \japhug{χcʰa}{right} and \japhug{ʁe}{left}, the latter two being mainly used as adverbs.


\subsection{Verb-Verb compounds} \label{sec.v.v.compounds}
There are two types of Verb-Verb nominal compounds in Japhug, action nominals (involving transitive action verbs) and degree nouns (with adjectival stative verbs).

\subsubsection{Action nominals} \label{sec.v.v.compounds.action}
Action nominals built from two verb roots are not common in Japhug. Some of such action nominals are made from verbs with complementary of near-identical meanings, for instance \japhug{joʁβzɯr}{tidying up} from \japhug{joʁ}{raise} and \japhug{βzɯr}{move}. This noun occurs in a light verb construction as in (\ref{ex:joRBzWr}). The denominal compound verb  \japhug{rɤjoʁβzɯr}{tidy up} has a meaning verb close to this construction (see § XXX).

\begin{exe}
\ex \label{ex:joRBzWr}
 \gll joʁβzɯr tɤ-βzu-t-a \\
 tidying.up \textsc{pfv}-do-\textsc{pst:tr-1sg} \\
 \glt `I did some tidying up.' (elicited)
\end{exe}

Another type of verb-verb action nominal are made from verbs with opposite meanings, for instance \japhug{βʁɤnŋo}{winning and losing} from \japhug{βʁa}{win} and \japhug{nŋo}{lose}, which is used with existential verbs as in (\ref{ex:BRAnNo.maNEndZi}).

\begin{exe}
\ex \label{ex:BRAnNo.maNEndZi}
 \gll βʁɤnŋo maŋe-ndʑi \\
winning.and.losing not.exist:\textsc{sens}-\textsc{du} \\
\glt `One cannot decide who (of the two of them) is winning and who is losing.' 
\end{exe}

At an earlier stage, such compound action nominals may have been common, as is suggested by the existence of denominal compound verbs without corresponding noun, such as \japhug{raχtɯtsɣe}{do commerce} (from \japhug{χtɯ}{buy} and \japhug{ntsɣe}{sell}, § XXX on the \forme{-n-} element).

\subsubsection{Degree nouns} \label{sec.v.v.compounds.degree}
The productive way of building degree nouns in Japhug is by adding the prefix \forme{tɯ-} to an adjectival stative verb (§ XXX), but an alternative formation involves the compounding of two antonymic verbs, such as \japhug{jpumxtsʰɯm}{thickness} from \japhug{jpum}{be thick} and \japhug{xtsʰɯm}{be thin}. All known examples are collected in Table \ref{tab:degree.comp}; note that the first member of these compounds is in \textit{status constructus} if possible.

\begin{table}
\caption{Degree compound nouns} \label{tab:degree.comp}
\begin{tabular}{llll}
 \lsptoprule 
 Compound & First verb & Second verb \\
 \midrule
\japhug{jpumxtsʰɯm}{thickness} (diameter) &\japhug{jpum}{be thick} &\japhug{xtsʰɯm}{be thin} \\
\japhug{jaʁmba}{thickness} (of a sheet)&\japhug{jaʁ}{be thick} &\japhug{mba}{be thin} \\
\japhug{xtɯrɲɟi}{length} &\japhug{xtɯt}{be short} &\japhug{rɲɟi}{be long} \\
\japhug{xtɕɯxte}{size} &\japhug{xtɕi}{be small} &\japhug{wxti}{be be} \\
 \lspbottomrule
\end{tabular}
\end{table}

In the case of \japhug{xtɕɯxte}{size}, the second element \forme{xte} is a variant also found in the derived verb \japhug{mɯxte}{be the majority}, probably a remnant of a former \forme{*i/e} alternation still observed in the verb  \japhug{ɣi}{come} (§ XXX).

These nouns can further derive denominal verbs in \forme{a-} meaning `of uneven X', for instance   \japhug{ajpomxtsʰɯm}{having uneven thickness}  (with unexplained \ipa{a} / \ipa{o} alternation, see § XXX).

\subsection{Adverb-Verb compounds} \label{sec.adv.v.compounds}
Adverb-Verb compounds are relative marginal. Compounds with \japhug{kɯzɣa}{a long time, many times} in \textit{status constructus} \forme{kɯzɣɤ-} followed by a verb are however attested, as \japhug{kɯzɣɤ-ɕar}{searching a long time} from the verb \japhug{ɕar}{search} in (\ref{ex:kWZGACar}). These compounds only occur as objects of the light verb \japhug{βzu}{make}. This construction is studied in more detail in section XXX (see also \citealt[252]{jacques16complementation}).

\begin{exe}
\ex \label{ex:kWZGACar}
\gll kɯ-xtɕi nɯ ɣɯ pjɤ-me tɕe, tɕendɤre rca kɯzɣɤ-ɕar ʑo ɲɤ-βzu-nɯ ri pjɤ-me.\\
\textsc{nmlz}:S/A-be.small \textsc{dem} \textsc{gen} \textsc{ifr}.\textsc{ipfv}-not.exist \textsc{lnk} \textsc{lnk} unexpectedly long.time-search \textsc{emph} \textsc{ifr}-do-\textsc{pl}  \textsc{lnk} \textsc{ifr}.\textsc{ipfv}-not.exist \\
\glt `The (pigeon skin) of the youngest girl was not there, there looked for it a long time but it was not there.' (the flood 2002, 55)
\end{exe}

\subsection{Noun-Verb compounds} \label{sec.n.v.compounds}
Noun-Verb compounds include three main types, Subject-Verb, Object-Verb and Adjunct-Verb compounds. Participles or other nominalized verbs forms are treated in sections \ref{sec.n.n.compounds}, but criteria to discriminate between ambiguous forms in cases of homophony between noun and verb are provided in \ref{sec:object.verb.compounds}.

\subsubsection{Subject-Verb compounds} \label{sec:subject.verb.compounds}
Subject-Verb compounds exclusively occur with intransitive verbs, mainly adjectival stative verbs. The noun is generally in \textit{status constructus} (see Table \ref{tab:subj.v.compounds} below). Three types are found depending on their meaning: \textit{karmadhāraya}, \textit{bahuvrīhi} or action nominals.

 Compounds of the \textit{karmadhāraya} type are equivalent to a participial relative with the intransitive subject as the relativized element. If the nominal and verbal elements are represented as N and V respectively, a \textit{karmadhāraya} NV compound means  `N which Vs'. They  are common with stative verbs of colour such as \japhug{ɲaʁ}{be black} or \japhug{wɣrum}{be white} as in Table \ref{tab:subj.v.compounds}. 
 
\begin{table}
\caption{Examples of Subject-Verb  \textit{karmadhāraya} compound nouns} \label{tab:subj.v.compounds}
\begin{tabular}{llllll}
\lsptoprule
 Compound& Base Noun & Verb\\
 \midrule
\japhug{tɤɕɤɲaʁ}{black barley} & \japhug{tɤɕi}{barley} & \japhug{ɲaʁ}{be black} \\
\japhug{tɤɕɤɣrum}{white barley} & & \japhug{wɣrum}{be white}  \\
\japhug{mtsʰalɤɲaʁ}{black nettle} & \japhug{mtsʰalu}{nettle} & \japhug{ɲaʁ}{be black} \\
\japhug{mtsʰalɤɣrum}{white nettle} & & \japhug{wɣrum}{be white}  \\
\japhug{qartsɯɲaʁ}{cold winter} & \japhug{qartsɯ}{winter} & \japhug{ɲaʁ}{be black} \\
\japhug{pɣɤɲaʁ}{Pucrasia macrolopha} & \japhug{pɣa}{bird} &   \\
\japhug{tɤmtɯɲaʁ}{deadlock} & \japhug{tɤ-mtɯ}{knot} &   \\
\lspbottomrule
\end{tabular}
\end{table}

The compounds in Table \ref{tab:subj.v.compounds} are highly lexicalized; in the case for instance of \japhug{pɣɤɲaʁ}{Pucrasia macrolopha}, this bird is not even black as the speakers themselves point out (\ref{ex:pGAYaR}).

\begin{exe}
\ex \label{ex:pGAYaR}
 \gll pɣɤɲaʁ kɤ-ti ci tu tɕe, nɯnɯ ʁo lɯski li nɯ pɣa ŋu, tɕeri mɤ-ɲaʁ ma ɲɯ-mpɕɤr. ɯ-muj nɯra wuma ʑo ɲɯ-mpɕɤr qhe kɯ-tu ra ɲɯ-nɤmbju ʑo \\
 Pucrasia.macrolopha \textsc{nmlz}:P-say \textsc{indef} exist:\textsc{fact} \textsc{lnk} \textsc{dem} \textsc{advers} of.course again \textsc{dem} bird be:\textsc{fact} but \textsc{neg}-be.black:\textsc{fact} \textsc{lnk} \textsc{sens}-be.beautiful \textsc{3sg.poss}-feather \textsc{dem:pl} really \textsc{emph}   \textsc{sens}-be.beautiful \textsc{lnk} \textsc{nmlz}:S/A-exist \textsc{pl} \textsc{sens}-be.brilliant \textsc{emph} \\
 \glt `The Pucrasia macrolopha is of course also a bird (like the previous ones we talked about), but it is not black, it is beautiful, its feathers are very beautiful and those that are there (visible) are iridescent.'
\end{exe}

More complex NV compounds of this type are found, such as \japhug{tɯ-jaʁndzumɤpaχcɤl}{middle finger} from \japhug{tɯ-jaʁndzu}{finger} and \japhug{mɤpaχcɤl}{be in the middle} (itself a denominal verb from \japhug{ɯ-χcɤl}{middle}).

In such compounds, some stative verbs occur with a \forme{-x-} element in individual forms. This is the case of \japhug{tɤlɤxcʰi}{fresh milk} from \japhug{tɤ-lu}{milk} and \japhug{cʰi}{be sweet}. It is possible that this velar fricative represents the remnant of a participle prefix \forme{kɯ-}; note however its presence in some derivations, in particular causative and tropative (\japhug{nɤxcʰi}{to find sweet}, § XXX).
 
\textit{Bahuvrīhi} compounds are equivalent to a participial relative with the possessor of the subject as the relativized element (section XXX); In other words, a \textit{bahuvrīhi} NV means `(person/animal/entity) whose N Vs'. Examples are considerably fewer than \textit{karmadhāraya}-s, but include \japhug{ɕnɤsti}{person with a stuffy nose} (from \japhug{tɯ-ɕna}{nose} and \japhug{asti}{be blocked}, see the discussion in \ref{sec:object.verb.compounds}) and  \japhug{ɕnaβndʑɣi}{snotty-nosed kid} (from \japhug{tɯ-ɕnaβ}{snot} and a verbal root  \forme{-ndʑɣi} attested in \japhug{nɤndʑɣi}{have (snot)}). 
 
A particularly interesting Noun-Verb \textit{bahuvrīhi} is the plant name \japhug{kɯngɯttɤrqʰɤɴɢaʁ}{unidentified plant}, which comprises three elements: the numeral \japhug{kɯngɯt}{nine}, the IPN \japhug{tɤ-rqʰu}{skin, hull} and the intransitive verb \japhug{ɴɢaʁ}{peel, shed skin} (anticausative of \japhug{qaʁ}{peel}, see § XXX). This compound is to be parsed [\forme{kɯngɯt-tɤrqʰɤ-}][\forme{ɴɢaʁ}] from a morphological point of view, as its meaning is `(plant) whose nine skins shed off' as is explained in the text excerpt (\ref{ex:kWngWttArqhANGaR}): the first element \forme{kɯngɯt-tɤrqʰɤ-}\footnote{Note that this compound has Numeral-Noun order as in other examples (see \ref{sec:bahuvrihi.n.n}).} corresponds to the intransitive subject of \japhug{ɴɢaʁ}{peel, shed skin}. 

\begin{exe}
\ex \label{ex:kWngWttArqhANGaR}
\gll kɯngɯttɤrqʰɤɴɢaʁ ɯ-rmi kɯra nɯnɯ tɕendɤre, ɯ-rqʰu kɯ-dɯ\redp{}dɤn ʑo pjɯ-ɴɢaʁ ɲɯ-ŋu. nɯnɯ tɯ-mpɕar nɤ tɯ-mpɕar, tɯ-mpɕar nɤ tɯ-mpɕar, pɯ-ɴɢaʁ qʰe ɯ-ŋgɯ li mɤʑɯ ɲɯ-βze qʰe, \\
plant.name \textsc{3sg.poss}-name \textsc{dem:prox:pl} \textsc{dem} \textsc{lnk} \textsc{3sg.poss}-skin \textsc{nmlz}:S/A-\textsc{emph}\redp{}be.many \textsc{emph} \textsc{ipfv}-\textsc{anticaus}:peel \textsc{sens}-be \textsc{dem} one-leaf \textsc{lnk}  one-leaf  one-leaf \textsc{lnk}  one-leaf \textsc{pfv}-\textsc{anticaus}:peel \textsc{lnk} \textsc{3sg.poss}-inside again yet \textsc{ipfv}-grow \textsc{lnk}  \\
\glt `As for the name of the \forme{kɯngɯttɤrqʰɤɴɢaʁ}, (it is because) it has a lot of skins that shed off, one after the other, and after one has shed off, another one grows again inside.' (14-sWNgWJu, 72-5)
\end{exe}

Yet, from a phonological point of view, the form should rather be parsed as [\forme{kɯngɯt-}][\forme{tɤrqʰɤ-ɴɢaʁ}], as the phonological integration between \forme{tɤrqʰɤ-} in \textit{status constructus} and the following verb root is stronger than that between the numeral \japhug{kɯngɯt}{nine} and the rest, as shown by the preservation of the final \forme{-t} (§ XXX on internal sandhi).

Action nominals NV compounds are rare with intransitive verbs. Examples include \japhug{pɣɤmbri}{bird song} from \japhug{pɣa}{bird} and the intransitive \japhug{mbri}{cry, sing} or \japhug{snɯɲaʁ}{harming people} from  \japhug{tɯ-sni}{heart} and \japhug{ɲaʁ}{be black}, a compound serving as the base of many denominal verbs (§ XXX).

 
\subsubsection{Object-Verb compounds} \label{sec:object.verb.compounds}
Object-Verb nominal compounds in Japhug are very productive, and can be classified into two main types: actor OV compounds, and action OV compounds.

Actor OV compounds are common in names of trades, animals and even plants, such as \japhug{rŋɯlfɕi}{silversmith}, \japhug{βɣɤru}{miller}, \japhug{zrɯɣndza}{praying mantis} and \japhug{tɤtɕɯβraʁ}{burdock}. The first of these examples, from the Tibetan loanword \japhug{rŋɯl}{silver} and the labile verb \japhug{fɕi}{forge}, requires little explanation. Some compounds present significant morphological alterations, as  \japhug{βɣɤru}{miller}, which comes from the \textit{status constructus} of \japhug{βɣa}{mill} and the non-reduplicated form of the verb \japhug{rɯru}{guard, look after} (§ XXX). In addition, some compounds of this type do not make much sense without some cultural background; as an illustration of how the Japhug corpus can be used to better understand the origin of these compounds, I discuss below the latter two nouns.

The compound \japhug{zrɯɣndza}{praying mantis} derives from \japhug{zrɯɣ}{louse} and \japhug{ndza}{eat}, and literally means `louse eater', a descriptive term based on the feeding habits of that insect, as described in (\ref{ex:zrWGndza}).

\begin{exe}
\ex \label{ex:zrWGndza}
\gll nɯ ɯ-taʁ ri zrɯɣndza kɤ-ndo-tɕi tɕe, .... tɕendɤre zrɯɣ rcanɯ lɤŋɤtʂɤɣ jamar ʑo ɯ-ɕki kɤ-ta-tɕi. tɕe kɯ-mɤku nɯra tɕe, tɕe zrɯɣ nɯ lonba ʑo cʰɯ-mqlaʁ tɕe tu-ndze ɲɯ-ŋu. tɕendɤre kɯ-maqʰu tɕe nɤki ɲɯ-ŋu,  tɕendɤre ku-nɯni kɯ-fse qhe, ɯ-ŋgɯ nɯnɯ,  ɯ-se nɯ lu-nɯ-tɕɤt qʰe cʰɯ-mqlaʁ ɲɯ-ŋu. \\
\textsc{dem} \textsc{3sg}-on \textsc{loc} praying.mantis \textsc{pfv}-take-\textsc{1du} \textsc{lnk} .... \textsc{lnk} louse \textsc{foc}:\textsc{unexp} five.or.six about \textsc{emph} \textsc{3sg}-\textsc{dat} \textsc{pfv}:\textsc{east}-put-\textsc{1du} \textsc{lnk} \textsc{nmlz}:S/A-be.first \textsc{dem:pl} \textsc{lnk} \textsc{lnk} louse \textsc{dem} \textsc{all} \textsc{emph} \textsc{ipfv}-swallow \textsc{lnk} \textsc{ipfv}-eat[III] \textsc{sens}-be \textsc{lnk} \textsc{nmlz}:S/A-be.after \textsc{lnk} \textsc{dem}:\textsc{cataph} sens-be \textsc{lnk} \textsc{ipfv}-suck[III] \textsc{nmlz}:S/A-be.like \textsc{lnk} \textsc{3sg.poss}-inside \textsc{dem} \textsc{3sg.poss}-blood \textsc{dem} \textsc{ipfv}:\textsc{upstream}-\textsc{auto}-take.out \textsc{lnk}  \textsc{ipfv}-swallow \textsc{sens}-be \\
\glt `(When we were little, one of my classmate had a lot of lice, and) we took a praying mantis (and put it on his clothes), then put five or six lice near it; the first ones, it swallowed them whole, and the following ones,  it did the following, it would kind of suck them, drink the blood inside them, and then swallow it (and then throw them away).' (26-zrWGndza, 25-35)
\end{exe}

The nouns \japhug{tɤtɕɯβraʁ}{burdock} from (\japhug{tɤ-tɕɯ}{son, boy} and \japhug{βraʁ}{attach}) and  \japhug{tɕʰemeβraʁ}{little burdock} (with \japhug{tɕʰeme}{girl} as first element) literally mean `attaching boys/girls'; an explanation for these names from local folklore is provided in (\ref{ex:tAtCWBraR}).

\begin{exe}
\ex \label{ex:tAtCWBraR}
\gll tɤtɕɯβraʁ tɕe, ɕɯ kɯ pa-mto nɯnɯ tɕe tɕe nɯ ɣɯ ɯʑɤɣ maʁ nɤ, ɯ-kʰa ɣɯ maʁ nɤ,  ɯ-kɯmdza kɯ-fse ra ɣɯ, nɯ-tɕɯ maʁ nɤ nɯ-me tu tu-ti-nɯ ɲɯ-ŋu. tɕe nɯ nɯ-kɯmdza kɯ-fse kɯ-ɤrɕɤt ra, nɯ-skʰrɯ mɤ-kɯ-βdi a-pɯ-tu tɕe, ``wo ... ɯ-rɟit tɤ-tɕɯ sci ma tɤtɕɯβraʁ pɯ-mto-t-a" \\
burdock \textsc{lnk} who \textsc{erg}  pfv:3\fl{}3'-see \textsc{dem} \textsc{lnk} \textsc{lnk} \textsc{dem} \textsc{gen} \textsc{3sg:gen} not.be:\textsc{fact} \textsc{lnk} \textsc{3sg.poss}-house \textsc{gen} not.be:\textsc{fact} \textsc{lnk}  \textsc{3sg.poss}-relative \textsc{nmlz}:S/A-be.like \textsc{pl} \textsc{gen} \textsc{3pl.poss}-son not.be:\textsc{fact} \textsc{lnk} \textsc{3pl.poss}-daughter exist:\textsc{fact} \textsc{ipfv}-say-\textsc{pl} \textsc{sens}-be \textsc{lnk} \textsc{dem} \textsc{3pl.poss}-relative  \textsc{nmlz}:S/A-be.like  \textsc{nmlz}:S/A-be.related \textsc{pl}  \textsc{3pl.poss}-body  \textsc{neg}-\textsc{nmlz}:S/A-be.well \textsc{irr}-\textsc{ipfv}-exist \textsc{lnk} \textsc{interj} .... \textsc{3sg.poss}-child \textsc{indef.poss}-son \textsc{lnk} be.born:\textsc{fact} burdock \textsc{pfv}-see-\textsc{tr:pst}-\textsc{1sg} \\
\glt `The burdock, whoever saw it will have a boy or a girl, him or someone from his house or among his relatives. If someone among his relatives is pregnant, he will say `her child will be a boy, as I saw a burdock.'' (26-NalitCaRmbWm, 109+)
\end{exe}

Action OV compounds generally occur in constructions with light verbs such as \japhug{lɤt}{throw, release} or \japhug{βzu}{make}. Examples include \japhug{cʰɤtsʰi}{alcohol drinking} from \japhug{cʰa}{alcohol} and \japhug{tsʰi}{drink} (compare with the denominal verb  \japhug{ɣɯcʰɤtsʰi}{drink too much alcohol}, section XXX) or \japhug{ʁrɯrpu}{hitting with horns} (not goring) from \japhug{ta-ʁrɯ}{horn} and \japhug{rpu}{bump into}  (see also \japhug{nɯʁrɯrpu}{hit with horns}).\footnote{The object of \japhug{rpu}{bump into} is the body part bumping into something, § XXX.} A complete list of such compounds is provided in section XXX, together with corresponding denominal quasi-incorporating verbs, which are more common in texts. 

These nouns mainly occur with light verbs such as \japhug{lɤt}{throw, release} as in example (\ref{ex:RrWrpu}) (§ XXX), but are also found in other constructions as in (\ref{ex:chAtshi.koGAtChom}), where a free object \japhug{cʰa}{alcohol} with the bare infinitive \forme{ɯ-tsʰi} could also be used (see § XXX).

\begin{exe}
\ex \label{ex:RrWrpu}
 \gll jla kɯ a-taʁ ʁrɯrpu ta-lɤt \\
 hybrid.yak \textsc{erg} \textsc{1sg}-on hitting.with.horns \textsc{pfv}:3\fl{}3'-throw \\
 \glt `The hybrid yak hit me with his horn.' (elicited)
\end{exe}

\begin{exe}
\ex \label{ex:chAtshi.koGAtChom}
 \gll cʰɤtsʰi ko-ɣɤ-tɕʰom tɕe  \\
 alochol.drinking \textsc{ifr}-\textsc{caus}-be.too.much \textsc{lnk} \\
\glt `He had drunk too much alcohol.' (150829 jidian-zh, 16)
\end{exe}
Not all compounds whose second element originates from a transitive verb are Object-Verb (or Adjunct-Verb) compounds. Two potentially ambiguous cases must be pointed out. 

First, there are Noun-Noun compounds whose second element is an IPN deriving from a transitive verb (see § XXX), but which loses its possessive prefix as is regular in compounding (\ref{sec:possessive.prefix.second.compounds}). In such cases the resulting Noun-Noun compound is not formally distinguishable from an Noun-Verb compound, and only the meaning can be used to differentiate between the two classes. For instance, the plant name \japhug{tʂɤɕpʰɤt}{plantain} has the \textit{status constructus} of \japhug{tʂu}{path} as a first element, while its second part \forme{-ɕpʰɤt} can be interpreted as either directly from the verb \japhug{ɕpʰɤt}{patch} (`road patcher') or from the derived noun \japhug{tɤ-ɕpʰɤt}{patch (n)} (a piece of fabric used to patch worn clothes) (`road patch'). In this particular case, the second interpretation is more likely, and hence \japhug{tʂɤɕpʰɤt}{plantain} is better analyzed as a Noun-Noun compound.

Second, when the second element of a Noun-Verb compound is a \forme{a-} passive verb (see § XXX), the \forme{a-/ɤ-} prefix is absorbed by the first element of the compound and becomes invisible. In the resulting form, the second element superficially looks similar to the transitive verb. For instance, the noun \japhug{ɕnɤsti}{person with a stuffy nose} appears to derive from \japhug{tɯ-ɕna}{nose} and the transitive verb  \japhug{sti}{block}. However, semantics rules out such a derivation: the compound is a \textit{bahuvrīhi} literally meaning `whose nose is blocked' (see \ref{sec:subject.verb.compounds}), and cannot be interpreted as `(person) blocking noses', the expected meaning of an Object-Verb compound. Since the passive \japhug{asti}{be blocked} of  \japhug{sti}{block} is well-attested, as shown by (\ref{ex:pjAkAstici}), it is better to analyze \japhug{ɕnɤsti}{person with a stuffy nose} as a Subject-Verb \textit{bahuvrīhi} compound (\ref{sec:subject.verb.compounds}) derived from that passive form.

\begin{exe}
\ex \label{ex:pjAkAstici}
\gll maka ɲɯ́-wɣ-ɕɯɣ-mu mɯ-pjɤ-cʰa ma mɯ-pjɤ-mtsʰɤm matɕi ɯ-rna pjɤ-k-ɤ-sti-ci. \\
at.all \textsc{ipfv}-\textsc{inv}-\textsc{caus}-be.afraid \textsc{neg}-\textsc{ifr.ipfv}-can \textsc{lnk} \textsc{neg}-\textsc{ifr}-hear because \textsc{3sg.poss}-ear \textsc{ifr.ipfv}-\textsc{evd}-\textsc{pass}-block-\textsc{evd} \\
\glt `The noise could not frighten him, as he did not hear it, because his ears were blocked.' (140514 huishuohua de niao, 203)
\end{exe}

\subsubsection{Adjunct-Verb compounds} \label{sec:adjunct.verb.compounds}
Adjunct-Verb compounds are all action nouns, the nominal element being either locative or instrument. Like other action nominal compounds, they can undergo denominal derivation to become quasi-incorporating verb (§ XXX).  

Adjunct-Verb compounds can be made from transitive verbs, as \japhug{zgrɯtɕʰɯ}{nudge} and \japhug{kɤtɕʰɯ}{headbutt} with the verb \japhug{tɕʰɯ}{gore} as second element and the body parts  \japhug{tɯ-zgrɯ}{elbow} and  \japhug{tɯ-ku}{head} as first element. Here the body parts are instrumental adjuncts, as the object of \japhug{tɕʰɯ}{gore} is the person being gored/hit, not the part of the body one uses. These nouns occur with the light verb \japhug{lɤt}{throw, release} as in (\ref{ex:zgrWtChW}). 

\begin{exe}
\ex \label{ex:zgrWtChW}
\gll zgrɯtɕʰɯ tɤ-lat-a \\
nudge \textsc{pfv}-throw-\textsc{1sg} \\
\glt `I nudged (him).' (elicited)
\end{exe}

Examples from intransitive verbs are also attested, as \japhug{qʰaru}{look back} from the locative noun \japhug{ɯ-qʰu}{back} and the semi-transitive \japhug{ru}{look at}. It is also used with  \japhug{lɤt}{throw, release} as in (\ref{ex:qharu}).

\begin{exe}
\ex \label{ex:qharu}
\gll ɯʑo nɯ  tatpa ta-ta ma qʰaru mucin ʑo mɯ-pa-lɤt nɤ tɤ-ari ɲɯ-ŋu. \\
\textsc{3sg} \textsc{dem} faith \textsc{pfv}:3\fl{}3'-put \textsc{lnk} look.back at.all emph \textsc{neg}-\textsc{pfv}:3\fl{}3'-throw \textsc{lnk} \textsc{pfv}:\textsc{up}-go[II] \textsc{sens}-be \\
\glt `He had faith, did not look back at all and (succeeded in) going up to (the abode of the goads). (Norbzang, 129)
\end{exe}

The quasi-incorporating verbs \japhug{sɯzgrɯtɕʰɯ}{nudge}, \japhug{nɤkɤtɕʰɯ}{gore, give a headbutt}, and \japhug{nɤqʰaru}{look back} are considerably more common that light verb constructions with compound action nouns such as (\ref{ex:zgrWtChW}) and (\ref{ex:qharu}).

There is a very productive Noun-Verb compound formation with the noun \japhug{kʰramba}{lie} as first element, meaning `pretending to do  X', compatible with both transitive and intransitive verbs. It occurs in a light verb construction with \japhug{βzu}{make} as in (\ref{ex:khrambatshi}). This construction is studied in more detail in section XXX (see also \citealt[252]{jacques16complementation}).

\begin{exe}
\ex \label{ex:khrambatshi}
\gll ʑara kɯ cʰa nɯ kʰramba-tsʰi ka-βzu-nɯ,  \\
\textsc{3pl} \textsc{erg} alcohol \textsc{dem} lie-drink \textsc{pfv}:3\fl{}3'-make-\textsc{pl} \\
\glt `They pretended to drink alcohol.' (Norbzang, 100)
\end{exe}

The compound \japhug{mɲaʁmtsaʁ}{grasshoper} from \japhug{tɯ-mɲaʁ}{eye} and \japhug{mtsaʁ}{jump} is obscure, but unlikely to be a \textit{bahuvrīhi} `whose eyes jump', and should rather be analyzed as an adjunct compound (maybe `jumping with (big) eyes', as if from a comitative adverb \ref{sec:comitative.adverb}). 

\subsection{Verb-Noun compounds} \label{sec.v.n.compounds}
Verb-Noun compounds are extremely rare in Japhug, as they are in general in Trans-Himalayan languages other than Chinese.  

Adjectival stative verbs nearly always occur as second element in compounds with a noun (\ref{sec:subject.verb.compounds}), but the opposite order is attested in \japhug{sɤŋɤβdi}{disagreeable smell} from \japhug{tɤ-di}{smell} and \japhug{sɤŋɤβ}{be disagreeable},\footnote{The meaning of this verb is difficult to render exactly in English; the best approximation would be `which does not tempt one' (French `qui ne donne pas envie'); it can be used with an infinite complement (§ XXX). It is of denominal origin (§ XXX), from the same nominal root as the transitive \japhug{nɤŋɤβ}{be embarrassed to} (Chinese \zh{觉得不好意思}). }  a noun which can occur with the intransitive verb \japhug{mnɤm}{smell} as in (\ref{ex:sANABdi}).

\begin{exe}
\ex \label{ex:sANABdi}
\gll sɤŋɤβ-di ʑo ɲɯ-mnɤm \\
disagreeable-smell \textsc{emph} \textsc{sens}-smell \\
\glt `There is a disagreeable smell.' (elicited)
\end{exe}
A possible example of Verb-Noun compound with a transitive verb is \japhug{ndzɤpri}{brown bear}, comprising \japhug{pri}{bear} and \japhug{ndza}{eat} -- as shown by (\ref{ex:ndzApri}) from a text about bears, it is considered by some native speakers of Japhug as a man eater, though this explanation could be folk-etymology. Note that this compound is also anomalous in that when transitive verbs are used in compounds with a noun, that noun is either an object (\ref{sec:object.verb.compounds}) or adjunct (\ref{sec:adjunct.verb.compounds}), never the subject.

\begin{exe}
\ex \label{ex:ndzApri}
\gll tɕe ndzɤpri kɤ-ti nɯ tɕe tɯrme tu-kɯ-ndza ɲɯ-ŋgrɤl \\
\textsc{lnk} brown.bear \textsc{inf}-say \textsc{dem} \textsc{lnk} people \textsc{ipfv}-\textsc{genr}:S/P-eat \textsc{sens}-be.usually.the.case \\
\glt `It eats people, so is it called \forme{ndzɤpri}.' (21-pri, 94)
\end{exe} 

We find several examples of nominal compounds whose structure is \forme{tɤ-}+Verb+Noun, where the verb is an adjectival stative verb. This category includes \japhug{tɤqiaβjmɤɣ}{lactarius sp.}, literally `bitter mushroom', from the noun \japhug{tɤjmɤɣ}{mushroom} see \ref{sec:frozen.indef} concerning the lost of \forme{tɤ-}) and the verb \japhug{qiaβ}{be bitter}, or \japhug{tɤmbextsa}{type of shoes} from \japhug{tɯ-xtsa}{shoe} and \japhug{mbe}{be old}. These should not be analyzed as Verb-Noun compounds however, as the first element originates from a nominalized form of the verb (such as \japhug{tɤ-mbe}{old thing}, see § \ref{sec:property.nouns} and § XXX on this derivation): they rather are a subtype of Noun-Noun compounds.

The same applies to compounds whose first element comes from a participle, such as \japhug{kɤrŋijmɤɣ}{type of mushroom} from \japhug{tɤjmɤɣ}{mushroom} with the S-participle \japhug{kɯ-ɤrŋi}{green one} \ipa{kɤrŋi} from the verb \japhug{arŋi}{be green}. Note that the compounding order is unexpected, as participles of adjectival stative verbs generally following the noun (§ XXX). Note that compounds with participles of transitive verbs as first element are also found, as for instance \japhug{kɯqurʑŋgri}{evening star}, literally `the star of the helper', for reasons explained in the following text (\ref{ex:kWqur.ZNgri}).

\begin{exe}
\ex \label{ex:kWqur.ZNgri}
\gll ɯnɯnɯ kɯɕɯŋgɯ tɕe kɯ-qur ju-kɯ-ɕe tɕe nɯnɯ, mɯ-nɯ-ɬoʁ mɤɕtʂa nɯ tu-kɯ-nɯna mɯ-pjɤ-jɤɣ ɲɯ-ŋu tɕe,  tɕe núndʐa kɯqurʑŋgri tu-sɤrmi-nɯ \\
\textsc{dem} before \textsc{lnk} \textsc{nmlz}:S/A-help \textsc{ipfv}-\textsc{genr}:S/P-go \textsc{lnk} \textsc{dem} \textsc{neg}-\textsc{pfv}:\textsc{west}-come.out until \textsc{dem} \textsc{ipfv}-\textsc{genr}:S/P-rest \textsc{neg}-\textsc{ifr}.\textsc{ipfv}-be.possible \textsc{sens}-be \textsc{lnk} \textsc{lnk} for.this.reason evening.star \textsc{ipfv}-call-\textsc{pl} \\
\glt `Long ago, when one would go helping, one was not supposed to rest until it comes out, and for this reason it was called `star of the helper'.' (29-mWBZi, 62)
\end{exe}

 

\section{Noun class prefixes} \label{sec:class.prefixes}
Noun class prefixes are prefixal elements that occur in some nouns, whose root cannot occur on its own, except for a few rare exceptions (such as \japhug{qapɣɤmtɯmtɯ}{hoopoe} discussed in \ref{sec:uvular.animal}). Uvular \forme{qa-/χ-/ʁ-} and velar \forme{kɯ-/x-/ɣ-} prefixes are attested, and occur on animal names, plant names and nouns referring to traditional objects. Additional body part class prefixes, in particular \forme{m-} are also present in Japhug.

Dental prefixal elements such as \forme{tɤ-} or \forme{tɯ-} are very common, but  are  better interpreted as frozen indefinite possessor prefixes (see \ref{sec:frozen.indef}), rather as noun class prefixes.

\subsection{Uvular animal name prefix} \label{sec:uvular.animal}
The uvular animal prefix has a plene form \forme{qa-} (Table \ref{tab:animal.qa}) and a reduced allomorph \forme{χ-/ʁ-}, attested in a few names like \japhug{ʁmbroŋ}{wild yak}, \japhug{rtɕʰɯrjɯ}{caterpillar} and \japhug{tɕʰɯχpri}{salamander}.

Note that \japhug{ʁmbroŋ}{wild yak} is a borrowing from Tibetan \tibet{འབྲོང་}{ⁿbroŋ}{wild yak}, a fact that possibly suggests that the \forme{χ-/ʁ-} prefix has some degree of productivity (see \citealt{jacques14snom}). 

The noun \japhug{qapɣɤmtɯmtɯ}{hoopoe} is clearly a compound containing the \textit{status constructus} of \japhug{pɣa}{bird} and the reduplicated form of the noun \japhug{ɯ-mtɯ}{crest}, to which the class prefix \forme{qa-} has been added. 

The allomorph \forme{qa-} is reduced to its non-syllabic variants \forme{χ-/ʁ-} when the prefixed noun occurs as second member of compound. The nouns \japhug{tɕʰɯχpri}{salamander} and \japhug{rtɕʰɯrjɯ}{caterpillar} are examples of this reduction. The former is a compound of \forme{tɕʰɯ-} (a syllable borrowed  from Tibetan \tibet{ཆུ་}{tɕʰu}{water}) and \forme{-χpri}, a variant of \japhug{qapri}{snake}. The latter comprises the syllable \forme{rtɕʰɯ-}, \textit{status constructus} of the unprefixed root of \japhug{tɯrtɕʰi}{type of vegetable (\zh{酸酸菜})}, and the second \forme{-ʁjɯ} is the reduced variant of \japhug{qajɯ}{worm}.

\begin{table}
\caption{Animal name \forme{qa-} prefix} \label{tab:animal.qa}
\begin{tabular}{l|l}
 \lsptoprule 
\japhug{qacʰɣa}{fox} &	\japhug{qandʐe}{earthworm} \\
\japhug{qaɕɣi}{big fly} &	\japhug{qandʐi}{anadromous fish} \\
\japhug{qaɕpa}{frog} &	\japhug{qandʑɣi}{fox} \\
\japhug{qajdo}{crow} &	\japhug{qaɲi}{mole} \\
\japhug{qajtʂʰa}{aegyptius monachus} &	\japhug{qapar}{dhole} \\
\japhug{qajɯ}{worm} &	\japhug{qapɣɤmtɯmtɯ}{hoopoe} \\
\japhug{qaɟy}{fish} &	\japhug{qapri}{fox} \\
\japhug{qala}{rabbit} &	\japhug{qarma}{crossoptilon} \\
\japhug{qaliaʁ}{eagle} &	\japhug{qartsʰaz}{deer} \\
\japhug{qambalɯla}{butterfly} &	\japhug{qartsʰi}{deer} \\
\japhug{qambrɯ}{male yak} &	\japhug{qaʑo}{sheep} \\
\japhug{qamtɕɯr}{shrew} &	\\
%\japhug{qacʰɣa}{fox} \\
%\japhug{qaɕɣi}{big fly} \\
%\japhug{qaɕpa}{frog} \\
%\japhug{qajdo}{crow} \\
%\japhug{qajtʂʰa}{aegyptius monachus} \\
%\japhug{qajɯ}{worm} \\
%\japhug{qaɟy}{fish} \\
%\japhug{qala}{rabbit} \\
%\japhug{qaliaʁ}{eagle} \\
%\japhug{qambalɯla}{butterfly} \\
%\japhug{qambrɯ}{male yak} \\
%\japhug{qamtɕɯr}{shrew} \\
%\japhug{qandʐe}{earthworm} \\
%\japhug{qandʐi}{anadromous fish} \\
%\japhug{qandʑɣi}{fox} \\
%\japhug{qaɲi}{mole} \\
%\japhug{qapar}{dhole} \\
% \japhug{qapɣɤmtɯmtɯ}{hoopoe} \\
%\japhug{qapri}{fox} \\
%\japhug{qarma}{crossoptilon} \\
%\japhug{qartsʰaz}{deer} \\
%\japhug{qartsʰi}{deer} \\
%\japhug{qaʑo}{sheep} \\
 \lspbottomrule
\end{tabular}
\end{table}

\subsection{Velar animal name prefix}  \label{sec:velar.class.prefix}
While most nouns beginning in \forme{kɯ-} are frozen participles (see § XXX), there is a residue of forms which cannot be accounted as deverbal nouns. Table \ref{tab:animal.kW} presents animal names that are not derivable from any verb root, and appear to bear a \forme{kɯ-} class prefix, distinct from the uvular one.
 
\begin{table}
\caption{Animal name \forme{kɯ-} prefix} \label{tab:animal.kW}
\begin{tabular}{ll}
 \lsptoprule 
\japhug{kɯɕpaz}{marmot} \\
\japhug{kɯjka}{pyrrhocorax} \\
\japhug{kɯmu}{tetraogallus tibetanus} \\
\japhug{kɯpɤz}{type of bug} \\
\japhug{kɯrtsɤɣ}{snow leopard} \\
\japhug{kɯrŋi}{beast} \\
\japhug{kɯrnɯ}{mite} \\
 \lspbottomrule
\end{tabular}
\end{table} 

There is a handful of nouns with reduced allomorphs \forme{ɣ-}, \forme{x-} or even metathesized as \forme{βɣ-} in some words, corresponding to \forme{kə-} in Situ (see the phonological discussion in \citealt[6]{jacques14antipassive}), including \japhug{xɕiri}{weasel}, \japhug{xtɯt}{wild cat},  \japhug{ɣzɯ}{monkey}, \japhug{ɣni}{flying squirrel}, \japhug{βɣɯz}{badger} and \japhug{βɣɤza}{fly}. The same allomorphy is observed between the subject participle \forme{kɯ-} (§ \ref{sec:subject.participles}) and the nominalization prefixes \forme{x-}/\forme{ɣ-} (§ \ref{sec:G.nmlz}).

\subsection{Uvular plant name prefix} \label{sec:uvular.plant}
Quite a number of plant names have a uvular class prefix \forme{qa-}, including both cultivated and wild plants (and even plant parts), such as \japhug{qaɕti}{peach}, \japhug{qaɟɤɣi}{oat}, \japhug{qampʰoʁ}{oak leaves},  \japhug{qandzi}{type of fir}, \japhug{qaʑmbri}{vine}, \japhug{qawɯz}{edelweiss} and many others.
 
\subsection{Other uses of the uvular class prefix} \label{sec:uvular.other}
In addition to animal, plant and body part names, the class prefix \forme{qa-} appears on some tools (\japhug{qajo}{earthen pot}, \japhug{qase}{leather rope}, \japhug{qarɤt}{rake}, 
\japhug{qapi}{flint stone}), names of periods of the year (\japhug{qartsɯ}{winter}, \japhug{qartsɤβ}{harvest}), materials (\japhug{qandʑi}{tin}, \japhug{qambɯt}{sand}) or natural forces like \japhug{qale}{wind}.

The reduced form \forme{ʁ-} of the class prefix occurs with the noun \japhug{qale}{wind} in some compounds such as \japhug{akɯcʰoʁle}{north/east wind} and the abstract IPN \japhug{ɯ-ʁle}{reputation} (and the verbs derived from it, such as \japhug{raʁle}{be polite}).
  
\subsection{Body part noun prefixes}  \label{ex:body.part.prefix}
The identification of class prefixes in body parts mainly rests on comparative evidence. Other Trans-Himalayan languages that preserve clusters such as Tibetan have in some names for body parts cluster that do not match those found in Japhug, for instance \tibet{མཁྲིས་པ་}{mkʰris.pa}{bile} and \tibet{སྐེ་}{ske}{neck} corresponding to the Japhug IPNs \japhug{tɯ-ɕkrɯt}{bile} and \japhug{tɯ-mke}{neck} respectively (see § \ref{sec:body.part}), suggesting that body part class prefixes such as \forme{ɕ-} and \forme{m-} have been added to these words in Gyalrongic and Tibetan independently.

Apart from the \forme{m-} and \forme{ɕ-/ʑ-} class prefixes, some APN body parts such as \japhug{qambɣo}{earwax} have a \forme{qa-}  prefix (\ref{sec:body.part}).

The only evidence for a derivational use of these class prefixes in Japhug is the noun \japhug{tɯ-mci}{saliva}, that may be derived from \japhug{tɯ-ci}{water} by addition of the \forme{m-} class prefix.\footnote{The noun \japhug{tɯ-mgɯr}{back} could be another example, but the verb root from which it could be derivable, \japhug{fkur}{carry on the back}, is likely to be a Tibetan loanword and has a different vocalism. } Given the fact that \japhug{tɯ-ci}{water} is a lexical innovation that is not even shared by Stau and Khroskyabs (see \ref{sec:earth.IPN}), this suggests that the class prefix \forme{m-} may have remained productive even relatively recently.

\section{Nominal derivations}
Nominal derivations pale in comparison of the rich verbal (§ XXX) and even ideophonic (§ XXX) derivations in Japhug. There is little derivational prefixation in nouns (aside from the collective \forme{kɤndʑi-} prefix and derivational uses of class prefixes, as seen in \ref{sec:class.prefixes} above), and nearly all of the suffixes or quasi-suffixes involved in these derivations are traceable to inalienably possessed nouns that are still attested in the language, and have thus nearly no antiquity. 

\subsection{Privative} \label{sec:privative}
The suffix \forme{-lu} can be combined with the \textit{status constructus} form of body part nouns, without possessive prefix, to derive a noun meaning `...less', `without ...' that can be used as a modifier (\ref{sec:unpossessible.nouns}). Examples attested in the  corpus are indicated in Table \ref{tab:privative.lu}, but this derivation appears to be productive.

\begin{table}
\caption{Privative \forme{-lu} suffix} \label{tab:privative.lu}
\begin{tabular}{l|l}
 \lsptoprule 
\japhug{ta-ʁrɯ}{horn} &\japhug{ʁrɯlu}{hornless} \\
\japhug{tɤ-jme}{tail} &\japhug{jmɤlu}{without tail}  \\
\japhug{tɯ-jaʁ}{hand} &\japhug{jaʁlu}{missing a hand} \\
\japhug{tɯ-ku}{head} &\japhug{kɤlu}{headless} \\
 \lspbottomrule
\end{tabular}
\end{table}

These privative forms can be used as modifiers of other nouns, and are placed after the nouns and before determiners such as demonstratives or numerals, as in (\ref{ex:RrWlu}) and (\ref{ex:jmAlu}).

\begin{exe}
\ex \label{ex:RrWlu}
\gll ʑɤni ɣɯ ftsoʁ ʁrɯlu ci ta-rku-nɯ ɲɯ-ŋu \\
\textsc{3du} \textsc{gen} female.hybrid.yak hornless \textsc{indef} \textsc{pfv}:3\fl3'-put.in-\textsc{pl} \textsc{sens}-be \\
\glt `They gave them a hornless female yak (to take with them back to the husband's home.' (2005-stod, 243)
\end{exe}

Privative nouns are systematically glossed in Japhug with possessor participial relatives in \japhug{kɯ-me}{not having} (see § XXX), as in (\ref{ex:jmAlu}) (see also example \ref{ex:kAlu} from section \ref{sec:karmadharaya.n.n})
 

\begin{exe}
\ex \label{ex:jmAlu}
\gll tɕe kɯju jmɤlu nɯnɯ tɯrme ɲɯ-ŋu, ɯ-jme kɯ-me nɯ tɕe, tɕe kɯju jmɤlu nɯnɯ ɲɯ-sɲu ɕti tɕe nɯ  nɯ-sɲu tɕe tɕe iɕqʰa tɯ-rɣi cʰɯ-kɯ-χtɤr nɯ nɯ-kɯ-sɲu tu-sɤrmi-nɯ. \\
\textsc{lnk} animal tailless \textsc{dem} man \textsc{sens}-be \textsc{3sg.poss}-tail \textsc{nmlz}:S/A-not.exist \textsc{dem} \textsc{lnk} \textsc{lnk} animal tailless  \textsc{dem} \textsc{ipfv}-be.crazy  be:\textsc{affirm}:\textsc{fact} \textsc{lnk} \textsc{dem} \textsc{pfv}-be.crazy \textsc{lnk} \textsc{lnk} the.aforementioned \textsc{indef.poss}-seed \textsc{ipfv}-\textsc{nmlz}:S/A-spread \textsc{dem} \textsc{pfv}-\textsc{nmlz}:S/A-be.crazy \textsc{ipfv}-call-\textsc{pl} \\
\glt `The `tailless animal' is the man, and `he becomes crazy', (when the crow say) that (people) became crazy, it means that they are sowing seeds.' (22-qajdo, 47-9)
\end{exe}

\subsection{Diminutive} \label{sec:diminutive}
There are four diminutive formations in Japhug, with the quasi-suffixes \forme{-pɯ}, \forme{-tsa}, \forme{-tɕɯ} and \forme{-li}.

The most productive is the \forme{-pɯ} suffixation. This transparent suffix comes from the noun \japhug{tɤ-pɯ}{offspring, young} (from Tibetan \tibet{བུ་}{bu}{son}). A diminutive formation based on the same noun also exists in Tibetan (\citealt{uray52diminutive},  \citealt[627]{hill14derivational}); whether the diminutive formation was independently innovated, or was borrowed from Tibetan is a question that deserves further investigation. It is also attested in Situ (\citealt{zhang16bragdbar}, \citealt[151]{lai17khroskyabs}).

Earlier diminutives are formed with the\textit{ status constructus} of the noun, for instance \japhug{tɕʰemɤpɯ}{young girl} from \japhug{tɕʰeme}{girl}, \japhug{staχpɯ}{pea} from \japhug{stoʁ}{broad bean}, or \japhug{kʰɯzɤpɯ}{puppy} from a non-attested form \forme{*kʰɯza}, propably itself the \forme{-tsa} diminutive of \japhug{kʰɯna}{dog}, borrowed from a Situ dialect.

More recent diminutives are directly formed with the base form, such as \japhug{qapripɯ}{little serpent}. This formation is extremely productive, and applies to plants, animals and even objects as in (\ref{ex:srWnloR}).

\begin{exe}
\ex \label{ex:srWnloR}
\gll tɕe srɯnloʁ-pɯ ci ɲɤ-kʰo tɕe \\
lnk ring-\textsc{dim} \textsc{indef} \textsc{ifr}-give \textsc{lnk} \\
\glt `He handed him a little ring.' (2011-4-smanmi, 120)
\end{exe}

The suffix \forme{-pɯ} is recursive, and examples of doubly suffixed nouns are found in the corpus, as in (\ref{ex:lhAndzxipWpW}) for instance. 

\begin{exe}
\ex \label{ex:lhAndzxipWpW}
\gll  ɬɤndʐi-pɯ-pɯ nɯra kɯ, ɯ-pʰoŋbu nɯra ko-sɤlɤɣɯ-nɯ ri, \\
demon-\textsc{dim}-\textsc{dim} \textsc{dem}:\textsc{pl} \textsc{erg} \textsc{3sg}.\textsc{poss}-body \textsc{dem}:\textsc{pl} \textsc{ifr}-link-\textsc{pl} \textsc{lnk} \\
\glt `The little little demons put back his body together, but...' (150909 xifangping-zh, 93)
\end{exe}

Suffixation with \forme{-pɯ} is the fused variant of the property noun construction with \japhug{ɯ-pɯ}{little one} described in section \ref{sec:property.nouns}.

A diminutive that is common to all Gyalrongic languages is the suffix \forme{-tsa}/\forme{-za} (Situ \forme{-tsa} or \forme{-za} (\citealt[163]{linxr93jiarongen}), Khroskyabs \forme{-ze} / \forme{-zə} / \forme{-zɑ}, \forme{-tsi} (\citealt[158]{lai17khroskyabs}), Stau \forme{-zə}), found in fossilized forms in nouns such as \japhug{kʰɯtsa}{bowl} and \japhug{βɣɤza}{fly},\footnote{The noun \japhug{βɣɤza}{fly} is cognate to Brag-dbar \forme{kəvɐ̂s}, Khroskyabs \forme{jvɑzɑ́} (\citealt{zhang16bragdbar}, \citealt[156]{lai17khroskyabs}) and originates from proto-Gyalrong \forme{*kpɔs-tsa} (\citealt[53]{jacques08zh}). } but still visible in diminutive forms like \japhug{paʁtsa}{piglet} (from \japhug{paʁ}{pig}). It originates from the noun `son' that is lost in Japhug but still attested in Situ and Khroskyabs (Wobzi \forme{zî} `young man'). 

In Japhug the \forme{-tsa} diminutive is not very productive; it applies to some nouns that already have a \forme{-pɯ} diminutive such as \japhug{stoʁtsa}{name of plant} from \japhug{stoʁ}{broad bean} (besides \japhug{staχpɯ}{pea}).

The third diminutive suffix \forme{-tɕɯ}, like the two preceding ones, originates from a noun meaning `offspring', \japhug{tɤ-tɕɯ}{son}, and requires \textit{status constructus}.

It is  used for animals (\japhug{kumpɣɤtɕɯ}{sparrow} from \japhug{kumpɣa}{fowl}) or inanimate objects (\japhug{kʰɤtɕɯ}{little house} from \japhug{kʰa}{house} or \japhug{lʁɤtɕɯ}{little gunny bag} from \japhug{lʁa}{gunny bag}). It occurs in some lexicalized forms such as \japhug{mbrɯtɕɯ}{knife}.\footnote{The root of this noun is metathesized from \forme{*mbɯr}; its cognates have a \forme{-tsa} diminutive in Situ (Brag-dbar \forme{mbərtsiɛ̄}, \citealt[228]{zhang16bragdbar}) and Khroskyabs (Wobzi \forme{(bərzé}, \citealt[115]{lai17khroskyabs}).}

The suffix \forme{-li} is the least productive of all diminutive formations, and the only that cannot be traced to an existing noun. It appears is \japhug{tɕʰemɤli}{little girl} (a synonym of \japhug{tɕʰemɤpɯ}{little girl}) and in \japhug{rgali}{young cow}.

\subsection{Augmentative} \label{sec:augmentative}
A handful of nouns, some of Tibetan origin, have an augmentation form in \forme{-te}, originally from a property noun \forme{*ɯ-te} `big' (related to the verb \japhug{wxti}{be big}).

Augmentatives include \japhug{tɕɣomte}{cultivated xanthoxylum} (from \japhug{tɕɣom}{xanthoxylum}), \japhug{tɯjite}{big field} (from \japhug{tɯ-ji}{field}, name of several fields in Kamnyu), \japhug{tɕʰɯte}{big river} (from \tibet{ཆུ་}{tɕʰu}{water, river}) and the \textit{bahuvrīhi} \japhug{ŋgute}{person with a big head} (from Tibetan \tibet{འགོ་}{ⁿgo}{head, top}, not attested independently).

\subsection{Derogative} \label{sec:derogative}
There are three derogative quasi-suffixes in Japhug, deriving designations of old or broken things: \forme{-do} and \forme{-mbe} `old X' and \forme{-ɴqra} `broken X'. These suffixes are the fused variants of the property nouns \japhug{ɯ-ɴqra}{broken one}, \japhug{ɯ-do}{old one} and \japhug{tɤ-mbe}{old thing}  (see \ref{sec:property.nouns}). 

The suffixes \forme{-do} and \forme{-mbe}, like their corresponding property nouns, differ in that the former occurs with animals and plants (\japhug{nɯŋa-do}{old cow}, \japhug{rɟɤlpu-do}{old king}), while the latter is used for inanimate objects.

 In a few cases, the suffixed noun is in status constructus (as \japhug{kʰɤɴqra}{ruin} from \japhug{kʰa}{house} and \forme{-ɴqra}, or \japhug{kʰɯdo}{old dog} (from \japhug{kʰɯna}{dog} and \forme{-do}, see \ref{sec:reduced.forms.compounds}). When the suffixed noun is an IPN, addition of a derogative suffix does not turn it into a APN, as in \japhug{tɯ-rcɤmbe}{old jacket} from \japhug{tɯ-rcu}{jacket} and \forme{-mbe} (unlike other types of compounds, § XXX).

\subsection{Inhabitant} \label{ex:inhabitant.pW}
The inhabitant suffix \forme{-pɯ} derives from the same noun \japhug{tɤ-pɯ}{offspring, young} from which the diminutive \forme{-pɯ} ultimately originates (see \ref{sec:diminutive}). It is used to derive nouns referring to inhabitants of a certain place, and occurs without \textit{status constructus}. For instance, from the village names of \forme{kɤmɲɯ} (the village whose speech is described in this grammar) and \forme{snarndi} (a village in Tshobdun), one derives \japhug{kɤmɲɯpɯ}{person from Kamnyu} and \japhug{snarndipɯ}{person from Snarndi} (see the text 26-tshubdWnpW in the corpus). Given the high productivity of this derivation, these nouns are not indicated in the dictionary, as it would unnecessarily inflate the number of entries.

Note that alternatively, adding the plural \forme{ra}  to a placename suffices to refer to the inhabitants of that place (§ \ref{sec:place.names}).

\subsection{Gender} \label{sec:gender}
There is no morphological expression of gender in Japhug. For animals, the nouns \japhug{pʰu}{male} and \japhug{mu}{female} (from Tibetan \tibet{ཕོ་}{pʰo}{male} and \tibet{མོ་}{mo}{female}) can be used on their own (as in \ref{ex:phu.mu}) or occur as second member of compounds, as \japhug{kumpɣapʰu}{rooster} and \japhug{kumpɣamu}{hen} from \japhug{kumpɣa}{fowl}, or \japhug{lɯlɤmu}{female cat} from \japhug{lɯlu}{cat}, with \textit{status constructus} of the first noun.

\begin{exe}
\ex \label{ex:phu.mu}
\gll tɤkʰe pɣɤtɕɯ ndɤre pʰu mu saχsɤl \\
stupid bird:\textsc{dim} on.the.other.hand male female be.clear:\textsc{fact} \\
\glt `The male and the female of the `stupid bird', as opposed (to the birds previously discussed), are easy to distinguish.' (23-scuz, 45)
\end{exe}

The suffixes \forme{-pa} and \forme{-mɯ} (from Tibetan \forme{-pa} and \forme{-mo} respectively) also occur for a handful of nouns, some of Tibetan origin (\japhug{srɯnmɯ}{râkshasî} from \tibet{སྲིན་མོ་}{srin.mo}{râkshasî}) but also some local names such as \japhug{ɴɢarpa}{male one quarter yak hybrid}  vs \japhug{ɴɢarmɯ}{female one quarter yak hybrid}.

The noun \japhug{paʁɟu}{boar} from \japhug{paʁ}{pig} has a suffix \forme{-ɟu} that is not found in any other word.

For some domestic animals, a lexical distinction is made between male and female animals (see Table \ref{tab:lexical.gender}).

\begin{table}
\caption{Lexical distinction of male and female animals} \label{tab:lexical.gender}
\begin{tabular}{l|l}
 \lsptoprule 
 Male & Female \\
 \midrule
\japhug{qambrɯ}{male yak} & \japhug{qra}{female yak} \\
\japhug{jla}{male hybrid yak} & \japhug{ftsoʁ}{female hybrid yak} \\
\japhug{mbala}{bull} & \japhug{nɯŋa}{cow}  \\
\japhug{zrɤβ}{he-goat} & (\japhug{tsʰɤnmu}{ewe})  \\
 \lspbottomrule
\end{tabular}
\end{table}

\subsection{Collective} \label{sec:collective}
While Japhug lacks number inflection, there are several four collective derivations: the social relation collective, four reduplicated collectives and the \textit{dvandva} collective.

\subsubsection{Social relation collective}  \label{sec:social.collective}
The first type of collective is a noun prefixed in \forme{kɤndʑi-} and built either from kinship or social relation terms (which can be either IPNs or APNs), designating a group of people linked to one another by a specific relation.\footnote{In previous publications, the transcription \forme{kɤndʑɯ-} has been used; see § \ref{sec:W.i.compounds} on the question of the contrast between \ipa{i} and \ipa{ɯ} following palatals and alveolo-palatals in non-final syllables.}

Two distinct types of social relation collectives should be distinguished: reciprocal and non-reciprocal collectives.

Reciprocal collectives (Table \ref{tab:reciprocal.collectives}) are from nouns designating an relationship in which all members of the group call each other by the same term; it can be non-kinship terms like `companion' or `friend' or kinship terms like \japhug{tɤ-sqʰaj}{sister (of a girl)} (on the use of this term see § XXX). 

\begin{table}
\caption{Reciprocal social relation collectives} \label{tab:reciprocal.collectives}
\begin{tabular}{lllllll}
 \lsptoprule 
 Collective & Base noun \\
\midrule
\japhug{kɤndʑiɣɯfsu}{friends} & \japhug{ɣɯfsu}{friend} \\
\japhug{kɤndʑiβzaŋsa}{friends} & \japhug{βzaŋsa}{friend} \\
\japhug{kɤndʑiɕaχpu}{friends} & \japhug{ɕaχpu}{friend} \\
\japhug{kɤndʑikɯmdza}{relatives} & \japhug{kɯmdza}{relative} \\
\japhug{kɤndʑirɣa}{neighbours} & \japhug{tɤ-rɣa}{neighbour} \\
\japhug{kɤndʑislamaχti}{classmates} & \japhug{slamaχti}{classmate} \\
\japhug{kɤndʑisqʰaj}{sisters} & \japhug{tɤ-sqʰaj}{sister (of a girl)} \\
\japhug{kɤndʑimɤtsa}{mother's sister's children} & \japhug{tɤ-mɤtsa}{mother's sister's child} \\
\japhug{kɤndʑitɤtɕɯχti}{friends (between boys)} & \japhug{tɤtɕɯχti}{friend (between boys)} \\
\japhug{kɤndʑitɕʰemɤχti}{friends (between girls)} & \japhug{tɕʰemɤχti}{friend (between girls)} \\
\japhug{kɤndʑixtɤɣ}{brothers} & \japhug{tɤ-xtɤɣ}{brother (of a boy)} \\
\japhug{kɤndʑiχti}{companions} & \japhug{tɯ-χti}{companion} \\
\japhug{kɤndʑizda}{companions} & \japhug{tɯ-zda}{companion} \\
 \lspbottomrule
\end{tabular}
\end{table}

Non-reciprocal collectives (Table \ref{tab:non.reciprocal.collectives}) are based on nouns designating unequal relationships, in which the members designate each other by different terms, in particular kinship terms involving relatives from different generations or different gender.  Aside from kinship terms, groups comprising domestic animals and their owners can also be formed by the same process from the name of the animal, as \japhug{kɤndʑimbro}{horseman and his horse} and \japhug{kɤndʑiftsoʁ}{female hybrid yak  and its owners} (\ref{ex:kAndZWftsWftsoR} below).

Non-reciprocal collectives are either formed from one of the two nouns, which can be either the one from the lower (\japhug{kɤndʑiɣe}{grandparents and grandchildren} ) or the higher generation (\japhug{kɤndʑiɲi}{paternal aunt and her nephews}), or by a combination of two kinship terms, the first of which undergoes in some cases changes to the point of being barely recognizable (\japhug{kɤndʑipɤmdɯ}{paternal uncle and his nephews}).\footnote{In the case of \japhug{kɤndʑiwɤɬaʁ}{maternal aunt and her nephews}, the origin of the element \forme{-wɤ-} is not identifiable.}

\begin{table}
\caption{Non-reciprocal social relation collectives} \label{tab:non.reciprocal.collectives}
\begin{tabular}{lllllll}
 \lsptoprule 
  Collective & Base noun \\
\midrule
\japhug{kɤndʑiɣe}{grandparents and grandchildren} & \japhug{tɤ-ɣe}{grandchild} \\
\japhug{kɤndʑiʁi}{siblings} & \japhug{ta-ʁi}{younger sibling} \\
\japhug{kɤndʑime}{parents and daughter} & \japhug{ɯ-me}{daughter} \\
\japhug{kɤndʑiɲi}{paternal aunt and her nephews} & \japhug{tɤ-ɲi}{father's sister} \\
\midrule
\japhug{kɤndʑimbro}{horseman and his horse} & \japhug{mbro}{horse} \\
\japhug{kɤndʑijla}{male hybrid yak and its owners} & \japhug{jla}{male hybrid yak} \\
\japhug{kɤndʑiftsoʁ}{female hybrid yak  and its owners} & \japhug{ftsoʁ}{female hybrid yak}  \\
\japhug{kɤndʑipaʁ}{pig and its owners} & \japhug{paʁ}{pig} \\
\japhug{kɤndʑiqaʑo}{sheep and its owners} & \japhug{qaʑo}{sheep} \\
\japhug{kɤndʑitsʰɤt}{goat and its owners} & \japhug{tsʰɤt}{goat} \\
\midrule
\japhug{kɤndʑirpɯftsa}{maternal uncle and his nephews} & \japhug{tɤ-rpɯ}{mother's uncle} \\
& \japhug{tɤ-ftsa}{sister's son} \\
\japhug{kɤndʑiwɤɬaʁ}{maternal aunt and her nephews} & \japhug{tɤ-ɬaʁ}{mother's sister} \\
\japhug{kɤndʑipɤmdɯ}{paternal uncle and his nephews} & \japhug{tɤ-mdɯ}{brother's child} \\
& \japhug{tɤ-βɣo}{father's brother} \\
\japhug{kɤndʑiwɤmɯsnom}{brother and sisters} & \japhug{tɤ-wɤmɯ}{brother (of a girl)} \\
& \japhug{tɤ-snom}{sister (of a boy)} \\
 \lspbottomrule
\end{tabular}
\end{table}

The collective nouns can be used as normal nouns and take case marking, numerals and other modifiers, as in   (\ref{ex:kAndZWxtAG.XsWm}).

\begin{exe}
\ex \label{ex:kAndZWxtAG.XsWm}
\gll  kɤndʑi-xtɤɣ χsɯm pjɤ-tu-nɯ \\
\textsc{coll}-brother three \textsc{ifr}.\textsc{ipfv}-exist-\textsc{pl} \\
\glt `There were three brothers.' (07-deluge, 1)
\end{exe}

Social relationship collectives are also found in Situ and Tshobdun (\citealt[107]{jackson98morphology}), where they have optional reduplication; in Japhug, reduplication is used by some speakers, as \japhug{kɤndʑiftsɯftsoʁ}{female hybrid yak  and its owners} in example (\ref{ex:kAndZWftsWftsoR}), from a story by Kunbzang Mtsho.

\begin{exe}
\ex \label{ex:kAndZWftsWftsoR}
 \gll kɤndʑi-ftsɯ\redp{}ftsoʁ χsɯm nɯ, tsʰɯntsʰɯn kɯ-pa kɤ-nɯ-ɬoʁ-nɯ ɲɯ-ŋu, \\
 \textsc{coll}-female.yak.hybrid three \textsc{dem} \textsc{idph}:II:in.order \textsc{inf}:\textsc{stat}-\textsc{aux} \textsc{pfv}:\textsc{east}-\textsc{auto}-come.out-\textsc{pl} \textsc{sens}-be \\
\glt `(The girl, her husband) and their female hybrid yak crossed (the large river) without damage.' (kunbzang2003, 186)
\end{exe}

The lists in Tables \ref{tab:reciprocal.collectives} and \ref{tab:non.reciprocal.collectives} comprise all most common social relation collectives, but is by no means a complete list. For instance, next to \japhug{kɤndʑirpɯftsa}{maternal uncle and his nephews} from \japhug{tɤ-rpɯ}{mother's uncle} and \japhug{tɤ-ftsa}{sister's son}, the terms \japhug{kɤndʑirpɯ}{maternal uncle and his nephews} and \japhug{kɤndʑiftsa}{nephew with his maternal uncles and aunts} are also possible though less common. However, some combinations are considered incorrect. For instance, Tshendzin considers that $\dagger$\forme{kɤndʑirʑaβ} (from \japhug{tɤ-rʑaβ}{wife}) is only found in children's language (\forme{nɯ tɤ-pɤtso ra kɯ tu-ti-nɯ ŋgrɤl} `children talk like that'), as the correct term  is \japhug{ʁzɤmi}{husband and wife} from Tibetan \tibet{བཟའ་མི་}{bza.mi}{husband and wife}.

There is in addition an irregular collective \japhug{kɤtsa}{parents and children}, with the same element \forme{-tsa} found in some diminutives (see \ref{sec:diminutive}), from an earlier word for `child'.

It can be used without any preceding noun as in (\ref{ex:kAtsa.ra}), but more commonly with a preceding noun, as in (\ref{ex:tAmu.kAtsa}) (note also \forme{tɤ-tɕɯ kɤtsa} `father and son' and \forme{tɕʰeme kɤtsa} `mother and daughter' from \japhug{tɤ-tɕɯ}{son, boy} and \japhug{tɕʰeme}{girl}).

\begin{exe}
\ex \label{ex:kAtsa.ra}
\gll tɕe tɤ-mu nɯ kɯ ɯ-pɯ nɯnɯ ju-ɕpʰɣɤm tɕe, ʑara kɤtsa ra stɯsti ʁɟa ʑo ɕe-nɯ ɲɯ-ra. \\
\textsc{lnk} \textsc{indef}.\textsc{poss}-mother \textsc{dem} \textsc{erg} \textsc{3sg}.\textsc{poss}-young dem \textsc{ipfv}-flee.with[III] \textsc{lnk} \textsc{3pl} parents.and.children \textsc{pl} alone completely \textsc{emph} go:\textsc{fact}-\textsc{pl} \textsc{sens}-have.to \\
\glt `And the mother (lioness) flees with her youngs, and they (mother and children) have to go alone (without the father).' (20-sWNgi, 75)
\end{exe}

\begin{exe}
\ex \label{ex:tAmu.kAtsa}
\gll
tɤ-mu kɤtsa ci pjɤ-tu-ndʑi tɕe \\
\textsc{indef}.\textsc{poss}-mother  parents.and.children \textsc{indef} \textsc{ifr}.\textsc{ipfv}-exist-\textsc{du} \textsc{lnk} \\
\glt `There was a mother and her son.' (2003tamukatsa, 1)
\end{exe}

Like comitative adverbs (§ \ref{sec:comitative.adverb}), it is clear that social relation collectives originate from participles of denominal verbs. The only example of the verbal denominal \forme{andʑɯ-} derivation from which they originate is \japhug{andʑɯrɣa}{be together as neighbours} from the IPN \japhug{tɤ-rga}{neighbourg}, as in (\ref{ex:YAndZWrGandZi}) (see § XXX on the denominal derivation). The social relation collective \japhug{kɤndʑirɣa}{neighbours} can thus be analyzed as the participle of this verb \forme{kɯ-ɤndʑirɣa}.  

\begin{exe}
\ex \label{ex:YAndZWrGandZi}
\gll ɲɯ-ɤndʑirɣa-ndʑi \\
\textsc{sens}-be.neighbours-\textsc{du} \\
\glt `They are one next to the other.' (elicitation)
\end{exe}

The \forme{andʑi-} denominal derivation being however restricted to this single example, from a synchronic point of view this collective formation is a strictly nominal derivation.

\subsubsection{Reduplicated collectives} \label{sec:redp.coll}
The reduplicated collectives are built using partial reduplication, and three distinct patterns exist.


First, some nouns allow standard partial reduplication with \forme{ɯ} in the reduplicated syllable (§ XXX) expressing a vague collective. This reduplication can apply to loanwords from Tibetan, such as \japhug{χsɯ\redp{}χsɤr}{things in gold} and \japhug{rŋɯ\redp{}rŋɯl}{things in silver} from \japhug{χsɤr}{gold} and \japhug{rŋɯl}{silver} (Tibetan \tibet{གསེར་}{gser}{gold} and \tibet{དངུལ་}{dŋul}{silver}).

\begin{exe}
\ex 
\gll a-χsɯ\redp{}χsɤr ra, a-rŋɯ\redp{}rŋɯl ra mɤ-ra kɯ ɕom rɟɤskɤt ɯ-taʁ tu-ɕe-a ŋu \\
\textsc{1sg}.\textsc{poss}-\textsc{coll}\redp{}gold \textsc{pl} \textsc{1sg}.\textsc{poss}-\textsc{coll}\redp{}silver \textsc{pl} \textsc{neg}-have.to:\textsc{fact} \textsc{erg} iron  stairs \textsc{3sg}.\textsc{poss}-on \textsc{ipfv}:\textsc{up}-go-\textsc{1sg} be:\textsc{fact} \\
\glt `I don't need things in gold or silver, I will go up the iron stairs.' (not the golden or silver stairs, 2005-Kunbzang, 215)
\end{exe}

Some nouns only appearing in reduplicated form are presumably ancient collectives, like \japhug{kʰrambaχtɯχtɤm}{lies} from a possible non-reduplicated form *\forme{kʰrambaχtɤm} (from  \tibet{ཁྲམ་པ་གཏམ་}{kʰram.pa.gtam}{deceiving words}).

%tɕendɤre nɯnɯ ʑɯ-ʑɯmkhɤm ʑo nɯ, nɯ-rɟɤlpu ɣɯ ɯ-me ci ʑo staʁlu pjɤ-tu nɯ ma pjɤ-k-ɤrɕo-ci, 
%nyimaowdzer2002, 86

Second, we find the reduplicated collectives with the vowel \ipa{a}, not \ipa{ɯ}, in the replicated syllable. Examples are few, as shown in Table \ref{tab:coll.n}, but several of them are borrowings from Tibetan. In one case, \japhug{fɕafɕɤt}{words}, the base word is a transitive verb (\japhug{fɕɤt}{tell}).

\begin{table}
\caption{Collective noun derivation} \label{tab:coll.n}
\begin{tabular}{l|lll}
 \lsptoprule 
 Base form & Collective & Tibetan \\
 \midrule
\japhug{rdɯl}{dust, dirt} & \japhug{rdardɯl}{dust, dirt} & \tibet{རྡུལ་}{rdul}{dust} \\
\japhug{tɯ-ntɕʰɯr}{fragment}  & \japhug{ɯ-ntɕʰantɕʰɯr}{fragments} & \\
\japhug{ɯ-zɯr}{side}  & \japhug{ɯ-zarzɯr}{sides} & \tibet{ཟུར་}{zur}{side, corner} \\
\japhug{ɯ-rkɯ}{side} & \japhug{ɯ-rkarkɯ}{sides} & \\
\japhug{fɕɤt}{tell}  & \japhug{fɕafɕɤt}{words} &  \tibet{བཤད་}{bɕad}{explain, tell} \\
 \lspbottomrule
\end{tabular}
\end{table}

Reduplicated collective nouns in \forme{a-} can be used without number clitic, as in (\ref{ex:WntChantChWr}), but they often appear with the \japhug{ra}{plural} as in (\ref{ex:rdardWl}).

\begin{exe}
\ex \label{ex:WntChantChWr}
\gll znɤrɣama nɯ mtʰa ɯ-kɤcu ŋu. tɕe nɯnɯtɕu tɯ-ji ɯ-ntɕhantɕhɯr pɯ-dɤn, jinde kʰro ɲɤ-s-qapɯ-nɯ,\\
p.n. \textsc{dem} p.n. \textsc{3sg.poss}-east be:\textsc{fact} \textsc{lnk} \textsc{dem:pl} \textsc{indef}.\textsc{poss}-field \textsc{3sg.poss}-fragment:\textsc{coll} \textsc{pst}.\textsc{ipfv}-be:many now much \textsc{ifr}-\textsc{caus}-be.fallow-\textsc{pl}\\
\glt `Znargama ('The place where one calls the rain') is on the east of Mtha, there used to be many little fragments of fields, but now people have left them become fallow.' (150903 kAmYW tWji3, 19)
\end{exe}

\begin{exe}
\ex \label{ex:rdardWl}
\gll tɕe tɤɕi nɯ tú-wɣ-χtɕi tɕʰɣaʁtɕʰɣaʁ ʑo tɕe, rdardɯl nɯra ɲɯ́-wɣ-ɣɤ-me tɕe \\
\textsc{lnk} barley \textsc{dem} \textsc{ipfv}-\textsc{inv}-wash \textsc{idph}:II:completely.clean \textsc{emph} \textsc{lnk} dush:\textsc{coll} \textsc{dem:pl} \textsc{ipfv}-\textsc{inv}-\textsc{caus}-not.exist \textsc{lnk} \\
\glt `Then one washes the barley very thoroughly, one removes all the dirt.' (2002tWsqar, 118)
\end{exe}
 
The noun \japhug{rgargɯn}{old person} has the form of a collective noun as those of Table \ref{tab:coll.n}, but it is commonly used with singular or dual referents (as in \ref{ex:rgargWn}). It could be originally the collective form of a borrowing from Tibetan \tibet{རྒན་པོ་}{rgan.po}{old person}, though the expected form would have been $\dagger$\forme{rga-rgɤn}. 
 
\begin{exe}
\ex \label{ex:rgargWn}
\gll  rgargɯn ni kɤ-fstɯn pɯ-ra \\
old.person \textsc{du} \textsc{inf}-serve \textsc{pst.ipfv}-have.to \\
\glt `She had to take care of two old people.' (14-tApitaRi, 34)
\end{exe}

A third reduplicated collective derivation is only attested by one example, the form \japhug{qajɯqaja}{all kinds of worms} (see \ref{ex:qajWqaja}) which derives from \japhug{qajɯ}{worm}  by reduplicating the whole word and changing the last rime to \ipa{-a}, a reduplication template reminiscent of that found in Khroskyabs (see \citealt{lai13fuyin}, \citealt[22-24]{lai17khroskyabs}).

\begin{exe}
\ex \label{ex:qajWqaja}
\gll
tɯ-ci ɯ-ŋgɯ qajɯqaja tʰamtɕɤt, sɯŋgɯ ɣɯ ɯ-rɯdaʁ kɯ-xtɕi kɯ-wxti, mɤʑɯ pɣa nɯnɯra lonba ʑo kɤ-fsraŋ kɯ-ra ɲɯ-ɕti ma \\
\textsc{indef.poss}-water \textsc{3sg}-inside worm:\textsc{coll} all forest \textsc{gen} animal \textsc{nmlz}:S/A-be.small \textsc{nmlz}:S/A-be.big yet bird \textsc{dem:pl} all \textsc{emph} \textsc{inf}-protect \textsc{inf:stat}-have.to \textsc{sens}-be:\textsc{affirm} \textsc{lnk} \\
\glt `All the creatures in the water, the small and big animals of the forest, and also the birds have to be protected.' (160703 jingyu, 43)
\end{exe}

A fourth type of collective has the vowel \forme{-e} in the reduplicated syllable. It is attested by the noun \japhug{ɯ-ʁɟoʁɟe}{all kinds of diluted drinks} from \japhug{ɯ-ʁɟo}{diluted drink} (derived from the verb \japhug{ʁɟo}{rinse}).

\subsubsection{Dvandva collective} \label{sec:dvandva.coll}
The \textit{dvandva} collective is derived from two nouns, the first one in \textit{status constructus} form followed by the element \forme{-lɤ-}  and then by the second noun stem without possessive prefix. Known forms are listed in Table \ref{tab:dvandva.coll.n}.  

 \begin{table}
\caption{Dvandva collectives} \label{tab:dvandva.coll.n}
\begin{tabular}{l|lll}
 \lsptoprule 
Collective & First noun & Second Noun \\
 \midrule
 \japhug{tɯ-kɤlɤmɲaʁ}{facial features} & \japhug{tɯ-ku}{head} & \japhug{tɯ-mɲaʁ}{eye} \\
\japhug{tɯ-mɤlɤjaʁ}{the four limbs} & \japhug{tɯ-mi}{leg, foot} & \japhug{tɯ-jaʁ}{arm, hand} \\
 \japhug{ɯ-kɤlɤjme}{head upside down} & \japhug{tɯ-ku}{head} & \japhug{tɤ-jme}{tail} \\
  \japhug{kɯmɤlɤxso}{in vain} & \japhug{kɯ-me}{not existing} & \japhug{ɯ-xso}{empty, normal} \\
 \lspbottomrule
\end{tabular}
\end{table}

Among the examples in Table \ref{tab:dvandva.coll.n},  \japhug{ɯ-kɤlɤjme}{head upside down}  and   \japhug{kɯmɤlɤxso}{in vain} are mainly used adverbially. The first one is mainly used with verbs such as \japhug{ɕtʰɯz}{turn towards} and \japhug{ru}{look} at, as in (\ref{ex:WkAlAjme}).

\begin{exe}
\ex \label{ex:WkAlAjme}
 \gll tɕe nɯ ɯ-sta nɯ lɤtɕʰom nɯ ɲɯ́-wɣ-ʁɟo ʑo kʰrɯŋkʰrɯŋ ʑo qʰe tɕe ɯ-kɤlɤjme pjɯ́-wɣ-ɕtʰɯz qʰe, ɯ-mŋu nɯ pa pjɯ́-wɣ-ɕtʰɯz \\
 \textsc{lnk} \textsc{dem} \textsc{3sg.poss}-place \textsc{dem} churning.bucket \textsc{dem} \textsc{ipfv}-\textsc{inv}-rinse \textsc{emph}  \textsc{idpf}:II:completely.clean \textsc{emph} \textsc{lnk} \textsc{lnk} \textsc{3sg.poss}-head.upside.down \textsc{ipfv}:\textsc{down}-\textsc{inv}-turn.towards \textsc{lnk} \textsc{3sg.poss}-opening \textsc{dem} down   \textsc{ipfv}:\textsc{down}-\textsc{inv}-turn.towards \\
 \glt `One rinses the churning bucket very clean, and put it upside down at its place, the opening down.'(30-macha, 66)
\end{exe}

The adverb \japhug{kɯmɤlɤxso}{in vain} combines the subject participle of \japhug{me}{not exist} with the property noun \japhug{ɯ-xso}{empty, normal} (a lexicalized participle, whose uses and etymology are described in \ref{sec:property.nouns}). It is originally probably a noun used adverbially (§ XXX).


The  \forme{-lɤ-/-la-} element found in collective \textit{dvandva}-s is also attested in approximate numerals (see § XXX) and in adverbs such \japhug{tɯxpalɤskɤr}{during the whole year} (from \japhug{tɯ-xpa}{one year} and \japhug{fskɤr}{turn around}) and  \japhug{rtsɯɕaŋlaŋmtɕɤt}{all the plants} (from \japhug{rtsɯɕaŋ}{plant} and \japhug{tʰamtɕɤt}{all}, respectively from Tibetan \tibet{རྩི་ཤིང༌}{rtsi.ɕiŋ}{plant} and \tibet{ཐམས་ཅད་}{tʰams.tɕad}{all}).
 
 
The IPN \japhug{ɯ-ŋgɯmɤpɕi}{the inside and the outside} from the locative relator nouns \japhug{ɯ-ŋgɯ}{inside} and \japhug{ɯ-pɕi} (the latter from Tibetan \tibet{ཕྱི་}{pʰʲi}{outside}) has a related structure, but with the linking morpheme \forme{-mɤ-} instead of \forme{-lɤ-}.

\subsection{Superlative} \label{sec:superlative.XCWX}
While there is no adjectival superlative derivation in Japhug (superlative constructions are synthetic, see § XXX), we find nevertheless a derivation applied to locative nouns (§ \ref{sec:relator.location}), expressing the most extreme location. As shown in Table \ref{tab:superlative.n}, it is built by adding an element \forme{-ɕɯ-} followed by a complete copy of the root of the noun without \textit{status constructus} alternation or partial replication; the resulting noun is still an inalienably possessed locative noun. Example (\ref{ex:WqaCWqa}) illustrates the use of one of these forms.

\begin{table}
\caption{Superlative noun derivation} \label{tab:superlative.n}
\begin{tabular}{l|lll}
 \lsptoprule
\japhug{tɯ-ku}{head, top} & \japhug{ɯ-kuɕɯku}{the highest place} \\
\japhug{tɯ-qa}{root, paw, bottom} & \japhug{ɯ-qaɕɯqa}{the deepest place} \\
\japhug{ɯ-rkɯ}{side} & \japhug{ɯ-rkɯɕɯrkɯ}{the place most on the side} \\
\japhug{ɯ-zɯr}{side} & \japhug{ɯ-zɯrɕɯzɯr}{the place most on the side} \\
 \lspbottomrule
\end{tabular}
\end{table}

\begin{exe}
\ex \label{ex:WqaCWqa}
\gll rɟɤmtsʰu ɯ-qaɕɯqa pjɯ-ɕe tɕe, nɯnɯ ɯ-kɤ-nɤ-mɯm nɯra ɕ-tu-nɯ-tɕɤt ɲɯ-ŋu. \\
ocean \textsc{3sg.poss}-bottom:\textsc{super} \textsc{ipfv}:\textsc{down}-go  \textsc{lnk} \textsc{dem} \textsc{3sg.poss}-\textsc{nmlz}:P-\textsc{trop}-be.tasty \textsc{dem:pl} \textsc{transloc-ipfv}-\textsc{auto}-take.out \textsc{sens}-be \\
\glt `(The sperm whale) goes to the lowest depths of the ocean and catches the things it likes to eat.' (160703 jingyu, 24)
\end{exe}

\section{Denominal adverbs}

\subsection{Comitative adverbs} \label{sec:comitative.adverb}
The highly productive comitative adverb formation derives from nouns adverbs meaning `having X' , `together with X', `including X' or in the case of clothes or covers `wearing X'.

Comitative adverbs are built by partially reduplicating the last syllable of the noun stem (following the morphophonological rules in XXX) and prefixing either \forme{kɤ́-} or \forme{kɤɣɯ-}. This derivation applies to native words and loanwords from Tibetan. From instance, \japhug{χɕɤlmɯɣ}{glasses} (from Tibetan \tibet{ཤེལ་མིག་}{ɕel.mig}{glasses}) yields \forme{kɤ́-χɕɤlmɯ\tld{}lmɯɣ} or \forme{kɤɣɯ-χɕɤlmɯ\tld{}lmɯɣ} `together with glasses'.\footnote{Note that reduplication disregards morpheme boundaries, as the coda of \japhug{χɕɤl}{glass} (from \tibet{ཤེལ་}{ɕel}{glass}) is reduplicated with the following syllable. } 

No semantic difference between the comitative adverbs in \forme{kɤ́-} and those in \forme{kɤɣɯ-} has been detected; both are fully productive and can be built from the same nouns. As argued in \citet{jacques17comitative}, the \forme{kɤɣɯ-} form is inherited (from proto-Gyalrong \forme{*kɐwə-}), while \forme{kɤ́-} is borrowed from Tshobdun \forme{ko-}, the exact cognate of \forme{kɤɣɯ-}  (\citealt[107]{jackson98morphology}). The prefix \forme{kɤɣɯ-} and its Tshobdun cognate \forme{ko-} both originate from the S-participle \forme{kɯ-} (see § XXX) of the proprietive \forme{aɣɯ-} denominal derivation (see § XXX), attesting a \textsc{proprietive} $\Rightarrow$ \textsc{comitative} grammaticalization pathway  (\citealt{jacques17comitative}). 

When the base noun is an IPN, it is possible to build a comitative adverb with the indefinite possessor prefix or with the bare stem. For instance, from \japhug{tɤ-rte}{hat} one can derive both \forme{kɤ́-rtɯ\tld{}rte} / \forme{kɤɣɯ-rtɯ\tld{}rte} `with his/her hat' and \forme{kɤ́-tɤ-rtɯ\tld{}rte} /  \forme{kɤɣɯ-tɤ-rtɯ\tld{}rte} `with a hat' with the indefinite possessor prefix \forme{tɤ-}. These two sets of forms have different meanings: the former \forme{kɤ́-rtɯ\tld{}rte} / \forme{kɤɣɯ-rtɯ\tld{}rte} mean `wearing one's hat' (example \ref{ex:kAGWrtWrte}), while the latter \forme{kɤ́-tɤ-rtɯ\tld{}rte} /  \forme{kɤɣɯ-tɤ-rtɯ\tld{}rte} imply that the subject is not wearing the hat (\ref{ex:kAGWtArtWrte}); preserving the indefinite possessor in the derived form alienabilizes the IPN (see § \ref{sec:alienabilization}).

\begin{exe}
\ex \label{ex:kAGWrtWrte}
\gll kɤɣɯ-rtɯ\tld{}rte 	ʑo 	kʰa	ɯ-ŋgɯ	lɤ-tɯ-ɣe	\\
\textsc{comit}-hat \textsc{emph} house \textsc{3sg}-inside \textsc{pfv}-2-come[II] \\
\glt `You came inside the house wearing your hat.' (You were expected to take it off before coming in, elicited)
\end{exe}

\begin{exe}
\ex \label{ex:kAGWtArtWrte}
\gll  laχtɕʰa	kɤɣɯ-tɤ-rtɯ\redp{}rte	ʑo	ta-ndo \\
thing \textsc{comit-indef.poss}-hat \textsc{emph} \textsc{pfv}:3$\rightarrow$3'-take \\
\glt `He took the things together with the hat.' (Not wearing it, elicited)
\end{exe}

Example (\ref{ex:tsxha.kAtAlWlu}) illustrates the use of the alienabilized comitative adverb \japhug{kɤ́tɤlɯlu}{with milk} (from \japhug{tɤ-lu}{milk}) for the expression `milk tea' -- the inalienably form \japhug{kɤ́lɯlu}{with its milk} is only compatible with the animal producing the milk.

\begin{exe}
\ex \label{ex:tsxha.kAtAlWlu}
\gll   ʁja ku-te ɲɯ-ŋu, tʂʰa kɤ́-tɤlɯ\redp{}lu tú-wɣ-sɯ-rku kɯnɤ \\
verdigris \textsc{ipfv}-put[III] \textsc{sens}-be tea \textsc{comit}-milk \textsc{ipfv}-\textsc{inv}-\textsc{caus}-put.in also \\
\glt  `It gets verdigris, even when uses it to pour milk tea.'  (30-Com, 93-4)
\end{exe}

Comitative adverbs can be used as sentential adverbs, with scope over the whole sentence (\ref{ex:kAGWrtWrte}, \ref{ex:kAGWtArtWrte}, \ref{ex:kAsnWsno}). 

\begin{exe}
\ex \label{ex:kAsnWsno}
\gll kɤ́-snɯ\tld{}sno 	ʑo 	kɤ-rŋgɯ \\
\textsc{comit}-saddle \textsc{emph} \textsc{pfv}-lie.down \\
\glt `(The horse) slept with its saddle.' (elicited)
\end{exe}

Alternatively the comitative adverb is used as noun modifier (even in fixed expressions, as in \ref{ex:tsxha.kAtAlWlu}), and either follow (\ref{ex:tsxha.kAtAlWlu}, \ref{ex:kAjWjaR}) or precede (\ref{ex:kArnWrna}, \ref{ex:kAthAlwWlwa}) the noun which it modifies.

The noun in question can either correspond to the object (\ref{ex:tsxha.kAtAlWlu}, \ref{ex:kAjWjaR}, \ref{ex:kAthAlwWlwa}), the intransitive subject (\ref{ex:kArnWrna}, \ref{ex:kAsnWsno}) or even the transitive subject (\ref{ex:kArJWrJit.kW}). This last option is not attested in the text corpus, but speakers have no trouble producing sentences of this type.


\begin{exe}
\ex \label{ex:kAjWjaR}
\gll tɤ-sno 	kɤ́-jɯ\redp{}jaʁ 	nɯ 	lu-ta-nɯ \\
\textsc{indef.poss}-saddle \textsc{comit}-hand \textsc{dem} \textsc{ipfv}-put-\textsc{pl} \\
\glt `(Then), they put the saddle with its handles.' (30-tAsno, 77)
\end{exe}
 

\begin{exe}
\ex \label{ex:kArnWrna}
\gll pɣɤkʰɯ 	nɯ 	ɯ-ku 	nɯnɯ 	lɯlu 	tsa 	ɲɯ-fse, 	ɯ-mtsioʁ 	ɣɤʑu 	ma kɤ́-rnɯ\redp{}rna 	lɯlu 	ɯ-tɯ-fse 	ɲɯ-sɤre 	ʑo. \\
owl \textsc{dem} \textsc{3sg.poss}-head \textsc{dem} cat a.little \textsc{sens}-be.like \textsc{3sg.poss}-beak exist:\textsc{sens} a.part.from \textsc{comit}-ear cat \textsc{3sg-nmlz:degree}-be.like \textsc{sens}-be.extremely/be.funny \textsc{emph} \\
\glt `The owl's head looks a little like that of a cat, apart from the fact that it has a beak, it looks very much like a cat with its ears.' (22-pGAkhW, 7)
\end{exe}


\begin{exe}
\ex \label{ex:kAthAlwWlwa}
\gll kɤ́-tʰɤlwɯ\tld{}lwa 	ɯ-zrɤm 	ra 	kɯnɤ 	cʰɯ́-wɣ-ɣɯt 	pjɯ́-wɣ-ji 	ri 	maka 	tu-ɬoʁ 	mɯ́j-cʰa \\  
\textsc{comit}-earth \textsc{3sg.poss}-root \textsc{pl} also \textsc{ipfv-inv}-bring \textsc{ipfv-inv}-plant but at.all \textsc{ipfv}-come.out \textsc{neg:sens}-can \\
\glt `Even if one takes its root with earth (around it) and plant it, it cannot grow.' (15-babW, 121)
\end{exe}


\begin{exe}
\ex \label{ex:kArJWrJit.kW}
\gll lɯlu	kɤ́-rɟɯ\tld{}rɟit	ra	kɯ	ʑo	βʑɯ	 to-ndza-nɯ. \\
cat \textsc{comit}-offspring \textsc{pl}  \textsc{erg} \textsc{emph} mouse \textsc{ifr}-eat-\textsc{pl} \\
\glt `The cat and its young ate the mouse.' (elicited)
\end{exe}

The comitative adverbs have additional meanings in certain contexts. With the verb \japhug{fse}{be like},  comitative adverbs from body parts occurring with names of animals, as in (\ref{ex:kArnWrna}) and (\ref{ex:BZW.kAmtChWmtChi}), mean `to have a body part that looks like that of the other animal'.

\begin{exe}
\ex \label{ex:BZW.kAmtChWmtChi}
\gll
li βʑɯ kɤ́-mtɕʰɯ\redp{}mtɕʰi ci nɯ ɲɯ-fse  \\
again mouse \textsc{comit}-mouth \textsc{indef} \textsc{dem} \textsc{sens}-be.like \\
\glt `(The bat's) mouth is like that of a mouse.' (literally `It looks like a mouse with its mouth.' 25-qarmWrwa, 12)
\end{exe}

Nouns incorporated into comitative adverbs lose their nominal status and cannot be determined by relative clauses (including attributive adjectives), numerals or demonstratives. In a sentence such as \ref{ex:kAGWNkhWNkhor} for instance, the attributive participial relative [\forme{kɯ\tld{}kɯ-ŋɤn}] `all the ones who are evil' does not determine \forme{kɤɣɯ-ŋkhɯ\tld{}ŋkhor} `with his subjects', a syntactic structure which would correspond to the translation `with all his evil subjects'. Rather, it determines the head noun together with the comitative adverb  \forme{rɟɤlpu} \forme{kɤɣɯ-ŋkhɯ\tld{}ŋkhor} `the king with his subjects', which implies the translation given below.

\begin{exe}
\ex \label{ex:kAGWNkhWNkhor}
\gll rɟɤlpu 	kɤɣɯ-ŋkhɯ\tld{}ŋkhor 	kɯ\tld{}kɯ-ŋɤn  	ʑo 	to-ndo 	tɕe, 	tɕendɤre 	kɯ-mɤku 	nɯ 	sɤtɕʰa 	kɯ\tld{}kɯ-sɤ-scit 	ʑo 	jo-tsɯm 	ɲɯ-ŋu 	ri 	kɯ-maqʰu 	tɕe, 	kɯ\tld{}kɯ-sɤɣ-mu 	ʑo 	jo-tsɯm 	tɕe \\
king \textsc{comit}-subjects \textsc{total}\tld{}\textsc{nmlz}:S/A-be.bad \textsc{emph} \textsc{ifr}-take \textsc{lnk}  \textsc{lnk} \textsc{nmlz}:S/A-be.before \textsc{dem} place \textsc{total}\tld{}\textsc{nmlz}:S/A-\textsc{deexp}-be.happy \textsc{emph} \textsc{ifr}-take.away \textsc{sens}-be \textsc{lnk} \textsc{nmlz}:S/A-be.after \textsc{lnk} \textsc{total}\tld{}\textsc{nmlz}:S/A-\textsc{deexp}-fear \textsc{emph} \textsc{ifr}-take.away \textsc{lnk} \\
\glt `She took the king and his subjects, all the evil ones, in the beginning she took them to nice places, but later she took them to fearful places.' (Norbzang2012, 390)
\end{exe}

Other denominal adverb formations are also attested in Japhug, but are studied in the sections on time nominals (§ XXX) and locational nouns (§  XXX) in other chapters.


\subsection{Other denominal adverbs} \label{sec:other.denominal.adverbs}
Partial reduplication of nouns, in addition to the reduplicated collectives (§ \ref{sec:redp.coll}), can also derive location adverbs such as \japhug{tʂɯtʂu}{on the road} from \japhug{tʂu}{road} with a prolative meaning, as in (\ref{ex:tsxWtsxu}).

\begin{exe}
\ex \label{ex:tsxWtsxu}
\gll cʰa ra tʂɯ\redp{}tʂu kú-wɣ-nɯ-tsʰi tɕe \\
alcohol \textsc{pl} path\redp{}\textsc{location}  \textsc{ipfv}-\textsc{inv}-\textsc{auto}-drink \textsc{lnk} \\
\glt `One drinks alcohol on the way (back home).' (2010-histoire10)
\end{exe}
