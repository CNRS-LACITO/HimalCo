\chapter{Nominal morphology}
This chapter does not treat of grammatical categories expressed by independent words or clitics, such as number and grammatical relations (discussed in XXX and XXX), and focuses on possessive prefixes, compounding and  noun derivations. Nominalization (including lexicalized deverbal nouns) and denominal verbalization are treated in chapters XXX and XXX respectively.

\section{Possessive prefixes}  \label{sec:possessive.prefixes}
Japhug nouns are divided into two main categories, inalienably possessed nouns, which require a possessive prefix (\ref{sec:inalienably.possessed}) and common nouns which can occur with or without possessive prefix. 

\subsection{Possessive paradigm} \label{sec:possessive.paradigm}
The paradigm of possessive prefixes in Japhug is indicated in Table \ref{tab:possessive.prefixes}. It presents obvious commonalities with the personal pronouns (section \ref{sec:pers.pronouns}) and the indexation suffixes (section XXX), a question studied in more detail in XXX. 

\begin{table}[h] \centering
\caption{Possessive prefixes }\label{tab:possessive.prefixes}
\begin{tabular}{lllllllll} \lsptoprule
 Prefix & Person \\
\midrule
\forme{a-}  &		1\sg{} \\
\forme{nɤ-}  &			2\sg{} \\
\forme{ɯ-}  &			3\sg{} \\
\midrule
\forme{tɕi-}  &			1\du{} \\
\forme{ndʑi-}  &		2/3\du{} \\	
\midrule
\forme{i-}  &			1\pl{} \\
\forme{nɯ-}  &			2/3\pl{} \\
\midrule
\forme{tɯ-/tɤ-}  &			indefinite \\
\forme{tɯ-}  &			generic \\
\lspbottomrule
\end{tabular}
\end{table}

In the possessive paradigm, the contrast between second and third person is neutralized in the dual and plural, while it is preserved in pronouns and person indexation.

Unlike languages like Situ which have two series of possessive pronouns with the same initial consonant but distinct vocalism (\citealt[168-169]{linxr93jiarong}),\footnote{\citet[118-119]{prins16kyomkyo} analyzes the vowel as part of the nominal root.} Japhug preserves the vowel contrast \ipa{ɯ} vs \ipa{ɤ} only with the indefinite possessor form of inalienably possessed nouns; with definite possessors, the contrast is neutralized.

Stacking of possessive prefixes is not allowed in Japhug, with the exception of the combination of a definite possessor prefix with an indefinite possessor prefix \forme{tɯ-} or \forme{tɤ-} to turn an inalienably possessed noun into an alienably possessed one (see section \ref{sec:alienabilization}).

Possession cannot be expressed without a possessive prefix on the possessee. Possessive prefixes can be used on any noun, including recent borrowings from Chinese (or quasi-code switching), as \zh{老家} \forme{lǎojiā} `place of origin; old house' in (\ref{ex:aʑo.GW.alaojia}).

\begin{exe}
\ex \label{ex:aʑo.GW.alaojia}
\gll
aʑo ɣɯ a-<laojia> ɣɯ ɯ-lɤcu nɯre ri ku-rɤʑi-nɯ ŋu \\
\textsc{1sg} \textsc{gen} \textsc{1sg.poss}-old.house \textsc{gen} \textsc{3sg.poss}-upstream there \textsc{loc} \textsc{ipfv}-stay-\textsc{pl} be:\textsc{fact} \\
\glt `They live in a place upstream from my old house.' (14-tApitaRi, 238)
\end{exe}

In the case of first or second person possessors, it is possible to have simply a possessive prefix on the noun, (\japhug{a-ɣɲi}{my friend}, \japhug{a-mbro}{my horse} and \japhug{a-ʁgra}{my enemy} in \ref{ex:ambro}), a personal pronoun and a possessive prefix (same person and number, as in \ref{ex:aʑo.ambro}) or even a pronoun, the genitive clitic \forme{ɣɯ} and a possessive prefix as in (\ref{ex:aʑo.GW.alaojia}).

 \begin{exe}
\ex \label{ex:ambro} 
\gll a-ɣɲi ci tɯ\redp{}tɯ-ŋu nɤ, a-mbro ɯ-lwa ɯ-taʁ kɤ-zo, a-ʁgra ci tɯ\redp{}tɯ-ŋu nɤ, a-mbro ɯ-jme ɯ-taʁ kɤ-zo \\
\textsc{1sg.poss}-friend \textsc{indef} \textsc{cond}\redp{}2-be:\textsc{fact} \textsc{lnk} \textsc{1sg.poss}-horse \textsc{3sg.poss}-mane \textsc{3sg}-on \textsc{imp}-land \textsc{1sg.poss}-enemy \textsc{indef} \textsc{cond}\redp{}2-be:\textsc{fact} \textsc{lnk} \textsc{1sg.poss}-horse \textsc{3sg.poss}-tail  \textsc{3sg}-on \textsc{imp}-land  \\
\glt `If you are my friend, land on my horse's mane, if you are my enemy, land on my horse's tail.' (2002qaCpa, 196)
\end{exe}

\begin{exe}
\ex \label{ex:aʑo.ambro}
\gll aʑo a-mbro nɤrwɯrɯnbotɕʰi ŋu, tɯ-sŋi χpaχtsʰɤt ci ɲɯ́-wɣ-tsɯm-a cʰa \\
\textsc{1sg} \textsc{1sg.poss}-horse p.n. be:\textsc{fact} one-day yojana \textsc{indef} i\textsc{pfv:west}-\textsc{inv}-take.away-\textsc{1sg} can:\textsc{fact} \\
\glt `My horse is Norbu Rinpoche, he can make me cross one yojana per day.' (2003smanmi2, 54)
\end{exe}

Possessive prefixes are also used to express beneficiaries, recipients and other oblique arguments, such as the `person needing' in the construction with the verb \japhug{ra}{need, have to},  as in (\ref{ex:ambro.tARndo.kWtso}).\footnote{See section XXX for a more detailed account of the expression of beneficiaries in Japhug.}

 \begin{exe}
\ex \label{ex:ambro.tARndo.kWtso}
\gll a-mbro taʁndo kɯ-tso ci tɕi ra \\
\textsc{1sg.poss}-horse speech \textsc{nmls}:S/A-understand one also have.to:\textsc{fact} \\
\glt `I also need a horse who understands speech.' (2003kAndzwsqhaj2, 52)
\end{exe}

In the case of beneficiaries and recipients, if a genitive pronoun or genitive phrase is present, the presence of a possessive prefix is possible (\ref{ex:aZWG.akWra}) but not obligatory (\ref{ex:aZWG.kWra}), in particular in the case of possessed nouns that already have a definite possessor (\ref{ex:aZWG.Wlu}).

 \begin{exe}
\ex \label{ex:aZWG.akWra}
\gll aʑɯɣ a-kɯ-ra ci tu tɕe nɯ `ɣa' tɤ-ti ra \\
\textsc{1sg:gen} \textsc{1sg.poss}-\textsc{nmlz}:S/A-have.to \textsc{indef} exist:\textsc{fact} \textsc{lnk} \textsc{dem} yes \textsc{imp}-say have.to:\textsc{fact} \\
\glt `There is one thing I need, and you have to say `yes' to it.' (140429 ingwa wangzi, 47)
\end{exe}

 \begin{exe}
\ex \label{ex:aZWG.kWra}
\gll  aʑɯɣ kɯ-ra me \\
\textsc{1sg:gen} \textsc{nmlz}:S/A-have.to not.exist:\textsc{fact} \\
\glt `I don't need anything.' (2005slobdpon2, 275)
\end{exe}

 \begin{exe}
\ex \label{ex:aZWG.Wlu}
\gll aʑɯɣ ɯ-lu ra \\
\textsc{1sg:gen} \textsc{3sg.poss}-milk have.to:\textsc{fact} \\
\glt `I need its milk.' (02-deluge2012, 12)
\end{exe}


\subsection{Inalienably possessed nouns} \label{sec:inalienably.possessed}
Inalienably possessed nouns (IPN) differ from non-inalienably possessed nouns (NIPN) in that they require the presence of a possessive prefix. 

When the possessive prefix is definite, IPNs are not formally distinguishable from NIPNs; for instance, \japhug{a-pi}{my elder sibling} and \japhug{a-mbro}{my horse} both take the 1\sg{} \forme{a-} prefix and no direct clue indicates that the first noun is IPN and that the second one is NIPN.

The citation form however differs between IPN and NIPN: the former must take an indefinite possessor prefix (or in some cases a 3\sg{} \forme{ɯ-}), while the latter can occur without possessive prefix, as for instance \japhug{tɤ-pi}{elder sibling} (with the indefinite \forme{tɤ-}; the bare root $\dagger$\forme{pi} is not a correct form) vs \japhug{mbro}{horse} (without prefix).

Contrast \japhug{tɤ-ma}{mother} \japhug{ta-ma}{work} 
\subsubsection{Body parts} \label{sec:body.part}
Nearly all body parts are IPN with the indefinite possessor \forme{tɯ-}. These include native words, but also borrowings from Tibetan such as \japhug{tɯ-qʰoχpa}{organs, state of mind} from Tibetan \tibet{ཁོག་པ་}{kʰog.pa}{insides} (see \ref{sec:uvular.harmony} on the phonology of this word).

Body parts IPN selecting the prefix \forme{tɤ-} are mainly liquids from the body such as \japhug{tɤ-se}{blood}, \forme{tɤ-spɯ}{pus} and \japhug{tɤ-lu}{milk} (though some liquids also take the prefix \forme{tɯ-}, for instance \japhug{tɯ-ɕtʂi}{sweat}), hair (\japhug{tɤ-rme}{hair, fur}, \japhug{tɤ-kɤrme}{hair (head)}) and some animal body parts \japhug{tɤ-jme}{tail}, \japhug{tɤ-ŋkɯ}{pig skin}, \japhug{tɤ-rkʰom}{feather rachis}).

\subsubsection{Kinship terms}
\subsubsection{Alienabilization of IPN} \label{sec:alienabilization}

\subsection{Indefinite vs generic possessor}

\subsection{How did words like `sky', `earth' and `water' become inalienably possessed in Japhug?}

\section{Status constructus}

\subsection{Second member of compounds}
Irregular forms for second members of compounds are very rare.  The noun \japhug{ftɕɤru}{path in the middle of the fields} is a compound of \japhug{ftɕar}{summer} and \japhug{tʂu}{path} (such paths are made during summer to allow workers to work in the field without damaging the crops). The first element \forme{ftɕɤ-} is the \textit{status constructus} of \forme{ftɕar} (with loss of final consonant) and the form \forme{-ru} for the second member of the compound is a clue that \forme{tʂu} comes from earlier \forme{*t-ro} with a dental stop+\ipa{r} cluster changing to a retroflex affricate (see section XXX) -- the \forme{*t-} element being prefixal (perhaps a fossilized indefinite possessor prefix).

\section{Compounding}
\subsection{Noun-Noun compounds}
\subsection{Verb-Verb compounds}
\subsection{Adverb-Verb compounds}
\subsection{Noun-Verb compounds}
\subsection{Verb-Noun compounds}
Verb-Noun compounds are extremely rare in Japhug, as they are in general in Trans-Himalayan languages other than Chinese.  A possible example is \japhug{ndzɤpri}{brown bear}, compound of \japhug{pri}{bear} and \japhug{ndza}{eat} -- as shown by (\ref{ex:ndzApri}) from a text about bears, it is considered by some native speakers of Japhug as a man eater, though this explanation could be folk-etymology.

\begin{exe}
\ex \label{ex:ndzApri}
\gll tɕe ndzɤpri kɤ-ti nɯ tɕe tɯrme tu-kɯ-ndza ɲɯ-ŋgrɤl \\
\textsc{lnk} brown.bear \textsc{nmlz}:S-say \textsc{dem} \textsc{lnk} people \textsc{ipfv}-\textsc{genr}:S/P-eat \textsc{sens}-be.usually.the.case \\
\glt `The brown bear, it eats people.' (21-pri, 94)
\end{exe} 


We find several examples of nominal compounds whose structure is \forme{tɤ-}+Verb+Noun, where the verb is an adjectival stative verb. This category includes \japhug{tɤqiaβjmɤɣ}{lactarius sp.}, literally `bitter mushroom', from the noun \japhug{tɤjmɤɣ}{mushroom} and the verb \japhug{qiaβ}{be bitter}, and \japhug{tɤmbextsa}{type of shoes} from \japhug{tɯ-xtsa}{shoe} and \japhug{mbe}{be old}. These should not be analyzed as Verb-Noun compounds however, as the first element originates from a nominalized form of the verb (such as \japhug{tɤ-mbe}{old thing}, see section XXX on this derivation): they rather are a subtype of Noun-Noun compounds.

\section{Noun class prefixes}
\section{Nominal derivations}
\subsection{Collective}
While Japhug lacks number inflection, there is a marginal collective derivation. It is built by partial reduplication, but differs from the most common reduplication described in section XXX in that the vowel in the replicated syllable replacing the rhyme is \ipa{a}, not \ipa{ɯ}. Examples are few, as shown in Table \ref{tab:coll.n}, but several of them are borrowings from Tibetan.

\begin{table}
\caption{Collective noun derivation} \label{tab:coll.n}
\begin{tabular}{l|lll}
 \lsptoprule 
 Base form & Collective & Tibetan \\
 \midrule
\japhug{rdɯl}{dust, dirt} & \japhug{rdardɯl}{dust, dirt} & \tibet{རྡུལ་}{rdul}{dust} \\
\japhug{tɯ-ntɕʰɯr}{fragment}  & \japhug{ɯ-ntɕʰantɕʰɯr}{fragments} & \\
\japhug{ɯ-zɯr}{side}  & \japhug{ɯ-zarzɯr}{sides} & \tibet{ཟུར་}{zur}{side, corner} \\
\japhug{ɯ-rkɯ}{side} & \japhug{ɯ-rkarkɯ}{sides} & \\
 \lspbottomrule
\end{tabular}
\end{table}

Collective nouns can be used without number clitic, as in (\ref{ex:WntChantChWr}), but they often appear with the \japhug{ra}{plural} as in (\ref{ex:rdardWl}).

\begin{exe}
\ex \label{ex:WntChantChWr}
\gll znɤrɣama nɯ mtʰa ɯ-kɤcu ŋu. tɕe nɯnɯtɕu tɯ-ji ɯ-ntɕhantɕhɯr pɯ-dɤn, jinde kʰro ɲɤ-s-qapɯ-nɯ,\\
p.n. \textsc{dem} p.n. \textsc{3sg.poss}-east be:\textsc{fact} \textsc{lnk} \textsc{dem:pl} \textsc{indef}.\textsc{poss}-field \textsc{3sg.poss}-fragment:\textsc{coll} \textsc{pst}.\textsc{ipfv}-be:many now much \textsc{ifr}-\textsc{caus}-be.fallow-\textsc{pl}\\
\glt `Znargama ('The place where one calls the rain') is on the east of Mtha, there used to be many little fragments of fields, but now people have left them become fallow.' (150903 kAmYW tWji3, 19)
\end{exe}

\begin{exe}
\ex \label{ex:rdardWl}
\gll tɕe tɤɕi nɯ tú-wɣ-χtɕi tɕʰɣaʁtɕʰɣaʁ ʑo tɕe, rdardɯl nɯra ɲɯ́-wɣ-ɣɤ-me tɕe \\
\textsc{lnk} barley \textsc{dem} \textsc{ipfv}-\textsc{inv}-wash \textsc{idph}:II:completely.clean \textsc{emph} \textsc{lnk} dush:\textsc{coll} \textsc{dem:pl} \textsc{ipfv}-\textsc{inv}-\textsc{caus}-not.exist \textsc{lnk} \\
\glt `Then one washes the barley very thoroughly, one removes all the dirt.' (2002tWsqar, 118)
\end{exe}
 
The noun \japhug{rgargɯn}{old person} has the form of a collective noun as those of Table \ref{tab:coll.n}, but it is commonly used with singular or dual referents (as in \ref{ex:rgargWn}). It could be originally the collective form of a borrowing from Tibetan \tibet{རྒན་པོ་}{rgan.po}{old person}, though the expected form would have been $\dagger$\forme{rga-rgɤn}. 
 
\begin{exe}
\ex \label{ex:rgargWn}
\gll  rgargɯn ni kɤ-fstɯn pɯ-ra \\
old.person \textsc{du} \textsc{inf}-serve \textsc{pst.ipfv}-have.to \\
\glt `She had to take care of two old people.' (14-tApitaRi, 34)
\end{exe}

\subsection{Superlative}
While there is no adjectival superlative derivation in Japhug (superlative constructions are synthetic, see section XXX), we find nevertheless a derivation applied to locative nouns, expressing the most extreme location. As shown in Table \ref{tab:superlative.n}, it is built by adding an element \forme{-ɕɯ-} followed by a complete copy of the root of the noun without \textit{status constructus} alternation or partial replication; the resulting noun is still an inalienably possessed locative noun. Example (\ref{ex:WqaCWqa}) illustrate the use of one of these forms.

\begin{table}
\caption{Superlative noun derivation} \label{tab:superlative.n}
\begin{tabular}{l|lll}
 \lsptoprule
\japhug{tɯ-ku}{head, top} & \japhug{ɯ-kuɕɯku}{the highest place} \\
\japhug{tɯ-qa}{root, paw, bottom} & \japhug{ɯ-qaɕɯqa}{the deepest place} \\
\japhug{ɯ-rkɯ}{side} & \japhug{ɯ-rkɯɕɯrkɯ}{the place most on the side} \\
\japhug{ɯ-zɯr}{side} & \japhug{ɯ-zɯrɕɯzɯr}{the place most on the side} \\
 \lspbottomrule
\end{tabular}
\end{table}

\begin{exe}
\ex \label{ex:WqaCWqa}
\gll rɟɤmtsʰu ɯ-qaɕɯqa pjɯ-ɕe tɕe, nɯnɯ ɯ-kɤ-nɤ-mɯm nɯra ɕ-tu-nɯ-tɕɤt ɲɯ-ŋu. \\
ocean \textsc{3sg.poss}-bottom:\textsc{super} \textsc{ipfv}:\textsc{down}-go  \textsc{lnk} \textsc{dem} \textsc{3sg.poss}-\textsc{nmlz}:P-\textsc{trop}-be.tasty \textsc{dem:pl} \textsc{transloc-ipfv}-\textsc{auto}-take.out \textsc{sens}-be \\
\glt `(The sperm whale) goes to the lowest depths of the ocean and catches the things it likes to eat.' (160703 jingyu, 24)
\end{exe}
\section{Denominal adverbs}
\subsection{Comitative adverbs}
\subsection{Time adverbs}
