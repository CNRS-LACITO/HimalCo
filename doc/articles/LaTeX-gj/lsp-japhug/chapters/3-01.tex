\chapter{Nominal morphology}
This chapter does not treat of grammatical categories expressed by independent words or clitics, such as number and grammatical relations (discussed in XXX and XXX), and focuses on possessive prefixes, compounding and  noun derivations. Nominalization (including lexicalized deverbal nouns) and denominal verbalization are treated in chapters XXX and XXX respectively.

\section{Possessive prefixes}  \label{sec:possessive.prefixes}
Japhug nouns are divided into two main categories, inalienably possessed nouns, which require a possessive prefix (\ref{sec:inalienably.possessed}) and common nouns which can occur with or without possessive prefix. 

\subsection{Possessive paradigm} \label{sec:possessive.paradigm}
The paradigm of possessive prefixes in Japhug is indicated in Table \ref{tab:possessive.prefixes}. It presents obvious commonalities with the personal pronouns (section \ref{sec:pers.pronouns}) and the indexation suffixes (section XXX), a question studied in more detail in XXX. 

\begin{table}[h] \centering
\caption{Possessive prefixes }\label{tab:possessive.prefixes}
\begin{tabular}{lllllllll} \lsptoprule
 Prefix & Person \\
\midrule
\forme{a-}  &		1\sg{} \\
\forme{nɤ-}  &			2\sg{} \\
\forme{ɯ-}  &			3\sg{} \\
\midrule
\forme{tɕi-}  &			1\du{} \\
\forme{ndʑi-}  &		2/3\du{} \\	
\midrule
\forme{i-}  &			1\pl{} \\
\forme{nɯ-}  &			2/3\pl{} \\
\midrule
\forme{tɯ-/tɤ-/ta-}  &			indefinite \\
\forme{tɯ-}  &			generic \\
\lspbottomrule
\end{tabular}
\end{table}

In the possessive paradigm, the contrast between second and third person is neutralized in the dual and plural, while it is preserved in pronouns and person indexation.

Unlike languages like Situ which have two series of possessive pronouns with the same initial consonant but distinct vocalism (\citealt[168-169]{linxr93jiarongen}),\footnote{\citet[118-119]{prins16kyomkyo} analyzes the vowel as part of the nominal root.} Japhug preserves the vowel contrast \ipa{ɯ} vs \ipa{ɤ} only with the indefinite possessor form of inalienably possessed nouns; with definite possessors, the contrast is neutralized.

Stacking of possessive prefixes is not allowed in Japhug, with the exception of the combination of a definite possessor prefix with an indefinite possessor prefix \forme{tɯ-} or \forme{tɤ-} to turn an inalienably possessed noun into an non-inalienably possessed one (see section \ref{sec:alienabilization}).

\subsubsection{The expression of possession}
Possession cannot be expressed without a possessive prefix on the possessee. Possessive prefixes can be used on any noun, including recent borrowings from Chinese (or quasi-code switching), as \zh{老家} \forme{lǎojiā} `place of origin; old house' in (\ref{ex:aZo.GW.alaojia}). They also occur on most non-finite verbal forms, including participles (see XXX and XXX), bare infinitives (XXX) and degree nominal (XXX).

\begin{exe}
\ex \label{ex:aʑo.GW.alaojia}
\gll
aʑo ɣɯ a-<laojia> ɣɯ ɯ-lɤcu nɯre ri ku-rɤʑi-nɯ ŋu \\
\textsc{1sg} \textsc{gen} \textsc{1sg.poss}-old.house \textsc{gen} \textsc{3sg.poss}-upstream there \textsc{loc} \textsc{ipfv}-stay-\textsc{pl} be:\textsc{fact} \\
\glt `They live in a place upstream from my old house.' (14-tApitaRi, 238)
\end{exe}

In the case of first or second person possessors, it is possible to have simply a possessive prefix on the noun, (\japhug{a-ɣɲi}{my friend}, \japhug{a-mbro}{my horse} and \japhug{a-ʁgra}{my enemy} in \ref{ex:ambro}), a personal pronoun and a possessive prefix (same person and number, as in \ref{ex:aZo.ambro}) or even a pronoun, the genitive clitic \forme{ɣɯ} and a possessive prefix as in (\ref{ex:aZo.GW.alaojia}).

 \begin{exe}
\ex \label{ex:ambro} 
\gll a-ɣɲi ci tɯ\redp{}tɯ-ŋu nɤ, a-mbro ɯ-lwa ɯ-taʁ kɤ-zo, a-ʁgra ci tɯ\redp{}tɯ-ŋu nɤ, a-mbro ɯ-jme ɯ-taʁ kɤ-zo \\
\textsc{1sg.poss}-friend \textsc{indef} \textsc{cond}\redp{}2-be:\textsc{fact} \textsc{lnk} \textsc{1sg.poss}-horse \textsc{3sg.poss}-mane \textsc{3sg}-on \textsc{imp}-land \textsc{1sg.poss}-enemy \textsc{indef} \textsc{cond}\redp{}2-be:\textsc{fact} \textsc{lnk} \textsc{1sg.poss}-horse \textsc{3sg.poss}-tail  \textsc{3sg}-on \textsc{imp}-land  \\
\glt `If you are my friend, land on my horse's mane, if you are my enemy, land on my horse's tail.' (2002qaCpa, 196)
\end{exe}

\begin{exe}
\ex \label{ex:aZo.ambro}
\gll aʑo a-mbro nɤrwɯrɯnbotɕʰi ŋu, tɯ-sŋi χpaχtsʰɤt ci ɲɯ́-wɣ-tsɯm-a cʰa \\
\textsc{1sg} \textsc{1sg.poss}-horse p.n. be:\textsc{fact} one-day yojana \textsc{indef} i\textsc{pfv:west}-\textsc{inv}-take.away-\textsc{1sg} can:\textsc{fact} \\
\glt `My horse is Norbu Rinpoche, he can make me cross one yojana per day.' (2003smanmi2, 54)
\end{exe}

It is possible to have a first singular possessive preceded by a first plural pronoun, as in (\ref{ex:iZo.amu}) (see XXX for other examples of person mismatch involving \textsc{1pl} pronouns).

\begin{exe}
\ex \label{ex:iZo.amu}
\gll iʑo a-mu nɯ tʰamtʰam kɯrcɤsqaptɯɣ tʰɯ-azɣɯt ŋu. \\
\textsc{1pl} \textsc{1sg.poss}-mother \textsc{dem} now 81 \textsc{pfv}-reach  be:\textsc{fact} \\
\glt `My mother is now 81.' (2010-histoire09-2, 15)
\end{exe}

\subsubsection{Definiteness and obviation}
Nouns with a definite possessor in Japhug can be indefinite, unlike in most languages of Europe. Nouns with a definite possessor can occur an indefinite determiner (example \ref{ex:ambro}  above), and with a quantifier as in (\ref{ex:Wzda.tWrdoR}) mean a certain number of persons out of a group (`one of his X').

 \begin{exe}
\ex \label{ex:Wzda.tWrdoR}
\gll tɤ-tɕɯ nɯ kɯ ɯ-zda tɯ-rdoʁ ɯ-pʰe to-ti, tɯ-rdoʁ nɯ kɯ li ci ɯ-pʰe tɕe ɲɤ-k-ɤ-sɯ-ɤmɯ-mtsʰɯ\redp{}mtsʰɤm-nɯ \\
\textsc{indef.poss}-son \textsc{dem} \textsc{erg} \textsc{3sg.poss}-companion one-\textsc{cl} \textsc{3sg-dat} \textsc{ifr}-say one-\textsc{cl}  \textsc{dem} \textsc{erg} again \textsc{indef} \textsc{3sg-dat} \textsc{lnk}   \textsc{ifr}-\textsc{evd}-\textsc{pass}-\textsc{caus}-\textsc{recip}-hear-\textsc{pl} \\
\glt `The boy told one of his companion, and that one another one, and they informed each other.' (x1-sloXpWn, 82)
\end{exe}

Unlike Algonquian languages, but like Mapudungun (\citealt{haude16symmetrical}), possessed nouns are not automatically obviative, and inverse marking on the verb is not required if a possessed noun is subject, and its possessor object of a transitive verb, as shown by example (\ref{ex:prox.naBde}) where the direct form \forme{na-βde} appears (see \citealt{jacques10inverse} for other examples, and section XXX on inverse marking). The inverse \forme{nɯ́-wɣ-βde} is also possible in exactly the same context -- example  (\ref{ex:obv.nWwGBde}) comes from the same text and refers to the same event.

\begin{exe}
\ex \label{ex:prox.naBde}
\gll ɯ-rʑaβ nɯ kɯ na-βde \\
\textsc{3sg.poss}-wife \textsc{dem} \textsc{erg} \textsc{pfv}:3\fl3'-throw.away \\
\glt `His wife left him.' (14-tApitaRi, 289)
\end{exe}

\begin{exe}
\ex \label{ex:obv.nWwGBde}
\gll
ɯ-rʑaβ cʰo ɯ-tɕɯ nɯ ʁnaʁna kɯ nɯ́-wɣ-βde qʰe \\
\textsc{3sg.poss}-wife \textsc{comit} \textsc{3sg.poss}-son \textsc{dem} both \textsc{erg} \textsc{pfv}-\textsc{inv}-throw.away \textsc{lnk} \\
\glt `His wife and his son left him.' (14-tApitaRi, 294)
\end{exe}

\subsubsection{Other uses of possessive prefixes}
Possessive prefixes are also used to express beneficiaries, recipients and other oblique arguments, such as the `person needing' in the construction with the verb \japhug{ra}{need, have to},  as in (\ref{ex:ambro.tARndo.kWtso}).\footnote{See section XXX for a more detailed account of the expression of beneficiaries in Japhug.}

 \begin{exe}
\ex \label{ex:ambro.tARndo.kWtso}
\gll a-mbro taʁndo kɯ-tso ci tɕi ra \\
\textsc{1sg.poss}-horse speech \textsc{nmls}:S/A-understand one also have.to:\textsc{fact} \\
\glt `I also need a horse who understands speech.' (2003kAndzwsqhaj2, 52)
\end{exe}
In the case of beneficiaries and recipients, if a genitive pronoun or genitive phrase is present, the presence of a possessive prefix is possible (\ref{ex:aZWG.akWra}) but not obligatory (\ref{ex:aZWG.kWra}), in particular in the case of possessed nouns that already have a definite possessor (\ref{ex:aZWG.Wlu}).

 \begin{exe}
\ex \label{ex:aZWG.akWra}
\gll aʑɯɣ a-kɯ-ra ci tu tɕe nɯ `ɣa' tɤ-ti ra \\
\textsc{1sg:gen} \textsc{1sg.poss}-\textsc{nmlz}:S/A-have.to \textsc{indef} exist:\textsc{fact} \textsc{lnk} \textsc{dem} yes \textsc{imp}-say have.to:\textsc{fact} \\
\glt `There is one thing I need, and you have to say `yes' to it.' (140429 ingwa wangzi, 47)
\end{exe}

 \begin{exe}
\ex \label{ex:aZWG.kWra}
\gll  aʑɯɣ kɯ-ra me \\
\textsc{1sg:gen} \textsc{nmlz}:S/A-have.to not.exist:\textsc{fact} \\
\glt `I don't need anything.' (2005slobdpon2, 275)
\end{exe}

 \begin{exe}
\ex \label{ex:aZWG.Wlu}
\gll aʑɯɣ ɯ-lu ra \\
\textsc{1sg:gen} \textsc{3sg.poss}-milk have.to:\textsc{fact} \\
\glt `I need its milk.' (02-deluge2012, 12)
\end{exe}

\subsection{Inalienably possessed nouns} \label{sec:inalienably.possessed}
Inalienably possessed nouns (IPN) differ from non-inalienably possessed nouns (NIPN) in that they require the presence of a possessive prefix. 

When the possessive prefix is definite, IPNs are not formally distinguishable from NIPNs; for instance, \japhug{a-pi}{my elder sibling} and \japhug{a-mbro}{my horse} both take the 1\sg{} \forme{a-} prefix and no direct clue indicates that the first noun is IPN and that the second one is NIPN.

The citation form however differs between IPN and NIPN: the former must take an indefinite possessor prefix (or in some cases a 3\sg{} \forme{ɯ-}), while the latter can occur without possessive prefix, as for instance \japhug{tɤ-pi}{elder sibling} (with the indefinite \forme{tɤ-}; the bare root $\dagger$\forme{pi} is not a correct form) vs \japhug{mbro}{horse} (without prefix).

IPN are divided into four classes depending on their citation form. There are three distinct forms for the indefinite possessor prefix (\forme{tɯ-}, \forme{tɤ-}, \forme{ta-}) whose distribution is not completely predictable on the basis of phonology or semantics (though some generalizations are provided below). In addition, some IPN always only take definite possessive prefixes. The contrast between these four classes is neutralized when the noun takes a definite possessor prefix (unlike in Situ, see \citealt[168-169]{linxr93jiarongen} and \citealt[118-119]{prins16kyomkyo}).

The most common allomorph of the indefinite possessor prefix is \forme{tɯ-}. IPN which select this allomorph, such as \japhug{tɯ-jaʁ}{hand}, have identical indefinite and generic possessor forms (see section \ref{sec:indef.genr.poss}).

The allomorph \forme{tɤ-} is also very common, in particular with kinship terms and some body parts (see \ref{sec:body.part} and \ref{sec:kinship}). The form \forme{ta-} is mainly a phonological variant of \forme{tɤ-}, occurring mainly with nouns whose stem begins with a uvular such as \japhug{ta-ʁrɯ}{horn} or \japhug{ta-ʁi}{younger sibling}. The contrast between \ipa{ɤ} and \ipa{a} is very difficult to perceive before uvulars with some speakers (see section XXX), and the transcription adopted in this grammar (and the online corpus and dictionary) is based on the slow syllable-by-syllable pronunciation of these words by Tshendzin. Two IPNs, however, \japhug{ta-ma}{work} and \japhug{ta-mar}{butter}, have the \forme{ta-} allomorph with an initial \forme{m-}, probably originally due to vowel harmony (see XXX).

The minimal pair between \japhug{tɤ-ma}{mother} and \japhug{ta-ma}{work} shows that this vowel contrast, however marginal, is distinctive, and that even if the two allomorphs \forme{tɤ-} and \forme{ta-} were originally phonologically conditioned, it is not the case any more in Kamnyu Japhug.

Some IPN never occur with indefinite possessor prefix, for instance \japhug{ɯ-tʰoʁ}{ground} is only attested with the 3\sg{} \forme{ɯ-} prefix. For some IPNs, the indefinite possessor form is difficult to elicit and in case of doubt the third singular form is given in the dictionary \citet{jacques16japhug} (for instance \japhug{ɯ-mdoʁ}{colour}). Future investigations may reveal an indefinite possessor form for some of these nouns.

When denominal verbs are derived from IPNs, the vocalism of the denominal prefix tends to be the same as that of the  indefinite possessor prefix (for instance \japhug{tɤ-βɟu}{mattress} \fl{} \japhug{nɤβɟu}{use as a mattress}, not $\dagger$\forme{nɯβɟu}), though there are exceptions (\japhug{tɯ-rpaʁ}{shoulder} \fl{} \japhug{mɤrpaʁ}{carry on the shoulder}), as discussed in section XXX.

\subsubsection{Conversion from NIPN to IPN} \label{sec:nipn.to.ipn}
NIPNs can be converted to IPNs  without phonological alternation. However, note the case of \japhug{ɯ-ʁle}{reputation} from \japhug{qale}{wind} with reduced form \forme{ʁ-} of the class prefix \forme{qa-}, as some second members of compounds (see \ref{sec:second.member.alternation} and \ref{sec:class.prefixes}).

\subsubsection{Body parts} \label{sec:body.part}
The great majority of body parts are IPN with the indefinite possessor \forme{tɯ-}. These include native words, but also borrowings from Tibetan such as \japhug{tɯ-qʰoχpa}{organs, state of mind} from Tibetan \tibet{ཁོག་པ་}{kʰog.pa}{insides} (see \ref{sec:uvular.harmony} on the phonology of this word).

Body parts IPN selecting the prefix \forme{tɤ-} are mainly liquids from the body such as \japhug{tɤ-se}{blood}, \japhug{tɤ-spɯ}{pus} and \japhug{tɤ-lu}{milk} (though some liquids also take the prefix \forme{tɯ-}, for instance \japhug{tɯ-ɕtʂi}{sweat}), hair (\japhug{tɤ-rme}{hair, fur}, \japhug{tɤ-kɤrme}{hair (head)}) and some body parts of animals (\japhug{tɤ-jme}{tail}, \japhug{tɤ-ŋkɯ}{pig skin}, \japhug{tɤ-rkʰom}{feather rachis}).

Parts of plants on the other hand mainly have the prefix \forme{tɤ-}, as \japhug{tɤ-jwaʁ}{leaf}, \japhug{tɤ-tsrɯ}{sprout}, \japhug{tɤ-zrɤm}{root} etc.

NIPN body parts are rare; examples include \japhug{qame}{mole} and \japhug{qambɣo}{earwax}, with a \forme{qa-} class prefix (see section \ref{sec:class.prefixes}).

\subsubsection{Kinship terms} \label{sec:kinship}
The great majority of kinship terms select the indefinite possessor prefix \forme{tɤ-} or \forme{ta-} (see section XXX for a description of the kinship system). The only kinship terms in \forme{tɯ-} are \japhug{tɯ-me}{daughter} (but this form is not attested in the text corpus) and \japhug{tɯ-lɤt}{second sibling}; however, the \forme{tɯ-} prefix in the latter word is becoming non-analyzable, as shown by examples such as (\ref{ex:tWlAt}). 

\begin{exe}
\ex \label{ex:tWlAt}
\gll  nɯ-me tɯ-lɤt nɯ ɲɤ-mbi-nɯ \\
\textsc{3pl.poss}-daughter \textsc{indef.poss}-second.sibling \textsc{dem} \textsc{ifr}-give-\textsc{pl} \\
\glt `They gave him their second daughter.' (2002qaCpa, 40)
\end{exe}

\subsubsection{Property nouns} \label{sec:property.nouns}
Property nouns are a subclass of IPN that designate a entity that possesses a particular (mainly derogative) characteristic. They generally follow another noun as in (\ref{ex:penzi.WpW}) and (\ref{ex:kha.WNqra}). In the /noun+property noun/ phrase, the latter is the syntactic head but semantically modifies the former (see section XXX on the various property modifiers in Japhug).  

\begin{exe}
\ex \label{ex:penzi.WpW}
 \gll <penzi> ɯ-pɯ, sɤlaŋpʰɤn ɯ-pɯ jamar ɲɯ-wxti cʰa  \\
 basin \textsc{3sg.poss}-little.one   basin \textsc{3sg.poss}-little.one  about \textsc{ipfv}-be.big can:\textsc{fact} \\
 \glt `It can grow about as big as a little basin.' (18-NGolo, 48)
\end{exe}

\begin{exe}
\ex \label{ex:kha.WNqra}
 \gll
kʰa ɯ-ɴqra tɕe znde ɯ-mbe ma tʰam kɯ-tu me. \\
house \textsc{3sg.poss}-broken.one lnk wall \textsc{3sg.poss}-old.one apart.from now \textsc{nmlz}:S/A-exist not.exist:\textsc{fact} \\ 
\glt `Now there is nothing (there), apart from a ruin and old walls. (140522 tshupa, 58)
\end{exe}

These phrases can be turned into compounds made of the first noun and a quasi-suffix corresponding to the property noun, as there is a one-to-one correspondence between them and the diminutive and derogative suffixes described in section \ref{sec:diminutive} and \ref{sec:derogative} (Table \ref{tab:property.nouns}). In the case of \forme{sɤlaŋpʰɤn ɯ-pɯ}  from example (\ref{ex:penzi.WpW}) for instance, it is possible to say \japhug{sɤlaŋpʰɤn-pɯ}{little basin} as one word. In some cases the corresponding noun has \textit{status constructus} on the first element, as in \japhug{kʰɤɴqra}{ruin} from \japhug{kʰa}{house} and \japhug{ɯ-ɴqra}{broken one}, a form which occurs in (\ref{ex:khANqra}), in the same text as  (\ref{ex:kha.WNqra}) (referring to the same house). The opposite however is not always possible; for instance, lexicalized diminutives like \japhug{staχpɯ}{pea} from \japhug{stoʁ}{broad bean} cannot be turned into a phrase with \japhug{ɯ-pɯ}{little one} as second element.

\begin{exe}
\ex \label{ex:khANqra}
 \gll tɕe nɯ tɤtsoʁsta nɯnɯ kʰɤɴqra ɕti tʰam tɕe kɯ-rɤʑi me \\
\textsc{lnk} \textsc{dem} place.name \textsc{dem} ruin be:\textsc{affirm}:fact  now \textsc{lnk} \textsc{nmlz}:S/A-stay not.exist:\textsc{fact} \\
\glt `Now Tatsogsta (`the place of silverweed') is a ruin, nobody lives there.' (140522 tshupa, 56)
\end{exe}

\begin{table}
\caption{Property nouns and corresponding quasi-suffixes} \label{tab:property.nouns}
\begin{tabular}{l|ll}
\lsptoprule
Property Noun & Suffix& \\
\midrule
\japhug{ɯ-pɯ}{little one} & \forme{-pɯ} &diminutive \\
\japhug{ɯ-ɴqra}{broken one} &  \forme{-ɴqra} &derogative \\
\japhug{ɯ-do}{old one} &  \forme{-do} & \\
\japhug{tɤ-mbe}{old thing} &  \forme{-mbe} & \\
\japhug{ɯ-rqɯ}{cold thing} &  \forme{-rqɯ} & \\
\lspbottomrule
\end{tabular}
\end{table}

The property nouns \japhug{ɯ-do}{old one}  and \japhug{tɤ-mbe}{old thing} differ in that the former one is used for living things (including animals and plants), while the second occurs with inanimate objects. The quasi-suffix \forme{-rqɯ} is mainly used in \japhug{tɯ-cirqɯ}{cold water}.

Property nouns are not commonly used with an indefinite possessor prefix; in attested examples, it is always \forme{tɤ-}. They origins are diverse: \japhug{ɯ-pɯ}{little one} derives from \japhug{tɤ-pɯ}{offspring, young} (see \ref{sec:diminutive}), while \japhug{tɤ-mbe}{old thing} and \japhug{ɯ-do}{old thing}  originate  from \japhug{mbe}{be old} and \japhug{do}{be old (of plants)} by deverbal derivation (section XXX). Some \forme{tɤ-} prefixed nouns of verbal origin like \japhug{tɤkʰe}{idiot, fool} (from \japhug{kʰe}{be stupid}) may come from former property nouns.
 
\subsubsection{Alienabilization of IPN} \label{sec:alienabilization}
 It is possible to turn an IPN  into an NIPN one by adding a definite possessive prefix before the indefinite one; this is the only case of possessive prefix stacking in Japhug. This process is very productive, and better illustrated by minimal pairs; the following examples involve the IPNs \japhug{tɯ-ci}{water}, \japhug{tɤ-lu}{milk} and \japhug{tɤ-muj}{feather}.
 
The noun \japhug{tɯ-ci}{water} with a definite possessor (\japhug{ɯ-ci}{its juice/water}) refers either to the juice of a plant, or to water in which a plant has been soaked as in (\ref{ex:Wci})

  \begin{exe}
\ex \label{ex:Wci}
 \gll  ɯʑo tɯ-ci kɯ-sɤ-ɕke ɯ-ŋgɯ pjɯ́-wɣ-ɣɤ-la, tɕe nɯ ɣɯ ɯ-ci ɯ-ŋgɯ nɯtɕu tɯ-mi pjɯ́-wɣ-ɣɤ-la tɕe nɯnɯ, χtɕoŋ nɯ ɲɯ-pʰɤn ɲɯ-ti-nɯ ri, \\
\textsc{3sg}  \textsc{indef.poss}-water \textsc{nmlz}:S/A-\textsc{deexp}-burn \textsc{3sg}-in \textsc{ipfv}-\textsc{inv}-\textsc{caus}-soak \textsc{lnk} \textsc{dem} \textsc{gen} \textsc{3sg.poss}-water \textsc{3sg}-in \textsc{dem:loc}  \textsc{genr.poss}-foot  \textsc{ipfv}-\textsc{inv}-\textsc{caus}-soak \textsc{lnk} \textsc{dem}, rheumatism \textsc{dem} \textsc{sens}-be.efficient \textsc{sens}-say-\textsc{pl} \textsc{lnk}  \\
 \glt `One puts it in hot water, and then one puts one's feet in that water, and it efficient against rheumatism, they say.' (20-sWrna, 144)
 \end{exe}

The alienabilized form \japhug{ɯ-tɯ-ci}{its water}, as in (\ref{ex:WtWci}), is used to talk about water given to an animal to drink, or water (artificially irrigated or not) absorbed by a plant.

 \begin{exe}
\ex \label{ex:WtWci}
 \gll  tɕeri ɯ-tɯ-ci wuma ʑo na-ʁzi tɕe, ɯ-tɯ-ci nɯ mɯ-pjɯ-mbrɤt ɲɯ-ra. \\
 but \textsc{3sg.poss}-\textsc{indef.poss}-water really \textsc{emph} \textsc{trop}-need:\textsc{fact} \textsc{lnk} \textsc{3sg.poss}-\textsc{indef.poss}-water dem \textsc{neg-ipfv-anticaus}:break \textsc{sens}-have.to \\
 \glt  `But it needs water a lot, it needs to have water continuously.'  (07-Zmbri, 11)
 \end{exe}
 
 When a definite possessor is present on the noun \japhug{tɤ-lu}{milk} in a form such as \japhug{ɯ-lu}{her milk}, that prefix refers to the animal producing the milk, as in (\ref{ex:Wlu}).
 
 \begin{exe}
\ex \label{ex:Wlu}
 \gll 
tɤ-pi kɯ-wxti nɯ kɯ nɯŋa ɣɯ ɯ-lu nɯ cʰondɤre  ɯ-ɕa nɯ to-nɯ-ndo. \\
\textsc{indef.poss}-elder.sibling \textsc{nmlz}:S/A-be.big \textsc{dem} \textsc{erg} cow \textsc{gen} \textsc{3sg.poss}-milk \textsc{dem} \textsc{comit} \textsc{3sg.poss}-meat \textsc{dem} \textsc{ifr}-\textsc{auto}-take \\
\glt `The elder brother took the cow's milk and meat.' (02-deluge2012, 19)
 \end{exe}

The form \japhug{ɯ-tɤ-lu}{his/its milk} with alienabilization is used on the other hand when indicating the person or animal drinking the milk, as in (\ref{ex:WtAlu}).

  \begin{exe}
\ex \label{ex:WtAlu}
 \gll tɕe ɯ-tɤ-lu pjɯ́-wɣ-rku tɕe nɯnɯ pjɯ-tsʰi qʰe, tɯ-sŋi tɕe tɯ-kʰɯtsa jamar tɯ-rdoʁ kɯ pjɯ-tsʰi ɲɯ-cʰa. \\
\textsc{lnk} \textsc{3sg.poss}-\textsc{indef.poss}-milk \textsc{ipfv}:\textsc{down}-\textsc{inv}-put.in  \textsc{lnk} \textsc{dem} \textsc{ipfv}-drink \textsc{lnk} one-day \textsc{lnk} one-bowl about one-\textsc{cl} \textsc{erg} \textsc{ipfv}-drink \textsc{sens}-can \\
\glt `People pour drink for it (the cat) to drink, and it drinks it, it can drink about a bowl of milk per day.' (21-lWLU, 45)
  \end{exe}

The IPN \japhug{tɤ-muj}{feather} takes as its possessor a bird (or a bird body part such as `wings'), as in (\ref{ex:Wmuj}).

    \begin{exe}
\ex \label{ex:Wmuj}
 \gll   jinde tɕe ɯ-kɯ-sat koŋla maŋe tɕe, nɯ qarma ɯ-muj kɯnɤ tɯ-jaʁ mɯ́j-ɣi wo \\
 nowadays \textsc{lnk} \textsc{3sg.poss}-\textsc{nmlz}:S/A-kill completely not.exist:C \textsc{lnk} \textsc{dem} crossoptilon \textsc{3sg.poss}-feather also \textsc{genr.poss}-hand \textsc{neg}:\textsc{sens}-come \textsc{sfp} \\
 \glt `Nowadays, nobody kills them, and one cannot get crossoptilon feathers.' (23-qapGAmtWmtW, 173)
  \end{exe}

Its alienabilized form, such as \japhug{ɯ-tɤ-muj}{his feather} in (\ref{ex:WtAmuj}), is used when the feather is detached from the body of the bird, and belongs to a human.
  
\begin{exe}
\ex \label{ex:WtAmuj}
 \gll tɤtɕɯpɯ kɯ-xtɕi nɯ ɣɯ ɯ-tɤ-muj nɯ li ɯ-tʰoʁ nɯtɕu pjɤ-nɯ-jɣɤt  \\
 boy:\textsc{dim} \textsc{nmlz}:S/A-be.small dem gen \textsc{3sg.poss}-\textsc{indef.poss}-feather \textsc{dem} again \textsc{3sg.poss}-ground \textsc{dem:loc} \textsc{ifr}:\textsc{down}-\textsc{auto}-go.back  \\
 \glt `The younger boy's feather fell back on the ground again.' (140510 sanpian yumao, 68)
\end{exe}
 
As the examples above show, alienabilized IPNs mainly occur to refer to body parts or bodily fluids that have been removed from an animal, and are used or owned by a human or another animal on which they do not grow. The referent marked by the possessive prefix can be beneficiary as in (\ref{ex:WtAlu}) or possessor as in (\ref{ex:WtAmuj}).
 
 A distinct process of alienabilization is the loss of possessive prefixes in derivations, for instance with the privative suffix (\ref{sec:privative}) or with the derogative derivation (\ref{sec:derogative}).
 
\subsubsection{Frozen indefinite possessors} \label{sec:frozen.indef}
NIPNs with a disyllabic root whose first element is \forme{tɯ-} or \forme{tɤ-}, with the exception of loanwords such as \japhug{tɯrsa}{grave} (from \tibet{དུར་ས་}{dur.sa}{grave}) or counted nouns (see section XXX), are mainly ancient IPN whose indefinite possessive prefix \forme{tɯ-} has become frozen and reanalyzed as part of the root. Comparison with other Gyalrong languages can demonstrate that such reanalysis took place in Japhug.

For instance, the noun \japhug{tɯrme}{man} is NIPN in Japhug, but in Situ the 3\sg{} form of \forme{tə-rmî} `man' is \forme{wǝ-rmî}  (\citealt[183;197]{lin09phd}), showing that \forme{tə-} is the indefinite possessive prefix, cognate of Japhug \forme{tɯ-}. This shift may be due to the fact that the 3\sg{} form is used in Situ in constructions where the non-possessed form is preferred in Japhug, such as in prenominal relatives (\citealt[190]{lin09phd}), and was therefore less prone to lexicalization.

The noun \japhug{tɤjmɤɣ}{mushroom} is NIPN, but the stem \forme{jmɤɣ-} appears as first element of compounds such as \japhug{jmɤɣni}{russula}, suggesting that it is a former IPN occurring without its indefinite possessor prefix in this compound (see \ref{sec.compounds.first.other.alternations}). The status of \japhug{tɤjmɤɣ}{mushroom} as a former IPN is less surprising if one takes into account the likely etymological relationship with Chinese \zh{帽} \forme{mawH} `hat' (from \forme{*mˤuk-s}).  If the noun for `mushroom' in Japhug and other Gyalrongic languages comes from `hat' (cf Breton \forme{tog touseg} `toad hat' for `mushroom'), it is expected that it would become an IPN (like \japhug{tɤ-rte}{hat}), and for the indefinite possessor \forme{tɤ-} to become frozen after the noun ceases to be a term for head covers.
 
\subsubsection{Unusual IPNs in Japhug} \label{sec:earth.IPN}
While IPN membership of nouns such as body parts of kinship terms is expected from a crosslinguistic point of view, Japhug has IPNs for nouns referring to natural entities such as \japhug{tɯ-mɯ}{sky}, \japhug{tɯ-ci}{water} or \japhug{ɯ-tʰoʁ}{ground}, a highly unusual fact. Explanations should be sought in the etymology of these words, and solved on an item per item basis. As an example of the type of accounts that historical linguistics can provide for such nouns, I discuss here the case of \japhug{ɯ-tʰoʁ}{ground}.

The Japhug noun \japhug{ɯ-tʰoʁ}{ground} cannot take any possessive prefix other than 3\sg{} \forme{ɯ-}, not event he indefinite possessor prefix. It has no known cognates in other Gyalrongic languages, but is a perfect match for a Tibetan word with the shape \forme{tʰog} (compare the other borrowed noun \japhug{tʰoʁ}{thunder} from Tibetan \tibet{ཐོག་}{tʰog}{thunder}). The etymology of this word requires a four steps scenario.

First, Japhug borrowed the Tibetan relator noun  \tibet{ཐོག་ཏུ་}{tʰog(tu)} `on' as \forme{ɯ-tʰoʁ} *`on' (not attested), adding a third person possessive prefix like all relator nouns (see section XXX). This relator noun was in competition with the existing native equivalent \japhug{ɯ-taʁ}{on}.\footnote{It is not surprising in Japhug to have several competing relator noun for the same functional slot; the same is true of the dative \ipa{ɯ-ɕki} and \ipa{ɯ-pʰe}, see seciton XXX. }
  
  Second, it  became restricted to the collocation \forme{*sɤtɕʰa ɯ-tʰoʁ zɯ} `on the ground' (not attested), with the native locative \forme{zɯ} and the  noun of Tibetan origin \japhug{sɤtɕʰa}{earth}.
  
    Third, the collocation  \forme{*sɤtɕʰa ɯ-tʰoʁ zɯ}`on the ground', becoming tautological, was reduced to \japhug{ɯ-tʰoʁ zɯ}{on the ground} (attested).
 
 Fourth, the noun \japhug{ɯ-tʰoʁ}{ground} was created by backformation from the locative phrase \forme{ɯ-tʰoʁ zɯ} `on the ground'. The fact that the locative postposition \ipa{zɯ} is always optional (section XXX) undoubtedly made this step easier.  Thus, Japhug attests an example of degrammation (see \citealt[135]{norde09degrammaticalization}) from a relator noun meaning `on' (with or without motion) to a common noun meaning `ground'. 
 
\subsection{Indefinite vs generic possessor} \label{sec:indef.genr.poss}
The generic possessive prefix \forme{tɯ-} is formally identical to the indefinite possessor prefix of some IPNs, but must be strictly distinguished from it. Four criteria can be used to determine if a \forme{tɯ-} prefix is generic, rather than indefinite.

First, the generic possessor prefix appears on NIPNs, as in example (\ref{ex:tWlaXtCha}) with \japhug{tɯ-kʰa}{one's house} and \japhug{tɯ-laχtɕʰa}{one's things}, the generic forms of \japhug{kʰa}{house} and \japhug{laχtɕʰa}{thing}.

\begin{exe}
\ex \label{ex:tWlaXtCha}
\gll tɕe  	aʁɤndɯndɤt  	ʑo  	ku-zo  	qhe  	ɯ-qe  	ku-lɤt  	qʰe	wuma  ʑo  	tɯ-kʰa  	cʰo  	tɯ-laχtɕʰa  	ra  	sɯ-ɴqʰi.  \\
\textsc{lnk} everywhere \textsc{emph} \textsc{ipfv}-land \textsc{lnk} \textsc{3sg.poss}-feces \textsc{ipfv}-throw \textsc{lnk} really \textsc{emph} \textsc{genr.poss}-house \textsc{comit} \textsc{genr.poss}-thing \textsc{pl} \textsc{caus}-be.dirty:\textsc{fact} \\
\glt `(Flies) land everywhere, shit on it and make one's houses and things dirty.' (25 akWzgumba, 59)
\end{exe}

Second, the generic \forme{tɯ-} occurs on IPNs that normally select the \forme{tɤ-} indefinite possessive prefix, such as \japhug{tɯ-rɟit}{one's child} and \japhug{tɯ-rpɯ}{one's maternal uncle} in examples (\ref{ex:tWrJit}) and (\ref{ex:tWrpW}), by contrast with the citation forms  \japhug{tɤ-rɟit}{child} and \japhug{tɤ-rpɯ}{maternal uncle}.

\begin{exe}
\ex \label{ex:tWrJit}
\gll nɯ 	kɯ-fse 	tɕe 	tɯʑo 	tɯ-rɟit 	kɯnɤ 	ʑa 	mɤ-sci 	tu-ti-nɯ \\
\textsc{dem} \textsc{nmlz}:S-be.like \textsc{lnk} \textsc{genr} \textsc{genr.poss}-child also early \textsc{neg-fact}:be.born \textsc{ipfv}-say-\textsc{pl} \\
\glt `People say that in this way, one's child will be born late.' (27 qartshaz, 111)
\end{exe}

\begin{exe}
\ex \label{ex:tWrpW}
\gll  tɯ-rpɯ 	ɯ-rɟit 	ɯ-ɕki 	tɕe 	tɕe 	``a-rpɯ a-ɬaʁ" 	tu-kɯ-ti 	ŋu. \\
\textsc{genr.poss}-uncle \textsc{3sg.poss}-offspring \textsc{3sg-dat} \textsc{lnk} \textsc{lnk} \textsc{1sg.poss}-uncle \textsc{1sg.poss}-aunt \textsc{ipfv-genr}-say  be:\textsc{fact} \\
\glt `One has to say `my maternal uncle, my maternal aunt to one's maternal uncle's sons and daughters.' (140425 kWmdza01, 69)
\end{exe}

The used of the generic possessive \japhug{tɯ-rpɯ}{one's maternal uncle} in (\ref{ex:tWrpW}) can be contrasted with the indefinite possessed form with \forme{tɤ-} in example (\ref{ex:tArpW}).

\begin{exe}
\ex  \label{ex:tArpW}
\gll
nɤʑo 	tɤ-rpɯ 	ɯ-rɟit 	a-pɯ-tɯ-ŋu, 	tɕe 	tɕe 	aʑo 	kɯ 	`a-rpɯ' 	tu-ti-a 	kɯ-ra.  \\
\textsc{2sg} \textsc{indef.poss}-uncle \textsc{3sg.poss}-offspring \textsc{irr-ipfv}-2-be \textsc{lnk} \textsc{lnk} \textsc{1sg} \textsc{erg}  \textsc{1sg.poss}-uncle \textsc{ipfv}-say-\textsc{1sg} \textsc{nmlz:S/A}-have.to  \\
\glt `If you are the maternal uncle's son, (and I am the nephew) I have to say `my uncle' (to you).'  (hist140425 kWmdza, 114)
\end{exe}

Third, in the case of IPNs whose indefinite possessive is \forme{tɯ-}, such as \japhug{tɯ-mtɕʰi}{mouth}, the indefinite and generic forms are homophonous, but are nevertheless distinguishable. In the case of a generic form the generic pronoun \japhug{tɯʑo}{one} (\ref{sec:genr.pro}) can always be added as in (\ref{ex:tWZo.tWmtChi}). 

\begin{exe}
\ex  \label{ex:tWZo.tWmtChi}
\gll tɯʑo sɤz kɯ-mna, kɯ-ɤʑɯχtso ra a-pɯ-ŋu nɤ,  tɯʑo tɯ-mtɕʰi maŋtaʁ nɯtɕu ɲɯ-ɬoʁ ŋu. \\
\textsc{genr} \textsc{comp} \textsc{nmlz}:S/A-be.better \textsc{nmlz}:S/A-be.clean \textsc{pl} \textsc{irr}-\textsc{ipfv}-be \textsc{lnk}  \textsc{genr} \textsc{genr.poss}-mouth above \textsc{dem:pl} \textsc{ipfv}-come.out be:\textsc{fact} \\
\glt `If (one uses the bowl of) someone who is cleaner than oneself, the (pimple) will appear above one's mouth.' (25-khArWm, 11)
\end{exe}

Even when the generic pronoun is not present, it is possible to identify generic possessors, as they are coreferent with the generic argument indexed on the verb (by \forme{kɯ-} for S and P, and \forme{wɣɯ-} for A, see section XXX). For instance, in (\ref{ex:genr.tWmtChi}), we know that  the \forme{tɯ-} prefixes in \japhug{tɯ-mtɕʰi}{one's mouth} and  \japhug{tɯ-ɕɣa}{one's teeth} are generic and not indefinite possessor because they refer to the same generic human as the transitive subject of the verbs \japhug{pʰɯt}{take out} and \japhug{ndza}{eat} in the previous clause, marked by the inverse prefix.

\begin{exe}
\ex  \label{ex:genr.tWmtChi}
\gll
tɕe ɲɯ́-wɣ-pʰɯt tɕe tú-wɣ-ndza ŋgrɤl ri, wuma ʑo tɯ-mtɕʰi cʰo tɯ-ɕɣa ra ɲɯ-sɯɣ-ɲaʁ ŋu. \\
\textsc{lnk} \textsc{ipfv}-\textsc{inv}-take.out \textsc{lnk} \textsc{ipfv}-\textsc{inv}-eat be.usually.the.case:\textsc{fact} but really \textsc{emph}  \textsc{genr.poss}-mouth \textsc{comit} \textsc{genr.poss}-tooth \textsc{pl}  \textsc{ipfv}-\textsc{caus}-be.black be:\textsc{fact} \\
\glt `One can pluck it and eat it, but it causes one's mouth and teeth to become black.' (11-qarGW, 70) 
\end{exe}

Fourth, possessed case markers such as the dative \forme{-ɕki} (section XXX) do not have indefinite possessive forms, and therefore if prefixed in \forme{tɯ-}, it will always mark a generic possessor, as in (\ref{ex:tWCki}) -- such forms are often preceded by the generic pronoun \japhug{tɯʑo}{one} anyway.

\begin{exe}
\ex  \label{ex:tWCki}
\gll ma tɯ-ɕki wuma ʑo ʑɣɤ-sɯ-ɤrmbat tɕe núndʐa kʰe tu-ti-nɯ ɲɯ-ŋu. \\
\textsc{lnk} \textsc{genr-dat} really \textsc{emph} \textsc{refl}-\textsc{caus}-be.near:\textsc{fact} \textsc{lnk} for.this.reason stupid:\textsc{fact} \textsc{ipfv}-say-\textsc{pl} \textsc{sens}-be \\
\glt `It easily comes near oneself, so people call it `stupid'.' (23-scuz, 62) 
\end{exe}

\subsubsection{Comparative perspectives} \label{sec:indef.t.comparative}
Indefinite and generic possessive dental stop prefixes are found in all Gyalrong languages (\citealt{jackson98morphology}), but only indirect traces exist in Khroskyabs  (\citealt[155]{lai17khroskyabs}). 

Outside of Gyalrongic, potential cognates of these prefixes include the `relational prefix' \forme{tə-} in Ao (\citealt[84-85]{coupe07mongsen}, as first noticed by \citealt[141-2]{wolfenden29outlines}) and some \forme{d-} or \forme{g-} prefixes in body parts in Tibetan (see \citealt{jacques14snom}).

\section{Status constructus}
The term \textit{status constructus} is used in Gyalrongic linguistics (\citealt{jacques12incorp}, \citealt[163-4]{lai17khroskyabs}) to refer to the non-autonomous form of (mainly nominal, but also verbal and adverbial) roots occurring as non-final element of compounds. The use of this term, adopted from Semitic linguistics,  differs from works such as \citet{creissels06hongrois} or \citet{creissels17construct} in which `construct form' refers to a specific form used that is obligatory on the head noun in specific noun-modifier constructions (including with a possessive marker). Given the fact in that Japhug and other Gyalrongic languages, nominal compounds generally follow modifier-head order (the opposite of Semitic), the form undergoing \textit{status constructus} alternation in Japhug is generally the modifier noun.\footnote{In addition, in Japhug the possessed form of nouns does present morphological alternations (\ref{sec:possessive.paradigm}), except  in exceptional cases (\ref{sec:nipn.to.ipn}).}

In this section, the various types of alternations attested for first or other non-final members of compounds are described, in particular vowel alternation (the most common type). Additionally, exceptional changes to the final members of compounds are discussed in \ref{sec:final.compounds}.

\subsection{Vowel alternations in non-final members of compounds} 
Regular \textit{status constructus} is Japhug applies to open syllables, and involves shift of all vowels to either \ipa{ɯ} or \ipa{ɤ} back unrounded vowels, following the correspondences in Table \ref{tab:sc.regular}.

\begin{table}
\caption{Regular \textit{status constructus} in Japhug} \label{tab:sc.regular}
\begin{tabular}{lllll}
\lsptoprule
Base & SC & Example \\
\ipa{-a} &\ipa{-ɤ} & \\
\ipa{-e} &\ipa{-ɤ} & \japhug{tɕʰemɤpɯ}{little girl} from  \japhug{tɕʰeme}{girl} + \japhug{ɯ-pɯ}{little one} \\
\ipa{-i} &\ipa{-ɯ} & \japhug{ɯ-χtɯrca}{with the others} \\
&&from  \japhug{tɯ-χti}{companion} + \japhug{tɤ-rca}{together with} \\
\ipa{-o} &\ipa{-ɤ} &  \japhug{mbrɤsno}{horse saddle} from  \japhug{mbro}{hrose} + \japhug{tɤ-sno}{saddle}\\
\ipa{-u} &\ipa{-ɤ} & \japhug{tɤ-kɤrme}{head hair} from  \japhug{tɯ-ku}{head} + \japhug{tɤ-rme}{hair} \\
\lspbottomrule
\end{tabular}
\end{table}

There are irregular cases of \ipa{i} changing to \ipa{ɤ}, as in \japhug{qaprɤftsa}{centipede} from \japhug{qapri}{snake} and \japhug{tɤ-ftsa}{nephew}.

\subsection{Other alternations} \label{sec.compounds.first.other.alternations}
Apart from regular vowel change, three types of alternations are observed in non-final member of compounds. First, there are very rare cases of loss of coda, which can be listed as follows:

\begin{itemize}
\item Loss of \ipa{-β}: 

\japhug{ɴqiaβ}{dark side of the mountain}  + \japhug{zwɤr}{mugwort} \fl{}  \japhug{ɴqiazwɤr}{Artemisia sp.}  
\item Loss of  \ipa{-t}: 

\japhug{xtɯt}{be short} + \japhug{rɲɟi}{be long} \fl{} \japhug{xtɯrɲɟi}{length (n)}  

 \japhug{tsʰɤt}{goat} + \japhug{ta-ʁrɯ}{horn} \fl{} \japhug{tsʰɤʁrɯ}{goat horn}  
\item Loss of \ipa{-z}: 

\japhug{qartsʰaz}{deer}  + \japhug{tɯ-ndʐi}{skin} \fl{}  \japhug{qartsʰɤndʐi}{deer hide}  
\item Loss of \ipa{-r}:

 \japhug{zwɤr}{mugwort} + \japhug{wɣrum}{be white} \fl{} \japhug{zwɤɣrum}{Artemisia sp.}  

\japhug{ɕɤr}{night} + \japhug{ɯ-χcɤl}{middle} \fl{}  \japhug{ɕɤχcɤl}{middle of the night}  
\item Loss of \ipa{-ɣ}:

 \japhug{tɤjmɤɣ}{mushroom}  + \japhug{tɯ-sti}{alone}  \fl{}  \japhug{jmɤtɤsti}{species of mushroom}  
 
\japhug{tɯ-mtʰɤɣ}{waist}  + \japhug{rŋgɤβ}{attach} \fl{}  \japhug{tɯ-mtʰɤrɴɢɤβ}{part of the trouser where one can tuck things in} 
\item Loss of \ipa{-ʁ}: 

\japhug{ɕoʁ}{buckwheat} + \japhug{wɣrum}{be white} \fl{}  \japhug{ɕɤɣrum}{buckwheat sp}  

\japhug{paʁ}{pig} + \japhug{tɤ-qa}{foot} \fl{}  \japhug{pɤqa}{stuffed pig feet}  
\end{itemize}

All of these examples are lexicalized, and cannot be predicted by any rule based on phonology (the presence of a cluster in the following element is irrelevant, for instance). Some of them occur with other alternations in the second syllable (cf \ref{sec:second.member.alternation}).

Second, some nouns have reduced forms when occurring as first member of compounds. For instance, the noun \japhug{nɯŋa}{cow} has the form \forme{ŋɤ-} in the compounds \japhug{ŋɤnɯ}{udder} (with \japhug{tɯ-nɯ}{teat}) and \japhug{ŋɤqe}{cow dung} (with \japhug{tɯ-qe}{shit, dung}), which is the regular \textit{status constructus} from a stem \forme{ŋɤ-}. The apparent `loss' of a \forme{nɯ-} element is due to the fact that the noun \japhug{nɯŋa}{cow} is itself an ancient compound comprising \japhug{tɯ-nɯ}{teat}) as first element `the bovid with udders').

Third, some IPNs or alienabilized former IPNs lose their possessive prefix (or frozen indefinite possessive \forme{tɯ-/tɤ-}, see \ref{sec:frozen.indef}), as for instance the noun  \japhug{jmɤrtaʁ}{silverfish}, which comes from \japhug{tɤ-jme}{tail} and \japhug{artaʁ}{be forked} (`forked tail').\footnote{The verb \japhug{artaʁ}{be forked} itself is denominal from \japhug{tɤ-jwaʁ}{branch}.} Its first element \japhug{tɤ-jme}{tail} loses the prefix \forme{tɤ-} and undergoes regular vowel alternation.

Similar examples are particularly common with \japhug{tɯ-xtsa}{shoe}, as mainly parts of the shoes are referred to by NIPN compounds with \forme{xtsɤ-} as first element (\japhug{xtsɤɕna}{tip of the shoe}, \japhug{xtsɤrkɯ}{sides of the shoe} etc).

\subsection{Final member of compounds} \label{sec:final.compounds}
\subsubsection{Loss of possessive prefix}

\subsubsection{Alternations} \label{sec:second.member.alternation}
Morphophonological alternations affecting second members of compounds are very rare in Japhug. Several cases are found with animal nouns with the uvular class prefix \forme{qa-}, which has a variant \ipa{χ-/ʁ-} when occurring as second member of compounds (see \ref{sec:uvular.animal} and \ref{sec:uvular.other}).

Other alternations are restricted to specific lexical items, which are discussed below one by one.

The IPN \japhug{tɯ-mtʰɤrɴɢɤβ}{part of the trouser where one can tuck things in} (a noun whose meaning is better explained by an example sentence like \ref{ex:WmthArNGAB}) is a compound of the noun \japhug{tɯ-mtʰɤɣ}{waist} with the transitive verb \japhug{rŋgɤβ}{attach}, which appears as a uvularized allomorph \forme{-rɴɢɤβ} not attested otherwise; it is unclear why uvularization took place in this word (dissimilation with the coda \ipa{-ɣ} of the previous root is unlikely).

\begin{exe}
\ex \label{ex:WmthArNGAB}
\gll tsʰi tɤ-mda tɕe nɯ ɯʑo ɯ-cʰɤmdɤru nɯ pjɯ-nɯ-rʁe tɕe pjɯ-nɯ-tsʰi, mɯ-na-tsʰi tɕe tɕe li tu-nɯ-χɕoʁ tɕe ɯ-mtʰɤrɴɢɤβ cʰɯ-nɯ-rʁe \\
drink:\textsc{fact} \textsc{pfv}-be.the.moment \textsc{lnk} \textsc{dem} \textsc{3sg} \textsc{3sg.poss}-drinking.straw \textsc{dem} \textsc{ipfv}:\textsc{down}-\textsc{auto}-insert \textsc{lnk} \textsc{ipfv}:\textsc{down}-\textsc{auto}-drink neg-pfv:3\fl3'-drink \textsc{lnk} \textsc{lnk} again \textsc{ipfv}:\textsc{up}-\textsc{auto}-take.out \textsc{lnk} \textsc{3sg.poss}-tuck \textsc{ipfv}:\textsc{downstream}-\textsc{auto}-insert \\
\glt `When it is time to drink, he inserts his straw (into the jar) and drinks from it, and when he does not drink any more, he takes it out and tucks it back into his trousers.' (30-tChorzi, 45)
\end{exe}

The noun \japhug{ftɕɤru}{path in the middle of the fields} is a compound of \japhug{ftɕar}{summer} and \japhug{tʂu}{path} (such paths are made during summer to allow workers to work in the field without damaging the crops). The first element \forme{ftɕɤ-} is the \textit{status constructus} of \forme{ftɕar} (with loss of final consonant) and the form \forme{-ru} for the second member of the compound is a clue that \forme{tʂu} comes from earlier \forme{*t-ro} with a dental stop+\ipa{r} cluster changing to a retroflex affricate (see section XXX) -- the \forme{*t-} element being prefixal (perhaps a fossilized indefinite possessor prefix).

The noun \japhug{jmɤɣni}{russula} clearly derives from \japhug{tɤjmɤɣ}{mushroom} and \japhug{ɣɯrni}{be red}, but while the loss of the \forme{tɤ-} prefix can be explained (see \ref{sec:frozen.indef}), the form of the second element (without \forme{r-} preinitial) is a mystery.
 

\section{Compounding}
\subsection{Noun-Noun compounds}
\subsection{Verb-Verb compounds}
\subsection{Adverb-Verb compounds}
\subsection{Noun-Verb compounds}
\subsection{Verb-Noun compounds}
Verb-Noun compounds are extremely rare in Japhug, as they are in general in Trans-Himalayan languages other than Chinese.  A possible example is \japhug{ndzɤpri}{brown bear}, compound of \japhug{pri}{bear} and \japhug{ndza}{eat} -- as shown by (\ref{ex:ndzApri}) from a text about bears, it is considered by some native speakers of Japhug as a man eater, though this explanation could be folk-etymology.

\begin{exe}
\ex \label{ex:ndzApri}
\gll tɕe ndzɤpri kɤ-ti nɯ tɕe tɯrme tu-kɯ-ndza ɲɯ-ŋgrɤl \\
\textsc{lnk} brown.bear \textsc{inf}-say \textsc{dem} \textsc{lnk} people \textsc{ipfv}-\textsc{genr}:S/P-eat \textsc{sens}-be.usually.the.case \\
\glt `It eats people, so is it called \forme{ndzɤpri}.' (21-pri, 94)
\end{exe} 

We find several examples of nominal compounds whose structure is \forme{tɤ-}+Verb+Noun, where the verb is an adjectival stative verb. This category includes \japhug{tɤqiaβjmɤɣ}{lactarius sp.}, literally `bitter mushroom', from the noun \japhug{tɤjmɤɣ}{mushroom} and the verb \japhug{qiaβ}{be bitter}, and \japhug{tɤmbextsa}{type of shoes} from \japhug{tɯ-xtsa}{shoe} and \japhug{mbe}{be old}. These should not be analyzed as Verb-Noun compounds however, as the first element originates from a nominalized form of the verb (such as \japhug{tɤ-mbe}{old thing}, see sections \ref{sec:property.nouns} and XXX on this derivation): they rather are a subtype of Noun-Noun compounds.

\section{Noun class prefixes} \label{sec:class.prefixes}
Noun class prefixes are prefixal elements that occur in some nouns, whose root cannot occur on its own. Uvular \forme{qa-/χ-/ʁ-} and velar \forme{kɯ-/x-/ɣ-} prefixes are attested, and occur on animal names, plant names and nouns referring to traditional objects.

\subsection{Uvular animal name prefix} \label{sec:uvular.animal}
The uvular animal prefix has a plene form \forme{qa-} (Table \ref{tab:animal.qa}) and a reduced allomorph \forme{χ-/ʁ-}, attested in a few names like \japhug{ʁmbroŋ}{wild yak}, \japhug{rtɕʰɯrjɯ}{caterpillar} and \japhug{tɕʰɯχpri}{salamander}.

Note that \japhug{ʁmbroŋ}{wild yak} is a borrowing from Tibetan \tibet{འབྲོང་}{ⁿbroŋ}{wild yak}, a fact that possibly suggests that the \forme{χ-/ʁ-} prefix has some degree of productivity (see \citealt{jacques14snom}). 

The noun \japhug{qapɣɤmtɯmtɯ}{hoopoe} is clearly a compound containing the \textit{status constructus} of \japhug{pɣa}{bird} and the reduplicated form of the noun \japhug{ɯ-mtɯ}{crest}, to which the class prefix \forme{qa-} has been added.

The allomorph \forme{qa-} is reduced to its non-syllabic variants \forme{χ-/ʁ-} when the prefixed noun occurs as second member of compound. The nouns \japhug{tɕʰɯχpri}{salamander} and \japhug{rtɕʰɯrjɯ}{caterpillar} are examples of this reduction. The former is a compound of \forme{tɕʰɯ-} (a syllable borrowed  from Tibetan \tibet{ཆུ་}{tɕʰu}{water}) and \forme{-χpri}, a variant of \japhug{qapri}{snake}. The latter comprises the syllable \forme{rtɕʰɯ-}, \textit{status constructus} of the unprefixed root of \japhug{tɯrtɕʰi}{type of vegetable (\zh{酸酸菜})}, and the second \forme{-ʁjɯ} is the reduced variant of \japhug{qajɯ}{worm}.

\begin{table}
\caption{Animal name \forme{qa-} prefix} \label{tab:animal.qa}
\begin{tabular}{l|l}
 \lsptoprule 
\japhug{qacʰɣa}{fox} &	\japhug{qandʐe}{earthworm} \\
\japhug{qaɕɣi}{big fly} &	\japhug{qandʐi}{anadromous fish} \\
\japhug{qaɕpa}{frog} &	\japhug{qandʑɣi}{fox} \\
\japhug{qajdo}{crow} &	\japhug{qaɲi}{mole} \\
\japhug{qajtʂʰa}{aegyptius monachus} &	\japhug{qapar}{dhole} \\
\japhug{qajɯ}{worm} &	\japhug{qapɣɤmtɯmtɯ}{hoopoe} \\
\japhug{qaɟy}{fish} &	\japhug{qapri}{fox} \\
\japhug{qala}{rabbit} &	\japhug{qarma}{crossoptilon} \\
\japhug{qaliaʁ}{eagle} &	\japhug{qartsʰaz}{deer} \\
\japhug{qambalɯla}{butterfly} &	\japhug{qartsʰi}{deer} \\
\japhug{qambrɯ}{male yak} &	\japhug{qaʑo}{sheep} \\
\japhug{qamtɕɯr}{shrew} &	\\
%\japhug{qacʰɣa}{fox} \\
%\japhug{qaɕɣi}{big fly} \\
%\japhug{qaɕpa}{frog} \\
%\japhug{qajdo}{crow} \\
%\japhug{qajtʂʰa}{aegyptius monachus} \\
%\japhug{qajɯ}{worm} \\
%\japhug{qaɟy}{fish} \\
%\japhug{qala}{rabbit} \\
%\japhug{qaliaʁ}{eagle} \\
%\japhug{qambalɯla}{butterfly} \\
%\japhug{qambrɯ}{male yak} \\
%\japhug{qamtɕɯr}{shrew} \\
%\japhug{qandʐe}{earthworm} \\
%\japhug{qandʐi}{anadromous fish} \\
%\japhug{qandʑɣi}{fox} \\
%\japhug{qaɲi}{mole} \\
%\japhug{qapar}{dhole} \\
% \japhug{qapɣɤmtɯmtɯ}{hoopoe} \\
%\japhug{qapri}{fox} \\
%\japhug{qarma}{crossoptilon} \\
%\japhug{qartsʰaz}{deer} \\
%\japhug{qartsʰi}{deer} \\
%\japhug{qaʑo}{sheep} \\
 \lspbottomrule
\end{tabular}
\end{table}

\subsection{Velar animal name prefix} 
While most nouns beginning in \forme{kɯ-} are frozen participles (see section XXX), there is a residue of forms which cannot be accounted as deverbal nouns. Table \ref{tab:animal.kW} presents animal names that are not derivable from any verb root, and appear to bear a \forme{kɯ-} class prefix, distinct from the uvular one.
 
\begin{table}
\caption{Animal name \forme{kɯ-} prefix} \label{tab:animal.kW}
\begin{tabular}{ll}
 \lsptoprule 
\japhug{kɯɕpaz}{marmot} \\
\japhug{kɯjka}{pyrrhocorax} \\
\japhug{kɯmu}{tetraogallus tibetanus} \\
\japhug{kɯpɤz}{type of bug} \\
\japhug{kɯrtsɤɣ}{snow leopard} \\
\japhug{kɯrŋi}{beast} \\
 \lspbottomrule
\end{tabular}
\end{table} 

There is a handful of nouns with reduced allomorphs \forme{ɣ-}, \forme{x-} or even metathesized as \forme{βɣ-} in some words, corresponding to \forme{kə-} in Situ (see the phonological discussion in \citealt[6]{jacques14antipassive}), including \japhug{xɕiri}{weasel}, \japhug{xtɯt}{wild cat},  \japhug{ɣzɯ}{monkey}, \japhug{ɣni}{flying squirrel}, \japhug{βɣɯz}{badger} and \japhug{βɣɤza}{fly}.

\subsection{Uvular plant name prefix} \label{sec:uvular.plant}
Quite a number of plant names have a uvular class prefix \forme{qa-}, including some both cultivated plants and yet unidentified wild plants, including \japhug{qaɕti}{peach}, \japhug{qaɟɤɣi}{oat}, \japhug{qampʰoʁ}{oak leaves},  \japhug{qandzi}{type of fir}, \japhug{qaʑmbri}{vine}, \japhug{qawɯz}{edelweiss} and many others.
 
\subsection{Other uses of the uvular class prefix} \label{sec:uvular.other}
In addition to animal, plant and body part names, the class prefix \forme{qa-} appears on some tools (\japhug{qajo}{earthen pot}, \japhug{qase}{leather rope}, \japhug{qarɤt}{rake}, 
\japhug{qapi}{flint stone}), names of periods of the year (\japhug{qartsɯ}{winter}, \japhug{qartsɤβ}{harvest}), materials (\japhug{qandʑi}{tin}, \japhug{qambɯt}{sand}) or natural forces like \japhug{qale}{wind}.

The reduced form \forme{ʁ-} of the class prefix occurs with the noun \japhug{qale}{wind} in some compounds such as \japhug{akɯcʰoʁle}{north/east wind} and the abstract IPN \japhug{ɯ-ʁle}{reputation} (and the verbs derived from it, such as \japhug{raʁle}{be polite}).
  
\subsection{Body part noun prefixes}  \label{ex:body.part.prefix}
The identification of class prefixes in body parts mainly rests on comparative evidence. Other Trans-Himalayan languages that preserve clusters such as Tibetan have in some names for body parts cluster that do not match those found in Japhug, for instance \tibet{མཁྲིས་པ་}{mkʰris.pa}{bile} and \tibet{སྐེ་}{ske}{neck} corresponding to the Japhug IPNs \japhug{tɯ-ɕkrɯt}{bile} and \japhug{tɯ-mke}{neck} respectively (see section \ref{sec:body.part}), suggesting that body part class prefixes such as \forme{ɕ-} and \forme{m-} have been added to these words in Gyalrongic and Tibetan independently.

Apart from the \forme{m-} and \forme{ɕ-/ʑ-} class prefixes, some NIPN body parts such as \japhug{qambɣo}{earwax} have a \forme{qa-}  prefix (\ref{sec:body.part}).

The only evidence for a derivational use of these class prefixes in Japhug is the noun \japhug{tɯ-mci}{saliva}, that may be derived from \japhug{tɯ-ci}{water} by addition of the \forme{m-} class prefix.\footnote{The noun \japhug{tɯ-mgɯr}{back} could be another example, but the verb root from which it could be derivable, \japhug{fkur}{carry on the back}, is likely to be a Tibetan loanword and has a different vocalism. }

\section{Nominal derivations}
\subsection{Privative} \label{sec:privative}
The suffix \forme{-lu} can be combined with the \textit{status constructus} form of body part nouns, without possessive prefix, to derive a adjectival noun meaning `...less', `without ...'. Examples attested in the  corpus are indicated in Table \ref{tab:privative.lu}, but this derivation appears to be productive.

\begin{table}
\caption{Privative \forme{-lu} suffix} \label{tab:privative.lu}
\begin{tabular}{l|l}
 \lsptoprule 
\japhug{ta-ʁrɯ}{horn} &\japhug{ʁrɯlu}{hornless} \\
\japhug{tɤ-jme}{tail} &\japhug{jmɤlu}{without tail}  \\
\japhug{tɯ-jaʁ}{hand} &\japhug{jaʁlu}{missing a hand} \\
\japhug{tɯ-ku}{head} &\japhug{kɤlu}{headless} \\
 \lspbottomrule
\end{tabular}
\end{table}

These adjectival nouns can be used as modifiers of other nouns, and are placed after the nouns and before determiners such as demonstratives or numerals, as in (\ref{ex:RrWlu}) and (\ref{ex:jmAlu}).

\begin{exe}
\ex \label{ex:RrWlu}
\gll ʑɤni ɣɯ ftsoʁ ʁrɯlu ci ta-rku-nɯ ɲɯ-ŋu \\
\textsc{3du} \textsc{gen} female.hybrid.yak hornless \textsc{indef} \textsc{pfv}:3\fl3'-put.in-\textsc{pl} \textsc{sens}-be \\
\glt `They gave them a hornless female yak (to take with them back to the husband's home.' (2005-stod, 243)
\end{exe}

Privative nouns are systematically glossed in Japhug with possessor participial relatives in \japhug{kɯ-me}{not having} (see section XXX), as in examples (\ref{ex:kAlu}) and (\ref{ex:jmAlu}).

\begin{exe}
\ex \label{ex:kAlu}
\gll ɯ-taʁ ɯ-mnɯ kɯnɤ kɯ-zri tu-ɬoʁ mɯ́j-cʰa tɕe, nɯ-kɤ-ʁndzɤr ʑo ɲɯ-fse tɕe nɯ ʑmbrɯkɤlu tu-kɯ-ti ŋu. tɕe nɯ ɯ-ku kɯ-me kɤ-ti ɲɯ-ŋu.  kɤlu nɯ ɯ-ku kɯ-me kɤ-ti ɲɯ-ŋu.\\
\textsc{3sg}-on \textsc{3sg.poss}-new.twig also \textsc{nmlz}:S/A-be.long \textsc{ipfv}:\textsc{up}-come.out \textsc{neg:sens}-can \textsc{lnk} \textsc{pfv}-\textsc{nmlz}:P-cut \textsc{emph} \textsc{sens}-be.like \textsc{lnk} \textsc{dem} plant.name \textsc{ipfv}-\textsc{genr}-say be:\textsc{fact} \textsc{lnk}  \textsc{dem} \textsc{3sg.poss}-head \textsc{nmlz}:S/A-not.exist \textsc{inf}-say \textsc{sens}-be headless \textsc{dem} \textsc{3sg.poss}-head \textsc{nmlz}:S/A-not.exist \textsc{inf}-say \textsc{sens}-be\\
\glt `Its new twigs cannot grow very long, and look like they have been sawed short, therefore it is called `headless willow'. `Headless' means `without head'.'(07-Zmbri, 34-36)
\end{exe}

\begin{exe}
\ex \label{ex:jmAlu}
\gll tɕe kɯju jmɤlu nɯnɯ tɯrme ɲɯ-ŋu, ɯ-jme kɯ-me nɯ tɕe, tɕe kɯju jmɤlu nɯnɯ ɲɯ-sɲu ɕti tɕe nɯ  nɯ-sɲu tɕe tɕe iɕqʰa tɯ-rɣi cʰɯ-kɯ-χtɤr nɯ nɯ-kɯ-sɲu tu-sɤrmi-nɯ. \\
\textsc{lnk} animal tailless \textsc{dem} man \textsc{sens}-be \textsc{3sg.poss}-tail \textsc{nmlz}:S/A-not.exist \textsc{dem} \textsc{lnk} \textsc{lnk} animal tailless  \textsc{dem} \textsc{ipfv}-be.crazy  be:\textsc{affirm}:\textsc{fact} \textsc{lnk} \textsc{dem} \textsc{pfv}-be.crazy \textsc{lnk} \textsc{lnk} the.aforementioned \textsc{indef.poss}-seed \textsc{ipfv}-\textsc{nmlz}:S/A-spread \textsc{dem} \textsc{pfv}-\textsc{nmlz}:S/A-be.crazy \textsc{ipfv}-call-\textsc{pl} \\
\glt `The `tailless animal' is the man, and `he becomes crazy', (when the crow say) that (people) became crazy, it means that they are sowing seeds.' (22-qajdo, 47-9)
\end{exe}

\subsection{Diminutive} \label{sec:diminutive}
There are three diminutive formations in Japhug, with the quasi-suffixes \forme{-pɯ}, \forme{-tsa} and \forme{-tɕɯ}.

The most productive is the \forme{-pɯ} suffixation. This transparent suffix comes from the noun \japhug{tɤ-pɯ}{offspring, young} (from Tibetan \tibet{བུ་}{bu}{son}). A diminutive formation based on the same noun also exists in Tibetan (\citealt{uray52diminutive},  \citealt[627]{hill14derivational}); whether the diminutive formation was independently innovated, or was borrowed from Tibetan is a question that deserves further investigation. It is also attested in Situ (\citealt{zhang16bragdbar}, \citealt[151]{lai17khroskyabs}).

Earlier diminutives are formed with the\textit{ status constructus} of the noun, for instance \japhug{tɕʰemɤpɯ}{young girl} from \japhug{tɕʰeme}{girl}, \japhug{staχpɯ}{pea} from \japhug{stoʁ}{broad bean}, or \japhug{kʰɯzɤpɯ}{puppy} from a non-attested form \forme{*kʰɯza}, propably itself the \forme{-tsa} diminutive of \japhug{kʰɯna}{dog}, borrowed from a Situ dialect.

More recent diminutives are directly formed with the base form, such as \japhug{qapripɯ}{little serpent}. This formation is extremely productive, and applies to plants, animals and even objects as in (\ref{ex:srWnloR}).

\begin{exe}
\ex \label{ex:srWnloR}
\gll tɕe srɯnloʁ-pɯ ci ɲɤ-kʰo tɕe \\
lnk ring-\textsc{dim} \textsc{indef} \textsc{ifr}-give \textsc{lnk} \\
\glt `He handed him a little ring.' (2011-4-smanmi, 120)
\end{exe}

Suffixation with \forme{-pɯ} is the fused variant of the property noun construction with \japhug{ɯ-pɯ}{little one} described in section \ref{sec:property.nouns}.

A diminutive that is common to all Gyalrongic languages is the suffix \forme{-tsa}/\forme{-za} (Situ \forme{-tsa} or \forme{-za} (\citealt[163]{linxr93jiarongen}), Khroskyabs \forme{-ze} / \forme{-zə} / \forme{-zɑ}, \forme{-tsi} (\citealt[158]{lai17khroskyabs}), Stau \forme{-zə}), found in fossilized forms in nouns such as \japhug{kʰɯtsa}{bowl} and \japhug{βɣɤza}{fly},\footnote{The noun \japhug{βɣɤza}{fly} is cognate to Brag-dbar \forme{kəvɐ̂s}, Khroskyabs \forme{jvɑzɑ́} (\citealt{zhang16bragdbar}, \citealt[156]{lai17khroskyabs}) and originates from proto-Gyalrong \forme{*kpɔs-tsa} (\citealt[53]{jacques08zh}). } but still visible in diminutive forms like \japhug{paʁtsa}{piglet} (from \japhug{paʁ}{pig}). It originates from the noun `son' that is lost in Japhug but still attested in Situ and Khroskyabs (Wobzi \forme{zî} `young man'). 

In Japhug the \forme{-tsa} diminutive is not very productive; it applies to some nouns that already have a \forme{-pɯ} diminutive such as \japhug{stoʁtsa}{name of plant} from \japhug{stoʁ}{broad bean} (besides \japhug{staχpɯ}{pea}).

The third diminutive suffix \forme{-tɕɯ}, like the two preceding ones, originates from a noun meaning `offspring', \japhug{tɤ-tɕɯ}{son}, and requires \textit{status constructus}.

It is  used for animals (\japhug{kumpɣɤtɕɯ}{sparrow} from \japhug{kumpɣa}{fowl}) or inanimate objects (\japhug{kʰɤtɕɯ}{little house} from \japhug{kʰa}{house} or \japhug{lʁɤtɕɯ}{little gunny bag} from \japhug{lʁa}{gunny bag}). It occurs in some lexicalized forms such as \japhug{mbrɯtɕɯ}{knife}.\footnote{The root of this noun is metathesized from \forme{*mbɯr}; its cognates have a \forme{-tsa} diminutive in Situ (Brag-dbar \forme{mbərtsiɛ̄}, \citealt[228]{zhang16bragdbar}) and Khroskyabs (Wobzi \forme{(bərzé}, \citealt[115]{lai17khroskyabs}).}

\subsection{Derogative} \label{sec:derogative}
There are three derogative quasi-suffixes in Japhug, deriving designations of old or broken things: \forme{-do} and \forme{-mbe} `old X' and \forme{-ɴqra} `broken X'. These suffixes are the fused variants of the property nouns \japhug{ɯ-ɴqra}{broken one}, \japhug{ɯ-do}{old one} and \japhug{tɤ-mbe}{old thing}  (see \ref{sec:property.nouns}). 

The suffixes \forme{-do} and \forme{-mbe}, like their corresponding property nouns, differ in that the former occurs with animals and plants (\japhug{nɯŋa-do}{old cow}, \japhug{rɟɤlpu-do}{old king}), while the latter is used for inanimate objects.

 In a few cases, the suffixed noun is in status constructus (as \japhug{kʰɤɴqra}{ruin} from \japhug{kʰa}{house} and \forme{-ɴqra}). When the suffixed noun is an IPN, addition of a derogative suffix does not turn it into a NIPN, as in \japhug{tɯ-rcɤmbe}{old jacket} from \japhug{tɯ-rcu}{jacket} and \forme{-mbe} (unlike other types of compounds, see XXX).

\subsection{Gender}
There is no morphological expression of gender in Japhug. For animals, the nouns \japhug{pʰu}{male} and \japhug{mu}{female} (from Tibetan \tibet{ཕོ་}{pʰo}{male} and \tibet{མོ་}{mo}{female}) can be used on their own (as in \ref{ex:phu.mu}) or occur as second member of compounds, as \japhug{kumpɣapʰu}{rooster} and \japhug{kumpɣamu}{hen} from \japhug{kumpɣa}{fowl}, or \japhug{lɯlɤmu}{female cat} from \japhug{lɯlu}{cat}, with \textit{status constructus} of the first noun.

\begin{exe}
\ex \label{ex:phu.mu}
\gll tɤkʰe pɣɤtɕɯ ndɤre pʰu mu saχsɤl \\
stupid bird:\textsc{dim} on.the.other.hand male female be.clear:\textsc{fact} \\
\glt `The male and the female of the `stupid bird', as opposed (to the birds previously discussed), are easy to distinguish.' (23-scuz, 45)
\end{exe}

The suffixes \forme{-pa} and \forme{-mɯ} (from Tibetan \forme{-pa} and \forme{-mo} respectively) also occur for a handful of nouns, some of Tibetan origin (\japhug{srɯnmɯ}{râkshasî} from \tibet{སྲིན་མོ་}{srin.mo}{râkshasî}) but also some local names such as \japhug{ɴɢarpa}{male one quarter yak hybrid}  vs \japhug{ɴɢarmɯ}{female one quarter yak hybrid}.

The noun \japhug{paʁɟu}{boar} from \japhug{paʁ}{pig} has a suffix \forme{-ɟu} that is not found in any other word.

For some domestic animals, a lexical distinction is made between male and female animals (see Table \ref{tab:lexical.gender}).

\begin{table}
\caption{Lexical distinction of male and female animals} \label{tab:lexical.gender}
\begin{tabular}{l|l}
 \lsptoprule 
 Male & Female \\
 \midrule
\japhug{qambrɯ}{male yak} & \japhug{qra}{female yak} \\
\japhug{jla}{male hybrid yak} & \japhug{ftsoʁ}{female hybrid yak} \\
\japhug{mbala}{bull} & \japhug{nɯŋa}{cow}  \\
\japhug{zrɤβ}{he-goat} & (\japhug{tsʰɤnmu}{ewe})  \\
 \lspbottomrule
\end{tabular}
\end{table}

\subsection{Collective}
While Japhug lacks number inflection, there are several marginal collective derivations. 

The first one is built by partial reduplication, but differs from the most common reduplication described in section XXX in that the vowel in the replicated syllable replacing the rhyme is \ipa{a}, not \ipa{ɯ}. Examples are few, as shown in Table \ref{tab:coll.n}, but several of them are borrowings from Tibetan.

\begin{table}
\caption{Collective noun derivation} \label{tab:coll.n}
\begin{tabular}{l|lll}
 \lsptoprule 
 Base form & Collective & Tibetan \\
 \midrule
\japhug{rdɯl}{dust, dirt} & \japhug{rdardɯl}{dust, dirt} & \tibet{རྡུལ་}{rdul}{dust} \\
\japhug{tɯ-ntɕʰɯr}{fragment}  & \japhug{ɯ-ntɕʰantɕʰɯr}{fragments} & \\
\japhug{ɯ-zɯr}{side}  & \japhug{ɯ-zarzɯr}{sides} & \tibet{ཟུར་}{zur}{side, corner} \\
\japhug{ɯ-rkɯ}{side} & \japhug{ɯ-rkarkɯ}{sides} & \\
 \lspbottomrule
\end{tabular}
\end{table}

Collective nouns can be used without number clitic, as in (\ref{ex:WntChantChWr}), but they often appear with the \japhug{ra}{plural} as in (\ref{ex:rdardWl}).

\begin{exe}
\ex \label{ex:WntChantChWr}
\gll znɤrɣama nɯ mtʰa ɯ-kɤcu ŋu. tɕe nɯnɯtɕu tɯ-ji ɯ-ntɕhantɕhɯr pɯ-dɤn, jinde kʰro ɲɤ-s-qapɯ-nɯ,\\
p.n. \textsc{dem} p.n. \textsc{3sg.poss}-east be:\textsc{fact} \textsc{lnk} \textsc{dem:pl} \textsc{indef}.\textsc{poss}-field \textsc{3sg.poss}-fragment:\textsc{coll} \textsc{pst}.\textsc{ipfv}-be:many now much \textsc{ifr}-\textsc{caus}-be.fallow-\textsc{pl}\\
\glt `Znargama ('The place where one calls the rain') is on the east of Mtha, there used to be many little fragments of fields, but now people have left them become fallow.' (150903 kAmYW tWji3, 19)
\end{exe}

\begin{exe}
\ex \label{ex:rdardWl}
\gll tɕe tɤɕi nɯ tú-wɣ-χtɕi tɕʰɣaʁtɕʰɣaʁ ʑo tɕe, rdardɯl nɯra ɲɯ́-wɣ-ɣɤ-me tɕe \\
\textsc{lnk} barley \textsc{dem} \textsc{ipfv}-\textsc{inv}-wash \textsc{idph}:II:completely.clean \textsc{emph} \textsc{lnk} dush:\textsc{coll} \textsc{dem:pl} \textsc{ipfv}-\textsc{inv}-\textsc{caus}-not.exist \textsc{lnk} \\
\glt `Then one washes the barley very thoroughly, one removes all the dirt.' (2002tWsqar, 118)
\end{exe}
 
The noun \japhug{rgargɯn}{old person} has the form of a collective noun as those of Table \ref{tab:coll.n}, but it is commonly used with singular or dual referents (as in \ref{ex:rgargWn}). It could be originally the collective form of a borrowing from Tibetan \tibet{རྒན་པོ་}{rgan.po}{old person}, though the expected form would have been $\dagger$\forme{rga-rgɤn}. 
 
\begin{exe}
\ex \label{ex:rgargWn}
\gll  rgargɯn ni kɤ-fstɯn pɯ-ra \\
old.person \textsc{du} \textsc{inf}-serve \textsc{pst.ipfv}-have.to \\
\glt `She had to take care of two old people.' (14-tApitaRi, 34)
\end{exe}

The second collective derivation is only attested by one example, the form \japhug{qajɯqaja}{all kinds of worms} (see example \ref{ex:qajWqaja}) which derives from \japhug{qajɯ}{worm}  by reduplicating the  whole word and changing the last rime to \ipa{-a}, a reduplication template reminiscent of that found in Khroskyabs (see \citealt{lai13fuyin}, \citealt[22-24]{lai17khroskyabs}).

\begin{exe}
\ex \label{ex:qajWqaja}
\gll
tɯ-ci ɯ-ŋgɯ qajɯqaja tʰamtɕɤt, sɯŋgɯ ɣɯ ɯ-rɯdaʁ kɯ-xtɕi kɯ-wxti, mɤʑɯ pɣa nɯnɯra lonba ʑo kɤ-fsraŋ kɯ-ra ɲɯ-ɕti ma \\
\textsc{indef.poss}-water \textsc{3sg}-inside worm:\textsc{coll} all forest \textsc{gen} animal \textsc{nmlz}:S/A-be.small \textsc{nmlz}:S/A-be.big yet bird \textsc{dem:pl} all \textsc{emph} \textsc{inf}-protect \textsc{inf:stat}-have.to \textsc{sens}-be:\textsc{affirm} \textsc{lnk} \\
\glt `All the creatures in the water, the small and big animals of the forest, and also the birds have to be protected.' (160703 jingyu, 43)
\end{exe}

\subsection{Superlative}
While there is no adjectival superlative derivation in Japhug (superlative constructions are synthetic, see section XXX), we find nevertheless a derivation applied to locative nouns, expressing the most extreme location. As shown in Table \ref{tab:superlative.n}, it is built by adding an element \forme{-ɕɯ-} followed by a complete copy of the root of the noun without \textit{status constructus} alternation or partial replication; the resulting noun is still an inalienably possessed locative noun. Example (\ref{ex:WqaCWqa}) illustrates the use of one of these forms.

\begin{table}
\caption{Superlative noun derivation} \label{tab:superlative.n}
\begin{tabular}{l|lll}
 \lsptoprule
\japhug{tɯ-ku}{head, top} & \japhug{ɯ-kuɕɯku}{the highest place} \\
\japhug{tɯ-qa}{root, paw, bottom} & \japhug{ɯ-qaɕɯqa}{the deepest place} \\
\japhug{ɯ-rkɯ}{side} & \japhug{ɯ-rkɯɕɯrkɯ}{the place most on the side} \\
\japhug{ɯ-zɯr}{side} & \japhug{ɯ-zɯrɕɯzɯr}{the place most on the side} \\
 \lspbottomrule
\end{tabular}
\end{table}

\begin{exe}
\ex \label{ex:WqaCWqa}
\gll rɟɤmtsʰu ɯ-qaɕɯqa pjɯ-ɕe tɕe, nɯnɯ ɯ-kɤ-nɤ-mɯm nɯra ɕ-tu-nɯ-tɕɤt ɲɯ-ŋu. \\
ocean \textsc{3sg.poss}-bottom:\textsc{super} \textsc{ipfv}:\textsc{down}-go  \textsc{lnk} \textsc{dem} \textsc{3sg.poss}-\textsc{nmlz}:P-\textsc{trop}-be.tasty \textsc{dem:pl} \textsc{transloc-ipfv}-\textsc{auto}-take.out \textsc{sens}-be \\
\glt `(The sperm whale) goes to the lowest depths of the ocean and catches the things it likes to eat.' (160703 jingyu, 24)
\end{exe}
\section{Denominal adverbs}
\subsection{Comitative adverbs} \label{sec:comitative.adverb}
\subsection{Time adverbs}
