\chapter{The noun phrase}
%\section{Indefiniteness} \label{sec:indef}
%Japhug does not have have genuine definite or indefinite articles, though it presents several strategies to mark definiteness. First, it is possible to use a noun with an indefinite or a definite marker. Second, Japhug has a series of indefinite pronouns, and interrogative pronouns can also be used as indefinites in some contexts. Third, non-referentiality or non-identifiability of an entity can be expressed by various syntactic constructions, including relative clauses and incorporation.
%
%\subsection{Indefinite and definite markers}
%Like many languages (\citealt[130]{creissels06sgit1}), Japhug  uses bare nouns without any definiteness marking. Bare nouns are most often non-referential:

%  \begin{exe}
%\ex
%\gll ʁnaʁna 		tɕheme 	tɯ-tɤ-tu 	nɤ, 	kɤndʑɯsqhɤj 	tu-kɤ-sɯ-βzu \\
%both \emphat{} \gen{} \textsc{3du}.\poss{}-child girl \textsc{if}-\aor{}-have \textsc{if} group.of.sisters \ipf{}-\inftv{}-\caus{}-make \\
%\glt If both of them have girls, let them be sisters. (zrɤntɕɯ tɯrme 4)
%\end{exe}
%
%They are  used in nominal predicates with a copula:
%
%  \begin{exe}
%\ex
%\gll aʑo 	tɕʰeme 	ɲɯ-ŋu-a  tɕe \\
%I girl \const{}-be-\textsc{1sg} \coord{} \\
%\glt I am a girl. (Nyima Vodzer 144)
%\end{exe}
%
%Bare nouns are rare with referential nouns (except in answers to questions), but examples can be found:
%
%  \begin{exe}
%\ex
%\gll qacʰɣa 	kɯ 	maχtɕɯ tɤ-tɯt-a nɯ mɤ-tɯ-ste 	ti 	ɲɯ-ŋu  \\
%fox \erg{} in.principle \aor{}-say[II]-\textsc{1sg} \compl{} \negat{}-2-\npst{}:do.this.way[III] \npst{}:say \ipf{}-be \\
%\glt The fox says: ``You do not do as I told you to." (The fox, 44)
%\end{exe}
%
%Personal names generally occur as bare nouns, without any definiteness marker (though, as we will see, these markers are not agrammatical with personal names):
%
%  \begin{exe}
%\ex
%\gll  ɯrɟɤnpanma 	kɯ 	ʁlaŋsaŋtɕhin 	ɯ-ɕki  \\
% Padmasambhava \erg{} Gesar \textsc{3sg}-\textsc{dat} \\
%\glt Padmasambhava (told) Gesar:
%\end{exe}
%
%Bare nouns are however relatively rare. Most nominal phrases contain either a demonstrative, an indefinite or a topic marker (cf chapter \ref{chapt:noun.phrase}). 
%
%
%The numeral ci is used as an indefinite marker, placed at the end of the noun phrase:
%  \begin{exe}
%\ex
%\gll tɕʰeme 	kɯ-mpɕɯ-mpɕɤr 	ci 	ɲɤ-nɯɬoʁ \\
%girl \nmlz{}:\stat{}-\textsc{intens}-beautiful \textsc{indef}  \evd{}-appear \\
%\glt A very beautiful girl appeared (out of it). (The flood, 39)
%\end{exe}
%ci must be used to introduce a new referent in a story.
%
%On its own, ci can also serve as an indefinite pronoun, meaning ``one of (a group)'':
%  \begin{exe}
%\ex
%\gll ci 	ɣɯ 	tɤ-tɕɯ, 	ci 	ɣɯ 	tɕheme 	tɯ-tɤ-tu 	nɤ, 	ʁzɤmi 	ku-kɤ-sɯ-βzu \\
%\textsc{indef} \gen{} \neu{}-boy \textsc{indef} \gen{} girl \textsc{if}-\aor{}-have \textsc{if} husband.and.wife \ipf{}-\inftv{}-\caus{}-make \\
%\glt If one of them has a boy, and the other one has a girl, let us make them husband and wife. (zrɤntɕɯ tɯrme 5)
%\end{exe}
%
%
%%ci thɯ-kɯ-rgɯ-rgɤz ɲɯ-ɕti tɕe, ci kɯ-xtɕɯ-xtɕi ɲɯ-ɕti tɕe,
%%Nyima wodzer4.127
%
%Japhug has no definite article. The distribution of the ubiquitous marker \textit{nɯ} is close to that of a definite marker, but since it does not normally appear in focalized noun phrases such as answers to questions,  it is more statisfying to treat it as a topic marker (see section \ref{sec:topic}).
%
%The marker  iɕqʰa ``the aforementioned'' indicates both definiteness and topicality.\footnote{Used alone, iɕqʰa is a temporal adverb meaning ``just before'', see \ref{chapt:adv}.} It is used on referents that have been previously mentioned in the same story, usually only a few sentences back:
%  \begin{exe}
%\ex \label{ex:indef}
%\gll \textbf{``razri} 	\textbf{kɤtɯm} 	\textbf{ci} 	ɲɯ-ra, 	taqaβ 	ci 	ɲɯ-ra" to-ti qhe   \\
% thread ball \textsc{indef} \const{}-need needle \textsc{indef} \const{}-need \evd{}-say \coord{}  \\
%\glt He told (Rgyabza) ``I need a ball of thread and a needle''.
%\ex \label{ex:icqha}
%\gll tɕendɤre 	ɲo-kho 	qhe, 	tɕe 	ɯ-ndzɤtshi 	kɤ-tsɯm 	nɯ 	tɕu 	qhe 	tɕe, \textbf{iɕqʰa} 	\textbf{kɤtɯm} 	\textbf{nɯ }	ɯʑo 	kɯ 	ko-ndo, 	taqaβ-rna 	nɯ 	ɲɤ-rku qhe,  \\
% \coord{} \evd{}-give \coord{} \coord{} \textsc{3sg}.\poss{}-meal \inftv{}-bring \textsc{compl}  \loc{} \coord{} \coord{} the.aforementioned ball \topic{} he \erg{} \evd{}-take needle-ear \topic{} \evd{}put.in \coord{} \\
%\glt She gave it to him. While (people) brought his meal, he took the ball of thread and put it into the ear of the needle. (Gesar 270-2)
%\end{exe}
%The referent ``ball of thread'', first introduced in sentence \ref{ex:indef}, appears again two sentences later with both the topic markers nɯ and \textit{iɕqʰa}. 
%
%Unlike nɯ, iɕqʰa is relatively rare depending one the speaker and the type of discourse. It is possible to find stories long more than 80 sentences without any occurrence of iɕqʰa. 
%
%There seems to be a limit to the number of sentences that can separate a noun phrase in iɕqʰa from its preceding occurrence (probably no more than five-six), but this topic deserves of systematic study based on all available stories.
%
% iɕqʰa as a definite marker not only occurs with nouns, but also with bare demonstratives such as nɯnɯ and with personal names:
%
%  \begin{exe}
%\ex
%\gll  tɕendɤre 	iɕqʰa 	ʁlaŋsaŋtɕʰin 	χsɯm 	ma 	mɯ-tɤ-kɯ-rʑaʁ 	nɯ, \\
% \coord{} the.aforementioned Gesar three apart.from \negat{}-\aor{}-\nmlz{}:S-pass.day \topic{} \\
%\glt Gesar, who was only three days old,  (Gesar 81)
%\end{exe}
%
%negative with maka
%ɯʑo kɯ ta-tɯt maka kɯ-tu me
%He said nothing 什么也没有说
%\subsection{Competing constructions}
%
%\subsubsection{Relatives}
%nɤʑo	nɯ-nɯ-ɣɤwu	ma,	nɤ-kɯ-nɯɣmu	me	ma	ma-ta-mbi
%Frog 38
%
%
%iɕqha tɯrme kɯngo ɣɤʑu tɕe, ɯkɯrtoʁ jaria wo!
%Someone was ill \wav{8_kWngoGAZu}
%
%
%	kɤ-mɯnmu	kɯ́nɤ	mɯ-pjɤ-mɯnmu				
%	Raven 58
%
%
%A	78	nɤj	thɤstɯɣ	tɯ-chɯ-cha	ʑo	nɤ,	a-tɤ-tɯ-nɯ-rke	qhe	nɯnɯ	nɤʑo	ɣɯ	nɤ-rkus	ŋu"	to-ti,
%
%
%tɕhi kɯfse pɯtɯnɯmbɣom kɯ́nɤ, zgo nɯ kutɯpɣaʁ ndɤre ra
%\wav{8_kutWnWmbGom}
%
%nɯ apɯŋu ma dianhua ɯkɯlɤt ɣɤʑu!
%\wav{WkWlAt}
%%Negation
%ɯ-ɕɯ-kɯ-phɯt ra kɯ-tu me