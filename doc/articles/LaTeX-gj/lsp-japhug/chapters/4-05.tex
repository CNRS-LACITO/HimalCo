\chapter{Voice derivations}

\section{Fossil derivations}

\subsection{Applicative \forme{-t}}
Beside the productive prefixal \forme{nɯ-} applicative (sec XXX), Japhug has vestigial traces of a \forme{-t} applicative suffix, better attested in Kiranti and West Himalayish languages (see \citealt{michailovsky85dental}, \citet{jacques15derivational.khaling} and \citealt{jacques16ssuffixes} for comparative studies of this suffix). Only two examples of this derivation exist in Japhug: \japhug{ɣɯt}{bring} and \japhug{mdɯt}{strongly wish for}. 

The verb of manipulation \japhug{ɣɯt}{bring} derives from the motion verb \japhug{ɣi}{come}; the vowel alternation is regular as pre-Japhug \forme{*i} changes to \ipa{ɯ} in closed syllables. With a motion verb such as `come', the effect of the applicative (\ref{ex:bring1}) is similar to a causative  (\ref{ex:bring2}). 

\begin{exe}
\ex \label{ex:bring1}
\glt `come with X' $\rightarrow$ `bring'
\ex \label{ex:bring2}
\glt `cause X to come' $\rightarrow$ `bring'
\end{exe}

The transitive verb \japhug{mdɯt}{strongly wish for} is historically related to the verb \japhug{mdɯ}{live up to}, and constitutes another example of the \forme{-t} applicative, though it is less immediately obvious than in the case of \japhug{ɣɯt}{bring} because each of the verbs has undergone semantic specialization after the derivation took palce.

The verb \forme{mdɯ} is semi-transitive (sec XXX), and takes as its semi-object the lifespan; it can be applied to plants, animals and humans, as shown by examples (\ref{ex:chWmdW}) and (\ref{ex:chWmdWa}). It selects the  downstream' series of directional prefixes (\forme{tʰɯ-}, \forme{cʰɯ-}).

 \begin{exe}
\ex \label{ex:chWmdW}
\gll tɕe nɯŋa ɯʑo nɯnɯ, sqamŋu-xpa jamar cʰɯ-mdɯ ɲɯ-ŋgrɤl\\
\textsc{lnk} cow \textsc{3sg} \textsc{dem} fifteen-year about \textsc{ipfv}-live.up.to \textsc{sens}-be.usually.the.case \\
\glt `A cow itself can live up to fifteen years.' (05-qaZo, 142)
\end{exe}

 \begin{exe}
\ex \label{ex:chWmdWa}
\gll ``nɤʑo nɯ kʰrɯtsu-xpa a-tʰɯ-tɯ-mdɯ ra nɤ" to-ti ɲɯ-ŋu. tɕe ``aʑo kɯnɤ kʰrɯtsu cʰondɤre tɯ-rʑaʁ nɯnɯ cʰɯ-mdɯ-a ra" to-ti \\
\textsc{2sg} \textsc{dem} ten.thousand-year \textsc{irr}-\textsc{pfv}-2-live.up.to have.to:\textsc{fact} \textsc{sfp} \textsc{ifr}-say \textsc{sens}-be \textsc{lnk} \textsc{1sg} also  ten.thousand-year \textsc{comit} one-day \textsc{dem} \textsc{ipfv}-live.up.to-1sg have.to:\textsc{fact} \textsc{ifr}-say \\
\glt `He said: `May you live ten thousand years! I want to live one thousand years and one more day.' (150830 afanti-zh, 64)
\end{exe}

The meaning `live until/up to' is however a semantic innovation in Japhug: its Situ cognate \forme{mdə́} means `reach' as a motion verb. Japhug has restricted the meaning of this verb to a very specific context.

The verb \japhug{mdɯt}{strongly wish for} is morphologically transitive, and can take as its object an infinitive complement as in (\ref{ex:chWmdWta}). It shares with \japhug{mdɯ}{live up to} the `downstream' directional prefixes (\forme{cʰɯ-}).

 \begin{exe}
\ex \label{ex:chWmdWta}
\gll aʑo kɯrɯ-skɤt kɤ-βzjoz nɯ cʰɯ-mdɯt-a ʑo ɕti \\
\textsc{1sg} Tibetan-language \textsc{inf}-learn \textsc{dem} \textsc{ipfv}-strongly.wish \textsc{emph} be.\textsc{affirm}:\textsc{fact} \\
\glt `I strong wish to learn Tibetan/Gyalrong.' (elicited)
\end{exe}

The precise meaning of \forme{mdɯt}  is to wish for something that one is confident he can realize. If one accepts that the idea the original meaning of Japhug \japhug{mdɯ}{live up to} was `reach' as in Situ, the meaning `wish for' of the verb \forme{mdɯt} has the same relationship to that of the base verb as English `reach for' (`reach for the stars') to the verb `reach', with a conative interpretation `try to reach'.  The addition of the suffix \forme{-t} turns the semi-transitive (morphologically intransitive) \forme{mdɯ} into a transitive verb whose A corresponds to the S of the base verb. This applicative derivation from a semi-transitive verb is not unique in Japhug; the transitive verb \japhug{nɯrga}{like} from the verb \japhug{rga}{like, be happy} with the \forme{nɯ-} applicative is another similar example (see section XXX).


