\chapter{Syllable structure and sandhi} \label{sec:syllable}

\section{Neutralization, quasi-neutralization and free variation}

\subsection{The contrast between \ipa{-oŋ} and \ipa{-aŋ}}

\subsection{The contrast between \ipa{ɯ} and \ipa{i} after palatal and alveolo-palatal consonants}
The contrast between \ipa{ɯ} and \ipa{i} is partially or completely neutralized after palatal (\ipa{c}, \ipa{cʰ}, \ipa{ɟ}, \ipa{ɲɟ}, \ipa{ɲ} and \ipa{j}) and alveolo-palatal (\ipa{tɕ}, \ipa{tɕʰ}, \ipa{dʑ}, \ipa{ndʑ}, \ipa{ɕ}, \ipa{ʑ}) consonants in some contexts.

In stressed open syllables, in particular in word-final position, the contrast between \ipa{ɯ} and \ipa{i} is however very clear after all consonants. The existence of minimal pairs such as \japhug{cɯ}{stone} and \japhug{ci}{one} or the stem alternations between \forme{ndʑɯ} (stem I) and \forme{ndʑi} (stem III) in the paradigm of the verb \japhug{ndʑɯ}{accuse} (§ XXX), show that this contrast is not neutralized after palatal and alveolo-palatal consonants in this context.

The contrast between \ipa{ɯ} and \ipa{i} is completely neutralized in most closed syllables and near-neutralized in unstressed syllables, including verbal suffixes, prefixes, and non-final element of compounds; each of these contexts is extensively discussed below.

\subsubsection{Neutralization in closed syllables}
In closed syllables, the contrast between \ipa{ɯ} and \ipa{i} is almost completely neutralized after palatal (\ipa{c}, \ipa{cʰ}, \ipa{ɟ}, \ipa{ɲɟ}, \ipa{ɲ} and \ipa{j}) and alveolo-palatal (\ipa{tɕ}, \ipa{tɕʰ}, \ipa{dʑ}, \ipa{ndʑ}, \ipa{ɕ}, \ipa{ʑ}) consonants; the archiphoneme \archi{ɯ,i}  is realized as \phonet{ɯ} before \forme{-ɣ}, \forme{-β} (and its variant \forme{-p} with some ideophones, § XXX), and as \phonet{i} before \forme{-m}, \forme{-r}, \forme{-n} and \forme{-l}, as shown by Table \ref{tab:palatal.WC.iC}. In the orthography adopted in the present work, the archiphoneme \archi{ɯ,i} is however transcribed as \forme{ɯ} in all these contexts.

\begin{table}
\caption{Realizations of the archiphoneme \archi{ɯ,i} following palatal and alveolo-palatal consonants in close syllables} \centering \label{tab:palatal.WC.iC}
\begin{tabular}{lllll}
\toprule
Coda & Example & Realization \\
\midrule
\forme{-β}/\forme{-p} & \japhug{cʰɯβ}{ideophone of an object breaking} &\phonet{cʰɯβ} \\
\forme{-ɣ}  & \japhug{rɟɯɣ}{run} &\phonet{rɟɯɣ} \\
\midrule
\forme{-m}  & \japhug{jɯm}{be nice (of weather)} &\phonet{jim} \\
\forme{-n}  & \japhug{jaftɕɯn}{stirrup} &\phonet{jaftɕin} \\
\forme{-r}  & \japhug{mtɕɯr}{turn} &\phonet{mtɕir} \\
\forme{-l}  & \japhug{rɲɯl}{wither} &\phonet{rɲil} \\
\bottomrule
\end{tabular}
\end{table}
The contrast between \ipa{ɯ} and \ipa{i} used to be neutralized before the codas \forme{-z} and \forme{-t}, with the archiphoneme \archi{ɯ,i} realized as \phonet{i}. However, new rhymes \forme{-ɯt} and \forme{-ɯz} contrasting with \forme{-it} and \forme{-iz} have been reintroduced by the transitive 1/2\textsc{sg}\fl{}3 past suffix \forme{-t} or \forme{-z} (depending on the dialect of Japhug, see § XXX).

Minimal pairs between open syllable \forme{-ɯ} stem verbs taking the \forme{-t} suffix (such as \ipa{tɤ-tɯ-cɯ-t} `you opened it' \textsc{pfv}-2-open-\textsc{pst}) and \forme{-it} stem verbs (\ipa{lɤ-tɯ-cit} `you moved' \textsc{pfv}-2-move) can easily be found. Even if this contrast is extremely marginal and restricted to this morphological context, \ipa{ɯ} and \ipa{i} cannot be considered to be neutralized before \forme{-z} or \forme{-t} (depending on the dialect of Japhug).

\subsubsection{Non-final elements of compounds} \label{sec:W.i.compounds}
In compounds and other polysyllabic stems, the contrast between contrast between \ipa{ɯ} and \ipa{i} is very difficult to perceive in non-final open syllables with palatal or alveolo-palatal consonant onsets, and Tshendzin has, during our decade-long collaboration, expressed conflicting views about whether a contrast does or does not exist in this context.

While there are no minimal pairs only distinguished by the \ipa{ɯ} vs \ipa{i} contrast in non-final syllables after palatal or alveolo-palatal consonants, it seems now clear that some words are consistently pronounced with \forme{ɯ} rather than \forme{i} in this context. This unstable contrast, which does not carry much information load, is likely to differ at the idiolectal level. The orthography used in this work reflects a normalization based on Tshendzin's judgements (rechecked several times).
\begin{paragraph}{\forme{tɕʰi°} vs \forme{tɕʰɯ°}}
Many nouns in Japhug have \forme{tɕʰi°} or \forme{tɕʰɯ°} as first element, in particular due to words of Tibetan origin containing \tibet{ཆུ་}{tɕʰu}{water} or \tibet{མཆུ་}{mtɕʰu}{lip}. The variant \forme{tɕʰi°}  is found in the great majority of these words. In some cases like \japhug{tɕʰira}{water jar} (from \tibet{ཆུ་ར་}{tɕʰu.ra}{water container}), I have used an etymologizing transcription with \forme{ɯ} (\forme{tɕʰɯra}) in previous works (in particular \citealt{jacques15japhug}), but changed my mind after careful rechecking (using for instance the stem I and III of the verb \japhug{tɕʰɯ}{gore} for comparison). 

The variant \forme{tɕʰɯ°} is restricted essentially to words with a a cluster with a uvular or a labial fricative as first element (such as \japhug{rtɕʰɯʁjɯ}{caterpillar}, \japhug{tɕʰɯχpri}{newt}, \japhug{tɕʰɯβroʁ}{type of tsampa}), but also found in \japhug{tɕʰɯmɲɯɣ}{water hole} (from \tibet{ཆུ་མིག་}{tɕʰu.mig}{water hole}) and \japhug{tɕʰɯzɯ}{type of weaving tool}.

The noun \japhug{tɯ-mtɕʰi}{mouth} (from \tibet{མཆུ་}{mtɕʰu}{lip}) has an unexpected \forme{i} (the expected form would be $\dagger$\forme{mtɕʰɯ}). It is possible that it was extracted from a compound (like \japhug{tɯ-mtɕʰirme}{moustaches}) where \ipa{ɯ} and \ipa{i} are neutralized as \phonet{i}, and then that this phonetic variant was reinterpreted as the independent root.
\end{paragraph}
\begin{paragraph}{\forme{ndʑi°} vs \forme{ndʑɯ°} }
Compounds with \forme{ndʑi°} or \forme{ndʑɯ°} as non-final element are not common, but social relation collective nouns (§ \ref{sec:social.collective}) containing the prefix \forme{kɤndʑi-}, are particular numerous. Words with the variant \forme{ndʑɯ°} are very rare: the two verbs \japhug{ndʑɯrpɯt}{be numb} and \japhug{andʑɯβri}{protect each other} (with the rare reciprocal prefix \forme{andʑɯ-} of denominal origin, see § XXX), and the nouns \japhug{ndʑɯrwɯz}{Sonchus sp} and \japhug{ndʑɯnɯ}{Angelica}.
\end{paragraph}
\begin{paragraph}{\forme{ci°} vs \forme{cɯ°} }
Compounds comprising \japhug{cɯ}{stone} or the root of \japhug{tɯ-ci}{water} as non-final element are common, and the majority of these words have the variant \forme{ci°}, even when the noun comes from \japhug{cɯ}{stone} (for instance \japhug{ciχɕiz}{stony earth}). Exceptions include \japhug{cɯrmbɯ}{stone heap}, \japhug{scɯʁzɯɣ}{appearance} (from \tibet{སྐྱེ་གཟུགས་}{skʲe.gzugs}{physical appearance}) and \japhug{ɣɤcɯqʰlɯβ}{making noise (of water when agitated)}.\footnote{This verb is a denominal noun-ideophone compound, see § XXX.}
\end{paragraph}
\begin{paragraph}{\forme{cʰi°} vs \forme{cʰɯ°} }
Nouns with \forme{cʰi°} or \forme{cʰɯ°} include in particular Tibetan loanwords in \tibet{ཁྱི་}{kʰʲi}{dog}. The form \forme{cʰi°} occurs in most cases (for instance \japhug{cʰisɲu}{rabbies} from \tibet{ཁྱི་སྨྱོ་}{kʰʲi.smʲo}{rabbies}), but exceptions include \japhug{cʰɯmu}{female dog}, \japhug{cʰɯrdom}{roaming dog} and \japhug{nɯcʰɯra}{keep guard} (rechecked using the minimal pair \japhug{cʰi}{be sweet} vs \japhug{tɤ-cʰɯ}{wedge}).
\end{paragraph}
\begin{paragraph}{\forme{ji°} vs \forme{jɯ°} }
Compounds with non-final \forme{jɯ°}  are extremely rare, limited to the two Tibetan loanwords \japhug{jɯɣi}{writing} (from \tibet{ཡི་གེ་}{ji.ge}{letter}) and \japhug{pjɯrɯ}{coral} (from \tibet{བྱུ་རུ་}{bʲu.ru}{coral}).
\end{paragraph}
 

\subsubsection{Verbal prefixes and reduplicated forms}

\subsubsection{Verbal suffixes and possessive prefixes}

\subsection{The contrast between \ipa{ɤ} and \ipa{a}}