\chapter{Syllable structure and sandhi} \label{sec:syllable}

\section{Difficult cases}

\subsection{The contrast between \ipa{-oŋ} and \ipa{-aŋ}}

\subsection{The contrast between \ipa{ɯ} and \ipa{i} after palatal and alveolo-palatal consonants}
The contrast between \ipa{ɯ} and \ipa{i} is very clear after all consonants in stressed open syllables, in particular in word-final position. The existence of minimal pairs such as \japhug{cɯ}{stone} and \japhug{ci}{one} or the stem alternations between \forme{ndʑɯ} (stem I) and \forme{ndʑi} (stem III) in the paradigm of the verb \japhug{ndʑɯ}{accuse} (§ XXX), show that this contrast is not neutralized after palatal and alveolo-palatal consonants in this context.

In closed syllables however, the contrast between \ipa{ɯ} and \ipa{i} is almost completely neutralized after palatal (\ipa{c}, \ipa{cʰ}, \ipa{ɟ}, \ipa{ɲɟ}, \ipa{ɲ} and \ipa{j}) and alveolo-palatal (\ipa{tɕ}, \ipa{tɕʰ}, \ipa{dʑ}, \ipa{ndʑ}, \ipa{ɕ}, \ipa{ʑ}) consonants; the archiphoneme \archi{ɯ,i}  is realized as \phonet{ɯ} before \forme{-ɣ}, \forme{-β} (and its variant \forme{-p} with some ideophones, § XXX), and as \phonet{i} before \forme{-m}, \forme{-r}, \forme{-n} and \forme{-l}, as shown by Table \ref{tab:palatal.WC.iC}.

\begin{table}
\caption{Realizations of the archiphoneme \archi{ɯ,i} following palatal and alveolo-palatal consonants in close syllables} \centering \label{tab:palatal.WC.iC}
\begin{tabular}{lllll}
\toprule
Coda & Example & Realization \\
\midrule
\forme{-β}/\forme{-p} & \japhug{cʰɯβ}{ideophone of an object breaking} &\phonet{cʰɯβ} \\
\forme{-ɣ}  & \japhug{rɟɯɣ}{run} &\phonet{rɟɯɣ} \\
\midrule
\forme{-m}  & \japhug{jɯm}{be nice (of weather)} &\phonet{jim} \\
\forme{-n}  & \japhug{jaftɕɯn}{stirrup} &\phonet{jaftɕin} \\
\forme{-r}  & \japhug{mtɕɯr}{turn} &\phonet{mtɕir} \\
\forme{-l}  & \japhug{rɲɯl}{turn} &\phonet{rɲil} \\
\bottomrule
\end{tabular}
\end{table}
The contrast between \ipa{ɯ} and \ipa{i} used to be neutralized before the codas \forme{-z} and \forme{-t}, with the archiphoneme \archi{ɯ,i} realized as \phonet{i}. However, new rhymes \forme{-ɯt} and \forme{-ɯz} contrasting with \forme{-it} and \forme{-iz} have been reintroduced by the transitive 1/2\textsc{sg}\fl{}3 past suffix \forme{-t} or \forme{-z} (depending on the dialect of Japhug, see § XXX).

Minimal pairs between open syllable \forme{-ɯ} stem verbs taking the \forme{-t} suffix (such as \ipa{tɤ-tɯ-cɯ-t} `you opened it' \textsc{pfv}-2-open-\textsc{pst}) and \forme{-it} stem verbs (\ipa{lɤ-tɯ-cit} `you moved' \textsc{pfv}-2-move) can easily be found. Even if this contrast is extremely marginal and restricted to this morphological context, \ipa{ɯ} and \ipa{i} cannot be considered to be neutralized before \forme{-z} or \forme{-t} (depending on the dialect of Japhug).


\subsection{The contrast between \ipa{ɤ} and \ipa{a} before uvulars}