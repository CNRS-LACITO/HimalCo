\chapter{Syllable structure and sandhi} \label{sec:syllable}

\section{Neutralization, quasi-neutralization and free variation}

\subsection{The contrast between \ipa{-oŋ} and \ipa{-aŋ}} \label{sec:aN.oN.free}
Inherited Japhug words never have final \forme{-ŋ} (in particular, proto-Gyalrong \forme{*-aŋ} shifts to \forme{-o}, see § XXX); words with this coda belong to four groups: Tibetan loanwords, ideophones, borrowings from other Gyalrong varieties (in the case of \japhug{rkaŋ}{be robust}, see § XXX) or function words with irregular syllable fusion (\japhug{koŋla}{really} from \forme{kɯŋula}, see § XXX).

Despites some minimal pairs (\japhug{caŋ}{dammed wall} vs \japhug{coŋ}{damage, loss}, respectively from \tibet{གྱང་}{gʲaŋ}{dammed wall} and \tibet{གྱོང་}{gʲoŋ}{loss}), the contrast between \ipa{-oŋ} and \ipa{-aŋ} in Kamnyu Japhug has some degree of instability: some words in \ipa{-oŋ} and \ipa{-aŋ}, but not all, allow free variation between the two pronunciations. This free variation is presumably due to influence from neighbouring dialects of Japhug such as that of Rqakyo, where words with \ipa{-oŋ} and \ipa{-oʁ} in Kamnyu are pronounced with a more open vowel.

Words with stable \forme{-aŋ} in final syllable include \japhug{rkaŋ}{be robust}, \japhug{fsaŋ}{fumigation}, \japhug{kʰɯrtʰaŋ}{administrative position}, \japhug{mkʰɤrmaŋ}{people}, \japhug{praʁkʰaŋ}{cave}, \japhug{rɲaŋ}{be old}. In non-final position, stable  \forme{-aŋ} is found in particular in words whose final syllable contains \ipa{a}, as in \japhug{fsraŋma}{protecting deity} (as opposed to unstable \forme{fsroŋ} / \forme{fsraŋ} `protect', see Table \ref{tab:aN.oN.free}).

Words with stable \forme{-oŋ} in final syllable include \japhug{ɕoŋtɕa}{wood} (as opposed to \forme{ɕaŋβzu} / \forme{ɕoŋβzu} `carpentry' with unstable rhyme), \japhug{koŋla}{really}, \japhug{pʰoŋ}{bottle}, \japhug{qɯmdroŋ}{wild goose}, \japhug{ʁmbroŋ}{wild yak} or \japhug{tɯ-phoŋbu}{body}.

Table \ref{tab:aN.oN.free} presents a list of words with unstable \ipa{-oŋ} / \ipa{-aŋ}, mostly loanwords from Tibetan, coming either from rhymes in \ipa{a}+nasal or from rhymes in rounded vowel+nasal. It also includes the egressive postpositions with \forme{ɕaŋ-} as first element such as \forme{ɕoŋtaʁ} / \japhug{ɕaŋtaʁ}{up from} (see §\ref{sec:egressive}).

For all these words, the orthography used in this work and in the corpus generalizes the \forme{-aŋ} variant,  which is considered by Tshendzin to be more `correct'.

\begin{table}[H]
\caption{Words with free variation between \ipa{-oŋ} and \ipa{-aŋ} in Kamnyu Japhug} \label{tab:aN.oN.free}
\begin{tabular}{llllll}
\lsptoprule
\ipa{-aŋ} variant & \ipa{-oŋ} variant &Etymology \\
\midrule
\japhug{raŋri}{each} & \forme{roŋri} & \tibet{རང་རེ་}{raŋ.re}{each} \\
\japhug{ɕoŋβzu}{carpentry} & \forme{ɕaŋβzu} & \tibet{ཤིང་བཟོ་}{ɕiŋ.bzo}{carpentry} \\
\japhug{fsraŋ}{protect, save} & \forme{fsroŋ} & \tibet{བསྲུངས་}{bsruŋs}{save} \\
\japhug{tʂaŋka}{(gold, silver) coin} & \forme{tʂoŋka} & \tibet{	ཊམ་ཀ}{ṭam.ka}{coin} \\
\forme{ɕaŋ-} `egressive' &\forme{ɕoŋ-} & \\
\lspbottomrule
\end{tabular}
\end{table}

\subsection{The contrast between \ipa{ɯ} and \ipa{i} after palatal and alveolo-palatal consonants}
The contrast between \ipa{ɯ} and \ipa{i} is partially or completely neutralized after palatal (\ipa{c}, \ipa{cʰ}, \ipa{ɟ}, \ipa{ɲɟ}, \ipa{ɲ} and \ipa{j}) and alveolo-palatal (\ipa{tɕ}, \ipa{tɕʰ}, \ipa{dʑ}, \ipa{ndʑ}, \ipa{ɕ}, \ipa{ʑ}) consonants in some contexts.

In stressed open syllables, in particular in word-final position, the contrast between \ipa{ɯ} and \ipa{i} is however very clear after all consonants. The existence of minimal pairs such as \japhug{cɯ}{stone} and \japhug{ci}{one} or the stem alternations between \forme{ndʑɯ} (stem I) and \forme{ndʑi} (stem III) in the paradigm of the verb \japhug{ndʑɯ}{accuse} (§ XXX), show that this contrast is not neutralized after palatal and alveolo-palatal consonants in this context.

The contrast between \ipa{ɯ} and \ipa{i} is completely neutralized in most closed syllables and near-neutralized in unstressed syllables, including verbal suffixes, prefixes, and non-final element of compounds; each of these contexts is extensively discussed below.

\subsubsection{Neutralization in closed syllables}
In closed syllables, the contrast between \ipa{ɯ} and \ipa{i} is almost completely neutralized after palatal (\ipa{c}, \ipa{cʰ}, \ipa{ɟ}, \ipa{ɲɟ}, \ipa{ɲ} and \ipa{j}) and alveolo-palatal (\ipa{tɕ}, \ipa{tɕʰ}, \ipa{dʑ}, \ipa{ndʑ}, \ipa{ɕ}, \ipa{ʑ}) consonants; the archiphoneme \archi{ɯ,i}  is realized as \phonet{ɯ} before \forme{-ɣ}, \forme{-β} (and its variant \forme{-p} with some ideophones, § XXX), and as \phonet{i} before \forme{-m}, \forme{-r}, \forme{-n} and \forme{-l}, as shown by Table \ref{tab:palatal.WC.iC}. In the orthography adopted in the present work, the archiphoneme \archi{ɯ,i} is however transcribed as \forme{ɯ} in all these contexts.

\begin{table}
\caption{Realizations of the archiphoneme \archi{ɯ,i} following palatal and alveolo-palatal consonants in close syllables} \centering \label{tab:palatal.WC.iC}
\begin{tabular}{lllll}
\toprule
Coda & Example & Realization \\
\midrule
\forme{-β}/\forme{-p} & \japhug{cʰɯβ}{ideophone of an object breaking} &\phonet{cʰɯβ} \\
\forme{-ɣ}  & \japhug{rɟɯɣ}{run} &\phonet{rɟɯɣ} \\
\midrule
\forme{-m}  & \japhug{jɯm}{be nice (of weather)} &\phonet{jim} \\
\forme{-n}  & \japhug{jaftɕɯn}{stirrup} &\phonet{jaftɕin} \\
\forme{-r}  & \japhug{mtɕɯr}{turn} &\phonet{mtɕir} \\
\forme{-l}  & \japhug{rɲɯl}{wither} &\phonet{rɲil} \\
\bottomrule
\end{tabular}
\end{table}

The contrast between \ipa{ɯ} and \ipa{i} used to be neutralized before the codas \forme{-z} and \forme{-t}, with the archiphoneme \archi{ɯ,i} realized as \phonet{i}. However, new rhymes \forme{-ɯt} and \forme{-ɯz} contrasting with \forme{-it} and \forme{-iz} have been reintroduced by the transitive 1/2\textsc{sg}\fl{}3 past suffix \forme{-t} or \forme{-z} (depending on the dialect of Japhug, see § XXX).

Minimal pairs between open syllable \forme{-ɯ} stem verbs taking the \forme{-t} suffix (such as \ipa{tɤ-tɯ-cɯ-t} `you opened it' \textsc{pfv}-2-open-\textsc{pst}) and \forme{-it} stem verbs (\ipa{lɤ-tɯ-cit} `you moved' \textsc{pfv}-2-move) can easily be found. Even if this contrast is extremely marginal and restricted to this morphological context, \ipa{ɯ} and \ipa{i} cannot be considered to be neutralized before \forme{-z} or \forme{-t} (depending on the dialect of Japhug).

\subsubsection{Non-final elements of compounds} \label{sec:W.i.compounds}
In compounds and other polysyllabic stems, the contrast between contrast between \ipa{ɯ} and \ipa{i} is very difficult to perceive in non-final open syllables with palatal or alveolo-palatal consonant onsets, and Tshendzin has, during our decade-long collaboration, expressed conflicting views about whether a contrast does or does not exist in this context.

While there are no minimal pairs only distinguished by the \ipa{ɯ} vs \ipa{i} contrast in non-final syllables after palatal or alveolo-palatal consonants, it seems now clear that some words are consistently pronounced with \forme{ɯ} rather than \forme{i} in this context. This unstable contrast, which does not carry much information load, is likely to differ at the idiolectal level. The orthography used in this work reflects a normalization based on Tshendzin's judgements (rechecked several times).
\begin{paragraph}{\forme{tɕʰi°} vs \forme{tɕʰɯ°}}
Many nouns in Japhug have \forme{tɕʰi°} or \forme{tɕʰɯ°} as first element, in particular due to words of Tibetan origin containing \tibet{ཆུ་}{tɕʰu}{water} or \tibet{མཆུ་}{mtɕʰu}{lip}. The variant \forme{tɕʰi°}  is found in the great majority of these words. In some cases like \japhug{tɕʰira}{water jar} (from \tibet{ཆུ་ར་}{tɕʰu.ra}{water container}), I have used an etymologizing transcription with \forme{ɯ} (\forme{tɕʰɯra}) in previous works (in particular \citealt{jacques15japhug}), but changed my mind after careful rechecking (using for instance the stem I and III of the verb \japhug{tɕʰɯ}{gore} for comparison). 

The variant \forme{tɕʰɯ°} is restricted essentially to words with a a cluster with a uvular or a labial fricative as first element (such as \japhug{rtɕʰɯʁjɯ}{caterpillar}, \japhug{tɕʰɯχpri}{newt}, \japhug{tɕʰɯβroʁ}{type of tsampa}), but also found in \japhug{tɕʰɯmɲɯɣ}{water hole} (from \tibet{ཆུ་མིག་}{tɕʰu.mig}{water hole}) and \japhug{tɕʰɯzɯ}{type of weaving tool}.

The noun \japhug{tɯ-mtɕʰi}{mouth} (from \tibet{མཆུ་}{mtɕʰu}{lip}) has an unexpected \forme{i} (the expected form would be $\dagger$\forme{mtɕʰɯ}). It is possible that it was extracted from a compound (like \japhug{tɯ-mtɕʰirme}{moustaches}) where \ipa{ɯ} and \ipa{i} are neutralized as \phonet{i}, and then that this phonetic variant was reinterpreted as the independent root.
\end{paragraph}
\begin{paragraph}{\forme{ndʑi°} vs \forme{ndʑɯ°} } \label{sec:compounds.ndZi.ndZW}
Compounds with \forme{ndʑi°} or \forme{ndʑɯ°} as non-final element are not common, but social relation collective nouns (§ \ref{sec:social.collective}) containing the prefix \forme{kɤndʑi-}, are particular numerous. Words with the variant \forme{ndʑɯ°} are very rare: the two verbs \japhug{ndʑɯrpɯt}{be numb} and \japhug{andʑɯβri}{protect each other} (with the rare reciprocal prefix \forme{andʑɯ-} of denominal origin, see § XXX), and the nouns \japhug{ndʑɯrwɯz}{Sonchus sp} and \japhug{ndʑɯnɯ}{Angelica} (minimal pair with \forme{ndʑi-nɯ} `their breast', see § \ref{sec:verbal.suffixes.possessive.prefixes.i.W})
\end{paragraph}
\begin{paragraph}{\forme{ci°} vs \forme{cɯ°} }
Compounds comprising \japhug{cɯ}{stone} or the root of \japhug{tɯ-ci}{water} as non-final element are common, and the majority of these words have the variant \forme{ci°}, even when the noun comes from \japhug{cɯ}{stone} (for instance \japhug{ciχɕiz}{stony earth}). Exceptions include \japhug{cɯrmbɯ}{stone heap}, \japhug{scɯʁzɯɣ}{appearance} (from \tibet{སྐྱེ་གཟུགས་}{skʲe.gzugs}{physical appearance}) and \japhug{ɣɤcɯqʰlɯβ}{making noise (of water when agitated)}.\footnote{This verb is a denominal noun-ideophone compound, see § XXX.}
\end{paragraph}
\begin{paragraph}{\forme{cʰi°} vs \forme{cʰɯ°} } \label{sec:compounds.chi.chW}
Nouns with \forme{cʰi°} or \forme{cʰɯ°} include in particular Tibetan loanwords in \tibet{ཁྱི་}{kʰʲi}{dog}. The form \forme{cʰi°} occurs in most cases (for instance \japhug{cʰisɲu}{rabbies} from \tibet{ཁྱི་སྨྱོ་}{kʰʲi.smʲo}{rabbies}), but exceptions include \japhug{cʰɯmu}{female dog}, \japhug{cʰɯrdom}{roaming dog} and \japhug{nɯcʰɯra}{keep guard} (rechecked using the minimal pair \japhug{cʰi}{be sweet} vs \japhug{tɤ-cʰɯ}{wedge}).
\end{paragraph}
\begin{paragraph}{\forme{ɟi°} vs \forme{ɟɯ°} }
Words containing the status contructus of \japhug{ɟu}{bamboo} can have either the vowel \ipa{i} (as \japhug{ɟispjɤt}{person making bamboo baskets}) or \ipa{ɯ} (\japhug{ɟɯmɢom}{bamboo tweezers}). Other words in  \forme{ɟi°} include \japhug{ɟiga}{tortuous path} and its derived forms.

The root of \japhug{tɯ-rɟɯ}{fortune} commonly occur as first element of compounds, and have the vowel \ipa{ɯ}, as in \japhug{rɟɯrŋom}{coveting other people's fortune}. The syllable \forme{-rɟɯ-} also occurs as the status constructus of \japhug{tɤ-rɟit}{child} in the word \japhug{tɤrɟɯsti}{only child}.

\end{paragraph}
\begin{paragraph}{\forme{ji°} vs \forme{jɯ°} } \label{sec:compounds.ji.jW}
Compounds with non-final \forme{jɯ°}  are extremely rare, limited to the two Tibetan loanwords \japhug{jɯɣi}{writing} (from \tibet{ཡི་གེ་}{ji.ge}{letter}) and \japhug{pjɯrɯ}{coral} (from \tibet{བྱུ་རུ་}{bʲu.ru}{coral}).
\end{paragraph}
 

\subsubsection{Verbal prefixes and reduplicated forms}
The vowels \ipa{ɯ} and \ipa{ɤ} are by far the most common ones in prefixes in Japhug. In the case of verbal prefixes with palatal and alveolo-palatal onsets, the same phonological problem as in compounds (§ \ref{sec:W.i.compounds}) is observed, namely the question whether the contrast between \ipa{ɯ} and \ipa{i} have been neutralized or not, and whether transcribing the vowel as \forme{i} rather than \forme{ɯ} would be most appropriate.

The prefixes concerned by this problem are few, and are exhaustively listed in Table \ref{tab:pref.palatal}. For prefixes in \forme{ɕV-} and \forme{cʰV-}, testing the vowel is possible using minimal pairs (\japhug{ɕɯ}{who} vs \japhug{ɕi}{polar interrogative} and \japhug{cʰi}{be sweet} vs vs \japhug{tɤ-cʰɯ}{wedge} as in § \ref{sec:compounds.chi.chW}). For \forme{pjV-} and \forme{jV} prefixes, comparison is possible with the first syllable of the nouns \japhug{jɯɣi}{writing} and \japhug{pjɯrɯ}{coral} (§ \ref{sec:compounds.ji.jW}). 

The main consultants on whose expertise this work has been written have presenting conflicting judgments regarding the nature of the vowel in these prefixes. After some hesitation, Tshendzin considers it to be \forme{i} for all of these prefixes (for instance, she considers the vowel of \japhug{pjɯrɯ}{coral} to be different from that of \forme{pjɯ-ɣi} `he comes down'). I nevertheless keep here the orthographic `etymological' transcription with the vowel \forme{ɯ} for three reasons.

First, the contrast is marginal, if existent at all, for most speakers, and given the limited number of prefixes affected, it is possible to automatically change the transcription from \forme{ɯ} to \forme{i} only in these prefixes without loss of information (this can be necessary for instance as a pre-treatment for the purposes of automatic transcription training). 

Second, the prefixes in Table \ref{tab:pref.palatal} are affected by vowel fusion (§ XXX) and coalescence with the inverse prefix \forme{-wɣ} (§ XXX) like other prefixes in \forme{Cɯ-}, and these morphophonological phenomena are easier to state by assuming the underlying vowel \forme{ɯ} for all prefixes. 

Third, the assimilation from \forme{ɯ} to \forme{i} following palatal and alveolo-palatal consonants appears to be general in the morphology of the language, including partially reduplicated forms, whose rhyme is replaced by \forme{-ɯ} (§ XXX). For instance, Tshendzin the replicant here transcribed as \forme{-ɕɯ-} of the reduplicated form \japhug{nɤɕɯɕe}{go everywhere} (from \japhug{ɕe}{go}) to be phonetically closer to \japhug{ɕi}{polar interrogative} than to \japhug{ɕɯ}{who}. Adopting a transcription with \forme{i} in the prefixes in Table \ref{tab:pref.palatal} would therefore only make sense if reduplicated forms of syllable with palatal and alveolo-palatal consonants are also transcribed with \forme{i} (thus \forme{nɤɕiɕe} instead of \forme{nɤɕɯɕe}). However, such an orthography would cause unnecessary problems when automatically researching for reduplicated forms in the corpus, and since in this case too the pronunciation of the grapheme \forme{ɯ} as \ipa{i} is completely predictable, and could be corrected automatically without difficulty, I choose to keep an etymological transcription. 

\begin{table}
\caption{Verbal prefixes with palatal or alveolo-palatal onsets and high unrounded vowels} \centering \label{tab:pref.palatal}
\begin{tabular}{lll}
\lsptoprule
Prefix & Function & Reference \\
\midrule
\forme{pjɯ/i-} & Orientation prefix, series B, down & § XXX \\
\forme{cʰɯ/i-} & Orientation prefix, series B, downstream & § XXX \\
\forme{ɲɯ/i-} & Orientation prefix, series B, west; sensory & § XXX \\
\forme{jɯ/i-} & conative & § XXX \\
\forme{ɕɯ/i-} & translocative associated motion & § \ref{sec:translocative.morpho} \\
\forme{ɕɯ/i-} & apprehensive & § \ref{sec:apprehensive} \\
\forme{ɕɯ/i-} & causative & § XXX \\
\forme{ɕɯ/i-} & denominal & § XXX \\
\lspbottomrule
\end{tabular}
\end{table}

In other words, in the transcription adopted in this work I transcribe the contrast between \forme{ɯ} and \forme{i} before palatal and alveolo-palatals when non-predictable (in noun and verb stems), but neglect it and use \forme{ɯ} throughout in verbal prefixes and replicated syllables.

\subsubsection{Verbal suffixes and possessive prefixes} \label{sec:verbal.suffixes.possessive.prefixes.i.W}
A few unstressed verb suffixes have alveolo-palatal onsets and high unrounded vowels: the dual indexation suffixes \textsc{1du} \forme{-tɕi} and \textsc{2/3du} \forme{-ndʑi} (§ XXX) and the modal suffix \forme{-ci} (§ XXX). It is clear that these suffixes originally had a schwa-like vowel rather than a front vowel as in most Gyalrong languages (see § XXX and § XXX), but here too Tshendzin considers the vowel of these suffixes to be \ipa{i} rather than \ipa{ɯ} (the suffix \forme{-ci} for instance according to her resembles \japhug{ci}{one} more than \japhug{cɯ}{stone}), and since unlike verbal prefixes no vowel fusion or other phenomena takes place with these suffixes, using an etymological notation was an unnecessary complication.

In the prefixal possessive paradigm (§ \ref{sec:possessive.paradigm}), we also find a series of prefixes with palatal and alveolo-palatal onsets and high unrounded vowels: \textsc{1du} \forme{tɕi-}, \textsc{1pl} \forme{ji-} and \textsc{2/3du} \forme{ndʑi-}. Unlike verbal prefixes, but like the verbal indexation suffixes, possessive prefixes in the Kamnyu dialect of Japhug never undergo vowel fusion and remain invariable. The vowel is phonologically \ipa{i}, as shown by the minimal pair between \japhug{ndʑɯnɯ}{Angelica} and the dual possessive \forme{ndʑi-nɯ} `their breast' (§ \ref{sec:compounds.ndZi.ndZW}). For these reasons, a phonological (with \forme{-i}) rather than etymological (with \forme{-ɯ}) transcription was preferred for these prefixes.

\subsection{The contrast between \ipa{ɤ} and \ipa{e} after palatal and alveolo-palatal consonants}
The vowels \ipa{ɤ} and \ipa{e} in Kamnyu Japhug contrast in very few contexts. In word-final stressed open syllables, only \ipa{e} is found, \ipa{ɤ} being attested only in clitics and unstressed final syllables in words such as \japhug{kɯnɤ}{also}. In closed syllables, \ipa{e} is only found with the coda \ipa{-t} in the past \textsc{2sg}\fl{}3 of transitive \forme{-e} stem verbs (§ XXX); the word-final \forme{-et} vs \forme{-ɤt} are the only cases where \ipa{ɤ} and \ipa{e} are clearly contrastive.

In non-final open syllables, there are no examples of minimal pair involving a contrast between \ipa{ɤ} and \ipa{e}. Phonetic \phonet{e} is clearly heard after palatal and alveolo-palatal consonants when the vowel of the following syllable is \ipa{e}, for instance in words such as \japhug{tɕʰeme}{girl} or \japhug{sɤrɲɟele}{extend (limbs)}, but in other contexts I generally transcribe \forme{ɤ} throughout, except when the word is an obvious compound whose elements are recognized by the speakers (for instance \japhug{tɕetʰa}{soon} from \japhug{tɕe}{then} and \japhug{tʰa}{soon}). 

\subsection{The contrast between \ipa{ɤ} and \ipa{a}} \label{sec:A.vs.a.prefixes}
The contrast between \ipa{ɤ} and \ipa{a}, while clear in most contexts, is difficult to perceive in unstressed syllables when followed by a uvular (in particular when followed by a cluster with a uvular preinitial) and in syllables containing \forme{a} (due to the retrograde assimilation of height).

In syllable-by-syllable slow pronunciation, Tshendzin however makes it clear that some unstressed syllables preceding a uvular have \ipa{a} rather than \ipa{ɤ}. This concerns the indefinite possessive prefix in words such as \japhug{ta-ʁi} {younger sibling}, \japhug{ta-ʁrɯ}{horn}, \japhug{ta-ʁri}{dirt}, \japhug{ta-ʁjɯβ}{shadow}, \japhug{ta-ʁrɯm}{light, shadow}, \japhug{ta-χpi}{shape, model}, \japhug{ta-ʁrɤt}{charcoal}, \japhug{ta-ʁɟaz}{soot} and \japhug{ta-ʁa}{free time} (§ \ref{sec:inalienably.possessed}), and the fossilized prefix in nouns such as \japhug{taqaβ}{needle}. The noun \japhug{tɤ-ʁar}{wing} however has \forme{tɤ-} rather \forme{ta-}.

Another context where the variant \forme{ta-} is found is with the \forme{m-} initial nouns \japhug{ta-ma}{work} and \japhug{ta-mar}{butter}. Note that a contrast exists with the honorific noun \japhug{tɤ-ma}{mother} (§ \ref{sec:inalienably.possessed}), showing that the two allomorphs \forme{tɤ-} and \forme{ta-} are synchronically distinct in the Kamnyu variety at least for some speakers.

With some verbal prefixes, in particular the antipassive prefixes \forme{rɤ-/ra-} and \forme{sɤ-/sa-} (§ XXX) , the deexperiencer \forme{sɤ-/sa-} (§ XXX)  and the tropative \forme{nɤ(ɣ)-/na-} (§ XXX) and several denominal prefixes (§ XXX), the \forme{a} allomorph is found where followed by a stem with a uvular preinitial. For instance, the antipassive of \japhug{χtɯ}{buy} is \japhug{raχtɯ}{buy (things)} with the variant \forme{ra-} rather than \forme{rɤ-}. Not all prefixes in \forme{ɤ} present this alternation: for instance, the causative, facilitative and denominal prefixes \forme{ɣɤ-} never have a variant $\dagger$\forme{ɣa-}.

With inflectional morphology however, in particular orientation prefixes (§ XXX), no such phonologically determined alternation is found. For instance, the series A prefix \forme{tɤ-} `up' remains unchanged when preceding stems with a uvular preinitial as in \forme{tɤ-χtɯ-t-a} \textsc{pfv}-buy-\textsc{pst}:\textsc{tr}-\textsc{1sg}  I bought it', where the prefix \forme{tɤ-} is different from the corresponding series C prefix \forme{ta-} in \forme{ta-χtɯ} \textsc{pfv}:3\fl{}3'-buy `he bought it'.

