\chapter{Pronouns and Demonstratives}
\section{Pronouns and possessive prefixes} \label{sec:pronouns}

%\\ipa\{([\w-]*)\}
%\1

%(\d)\\(\w\w)\{\}
%\\textsc{\1\2}

The pronominal system of Japhug distinguishes singular, dual and plural, but does not present any inclusive / exclusive distinction as other languages such as Tshobdun (see \citealt{jackson98morphology}).


Alongside the free pronouns, a system of pronominal prefixes is used not only to express  possession on noun (see section \ref{sec:possession}  for an account of the possessive constructions), but also appears in various constructions in the verbal system. These prefixes do not distinguish the second and the third person in the dual and plural forms:


\begin{table}[h] \centering
\caption{Pronouns and possessive prefixes }\label{tab:pronoun}
\begin{tabular}{lllllllll} \lsptoprule
 Free pronoun & Prefix & \\
\midrule
 \forme{aʑo},    \forme{aj} &	\forme{a-}  &		1\sg{} \\
\forme{nɤʑo},  \forme{nɤj} &	\forme{nɤ-}  &			2\sg{} \\
\forme{ɯʑo}  &	\forme{ɯ-}  &			3\sg{} \\
\forme{tɕiʑo}  &	\forme{tɕi-}  &			1\du{} \\
\forme{ndʑiʑo}  &	\forme{ndʑi-}  &		2\du{} \\	
\forme{ʑɤni}  &	\forme{ndʑi-}  &		3\du{} \\	
\forme{iʑo}, \forme{iʑora},   \forme{iʑɤra}   &	\forme{i-}  &			1\pl{} \\
\forme{nɯʑo}, \forme{nɯʑora},   \forme{nɯʑɤra}  &	\forme{nɯ-}  &			2\pl{} \\
\forme{ʑara}  &	\forme{nɯ-}  &			3\pl{} \\
\lspbottomrule
\end{tabular}
\end{table}

Free pronouns and possessive prefixes are remarkably similar in Kamnyu Japhug. In the eastern Japhug dialects, a different \textsc{1sg} pronoun distinct from the possessive prefix  is used: \forme{ŋa} (possibly borrowed from Situ \forme{ŋā}). In the table above, we observe that all the pronouns except the third person dual and plural are formed by adding the root \forme{-ʑo} to the corresponding possessive prefix. This root is that of the generic pronoun \forme{tɯ-ʑo}, which appears mainly in gnomic statements:

%tɯʑo tɯkɤsɯso nɯ tukɯnɯti ɕti ma
%One says what he thinks
%\ipa{8_tWZo}

\begin{exe}
\ex
\gll 
tɯ-zda pjɯ́-wɣ-z-ɣɤtɕa, \textbf{tɯʑo}  ntsɯ  pjɯ-kɯ-ʑɣɤ-ɣɤŋgi   	tɕe,  pɯ-kɯ-nɯ-ɣɤtɕa 	kɯ́nɤ   	pjɯ-kɯ-ʑɣɤ-ɣɤŋgi   	tɕe,    ɯ-mbrɤzɯ   	kɯ-tu   	me  	tu-kɯ-ti   	ɲɯ-ŋu.   \\
\textsc{genr.poss}-companion \textsc{ipfv-inv-caus}-be.wrong oneself always \textsc{ipfv-genr:S/P-refl}-be.right \textsc{lnk} \textsc{pfv-genr:S/P-auto}-be.wrong also \textsc{ipfv-genr:S/P-refl}-be.right lnk \textsc{3sg.poss}-result \textsc{nmlz:S/A}-have  not.exist:\textsc{fact} \textsc{ipfv-genr}-say \textsc{sens}-be \\
\glt  `If one considers that one's companion is wrong, and always considers himself to be right even if one is wrong, there is can be no good result.' (Mouse and sparrow, 80-82)
\end{exe} 

The first and second person singular pronouns  \forme{aʑo} and \forme{nɤʑo} also have the shorter monosyllabic forms \forme{aj} and \forme{nɤj} respectively. These short forms are considerably less common in stories (in the reported speech of the characters), but appear frequently in free conversations.

Third person pronouns can be used with inanimate referents, as the third person dual \forme{ʑɤni} in example (\ref{ex:rNgW}).

\begin{exe}
\ex \label{ex:rNgW}
\gll tɕe   	rŋgɯ   	nɯ   	to-k-ɤmɯrpu-ndʑi-ci   	tɕe,   	tɕendɤre   	ʑɤni   	pjɤ-nɯ-ɴɢrɯ-ndʑi   \\
\textsc{lnk} boulder \textsc{dem} \textsc{ifr-evd}-bump.into:\textsc{recip}-\textsc{du-evd} \textsc{lnk} \textsc{lnk} \textsc{3du} \textsc{ifr-auto}-crush-\textsc{du} \\
\glt `The boulders bumped into each other and they were pulverized.' (smanmi4.82-83)
\end{exe}

\section{Genitive forms} \label{sec:pronouns.gen}
The form of pronouns and personal prefixes undergoes few morphophonological changes in combination with postpositions and relational nouns. However, in combination with the genitive postposition \forme{ɣɯ} (cf XXX), some  personal pronouns have special forms indicated in Table  \ref{tab:pronoun.gen}.

\begin{table}[h] \centering
\caption{Pronouns and possessive prefixes }\label{tab:pronoun.gen}
\begin{tabular}{lllllllll} \lsptoprule
 Free pronoun & Genitive & \\
\midrule
 \forme{aʑo}  &	\forme{aʑɯɣ}  &		\textsc{1sg} \\ 
\forme{nɤʑo}  &	\forme{nɤʑɯɣ}  &			\textsc{2sg} \\ 
\forme{ɯʑo}  &	\forme{ɯʑɤɣ}  &			\textsc{3sg} \\ 
\forme{tɕiʑo}  &	\forme{tɕiʑɤɣ}  &			\textsc{1du} \\ 
\forme{ndʑiʑo}  &	\forme{ndʑiʑɤɣ}  &		\textsc{2du} \\	 
\forme{ʑɤni}  &	\forme{ʑɤniɣɯ}  &		\textsc{3du} \\	 
\forme{iʑo}  &	\forme{iʑɤɣ}, 	\forme{iʑɤra ɣɯ}   &			\textsc{1pl} \\ 
\forme{nɯʑo}  &	\forme{nɯʑɤɣ}, 	\forme{nɯʑɤra ɣɯ}  &			\textsc{2pl} \\ 
\forme{ʑara}  &	\forme{ʑaraɣ},   \forme{ʑara ɣɯ}&			\textsc{3pl}  \\  
\lspbottomrule
\end{tabular}
\end{table}

While some degree of variation exists with dual and plural pronouns (for instance the regular \forme{iʑo ɣɯ} is found alongside \forme{iʑɤɣ} and \forme{iʑɤra ɣɯ}), for the singular pronouns only one form is attested.

\begin{exe}
\ex
\gll aʑɯɣ 	ndʐa 	ŋu 	ɕi, 	nɤʑɯɣ 	ndʐa 	ŋu, 	ɤj 	mɯ́j-tso-a   \\
\textsc{1sg:gen} reason be:\textsc{fact} \textsc{qu} \textsc{2sg:gen} reason be:\textsc{fact} \textsc{1sg} \textsc{neg:sens}-understand-\textsc{1sg} \\
\glt  `I don't know if it is because of me, or because of you.' (that the phone line is not working well) (phone conversation, 2011) %\wav{8_ndzxa})
\end{exe} 

In the genitive forms of the pronouns, the vowel of the genitive marker is generally dropped, and the pronominal root \forme{-ʑo} undergoes vowel change to \forme{-ʑɯɣ} (in the case of first and second person) and \forme{-ʑɤɣ} (in other forms). Note that \forme{ʑaraɣ} is the only case of the rhyme \ipa{aɣ} in Japhug.

\section{The emphatic use of pronouns} \label{sec:pronouns.emph}
In addition to their referential and anaphoric use, pronouns in Japhug can be used in an emphatic way in combination with the particle \forme{ʑo}, as in  (\ref{ex:WZo.Zo}).

\begin{exe}
\ex \label{ex:WZo.Zo}
\gll aʑo ɯʑo ʑo kɤ-mto mɯ-pɯ-rɲo-t-a. \\
\textsc{1sg} \textsc{3sg} \textsc{emph} \textsc{inf}-see \textsc{neg-pfv}-experience-\textsc{pst:tr-1sg} \\
\glt `I never saw it itself.' (24-kWmu, 7)
\end{exe} 

In combination with the autobenefactive \forme{nɯ-} on the verb, pronouns express the meaning `do X on one's own'. In the case of transitive verbs, the pronoun in this use does not take the ergative even if the referent is the transitive subject (example \ref{ex:pjWnWtCAtnW}, where \japhug{tɕɤt}{take out} is transitive).

\begin{exe}
\ex
\gll tɕe ɲɯ-tɯ-nɤm qhe, tɕe ʑara ku-nɯ-nɯɣi-nɯ ŋu ɕi? \\
\textsc{lnk} \textsc{ipfv:east}-2-chase[III] \textsc{lnk} \textsc{lnk} \textsc{3pl} \textsc{ipfv:west}-auto-come.back-\textsc{pl} be:\textsc{fact} \textsc{qu} \\
\glt `Do you chase them, or do they come back home on their own?' (taRrdo conversation, 29)
\end{exe} 

\begin{exe}
\ex \label{ex:pjWnWtCAtnW}
\gll tɕe lu-nɯ-rɤji-nɯ tɕe, nɯ-kɤ-ndza nɯra ʑara pjɯ-nɯ-tɕɤt-nɯ pjɤ-ŋu tɕe \\
\textsc{lnk} \textsc{ipfv-auto}-plant.crops-\textsc{pl} \textsc{lnk} \textsc{3pl.poss-nmlz:P}-eat \textsc{dem:pl} \textsc{3pl} \textsc{ipfv-auto}-take.out-pl \textsc{ifr.ipfv}-be \textsc{lnk} \\
\glt `They planted crops, and earned their food on their own.' (about lepers, who were settled in the special place by the government, 25-khArWm, 70)
\end{exe}

This construction is also attested with first of second person pronouns, as in (\ref{ex:aZo.Zo}).

\begin{exe}
\ex \label{ex:aZo.Zo}
\gll aʑo ʑo nɯnɯ ɕ-pjɯ-sat-a ra \\
\textsc{1sg} \textsc{emph} \textsc{dem} \textsc{transloc-ipfv}-kill-\textsc{1sg} have.to:\textsc{fact} \\
\glt `I have to kill her myself.' (140504 baixuegongzhu, 117)
\end{exe}

\section{Interrogative pronouns}
The interrogative pronouns in Japhug are indicated in Table \ref{tab:interrog.pronoun}. These pronouns are used in independent interrogative clauses (\ref{ex:tChi.pWNu}), in subordinate clauses (\ref{ex:tChi.kWNu}), and also in correlatives (\ref{ex:NotCu.WsAzrAZi}), and also occur to express indefinitness.

\begin{exe}
\ex \label{ex:tChi.pWNu}
\gll
tɕe mɤʑɯ tɕʰi pɯ-ŋu? \\
\textsc{lnk} yet what \textsc{pst.ipfv}-be \\
\glt `What was there (after this one)?' (12-ndZiNgri, 100)
\end{exe}  

\begin{exe}
\ex \label{ex:tChi.kWNu}
\gll ɯʑo tɕʰi kɯ-ŋu nɯ ko-tso-nɯ tɕe tɕe cʰɤ́-wɣ-tɕɤt \\
\textsc{3sg} what \textsc{nmlz}:S/A-be \textsc{dem} \textsc{ifr}-understand-\textsc{pl} \textsc{lnk} \textsc{lnk} \textsc{ifr:downstream-inv}-take.out \\
\glt `They understood what he was, and expelled him (from their group).' (140427 hanya yu gezi-zh, 19)
\end{exe}  

\begin{exe}
\ex \label{ex:NotCu.WsAzrAZi}
\gll 
ɯ-pʰoŋbu tɕʰi kɯ-fse nɯ, ŋotɕu ɯ-sɤz-rɤʑi nɯnɯ ɣɯ kɯ-nɯtsa kɯ-fse ɲɯ-ɕti tɕe \\
3sg.poss-body what \textsc{nmlz}:S/A-be.like \textsc{dem} where \textsc{3sg-nmlz:oblique}-remain \textsc{dem} \textsc{gen}  nmlz:S/A-fit \textsc{nmlz}:S/A-be.like \textsc{sens}-be:\textsc{affirm} \textsc{lnk} \\
\glt `The way its body is like is well-fitted to the place where it lives.' (19-rNamoN, 24)
\end{exe}  

\begin{table}[h] \centering
\caption{Interrogative pronouns }\label{tab:interrog.pronoun}
\begin{tabular}{lllllllll} \lsptoprule
\japhug{tɕʰi}{what} \\
\japhug{ɕɯ}{who} \\
\japhug{tʰɤstɯɣ}{how many} \\
\japhug{tʰɤjtɕu}{when} \\
\japhug{ŋotɕu}{where}, \japhug{ŋoj}{where} \\
\japhug{tɕʰindʐa}{why} \\
\lspbottomrule
\end{tabular}
\end{table}


\subsection{\japhug{tɕʰi}{what}} \label{sec:tChi}
The interrogative pronoun  `what' considerably varies across Japhug dialects. In Kamnyu we find \forme{tɕʰi}, apparently borrowed from Tibetan \forme{tɕʰi}. Neighbouring dialects of Gdongbrgyad area have either \forme{tsʰi} (in Mangi) or \forme{tʰi} (in Rqaco), which represents the original Rgyalrongic root for this interrogative pronoun (cognate with Tibetan \tibet{ཆི་}{tɕʰi}{what} and Limbu \forme{the}). Even in the Kamnyu dialect, the form \forme{tsʰi-} is directly attested in the indefinite \japhug{tsʰitsuku}{some}. Mangi Japhug shares with Kamnyu the sound change \forme{*tʰi} $\rightarrow$ \forme{tsʰi} which also affect the verb \japhug{tsʰi}{drink}.

The Eastern dialects of Gsardzong and Datshang have \forme{xto} instead, a word of unknown etymology.

In the Kamnyu dialect, \japhug{tɕʰi}{what} is by far the most common interrogative pronoun in the corpus. In interrogative clauses, it can be used to ask about objects, non-human animals (\ref{ex:nAmbro}) and names of persons (\ref{ex:tChi.tWrmi}).

\begin{exe}
\ex \label{ex:nAmbro}
\gll
nɤʑo nɤ-mbro nɯ tɕʰi ŋu \\
\textsc{2sg} 2sg.poss-horse \textsc{dem} what be:\textsc{fact} \\
\glt `Who is your horse?' (about a sentient horse, 2003smanmi-tamu, 53)
\end{exe}  

\begin{exe}
\ex \label{ex:tChi.tWrmi}
\gll tɕʰi tɯ-rmi? \\
what 2-be.called:\textsc{fact} \\
\glt `What is your name?' (heard in context)
\end{exe}  

As in many languages, this interrogative pronoun (instead of the pronoun \japhug{ɕɯ}{who}) is also used in questions about classification of persons (\citealt{idiatov07nonselective}), including social affiliation (\ref{ex:tChi.WrWG}, and \ref{ex:tChi.kWNu} above) and biological affiliation (\ref{ex:tChi.tosci}).

\begin{exe}
\ex \label{ex:tChi.WrWG}
\gll ɯtɤz nɯʑo tɕʰi ɯ-rɯɣ tɯ-ŋu-nɯ? \\
finally \textsc{2pl} what  3sg.poss-race  2-be:\textsc{fact}-\textsc{pl} \\
\glt `Finally, what race (of being) are you?' (smanmi2003, 172)
\end{exe}  

There is no specific interrogative pronoun to ask about manner like English `how', and Japhug expresses this meaning by combining \forme{tɕʰi} with the verbs \japhug{fse}{be like...} or \japhug{stu}{do like...}, as in examples (\ref{ex:tChi.tAtWfsendZi}), (\ref{ex:tChi.atAfsej}) and (\ref{ex:tChi.Zo.tuwGBzu}).

\begin{exe}
\ex \label{ex:tChi.tAtWfsendZi}
\gll a-ʁi, ki kɯ-fse tɤjpɣom kɯ-wxti nɯtɕu, kɤ-ɕe tɕʰi tɤ-tɯ-fse-ndʑi?? \\
\textsc{1sg.poss}-younger.sibling this \textsc{nmlz}:S/A-be.like ice \textsc{nmlz}:S/A-be.big \textsc{dem:loc} inf-go what \textsc{pfv}-2-be.like-\textsc{du} \\
\glt `Sister, how did you cross such a big block of ice?' (stodtWphu2005, 156)
\end{exe}  

   \begin{exe}
\ex \label{ex:tChi.atAfsej}
\gll  kɤ-pʰɣo tɕʰi a-tɤ-fse-j    \\
inf-flee what \textsc{irr-pfv}-be.like-\textsc{1pl} \\
\glt  `How do we flee?' (Slobdpon 69)
\end{exe} 

\begin{exe}
\ex \label{ex:tChi.Zo.tuwGBzu}
\gll nɤ-smɤn tɤ-sɯ-βzu-t-a ri maka mɯ́j-phɤn, tɕe tɕʰi ʑo tú-wɣ-stu phɤn \\
\textsc{3sg.poss}-medicine \textsc{pfv-caus}-make-\textsc{pst:tr-1sg} but at.all \textsc{neg:sens}-be.efficient \textsc{lnk} what \textsc{emph} \textsc{ipfv-inv}-do.like be.efficient:\textsc{fact} \\
\glt `I had medicine made for you but it does not work, how should we do for it to work?' (nyima wodzer 2002, 22) 
\end{exe}  

 
The pronoun \japhug{tɕʰi}{what} on its own can occur in questions about the reason or the purpose of a particular state of affair, as in (\ref{ex:tChi.Zo.tuwGBzu}) and (\ref{ex:tChi.YWtWnAre}).

\begin{exe}
\ex \label{ex:tChi.Zo.tuwGBzu}
\gll  aʑo tɕʰi a-pɯ-ŋu-a? \\
\textsc{1sg} what \textsc{irr-ipfv}-be-1sg \\
\glt `How can it be me?' (2003sras, 61)
\end{exe}  

\begin{exe}
\ex \label{ex:tChi.YWtWnAre}
\gll  a-tɤɕime, tɕʰi ɲɯ-tɯ-nɤre ŋu? \\
 \textsc{1sg.poss}-lady what \textsc{sens}-2-laugh be:\textsc{fact} \\
 \glt `My lady, why are you laughing?'  (Not `what are you laughing at ?', 2002qaCpa, 102)
\end{exe}  

When referring to purpose or reason, it is possible to combine  \japhug{tɕʰi}{what} with the nouns \japhug{ɯ-spa}{its material} and \japhug{ɯ-ndʐa}{its reason} (as the pronoun \japhug{tɕʰindʐa}{why})  respectively, as in (\ref{ex:tChi.Wspa.pWNu}) and (\ref{ex:tChi.YWtWɣAwu}). Note that examples (\ref{ex:tChi.YWtWnAre}) and (\ref{ex:tChi.YWtWɣAwu}) are from the same story, just a few lines away, in the same context; the construction in (\ref{ex:tChi.YWtWɣAwu}) is a more explicit variant of that in (\ref{ex:tChi.YWtWnAre}).

\begin{exe}
\ex \label{ex:tChi.Wspa.pWNu}
\gll tɕe tɕʰi ɯ-spa pɯ-ŋu mɤ-xsi ma tɕe nɯ kɯ-fse pjɤ-tu  \\
\textsc{lnk} what \textsc{3sg.poss}-material \textsc{pst.ipfv}-be \textsc{neg-genr}:know \textsc{lnk} \textsc{lnk} \textsc{dem} \textsc{nmlz}:S/A-be.like \textsc{ifr.ipfv}-exist \\
\glt `It is not known what it was for, but there was something like that.' (hist140522 GJW, 18)
\end{exe}  

\begin{exe}
\ex \label{ex:tChi.YWtWɣAwu}
\gll tɕʰindʐa ɲɯ-tɯ-ɣɤwu ŋu? \\
why \textsc{sens}-2-cry be:\textsc{fact} \\
\glt `Why are you crying?' (2002qaCpa, 94)
\end{exe} 

The pronoun \forme{tɕʰi} takes case marking with genitive \forme{ɣɯ} and the instrumental/ergative \forme{kɯ}, as in (\ref{ex:tChi.kW}).

\begin{exe}
\ex \label{ex:tChi.kW}
\gll tɕe tɕʰi kɯ tu-sɯ-βze ŋu mɤxsi ma nɯ kɯ-fse nɯ, sɯku ri ku-ndzoʁ ŋu \\
\textsc{lnk} what \textsc{erg} \textsc{ipfv}-\textsc{caus}-make[III] be:\textsc{fact} \textsc{neg}-\textsc{genr}-know \textsc{lnk} \textsc{dem} \textsc{nmlz}:S/A-be.like \textsc{dem} top.of.trees \textsc{loc} \textsc{ipfv}-\textsc{anticaus}:attach be:\textsc{fact} \\
\glt `I don't what it (the wasp) uses to make it (its nest), it is attached on trees.' (26-ndzWrnaR, 55)
\end{exe} 
In combination with the adverb \forme{jarma} / \japhug{jamar}{about}, it can be used to indicate a quantity, instead of \japhug{tʰɤstɯɣ}{how many, how much} (section \ref{sec:thAstWG}).

\begin{exe}
\ex \label{ex:tChi.jamar}
\gll tu-ɕtʂam-a tɕe tɕʰi jamar ʑo ɣɤʑu kɯ? \\
\textsc{ipfv}-measure[III]-\textsc{1sg} \textsc{lnk} what about \textsc{emph} exist:\textsc{sens} \textsc{sfp} \\
\glt `I will measure it with a scoop to see how much (gold) there is.' (140512alibaba-zh, 59)
\end{exe}  

\begin{exe}
\ex \label{ex:tChi.jamar.kondza}
\gll kʰɯtsa ɯ-ŋgɯ tɯ-ci tu-rku-nɯ tɕe, nɯnɯtɕu tɤŋe nɯ pjɯ-sɯ-ntɕʰɤr-nɯ tɕe, tɕe tɕʰi jamar ko-ndza nɯnɯ, nɯnɯ ɯ-ŋgɯ nɯtɕu pjɯ-ru-nɯ tɕe,  nɯnɯ tu-rtoʁ-nɯ pjɤ-ŋgrɤl.   \\
bowl \textsc{3sg}-inside \textsc{indef.poss}-water \textsc{ipfv}-put.in-\textsc{pl} \textsc{lnk} \textsc{dem:loc} sun \textsc{dem} \textsc{ipfv-caus}-illuminate-\textsc{pl} \textsc{lnk} \textsc{lnk} what about \textsc{ifr}-eat \textsc{dem} \textsc{dem} \textsc{3sg}-inside \textsc{ipfv}:\textsc{down}-look.at-\textsc{pl} dem \textsc{ipfv}-see-\textsc{pl} \textsc{ifr.ipfv}-be.usually.the.case \\
\glt `They used to put water in a bowl and let the sunlight reflect into it; they could see how much (of the sun) had been occulted (`eaten' by the eclipse).' (29-mWBZi, 130)
\end{exe}  

\begin{exe}
\ex
\gll  zgo 	tʰɤstɯɣ 	ja-nnɯ-pɣaʁ-ndʑi, 	tɯ-ci 	tɕʰi 	jarma 	ja-nnɯ-pjɤl-ndʑi 	mɤ-xsi 	ma,       \\
 mountain how.many \textsc{pfv}:3$\rightarrow$3'-\textsc{auto}-turn.over-\textsc{du} \textsc{indef.poss}-water what about \textsc{pfv}:3$\rightarrow$3'-\textsc{auto}-cross-\textsc{du} \textsc{neg-genr}:know \textsc{lnk} \\
\glt `It is not known how many mountains they crossed, around how many rivers they went.'  (2002qajdo, 50)
\end{exe}  

It is possible to combine \forme{tɕʰi jamar} with a adjective to express approximate comparison, as in (\ref{ex:tChi.kWzri}).

\begin{exe}
\ex \label{ex:tChi.kWzri}
\gll lɯlu ɣɯ tɕe ɯʑo ɯ-pʰoŋbu tɕʰi kɯ-zri jamar ɯ-jme nɯ kɯnɤ zri ri \\
cat \textsc{gen}\textsc{lnk} \textsc{3sg} \textsc{3sg.poss}-body nmlz:S/A \textsc{3sg.poss}-tail \textsc{dem} also be.big:\textsc{fact} but \\
\glt `The cat, its body is about as long as its tail, but...' (27-qartshAz, 219)
\end{exe}  

In correlative clauses, the pronoun \japhug{tɕʰi}{what} can also be used to refer to a quantity without the adverb \japhug{jamar}{about} (example \ref{ex:tChi.tAkWsci}).

\begin{exe}
\ex \label{ex:tChi.tAkWsci}
\gll  
tɤ-rɟit tɕʰi tɤ-kɯ-sci nɯ ʑo ɣɯ-tɕɤt kɯ-ra pjɤ-ɕti tɕe,   \\
\textsc{indef.poss}-child what \textsc{pfv-nmlz:S/A}-be.born \textsc{dem} \textsc{emph} \textsc{inv}-take.out:\textsc{fact} \textsc{inf:stat}-have.to \textsc{ipfv.ifr}-be:\textsc{affirm} \\
\glt `However many children were born, one had to raise them.' (tApAtso kAnWBdaR I, 9)
\end{exe}  

However, in independent interrogative clauses, \japhug{tɕʰi}{what} cannot refer to quantities. Sentence (\ref{ex:tChi.tosci}) thus can only mean `Was it a boy or a girl' not `How many children did she have?'.

\begin{exe}
\ex \label{ex:tChi.tosci}
\gll  ɯ-rɟit tɕʰi to-sci \\
\textsc{3sg.poss}-child what \textsc{ifr}-be.born \\
\glt `Was it a boy or a girl?'
\end{exe}  

\subsection{\japhug{ɕɯ}{who}}
The interrogative pronoun \japhug{ɕɯ}{who} occurs in questions about the identification of a human referent. It can occur in all syntactic roles, and does not have special ergative or genitive forms (see examples \ref{ex:CW.kW} and \ref{ex:CW.GW}). It is the probable cognate of a etymon widespread in the Trans-Himalayan family (for instance, Tibetan \tibet{སུ་}{su}{who}).

\begin{exe}
\ex  \label{ex:CW.tWNu}
\gll ma-tɯ-nɯqaɟy ma ɕɯ tɯ-ŋu mɤ-xsi \\
\textsc{neg:imp}-2-fish \textsc{lnk} who 2-be:\textsc{fact} \textsc{neg-genr}:know   \\
\glt `Don't fish, I don't who you are.' (gesar, 369)
\end{exe}  

\begin{exe}
\ex  \label{ex:CW.kW}
\gll  mɤ-ta-mbi nɤʑo qaɕpa ɕɯ kɯ tɯ́-wɣ-mbi    \\
\textsc{neg}-1$\rightarrow$2-give:\textsc{fact} \textsc{2sg} frog who \textsc{erg} 2-\textsc{inv}-give:\textsc{fact}  \\
\glt `We won't give her to you, who would give her to you, a frog?'   (2002 qaCpa, 09)
\end{exe} 

 
\begin{exe}
\ex  \label{ex:CW.GW}
\gll  ɕɯ ɣɯ ʑo ɲɯ-kʰam-a ra kɯɣe?    \\
who \textsc{gen} \textsc{emph} \textsc{ipfv}-give:III-\textsc{1sg} \textsc{sfp} \\
\glt `Whom should I give (her) to (in marriage)?' (140508 benling gaoqiang de si xiongdi-zh, 222)
\end{exe}  

\subsection{\japhug{tʰɤstɯɣ}{how many, how much} and \japhug{tʰɤjtɕu}{when}} \label{sec:thAstWG}
To ask about precise quantities, \japhug{tʰɤstɯɣ}{how many, how much} occurs rather than \forme{tɕʰi jamar} as seen above (section \ref{ex:tChi.jamar}).

\begin{exe}
\ex \label{ex:thAstWG.tWkhAm}
 \gll    nɤʑo 	tʰɤstɯɣ 	tɯ-kʰɤm?    \\
 you how.much 2-give[III]:\textsc{fact}  \\
\glt  `How much (money) do you give (for it)?' (Bargaining, 13)
\end{exe} 

The pronoun \forme{tʰɤstɯɣ} has a conjunct form \forme{tʰɤstɯ-} when used with classifiers (in \ref{ex:thAstWmaR}, with the classifier \ipa{-maʁ} `size of shoes' from Chinese \zh{码} \forme{mǎ}).

 \begin{exe}
\ex \label{ex:thAstWmaR}
 \gll   nɤ-xtsa nɯ tʰɤstɯ-maʁ tu-tɯ-ŋge ŋu   \\
\textsc{2sg.poss}-shoe \textsc{dem} how.many-size \textsc{ipfv}-2-wear[III] be:\textsc{fact} \\ 
\glt `What is the size of your shoes?'  (Conversation, 2015)
\end{exe} 

Combined with the noun \japhug{tɤ-rʑaʁ}{time}, 	\forme{tʰɤstɯɣ} can be used to ask about a lenght of time (\ref{ex:thAstWG}).

\begin{exe}
\ex \label{ex:thAstWG}
 \gll   nɤʑo 	tɤ-rʑaʁ 	tʰɤstɯɣ 	jamar 	tɤ-tsu tɕe 	kɤ-tɯ-spa-t?  \\
 you \textsc{indef.poss}-time how.many about \textsc{pfv}-pass \textsc{lnk} \textsc{pfv}-2-be.able-\textsc{pst:tr} \\
\glt   `How long did you need to learn it?' (elicited)
\end{exe} 

The phrase \forme{tɤ-rʑaʁ tʰɤstɯɣ} (or alternatively \forme{tɯtsʰot tʰɤstɯɣ}) in collocation with the verb \japhug{zɣɯt}{reach}, is also employed for asking about clock time, as in (\ref{ex:thAstWG.kozGWt})

 \begin{exe}
\ex \label{ex:thAstWG.kozGWt}
 \gll   tɤ-rʑaʁ 	tʰɤstɯɣ ko-zɣɯt? \\
  \textsc{indef.poss}-time how.many  \textsc{ifr}-reach \\
  \glt `What is the time?' (heard in context)
  \end{exe} 
  
This meaning can also be expressed by the pronoun \japhug{tʰɤjtɕu}{when}  (example \ref{ex:thAjtCu}).

\begin{exe}
\ex \label{ex:thAjtCu}
\gll  tʰɤjtɕu 	lɤ-tɯ-nɯɣe 	pɯ-ŋu 	ra 	nɤ?    \\
 when \textsc{pfv}-2-come.back[II] \textsc{pst.ipfv}-be \textsc{pl} \textsc{sfp} \\
\glt  When did you come back home? (taRrdo conversation, 01)
\end{exe} 

The element \ipa{tʰɤ-} in the pronouns \japhug{tʰɤjtɕu}{when}  and \japhug{tʰɤstɯɣ}{how many, how much} is the \textit{status constructus} form of proto-Japhug \forme{*tʰi}, the inherited form of the pronoun `what' (see section \ref{sec:tChi}). The element \forme{-tɕu} in \japhug{tʰɤjtɕu}{when} is related to the locative \forme{tɕu} (see XXX).

\subsection{\japhug{ŋotɕu}{where}}

The interrogative pronoun \japhug{ŋotɕu}{where} and its reduced form \forme{ŋoj} can be used to ask either about a location (\ref{ex:NotCu.kutWrAZi}), a direction towards (examples \ref{ex:NotCu.tWCe} and \ref{ex:Noj.nari}) or from (\ref{ex:NotCu.jAtWGenW}) a certain place. The second syllable of this pronoun \forme{-tɕu} comes from the locative postposition \forme{tɕu}, but the first part is etymologically obscure.
 
\begin{exe}
\ex \label{ex:NotCu.kutWrAZi}
\gll     ŋotɕu ku-tɯ-rɤʑi?   \\
  where \textsc{pres.egoph}-2-stay \\
\glt `Where are you?" (Conversation, 2005)
\end{exe} 

\begin{exe}
\ex \label{ex:NotCu.tWCe}
\gll   ŋotɕu tɯ-ɕe? \\
 where 2-go:\textsc{fact} \\
\glt `Where are you going to?' (Common greeting used when one meets someone on the road)
 \end{exe} 
 
\begin{exe}
\ex \label{ex:Noj.nari}
\gll     qala ŋoj nɯ-ari  \\
  rabbit where \textsc{pfv:west}-go[II] \\
\glt Where did the rabbit go?  (qala2002, 21)
\end{exe} 

\begin{exe}
\ex \label{ex:NotCu.jAtWGenW}
\gll  nɯʑɤra ŋotɕu jɤ-tɯ-ɣe-nɯ? ŋotɕu ɕ-pɯ-tɯ-tu-nɯ? \\
\textsc{2pl} where \textsc{pfv}-2-come[II]-\textsc{pl} where \textsc{transloc-pfv}-2-exist-\textsc{pl} \\
\glt `Where are you from? Where have you been?' (2003sras, 57)
\end{exe} 

The pronoun \japhug{ŋotɕu}{where} is not exclusively used in question about place or direction, we also find it in the expression in (\ref{ex:NotCu.YWNgrAl}).

 \begin{exe}
\ex \label{ex:NotCu.YWNgrAl}
\gll     kɯki 	ŋotɕu 	ɲɯ-ŋgrɤl?   \\
 this where \textsc{ipfv}-be.usually.the.case \\
\glt `How could this be possible?'  (qajdoskAt 2002, 32)
\end{exe} 

This sentence is used to express indignation (as in Chinese \zh{哪有这样的道理?}).\footnote{In the story from which it is quoted, the husband says this sentence after his wife, quoting the words of a raven, says that she will have luck, not her husband, who thus reacts in anger. }

The pronoun \forme{ŋotɕu} in \textit{status constructus} form \forme{ŋɤtɕɯ-} occurs in the delocutive expression \japhug{ŋɤtɕɯkɤti,kʰɯ}{obey to everything} in a compound with the infinitive \forme{kɤ-ti} of the verb \japhug{ti}{say}, and in collocation with \japhug{kʰɯ}{agree}, as in (\ref{ex:NotCu.YWNgrAl}).\footnote{the causative \japhug{ŋɤtɕɯkɤti,sɯkʰɯ}{cause to obey to everything} also exists.} 
This expression originates presumably from a phrase such as `agree (\forme{kʰɯ}) to whatever (\forme{ŋotɕu}) he says (\forme{ti})'.

 \begin{exe}
\ex \label{ex:NotCu.YWNgrAl}
\gll  ɯ-tɕɯ kɯβde nɯra wuma ʑo ŋɤtɕɯkɤti pjɤ-kʰɯ-nɯ  \\
3sg.poss-son four dem:pl really emph obey.to.everything(1) \textsc{ifr.ipfv}-obey.to.everything(2)-\textsc{pl} \\
\glt `His four sons were very obedient.' (140508 benling gaoqiang de si xiongdi-zh, 15)
\end{exe} 



\section{Indefinitive pronouns} \label{sec:indef.pro}
 Japhug has a handful of indefinite pronouns, indicated in Table \ref{tab:indef.pronoun}. They do not form a complete paradigm, and other constructions, in particular generic nouns and free relatives occur to express meanings for which no indefinite pronoun exists (see section XXX).

There are no negative indefinite pronouns, and indefinite pronouns are almost never under the scope of negation (except in translations from Chinese). They also never occur as standard of comparison.\footnote{Examples such as `In Freiburg the weather is better than anywhere in Germany' (\citealt[2]{haspelmath97indef}) would not be expressible with an indefinite pronoun, see section XXX.}
 

\begin{table}[H] \centering
\caption{Indefinite pronouns }\label{tab:indef.pronoun}
\begin{tabular}{lllllll} \lsptoprule
\forme{tʰɯci}, \japhug{tʰɯtʰɤci}{something} \\
\japhug{tsʰitsuku}{whatever} \\
\japhug{ɕɯmɤɕɯ}{whoever, anybody} \\
\japhug{ciscʰiz}{somewhere} \\
\japhug{ŋotɕuŋɤndɤt}{everywhere, anywhere} \\
\lspbottomrule
\end{tabular}
\end{table}

\subsection{\japhug{tʰɯci}{something} }
The indefinite pronoun \japhug{tʰɯci}{something} derives from the \textit{status constructus} of the proto-Japhug pronoun \forme{*tʰi} `what' (see \ref{sec:tChi} above) with the indefinite determiner and numeral \japhug{ci}{one}. Its reduplicated form \forme{tʰɯtʰɤci} has an irregular vocalism \ipa{ɤ} ($\dagger$\forme{tʰɯtʰɯci} would have been expected instead).

 It can designate specific referents, whose nature is known to the speaker but unknown to the addressee (as in \ref{ex:thWthAci.Zo.pjWtu}),\footnote{Example (\ref{ex:thWthAci.Zo.pjWtu}) is from a tale about a rabbit tricking a snow leopard; the difference of knowledge between the speaker and the addressee concerning the nature of the `something' is crucial to the plot. }.

\begin{exe}
\ex  \label{ex:thWthAci.Zo.pjWtu}
\gll tu-nɯsman-a jɤɣ ri, mɤʑɯ ɯ-ftɕaka tsuku pjɯ-tu ra wo, tɕe tʰɯtʰɤci ʑo pjɯ-tu ra \\
\textsc{ipfv}-treat-\textsc{1sg} be.possible:\textsc{fact} but yet \textsc{3sg.poss}-manner some \textsc{ipfv}-exist have.to:\textsc{fact} \textsc{sfp} \textsc{lnk} something \textsc{emph} \textsc{ipfv}-exist have.to:\textsc{fact} \\
\glt `I can treat (your illness), but yet another method is needed, something (else) is needed.'  (140427 qala cho kWrtsAg, 48-49)
\end{exe}

The pronoun \forme{tʰɯci} also occurs to refer to things whose name is unknown to the speaker (as in \ref{ex:gser.zhwa} and \ref{ex:thWci.khWtsa}), even if he/she may have seen the object.
 
\begin{exe}
\ex \label{ex:gser.zhwa}
\gll tɕe nɯ nɯ-rte nɯ tɕʰi ŋu ma tʰɯci ci ``-ʑa" tu-ti ŋu, χsɤrʑa! \\
\textsc{lnk} \textsc{dem} \textsc{3pl.poss}-hat \textsc{dem} what be:\textsc{fact} \textsc{lnk} something \textsc{indef} ... \textsc{ipfv}-say be:\textsc{fact} golden.hat \\
\glt `How is their hat (called), something in `ʑa'.... yes, \tibet{གསེར་ཞྭ་}{gser.ʑʷa}{golden hat}!' (30-mboR, 102)
\end{exe}

\begin{exe}
\ex \label{ex:thWci.khWtsa}
\gll  tɕe tɤ-ndʑɯɣ nɯ kɯnɤ, tʰɯci kʰɯtsa kɯ-fse ɯ-ŋgɯ tu-rku-nɯ tɕe   \\
\textsc{lnk} \textsc{indef.poss}-resin \textsc{dem} also something bowl \textsc{nmlz}:S/Abe.like \textsc{3sg}-inside \textsc{ipfv}-put.in-\textsc{pl} \textsc{lnk}   \\
\glt `The resin, people put it into something like a bowl.'' (07-tAtho, 44)
\end{exe}

It is also used for non-specific referents whose nature is entirely unknown, as in  (\ref{ex:thWthAci.tannWrkunW}) and (\ref{ex:thWmqlaR}).

\begin{exe}
\ex \label{ex:thWthAci.tannWrkunW}
\gll   tɕe mɤʑɯ tʰɯtʰɤci ta-nnɯ-rku-nɯ kɯma  \\
\textsc{lnk} yet something \textsc{pfv}:3$\rightarrow$3'-\textsc{auto}-put.in-\textsc{pl} \textsc{sfp} \\
\glt `They also probably gave them something else.' (02-deluge2012, 120)
 \end{exe}
 
  \begin{exe}
\ex \label{ex:thWmqlaR}
\gll 
 tʰɯ-mqlaʁ tʰɯ-mqlaʁ ma tʰɯci fse ci ndʐa cʰɯ-ɕe ɕti \\
 \textsc{imp}:swallow  \textsc{imp}:swallow \textsc{lnk} something be.like:\textsc{fact} \textsc{indef} reason \textsc{ipfv:downstream}-go be:\textsc{affirm:fact} \\
\glt `Swallow it, swallow it, it comes down (into your throat) for some reason.' (2005-stod-kunbzang, 87)
  \end{exe}

The reduplicated form \forme{tʰɯtʰɤci}, especially in combination with \japhug{fse}{be like}, can also mean `whatever (happened)', as in (\ref{ex:thWthAci.kWfse}).
 
 \begin{exe}
\ex \label{ex:thWthAci.kWfse}
\gll  slama ra ɣɯ tʰɯtʰɤci kɯ-fse, kɤ-rɤ-βzjoz ra ɲɯ-stu mɯ́j-stu-nɯ, nɯ-stu ɲɯ-nɤma-nɯ mɯ́j-nɤma-nɯ,  nɯnɯra nɯ-pʰama ra nɯ-ɕki kɯ-rɤfɕɤt ɲɯ-ra. \\
student \textsc{pl} \textsc{gen} something \textsc{nmlz}:S/A-be.like \textsc{inf-antipass}-learn \textsc{pl} \textsc{sens}-try.hard-\textsc{pl} \textsc{neg:sens}-try.hard-\textsc{pl} \textsc{3sg.poss}-right \textsc{sens}-do-\textsc{pl} \textsc{neg:sens}-do-\textsc{pl} \textsc{dem:pl} \textsc{3pl.poss}-parent \textsc{pl} \textsc{3pl-dat} \textsc{genr}:S/P-tell \textsc{sens}-have.to \\
\glt `One has to tell the parents whatever concerns the students, whether they study seriously and try hard or not.'   (150901 tshuBdWnskAt, 18)
 \end{exe}
  
The pronoun \japhug{tʰɯci}{something}  can also occur as determiner of a noun (or a headless relative clause) as an indefinite determiner `some'. This use is found in native texts (example \ref{ex:thWci.khWtsa} above with the relative \forme{kʰɯtsa kɯ-fse} `which is like a bowl'), but it is most common in texts translated from Chinese, always with the indefinite determiner \japhug{ci}{one} after the noun or the relative clause, as in (\ref{ex:thWci.laXCi}) and (\ref{ex:thWthAci.akAspa}). 

   \begin{exe}
\ex \label{ex:thWci.laXCi}
\gll   tʰɯci laχɕi ci ɕ-pɯ-nɯ-βzjoz-nɯ tɕe, jɤ-ɕe-nɯ ra \\
something trade \textsc{indef} \textsc{transloc-imp-auto}-learn-\textsc{pl} \textsc{lnk} \textsc{imp}-go-\textsc{pl} have.to:\textsc{fact} \\
\glt `Go and learn some trade!' (140508 benling gaoqiang de si xiongdi-zh, 29)
 \end{exe}

   \begin{exe}
\ex \label{ex:thWthAci.akAspa}
\gll 
 laχɕi ci pjɯ-βzjoz-a, tʰɯci a-kɤ-spa ci a-pɯ-tu ɲɯ-ra  \\
 trade \textsc{indef} \textsc{ipfv}-learn-\textsc{1sg} something \textsc{1sg.poss-nmlz:P}-be.able \textsc{indef} \textsc{irr-pfv}-exist \textsc{sens}-have.to \\
 \glt `I have to learn a trade, to have something I am able to do.' (150902 luban-zh, 12)
 \end{exe}
 
The determiner \japhug{kɯmaʁ}{other} (on which see XXX) is placed before \japhug{tʰɯci}{something}, as in (\ref{ex:kWmaR.thWci}). 
 
\begin{exe}
\ex \label{ex:kWmaR.thWci}
\gll    ki mbro ki ɲɯ-kɤ-ntsɣe tɕe, kɯmaʁ tʰɯci ɲɯ-kɤ-sɤndu to-nɯkrɤz-ndʑi \\
\textsc{dem:prox} horse \textsc{dem:prox} \textsc{ipfv-inf}-sell \textsc{lnk} other  something   \textsc{ipfv-inf}-exchange \textsc{ifr}-discuss-\textsc{du} \\
 \glt `They discussed about selling their horse, and exchanging it for something else.' (150822 laoye zuoshi zongshi duide, 41)
\end{exe}


\subsection{\japhug{tsʰitsuku}{whatever}}
The pronoun \japhug{tsʰitsuku}{whatever}, unlike  \japhug{tʰɯci}{something}, is not used for specific referents. Example (\ref{ex:tshitsuku.kuwGsqa}) illustrates its most common use. 

\begin{exe}
\ex \label{ex:tshitsuku.kuwGsqa}
\gll  
kɤ-nɯβlɯ tɕe ɕkrɤz wuma ʑo pe ma nɯnɯ, nɯnɯ ɣɯ ɯ-smɯmba nɯ sɤɕke, tɕendɤre tsʰitsuku kú-wɣ-sqa tɕe, ʑaʑa ʑo ku-ɣɤ-smi cha tsʰitsuku tú-wɣ-sɯ-ɤla tɕe, ʑaʑa tu-sɯ-ɤle cʰa. \\
\textsc{inf}-burn \textsc{lnk} oak really \textsc{emph} be.good:\textsc{fact} \textsc{lnk} \textsc{dem} \textsc{dem} \textsc{gen} \textsc{3sg.poss}-flame \textsc{dem} burning \textsc{lnk} whatever \textsc{ipfv-inv}-cook \textsc{lnk} soon \textsc{emph}  \textsc{ipfv-caus}-be.cooked can:\textsc{fact} whatever \textsc{ipfv-inv-caus}-be.boiling \textsc{lnk} soon  \textsc{ipfv-caus-caus}-be.boiling[III] can:\textsc{fact} \\
\glt `For burning, oak is very good, the flames (from its wood) are very hot, whatever one cooks, it cooks it quickly, whatever one boils, it boils it quickly.' (08-CkrAz, 4-5)
\end{exe}
%nɤʑo kɯ rcanɯ, tɯ-tso ɯ-tɯ-me nɯ, maka /ji/ tɕi-rca jɤ-ɣi tɕe, nɯ sɤznɤ tshitsuku a-pɯ-tɯ-mtɤm tɕe a-pɯ-tɯ-nɯtɯtso ɲɯ-mna
%140510_fengwang, 15
 
In many cases, it is better translated as `all kinds of things', as in (\ref{ex:tshitsuku.YWznAme}).

\begin{exe}
\ex \label{ex:tshitsuku.YWznAme}
\gll  
tɕe nɯtɕu kɯnɤ ɯ-jaʁ ɯ-ntsi tɤɲi pjɯ-sɤtse  ɯ-jaʁ ɯ-ntsi kɯ tsʰitsuku ɲɯ-z-nɤme qhe, ʑara nɯ-ndzɤtsʰi tu-βze, fsapaʁ ra nɯ-ndzɤtsʰi ɲɯ-βze \\
\textsc{lnk} \textsc{dem:loc} also \textsc{3sg.poss}-hand \textsc{3sg.poss}-one.of.a.pair staff \textsc{ipfv}-plant[III]  \textsc{3sg.poss}-hand \textsc{3sg.poss}-one.of.a.pair \textsc{erg} whatever \textsc{ipfv-caus}-do[III] \textsc{lnk} \textsc{3pl} \textsc{3pl.poss}-food \textsc{ipfv}-make[III] animal \textsc{pl} \textsc{3pl.poss}-food \textsc{ipfv}-make[III]  \\
\glt `Even like that, she supports herself with a staff in one hand, and with the other hand she does all kinds of things, makes their food, she makes food for the animals.' (14-tApitaRi, 54)
\end{exe}
%tshitsuku ɲɯ́-wɣ-mbi, ɲɯ́-wɣ-jtshi, tú-wɣ-raχtɕɤz tɕe, ʑɯrɯʑɤri tɕe tɕendɤre ku-kɯ-nɯfse ɲɯ-ŋu

As other indefinite pronouns, \japhug{tsʰitsuku}{whatever} is not normally used with negation, but such sentences do occur in the corpus in translations from Chinese, as (\ref{ex:tshitsuku.mWtoti}). They are not not idiomatic Japhug, and even only marginally grammatical.

\begin{exe}
\ex \label{ex:tshitsuku.mWtoti}
\gll   tsʰitsuku mɯ-to-ti, qʰe tɕendɤre kɯ-rŋgɯ jo-nɯɕe qʰe ko-nɯ-rŋgɯ. \\
whatever \textsc{neg-ifr}-say \textsc{lnk} \textsc{lnk} \textsc{nmlz}:S/A-lay.down \textsc{ifr}-go.back \textsc{lnk} \textsc{ifr-auto}-lay.down \\
\glt `He did not said anything, went back to sleep and laid down in bed.' (150902 qixian-zh, 91)
\end{exe}

 \subsection{\japhug{ɕɯmɤɕɯ}{whoever, anybody}}
 There is no indefinite pronoun for human referents `somebody' in Japhug  corresponding to \japhug{tʰɯci}{something}. 
 
\subsection{Interrogative pronouns used as indefinites}
%\begin{exe} 
% \ex 
%\gll   \jg{nɯnɯ} \jg{ŋotɕu} \jg{tɤ-tɯ-nɯ-tɕhɯ-ntɕhoz} \jg{khɯ} \\
%this where \aor{}-2-\auto{}-\redp{}-use \npst{}:be.possible \\
% \glt You can use (this word) wherever you want. (Dpalcan 2010)
%\end{exe} 
%ŋotɕu chiz jó-wɣ-nɯ-sɤzɣɯt-a-nɯ kɯ
%pɣɤtɕɯ	nɯ	sɯ-ku	nɯ	ra	tɕhi	kɤ-cha	ʑo	pjɤ-kra	tɕe,
%29
%
%ŋotɕu	tɯ-ʁjis	tɕhi	kɯ-ɣi	nɯ	tú-ɣ-nɯχtɯ	tɕe	ɲɯ-pe,
%83 thaXtsa
%
%135	tɯ-mgo	zmɤrɤβ	nɯ	tɕhi	pɯ-nɯ-ŋu-ŋu	ʑo	tú-ɣ-nɯ-βzu	jɤk	ma	khɯ,
%tsampa
%
%
%"tɕe mɤʑɯ thɯthɤci ʑo mɯ́jnaχtɕɯɣ kɯ akɤtɯχpjɤt ra" toti
%yici bi yici you jinbu 17
%
%
%A	70	tɕe	ɕnɤloq	nɯnɯ	reri	nɯ	tɯ-kɤ-taq	nɯ	lú-ɣ-βraq	qhe,	tɕe	ɕnɤloq	nɯ	cɯschɯs	kú-ɣ-βraq	qhe,
%C	70	就把要织的带子套住在鼻圈上, 鼻圈随便拴在什么地方
%A	83	tɕe	nɯnɯ	ra,	tɯ-...	ŋotɕu	tɯ-ʁjis	tɕhi	kɯ-ɣi	nɯ	tú-ɣ-nɯχtɯ	tɕe	ɲɯ-pe,
%B	83
%C	83
%D	83	那些(线),你要想要什么线就可以买什么就好了,
%
%ɕu pɯnɯŋɯŋu kɯ, aʑo amu mpɕɤr ɲɯsɯsɤm ɕti
%
%ɕu pɯnɯŋɯŋu nɤ, tɕhi pɯnɯfsɯfse, ɯkɯnɤmpɕɤr tu ɕti
%There is always someone who finds you beautiful
%8_WkWnAmpCAr
%
%tɯʑo mɯpɯ́wɣnɤmpɕɤr kɯ́nɤ, kɯmaʁ tɯrme ɯkɯnɤmpɕɤr tu ɕti
%8_WkWnAMpCAr2
%lú-wɣ-sti tɕe tɕe nɯ ɯ-ŋgɯ tɕhi pɯ-nɯ-ŋɯ-ŋu nɯ ɲɯ-mɲɤt mɯ́j-cha
%tɕhi tɤ-tɯ-nɯ-tɯ-tɯt ʑo ju-ɣi ɕti
%sɤtɕha ŋotɕu kɯ-tu tu-nɯ-ɬoʁ ɲɯ-ɕti.
%tham qhe ɕɯmɤɕɯ kɯ ku-nɯ-ntɕhoz-nɯ ɕti
\section{Demonstratives} \label{sec:demonstratives}
\section{Other pronominal-like elements} \label{sec:other.pro}
\japhug{kɯmaʁ}{other}