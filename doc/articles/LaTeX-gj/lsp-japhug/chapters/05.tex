\chapter{Pronouns and indefinites}
\section{Pronouns and possessive prefixes} \label{sec:pronouns}

%\\ipa\{([\w-]*)\}
%\1

%(\d)\\(\w\w)\{\}
%\\textsc{\1\2}

The pronominal system of Japhug distinguishes singular, dual and plural, but does not present any inclusive / exclusive distinction as other languages such as Tshobdun (see \citealt{jackson98morphology}).


Alongside the free pronouns, a system of pronominal prefixes is used not only to express  possession on noun (see section \ref{sec:possession}  for an account of the possessive constructions), but also appears in various constructions in the verbal system. These prefixes do not distinguish the second and the third person in the dual and plural forms:


\begin{table}[h] \centering
\caption{Pronouns and possessive prefixes }\label{tab:pronoun}
\begin{tabular}{lllllllll} \lsptoprule
 Free pronoun & Prefix & \\
\midrule
 \forme{aʑo},    \forme{aj} &	\forme{a-}  &		1\sg{} \\
\forme{nɤʑo},  \forme{nɤj} &	\forme{nɤ-}  &			2\sg{} \\
\forme{ɯʑo}  &	\forme{ɯ-}  &			3\sg{} \\
\forme{tɕiʑo}  &	\forme{tɕi-}  &			1\du{} \\
\forme{ndʑiʑo}  &	\forme{ndʑi-}  &		2\du{} \\	
\forme{ʑɤni}  &	\forme{ndʑi-}  &		3\du{} \\	
\forme{iʑo}, \forme{iʑora},   \forme{iʑɤra}   &	\forme{i-}  &			1\pl{} \\
\forme{nɯʑo}, \forme{nɯʑora},   \forme{nɯʑɤra}  &	\forme{nɯ-}  &			2\pl{} \\
\forme{ʑara}  &	\forme{nɯ--}  &			3\pl{} \\
\lspbottomrule
\end{tabular}
\end{table}

Free pronouns and possessive prefixes are remarkably similar in Kamnyu Japhug. In the eastern Japhug dialects, a different \textsc{1sg} pronoun distinct from the possessive prefix  is used: \forme{ŋa}. In the table above, we observe that all the pronouns except the third person dual and plural are formed by adding the root \forme{-ʑo} to the corresponding possessive prefix. This root is that of the generic pronoun \forme{tɯ-ʑo}, which appears mainly in gnomic statements:

%tɯʑo tɯkɤsɯso nɯ tukɯnɯti ɕti ma
%One says what he thinks
%\ipa{8_tWZo}

\begin{exe}
\ex
\gll 
tɯ-zda pjɯ́-wɣ-z-ɣɤtɕa, \textbf{tɯʑo}  ntsɯ  pjɯ-kɯ-ʑɣɤ-ɣɤŋgi   	tɕe,  pɯ-kɯ-nɯ-ɣɤtɕa 	kɯ́nɤ   	pjɯ-kɯ-ʑɣɤ-ɣɤŋgi   	tɕe,    ɯ-mbrɤzɯ   	kɯ-tu   	me  	tu-kɯ-ti   	ɲɯ-ŋu.   \\
\textsc{genr.poss}-companion \textsc{ipfv-inv-caus}-be.wrong oneself always \textsc{ipfv-genr:S/P-refl}-be.right \textsc{lnk} \textsc{pfv-genr:S/P-auto}-be.wrong also \textsc{ipfv-genr:S/P-refl}-be.right lnk \textsc{3sg.poss}-result \textsc{nmlz:S/A}-have  not.exist:\textsc{fact} \textsc{ipfv-genr}-say \textsc{sens}-be \\
\glt  `If one considers that one's companion is wrong, and always considers himself to be right even if one is wrong, there is can be no good result.' (Mouse and sparrow, 80-82)
\end{exe} 


The first and second person singular pronouns  \forme{aʑo} and \forme{nɤʑo} also have the shorter monosyllabic forms \forme{aj} and \forme{nɤj} respectively. These short forms are considerably less common in stories (in the reported speech of the characters), but appear frequently in free conversations.


Third person pronouns can be used with inanimate referents, as the third person dual \forme{ʑɤni} in example (\ref{ex:rNgW}).

\begin{exe}
\ex \label{ex:rNgW}
\gll tɕe   	rŋgɯ   	nɯ   	to-k-ɤmɯrpu-ndʑi-ci   	tɕe,   	tɕendɤre   	ʑɤni   	pjɤ-nɯ-ɴɢrɯ-ndʑi   \\
\textsc{lnk} boulder \textsc{dem} \textsc{ifr-evd}-bump.into:\textsc{recip}-\textsc{du-evd} \textsc{lnk} \textsc{lnk} \textsc{3du} \textsc{ifr-auto}-crush-\textsc{du} \\
\glt `The boulders bumped into each other and they were pulverized.' (smanmi4.82-83)
\end{exe}




\section{Genitive forms} \label{sec:pronouns.gen}
The form of pronouns and personal prefixes undergoes few morphophonological changes in combination with postpositions and relational nouns. However, in combination with the genitive suffix \forme{ɣɯ} (cf XXX), some  personal pronouns have special forms indicated in Table  \ref{tab:pronoun.gen}.


% (cf. \ref{sec:possession})


\begin{table}[h] \centering
\caption{Pronouns and possessive prefixes }\label{tab:pronoun.gen}
\begin{tabular}{lllllllll} \lsptoprule
 Free pronoun & Genitive & \\
\midrule
 \forme{aʑo}  &	\forme{aʑɯɣ}  &		\textsc{1sg} \\ 
\forme{nɤʑo}  &	\forme{nɤʑɯɣ}  &			\textsc{2sg} \\ 
\forme{ɯʑo}  &	\forme{ɯʑɤɣ}  &			\textsc{3sg} \\ 
\forme{tɕiʑo}  &	\forme{tɕiʑɤɣ}  &			\textsc{1du} \\ 
\forme{ndʑiʑo}  &	\forme{ndʑiʑɤɣ}  &		\textsc{2du} \\	 
\forme{ʑɤni}  &	\forme{ʑɤniɣɯ}  &		\textsc{3du} \\	 
\forme{iʑo}  &	\forme{iʑɤɣ}, 	\forme{iʑɤra ɣɯ}   &			\textsc{1pl} \\ 
\forme{nɯʑo}  &	\forme{nɯʑɤɣ}, 	\forme{nɯʑɤra ɣɯ}  &			\textsc{2pl} \\ 
\forme{ʑara}  &	\forme{ʑaraɣ},   \forme{ʑara ɣɯ}&			\textsc{3pl}  \\  
\lspbottomrule
\end{tabular}
\end{table}

While some degree of variation exists with dual and plural pronouns (for instance the regular \forme{iʑo ɣɯ} is found alongside \forme{iʑɤɣ} and \forme{iʑɤra ɣɯ}), for the singular pronouns only one form is attested.

\begin{exe}
\ex
\gll aʑɯɣ 	ndʐa 	ŋu 	ɕi, 	nɤʑɯɣ 	ndʐa 	ŋu, 	ɤj 	mɯ́j-tso-a   \\
\textsc{1sg:gen} reason be:\textsc{fact} \textsc{qu} \textsc{2sg:gen} reason be:\textsc{fact} \textsc{1sg} \textsc{neg:sens}-understand-\textsc{1sg} \\
\glt  `I don't know if it is because of me, or because of you.' (that the phone line is not working well) (phone conversation, 2011) %\wav{8_ndzxa})
\end{exe} 

In the genitive forms of the pronouns, the vowel of the genitive marker is generally dropped, and the pronominal root \forme{-ʑo} undergoes vowel change to \forme{-ʑɯɣ} (in the case of first and second person) and \forme{-ʑɤɣ} (in other forms). Note that \forme{ʑaraɣ} is the only case of the rhyme \ipa{aɣ} in Japhug.


\section{Interrogative pronouns}
The interrogative pronouns in Japhug are indicated in Table \ref{tab:interrog.pronoun}.

\begin{table}[h] \centering
\caption{Interrogative pronouns }\label{tab:interrog.pronoun}
\begin{tabular}{lllllllll} \lsptoprule
\japhug{tɕʰi}{what} \\
\japhug{ɕɯ}{who} \\
\japhug{tʰɤstɯɣ}{how many} \\
\japhug{tʰɤjtɕu}{when} \\
\japhug{ŋotɕu}{where}, \japhug{ŋoj}{where} \\
\japhug{tɕʰindʐa}{why} \\
\lspbottomrule
\end{tabular}
\end{table}

The pronoun interrogative  `what' varies considerably across Japhug dialects. In Kamnyu we find \forme{tɕʰi}, visibly borrowed from Tibetan \forme{tɕʰi}. Neighbouring dialects of Gdongbrgyad xiang have either \forme{tsʰi} (in Mangi) or \forme{tʰi} (in Rqaco), which represents the original Rgyalrongic root for this interrogative pronoun. The Eastern dialects of Gsardzong and Datshang have \forme{xto} instead, a word of unknown etymology.

The interrogative \japhug{ɕɯ}{who} is used with a human referent. It can occur in all syntactic roles, and does not have special ergative or genitive forms (see examples \ref{ex:CW.kW} and \ref{ex:CW.GW}).

\begin{exe}
\ex  \label{ex:CW.kW}
\gll  mɤ-ta-mbi 	nɤʑo 	qaɕpa 	ɕɯ 	kɯ 	tɯ́-wɣ-mbi    \\
\textsc{neg}-1$\rightarrow$2-give:\textsc{fact} \textsc{2sg} frog who \textsc{erg} 2-\textsc{inv}-give:\textsc{fact}  \\
\glt `We won't give her to you, who would give her to you, a frog?'   (2002 qaCpa, 09)
\end{exe} 

 
\begin{exe}
\ex  \label{ex:CW.GW}
\gll  ɕɯ ɣɯ ʑo ɲɯ-kham-a ra kɯɣe?    \\
who \textsc{gen} \textsc{emph} \textsc{ipfv}-give:III-\textsc{1sg} \textsc{sfp} \\
\glt `Whom should I give (her) to (in marriage)?' (140508 benling gaoqiang de si xiongdi, 222)
\end{exe}  


To ask about a quantity, the pronoun \japhug{tʰɤstɯɣ}{how many, how much} or \japhug{tɕʰi jarmar}{how much} (see XXX on the forms of this adverb) can be used.

\begin{exe}
\ex
\gll tu-ɕtʂam-a tɕe tɕʰi jamar ʑo ɣɤʑu kɯ? \\
\textsc{ipfv}-measure[III]-\textsc{1sg} \textsc{lnk} what about \textsc{emph} exist:\textsc{sens} \textsc{sfp} \\
\glt `I will measure it with a scoop to see how much (gold) there is.' (140512alibaba, 59)
\end{exe}  

\begin{exe}
\ex
\gll  zgo 	tʰɤstɯɣ 	ja-nnɯ-pɣaʁ-ndʑi, 	tɯ-ci 	tɕʰi 	jarma 	ja-nnɯ-pjɤl-ndʑi 	mɤ-xsi 	ma,       \\
 mountain how.many \textsc{pfv}:3$\rightarrow$3'-\textsc{auto}-turn.over-\textsc{du} \textsc{indef.poss}-water what about \textsc{pfv}:3$\rightarrow$3'-\textsc{auto}-cross-\textsc{du} \textsc{neg-genr}:know \textsc{lnk} \\
\glt `It is not known how many mountains they crossed, around how many rivers they went.'  (2002qajdo, 50)
\end{exe}  
  
%tɤ-rɟit tɕhi tɤ-kɯ-sci nɯ ʑo ɣɯ-tɕɤt kɯ-ra pjɤ-ɕti tɕe,   
  
For precise figures, in particular for money, only \forme{tʰɤstɯɣ} is used (see \ref{ex:thAstWG.tWkhAm}).
\begin{exe}
\ex \label{ex:thAstWG.tWkhAm}
 \gll    nɤʑo 	tʰɤstɯɣ 	tɯ-kʰɤm?    \\
 you how.much 2-give[III]:\textsc{fact}  \\
\glt  `How much do you give (for it)?' (Bargaining, 13)
\end{exe} 

The pronoun \forme{tʰɤstɯɣ} as a conjunct form \forme{tʰɤstɯ-} when used with classifiers (in \ref{ex:thAstWmaR}, with the classifier \ipa{-maʁ} `size of shoes' from Chinese \zh{码} \forme{mǎ}).

 \begin{exe}
\ex \label{ex:thAstWmaR}
 \gll   nɤ-xtsa nɯ tʰɤstɯ-maʁ tu-tɯ-ŋge ŋu   \\
\textsc{2sg.poss}-shoe \textsc{dem} how.many-size \textsc{ipfv}-2-wear[III] be:\textsc{fact} \\ 
\glt `What is the size of your shoes?'  (Conversation, 2015)
\end{exe} 

%

Combined with the noun \japhug{tɤ-rʑaʁ}{time}, 	\forme{tʰɤstɯɣ} can be used  XXX

\begin{exe}
\ex
 \gll   nɤʑo 	tɤ-rʑaʁ 	thɤstɯɣ 	jamar 	kɤ-βzjoz 	kɤ-tɯ-spa-t?  \\
 you \textsc{indef.poss}-time how.many about \textsc{inf}-learn \textsc{pfv}-2-be.able-\textsc{pst:tr} \\
\glt   `How long did you need to learn it?' (elicitation, Dpalcan)
\end{exe} 
 
 \begin{exe}
\ex
 \gll   tɤ-rʑaʁ 	thɤstɯɣ ko-zɣɯt? \\
  \textsc{indef.poss}-time how.many  \textsc{ifr}-reach \\
  \glt `What is the time?' (heard in context)
  \end{exe} 
%
%To ask about the precise moment when an event took place, the pronoun \ipa{tʰɤjtɕu} ``when'' should be used instead:
%
%\begin{exe}
%\ex
%\gll  tʰɤjtɕu 	lɤ-tɯ-nɯɣe 	pɯ-ŋu 	ra 	nɤ?    \\
% when \aor{}-2-come.back[II] \pst{}.\ipf{}-be \pl{} \textsc{part} \\
%\glt  When did you come back home? (Tarrdo conversation, 01)
%\end{exe} 
%
%The interrogative ŋotɕu and its reduced form ŋoj can be used to ask either about a location or a direction:
%
%\begin{exe}
%\ex
%\gll     ŋotɕu ku-tɯ-rɤʑi?   \\
%  where \pres{}-2-stay \\
%\glt Where are you? (Conversation, 2005)
%\end{exe} 
% 
%
% 
%\begin{exe}
%\ex
%\gll     qala ŋoj nɯ-ari  \\
%  rabbit where \aor{}:west-go[II]\\
%\glt Where did the rabbit go?  (The rabbit and the bear, 21)
%\end{exe} 
% 
% ŋotɕu also appears in the following idiomatic expression:
% 
% \begin{exe}
%\ex
%\gll     kɯki 	ŋotɕu 	ɲɯ-ŋgrɤl?   \\
% this where \ipf{}-be.usually.the.case \\
%\glt `How could this be possible?'  (The raven, 32)
%\end{exe} 
%
%This sentence is used to express indignation (as in Chinese \zh{哪有这样的道理?}); in the story from which it is quoted, the husband says this sentence after his wife, quoting the words of a raven, says that she will have luck, not her husband. 
%
%%ŋɤtɕɯkɤti
%
% tɕʰi is by far the most common of all interrogative pronouns, and can appear in a broad variety of contexts. It can be used to ask about a thing:
%
%
% \begin{exe}
%\ex
%\gll    nɯnɯ 	tɕʰi 	pɯ-rmi 	kɯ?       \\
%that what \pst{}.\ipf{}-be.called \textsc{qu} \\
%\glt  What was he called, again? (Gesar, 249)
%\end{exe} 
% 
% tɕʰi also occurs rhetorical questions with the irrealis (\ref{sub:irrealis}), in the meaning ``how could it be that...'':
% 
% \begin{exe}
%\ex
%\gll   tʰaʁɕa 	ʁo 	tɕʰi 	a-pɯ-rtaβ-a 	kɯ \\
% weaving.comb \textsc{advers} what \irr{}-\pfv{}-hit-\textsc{1sg} \textsc{qu} \\
%\glt  How (could you imagine that) I would hit her with the weaving comb? (The bird, 61)
%\end{exe} 
%
%  \begin{exe}
%\ex
%\gll  aʑo 	tɕhi 	a-pɯ-ŋu-a? \\
%\textsc{1sg} what \irr{}-\ipf{}-be-\textsc{1sg}  \\
%\glt  How could it be me? (not to be interpreted as: ``What would I be?", The prince, 60)
%\end{exe} 
%
%
%To ask about manner, tɕhi is either combined with the stative verb fse ``to be in such a way, to be like'' or the transitive stu ``to do in such a way'':
%
%   \begin{exe}
%\ex
%\gll  iʑɤra 	ɣɯ 	tɕhi 	tu-fse 	tɕe 	ji-tɯ-ci 	ɣɤʑu 	tu-tɯ-the 	ɯ-tɯ́-cha  \\
%we \gen{} what  \ipf{}-be.in.such.a.way \coord{} \textsc{1pl}.\poss{}-\neu{}-water be.there.\textsc{sensory} \ipf{}-2-ask[III] \textsc{qu}-2-\npst{}:be  \\
%\glt  Can you ask how can we have water? (The divination2, 14)
%\end{exe} 
% 
%   \begin{exe}
%\ex
%\gll  nɯnɯ 	ra 	kɯ 	tɕʰi 	a-tɤ-stu-nɯ 	tɕe 	nɯ-tɯ-ci 	ɣɤʑu   \\
%\dem{}.\textsc{distal} \pl{} \erg{} what \irr{}-\pfv{}-do.this.way-\pl{} \coord{} \textsc{3pl}.\poss{}-\neu{}-water be.there.\textsc{sensory}  \\
%\glt  What should these (people) do so that they can have water? (The divination2, 38)
%\end{exe} 
%
%Alternatively to the construction above (with both verbs in non-finite forms) the verb fse can be used as the main predicate with a verbal complement:
%   \begin{exe}
%\ex
%\gll  kɤ-pʰɣo tɕʰi a-tɤ-fse-j    \\
%\inftv{}-flee what \irr{}-\pfv{}-be.in.such.a.way-\textsc{1pl}   \\
%\glt  How do we flee? (Slobdpon 69)
%\end{exe} 
%
%To ask about the reason of an action,  tɕʰi is combined with the noun ɯ-ndʐa ``reason'' as tɕʰindʐa ``why'':
%  \begin{exe}
%\ex
%\gll    ama 	a-pi 	khu 	tɕʰindʐa 	ku-tɯ-nɤpʰɯpʰɣo 	tɕe 	nɤʑo 	kɯ-fse 	kɯ-sɤɣ-mu 	me   \\
%\textsc{part}:surprise \textsc{1sg}.\poss{}-elder.sibling   tiger why \pres{}-2-\textsc{atelic}:flee \coord{} \textsc{2sg} \nmlz{}:\stat{}-be.in.such.a.way \nmlz{}:\stat{}-\textsc{deexp}-fear \npst{}:not.be.there \\
%\glt Oh, elder brother Tiger, why are you running around, there is nothing as fearsome as you?  (The tiger, 26)
%\end{exe} 
%
% 
%
%\section{Indefinite and definite marking}
%Japhug does not have have genuine definite or indefinite articles, though and presents several strategies to mark definiteness. First, it is possible to use a noun with an indefinite or a definite marker. Second, Japhug has a series of indefinite pronouns, and interrogative pronouns can also be used as indefinites in some contexts. Third, non-referentiality or non-identifiability of an entity can be expressed by various syntactic constructions, including relative clauses and incorporation.
%
%%identifiability+referentiality
%
%\subsection{Indefinite and definite markers}
%Like many languages (\citealt[130]{creissels06sgit1}), Japhug  uses bare nouns without any definiteness marking. Bare nouns are most often non-referential:
%
%  \begin{exe}
%\ex
%\gll ʁnaʁna 		tɕheme 	tɯ-tɤ-tu 	nɤ, 	kɤndʑɯsqhɤj 	tu-kɤ-sɯ-βzu \\
%both \emphat{} \gen{} \textsc{3du}.\poss{}-child girl \textsc{if}-\aor{}-have \textsc{if} group.of.sisters \ipf{}-\inftv{}-\caus{}-make \\
%\glt If both of them have girls, let them be sisters. (zrɤntɕɯ tɯrme 4)
%\end{exe}
%
%They are  used in nominal predicates with a copula:
%
%  \begin{exe}
%\ex
%\gll aʑo 	tɕʰeme 	ɲɯ-ŋu-a  tɕe \\
%I girl \const{}-be-\textsc{1sg} \coord{} \\
%\glt I am a girl. (Nyima Vodzer 144)
%\end{exe}
%
%Bare nouns are rare with referential nouns (except in answers to questions), but examples can be found:
%
%  \begin{exe}
%\ex
%\gll qacʰɣa 	kɯ 	maχtɕɯ tɤ-tɯt-a nɯ mɤ-tɯ-ste 	ti 	ɲɯ-ŋu  \\
%fox \erg{} in.principle \aor{}-say[II]-\textsc{1sg} \compl{} \negat{}-2-\npst{}:do.this.way[III] \npst{}:say \ipf{}-be \\
%\glt The fox says: ``You do not do as I told you to." (The fox, 44)
%\end{exe}
%
%Personal names generally occur as bare nouns, without any definiteness marker (though, as we will see, these markers are not agrammatical with personal names):
%
%  \begin{exe}
%\ex
%\gll  ɯrɟɤnpanma 	kɯ 	ʁlaŋsaŋtɕhin 	ɯ-ɕki  \\
% Padmasambhava \erg{} Gesar \textsc{3sg}-\textsc{dat} \\
%\glt Padmasambhava (told) Gesar:
%\end{exe}
%
%Bare nouns are however relatively rare. Most nominal phrases contain either a demonstrative, an indefinite or a topic marker (cf chapter \ref{chapt:noun.phrase}). 
%
%
%The numeral ci is used as an indefinite marker, placed at the end of the noun phrase:
%  \begin{exe}
%\ex
%\gll tɕʰeme 	kɯ-mpɕɯ-mpɕɤr 	ci 	ɲɤ-nɯɬoʁ \\
%girl \nmlz{}:\stat{}-\textsc{intens}-beautiful \textsc{indef}  \evd{}-appear \\
%\glt A very beautiful girl appeared (out of it). (The flood, 39)
%\end{exe}
%ci must be used to introduce a new referent in a story.
%
%On its own, ci can also serve as an indefinite pronoun, meaning ``one of (a group)'':
%  \begin{exe}
%\ex
%\gll ci 	ɣɯ 	tɤ-tɕɯ, 	ci 	ɣɯ 	tɕheme 	tɯ-tɤ-tu 	nɤ, 	ʁzɤmi 	ku-kɤ-sɯ-βzu \\
%\textsc{indef} \gen{} \neu{}-boy \textsc{indef} \gen{} girl \textsc{if}-\aor{}-have \textsc{if} husband.and.wife \ipf{}-\inftv{}-\caus{}-make \\
%\glt If one of them has a boy, and the other one has a girl, let us make them husband and wife. (zrɤntɕɯ tɯrme 5)
%\end{exe}
%
%
%%ci thɯ-kɯ-rgɯ-rgɤz ɲɯ-ɕti tɕe, ci kɯ-xtɕɯ-xtɕi ɲɯ-ɕti tɕe,
%%Nyima wodzer4.127
%
%Japhug has no definite article. The distribution of the ubiquitous marker \textit{nɯ} is close to that of a definite marker, but since it does not normally appear in focalized noun phrases such as answers to questions,  it is more statisfying to treat it as a topic marker (see section \ref{sec:topic}).
%
%The marker  iɕqʰa ``the aforementioned'' indicates both definiteness and topicality.\footnote{Used alone, iɕqʰa is a temporal adverb meaning ``just before'', see \ref{chapt:adv}.} It is used on referents that have been previously mentioned in the same story, usually only a few sentences back:
%  \begin{exe}
%\ex \label{ex:indef}
%\gll \textbf{``razri} 	\textbf{kɤtɯm} 	\textbf{ci} 	ɲɯ-ra, 	taqaβ 	ci 	ɲɯ-ra" to-ti qhe   \\
% thread ball \textsc{indef} \const{}-need needle \textsc{indef} \const{}-need \evd{}-say \coord{}  \\
%\glt He told (Rgyabza) ``I need a ball of thread and a needle''.
%\ex \label{ex:icqha}
%\gll tɕendɤre 	ɲo-kho 	qhe, 	tɕe 	ɯ-ndzɤtshi 	kɤ-tsɯm 	nɯ 	tɕu 	qhe 	tɕe, \textbf{iɕqʰa} 	\textbf{kɤtɯm} 	\textbf{nɯ }	ɯʑo 	kɯ 	ko-ndo, 	taqaβ-rna 	nɯ 	ɲɤ-rku qhe,  \\
% \coord{} \evd{}-give \coord{} \coord{} \textsc{3sg}.\poss{}-meal \inftv{}-bring \textsc{compl}  \loc{} \coord{} \coord{} the.aforementioned ball \topic{} he \erg{} \evd{}-take needle-ear \topic{} \evd{}put.in \coord{} \\
%\glt She gave it to him. While (people) brought his meal, he took the ball of thread and put it into the ear of the needle. (Gesar 270-2)
%\end{exe}
%The referent ``ball of thread'', first introduced in sentence \ref{ex:indef}, appears again two sentences later with both the topic markers nɯ and \textit{iɕqʰa}. 
%
%Unlike nɯ, iɕqʰa is relatively rare depending one the speaker and the type of discourse. It is possible to find stories long more than 80 sentences without any occurrence of iɕqʰa. 
%
%There seems to be a limit to the number of sentences that can separate a noun phrase in iɕqʰa from its preceding occurrence (probably no more than five-six), but this topic deserves of systematic study based on all available stories.
%
% iɕqʰa as a definite marker not only occurs with nouns, but also with bare demonstratives such as nɯnɯ and with personal names:
%
%  \begin{exe}
%\ex
%\gll  tɕendɤre 	iɕqʰa 	ʁlaŋsaŋtɕʰin 	χsɯm 	ma 	mɯ-tɤ-kɯ-rʑaʁ 	nɯ, \\
% \coord{} the.aforementioned Gesar three apart.from \negat{}-\aor{}-\nmlz{}:S-pass.day \topic{} \\
%\glt Gesar, who was only three days old,  (Gesar 81)
%\end{exe}
%
%\subsection{Indefinitive pronouns}
% Japhug has some indefinite pronouns that are etymologically linked with corresponding interrogatives.
%
%
%\begin{table}[H] \centering
%\caption{Indefinite pronouns }\label{tab:indef.pronoun}
%\begin{tabular}{lllllllll} \toprule
%Form & Meaning   \\
%\midrule
%tʰɯci, tʰɯtʰɤci & something \\
%cɯscʰɯz & somewhere \\
%tsʰitsuku & whatever \\
%\bottomrule
%\end{tabular}
%\end{table}
%


%
%tʰɯci, tʰɯtʰɤci
%
%
%
%cɯscʰɯz
%
%tshi tsu ku
%
%\begin{exe} 
% \ex 
%\gll   \jg{nɯnɯ} \jg{ŋotɕu} \jg{tɤ-tɯ-nɯ-tɕhɯ-ntɕhoz} \jg{khɯ} \\
%this where \aor{}-2-\auto{}-\redp{}-use \npst{}:be.possible \\
% \glt You can use (this word) wherever you want. (Dpalcan 2010)
%\end{exe} 
%
%pɣɤtɕɯ	nɯ	sɯ-ku	nɯ	ra	tɕhi	kɤ-cha	ʑo	pjɤ-kra	tɕe,
%29
%
%ŋotɕu	tɯ-ʁjis	tɕhi	kɯ-ɣi	nɯ	tú-ɣ-nɯχtɯ	tɕe	ɲɯ-pe,
%83 thaXtsa
%
%135	tɯ-mgo	zmɤrɤβ	nɯ	tɕhi	pɯ-nɯ-ŋu-ŋu	ʑo	tú-ɣ-nɯ-βzu	jɤk	ma	khɯ,
%tsampa
%
%
%"tɕe mɤʑɯ thɯthɤci ʑo mɯ́jnaχtɕɯɣ kɯ akɤtɯχpjɤt ra" toti
%yici bi yici you jinbu 17
%
%
%A	70	tɕe	ɕnɤloq	nɯnɯ	reri	nɯ	tɯ-kɤ-taq	nɯ	lú-ɣ-βraq	qhe,	tɕe	ɕnɤloq	nɯ	cɯschɯs	kú-ɣ-βraq	qhe,
%C	70	就把要织的带子套住在鼻圈上, 鼻圈随便拴在什么地方
%A	83	tɕe	nɯnɯ	ra,	tɯ-...	ŋotɕu	tɯ-ʁjis	tɕhi	kɯ-ɣi	nɯ	tú-ɣ-nɯχtɯ	tɕe	ɲɯ-pe,
%B	83
%C	83
%D	83	那些(线),你要想要什么线就可以买什么就好了,
%
%ɕu pɯnɯŋɯŋu kɯ, aʑo amu mpɕɤr ɲɯsɯsɤm ɕti
%
%ɕu pɯnɯŋɯŋu nɤ, tɕhi pɯnɯfsɯfse, ɯkɯnɤmpɕɤr tu ɕti
%There is always someone who finds you beautiful
%8_WkWnAmpCAr
%
%tɯʑo mɯpɯ́wɣnɤmpɕɤr kɯ́nɤ, kɯmaʁ tɯrme ɯkɯnɤmpɕɤr tu ɕti
%8_WkWnAMpCAr2
%
%
%
%negative with maka
%ɯʑo kɯ ta-tɯt maka kɯ-tu me
%He said nothing 什么也没有说
%\subsection{Competing constructions}
%
%\subsubsection{Relatives}
%nɤʑo	nɯ-nɯ-ɣɤwu	ma,	nɤ-kɯ-nɯɣmu	me	ma	ma-ta-mbi
%Frog 38
%
%
%iɕqha tɯrme kɯngo ɣɤʑu tɕe, ɯkɯrtoʁ jaria wo!
%Someone was ill \wav{8_kWngoGAZu}
%
%
%	kɤ-mɯnmu	kɯ́nɤ	mɯ-pjɤ-mɯnmu				
%	Raven 58
%
%
%A	78	nɤj	thɤstɯɣ	tɯ-chɯ-cha	ʑo	nɤ,	a-tɤ-tɯ-nɯ-rke	qhe	nɯnɯ	nɤʑo	ɣɯ	nɤ-rkus	ŋu"	to-ti,
%
%
%tɕhi kɯfse pɯtɯnɯmbɣom kɯ́nɤ, zgo nɯ kutɯpɣaʁ ndɤre ra
%\wav{8_kutWnWmbGom}
%
%nɯ apɯŋu ma dianhua ɯkɯlɤt ɣɤʑu!
%\wav{WkWlAt}
%
%\section{Demonstratives} \label{sec:demonstratives}