\chapter{The noun phrase} \label{chap:noun.phrase}


\section{Noun modifiers and determiners}
This section discusses all nouns modifiers and determiners except relative clauses (§ XXX) and complement clauses (§ XXX). 
 
\subsection{Number}  \label{sec:number.determiners}
Japhug has two number markers, the dual \forme{ni} and the plural \forme{ra}. These clitics are not obligatory for non-singular arguments (even in the case of human referents), and do not necessary trigger plural or dual agreement on the verb. 

\subsubsection{Dual} \label{sec:dual.determiners}
The dual \forme{ni} is historically related to the numeral \forme{ʁnɯz} (§ \ref{sec:one.to.ten}), but their relationship is synchronically opaque. It combines with the proximal and distal demonstratives \forme{ki} and \forme{nɯ} respectively to form the dual demonstratives \forme{kɯni} and \forme{nɯni} (§ \ref{sec:demonstrative.pronouns}, § \ref{sec:demonstrative.determiners}).

There is no semantic restriction on the use of \forme{ni}, it most often occurs with human referents (\ref{ex:awW.cho.aRi.ni}, \ref{ex:Wmu.Wwa.ni}, \ref{ex:tCiZo.ni}, \ref{ex:ni.ndZisroR}), but is also commonly attested with animals (\ref{ex:ʁnWz.ni}) inanimate objects (\ref{ex:ni.RnaRna}), and placenames (\ref{ex:rgWnba.ni}).

\begin{exe}
\ex \label{ex:rgWnba.ni}
\gll prɤɕta cʰo rgɯnba ni ndʑi-pɤrtʰɤβ ri ŋu \\
pl.n. \textsc{comit} monastery \textsc{du} \textsc{3du}.\textsc{poss}-between \textsc{loc} be:\textsc{fact} \\
\glt `It is between Prashta and the monastery.' (140522 Kamnyu zgo, 115)
\end{exe}

The dual can follow the numeral \japhug{ʁnɯz}{two}, as in (\ref{ex:ʁnWz.ni}). This combination is however very rare (only 13 examples in the corpus out of hundreds of dual \forme{ni}). The opposite order (dual followed by numeral) is not grammatical.

\begin{exe}
\ex \label{ex:ʁnWz.ni}
\gll mbɣɤru nɯ jla ʁnɯz ni ndʑi-tʰɤβ ri ɲɯ-ɕe tɕe \\
plough.beam \textsc{dem} hybrid.yak two \textsc{3du}.\textsc{poss}-between \textsc{loc} \textsc{ipfv}:\textsc{west}-go \textsc{lnk} \textsc{lnk} \\
\glt `The beam of the plough goes between the two hybrid yaks.' 
\end{exe}


The adverb \japhug{ʁnaʁna}{both} (§ XXX) commonly co-occurs with dual, as in (\ref{ex:ni.RnaRna}).
%tɤ-pi ʁnaʁna ʑo pɯ́-wɣ-sat-ndʑi ɲɯ-ŋu. 

\begin{exe}
\ex \label{ex:ni.RnaRna}
\gll zaŋ cʰo raʁ ni ʁnaʁna ʑo ʁja ku-te ɲɯ-ŋu \\
copper \textsc{comit} brass \textsc{du} both \textsc{emph} verdigris \textsc{ipfv}-put[III] \textsc{sens}-be \\
\glt `Both copper and brass can get verdigris.' (30-Com, 101)
\end{exe}

The marker \forme{ni} can appear with a noun phrase comprising two nouns (each with singular referents) linked by the comitative \forme{cʰo} (§ \ref{sec:comitative}).

\begin{exe}
\ex \label{ex:awW.cho.aRi.ni}
\gll  tɕe a-wɯ cʰo a-ʁi ni pjɯ-tɯ-sat mɤ-jɤɣ \\
\textsc{lnk} \textsc{1sg}.\textsc{poss}-grandfather \textsc{comit} \textsc{1sg}.\textsc{poss}-younger.sibling \textsc{du} \textsc{ipfv}-2-kill \textsc{neg}-be.possible:\textsc{fact} \\
\glt `You cannot kill my grandfather and my younger brother.' (2011-05-nyima, 133)
\end{exe}

The dual can also be used with noun dyads (§ \ref{sec:dyads}), as in (\ref{ex:Wmu.Wwa.ni}). 

\begin{exe}
\ex \label{ex:Wmu.Wwa.ni}
\gll   ɯ-mu ɯ-wa ni kɯ ɲɯ-z-nɤja-ndʑi qʰe \\
\textsc{3sg}.\textsc{poss}-mother \textsc{3sg}.\textsc{poss}-father \textsc{du} \textsc{erg} \textsc{ipfv}-\textsc{caus}-be.a.pity-\textsc{du} \textsc{lnk} \\
\glt `Her parents would not be parted from her.' (14-tApitaRi, 305)
\end{exe}

The third person dual pronoun \forme{ʑɤni} is build by combining the pronominal root \forme{-ʑo-} with the dual \forme{ni} (§ \ref{sec:pers.pronouns}), and is not attested in combination with the dual. The first and second dual pronouns \forme{tɕiʑo} and \forme{ndʑiʑo}, do occur with the dual marker as in (\ref{ex:tCiZo.ni}), though examples are very rare.

\begin{exe}
\ex \label{ex:tCiZo.ni}
\gll  tɕiʑo ni wuma ʑo pɯ-amɯmi-tɕi tɕe \\
\textsc{1du} \textsc{du} really \textsc{emph} \textsc{pst}.\textsc{ipfv}-be.in.good.terms-\textsc{1du} \textsc{lnk} \\
\glt `We were in harmony together.' (140512 fushang he yaomo-zh, 85)
\end{exe}

Noun phrases with the dual \forme{ni} are always correlated with a dual prefix on the following noun in possessive constructions or with relator nouns, as in (\ref{ex:rgWnba.ni}), (\ref{ex:ʁnWz.ni}) and (\ref{ex:ni.ndZisroR}). Not a single example of a noun phrase in \forme{ni} followed by a noun with singular of plural possessive prefix is found in the corpus.

\begin{exe}
\ex \label{ex:ni.ndZisroR} 
\gll ɯ-pi ni ndʑi-sroʁ ko-ri tɕe \\
\textsc{3sg}.\textsc{poss}-elder.sibling \textsc{du} \textsc{3du}.\textsc{poss}-life \textsc{ifr}-save \textsc{lnk} \\
\glt `He saved the life of his two brothers.' (qachGa 2012, 139)
\end{exe}

The marker \forme{ni} is not obligatory with dual referents, in particular when the numeral \japhug{ʁnɯz}{two} is present. An overt noun phrase without dual marking can trigger indexation on the verb, especially with collectives expressing a pair of individuals as \japhug{ʁzɤmi}{husband and wife} in (\ref{ex:RjWmbrWg.RzAmi}), but also with other types of noun phrases as in (\ref{ex:nW.talWlAtndZi}).

\begin{exe}
\ex \label{ex:RjWmbrWg.RzAmi}
\gll  kɯɕɯŋgɯ tɕe tɕe atu <qinghai> ʑɴɢɯloʁ nɯtɕu tɕe, ʁjɯmbrɯɣ ʁzɤmi ci pjɤ-tu-ndʑi tɕe,  \\
in.former.times \textsc{lnk}  \textsc{lnk} up.there p.n. p.n. \textsc{dem}:\textsc{loc} \textsc{lnk} dragon husband.and.wife one \textsc{ifr}.\textsc{ipfv}-exist-\textsc{du} \textsc{lnk} \\
\glt `In former times, in Qinghai, in the Mgolog area, there was a couple of dragons.' (150820 qaprANar, 44)
\end{exe}

\begin{exe}
\ex \label{ex:nW.talWlAtndZi}
\gll  ʁdɯxpanaχpu ɯ-tɕɯ cʰo aʑo a-tɕɯ nɯ tɤ-alɯlɤt-ndʑi tɕe, \\
p.n. \textsc{3sg}.\textsc{poss}-son \textsc{comit} \textsc{1sg} \textsc{1sg}.\textsc{poss}-son \textsc{dem} \textsc{pfv}-fight-\textsc{du} \textsc{lnk} \\
\glt `The son of Gdugpa Nagpo and my son were fighting.' (28-smAnmi, 280)
\end{exe}

Such examples are however surprisingly rare in the corpus; dual indexation is most often correlated with a dual marker on the corresponding noun phrase, if overt.

The numeral \japhug{ʁnɯz}{two} without the dual also triggers dual indexation, as in (\ref{ex:RnWz.tundZi}).

\begin{exe}
\ex \label{ex:RnWz.tundZi}
\gll   sɯŋgɯ zɯ tɯrme wuma ʑo kɯ-wxti ʁnɯz tu-ndʑi tɕe\\
forest \textsc{loc} person really \textsc{emph} \textsc{nmlz}:S/A-be.big two exist:\textsc{fact}-\textsc{du} \textsc{lnk}\\
\glt `In the forest, there are two giants.'  (140428 yonggan de xiaocaifeng, 172)
\end{exe}

Dual marking on a noun phrase is not necessarily correlated with dual indexation on the verb, especially, but not exclusively, with inanimate referents, as in (\ref{ex:ni.tomto}). This question is studied in more detail in § XXX.

\begin{exe}
\ex \label{ex:ni.tomto}
\gll  ɯ-mɲaʁ χcʰoʁe ni to-mto. \\
\textsc{3sg}.\textsc{poss}-eye left.and.right \textsc{du} \textsc{pfv}-have.sight \\
\glt `His left and right eyes recovered sight.' (140517 mogui de jing, 105)
\end{exe}

However, a noun phrase with \forme{ni} is never correlated with a plural indexation marker on the verb. Apparent exceptions are either speech errors (a topic treated in § XXX), or cases of ambiguous indexation, as in (\ref{ex:paznAkharnW}).

\begin{exe}
\ex \label{ex:paznAkharnW}
 \gll  nɤ-pi ni kɯ nɤʑo nɯɣi kɤ-sɯso kɯ ʁmaʁ χsɯ-tɤxɯr kɯ pa-z-nɤkʰar-nɯ ɕti tɕe, \\
 \textsc{2sg}.\textsc{poss}-elder.sibling \textsc{du} \textsc{erg} \textsc{2sg} come.back:\textsc{fact} \textsc{inf}-think \textsc{erg} soldier three-round \textsc{erg} \textsc{pfv}:3\fl{}3'-\textsc{caus}-surround-\textsc{pl} be.\textsc{affirm}:\textsc{fact} \textsc{lnk} \\
 \glt `Your two elder brothers, thinking that you are coming back, had (the palace) guarded on all sides by three rows of soldiers.' (qachGa2012, 157)
\end{exe}

Example (\ref{ex:paznAkharnW}) is not completely straightforward, and deserves a detailed comment. The form \forme{paznɤkʰarnɯ} can be parsed as either \forme{pɯ-az-nɤkʰar-nɯ} \textsc{pst}.\textsc{ipfv}-\textsc{prog}-surround-\textsc{pl} `They were guarding it' with vowel fusion (§ XXX) or \forme{pa-z-nɤkʰar-nɯ} \textsc{pfv}:3\fl{}3'-\textsc{caus}-surround-\textsc{pl} `(He/they) had them guard it'. Context makes it clear here that the second option is the correct one, in particular because in the same passage in another version of the same story, we find the verb \forme{pa-sɯ-lɤt} \textsc{pfv}:3\fl{}3'-\textsc{caus}-throw `he had (them) make' with the perfective 3\fl{}3' form of a causative verb (\citealt[242]{jacques16complementation}, § XXX). Moreover, while the phrase \forme{nɤ-pi ni kɯ}  `your two elder brothers' could in principle belong to the infinitival clause in \forme{kɤ-sɯso}\footnote{Incidentally, note that this infinitival clause contains another complement in Hybrid Reported Speech, see § XXX.}, it is clear from the context and the explanations provided by native speakers that \forme{nɤ-pi ni kɯ} is the causer, and \forme{ʁmaʁ χsɯ-tɤxɯr kɯ} `three rows of soldiers' is the causee (also marked by the ergative, see § \ref{sec:causee.kW}). 

We thus observe plural indexation \forme{-nɯ} on the main verb \forme{pa-z-nɤkʰar-nɯ}, while the subject \forme{nɤ-pi ni kɯ}  has a dual marker. However, this is neither a counterexample to the number indexation rule stated above, nor a speech error: rather, it is a consequence of the fact that causees rather than causers can trigger number indexation on the verb in specific cases (see § XXX).

\subsubsection{Plural} \label{sec:plural.determiners}
The plural marker \forme{ra}, like the dual, follows the noun and most of its modifiers, and fuses with the demonstratives \forme{ki} and \forme{nɯ} respectively to build the plural demonstratives \forme{kɯra} and \forme{nɯra} (§ \ref{sec:demonstrative.pronouns}, § \ref{sec:demonstrative.determiners}). The etymology of the plural marker \forme{ra} is unknown, but a potential cognate exists in Pumi (\forme{=ɹə}, (\citealt[135]{daudey14grammar}; Japhug \forme{-a} regularly corresponds to Pumi \forme{-ə} in the native vocabulary). It should not be confused with the auxiliary verb \japhug{ra}{have to, need} (§ XXX), though there are cases where some ambiguity may occur (§ XXX).

Like the dual \forme{ni}, the plural \forme{ra} is compatible with both animate and inanimate referents, as in (\ref{ex:si.ra.cho}) and (\ref{ex:rdAstaR.ra}). It can be a plain marker of plurality as in (\ref{ex:si.ra.cho}).

\begin{exe}
\ex \label{ex:si.ra.cho}
\gll kɯmaʁ si ra cʰo nɯ-mdoʁ mɤ-naχtɕɯɣ \\
other tree \textsc{pl} \textsc{comit} \textsc{3pl}.\textsc{poss}-colour \textsc{neg}-be.the.same:\textsc{fact} \\
\glt `Its colour is different from that of the other trees.' (11-qrontshom, 56)
\end{exe} 

The marker \forme{ra} is also often an associative plural, understandable as `and other things', as in (\ref{ex:rdAstaR.ra}).

\begin{exe}
\ex \label{ex:rdAstaR.ra}
\gll rdɤstaʁ ra pjɯ-tʂaβ-nɯ qʰe tɯrme tu-xtsɯɣ ɲɯ-ŋu \\
stone \textsc{pl} \textsc{ipfv}-cause.to.fall-\textsc{pl} \textsc{lnk} people \textsc{ipfv}-hit \textsc{sens}-be \\
\glt `(Goats and sheep, as they climb high) cause stones (and other things) to fall and these hit people.' (tshAt-qaZo-kAlAG, 4)
\end{exe} 

The plural can follow numerals (even without head noun) to express an approximative number, as in (\ref{ex:XsWm.kWBde}).\footnote{Note that in (\ref{ex:XsWm.kWBde}) \forme{ci ci} is the expression for `sometimes', not used as a numeral, see § XXX.} 

\begin{exe}
\ex \label{ex:XsWm.kWBde}
\gll ci ci χsɯm kɯβde ra ɲɯ-lɤt ɲɯ-ŋgrɤl. tsuku tɕe ʁnɯz jamar ma mɯ́j-lɤt,\\
one one three four \textsc{pl} \textsc{sens}-throw \textsc{sens}-be.usually.the.case. some \textsc{lnk} two about apart.from \textsc{neg}:\textsc{sens}-throw \\
\glt  `Sometimes (dogs) have three or four (litters), some only have two.' (05-khWna, 22)
\end{exe} 

The plural marker \forme{ra} can also indicate approximate location, with or without locative markers. In (\ref{ex:kha.ra}), we find approximate location \forme{ra} in \forme{kʰa ra} `(everywhere) in the house, around the house' and \forme{tɯ-ji ɯ-ngɯ ra} `in the fields', and in (\ref{ex:nWrNa.ra}) with body parts.

This use of \forme{ra} can convey a meaning of distributed location, and is often combined with the adverb \japhug{aʁɤndɯndɤt}{everywhere} (§ \ref{sec:aRandWndAt}). It is reminiscent of plural markers in Kirghiz and Old Japanese, which combine collective, hypocoristic and approximate locative meanings (see \citealt[195]{antonov07ra}).

\begin{exe}
\ex \label{ex:kha.ra}
\gll βʑɯ nɯ wuma ʑo ŋɤn tɕe, tɕendɤre aʁɤndɯndɤt ʑo kʰa ra cʰɯ-rɤpɯ. tɯ-ji ɯ-ngɯ ra cʰɯ-rɤpɯ, \\
mouse \textsc{dem} really \textsc{emph} be.evil:\textsc{fact} \textsc{lnk} \textsc{lnk} everywhere \textsc{emph} house \textsc{pl} \textsc{ipfv}-bear.young \textsc{indef}.\textsc{poss}-field \textsc{3sg}.\textsc{poss}-inside \textsc{pl}  \textsc{ipfv}-bear.young \\
\glt `The mouse is fierce, it has youngs everywhere in the house, and has youngs in the fields.' (27-spjaNkW, 166)
\end{exe} 

\begin{exe}
\ex \label{ex:nWrNa.ra}
\gll nɯ-βri ra ɲɯ-ɬoʁ, nɯ-mke nɯra ɲɯ-ɬoʁ nɯ-rŋa ra brɤβbrɤβ ʑo ɲɯ-ɬoʁ ɲɯ-ŋu. \\
\textsc{3pl}.\textsc{poss}-body \textsc{pl} \textsc{ipfv}-come.out \textsc{3pl}.\textsc{poss}-neck \textsc{dem:pl} \textsc{ipfv}-come.out \textsc{3pl}.\textsc{poss}-face \textsc{pl} \textsc{idph}:II:covered.by.tiny.bumps \textsc{emph} \textsc{ipfv}-come.out  \textsc{sens}-be \\
\glt `(People who suffer from this disease have little blisters) appearing on their body, on their neck and all over their face.' (27-kharwut, 58)
\end{exe} 

The marker \forme{ra} even occurs with referents which are clearly singular, not only in the approximative location function, but also in examples such as (\ref{ex:tAwi.ra}) where the reason for the presence of \forme{ra} is less immediately obvious. In (\ref{ex:tAwi.ra}), a sentence taken from the translation of Rotkäppchen into Japhug (from Chinese, though here the presence of \forme{ra} cannot be due to calque), the function of the plural on the phrase \forme{tɤ-wi ra} `the grandmother' is more subtle: it conveys the idea idea that the impersonation takes on several aspects of the grandmother, not only her physical appearance, but also her voice, as implied by the second clause. 

\begin{exe}
\ex \label{ex:tAwi.ra}
\gll  qapar nɯ kɯ li, [...] tɤ-wi ra to-nɯɕpɯz tɕe, tɕe ɯ-skɤt ra cʰɤ-sɯ-ɤmtɕoʁ ʑo tɕe nɯra to-ti. \\
dhole \textsc{dem} \textsc{erg} again { } \textsc{indef}.\textsc{poss}-grandmother \textsc{pl} \textsc{ifr}-impersonate \textsc{lnk} \textsc{lnk} \textsc{3sg}.\textsc{poss}-voice \textsc{pl} \textsc{ifr}-\textsc{caus}-be.sharp \textsc{emph} \textsc{lnk} \textsc{dem}:\textsc{pl} \textsc{ifr}-say \\
\glt `The wolf was pretending to be the grandmother, and said these (words) with a sharp voice.' (140428 xiaohongmao-zh, 95-96)
\end{exe} 

Just like noun phrases with dual \forme{ni} correlate with dual possessive prefixe (see \ref{ex:ni.ndZisroR} in § \ref{sec:dual.determiners}), those with plural \forme{ra} can only be coreferent with a plural possessive prefix, as \forme{nɯ-} in (\ref{ex:si.ra.nWmat}).

\begin{exe}
\ex \label{ex:si.ra.nWmat}
 \gll  sɯku tɕe tʰɣe kɯ-fse, kɯmaʁ si ra nɯ-mat nɯra ɕ-pjɯ-nɯ-pʰɯt tɕe tu-ndze ɲɯ-ŋu.\\
tree \textsc{lnk} acorn \textsc{nmlz}:S/A-be.like other tree \textsc{pl} \textsc{3pl}.\textsc{poss}-fruit \textsc{dem}:\textsc{pl} \textsc{transloc}-\textsc{ipfv}:\textsc{down}-\textsc{auto}-pluck \textsc{lnk} \textsc{ipfv}-eat[III] \textsc{sens}-be \\
\glt `On the trees, (the bear) plucks acorn or fruits from other trees to eat.' (21-pri, 44)
\end{exe}

Apparent counterexamples such as (\ref{ex:WtaR.ra.Wmat}), where \forme{ra} is followed by a noun with the singular possessive prefix \forme{ɯ-}, occur when the preceding noun phrase is not the possessor of the following noun. For instance, in (\ref{ex:WtaR.ra.Wmat}) \forme{ra} has the vague locative function, and the phrase \forme{tɯ-ŋga ɯ-taʁ ra} `on the clothes' is not the possessor of \japhug{ɯ-mat}{its fruits}, it is a locative adjunct.

\begin{exe}
\ex \label{ex:WtaR.ra.Wmat}
 \gll tɯ-ŋga ɯ-taʁ ra ɯ-mat bɤbɤβ ʑo ku-ndzoʁ. \\
 \textsc{indef}.\textsc{poss}-clothes \textsc{3sg}.\textsc{poss}-on \textsc{pl} \textsc{3sg}.\textsc{poss}-fruit \textsc{idph}:II:in.clusters \textsc{emph} \textsc{ipfv}-\textsc{anticaus}:attach \\
\glt `Its seeds attach on clothes in clusters.' (18-qromJoR, 169)
\end{exe}

The plural \forme{ra} very commonly occurs with headless relatives, with or without a demonstrative, as in (\ref{ex:nW.tCaGi}), where we find both relatives followed by \forme{nɯnɯra} and another one followed by \forme{ra}.

\begin{exe}
\ex \label{ex:nW.tCaGi}
\gll [kɤ-ti mɤ-kɯ-pe kɯ-fse tu-kɯ-ti] nɯnɯra tɕe, [[kɤ-nɯtsɯ kɯ-ra] ra kɯnɤ tu-kɯ-ti] nɯnɯra, 
tɯrme ra kɯnɤ, tɕaɣi tu-sɤrmi-nɯ ŋgrɤl.  \\
\textsc{inf}-say \textsc{neg}-\textsc{nmlz}.S/A-be.good \textsc{nmlz}.S/A-be.like \textsc{ipfv}-\textsc{nmlz}.S/A-say \textsc{dem}:\textsc{pl} \textsc{lnk} \textsc{inf}-hide \textsc{nmlz}.S/A-have.to \textsc{pl} also \textsc{ipfv}-\textsc{nmlz}.S/A-say \textsc{dem}:\textsc{pl} people \textsc{pl} also  parrot \textsc{ipfv}-call-\textsc{pl} be.usually.the.case:\textsc{fact} \\
\glt `Those who say things that one should not say, who say even what should be concealed, even (if they are) people, they call them `parrots'. (24-qro, 125)
\end{exe} 

%ɯʑo sɤz pɣɤtɕɯ kɯ-xtɕi nɯra tu-ndze ɲɯ-ŋu. tɕe nɯnɯ tu-ti-nɯ ɲɯ-ŋu tɕe ɯ-mɤ-ŋu ma,
%ta-ndza ra pɯ́-wɣ-mto me
 
The plural \forme{ra} also occurs between auxiliaries and the preceding complement clause with a verb in finite (\ref{ex:GWkWnWru.ra}) or non-finite (\ref{ex:kAnAjaR.ra}) form, with a vague implication that additional related actions are concerned.

\begin{exe}
\ex \label{ex:GWkWnWru.ra}
 \gll li tɯ-ji ɯ-ŋgɯ ra ɣɯ-ku-nɯru ra ŋgrɤl. \\
 again \textsc{indef}.\textsc{poss}-field \textsc{3sg}.\textsc{poss}-inside \textsc{pl} \textsc{cisloc}-\textsc{ipfv}-eat.crops \textsc{pl} be.usually.the.case:\textsc{fact} \\
\glt `It also (usually) comes to eat crops in the fields.' (24-ZmbrWpGa, 37)
\end{exe}

\begin{exe}
\ex \label{ex:kAnAjaR.ra}
 \gll  ɣɤmdzu tɕe nɯnɯ kɤ-nɤjaʁ ra mɤ-sɤ-nɤz tɕe \\
be.thorny:\textsc{fact} \textsc{lnk} \textsc{dem} \textsc{inf}-touch \textsc{pl} \textsc{neg}-\textsc{deexp}-dare:\textsc{fact} \textsc{lnk} \\
\glt `It is thorny and one does not dare to touch it with the hand.' (11-qrontshom, 91)
\end{exe}

The marker \forme{ra} following a locative noun or adverb can have the meaning `the people/things from X', as in (\ref{ex:alo.ra}), without the need to add a demonstrative (cf \ref{ex:aki.nW} § \ref{sec:demonstrative.determiners}).

\begin{exe}
\ex \label{ex:alo.ra}
 \gll alo ra ɲɯ-mbɣom-nɯ qʰe \\
 upstream \textsc{pl} \textsc{sens}-be.in.a.hurry-\textsc{pl} \textsc{lnk} \\
 \glt `Those in the village, they (do things) in hurry.' (conversation140510 tshering, 175)
\end{exe}

\subsection{Demonstratives} \label{sec:demonstrative.determiners}
Japhug demonstrative determiners are formally identical  to the demonstrative pronouns (§ \ref{sec:demonstrative.pronouns}). They distinguish between proximal and distal demonstratives with different roots, and fuse with the dual and plural markers studied in § \ref{sec:number.determiners}; the proximal \forme{ki} undergoes change to \forme{kɯ-} in those fused forms.

As with the demonstrative pronouns, there are three sets of demonstratives, the base form, the reduplicated one (obtained by reduplicating the first syllable), and the emphatic one, with added \forme{ɯ-} prefix. Note that the latter two sets are not attested in the dual for determiners in the corpus, but the forms exist and are easily deducible from the corresponding plural ones. In addition, there is a medial demonstrative \forme{nɤki} which occurs in prenominal position.

\begin{table}
\caption{Demonstrative determiners}\label{tab:dem.determiners}
\begin{tabular}{ll|l|ll} 
\lsptoprule
&Base form & Reduplicated & Emphatic \\
\midrule
\textsc{prox.sg} & \forme{ki} & \forme{kɯki} &  \forme{ɯkɯki}  \\
\textsc{dist.sg} & \forme{nɯ} &  \forme{nɯnɯ} & \forme{ɯnɯnɯ} \\
\midrule
\textsc{prox.pl} & \forme{kɯni} & X &  X \\
\textsc{dist.pl} & \forme{nɯni} &  X & X \\
\midrule
\textsc{prox.pl} & \forme{kɯra} & \forme{kɯkɯra} &  \forme{ɯkɯkɯra}  \\
\textsc{dist.pl} & \forme{nɯra} &  \forme{nɯnɯra} & \forme{ɯnɯnɯra} \\
\midrule
\textsc{medial} &  \forme{nɤki} \\
\lspbottomrule
\end{tabular}
\end{table}

In Japhug, as in other Gyalrong languages, demonstrative determiners can be either/both pre- and postnominal, as shown by an example such as (\ref{ex:ki.srWnloRpW.ki}) with the proximal \forme{ki} both before and after the noun \japhug{srɯnloʁpɯ}{little ring}.

\begin{exe}
\ex \label{ex:ki.srWnloRpW.ki}
 \gll aʑo ɣɯ-ɕaβ-a tɤ-ŋu tɕe, ki srɯnloʁ-pɯ ki ɲɯ-ɕtʰɯz-a tɕe,  \\
 \textsc{1sg} \textsc{inv}-catch.up:\textsc{fact}-\textsc{1sg} \textsc{pfv}-be \textsc{lnk} \textsc{dem}.\textsc{prox} ring-\textsc{dim} \textsc{dem}.\textsc{prox} \textsc{ipfv}:\textsc{west}-turn.toward-\textsc{1sg} \\
\glt `When (the râkshasas) will be about to catch up with me, I will  turn this little ring towards west (in their direction).' (28-smAnmi, 222)
\end{exe}

All possible combinations of base demonstratives (B) and reduplicated demonstratives (R) are attested as pre- or postnominal determiners:

\begin{itemize}
\item BNB: \forme{ki} N \forme{ki}, \forme{nɯ} N \forme{nɯ} (\ref{ex:ki.srWnloRpW.ki})
\item RNB: \forme{kɯki} N \forme{ki}, \forme{nɯnɯ} N \forme{nɯ} (\ref{ex:kWki.tAYi.ki})
\item BNR: \forme{ki} N \forme{kɯki}, \forme{nɯ} N \forme{nɯnɯ} (\ref{ex:ki.rgAtpu.kWki})
\item RNR: \forme{kɯki} N \forme{kɯki}, \forme{nɯnɯ} N \forme{nɯnɯ} (\ref{ex:kWki.qingjiao.kWki})
\end{itemize}  

The types BNB and RNB, with the postnominal determiner as a base demonstrative, are by far the most common ones in the corpus.

\begin{exe}
\ex \label{ex:kWki.tAYi.ki}
 \gll  aʑo kɯki tɤɲi ki lu-nɤkʰɯkʰrɯt-a tɕe \\
 \textsc{1sg} \textsc{dem}.\textsc{prox} staff \textsc{dem}.\textsc{prox} \textsc{ipfv}:\textsc{upstream}-drag-\textsc{1sg} \textsc{lnk} \\
 \glt `I will drag along this staff (on the ground).' (Kunbzang2003, 225)
\end{exe}
 

\begin{exe}
\ex \label{ex:kWki.qingjiao.kWki}
 \gll iɕqʰa kɯki <qingjiao> kɯki tɕe, ɯ-qa kɯ-wɣrum ɲɯ-ŋu. \\
 the.aforementioned \textsc{dem}.\textsc{prox} plant.name \textsc{dem}.\textsc{prox} \textsc{lnk} \textsc{3sg}.\textsc{poss}-root \textsc{nmlz}:S/A-be.white \textsc{sens}-be \\
 \glt `This (plant that is called) \textit{qingjiao} (in Chinese), its root is white (unlike the other \textit{qingjiao} whose root is red).' (17-ndZWnW, 81)
\end{exe}

\begin{exe}
\ex \label{ex:ki.rgAtpu.kWki}
 \gll ki rgɤtpu kɯki kɯ, iɕqʰa, qaʑo nɯ to-mtsʰi qʰe, li tʂu kɯ-wxti nɯtɕu jo-ɕe tɕe, \\
\textsc{dem}.\textsc{prox} old.man \textsc{dem}.\textsc{prox} \textsc{erg} the.aforementioned sheep \textsc{dem} \textsc{ifr}-lead \textsc{lnk} again road \textsc{nmlz}:S/A-be.big \textsc{dem}:\textsc{loc} \textsc{ifr}-go \textsc{lnk} \\
\glt `The old man, leading the sheep, went to the big road.' (150822 laoye zuoshi zongshi duide-zh, 101
\end{exe}

The emphatic form is only used prenominally as in (\ref{ex:WkWki.arZaB.kWki}) to differentiate in case of confusion -- in this case, because the story is about two persons designated by the term \japhug{tɤ-rʑaβ}{wife}, even if they have different possessors (\textsc{3sg} vs \textsc{1sg}).

\begin{exe}
\ex \label{ex:WkWki.arZaB.kWki}
 \gll   nɯ ɯ-rʑaβ nɯ kɯ, ɯkɯki a-rʑaβ kɯki, kɯki ɕkom ki na-sɯ-ɤβzu tɕe, \\
 \textsc{dem} \textsc{3sg}.\textsc{poss}-wife \textsc{dem} \textsc{erg} \textsc{dem}.\textsc{prox}.\textsc{emph} \textsc{1sg}.\textsc{poss}-wide \textsc{dem}.\textsc{prox} \textsc{dem}.\textsc{prox} muntjac \textsc{dem}.\textsc{prox}  \textsc{pfv}:3\fl{}3'-\textsc{caus}-become \textsc{lnk} \\
\glt `His wife turned this wife of mine into this muntjac.' (140512 fushang he yaomo-zh, 187)
\end{exe}

When the postnominal demonstrative is in plural or dual form, the prenominal one is generally unmarked for number, as in (\ref{ex:kWki.tCheme.kWra}).

\begin{exe}
\ex \label{ex:kWki.tCheme.kWra}
 \gll kɯki tɕʰeme kɯra nɯ-rca aʑo tu-ɕe-a ɲɯ-ntsʰi ma mɯ́j-pe \\
 \textsc{dem} girl \textsc{dem}:\textsc{pl} \textsc{3pl}.\textsc{poss}-following \textsc{1sg} \textsc{ipfv}:\textsc{up}-go-\textsc{1sg} \textsc{sens}-have.better apart.from \textsc{neg}:\textsc{sens}-be.good \\
 \glt `I have no other choice but to go (to heaven) with these girls.' (31-deluge, 61)
 \end{exe}

However, there are also a few examples with plural marking on both pre- and postnominal demonstratives, as in (\ref{ex:nWnWra.pGa.nWra}), a remarkable phenomenon given the fact that the number markers are strictly postnominal. Plural marking on the prenominal demonstrative with a singular postnominal demonstrative is not attested.
 
\begin{exe}
\ex \label{ex:nWnWra.pGa.nWra}
 \gll nɯnɯra pɣa nɯra lonba ʑo ɲɤ-me-nɯ tɕe, ʁʑɯnɯ sqaptɯɣ ɲɤ-k-ɤpa-nɯ-ci. \\
 \textsc{dem.pl} bird  \textsc{dem.pl}  all \textsc{emph} \textsc{ifr}-not.exist \textsc{lnk} young.man eleven \textsc{ifr}-\textsc{evd}-become-\textsc{pl}-\textsc{evd} \\
 \glt `All those birds disappeared, and became eleven young men.' (140520 ye tiane-zh, 121)
\end{exe}

Proximal prenominal demonstratives can be combined with the postnominal \forme{nɯ}, as in (\ref{ex:kWki.Xpi.nW}), where the latter one is used as a topic marker. The opposite combination, a distal prenominal demonstrative with proximal postnominal one, is not attested in the corpus and presumably agrammatical.

\begin{exe}
\ex \label{ex:kWki.Xpi.nW}
 \gll kɯki χpi nɯ pɯpɯŋu nɤ,  \\
 \textsc{dem}.\textsc{prox} story \textsc{dem} \textsc{top} \textsc{lnk} \\
 \glt `As far as this story goes,' (11 examples in the corpus)
\end{exe}

The medial demonstrative \forme{nɤki}, used to designate referents closer to the addressee than the speaker, is found as a pronoun (§ \ref{sec:medial.dem.pro}), but also occurs as a prenominal determiner, with or without postnominal demonstrative (either proximal or distal), as in (\ref{ex:nAki.nAtAYi}) and (\ref{ex:nAki.nAtAri}). It is frequently used with a noun taking a second person possessive prefix -- note that the first syllable \forme{nɤ-} of the demonstrative \forme{nɤki} itself probably originates from the second singular possessive, as proposed in § \ref{sec:medial.dem.pro}.

\begin{exe}
\ex \label{ex:nAki.nAtAYi}
 \gll nɤki nɯ-tɤɲi ɯ-taʁ kɤ-rɤt nɯ ɯβrɤ-kɯ-z-nɤmɲo-a-nɯ \\
 \textsc{dem}:\textsc{medial} \textsc{2pl}.\textsc{poss}-staff \textsc{3sg}.\textsc{poss}-on \textsc{nmlz}:P-write \textsc{dem} \textsc{pot}-2\fl{}1-\textsc{caus}-watch-\textsc{1sg}-\textsc{pl} \\
 \glt `Would you show me what is written on that staff of yours?' (2003sras, 61)
\end{exe}

\begin{exe}
\ex \label{ex:nAki.nAtAri}
 \gll nɤki nɤ-tɤ-ri nɯ ŋotɕu pɯ-tu \\
 \textsc{dem}:\textsc{medial} \textsc{2sg}.\textsc{poss}-\textsc{indef}.\textsc{poss}-thread \textsc{dem} where \textsc{pst}.\textsc{ipfv}-exist \\
\glt `That thread of yours, where is it from?' (Norbzang2005, 180)
\end{exe}

The relative position of prenominal demonstrative and other pronominal elements is not free. The aforementioned topic marker \forme{iɕqʰa} strictly occurs before prenominal demonstratives (as in \ref{ex:kWki.qingjiao.kWki} and \ref{ex:kWki.XsAr.pGAtCW} respectively), while nominal modifiers such as \japhug{χsɤr}{gold} in \ref{ex:kWki.XsAr.pGAtCW} appear closer to the noun. Pronouns coreferent with a possessive prefix on the head noun, however, can be placed either after (\ref{ex:nW.aZo.aCArW.nW}) or before (\ref{ex:aZo.ki.aku.ki}) prenominal demonstratives.

\begin{exe}
\ex \label{ex:kWki.XsAr.pGAtCW}
 \gll kɯki χsɤr pɣɤtɕɯ ki nɤ-jaʁ ɲɯ-kham-a ŋu \\
\textsc{dem}.\textsc{prox} gold bird \textsc{dem}.\textsc{prox} \textsc{1sg}.\textsc{poss}-hand \textsc{ipfv}-give[III]-\textsc{1sg} be:\textsc{fact} \\
\glt `(If you succeed) I will give you this golden bird.' (2012qachGa, 46)
\end{exe}

\begin{exe}
\ex \label{ex:nW.aZo.aCArW.nW}
 \gll nɯtɕu a-tɯrsa ŋu, tɕe nɤʑo kɯ [nɯ aʑo a-ɕɤrɯ nɯnɯra] a-tɤ-tɯ-tɕɤt tɕe, \\
 \textsc{dem}:\textsc{loc} \textsc{1sg}.\textsc{poss}-tomb be:\textsc{fact} \textsc{lnk} \textsc{2sg} \textsc{erg} \textsc{dem} \textsc{1sg} \textsc{1sg}.\textsc{poss}-bone \textsc{dem}:\textsc{pl} \textsc{irr}-\textsc{pfv}-2-take.out \textsc{lnk} \\
\glt `My tomb is there, if you take out my bones (from it),' (150907 niexiaoqian-zh, 109)
\end{exe}

\begin{exe}
\ex \label{ex:aZo.ki.aku.ki}
 \gll  kɯki, aʑo [ki a-ku ki] pɯ-pʰɯt ra \\
 \textsc{dem}.\textsc{prox} \textsc{1sg}  \textsc{dem}.\textsc{prox} \textsc{1sg}.\textsc{poss}-head  \textsc{dem}.\textsc{prox} \textsc{imp}-cut have.to:\textsc{fact} \\
 \glt `Please behead me!' (140507 jinniao-zh, 292)
\end{exe}

The principles governing the presence and absence of the demonstrative determiners, and the choice of the various patterns described above, is particularly complex to describe and will be a topic for future research, when a larger corpus of texts will become available. While the proximal demonstratives always have some deictic function (although it may not be always appropriate to translate them with a demonstrative in other languages such as English), the distal demonstratives clearly contribute to marking topic (§ \ref{sec:topic}) and definiteness (§ \ref{sec:definiteness}), and disentangling these various functions is a complex matter.

The demonstratives \forme{nɯ} and \forme{nɯnɯ} are particularily common after relative clauses (either participial § XXX or finite ones § XXX) and complement clauses (§ XXX) but arguments against analyzing them as subordinators (like English `that') are presented in § XXX. 

Following locative adverbs or locative postpositional phrases, the distal and proximal demonstratives can be used to express the meaning `the one/those X' as in (\ref{ex:athi.ki}) and (\ref{ex:aki.nW}). Note that the number determiner \forme{ra} can also be used in the same way (example \ref{ex:alo.ra} § \ref{sec:plural.determiners}) even without being combined with a demonstrative.

\begin{exe}
\ex \label{ex:athi.ki}
\gll amaŋ amaŋ, atʰi ki kɯ `a-βɣo mɤ-a<nɯ>tɯɣ-a tɕe a-scawa' ɲɯ-sɯsɤm ɲɯ-ŋu ɣe \\
\textsc{interj}:\textsc{surprise} \textsc{interj}:\textsc{surprise}  downstream \textsc{dem}.\textsc{prox} \textsc{erg} \textsc{1sg}.\textsc{poss}-uncle \textsc{neg}-<auto>meet:\textsc{fact}-\textsc{1sg} lnk 1sg.poss-poor.of \textsc{sens}-think[III] \textsc{sens}-be \textsc{sfp} \\
\glt `The one down there, he is thinking `Poor of me, I will not meet my lama', isn't he?' (2003kandZislama, 203)
\end{exe}

\begin{exe}
\ex \label{ex:aki.nW}
\gll a-pa, aki nɯ staʁlupa kɤ-βde ɯ-spa nɯ mɤ-nɯ-xsi ri, \\
\textsc{1sg}.\textsc{poss}-father down \textsc{dem} born.in.the.year.of.the.tiger \textsc{nmlz}:P-throw.away \textsc{3sg}.\textsc{poss}-material \textsc{dem} \textsc{neg}-\textsc{auto}-\textsc{genr}:know \textsc{lnk} \\
\glt `Father, the one down there, I don't know if he is a (boy) born in the year of the Tiger, to be thrown (in the lake), but...' (2011-05-nyima, 154)
\end{exe}

Note however that demonstratives or number markers are not absolutely necessary in such a context. A few (rare) examples of locative postpositional phrases meaning `the one at/in/from' without any modifier can be found, as in (\ref{ex:sWkAku.nWtCu}), where the postpositional phrase is directly followed by the dative, here used in its locative meaning `by (the side of), near, at' (§ \ref{sec:dative}). In this example, the phrase \forme{sɯkɤku nɯtɕu} does not mean  on the treetop', but `the man who is on the treetop'.\footnote{The story from which this example is taken is about three thieves who mistakenly steal a tiger during the night, believing it was an ox; one of the three thieves flees on the top of a tree -- his manner of fleeing being here the characteristic distinguishing him from the other two thieves.}

\begin{exe}
\ex \label{ex:sWkAku.nWtCu}
\gll  [sɯkɤku nɯtɕu] ɯ-pʰe nɯtɕu lo-zɣɯt-ndʑi tɕe  tɕe,\\
treetop \textsc{dem}:\textsc{loc} \textsc{3sg}.\textsc{poss}-\textsc{dat} \textsc{dem}:\textsc{loc}  \textsc{ifr}:\textsc{upstream}-reach-\textsc{du} \textsc{lnk} \textsc{lnk} \\
\glt `(The tiger and the fox) arrived at (the tree where the one who was) on the treetop (was).' (2012-x1-khu, 47)
\end{exe}

\subsection{Quantifiers} \label{sec:quantifiers.determiners}
This section only discusses universal, mid-scalar and specifically distributive quantifiers; numerals and counted nouns, which also serve as quantifiers (in particular distributive ones) are described in chapter \ref{chap:numerals}.

\subsubsection{Universal quantifiers} \label{sec:universal.quant}
The determiner \japhug{tʰamtɕɤt}{all}, from Tibetan \tibet{ཐམས་ཅད་}{tʰams.cad}{all}, is strictly postnominal, as in (\ref{si.thamCAt.kW}). It cannot be used as a pronoun, and there are no examples in the corpus of \japhug{tʰamtɕɤt}{all} following a personal pronoun.

\begin{exe}
\ex \label{si.thamCAt.kW}
 \gll   sɯŋgɯ kɤ-kɯ-nɯχtɕɤn tɕe tɕe si tʰamtɕɤt kɯ nɯnɯ pjɯ-kɯ-sat kɯ-ŋgrɤl ɲɯ-ŋu. \\
 forest \textsc{pfv}-\textsc{nmlz}:S/A-be.dangerous \textsc{lnk} \textsc{lnk} tree all \textsc{erg} \textsc{dem} \textsc{ipfv}-\textsc{genr}:S/P-kill  \textsc{nmlz}:S/A-be.usually.the.case \textsc{sens}-be \\
 \glt `The fierce/dangerous forest, it was (a place where) all the tree would could to kill (people thrown into it).' (28-smAnmi, 191)
\end{exe}

It can also follow a headless relative clause, as in (\ref{WkWndza.thamCAt.nWnWra}), and be followed by demonstratives.

\begin{exe}
\ex \label{WkWndza.thamCAt.nWnWra}
 \gll nɯ-zda rɯdaʁ ɯ-kɯ-ndza tʰamtɕɤt nɯnɯra nɯ-rmi lonba kɯrŋi tu-kɯ-ti ŋu \\
 \textsc{3pl}.\textsc{poss}-companion animal \textsc{3sg}.\textsc{poss}-\textsc{nmlz}:S/A-eat all \textsc{dem}:\textsc{pl}  \textsc{3pl}.\textsc{poss}-name all beast \textsc{ipfv}-\textsc{genr}-say be:\textsc{fact} \\
 \glt `All those which eat the other animals, their name, all of them, is `beast'.'  (150822 kWrNi, 8)
 \end{exe}
 
The combination of a demonstrative such as \japhug{nɯ}{this} with  \japhug{tʰamtɕɤt}{all} does not mean `all of this', but `so much, so many', as in (\ref{kha.nW.thamCAt}) (see § XXX for more examples of this construction).

\begin{exe}
\ex \label{kha.nW.thamCAt}
 \gll iʑora tʰɯ-dɤn-i qʰe, kʰa nɯ tʰamtɕɤt mɯ-ɲɯ-ɤmɯ-xtɕʰɯt-i qʰe \\ 
 \textsc{1pl} \textsc{pfv}-be.many-\textsc{1pl} \textsc{lnk} house \textsc{dem} all \textsc{neg}-\textsc{sens}-\textsc{recip}-have.enough.place-\textsc{1pl} \textsc{lnk} \\
 \glt `There was now more of us (than before), and so many of us could not fit in the house.' (14-tApitaRi, 103-104)
 \end{exe}
 
 Another universal quantifier,  \japhug{kɤsɯfse}{all}, is common as a pronoun (§ \ref{sec:quantifiers.pronouns}) or as an adverb (§ XXX). It is potentially analyzable as a determiner in examples like  (\ref{ex:kAtsa.ra.kAsWfse.kW}) where it follows the noun phrase \forme{kɤtsa ra} `parents and children', and takes the ergative \forme{kɯ}.
 
 \begin{exe}
\ex \label{ex:kAtsa.ra.kAsWfse.kW}
\gll  kɤtsa ra kɤsɯfse kɯ wuma ʑo pjɤ-nɯ-rga-nɯ  \\
parents.and.children \textsc{pl} all \textsc{erg} really \textsc{emph} \textsc{ifr}.\textsc{ipfv}-\textsc{appl}-like-\textsc{pl} \\
\glt `Everybody in the family liked her very much.' (140429 qingwa wangzi, 5)
  \end{exe}
  
A third universal quantifier \japhug{rmɯrmi}{all, all kinds of}, a borrowing from Situ (meaning `everybody'), is attested but only rarely used, in examples such as (\ref{rmWrmi.GW.nWrmi}).
  
\begin{exe}
\ex \label{rmWrmi.GW.nWrmi}
\gll  maka tɤ-rɤku rmɯrmi ɣɯ nɯ-rmi nɯ to-nɤrmi ri maka kɯm mɯ-pjɤ-ɲɟɯ   \\
at.all \textsc{indef}.\textsc{poss}-crops all \textsc{gen} \textsc{3pl}.\textsc{poss}-name \textsc{ifr}-say.name \textsc{lnk} at.all door \textsc{neg}-\textsc{ifr}-\textsc{anticaus}:open \\
\glt `He said the names of all kinds of crops, but the door did not open.' (140512 alibaba-zh, 107)
\end{exe}  

 A fourth universal quantifier, \japhug{lonba}{all} (from Tibetan \tibet{ལོན་པ་}{lon.pa}{reached, enough, completed}), exists in Japhug, but it is not used as a noun determiner and only occurs as an adverb (§ XXX).
% aʑo a-ʁi nɯra lonba aʑo kɯ tɤ-nɤpɯpa-t-a
% 140426 tApAtso kAnWBdaR4, 1
% 
% tɕe, ɯ-tɯ-mbro nɯnɯ cho nɯra
%lonba qaɕti cho naχtɕɯɣ-ndʑi ʑo
%09-sArsi, 17

An alternative construction with a meaning similar to a universal quantifier is the totalitative reduplication (§ XXX) \japhug{kɯ\redp{}kɯ-tu}{all who exist} of the participle of the existential verb \japhug{tu}{exist}, in a post-nominal or head-internal relative clauses, as in (\ref{ex:Wzda.kWkWtu.kW}). 

 \begin{exe}
\ex \label{ex:Wzda.kWkWtu.kW}
\gll tɕe [ɯ-zda ra kɯ\redp{}kɯ-tu] kɯ nɯ-rʑaβ na-nɯ-ɕar-nɯ ɲɯ-ŋu \\
\textsc{lnk} \textsc{3sg}.\textsc{poss}-companion \textsc{pl} \textsc{total}\redp{}\textsc{nmlz}:S/A-exist \textsc{erg} \textsc{3pl}.\textsc{poss}-wife \textsc{pfv}:3\fl{}3'-\textsc{auto}-look.for-\textsc{pl} \textsc{sens}-be \\
\glt `All of his companions took (other women) as their wives.' (Norbzang2005, 57)
  \end{exe}

\subsubsection{Mid-scalar quantifier} \label{sec:tsuku}
The quantifier \japhug{tsuku}{some} is generally used as a pronoun (§ \ref{sec:partitive.pronouns}), but it does occur as a prenominal determiner as in (\ref{ex:tsuku.tWrme}), or a postnominal one as in (\ref{ex:kWmtChW.tsuku}) and (\ref{ex:kWmWrkW.tsuku}). It is most often used in the corpus with human referents, but is compatible with inanimate objects, as shown by (\ref{ex:kWmtChW.tsuku}).

\begin{exe}
\ex \label{ex:tsuku.tWrme}
\gll tsuku tɯrme ra kú-wɣ-mtsɯɣ-nɯ tɕe mɯ́j-ʁdɯɣ, tsuku tɯrme ra [...] kú-wɣ-mtsɯɣ-nɯ tɕe tɕe, wuma ʑo cʰɯ́-wɣ-z-nɯɣmbɤβ-nɯ qʰe ɲɯ́-wɣ-z-nɯtɯfɕɤl-nɯ qʰe ku-rŋgɯ-nɯ ɲɯ-ra.\\
some people \textsc{pl} \textsc{ipfv}-\textsc{inv}-bite-\textsc{pl} \textsc{lnk} \textsc{neg}.\textsc{sens}-be.serious some people \textsc{pl} { } \textsc{ipfv}-\textsc{inv}-bite-\textsc{pl} \textsc{lnk} \textsc{lnk} really \textsc{emph} \textsc{ipfv}-\textsc{inv}-\textsc{caus}-swell-\textsc{pl} \textsc{lnk}  \textsc{ipfv}-\textsc{inv}-\textsc{caus}-have.diarrhea-\textsc{pl} \textsc{lnk} \textsc{ipfv}-lie.down-\textsc{pl} \textsc{sens}-have.to\\
\glt `Some people, when they are stung (by bees) are fine, other people, when they are stung, it causes them swelling and diarrhea and they have to lie down.' (26-ndzWrnaR, 65-67)
\end{exe}

\begin{exe}
\ex \label{ex:kWmtChW.tsuku}
\gll  `pjɯ-nɯβle-a ɲɯ-ra' ɲɤ-sɯso tɕe, kɯmtɕʰɯ tsuku ɲɤ-kʰo tɕe, \\
\textsc{ipfv}-cheat[III]-\textsc{1sg} \textsc{sens}-have.to \textsc{ifr}-think \textsc{lnk} toy some \textsc{ifr}-give \textsc{lnk} \\
\glt `She thought `Let's cheat him' and gave him some toys.' (Norbzang2012, 134)
\end{exe}

\begin{exe}
\ex \label{ex:kWmWrkW.tsuku}
\gll ri kɯ-mɯrkɯ tsuku pjɤ-tu-nɯ tɕe tɕe, \\
\textsc{lnk} \textsc{nmlz}:S/A-steal some \textsc{ifr}.\textsc{ipfv}-exist-\textsc{pl} \textsc{lnk} \textsc{lnk} \\
\glt `There were some thieves.' (X1-khu, 7)
\end{exe}

Note in (\ref{ex:tsuku.tWrme.tWrdoR}) the combination of the quantifier \japhug{tsuku}{some} with the counted noun \japhug{tɯ-rdoʁ}{one piece}, which expresses here a partitive meaning (thirteen of fifteen children for each of them, § \ref{sec:ICN}).

\begin{exe}
\ex \label{ex:tsuku.tWrme.tWrdoR}
\gll tsuku tɯrme tɯ-rdoʁ ɣɯ ɯ-rɟit, sqafsum jamar, sqamŋu jamar tu-kɯ-tu pjɤ-tu. \\
 some person one-piece \textsc{gen} \textsc{3sg}.\textsc{poss}-offspring thirteen about fifteen about \textsc{ipfv}-\textsc{genr}:S/A-exist \textsc{ifr}.\textsc{ipfv}-exist    \\
\glt  `Some (women) had thirteen or fifteen children.' (140426 tApAtso kAnWBdaR, 88)
\end{exe}
 
\subsubsection{Distributive quantifier} \label{sec:raNri}
 Although distributive meaning is generally expressed in Japhug with a counted noun (see in particular § \ref{sec:ICN} and § \ref{sec:CCN}), the postnominal determiner \japhug{raŋri}{each} and its variant \japhug{rɯri}{each} (from Tibetan \tibet{རང་རེ་}{raŋ.re}{each} and \tibet{རེ་རེ་}{re.re}{each}) can also express distributive meaning, as in (\ref{ex:tWtWpW.raNri}). 
 
\begin{exe}
\ex \label{ex:tWtWpW.raNri}
\gll paʁ rcanɯ, tɯ-tɯpɯ raŋri kɯ ʑo pjɯ-χsu-nɯ ra. \\
pig \textsc{unexpected} one-household each \textsc{erg} \textsc{emph} \textsc{ipfv}-raise-\textsc{pl} have.to:\textsc{fact} \\
 \glt `Each single household has to raise pigs.' (05-paR, 4)
 \end{exe}
 
 It can also be used with numerals, as in (\ref{ex:sqamNu.raNri}), where it refers specifically to days.

 \begin{exe}
\ex \label{ex:sqamNu.raNri}
\gll  sqamŋu raŋri ʑo zgo tu-ɕe pɯ-ŋu ɲɯ-ŋu, \\
fifteen \textsc{each} \textsc{emph} mountain \textsc{ipfv}:\textsc{up}-go \textsc{pst}.\textsc{ipfv}-be \textsc{sens}-be \\
\glt `Every fifteen days, she would go up the mountain.' (Norbzang2005, 57)
  \end{exe}

With a noun phrase with the quantifier \japhug{raŋri}{each} occurs in a clause with a counted noun, the scope of the two quantifiers is ambiguous, as in (\ref{ex:rirAB.raNri}).

\begin{exe}
\ex \label{ex:rirAB.raNri}
\gll rirɤβ raŋri χsɯ-tɤxɯr a-tɤ-tɯ-sɯ-lɤt tɕe, \\
mountain each three-round \textsc{irr}-\textsc{pfv}-2-\textsc{caus}-throw \textsc{lnk} \\
\glt `Drag her three times around each mountain.' (Kunbzang2005, 421)
 \end{exe}
 
When a noun with the determiner \forme{raŋri} is possessor, its possessum is often  followed by the distributive determiner \japhug{tɯka}{each} (and its reduplicated variant \forme{tɯkaka}) as in (\ref{ex:Wmat.raNri}).
 
  \begin{exe}
\ex \label{ex:Wmat.raNri}
\gll   iɕqʰa ɯ-mat raŋri ʑo nɯ ɯ-ru tɯka ntsɯ tu. \\
the.aforementioned \textsc{3sg}.\textsc{poss}-fruit each \textsc{emph} \textsc{dem} \textsc{3sg}.\textsc{poss}-stalk own always exist:\textsc{fact} \\
\glt `Each of its fruits has its own stalk.' (17-thowum, 34)
  \end{exe}
  
The  determiner \japhug{tɯka}{each} can also be used without a possessor in  \forme{raŋri}, for example with the distributive pronouns \japhug{ʑaka}{each his own} and \japhug{ʑakastaka}{each his own} (§ \ref{sec:distributive.pronouns})  as in (\ref{ex:nWkho.tWka}).

   \begin{exe}
\ex \label{ex:nWkho.tWka}
\gll      ʑakastaka nɯ-kʰo tɯka pjɤ-tu tɕe \\
each.his.own \textsc{3pl}.\textsc{poss}-room each \textsc{ifr}.\textsc{ipfv}-exist \textsc{lnk} \\
\glt `Each of them had her own room.' (140508 shie ge tiaowu de gongzhu, 85)
   \end{exe}
   
In addition to possessors, \japhug{tɯka}{each} also follows objects, with broad scope on the whole action (\ref{ex:pCaR.tWka}).

    \begin{exe}
\ex \label{ex:pCaR.tWka}
\gll  pɕaʁ tɯka to-βzu-nɯ tɕe jo-nɯ-ɕe-nɯ.  \\
reverence each \textsc{ifr}-make-\textsc{pl} \textsc{lnk} \textsc{ifr}-\textsc{vert}-go-\textsc{pl} \\
\glt `They made a reverence each and went back.' (28-smAnmi, 176)
   \end{exe}
 
\subsection{Indefinite and definite markers} \label{sec:indefinite.markers}

\subsubsection{Indefinite article} \label{sec:indef.article}
The form \japhug{ci}{one} has among its many functions (in addition to pronoun, numeral and adverb, see § \ref{sec:ci.someone}, § \ref{sec:other.pro}, § \ref{sec:partitive.pronouns}, § \ref{sec:identity.modifier}, § \ref{sec:one.to.ten} and § XXX) that of singular indefinite article, as in (\ref{ex:ci.indef}) and (\ref{ex:ci.chAGi}). It is typically used to introduce a new referent in a story.

\begin{exe}
\ex \label{ex:ci.indef}
\gll tɕʰeme kɯ-mpɕɯ\redp{}mpɕɤr ci ɲɤ-nɯ-ɬoʁ \\
girl \textsc{nmlz}:S/A-\textsc{emph}\redp{}beautiful \textsc{indef} \textsc{ifr}-\textsc{auto}-come.out \\
\glt `A very beautiful girl appeared (out of it).' (The flood, 39)
\end{exe}

\begin{exe}
\ex \label{ex:ci.chAGi}
\gll tɕɤlo tɕe tɤ-tɕɯ ci cʰɤ-ɣi qʰe, \\
upstream \textsc{lnk} \textsc{indef}.\textsc{poss}-son \textsc{indef} \textsc{ifr}:\textsc{downstream}-come \textsc{lnk} \\
\glt `A boy came from upstream.' (2003-kWBRa, 41)
\end{exe}

Although \forme{ci} can be used as a partitive pronoun `one of them' (§ \ref{sec:partitive.pronouns}), as a postnominal determiner it does not have partitive meaning. To express a meaning such as `one of the boys', a CN such as \japhug{tɯ-rdoʁ}{one piece} is used instead (§ \ref{sec:ICN}). 

Note that when used as a prenominal modifier, \forme{ci} has a completely different (definite) meaning `the other X' (§ \ref{sec:identity.modifier}). However, the indefinite \forme{ci} is attested in prenominal position if preceded by the prenominal identity modifier \japhug{kɯmaʁ}{other}, as in (\ref{ex:kWmaR.ci.nArZaB}), though the exact syntactic analysis of such sentences may require more research (it is possible that \forme{nɤ-rʑaβ} here is an essive adjunct § \ref{sec:essive.abs}, and does not belong to the same constituent as \forme{kɯmaʁ ci}).

\begin{exe}
\ex \label{ex:kWmaR.ci.nArZaB}
\gll  nɤʑo kɯmaʁ ci nɤ-rʑaβ nɯ-nɯ-ɕar kɯ mna  \\
\textsc{2sg} other \textsc{indef} \textsc{2sg}.\textsc{poss}-wife \textsc{imp}-\textsc{auto}-search \textsc{erg} be.better:\textsc{fact} \\
\glt `You should look for another wife.' (150909 xiaocui, 163)
\end{exe}

There are no dual or plural indefinite articles in Japhug. The plural marker \forme{ra} can occur after the indefinite \forme{ci}, but with a vague associative meaning `and other things' as in (\ref{ex:ci.ra}).

\begin{exe}
\ex \label{ex:ci.ra}
 \gll  ndʑi-tɕɯ ci, ndʑi-me ci ra to-tu. \\
 \textsc{3du}.\textsc{poss}-son \textsc{indef}  \textsc{3du}.\textsc{poss}-girl \textsc{indef} \textsc{pl} \textsc{ifr}-exist \\
 \glt  `They$_{du}$ had a boy and a girl (etc).' (150827 tianluo-zh, 155)
\end{exe}

 The indefinite \forme{ci} is not obligatory for indefinite referents (whether specific or non-specific), and bare NPs can used as \japhug{fsapaʁ}{animal} and \japhug{qapar}{dhole} in example (\ref{ex:ci.ra2}).
 

\begin{exe}
\ex \label{ex:ci.ra2}
 \gll  fsapaʁ nɯ-me, a-pɯ-si qhe, `nɯ qapar kɯ ta-ndza ŋu ma' tu-ti-nɯ ɕti ma, \\
 animal \textsc{pfv}-not.exist \textsc{irr}-\textsc{pfv}-die \textsc{lnk} \textsc{dem} dhole \textsc{erg} \textsc{pfv}:3\fl{}3'-eat be:\textsc{fact} \textsc{sfp} \textsc{ipfv}-say-\textsc{pl} be.\textsc{affirm}:\textsc{fact} \textsc{lnk}  \\
 \glt `When an animal disappears, dies, people say `A dhole ate it.' (28-qapar, 
\end{exe}


\subsubsection{Indefinite pronoun as modifier} \label{sec:indefinite}
The indefinite pronoun \japhug{tʰɯci}{something} (§ \ref{sec:thWci}) has marginal uses as a prenominal indefinite modifier, as in  (\ref{ex:thWci.laXCi}), (\ref{ex:thWci.WjmNo}) and (\ref{ex:laXtCha.ci.nWnW}) below. 

\begin{exe}
\ex \label{ex:thWci.laXCi}
\gll   tʰɯci laχɕi ci ɕ-pɯ-nɯ-βzjoz-nɯ tɕe, jɤ-ɕe-nɯ ra \\
something trade \textsc{indef} \textsc{transloc-imp-auto}-learn-\textsc{pl} \textsc{lnk} \textsc{imp}-go-\textsc{pl} have.to:\textsc{fact} \\
\glt `Go and learn some trade!' (140508 benling gaoqiang de si xiongdi-zh, 29)
 \end{exe}
 
 This construction arose perhaps from the use of the pronoun \forme{tʰɯci} as head of a postnominal relative clause with the verb \japhug{fse}{be like}, as illustrated by examples like (\ref{ex:thWci.kAnWsaXCWB}) or (\ref{ex:thWci.akAspa}) in § \ref{sec:thWci}. Turning the verb \japhug{fse}{be like} to a finite form as in (\ref{ex:thWci.WjmNo}) could cause the indefinite \forme{tʰɯci}, head of the relative in (\ref{ex:thWci.kAnWsaXCWB}), to be reanalyzed as the prenominal modifier of the immediately adjacent noun in (\ref{ex:thWci.WjmNo}).

 \begin{exe}
\ex \label{ex:thWci.kAnWsaXCWB}
\gll nɯra [tʰɯci [kɤ-nɯsaχɕɯβ kɯ-fse]] pɯ-ŋu wo.  \\
\textsc{dem}:\textsc{pl} something \textsc{inf}-have.a.contest \textsc{nmlz}:S/A-be.like \textsc{pst}.\textsc{ipfv}-be \textsc{sfp} \\
\glt `It was like a kind of contest.' (160706 thotsi, 16)
 \end{exe}
 
\begin{exe}
\ex \label{ex:thWci.WjmNo}
\gll [tʰɯci ɯ-jmŋo] ci ʑo pɯ-fse ri \\
something \textsc{3sg}.\textsc{poss}-dream one \textsc{emph} \textsc{pst}.\textsc{ipfv}-be.like \textsc{lnk} \\
\glt `It looked like (he had had) some dream.' (Lobzang2005, 74)
 \end{exe}
 
 
\subsubsection{The marking of definiteness} \label{sec:definiteness}
Japhug has no dedicated definite determiner, but  \forme{nɯ} and \forme{nɯnɯ}  as demonstrative determiners (\ref{sec:demonstrative.determiners}) and as topic markers (\ref{sec:topic}) and the prenominal aforementioned topic marker \forme{iɕqʰa} (§ \ref{sec:iCqha}) are generally used with definite referents.  

Example (\ref{ex:ci.joGi}) illustrates a typical example with the determiner \forme{nɯ}; the indefinite article \forme{ci} (§ \ref{sec:indef.article}) occurs in the first introduction of a new referent in the story as in the first clause of example (\ref{ex:ci.joGi}), but on the following occurrence of the same noun \forme{nɯ} is found.

\begin{exe}
\ex \label{ex:ci.joGi}
 \gll  tɕe qajdo ci jo-ɣi tɕe, tɕe qajdo nɯ kɯ `mo laz tu, pʰo laz me' to-ti. \\
 \textsc{lnk} crow \textsc{indef} \textsc{ifr}-come \textsc{lnk} \textsc{lnk} crow \textsc{dem} \textsc{erg} girl karma exist:\textsc{fact} boy karma not.exist:\textsc{fact} \textsc{ifr}-say \\
 \glt `A crow came. The crow said: `The girl will have chance, the boy won't.'' (28-qAjdoskAt, 8)
\end{exe}

However, although nouns phrases followed by \forme{nɯ} and \forme{nɯnɯ} more often than not denote definite referents, these determiners cannot be analyzed as definite articles, as noun phrases with \forme{nɯ} or \forme{nɯnɯ} can in certain cases have indefinite referents. 

A very clear case of use of \forme{nɯ} with an indefinite referent occurs on nouns serving as heads of head-internal relative clauses. A well-attested typological generalization is that in this type of relative clauses, definiteness marking is agrammatical (see \citealt{basilico96internally} and § XXX). In Khroskyabs, \citet[636]{lai17khroskyabs} reports that the definiteness marker \forme{=tə} is indeed not accepted on the head noun of head-internal relatives. In Japhug however, \forme{nɯ} does occur in such a syntactic context. For instance, in (\ref{ex:tAnmaR.nW.kW}), the head \forme{tɤ-nmaʁ nɯ kɯ} is subject of the participle \japhug{ɲɯ-kɯ-nɯ-ɕar}{looking for}, and is embedded in the participial relative clause indicated in brackets -- the presence of the ergative \forme{kɯ} precludes to analyze it as a post-nominal relative (§ XXX). From the meaning of the sentence the head \japhug{tɤ-nmaʁ}{husband} is clearly indefinite non-specific non-generic  (see \citealt[286-291]{lehmann84relativsatz}). The fact that it takes the marker \forme{nɯ} shows that this marker, unlike Khroskyabs \forme{=tə}, is not primarily marking definiteness.

\begin{exe}
\ex \label{ex:tAnmaR.nW.kW}
 \gll tɕeri [tɤ-nmaʁ nɯ kɯ ɯ-rʑaʁ kɯ-ɤntɕʰɯ ɲɯ-kɯ-nɯ-ɕar], aʁɤndɯndɤt tɤndɤɣri tu-kɯ-βzu pjɤ-tu.  \\
but  \textsc{indef}.\textsc{poss}-husband \textsc{dem} \textsc{erg} \textsc{3sg}.\textsc{poss}-wife  \textsc{nmlz}:S/A-be.many \textsc{ipfv}-\textsc{nmlz}:S/A-\textsc{auto}-search everywhere  illegitimate.child  \textsc{ipfv}-\textsc{nmlz}:S/A-make \textsc{ifr}.\textsc{ipfv}-exist \\
\glt `However there were husbands who were looking for several women and had illegitimate children.' (140427 tAndAGri, 3)
\end{exe}

Other cases of indefinite noun phrase with \forme{nɯ} are observed with left-dislocated topics. In example (\ref{ex:RnWz.nWnW}), we find a type of tail-head linkeage  (§ XXX) where both the noun phrase \japhug{spjaŋkɯ ʁnɯz}{two wolves} and the verb \japhug{ɲɤ-k-ɤtɯɣ-ci}{he met} are repeated; in the second occurrence, the noun phrase is topicalized and is followed by the topic marker \forme{nɯnɯ}, with a slight pause of hesitation. The determiner \forme{nɯnɯ} in this clause, unlike \forme{nɯ} in (\ref{ex:ci.joGi}), does not mark definiteness: that clause cannot be understood as `He met the two wolves'.

\begin{exe} 
\ex \label{ex:RnWz.nWnW} 
 \gll spjaŋkɯ ʁnɯz ɲɤ-k-ɤtɯɣ-ci. spjaŋkɯ ʁnɯz nɯnɯ, tɕendɤre ɲɤ-k-ɤtɯɣ-ci tɕe iɕqʰa, kɯ-rɤ-ntɕʰa nɯ wuma ʑo ɲɤ-mu. \\ 
 wolf two \textsc{ifr}-\textsc{evd}-meet-\textsc{evd}  wolf two \textsc{dem} \textsc{lnk} \textsc{ifr}-\textsc{evd}-meet-\textsc{evd} \textsc{lnk} the.aforementioned \textsc{nmlz}:S/A-\textsc{a.pass}:\textsc{n.hum}-kill \textsc{dem} really \textsc{emph} \textsc{ifr}-be.afraid \\ 
 \glt `He$_i$ (the butcher) met two wolves. He$_i$ met two wolves, and the butcher$_i$ was very much afraid.' (150902 liaozhai lang-zh, 7-8)
\end{exe}

The determiners \forme{nɯ} or \forme{nɯnɯ} are followed by the indefinite singular article \forme{ci} in the corpus if both have scope on the same noun. In all cases with \forme{ci} followed by \forme{nɯ} (other than the identity pronoun in § \ref{sec:other.pro}), or of \forme{nɯ} followed by \forme{ci} in the corpus, they belong to different constituents. For instance, in (\ref{ex:ci.YAZGAsAphAr}), \forme{ci} is in adverbial use (`a little, once', see § XXX) and does not belong to the preceding noun phrase.  

\begin{exe}
\ex \label{ex:ci.YAZGAsAphAr}
\gll [tɕʰeme nɯ] ci ɲɤ-ʑɣɤ-sɤpʰɤr qʰe  \\
girl \textsc{dem} one \textsc{ifr}-\textsc{refl}-shake \textsc{lnk} \\
\glt `The girl shook herself.' (02-deluge2012, 125)
\end{exe}

In (\ref{ex:laXtCha.ci.nWnW}) although \forme{nɯnɯ} follows \forme{ci}, it has scope over the both preceding phrases, which are left-dislocated and followed by a pause.

\begin{exe}
\ex \label{ex:laXtCha.ci.nWnW}
\gll  kɤ-xtɕɤr tɕe nɯnɯ tɕe tɕe iɕqʰa, [[tʰɯci tɯmbri tɤ-ri kɯ-fse kɯ] [laχtɕʰa ci] nɯnɯ], ci kú-wɣ-sɯ-pa tɕe, kú-wɣ-xtɕɤr, \\
\textsc{inf}-attach \textsc{lnk} \textsc{dem} \textsc{lnk} \textsc{lnk} the.aforementioned something rope \textsc{indef}.\textsc{poss}-thread \textsc{nmlz}:S/A-be.like \textsc{erg} thing \textsc{indef} \textsc{dem} one \textsc{ipfv}-\textsc{inv}-\textsc{caus}-do \textsc{lnk} \textsc{ipfv}-\textsc{inf}-attach \\
\glt ``To attach' (means), to put together, attach something with something like a rope or a thread.'  (150902 kAxtCAr, 2-3)
\end{exe}

However, the indefinite article \forme{ci} can be followed by \forme{nɯ} as in (\ref{ex:kWrZi.ci.nW}).

\begin{exe}
\ex \label{ex:kWrZi.ci.nW}
\gll nɯ, kɯ-rʑi ci nɯ ɣɤʑu maʁ kɯ \\
\textsc{dem} \textsc{nmlz}:S/A-be.heavy \textsc{indef} \textsc{dem} exist:\textsc{sens} not.be:\textsc{fact} \textsc{sfp} \\
\glt `This is not a difficult thing (to do).' (divination2005, 15)
\end{exe}

The aforementioned topic marker \forme{iɕqʰa} (§ \ref{sec:iCqha}) is almost always used with definite referents when prenominal, as in (\ref{ex:RnWz.nWnW}) above, and is the closest candidate to be analyzed as a definiteness marker in Japhug. It does occur with non-specific generic referents as in (\ref{ex:lWlAmu}), including some that are very clearly indefinite as in (\ref{ex:lApWG}); note the absence of postnominal determiner \forme{nɯ} (\ref{ex:lApWG}).

\begin{exe}
\ex \label{ex:lWlAmu}
 \gll iɕqʰa lɯlɤmu nɯ tʰɯ-rɤpɯ tɕe tɕe ɯ-sŋi tɕe kɤ-nɯ-rŋgɯ nɯ stʰɯci mɯ́j-tsu ma ɯ-pɯ ra χse ɲɯ-ra tɕe, \\
 the.aforementioned female.cat \textsc{dem} \textsc{ipfv}-bear.young \textsc{lnk} \textsc{lnk} \textsc{3sg}.\textsc{poss}-day \textsc{lnk} \textsc{inf}-\textsc{auto}-lie.down \textsc{dem} so.much \textsc{neg}:\textsc{sens}-have.time.to \\
 \glt `A/the female cat (unlike male cats), when it had had youngs, does not have time to sleep during the day, as it has to feed its youngs.' (21-lWLU, 
\end{exe}

\begin{exe}
\ex \label{ex:lApWG}
\gll  iɕqʰa lɤpɯɣ ɯ-rɣi ʑo fse. \\
the.aforementioned radish \textsc{3sg}.\textsc{poss}-seed \textsc{emph} be.like:\textsc{fact} \\
\glt `It looks like a radish seed.' (hist-26-qro-fourmi, 61)
\end{exe}

In  (\ref{ex:laXtCha.ci.nWnW}), \forme{iɕqʰa}  also precedes two phrases involving indefinite referents, but  there is a marked pause, and this is a case of \forme{iɕqʰa} in its function as speech filler (see § XXX).

\subsubsection{Absence of definiteness marking}
Like many languages (\citealt[130]{creissels06sgit1}), Japhug uses bare nouns without any definiteness marking. Bare nouns are most often non-referential, as \japhug{tɕʰeme}{girl} in (\ref{ex:tCheme.tWtAtu}).

\begin{exe}
\ex \label{ex:tCheme.tWtAtu}
\gll ʁnaʁna tɕʰeme tɯ\redp{}tɤ-tu nɤ, kɤndʑisqʰaj tu-kɤ-sɯ-βzu \\
both girl \textsc{cond}\redp{}\textsc{pfv}-exist \textsc{lnk} \textsc{coll}:sister \textsc{ipfv}-\textsc{inf}-\textsc{caus}-make \\
\glt `If both of them have girls, let them be sisters.' (zrAntCW, 4)
\end{exe}

Bare nouns are less common with referential nouns (except in answers to questions), but examples can be found, as \japhug{qacʰɣa}{fox} in (\ref{ex:qachGa.kW}).

\begin{exe}
\ex \label{ex:qachGa.kW}
\gll qacʰɣa 	kɯ maχtɕɯ tɤ-tɯt-a nɯ mɤ-tɯ-ste ti ɲɯ-ŋu \\
fox \textsc{erg} I.told.you.so \textsc{pfv}-say[II]-\textsc{1sg} \textsc{dem} \textsc{neg}-2-do.like[III]:\textsc{fact} say:\textsc{fact} \textsc{sens}-be \\
\glt `The fox says: `You do not do as I told you to." (2003qachGa, 44)
\end{exe}

Personal names generally occur as bare nouns, without any definiteness marker as in (\ref{ex:WrJAnpanma}), but there are no constraints against co-occurrence of personal names with the determiner \forme{nɯ} either (see § \ref{sec:personal.names.modifiers}).

\begin{exe}
\ex \label{ex:WrJAnpanma}
\gll  ɯrɟɤnpanma kɯ ʁlaŋsaŋtɕhin ɯ-ɕki  \\
 Padmasambhava \textsc{erg} Gesar \textsc{3sg}-\textsc{dat} \\
\glt `Padmasambhava (told) Gesar.' (Gesar, 2)
\end{exe}

 \subsection{Topic markers} \label{sec:topic}
 
  \subsubsection{Delimitative topic} \label{sec:delimitative}
The delimitative topic marker \forme{pɯ\redp{}pɯ-ŋu nɤ} `as for..., concerning...' is transparently derived from the past imperfective of the verb `be' in conditional form `if it was...' (with reduplication of the first syllable, see § XXX), as other copulas such as affirmative \japhug{ɕti}{be} and \japhug{maʁ}{not be} in (\ref{ex:pWpWmaʁ}).

\begin{exe}
\ex \label{ex:pWpWmaʁ}
\gll nɯnɯ koŋla ʑo tɤɕime pɯ\redp{}pɯ-maʁ nɤ \\
\textsc{dem} really \textsc{emph} princess \textsc{cond}\redp{}\textsc{pst.ipfv}-not.be lnk \\
\glt `If she was not really a princess,' (140519 wandou gongzhu, 71)
\end{exe}

The delimitative construction generally has scope over a noun phrase, which can have an additional demonstrative \forme{nɯ} as topicalizer as in (\ref{ex:nW.pWpWNunA}) (see § \ref{sec:nW.topic}).

\begin{exe}
\ex \label{ex:nW.pWpWNunA}
\gll a-mu nɯ pɯ\redp{}pɯ-ŋu nɤ, qhlɯ ʁdɯxpakɤrpu ɣɯ ɯ-me stu kɯ-xtɕi nɯ a-mu ɲɯ-pe, \\
\textsc{1sg}.\textsc{poss}-mother \textsc{dem} \textsc{cond}\redp{}\textsc{pst.ipfv}-be \textsc{lnk} nâga p.n \textsc{gen} \textsc{3sg}.\textsc{poss}-daughter most \textsc{nmlz}:S/A-be.small \textsc{dem} \textsc{1sg}.\textsc{poss}-mother \textsc{sens}-be.good \\
\glt `As for my mother, the daughter of the Nâga Gdugpa dkarpo is good to be my mother.' (Gesar, 5)
\end{exe}

In this construction, the verb is in the process of becoming grammaticalized as a topic particle. It is possible to find examples where the verb still takes person indexation in the delimitative construction when the topicalized element is a first or second person pronoun, as in (\ref{ex:pWpWNuanA}). 

\begin{exe}
\ex \label{ex:pWpWNuanA}
\gll aʑo pɯ\redp{}pɯ-ŋu-a nɤ, kɤndʑiʁi kɯmŋu tu-j, \\
\textsc{1sg} \textsc{cond}\redp{}\textsc{pst.ipfv}-be-\textsc{1sg} \textsc{lnk} siblings five exist:\textsc{fact}-\textsc{1sg} \\
\glt `Concerning me, we are five brothers and sisters.' (hist140501 tshering skyid, 1)
\end{exe}

However, there are also examples with first or second person pronoun without indexation on the delimitative marker, as in (\ref{ex:pWpWNunA}), (\ref{ex:pWpWNunA2}) and (\ref{ex:pWpWNunA3}), where a first person singular form \forme{pɯ\redp{}pɯ-ŋu-a nɤ} or second person \forme{pɯ\redp{}pɯ-tɯ-ŋu nɤ} would have been expected. Such examples show that \forme{pɯpɯŋunɤ} has ceased to be analyzed as a verb form at least in these cases. Moreover, third person plural and dual indexation is hardly ever found in the delimitative construction.

\begin{exe}
\ex \label{ex:pWpWNunA}
\gll nɤʑo pɯpɯŋunɤ, ɬɤndʐi ra ɣɯ nɯ-kɯ-βʁa, nɯ-rɟɤlpu tɯ-ŋu \\
\textsc{2sg} as.for demon \textsc{pl} \textsc{gen} \textsc{3pl.poss}-\textsc{nmlz}:S/A-be.victorious \textsc{3pl.poss}-king 2-be:\textsc{fact} \\
\glt `You, you are the king of the demons.' (hist140512 fushang he yaomo-zh, 61)
\end{exe}

\begin{exe}
\ex \label{ex:pWpWNunA2}
\gll  aʑo kɯ-fse pɯpɯŋunɤ, ɕɯŋgɯ sɤ-xtɕɯ\redp{}xtɕi nɯtɕu, χpɯn lɤ-kɤ-ta, \\
\textsc{1sg} \textsc{nmlz}:S/A-be.like as.for  before \textsc{conv}-\redp{}be.small \textsc{dem}:\textsc{loc} monk \textsc{pfv}:\textsc{upstream}-\textsc{nmlz}:P-put \\
\glt `For instance me, (I was) sent to become monk early in my childhood.' (160721 XpWN, 7)
  \end{exe}

\begin{exe}
\ex \label{ex:pWpWNunA3}
\gll aʑo pɯpɯŋunɤ, nɯnɯ [...] aʑo ɣɯ a-ndʐa nɯ tu-o<nɯ>lɯlat-a pɯ-ŋu tɕe, \\
\textsc{1sg} as.for \textsc{dem} { } \textsc{1sg} \textsc{gen} \textsc{1sg}.\textsc{poss}-reason \textsc{dem} \textsc{ipfv}-<\textsc{auto}>fight-\textsc{1sg} \textsc{pst}.\textsc{ipfv}-be \textsc{lnk} \\
\glt  As for me, I was fighting for my own sake.' (140512 abide he mogui-zh, 92)
 \end{exe}
 
A short form \forme{ŋunɤ} instead of \forme{pɯpɯŋu nɤ} is also attested, as in (\ref{ex:WNga.ra.NunA}).

\begin{exe}
\ex \label{ex:WNga.ra.NunA}
\gll ma ɯ-ŋga ra ŋunɤ, maka wuma ʑo ko-ɴqhi ma. \\
\textsc{lnk} \textsc{3sg}.\textsc{poss}-clothes \textsc{pl} as.for at.all really \textsc{emph} \textsc{ifr}-be.dirty \textsc{lnk} \\
\glt `As for his clothes, they had become very dirty.' (conversation 140510)
\end{exe}
 
The delimitative topic  construction is appropriate to introduce the main topic of a following discourse (as in \ref{ex:pWpWNuanA} and \ref{ex:pWpWNunA2}), but can be used for contrastive topics, as in example (\ref{ex:pWpWNunA3}) where the speaker expresses a contrast between his and the addresses action (`you, you were fighting for the sake of other people').


 \subsubsection{Aforementioned topic} \label{sec:iCqha}
 The marker \japhug{iɕqʰa}{the aforementioned}  is used on referents that have been previously mentioned in the same story, usually only a few sentences back. It is strictly prenominal. 
 
Example (\ref{ex:iCqha.aforementioned}) illustrates the most typical use of this marker. Sentence (\ref{ex:kAtWm}) introduces a new reference, \japhug{kɤtɯm}{ball of thread} marked with the indefinite article \forme{ci} (§ \ref{sec:indef.article}). Three clauses later in (\ref{ex:iCqha.kAtWm}), the same referent occurs again with two topic markers, the postnominal \textit{nɯ} and the prenominal \textit{iɕqʰa}.
 
 
\begin{exe}
\ex \label{ex:iCqha.aforementioned}
\begin{xlist}
\ex \label{ex:kAtWm}
\gll `razri \textbf{kɤtɯm} \textbf{ci} ɲɯ-ra, taqaβ ci ɲɯ-ra' to-ti qʰe   \\
 thread ball \textsc{indef} \textsc{sens}-need needle \textsc{indef} \textsc{sens}-need \textsc{ifr}-say \textsc{lnk}  \\
\glt `He told (Rgyabza) `I need a ball of thread and a needle.''  
\ex  
\gll tɕendɤre ɲɤ-kʰo qʰe,  \\
\textsc{lnk} \textsc{ifr}-give \textsc{lnk}   \\
\glt `She gave it to him.'
\ex 
\gll  tɕe ɯ-ndzɤtsʰi ka-tsɯm-nɯ nɯtɕu qʰe tɕe,   \\
 \textsc{lnk} \textsc{3sg}.\textsc{poss}-meal \textsc{pfv}:3\fl{}3'-bring-\textsc{pl} \textsc{dem}:\textsc{loc}  \textsc{lnk} \textsc{lnk}    \\
\glt `When they brought his meal,'
\ex \label{ex:iCqha.kAtWm}
\gll   \textbf{iɕqʰa} \textbf{kɤtɯm} \textbf{nɯ} ɯʑo kɯ ko-ndo, \\
   the.aforementioned ball \textsc{dem} \textsc{3sg} \textsc{erg} \textsc{ifr}-take \\
\glt `he took the ball of thread, and...' (Gesar 270-272)
\end{xlist}
\end{exe}
 
A systematic study of the use of the topic marker \forme{iɕqʰa} in Japhug must overcome two inherent difficulties. First, this topic marker is homophonous with (and historically related to) the speech filler \forme{iɕqʰa} (§ XXX) and with the adverb \japhug{iɕqʰa}{just now}, which can also precede noun phrases. Listening to the sound files can help distinguishing between the three, as the speech filler is always followed by a pause (and optionally by the demonstrative \forme{nɯ}), but there are still ambiguous sentences (see below). Second, \forme{iɕqʰa} occurs on nouns designating entities that the speaker considers to have been previously referred to in the conversation, even if they are not present in the same recording. 

For instance in (\ref{ex:iCqha.pɣArnoR}) the noun \japhug{pɣɤrnoʁ}{a species of fungus} is used with \forme{iɕqʰa}, although this name does not occur before in the same text; it was however mentioned the day before in another recording.

\begin{exe}
\ex \label{ex:iCqha.pɣArnoR}
\gll nɯ zdɯmqe cʰo iɕqʰa, pɣɤrnoʁ nɯni ndʑi-tsʰɯɣa wuma ʑo naχtɕɯɣ. \\
\textsc{dem} fungi.sp. \textsc{comit} the.aforementioned fungi.sp. \textsc{dem}:\textsc{du} \textsc{3du}.\textsc{poss}-form really \textsc{emph} be:identical:\textsc{fact} \\
\glt `The \forme{zdɯmqe} and the \forme{pɣɤrnoʁ} are very similar.' (23-mbrAZim, 82)
\end{exe}

 
The topic marker \forme{iɕqʰa} transparently comes from the adverb \japhug{iɕqʰa}{just now} (§ XXX). The pivot constructions that allowed reanalysis from adverb to prenominal topic marker are very probably headless relatives (§ XXX) as in  (\ref{ex:iCqha.tAtWta}), or complement clauses as in (\ref{ex:iCqha.ZnWzmWnmuta}). 

\begin{exe}
\ex \label{ex:iCqha.tAtWta}
 \gll  [iɕqʰa tɤ-tɯt-a] nɯ tú-wɣ-stu qʰe, \\
 just.now \textsc{ifr}-say[II]-\textsc{1sg} \textsc{dem} \textsc{ipfv}-\textsc{inv}-do.like \textsc{lnk} \\
\glt `One does as I just said, and...' (2002tWsqar, 139)
\end{exe}

\begin{exe}
\ex \label{ex:iCqha.ZnWzmWnmuta}
 \gll iɕqʰa [ʑ-nɯ-z-mɯnmu-t-a] nɯ mɯ-pjɤ-pe rcama.  \\
the.aforementioned  \textsc{transloc}-\textsc{pfv}-\textsc{caus}-move-\textsc{pst}:\textsc{tr}-\textsc{1sg} \textsc{dem} \textsc{neg}-\textsc{ifr}.\textsc{ipfv}-be.good \textsc{fsp} \\
\glt `It was probably not a good thing that I had moved them (as I said above).' (150819 kumpGa, 45)
 \end{exe}
 
 These sentences are still synchronically ambiguous in Japhug; in  (\ref{ex:iCqha.ZnWzmWnmuta}) the context makes it clear that \forme{iɕqʰa} is the topic marker (since the fact of having moved (the eggs) had been told a few sentences back) and not an adverb `just now' with a temporal reference in the past, as the meaning would be `it was probably not a good thing that I had just moved them' (an impossible interpretation in this context, since this sentence is an explanation why several eggs had not given chicks, several days after they had been brought to another place). However, extracted from the context, both interpretation would be equally possible for (\ref{ex:iCqha.ZnWzmWnmuta}), and correspond to two distinct syntactic structures.

With postnominal (§ XXX) or left-headed head-internal relative clauses (§ XXX) as in (\ref{ex:tWrpa.thafse}), \forme{iɕqʰa} can also be ambiguous. Since the adverb \japhug{iɕqʰa}{just now} can occur both before the object (\ref{ex:tWrpa.thWfseta}) or before the verb (\ref{ex:tWrpa.thWfseta2}) in an independent clause, a relative such as (\ref{ex:tWrpa.thafse}) can be either interpreted `the axe (mentioned above) that he had whetted' (with the topic marker \forme{iɕqʰa} outside of the relative clause, having scope on its head) and `the axe that he had just whetted' with the adverb \japhug{iɕqʰa}{just now} inside the relative clause.

 \begin{exe}
\ex \label{ex:tWrpa.thafse}
 \gll  tɕendɤre <luban> kɯ iɕqʰa [tɯrpa tʰa-fse] nɯ to-ndo tɕe, \\
 \textsc{lnk} p.n. \textsc{erg} the.aforementioned axe \textsc{pfv}:3\fl{}3'-whet \textsc{dem} \textsc{ifr}-take \textsc{lnk} \\
 \glt `Luban took the axe that he had whetted.' (150902 luban-zh, 90)
 \end{exe}

  \begin{exe}
  \ex 
  \begin{xlist}
\ex \label{ex:tWrpa.thWfseta}
 \gll   iɕqʰa tɯrpa tʰɯ-fse-t-a \\
just.now axe \textsc{pfv}-whet-\textsc{pst}:\textsc{tr}-\textsc{1sg} \\
\ex \label{ex:tWrpa.thWfseta2}
 \gll   tɯrpa  iɕqʰa tʰɯ-fse-t-a \\
 axe just.now \textsc{pfv}-whet-\textsc{pst}:\textsc{tr}-\textsc{1sg} \\
 \glt `I just whetted a/the axe.' (elicited)
 \end{xlist}
 \end{exe}

The use of \forme{iɕqʰa} as a topic marker with nouns (as in \ref{ex:iCqha.kAtWm} above) probably took place by reanalysis of the adverb in headless or postnominal relatives, or in complment clauses as above, then generalized to all noun phrases even those without subordinate clause.

 \subsubsection{Adversative topic} \label{sec:adversative.topic}
There are two adversative topic markers in Japhug \forme{ʁo} and \forme{ndɤre}. The former is similar in meaning to Mandarin \ch{倒}{dào}{instead, on the other hand}, and occurs in contexts with a strong adversative meaning `however, but, on the other hand' as in (\ref{ex:Ro.pWtu}).

\begin{exe}
\ex \label{ex:Ro.pWtu}
\gll jinde ku-nɯ-tu ɕi kɯma mɤ-xsi ma kɯɕɯŋgɯ ʁo pɯ-tu, \\
nowadays \textsc{dubit}-\textsc{auto}-exist \textsc{qu} \textsc{sfp} \textsc{neg}-\textsc{genr}:know \textsc{lnk} in.former.times \textsc{top}.\textsc{advers} \textsc{ipfv}.\textsc{pst}-exist \\
\glt `It is not clear whether it is still to be found nowadays, but it did exist in former times.' (23-scuz, 30)
\end{exe}

The marker \forme{ʁo} also occurs in two constructions meaning `of course'. First, it is found in the `X \forme{ʁo} X' construction meaning `of course (it is)  X', as in (\ref{ex:nAZo.Ro.nAZo}), the answer to the question in (\ref{ex:nABJu.YWCara.Ci}) which presents two alternatives.

\begin{exe}
\ex
\begin{xlist}
\ex  \label{ex:nABJu.YWCara.Ci}
\gll `a-tɤɕime, nɤ-βɟu ɲɯ-ɕar-a ɕi, aʑo tu-ozgrɯ-a' nɯra to-ti, `ma nɤ-pi ɣɯ aʑo tɤ-azgrɯ-a ɕti' to-ti  \\
\textsc{1sg}.\textsc{poss}-lady \textsc{2sg}.\textsc{poss}-mat \textsc{ipfv}-search-\textsc{1sg} \textsc{qu} \textsc{1sg} \textsc{ipfv}-bow-\textsc{1sg} \textsc{dem}:\textsc{pl} \textsc{ifr}-say \textsc{lnk} \textsc{1sg}.\textsc{poss}-elder.sibling \textsc{gen} \textsc{1sg} \textsc{pfv}-bow-\textsc{1sg} be.\textsc{affirm}:\textsc{fact} \textsc{ifr}-say \\
\glt `He said `My lady, should I look for a cushion for you, or should I bow (for you to sit on my back)', and he said  `Because I bow for your elder sister (to sit).'
\ex  \label{ex:nAZo.Ro.nAZo}
\gll  `nɤʑo ʁo nɤʑo ma, a-βɟu ɲɯ-tɯ-ɕar kɯ-ɤtsɯtsu me' to-ti.   \\
\textsc{2sg} \textsc{top}.\textsc{advers} \textsc{2sg} \textsc{lnk} \textsc{1sg}.\textsc{poss}-mat \textsc{ipfv}-2-search \textsc{inf}.\textsc{stat}-have.time  not.exist:\textsc{fact} \textsc{ifr}-say \\
\glt  `Of course (I will sit on) you, there is no time to look for a mat for me.' (2014-kWlAG, 195)
\end{xlist}
\end{exe}

Second, \forme{ʁo} is commonly used with the adverb \japhug{lɯski}{of course}, as in (\ref{ex:Ro.lWski}), not necessarily with any adversative meaning.

\begin{exe}
\ex \label{ex:Ro.lWski}
\gll  pɣɤɲaʁ kɤ-ti ci tu tɕe, nɯnɯ ʁo lɯski li nɯ pɣa ŋu \\
pheasant \textsc{nmlz}:P-say \textsc{indef} exist:\textsc{fact} \textsc{lnk} \textsc{dem} \textsc{top}.\textsc{advers} of.course again \textsc{dem} bird be:\textsc{fact} \\
\glt `There is a bird called \forme{pɣɤɲaʁ} (\textit{Pucrasia macrolopha}), this one, of course (since its name contains \japhug{pɣa}{bird}, § \ref{sec:subject.verb.compounds}, Table \ref{tab:subj.v.compounds}) is also a bird (like those previously discussed).' (23-pGAYaR, 2)
\end{exe}

The marker \forme{ndɤre} presents a milder adversative meaning `as far as X is concerned, unlike some other (people)' as in (\ref{ex:aZo.ndAre.rgaa}).  

\begin{exe}
\ex \label{ex:aZo.ndAre.rgaa}
\gll tsuku kɯ-rga tu, tsuku mɤ-kɯ-rga tu. aʑo ndɤre rga-a. \\
some \textsc{nmlz}:S/A-like exist:\textsc{fact} some \textsc{neg}-\textsc{nmlz}:S/A-like exist:\textsc{fact} \textsc{1sg} \textsc{top.advers} like:\textsc{fact}-\textsc{1sg} \\
\glt `Some like it, some don't; as far as I am concerned, I like it.' (07-tCGom2, 8)
\end{exe}

In (\ref{ex:nW.ndAre.wuma}), the use of \forme{ndɤre} suggests the meaning `as opposed to other possible missions'.

\begin{exe}
\ex \label{ex:nW.ndAre.wuma}
\gll a a-pa, nɯ ndɤre wuma ʑo ɴqa, sɤɣʑɯr. \\
\textsc{interj} \textsc{1sg}.\textsc{poss}-father \textsc{dem} \textsc{top.advers} really \textsc{emph} be.difficult:\textsc{fact} be.dangerous:\textsc{fact} \\
\glt `Ah father, this (mission on which you send me) is very difficult and dangerous indeed.' (28-smAnmi, 72)
\end{exe}

In (\ref{ex:jWGmWr.ndAre}), \forme{ndɤre} has a clear adversative meaning `this evening, on the other hand' (as opposed to the previous evenings).

\begin{exe}
\ex \label{ex:jWGmWr.ndAre}
\gll jɯfɕɯr tɯrmɯ tɕe nɤ-pi tɯlɤt nɯ ɯ-taʁ ko-ɴqoʁ-a ri mɯ-tɤ́-wɣ-tsɯm-a tɕe,
jɯɣmɯr ndɤre nɤʑo tu-kɯ-tsɯm-a ra ma tɕe kutɕu aʑo-sti ma maŋe-a tɕe, \\
 yesterday dusk \textsc{lnk} \textsc{2sg}.\textsc{poss}-elder.sibling  second.sibling \textsc{dem} \textsc{3sg}.\textsc{poss}-on \textsc{ifr}-hang-\textsc{1sg} \textsc{lnk} \textsc{neg}-\textsc{pfv:up}-\textsc{inv}-take.away-\textsc{1sg} \textsc{lnk} this.evening \textsc{top.advers}  \textsc{2sg} \textsc{ipfv}:\textsc{up}-2$\rightarrow$1-take.away-\textsc{1sg} have.to:\textsc{fact} \textsc{lnk} \textsc{lnk} here \textsc{1sg}-alone apart.from  not.exist:\textsc{sens}-\textsc{1sg} \textsc{lnk}  \\
\glt  `Yesterday at dusk I clung onto your second eldest sister but she did not take me away, this evening take me away, I am all alone here.' (07-deluge, 56-57)
\end{exe}

The phrase \forme{nɯ sɤznɤ} `even, rather than that etc' (§ \ref{sec:comparative}) is also used as an adversative topic marker similar to \forme{ʁo} (see in particular example \ref{ex:nW.sAznA.YWwxti}).

\subsubsection{The demonstrative \forme{nɯ} as a topic marker} \label{sec:nW.topic}
The postnominal determiner \forme{nɯ} and its reduplicated form \forme{nɯnɯ} is one of the most common words in Japhug, and has a considerable number of functions. It is used as a demonstrative (\ref{sec:demonstrative.determiners}), contributes to expressing definiteness (\ref{sec:definiteness}) and could be argued to be a subordinator (an analysis not adopted in the present work, see § XXX).

In addition, it is commonly used to mark topic: left-dislocated noun phrases generally (though not compulsorily) take this determiner. For instance, in texts presenting animals or plants, their name on first occurrence is left dislocated and followed by the determiner \forme{nɯ}, as in (\ref{ex:qawWz.nW}).

\begin{exe}
\ex \label{ex:qawWz.nW}
\gll  qawɯz nɯ, (qawɯz nɯ pɯ-tɯ-mto-t, ɣe?)  qawɯz nɯnɯ, nɤki, kɯɕɯŋgɯ tɕe, \\
Edelweiss \textsc{dem} Edelweiss \textsc{dem} \textsc{pfv}-2-see-\textsc{pst}:\textsc{tr} \textsc{sfp} Edelweiss \textsc{dem} \textsc{filler} before \textsc{lnk} \\
\glt `The edelweiss, (you saw Edelweiss before, right?)... The edelweiss, in former times,' (15-babW, 177)
\end{exe}

It also occurs with personal pronouns, as in (\ref{ex:mWNi.zW}), a sentence where the narrator talks about his personal situation, as opposed to that of his parents who were mentioned in the previous lines.

\begin{exe}
\ex \label{ex:mWNi.zW}
\gll aʑo nɯ, mɯŋi zɯ kɯ-rɤ-βzjoz ɲɯ-ɕe-a pɯ-ŋu. \\
\textsc{1sg} \textsc{dem} pl.n. \textsc{loc} \textsc{nmlz}:S/A-\textsc{antipass}-learn \textsc{ipfv}:\textsc{west}-go-\textsc{1sg} \textsc{pst}.\textsc{ipfv}-be \\
\glt `As for me, I was going to school in Mungi.' (2010-09, 22)
\end{exe}

In its function as a topicalizer, the determiner \forme{nɯ} can follow a noun with postnominal demonstratives, as in (\ref{ex:kWki.nW}). However, due to the difficulty of systematically sorting out the topicalization and demonstrative functions of this marker, I do not attempt to reflect this distinction in the glosses, and use  \textsc{dem} everywhere.

\begin{exe}
\ex \label{ex:kWki.nW}
\gll tɕeri kɯki mɯntoʁ kɯki nɯ pɯpɯŋunɤ, wuma ʑo kɯ-ʑru, kɯ-pe, \\
\textsc{lnk} \textsc{dem}.\textsc{prox} flower \textsc{dem}.\textsc{prox} \textsc{dem} as.far really \textsc{emph} \textsc{nmlz}:S/A-be.strong \textsc{nmlz}:S/A-be.good \\ 
\glt `But concerning this flower, so precious and nice' (150820 meili de meiguihua, 58)
\end{exe}

\subsubsection{The linker \forme{tɕe} as a topic marker} \label{sec:tCe.topic}
The word \forme{tɕe}, which originates from a locative postposition (§\ref{sec:locative.j}), is mainly used in Japhug as a linker (§ XXX) and as a postposition (§ \ref{sec:tCe.postposition}), one of the most common words in the corpus.

In addition, it can serve as a topic marker, following left-dislocated noun or postpositional phrases (\ref{ex:tsuku.kW.tCe}).

\begin{exe}
\ex \label{ex:tsuku.kW.tCe}
\gll tsuku kɯ tɕe lɤpɯɣ ra mbɯsɯt chɯ-lɤt-nɯ tɕe nɯra ɲɯ-rku-nɯ ɲɯ-ŋu \\
some \textsc{erg} \textsc{lnk} radish \textsc{pl} grating \textsc{ipfv}-throw-\textsc{pl} \textsc{lnk} \textsc{dem}.\textsc{pl} \textsc{ipfv}-put.in-\textsc{pl} \textsc{sens}-be \\
\glt `Some people, they grate radish and use it as filling (for the sausage).' (05-paR, 77)
\end{exe}

 \subsection{Focus markers} \label{sec:focus}
   \subsubsection{Unexpected focus} \label{sec:unexpected}
  The unexpected/high degree marker \forme{rcanɯ} or \forme{rca}, which was grammaticalized from the  secutive relator noun \japhug{ɯ-rca}{following} (§ \ref{sec:secutive}). It indicates that the phrase or clause preceding it is topical, and the situation or action described by the predicate that follows is unexpected (\ref{ex:nAZo.rcanW}), intensifies to a noticeable (and not foreseeable) extent (\ref{ex:tokAnWmqajndZic.tCe.rcanW}) or occurs with a remarkably high degree or intensity, with  (\ref{ex:mbro.rcanW}) or without (\ref{ex:apWme.rcanW}) surprise.

\begin{exe}
\ex \label{ex:nAZo.rcanW}
 \gll  wo nɤʑo rcanɯ tɕʰi ɲɯ-tɯ-nɤme ŋu ma,  aʑo tɯ-mɯ kɯ pɯ-kɯ-sɯ-χtɕi-a, tɤndʐo nɯ! \\
 \textsc{interj} \textsc{2sg} \textsc{foc}:\textsc{unexp} what \textsc{sens}-2-do[III] be:\textsc{fact} \textsc{lnk} \textsc{1sg} \textsc{indef}.\textsc{poss}-sky \textsc{erg} \textsc{pfv}-2\fl{}1-\textsc{caus}-wash-\textsc{1sg} cold \textsc{sfp} \\
\glt `You, what are you doing, you caused me to be drenched by the rain.' (kWlAG2014, 157) \\
\end{exe}

\begin{exe}
\ex \label{ex:tokAnWmqajndZic.tCe.rcanW}
 \gll to-k-ɤnɯmqaj-ndʑi-ci tɕe rcanɯ, ʑɯrɯʑɤri tɕe ko-k-ɤndɯndo-ndʑi-ci, \\
 \textsc{ifr}-\textsc{evd}-\textsc{recip}:scold-\textsc{du-evd} \textsc{lnk}  \textsc{foc}:\textsc{unexp} progressively \textsc{lnk}   \textsc{ifr}-\textsc{evd}-\textsc{recip}:take-\textsc{du-evd} \\
 \glt `They scolded each other and progressively started to fight, ' (lWlu2002, 52)
\end{exe}

 \begin{exe}
\ex \label{ex:mbro.rcanW}
 \gll mbro rcanɯ ɯ-xɕɤt kɯ-tɯ\redp{}tu ʑo nɯ-ntsʰɤr ɲɯ-nu, \\
 horse \textsc{foc}:\textsc{unexp} \textsc{3sg}.\textsc{poss}-strength \textsc{nmlz}:S/A-\textsc{emph}\redp{}exist \textsc{emph} \textsc{pfv}-neigh \textsc{sens}-be \\ 
 \glt `The horse neighed with all his strength.' (qachGa2003, 158)
\end{exe}

The marker \forme{rcanɯ} is particularly common in the degree construction with a \forme{tɯ-} degree nominal (§ XXX), as in (\ref{ex:apWme.rcanW}). In this particular construction,  \forme{rcanɯ} does not necessarily express unexpectedness.

\begin{exe}
\ex \label{ex:apWme.rcanW}
 \gll  tɕe nɯnɯ lɯlu a-pɯ-me rcanɯ, βʑɯ ɯ-tɯ-ŋɤn saχaʁ. \\
 \textsc{lnk} \textsc{dem} cat \textsc{irr}-\textsc{ipfv}-not.exist \textsc{foc}:\textsc{unexp} mouse
 \textsc{3sg}.\textsc{poss}-\textsc{nmlz}:\textsc{degree}-be.evil be.extremely:\textsc{fact} \\ 
 \glt `If there are no cats, the mice are extremely fierce (cause a lot of damages).' (21-lWlu, 32) 
\end{exe}

 \subsubsection{Additive and scalar focus marker \forme{kɯnɤ} } \label{sec:kWnA}
The additive and scalar focus marker \japhug{kɯnɤ}{also, even} follows the constituent over which it has scope, which can be noun phrases, postpositional phrases but also subordinate clauses (these are treated in § XXX). As with other function words with the syllable \forme{nɤ} as last element (§ XXX), the stress is on the first syllable (\forme{kɯ́nɤ}) and the vowel on the second syllable is often elited (a pronunciation \forme{kɯn} is often heard). 

The marker \forme{kɯnɤ} expresses both additive focus, as in (\ref{ex:aZo.kWNA.staRlupa}), and scalar focus, as in (\ref{ex:WNgWz.kWnA.tunAndWtnW}) in affirmative sentences. It is also compatible with negative verb forms, as in (\ref{ex:tWrdoR.kWnA}), expressing the meaning `not even' (see also \japhug{cinɤ}{(not) even one} in § \ref{sec:cinA}).

\begin{exe}
\ex \label{ex:aZo.kWNA.staRlupa}
\gll aʑo kɯnɤ staʁlupa ŋu-a tɕe \\
\textsc{1sg} also born.in.the.tiger.year be:\textsc{fact}-\textsc{1sg} \textsc{lnk} \\
\glt `Me too (like you), I am of the Tiger year.' (2011-05-nyima, 168)
\end{exe}

\begin{exe}
\ex \label{ex:WNgWz.kWnA.tunAndWtnW}
\gll ʑara ʑo ɯ-ŋgɯz kɯnɤ tu-nɤndɯt-nɯ tɕe nɯ kɯ-βʁa ɣɤʑu, kɯ-nŋo ɣɤʑu qʰe, \\
\textsc{3pl} \textsc{emph} \textsc{3sg}.\textsc{poss}-among:\textsc{loc} also \textsc{ipfv}-fight-\textsc{pl} \textsc{lnk} \textsc{dem} \textsc{nmlz}:S/A-win \textsc{sens}:exist \textsc{nmlz}:S/A-lose  \textsc{sens}:exist \textsc{lnk} \\
\glt `Even among themselves, they fight, and there are winners and losers.' (20-sWNgi, 62-63)
\end{exe}
 
\begin{exe}
\ex \label{ex:tWrdoR.kWnA}
\gll tɯ-sŋi mɯntoʁ tɯ-rdoʁ kɯnɤ ci ci tɕe mɯ́j-stʰɯt \\
one-day flower one-piece also once once \textsc{lnk} \textsc{neg}:\textsc{sens}-finish \\
\glt `Sometimes one cannot finish even one pattern (on the belt) in one day.' (2011-06-thaXtsa, 47)
\end{exe}

As an additive focus marker, \forme{kɯnɤ} can be repeated on all the nouns designating the members of a group sharing a particular property, in the construction $X$ \forme{kɯnɤ}, $Y$ \forme{kɯnɤ}  `both $X$ and $Y$', as in (\ref{ex:Dpalcan.kWnA}).

\begin{exe}
\ex \label{ex:Dpalcan.kWnA}
 \gll a-pɯ-ŋu tɕe, aʑo kɯnɤ taʁrdo rɟitpa a-pɯ-ŋu-a, χpɤltɕin kɯnɤ taʁrdo rɟitpa a-pɯ-ŋu, ... nɯ tɕi-rɟit nɯni tɕe taʁrdo rɟitpa ma nɯ ma kɯmaʁ rɟitpa nɯ kɤ-rtsi me.  \\
 \textsc{irr}-\textsc{ipfv}-be \textsc{lnk} \textsc{1sg} also pl.n. lineage  \textsc{irr}-\textsc{ipfv}-be-\textsc{1sg}  p.n. also pl.n. lineage  \textsc{irr}-\textsc{ipfv}-be { } \textsc{dem} \textsc{1du}.\textsc{poss}-offspring \textsc{dem}:\textsc{du} \textsc{lnk} pl.n. lineage \textsc{lnk} \textsc{dem} apart.from other lineage \textsc{dem} \textsc{nmlz}:O-count not.exist:\textsc{fact} \\
 \glt `For instance suppose that both Dpalcan and I were from Taqrdo lineage, then our two children would only count as members of the Taqrdo lineage and no other lineage.' (140426 rJitpa, 13-15)
\end{exe}

The scope of  \forme{kɯnɤ} is generally exclusively on the constituent that it immediately follows, but there are cases where the scope is more extensive. In (\ref{ex:aZo.kWnA.akAsWso}), \forme{kɯnɤ} occurs between the pronoun \forme{aʑo} and the following participial verb form, which bears a \textsc{1sg} possessive prefix \forme{a-} coreferent with that pronoun (see also \ref{ex:aZWG.kWnA} below). The semantic scope of \forme{kɯnɤ} here is on the whole relative \forme{aʑo a-kɤ-sɯso} `(the things) that I want' rather than exclusively on the pronoun \forme{aʑo}.

\begin{exe}
\ex \label{ex:aZo.kWnA.akAsWso}
 \gll aʑo kɯnɤ a-kɤ-sɯso nɯ tɤ-stu-nɯ ra \\
 \textsc{1sg} also \textsc{1sg}.\textsc{poss}-\textsc{nmlz}:O-think \textsc{dem} \textsc{imp}-do.like-\textsc{pl} have.to:\textsc{fact} \\
 \glt `(I will do as you say, but) do also the things I want.' (2003kAndzwsqhaj2, 47)
\end{exe}

The focus marker \forme{kɯnɤ} is found with nouns or pronouns in core argument function, including S (\ref{ex:kWnA.nArca}), O (\ref{ex:nWXpWm.kWnA}), and semi-objects (\ref{ex:kWnA.mAsna}).  Examples with transitive subjects are presented below (\ref{ex:nWra.kWnA} and \ref{ex:Wzda.ra.kWnA}).

 \begin{exe}
\ex \label{ex:kWnA.nArca}
\gll aʑo kɯnɤ nɤ-rca ɣi-a ɕti  \\
\textsc{1sg} also \textsc{2sg}.\textsc{poss}-following come:\textsc{fact}-\textsc{1sg} be.\textsc{affirm}:\textsc{fact} \\
\glt `I am coming with you too.' (2011-05-nyima, 171)
 \end{exe}
 
   \begin{exe}
\ex \label{ex:nWXpWm.kWnA}
\gll    ma nɯ-χpɯm kɯnɤ kʰro mɤ-kɯ-fkaβ kɯ-fse ku-rɤʑi-nɯ  \\
lnk 3pl.poss-knee also much \textsc{neg}-\textsc{nmlz}:S/A-cover \textsc{nmlz}:S/A-be.like \textsc{ipfv}-stay-\textsc{pl} \\
\glt `(Gents) would (wear trousers that did) not cover much even their knees.'  (30-rkAsnom, 5) 
  \end{exe}
  
  \begin{exe}
 \ex \label{ex:kWnA.mAsna}
 \gll   ɯ-ru nɯra laʁdɯn ɯ-jɯ kɯnɤ mɤ-sna, ma mɤ-ngɯt. \\
 \textsc{3sg}.\textsc{poss}-trunk \textsc{dem}:\textsc{pl} tool \textsc{3sg}.\textsc{poss}-handle also \textsc{neg}-be.worth \textsc{lnk}  \textsc{neg}-be.strong:\textsc{fact} \\
 \glt `(The wood from) its trunk is not even good (enough to be used to make) tool handles, as it is not strong.'  (17-xCAj, 79)
  \end{exe}

It also occurs with all types of oblique arguments and adjuncts, including genitive (\ref{ex:aZWG.kWnA}), dative (\forme{ɯ-ɕki} \ref{ex:nWCki.kWnA}),  locational adjuncts in \forme{tɕu} (\ref{ex:kutCu.kWnA}) or \forme{ri} (\ref{ex:ri.kWnA}), temporal adjuncts (\ref{ex:ftCAXcAl.kWnA}) or adjuncts expressing manner or cause (\ref{ex:nWtCu.kWnA2}).  
  
   \begin{exe}
\ex \label{ex:aZWG.kWnA}
\gll aʑɯɣ kɯnɤ a-mpʰrɯmɯ a-pɯ-tɯ-sɯ-re ɯ-tɯ́-cʰa \\
\textsc{1sg}:\textsc{gen} also \textsc{1sg}.\textsc{poss}-divination \textsc{irr}-\textsc{pfv}-2-\textsc{caus}-look[III] \textsc{qu}-2-can:\textsc{fact} \\
\glt `Can you ask (the monk) to make a divination for me too?' (The divination, 31)
\end{exe}  
  
   \begin{exe}
\ex \label{ex:nWCki.kWnA}
\gll  tɯ-pi ɣɯ ɯ-nmaʁ ra nɯ-ɕki kɯnɤ `a-pi' tu-kɯ-ti ɕti ma nɯ ma kupa kɯ-fse ʑaka ɯ-rmi me. \\
\textsc{genr}.\textsc{poss}-elder.sibling \textsc{gen} \textsc{3sg}.\textsc{poss}-husband \textsc{pl} \textsc{3pl}.\textsc{poss}-\textsc{dat} also \textsc{1sg}.\textsc{poss}-elder.sibling \textsc{ipfv}-\textsc{genr}-say be.\textsc{affirm}:\textsc{fact} \textsc{lnk} \textsc{dem} apart.from Chinese \textsc{nmlz}:S/A-be.like each \textsc{3sg}.\textsc{poss}-name not.exist:\textsc{fact} \\
\glt  `One calls one's sister's husband (and others from his family) `my elder brother', there are no other special terms as in Chinese.' (140425 kWmdza05)
\end{exe}


  \begin{exe}
\ex \label{ex:kutCu.kWnA}
\gll  kutɕu kɯnɤ nɯ ɲɯ-fse, jɯfɕɯndʐi ra kɯ-xtɕɯ\redp{}xtɕi tɤ-ɣɤndʐo kɯ-fse ri, ɕɤxɕo tɕe kɯ-xtɕɯ\redp{}xtɕi ɲɯ-ʑi kɯ-fse \\
here also \textsc{dem} \textsc{sens}-be.like a.few.days.ago \textsc{nmlz}:S/A-\textsc{emph}\redp{}be.small \textsc{pfv}-be.cold \textsc{nmlz}:S/A-be.like \textsc{lnk} the.last.days \textsc{lnk} \textsc{nmlz}:S/A-\textsc{emph}\redp{}be.small \textsc{sens}-subside \textsc{nmlz}:S/A-be.like \\
\glt `It is like that here too, a few days ago the weather became a little cold, but the last days it has eased a bit.' (conversation, 141027)
  \end{exe}
  
    \begin{exe}
\ex \label{ex:ri.kWnA}
\gll   maldzɯ nɯ, nɯ ɯ-tʰɤcu tsa ri kɯnɤ ɣɤʑu. qarɣɤpɤt ɯ-rca ri kɯnɤ tu-ɬoʁ ɲɯ-ŋu. \\
plant.name \textsc{dem} \textsc{dem} \textsc{3sg}.\textsc{poss}-downstream a.little \textsc{loc} also exist:\textsc{sens} plant.name \textsc{3sg}.\textsc{poss}-among \textsc{loc} also \textsc{ipfv}-come.out \textsc{sens}-be \\
\glt `The \forme{maldzɯ} plant, it is also found in places of slightly lower altitude, but grows also in the same places as  \forme{qarɣɤpɤt} plants.' (18-qromJoR, 81-82)
    \end{exe}
    
\begin{exe}
\ex \label{ex:ftCAXcAl.kWnA}
\gll   kukutɕu ftɕɤχcɤl kɯnɤ <baonuanyi> tu-tɯ-ŋge pɯ-ɕti. \\
  here mid.summer also warm.clothes \textsc{ipfv}-2-wear[III] \textsc{pst}.\textsc{ipfv}-be.\textsc{affirm} \\
  \glt `Here you were wearing warm clothes even in mid summer.' (conversation, 141017)
    \end{exe}
    
    \begin{exe}
\ex \label{ex:nWtCu.kWnA2}
\gll    tɕe nɯtɕu kɯnɤ ɯ-jaʁ ɯ-ntsi tɤɲi pjɯ-sɤtse, ɯ-jaʁ ɯ-ntsi kɯ tsʰitsuku ɲɯ-z-nɤme qʰe, \\
\textsc{lnk} \textsc{dem}:\textsc{loc} also \textsc{3sg}.\textsc{poss}-hand \textsc{3sg}.\textsc{poss}-one.of.a.pair erg various.things \textsc{ipfv}-\textsc{caus}-do[III] \textsc{lnk}  \\
\glt `Even like that (despite the pain in her legs), she props herself with a cane using one hand, and does all kinds of things with her other hand.' (14-tApitaRi, 52)
\end{exe}

Although \japhug{kɯnɤ}{also, even} can be combined with most postpositions and relator nouns as shown by the examples above, it is however incompatible with the ergative \forme{kɯ}. For instance, in  (\ref{ex:nWra.kWnA}), although the demonstrative pronoun \forme{nɯra} `they, those' in the second clause is the subject of the transitive verb \japhug{ndza}{eat}, it does not take the ergative \forme{kɯ} as would be expected (§ \ref{sec:A.kW}). The same applies to \forme{ɯ-zda ra} `his companions', subject of the transitive verb \forme{na-nɯ-ɕar-nɯ} `they looked for themselves' in (\ref{ex:Wzda.ra.kWnA}), 

  \begin{exe}
\ex \label{ex:nWra.kWnA}
\gll ɯ-pɯ nɯra li ju-ɣi-nɯ qʰe, nɯra kɯnɤ ɣɯ-tu-ndza-nɯ. \\
\textsc{3sg}.\textsc{poss}-young \textsc{dem}:\textsc{pl} again \textsc{ipfv}-come-\textsc{pl} \textsc{lnk} \textsc{dem}:\textsc{pl} also \textsc{cisloc}-\textsc{ipfv}-eat-\textsc{pl} \\
\glt `Its youngs also come and they too eat it.' (20-sWNgi, 59-60)
  \end{exe}
  
    \begin{exe}
\ex \label{ex:Wzda.ra.kWnA}
\gll   ɯ-zda ra kɯnɤ nɯ-rʑaβ tɯka na-nɯ-ɕar-nɯ ɲɯ-ŋu \\
\textsc{3sg}.\textsc{poss}-companion \textsc{pl} also \textsc{3sg}.\textsc{poss}-wife each \textsc{pfv}:3\fl{}3'-\textsc{auto}-search \textsc{sens}-be \\
\glt `His companions also took each a wife for himself (among the women of the island).' (2005Norbzang, 44)
    \end{exe}
    
The combinations $\dagger$\forme{kɯ kɯnɤ} or $\dagger$\forme{kɯnɤ kɯ} are unattested, and not accepted by native speakers. The contrast between absolutive and ergative noun phrases is therefore neutralized in additive or scalar focus with \forme{kɯnɤ}. Note that other focus markers, such as \forme{ri} and \forme{tɕi} (see \ref{ex:tCi.ndze} in § \ref{sec:ri.additive}) differ from \forme{kɯnɤ} in this regard.

Four distinct facts converge to suggest that the first syllable of \forme{kɯnɤ} is historically related to the ergative postposition \forme{kɯ}: (i) the incompatibility of co-occurrence of \forme{kɯnɤ} and \forme{kɯ}; (ii) the stress on the first syllable in \forme{kɯ́nɤ}; (iii) the similar \forme{-nɤ} element in the other scalar focus marker \japhug{cinɤ}{(not) even one} (§ \ref{sec:cinA}) (iv) the existence of the linker \forme{nɤ}, possibly of Tibetan origin (§ XXX). A detailed examination of this topic is however impossible on the basis Japhug-internal evidence, and will require extensive syntactic comparison between Gyalrong languages.

The adverb \japhug{tɤmtɯkɯnɤ}{specially, on purpose} (§ XXX) appears to be a lexicalized combination of the noun \japhug{tɤ-mtɯ}{knot} and the focus marker \forme{kɯnɤ}.

 \subsubsection{Correlative additive focus markers \forme{ri} and \forme{tɕi}} \label{sec:ri.additive} 
 The additive focus markers \forme{ri} and \forme{tɕi}  are used in enumerations, repeated after each noun referring to  members of a group, to focus on the fact that their referents share a common property (or properties that are semantically close enough), as in (\ref{ex:ri.kWsthWci.WWmpCar}) and (\ref{ex:tCi.tulhoR.cha}) (see additional examples in \citealt[313-314]{jacques14linking}).
 
 \begin{exe}
\ex \label{ex:ri.kWsthWci.WWmpCar}
 \gll  a-rʑaβ ri kɯstʰɯci ɲɯ-mpɕɤr, a-mbro ri kɯstʰɯci ɲɯ-ʑru, a-pɣɤtɕɯ ri kɯstʰɯci ɲɯ-mpɕɤr tɕe, \\
 \textsc{1sg}.\textsc{poss}-wife also so.much \textsc{sens}-be.beautiful  \textsc{1sg}.\textsc{poss}-horse also so.much \textsc{sens}-be.strong  \textsc{1sg}.\textsc{poss}-bird also so.much \textsc{sens}-be.beautiful \textsc{lnk} \\
 \glt `My wife is so beautiful, my horse so strong, my bird so beautiful.' (2003qachga, 116)
 \end{exe}
 
  \begin{exe}
\ex \label{ex:tCi.tulhoR.cha}
 \gll  ɴqiaβ tɕi tu-ɬoʁ cʰa, zrɯ tɕi tu-ɬoʁ cʰa, \\
 dark.side.of.the.mountain also \textsc{ipfv}-come.out can:\textsc{fact}   sunny.side.of.the.mountain also \textsc{ipfv}-come.out can:\textsc{fact}  \\
 \glt `It can grow in both the dark and the sunny sides of the mountains.' (17-thowum, 14)
  \end{exe}
  
The correlative focus markers \forme{ri} and \forme{tɕi} can occur after any noun phrase or postpositional phrase, including with the ergative  \forme{kɯ} as shown by (\ref{ex:tCi.ndze}), unlike the marker \japhug{kɯnɤ}{even, also} (see examples \ref{ex:nWra.kWnA} and \ref{ex:Wzda.ra.kWnA}, § \ref{sec:kWnA}).
  
  \begin{exe}
\ex \label{ex:tCi.ndze}
 \gll paʁ kɯ tɕi ndze, nɯŋa kɯ tɕi ndze, jla kɯ tɕi ndze.   \\
 pig \textsc{erg} also eat[III]:\textsc{fact}  cow \textsc{erg} also eat[III]:\textsc{fact}  hybrid.yak \textsc{erg} also eat[III]:\textsc{fact}  \\
 \glt `Pigs eat it, cows eat it, hybrid yaks eat it.' (18-NGolo, 171)
  \end{exe}

The focus markers \forme{ri} and \forme{tɕi} can have scope on only part of the noun/propositional phrase, and even on the relator nouns as in (\ref{ex:WNgW.tCi}).

   \begin{exe}
\ex \label{ex:WNgW.tCi}
 \gll   sɤtɕʰa ɯ-ŋgɯ tɕi ɣɤʑu, sɤtɕʰa ɯ-taʁ tɕi ʑo ɣɤʑu \\
 ground \textsc{3sg}.\textsc{poss}-inside also exist:\textsc{sens}  ground \textsc{3sg}.\textsc{poss}-inside also \textsc{emph} exist:\textsc{sens} \\
 \glt `It is found both inside the ground, and on the ground.' (25-GdAso, 17)
    \end{exe}
    
Alternatively, it is possible to enumerate distinct related properties of the same referent using \forme{ri} (this usage is not found with \forme{tɕi}), but that marker still follows the noun phrase (correlative \forme{ri} can follow verbs, but only in a specific construction, see \ref{ex:ri.kWmWm.ri} below). In this case the referent cannot be elided, and must be repeated in both clauses, at least as a third person pronoun \forme{ɯʑo} as in (\ref{ex:WlWz.ri.pjAxtCi}). 

  \begin{exe}
\ex \label{ex:WlWz.ri.pjArZi}
 \gll pʰaʁrgot nɯnɯ ɯʑo ri pjɤ-rʑi, ɯʑo ri pjɤ-tsʰu tɕe \\
 boar \textsc{dem} \textsc{3sg} also \textsc{ifr}.\textsc{ipfv}-be.heavy \textsc{3sg} also \textsc{ifr}.\textsc{ipfv}-be.fat \textsc{lnk} \\ 
\glt  `The boar, it was heavy and fat.' (140428 yonggan de xiaocaifeng-zh, 244)
 \end{exe}

A variant of this construction is found with internally-headed relative clauses in apposition, taking the third person pronoun \forme{ɯʑo} as head, as in (\ref{ex:WZo.ri.kWwxti}).

\begin{exe}
\ex \label{ex:WZo.ri.kWwxti}
\gll  [ɯʑo ri kɯ-wxti], [ɯʑo ri kɯ-sɤjlɯ\redp{}jloʁ] ci pjɤ-ŋu. \\
\textsc{3sg} also \textsc{nmlz}:S/A-be.big \textsc{3sg} also \textsc{nmlz}:S/A-\textsc{emph}\redp{}be.big \textsc{indef} \textsc{ifr}.\textsc{ipfv}-be \\
\glt `(The toad) was a big and disgusting (creature).' (150818 muzhi guniang, 86)
\end{exe}

 
The correlative construction can involve the possessor of an IPN, as in (\ref{ex:WlWz.ri.pjAxtCi}), where in the first clause the referent `the girl' is possessor of the intransitive subject (literally `her age was small', § XXX) and in second it corresponds to the intransitive subject, realized as a third person pronoun \forme{ɯʑo} `she'.

  \begin{exe}
\ex \label{ex:WlWz.ri.pjAxtCi}
 \gll tɕʰeme nɯ ɯ-lɯz ri pjɤ-xtɕi, ɯʑo ri pjɤ-mpɕɤr,  \\
 girl \textsc{dem} \textsc{3sg}.\textsc{poss}-age also \textsc{ifr}.\textsc{ipfv}-be.small \textsc{3sg} also \textsc{ifr}.\textsc{ipfv}-be.beautiful \\
\glt `The girl was young and beautiful.' (150909 hua pi-zh, 10)
 \end{exe}
 
 More complex correlations, involving different subjects and predicates related to another referent, are also possible as shown by example (\ref{ex:lWlu.kW}), where \forme{ri} occurs after the intransitive subject \japhug{tɯ-ci}{water}, after the transitive subject \japhug{lɯlu}{cat} with the ergative and after the finite verb \japhug{tu-ɕe}{it goes up} (on which see below and refer to § XXX).
 
 \begin{exe}
\ex   \label{ex:lWlu.kW}
\gll <yancong> ku-kɯ-rɤloʁ tɕe ɯ-taʁ tɯ-ci ri mɯ́j-ɣi lɯlu kɯ ri mɯ-ɲɯ́-wɣ-ɕaβ qapri tu-ɕe ri mɯ́j-cʰa tɕe \\
 chimney \textsc{ipfv}-\textsc{genr}:S/P-make.a.nest \textsc{lnk} \textsc{3sg}.\textsc{poss}-on \textsc{indef}.\textsc{poss}-water also \textsc{neg}:\textsc{sens}-come cat \textsc{erg} also \textsc{neg}-\textsc{ipfv}-\textsc{inv}-catch snake \textsc{ipfv}:\textsc{up}-go also \textsc{neg}:\textsc{sens}-can \textsc{lnk} \\
 \glt `(The sparrows) make their nest in the chimney, (because) water cannot come up there, the cats cannot catch them, and the snakes cannot go up there.' (22-kumpGatCW, 69)
 \end{exe}
 
 The marker \forme{ri} is homophonous with the locative \forme{ri} (§ \ref{sec:locative}), and in cases with an enumeration of locative adjuncts, there can be ambiguity between the two. In (\ref{ex:Xcha.ri.ci}), \forme{ri} is analyzed as a locative because of the position of the determiner \forme{ci}, and also because it can be replaced with other locative postpositions.
 
 \begin{exe}
\ex \label{ex:Xcha.ri.ci}
\gll   χcʰa ri ci, ɯ-ʁe ri ci ɯ-jme cʰɯ-ɬoʁ ɲɯ-ŋu. \\
right \textsc{loc} one  \textsc{3sg}.\textsc{poss}-left \textsc{loc} one \textsc{3sg}.\textsc{poss}-tail \textsc{ipfv}:\textsc{downstream}-come.out \textsc{sens}-be \\
\glt `It has one tail on the right, and one on the left.' (26-qro, 116)
\end{exe}

The marker \forme{ri} can follow verbs only if combined with an existential verb, a copula or a modal auxiliary verb as main predicate (meaning `both $X$ and $Y$' with positive copulas, and `neither $X$ nor $Y$' with negative ones). In this type of construction, verbs are mostly in non-finite form, as in (\ref{ex:ri.kWmWm.ri}). Examples with finite verbs however do exist; this topic is treated in § XXX. %ɲɯ-ɣɤwu ri kɯ-maʁ, ɲɯ-nɤre ri kɯ-maʁ kɯ-fse ɲɤ-k-ɤβzu-ci  ; tu-rɯɕmi ri mɤ-kɯ-khɯ, chɯ-nɯrɤɣo ri mɤ-kɯ-khɯ ci ɲɤ-k-ɤβzu-ci. ; tu-ndzur ri pjɤ-maʁ, ku-omdzɯ ri pjɤ-maʁ.

 \begin{exe}
\ex \label{ex:ri.kWmWm.ri}
 \gll   nɯ pɯ́-wɣ-ta ri  kɯroz kɯ-mɯm ri maŋe, kɯroz mɤ-kɯ-ɣɤ-mɲɤt ri maŋe qʰe, \\
 \textsc{dem} \textsc{pfv}-\textsc{inv}-put \textsc{lnk} specially \textsc{nmlz}:S/A-be.tasty also not.exist:\textsc{sens} specially \textsc{neg}-\textsc{nmlz}:S/A-\textsc{facil}-be.spoiled also not.exist:\textsc{sens} \textsc{lnk} \\
 \glt `When if one puts (a seal on the bread), there is nothing especially tasty about it, and nothing special concerning the preservation (of the bread).' (160706 thotsi, 27)
  \end{exe}
  

  
 \subsubsection{Scalar focus marker \forme{cinɤ}} \label{sec:cinA} 
 The focus marker \japhug{cinɤ}{(not) even one} exclusively occurs with a negative verb. Like \japhug{kɯnɤ}{also, even}, this marker has stress on the first syllable \forme{cínɤ}, which is obviously related to the numeral \japhug{ci}{one} (§ \ref{sec:one.to.ten}, § \ref{sec:indef.article}).
 
 The marker \forme{cinɤ} has scope over the constituent that immediately precedes it, generally a noun phrase including or consisting of a CN, as in (\ref{ex:tWrdoR.cinA3}), but also object and subject participial relative clauses as in (\ref{ex:zrWG.kAmto.cinA}), (\ref{ex:WrNa.WkWru.cinA}) and (\ref{ex:lukWpGaR.nW.cinA}).
 
 \begin{exe}
\ex \label{ex:tWrdoR.cinA3}
\gll tsuku kɯ qʰe tɯ-rdoʁ cinɤ mɤ-kɯ-mto tu. \\
some erg lnk one-piece even neg-nmlz:S/A-see exist:fact \\
\glt `There are some people who (cannot) even find a single one.' (20-grWBgrWB, 36)
 \end{exe} 

 \begin{exe}
\ex \label{ex:zrWG.kAmto.cinA}
\gll  ma tɕe jinde nɯ zrɯɣ kɤ-mto cinɤ maŋe. \\
\textsc{lnk} \textsc{lnk} nowadays \textsc{dem} louse \textsc{nmlz:P}-see even not.exist:\textsc{sens} \\
\glt `Nowadays there isn't even a single louse to be seen/one cannot even see a single louse.' (21-mdzadi, 77)
\end{exe} 

\begin{exe}
\ex \label{ex:WrNa.WkWru.cinA}
\gll ɯ-rŋa ɯ-kɯ-ru cinɤ ʑo pjɤ-me \\
3sg.poss-face 3sg.poss-nmlz:S/A-look even \textsc{emph} \textsc{ipfv}.\textsc{ifr}-not.exist \\
\glt `Not even one (of the thieves) looked at it/The (thieves) did not even so much as looked at it.' (140426 luozi he qiangdao)
\end{exe}

\begin{exe}
\ex \label{ex:lukWpGaR.nW.cinA}
\gll tɕe ɯ-ɲɯ-kɯ-ɣɤ-rkɯn nɯ ɲɯ-dɤn ma lu-kɯ-pɣaʁ nɯ tɯ-rdoʁ cinɤ ʑo maŋe \\
\textsc{lnk} \textsc{3sg}.\textsc{poss}-\textsc{ipfv}-\textsc{nmlz}:S/A-\textsc{caus}-be.few \textsc{dem} \textsc{sens}-be.many \textsc{lnk} \textsc{ipfv}:\textsc{upstream}-\textsc{nmlz}:S/A-plough \textsc{dem} one-piece even \textsc{emph} not.exist:\textsc{sens} \\
\glt `A lot of people diminish their fields, and not a single of them opens new fields.' (150903 friche, 6)
\end{exe}

In the case of relative clauses before \forme{cinɤ}, there is some ambiguity as to whether the scope of the focus marker is on the head of the relative or on the main verb of the relative clause, hence the two proposed translations above for (\ref{ex:zrWG.kAmto.cinA}) and (\ref{ex:WrNa.WkWru.cinA}).

It is not possible to use \forme{cinɤ} with scope over transitive subjects, followed by the ergative.

The form \forme{cinɤ} also occurs in the expression \forme{ŋu cinɤ maʁ kɯ} `in any case it is not', as in (\ref{ex:Nu.cinA.maR.kW}), literally `It is not even the case that...' ; in this construction, only the first verb \japhug{ŋu}{be} receives person indexation, as shown by (\ref{ex:Nua.cinA.maR.kW}). In addition to \japhug{ŋu}{be}, a few other verbs such as \japhug{fse}{be like} can occur with \forme{ci nɤ maʁ kɯ} `anyway X does not' .

 \begin{exe}
\ex \label{ex:Nu.cinA.maR.kW}
\gll qajdo kɯ tɕʰi mɤ-nɯ-ti ɕti nɤ, a-tɤ-nɯ-ti ma ŋu cinɤ maʁ kɯ, nɯ sɤznɤ kɯ-scɯ-scit rɤʑi-tɕi \\
crow \textsc{erg} what \textsc{neg}-\textsc{auto}-say:\textsc{fact} be.\textsc{affirm}:\textsc{fact} \textsc{lnk} \textsc{irr}-\textsc{pfv}-\textsc{auto}-say \textsc{lnk} be:\textsc{fact} even not.be:\textsc{fact} \textsc{sfp} \textsc{dem} \textsc{comp} \textsc{nmlz}:S/A-\textsc{emph}\redp{}happy stay:\textsc{fact}-\textsc{1du} \\
\glt `What would not a crow say (a crow tells only lies), let it say as it wants, in any case it is not (true), let us rather live (together) happily.' (28-qAjdoskAt, 28)
\end{exe} 

 \begin{exe}
\ex \label{ex:Nua.cinA.maR.kW}
\gll  kɯ-mɯrkɯ ŋu-a cinɤ maʁ kɯ  \\
\textsc{nmlz}:S/A-steal be:\textsc{fact}-\textsc{1sg} even not.be \textsc{sfp} \\
\glt `Anyway it is not me who is the thief.' (elicited)
\end{exe}

\subsubsection{Restrictive focus} \label{sec:restrictive.focus} 
 The most common way to express restrictive focus in Japhug is to combine the exceptive \japhug{ma}{apart from} (and its reduplicated variant \forme{mɯma} § \ref{sec:exceptive}) with a negative predicate. This can be a verb with a negative prefix as in (\ref{ex:XsArZaR}), or a negative existential verb as in (\ref{ex:Wmi.Wntsi.ma.me}).
 
 \begin{exe}
\ex  \label{ex:XsArZaR}
\gll   χsɤ-rʑaʁ ma mɯ-pɯ-tsu-a ɲɤ-sɯso ri χsɯ-xpa pjɤ-tsu tɕe,  \\
three-day apart.from \textsc{neg}-\textsc{pfv}-pass-\textsc{1sg} \textsc{ifr}-think \textsc{lnk} three-year \textsc{ifr}-pass \textsc{lnk} \\
\glt `He thought that he had spent only three days, but three years had passed.' (2011-4-smanmi, 178)
  \end{exe}
  
  \begin{exe}
\ex  \label{ex:Wmi.Wntsi.ma.me}
\gll  rkoŋɟɤl nɯnɯ, ɯ-mi ɯ-ntsi nɯ ma me kʰi.   \\
one.legged.demon \textsc{dem} \textsc{3sg}.\textsc{poss}-leg \textsc{3sg}.\textsc{poss}-one.of.a.pair \textsc{dem} apart.from not.exist:\textsc{fact} \textsc{hearsay} \\
\glt  `It is said that one-legged demons only had one leg.' (140510 rkoNJAl, 4)
  \end{exe}
  
In the case of restrictive focalization on locative or temporal phrases,   the terminative \japhug{mɤɕtʂa}{until} (§ \ref{sec:terminative}) occurs instead of the exceptive.
  
\begin{exe}
\ex  \label{ex:nWnWtCu.mACtsxa.pjAme}
\gll  nɯnɯtɕu mɤɕtʂa mtsʰalu pjɤ-me qʰe \\
\textsc{dem}:\textsc{loc} until nettle  \textsc{ifr.ipfv}-not.exist \textsc{lnk} \\
\glt `It was only there that there was nettle.' =  `There was no nettle until there.' (140520 ye tian'e, 319)
\end{exe}
    
The restrictive focus construction implies the presence of a noun phrase with a numeral or a CN when the restriction bears on the quantity, but restriction can also be qualitative, without quantifier, as in (\ref{ex:karGi.Zo.kWfse.ma.me}).

\begin{exe}
\ex \label{ex:karGi.Zo.kWfse.ma.me}
 \gll   ɯ-mat nɯnɯ na-lɤt ɕɯmɯma nɤ kɯ-ndɯ\redp{}ndɯβ ʑo ma me, karɣi ʑo kɯ-fse ma me  \\
 \textsc{3sg}.\textsc{poss}-fruit \textsc{dem} \textsc{pfv}:3\fl{}3'-throw just \textsc{lnk}  \textsc{nmlz}:S/A-\textsc{emph}\redp{}small \textsc{emph} apart.from not.exist:\textsc{fact} turnip.seed \textsc{emph} \textsc{nmlz}:S/A-be.like apart.from not.exist:\textsc{fact} \\
 \glt  `When the fruit of (xanthoxyllum) has just come out, there is only something very small, only like a turnip seed.'  (07-tCGom, 7)
  \end{exe}
  
The restrictive focus construction can be combined with a scalar focus in \forme{kɯnɤ} (see §  \ref{sec:kWnA}), as in (\ref{ex:ma.kWme.kWnA}). In this example, \forme{kɯnɤ} has scope over the subordinate clause \forme{stɯsti ma kɯ-me}, which is ambiguous between a participial headless relative (§ XXX) `consisting of only a female all alone' and a manner infinitival clause (§ XXX; in this case the gloss of \forme{kɯ-me} would be \textsc{inf}:\textsc{stat}-not.exist) `even (when) there is only a female all alone'.

  \begin{exe}
\ex \label{ex:ma.kWme.kWnA}
\gll  mu ma, stɯsti ma kɯ-me kɯnɤ cʰɯ-rɤŋgɯm ɲɯ-ɕti. \\
female apart.from alone apart.from \textsc{nmlz}:S/A-not.exist also \textsc{ipfv}-lay.eggs \textsc{sens}-be.\textsc{affirm} \\
\glt `Even only a female (hen) alone does lay eggs.' (150819 kumpGa, 11)
\end{exe}
   
A second possibility to express restrictive focus is the use of the adverb \japhug{ʁɟa}{completely, all} (§ XXX) with scope on a  noun phrase rather than the whole clause as in (\ref{ex:RJa.tunWndze}).\footnote{The form \forme{ʁɟa} possibly originates from the first syllable of Tibetan \tibet{གཡའ་མ་}{gja.ma}{stone slab}, through a meaning `bare rock'.}  

\begin{exe}
\ex \label{ex:stAmku.RJa}
\gll alo mbroχpa ra tɕe tɕe nɤki qra cʰo qambrɯ ra ɣɯ nɯ-ɣli nɯnɯ
tɕe nɯ tu-wum-nɯ, tu-sɯɣ-rom-nɯ mbroχpa sɤtɕʰa tɕe stɤmku ʁɟa ɲɯ-ɕti ma si maŋe tɕe tɕe    \\
upstream nomad \textsc{pl} \textsc{lnk} \textsc{lnk} \textsc{filler} female.yak \textsc{comit} male.yak \textsc{pl} \textsc{gen} \textsc{3pl}.\textsc{poss}-dung \textsc{dem} \textsc{lnk} \textsc{dem} \textsc{ipfv}-gather-\textsc{pl} \textsc{ipfv}-\textsc{caus}-be.dry-\textsc{pl} nomad place \textsc{lnk} grassland completely \textsc{sens}-be.\textsc{affirm} \textsc{lnk} tree not.exist:\textsc{sens} \textsc{lnk} \textsc{lnk}  \\
\glt `Upstream, in the nomad areas, they gather and dry yak dung, as in nomad places there is only grassland, there no trees.' (05-tamar, 7-10)
\end{exe}

The adverb \forme{ʁɟa} (here used rather as a noun modifier) is related to the denominal verb \japhug{aʁɟa}{be bald, be bare} (see § XXX on the \forme{a-} derivation), which can be applied to nouns such as \japhug{stɤmku}{grassland} and \japhug{zgo}{mountain}.
 
\begin{exe}
\ex \label{ex:RJa.tunWndze}
 \gll qajɯ ʁɟa tu-nɯ-ndze, ma nɯ ma tɤ-rɤku kɯ-fse ra ndze mɤ-ŋgrɤl. \\
 bug completely \textsc{ipfv}-\textsc{auto}-eat[III] \textsc{lnk} \textsc{dem} apart.from \textsc{indef}.\textsc{poss}-harvest \textsc{nmlz}:S/A-be.like \textsc{pl} eat[III]:\textsc{fact} \textsc{neg}-be.usually.the.case:\textsc{fact} \\ 
\glt `It only eats insects, it does not eat cultivated plants.' (140511 qamtsWrmdzu, 16)
\end{exe}

While in (\ref{ex:RJa.tunWndze})  and (\ref{ex:stAmku.RJa}) it remains ambiguous whether \forme{ʁɟa} forms a syntactic constituent with the previous nouns or the following verb, in (\ref{ex:RJa.kW}) the presence of the ergative makes it clear that \forme{ʁɟa} is not a clausal adverb, and belongs to the postpositional phrase headed by \forme{kɯ}.

\begin{exe}
\ex \label{ex:RJa.kW}
 \gll [tɤ-lu cʰo tɯkrimgo ʁɟa kɯ] cʰɯ-z-ɣɤ-wxti-nɯ. \\
 \textsc{indef}.\textsc{poss}-milk \textsc{comit} doughnut completely \textsc{erg} \textsc{ipfv}-\textsc{caus}-\textsc{caus}-be.big-\textsc{pl} \\
\glt `They (used to) raise up (the babies) by feeding them milk and doughnuts only.' (140426 tApAtso kAnWBdaR, 102)
\end{exe}

The same applies to (\ref{ex:Wru.RJa.nW}), where the presence of the demonstrative \forme{nɯ} after \forme{ʁɟa} shows that it belongs to the same noun phrase.

\begin{exe}
\ex \label{ex:Wru.RJa.nW}
 \gll ɯ-rdoʁ nɯ-me tɕe, [ɯ-ru ʁɟa nɯ], pɯ-kɤ-tɤβ nɯnɯ, taʁndzɤr ɯ-ŋgɯ tú-wɣ-rku tɕe, \\
 \textsc{3sg}.\textsc{poss}-grain \textsc{pfv}-not.exist \textsc{lnk} \textsc{3sg}.\textsc{poss}-stalk completely \textsc{dem} \textsc{pfv}-\textsc{nmlz}:P-thresh \textsc{dem} feeding.emmer \textsc{3sg}.\textsc{poss}-inside \textsc{ipfv}-\textsc{inv}-put.in \textsc{lnk} \\
 \glt `When all the grains have been removed, the bare stalks, the one that have been threshed, one puts them in a feeding emmer.' (140513 tWrtsi, 5)
\end{exe}

A reduplicated emphatic form \forme{ʁɟɯ\redp{}ʁɟa} is also found as in (\ref{ex:RJWRJa.kW})

\begin{exe}
\ex \label{ex:RJWRJa.kW}
 \gll χtɕɤnzɤn ʁɟɯ\redp{}ʁɟa kɯ ʑo pɯ́-wɣ-nɤjo ɕti ɲɯ-ŋu.  \\
beast \textsc{emph}\redp{}completely \textsc{erg} \textsc{emph} \textsc{pst}.\textsc{ipfv}-\textsc{inv}-wait be.\textsc{affirm}:\textsc{fact} \textsc{sens}-be \\
\glt `It was all wild beasts waiting for him (there).' (Norbzang2005, 308)
 \end{exe}
 
A third option to express restrictive focus is the IPN \forme{ɯ-jlu}, which is used in the meaning `uncooked' as a property IPN (§ \ref{sec:property.nouns}), but has become grammaticalized as a restrictive marker `exclusively, without anything else' (presumably from an intermediate meaning `plain'), as in (\ref{ex:Wjlu.Zo}).

\begin{exe}
\ex \label{ex:Wjlu.Zo}
 \gll srɤz nɯ kɯ tɕʰoz ɯ-jlu ʑo pjɯ-nɯjɤntɤn pɯ-ɕti ma jɯm nɯ mɯ-pjɤ-ɕar ɲɯ-ŋu, \\
prince \textsc{dem} \textsc{erg}  religion \textsc{3sg}.\textsc{poss}-exclusively \textsc{emph} \textsc{ipfv}-be.assiduous.in  \textsc{pst}.\textsc{ipfv}-be.\textsc{affirm} \textsc{lnk} wife \textsc{dem} \textsc{neg}-\textsc{ifr}.\textsc{ipfv}-look.for \textsc{sens}-be \\
 \glt `The prince was focused exclusively in the study of religion, and was not looking for a wife.' (sras2003, 3)
 \end{exe}

For the expression of restrictive focus with temporal noun phrases or clauses, the postposition \japhug{kóʁmɯz}{only after} can also be used, especially with the demonstrative in the expression \japhug{nɯ kóʁmɯz nɤ}{only then} (§ \ref{sec:temporal.postpositions}, § XXX).



\subsection{Identity modifiers} \label{sec:identity.modifier}
There is no specific identity modifier `the same' in Japhug. The only way to express this meaning is to use the S-participle of the verb \japhug{naχtɕɯɣ}{be the same} (a denumeral verb of Tibetan origin, § \ref{sec:tibetan.numerals}, see also § \ref{sec:comitative} on the syntax of this stative verb and § XXX on its derivation) in a relative clause, as in (\ref{ex:tArmi.kWnaXtCWG}) (a possessor relative, § XXX). This participle is also used adverbially (see § XXX).

\begin{exe}
\ex \label{ex:tArmi.kWnaXtCWG}
\gll tɤ-rmi kɯ-naχtɕɯɣ pjɤ-dɤn wo kɤmɲɯ, nɤki kɯrɯ ra tɕe. \\
\textsc{indef}.\textsc{poss}-name \textsc{nmlz}:S/A-be.the.same \textsc{ifr}.\textsc{ipfv}-be.many \textsc{sfp} pl.n. \textsc{filler} Tibetan \textsc{pl} \textsc{lnk} \\
\glt `There were many people who had identical names, in Kamnyu, among the Tibetans.' (140522 tshupa, 161)
\end{exe}


There are two prenominal modifiers expressing non-identity in Japhug: \japhug{kɯmaʁ}{other} and the numeral \japhug{ci}{one}, which in prenominal position means `the other one' (in postnominal position, it is used as an indefinite article, see § \ref{sec:indef.article}). Both of these words can also be used as pronouns, though \forme{ci} requires to be combined with the demonstrative \forme{nɯ} in this usage (see § \ref{sec:other.pro}).

The modifier \forme{kɯmaʁ} is prenominal in its meaning `other', as in (\ref{ex:kWmaR.tWrme}). 

\begin{exe}
\ex \label{ex:kWmaR.tWrme}
\gll tɯ-zda nɯ ma kɯmaʁ tɯrme a-pɯ-me tɕe, kʰa ra aʁɤndɯndɤt ɲɯ-ɤ<nɯ>ɣro ɲɯ-ŋu ɲɯ-ti. \\
\textsc{genr}.\textsc{poss}-companion \textsc{dem} apart.from other person \textsc{irr}-\textsc{ipfv}-not.exist \textsc{lnk} house \textsc{pl} everywhere \textsc{ipfv}-<\textsc{auto}>play \textsc{sens}-be \textsc{sens}-say \\
\glt `(Our neighbour) says that if there are no other persons apart from family members, (the monkey) would play everywhere in the house.' (19-GzW2, 10)
\end{exe}

There are apparent examples of \japhug{kɯmaʁ}{other} in postnominal position, as in (\ref{ex:kWmaR.taXtW}) and (\ref{ex:kWmaR.tanWsWBzu}), but in such sentences \forme{kɯmaʁ} is a preverbal adverb, not a noun modifier, with a slightly different meaning `anew'. In (\ref{ex:kWmaR.taXtW}), the usage of \forme{kɯmaʁ} is very similar to its Chinese equivalent \ch{另外}{lìngwài}{other} in the corresponding Chinese sentence \zh{阿兰另外给我买了一部手机}, where the preverbal position of \ch{另外}{lìngwài}{other} clearly shows that it is not a noun modifier. 

\begin{exe}
\ex \label{ex:kWmaR.taXtW}
\gll <alan> kɯ a-<dianhua> kɯmaʁ ta-χtɯ \\
p.n. \textsc{erg} \textsc{1sg}.\textsc{poss}-phone other \textsc{pfv}:3\fl{}3'-buy \\
\glt `Alan bought me a new phone.' (conversation, 17-03-27)
\end{exe}

\begin{exe}
\ex \label{ex:kWmaR.tanWsWBzu}
\gll a-ʁi kɯ kʰa kɯmaʁ ta-nɯ-sɯ-βzu qʰe, \\
\textsc{1sg}.\textsc{poss}-younger.sibling \textsc{erg} house other \textsc{pfv}:3\fl{}3'-\textsc{auto}-\textsc{caus}-make \textsc{lnk} \\
\glt `My brother made himself a new house.' (14-tApitaRi, 304)
\end{exe}

The identity determiner \japhug{kɯmaʁ}{other} is grammaticalized from the S-participle of the verb \japhug{maʁ}{not be}, \forme{kɯ-maʁ} `who/which is not X' (see also § \ref{sec:lexicalized.subject.participle}), which is still widely used, as in (\ref{ex:tChWrtsAm.kWmaR}) and (\ref{ex:sthWci.kWmaR}).


%\begin{exe}
%\ex \label{ex:Wstu.kWmaR}
%\gll ɯ-stu kɯ-maʁ me, kɯki mɤ-kɯ-pe me \\
%\textsc{3sg}.\textsc{poss}-truth \textsc{nmlz}:S/A-not.be not.exist:\textsc{fact} dem.\textsc{prox} \textsc{neg}-\textsc{nmlz}:S/A-be.good not.exist:\textsc{fact} \\
%\glt ` (28-smAnmi, 16)
%\end{exe}

\begin{exe}
\ex \label{ex:tChWrtsAm.kWmaR}
\gll mɤʑɯ [tɕʰirtsɤm kɯ-maʁ] nɯnɯ tɕe, tú-wɣ-χtɕi ma nɯ ma kɤ-sqa (mɤ-ra) \\
yet type.of.tsampa \textsc{nmlz}:S/A-not.be \textsc{dem} \textsc{lnk} \textsc{ipfv}-\textsc{inv}-wash \textsc{lnk} \textsc{dem} apart.from \textsc{inf}-boil \textsc{neg}-have.to:\textsc{fact} \\
\glt `The tsampa that is not `chu.rtsam', one needs to wash it, but not to boil it.' (2002tWsqar, 112)
\end{exe}

\begin{exe}
\ex \label{ex:sthWci.kWmaR}
\gll  [ɯ-rkɯ wuma ʑo stʰɯci kɯ-maʁ] nɯtɕu tɤ-ri ci kú-wɣ-lɤt \\
\textsc{3sg}.\textsc{poss}-side really \textsc{emph} so.much \textsc{nmlz}:S/A-not.be \textsc{dem}:\textsc{loc} \textsc{indef}.\textsc{poss}-thread once \textsc{ipfv}-\textsc{inv}-throw \\
\glt `One sews a thread at a place which is not too much on the border (of the patch)'. (12-kAtsxWb, 16)
\end{exe}


The modifier \forme{ci} differs from \forme{kɯmaʁ} in that it is necessarily definite, meaning `the other one', as in (\ref{ex:ci.rWdaR}), where it refers to an animal that it chased by lions, which was previously mentioned in the text.

\begin{exe}
\ex \label{ex:ci.rWdaR}
\gll ʑɯrɯʑɤri qʰe ci rɯdaʁ nɯ dɯxpa ma nɯ-kɤ-ndza ɯ-spa ɲɯ-ɕti qʰe, qʰe pjɯ-ndʐaβ qʰe mɯ-ɲɯ-cʰa qʰe, \\
progressively \textsc{lnk} other.one animal \textsc{dem} poor.of \textsc{lnk} \textsc{3pl}.\textsc{poss}-\textsc{nmlz}:P-eat \textsc{3sg}.\textsc{poss}-material \textsc{sens}-be.affirm \textsc{lnk} \textsc{lnk} \textsc{ipfv}-\textsc{anticaus}:make.fall \textsc{lnk} \textsc{neg}-\textsc{ipfv}-can \textsc{lnk} \\
\glt `The other animal, poor of him, it is their prey, progressively it falls down and cannot stand it anymore.' (20-sWNgi, 43)
\end{exe}

Interestingly, the determiner \forme{ci} does not have scope over other noun modifiers. For instance, in (\ref{ex:ci.tCheme.kWNAn}), the noun \japhug{tɕʰeme}{woman} occurs with an attributive adjective in participial form \forme{kɯ-ŋɤn} `who is evil' (a relative clause, see § \ref{sec:attributes}), but the meaning is not `the other evil woman' as could have been expected (since the woman who is the subject of the sentence is, by contrast, a kind person), and rather must be `the other woman, the evil one'. There is no pause in the recording that could lead us to suppose that \forme{kɯ-ŋɤn} here is an apposition -- it is rather a postnominal relative.

\begin{exe}
\ex \label{ex:ci.tCheme.kWNAn}
\gll nɤki, tɕʰeme nɯ ɯ-ɕki ɯ-kɯ-sɤja jo-ɕe, ci tɕʰeme kɯ-ŋɤn nɯ ɯ-ɕki. \\
\textsc{filler} women \textsc{dem} \textsc{3sg}-\textsc{dat} \textsc{3sg}.\textsc{poss}-\textsc{nmlz}:S/A-give.back \textsc{ifr}-go other.one woman \textsc{nmlz}:S/A-be.evil \textsc{dem} \textsc{3sg}-\textsc{dat} \\
\glt `She went to give it back to the woman, the other one, the evil woman.' (140515 jiesu de laoren-zh, 90)
\end{exe}

%a-ʁi kɯnɤ tɯrme kha kɤ-sɯxɕe mɯ́j-khɯ qhe,

%ci qhɤjmbaʁ nɯ kɯ-jaʁ kɯ-fse nɯnɯ 
%mtshalu ɯ-cu tɕe nɤki,
%tɯ-mgo zmɤrɤβ kú-wɣ-nɯ-lɤt sna.
%16-RlWmsWsi
%li ci /ɯt/ ɯ-tɯphu nɯ tɤpu qhɤjmbaʁ tu-ti-nɯ ŋu tɕe,
\subsection{Attributes} \label{sec:attributes}
Japhug has several sub-parts of speech which could be described as `adjectives': stative verbs, adverbs and nouns which express properties (rather than actions or entities). Property words used as noun modifiers are collectively designated by the term \textit{attributes}. This heterogenous class excludes the property nouns described in § \ref{sec:property.nouns}, which are the syntactic heads of the noun phrase.

Three types of attributes are distinguished: attributive postnominal (noun or adverb) modifiers, prenominal modifiers and participial \forme{kɯ-} relatives (mainly postnominal or head-internal). The constructions mentioned in this section are all described in more details elsewhere in the grammar, and for this reason the discussion is kept brief.

\subsubsection{Attributive postnominal modifiers} \label{ex:attributive.postnominal}
In addition to the postnominal markers studied above (numeral and number § \ref{sec:number.determiners}, demonstratives § \ref{sec:demonstrative.determiners}, quantifiers § \ref{sec:quantifiers.determiners}, definiteness markers § \ref{sec:indef.article}, topic and focus markers), there are a certain number of nouns that can serve a post-nominal modifiers.

An entire class of such nouns consists of the privative nouns in \forme{-lu} `...less', described in § \ref{sec:privative}.

The word \japhug{wuma}{real, really} from Tibetan \tibet{ངོ་མ་}{ŋo.ma}{real, true} is generally used adverbially as an intensifier, in particular with stative verbs (§ XXX), but also occurs as a postnominal modifier meaning ` real', its original meaning, as in (\ref{ex:lhAndzxi.wuma} and (\ref{ex:tAtsoR.wuma}).

\begin{exe}
\ex \label{ex:lhAndzxi.wuma}
\gll ɬɤndʐi wuma nɯ nɤʑo ɲɯ-tɯ-ŋu ma aʑo ɬɤndʐi ɲɯ-maʁ-a \\
demon real \textsc{dem} \textsc{2sg} \textsc{sens}-2-be \textsc{lnk} \textsc{1sg} demon \textsc{sens}-not.be-\textsc{1sg} \\
\glt `You are the real demon, not me.' (2002lhandzi, 12)
\end{exe}

\begin{exe}
\ex \label{ex:tAtsoR.wuma}
\gll ɯ-qa nɯ qarŋe, tɤtsoʁ wuma nɯ. \\
\textsc{3sg}.\textsc{poss}-root \textsc{dem} be.yellow:\textsc{fact} silverweed real \textsc{dem} \\
\glt `Its root is yellow, the real silverweed.' (19-khWlu, 74)
\end{exe}

In addition, some adverbs, which are more often used with scope over the whole clause, can occur as postnominal modifiers, in particular the comitative adverbs (§ \ref{sec:comitative.adverb}). In (\ref{ex:CWNArWra.kW}), we clearly see that the adverb \japhug{ɕɯŋarɯra}{each better than the other} is a postnominal modifier of \japhug{rɟɤlpu}{king} and not a sentential adverb, as it is followed by the ergative \forme{kɯ}.

\begin{exe}
\ex \label{ex:CWNArWra.kW}
\gll   rɟɤlpu ɕɯŋarɯra kɯ ta-tʰu-nɯ ɕti ri, mɯ-tɤ-nɤla-j ɕti tɕe, \\
king each.better.than.the.other \textsc{erg} \textsc{pfv}:3\fl{}3'-ask-\textsc{pl} be.\textsc{affirm}:\textsc{fact} \textsc{lnk} \textsc{neg}-\textsc{pfv}-agree-\textsc{1sg} \\
\glt `(Many) kings, all better than the other, asked for (my daughters in marriage), but we did not agree.' (2003qachGa, 71)
\end{exe}

\subsubsection{Attributive prenominal modifiers}   \label{ex:attributive.prenominal}
There are two types of prenominal modifiers, the possessors (which are followed by a possessum taking a possessive prefix coreferent with the possessor, § \ref{ex:prefix.expression.of.possession}), but also   are essentially placenames and other unpossessible nouns (§ \ref{sec:place.names}) and nouns expressing the material from which an object is a made, as \japhug{χsɤr}{gold}, \japhug{rŋɯl}{silver} and \japhug{si}{wood} 
in (\ref{ex:rNWl.rJAskAt}).

\begin{exe}
\ex \label{ex:rNWl.rJAskAt}
\gll a-tɤɕime, nɤʑo rŋɯl rɟɤskɤt ɯ-taʁ tɯ-ɕe ɕi, χsɤr rɟɤskɤt ɯ-taʁ tɯ-ɕe ɕi, ɕom rɟɤskɤt ɯ-taʁ tɯ-nɯ-ɕe? \\
\textsc{1sg}.\textsc{poss}-lady \textsc{2sg} silver stair \textsc{3sg}.\textsc{poss}-on 2-go:\textsc{fact}  \textsc{qu} gold stair \textsc{3sg}.\textsc{poss}-on 2-go:\textsc{fact} \textsc{qu} wood stair \textsc{3sg}.\textsc{poss}-on 2-\textsc{auto}-go:\textsc{fact} \\
\glt `My lady, will you go on the silver stairs, the golden stairs or the wooden stairs?' (2014-kWlAG, 369)
\end{exe}

When a prenominal modifier is present, possessive prefixes on the head noun can be neutralized to indefinite possessor, or the modifier and the head noun can be fused as a compound noun, with the possessive prefixes occurring in leftmost position (see the discussion in § \ref{sec:possessive.prefixes.prenominal}).


\subsubsection{Participial relatives} \label{ex:attributive.participles.stative.verbs}
Most words expressing properties in Japhug are a subclass of stative verbs (§ XXX), and cannot serve as attributes without being embedded into a relative clause. Since intransitive subjects can only be relativized using \forme{kɯ-} participial relative clauses (§ XXX), attributive adjectival stative verbs are always in this form, as \forme{kɯ-pe} `good one, which is good' in (\ref{ex:wuma.Zo.kWpe}); the relative  \forme{wuma ʑo tɕʰeme kɯ-pe} in this example is head-internal (§ XXX), as shown by the position of the intensifier \forme{wuma ʑo}, and literally means `(a) woman who is/was really nice.'

\begin{exe}
   \ex  \label{ex:wuma.Zo.kWpe}
\gll  ɯ-rʑaβ βdaʁmu nɯ [wuma ʑo tɕʰeme kɯ-pe] ci pjɤ-ŋu. \\
\textsc{3sg}.\textsc{poss}-wife lady \textsc{dem} really \textsc{emph} woman \textsc{nmlz}:S/A-be.good \textsc{indef} \textsc{ifr}.\textsc{ipfv}-be \\
\glt `His wife, the queen, was a very nice woman.' (28-smAnmi, 4)
\end{exe}  

In the case of shorter relative clauses it is not always clear whether we have a head-internal, or a postnominal one (§ XXX). Prenominal relatives with an adjectival stative verb such as \forme{stu kɯ-mna} `the best one, the leader' in (\ref{ex:stu.kWmna.tCheme}) are very rare.


\begin{exe}
\ex \label{ex:stu.kWmna.tCheme}
\gll  rɟɤlpu nɤrɯβzaŋ nɯ kɯ, nɤki, [stu kɯ-mna tɕʰeme] nɯ ɲɤ-nɯ-ɕar ɲɯ-ŋu \\
king p.n. \textsc{dem} \textsc{erg} filler most \textsc{nmlz}:S/A-be.better woman \textsc{dem} \textsc{ifr}-\textsc{auto}-search \textsc{sens}-be \\
\glt `King Norbzang chose for himself the woman leader.' (Norbzang2012, 41)
\end{exe} 


Attributive relative clauses may contain comparative constructions as in (\ref{ex:WZo.sAz.rWdaR.kWxtCi}), or semi-objects; a more detailed account is presented in § XXX.

\begin{exe}
\ex \label{ex:WZo.sAz.rWdaR.kWxtCi}
\gll ɯʑo sɤz rɯdaʁ kɯ-xtɕi nɯra tu-ndze \\
\textsc{3sg} \textsc{comp} animal \textsc{nmlz}:S/A-be.small \textsc{dem}:\textsc{pl} \textsc{ipfv}-eat[III] \\
\glt `It eats the animals that are smaller than itself.' (20-sWNgi, 23)
\end{exe}

As shown in § XXX, in head-internal relative clauses the same determiner can appear on the head noun and repeated after the whole relative. The same is found with adjectival participial relatives, as in (\ref{ex:qajW.ci.kArNi.ci}) with the indefinite \japhug{ci}{one}, though this usage is rare, in most cases only one of the two determiners is used (either the one inside the relative or the external one, § XXX).

\begin{exe}
\ex \label{ex:qajW.ci.kArNi.ci}
\gll tɕe [qajɯ ci kɯ-ɤrŋi] ci ŋu. \\
\textsc{lnk} bug \textsc{indef} \textsc{nmlz}:S/A-be.green \textsc{indef} be:\textsc{fact} \\
\glt `It is a green/black bug.' (26-zrWGndza)
\end{exe}

Subject prenominal relatives are almost not attested with stative verbs, but are found with some intransitive dynamic verbs, as in the lexicalized expression in (\ref{ex:kWrla.kha}), where prenominal position is required.

\begin{exe}
\ex \label{ex:kWrla.kha}
\gll kɯ-rlaʁ kʰa \\
\textsc{nmlz}:S/A-disappear house \\
\glt `An abandonned house.' 
\end{exe}

 \section{Noun coordination}
Japhug lacks a dedicated noun coordinator. Nouns can be either coordinated by using the position \forme{cʰo} (§ \ref{sec:coordinator.cho}), or by  juxtaposition without any linking element (§ \ref{sec:bare.coordination}). 

\subsection{Coordination or embedded phrase} \label{sec:coordinator.cho}
The closest thing to a noun coordinator in Japhug is the comitative marker \forme{cʰo}; it can be used both to connect finite clauses (§ XXX) or nouns as in (\ref{ex:qandZGi.cho.qaliaR}), with the plural indexation on the verb reflecting the whole transitive subject constituent \forme{qandʑɣi cʰo qaliaʁ ra kɯ}.

\begin{exe}
\ex \label{ex:qandZGi.cho.qaliaR}
 \gll  qandʑɣi cʰo qaliaʁ ra kɯ cʰɯ-nɯ-tsɯm-nɯ tu-ndza-nɯ ŋgrɤl. \\
falcon \textsc{comit} eagle \textsc{pl} \textsc{erg} \textsc{ipfv}:\textsc{downstream}-\textsc{vert}-take.away-\textsc{pl} \textsc{ipfv}-eat-\textsc{pl} be.usually.the.case:\textsc{fact} \\
\glt `Falcons and eagles take them (the moles) and eat them.' (28-qapar, 201)
\end{exe}

There is however evidence that \forme{cʰo} is a postposition: first, some verbs select an oblique argument with the comitative (§ \ref{sec:comitative}), and second, while \forme{cʰo}  cannot be used without a preceding noun phrase (or clause), the following noun is optional. For these reasons, rather than assuming a `flat' structure as in Figure \ref{fig:qanZGi}, I consider \forme{qandʑɣi cʰo} to be a postpositional phrase  used as an adnominal modifier of the noun \japhug{qaliaʁ}{eagle}, as in Figure \ref{fig:qanZGi2}.

\begin{figure} 
\caption{\forme{cʰo} as a coordinator} \label{fig:qanZGi} \centering
\Tree [.PostP [.NP  [.N' [.N \forme{qandʑɣi} ] [.Coord \forme{cʰo} ]  [.N \forme{qaliaʁ} ] ] [.D \forme{ra} ] ] [.Post \forme{kɯ} ] ]
\end{figure}

\begin{figure} 
\caption{\forme{cʰo} as a postposition} \label{fig:qanZGi2} \centering
\Tree [.PostP [.NP  [.N' [.PostP [.N \forme{qandʑɣi} ] [.Post \forme{cʰo} ] ]  [.NP \forme{qaliaʁ} ] ] [.D \forme{ra} ] ] [.Post \forme{kɯ} ] ]
\end{figure}

Even without number marking, the constituent comprising the noun and the \forme{cʰo} postpositional phrase is indexed with non-singular indexation, dual in the case of (\ref{ex:dpalcan.cho.alan}).

 \begin{exe}
\ex \label{ex:dpalcan.cho.alan}
 \gll  χpɤltɕin cʰo alan kɯ ko-ndo-ndʑi tɕe, \\
 n.p. \textsc{comit} n.p. \textsc{erg} \textsc{ifr}-take-\textsc{du} \textsc{lnk} \\
 \glt `Dpalcan and Alan caught (one).' (24-qro, 101)
 \end{exe}
 
\subsection{Bare coordination} \label{sec:bare.coordination}

\subsubsection{Enumeration} \label{sec:noun.enumeration}
Enumerations are the listing of a series of nouns, often with a specific (rising) intonation and a pause between item, and without any coordinating element including the postposition \forme{cʰo}. In Japhug, quite lengthy enumerations are attested in the corpus, as shown by (\ref{ex:mbro.etc}) with seven nouns.

\begin{exe}
\ex \label{ex:mbro.etc}
 \gll mbro, jla, nɯŋa, mbala, tsʰɤt, qaʑo, paʁ, nɯra nɯtɕu ʁɟa z-ɲɯ́-wɣ-lɤɣ pɯ-ŋu. \\
 horse hybrid.yak cow bull goat sheep pig \textsc{dem}:\textsc{pl} \textsc{dem}:\textsc{loc} completely \textsc{transloc}-\textsc{ipfv}-\textsc{inv}-graze \textsc{pst}.\textsc{ipfv}-be \\
\glt `People used to graze there horses, hybrid yaks, cows, bulls, goats, sheep ans pigs.' (140522 Kamnyu zgo, 156)
\end{exe}

Enumerations with only two or three nouns and without specific intonation are also found, as in (\ref{ex:tshAt.qaZo}). When the order of the nouns is rigid (which is not the case in \ref{ex:tshAt.qaZo}, since \forme{qaʑo tsʰɤt} `sheep and goats' is also attested), the construction belongs to a distinct category: that of noun dyads (§ \ref{sec:dyads}).

\begin{exe}
\ex \label{ex:tshAt.qaZo}
 \gll  qʰe tsʰɤt qaʑo ra ɣɯ nɯ-ndza nɯra ɲɯ-sna.  \\
 \textsc{lnk} goat sheep \textsc{pl} \textsc{gen} \textsc{3pl}.\textsc{poss}-food \textsc{dem}:\textsc{pl} \textsc{sens}-be.good \\
\glt `It is good as fodder for goats and sheep.' (16-RlWmsWsi, 65)
\end{exe}

\subsubsection{Noun dyads} \label{sec:dyads}
Noun dyads are a pair of nouns occurring in a fixed order, without intervening linker or postposition, and sharing their number and case markers. A good example is provided by the expression `parents' comprising the kinship terms \japhug{tɤ-mu}{mother} and \japhug{tɤ-wa}{father}, as in (\ref{ex:amu.awa.ni.GW}). Note that while number and case markers are shared by both nouns, each of them takes its own possessive prefix, and both prefixes are coreferent. 

\begin{exe}
\ex \label{ex:amu.awa.ni.GW}
 \gll nɯ a-mu a-wa ni ɣɯ ŋu \\
 \textsc{dem}  \textsc{1sg}.\textsc{poss}-mother \textsc{1sg}.\textsc{poss}-father \textsc{du} \textsc{gen} be:\textsc{fact} \\
 \glt `This is for my parents.' (meimei de gushi)
\end{exe}

The dyad for `parents' has a honorific variant, originally used for noblemen in the traditional society. It comprises the terms \japhug{tɤ-pa}{father} and \japhug{tɤ-ma}{mother}, which are borrowed from Tibetan \tibet{ཨ་ཕ་}{ʔa.pʰa}{father}  and \tibet{ཨ་མ་}{ʔa.ma}{mother} respectively. Interesting, the honorific expression follows the `father-mother' order (as in example \ref{ex:apa.ama}), while the native one puts `mother' in the first place.

\begin{exe}
\ex \label{ex:apa.ama}
 \gll nɯ kɯ-fse a-pa a-ma ni kɯ ɲɯ-ti-ndʑi tɕe \\
 \textsc{dem} \textsc{nmlz}:S/A-be.like \textsc{1sg}.\textsc{poss}-father  \textsc{1sg}.\textsc{poss}-mother \textsc{du} \textsc{erg} \textsc{sens}-say-\textsc{du} \textsc{lnk} \\
 \glt `My parents say this.' (2003nyima2, 94)
\end{exe}

Other common dyads include \forme{rgɤtpu rgɤnmɯ} `old man(men) and woman(women)', \forme{tɤ-tɕɯ tɕʰeme} `boy(s) and girl(s)' with APNs. They are most commonly used as collectives with indefinite referents as in (\ref{ex:tAtCW.tCheme.tWsAmdzW}), but are also attested with definite ones, as in (\ref{ex:rgAtpu.rgAnmW}).

\begin{exe}
\ex \label{ex:tAtCW.tCheme.tWsAmdzW}
 \gll  tɤ-tɕɯ tɕʰeme tɯ-sɤ-ɤmdzɯ ʑaka tu \\
 \textsc{indef}.\textsc{poss}-son girl \textsc{genr}.\textsc{poss}-\textsc{nmlz}:\textsc{oblique}-sit each \textsc{exist}:fact \\
\glt `Gents and ladies each have (different) seating places.' (31-khAjmu, 10)
\end{exe}

\begin{exe}
\ex \label{ex:rgAtpu.rgAnmW}
 \gll rgɤtpu rgɤnmɯ ni kɯ kɯki tɤ-pɤtso χsɯm ki kɤsɯfse ʑo cʰɤ-ɣɤ-wxti-ndʑi. \\
 old.man old.woman \textsc{du} \textsc{erg} \textsc{dem}.\textsc{prox} \textsc{indef}.\textsc{poss}-child three \textsc{dem.prox} all \textsc{emph} \textsc{ifr}-\textsc{caus}-be.big-\textsc{du} \\
\glt `The old man and the old woman raised all these three children.' (140514 huishuohua de niao-zh, 60)
\end{exe}

Another type of noun dyad comprises two abstract nouns, which can be used as manner adjuncts with the ergative (see example \ref{ex:tAmqe.tAndWt.kW} § \ref{sec:manner.nominal.kW}) or in the degree construction as in (\ref{ex:tAre.tAJaR}) with the dyad \forme{tɤ-re tɤ-ɟaʁ} `chatting and laughing' (only \japhug{tɤ-re}{laugh} exists as an independent word). Some of these dyads are nominalized forms of bipartite verbs (§ XXX).

\begin{exe}
\ex \label{ex:tAre.tAJaR}
 \gll  nɯtɕu rcanɯ maka tɤ-re tɤ-ɟaʁ pjɤ-saχaʁ \\
 dem:loc unexpectedly completely \textsc{indef}.\textsc{poss}-laugh \textsc{indef}.\textsc{poss}-laugh \textsc{indef}.\textsc{poss}-chatting.and.laughing \textsc{ifr}.\textsc{ipfv}-be.extremely \\
 \glt `There, (the râkshasî) were chatting and laughing a lot.' (2011-05-nyima, 34)
\end{exe}

\section{Word order in the noun phrase}
The relative order of the elements in the noun phrase in Japhug is relatively rigid.  Examples of relatively complex nouns phrases such as (\ref{ex:N.Adj.Num}) and (\ref{ex:Dem.N.Num.Dem}) illustrate the orders N(oun)-Adj(ective)-Num(eral) and Dem(onstrative)-N(oun)-Num(eral)-Dem(onstrative) respectively, considering here attributive adjectival stative verbs in participial form to be the main `adjective' class in Japhug, see § XXX and § \ref{ex:attributive.participles.stative.verbs}).
 
\begin{exe}
\ex \label{ex:N.Adj.Num}
 \gll sɯŋgɯ zɯ, [tɯrme wuma ʑo kɯ-wxti ʁnɯz] tu-ndʑi tɕe \\
 forest \textsc{loc} people really \textsc{emph} \textsc{nmlz}:S/A-be.big two exist:\textsc{fact}-\textsc{du} \textsc{lnk} \\
 \glt `In the forest, there are two very big persons.' (140428 yonggan de xiaocaifeng-zh, 171)
\end{exe}

\begin{exe}
\ex \label{ex:Dem.N.Num.Dem}
 \gll [ɯkɯki ɕnat ʁnɯz kɯni] kɯ, \\
 \textsc{dem}.\textsc{prox} heddle two \textsc{dem}.\textsc{prox}:\textsc{du} \textsc{erg} \\
 \glt `(The weaving is done) with these two heddles.' (vid-20140429090403, 47)
\end{exe}

From such examples, it can be extrapolated that the most basic word order in the noun phrase in Japhug is Dem-N-Adj-Num-Dem. Although no noun phrase in the corpus presents all five elements, it is easy to elicitate such an example. This order is not unusually crosslinguistically: \citet{cinque05universal20} notes that the orders  Dem-N-Adj-Num and N-Adj-Num-Dem order are both widely attested.

Although the indefinite marker \japhug{ci}{one} and the topic \forme{nɯ} can appear between the noun and the adjective (being here a head-internal participial relative clause, § \ref{ex:attributive.participles.stative.verbs}), as shown by examples (\ref{ex:qajW.ci.kArNi.ci}) above and (\ref{ex:GJW.ci.kWmbWmbro}) below, numerals are not attested in this position.

\begin{exe}
\ex \label{ex:GJW.ci.kWmbWmbro}
 \gll ɣɟɯ ci kɯ-mbɯ\redp{}mbro ʑo pjɤ-tu ɲɯ-ŋu tɕe   \\
 watchtower \textsc{indef} \textsc{nmlz}:S/A-\textsc{emph}\redp{}be.high \textsc{emph} \textsc{ifr}.\textsc{ipfv}-exist \textsc{sens}-be \textsc{lnk} \\
\glt  `There was a very big tower.' (Norbzang2012, 54)
\end{exe}

There are other attributes than adjectival stative verbs in Japhug (§ \ref{ex:attributive.postnominal} and § \ref{ex:attributive.prenominal}), in particular prenominal attributes as \japhug{rŋɯl}{silver} in (\ref{ex:rNWl.qhoRqhoR}), but it is problematic to use such examples as evidence for a Adj-N-Num-Dem, given the fact that prenominal attributes are always essentially nouns used as modifiers.

 \begin{exe}
\ex \label{ex:rNWl.qhoRqhoR}
\gll  rŋɯl qʰoʁqʰoʁ χsɯm nɯ ɲɤ-nɯ-ɬoʁ. \\
silver ingot three \textsc{dem} \textsc{ifr}-\textsc{auto}-come.out \\
\glt `The three silver ingots had come out.' (28-qAjdoskAt, 178)
\end{exe}

The position of the prenominal demonstrative can alternatively be filled by the identity modifiers \japhug{kɯmaʁ}{other} or \japhug{ci}{the other one} (§ \ref{sec:identity.modifier}). These elements can be preceded by either a comitative \forme{cʰo} phrase (§ \ref{sec:coordinator.cho}) or by the aforementioned topic marker \forme{iɕqʰa} (§ \ref{sec:iCqha}, cf example \ref{ex:iCqha.ci.rJAlpu}), the leftmost element of a noun phrase.

 \begin{exe}
\ex \label{ex:iCqha.ci.rJAlpu}
\gll rɟɤlpu ɯ-tɕɯ nɯ kɯ, iɕqʰa ci rɟɤlpu nɯnɯ, <xila> rɟɤlpu nɯ ɯ-ɕki, \\
king \textsc{3sg}.\textsc{poss}-son \textsc{dem} \textsc{erg} the.aforementioned other.one king \textsc{dem} pl.n. king \textsc{dem} \textsc{3sg}.\textsc{poss}-\textsc{dat} \\
\glt `The king's son (told) the other king, the king of Greece' (140518 huifei de muma-zh, 177
\end{exe}


%{Nominal predicates} \label{sec:nominal.predicates}
