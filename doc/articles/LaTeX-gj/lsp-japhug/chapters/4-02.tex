\chapter{Person indexation} \label{chap:indexation}
In Japhug, person indexation is the defining feature of finite verbs, as opposed to non-finite verbs (§\ref{chap:non-finite}) and other parts of speech. Japhug finite verb forms index one or two arguments, depending on the transitivity of the verb, using a combination of prefixes, suffixes and stem alternation. No verb indexes more than two arguments. The indexation system is very close to a canonical direct-inverse system (§\ref{sec:direct-inverse}).

This chapter first presents intransitive and transitive conjugations, investigates the issue of agreement mismatch, and then discusses the origin of person indexation affixes. In addition, it documents the analogical extension of person indexation suffixes to non-finite verb forms in some specific contexts.

\section{Intransitive verbs} \label{sec:intr.indexation}
Intransitive verbs comprise dynamic, stative and semi-transitive verbs. All of the verbs have in common the property of indexing one argument, the intransitive subject, which when overt is in absolutive form (§\ref{sec:absolutive.S}).

\subsection{The intransitive paradigm} \label{sec:intransitive.paradigm}
Table \ref{tab:intransitive.indexation} illustrates the paradigm of intransitive verbs in Kamnyu Japhug, using the verb \japhug{ɕe}{go} in the Factual non-past\footnote{This TAM category is chosen to illustrate the paradigms due to the fact that it does not bear any orientation prefix, but at the same time presents stem alternation in the transitive paradigm.} as an example. Other Japhug dialects have slightly different indexation suffixes, a question discussed in §\ref{sec:indexation.suffixes.history} with comparative evidence from other Gyalrong languages.

There is no stem alternation related to person indexation in the intransitive paradigm in any Japhug dialect. The invariable stem is represented with the symbol \ro{} in Table \ref{tab:intransitive.indexation}.\footnote{This notation follows the Kirantological tradition (for instance \citealt{driem93dumi}). }

\begin{table}[H] \centering
\caption{The intransitive conjugation in Japhug}\label{tab:intransitive.indexation}
\begin{tabular}{lllllllll} \lsptoprule
Person & Form & \japhug{ɕe}{go} (Factual non-past) \\
\midrule
\textsc{1sg} & \ro{}-\forme{a} & \forme{ɕe-a} \\
\textsc{1du} & \ro{}-\forme{tɕi} & \forme{ɕe-tɕi} \\
\textsc{1pl} & \ro{}-\forme{ji} & \forme{ɕe-j} \\
\midrule
\textsc{2sg} & \forme{tɯ}-\ro{} & \forme{tɯ-ɕe} \\
\textsc{2du} & \forme{tɯ}-\ro{}-\forme{ndʑi} & \forme{tɯ-ɕe-ndʑi} \\
\textsc{2pl} & \forme{tɯ}-\ro{}-\forme{nɯ} & \forme{tɯ-ɕe-nɯ} \\
\midrule
\textsc{3sg} & \ro{} & \forme{ɕe} \\
\textsc{3du} & \ro{}-\forme{ndʑi} & \forme{ɕe-ndʑi} \\
\textsc{3pl} & \ro{}-\forme{nɯ} & \forme{ɕe-nɯ} \\
\midrule
generic & \forme{kɯ}-\ro{} & \forme{kɯ-ɕe} \\
\lspbottomrule
\end{tabular}
\end{table}

In the intransitive paradigm, five suffixes and two prefixes are found. The stress is always on the last syllable of the verb stem, and all person indexation suffixes, including \forme{-a}, are unstressed and sometimes are even devoiced (§XXX). Unlike other languages of the Trans-Himalayan, such as Khaling (where the dual inclusive and the third dual are homophonous, see \citealt[1113]{jacques12khaling}), in Japhug all slots in the intransitive paradigm are distinct, without ambiguity.  

\subsubsection{First person}

First person subjects are indexed by a set of three suffixes marking both person and number: \forme{-a}, \forme{-tɕi} and \forme{-ji} respectively for first singular, dual and plural. As in the pronominal paradigms (§\ref{sec:pers.pronouns}), there is no inclusive/exclusive distinction in Japhug.

The \textsc{1sg} \forme{-a} suffix is the only suffix in Japhug with a vowel other than \forme{ɯ} (or \forme{i} after palatal and alveolo-palatal consonants, §\ref{sec:W.i.contrast}), and is the only indexation suffix that can be followed by another indexation suffix in the transitive paradigm (§\ref{sec:double.number.indexation}). The \forme{-a} \textsc{1sg} person index is among the suffixes revealing the underlying form of the codas: \forme{-β}, \forme{-ɣ}, \forme{-ʁ}, \forme{-z}, which become unvoiced in some contexts (§XXX) are realized as voiced (see for instance in Table \ref{tab:verb.stem.1sg} below; \forme{-β} is realized \phonet{-w-} in this context, see §XXX), but the coda \forme{-t} remains unvoiced (for instance \forme{scit-a} be.happy-\textsc{1sg} `I am happy'). The codas are resyllabified; for instance \forme{scit-a} is syllabified as \forme{sci/ta}).

 Some verb stems (independently of transitivity) undergo predictable phonological alterations when followed by \forme{-a}. With verb stems whose last syllable is an open syllable,  the \forme{-a} suffix merges with the vowel of the last syllable. With closed syllable verb stem in \forme{-ɤC} (C representing a coda), the \textsc{1sg} suffix causes vowel assimilation. These phonological rules are presented in Table \ref{tab:verb.stem.1sg}.

When the verb stem ends in \forme{-a}, the \textsc{1sg} suffix merges with the stem as \phonet{a} in Kamnyu Japhug, resulting in homophony between the \textsc{1sg} and the \textsc{3sg} forms. The surface form \phonet{rga} corresponds to both \textsc{1sg} \japhug{rga-a}{I like it} and \textsc{3sg} \japhug{rga}{he likes it}. The fused and invisible suffix is systematically indicated in the orthography used in this grammar. In some dialects of Japhug, a long vowel occurs in the \textsc{1sg}, which thus remains distinct from the \textsc{3sg}.

When the verb stem ends in the mid-high vowels \forme{-e} and \forme{-o}, these vowels become the corresponding high vowels \forme{-i} and \forme{-u} when followed by \textsc{1sg} in Kamnyu Japhug. This alternation does not occur in all Japhug dialects.

With verb stem ending in \forme{-ɤt}, \forme{-ɤn}, \forme{-ɤβ}, \forme{-ɤm}, \forme{-ɤr}, \forme{-ɤl} and \forme{-ɤz},  the \textsc{1sg} suffix causes non-optional vowel assimilation \forme{-ɤC-a} $\Rightarrow$ \ipa{-aCa}; Table \ref{tab:verb.stem.1sg} provides examples for all rhymes of this type. In the orthography employed in this grammar, these forms are transcribed as \forme{aC-a} rather than the underlying \forme{ɤC-a} (\forme{jɣat-a} rather than \forme{jɣɤt-a}), to indicate the fact that \forme{ɤ} $\Rightarrow$ \forme{a} assimilation is obligatory in this context.

\begin{table}
\caption{Predictable phonological alternations on the verb stem caused by the \forme{-a} \textsc{1sg} suffix in Kamnyu Japhug} \label{tab:verb.stem.1sg}
\begin{tabular}{llllll}
\lsptoprule
Rhyme of the  & Result of  &Examples \\
last syllable & fusion with  \\
of the verb stem & the \textsc{1sg} suffix \\
\midrule
\forme{-e} & \phonet{-ia} & \forme{ɕe-a} $\Rightarrow$ \phonet{ɕia} `I will go there' \\
\forme{-o} & \phonet{-ua} & \forme{tso-a} $\Rightarrow$ \phonet{tsua} `I understand it' \\
\forme{-a} & \phonet{-a} & \forme{rga-a} $\Rightarrow$ \phonet{rga} `I like it' \\
\midrule
\forme{-ɤβ} & \phonet{-awa} & \forme{tʰɯ-rdɤβ-a} $\Rightarrow$ \phonet{tʰɯrdawa} `I lost money' \\
\forme{-ɤm} & \phonet{-ama} & \forme{mtsʰɤm-a} $\Rightarrow$ \phonet{mtsʰama} `I hear it' \\
\forme{-ɤt} & \phonet{-ata} & \forme{jɣɤt-a} $\Rightarrow$ \phonet{jɣata} `I will come back' \\
\forme{-ɤn} & \phonet{-ana} & \forme{tu-nɯsmɤn-a} $\Rightarrow$ \phonet{tunɯsmana} \\
&&  `I will will treat it' \\
\forme{-ɤr} & \phonet{-ara} & \forme{pɯ-atɤr-a} $\Rightarrow$ \phonet{patara} `I fell down' \\
\forme{-ɤl} & \phonet{-ala} & \forme{nɯ-nɯtɯfɕɤl-a} $\Rightarrow$ \phonet{nɯ-nɯtɯfɕal-a}\\
&& `I had diarrhea' \\
\forme{-ɤz} & \phonet{-aza} & \forme{mkʰɤz-a} $\Rightarrow$ \phonet{mkʰaza} `I am expert at it' \\
\lspbottomrule
\end{tabular}
\end{table}

The first dual \forme{-tɕi} suffix (\forme{-tsə} in some dialects of Japhug, §\ref{sec:indexation.suffixes.history}) only causes regular devoicing assimilation on the coda of the verb stem: \forme{-z}, \forme{-r}, \forme{-ɣ}, \forme{-ʁ} are realized as \forme{-s}, \forme{-ʂ}, \forme{-x}, \forme{-χ} when followed by \forme{-tɕi} (for instance \forme{mkʰɤz-tɕi} is pronounced \phonet{mkʰɛ́stɕi}). The labial coda \forme{-β} is not affected. 

The first plural \forme{-ji} has two allomorphs, \forme{-j} and \forme{-i}. The first one occurs on verb stems ending in open syllables, for instance \japhug{ɕe-j} `we (will) go', and the second follows verb stems in closed syllables, such as \forme{scit-i} `we are happy', with resyllabification of the coda (\forme{sci/ti}). Like the \forme{-a} suffix discussed above, the suffix \forme{-i} reveals the underlying form of the codas. The contrast between \forme{-ɯ} and \forme{-i} is neutralized when followed by the \textsc{1pl} suffix:; for instance, the last syllable of \forme{smi tʰɯ-βlɯ-j} `we made a fire'  and \forme{lɤpɯɣ pɯ-βli-j} `we planted radish' is considered to be homophonous by Tshendzin (§XXX).

\subsubsection{Non-first person}

Second and third person forms have the same set of suffixes (zero, \forme{-ndʑi} and \forme{-nɯ} for singular,dual and plural respectively) and only differ by the presence of a \forme{tɯ-} prefix in second person forms. Unlike in Situ (\citealt[197-208]{linxr93jiarong}), there is no distinct second person suffix in the \textsc{2sg}.

The non-first person dual and plural suffixes \forme{-ndʑi} and \forme{-nɯ} (some Japhug dialects have \forme{-ndzə} in the dual instead, see §\ref{sec:indexation.suffixes.history}) nasalize the coda \forme{-t} to \phonet{n}, which is not audible before \forme{-ndʑi} and results in a geminate in the plural. For instance, \forme{scit-ndʑi} and \forme{scit-nɯ} are realized as \phonet{scíndʑi} and \phonet{scínnɯ} respectively. The vowel \forme{-i} and \forme{-ɯ} is often elided, resulting in apparent \forme{-n} codas. The contrast between the codas \forme{-n} and \forme{-t} is neutralized in these forms: the last two syllables of both \forme{tu-nɤndɯt-nɯ} \textsc{ipfv}-fight-\textsc{pl} `they fight (over it)' and \forme{pjɯ-ndɯn-nɯ}  \textsc{ipfv}-read-\textsc{pl} `they read/recite it' are thus realized as \phonet{-ndɯ́nnɯ}.

The second person \forme{tɯ-} prefix fuses with the initial \forme{a-} of contracting verbs (§XXX). The result of vowel fusion is \forme{tɯ-a-} $\Rightarrow$ \phonet{ta} in the Factual Non-past (\japhug{tɯ-atɤr}{you will fall down}) or the Past Perfective (\japhug{jɤ-tɯ-ari}{you went there}), but \forme{tɯ-ɤ-} $\Rightarrow$ \phonet{tɤ} in Irrealis, Imperative, Imperfective or Prohibitive (\japhug{ma-tɤ-tɯ-ɤɕqʰe}{don't cough}) forms. Some irregular verbs have unpredictable second person forms (§\ref{sec:intr.person.irregular}). The generic intransitive subject prefix \forme{kɯ-} (also used for the object of transitive verbs, see \ref{sec:indexation.generic.tr}) follows the same rules of vowel fusion as the second person prefix.


\subsection{Irregular intransitive verbs} \label{sec:intr.person.irregular}
In comparison with Zbu (\citealt{gong18these}), Japhug only has very few irregular verbs. Irregularities related to person marking in Japhug all involve the prefixes.

The second person forms of sensory existential verbs \japhug{ɣɤʑu}{exist} and \japhug{maŋe}{not exist} are infixed rather prefixed. The infixed forms are \forme{ɣɤtɤʑu} and \forme{mataŋe} respectively, as in (\ref{ex:GAtAZu})  (from \citealt[91]{jacques12agreement}) and  (\ref{ex:kAmtshAm.mataNe}).

\begin{exe}
\ex \label{ex:GAtAZu}
\gll iɕqʰa tɯrme ra nɯ-rca ɣɤ<tɤ>ʑu \\
the.aforementioned person \textsc{pl} \textsc{3pl}.\textsc{poss}-following <2sg>exist:\textsc{sens} \\
\glt `(I saw) you among these people.' (elicited)
\end{exe}

\begin{exe}
\ex \label{ex:kAmtshAm.mataNe}
\gll kɤ-mtsʰɤm maka ma<ta>ŋe tɕe, nɤ-kɯ-mŋɤm tu ɯβrɤ-ŋu ma, mɤ-kɯ-pe tu ɯβrɤ-ŋu ma nɯra nɯ-sɯso-t-a. \\
\textsc{inf}-hear at.all <\textsc{2sg}>not.exist:\textsc{sens} \textsc{lnk} \textsc{2sg}.\textsc{poss}-\textsc{nmlz}:S/A-hurt exist:\textsc{fact} \textsc{opt}-be:\textsc{fact} \textsc{sfp} \textsc{2sg}.\textsc{poss}-\textsc{neg}-\textsc{nmlz}:S/A-be.good:\textsc{fact} exist:\textsc{fact} \textsc{opt}-be:\textsc{fact} \textsc{sfp} \textsc{dem}:\textsc{pl} \textsc{pfv}-think-\textsc{pst}:\textsc{tr}-\textsc{1sg} \\
\glt `(I) have not heard at all about you (for some time), I was wondering whether you have some disease, whether something bad happened to you.' (phone conversation, 16-12-28)
\end{exe}

These are not the only infixed forms in the paradigm of these verbs: the generic person \forme{kɯ-} is also infixed (\forme{ɣɤkɤʑu}, \forme{makaŋe}) as is the spontaneous-autobenefactive \forme{nɯ-} (§XXX).

The verb \japhug{zɣɯt}{reach, arrive} has in part of its paradigm forms that are identical to those of contracting verbs (§XXX). In the Past Perfective, it has two alternative second person forms in free variation, the regular \forme{jɤ-tɯ-zɣɯt} and the form \forme{jɤ-tɯ-azɣɯt} with an additional \forme{a-}, illustrated by (\ref{ex:jAtWzGWt.tCe}) and  (\ref{ex:jAtazGWt.mACtsxa}) respectively, coming from two versions of the same story by the same speaker. 

\begin{exe}
\ex \label{ex:jAtazGWt.mACtsxa}
\gll  a-rkɯ mɯ-jɤ-tɯ-azɣɯt mɤɕtʂa mɯ-pɯ-ta-mtsʰɤm tɕe \\
\textsc{1sg}.\textsc{poss}-side \textsc{neg}-\textsc{pfv}-2-arrive until \textsc{neg}-\textsc{pfv}-1\fl{}2-hear \textsc{lnk} \\
\glt `I did not feel your (presence) until you arrived near me.' (Norbzang2012, 260)
\end{exe}

\begin{exe}
\ex \label{ex:jAtWzGWt.tCe}
\gll jɤ-tɯ-zɣɯt tɕe, nɤki, aʑo a-kʰa a-jɤ-tɯ-z-mɤke ma nɤj nɤ-kʰa a-mɤ-jɤ-tɯ-z-mɤke ra mɯ-tɤ-tɯ-tɯt \\
\textsc{pfv}-2-arrive \textsc{lnk} \textsc{filler} \textsc{1sg} \textsc{1sg}.\textsc{poss}-house \textsc{irr}-\textsc{pfv}-2-\textsc{caus}-be.first[III] \textsc{lnk} 
\textsc{2sg} \textsc{2sg}.\textsc{poss}-house \textsc{irr}-\textsc{neg}-\textsc{pfv}-2-\textsc{caus}-be.first[III] have.to:\textsc{fact} \textsc{neg}-\textsc{pfv}-2-say[III] \\
\glt `You did not say ``When you arrive, don't go first to your house, come to my house first.'' (Norbzang2005, 261)
\end{exe}

The paradigm of this verb otherwise includes non-optional contracting (\forme{jɤ-azɣɯt} `he arrived') and non-contracting forms (the immediate converb \forme{ju-tɯ-zɣɯt} `as soon as X arrived', §\ref{sec:immediate.converb}).

\subsection{Semi-transitive verbs} \label{sec:semi.transitive}
Semi-transitive verbs have the same paradigm as plain intransitive verbs, and lack the morphological properties of transitive verbs (§\ref{sec:transitivity.morphology}). Their intransitive subject is in absolutive form. However, they take a semi-object (§\ref{sec:semi.object}), also in absolutive form, as \japhug{paχɕi}{apple} in (\ref{ex:paXCi.ci.taroa}). These semi-objects do present some objectal properties (§XXX).

\begin{exe}
\ex \label{ex:paXCi.ci.taroa}
\gll  tɕe aʑo tʰam kɯki, paχɕi ci tɤ-aro-a tɕe tɕendɤre, 
[...] nɯʑora kɯnɤ ta-sɯ-ɤʁe-nɯ ra \\
\textsc{lnk} \textsc{1sg} now \textsc{dem}.\textsc{prox} apple \textsc{indef} \textsc{pfv}-have-\textsc{1sg} \textsc{lnk} \textsc{lnk} { } \textsc{2pl} also 1\fl{}2-\textsc{caus}-have.to.eat:\textsc{fact}-\textsc{pl} have.to:\textsc{fact} \\
\glt `Now that I have (was given) this apple, I will give it to you also to eat.' (150904 zhongli-zh, 35)
\end{exe}

Unlike transitive verbs, which can index the number of the object if the subject is \textsc{1sg} (§\ref{sec:double.number.indexation}), semi-transitive verbs cannot stack a person index after the \textsc{1sg} \forme{-a}. For instance, in (\ref{ex:XsWm.aroa}), although the object is plural, a form such as $\dagger$\forme{aroa-a-nɯ} with the \forme{-nɯ} plural prefix is strictly prohibited. 

\begin{exe}
\ex   \label{ex:XsWm.aroa}
 \gll aʑo tɤ-rɟit χsɯm aro-a   \\
I \textsc{indef.poss}-child three have:\textsc{fact}-\textsc{1sg} \\
 \glt `I have three children.' (elicited)
\end{exe} 

The subject of some semi-transitive verbs, in particular \japhug{tso}{know, understand} and \japhug{ʑɣɤpa}{pretend}, can be optionally marked with the ergative like a transitive subject (§\ref{sec:S.kW}), as \forme{tɤ-mu nɯ kɯ} in (\ref{ex:kW.mWpjAtso}) and \forme{βdaʁmu nɯ kɯ} in (\ref{ex:kW.toZGApa}). 

\begin{exe}
\ex   \label{ex:kW.mWpjAtso}
 \gll  tɕendɤre [tɤ-mu nɯ kɯ] ɕɯ ŋu nɯ maka mɯ-pjɤ-tso tɕeri \\
\textsc{lnk} \textsc{indef}.\textsc{poss}-mother \textsc{dem} \textsc{erg} who be:\textsc{fact} \textsc{dem} at.all \textsc{neg}-\textsc{ifr}.\textsc{ipfv}-know \textsc{lnk} \\
\glt `The old woman did not realize who it was.' (2002qaCpa, 242)
\end{exe}

\begin{exe}
\ex   \label{ex:kW.toZGApa}
 \gll iɕqʰa βdaʁmu nɯ kɯ [wuma ʑo ɯ-sɯm kɯ-sna] to-ʑɣɤpa \\
 the.aforementioned lady \textsc{dem} \textsc{erg} really \textsc{emph} \textsc{3sg}.\textsc{poss}-mind nmlz:S/A-be.good \textsc{ifr}-pretend \\
 \glt `The lady pretended to be a good person.' (140520 ye tiane-zh, 44)
\end{exe}


Some semi-transitive verbs can take both nominal semi-object and complement clauses. For instance, \forme{tso} (which can be translated as `know', `understand' or `realize' depending on the context) occurs with nouns referring to speech or meaning as semi-object (as in \ref{ex:apWtWtso.smWlAm}), finite relative clauses (\ref{ex:rCanW.mWkAtsoa}) and also participial clauses (\ref{ex:kWNu.kutsoa}; see §XXX concerning the analysis of such clauses).

\begin{exe}
\ex   \label{ex:apWtWtso.smWlAm}
 \gll pja mɯndʐamɯχtɕɯɣ nɯ-skɤt a-pɯ-tɯ-tso smɯlɤm! \\
 bird all.type \textsc{3pl}.\textsc{poss}-speech \textsc{irr}-\textsc{pfv}-2-understand wish \\
 \glt `May you understand the speech of all the species of birds!'(2003kandZislama, 85)
\end{exe}

\begin{exe}
\ex   \label{ex:rCanW.mWkAtsoa}
 \gll ci ta-pa-tɕi, nɯstʰɯci tɤ-nɤrʑaʁ ri, [nɯstʰɯci nɤ-ku ʑru rcanɯ] mɯ-kɤ-tso-a \\
 one \textsc{pfv}:3\fl{}3'-do-\textsc{du} so.much \textsc{pfv}-pass(time) so.much \textsc{2sg}.\textsc{poss}-head be.strong:\textsc{fact} \textsc{foc}:\textsc{unexp} \textsc{neg}-\textsc{pfv}-know-\textsc{1sg} \\
\glt `So much time has passed since we have married, I did not realize that your hair was so long.' (Kunbzang2003, 467)
\end{exe}

\begin{exe}
\ex   \label{ex:kWNu.kutsoa}
 \gll  tɕe [tɕʰi ɯ-skɤt kɯ-ŋu ra] ku-tso-a ɲɯ-ra ma tu-tɯ-ti stɯsti, mɯ́j-ɕɯftaʁ-a ɲɯ-ti \\
\textsc{lnk} what \textsc{3sg}.\textsc{poss}-speech \textsc{nmlz}:S/A-be \textsc{pl} \textsc{ipfv}-understand-\textsc{1sg} \textsc{sens}-have.to \textsc{lnk} \textsc{ipfv}-2-say alone \textsc{neg}:\textsc{sens}-remember-\textsc{1sg} \textsc{sens}-say \\
 \glt `He says: `I need to understand what it is about (what objects these words refer to), otherwise if you only speak (if you only explain orally) I won't remember.'' (conversation 14-05-10, 79)
\end{exe}
%\begin{exe}
%\ex   \label{ex:ŋundZi.mWkAtsoa}
% \gll pʰu ŋu ɕi, pʰu ci mu ŋu-ndʑi nɯ mɯ-kɤ-tso-a ma \\
% male be:\textsc{fact} \textsc{qu} male \textsc{indef} female be:\textsc{fact}-\textsc{du} \textsc{dem} \textsc{neg}-\textsc{pfv}-know-\textsc{1sg} \\
% \glt `I did not get to know whether it is the male that is like that, or both whether both the male and the female are.' (24-ZmbrWpGa, 81)
%\end{exe} 
Among semi-transitive verbs, we find the following subclasses:

\begin{itemize}
\item Verbs of cognition and perception: \japhug{tso}{know, understand}, \japhug{sɤŋo}{listen}
\item Verbs of evaluation: \japhug{rga}{like}, \japhug{stu}{believe}
\item Modal verbs: \japhug{cʰa}{can}
\item Verbs of possession:  \japhug{aro}{have}
\item Copulas: \japhug{ŋu}{be}, \japhug{maʁ}{not be}
\item Verbs of assignation: \japhug{rmi}{be called}, \japhug{artsi}{count as}, \japhug{fse}{be like}
\item Verbs requiring an argument expressing time: \japhug{acʰɤt}{have X years of difference}, \japhug{tsu}{pass X time}
\item Verbs of pretense:  \japhug{ʑɣɤpa}{pretend}
\item Verbs of obtention: \japhug{aʁe}{have to eat}, \japhug{βɟɤt}{get, obtain}
\item Some adjectival stative verbs: \japhug{mkʰɤz}{be expert}, \japhug{pʰɤn}{be efficient}
\end{itemize}

Most semi-transitive verbs are underived bare roots. The only obviously derived verbs are \japhug{ʑɣɤpa}{pretend}, which comes from the reflexive (§XXX) of the verb \japhug{pa}{do} and \japhug{artsi}{count as}, passive of \japhug{rtsi}{count}. The verb \japhug{aro}{have} might be denominal from \japhug{tɤ-ro}{surplus, leftover}.

Most semi-transitive verbs do not usually take a human semi-object, so that sentences with a first or second person semi-object are generally clumsy to build. For some of the verbs above, applicative forms are used when a first or second person object is needed, for instance \japhug{nɯrga}{like} and \japhug{nɤstu}{believe}. The verbs \japhug{stu}{believe} and \japhug{nɤstu}{believe} differ in that the semi-object of the former refers to words (in general, a complement clause; `believe that X') while the object of the latter is a person (`believe him'). 

However, examples with subjects and semi-objects both either first or second person are attested. For instance, (\ref{ex:WYWfsea}) shows a very spontaneous use of a \textsc{2sg} semi-object with a \textsc{1sg} subject with the verb \japhug{fse}{be like}. Only the subject is indexed (with the suffix \forme{-a}) and the use of the transitive \forme{ta-} 1\fl{}2 portmanteau prefix (§\ref{sec:indexation.local}) here would be nonsensical. 

\begin{exe}
\ex \label{ex:WYWfsea}
\gll a-ʁi, nɤʑo ɯ-ɲɯ-fse-a? \\
\textsc{1sg}.\textsc{poss}-younger.sibling \textsc{2sg} \textsc{qu}-\textsc{sens}-be.like-\textsc{1sg} \\
\glt `Sister, do I look like you?' (2014-kWlAG, 475)
\end{exe}


Some semi-transitive verbs are labile; some have a transitive counterpart, while other ones have a plain intransitive one (§\ref{sec:semi.tr.labile}). The meaning of the verb also slightly changes depending on transitivity (for instance, \forme{rga} means `like' when semi-transitive, and `be happy' when stative intransitive).

The copulas are a distinct subclass of semi-transitive verbs, in that their semi-object is a predicative noun. Bare predicative nouns without any verbal element do occur in the corpus, but are rare (§XXX). Person indexation on copula generally follows the subject, but we do find examples in which the indexation agrees with the predicative noun, as in (\ref{ex:stAmku.nWra.NunW}), where the verb has plural indexation like the predicative noun phrase \forme{stɤmku nɯra}, whereas the subject is in the singular. Note that the plural \forme{nɯra} here cannot be interpreted as approximate location (§\ref{sec:plural.determiners}), because in the next sentence the grasslands are analogically referred to by the plural demonstrative pronoun \forme{nɯra} (§\ref{sec:anaphoric.demonstrative.pro}).

\begin{exe}
\ex \label{ex:stAmku.nWra.NunW}
\gll  stu ɯ-sɤ-dɤn nɯ [stɤmku nɯra] ŋu-nɯ. tɕe nɯra nɯ-ŋgɯ tɕe tu-ɬoʁ ŋu tɕe \\
most \textsc{3sg}.\textsc{poss}-\textsc{nmlz}:\textsc{obl}-be.many \textsc{dem} grassland \textsc{dem}:\textsc{pl} be:\textsc{fact}-\textsc{pl} \textsc{lnk} \textsc{dem}:\textsc{pl} \textsc{3pl}.\textsc{poss}-inside \textsc{loc} \textsc{ipfv}-come.out be:\textsc{fact} \textsc{lnk} \\
\\
\glt `The place where it is most numerous is the grasslands, and it grows in these' (19-qachGa mWntoR, 24-25)
\end{exe}

\subsection{Intransitive verbs with oblique arguments} \label{sec:intr.goal}
Semi-transitive verbs have to be distinguished from motion verbs (or perception verbs) with a goal (§\ref{absolutive.goal}), such as \japhug{ɕe}{go}, \japhug{ɣi}{come} or \japhug{ru}{look at}. These verbs are morphologically intransitive, lacking the morphological characteristics of transitive verbs (§\ref{sec:transitivity.morphology}).

With these verbs, the goal can occur in absolutive form, and superficially resembles a semi-object, as \japhug{sɯŋgɯ}{forest} in (\ref{ex:sWNgW.joCendZi}). Indexation obligatorily occurs with the subject (for example, the \textsc{3du} form in \ref{ex:sWNgW.joCendZi}), never with the goal. As in the case of semi-transitive verbs, number stacking on the 1sg \forme{-a} is not possible (§\ref{sec:semi.transitive}, example \ref{ex:XsWm.aroa}).

\begin{exe}
\ex   \label{ex:sWNgW.joCendZi}
 \gll ʁnɯz ni, [sɯŋgɯ] jo-ɕe-ndʑi. \\
two \textsc{du} forest \textsc{ifr}-go-\textsc{du} \\
\glt `Two (men) went into the forest.' (26-tAGe, 1)
\end{exe}

However, unlike semi-objects, these goals can optionally take locative postpositions, such as \forme{zɯ} in (\ref{ex:sWNgW.zW.joCe}).

\begin{exe}
\ex   \label{ex:sWNgW.zW.joCe}
 \gll tɤ-pɤtso nɯnɯ li [sɯŋgɯ zɯ] jo-ɕe. \\
 \textsc{indef}.\textsc{poss}-child \textsc{dem} again forest \textsc{loc} \textsc{ifr}-go \\
 \glt `The child went again into the forest.' (140428 yonggan de xiaocaifen-zh, 230)
\end{exe}

Dative marking on the goals is also well-attested, as in (\ref{ex:sWNgW.WCki.joCe}) -- with motion verbs, it translates as `towards X'.

\begin{exe}
\ex   \label{ex:sWNgW.WCki.joCe}
 \gll tɕhemɤpɯ nɯ kɯ ɯ-wa cʰo ɯ-pi nɯra ɲɤ-βde tɕe, sɯŋgɯ ɯ-ɕki tɕe jo-ɕe. \\
girl \textsc{dem} \textsc{erg} \textsc{3sg}.\textsc{poss}-father \textsc{comit} \textsc{3sg}.\textsc{poss}-elder.sibling \textsc{dem}:\textsc{pl} \textsc{ifr}-leave \textsc{lnk} forest \textsc{3sg}.\textsc{poss}-\textsc{dat} \textsc{loc} \textsc{ifr}-go \\
\glt `The girl left her father and her brothers, and went toward the forest.' (140506 shizi he huichang de bailingniao, 76)
\end{exe}

The subject of intransitive verbs with goals is in absolutive form, except when shared with a transitive verb in another clause, as \forme{tɕhemɤpɯ nɯ kɯ} in (\ref{ex:sWNgW.WCki.joCe}), which owes its ergative marking to the transitive verb \forme{ɲɤ-βde} `She left them'. The verb \japhug{rpu}{bump} (which takes as goal the surface of physical contact) however can take ergative subjects, as it is labile and can be conjugated transitively (§\ref{sec:goal.labile}).

Some verbs, such as \japhug{atɯɣ}{meet}, select an oblique comitative argument in \forme{cʰo} (§\ref{sec:comitative}).

\subsection{Intrinsically non-singular subjects}
Some intransitive verbs have an intrinsic reciprocal meaning, and do not occur in singular form. This category includes most derived reciprocal verbs (§XXX), but also some historical reciprocal verbs that are synchronically non-analyzable such as \japhug{amɯmi}{be in good terms}, and denominal verbs in \forme{a-} like \japhug{anɯmqaj}{fight} or \japhug{aɕɣa}{be of the same age} (§XXX). Example (\ref{ex:atAtAnWmqajnW}) provides some examples of verbs of this type. In the corpus, these verbs only occur in dual or plural form.

\begin{exe}
\ex   \label{ex:atAtAnWmqajnW}
 \gll tɕe a-pi a-ʁi ra kutɕu a-nɯ-tɯ-ɤnɯɣro-nɯ, ci ci a-tɤ-tɯ-ɤnɯmqaj-nɯ, ci ci a-tɤ-tɯ-ɤmɯmi-nɯ qʰe a-kɤ-tɯ-nɯ-rɤʑi-nɯ, \\
\textsc{lnk} \textsc{1sg}.\textsc{poss}-elder.sibling  \textsc{1sg}.\textsc{poss}-younger.sibling \textsc{pl}  here \textsc{irr}-\textsc{pfv}-2-<\textsc{auto}>play-\textsc{pl} once once \textsc{irr}-\textsc{pfv}-2-fight-\textsc{pl} once once \textsc{irr}-\textsc{pfv}-2-be.in.good.terms-\textsc{pl} \textsc{lnk} \textsc{irr}-\textsc{pfv}-2-\textsc{auto}-stay-\textsc{pl}\\
\glt `Brothers, stay here and play, fight from time to time, reconcile with each other from time to time.' (2003kandzwsqhaj, 43)
\end{exe}

The subject of these verbs can be a noun phrase comprising a comitative postpositional phrase in \forme{cʰo} (see §\ref{sec:comitative} and §\ref{sec:coordinator.cho}); number indexation on the verb reflects the addition of the added number of all nominals in the noun phrase. For example, in (\ref{ex:cho.pjAkAmWmindZi}) and (\ref{ex:cho.aCGAtCi}), the verbs have dual indexation, referring to the total number of individuals in the subject noun phrase connected by the comitative \forme{cʰo}.

\begin{exe}
\ex   \label{ex:cho.pjAkAmWmindZi}
 \gll  <maerjina> nɯ cʰo <alibaba> ni wuma ʑo pjɤ-k-ɤmɯmi-ndʑi tɕe \\
p.n. \textsc{dem} \textsc{comit} p.n. \textsc{du} really \textsc{emph} \textsc{pst}.\textsc{ifr}-\textsc{evd}-be.in.good.terms-\textsc{du} \textsc{lnk} \\
\glt `Maerjina and Alibaba were in very good terms.' (140512 alibaba-zh, 306)
\end{exe}

\begin{exe}
\ex   \label{ex:cho.aCGAtCi}
 \gll  nɤj nɤ-mu cʰo aʑo ni aɕɣa-tɕi \\
 \textsc{2sg} \textsc{2sg}.\textsc{poss}-mother \textsc{comit} \textsc{1sg} \textsc{du} be.of.the.same.age:\textsc{fact}-\textsc{1du} \\
 \glt `I have the same age as your mother.' (`You mother and I have the same age') (elicited)
\end{exe} 

The verb \japhug{acʰɤt}{have X years of difference} has an intrinsically non-singular subject, and is at the same time semi-transitive (§\ref{sec:semi.transitive}), taking as semi-object a temporal noun phrase expressing the age difference between the members of the group referred to by the subject, for instance \forme{ʁnɯ-pɤrme nɤ ʁnɯ-pɤrme} `two years each' in (\ref{ex:RnWpArme.machAti}).

\begin{exe}
\ex   \label{ex:RnWpArme.machAti}
 \gll tɕe iʑo kɤndʑiʁi ra ʁnɯ-pɤrme nɤ ʁnɯ-pɤrme ntsɯ ma mɤ-acʰɤt-i \\
 \textsc{lnk} \textsc{1pl} \textsc{coll}:sibling \textsc{pl} two-year \textsc{lnk} two-year always apart.from \textsc{neg}-differ.in.age:\textsc{fact}-\textsc{1pl} \\
\glt `We brother and sisters were born in two years intervals.' (if ranked by birth order, each couple of adjacent sibling differ in age from each other by two years each) (14-tApitaRi, 243)
 \end{exe}
 
The verb \japhug{alɯlɤt}{fight}, historically the reciprocal of \japhug{lɤt}{throw, release} (§XXX), almost always has non-singular indexation, as in (\ref{ex:qro.ni.YAlWlAtndZi}).


\begin{exe}
\ex   \label{ex:qro.ni.YAlWlAtndZi}
 \gll  tɕeki qro ni ɲɯ-ɤlɯlɤt-ndʑi \\
 down ant \textsc{du} \textsc{sens}-fight-\textsc{du} \\
\glt `Down there two ants are fighting.' (conversation140501-01)
\end{exe}

However, examples with singular indexation are also attested, for instance (\ref{ex:tAtalWlAt}) and (\ref{ex:tutalWlAt}) with \textsc{2sg} form. Both are from texts translated from Chinese, but were not considered infelicitous by Tshendzin. The singular verb forms might be due to calquing, but are not radically ungrammatical.\footnote{The Chinese original sentences of examples (\ref{ex:tAtalWlAt}) and (\ref{ex:tutalWlAt}) are \ch{你是为众人的利益而战}{nǐ shì wèi zhòngrén de lìyì érzhàn}{You were fighting for the interest of the people} and \ch{你不该和舅舅动手}{nǐ bùgāi hé jiùjiù dòngshǒu}{You should not get into a fight with your uncle} respectively.  } In (\ref{ex:tutalWlAt}), instead of using dual indexation, the person with whom the subject fights is marked with the dative.

\begin{exe}
\ex   \label{ex:tAtalWlAt}
 \gll  ki ɕɯŋgɯ ki pɯpɯŋunɤ, nɤʑo kɯ iɕqʰa mkʰɤrmaŋ ɣɯ nɯ-ndʐa kɯ tɤ-tɯ-alɯlɤt pɯ-ŋu tɕe, \\
 \textsc{dem}.\textsc{prox} before \textsc{dem}.\textsc{prox} \textsc{top} \textsc{2sg} \textsc{erg} \textsc{filler} people \textsc{gen} \textsc{3pl}.\textsc{poss}-reason \textsc{erg} \textsc{pfv}-2-fight \textsc{pst}.\textsc{ipfv}-be \textsc{lnk} \\
\glt `The previous time, you fought for the sake of the people.' (140512 abide he mogui-zh, 89)
 \end{exe}

\begin{exe}
\ex   \label{ex:tutalWlAt}
 \gll  tɕe nɤʑo nɤ-rpɯ ɯ-ɕki, nɤkinɯ, tu-tɯ-ɤlɯlɤt ndɤre mɤ-pe \\
 \textsc{lnk} \textsc{2sg} \textsc{2sg}.\textsc{poss}-MB \textsc{3sg}.\textsc{poss}-\textsc{dat} \textsc{filler} \textsc{ipfv}-2-fight \textsc{lnk} \textsc{neg}-be.good:\textsc{fact} \\
 \glt `It is not good for you to fight with your uncle.' (150826 baoliandeng-zh, 185)
  \end{exe}
  
\subsection{Invariable intransitive verbs}
%thɯ-nɯɕe ma, mɤ-tɯ-ra

\section{Transitive verbs}

\subsection{The morphological marking of transitivity in Japhug} \label{sec:transitivity.morphology}

%The past tense \forme{-t} suffix

\subsection{The direct-inverse system} \label{sec:direct-inverse}

\subsubsection{Mixed configurations} \label{sec:indexation.moxed}

\subsubsection{Non-local configurations} \label{sec:indexation.non.local}

\subsubsection{Local configurations} \label{sec:indexation.local}

\subsubsection{Generic indexation} \label{sec:indexation.generic.tr}

\subsubsection{Double number indexation}  \label{sec:double.number.indexation}

\begin{landscape}
\begin{table}[H]
\caption{Japhug transitive and intransitive paradigms}\label{tab:japhug.tr}
\resizebox{\columnwidth}{!}{
\begin{tabular}{l|l|l|l|l|l|l|l|l|l|l|}
\textsc{} & 	\textsc{1sg} & 	  \textsc{1du} & 	\textsc{1pl} & 	\textsc{2sg} & 	\textsc{2du} & 	\textsc{2pl} & 	\textsc{3sg} & 	\textsc{3du} & 	\textsc{3pl} & 	\textsc{3'} \\ 	
\hline
\textsc{1sg} & \multicolumn{3}{c|}{\grise{}} &	\forme{} & 	\forme{} & 	\forme{} & 	\forme{\sigc{}-a}   & 	 \forme{\sigc{}-a-ndʑi} & 	 \forme{\sigc{}-a-nɯ} & 	\grise{} \\	
\cline{8-10}
\textsc{1du} & 	\multicolumn{3}{c|}{\grise{}} &	\forme{ta-\siga{}} & 	\forme{ta-\siga{}-ndʑi} & 	\forme{ta-\siga{}-nɯ} & 	\multicolumn{3}{c|}{ \forme{\siga{}-tɕi}}  & 	\grise{} \\	
\cline{8-10}
\textsc{1pl} & 	\multicolumn{3}{c|}{\grise{}} & 	  & 	&  & 	\multicolumn{3}{c|}{ \forme{\siga{}-ji}}  & 	\grise{} \\	
\hline
\textsc{2sg} & 	\forme{kɯ-\siga{}-a} & 	\forme{} & 	\forme{} & 	\multicolumn{3}{c|}{\grise{}}&	\multicolumn{3}{c|}{\forme{tɯ-\sigc{}}} & 	\grise{} \\	
\cline{2-2}
\cline{8-10}
\textsc{2du} & 	\forme{kɯ-\siga{}-a-ndʑi} & 	\forme{kɯ-\siga{}-tɕi} & 	\forme{kɯ-\siga{}-ji} & 	\multicolumn{3}{c|}{\grise{}} &	\multicolumn{3}{c|}{\forme{tɯ-\siga{}-ndʑi}} & 	\grise{} \\	
\cline{2-2}
\cline{8-10}
\textsc{2pl} & 	\forme{kɯ-\siga{}-a-nɯ} & 	\forme{} & 	\forme{} & 	\multicolumn{3}{c|}{\grise{}}&	\multicolumn{3}{c|}{\forme{tɯ-\siga{}-nɯ}} & 	\grise{} \\	
\hline
\textsc{3sg} &  	\forme{wɣɯ́-\siga{}-a} & 	\forme{} & 	\forme{} & 	\forme{} & 	\forme{} & 	\forme{} & \multicolumn{3}{c|}{\grise{}} &	\forme{\sigc{}} \\ 	
\cline{2-2}
\cline{11-11}
\textsc{3du} &  	\forme{wɣɯ́-\siga{}-a-ndʑi} & 	 \forme{wɣɯ́-\siga{}-tɕi} & 		\forme{wɣɯ́-\siga{}-ji} & 	\forme{tɯ́-wɣ-\siga{}} & 	\forme{tɯ́-wɣ-\siga{}-ndʑi} & 	\forme{tɯ́-wɣ-\siga{}-nɯ} & 	\multicolumn{3}{c|}{\grise{}} &	\forme{\siga{}-ndʑi} \\ 
\cline{2-2}	
\cline{11-11}
\textsc{3pl} &  	\forme{wɣɯ́-\siga{}-a-nɯ} & 	\forme{} & 	\forme{} & 	\forme{} & 	\forme{} & 	\forme{} & \multicolumn{3}{c|}{\grise{}} &	\forme{\siga{}-nɯ} \\ 	
\hline
\textsc{3'} & 	\multicolumn{6}{c|}{\grise{}} &	\forme{wɣɯ́-\siga{}} & 	\forme{wɣɯ́-\siga{}-ndʑi} & 	\forme{wɣɯ́-\siga{}-nɯ} & 	\grise{} \\	
	\hline	\hline
\textsc{intr}&\forme{\siga{}-a}&\forme{\siga{}-tɕi}&\forme{\siga{}-ji}&\forme{tɯ-\siga{}}&\forme{tɯ-\siga{}-ndʑi}&\forme{tɯ-\siga{}-nɯ}&\forme{\siga{}}&\forme{\siga{}-ndʑi} &\forme{\siga{}-nɯ}& 	\grise{} \\	
\hline
\end{tabular}}
\end{table}


\begin{table}[H]
\caption{The paradigm of the verb \japhug{mto}{see} in the Factual non-past}\label{tab:mto.paradigm}
\resizebox{\columnwidth}{!}{
\begin{tabular}{l|l|l|l|l|l|l|l|l|l|l|}
\textsc{} & 	\textsc{1sg} & 	  \textsc{1du} & 	\textsc{1pl} & 	\textsc{2sg} & 	\textsc{2du} & 	\textsc{2pl} & 	\textsc{3sg} & 	\textsc{3du} & 	\textsc{3pl} & 	\textsc{3'} \\ 	
\hline
\textsc{1sg} & \multicolumn{3}{c|}{\grise{}} &	\forme{} & 	\forme{} & 	\forme{} & 	\forme{mtam-a}   & 	 \forme{mtam-a-ndʑi} & 	 \forme{mtam-a-nɯ} & 	\grise{} \\	
\cline{8-10}
\textsc{1du} & 	\multicolumn{3}{c|}{\grise{}} &	\forme{ta-mto} & 	\forme{ta-mto-ndʑi} & 	\forme{ta-mto-nɯ} & 	\multicolumn{3}{c|}{ \forme{mto-tɕi}}  & 	\grise{} \\	
\cline{8-10}
\textsc{1pl} & 	\multicolumn{3}{c|}{\grise{}} & 	  & 	&  & 	\multicolumn{3}{c|}{ \forme{mto-j}}  & 	\grise{} \\	
\hline
\textsc{2sg} & 	\forme{kɯ-mto-a} & 	\forme{} & 	\forme{} & 	\multicolumn{3}{c|}{\grise{}}&	\multicolumn{3}{c|}{\forme{tɯ-mtɤm}} & 	\grise{} \\	
\cline{2-2}
\cline{8-10}
\textsc{2du} & 	\forme{kɯ-mto-a-ndʑi} & 	\forme{kɯ-mto-tɕi} & 	\forme{kɯ-mto-j} & 	\multicolumn{3}{c|}{\grise{}} &	\multicolumn{3}{c|}{\forme{tɯ-mto-ndʑi}} & 	\grise{} \\	
\cline{2-2}
\cline{8-10}
\textsc{2pl} & 	\forme{kɯ-mto-a-nɯ} & 	\forme{} & 	\forme{} & 	\multicolumn{3}{c|}{\grise{}}&	\multicolumn{3}{c|}{\forme{tɯ-mto-nɯ}} & 	\grise{} \\	
\hline
\textsc{3sg} &  	\forme{wɣɯ́-mto-a} & 	\forme{} & 	\forme{} & 	\forme{} & 	\forme{} & 	\forme{} & \multicolumn{3}{c|}{\grise{}} &	\forme{mtɤm} \\ 	
\cline{2-2}
\cline{11-11}
\textsc{3du} &  	\forme{wɣɯ́-mto-a-ndʑi} & 	 \forme{wɣɯ́-mto-tɕi} & 		\forme{wɣɯ́-mto-j} & 	\forme{tɯ́-wɣ-mto} & 	\forme{tɯ́-wɣ-mto-ndʑi} & 	\forme{tɯ́-wɣ-mto-nɯ} & 	\multicolumn{3}{c|}{\grise{}} &	\forme{mto-ndʑi} \\ 
\cline{2-2}	
\cline{11-11}
\textsc{3pl} &  	\forme{wɣɯ́-mto-a-nɯ} & 	\forme{} & 	\forme{} & 	\forme{} & 	\forme{} & 	\forme{} & \multicolumn{3}{c|}{\grise{}} &	\forme{mto-nɯ} \\ 	
\hline
\textsc{3'} & 	\multicolumn{6}{c|}{\grise{}} &	\forme{wɣɯ́-mto} & 	\forme{wɣɯ́-mto-ndʑi} & 	\forme{wɣɯ́-mto-nɯ} & 	\grise{} \\	
\hline
\end{tabular}}
\end{table}
\end{landscape}


\subsection{Ditransitive verbs}

\subsubsection{Indirective}
\subsubsection{Secundative}
%including stu
\subsubsection{Causative}
\subsection{The function of the direct/inverse contrast in non-local configurations}

\subsection{An irregular verb}

\section{Labile verbs}
\subsection{Transitive-intransitive labile verbs}
\subsection{Transitive-intransitive labile verbs with oblique arguments} \label{sec:goal.labile}
%rpu
\subsection{Semi-transitive labile verbs}\label{sec:semi.tr.labile}
%sɤŋo, tso me, rga
\section{Agreement mismatch}
\subsection{Optional number indexation} \label{sec:optional.indexation}
%"ma nɯra aʑo a-pi ŋu-nɯ wo" to-ti.

\subsection{Plural as honorific} \label{sec:honorific.indexation}
%tɕendɤre, wo nɯ kɯ-fse ci tɯ-ŋu, nɯ mɤ-kɯ-naχtɕɯɣ ci tɯ-ŋu tɯ-ŋu-nɯ 
\subsection{Partitive indexation} \label{sec:partitive.indexation}
Unlike with second or third persons, with first persons number indexation is absolutely compulsory. Apparent examples of mismatch however do exist, but are confined to a very specific partitive use of dual of plural number. With the interrogative pronoun \japhug{ɕɯ}{who} (§\ref{sec:CW.pronoun}), in particular, indexation on the verb can be non-singular with the specific partitive meaning `who among $X$', in particular in comparative constructions as in (\ref{ex:CW.kW.YWmpCArtCi}), with \textsc{1du} indexation on the verb although this sentence implies that only one of the two sisters is the most beautiful (see §XXX on this comparative construction, and \citet{jacques16comparative} and §\ref{sec:comparee.kW} on the use of the ergative here).

\begin{exe}
\ex   \label{ex:CW.kW.YWmpCArtCi}
 \gll  a-ʁi, nɤki tɕetʰi tɕe, tɯ-ci ɯ-ŋgɯ ɕ-pɯ-ru tɕe, ɕɯ kɯ ɲɯ-mpɕɤr-tɕi kɯ? \\
 \textsc{1sg}.\textsc{poss}-younger.sibling \textsc{filler} downstream \textsc{loc} \textsc{indef}.\textsc{poss}-water \textsc{3sg}.\textsc{poss}-inside \textsc{transloc}-\textsc{imp}:\textsc{down}-look \textsc{lnk} who \textsc{erg} \textsc{sens}-be.beautiful-\textsc{1sg} \textsc{sfp} \\
 \glt `Sister, go and look down there in the water, who is the most beautiful of us?' (2014-kWlAG, 477)
\end{exe} 


%ɕɯ kɯ nɯ stu ʑo, nɤkinɯ, kɯ-sɤmtshɤr, ʑɯmkhɤm ɯ-ku kɯ-rkɯn kɯ-fse kɤ-ɣɯt kɯ-cha nɯnɯ kɯ,
%kɯki @nuoha nɯ nɯ-rʑaβ ku-tɯ-nɯ-sɯβzu-nɯ jɤɣ" to-ti
%tɯ-rdoʁ kɯ a-sci a-thɯ-ndo-nɯ ntshi ɲɯ-sɯsɤm pjɤ-ŋu 

Another type of partitive indexation is found with \textsc{1pl} pronouns and third person indexation, as in (\ref{ex:iZora.tutinW}), with two verbs in \textsc{3pl}\fl{}3 form and the \textsc{1pl} pronoun in topicalized position meaning `some among us'.

\begin{exe}
\ex   \label{ex:iZora.tutinW}
 \gll  iʑora tɕe ɕkɤpʰɤr tu-ti-nɯ tsuku kɯ ɕkɤjwaʁ tu-ti-nɯ ŋu ma \\
 \textsc{1pl} \textsc{lnk} wild.chives \textsc{ipfv}-say-\textsc{pl} some \textsc{erg}  wild.chives \textsc{ipfv}-say-\textsc{pl} be:\textsc{fact} \textsc{lnk} \\
\glt `Among us, some call it \forme{ɕkɤpʰɤr}, some \forme{ɕkɤjwaʁ}.'(07-Cku, 82)
\end{exe} 

\subsection{First person and generic}

Another type of agreement mismatch observed with first person concerns \textsc{1pl} and generic person. In (\ref{ex:tWZAra.pWxtCij}), the adjectival stative verb \japhug{xtɕi}{be small} bears \textsc{1pl} indexation, but the corresponding overt pronoun in the sentence is the generic person \japhug{tɯʑɤra}{one} (§\ref{sec:genr.pro}). The generic form \forme{pɯ-kɯ-xtɕi}, as in (\ref{ex:pWkWxtCi.tCe.pWwGmto}) would be possible in the exactly the same context, clearly including the first person.

\begin{exe}
\ex   \label{ex:tWZAra.pWxtCij}
 \gll tɯʑɤra pɯ-xtɕi-j tɕe, \\
 \textsc{genr} \textsc{pst}.\textsc{ipfv}-be.small-\textsc{1pl} \textsc{lnk} \\
 \glt `When we were young.' (17-ndZWnW, 52)
\end{exe}

\begin{exe}
\ex   \label{ex:pWkWxtCi.tCe.pWwGmto}
 \gll tɕe jinde aj pɯ-mto-t-a me ri, pɯ-kɯ-xtɕi tɕe pɯ́-wɣ-mto \\
 \textsc{lnk} now \textsc{1sg} \textsc{pfv}-see-\textsc{pst}:\textsc{tr}-\textsc{1sg} not.exist:\textsc{fact} \textsc{lnk} \textsc{pst}.\textsc{ipfv}-\textsc{genr}:S/P-be.small \textsc{lnk} \textsc{pfv}-\textsc{inv}-see \\
\glt `I have not seen it recently, but when we were young, we did see it.' (22-qomndroN, 35)
\end{exe}

Conversely, in (\ref{ex:iZo.ci.YWwGphWt}), the \textsc{1pl} pronoun \forme{iʑo} seems to agree with a verb in transitive subject generic form (\ref{sec:indexation.generic.tr}). However, here the use of the generic form with the imperfective on the verb implies a slight deontic or gnomic meaning; the same configuration is also found in procedural texts as in (\ref{ex:iZo.luWGnWBzu}).

\begin{exe}
\ex   \label{ex:iZo.ci.YWwGphWt}
 \gll  iʑo kɯ-mɤku pɤjkʰu, ɯ-ɕɣa kɯ-mtɕoʁ nɯ ci ɲɯ́-wɣ-pʰɯt \\
 \textsc{1pl} \textsc{nmlz}:S/A-be.first still \textsc{3sg}.\textsc{poss}-tooth \textsc{nmlz}:S/A-be.sharp \textsc{dem} a.little \textsc{ipfv}-\textsc{inv}-take.out \\
\glt `Let us first take out its sharp teeth.' (150908 menglang-zh, 80)
\end{exe}

\begin{exe}
\ex   \label{ex:iZo.luWGnWBzu}
 \gll iʑo kɯrɯ ra, nɤkinɯ, qajɣi lú-wɣ-nɯ-βzu tɕe \\
\textsc{1pl} Tibetan \textsc{pl} \textsc{filler} bread \textsc{ipfv}-\textsc{inv}-\textsc{auto}-make \textsc{lnk} \\
\glt `We Tibetans, when we make bread,' (160706 thotsi, 1)
\end{exe}
\section{The historical relationshop between person indexation suffixes and possessive prefixes} \label{sec:indexation.suffixes.history}

\section{The origin of portmanteau prefixes}

\section{Person indexation on non-finite predicative words} \label{sec:non.finite.indexation}
