\chapter{Pronouns and Demonstratives}
\section{Personal pronouns} \label{sec:pers.pronouns}

%\\ipa\{([\w-]*)\}
%\1

%(\d)\\(\w\w)\{\}
%\\textsc{\1\2}

The pronominal system of Japhug distinguishes singular, dual and plural. Alongside the free pronouns, a system of pronominal prefixes is used not only to express possession on noun (see § XXX for an account of the possessive constructions), but also appears in various constructions in the verbal system. These prefixes do not distinguish the second and the third person in the dual and plural forms; their use is described in section \ref{sec:possessive.paradigm}.

\begin{table}[h] \centering
\caption{Pronouns and possessive prefixes }\label{tab:pronoun}
\begin{tabular}{lllllllll} \lsptoprule
 Free pronoun & Prefix & \\
\midrule
 \forme{aʑo},    \forme{aj} &	\forme{a-}  &		1\sg{} \\
\forme{nɤʑo},  \forme{nɤj} &	\forme{nɤ-}  &			2\sg{} \\
\forme{ɯʑo}  &	\forme{ɯ-}  &			3\sg{} \\
\midrule
\forme{tɕiʑo}  &	\forme{tɕi-}  &			1\du{} \\
\forme{ndʑiʑo}  &	\forme{ndʑi-}  &		2\du{} \\	
\forme{ʑɤni}  &	\forme{ndʑi-}  &		3\du{} \\	
\midrule
\forme{iʑo}, \forme{iʑora},   \forme{iʑɤra}   &	\forme{i-}  &			1\pl{} \\
\forme{nɯʑo}, \forme{nɯʑora},   \forme{nɯʑɤra}  &	\forme{nɯ-}  &			2\pl{} \\
\forme{ʑara}  &	\forme{nɯ-}  &			3\pl{} \\
\lspbottomrule
\end{tabular}
\end{table}

Free pronouns and possessive prefixes are remarkably similar in Kamnyu Japhug. In the eastern Japhug dialects, a different \textsc{1sg} pronoun distinct from the possessive prefix  is used: \forme{ŋa} (possibly borrowed from Situ \forme{ŋā}). In the table above, we observe that all the pronouns except the third person dual and plural are formed by adding the root \forme{-ʑo} to the corresponding possessive prefix. 

The first and second person singular pronouns  \forme{aʑo} and \forme{nɤʑo} also have the shorter monosyllabic forms \forme{aj} and \forme{nɤj} respectively. These short forms are considerably less common in stories (in the reported speech of the characters), but appear frequently in free conversations.

Japhug lacks any inclusive / exclusive distinction, unlike other Gyalrongic languages such as Tshobdun, Situ or Khroskyabs (see \citealt{jackson98morphology}, \citealt[177]{linxr93jiarongen}, \citealt[92]{prins16kyomkyo}, \citealt[170]{lai17khroskyabs}). Example  (\ref{ex:tCiZo.CetCi}) shows the dual pronoun \japhug{tɕiʑo}{we (dual)} in inclusive use (it is clear from the context that the son tells his mother to come with him), and (\ref{ex:tCiZo.tCitAYi}) illustrates the same pronoun in exclusive use. Similar pairs of examples can be found with the first plural pronoun \japhug{iʑo}{we (plural} and its variants.

\begin{exe}
\ex \label{ex:tCiZo.CetCi}
\gll a-mu tɕetʰa tɕiʑo kɯnɤ ɕe-tɕi \\
\textsc{1sg.poss}-mother later \textsc{1du} also go:\textsc{fact}-\textsc{1du} \\
\glt `Mother, you and I will go too.' (2003tWxtsa, 138)
\end{exe}

\begin{exe}
\ex \label{ex:tCiZo.tCitAYi}
\gll nɯʑora ɣɯ nɯ-ɕɤmɯɣdɯ cʰo kɯ-fse nɯ ɯ-tsʰɤt nɯ, tɕiʑo ɣɯ tɕi-tɤɲi tɯ-ldʑa pɯ-tu tɕe, nɯ kɤ-nɯ-tʰɯ-tɕi ɕti wo \\
\textsc{2pl} \textsc{gen} \textsc{2pl.poss}-gun \textsc{comit} \textsc{nmlz}:S/A-be.like \textsc{dem} \textsc{3sg}-instead \textsc{dem} \textsc{1du} \textsc{gen} \textsc{1du.poss}-staff one-\textsc{cl} \textsc{pst.ipfv}-exist \textsc{lnk} \textsc{dem} \textsc{pfv}-\textsc{auto}-spread-\textsc{1du} be:\textsc{affirm}:\textsc{fact} \textsc{sfp} \\
\glt `Instead of guns and other things like you, we only had a staff, and we used it as a bridge (to cross the river).' (2003kunbzang, 164)
\end{exe}

Third person pronouns can be used with inanimate referents, as the third person dual \forme{ʑɤni} in example (\ref{ex:rNgW}).

\begin{exe}
\ex \label{ex:rNgW}
\gll tɕe   	rŋgɯ   	nɯ   	to-k-ɤmɯrpu-ndʑi-ci   	tɕe,   	tɕendɤre   	ʑɤni   	pjɤ-nɯ-ɴɢrɯ-ndʑi   \\
\textsc{lnk} boulder \textsc{dem} \textsc{ifr-evd}-bump.into:\textsc{recip}-\textsc{du-evd} \textsc{lnk} \textsc{lnk} \textsc{3du} \textsc{ifr-auto}-crush-\textsc{du} \\
\glt `The boulders bumped into each other and they were pulverized.' (smanmi4.82-83)
\end{exe}

In some contexts, demonstrative pronouns rather than person pronouns are used to refer to a third person, even human (see § \ref{sec:demonstrative.pronouns}).

Personal pronouns are not used as head of relative clauses (as in Chinese \zh{……的你} `you who are ...'), though there are case of relativization of first or second person possessor, as in (\ref{ex:amu.kWme}) (see § XXX).

\begin{exe}
\ex \label{ex:amu.kWme}
\gll aʑo nɯ a-mu kɯ-me ŋu-a tɕe tɕe \\
1sg \textsc{dem} \textsc{1sg.poss}-mother \textsc{nmlz}:S/A-exist be:\textsc{fact-1sg} \textsc{lnk} \textsc{lnk} \\
\glt `I am someone who does not have a mother.' (2003Nyimawodzer2, 12)
\end{exe}

Personal pronouns can take determiners, in particular the demonstrative \forme{nɯ} as in (\ref{ex:amu.kWme}), numerals (§ \ref{sec:uses.numerals}) and can also precede a noun in apposition, in expressions such as \forme{iʑo kɯrɯ} `we, Tibetans' (\ref{ex:iZo.kWrW}) or \forme{nɤʑo qaɕpa} `you frog' in (\ref{ex:nAZo.qaCpa}).

\begin{exe}
\ex \label{ex:iZo.kWrW}
\gll
iʑo kɯrɯ tɕe pɤjka tu-nɯ-ti-j ŋu tɕe, \\
\textsc{1pl} Tibetan \textsc{lnk} species.of.squash \textsc{ipfv}-\textsc{auto}-say-\textsc{1pl} \textsc{lnk} \\
\glt `We Tibetans call it \forme{pɤjka}.' (16-CWrNgo, 71)
\end{exe}

\begin{exe}
\ex  \label{ex:nAZo.qaCpa}
\gll  nɯ-nɯ-nɤre ma nɤʑo qaɕpa nɤ-rʑaβ nɤ-kɯ-mbi kɯ-tu me   \\
\textsc{imp-auto}-laugh \textsc{lnk} \textsc{2sg} frog \textsc{2sg.poss}-wife \textsc{2sg.poss}-\textsc{nmlz}:S/A-give \textsc{nmlz}:S/A-exist not.exist:\textsc{fact} \\
\glt `Laugh as you wish, nobody will give you a wife, you frog.'   (2002 qaCpa, 176)
\end{exe} 

Personal pronouns occur as member of compounds only in one construction: with the root \forme{-sɯso} `as X wish' (from the verb \japhug{sɯso}{think}), as in example (\ref{ex:ʑara.sWso}). This construction is discussed in more detail in section XX.

\begin{exe}
\ex \label{ex:ʑara.sWso}
\gll a-zda ra ʑara-sɯso tu-nɯ-nɤŋkɯŋke-nɯ ɲɯ-kʰɯ \\
\textsc{1sg.poss}-companion \textsc{pl} \textsc{3pl}-as.wish \textsc{ipfv}-\textsc{auto}-go.here.and.there-\textsc{pl} \textsc{sens}-be.possible \\
\glt `The other (snakes) can go here and there as they wish.' (The divination, 43)
\end{exe}

As in most languages with polypersonal indexation, pronouns (especially first and second person pronouns) are never obligatory, and a finite verb form without overt argument NPs is a perfectly well-formed sentence (see § XXX). 

\section{Generic pronoun}  \label{sec:genr.pro}
The generic pronoun \japhug{tɯʑo}{one} has the same morphological structure as personal pronouns as seen in the previous section, combining the generic possessive prefix \ipa{tɯ-} with the pronominal root \forme{-ʑo}. Note that this generic possessive has to be strictly distinguished from the homophonous indefinite possessive prefix \forme{tɯ-} (see § \ref{sec:indef.genr.poss}).

In Japhug, sentences have at most one generic human referent. If this referent is core argument, the verb has generic indexation (\forme{kɯ-} for S/P, \forme{wɣ-} for A, as in the following examples; see also section XXX). The generic argument can be realized as the generic pronoun \forme{tɯʑo} as in (\ref{ex:pjWkWZGAGANgi}) or by a generic noun (such as \japhug{tɯrme}{person}, see § XXX).

\begin{exe}
\ex \label{ex:pjWkWZGAGANgi}
\gll tɯ-zda pjɯ́-wɣ-z-ɣɤtɕa, \textbf{tɯʑo}  ntsɯ  pjɯ-kɯ-ʑɣɤ-ɣɤŋgi   	tɕe,  pɯ-kɯ-nɯ-ɣɤtɕa 	kɯ́nɤ   	pjɯ-kɯ-ʑɣɤ-ɣɤŋgi   	tɕe,    ɯ-mbrɤzɯ   	kɯ-tu   	me  	tu-kɯ-ti   	ɲɯ-ŋu.   \\
\textsc{genr.poss}-companion \textsc{ipfv-inv-caus}-be.wrong oneself always \textsc{ipfv-genr:S/P-refl}-be.right \textsc{lnk} \textsc{pfv-genr:S/P-auto}-be.wrong also \textsc{ipfv-genr:S/P-refl}-be.right lnk \textsc{3sg.poss}-result \textsc{nmlz:S/A}-have  not.exist:\textsc{fact} \textsc{ipfv-genr}-say \textsc{sens}-be \\
\glt  `If one considers that one's companion is wrong, and always considers himself to be right even if one is wrong, there is can be no good result.' (Mouse and sparrow, 80-82)
\end{exe} 

The generic pronoun can occur before a noun with the generic possessive as in \forme{tɯʑo tɯ-skɤt}  `one's language' in example (\ref{ex:tWZo.tWskAt}); this contributes to disambiguating between the indefinite possessive and the generic possessive in the case of inalienably possessed nouns (thus on its own \forme{tɯ-skɤt} can mean either `a language' or `one's language').

\begin{exe}
\ex \label{ex:tWZo.tWskAt}
\gll tɕendɤre tɯʑo tɯ-skɤt ʑara ɣɯ-sɯxɕɤt ɲɯ-ra, ʑara nɯ-skɤt tɯʑo kɯ-sɯxɕɤt ɲɯ-ra \\
\textsc{lnk} \textsc{genr} \textsc{genr.poss}-language \textsc{3pl} \textsc{inv}-teach \textsc{sens}-have.to \textsc{3pl} \textsc{3pl.poss}-language \textsc{genr} \textsc{genr}:S/P-teach \textsc{sens}-have.to \\
\glt `One has to teach them one's language, and they have to teach you their language.'  (150901 tshuBdWnskAt, 29)
\end{exe} 

When occurring in A function, the generic pronoun \forme{tɯʑo} obligatorily receives the ergative \forme{kɯ} as in (\ref{ex:tWZo.kW}) (note that in example \ref{ex:tWZo.tWskAt}, although the generic referent is A in the first clause, \forme{tɯʑo} does not take ergative because it is a determiner of \forme{tɯ-skɤt}). 

\begin{exe}
\ex \label{ex:tWZo.kW}
\gll tɯʑo kɯ tɯ-χti ɲɯ́-wɣ-nɯ-ɕar kɯ-maʁ kɯ,  tɯ-pʰama ra kɯ tɯ-χti ɲɯ-ɕar-nɯ tɕe tɯ-sɯm pɯ-a<nɯ>ri nɤ ju-kɯ-ɕe,
mɯ-pɯ-a<nɯ>ri nɤ ju-kɯ-ɕe pɯ-ra \\
\textsc{genr} \textsc{erg} \textsc{genr.poss}-spouse \textsc{ipfv-inv-auto}-search \textsc{inf:stat}-not.be \textsc{erg} \textsc{genr.poss}-parent \textsc{pl} \textsc{erg} \textsc{genr.poss}-spouse \textsc{ipfv}-search-\textsc{pl} \textsc{lnk} \textsc{genr.poss}-mind \textsc{pst.ipfv}-<\textsc{auto}>go[II] \textsc{lnk} \textsc{ipfv}-\textsc{genr}:S/P-go, \textsc{neg-pst.ipfv}-<\textsc{auto}>go[II] \textsc{lnk} \textsc{ipfv-genr}:S/P-go \textsc{pst.ipfv}-have.to \\
\glt `One could not choose one's spouse, one's parents chose one's spouse, and one had to go whether one liked it or not.' (14-tApitaRi, 212-215)
\end{exe} 

Other cases like dative are treated like inalienably possessed nouns (see § XXX); for instance, when the generic argument is in the dative, the forms \forme{tɯ-ɕki} or \forme{tɯ-pʰe} `to one' occur, with no indexation on the verb selecting this dative argument, as in example (\ref{ex:tWZo.tWCki}).

\begin{exe}
\ex \label{ex:tWZo.tWCki}
\gll
tɯ-ɲi ɣɯ ɯ-rɟit nɯra kɯ tɯʑo tɯ-ɕki ``a-rpɯ", tɤ-tɕɯ pɯ-kɯ-ŋu nɤ ``a-rpɯ" tu-ti-nɯ, tɕʰeme pɯ-kɯ-ŋu nɤ ``a-ɬaʁ" tu-ti-nɯ kɯ-ra ŋu \\
\textsc{genr.poss}-FZ \textsc{gen} \textsc{3sg.poss}-child \textsc{dem:pl} \textsc{erg} \textsc{genr} \textsc{genr}-dat \textsc{1sg.poss}-MB \textsc{indef.poss}-son \textsc{pst.ipfv-genr}:S/P-be if \textsc{1sg.poss}-MB \textsc{ipfv}-say-\textsc{pl} girl \textsc{pst.ipfv-genr}:S/P-be  if \textsc{1sg.poss}-MZ \textsc{ipfv}-say-\textsc{pl} \textsc{inf.stat}-have.to be:\textsc{fact} \\
\glt `One's father's sister's children have to call oneself `my mather uncle' if one is a boy, `my mather aunt' if one is a girl.'  (see § XXX about Omaha kinship, 140425kWmdza03, 1)
\end{exe} 

As examples (\ref{ex:pjWkWZGAGANgi}) to (\ref{ex:tWZo.tWCki}) illustrate, generic agreement between pronoun, possessive prefix and verb indexation is very systematic, and suffers no exception.


\section{Genitive forms} \label{sec:pronouns.gen}
The form of pronouns and personal prefixes undergoes few morphophonological changes in combination with postpositions and relational nouns. However, in combination with the genitive postposition \forme{ɣɯ} (cf XXX), some  personal pronouns have special forms indicated in Table  \ref{tab:pronoun.gen}.

\begin{table}[h] \centering
\caption{Pronouns and possessive prefixes }\label{tab:pronoun.gen}
\begin{tabular}{lllllllll} \lsptoprule
 Free pronoun & Genitive & \\
\midrule
 \forme{aʑo}  &	\forme{aʑɯɣ}  &		\textsc{1sg} \\ 
\forme{nɤʑo}  &	\forme{nɤʑɯɣ}  &			\textsc{2sg} \\ 
\forme{ɯʑo}  &	\forme{ɯʑɤɣ}  &			\textsc{3sg} \\ 
\forme{tɕiʑo}  &	\forme{tɕiʑɤɣ}  &			\textsc{1du} \\ 
\forme{ndʑiʑo}  &	\forme{ndʑiʑɤɣ}  &		\textsc{2du} \\	 
\forme{ʑɤni}  &	\forme{ʑɤniɣɯ}  &		\textsc{3du} \\	 
\forme{iʑo}  &	\forme{iʑɤɣ}, 	\forme{iʑɤra ɣɯ}   &			\textsc{1pl} \\ 
\forme{nɯʑo}  &	\forme{nɯʑɤɣ}, 	\forme{nɯʑɤra ɣɯ}  &			\textsc{2pl} \\ 
\forme{ʑara}  &	\forme{ʑaraɣ},   \forme{ʑara ɣɯ}&			\textsc{3pl}  \\  
\lspbottomrule
\end{tabular}
\end{table}

While some degree of variation exists with dual and plural pronouns (for instance the regular \forme{iʑo ɣɯ} is found alongside \forme{iʑɤɣ} and \forme{iʑɤra ɣɯ}), for the singular pronouns only one form is attested.

\begin{exe}
\ex
\gll aʑɯɣ 	ndʐa 	ŋu 	ɕi, 	nɤʑɯɣ 	ndʐa 	ŋu, 	aj 	mɯ́j-tso-a   \\
\textsc{1sg:gen} reason be:\textsc{fact} \textsc{qu} \textsc{2sg:gen} reason be:\textsc{fact} \textsc{1sg} \textsc{neg:sens}-understand-\textsc{1sg} \\
\glt  `I don't know if it is because of me, or because of you.' (that the phone line is not working well) (phone conversation, 2011) %\wav{8_ndzxa})
\end{exe} 

In the genitive forms of the pronouns, the vowel of the genitive marker is generally dropped, and the pronominal root \forme{-ʑo} undergoes vowel change to \forme{-ʑɯɣ} (in the case of first and second person) and \forme{-ʑɤɣ} (in other forms). Note that \forme{ʑaraɣ} is the only case of the rhyme \ipa{aɣ} in Japhug.

When genitive pronouns occur as determiners of nouns (including in the possessive existential construction, see § XXX), these nouns almost always take a possessive prefix coreferent with the genitive pronoun, as in (\ref{ex:tɕithAfkAlAGi}).

\begin{exe}
\ex \label{ex:tɕithAfkAlAGi}
\gll 
tɕiʑɤɣ tɕi-tʰɤfkɤlɤɣi tɯ-ɕkat pɯ-tu tɕe, nɯ kɤ-nɯ-χtɤr-tɕi ɕti wo \\
\textsc{1du:gen} 2du-plant.ash one-load \textsc{pst.ipfv}-exist \textsc{lnk} \textsc{dem} \textsc{pfv-auto}-spread-\textsc{1du} be:\textsc{affirm}:\textsc{fact} \textsc{sfp} \\
\glt `We had one load of plant ash, and spill it there.' (2003kunbzang, 171)
\end{exe} 

The genitive pronouns can be used as possessive pronouns (`mine', `my own' etc) and take the determiner \forme{nɯ} and the plural \forme{ra}, as in (\ref{ex:aZWG.nW}) and (\ref{ex:WZAG.nWra}).

\begin{exe}
\ex \label{ex:aZWG.nW}
\gll ``tɕe ɣnɤsqaptɯ-rʑaʁ tu-tsu tɕe ɲɯ-ʁaʁ ŋu" ɲɯ-ti-nɯ ri, aʑɯɣ nɯ ɣnɤsqamnɯz tɤ-rʑaʁ mɤɕtʂa mɯ-nɯ-ʁaʁ. \\
\textsc{lnk} eleven-night \textsc{ipfv}-pass \textsc{lnk} \textsc{ipfv}-hatch be:\textsc{fact} \textsc{sens}-say-\textsc{pl} \textsc{lnk} \textsc{1sg:gen} \textsc{dem} twelve one-night  until \textsc{neg-pfv}-hatch \\
\glt `People say that (chicken eggs) hatch after eleven days, mine took twelve days to hatch.' (150819 kumpGa)
\end{exe} 

\begin{exe}
\ex \label{ex:WZAG.nWra}
\gll ɯʑɤɣ nɯra tu-nɯ-ɣɤ-βdi tɕe, ɕɯ-sɤ-sqɤr mɤ-ra \\
\textsc{3sg:gen} \textsc{dem:pl} \textsc{ipfv-auto-caus}-be.good \textsc{lnk} \textsc{transloc-antipass}-hire \textsc{neg}-have.to:\textsc{fact} \\
\glt `He repairs his own (machines) himself, he does not need to ask other people.' (14-tApitaRi, 168)
\end{exe} 

\section{The emphatic use of pronouns} \label{sec:pronouns.emph}
In addition to their referential and anaphoric functions, pronouns in Japhug can be used in an emphatic way in combination with the particle \forme{ʑo}, as in  (\ref{ex:WZo.Zo}).

\begin{exe}
\ex \label{ex:WZo.Zo}
\gll aʑo ɯʑo ʑo kɤ-mto mɯ-pɯ-rɲo-t-a. \\
\textsc{1sg} \textsc{3sg} \textsc{emph} \textsc{inf}-see \textsc{neg-pfv}-experience-\textsc{pst:tr-1sg} \\
\glt `I never saw it itself.' (24-kWmu, 7)
\end{exe} 

In combination with the autobenefactive \forme{nɯ-} on the verb, pronouns express the meaning `do X on one's own'. In the case of transitive verbs, the pronoun in this use does not take the ergative even if the referent is the transitive subject (example \ref{ex:pjWnWtCAtnW}, where \japhug{tɕɤt}{take out} is transitive).

\begin{exe}
\ex
\gll tɕe ɲɯ-tɯ-nɤm qhe, tɕe ʑara ku-nɯ-nɯɣi-nɯ ŋu ɕi? \\
\textsc{lnk} \textsc{ipfv:east}-2-chase[III] \textsc{lnk} \textsc{lnk} \textsc{3pl} \textsc{ipfv:west}-auto-come.back-\textsc{pl} be:\textsc{fact} \textsc{qu} \\
\glt `Do you chase them, or do they come back home on their own?' (taRrdo conversation, 29)
\end{exe} 

\begin{exe}
\ex \label{ex:pjWnWtCAtnW}
\gll tɕe lu-nɯ-rɤji-nɯ tɕe, nɯ-kɤ-ndza nɯra ʑara pjɯ-nɯ-tɕɤt-nɯ pjɤ-ŋu tɕe \\
\textsc{lnk} \textsc{ipfv-auto}-plant.crops-\textsc{pl} \textsc{lnk} \textsc{3pl.poss-nmlz:P}-eat \textsc{dem:pl} \textsc{3pl} \textsc{ipfv-auto}-take.out-pl \textsc{ifr.ipfv}-be \textsc{lnk} \\
\glt `They planted crops, and earned their food on their own.' (about lepers, who were settled in the special place by the government, 25-khArWm, 70)
\end{exe}

This construction is also attested with first of second person pronouns, as in (\ref{ex:aZo.Zo}).

\begin{exe}
\ex \label{ex:aZo.Zo}
\gll aʑo ʑo nɯnɯ ɕ-pjɯ-sat-a ra \\
\textsc{1sg} \textsc{emph} \textsc{dem} \textsc{transloc-ipfv}-kill-\textsc{1sg} have.to:\textsc{fact} \\
\glt `I have to kill her myself.' (140504 baixuegongzhu, 117)
\end{exe}

The emphatic pronoun \japhug{raŋ}{oneself} borrowed from Tibetan \tibet{རང་}{raŋ}{oneself}, can also be used with any person, though this usage is not very common. It can occur with the autobenefactive (\ref{ex:aZo.raN}) or without it (\ref{ex:aZo.raN.Zo}).

\begin{exe}
\ex \label{ex:aZo.raN}
\gll
nɤʑo tu-tɯ-ti mɤ-ra ma aʑo raŋ tu-nɯ-ti-a jɤɣ \\
\textsc{2sg} \textsc{ipfv}-2-say \textsc{neg}-have.too:\textsc{fact} \textsc{lnk} \textsc{1sg} oneself \textsc{ipfv}-\textsc{auto}-say-\textsc{1sg} be.possible:\textsc{fact} \\
\glt `You don't need to say it, I can say it myself.' (elicited)
\end{exe}

\begin{exe}
\ex \label{ex:aZo.raN.Zo}
\gll aʑo raŋ ʑo ju-ɕe-a ra \\
\textsc{1sg} oneself \textsc{emph} \textsc{ipfv}-go-\textsc{1sg} have.to:\textsc{fact} \\
\glt `I have to go there myself.' (150830 afanti-zh, 96)
\end{exe}
\section{Interrogative pronouns}
The interrogative pronouns in Japhug are indicated in Table \ref{tab:interrog.pronoun}. These pronouns are used in independent interrogative clauses (\ref{ex:tChi.pWNu}), in subordinate clauses (\ref{ex:tChi.kWNu}), and also in correlatives (\ref{ex:NotCu.WsAzrAZi}), and also occur to express non-specific referents (these uses are described in section  \ref{sec:interrogative.indef}, after the indefinite pronouns).

\begin{exe}
\ex \label{ex:tChi.pWNu}
\gll
tɕe mɤʑɯ tɕʰi pɯ-ŋu? \\
\textsc{lnk} yet what \textsc{pst.ipfv}-be \\
\glt `What was there (after this one)?' (12-ndZiNgri, 100)
\end{exe}  

\begin{exe}
\ex \label{ex:tChi.kWNu}
\gll ɯʑo tɕʰi kɯ-ŋu nɯ ko-tso-nɯ tɕe tɕe cʰɤ́-wɣ-tɕɤt \\
\textsc{3sg} what \textsc{nmlz}:S/A-be \textsc{dem} \textsc{ifr}-understand-\textsc{pl} \textsc{lnk} \textsc{lnk} \textsc{ifr:downstream-inv}-take.out \\
\glt `They understood what he was, and expelled him (from their group).' (140427 hanya yu gezi-zh, 19)
\end{exe}  

\begin{exe}
\ex \label{ex:NotCu.WsAzrAZi}
\gll 
ɯ-pʰoŋbu tɕʰi kɯ-fse nɯ, ŋotɕu ɯ-sɤz-rɤʑi nɯnɯ ɣɯ kɯ-nɯtsa kɯ-fse ɲɯ-ɕti tɕe \\
3sg.poss-body what \textsc{nmlz}:S/A-be.like \textsc{dem} where \textsc{3sg-nmlz:oblique}-remain \textsc{dem} \textsc{gen}  nmlz:S/A-fit \textsc{nmlz}:S/A-be.like \textsc{sens}-be:\textsc{affirm} \textsc{lnk} \\
\glt `The way its body is like is well-fitted to the place where it lives.' (19-rNamoN, 24)
\end{exe}  

\begin{table}[h] \centering
\caption{Interrogative pronouns }\label{tab:interrog.pronoun}
\begin{tabular}{lllllllll} \lsptoprule
\japhug{tɕʰi}{what} \\
\japhug{ɕɯ}{who} \\
\japhug{tʰɤstɯɣ}{how many} \\
\japhug{tʰɤjtɕu}{when} \\
\japhug{ŋotɕu}{where}, \japhug{ŋoj}{where} \\
\japhug{tɕʰindʐa}{why} \\
\lspbottomrule
\end{tabular}
\end{table}


\subsection{\japhug{tɕʰi}{what}} \label{sec:tChi}
The interrogative pronoun  `what' considerably varies across Japhug dialects. In Kamnyu we find \forme{tɕʰi}, apparently borrowed from Tibetan \forme{tɕʰi}. Neighbouring dialects of Gdongbrgyad area have either \forme{tsʰi} (in Mangi) or \forme{tʰi} (in Rqaco), which represents the original Rgyalrongic root for this interrogative pronoun (cognate with Tibetan \tibet{ཆི་}{tɕʰi}{what} and Limbu \forme{the}). Even in the Kamnyu dialect, the form \forme{tsʰi-} is directly attested in the indefinite \japhug{tsʰitsuku}{some} (\ref{sec:tshitsuku}). Mangi Japhug shares with Kamnyu the sound change \forme{*tʰi} \fl{} \forme{tsʰi} which also affects the verb \japhug{tsʰi}{drink} (this sound change occurred after the pronoun  \forme{*tʰi} underwent \textit{status constructus} alternation to \forme{tʰɯ-} and was used to build the indefinite pronoun \japhug{tʰɯci}{something}, see \ref{sec:thWci}).

The Eastern dialects of Gsardzong and Datshang have \forme{xto} instead, a word of unknown etymology.

In the Kamnyu dialect, \japhug{tɕʰi}{what} is by far the most common interrogative pronoun in the corpus. In interrogative clauses, it can be used to ask about objects, non-human animals (\ref{ex:nAmbro}) and names of persons (\ref{ex:tChi.tWrmi}).

\begin{exe}
\ex \label{ex:nAmbro}
\gll
nɤʑo nɤ-mbro nɯ tɕʰi ŋu \\
\textsc{2sg} 2sg.poss-horse \textsc{dem} what be:\textsc{fact} \\
\glt `Who is your horse?' (about a sentient horse, 2003smanmi-tamu, 53)
\end{exe}  

\begin{exe}
\ex \label{ex:tChi.tWrmi}
\gll tɕʰi tɯ-rmi? \\
what 2-be.called:\textsc{fact} \\
\glt `What is your name?' (heard in context)
\end{exe}  

As in many languages, this interrogative pronoun (instead of the pronoun \japhug{ɕɯ}{who}) is also used in questions about classification of persons (\citealt{idiatov07nonselective}), including social affiliation (\ref{ex:tChi.WrWG}, and \ref{ex:tChi.kWNu} above) and biological affiliation (\ref{ex:tChi.tosci}).

\begin{exe}
\ex \label{ex:tChi.WrWG}
\gll ɯtɤz nɯʑo tɕʰi ɯ-rɯɣ tɯ-ŋu-nɯ? \\
finally \textsc{2pl} what  3sg.poss-race  2-be:\textsc{fact}-\textsc{pl} \\
\glt `Finally, what race (of being) are you?' (smanmi2003, 172)
\end{exe}  

There is no specific interrogative pronoun to ask about manner like English `how', and Japhug expresses this meaning by combining \forme{tɕʰi} with the verbs \japhug{fse}{be like...} or \japhug{stu}{do like...}, as in examples (\ref{ex:tChi.tAtWfsendZi}), (\ref{ex:tChi.atAfsej}) and (\ref{ex:tChi.Zo.tuwGBzu}).

\begin{exe}
\ex \label{ex:tChi.tAtWfsendZi}
\gll a-ʁi, ki kɯ-fse tɤjpɣom kɯ-wxti nɯtɕu, kɤ-ɕe tɕʰi tɤ-tɯ-fse-ndʑi?? \\
\textsc{1sg.poss}-younger.sibling this \textsc{nmlz}:S/A-be.like ice \textsc{nmlz}:S/A-be.big \textsc{dem:loc} inf-go what \textsc{pfv}-2-be.like-\textsc{du} \\
\glt `Sister, how did you cross such a big block of ice?' (stodtWphu2005, 156)
\end{exe}  

   \begin{exe}
\ex \label{ex:tChi.atAfsej}
\gll  kɤ-pʰɣo tɕʰi a-tɤ-fse-j    \\
\textsc{inf}-flee what \textsc{irr-pfv}-be.like-\textsc{1pl} \\
\glt  `How will we flee?' (Norbzang 69)
\end{exe} 

\begin{exe}
\ex \label{ex:tChi.Zo.tuwGBzu}
\gll nɤ-smɤn tɤ-sɯ-βzu-t-a ri maka mɯ́j-pʰɤn, tɕe tɕʰi ʑo tú-wɣ-stu pʰɤn \\
\textsc{3sg.poss}-medicine \textsc{pfv-caus}-make-\textsc{pst:tr-1sg} but at.all \textsc{neg:sens}-be.efficient \textsc{lnk} what \textsc{emph} \textsc{ipfv-inv}-do.like be.efficient:\textsc{fact} \\
\glt `I had medicine made for you but it does not work, how should we do for it to work?' (nyima wodzer 2002, 22) 
\end{exe}  

 
The pronoun \japhug{tɕʰi}{what} on its own can occur in questions about the reason or the purpose of a particular state of affair, as in (\ref{ex:tChi.apWNua}) and (\ref{ex:tChi.YWtWnAre}).

\begin{exe}
\ex \label{ex:tChi.apWNua}
\gll  aʑo tɕʰi a-pɯ-ŋu-a? \\
\textsc{1sg} what \textsc{irr-ipfv}-be-1sg \\
\glt `How can it be me?' (2003sras, 61)
\end{exe}  

\begin{exe}
\ex \label{ex:tChi.YWtWnAre}
\gll  a-tɤɕime, tɕʰi ɲɯ-tɯ-nɤre ŋu? \\
 \textsc{1sg.poss}-lady what \textsc{sens}-2-laugh be:\textsc{fact} \\
 \glt `My lady, why are you laughing?'  (Not `what are you laughing at ?', 2002qaCpa, 102)
\end{exe}  

When referring to purpose or reason, it is possible to combine  \japhug{tɕʰi}{what} with the nouns \japhug{ɯ-spa}{its material} and \japhug{ɯ-ndʐa}{its reason} (as the pronoun \japhug{tɕʰindʐa}{why})  respectively, as in (\ref{ex:tChi.Wspa.pWNu}) and (\ref{ex:tChi.YWtWɣAwu}). Note that examples (\ref{ex:tChi.YWtWnAre}) and (\ref{ex:tChi.YWtWɣAwu}) are from the same story, just a few lines away, in the same context; the construction in (\ref{ex:tChi.YWtWɣAwu}) is a more explicit variant of that in (\ref{ex:tChi.YWtWnAre}).

\begin{exe}
\ex \label{ex:tChi.Wspa.pWNu}
\gll tɕe tɕʰi ɯ-spa pɯ-ŋu mɤ-xsi ma tɕe nɯ kɯ-fse pjɤ-tu  \\
\textsc{lnk} what \textsc{3sg.poss}-material \textsc{pst.ipfv}-be \textsc{neg-genr}:know \textsc{lnk} \textsc{lnk} \textsc{dem} \textsc{nmlz}:S/A-be.like \textsc{ifr.ipfv}-exist \\
\glt `It is not known what it was for, but there was something like that.' (hist140522 GJW, 18)
\end{exe}  

\begin{exe}
\ex \label{ex:tChi.YWtWɣAwu}
\gll tɕʰindʐa ɲɯ-tɯ-ɣɤwu ŋu? \\
why \textsc{sens}-2-cry be:\textsc{fact} \\
\glt `Why are you crying?' (2002qaCpa, 94)
\end{exe} 

The pronoun \forme{tɕʰi} takes case marking with genitive \forme{ɣɯ} and the instrumental/ergative \forme{kɯ}, as in (\ref{ex:tChi.kW}).

\begin{exe}
\ex \label{ex:tChi.kW}
\gll tɕe tɕʰi kɯ tu-sɯ-βze ŋu mɤxsi ma nɯ kɯ-fse nɯ, sɯku ri ku-ndzoʁ ŋu \\
\textsc{lnk} what \textsc{erg} \textsc{ipfv}-\textsc{caus}-make[III] be:\textsc{fact} \textsc{neg}-\textsc{genr}-know \textsc{lnk} \textsc{dem} \textsc{nmlz}:S/A-be.like \textsc{dem} top.of.trees \textsc{loc} \textsc{ipfv}-\textsc{anticaus}:attach be:\textsc{fact} \\
\glt `I don't what it (the wasp) uses to make it (its nest), it is attached on trees.' (26-ndzWrnaR, 55)
\end{exe} 
In combination with the adverb \forme{jarma} / \japhug{jamar}{about}, it can be used to indicate a quantity, instead of \japhug{tʰɤstɯɣ}{how many} (section \ref{sec:thAstWG}).

\begin{exe}
\ex \label{ex:tChi.jamar}
\gll tu-ɕtʂam-a tɕe tɕʰi jamar ʑo ɣɤʑu kɯ? \\
\textsc{ipfv}-measure[III]-\textsc{1sg} \textsc{lnk} what about \textsc{emph} exist:\textsc{sens} \textsc{sfp} \\
\glt `I will measure it with a scoop to see how much (gold) there is.' (140512alibaba-zh, 59)
\end{exe}  

\begin{exe}
\ex \label{ex:tChi.jamar.kondza}
\gll kʰɯtsa ɯ-ŋgɯ tɯ-ci tu-rku-nɯ tɕe, nɯnɯtɕu tɤŋe nɯ pjɯ-sɯ-ntɕʰɤr-nɯ tɕe, tɕe tɕʰi jamar ko-ndza nɯnɯ, nɯnɯ ɯ-ŋgɯ nɯtɕu pjɯ-ru-nɯ tɕe,  nɯnɯ tu-rtoʁ-nɯ pjɤ-ŋgrɤl.   \\
bowl \textsc{3sg}-inside \textsc{indef.poss}-water \textsc{ipfv}-put.in-\textsc{pl} \textsc{lnk} \textsc{dem:loc} sun \textsc{dem} \textsc{ipfv-caus}-illuminate-\textsc{pl} \textsc{lnk} \textsc{lnk} what about \textsc{ifr}-eat \textsc{dem} \textsc{dem} \textsc{3sg}-inside \textsc{ipfv}:\textsc{down}-look.at-\textsc{pl} dem \textsc{ipfv}-see-\textsc{pl} \textsc{ifr.ipfv}-be.usually.the.case \\
\glt `They used to put water in a bowl and let the sunlight reflect into it; they could see how much (of the sun) had been occulted (`eaten' by the eclipse).' (29-mWBZi, 130)
\end{exe}  

\begin{exe}
\ex
\gll  zgo 	tʰɤstɯɣ 	ja-nnɯ-pɣaʁ-ndʑi, 	tɯ-ci 	tɕʰi 	jarma 	ja-nnɯ-pjɤl-ndʑi 	mɤ-xsi 	ma,       \\
 mountain how.many \textsc{pfv}:3\fl3'-\textsc{auto}-turn.over-\textsc{du} \textsc{indef.poss}-water what about \textsc{pfv}:3\fl3'-\textsc{auto}-cross-\textsc{du} \textsc{neg-genr}:know \textsc{lnk} \\
\glt `It is not known how many mountains and rivers they crossed.'  (2002qajdo, 50)
\end{exe}  

It is possible to combine \forme{tɕʰi jamar} with a adjective to express approximate comparison, as in (\ref{ex:tChi.kWzri}).

\begin{exe}
\ex \label{ex:tChi.kWzri}
\gll lɯlu ɣɯ tɕe ɯʑo ɯ-pʰoŋbu tɕʰi kɯ-zri jamar ɯ-jme nɯ kɯnɤ zri ri \\
cat \textsc{gen} \textsc{lnk} \textsc{3sg} \textsc{3sg.poss}-body what \textsc{nmlz}:S/A-be.long about \textsc{3sg.poss}-tail \textsc{dem} also be.long\textsc{fact} but \\
\glt `The cat, its body is about as long as its tail, but...' (27-qartshAz, 219)
\end{exe}  

In correlative clauses, the pronoun \japhug{tɕʰi}{what} can also be used to refer to a quantity without the adverb \japhug{jamar}{about} (example \ref{ex:tChi.tAkWsci}).

\begin{exe}
\ex \label{ex:tChi.tAkWsci}
\gll  
tɤ-rɟit tɕʰi tɤ-kɯ-sci nɯ ʑo ɣɯ-tɕɤt kɯ-ra pjɤ-ɕti tɕe,   \\
\textsc{indef.poss}-child what \textsc{pfv-nmlz:S/A}-be.born \textsc{dem} \textsc{emph} \textsc{inv}-take.out:\textsc{fact} \textsc{inf:stat}-have.to \textsc{ipfv.ifr}-be:\textsc{affirm} \textsc{lnk} \\
\glt `However many children were born, one had to raise them.' (tApAtso kAnWBdaR I, 9)
\end{exe}  

However, in independent interrogative clauses, \japhug{tɕʰi}{what} cannot refer to quantities. Sentence (\ref{ex:tChi.tosci}) thus can only mean `Was it a boy or a girl' not `How many children did she have?'.

\begin{exe}
\ex \label{ex:tChi.tosci}
\gll  ɯ-rɟit tɕʰi to-sci \\
\textsc{3sg.poss}-child what \textsc{ifr}-be.born \\
\glt `Was it a boy or a girl?'
\end{exe}  

\subsection{\japhug{ɕɯ}{who}}
The interrogative pronoun \japhug{ɕɯ}{who} occurs in questions about the identification of a human referent. It can occur in all syntactic roles, and does not have special ergative or genitive forms (see examples \ref{ex:CW.kW.tWwGmbi} and \ref{ex:CW.GW}). It is the probable cognate of a etymon widespread in the Trans-Himalayan family (for instance, Tibetan \tibet{སུ་}{su}{who}).

\begin{exe}
\ex  \label{ex:CW.tWNu}
\gll ma-tɯ-nɯqaɟy ma ɕɯ tɯ-ŋu mɤ-xsi \\
\textsc{neg:imp}-2-fish \textsc{lnk} who 2-be:\textsc{fact} \textsc{neg-genr}:know   \\
\glt `Don't fish, I don't who you are.' (gesar, 369)
\end{exe}  

\begin{exe}
\ex  \label{ex:CW.kW.tWwGmbi}
\gll  mɤ-ta-mbi nɤʑo qaɕpa ɕɯ kɯ tɯ́-wɣ-mbi    \\
\textsc{neg}-1\fl2-give:\textsc{fact} \textsc{2sg} frog who \textsc{erg} 2-\textsc{inv}-give:\textsc{fact}  \\
\glt `We won't give her to you, who would give her to you, a frog?'   (2002 qaCpa, 09)
\end{exe} 
 
\begin{exe}
\ex  \label{ex:CW.GW}
\gll  ɕɯ ɣɯ ʑo ɲɯ-kʰam-a ra kɯɣe?    \\
who \textsc{gen} \textsc{emph} \textsc{ipfv}-give:III-\textsc{1sg} \textsc{sfp} \\
\glt `Whom should I give (her) to (in marriage)?' (140508 benling gaoqiang de si xiongdi-zh, 222)
\end{exe}  

Forms related to \japhug{ɕɯ}{who} in Japhug include the indefinite pronoun \japhug{ɕɯmɤɕɯ}{whoever, anybody} (\ref{sex:CWmACW}) and \japhug{ɕɯŋarɯra}{each better than the other} (XXX).

\subsection{\japhug{tʰɤstɯɣ}{how many} and \japhug{tʰɤjtɕu}{when}} \label{sec:thAstWG}
To ask about precise quantities, \japhug{tʰɤstɯɣ}{how many} (or `how much') occurs rather than \forme{tɕʰi jamar} as seen above (section \ref{ex:tChi.jamar}).

\begin{exe}
\ex \label{ex:thAstWG.tWkhAm}
 \gll    nɤʑo 	tʰɤstɯɣ 	tɯ-kʰɤm?    \\
 you how.much 2-give[III]:\textsc{fact}  \\
\glt  `How much (money) do you give (for it)?' (Bargaining, 13)
\end{exe} 

It can be used for any countable quantity, including for people, as in (\ref{ex:thAstWG.tWtunW}).

\begin{exe}
\ex \label{ex:thAstWG.tWtunW}
 \gll
tsʰupa tʰɤstɯɣ tɯ-tu-nɯ ŋu? \\
village how.much 2-exist:\textsc{fact}-\textsc{pl} be:\textsc{fact} \\
\glt `How many (people) are you in the village?' (conversation, 140501)
\end{exe} 

The pronoun \forme{tʰɤstɯɣ} has a conjunct form \forme{tʰɤstɯ-} when used with classifiers (in \ref{ex:thAstWmaR}, with the classifier \ipa{-maʁ} `size of shoes' from Chinese \zh{码} \forme{mǎ}).

 \begin{exe}
\ex \label{ex:thAstWmaR}
 \gll   nɤ-xtsa nɯ tʰɤstɯ-maʁ tu-tɯ-ŋge ŋu   \\
\textsc{2sg.poss}-shoe \textsc{dem} how.many-size \textsc{ipfv}-2-wear[III] be:\textsc{fact} \\ 
\glt `What is the size of your shoes?'  (Conversation, 2015)
\end{exe} 

Combined with the noun \japhug{tɤ-rʑaʁ}{time}, 	\forme{tʰɤstɯɣ} can be used to ask about a length of time (\ref{ex:thAstWG}).

\begin{exe}
\ex \label{ex:thAstWG}
 \gll   nɤʑo 	tɤ-rʑaʁ 	tʰɤstɯɣ 	jamar 	tɤ-tsu tɕe 	kɤ-tɯ-spa-t?  \\
 you \textsc{indef.poss}-time how.many about \textsc{pfv}-pass \textsc{lnk} \textsc{pfv}-2-be.able-\textsc{pst:tr} \\
\glt   `How long did you need to learn it?' (elicited)
\end{exe} 

The phrase \forme{tɤ-rʑaʁ tʰɤstɯɣ} (or alternatively \forme{tɯtsʰot tʰɤstɯɣ}) in collocation with the verb \japhug{zɣɯt}{reach}, is also employed for asking about clock time, as in (\ref{ex:thAstWG.kozGWt})

 \begin{exe}
\ex \label{ex:thAstWG.kozGWt}
 \gll   tɤ-rʑaʁ 	tʰɤstɯɣ ko-zɣɯt? \\
  \textsc{indef.poss}-time how.many  \textsc{ifr}-reach \\
  \glt `What is the time?' (heard in context)
  \end{exe} 
  
This meaning can also be expressed by the pronoun \japhug{tʰɤjtɕu}{when}  (example \ref{ex:thAjtCu}).

\begin{exe}
\ex \label{ex:thAjtCu}
\gll  tʰɤjtɕu 	lɤ-tɯ-nɯɣe 	pɯ-ŋu 	ra 	nɤ?    \\
 when \textsc{pfv}-2-come.back[II] \textsc{pst.ipfv}-be \textsc{pl} \textsc{sfp} \\
\glt  When did you come back home? (taRrdo conversation, 01)
\end{exe} 

The element \ipa{tʰɤ-} in the pronouns \japhug{tʰɤjtɕu}{when}  and \japhug{tʰɤstɯɣ}{how many} is the \textit{status constructus} form of proto-Japhug \forme{*tʰi}, the inherited form of the pronoun `what' (see § \ref{sec:tChi}). The element \forme{-tɕu} in \japhug{tʰɤjtɕu}{when} is related to the locative \forme{tɕu} (see § XXX).

\subsection{\japhug{ŋotɕu}{where}} \label{sec:NotCu}

The interrogative pronoun \japhug{ŋotɕu}{where} and its reduced form \forme{ŋoj} can be used to ask either about a location (\ref{ex:NotCu.kutWrAZi}), a direction towards (examples \ref{ex:NotCu.tWCe} and \ref{ex:Noj.nari}) or from (\ref{ex:NotCu.jAtWGenW}) a certain place. The second syllable of this pronoun \forme{-tɕu} comes from the locative postposition \forme{tɕu}, but the first part is etymologically obscure.
 
\begin{exe}
\ex \label{ex:NotCu.kutWrAZi}
\gll     ŋotɕu ku-tɯ-rɤʑi?   \\
  where \textsc{pres.egoph}-2-stay \\
\glt `Where are you?" (Conversation, 2005)
\end{exe} 

\begin{exe}
\ex \label{ex:NotCu.tWCe}
\gll   ŋotɕu tɯ-ɕe? \\
 where 2-go:\textsc{fact} \\
\glt `Where are you going to?' (Common greeting used when one meets someone on the road)
 \end{exe} 
 
\begin{exe}
\ex \label{ex:Noj.nari}
\gll     qala ŋoj nɯ-ari  \\
  rabbit where \textsc{pfv:west}-go[II] \\
\glt Where did the rabbit go?  (qala2002, 21)
\end{exe} 

\begin{exe}
\ex \label{ex:NotCu.jAtWGenW}
\gll  nɯʑɤra ŋotɕu jɤ-tɯ-ɣe-nɯ? ŋotɕu ɕ-pɯ-tɯ-tu-nɯ? \\
\textsc{2pl} where \textsc{pfv}-2-come[II]-\textsc{pl} where \textsc{transloc-pfv}-2-exist-\textsc{pl} \\
\glt `Where are you from? Where have you been?' (2003sras, 57)
\end{exe} 

With the determiner \forme{nɯ}, the pronoun \forme{ŋotɕu} means `which (of several places)', as in (\ref{ex:NotCu.nW.Nu}) and (\ref{ex:NotCu.nW.Wku.Nu}).

\begin{exe}
\ex \label{ex:NotCu.nW.Nu}
\gll kʰa raŋri ɣɯ ʑo ɯ-ftaʁ pjɤ-tu ɕti ma, tɕe ŋotɕu nɯ ŋu, ŋotɕu nɯ maʁ mɯ-pjɤ-saχsɤl. \\
house each \textsc{gen} \textsc{emph} \textsc{3sg.poss}-mark \textsc{ifr.ipfv}-exist be:\textsc{affirm:fact} \textsc{lnk} \textsc{lnk} where \textsc{dem} be:\textsc{fact}  where \textsc{dem} be:\textsc{fact} \textsc{neg-ifr.ipfv}-be.clear \\
\glt `There was a mark on each of the houses, and one could not tell which (house) was (Alibaba's) and which was not.' (140512 alibaba-zh, 189-190)
\end{exe} 

\begin{exe}
\ex \label{ex:NotCu.nW.Wku.Nu}
\gll qaprɤftsa nɯnɯ, cici jɤ-ari tɕe ɯ-ku ju-z-mɤke, cici tɕe ɯ-jme ju-zmɤke ɲɯ-ɕti tɕe
ŋotɕu nɯ ɯ-ku ŋu, ŋotɕu ɯ-jme ŋu, mɯ́j-saχsɤl \\
centipede \textsc{dem} sometimes \textsc{pfv}-go \textsc{lnk} \textsc{3sg.poss}-head \textsc{ipfv-caus}-be.first[III] \textsc{lnk} \textsc{3sg.poss}-tail \textsc{ipfv-caus}-be.first[III] sens-be:affirm lnk where \textsc{dem} \textsc{3sg.poss}-head be:\textsc{fact} where \textsc{3sg.poss}-head be:\textsc{fact} \textsc{neg.sens}-be/clear \\
\glt `The centipede, when it moves, sometimes its head goes first, sometimesits tail goes first, it is not each to tell which is its head and which is its tail.' (21-qaprAftsa, 12)
\end{exe} 


With generic nouns such as \japhug{tɯrme}{person}, \forme{ŋotɕu} can serve as prenominal determiner to mean `a person from where', as in (\ref{ex:NotCu.tWrme}).

\begin{exe}
\ex \label{ex:NotCu.tWrme}
\gll ŋotɕu tɯrme tɯ-ŋu? \\
where person 2-be:\textsc{fact} \\
\glt `Where are you from?' (2011-05-nyima, 83)
\end{exe} 

In participial relatives with subject participle (in \forme{kɯ-}, see § XXX), \japhug{ŋotɕu}{where} can occur to express relativization of locative adjuncts, as in  (\ref{ex:NotCu.kWtu}); see § XXX for a discussion of the other available constructions.

\begin{exe}
\ex \label{ex:NotCu.kWtu}
\gll kɯ-me nɯra qʰe me,  ŋotɕu kɯ-tu nɯ qʰe kɯ-dɯ\redp{}dɤn tu-ɬoʁ ŋu. \\
\textsc{nmlz}:S/A-not.exist dem:pl lnk not.exist:\textsc{fact} where \textsc{nmlz}:S/A-exist \textsc{dem} \textsc{lnk} \textsc{nmlz}:S/A-\textsc{emph}\redp{}be.many \textsc{ipfv}-come.out be:\textsc{fact} \\
\glt `In (places) where it is not found, there is none, but in (places) where it is found, it grows in great number.' (21-jmAGni, 91)
\end{exe} 

The pronoun \japhug{ŋotɕu}{where} is not exclusively used in question about place or direction, we also find it in the expression in (\ref{ex:NotCu.YWNgrAl}).

 \begin{exe}
\ex \label{ex:NotCu.YWNgrAl}
\gll     kɯki 	ŋotɕu 	ɲɯ-ŋgrɤl?   \\
 this where \textsc{ipfv}-be.usually.the.case \\
\glt `How could this be possible?'  (qajdoskAt 2002, 32)
\end{exe} 

This sentence is used to express indignation (as in Chinese \zh{哪有这样的道理?}).\footnote{In the story from which it is quoted, the husband says this sentence after his wife, quoting the words of a raven, says that she will have luck, not her husband, who thus reacts in anger. }



\section{Indefinite pronouns} \label{sec:indef.pro}
 Japhug has a handful of indefinite pronouns, indicated in Table \ref{tab:indef.pronoun}. They do not form a complete paradigm, and other constructions, in particular generic nouns and free relatives occur to express meanings for which no indefinite pronoun exists (see § XXX).

There are no negative indefinite pronouns, and indefinite pronouns are almost never under the scope of negation (except in translations from Chinese). They also never occur as standard of comparison.\footnote{Examples such as `In Freiburg the weather is better than anywhere in Germany' (\citealt[2]{haspelmath97indef}) would not be expressible with an indefinite pronoun, see § XXX.}
 

\begin{table}[H] \centering
\caption{Indefinite pronouns }\label{tab:indef.pronoun}
\begin{tabular}{lllllll} \lsptoprule
\forme{tʰɯci}, \japhug{tʰɯtʰɤci}{something} \\
\japhug{tsʰitsuku}{whatever} \\
\japhug{ɕɯmɤɕɯ}{whoever, anybody} \\
\japhug{ciscʰiz}{somewhere} \\ 
\lspbottomrule
\end{tabular}
\end{table}

\subsection{\japhug{tʰɯci}{something} } \label{sec:thWci}
The indefinite pronoun \japhug{tʰɯci}{something} derives from the \textit{status constructus} of the proto-Japhug pronoun \forme{*tʰi} `what' (see \ref{sec:tChi} above) with the indefinite determiner and numeral \japhug{ci}{one}. Note that vowel alternation bleeds the sound change \ipa{*tʰi} \fl{}  \ipa{tsʰi}, otherwise a form such as $\dagger$\forme{tsʰɯci} would have been expected. Its reduplicated form \forme{tʰɯtʰɤci} has an irregular vocalism \ipa{ɤ} ($\dagger$\forme{tʰɯtʰɯci} would have been expected instead).

 It can designate specific referents, whose nature is known to the speaker but unknown to the addressee (as in \ref{ex:thWthAci.Zo.pjWtu}),\footnote{Example (\ref{ex:thWthAci.Zo.pjWtu}) is from a tale about a rabbit tricking a snow leopard; the difference of knowledge between the speaker and the addressee concerning the nature of the `something' is crucial to the plot. }.

\begin{exe}
\ex  \label{ex:thWthAci.Zo.pjWtu}
\gll tu-nɯsman-a jɤɣ ri, mɤʑɯ ɯ-ftɕaka tsuku pjɯ-tu ra wo, tɕe tʰɯtʰɤci ʑo pjɯ-tu ra \\
\textsc{ipfv}-treat-\textsc{1sg} be.possible:\textsc{fact} but yet \textsc{3sg.poss}-manner some \textsc{ipfv}-exist have.to:\textsc{fact} \textsc{sfp} \textsc{lnk} something \textsc{emph} \textsc{ipfv}-exist have.to:\textsc{fact} \\
\glt `I can treat (your illness), but yet another method is needed, something (else) is needed.'  (140427 qala cho kWrtsAg, 48-49)
\end{exe}

The pronoun \forme{tʰɯci} also occurs to refer to things whose name is unknown to the speaker (as in \ref{ex:gser.zhwa} and \ref{ex:thWci.khWtsa}), even if he/she may have seen the object.
 
\begin{exe}
\ex \label{ex:gser.zhwa}
\gll tɕe nɯ nɯ-rte nɯ tɕʰi ŋu ma tʰɯci ci ``-ʑa" tu-ti ŋu, χsɤrʑa! \\
\textsc{lnk} \textsc{dem} \textsc{3pl.poss}-hat \textsc{dem} what be:\textsc{fact} \textsc{lnk} something \textsc{indef} ... \textsc{ipfv}-say be:\textsc{fact} golden.hat \\
\glt `How is their hat (called), something in `ʑa'.... yes, \tibet{གསེར་ཞྭ་}{gser.ʑʷa}{golden hat}!' (30-mboR, 102)
\end{exe}

\begin{exe}
\ex \label{ex:thWci.khWtsa}
\gll  tɕe tɤ-ndʑɯɣ nɯ kɯnɤ, tʰɯci kʰɯtsa kɯ-fse ɯ-ŋgɯ tu-rku-nɯ tɕe   \\
\textsc{lnk} \textsc{indef.poss}-resin \textsc{dem} also something bowl \textsc{nmlz}:S/Abe.like \textsc{3sg}-inside \textsc{ipfv}-put.in-\textsc{pl} \textsc{lnk}   \\
\glt `The resin, people put it into something like a bowl.'' (07-tAtho, 44)
\end{exe}

It is also used for non-specific referents whose nature is entirely unknown, as in  (\ref{ex:thWthAci.tannWrkunW}) and (\ref{ex:thWmqlaR}).

\begin{exe}
\ex \label{ex:thWthAci.tannWrkunW}
\gll   tɕe mɤʑɯ tʰɯtʰɤci ta-nnɯ-rku-nɯ kɯma  \\
\textsc{lnk} yet something \textsc{pfv}:3\fl3'-\textsc{auto}-put.in-\textsc{pl} \textsc{sfp} \\
\glt `They also probably gave them something else.' (02-deluge2012, 120)
 \end{exe}
 
  \begin{exe}
\ex \label{ex:thWmqlaR}
\gll 
 tʰɯ-mqlaʁ tʰɯ-mqlaʁ ma tʰɯci fse ci ndʐa cʰɯ-ɕe ɕti \\
 \textsc{imp}:swallow  \textsc{imp}:swallow \textsc{lnk} something be.like:\textsc{fact} \textsc{indef} reason \textsc{ipfv:downstream}-go be:\textsc{affirm:fact} \\
\glt `Swallow it, swallow it, it comes down (into your throat) for some reason.' (2005-stod-kunbzang, 87)
  \end{exe}

The reduplicated form \forme{tʰɯtʰɤci}, especially in combination with \japhug{fse}{be like}, can also mean `whatever (happened)', as in (\ref{ex:thWthAci.kWfse}).
 
 \begin{exe}
\ex \label{ex:thWthAci.kWfse}
\gll  slama ra ɣɯ tʰɯtʰɤci kɯ-fse, kɤ-rɤ-βzjoz ra ɲɯ-stu mɯ́j-stu-nɯ, nɯ-stu ɲɯ-nɤma-nɯ mɯ́j-nɤma-nɯ,  nɯnɯra nɯ-pʰama ra nɯ-ɕki kɯ-rɤfɕɤt ɲɯ-ra. \\
student \textsc{pl} \textsc{gen} something \textsc{nmlz}:S/A-be.like \textsc{inf-antipass}-learn \textsc{pl} \textsc{sens}-try.hard-\textsc{pl} \textsc{neg:sens}-try.hard-\textsc{pl} \textsc{3sg.poss}-right \textsc{sens}-do-\textsc{pl} \textsc{neg:sens}-do-\textsc{pl} \textsc{dem:pl} \textsc{3pl.poss}-parent \textsc{pl} \textsc{3pl-dat} \textsc{genr}:S/P-tell \textsc{sens}-have.to \\
\glt `One has to tell the parents whatever concerns the students, whether they study seriously and try hard or not.'   (150901 tshuBdWnskAt, 18)
 \end{exe}
  
The non-reduplicated form \forme{tʰɯci} occurs in a correlative construction with the form \forme{mɯci} to mean `this and that', an expression that is used especially in reporting speech from another person when the speaker does not want to bother reporting in details the exact words that have been said.

\begin{exe}
\ex \label{ex:thWci.mWci}
\gll 
tʰɯci nɤme-a ra, mɯci nɤ-me-a ra \\
something do[III]:fact-\textsc{1sg} have.to:\textsc{fact} something do[III]:fact-\textsc{1sg} have.to:\textsc{fact} \\
\glt `I have to do this and that (so I cannot do X)' (elicitation)
 \end{exe}
 
The pronoun \japhug{tʰɯci}{something}  can also occur as determiner of a noun (or a headless relative clause. This use is found in native texts (example \ref{ex:thWci.khWtsa} above with the relative \forme{kʰɯtsa kɯ-fse} `which is like a bowl'), but it is most common in texts translated from Chinese, always with the indefinite determiner \japhug{ci}{one} after the noun or the relative clause, as in (\ref{ex:thWci.laXCi}) and (\ref{ex:thWthAci.akAspa}). 

\begin{exe}
\ex \label{ex:thWci.laXCi}
\gll   tʰɯci laχɕi ci ɕ-pɯ-nɯ-βzjoz-nɯ tɕe, jɤ-ɕe-nɯ ra \\
something trade \textsc{indef} \textsc{transloc-imp-auto}-learn-\textsc{pl} \textsc{lnk} \textsc{imp}-go-\textsc{pl} have.to:\textsc{fact} \\
\glt `Go and learn some trade!' (140508 benling gaoqiang de si xiongdi-zh, 29)
 \end{exe}

   \begin{exe}
\ex \label{ex:thWthAci.akAspa}
\gll 
 laχɕi ci pjɯ-βzjoz-a, tʰɯci a-kɤ-spa ci a-pɯ-tu ɲɯ-ra  \\
 trade \textsc{indef} \textsc{ipfv}-learn-\textsc{1sg} something \textsc{1sg.poss-nmlz:P}-be.able \textsc{indef} \textsc{irr-pfv}-exist \textsc{sens}-have.to \\
 \glt `I have to learn a trade, to have something I am able to do.' (150902 luban-zh, 12)
 \end{exe}
 
The determiner \japhug{kɯmaʁ}{other} (on which see § XXX) is placed before \japhug{tʰɯci}{something}, as in (\ref{ex:kWmaR.thWci}). 
 
\begin{exe}
\ex \label{ex:kWmaR.thWci}
\gll    ki mbro ki ɲɯ-kɤ-ntsɣe tɕe, kɯmaʁ tʰɯci ɲɯ-kɤ-sɤndu to-nɯkrɤz-ndʑi \\
\textsc{dem:prox} horse \textsc{dem:prox} \textsc{ipfv-inf}-sell \textsc{lnk} other  something   \textsc{ipfv-inf}-exchange \textsc{ifr}-discuss-\textsc{du} \\
 \glt `They discussed about selling their horse, and exchanging it for something else.' (150822 laoye zuoshi zongshi duide, 41)
\end{exe}


\subsection{\japhug{tsʰitsuku}{whatever}} \label{sec:tshitsuku}
The pronoun \japhug{tsʰitsuku}{whatever} combines the  interrogative pronoun \japhug{tsʰi}{what} (replaced by \japhug{tɕʰi}{what}, a borrowing from Tibetan in Kamnyu Japhug, but still attested in Mangi village, see \ref{sec:tChi} above) with the mid-scalar quantifier  \japhug{tsuku}{some}.  Unlike  \japhug{tʰɯci}{something}, is not used for specific referents.  Example (\ref{ex:tshitsuku.kuwGsqa}) illustrates its most common use. The variant form \forme{tʰitsuku}, without the sound change \forme{*tʰi} \fl{} \forme{tsʰi} is also used by speakers of the Kamnyu dialect.

\begin{exe}
\ex \label{ex:tshitsuku.kuwGsqa}
\gll  
kɤ-nɯβlɯ tɕe ɕkrɤz wuma ʑo pe ma nɯnɯ, nɯnɯ ɣɯ ɯ-smɯmba nɯ sɤɕke, tɕendɤre tsʰitsuku kú-wɣ-sqa tɕe, ʑaʑa ʑo ku-ɣɤ-smi cʰa, tsʰitsuku tú-wɣ-sɯ-ɤla tɕe, ʑaʑa tu-sɯ-ɤle cʰa. \\
\textsc{inf}-burn \textsc{lnk} oak really \textsc{emph} be.good:\textsc{fact} \textsc{lnk} \textsc{dem} \textsc{dem} \textsc{gen} \textsc{3sg.poss}-flame \textsc{dem} burning \textsc{lnk} whatever \textsc{ipfv-inv}-cook \textsc{lnk} soon \textsc{emph}  \textsc{ipfv-caus}-be.cooked can:\textsc{fact} whatever \textsc{ipfv-inv-caus}-be.boiling \textsc{lnk} soon  \textsc{ipfv-caus-caus}-be.boiling[III] can:\textsc{fact} \\
\glt `For burning, oak is very good, the flames (from its wood) are very hot, whatever one cooks, it cooks it quickly, whatever one boils, it boils it quickly.' (08-CkrAz, 4-5)
\end{exe}
%nɤʑo kɯ rcanɯ, tɯ-tso ɯ-tɯ-me nɯ, maka /ji/ tɕi-rca jɤ-ɣi tɕe, nɯ sɤznɤ tshitsuku a-pɯ-tɯ-mtɤm tɕe a-pɯ-tɯ-nɯtɯtso ɲɯ-mna
%140510_fengwang, 15
 
In many cases, it is better translated as `all kinds of things', as in (\ref{ex:tshitsuku.YWznAme}).

\begin{exe}
\ex \label{ex:tshitsuku.YWznAme}
\gll  
tɕe nɯtɕu kɯnɤ ɯ-jaʁ ɯ-ntsi tɤɲi pjɯ-sɤtse  ɯ-jaʁ ɯ-ntsi kɯ tsʰitsuku ɲɯ-z-nɤme qhe, ʑara nɯ-ndzɤtsʰi tu-βze, fsapaʁ ra nɯ-ndzɤtsʰi ɲɯ-βze \\
\textsc{lnk} \textsc{dem:loc} also \textsc{3sg.poss}-hand \textsc{3sg.poss}-one.of.a.pair staff \textsc{ipfv}-plant[III]  \textsc{3sg.poss}-hand \textsc{3sg.poss}-one.of.a.pair \textsc{erg} whatever \textsc{ipfv-caus}-do[III] \textsc{lnk} \textsc{3pl} \textsc{3pl.poss}-food \textsc{ipfv}-make[III] animal \textsc{pl} \textsc{3pl.poss}-food \textsc{ipfv}-make[III]  \\
\glt `Even like that, she supports herself with a staff in one hand, and with the other hand she does all kinds of things, makes their food, she makes food for the animals.' (14-tApitaRi, 54)
\end{exe}
%tshitsuku ɲɯ́-wɣ-mbi, ɲɯ́-wɣ-jtshi, tú-wɣ-raχtɕɤz tɕe, ʑɯrɯʑɤri tɕe tɕendɤre ku-kɯ-nɯfse ɲɯ-ŋu

As other indefinite pronouns, \japhug{tsʰitsuku}{whatever} is not normally used with negation, but such sentences do occur in the corpus in translations from Chinese, as (\ref{ex:tshitsuku.mWtoti}). They are not not idiomatic Japhug, and even only marginally grammatical.

\begin{exe}
\ex \label{ex:tshitsuku.mWtoti}
\gll   tsʰitsuku mɯ-to-ti, qʰe tɕendɤre kɯ-rŋgɯ jo-nɯɕe qʰe ko-nɯ-rŋgɯ. \\
whatever \textsc{neg-ifr}-say \textsc{lnk} \textsc{lnk} \textsc{nmlz}:S/A-lay.down \textsc{ifr}-go.back \textsc{lnk} \textsc{ifr-auto}-lay.down \\
\glt `He did not said anything, went back to sleep and laid down in bed.' (150902 qixian-zh, 91)
\end{exe}

 \subsection{\japhug{ɕɯmɤɕɯ}{whoever, anybody}} \label{sex:CWmACW}
 There is no indefinite pronoun for human referents `somebody' in Japhug  corresponding to \japhug{tʰɯci}{something} -- a generic noun with the indefinite determiner \japhug{ci}{one} such as \forme{tɯrme ci} `a man' is used instead. There is nevertheless a `free choice' pronoun \japhug{ɕɯmɤɕɯ}{whoever, anybody} (see \citealt[48-52]{haspelmath97indef} on the differences with universal quantifiers), which however is not very common. As example (\ref{ex:CWmACW.kW}) shows, it can take the ergative \forme{kɯ}, and the verb receives plural indexation (see § XXX for the use of the plural for indefinite referents). 
 
 \begin{exe}
\ex \label{ex:CWmACW.kW}
\gll tɕaχkɤr kʰɯtsa nɯ ʁo tʰam qʰe ɕɯmɤɕɯ kɯ ku-nɯ-ntɕʰoz-nɯ ɕti \\
tin bowl \textsc{dem} \textsc{advers} now \textsc{lnk} anybody \textsc{erg} \textsc{ipfv-auto}-use-\textsc{pl} be:\textsc{affirm:fact} \\
\glt `Now anybody can use tin bowls.' (unlike before, when only important people could use it, 160702 khWtsa, 26)
 \end{exe}
 
  \subsection{\japhug{ciscʰiz}{somewhere}}
The indefinite pronoun \japhug{ciscʰiz}{somewhere} comprises the indefinite \japhug{ci}{one} and the approximate locative \forme{(s)cʰiz} (see § XXX). It occurs with or without the locative postposition \forme{ri}, as in (\ref{ex:cischiz}) and (\ref{ex:cischiz.ri}). It can refer to static location, or motion from or towards a direction.

 \begin{exe}
\ex \label{ex:cischiz}
\gll
ciscʰiz, tɤtsʰoʁ ɯ-taʁ kɯ-fse, tɤ-jtsi ɯ-taʁ kɯ-fse, nɯnɯra, nɯnɯtɕu kú-wɣ-βraʁ tɕe, \\
somewhere nail \textsc{3sg-on} \textsc{nmlz}:S/A-be.like, \textsc{indef.poss}-pillar \textsc{3sg-on} \textsc{nmlz}:S/A-be.like,  \textsc{dem:pl} \textsc{dem:loc} \textsc{ipfv-inv}-attach \textsc{lnk} \\
\glt `One attaches (their noseband) somewhere, like on a nail, on a pillar.' (150902 kAxtCAr, 6)
 \end{exe}
 
 \begin{exe}
\ex \label{ex:cischiz.ri}
\gll nɯnɯ ciscʰiz ri tú-wɣ-z-nɯndzɯ tɕe ɲɯ́-wɣ-ta.\\
\textsc{dem} somewhere  \textsc{loc} \textsc{ipfv-inv-caus}-be.vertical \textsc{lnk} \textsc{ipfv:west-inv}-put\\
\glt `One puts it vertically somewhere.' (14-tasa, 62)
 \end{exe}
  
 
\subsection{Interrogative pronouns used as indefinites} \label{sec:interrogative.indef}
Non-specific indefinite referents can be expressed by interrogative pronouns in Japhug. One type of construction where this function is attested is correlatives, as in (\ref{ex:NotCu.lAtWrNgW}) and (\ref{ex:thAjtCu.fsaN}).
 
\begin{exe}
\ex \label{ex:NotCu.lAtWrNgW}
\gll a-pɯwɯ, ŋotɕu lɤ-tɯ-rŋgɯ ʑo qhe, nɯtɕu rɤʑi-tɕi ŋu ma, \\
\textsc{1sg}-donkey where \textsc{pfv:upstream}-2-lay.down \textsc{emph} \textsc{lnk} \textsc{dem:loc} stay:\textsc{fact}-\textsc{1du} be:\textsc{fact} because \\
\glt `My donkey, we will stay wherever you lay down.' (28-qAjdoskAt, 38)
\end{exe}  
 
\begin{exe}
\ex \label{ex:thAjtCu.fsaN}
\gll tʰɤjtɕu fsaŋ kɤ-ta tɤ-ra ʑo tɕe nɯnɯ tu-βlɯ-nɯ tɕe, \\
when fumigation \textsc{inf}-put \textsc{pfv}-have.to \textsc{emph} \textsc{lnk} \textsc{dem} \textsc{ipfv}-burn-\textsc{pl} \textsc{lnk} \\
\glt `Whenever there is need to make fumigations, they burn it.' (15-YaBrWG, 31)
\end{exe}  

This meaning also occurs in infinitival subordinate clauses, in particular in the expression \forme{tɕʰi kɤ-cʰa} `do whatever X can to Y', as in example (\ref{ex:tChi.kAcha.Zo}).

\begin{exe}
\ex \label{ex:tChi.kAcha.Zo}
\gll  tɕʰi kɤ-cʰa ʑo cʰɯ-pʰɯt-nɯ, \\
what \textsc{inf}-can \textsc{emph} \textsc{ipfv}-remove-\textsc{pl} \\
\glt `People do whatever they can to remove (this plant).' (12-Zmbroko, 119)
\end{exe}

The most common construction to express unspecified referents is built by combining an interrogative pronoun, the verb verb with partial reduplication on the last syllable of the stem, and in most cases the autobenefactive \forme{nɯ-} prefix (this use of the autobenefactive reminds of its occurrence in concessive conditionals, see § XXX).  

With \japhug{tɕʰi}{what}, this construction expresses the meaning `whatever; no matter what' in intransitive subject (\ref{ex:tChi.pWnWNWNu}), object (\ref{ex:tChi.tAtWnWtWtWt}) or semi-object (\ref{ex:tChi.kWstWstua}, see § XXX) functions.

\begin{exe}
\ex \label{ex:tChi.pWnWNWNu}
\gll lú-wɣ-sti tɕe tɕe nɯ ɯ-ŋgɯ tɕʰi pɯ-nɯ-ŋɯ\redp{}ŋu nɯ ɲɯ-mɲɤt mɯ́j-cʰa \\
\textsc{ipfv-inv}-block \textsc{lnk} \textsc{lnk} \textsc{dem} \textsc{3sg}-inside what \textsc{pst.ipfv}-\textsc{auto}-be \textsc{dem} \textsc{ipfv}-be.spoiled \textsc{neg:sens}-can \\
\glt `One seals (its opening) and whatever (food) is inside will not be spoiled.' (150828 kodAt, 14)
\end{exe}  

\begin{exe}
\ex \label{ex:tChi.tAtWnWtWtWt}
\gll tɕʰi tɤ-tɯ-nɯ-tɯ\redp{}tɯt ʑo ju-ɣi ɕti \\
what \textsc{pfv}-2-\textsc{auto}-say[II] \textsc{emph} \textsc{ipfv}-come be:\textsc{affirm:fact} \\
\glt  `Whatever you say will come.' (2003twxtsa, 117)
\end{exe}  

\begin{exe}
\ex \label{ex:tChi.kWstWstua}
\gll nɤʑo tɕʰi kɯ-stɯ\redp{}stu-a ʑo ŋu \\
\textsc{2sg} what 2\fl1-do.like-\textsc{1sg} \textsc{emph} be:\textsc{fact} \\
\glt `Whatever you do to me (will be fine).' (28-qAjdoskAt, 40)
\end{exe}

With \japhug{ɕɯ}{who}, the construction means `whoever; regardless of who; no matter who'. Examples are found with the non-specific referent in intransitive subject (\ref{ex:CW.pWnWNWNu}), transitive subject (\ref{ex:CW.kW.panWmtWmtonW}) or oblique argument (\ref{ex:CW.GW.nWnWkhWkhota}) functions. Note that it often occurs with plural indexation.

\begin{exe}
\ex \label{ex:CW.pWnWNWNu}
\gll tɯsqar nɯ kɯrɯ tɯrme ra mɤ-kɯ-rga maka ʑo me, ɕɯ pɯ-nɯ-ŋɯ\redp{}ŋu ʑo, tɯsqar a-pɯ-tu qʰe, tɕendɤre, nɯ-kɤ-ndza tu-rtaʁ ɕti, \\
tsampa \textsc{dem} Tibetan person \textsc{pl} \textsc{neg}-\textsc{nmlz}:S/A-like at.all \textsc{emph} not.exist:\textsc{fact} who  \textsc{pst.ipfv-auto}-be \textsc{emph} tsampa \textsc{irr}-\textsc{ipfv}-exist \textsc{lnk} \textsc{lnk} \textsc{3pl.poss}-\textsc{nmlz}:P-eat \textsc{ipfv}-be.enough be:\textsc{affirm}:\textsc{fact} \\
\glt `Among Tibetan people, everybody likes tsampa (`there is no one who does not like it'), no matter who, if they have tsampa, they have enough to eat.' (2002tWsqar2, 9)
\end{exe}

\begin{exe}
\ex \label{ex:CW.kW.panWmtWmtonW}
\gll tɕe ɕɯ kɯ pa-nɯ-mtɯ\redp{}mto-nɯ ʑo kɯki ɣɯ, nɯ-kʰa ɣɯ nɯ-mɯntoʁ nɯ cʰondɤre nɯ-ɕoŋpʰu nɯra tɕe, mɤʑɯ nɯ-<cai> nɯra, pjɯ-ɣɤmɯ-nɯ tɕe, \\
\textsc{lnk} who \textsc{erg} \textsc{pfv}:3\fl3'-see-\textsc{pl} \textsc{emph} \textsc{dem.prox} \textsc{gen} \textsc{3pl.poss}-house \textsc{gen} \textsc{3pl.poss}-flower \textsc{dem} \textsc{comit} \textsc{3pl.poss}-tree \textsc{dem:pl} \textsc{lnk} yet \textsc{3pl.poss}-vegetable \textsc{dem:pl} \textsc{ipfv}-praise-\textsc{pl} \textsc{lnk} \\
\glt `Whoever saw it, the flowers and the trees and the vegetables of their house, they praised it.' (150824 yuanding-zh, 30)
\end{exe}

\begin{exe}
\ex \label{ex:CW.GW.nWnWkhWkhota}
\gll tɤɕime ri tɯ-rdoʁ ma me, tɕendɤre nɯʑo ɕɯ ɣɯ nɯ-nɯ-kʰɯ\redp{}kʰo-t-a ʑo mɯ́j-nɯtɯtʂaŋ ɕti tɕe, \\
lady also one-\textsc{cl} apart.from not.exist:\textsc{fact} \textsc{lnk} \textsc{2pl} who \textsc{gen} \textsc{pfv}-\textsc{auto}-give-\textsc{pst:tr-1sg} \textsc{emph} \textsc{neg:sens}-be.fair be:\textsc{affirm:fact} \textsc{lnk} \\
\glt `There is only one princess, and regardless of whom among you all I give her hand to, it will be unfair.' (140508 benling gaoqiang de si xiongdi-zh, 227)
\end{exe}

Examples of this construction are also found with the pronoun \japhug{ŋotɕu}{where}, with the meaning `no matter where, wherever' (location or direction from or to).

\begin{exe}
\ex \label{ex:NotCu.nWnWlhWlhoR}
\gll ŋotɕu nɯ-ɬɯ\redp{}ɬoʁ ʑo wuma ʑo sɤɣdɯɣ \\
where \textsc{pfv}-come.out \textsc{emph} really \textsc{emph} be.annoying:\textsc{fact} \\
\glt `No matter where it grows, it is very annoying.' (5-khArWm, 19)
\end{exe}

\begin{exe}
\ex \label{ex:nWGtWta}
\gll 
ŋotɕu 	nɯ́-wɣ-tɯ\redp{}ta 	ʑo 	kɯpɤz 	ɲɯ-βze 	ɲɯ-ɕti\\
 where \textsc{ipfv-inv}-\textsc{indefinite}\textasciitilde{}put \textsc{emph} type.of.bug \textsc{ipfv}-grow \textsc{sens}-be.\textsc{assert}\\
\glt `Bugs will grow wherever you put (the meat).' (28-kWpAz, 48)
\end{exe}
 
 The pronoun \forme{ŋotɕu} in \textit{status constructus} form \forme{ŋɤtɕɯ-} occurs in the delocutive expression \japhug{ŋɤtɕɯkɤti,kʰɯ}{obey to everything} in a compound with the infinitive \forme{kɤ-ti} of the verb \japhug{ti}{say}, and in collocation with \japhug{kʰɯ}{agree}, as in (\ref{ex:NAtCWkAti}).\footnote{the causative \japhug{ŋɤtɕɯkɤti,sɯkʰɯ}{cause to obey to everything} also exists.} 
This expression originates presumably from a phrase such as `agree (\forme{kʰɯ}) to whatever (\forme{ŋotɕu}) he says (\forme{ti})', though the pronoun \japhug{tɕʰi}{what}, not \japhug{ŋotɕu}{where} is used in Japhug in the construction meaning `whatever' as in examples (\ref{ex:tChi.pWnWNWNu}) to (\ref{ex:tChi.kWstWstua}) above.

 \begin{exe}
\ex \label{ex:NAtCWkAti}
\gll  ɯ-tɕɯ kɯβde nɯra wuma ʑo ŋɤtɕɯkɤti pjɤ-kʰɯ-nɯ  \\
3sg.poss-son four dem:pl really emph obey.to.everything(1) \textsc{ifr.ipfv}-obey.to.everything(2)-\textsc{pl} \\
\glt `His four sons were very obedient.' (140508 benling gaoqiang de si xiongdi-zh, 15)
\end{exe} 

Good examples of this construction are not found in the corpus with the other interrogative pronouns, \japhug{tʰɤjtɕu}{when} or \japhug{tʰɤstɯɣ}{how many}, but they can also be used in the same way.

No example of multiple partitive use of interrogatives (\citealt[177]{haspelmath97indef}) is attested in the data at hand; mid-scalar quantifiers such as \japhug{tsuku}{some} occur instead (see § XXX). 

\section{Universal quantifiers} \label{sec:quantifiers}
Several quantifiers meanings `all' exist in Japhug (section XXX). Among them, \japhug{kɤsɯfse}{all} can be used in the meaning `everybody', as in example (\ref{ex:kAsWfse.kW}).

\begin{exe}
\ex \label{ex:kAsWfse.kW}
\gll kɤsɯfse kɯ ʑo ta-nɯ maʁ \\
all \textsc{erg} \textsc{emph} put:\textsc{fact}-\textsc{pl} not.be:\textsc{fact} \\
\glt `Not everybody puts it.' (160706 thotsi, 21)
\end{exe}

The interrogative pronoun \japhug{tɕʰi}{what}, appears with the plural demonstrative determiner \forme{kɯra} to mean `everything', as in example (\ref{ex:tChi.kWra}). It is not possible to express meanings such as `everybody' or `everywhere'  by combining the other pronouns \japhug{ɕɯ}{who} or \japhug{ŋotɕu} with the same demonstrative.

\begin{exe}
\ex \label{ex:tChi.kWra}
\gll ɯʑo tɕʰi kɯra ko-tso \\
\textsc{3sg} what \textsc{dem:prox:pl} \textsc{ifr}-understand \\
\glt `He understood everything.' (2002qajdoskAt, 115)
\end{exe}

There are two words, \forme{aʁɤndɯndɤt} and \forme{ŋotɕuŋɤndɤt}, which can be translated as `everywhere'.

The word \japhug{aʁɤndɯndɤt}{everywhere} is mainly used adverbially with the emphatic \forme{ʑo} as in (\ref{ex:aRAndWndAt.ʑo}), but there are examples where it occurs with the locative postposition \forme{ri} as (\ref{ex:aRAndWndAt.ri}) like a locative noun phrase, an observation suggesting that it can be analyzed as a pronoun, though no sentences with \japhug{aʁɤndɯndɤt}{everywhere} as subject (like `everywhere is quiet') are found in the corpus.

 \begin{exe}
\ex \label{ex:aRAndWndAt.ʑo}
\gll  aʁɤndɯndɤt ʑo kʰa ra cʰɯ-rɤpɯ. tɯ-ji ɯ-ngɯ ra cʰɯ-rɤpɯ, \\
everywhere \textsc{emph} house \textsc{pl} \textsc{ipfv}-litter \textsc{indef.poss}-field \textsc{3sg}-inside \textsc{pl} \textsc{ipfv}-litter \\
\glt `Mice have litter everywhere, in the house, in the fields.' (27-spjaNkW, 166)
\end{exe} 

 \begin{exe}
\ex \label{ex:aRAndWndAt.ri}
\gll nɯ fse ʑo aʁɤndɯndɤt ri tu-nnɯ-ɬoʁ qʰe, ɯ-zrɤm nɯra kɯ-tu maŋe. \\
\textsc{dem} be.like:\textsc{fact} \textsc{emph} everywhere \textsc{loc} \textsc{ipfv}-\textsc{auto}-come.out \textsc{lnk} \textsc{3sg.poss}-root \textsc{dem:pl} \textsc{nmlz}:S/A-exist not.exist:\textsc{sens} \\
\glt `It grows simply like that everywhere, it has no roots.' (20-sWrna,76)
\end{exe} 

When \japhug{aʁɤndɯndɤt}{everywhere} occurs under the scope of negation, it never expresses the meaning `nowhere', as shown by (\ref{ex:aRAndWndAt.me}) and (\ref{ex:aRAndWndAt.juCenW}).

\begin{exe}
\ex \label{ex:aRAndWndAt.me}
\gll stɤmku nɯra, tɯ-ci ɯ-rkɯ nɯra tu ma aʁɤndɯndɤt sthɯci me \\
plain \textsc{dem:pl} \textsc{indef.poss}-water \textsc{3sg.poss}-side  \textsc{dem:pl} exist:\textsc{fact} \textsc{lnk} everywhere so.much not.exist:\textsc{fact} \\
\glt `It is found in plains, or next to rivers, but it is not found everywhere.' (14-sWNgWJu, 53)
\end{exe} 

\begin{exe}
\ex \label{ex:aRAndWndAt.juCenW}
\gll tɕe ɕɤr tɕe cʰɯ-nɯ-ɬoʁ-nɯ tɕe, aʁɤndɯndɤt ju-ɕe-nɯ mɤ-kɯ-kʰɯ \\
\textsc{lnk} night \textsc{lnk} \textsc{ipfv:downstream-auto}-come.out-\textsc{pl} \textsc{lnk} everywhere \textsc{ipfv}-go-\textsc{pl} \textsc{neg}-\textsc{nmlz}:S/A-be.\textsc{possible} \\
\glt `(They make it) to prevent (animals) from coming out at night and going everywhere.' (150902 mkhoN, 21)
\end{exe} 

The word  \japhug{ŋotɕuŋɤndɤt}{everywhere} is semantically very close to \japhug{aʁɤndɯndɤt}{everywhere} but rarer; it may also be translated as `in all kinds of places'. It contains a partially reduplicated form of the interrogative pronoun \japhug{ŋotɕu}{where} (\ref{sec:NotCu}).

 \begin{exe}
\ex \label{ex:NotCuNondAt}
\gll ɕkrɤz ɯ-ŋgɯ tɕi ɲɯ-ɬoʁ, tɯrgi ɯ-ŋgɯ tɕi ɲɯ-ɬoʁ, ʑmbri ɯ-ŋgɯ tɕi ɲɯ-ɬoʁ,  mbraj ɯ-ŋgɯ tɕi ɲɯ-ɬoʁ, tɕe sɤjku sɯŋgɯ nɯra tɕi ɲɯ-ɬoʁ, tɕe ŋotɕuŋondɤt ʑo ɣɤʑu ɕti ri, stɤmku me, sɯŋgɯ ʁɟa ʑo tu-ɬoʁ ɲɯ-ŋu. \\
oak \textsc{3sg-inside} also \textsc{sens}-come.out fir \textsc{3sg-inside} also \textsc{sens}-come.out willow \textsc{3sg-inside} also \textsc{sens}-come.out red.birch \textsc{3sg-inside} also \textsc{sens}-come.out \textsc{lnk} birch  forest \textsc{dem:pl} also \textsc{sens}-come.out \textsc{lnk} everywhere \textsc{emph} exist:\textsc{sens} be:\textsc{affirm}  \textsc{lnk} plain whether  forest completely \textsc{emph} \textsc{ipfv}-come.out \textsc{sens}-be \\
\glt `(This mushroom) grows among oaks, among firs, among willows, among red or white birch forests, you find it everywhere, whether on plains or in forest.' (23-mbrAZim, 233-238)
\end{exe} 

\section{Identity pronoun} \label{sec:other.pro}
The word \japhug{kɯmaʁ}{other} occurs as a prenominal determiner (see § XXX, also for a discussion on its etymology), but it can also be used as a pronoun with any noun, and take determiners as in (\ref{ex:kWmaR.nWra}).

\begin{exe}
\ex \label{ex:kWmaR.nWra}
\gll ma kɯmaʁ nɯra aj mɯ́j-sɯχsal-a ri, tɤkʰepɣɤtɕɯ nɯ sɯχsal-a  \\
\textsc{lnk} other \textsc{dem:pl} 1sg neg:sens-recognize but bird.sp \textsc{dem} recognize:\textsc{fact}-\textsc{1sg} \\
\glt `The other ones I don't recognize them, but the \forme{tɤkʰepɣɤtɕɯ} bird, I do recognize it.' (23-scuz, 46)
\end{exe}

The indefinite \japhug{ci}{one} combined with the demonstrative determiner \forme{nɯ} (or \forme{nɯnɯ}) has the meaning `the other one' (the definite counterpart of \japhug{kɯmaʁ}{other}), as in (\ref{ex:ci.nW.kW}) and (\ref{ex:tWrdoR.ci.nW}).

\begin{exe}
\ex \label{ex:ci.nW.kW}
 \gll tɕe ɯ-jaʁ kɯ ki tu-ste lu-z-naʁje ɲɯ-ŋu ri, tɕe ci nɯ kɯ ɯ-jaʁ ku-mtsɯɣ ɲɯ-ɕti qʰe, \\
 \textsc{lnk} \textsc{3sg.poss}-hand \textsc{dem:prox} \textsc{ipfv}-do.like[III]  \textsc{ipfv}-reach.into[III] \textsc{sens}-be but \textsc{lnk} \textsc{indef} \textsc{dem} \textsc{erg} \textsc{3sg.poss}-hand  ipfv-bite sens-be:\textsc{affirm:fact} \textsc{lnk} \\
\glt `(The cat) reaches with its paw (into the whole), but the other one (the weasel) bites its paw.' (27-spjaNkW)
\end{exe}

\begin{exe}
\ex \label{ex:tWrdoR.ci.nW}
 \gll 
ɯ-me ʁnɯz pjɤ-tu tɕe, tɯ-rdoʁ nɯ χsɤrlɤsmɤn pjɤ-rmi, ci nɯ rŋɯlɤsmɤn pjɤ-rmi tɕe, \\
\textsc{3sg.poss}-daughter two \textsc{ifr.ipfv}-exist \textsc{lnk} one-\textsc{cl} \textsc{dem} gser.la.sman \textsc{ifr.ipfv}-be.called \textsc{indef} \textsc{dem} dngul.la.sman \textsc{ifr.ipfv}-be.called \textsc{lnk} \\
\glt `He had two daughters, one of them was called Gser.la.sman, and the other Dngul.la.sman.' (2003-kWBRa, 1-2)
\end{exe}

Alternatively to the construction in (\ref{ex:tWrdoR.ci.nW}) with \japhug{tɯ-rdoʁ}{one piece} and \forme{ci nɯ} to express the meaning one of them .... and the other ...', it is possible to use \forme{ci nɯ} two times in the same sentence to refer to more than one distinct persons or animals, as in (\ref{ex:ci.nW.2}).

\begin{exe}
\ex \label{ex:ci.nW.2}
 \gll tɕe ci nɯnɯ ju-ɕe ɯ-kʰɯkʰa ci nɯ kɯ ɯ-pu tu-ndze, cʰɯ-rɤɕi. \\
\textsc{lnk} \textsc{indef} \textsc{dem} \textsc{ipfv}-go \textsc{3sg}-while \textsc{indef} \textsc{dem} \textsc{erg} \textsc{3sg.poss}-intestine \textsc{ipfv}-eat[III] \textsc{ipfv}-pull \\
\glt `While one of the two (the prey) is (still) going, the other one (the predator) eats and pulls its intestine.'  (20-RmbroN, 76)
\end{exe}

The dual \forme{ci nɯni} `the other two' and plural \forme{ci nɯra} `the other ones' are also attested, as in (\ref{ex:ci.nWni}).

 \begin{exe}
\ex \label{ex:ci.nWni}
 \gll  ci nɯni ɣɯ nɯ, ndʑi-ta-mar rɟɤɣi pjɤ-ŋu tɕe tɕe nɯnɯ ɯʑo kɯ to-ndza \\
\textsc{indef} \textsc{dem:du} \gen \textsc{dem} \textsc{3du}.\textsc{poss}-\textsc{indef}.\textsc{poss}-butter tsampa \textsc{ipfv}.\textsc{ifr}-be:\textsc{fact} \textsc{lnk} \textsc{lnk} \textsc{dem} \textsc{3sg} \textsc{erg} \textsc{ifr}-eat \\
\glt `The tsampa of the other two (sisters) was butter tsampa, and she ate it.' (2003-kWBRa, 20)
\end{exe}

\section{Demonstrative pronouns} \label{sec:demonstrative.pronouns}
There are two basic demonstratives in Japhug, the proximal \japhug{ki}{this} and the distal one \japhug{nɯ}{that}, which also occur as demonstrative determiners (see § XXX). Table \ref{tab:dem.pronoun} illustrates the various demonstrative pronouns that are derived from these basic forms, with Reduplicated and Emphatic forms. There is in addition a Cataphoric pronoun \forme{nɤki}, discussed in section \ref{sec:cataph.pron}.

Plural and dual forms, as in the case of determiners, are formed by adding \forme{-ra} and \forme{-ni} suffixes to the demonstrative root, which undergoes \textit{status constructus} change \ipa{i} \fl{} \ipa{ɯ} in the case of proximal demonstratives. Plural forms are given in the table; dual forms are attested but rare and can be predicted (\japhug{kɯni}{these two} etc).

\begin{table}
\caption{Demonstrative pronouns}\label{tab:dem.pronoun}
\begin{tabular}{lllll} 
\lsptoprule
&Base form & Reduplicated & Emphatic \\
\midrule
\textsc{prox.sg} & \forme{ki} & \forme{kɯki} &  \forme{ɯkɯki}  \\
\textsc{dist.sg} & \forme{nɯ} &  \forme{nɯnɯ} & \forme{ɯnɯnɯ} \\
\midrule
\textsc{prox.pl} & \forme{kɯra} & \forme{kɯkɯra} &  \forme{ɯkɯkɯra}  \\
\textsc{dist.pl} & \forme{nɯra} &  \forme{nɯnɯra} & \forme{ɯnɯnɯra} \\
\lspbottomrule
\end{tabular}
\end{table}

\subsection{Anaphoric demonstrative pronouns} \label{sec:anaphoric.demonstrative.pro}

The basic demonstratives \forme{ki} and \forme{nɯ} are less often used as pronouns that the other ones (they mainly occur as determiners). They nevertheless do occur in all syntactic functions, including object (in particular with the verb \japhug{ti}{say}, as in \ref{ex:nW.toti}, where it refers to words that have been previously told to another animal), XXX and extended object (in particular with the verb \japhug{stu}{do like} as in \ref{ex:ki.tuste}).

\begin{exe}
\ex \label{ex:nW.toti}
 \gll  li nɯ to-ti ri, \\
 again \textsc{dem} \textsc{ifr}-say \textsc{lnk} \\
\glt `(Gesar) said the same thing to the (snow leopard).' (gesar, 286)
\end{exe}

\begin{exe}
\ex \label{ex:ki.tuste}
 \gll ɯ-mu nɯ ku-rqoʁ tɕe ki tu-ste tɕe \\
\textsc{3sg.poss}-mother  \textsc{dem} \textsc{ipfv}-hug \textsc{lnk} \textsc{dem:prox} \textsc{ipfv}-do.like[III] \textsc{lnk} \\
\glt `It hugs its mother like that.' (19-GzW, 30)
\end{exe}

The distal demonstratives \forme{nɯ} and \forme{nɯnɯ} serve as anaphoric pronouns with any type of referent, including humans, but are most appropriate for abstract concepts, inanimate objects or plants as in (\ref{ex:nWnW.kW.smi}), though as mentioned in section \ref{sec:pers.pronouns},  third person pronouns such as \japhug{ɯʑo}{he} can also have inanimate antecedents.

\begin{exe}
\ex \label{ex:nWnW.kW.smi}
 \gll tʂʰa kɤ-nɯ-ta tɤ-ra, smi kɤ-βlɯ tɤ-ra pɯ-nɯ-ŋu, tʰamaka sko-nɯ pɯ-nɯ-ŋu, tɕe \textbf{nɯnɯ} kɯ smi tu-sɯ-tɕɤt-nɯ. \\
 tea \textsc{inf}-\textsc{auto}-put \textsc{pfv}-have.to fire  \textsc{pfv}-burn \textsc{pfv}-have.to \textsc{pst.ipfv-auto}-be tobacco smoke:\textsc{fact}-\textsc{pl} \textsc{pst.ipfv-auto}-be \textsc{lnk} \textsc{dem} \textsc{erg} fire \textsc{ipfv}-\textsc{caus}-take.out-\textsc{pl} \\
 \glt `When they need to boil tea, to make a fire or smoke tobacco, people light up the fire with it.' (15-babW, 226-229)
\end{exe}

When a third person mentioned in a discussion is present, the pronoun \japhug{ɯʑo}{he} is not the optimal way of referring to him/her, and a proximal demonstrative, in particular the reduplicated \japhug{kɯki}{this one} is used instead. It can occur to present someone to someone else (\ref{ex:kWki.aslama}) (note that a similar usage exists in Western languages such as English in the same context) and even to talk about the actions of this person, as in  (\ref{ex:kWki.kW.taBzu}) and (\ref{ex:kWki.nW.ftsWntCi}).

\begin{exe}
\ex \label{ex:kWki.aslama}
 \gll kɯki a-slama ŋu \\
\textsc{dem.prox} \textsc{1sg.poss}-student be:\textsc{fact} \\
\glt `This a (former) student of mine.' (conversation 140510, 17)
\end{exe}

\begin{exe}
\ex \label{ex:kWki.kW.taBzu}
 \gll  kɯki kɯ ta-βzu? \\
 \textsc{dem.prox} \textsc{erg} \textsc{pfv}:3\fl3'-make \\
 \glt `Did she make it?' (conversation 140510, 152)
\end{exe}

As other pronouns (see § \ref{sec:pers.pronouns}), demonstrative pronouns can take the demonstrative determiner \forme{nɯ}, as in (\ref{ex:kWki.nW.ftsWntCi}).

\begin{exe}
\ex \label{ex:kWki.nW.ftsWntCi}
 \gll mɯ\redp{}mɤ-pɯ-jɤɣ tɕe mɤ-ɣi-tɕi ma \textbf{kɯki} nɯ fstɯn-tɕi ra ma tɕi-βɣe ɯ-ku thɯ-kɯ-ɣɤrndi  \\
\textsc{cond}\redp{}\textsc{neg}-\textsc{pst}.\textsc{ipfv}-be.acceptable \textsc{lnk} \textsc{neg}-come:\textsc{fact}-\textsc{1du} \textsc{lnk} \textsc{dem:prox} \textsc{dem} serve:\textsc{fact}-\textsc{1du} have.to:\textsc{fact} \textsc{lnk} \textsc{1du.poss}-orphan \textsc{3sg.poss}-head \textsc{pfv}-\textsc{nmlz}:S/A-support \\
\glt `If it is not possible (to take the old man with us) we will not come, as we have to serve him, he is the one who adopted us orphans when we were in dire straits.' (The old man is presumably present when this sentence is uttered; 2003nyima2, 122)
\end{exe}

The emphatic demonstrative pronouns (which are also used as determiners, see § XXX) are built by combining the reduplicated forms of demonstratives with the third person possessive prefix \forme{ɯ-}. They are about fifty time less common than corresponding reduplicated forms, but their function is essentially the same. In example (\ref{ex:WnWnW.kW}), \forme{ɯnɯnɯ} is an anaphoric pronoun whose antecedent is present in the immediately preceding clause.

\begin{exe}
\ex \label{ex:WnWnW.kW}
 \gll
tɕe ɯ-rqʰu kɯ-fse ci ɣɤʑu tɕe, \textbf{ɯnɯnɯ} kɯ ɯ-rdu nɯ tu-ɕɯ-fkaβ kɯ-fse ɲɯ-ŋu. \\
\textsc{lnk} \textsc{3sg.poss}-hull \textsc{nmlz}:S/A-be.like \textsc{indef}  exist:\textsc{sens} \textsc{lnk} \textsc{dem:emph:distal} \textsc{erg} \textsc{3sg.poss}-eyeball \textsc{dem} \textsc{ipfv}-\textsc{caus}-cover  \textsc{nmlz}:S/A-be.like \textsc{sens}-be \\
\glt `It has something like a membrane, and it covers its eyeball with it.' (description of the nictitating membrane of birds, 140513 sWNgWrmABja, 9)
\end{exe}

\subsection{Cataphoric pronoun} \label{sec:cataph.pron}
The demonstrative \forme{nɤki} stands out among other demonstrative pronouns in that it is most specifically used for cataphoric referents. It occurs especially when the speaker hesitates and uses it as a filler, followed by a clause with the same verb  (examples \ref{ex:nAki.YWNu} and \ref{ex:nAki.YAXtAr}).

\begin{exe}
\ex \label{ex:nAki.YWNu}
 \gll
qra nɯ kɯ, mbala na-lɤt nɤ tɕe \textbf{nɤki} ɲɯ-ŋu, jla ɲɯ-ŋu, \\
female.yak \textsc{dem} \textsc{erg} male.young.bovid \textsc{pfv}:3\fl3' \textsc{lnk} \textsc{lnk} \textsc{dem:cataph} \textsc{sens}-be male.hybrid.yak \textsc{sens}-be \\
\glt `When a female yak has a young (with a bull), it is..., it is a hybrid yak.' (05-qambrW, 64)
\end{exe}

\begin{exe}
\ex \label{ex:nAki.YAXtAr}
 \gll tɕe nɯ tɯ-ci ɣɯ ɯ-taʁ nɯnɯtɕu, \textbf{nɤki} ɲɤ-χtɤr, iɕqʰa <yujinxiang> kɤ-ti mɯntoʁ nɯ ɣɯ  ɯ-jwaʁ nɯ ɲɤ-χtɤr.  \\
\textsc{lnk} \textsc{dem} \textsc{indef.poss}-water \textsc{gen} \textsc{3sg}-on \textsc{dem:loc} \textsc{dem:cataph} \textsc{ifr}-spread the.aforementionned tulip \textsc{nmlz}:P-say flower \textsc{dem} \textsc{gen} \textsc{3sg.poss}-leaf \textsc{dem} \textsc{ifr}-spread \\
\glt She spilled on the water...  she spilled the petals of the flower called `tulip'.' (150818 muzhi guniang, 69)
\end{exe}

It is also used  when the speaker alerts the addressee that a long description follows as in (\ref{ex:nAki.tustunW}), as in English `(he said) the following'. Given the fact the Japhug is strictly verb-final and has pre-verbal complements (see § XXX), this is a strategy employed to avoid relegating the main verb to the end of the description.

 \begin{exe}
\ex \label{ex:nAki.tustunW}
 \gll
kʰopi kɯ ɴqiazwɤr ci ɲɯ-mɯm rca ɲɯ-saχaʁ ʑo tɕe \textbf{nɤki} tu-stu-nɯ ɲɯ-ŋu ɲɯ-ti, ɲɯ-pʰɯt-nɯ qʰe kɯ-zri... ki jamar ʑo kɯ-zri ɲɯ-pʰɯt-nɯ qʰe nɤki, ɯ-ku ɯ-mtɯ kɯ-fse nɯtɕu kú-wɣ-ndo qʰe tɕe ɯ-pa nɯ, ɯ-jwaʁ nɯ cʰɯ-χɕoʁ-nɯ ɲɯ-ŋu...  \\
p.n. \textsc{erg} bitter.wormwood \textsc{indef} \textsc{sens}-be.tasty \textsc{unexpect} \textsc{sens}-be.extremely \textsc{emph} \textsc{lnk} \textsc{dem:cataph} \textsc{ipfv}-do.like-\textsc{pl} \textsc{sens}-be \textsc{sens}-say \textsc{ipfv}-take.out-\textsc{pl} \textsc{lnk} \textsc{nmlz}:S/A-be.long \textsc{dem:prox} about \textsc{emph} \textsc{nmlz}:S/A-be.long \textsc{ipfv}-take.out-\textsc{pl} \textsc{lnk} \textsc{dem:cataph} \textsc{3sg.poss}-head  \textsc{3sg.poss}-crest \textsc{nmlz}:S/A-be.like  \textsc{dem:loc} \textsc{ipfv-inv}-take \textsc{lnk} \textsc{lnk} \textsc{3sg.poss}-under \textsc{dem}  \textsc{3sg.poss}-leaf \textsc{dem} \textsc{ipfv:downstream}-take.out-\textsc{pl}  \textsc{sens}-be \\
\glt `Kebei says that bitter wormwood is very tasty, and that they prepare it in the following way: they pluck (wormwoods) that are this big, take it by something that looks like a crest on the top, and prune away the leaves under it... (continued by several paragraphs)' (conversation 140510)
\end{exe}

The pronoun \forme{nɤki} is also used as a determiner (section XXX) and the speech filler \forme{nɤkinɯ} (section XXX) derives from the combination of  \forme{nɤki}  with the determiner \forme{nɯ}. The are no plural or dual forms of \forme{nɤki}. This pronoun possibly originates from a medial demonstrative, etymologically from the second person possessive \forme{nɤ-} and the proximal \forme{ki} `the one near you'.
