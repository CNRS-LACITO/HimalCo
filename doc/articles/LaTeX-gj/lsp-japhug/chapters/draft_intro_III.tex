\chapter*{Introduction}  
This draft is the first part of a grammar of Japhug. The examples are mainly based on a text corpus partially available on the Pangloss Archive (\url{http://lacito.vjf.cnrs.fr/pangloss/corpus/list\_rsc.php?lg=Japhug}, though additional examples from casual conversations and elicitations are also included.

This comprises five chapters, focusing on nominal morphology and on the structure of the noun phrase (to the exclusion of relative clauses). In total, the following 26 chapters are planned:

\begin{itemize}
\item 1-01 The Japhug language
\item 1-02 The Japhug Corpus
\item 2-01 Phonological inventory 
\item 2-02 Consonant clusters and partial reduplication
\item 2-03 Syllable structure and sandhi
\item 3-01 \textbf{Nominal morphology}
\item 3-02 \textbf{Pronouns, Demonstratives and Indefinites}
\item 3-03 \textbf{Numerals and counted nouns}
\item 3-04 \textbf{Postpositions and relator nouns}
\item 3-05 \textbf{The noun phrase}
\item 3-06 Ideophones
\item 3-07 Adverbs and sentence final particles
\item 4-01 The verbal template
\item 4-02 Person indexation
\item 4-03 Orientation and associated motion
\item 4-04 Non-finite verbal morphology
\item 4-05 Voice derivations
\item 4-06 Denominal derivations
\item 4-07 TAME
\item 4-08 Other verbal categories
\item 5-01 Relativization
\item 5-02 Complementation
\item 5-03 Temporal and conditional clauses
\item 5-04 Comparison
\item 5-06 Other subordinate clauses
\item 6-01 Language contact
\item 6-02 Kinship terminology
\item 6-03 Japhug in Sino-Tibetan perspective
\end{itemize}

 References to planned section that have not yet been written are indicated by § XXX.