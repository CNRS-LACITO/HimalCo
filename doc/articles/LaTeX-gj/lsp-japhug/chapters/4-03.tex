\chapter{Orientation and associated motion}
\section{Associated motion}
\subsection{Associated motion vs motion verb construction}
To express the meaning of motion prior to an action, associated motion prefixes are nearly two times as common as corresponding motion verb constructions (henceforth MVC) in the Japhug corpus. There is however a clear semantic difference between the two constructions, which was briefly described in \citet{jacques13harmonization}, but is presented here in more detail.

AM and MVC differ from each other in that in the former, the completion of both motion event and verbal action is presupposed, whereas in the case of the latter, the two can be separated. This difference is most conspicuous in past perfective forms, and can be observed in three main constructions: concessives (with negation of the verbal action), interrogatives and conditionals.

\subsubsection{Concessive} \label{sec:am.concessive}
A MVC  with the motion verb in perfective form can be followed by a clause negating the purposive action, as in (\ref{ex:nAkWrtoR}). In this example, only the motion is realized, while the action expressed by the verb \japhug{rtoʁ}{look} could not be accomplished.

\begin{exe}
\ex \label{ex:nAkWrtoR}
\gll nɤ-kɯ-rtoʁ jɤ-ɣe-a ri, mɯ-nɯ-atɯɣ-tɕi, mɯ-pɯ-ta-mto. \\
\textsc{1sg.poss}-\textsc{nmlz}:S/A-see \textsc{pfv}-come[II]-\textsc{1sg} \textsc{lnk} \textsc{neg}-\textsc{pfv}-meet-\textsc{1du} \textsc{neg}-\textsc{pfv}-1\fl2-see \\
\glt `I came to see you but I did not see you.' 
\end{exe}

With the corresponding AM verb form \japhug{ɣɯ-jɤ-ta-rtoʁ}{I came to see you}, negating the action of the verb is self-contradictory and nonsensical, and a sentence such as (\ref{ex:GWjAtartoR}) is incorrect.

\begin{exe}
\ex \label{ex:GWjAtartoR}
\gll $\dagger$ɣɯ-jɤ-ta-rtoʁ ri mɯ-pɯ-ta-mto \\
\textsc{cisloc}-\textsc{pfv}-1\fl2-look \textsc{lnk} \textsc{neg}-\textsc{pfv}-1\fl2-see \\
\glt Intended meaning: `I came to see you but I did not see you.' 
\end{exe}

\subsubsection{Interrogative} \label{sec:am.interrogative}
In interrogative clauses, MVCs are required to express meanings such as `What/who have you come/gone to X', as in example (\ref{ex:tChi.WkWpa}), an example which occurs many times in the corpus.

\begin{exe}
\ex \label{ex:tChi.WkWpa}
\gll tɕʰi ɯ-kɯ-pa jɤ-tɯ-ɣe? \\
what \textsc{3sg.poss}-do \textsc{pfv}-2-come[II] \\
\glt `What did you come to do?' (nine examples in the corpus)
\end{exe}

The difference between MVC and AM in interrogatives can be illustrated by comparing the minimal pair  (\ref{ex:tChi.WkWndza}) and (\ref{ex:tChi.GWtAtWndzat}). example (\ref{ex:tChi.WkWndza}), whcih has the same structure as (\ref{ex:tChi.WkWpa}), implies that the addressee has not eaten yet, while (\ref{ex:tChi.GWtAtWndzat}) with associated motion can only be used if the food ingestion has already taken place, and requires a different translation.

\begin{exe}
\ex \label{ex:tChi.WkWndza}
\gll tɕʰi ɯ-kɯ-ndza jɤ-tɯ-ɣe? \\
what \textsc{3sg.poss}-eat \textsc{pfv}-2-come[II] \\
\glt `What have you come to eat?' (elicited)
\end{exe}

\begin{exe}
\ex \label{ex:tChi.GWtAtWndzat}
\gll tɕʰi ɣɯ-tɤ-tɯ-ndza-t \\
what \textsc{cisloc}-\textsc{pfv}-2-eat-\textsc{pst:tr}    \\
\glt `What did you eat, after you came here?' (elicited)
\end{exe}

\subsubsection{Conditional} \label{sec:am.conditional}
The presuppositional difference between MVC and AM is also perceptible in the protasis of conditional clauses. 

With MVC in the protasis as in (\ref{ex:mWmAjAtWGe}), there is no presupposition that the verbal action took place, only the motion event constitutes a condition to the state of affair described in the apodosis.

\begin{exe}
\ex \label{ex:mWmAjAtWGe}
\gll nɤ-wa ɯ-kɯ-rtoʁ mɯ\redp{}mɤ-jɤ-tɯ-ɣe nɤ aʑo mɯ-pɯ-kɯ-mto-a. \\
\textsc{1sg.poss}-father \textsc{3sg.poss-}\textsc{nmlz}:S/A-look \textsc{cond}\redp{}\textsc{neg}-\textsc{pfv}-2-come[II] \textsc{lnk} \textsc{1sg} \textsc{neg}-\textsc{pfv}-2\fl{}1-\textsc{1sg} \\
\glt `If you had not come to see your father, you would not have seen me.' (you saw me, but your father was not here)
\end{exe}

By contrast, with AM, the verbal action necessarily took place, as in example (\ref{ex:mWmAGWjAtWrtoR}).

\begin{exe}
\ex \label{ex:mWmAGWjAtWrtoR}
\gll nɤ-wa  mɯ\redp{}mɤ-ɣɯ-jɤ-tɯ-rtoʁ nɤ pɯ-sɤzdɯxpa \\
\textsc{1sg.poss}-father \textsc{cond}\redp{}\textsc{neg}-\textsc{cisloc}-\textsc{pfv}-2-look \textsc{lnk} \textsc{pst.ipfv}-be.pitiful \\ 
\glt `If you had not come to see your father, he would have felt sorry.' (but you did saw him, so he does not feel sorry)
\end{exe}