\chapter{Orientation and associated motion}
\section{Associated motion} \label{sec:associated.motion}

\subsection{AM prefixes: morphology} \label{sec:am.prefixes}
Unlike Arandic and Tacanan languages, Japhug and other Gyalrong languages have simpler AM systems with only two prefixes,  as illustrated by examples (\ref{ex:GWpjWnWtshinW}) and (\ref{ex:CpjAnWtshi}).

\begin{exe}
\ex \label{ex:GWpjWnWtshinW}
\gll tɕe tɯ-ci ɣɯ-pjɯ-nɯ-tsʰi-nɯ  \\
\textsc{lnk} \textsc{indef}.\textsc{poss}-water \textsc{cisloc}-\textsc{ipfv}-\textsc{auto}-drink-\textsc{pl} \\
\glt `(The wild yaks) come to drink water there.' (20-RmbroN, 46)  
\end{exe}

\begin{exe}
\ex \label{ex:CpjAnWtshi}
\gll tɕe	tɯ-ci	ɕ-pjɤ-nɯ-tsʰi. \\
\textsc{lnk} \textsc{indef}.\textsc{poss}-water \textsc{transloc}-\textsc{ifr}-\textsc{auto}-drink  \\
\glt `She went to drink water.' (140428 mu e guniang, 72) 
\end{exe}

 AM prefixes in Japhug and other Gyalrong languages refer to a motion event occurring \textit{before} the action of the main verb, resulting in a prior temporal relation with respect to the main verb, as in  (\ref{ex:GWpjWnWtshinW}) and (\ref{ex:CpjAnWtshi}). There are no AM marlers for subsequent or concurrent motion.
 
 

\begin{table}[H]
\caption{Associated motion prefixes in Gyalrong languages} \centering \label{tab:am-gyalrong}
\begin{tabular}{lllll}
\toprule
&come & \textsc{cisloc} & go & \textsc{transloc} \\
\midrule
Japhug &  \forme{ɣi} &\forme{ɣɯ-} &\forme{ɕe} &\forme{ɕɯ-, ɕ-, ʑ-,z- } \\
Kyom-kyo (Situ) &\forme{vi} &\forme{və-} &\forme{tʃʰi} &\forme{ʃi-} \\
Brag-bar (Situ) &\forme{βʑê, və} &\forme{ɟɐ-} &\forme{tɕʰê} &\forme{ɕɐ-} \\
Tshobdun & \forme{wî}& \forme{o-} &\forme{ʃɐ̂} &\forme{ʃə-} \\
Zbu & \forme{və̂}& \forme{və-} &\forme{xwéʔ} &\forme{ɕə-} \\
\bottomrule
\end{tabular}
\end{table}

Table \ref{tab:am-gyalrong} presents the forms of AM prefixes in all four Gyalrong languages (\citealt{jacques19am-st}; data from \citealt{jacques13harmonization}, \citealt{gong18these}, \citealt{jackson14morpho}, \citealt[200-204]{zhang16bragdbar}, \citealt[497-500]{prins16kyomkyo}). As shown by the close phonetic resemblance between the motion verbs and the corresponding AM prefixes, there is little doubt that the latter have been grammaticalized from the former (The cislocative \forme{ɟɐ-} in Bragbar Situ being an exception); this event occurred in the common ancestor of all Gyalrong languages, rather than independently in each language, as is shown by the fact that the transitive verb `to bring, to fetch' (Japhug \forme{ru}, Bragbar Situ \forme{ró} / \forme{rô}) shares a common irregularity in all languages: it must appear with an associated motion prefix: in examples (\ref{sec:CtAruta}) and (\ref{sec:rECAroN}) from Japhug and Situ respectively removing the AM prefix would result in incorrect forms.

\begin{exe}
 \ex  \label{sec:CtAruta}
 \gll  \textbf{ɕ}-tɤ-ru-t-a  \\
 \textbf{\textsc{transl}}-\textsc{pfv}:\textsc{up}-bring-\textsc{pst}:\textsc{tr}-1\textsc{sg} \\
 \glt `I fetched it.' (Japhug)
 \ex  \label{sec:rECAroN}
 \gll   rə-\textbf{ɕɐ}-rô-ŋ  \\
\textsc{pfv}:\textsc{up}-\textbf{\textsc{transl}}-bring[II]-1\textsc{sg} \\
\glt `I fetched it.' (Situ)
\end{exe}

In the Japhug verbal template, AM prefixes occupy slot 3, just after irrealis and negative prefixes (§ XXX), but before orientation prefixes (as shown by all examples above) unlike Situ, where AM occur closer to the verb stem than the orientation prefixes (as shown by \ref{sec:rECAroN}).

\subsubsection{Cislocative} \label{sec:cislocative.morpho}
The cislocative \forme{ɣɯ-} is superficially homophonous with one allomorph the inverse prefix \forme{ɣɯ-} (§ XXX) and with one denominal prefix (§ XXX), but cannot be confused with either one; nevertheless some discussion concerning the distinction between the inverse and the cislocative can be useful to readers of Japhug texts.

Since the cislocative and the inverse do not occur in the same slot, and in particular respectively before and following the orientation prefixes, they two are easily distinguishable in forms having orientation prefixes such as (\ref{ex:GWtundze}) (where \forme{ɣɯ-} can only be the cislocative, not the inverse, since it occurs on the left of the orientation prefix \forme{tu-}).

\begin{exe}
\ex \label{ex:GWtundze}
 \gll qajɯ ra tu-ndze ma tɤ-rɤku ɣɯ-tu-ndze mɤ-ŋgrɤl \\
bugs \textsc{pl} \textsc{ipfv}-eat[III] \textsc{lnk} \textsc{indef}.\textsc{poss}-crops   \textsc{cisloc}-\textsc{ipfv}-eat[III] \textsc{neg}-be.usually.the.case:\textsc{fact} \\
\glt `It eats bugs, and does not come and eats the crops.' (24-ZmbrWpGa, 129) (Japhug)
\end{exe}

The only type of forms where a confusion could potentially arise is in the non-past factual, without second person arguments (which are indexed by prefixes, see § XXX). Even in such forms, since the inverse receives stress, and since it precludes stem III (§ XXX), verbs with stem alternation will not have ambiguous forms at least in the singular, as illustrated by the contrast between \forme{ɣɯ́-ndza} (inverse, stem I) in (\ref{ex:GWndza}) and \forme{ɣɯ-ndze} (cislocative, stem III) in (\ref{ex:GWndze}). 

\begin{exe}
\ex \label{ex:GWndza}
 \gll  tɕe ɯʑo nɯ ɲɤ-nɯ-pʰɣo matɕi tɕe, nɯ maʁ nɤ ɣɯ́-ndza pjɤ-ŋu. \\
 \textsc{lnk} \textsc{3sg} \textsc{dem} \textsc{ifr}-\textsc{auto}-flee \textsc{lnk} \textsc{lnk} \textsc{dem} not.be:\textsc{fact} \textsc{lnk} \textsc{inv}-eat:\textsc{fact} \textsc{ifr}-be \\
\glt `(The rabbit), he fled away, otherwise he would have been eaten.' (140427 qala cho kWrtsag, 76)
\end{exe}

\begin{exe}
\ex \label{ex:GWndze}
 \gll  tɤ-rɤku kɯ-fse ɣɯ-ndze mɤ-ŋgrɤl \\
 \textsc{indef}.\textsc{poss}-crop \textsc{nmlz}:S/A-be.like \textsc{cisloc}-eat[III]:\textsc{fact} \textsc{neg}-be.usually.the.case:\textsc{fact} \\
 \glt `It does not come and eat crops.' (23-pGAYaR, 26)
\end{exe}

It is perfectly possible for the inverse (in its non-initial allomorph \forme{-wɣ-}) to directly follow the cislocative, which in this case receives the stress, as \forme{ɣɯ́-wɣ-ndza} in (\ref{ex:GwWGndza}).

\begin{exe}
\ex \label{ex:GwWGndza}
 \gll kʰu a-jɤ-ɤzɣɯt ndɤ ɣɯ́-wɣ-ndza ɲɯ-ŋu \\
 tiger \textsc{irr}-\textsc{pfv}-arrive \textsc{lnk} \textsc{cisloc}-\textsc{inv}-eat:\textsc{fact} \textsc{sens}-be \\
\glt `If the tiger arrives, it (the tiger) will eat him (the man).' (kandZislama 2003, 100)
\end{exe}

\subsubsection{Translocative} \label{sec:translocative.morpho}
Unlike the cislocative, the translocative prefix presents an important degree of allomorphy. Besides the main allomorph \forme{ɕɯ-}, the consonantal allomorphs \forme{ɕ-}, \forme{ʑ-}, \forme{s-} and \forme{z-} are also found. These allomorphs are only attested in direct contact with orientation prefixes, following the rules in Table \ref{tab:translocative.allomorphs}.

\begin{table}
\caption{Allomorphs of the translocative prefix} \centering \label{tab:translocative.allomorphs}
\begin{tabular}{lllll}
\lsptoprule
Allomorph & Orientation prefix  \\
\midrule
\forme{ɕ-} & \forme{tu/ɤ/o/a-}, \forme{pɯ/a-}, \forme{tʰɯ/a-}, \forme{ku/ɤ/o/a-},  \forme{pjɯ/ɤ-}, (\forme{cʰɯ/ɤ-}) \\
\forme{ʑ-} & \forme{lu/ɤ/o/a-}, \forme{nɯ/a-}, (\forme{ɲɯ/ɤ-}) \\
\forme{s-} & \forme{cʰɯ/ɤ-}, (\forme{pjɯ/ɤ-})  \\
\forme{z-} & \forme{ju/ɤ/o/a-},  \forme{ɲɯ/ɤ-} \\
\lspbottomrule
\end{tabular}
\end{table}

The dental allomorphs \forme{s-} and \forme{z-}  are only found with orientation prefixes with a palatal consonant, and result from dissimilation in place of articulation from their alveolo-palatal counterparts. They harmonize in voicing with the following orientation prefix, \forme{s-} being found before the orientation prefixes in \forme{cʰ-} and \forme{pj-}, and \forme{z-} before those in \forme{j-} and \forme{ɲ-}.

The alveolo-palatal allomorphs are found with non-palatal orientation prefixes, and also harmonize in voicing with the following with the following consonant, \forme{ɕ-} occurring before unvoiced prefixes (\forme{ɕ-tu-kʰat-a} in \ref{ex:ZYWlata} below) and \forme{ʑ-} before voicing ones as in (\ref{ex:ZnaCar}).

\begin{exe}
\ex \label{ex:ZnaCar}
 \gll tɕendɤre ɯ-pɯ ra nɯ-ndza ʑ-na-ɕar  \\
 \textsc{lnk} \textsc{3sg}.\textsc{poss}-young \textsc{pl} ɕ.\textsc{poss}-food \textsc{transloc}-\textsc{pfv}:3\fl{}3'-search \\
\glt `The (cat mother) went to look for food for her youngs.' (21-lWLU, 80)
\end{exe}

The dental allomorphs are also used with the palatal orientation prefixes, as shown by the form \forme{ʑ-ɲɯ-lat-a} (instead of \forme{z-ɲɯ-lat-a}) in (\ref{ex:ZYWlata}). With \forme{cʰ-}, \forme{j-} and \forme{ɲ-} prefixes, the allomorph \forme{ʑ-} is very rare, but with \forme{pj-} prefixes interestingly, the dissimilatory effect is much more limited and \forme{ɕ-} is more common than \forme{s-}, as shown by the counts in Table \ref{tab:transloc.allomorphs.counts}.

\begin{exe}
\ex \label{ex:ZYWlata}
 \gll nɯnɯ ʑakastaka kɯ-tu nɯnɯ ɣɯ nɯ-<gongfen> ra ʑ-ɲɯ-lat-a ɕ-tu-kʰat-a pɯ-ra. \\
\textsc{dem} each \textsc{nmlz}:S/A-exist \textsc{dem} \textsc{gen} \textsc{3sg}.\textsc{poss}-work.point \textsc{pl} \textsc{transloc}-\textsc{ipfv}-throw-\textsc{1sg} \textsc{transloc}-\textsc{ipfv}-do.everywhere-\textsc{1sg} \textsc{pst}.\textsc{ipfv}-have.to  \\
\glt `I had to go everywhere to count work points for each one who was in the commune.' (2010-09, 79)
\end{exe}

\begin{table}[H]
\caption{Number of attestations of the allomorphs of the translocative prefix with palatal oriental prefix} \centering \label{tab:transloc.allomorphs.counts}
\begin{tabular}{lllll}
\toprule
Prefixes & \forme{ɕ-} & \forme{ʑ-} & \forme{s-} & \forme{z-}  \\
\midrule
\forme{pjV-} & 95 & &3& \\
\forme{cʰV-} & 1 & &53& \\
\forme{ɲV-} &  & 4& &96 \\
\forme{jV-} & & 0& & 62\\
\lspbottomrule
\end{tabular}
\end{table}

With prefixes other than orientation prefixes, such as infinitive \forme{kɤ-} (\ref{ex:CWkArANgWm}) or second person \forme{tɯ-} (\ref{ex:maCWtWnAtWi}), only the allomorph \forme{ɕɯ-} is found. The infinitive \forme{kɤ-} and the east/centripetal orientation prefix \forme{kɤ-} can thus be distinguished by their compatibilities with the allomorphs of the translocative prefix, the former occurring with \forme{ɕɯ-}, and the latter with \forme{ɕ-}.

\begin{exe}
\ex \label{ex:CWkArANgWm}
 \gll tɕendɤre nɯtɕu ɕɯ-kɤ-rɤŋgɯm ndɤre kumpɣɤtɕɯ mɯ́j-nɤz \\
 \textsc{lnk} \textsc{dem}:\textsc{loc} \textsc{transloc}-\textsc{inf}-lay.eggs \textsc{lnk} sparrow \textsc{neg}:\textsc{sens}-dare \\
 \glt `The sparrow does not dare to lay eggs there (on the ground).' (22-kumpGatCW, 90-91)
\end{exe}

\begin{exe}
\ex  \label{ex:maCWtWnAtWi}
 \gll ma-ɕɯ-tɯ-nɤtɯti \\
 \textsc{neg}-\textsc{transloc}-2-tell.everywhere \\
 \glt `Do go around talking about it.' (2002 qaCpa, 252)
\end{exe}

In factual non-past prefixless forms, the allomorph \forme{ɕɯ-} is also the only possible one, as in (\ref{ex:CWte.pjAra}) ($\dagger$\forme{ɕ-te} would be an incorrect form).

\begin{exe}
\ex  \label{ex:CWte.pjAra}
 \gll  tɕe ndʑi-ŋga ɕɯ-te pjɤ-ra \\
 \textsc{lnk} \textsc{3du}.\textsc{poss}-clothes \textsc{transloc}-put[III]:\textsc{fact} \textsc{pst}.\textsc{ifr}-have.to \\
 \glt `She had to make their beds (cover them with a quilt)'. (2003 kWBRa, 76)
\end{exe}

\subsection{Argument of motion}
The argument undergoing the motion event is always the subject (S/A), except in the case of causative constructions, when it can be either causer or causee (as in \ref{ex:GWchWsWXtWnW}).

\begin{exe}
\ex \label{ex:GWchWsWXtWnW}
\gll tɕe kupa-cʰu nɯra atʰi pɕoʁ nɯra, ɯ-pɕi nɯra kɯ kɯre ri ɣɯ-cʰɯ-sɯ-χtɯ-nɯ ŋu.  \\
\textsc{lnk} Chinese-\textsc{loc} \textsc{dem:pl} downstream direction \textsc{dem:pl} \textsc{3sg}-outside  \textsc{dem:pl}  \textsc{erg} here \textsc{loc} \textsc{cisloc}-\textsc{ipfv}:\textsc{downstream}-\textsc{caus}-buy-\textsc{pl} be:\textsc{fact} \\ 
\glt `People from the Chinese areas, people from outside send people to come here to buy (matsutake and sell them in the areas downstream).' (20 grWBgrWB 58)  
  \end{exe} 


\subsection{Motion verbs and AM prefixes}
In Japhug, there is no constraint on AM prefixes occurring on motion verbs with the same deixis. Examples (\ref{ex:GWjuGinW}) and (\ref{ex:CpjACe}) respectively illustrate the cislocative on the verb \japhug{ɣi}{come} and the translocative on the verb \japhug{ɕe}{go}. Such examples are not common enough to allow a clear analysis of the semantic value of the redundant AM in these examples.

\begin{exe}
\ex \label{ex:GWjuGinW}
 \gll <jiazhang> ra ju-ɣi-nɯ tɕe <laoshi> ɯ-ɕki, tɯ-ɕki ʑo ɣɯ-ju-ɣi-nɯ ɕti netɕi? \\
 parents \textsc{pl} \textsc{ipfv}-come-\textsc{pl} \textsc{lnk} teacher \textsc{3sg}.\textsc{poss}-\textsc{dat} \textsc{genr}.\textsc{poss}-\textsc{dat} \textsc{emph} \textsc{cisloc}-\textsc{ipfv}-come-\textsc{pl} be.\textsc{affirm}:\textsc{fact} \textsc{sfp} \\
 \glt `The parents come, come to the teachers (us).' (conversation140501 01, 60)
\end{exe}

\begin{exe}
\ex \label{ex:CpjACe}
 \gll li nɤki iɕqʰa nɯ tɤjlu kɤ-rku ɯ-ŋgɯ zɯ ɕ-pjɤ-ɕe \\
 again \textsc{dem} the.aforementioned \textsc{dem} flour \textsc{nmlz}:P-put.in \textsc{3sg}.\textsc{poss}-inside \textsc{loc} \textsc{transloc}-\textsc{ifr}:\textsc{down}-go \\
 \glt `He went into the bag of flour.' (140519 chou xiaoya-zh, 145)
\end{exe}

The opposite combinations, namely cislocative with \japhug{ɕe}{go} and translocative with \japhug{ɣi}{come}, are not grammatical. 

\subsection{Orientation and AM}
In Japhug, AM markers only specify deixis and the temporal relation between motion event and verbal action, but are neutral as regards the orientation of the motion event.

Orientation and AM markers occupy distinct prefixal slots. Non-orientable verbs (verbs expressing actions other than motion, manipulation, sight or actions with a single direction, see § XXX) select one or two lexicalized orientations (see § XXX). For instance, the verb \japhug{mɯrkɯ}{steal} occurs with the orientation `up' (with the orientation prefixes \forme{tɤ-}, \forme{ta-}, \forme{tu-}, \forme{to-}, see § XXX). 

When non-orientable verbs occurs with AM, the verb normally keeps the lexicalized orientation prefix, as in (\ref{ex:CtumWrki}), where \japhug{mɯrkɯ}{steal} is used with the expected \forme{tu-} `up' prefix. In this context, the motion relating to the act of stealing occurs at the same horizontal level and there is no upward motion; the orientation prefix is thus here irrelevant to the motion event, and is only the default lexicalized orientation for the verb \japhug{mɯrkɯ}{steal}.

\begin{exe}
\ex \label{ex:CtumWrki}
 \gll kɯ-nŋo nɯ qʰe ci ci ɕ-tu-mɯrki kɯ-fse ma nɯ ma mɯ-ɲɯ-ɤʁe. \\
\textsc{nmlz}:S/A-be.defeated \textsc{dem} \textsc{lnk} one one \textsc{transloc}-\textsc{ipfv}-steal[III] \textsc{nmlz}:S/A-be.like apart.from \textsc{dem} apart.from \textsc{neg}-\textsc{sens}-have.to.eat \\
\glt `The (lion) which is defeated steals a little out of it, but apart from that has nothing to eat.' (20-sWNgi, 65)
\end{exe}

However, alternatively, the orientation prefixes of verbs with an AM marker can also reflect the orientation of the motion event. In (\ref{ex:CpjWntGea}), the verb \japhug{ntsɣe}{sell} occurs with the orientation prefixes `up' (\forme{ɕ-tu-ntsɣe-a}) and `down'  (\forme{ɕ-pjɯ-ntsɣe-a}). The lexicalized orientation normally selected by  \japhug{ntsɣe}{sell} is the `west; centrifuge' one (§ XXX). In (\ref{ex:CpjWntGea}) the `up' and `down' prefixes clearly correlate with those found on the manipulation verb \japhug{ɣɯt}{bring} and express the direction of the motion event.

\begin{exe}
\ex \label{ex:CpjWntGea}
\gll rɟa ɣɯ ɯ-laχtɕʰa tu-ɣɯt-a tɕe pot zɯ ɕ-tu-ntsɣe-a, pot ɣɯ ɯ-laχtɕʰa pjɯ-ɣɯt-a tɕe, rɟa zɯ ɕ-pjɯ-ntsɣe-a. \\
China \textsc{gen} \textsc{3sg}.\textsc{poss}-thing  \textsc{ipfv}:\textsc{up}-bring-\textsc{1sg} \textsc{lnk} Tibet \textsc{loc} \textsc{transloc}-\textsc{ipfv}:\textsc{up}-sell-\textsc{1sg}  Tibet \textsc{gen} \textsc{3sg}.\textsc{poss}-thing  \textsc{ipfv}:\textsc{down}-bring-\textsc{1sg} \textsc{lnk} China \textsc{loc} \textsc{transloc}-\textsc{ipfv}:\textsc{down}-sell-\textsc{1sg} \\
\glt `I bring things (down) from central Tibet, and sell them to China.' (28-qAjdoskAt, 15-16)
\end{exe}
 
Similarly, in  (\ref{ex:ZlumWrkia}), the `upstream' prefix occurs with \japhug{mɯrkɯ}{steal}, to express the fact that the character steals from a place located downstream to bring it upstream. Interestingly, after a few occurrences of the verb \japhug{mɯrkɯ}{steal} with AM and `upstream' orientation, one finds examples of that verb with the same non-lexicalized orientation `upstream' but without the AM prefix in the same text, as in (\ref{ex:lomWrkW}).

\begin{exe}
\ex \label{ex:ZlumWrkia}
 \gll tɕetʰi tɤmuj jlɤrɯcɤrna ɣɯ ɯ-pʰe nɯtɕu kɯ-mɯrkɯ chɯ-ɕe-a ŋu tɕe. tsʰɤt ɯ-ʁrɯ ɣɯ ɯ-ci nɯnɯ ʑ-lu-mɯrki-a ri a-qʰu zɯ lɤ-ɣe-nɯ tɕe \\
 downstream p.n. p.n. gen \textsc{3sg}.\textsc{poss}-\textsc{dat} \textsc{dem}:\textsc{loc}  \textsc{nmlz}:S/A-steal  \textsc{ipfv}:\textsc{downstream}-go-\textsc{1sg} be:\textsc{fact} \textsc{lnk}  goat \textsc{3sg}.\textsc{poss}-horn \textsc{gen} \textsc{3sg}.\textsc{poss}-water \textsc{dem} \textsc{transloc}-\textsc{ipfv}:\textsc{upstream}-streal[III]-\textsc{1sg} \textsc{lnk} \textsc{1sg}.\textsc{poss}-after \textsc{loc} \textsc{pfv}:\textsc{upstream}-come[II]-\textsc{pl} \textsc{lnk}  \\
 \glt `(Tomorrow  morning) I will go downstream to steal from Tamuj Jlarukyarna, I will steal the water from the goat's horn, but when (the mountain god) comes after me...' (25-kAmYW-XpAltCin, 31-32)
\end{exe}

\begin{exe}
\ex \label{ex:lomWrkW}
 \gll lo-mɯrkɯ pjɤ-cʰa tɕe lo-ɣɯt ri \\
 \textsc{ifr}:\textsc{upstream}-steal \textsc{ifr}-can \textsc{lnk} \textsc{ifr}:\textsc{upstream}-bring \textsc{lnk} \\
\glt `He was able to steal it and brought it upstream.' (02-montagnes-kamnyu-cz, 31)
\end{exe}

In some limited contexts, it is thus possible for verbs without associated motion markers to use their orientation prefix to indicate a direction towards of from which the action is directed, with or without motion, as in (\ref{ex:lomWrkW}). 

In (\ref{ex:lupea.je}), similarly, we find the verb \forme{lu-pe-a} with the `upstream' orientation without associated motion marker; when this sentence was uttered, we were seating far from the door, and to close the door it was necessary to get up and walk a few meters (we were seated at a place closer to the river, so the door was `upstream', § XXX). The same sentence could have been uttered if the door had been at a hand's reach, and could thus have been closed without walking. Replacing \forme{lu-pe-a} in (\ref{ex:lupea.je}) by \forme{ʑ-lu-pe-a} \textsc{transloc}-\textsc{ipfv}:\textsc{upstream}-close[III]-\textsc{1sg}, with an associated motion marker, would make the second interpretation impossible.

\begin{exe}
\ex \label{ex:lupea.je}
\gll tɕi-kɯm lu-pe-a je! \\
\textsc{1du}.\textsc{poss}-door \textsc{ipfv}:\textsc{upstream}-close[III]-\textsc{1sg} \textsc{sfp} \\
\glt `Let me close the door (for us).' (conversation, 03-05-2018, Tshendzin)
\end{exe}

It is debatable whether the orientation prefixes in examples like (\ref{ex:lomWrkW}) and (\ref{ex:lupea.je}) could be analyzed as marking associated motion (in Guillaume's (\citeyear{guillaume16am} definition), as these prefixes do not actually imply that a motion event takes or does not take place. What they specify is that if the main action is linked with a translational motion event, that motion event follows the direction indicated by the prefix. In this grammar, the use of orientation prefixes in examples such as (\ref{ex:lomWrkW}) and (\ref{ex:lupea.je}) are therefore not considered to be cases of associated motion.

\subsection{Echo phenomena} \label{sec:AM.echo}
Previous literature on AM has reported the existence of `echo phenomena' in the use of AM markers (\citealt[251]{wilkins91associated.motion}, \citealt[681-683]{vuillermet12eseejja}, \citealt[128-130]{rose15am}, \citealt[11]{guillaume16am}), namely that the same motion event can be expressed by more than one AM marker. This phenomenon is common in Japhug narratives. Two subtypes of AM echo can be distinguished.

First, in examples such as (\ref{ex:CtAru}) and (\ref{ex:GWYWsloR}), a motion verb (\japhug{ɕe}{go} and \japhug{ɣi}{come} respectively) is followed by a verb with an AM prefix with the same deixis, though there is a single motion event.

\begin{exe}
\ex \label{ex:CtAru}
\gll tɕʰi ɯ-taʁ to-ɕe tɕe ɕ-tɤ-ru   \\
stairs \textsc{3sg}.\textsc{poss}-on \textsc{ifr}:\textsc{up}-go \textsc{lnk}  \textsc{transloc}-\textsc{up}:\textsc{pfv}-look \\
\glt `He went up the stairs and looked up.'  (08-kWqhi, 18)
\end{exe}

\begin{exe}
\ex \label{ex:GWYWsloR}
\gll kʰa mɯ-pɯ-rɤʑi tɕe tɕe, ftɕar nɯ wuma ʑo βɣɯz pjɤ-rɯŋɯŋɤn tɕe maka,
kɯmtʰoʁ ra kɯnɤ ju-ɣi ɣɯ-ɲɯ-sloʁ pjɤ-ŋu. \\
house \textsc{neg}-\textsc{pst}.\textsc{ipfv}-stay \textsc{lnk} \textsc{lnk} summer \textsc{dem} really \textsc{emph} badger \textsc{ifr}.\textsc{ipfv}-cause.damage \textsc{lnk} completely threshold \textsc{pl} also \textsc{ipfv}-come \textsc{cisloc}-\textsc{ipfv}-dig.up \textsc{ifr}.\textsc{ipfv}-be \\
\glt `He was not home, and that summer badgers were causing a lot of damages, they came and even dug up  the threshold of the house.'  (27-spjaNkW, 107)
\end{exe}

Second, we also find cases, such as (\ref{ex:GWtaBzu}), without a motion verb, but with two verbs redundantly prefixed with the same AM marker (here \forme{ɣɯ-}).

\begin{exe}
\ex \label{ex:GWtaBzu}
\gll  tɕe a-kʰa ra ɣɯ-ta-rɤroʁrɯz, 	a-mgo  ra ɣɯ-ta-βzu ŋu ɕi \\
\textsc{lnk} \textsc{1sg}.\textsc{poss}-house \textsc{pl} \textsc{cisloc}-\textsc{pfv}:3$\rightarrow$3'-tidy 
 \textsc{1sg}.\textsc{poss}-food \textsc{pl} \textsc{cisloc}-\textsc{pfv}:3$\rightarrow$3'-make be:\textsc{fact} \textsc{qu} \\ 
\glt `Is it (the neighbour's wife who took pity on me) and came to tidy my house and make food for me?'  (150827 tianluo, 76)
\end{exe}


Echo AM is required in serial verb constructions (\citealt[253-255]{jacques16complementation}, § XXX), as shown in (\ref{ex:CkunWrtCe}), where the verbs \japhug{stu}{do like} and the \japhug{nɯrtɕa}{tease} share the same person (3$\rightarrow$3'), TAM (imperfective) and AM (translocative) markers.

\begin{exe}
\ex \label{ex:CkunWrtCe}
\gll  kɯra ɕ-tu-ste tɕe ɕ-ku-nɯrtɕe ra pjɤ-ŋu. \\
\textsc{dem}:\textsc{prox}:\textsc{pl} \textsc{transloc}-\textsc{ipfv}-do.like[III] \textsc{lnk}  \textsc{transloc}-\textsc{ipfv}-tease[III] \textsc{pl} \textsc{ifr}.\textsc{ipfv}-be \\
\glt `(The mouse) went and teased (the cat) like that.' (150902 dashu, 31)
\end{exe}

\subsection{Associated motion vs motion verb construction}
To express the meaning of motion prior to an action, associated motion prefixes are nearly two times as common as corresponding motion verb constructions (henceforth MVC) in the Japhug corpus. There is however a clear semantic difference between the two constructions, which was briefly described in \citet{jacques13harmonization}, but is presented here in more detail.

AM and MVC differ from each other in that in the former, the completion of both motion event and verbal action is presupposed (AM is monoactional), whereas in the case of the latter, the two can be separated. This mono- vs. pluractionality contrast is most conspicuous in past perfective forms, and can be observed in four types of constructions: concessives (with negation of the verbal action), interrogatives, conditionals and complement clauses. 

Another difference between MVC and AM is the fact that while MVC require a volitional verb in the purposive complete, there is no such requirement for the AM markers.

\subsubsection{Concessive} \label{sec:am.concessive}
A MVC  with the motion verb in perfective form can be followed by a clause negating the purposive action, as in (\ref{ex:nAkWrtoR}). In this example, only the motion is realized, while the action expressed by the verb \japhug{rtoʁ}{look} could not be accomplished.

\begin{exe}
\ex \label{ex:nAkWrtoR}
\gll nɤ-kɯ-rtoʁ jɤ-ɣe-a ri, mɯ-nɯ-atɯɣ-tɕi, mɯ-pɯ-ta-mto. \\
\textsc{1sg.poss}-\textsc{nmlz}:S/A-see \textsc{pfv}-come[II]-\textsc{1sg} \textsc{lnk} \textsc{neg}-\textsc{pfv}-meet-\textsc{1du} \textsc{neg}-\textsc{pfv}-1\fl2-see \\
\glt `I came to see you but I did not see you.' 
\end{exe}

With the corresponding AM verb form \japhug{ɣɯ-jɤ-ta-rtoʁ}{I came to see you}, negating the action of the verb is self-contradictory and nonsensical, and a sentence such as (\ref{ex:GWjAtartoR}) is incorrect.

\begin{exe}
\ex \label{ex:GWjAtartoR}
\gll $\dagger$ɣɯ-jɤ-ta-rtoʁ ri mɯ-pɯ-ta-mto \\
\textsc{cisloc}-\textsc{pfv}-1\fl2-look \textsc{lnk} \textsc{neg}-\textsc{pfv}-1\fl2-see \\
\glt Intended meaning: `I came to see you but I did not see you.' 
\end{exe}

Additional minimal pairs of the same type are presented in \citet[202-203]{jacques13harmonization}.

Example (\ref{ex:mWjsWntsGe}) from a conversation illustrates this property also with a manipulative verb \japhug{ɣɯt}{bring}: the action of the purposive complement  \japhug{kɤ-ntsɣe}{to sell} is negated in the following clause (with an abilitative \forme{sɯ-}, see § XXX).

 \begin{exe}
\ex \label{ex:mWjsWntsGe}
 \gll   sɤnɤmmtsʰu kɯ kɤ-ntsɣe cʰɤ-ɣɯt ri mɯ́j-sɯ-ntsɣe ndɤre, \\
 p.n. \textsc{erg} \textsc{inf}-sell \textsc{ifr}:\textsc{downstream}-bring \textsc{lnk} \textsc{neg}:\textsc{sens}-\textsc{abil}-sell \textsc{lnk} \\
\glt `Bsod.nams.mtsho brought them (to Mbarkham) to sell, but could not sell it.' (conversation, 14.05.10)
 \end{exe}

 

\subsubsection{Interrogative} \label{sec:am.interrogative}
In interrogative clauses, MVCs are required to express meanings such as `What/who have you come/gone to X', as in example (\ref{ex:tChi.WkWpa}), an example which occurs nine times in the corpus.

\begin{exe}
\ex \label{ex:tChi.WkWpa}
\gll tɕʰi ɯ-kɯ-pa jɤ-tɯ-ɣe? \\
what \textsc{3sg.poss}-do \textsc{pfv}-2-come[II] \\
\glt `What did you come to do?' (nine examples in the corpus)
\end{exe}

The difference between MVC and AM in interrogatives can be illustrated by comparing the minimal pair  (\ref{ex:tChi.WkWndza}) and (\ref{ex:tChi.GWtAtWndzat}). Example (\ref{ex:tChi.WkWndza}), which has the same structure as (\ref{ex:tChi.WkWpa}), implies that the addressee has not eaten yet, while (\ref{ex:tChi.GWtAtWndzat}) with associated motion can only be used if the food ingestion has already taken place, and requires a different translation.

\begin{exe}
\ex \label{ex:tChi.WkWndza}
\gll tɕʰi ɯ-kɯ-ndza jɤ-tɯ-ɣe? \\
what \textsc{3sg.poss}-eat \textsc{pfv}-2-come[II] \\
\glt `What have you come to eat?' (elicited)
\end{exe}

\begin{exe}
\ex \label{ex:tChi.GWtAtWndzat}
\gll tɕʰi ɣɯ-tɤ-tɯ-ndza-t \\
what \textsc{cisloc}-\textsc{pfv}-2-eat-\textsc{pst:tr}    \\
\glt `What did you eat upon coming here?' (elicited)
\end{exe}

\subsubsection{Conditional} \label{sec:am.conditional}
The presuppositional difference between MVC and AM is also perceptible in the protasis of conditional clauses. 

With MVC in the protasis as in (\ref{ex:mWmAjAtWGe}), there is no presupposition that the verbal action took place, only the motion event constitutes a condition to the state of affair described in the apodosis.

\begin{exe}
\ex \label{ex:mWmAjAtWGe}
\gll nɤ-wa ɯ-kɯ-rtoʁ mɯ\redp{}mɤ-jɤ-tɯ-ɣe nɤ aʑo mɯ-pɯ-kɯ-mto-a. \\
\textsc{1sg.poss}-father \textsc{3sg.poss-}\textsc{nmlz}:S/A-look \textsc{cond}\redp{}\textsc{neg}-\textsc{pfv}-2-come[II] \textsc{lnk} \textsc{1sg} \textsc{neg}-\textsc{pfv}-2\fl{}1-\textsc{1sg} \\
\glt `If you had not come to see your father, you would not have seen me.' (you saw me, but your father was not here)
\end{exe}

By contrast, with AM, the verbal action necessarily took place, as in example (\ref{ex:mWmAGWjAtWrtoR}).

\begin{exe}
\ex \label{ex:mWmAGWjAtWrtoR}
\gll nɤ-wa  mɯ\redp{}mɤ-ɣɯ-jɤ-tɯ-rtoʁ nɤ pɯ-sɤzdɯxpa \\
\textsc{1sg.poss}-father \textsc{cond}\redp{}\textsc{neg}-\textsc{cisloc}-\textsc{pfv}-2-look \textsc{lnk} \textsc{pst.ipfv}-be.pitiful \\ 
\glt `If you had not come to see your father, he would have felt sorry.' (but you did saw him, so he does not feel sorry)
\end{exe}

\subsubsection{Complement clauses} \label{sec:am.complement}
In complement clauses, verbs with AM prefixes are attested, and complement taking verbs always have scope over both the action of the verb and motion event.

 
In (\ref{ex:mACWkAtshi}), the modal verb \japhug{cʰa}{can} and the double negations (with the specific meaning `cannot help', § XXX) have scope over both the motion event and the verbal action -- this example is taken from a passage in a story where the king reproaches a small child, who just returned from a mission he himself send him to accomplish, not to have first come to greet him on his return home; the child says these words to justify why he first went to see his mother before greeting the king -- from this context it is clear that both the motion event (to him mother's house, explaining the child's failure to go to see the king) and the action `drink milk' (the reason for that motion event) are equally important to the plot and inseparable. 

\begin{exe}
\ex \label{ex:mACWkAtshi}
\gll  tɯ-nɯ ɯ-kɯ-tsʰi ɲɯ-ɕti-a tɕe, jɤ-azɣɯt-a tɕe, tɯ-nɯ ci mɤ-ɕɯ-kɤ-tsʰi nɯ mɯ́j-cʰa-a \\
\textsc{indef}.\textsc{poss}-breast \textsc{3sg}.\textsc{poss}-\textsc{nmlz}:S/A-drink \textsc{sens}-be.\textsc{affirm}-\textsc{1sg} \textsc{lnk} \textsc{pfv}-arrive-\textsc{1sg} \textsc{lnk} \textsc{indef}.\textsc{poss}-breast  \textsc{indef} \textsc{neg}-\textsc{transloc}-\textsc{inf}-drink \textsc{dem} \textsc{neg}:\textsc{sens}-can-\textsc{1sg} \\
\glt `I am (a toddler) who (still) drinks (his mother's) milk, when I arrived, I could not help but go to drink milk.'  (Norbzang, 262)
 \end{exe}
 
 In (\ref{ex:CWkAmWrkW.mAtWcha}), the negated modal verb has also on the action of both the main verb and the motion event -- the guards would prevent the main character not only to steal, but also to the where the object to be stolen is found. 
 
\begin{exe}
\ex \label{ex:CWkAmWrkW.mAtWcha}
\gll ʁmaʁ χsɯ-tɤkʰar kɯ ɲɯ-ɤz-nɤkʰar-nɯ ɕti tɕe, ɕɯ-kɤ-mɯrkɯ mɤ-tɯ-cʰa  \\
solider three-rounds \textsc{erg} \textsc{sens}-\textsc{prog}-surround-\textsc{pl} be.\textsc{affirm}:\textsc{fact} \textsc{lnk}  \textsc{transloc}-\textsc{inf}-steal \textsc{neg}-2-can:\textsc{fact} \\
\glt `Three rounds of soldiers will be surrounding it, you will not be able to (go there and) steal it.' (2003qachga, 55)
   \end{exe}
   
 Examples (\ref{ex:GWkAcW}) and (\ref{ex:CWkAmtChot}) illustrate the scope of aspectual  auxiliary verbs (here  \japhug{atsu}{have the time to} and \japhug{mda}{be time to}) on both motion event and verbal action.  In (\ref{ex:CWkAmtChot}), note that the infinitive form with AM \japhug{ɕɯ-kɤ-mtɕʰot}{go and make offerings}  translates the Chinese festival \ch{清明节}{qīngmíngjié}{Tomb-Sweeping Day} (using a verb borrowed from Tibetan  \tibet{མཆོད་}{mtɕʰod}{make offerings}). There was no motion verb in the original text.
 
\begin{exe}
\ex \label{ex:GWkAcW}
\gll qʰe potɯrʑi kɯ nɤ-kɯm ɣɯ-kɤ-cɯ mɤ-atsu ma \\
\textsc{lnk} p.n. \textsc{erg} \textsc{2sg}.\textsc{poss}-door \textsc{cisloc}-\textsc{inf}-open \textsc{neg}-have.the.time.to \textsc{lnk} \\
\glt `Bod.rje does not have time to come and open the door for you.' (2010 meimei de gushi, 21)
\end{exe} 
  
\begin{exe}
\ex \label{ex:CWkAmtChot}
\gll tɯrsa ɕɯ-kɤ-mtɕʰot to-mda ɲɯ-ŋu \\
grave \textsc{transloc}-\textsc{inf}-make.offerings \textsc{ifr}-be.time.to \textsc{sens}-be \\
\glt `It was the time to (go and) make offerings for the graves.' (160630 abao-zh, 70)
 \end{exe} 
 
The same scopal effect also applies to  verbs with AM in complement clauses selected by a verb in the protasis, as in (\ref{ex:CWkAru}) and (\ref{ex:CWkAmWrkW}): the realization of the verbal action (in addition to that of the motion event) belongs to the condition.

\begin{exe}
\ex \label{ex:CWkAru}
\gll ɕɯ-kɤ-ru mɯ\redp{}mɤ-pɯ-tɯ-cha ŋu nɤ nɤ-srɤm nɤ-sroʁ lɤt-i \\
\textsc{transloc}-\textsc{inf}-bring \textsc{cond}\redp{}\textsc{neg}-2-can:\textsc{fact} be:\textsc{fact} \textsc{lnk} \textsc{1sg.poss}-root \textsc{1sg.poss}-life throw:\textsc{fact}-\textsc{1pl} \\ 
\glt `If you are not able to bring it here, we will have your root and your life.' (Norbzang, 10)
\end{exe}

\begin{exe}
\ex \label{ex:CWkAmWrkW}
\gll nɤʑo ɕɯ-kɤ-mɯrkɯ a-pɯ-tɯ-cʰa nɤ aʑo cʰɯ-sɯ-jɣat-a jɤɣ \\
\textsc{2sg} \textsc{transloc}-\textsc{inf}-steal \textsc{irr}-\textsc{ipfv}-2-can \textsc{lnk} \textsc{1sg} \textsc{ipfv}-\textsc{caus}-go.back-\textsc{1sg} be.agreed:\textsc{fact} \\
\glt `If you succeed stealing it (after having gone there), I can cause him to go back there.' (02-montagnes-kamnyu, 46)
\end{exe}

 
By contrast, in  (\ref{ex:kWrAma.kACe}), in the case of the infinitival complement \forme{kɯ-rɤma kɤ-ɕe} `go to work' with a purposive clause \forme{kɯ-rɤma} (§ XXX), the main verb \japhug{mda}{be time to} only has scope over the motion event expressed by the verb \japhug{ɕe}{go} -- the time that is indicated by the stars refers to the beginning of the journey to work, not the start of the work itself. Compare this example in particular with (\ref{ex:CWkAmtChot}) above, with the same auxiliary verb.
 
 \begin{exe}
\ex \label{ex:kWrAma.kACe}
\gll  tɕe kɯɕɯŋgɯ tɕe tɯtsʰot pɯ-me tɕe  nɯnɯ cʰɯ-ɬoʁ lu-ɕqʰlɤt nɯra ɕ-tu-kɯ-ru tɕe, nɯnɯ kɤ-rɤru mda mɤ-mda cʰondɤre kɯ-rɤma kɤ-ɕe mda mɤ-mda nɯtɕu ɕ-tu-kɯ-ru pɯ-ŋgrɤl. \\
 \textsc{lnk} long.ago \textsc{lnk} clock \textsc{pst}.\textsc{ipfv}-not.exist \textsc{lnk} \textsc{dem} \textsc{ipfv}:\textsc{downstream}-come.out \textsc{ipfv}:\textsc{upstream}-disappear \textsc{dem}:\textsc{pl} \textsc{transloc}-\textsc{ipfv}:up-\textsc{genr}:S/P-look \textsc{lnk} \textsc{dem} \textsc{inf}-get.up be.time:\textsc{fact} \textsc{neg}-be.time:\textsc{fact} \textsc{comit} \textsc{nmlz}:S/A-work \textsc{inf}-go be.time:\textsc{fact} \textsc{neg}-be.time:\textsc{fact} \textsc{dem}:\textsc{loc} \textsc{transloc}-\textsc{ipfv}:up-\textsc{genr}:S/P-look \textsc{pst}.\textsc{ipfv}-be.usually.the.case  \\
 \glt  `In former times, there was no clock, and people used to watch when (these stars) came out or disappeared (to know) whether it was time to get up or go to work.' (29-LAntshAm, 66)
  \end{exe}
  
\subsubsection{Volitionality and controllability}
An additional difference between AM and MVC has to do with volitionality and/or controllability. In the case of an MVC, the verb in the purposive clause, whose action follows the motion event, is always necessarily volitional and controllable. By contrast, in the case of AM, it is possible to find examples where the verbal action expresses a non-controllable event, such as the action of finding a lost object in example (\ref{ex:CpjAmto}) with echo phenomenon (cf § \ref{sec:AM.echo}). Note that there are no examples of the non-volitional verb \japhug{mto}{see} with the MVC in the corpus (the volitional \japhug{rtoʁ}{see, look} or \japhug{ru}{look} occur  instead). 

\begin{exe}
\ex  \label{ex:CpjAmto}
\gll  nɯɕɯmɯma ʑo tɯ-ci ɯ-ŋgɯ pjɤ-ɕe qʰe iɕqʰa tɤɕime kɯ ɯ-sɤcɯ pɯ-kɤ-nɯ-ɕlɯɣ nɯ ɕ-pjɤ-mto. \\
immediately \textsc{emph} \textsc{indef}.\textsc{poss}-water \textsc{3sg}.\textsc{poss}-inside \textsc{ifr}:\textsc{down}-go \textsc{lnk} the.aforementioned lady \textsc{erg} \textsc{3sg}.\textsc{poss}-key \textsc{pfv}:\textsc{down}-\textsc{nmlz}:P-\textsc{auto}-drop \textsc{dem} \textsc{transloc}-\textsc{ifr}-see \\
\glt `He went immediately into the water and saw there the key that the lady had dropped by mistake.' (140510 fengwang, 118)
\end{exe}

%tɕe saɕɯ nɯ tɯrgi ɯ-locu zɯ ɕ-to-ɬoʁ tɕe, => vraiment mouvement?