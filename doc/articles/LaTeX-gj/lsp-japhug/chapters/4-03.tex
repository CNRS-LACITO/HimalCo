\chapter{Orientation and associated motion}
\section{Associated motion}
\subsection{Associated motion vs motion verb construction}
To express the meaning of motion prior to an action, associated motion prefixes are nearly two times as common as corresponding motion verb constructions (henceforth MVC) in the Japhug corpus. There is however a clear semantic difference between the two constructions, which was briefly described in \citet{jacques13harmonization}, but is presented here in more detail.

AM and MVC differ from each other in that in the former, the completion of both motion event and verbal action is presupposed (AM is monoactional), whereas in the case of the latter, the two can be separated. This mono- vs. pluractionality contrast is most conspicuous in past perfective forms, and can be observed in four types of constructions: concessives (with negation of the verbal action), interrogatives, conditionals and complement clauses.

\subsubsection{Concessive} \label{sec:am.concessive}
A MVC  with the motion verb in perfective form can be followed by a clause negating the purposive action, as in (\ref{ex:nAkWrtoR}). In this example, only the motion is realized, while the action expressed by the verb \japhug{rtoʁ}{look} could not be accomplished.

\begin{exe}
\ex \label{ex:nAkWrtoR}
\gll nɤ-kɯ-rtoʁ jɤ-ɣe-a ri, mɯ-nɯ-atɯɣ-tɕi, mɯ-pɯ-ta-mto. \\
\textsc{1sg.poss}-\textsc{nmlz}:S/A-see \textsc{pfv}-come[II]-\textsc{1sg} \textsc{lnk} \textsc{neg}-\textsc{pfv}-meet-\textsc{1du} \textsc{neg}-\textsc{pfv}-1\fl2-see \\
\glt `I came to see you but I did not see you.' 
\end{exe}

With the corresponding AM verb form \japhug{ɣɯ-jɤ-ta-rtoʁ}{I came to see you}, negating the action of the verb is self-contradictory and nonsensical, and a sentence such as (\ref{ex:GWjAtartoR}) is incorrect.

\begin{exe}
\ex \label{ex:GWjAtartoR}
\gll $\dagger$ɣɯ-jɤ-ta-rtoʁ ri mɯ-pɯ-ta-mto \\
\textsc{cisloc}-\textsc{pfv}-1\fl2-look \textsc{lnk} \textsc{neg}-\textsc{pfv}-1\fl2-see \\
\glt Intended meaning: `I came to see you but I did not see you.' 
\end{exe}

Additional minimal pairs of the same type are presented in \citet[202-203]{jacques13harmonization}.

\subsubsection{Interrogative} \label{sec:am.interrogative}
In interrogative clauses, MVCs are required to express meanings such as `What/who have you come/gone to X', as in example (\ref{ex:tChi.WkWpa}), an example which occurs nine times in the corpus.

\begin{exe}
\ex \label{ex:tChi.WkWpa}
\gll tɕʰi ɯ-kɯ-pa jɤ-tɯ-ɣe? \\
what \textsc{3sg.poss}-do \textsc{pfv}-2-come[II] \\
\glt `What did you come to do?' (nine examples in the corpus)
\end{exe}

The difference between MVC and AM in interrogatives can be illustrated by comparing the minimal pair  (\ref{ex:tChi.WkWndza}) and (\ref{ex:tChi.GWtAtWndzat}). example (\ref{ex:tChi.WkWndza}), which has the same structure as (\ref{ex:tChi.WkWpa}), implies that the addressee has not eaten yet, while (\ref{ex:tChi.GWtAtWndzat}) with associated motion can only be used if the food ingestion has already taken place, and requires a different translation.

\begin{exe}
\ex \label{ex:tChi.WkWndza}
\gll tɕʰi ɯ-kɯ-ndza jɤ-tɯ-ɣe? \\
what \textsc{3sg.poss}-eat \textsc{pfv}-2-come[II] \\
\glt `What have you come to eat?' (elicited)
\end{exe}

\begin{exe}
\ex \label{ex:tChi.GWtAtWndzat}
\gll tɕʰi ɣɯ-tɤ-tɯ-ndza-t \\
what \textsc{cisloc}-\textsc{pfv}-2-eat-\textsc{pst:tr}    \\
\glt `What did you eat, after you came here?' (elicited)
\end{exe}

\subsubsection{Conditional} \label{sec:am.conditional}
The presuppositional difference between MVC and AM is also perceptible in the protasis of conditional clauses. 

With MVC in the protasis as in (\ref{ex:mWmAjAtWGe}), there is no presupposition that the verbal action took place, only the motion event constitutes a condition to the state of affair described in the apodosis.

\begin{exe}
\ex \label{ex:mWmAjAtWGe}
\gll nɤ-wa ɯ-kɯ-rtoʁ mɯ\redp{}mɤ-jɤ-tɯ-ɣe nɤ aʑo mɯ-pɯ-kɯ-mto-a. \\
\textsc{1sg.poss}-father \textsc{3sg.poss-}\textsc{nmlz}:S/A-look \textsc{cond}\redp{}\textsc{neg}-\textsc{pfv}-2-come[II] \textsc{lnk} \textsc{1sg} \textsc{neg}-\textsc{pfv}-2\fl{}1-\textsc{1sg} \\
\glt `If you had not come to see your father, you would not have seen me.' (you saw me, but your father was not here)
\end{exe}

By contrast, with AM, the verbal action necessarily took place, as in example (\ref{ex:mWmAGWjAtWrtoR}).

\begin{exe}
\ex \label{ex:mWmAGWjAtWrtoR}
\gll nɤ-wa  mɯ\redp{}mɤ-ɣɯ-jɤ-tɯ-rtoʁ nɤ pɯ-sɤzdɯxpa \\
\textsc{1sg.poss}-father \textsc{cond}\redp{}\textsc{neg}-\textsc{cisloc}-\textsc{pfv}-2-look \textsc{lnk} \textsc{pst.ipfv}-be.pitiful \\ 
\glt `If you had not come to see your father, he would have felt sorry.' (but you did saw him, so he does not feel sorry)
\end{exe}

\subsubsection{Complement clauses} \label{sec:am.conditional}
In complement clauses, verbs with AM prefixes are attested, and the monoactionality condition is still verified. 

In (\ref{ex:mACWkAtshi}) for instance, the modal verb \japhug{cʰa}{can} and the double negations (with the specific meaning `cannot help') have scope over both the motion event and the verbal action -- this example is taken from a passage in a story where the king reproaches a small child, who just returned from a mission he himself send him to accomplish, not to have first come to greet him on his return home; the child says these words to justify why he first went to see his mother before greeting the king -- from this context it is clear that both the motion event (to him mother's house, explaining the child's failure to go to see the king) and the action `drink milk' (the reason for that motion event) are equally important to the plot and inseparable. 

\begin{exe}
\ex \label{ex:mACWkAtshi}
\gll  tɯ-nɯ ɯ-kɯ-tsʰi ɲɯ-ɕti-a tɕe, jɤ-azɣɯt-a tɕe, tɯ-nɯ ci mɤ-ɕɯ-kɤ-tsʰi nɯ mɯ́j-cʰa-a \\
\textsc{indef}.\textsc{poss}-breast \textsc{3sg}.\textsc{poss}-\textsc{nmlz}:S/A-drink \textsc{sens}-be.\textsc{affirm}-\textsc{1sg} \textsc{lnk} \textsc{pfv}-arrive-\textsc{1sg} \textsc{lnk} \textsc{indef}.\textsc{poss}-breast  \textsc{indef} \textsc{neg}-\textsc{transloc}-\textsc{inf}-drink \textsc{dem} \textsc{neg}:\textsc{sens}-can-\textsc{1sg} \\
\glt `I am (a toddler) who (still) drinks (his mother's) milk, when I arrived, I could not help but go to drink milk.'  (Norbzang, 262)
 \end{exe}
 
By contrast, in  (\ref{ex:kWrAma.kACe}), in the case of the infinitival complement \forme{kɯ-rɤma kɤ-ɕe} `go to work' with a purposive clause \forme{kɯ-rɤma} (§ XXX), the main verb \japhug{mda}{be time to} only has scope over the motion event expressed by the verb \japhug{ɕe}{go} -- the time that is indicated by the stars refers to the beginning of the journey to work, not the start of the work itself.
 
 \begin{exe}
\ex \label{ex:kWrAma.kACe}
\gll  tɕe kɯɕɯŋgɯ tɕe tɯtsʰot pɯ-me tɕe  nɯnɯ cʰɯ-ɬoʁ lu-ɕqʰlɤt nɯra ɕ-tu-kɯ-ru tɕe, nɯnɯ kɤ-rɤru mda mɤ-mda cʰondɤre kɯ-rɤma kɤ-ɕe mda mɤ-mda nɯtɕu ɕ-tu-kɯ-ru pɯ-ŋgrɤl. \\
 \textsc{lnk} long.ago \textsc{lnk} clock \textsc{pst}.\textsc{ipfv}-not.exist \textsc{lnk} \textsc{dem} \textsc{ipfv}:\textsc{downstream}-come.out \textsc{ipfv}:\textsc{upstream}-disappear \textsc{dem}:\textsc{pl} \textsc{transloc}-\textsc{ipfv}:up-\textsc{genr}:S/P-look \textsc{lnk} \textsc{dem} \textsc{inf}-get.up be.time:\textsc{fact} \textsc{neg}-be.time:\textsc{fact} \textsc{comit} \textsc{nmlz}:S/A-work \textsc{inf}-go be.time:\textsc{fact} \textsc{neg}-be.time:\textsc{fact} \textsc{dem}:\textsc{loc} \textsc{transloc}-\textsc{ipfv}:up-\textsc{genr}:S/P-look \textsc{pst}.\textsc{ipfv}-be.usually.the.case  \\
 \glt  `In former times, there was no clock, and people used to watch when (these stars) came out or disappeared (to know) whether it was time to get up or go to work.' (29-LAntshAm, 66)
  \end{exe}
  
  This monoactionality constraint also applies to verbs with AM in complement clauses selected by a verb in the protasis, as in (\ref{ex:CWkAru}) and (\ref{ex:CWkAmWrkW}): the realization of both the verbal action and the motion event belong to the condition.  Note that in the case of (\ref{ex:CWkAmWrkW}), it would be clumsy to render the Japhug AM marker by a MVC in English -- `if you succeed going to steal it' would mean that only the motion event is part of the condition, not the act of stealing.

\begin{exe}
\ex \label{ex:CWkAru}
\gll ɕɯ-kɤ-ru mɯ\redp{}mɤ-pɯ-tɯ-cha ŋu nɤ nɤ-srɤm nɤ-sroʁ lɤt-i \\
\textsc{transloc}-\textsc{inf}-bring \textsc{cond}\redp{}\textsc{neg}-2-can:\textsc{fact} be:\textsc{fact} \textsc{lnk} \textsc{1sg.poss}-root \textsc{1sg.poss}-life throw:\textsc{fact}-\textsc{1pl} \\ 
\glt `If you are not able to bring it here, we will have your root and your life.' (2003qachga, 37)
\end{exe}

\begin{exe}
\ex \label{ex:CWkAmWrkW}
\gll nɤʑo ɕɯ-kɤ-mɯrkɯ a-pɯ-tɯ-cʰa nɤ aʑo cʰɯ-sɯ-jɣat-a jɤɣ \\
\textsc{2sg} \textsc{transloc}-\textsc{inf}-steal \textsc{irr}-\textsc{ipfv}-2-can \textsc{lnk} \textsc{1sg} \textsc{ipfv}-\textsc{caus}-go.back-\textsc{1sg} be.agreed:\textsc{fact} \\
\glt `If you succeed stealing it (after having gone there), I can cause him to go back there.' (02-montagnes-kamnyu, 46)
\end{exe}

