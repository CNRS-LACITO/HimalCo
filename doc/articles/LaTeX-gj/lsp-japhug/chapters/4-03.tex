\chapter{Orientation and associated motion}
\section{Associated motion}

\subsection{Motion verbs and AM prefixes}
In Japhug, there is no constraint on AM prefixes occurring on motion verbs with the same deixis. Examples (\ref{ex:GWjuGinW}) and (\ref{ex:CpjACe}) respectively illustrate the cislocative on the verb \japhug{ɣi}{come} and the translocative on the verb \japhug{ɕe}{go}. Such examples are not common enough to allow a clear analysis of the semantic value of the redundant AM in these examples.

\begin{exe}
\ex \label{ex:GWjuGinW}
 \gll <jiazhang> ra ju-ɣi-nɯ tɕe <laoshi> ɯ-ɕki, tɯ-ɕki ʑo ɣɯ-ju-ɣi-nɯ ɕti netɕi? \\
 parents \textsc{pl} \textsc{ipfv}-come-\textsc{pl} \textsc{lnk} teacher \textsc{3sg}.\textsc{poss}-\textsc{dat} \textsc{genr}.\textsc{poss}-\textsc{dat} \textsc{emph} \textsc{cisloc}-\textsc{ipfv}-come-\textsc{pl} be.\textsc{affirm}:\textsc{fact} \textsc{sfp} \\
 \glt `The parents come, come to the teachers (us).' (conversation140501 01, 60)
\end{exe}

\begin{exe}
\ex \label{ex:CpjACe}
 \gll li nɤki iɕqʰa nɯ tɤjlu kɤ-rku ɯ-ŋgɯ zɯ ɕ-pjɤ-ɕe \\
 again \textsc{dem} the.aforementioned \textsc{dem} flour \textsc{nmlz}:P-put.in \textsc{3sg}.\textsc{poss}-inside \textsc{loc} \textsc{transloc}-\textsc{ifr}:\textsc{down}-go \\
 \glt `He went into the bag of flour.' (140519 chou xiaoya-zh, 145)
\end{exe}

The opposite combinations, namely cislocative with \japhug{ɕe}{go} and translocative with \japhug{ɣi}{come}, are not grammatical. 

\subsection{Orientation and AM}
In Japhug, AM markers only specify deixis and the temporal relation between motion event and verbal action, but are neutral as regards the orientation of the motion event.

Orientation and AM markers occupy distinct prefixal slots. Non-orientable verbs (verbs expressing actions other than motion, manipulation, sight or actions with a single direction, see § XXX) select one or two lexicalized orientations (see § XXX). For instance, the verb \japhug{mɯrkɯ}{steal} occurs with the orientation `up' (with the orientation prefixes \forme{tɤ-}, \forme{ta-}, \forme{tu-}, \forme{to-}). 

When non-orientable verbs occurs with AM, the verb normally keeps the lexicalized orientation prefix, as in \ref{ex:CtumWrki}, where \japhug{mɯrkɯ}{steal} is used with the \forme{tu-} `up' prefix; the orientation prefix is thus irrelevant to the motion event. 

\begin{exe}
\ex \label{ex:CtumWrki}
 \gll kɯ-nŋo nɯ qʰe ci ci ɕ-tu-mɯrki kɯ-fse ma nɯ ma mɯ-ɲɯ-ɤʁe. \\
\textsc{nmlz}:S/A-be.defeated \textsc{dem} \textsc{lnk} one one \textsc{transloc}-\textsc{ipfv}-steal[III] \textsc{nmlz}:S/A-be.like apart.from \textsc{dem} apart.from \textsc{neg}-\textsc{sens}-have.to.eat \\
\glt `The (lion) which is defeated steals a little out of it, but apart from that has nothing to eat.' (20-sWNgi, 65)
\end{exe}

In exceptional cases, however, it is possible to use the orientation prefix corresponding to direction of the motion event, as in (\ref{ex:ZlumWrkia}), where the `upstream' prefix occurs with \japhug{mɯrkɯ}{steal}, to express the fact that the character steals from a place located downstream to bring it upstream; however, it is unclear even in this case if the presence of AM is solely responsible for the occurrence of the `upstream' prefix on this verb form, as the same verb with the `upstream' prefix  without AM marker is found in the same story (see \ref{ex:lomWrkW}).\footnote{The first occurrences of the verb \japhug{mɯrkɯ}{steal} with the  `upstream' prefix in the story have the translocative prefix, but that AM marker is elided in the second half of the story.  }

\begin{exe}
\ex \label{ex:ZlumWrkia}
 \gll tɕetʰi tɤmuj jlɤrɯcɤrna ɣɯ ɯ-pʰe nɯtɕu kɯ-mɯrkɯ chɯ-ɕe-a ŋu tɕe. tsʰɤt ɯ-ʁrɯ ɣɯ ɯ-ci nɯnɯ ʑ-lu-mɯrki-a ri a-qʰu zɯ lɤ-ɣe-nɯ tɕe \\
 downstream p.n. p.n. gen \textsc{3sg}.\textsc{poss}-\textsc{dat} \textsc{dem}:\textsc{loc}  \textsc{nmlz}:S/A-steal  \textsc{ipfv}:\textsc{downstream}-go-\textsc{1sg} be:\textsc{fact} \textsc{lnk}  goat \textsc{3sg}.\textsc{poss}-horn \textsc{gen} \textsc{3sg}.\textsc{poss}-water \textsc{dem} \textsc{transloc}-\textsc{ipfv}:\textsc{upstream}-streal[III]-\textsc{1sg} \textsc{lnk} \textsc{1sg}.\textsc{poss}-after \textsc{loc} \textsc{pfv}:\textsc{upstream}-come[II]-\textsc{pl} \textsc{lnk}  \\
 \glt `(Tomorrow  morning) I will go downstream to steal from Tamuj Jlarukyarna, I will steal the water from the goat's horn, but when (the mountain god) comes after me...' (25-kAmYW-XpAltCin, 31-32)
\end{exe}

\begin{exe}
\ex \label{ex:lomWrkW}
 \gll lo-mɯrkɯ pjɤ-cʰa tɕe lo-ɣɯt ri \\
 \textsc{ifr}:\textsc{upstream}-steal \textsc{ifr}-can \textsc{lnk} \textsc{ifr}:\textsc{upstream}-bring \textsc{lnk} \\
\glt `He was able to steal it and brought it upstream.' (02-montagnes-kamnyu-cz, 31)
\end{exe}

\subsection{Echo phenomena} \label{sec:AM.echo}
Previous literature on AM has reported the existence of `echo phenomena' in the use of AM markers (\citealt[251]{wilkins91associated.motion}, \citealt[681-683]{vuillermet12eseejja}, \citealt[128-130]{rose15am}, \citealt[11]{guillaume16am}), namely that the same motion event can be expressed by more than one AM marker. This phenomenon is common in Japhug narratives. Two subtypes of AM echo can be distinguished.

First, in examples such as (\ref{ex:CtAru}) and (\ref{ex:GWYWsloR}), a motion verb (\japhug{ɕe}{go} and \japhug{ɣi}{come} respectively) is followed by a verb with an AM prefix with the same deixis, though there is a single motion event.

\begin{exe}
\ex \label{ex:CtAru}
\gll tɕʰi ɯ-taʁ to-ɕe tɕe ɕ-tɤ-ru   \\
stairs \textsc{3sg}.\textsc{poss}-on \textsc{ifr}:\textsc{up}-go \textsc{lnk}  \textsc{transloc}-\textsc{up}:\textsc{pfv}-look \\
\glt `He went up the stairs and looked up.'  (08-kWqhi, 18)
\end{exe}

\begin{exe}
\ex \label{ex:GWYWsloR}
\gll kʰa mɯ-pɯ-rɤʑi tɕe tɕe, ftɕar nɯ wuma ʑo βɣɯz pjɤ-rɯŋɯŋɤn tɕe maka,
kɯmtʰoʁ ra kɯnɤ ju-ɣi ɣɯ-ɲɯ-sloʁ pjɤ-ŋu. \\
house \textsc{neg}-\textsc{pst}.\textsc{ipfv}-stay \textsc{lnk} \textsc{lnk} summer \textsc{dem} really \textsc{emph} badger \textsc{ifr}.\textsc{ipfv}-cause.damage \textsc{lnk} completely threshold \textsc{pl} also \textsc{ipfv}-come \textsc{cisloc}-\textsc{ipfv}-dig.up \textsc{ifr}.\textsc{ipfv}-be \\
\glt `He was not home, and that summer badgers were causing a lot of damages, they came and even dug up  the threshold of the house.'  (27-spjaNkW, 107)
\end{exe}

Second, we also find cases, such as (\ref{ex:GWtaBzu}), without a motion verb, but with two verbs redundantly prefixed with the same AM marker (here \forme{ɣɯ-}).

\begin{exe}
\ex \label{ex:GWtaBzu}
\gll  tɕe a-kʰa ra ɣɯ-ta-rɤroʁrɯz, 	a-mgo  ra ɣɯ-ta-βzu ŋu ɕi \\
\textsc{lnk} \textsc{1sg}.\textsc{poss}-house \textsc{pl} \textsc{cisloc}-\textsc{pfv}:3$\rightarrow$3'-tidy 
 \textsc{1sg}.\textsc{poss}-food \textsc{pl} \textsc{cisloc}-\textsc{pfv}:3$\rightarrow$3'-make be:\textsc{fact} \textsc{qu} \\ 
\glt `Is it (the neighbour's wife who took pity on me) and came to tidy my house and make food for me?'  (150827 tianluo, 76)
\end{exe}


Echo AM is required in serial verb constructions (\citealt[253-255]{jacques16complementation}, § XXX), as shown in (\ref{ex:CkunWrtCe}), where the verbs \japhug{stu}{do like} and the \japhug{nɯrtɕa}{tease} share the same person (3$\rightarrow$3'), TAM (imperfective) and AM (translocative) markers.

\begin{exe}
\ex \label{ex:CkunWrtCe}
\gll  kɯra ɕ-tu-ste tɕe ɕ-ku-nɯrtɕe ra pjɤ-ŋu. \\
\textsc{dem}:\textsc{prox}:\textsc{pl} \textsc{transloc}-\textsc{ipfv}-do.like[III] \textsc{lnk}  \textsc{transloc}-\textsc{ipfv}-tease[III] \textsc{pl} \textsc{ifr}.\textsc{ipfv}-be \\
\glt `(The mouse) went and teased (the cat) like that.' (150902 dashu, 31)
\end{exe}

\subsection{Associated motion vs motion verb construction}
To express the meaning of motion prior to an action, associated motion prefixes are nearly two times as common as corresponding motion verb constructions (henceforth MVC) in the Japhug corpus. There is however a clear semantic difference between the two constructions, which was briefly described in \citet{jacques13harmonization}, but is presented here in more detail.

AM and MVC differ from each other in that in the former, the completion of both motion event and verbal action is presupposed (AM is monoactional), whereas in the case of the latter, the two can be separated. This mono- vs. pluractionality contrast is most conspicuous in past perfective forms, and can be observed in four types of constructions: concessives (with negation of the verbal action), interrogatives, conditionals and complement clauses. 

Another difference between MVC and AM is the fact that while MVC require a volitional verb in the purposive complete, there is no such requirement for the AM markers.

\subsubsection{Concessive} \label{sec:am.concessive}
A MVC  with the motion verb in perfective form can be followed by a clause negating the purposive action, as in (\ref{ex:nAkWrtoR}). In this example, only the motion is realized, while the action expressed by the verb \japhug{rtoʁ}{look} could not be accomplished.

\begin{exe}
\ex \label{ex:nAkWrtoR}
\gll nɤ-kɯ-rtoʁ jɤ-ɣe-a ri, mɯ-nɯ-atɯɣ-tɕi, mɯ-pɯ-ta-mto. \\
\textsc{1sg.poss}-\textsc{nmlz}:S/A-see \textsc{pfv}-come[II]-\textsc{1sg} \textsc{lnk} \textsc{neg}-\textsc{pfv}-meet-\textsc{1du} \textsc{neg}-\textsc{pfv}-1\fl2-see \\
\glt `I came to see you but I did not see you.' 
\end{exe}

With the corresponding AM verb form \japhug{ɣɯ-jɤ-ta-rtoʁ}{I came to see you}, negating the action of the verb is self-contradictory and nonsensical, and a sentence such as (\ref{ex:GWjAtartoR}) is incorrect.

\begin{exe}
\ex \label{ex:GWjAtartoR}
\gll $\dagger$ɣɯ-jɤ-ta-rtoʁ ri mɯ-pɯ-ta-mto \\
\textsc{cisloc}-\textsc{pfv}-1\fl2-look \textsc{lnk} \textsc{neg}-\textsc{pfv}-1\fl2-see \\
\glt Intended meaning: `I came to see you but I did not see you.' 
\end{exe}

Additional minimal pairs of the same type are presented in \citet[202-203]{jacques13harmonization}.

Example (\ref{ex:mWjsWntsGe}) from a conversation illustrates this property also with a manipulative verb \japhug{ɣɯt}{bring}: the action of the purposive complement  \japhug{kɤ-ntsɣe}{to sell} is negated in the following clause (with an abilitative \forme{sɯ-}, see § XXX).

 \begin{exe}
\ex \label{ex:mWjsWntsGe}
 \gll   sɤnɤmmtsʰu kɯ kɤ-ntsɣe cʰɤ-ɣɯt ri mɯ́j-sɯ-ntsɣe ndɤre, \\
 p.n. \textsc{erg} \textsc{inf}-sell \textsc{ifr}:\textsc{downstream}-bring \textsc{lnk} \textsc{neg}:\textsc{sens}-\textsc{abil}-sell \textsc{lnk} \\
\glt `Bsod.nams.mtsho brought them (to Mbarkham) to sell, but could not sell it.' (conversation, 14.05.10)
 \end{exe}

 

\subsubsection{Interrogative} \label{sec:am.interrogative}
In interrogative clauses, MVCs are required to express meanings such as `What/who have you come/gone to X', as in example (\ref{ex:tChi.WkWpa}), an example which occurs nine times in the corpus.

\begin{exe}
\ex \label{ex:tChi.WkWpa}
\gll tɕʰi ɯ-kɯ-pa jɤ-tɯ-ɣe? \\
what \textsc{3sg.poss}-do \textsc{pfv}-2-come[II] \\
\glt `What did you come to do?' (nine examples in the corpus)
\end{exe}

The difference between MVC and AM in interrogatives can be illustrated by comparing the minimal pair  (\ref{ex:tChi.WkWndza}) and (\ref{ex:tChi.GWtAtWndzat}). Example (\ref{ex:tChi.WkWndza}), which has the same structure as (\ref{ex:tChi.WkWpa}), implies that the addressee has not eaten yet, while (\ref{ex:tChi.GWtAtWndzat}) with associated motion can only be used if the food ingestion has already taken place, and requires a different translation.

\begin{exe}
\ex \label{ex:tChi.WkWndza}
\gll tɕʰi ɯ-kɯ-ndza jɤ-tɯ-ɣe? \\
what \textsc{3sg.poss}-eat \textsc{pfv}-2-come[II] \\
\glt `What have you come to eat?' (elicited)
\end{exe}

\begin{exe}
\ex \label{ex:tChi.GWtAtWndzat}
\gll tɕʰi ɣɯ-tɤ-tɯ-ndza-t \\
what \textsc{cisloc}-\textsc{pfv}-2-eat-\textsc{pst:tr}    \\
\glt `What did you eat upon coming here?' (elicited)
\end{exe}

\subsubsection{Conditional} \label{sec:am.conditional}
The presuppositional difference between MVC and AM is also perceptible in the protasis of conditional clauses. 

With MVC in the protasis as in (\ref{ex:mWmAjAtWGe}), there is no presupposition that the verbal action took place, only the motion event constitutes a condition to the state of affair described in the apodosis.

\begin{exe}
\ex \label{ex:mWmAjAtWGe}
\gll nɤ-wa ɯ-kɯ-rtoʁ mɯ\redp{}mɤ-jɤ-tɯ-ɣe nɤ aʑo mɯ-pɯ-kɯ-mto-a. \\
\textsc{1sg.poss}-father \textsc{3sg.poss-}\textsc{nmlz}:S/A-look \textsc{cond}\redp{}\textsc{neg}-\textsc{pfv}-2-come[II] \textsc{lnk} \textsc{1sg} \textsc{neg}-\textsc{pfv}-2\fl{}1-\textsc{1sg} \\
\glt `If you had not come to see your father, you would not have seen me.' (you saw me, but your father was not here)
\end{exe}

By contrast, with AM, the verbal action necessarily took place, as in example (\ref{ex:mWmAGWjAtWrtoR}).

\begin{exe}
\ex \label{ex:mWmAGWjAtWrtoR}
\gll nɤ-wa  mɯ\redp{}mɤ-ɣɯ-jɤ-tɯ-rtoʁ nɤ pɯ-sɤzdɯxpa \\
\textsc{1sg.poss}-father \textsc{cond}\redp{}\textsc{neg}-\textsc{cisloc}-\textsc{pfv}-2-look \textsc{lnk} \textsc{pst.ipfv}-be.pitiful \\ 
\glt `If you had not come to see your father, he would have felt sorry.' (but you did saw him, so he does not feel sorry)
\end{exe}

\subsubsection{Complement clauses} \label{sec:am.complement}
In complement clauses, verbs with AM prefixes are attested, and complement taking verbs always have scope over both the action of the verb and motion event.

 
In (\ref{ex:mACWkAtshi}), the modal verb \japhug{cʰa}{can} and the double negations (with the specific meaning `cannot help', § XXX) have scope over both the motion event and the verbal action -- this example is taken from a passage in a story where the king reproaches a small child, who just returned from a mission he himself send him to accomplish, not to have first come to greet him on his return home; the child says these words to justify why he first went to see his mother before greeting the king -- from this context it is clear that both the motion event (to him mother's house, explaining the child's failure to go to see the king) and the action `drink milk' (the reason for that motion event) are equally important to the plot and inseparable. 

\begin{exe}
\ex \label{ex:mACWkAtshi}
\gll  tɯ-nɯ ɯ-kɯ-tsʰi ɲɯ-ɕti-a tɕe, jɤ-azɣɯt-a tɕe, tɯ-nɯ ci mɤ-ɕɯ-kɤ-tsʰi nɯ mɯ́j-cʰa-a \\
\textsc{indef}.\textsc{poss}-breast \textsc{3sg}.\textsc{poss}-\textsc{nmlz}:S/A-drink \textsc{sens}-be.\textsc{affirm}-\textsc{1sg} \textsc{lnk} \textsc{pfv}-arrive-\textsc{1sg} \textsc{lnk} \textsc{indef}.\textsc{poss}-breast  \textsc{indef} \textsc{neg}-\textsc{transloc}-\textsc{inf}-drink \textsc{dem} \textsc{neg}:\textsc{sens}-can-\textsc{1sg} \\
\glt `I am (a toddler) who (still) drinks (his mother's) milk, when I arrived, I could not help but go to drink milk.'  (Norbzang, 262)
 \end{exe}
 
 In (\ref{ex:CWkAmWrkW.mAtWcha}), the negated modal verb has also on the action of both the main verb and the motion event -- the guards would prevent the main character not only to steal, but also to the where the object to be stolen is found. 
 
\begin{exe}
\ex \label{ex:CWkAmWrkW.mAtWcha}
\gll ʁmaʁ χsɯ-tɤkʰar kɯ ɲɯ-ɤz-nɤkʰar-nɯ ɕti tɕe, ɕɯ-kɤ-mɯrkɯ mɤ-tɯ-cʰa  \\
solider three-rounds \textsc{erg} \textsc{sens}-\textsc{prog}-surround-\textsc{pl} be.\textsc{affirm}:\textsc{fact} \textsc{lnk}  \textsc{transloc}-\textsc{inf}-steal \textsc{neg}-2-can:\textsc{fact} \\
\glt `Three rounds of soldiers will be surrounding it, you will not be able to (go there and) steal it.' (2003qachga, 55)
   \end{exe}
   
 Examples (\ref{ex:GWkAcW}) and (\ref{ex:CWkAmtChot}) illustrate the scope of aspectual  auxiliary verbs (here  \japhug{atsu}{have the time to} and \japhug{mda}{be time to}) on both motion event and verbal action.  In (\ref{ex:CWkAmtChot}), note that the infinitive form with AM \japhug{ɕɯ-kɤ-mtɕʰot}{go and make offerings}  translates the Chinese festival \ch{清明节}{qīngmíngjié}{Tomb-Sweeping Day} (using a verb borrowed from Tibetan  \tibet{མཆོད་}{mtɕʰod}{make offerings}). There was no motion verb in the original text.
 
\begin{exe}
\ex \label{ex:GWkAcW}
\gll qʰe potɯrʑi kɯ nɤ-kɯm ɣɯ-kɤ-cɯ mɤ-atsu ma \\
\textsc{lnk} p.n. \textsc{erg} \textsc{2sg}.\textsc{poss}-door \textsc{cisloc}-\textsc{inf}-open \textsc{neg}-have.the.time.to \textsc{lnk} \\
\glt `Bod.rje does not have time to come and open the door for you.' (2010 meimei de gushi, 21)
\end{exe} 
  
\begin{exe}
\ex \label{ex:CWkAmtChot}
\gll tɯrsa ɕɯ-kɤ-mtɕʰot to-mda ɲɯ-ŋu \\
grave \textsc{transloc}-\textsc{inf}-make.offerings \textsc{ifr}-be.time.to \textsc{sens}-be \\
\glt `It was the time to (go and) make offerings for the graves.' (160630 abao-zh, 70)
 \end{exe} 
 
The same scopal effect also applies to  verbs with AM in complement clauses selected by a verb in the protasis, as in (\ref{ex:CWkAru}) and (\ref{ex:CWkAmWrkW}): the realization of the verbal action (in addition to that of the motion event) belongs to the condition.

\begin{exe}
\ex \label{ex:CWkAru}
\gll ɕɯ-kɤ-ru mɯ\redp{}mɤ-pɯ-tɯ-cha ŋu nɤ nɤ-srɤm nɤ-sroʁ lɤt-i \\
\textsc{transloc}-\textsc{inf}-bring \textsc{cond}\redp{}\textsc{neg}-2-can:\textsc{fact} be:\textsc{fact} \textsc{lnk} \textsc{1sg.poss}-root \textsc{1sg.poss}-life throw:\textsc{fact}-\textsc{1pl} \\ 
\glt `If you are not able to bring it here, we will have your root and your life.' (Norbzang, 10)
\end{exe}

\begin{exe}
\ex \label{ex:CWkAmWrkW}
\gll nɤʑo ɕɯ-kɤ-mɯrkɯ a-pɯ-tɯ-cʰa nɤ aʑo cʰɯ-sɯ-jɣat-a jɤɣ \\
\textsc{2sg} \textsc{transloc}-\textsc{inf}-steal \textsc{irr}-\textsc{ipfv}-2-can \textsc{lnk} \textsc{1sg} \textsc{ipfv}-\textsc{caus}-go.back-\textsc{1sg} be.agreed:\textsc{fact} \\
\glt `If you succeed stealing it (after having gone there), I can cause him to go back there.' (02-montagnes-kamnyu, 46)
\end{exe}

 
By contrast, in  (\ref{ex:kWrAma.kACe}), in the case of the infinitival complement \forme{kɯ-rɤma kɤ-ɕe} `go to work' with a purposive clause \forme{kɯ-rɤma} (§ XXX), the main verb \japhug{mda}{be time to} only has scope over the motion event expressed by the verb \japhug{ɕe}{go} -- the time that is indicated by the stars refers to the beginning of the journey to work, not the start of the work itself. Compare this example in particular with (\ref{ex:CWkAmtChot}) above, with the same auxiliary verb.
 
 \begin{exe}
\ex \label{ex:kWrAma.kACe}
\gll  tɕe kɯɕɯŋgɯ tɕe tɯtsʰot pɯ-me tɕe  nɯnɯ cʰɯ-ɬoʁ lu-ɕqʰlɤt nɯra ɕ-tu-kɯ-ru tɕe, nɯnɯ kɤ-rɤru mda mɤ-mda cʰondɤre kɯ-rɤma kɤ-ɕe mda mɤ-mda nɯtɕu ɕ-tu-kɯ-ru pɯ-ŋgrɤl. \\
 \textsc{lnk} long.ago \textsc{lnk} clock \textsc{pst}.\textsc{ipfv}-not.exist \textsc{lnk} \textsc{dem} \textsc{ipfv}:\textsc{downstream}-come.out \textsc{ipfv}:\textsc{upstream}-disappear \textsc{dem}:\textsc{pl} \textsc{transloc}-\textsc{ipfv}:up-\textsc{genr}:S/P-look \textsc{lnk} \textsc{dem} \textsc{inf}-get.up be.time:\textsc{fact} \textsc{neg}-be.time:\textsc{fact} \textsc{comit} \textsc{nmlz}:S/A-work \textsc{inf}-go be.time:\textsc{fact} \textsc{neg}-be.time:\textsc{fact} \textsc{dem}:\textsc{loc} \textsc{transloc}-\textsc{ipfv}:up-\textsc{genr}:S/P-look \textsc{pst}.\textsc{ipfv}-be.usually.the.case  \\
 \glt  `In former times, there was no clock, and people used to watch when (these stars) came out or disappeared (to know) whether it was time to get up or go to work.' (29-LAntshAm, 66)
  \end{exe}
  
\subsubsection{Volitionality and controllability}
An additional difference between AM and MVC has to do with volitionality and/or controllability. In the case of an MVC, the verb in the purposive clause, whose action follows the motion event, is always necessarily volitional and controllable. By contrast, in the case of AM, it is possible to find examples where the verbal action expresses a non-controllable event, such as the action of finding a lost object in example (\ref{ex:CpjAmto}) with echo phenomenon (cf § \ref{sec:AM.echo}). Note that there are no examples of the non-volitional verb \japhug{mto}{see} with the MVC in the corpus (the volitional \japhug{rtoʁ}{see, look} or \japhug{ru}{look} occur  instead). 

\begin{exe}
\ex  \label{ex:CpjAmto}
\gll  nɯɕɯmɯma ʑo tɯ-ci ɯ-ŋgɯ pjɤ-ɕe qʰe iɕqʰa tɤɕime kɯ ɯ-sɤcɯ pɯ-kɤ-nɯ-ɕlɯɣ nɯ ɕ-pjɤ-mto. \\
immediately \textsc{emph} \textsc{indef}.\textsc{poss}-water \textsc{3sg}.\textsc{poss}-inside \textsc{ifr}:\textsc{down}-go \textsc{lnk} the.aforementioned lady \textsc{erg} \textsc{3sg}.\textsc{poss}-key \textsc{pfv}:\textsc{down}-\textsc{nmlz}:P-\textsc{auto}-drop \textsc{dem} \textsc{transloc}-\textsc{ifr}-see \\
\glt `He went immediately into the water and saw there the key that the lady had dropped by mistake.' (140510 fengwang, 118)
\end{exe}

