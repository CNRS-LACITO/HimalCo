\chapter{The Japhug language}
This paper focuses on the Japhug language (local name \ipa{kɯrɯ skɤt}) of Kamnyu village (\ipa{kɤmɲɯ}, Chinese \textit{Ganmuniao} \zh{干木鸟}) in Gdong-brgyad area (\ipa{ʁdɯrɟɤt}, Chinese  \textit{Longerjia} \zh{龙尔甲}), Mbarkhams county (Chinese \textit{Maerkang} \zh{马尔康}), Rngaba prefecture, Sichuan province, China.
 
 Japhug belongs to the Sino-Tibetan family, and is one of the four Rgyalrong languages, alongside Tshobdun, Zbu and Situ.\footnote{See  \citet{jackson00sidaba} for an overview of the Rgyalrong group, whose closest relatives include Khroskyabs (\citealt{lai15person}) and Horpa (\citealt{jackson07shangzhai}). A text collection of Japhug with sound files is included in the Pangloss archive (\citealt{michailovsky14pangloss}). A short grammar (\citealt{jacques08}), a series of articles on morphosyntax (see for instance  \citealt{jacques13harmonization} and
 \citealt{jacques14antipassive}) and a dictionary (\citealt{jacques15japhug}) are available but little has been published specifically on its phonology. } 
 
The description is based on the author’s fieldwork, and the word list and the short story in the appendix have been provided by Tshendzin (Chenzhen \zh{陈珍}, female, born 1950), a retired schoolteacher (a native speaker of Japhug, bilingual in Sichuan Mandarin since childhood).