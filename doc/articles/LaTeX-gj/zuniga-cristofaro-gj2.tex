\documentclass[twoside,a4paper,11pt]{article} 
\usepackage{polyglossia}
\usepackage{natbib}
\usepackage{booktabs}
\usepackage{xltxtra} 
\usepackage{longtable}
 \usepackage{geometry}
\usepackage[usenames,dvipsnames,svgnames,table]{xcolor}
\usepackage{multirow,slashbox}
%\usepackage{gb4e} 
\usepackage{multicol}
\usepackage{graphicx}
\usepackage{float}
\usepackage{varioref,hyperref} 
\hypersetup{colorlinks=true,linkcolor=blue,citecolor=blue}
\usepackage{memhfixc}
\usepackage{lscape}
\usepackage{lineno}
\usepackage[footnotesize,bf]{caption}


%%%%%%%%%%%%%%%%%%%%%%%%%%%%%%%
\setmainfont[Mapping=tex-text,Numbers=OldStyle,Ligatures=Common]{Junicode} 
%\setsansfont[Mapping=tex-text,Ligatures=Common,Mapping=tex-text,Ligatures=Common,Scale=MatchLowercase]{Ubuntu} 
\newfontfamily\phon[Mapping=tex-text,Ligatures=Common,Scale=MatchLowercase]{Charis SIL} 
%\newfontfamily\phondroit[Mapping=tex-text,Ligatures=Common,Scale=MatchLowercase]{Doulos SIL} 
%\newfontfamily\greek[Mapping=tex-text,Scale=MatchLowercase]{Galatia SIL} 
\newcommand{\ipa}[1]{{\phon\textit{#1}}} 
\newcommand{\ipab}[1]{{\phon #1}}
\newcommand{\ipapl}[1]{{\phondroit #1}}
\newcommand{\captionft}[1]{{\captionfont #1}} 
%\newfontfamily\cn[Mapping=tex-text,Scale=MatchUppercase]{IPAGothic}%pour le chinois
%\newcommand{\zh}[1]{{\cn #1}}
\newcommand{\tgf}[1]{\mo{#1}}
%\newfontfamily\mleccha[Mapping=tex-text,Ligatures=Common,Scale=MatchLowercase]{Galatia SIL}%pour le grec

\newcommand{\sg}{\textsc{sg}}
\newcommand{\pl}{\textsc{pl}}
\newcommand{\grise}[1]{\cellcolor{lightgray}\textbf{#1}} 
\newcommand{\Σ}{\greek{Σ}}
\newcommand{\ro}{$\Sigma$}
\newcommand{\ra}{$\Sigma_1$} 
\newcommand{\rc}{$\Sigma_3$}  
\newfontfamily\cn[Mapping=tex-text,Ligatures=Common,Scale=MatchUppercase]{SimSun}%pour le chinois
\newcommand{\zh}[1]{{\cn #1}}

\newcommand{\abs}{\textsc{abs}}
\newcommand{\acc}{\textsc{acc}}
\newcommand{\adess}{\textsc{adess}}
\newcommand{\agent}{\textsc{a}}
\newcommand{\antierg}{\textsc{antierg}}
\newcommand{\allat}{\textsc{all}}
\newcommand{\aor}{\textsc{aor}}
\newcommand{\assert}{\textsc{assert}}
\newcommand{\assoc}{\textsc{assoc}}
\newcommand{\auto}{\textsc{auto}}
\newcommand{\caus}{\textsc{caus}}
\newcommand{\cis}{\textsc{cis}}
\newcommand{\classif}{\textsc{class}}
\newcommand{\concessif}{\textsc{concsf}}
\newcommand{\comit}{\textsc{comit}}
\newcommand{\conj}{\textsc{conj}}
\newcommand{\const}{\textsc{const}}
\newcommand{\conv}{\textsc{conv}}
\newcommand{\cop}{\textsc{cop}}
\newcommand{\dat}{\textsc{dat}}
\newcommand{\dem}{\textsc{dem}}
\newcommand{\detm}{\textsc{det}}
\newcommand{\dir}{\textsc{dir1}}
\newcommand{\du}{\textsc{du}}
\newcommand{\duposs}{\textsc{du.poss}}
\newcommand{\dur}{\textsc{dur}}
\newcommand{\dyn}{\textsc{dyn}}
\newcommand{\erg}{\textsc{erg}}
\newcommand{\fut}{\textsc{fut}}
\newcommand{\gen}{\textsc{gen}}
\newcommand{\hypot}{\textsc{hyp}}
\newcommand{\ideo}{\textsc{ideo}}
\newcommand{\imp}{\textsc{imp}}
\newcommand{\infin}{\textsc{inf}}
\newcommand{\ipf}{\textsc{ipfv}}
\newcommand{\instr}{\textsc{instr}}
\newcommand{\intens}{\textsc{intens}}
\newcommand{\intr}{\textsc{intr}}
\newcommand{\intrg}{\textsc{intrg}}
\newcommand{\inv}{\textsc{inv}}
\newcommand{\irreel}{\textsc{irr}}
\newcommand{\loc}{\textsc{loc}}
\newcommand{\masc}{\textsc{m}}
\newcommand{\med}{\textsc{med}}
\newcommand{\negat}{\textsc{neg}}
\newcommand{\neu}{\textsc{neu}}
\newcommand{\nmlz}{\textsc{nmlz}}
\newcommand{\nom}{\textsc{nom}}
\newcommand{\nonps}{\textsc{n.pst}}
\newcommand{\obj}{\textsc{o}}
\newcommand{\obv}{\textsc{obv}}
\newcommand{\opt}{\textsc{dir2}}
\newcommand{\perf}{\textsc{pfv}}
\newcommand{\pli}{\textsc{pi}}
\newcommand{\pe}{\textsc{pe}}
\newcommand{\plposs}{\textsc{pl.poss}}
\newcommand{\poss}{\textsc{poss}}
\newcommand{\pot}{\textsc{pot}}
\newcommand{\pret}{\textsc{pret}}
\newcommand{\prohib}{\textsc{prohib}}
\newcommand{ \prox}{\textsc{prox}}
\newcommand{\prs}{\textsc{prs}}
\newcommand{\pst}{\textsc{pst}}
\newcommand{\recip}{\textsc{recip}}
\newcommand{\redp}{\textsc{redp}}
\newcommand{\refl}{\textsc{refl}}
\newcommand{\sgposs}{\textsc{sg.poss}}
\newcommand{\stat}{\textsc{stat}}
\newcommand{\subj}{\textsc{s}}
\newcommand{\topic}{\textsc{top}}
\newcommand{\volit}{\textsc{vol}}


\let\eachwordone=\it

\begin{document}
\linenumbers
\title{Directionality of change and   (non-)canonical direct/inverse systems \footnote{We would like to thank XXX. We are responsible for any remaining errors. This research was funded by the HimalCo project (ANR-12-CORP-0006) and is related to the research strand LR-4.11 'Automatic paradigm generation and language description' of the Labex EFL (funded by the ANR/CGI). } }

\author{Guillaume JACQUES, Anton ANTONOV\\ CNRS-INALCO-EHESS, CRLAO}
%\date{}
\maketitle
\section{Introduction}

which general principles of language change can be deduced from these observations


\section{The (re)shaping of the conjunct order in several Algonquian languages }


\subsection{Proto-Algonquian}
The reconstruction conjunct order paradigm in Proto-Algonquian is uncontroversial. Table \ref{tab:protoalg.conj} (based upon \citealt{bloomfield46proto} and \citealt{goddard00cheyenne}) presents the proto-Algonquian system indicative mode of the conjunct order, directly attested by Fox (Kickapoo) and Miami (\citealt{costa03miami}).

The final *\ipa{--i} in the singular direct and inverse forms is the indicative suffix; in the subjunctive and participle it is replaced by *\ipa{-e} and *\ipa{-a} respectively.\footnote{The participle also presents a different set of endings for the plural forms, which will not be discussed here.} The consonant *\ipa{c} in all the suffixes represents *\ipa{t} palatalized before *\ipa{i}. The non-palatalized form appears in the subjunctive mode and in the participle. Thus, the \textsc{2sg$\rightarrow$3} participle suffix is *\ipa{-ata} (compare indicative *\ipa{-aci}). As we will see, most of the  languages that loose the final vowels generalize the non-palatalized forms in the indicative conjunct order paradigms by analogy with the subjonctive and participle.


\begin{table}[H]
\caption{Proto-Algonquian conjunct order indicative paradigm, VAI and VTA }
\label{tab:protoalg.conj} \centering
\resizebox{\textwidth}{!}{
\begin{tabular}{lllllllll}
\toprule
 \backslashbox{A}{P}  & 	1\sg  & 1\pli & 1\pe &  2\sg & 2\pl  &  3\sg & 3\pl &	3' \\ 
\midrule
1s & 	\grise{}   & 	\grise{} &  \grise{} & 	\ipa{-eθâni} & 	\ipa{-eθakokwe} & 	\ipa{-aki} & 	\ipa{-akwâwi} & 	\ipa{-emaki} \\ 	
1\pli & \grise{}   & 	\grise{} &  \grise{}	 & \grise{}  & \grise{} 	 & 	\ipa{-ankwe} & 	\ipa{} & 	\ipa{-emankwe} \\ 	
1\pe & 	\grise{}   & 	\grise{} &  \grise{}	 & 	\multicolumn{2}{c}{\ipa{-eθânke}} \ipa{} & 	\ipa{-akenci} & 	\ipa{} & 	\ipa{-emakenci} \\ 	
2\sg & 	\ipa{-iyani} & \grise{} 	 & 	\multirow{2}{*}{\ipa{-i-yânke}} & 	\grise{}  & \grise{} 	 & 	\ipa{-aci} & 	\ipa{-atwâwi} & 	\ipa{-emati} \\ 	
2\pl & 	\ipa{-iyêkwe} & \grise{} 	 & \multirow{-2}{*}{\ipa{}}  & \grise{} 	 & \grise{} 	 & 	\ipa{-êkwe} & 	\ipa{} & 	\ipa{-emêkwe} \\ 	
3\sg & 	\ipa{-ici} & 	\multirow{2}{*}{\ipa{-eθankwe}} & 	\multirow{2}{*}{\ipa{-iyamenci}} & 	\ipa{-eθki} & 	\multirow{2}{*}{\ipa{-eθâkwe}} & \grise{} & \grise{}  & 		\cellcolor{Dandelion}\ipa{-âci} \\ 	
3\pl & 	\ipa{-iwâci} & \multirow{-2}{*}{\ipa{}} & \multirow{-2}{*}{\ipa{}} & 	\ipa{-eθkwâwi} & \multirow{-2}{*}{\ipa{}} & \grise{}  & 	\grise{}  & 		\cellcolor{Dandelion}\ipa{-âwâci} \\ 	
3' & 	\ipa{-irici} & 	\ipa{} & 	\ipa{} & 	\ipa{-emeθki} & 	\ipa{} & 	\cellcolor{green}\ipa{-ekweti} & 	\cellcolor{green}	\ipa{-ekowâci} & 	\ipa{} \\\bottomrule 	
\intr & \ipa{-(y)âni}	& \ipa{-(y)ankwe}	 & \ipa{-(y)ânke}	 & \ipa{-(y)ani}	 & \ipa{-(y)êkwe} & \ipa{-ci} / \ipa{-ki}	 & \ipa{-wâci}	 &  \ipa{-rici}\\ 
\bottomrule
\end{tabular}
}
\end{table}	

The proto-Algonquian system is clearly not direct-inverse. Parts of it is tripartite, in particular the first and second singular and the first person plural exclusive forms. For instance, intransitive \textsc{1pe} *\ipa{--ânke} and transitive \textsc{1pe$\rightarrow$3} *\ipa{-akenci},  \textsc{3$\rightarrow$1pe} *\ipa{-iyamenci} are all marked by unrelated  morphemes ($S \ne A \ne P$).

Other forms present accusative alignment; for instance, the second plural has \ipa{-êkwe} in both intransitive and direct forms, but *\ipa{-âkwe}  in inverse ones ($S = A \ne P$). There are specific markers *\ipa{-i-} and *\ipa{-eθ-} for first person and second person patients respectively, that occur in all inverse and local forms.

The only suffix neutral as to the syntactic roles in the system is the plural inclusive *\ipa{-ankwe} (which however is combined with 
the second person patient suffix *\ipa{-eθ-} in inverse forms).

The following sections show how such a non-hierarchical system was independently reshaped as a (partial) direct-inverse system in several Algonquian languages by ousting the opaque forms and replacing them with transparent ones. 

 

\subsection{Plains Cree}
Table \vref{tab:cree.conj} presents the conjunct order paradigm of Modern Plains Cree.

\begin{table}[h]
\caption{Plains Cree Conjunct Order indicative paradigms.  (\citealp{wolfart96sketch})}
\label{tab:cree.conj} \centering
\resizebox{\textwidth}{!}{
\begin{tabular}{lllllllll}
\toprule
 \backslashbox{A}{P}  & 	1\sg  & 1\pli & 1\pe &  2\sg & 2\pl  &  3\sg & 3\pl &	3' \\ 
\midrule
1\sg   & 	\grise{}   & 	\grise{} &  \grise{} &	\ipa{-it-ân}  & \ipa{-it-ako-k}	& \ipa{-ak}   & 	\ipa{-ak-ik}  & 	\ipa{-im-ak}   \\ 
1pi & \grise{}   &\grise{} & \grise{} & \multicolumn{2}{c}{\grise{}}  & \ipa{-â-yahk} & \ipa{-â-yahko-k}  & 	\ipa{-im-â-yahk}   \\ 
1\pe & \grise{}   &\grise{} & \grise{} & \multicolumn{2}{c}{\ipa{-it-âhk}}   & \ipa{-â-yâhk} & \ipa{-â-yâhk-ik}  & 	\ipa{-im-â-yâhk}   \\ 
2\sg   & 	\ipa{-i-yan}   & \grise{}& \multirow{2}{*}{}	&	\grise{}   &  \grise{} & \ipa{-at}  & \ipa{-at-ik} &\ipa{-im-at}   \\ 
2\pl  & 	\ipa{-i-yêk} & \grise{}& \multirow{-2}{*}{ \ipa{-i-yâhk}} & \grise{}  & 	\grise{}   & 	\ipa{-â-yêk}  & \ipa{-â-yêko-k} & 	\ipa{-im-â-yêk}   \\
3\sg   & 	\ipa{-i-t}   & \ipa{-iko-yahk} & \ipa{-iko-yâhk} & \ipa{-isk}  & 	\ipa{-iko-yêk} & 	\grise{}  & \grise{}	 & 	\ipa{-(im)-â-t}   \\ 
3\pl   & 	\ipa{-i-c-ik}&  \ipa{-iko-yahko-k} & \ipa{-iko-yâhk-ik}   & 	\ipa{-isk-ik}   & 	\ipa{-iko-yêko-k} & 	\grise{} &	\grise{}  & 	\ipa{-(im)-â-c-ik}   \\ 
\multirow{2}{*}{3'}   & \multirow{2}{*}{}  &  \multirow{2}{*}{}  & \multirow{2}{*}{} & &  \multirow{2}{*}{}  &\multirow{2}{*}{}   & \multirow{2}{*}{} &  \ipa{-â-yi-t} \\ 
 \multirow{-2}{*}{} & \multirow{-2}{*}{\ipa{-iy-i-t}} & \multirow{-2}{*}{\ipa{-ikow-â-yahk}}   &  \multirow{-2}{*}{\ipa{-ikow-â-yâhk}} &  \multirow{-2}{*}{\ipa{-iy-isk}} &  \multirow{-2}{*}{\ipa{-ikow-â-yêk}}& \multirow{-2}{*}{\ipa{-iko-t}}  & \multirow{-2}{*}{\ipa{-iko-c-ik}} &  \ipa{-iko-yi-t}  \\ 
\bottomrule
\textsc{intr} & \ipa{-yân} & \ipa{ -yahk} & \ipa{-yâhk} &\ipa{ -yan} &\ipa{ -yêk} & \ipa{-t} & \ipa{-c-ik} & \ipa{-yi-t} \\
\bottomrule
\end{tabular}
}
\end{table}
 

Table \vref{tab:creedia.conj} presents the earliest attested stage in the conjunct order paradigm of Plains Cree.

\begin{table}[H]
\caption{19\textsuperscript{th} century Plains Cree Conjunct Order indicative paradigms.  (adapted from \citealp{wolfart96sketch} and \citealp{dahlstrom89change})}
\label{tab:creedia.conj} \centering
\resizebox{\textwidth}{!}{
\begin{tabular}{lllllllll}
\toprule
 \backslashbox{A}{P}  & 	1\sg  & 1\pli & 1\pe &  2\sg & 2\pl  &  3\sg & 3\pl &	3' \\ 
\midrule
1\sg   & 	\grise{}   & 	\grise{} &  \grise{} &	\ipa{-it-ân}  & \ipa{-it-ako-k}	& \ipa{-ak}   & 	\ipa{-ak-ik}  & 	\ipa{-im-ak}   \\ 
1\pli & \grise{}   &\grise{} & \grise{} & \multicolumn{2}{c}{\grise{}}  &  \ipa{-ahk} & \ipa{-ahko-k} & 	\ipa{-im-â-yahk}   \\ 
1\pe & \grise{}   &\grise{} & \grise{} & \multicolumn{2}{c}{\ipa{-it-âhk}}   &  \ipa{-ak-iht} &  \ipa{-ak-iht-ik}   & 	\ipa{-im-â-yâhk}   \\ 
2\sg   & 	\ipa{-i-yan}   & \grise{}& \multirow{2}{*}{}	&	\grise{}   &  \grise{} & \ipa{-at}  & \ipa{-at-ik} &\ipa{-im-at}   \\ 
2\pl  & 	\ipa{-i-yêk} & \grise{}& \multirow{-2}{*}{ \ipa{-i-yâhk}} & \grise{}  & 	\grise{}   & 	\ipa{-êk}  & \ipa{-êko-k} & 	\ipa{-im-â-yêk}   \\
3\sg   & 	\ipa{-i-t}   & \ipa{-it-ahk} & \ipa{-i-yam-iht}  & \ipa{-isk}  & 	\ipa{-it-êk} & 	\grise{}  & \grise{}	 & 	\ipa{-(im)-â-t}   \\ 
3\pl   & 	\ipa{-i-c-ik}&  \ipa{-it-ahko-k} & \ipa{-i-yam-iht-ik}   & 	\ipa{-isk-ik}   & 	\ipa{-it-êko-k} & 	\grise{} &	\grise{}  & 	\ipa{-(im)-â-c-ik}   \\ 
\multirow{2}{*}{3'}   & \multirow{2}{*}{}  &  \multirow{2}{*}{}  & \multirow{2}{*}{} & &  \multirow{2}{*}{}  &\multirow{2}{*}{}   & \multirow{2}{*}{} &  \ipa{-â-yi-t} \\ 
 \multirow{-2}{*}{} & \multirow{-2}{*}{\ipa{-iy-i-t}} & \multirow{-2}{*}{\ipa{-ikow-â-yahk}}   &  \multirow{-2}{*}{\ipa{-ikow-â-yâhk}} &  \multirow{-2}{*}{\ipa{-iy-isk}} &  \multirow{-2}{*}{\ipa{-ikow-â-yêk}}& \multirow{-2}{*}{\ipa{-iko-t}}  & \multirow{-2}{*}{\ipa{-iko-c-ik}} &  \ipa{-iko-yi-t}  \\ 
\bottomrule
\textsc{intr} & \ipa{-yân} & \ipa{ -yahk} & \ipa{-yâhk} &\ipa{ -yan} &\ipa{ -yêk} & \ipa{-t} & \ipa{-c-ik} & \ipa{-yi-t} \\
\bottomrule
\end{tabular}
}
\end{table}


%\resizebox{\textwidth}{!}{
%\begin{tabular}{lllllllll}
%\toprule
% \backslashbox{A}{P}  & 	1\sg  & 1\pli & 1\pe &  2\sg & 2\pl  &  3\sg & 3\pl &	3' \\ 
%\midrule
%1\sg   & 	\grise{}   & 	\grise{} &  \grise{} &	\ipa{-it-ân}  & \ipa{-it-ako-k}	& \ipa{-ak}   & 	\ipa{-ak-ik}  & 	\ipa{-im-ak}   \\ 
%1\pli & \grise{}   &\grise{} & \grise{} & \multicolumn{2}{c}{\grise{}}  &  \ipa{-ahk} & \ipa{-ahko-k} & 	\ipa{-im-â-yahk}   \\ 
%1\pe & \grise{}   &\grise{} & \grise{} & \multicolumn{2}{c}{\ipa{-it-âhk}}   &  \ipa{-ak-iht} &  \ipa{-ak-iht-ik}   & 	\ipa{-im-â-yâhk}   \\ 
%2\sg   & 	\ipa{-i-yan}   & \grise{}& \multirow{2}{*}{}	&	\grise{}   &  \grise{} & \ipa{-at}  & \ipa{-at-ik} &\ipa{-im-at}   \\ 
%2\pl  & 	\ipa{-i-yêk} & \grise{}& \multirow{-2}{*}{ \ipa{-i-yâhk}} & \grise{}  & 	\grise{}   & 	\ipa{-êk}  & \ipa{-êko-k} & 	\ipa{-im-â-yêk}   \\
%3\sg   & 	\ipa{-i-t}   & \ipa{-it-ahk} & \ipa{-i-yam-iht}  & \ipa{-isk}  & 	\ipa{-it-êk} & 	\grise{}  & \grise{}	 & 	\ipa{-(im)-â-t}   \\ 
%3\pl   & 	\ipa{-i-c-ik}&  \ipa{-it-ahko-k} & \ipa{-i-yam-iht-ik}   & 	\ipa{-isk-ik}   & 	\ipa{-it-êko-k} & 	\grise{} &	\grise{}  & 	\ipa{-(im)-â-c-ik}   \\ 
%\multirow{2}{*}{3'}   & \multirow{2}{*}{}  &  \multirow{2}{*}{}  & \multirow{2}{*}{} & &  \multirow{2}{*}{}  &\multirow{2}{*}{}   & \multirow{2}{*}{} &  \ipa{-â-yi-t} \\ 
% \multirow{-2}{*}{} & \multirow{-2}{*}{\ipa{-iy-i-t}} & \multirow{-2}{*}{\ipa{-ikow-â-yahk}}   &  \multirow{-2}{*}{\ipa{-ikow-â-yâhk}} &  \multirow{-2}{*}{\ipa{-iy-isk}} &  \multirow{-2}{*}{\ipa{-ikow-â-yêk}}& \multirow{-2}{*}{\ipa{-iko-t}}  & \multirow{-2}{*}{\ipa{-iko-c-ik}} &  \ipa{-iko-yi-t}  \\ 
%\bottomrule
%\textsc{intr} & \ipa{-yân} & \ipa{-yahk} & \ipa{-yâhk} &\ipa{-yan} &\ipa{-yêk} & \ipa{-t} & \ipa{-c-ik} & \ipa{-yi-t} \\
%\bottomrule
%\end{tabular}
%}
%\end{table}

Comparing Table \vref{tab:creedia.conj} with Table \vref{tab:cree.conj} we can easily see that the direct forms (in dandelion) and the inverse forms (in green), bearing the so-called `theme signs' \textit{-â-} (direct) vs. \textit{-ikw-} (inverse), have been generalized at the expense of older and less easily segmentable ones in the case of mixed scenarios with plural \textsc{sap}s. This reshaping of the system has taken place some time between the 19\textsuperscript{th} and beginning of the 20\textsuperscript{th} centuries and has only affected the conjunct order. Since it is commonly assumed in Algonquianist circles that the conjunct order is older than the independent order, this can only mean that the latter had either suffered a similar reshaping earlier or else that it was as regular as it is nowadays right from the start, a hypothesis which ties in well with the idea that it is younger and largely nominal from the outset.

If we accept that idea, any discussion of the direct/inverse system in Cree (and in Algonquian) and its origin(s) should be based upon the conjunct order forms and not the independent order ones as they (ie the conjunct order forms) are likely to preserve an older stage in the development of the system.

Turning to Table \vref{tab:creedia.conj}, the fact that there were less direct vs. inverse forms bearing the so-called `theme signs' \textit{-â-} vs. \textit{-ikw-}, means that at some point the system as we know it today had even less, if at all, forms showing these theme signs and that ultimately the direct vs inverse distinction which is a hallmark not only of Cree but of all Algonquian languages could prove to be a realtively recent innovation. Indeed, the VTA paradigm reconstructed for Proto-Algonquian by \cite{goddard00cheyenne} confirms this hypothesis (cf. Table \vref{tab:protoalg.conj}). Focussing on Cree, if at some stage the language did not have any direction markers, a legitimate question would be how speakers knew who did what to whom. A possible answer is that transitive verbs showed polypersonal agreement, indexing both A and O using different affixes in each case. Viewed from this perspective, the forms in Table \vref{tab:creedia.conj} could be analyzed as showing a suffix indexing a 1\sg\ patient which was \textit{-i}, and a suffix \textit{-it} in the case of a 2\sg\ patient. This follows from a comparison of the transitive with the intransitive paradigm, which reveals that in most cells the outermost suffix in a transitive form is identical to the suffix indexing the single argument of an intransitive verb, i.e. A=S, thus showing accusative alignment. Furthermore, the relative order of the person suffixes is O-A. Some forms with a third person agent have \textit{-t} (3$\rightarrow$1\sg) and others \textit{-k} (3$\rightarrow$2\sg), which mirrors the situation with intransitives where the choice of one or the other is conditioned by the final of the verb stem: if it ends in a vowel, the suffix is \textit{-t} and if it ends in a consonant the suffix is \textit{-k}, except that in the case of agent markers it is the preceding patient marker's final which matters. 
\subsection{Ojibwe}
Some Nishnaabemwin (Ojibwe) dialects  present innovations similar to those observed in Plains Cree, but limited to the inverse forms. Table \ref{tab:ojbw.conj}, based on data from \citet[295]{valentine01grammar}, presents the Nishnaabemwin conservative paradigm. 

As in Cree, Nishnaabemwin has generalized the non palatalized allomorphs of second and third person conjunct order suffixe; thus we find \textsc{2sg$\rightarrow$3sg} \ipa{--ad} corresponding to proto-Algonquian *\ipa{--ati} in the indicative conjunct order instead of expected *\ipa{--aj}; the subjunctive and participle forms, which respectively were *\ipa{--ate} and *\ipa{--ata}, were not palatalized, and the non-patalized form \ipa{-ad} was generalized to the indicative mode of the conjunct order after the loss of final vowels. This particularity is not shared by all Ojibwe dialects; for instance,  the Algonquin Ojibwe dialect described by \citealt{cuoq1866} has generalized the palatalized form, see \citealt[101]{bloomfield46proto}.
 

 
\begin{table}[H]
\caption{The conservative Ojibwe VTA and VAi paradigms }
\label{tab:ojbw.conj} \centering
\resizebox{\textwidth}{!}{
\begin{tabular}{llllllllll}
\toprule
 \backslashbox{A}{P}  & 	1\sg  & 1\pli & 1\pe &  2\sg & 2\pl  &  3\sg & 3\pl &	3'   \\ 
\midrule
1\sg   & 	\grise{}   & 	\grise{} &  \grise{} &	\ipa{-inân}  & \ipa{-ininagog}	& \ipa{-ag}   & 	\ipa{-agwaa}    \\ 
1\pli & \grise{}   &\grise{} & \grise{} & \multicolumn{2}{c}{\grise{}}  &  \ipa{-ang} & \ipa{-ang-waa} &   \\ 
1\pe & \grise{}   &\grise{} & \grise{} & \multicolumn{2}{c}{\ipa{-inaang}}   &  \ipa{-angid} &  \ipa{-angidwaa}     \\ 
2\sg   & 	P\ipa{-iyan}   & \grise{}& \multirow{2}{*}{}	&	\grise{}   &  \grise{} & \ipa{-ad}  & \ipa{-adwaa}     \\ 
2\pl  & 	P\ipa{-iyeg} & \grise{}& \multirow{-2}{*}{ \ipa{-iyaang}} & \grise{}  & 	\grise{}   & 	\ipa{-eg}  & \ipa{-egwaa}     \\
3\sg   & 	P\ipa{-id}   & \ipa{-ininang} & P\ipa{-iyangid}  & \ipa{-ik}  & 	\ipa{-inineg} & 	\grise{}  & \grise{}	 & 	\ipa{-aad}     \\ 
3\pl   & 	P\ipa{-iwaad}&  \ipa{-ininangwaa} & P\ipa{-iyangidwaa} & 	\ipa{-ikwaa}   & 	\ipa{-ininegwaa} & 	\grise{} &	\grise{}  & \ipa{-aawaad}  \\ 
 3' & & & & & &   \ipa{-igod} &  \ipa{-igowaad} \\
\bottomrule
\textsc{intr} & \ipa{-yaan} & \ipa{-yang} & \ipa{-yaang} &\ipa{-yan} &\ipa{-yeg} & \ipa{-d} / \ipa{-g} & \ipa{-waad} & \ipa{-nid} \\
\bottomrule
\end{tabular}
}
\end{table}

Some dialects of Nishnaabewin,  such as Parry Island, allow innovative forms as optional variants of the conservative ones as presented in Table \ref{tab:ojibwe.vta.2} (see \citealt[178-9]{valentine01grammar}). The conservative forms themselves have been reshaped in comparison with the paradigm recorded in the 19th century. This includes the generalization of \ipa{-eg} vocalism in \textsc{3$\rightarrow$2p}  from the \textsc{2p$\rightarrow$3}, the replacement of the \textsc{3$\rightarrow$1pe}  \ipa{--iyamintʃ}  by an analysable form made of the direct \ipa{-angid} and the first object theme sign \ipa{--i}, and the doubling of the second person theme sign \ipa{-in} (from *-eθ-). By comparison, we present the 19th century Algonquin forms from \citet[51]{cuoq1866}, which are directly inherited from proto-Algonquian.

\begin{table}[H]
\caption{The Ojibwe VTA paradigm inverse forms and their PA origins}
\centering \label{tab:ojibwe.vta.2}
\begin{tabular}{llllllll}
\toprule
& Innovative & Conservative & 19th century Ojibwe & Proto-Algonquian \\
&paradigm & paradigm&\\
\midrule
\textsc{3$\rightarrow$1s} &\ipa{--igo-yaan} \grise{}& 	\ipa{--id} & \ipa{--itʃ} &	 *\ipa{--iti} & 		\\
\textsc{3$\rightarrow$1pi} & 	\ipa{--igo-yang} \grise{}& 	\ipa{--ininang}  \grise{} &  	\ipa{--inang}  	 &*\ipa{--eθankwe} & 		\\
\textsc{3$\rightarrow$1pe} & 	\ipa{--igo-yaang} \grise{}& 	\ipa{--iyangid} \grise{}&	\ipa{--iyamintʃ}  &  *\ipa{--iyamenti} & 		\\
\midrule
\textsc{3$\rightarrow$2s} & 	\ipa{--igo-yan} \grise{}& 	\ipa{--ik} &	\ipa{--ik} &  *\ipa{--eθki} & 		\\
\textsc{3$\rightarrow$2p} & \ipa{--igo-yeg} \grise{}& 	\ipa{--inineg}  \grise{} & \ipa{--inaak}  & *\ipa{--eθâkwe} & 		\\
\midrule
3'\textsc{$\rightarrow$3s} & \ipa{--igod} & 	\ipa{--igod} &		\ipa{--igotʃ} &*\ipa{--ekweti} & 		\\
3'\textsc{$\rightarrow$3p} & \ipa{--igodwaa}   & 	\ipa{--igodwaa}  \grise{}&\ipa{--igowaatʃ} &*\ipa{--ekowaati} & 		\\
\bottomrule
\end{tabular}
\end{table}

This dialect of Nishnaabemwin goes further than Plains Cree in that not only plural inverse forms, but also singular one, are affected by the analogy. Direct forms, on the other hand, remain unchanged.

\subsection{Arapaho}

The paradigm reshaping that has occurred in Cree and Nishnaabemwin is not isolated. Among Algonquian languages, Arapaho provide an example of a language which reshaped the conjunct order even further. Before discussing the Arapaho VTA paradigm, we provide some information on the VAI   paradigm, which are necessary to understands what happened in the VTA. Proto-Algonquian reconstructions are systematically given in these section, as the drastic sound changes of Arapaho (see \citealt{goddard74arapaho}) have rendered the cognate forms barely recognizable. We cannot provide here a detailed account of Arapaho historical phonology, and defer the reader to Goddard's works for further information. Arapaho data used in this section are taken from \citet{salzmann67arapaho.verb} and \citet{cowell06arapaho}.

The Arapaho VAI conjunct order paradigm, as shown by \citet[16-7]{goddard65arapaho}, regularly derives from the proto-Algonquian Conjunct Order participle. Had they originated from the indicative Conjunct Order forms, the third person forms shoudl have been different: for instance, the third singular would have been **-θ $\leftarrow$ *\ipa{--ci}.

Table \ref{tab:arapaho.vai} shows the main  allomorphs for the Conjunct order suffixes in Arapaho and their Proto-Algonquian origins. The first plural exclusive \ipa{--'} originates from the indefinite S form *\ipa{--nki} (\citealt{goddard98morphology.arapaho}), replacing the \textsc{1pe} ending.

%a possible way to account for it is proposed by \citet[22]{goddard65arapaho}, who argues that possibly in the allomorph *\ipa{--ânke} the \ipa{â}


\begin{table}[H]
\caption{The Arapaho VAI paradigms and its proto-Algonquian origin}
\centering \label{tab:arapaho.vai}
\begin{tabular}{lllllll}
\toprule
Person &   Arapaho    & Proto-Algonquian\\
\midrule
1s & 	\ipa{--noo} &  	*\ipa{--yâni} & 	\\	
1pe & 	\ipa{--ni'} /  	\ipa{--'} \grise{} & 		 *\ipa{--yânke}	 \\	
1pi & 	\ipa{--no'} & 	 		*\ipa{--yankwe} & 	\\	
\midrule
2s & 	\ipa{--n} & 	 	*\ipa{--yani} & 	\\	
2p & 	\ipa{--nee} & 	  		*\ipa{--yêkwe} & 	\\	
\midrule
3s & 	\ipa{--t} /	\ipa{--'} & 		*\ipa{--ta} / \ipa{--ki}& 	\\	
3's & 	\ipa{--níθ} &  		*\ipa{--riciri} & 	\\	
3p & 	\ipa{--θi'} &  		*\ipa{--ciki} 	\\	
3'p & 	\ipa{--níθi} & 	 		*\ipa{--ricihi} 	\\	
\bottomrule
\end{tabular}
\end{table}

In comparison with the VAI paradigm, which is almost entirely inherited from proto-Algonquian, the VTA paradigms presents considerable reshaping; the account proposed here and the Proto-Algonquian reconstructions are largely based on  \citet[19-24]{goddard65arapaho} (in combination with  \citealt{goddard00cheyenne} for some details of the Proto-Algonquian paradigms). Table \ref{tab:arapaho.vta}   presents the regular endings of the VTA paradigm in Arapaho, taken from  \citet[487-490]{cowell06arapaho}. The  further obviative 3'$\rightarrow$3' direct and inverse forms are not included.

\begin{table}[H]
\caption{The Arapaho VTA paradigm}
\centering \label{tab:arapaho.vta}
\begin{tabular}{llllllllllll}
\toprule
 & 	1s & 	1i & 	1e & 	2s & 	2p & 	3s & 	3p & 	3' & 	\\
1s & \grise{} & 	\grise{} & 	\grise{} & 	\ipa{--éθen} & 	\ipa{--eθénee} & 	\ipa{--o'} & 	 & 	 & 	\\
1i & 	\grise{} & 	\grise{} & 	\grise{} & 	\grise{} & 	\grise{} & 	\ipa{--óóno'} & 	 & 	 & 	\\
1e & 	\grise{} & 	\grise{} & 	\grise{} & 	\ipa{--een} & 	\ipa{--eenee} & 	\ipa{--éét} & 	 & 	 & 	\\
2s & 	\ipa{--ín} / \ipa{--ún}& 	\grise{} & \ipa{--ínee} /	\ipa{--únee} & 	\grise{} & 	\grise{} & 	\ipa{--ót} & 	 & 	 & 	\\
2p & 	\ipa{--éi'een} & 	\grise{} & 	\ipa{--éi'éénee} & 	\grise{} & 	\grise{} & 	\ipa{--óónee} & 	 & 	 & 	\\
3s & 	\ipa{--éínoo} & 	\ipa{--éíno'} & 	\ipa{éi'éét} & 	\ipa{--éín} & 	\ipa{--éínee} & 	\grise{} & 	\grise{} & 	\ipa{--oot} & 	\\
3p & 	 & 	 & 	 & 	 & 	 & 	\grise{} & 	\grise{} & 	\ipa{--óóθi'} & 	\\
3' & 	 & 	 & 	 & 	 & 	 & 	\ipa{--éít} & 	\ipa{--éíθi'} &   & 	\\
\bottomrule
\end{tabular}
\end{table}

Given the complexity of the paradigm in 	Table \ref{tab:arapaho.vta}, we split the discussion  in three parts, analyzing the direct, inverse and local forms separately.

The direct forms of the VTA paradigm are compared with the corresponding reconstructed Proto-Algonquian forms in Table \ref{tab:arapaho.vta.1}; the Proto-Algonquian forms that do not match Arapaho are indicated in grey. This table shows that as in Plains Cree, while the singular direct forms are inherited, the SAP plural ones are reshaped by reanalyzing the third person ending \ipa{--oot} as \ipa{--oo-} + the VAI ending \ipa{--t} and generalizing this structure to the first and second person plural: \ipa{--óó-no'} \textsc{1pi} are \ipa{--óó-nee} \textsc{2p} built by combining the direct marker \ipa{--oo--} with the regular VAI endings.

The \textsc{1pe} \ipa{--éét} probably does not originate from inherited  *\ipa{--akenta}. This form should have yielded  either *\ipa{--ooot} or *\ipa{--eeet}. While it is not entirely impossible that vowel shortening would have happened, it is more satisfying to derive  \ipa{--éét}  from the indefinite X-3 form of the conjunct participle  *\ipa{--enta} (\citealt{goddard98morphology.arapaho}).

\begin{table}[H]
\caption{The Arapaho VTA paradigm direct forms and their PA origins}
\centering \label{tab:arapaho.vta.1}
\begin{tabular}{lllll}
\toprule
Form& Arapaho & Proto-Algonquian \\
\midrule
 \textsc{1s}$\rightarrow$3 & 	\ipa{--o'} & 	*\ipa{--aka} & 		\\		
\textsc{1pe}$\rightarrow$3 & 	\ipa{--éét}\grise{} & 	 *\ipa{--akenta} & 		\\		
\textsc{1pi}$\rightarrow$3 & 	\ipa{--óó-no'}\grise{} & *\ipa{--ankwe} & 		\\		
\midrule
\textsc{2s}$\rightarrow$3 & 	\ipa{--ót} & 	*\ipa{--ata} & 		\\		
\textsc{2p}$\rightarrow$3 & 	\ipa{--óó-nee} \grise{}& *\ipa{--êkwe} & 		\\		
\midrule
\textsc{3s}$\rightarrow$3' & 	\ipa{--oot} & 	*\ipa{--âta} & 		\\		
\textsc{3p}$\rightarrow$3' & 	\ipa{--óóθi'} & 	*\ipa{--âciki} & 		\\		
\bottomrule
\end{tabular}
\end{table}

By contrast with the direct paradigm, the inverse   VTA paradigm is almost entirely remade, as in Parry Island Nishnaabemwin: only the third person forms are inherited, as can be seen in Table \ref{tab:arapaho.vta.2}. As in the direct paradigm, the third person ending \ipa{--éít} was reanalyzed as \ipa{--ei--} + the VAI ending \ipa{--t} and all other forms were rebuilt on that model, replacing the inherited forms. All inverse forms follow this pattern, except the \textsc{3$\rightarrow$1pe} suffix, where *\ipa{--éi'} would have been expected. The attested \textsc{3$\rightarrow$1pe} form \ipa{--éi-'-éét} combines the expected form with the direct ending \ipa{--eet}.

\begin{table}[H]
\caption{The Arapaho VTA paradigm inverse forms and their PA origins}
\centering \label{tab:arapaho.vta.2}
\begin{tabular}{lllll}
\toprule
\textsc{3$\rightarrow$1s} & 	\ipa{--éí-noo}\grise{} & 	*\ipa{--iti} & 		\\
\textsc{3$\rightarrow$1pe} & 	\ipa{--éi-'-éét} \grise{}& *\ipa{--iyamenti} & 		\\
\textsc{3$\rightarrow$1pi} & 	\ipa{--éí-no'} \grise{}& 	*\ipa{--eθankwe} & 		\\
\midrule
\textsc{3$\rightarrow$2s} & 	\ipa{--éí-n} \grise{}& *\ipa{--eθki} & 		\\
\textsc{3$\rightarrow$2p} & 	\ipa{--éí-nee} \grise{}& *\ipa{--eθâkwe} & 		\\
\midrule
3'\textsc{$\rightarrow$3s} & 	\ipa{--éít} & 	*\ipa{--ekweti} & 		\\
3'\textsc{$\rightarrow$3p} & 	\ipa{--éíθi'} & 	*\ipa{--ekociki} & 		\\
\bottomrule
\end{tabular}
\end{table}

As the inverse paradigm, the local paradigm has undergone considerable analogy. Only the \textsc{2s$\rightarrow$1s} and \textsc{2p$\rightarrow$1s} are inherited and have escaped reshaping. 



\begin{table}[H]
\caption{The Arapaho VTA paradigm local forms and their PA origins}
\centering \label{tab:vta.3}
\begin{tabular}{lllll}
\toprule
Person & Arapaho & PA Conjunct    \\
\midrule 
\textsc{1s$\rightarrow$2s}& \ipa{--éθen} \grise{}& *\ipa{--eθâni}   \\
\textsc{1s$\rightarrow$2p}&\ipa{--eθénee} \grise{}& *\ipa{--eθakokwe} & \\
\textsc{1pe$\rightarrow$2s}&\ipa{--een} \grise{}& *\ipa{--eθânke} &   \\
\textsc{1pe$\rightarrow$2p}&\ipa{--eenee} \grise{}& *\ipa{--eθânke} &   \\
\midrule 
\textsc{2s$\rightarrow$1s}&\ipa{--ún} / \ipa{--ín} &   *\ipa{--iyani}     \\
\textsc{2s$\rightarrow$1pe}& \ipa{--éi'een}\grise{}& *\ipa{--iyânkwe} &  \\
\textsc{2p$\rightarrow$1s}&\ipa{--únee} / \ipa{--ínee} &  *\ipa{--iyêkwe} &   \\
\textsc{2p$\rightarrow$1pe}&\ipa{--éi'eenee}\grise{} & *\ipa{--iyânkwe} &   \\
\bottomrule
\end{tabular}
\end{table}

\citet[23]{goddard65arapaho} explains the forms \textsc{1pe$\rightarrow$2s} \ipa{--een} and \textsc{3$\rightarrow$1pe} \ipa{--éi-'-één} by proportional analogy, after the reshaping of the inverse paradigm had taken place: as direct and inverse forms were rebuilt by adding VAI endings to the first part of the third person endings \ipa{--oo--} and \ipa{--ei--} reanalyzed as direction markers, the final consonants \ipa{--t} and \ipa{--n} became   respectively \textsc{3sg} and \textsc{2sg} markers not only for  S, but also for P.

After that, even in forms where the \ipa{--t} was not a third person marker, in particular  \textsc{1pe$\rightarrow$3}   \ipa{--éét} and   \textsc{3$\rightarrow$1pe}   \ipa{--éi'éét}, it became reanalyzed as such and the forms  \textsc{1pe$\rightarrow$2}   \ipa{--één} and   \textsc{2$\rightarrow$1pe}   \ipa{--éi'één} were built by changing the final \ipa{--t} to \ipa{--n} on the model of the VAI and VTA inverse forms (see Table  \ref{tab:arapaho.analogy.local}).

\begin{table}[H]
\caption{Proportional analogy in the Arapaho local forms}
\centering \label{tab:arapaho.analogy.local}
\begin{tabular}{lllll}
\toprule
 Person &  Form &  Person &  Form\\
\midrule 
 VAI \textsc{3s} & \ipa{--t} &  VAI \textsc{2s} & \ipa{--n} \\
  \textsc{3'$\rightarrow$3s} & \ipa{--éí-\textbf{t}} &   \textsc{3$\rightarrow$2s} & \ipa{--éí-\textbf{n}} \\
  \midrule 
    \textsc{1pe$\rightarrow$3} & \ipa{--éé-\textbf{t}} &  \textsc{1pe$\rightarrow$2s} &  \grise{}\ipa{--éé-\textbf{n}} \\
  \textsc{3$\rightarrow$1pe} & \ipa{--éi'éé-\textbf{t}} &  \textsc{2s$\rightarrow$1pe} &  \grise{}\ipa{--éi'éé-\textbf{n}} \\
\bottomrule
\end{tabular}
\end{table}

From there, the \textsc{1s$\rightarrow$2s}  \ipa{--éθen} (instead of expected *\ipa{eθoon}) is likely to originate from the independent order \textsc{1s$\rightarrow$2s} ending \ipa{--éθ} $\leftarrow$ *\ipa{--eθe} to which the second person suffix \ipa{--n} from the VAI paradigm was added. 

The second plural forms \textsc{1s$\rightarrow$2p} \ipa{--eθénee}, \textsc{1pe$\rightarrow$2p} \ipa{--eenee}  and \textsc{2p$\rightarrow$1pe} \ipa{--éi'eenee} were built from the corresponding second singular forms by replacing the \textsc{2s}  \ipa{--n} marker with the \textsc{2p} one \ipa{--nee}, as shown in Table \ref{tab:arapaho.analogy.local2}.
 
 
 \begin{table}[H]
\caption{Proportional analogy in the Arapaho local forms -- second plural}
\centering \label{tab:arapaho.analogy.local2}
\begin{tabular}{lllll}
\toprule
 Person &  Form &  Person &  Form\\
\midrule 
 VAI \textsc{2s} & \ipa{--n} &  VAI \textsc{2p} & \ipa{--nee} \\
  \textsc{3$\rightarrow$2s} & \ipa{--éí-\textbf{n}} &   \textsc{3$\rightarrow$2p} & \ipa{--éí-\textbf{nee}} \\
\textsc{2s$\rightarrow$1s}&  \ipa{--í-\textbf{n}} & \textsc{2p$\rightarrow$1s}&  \ipa{--í-\textbf{nee}} \\
   \midrule 
    \textsc{1s$\rightarrow$2s}& \ipa{--éθe-\textbf{n}} & \textsc{1s$\rightarrow$2p}&\ipa{--eθé-\textbf{nee}} \grise{} \\
    \textsc{1pe$\rightarrow$2s}&\ipa{--ee-\textbf{n}} & \textsc{1pe$\rightarrow$2p}&\ipa{--ee-\textbf{nee}}\grise{} \\
    \textsc{2s$\rightarrow$1pe}& \ipa{--éi'ee-\textbf{n}}&\textsc{2p$\rightarrow$1pe}&\ipa{--éi'ee-\textbf{nee}}\grise{}\\
\bottomrule
\end{tabular}
\end{table}
 
The restructuring that took place in the Arapaho conjunct order goes one step further than that observed in the Cree paradigms: while the extent of reshaping in the direct paradigm is comparable, all inverse forms, and   all local forms except \textsc{2s$\rightarrow$1s} have been remade. The direct \ipa{--oo--} and inverse \ipa{--éí--} theme signs, which originally were restricted to non-local forms, were generalized to all inverse forms and even to the local \textsc{2$\rightarrow$1pe} forms.

Arapaho thus proves that a language can develop a near-canonical direct-inverse system from a tripartite one by generalizing the direct and inverse markers of the non-local forms to the mixed and local ones. 

\subsection{The directionality of analogy in polypersonal systems}

The three cases studied above allow to propose four generalizations concerning the directionality of analogy in ploypersobnal systems with a proximate / obviative distinction in the non-local forms. 
 
First, analogy operates from 3'$\rightarrow$3 to all inverse forms and from 3$\rightarrow$3' to direct forms. This is a particular case of   \citealt{watkins62celtic}'s law: analogy takes place from the third person to the other forms, by reanalyzing the third person ending as part of the verb stem. In the case of Algonquian, the final  \ipa{--t}  in the 3$\rightarrow$3' *\ipa{-ât-} and 3'$\rightarrow$3 forms *\ipa{-ekwet-} is reinterpreted as identical to the VAI third person \ipa{--t}.  The rule of combining the  theme signs *\ipa{-â-} and *\ipa{-ekwe}   with the corresponding VAI endings to obtain the direct and inverse forms  is generalized to other forms of the paradigm.

Second, this analogical process first applies to plural forms before touching singular forms, both in the case of direct and inverse paradigms.
 
Third, analogy can apply from direct forms to inverse and non-local forms (as shown by the reshaping of \textsc{3$\rightarrow$1pe} and \textsc{3$\rightarrow$2p} in Nishnaabemwin).

Four, analogy applies first to inverse forms before affecting direct forms. There appear not to be any hierarchy between inverse and local forms as to their sensitivity to analogy.
 
 
 Whether these four generalizations have a validity in language families other than Algonquian remains to be demonstrated, but may be used as a heuristic principle for diachronic studies on languages whose history is less well known.
 
\section{From or towards a canonical direct-inverse system?}

in families without an established reconstruction

\subsection{Direct-inverse systems in Sino-Tibetan}

The only   large language family beside Algonquian where direct / inverse systems are found is Sino-Tibetan, a family whose historical phonology is very poorly understood and historical morphology extremely difficult to study in a comparative perspective. Algonquian provides us with examples of attested morphological change, which can serve a model to understand how the person marking systems of Sino-Tibetan languages came to be the way they are.

In this Sino-Tibetan,  direct / inverse agreement systems are widely distributed, and have been described best in Rgyalrong and Kiranti languages. The present paper only focuses on these two branches.


\subsubsection{Zbu Rgyalrong}

Direct/inverse systems in Rgyalrong languages in general are quite complex as they usually index not only the person but also the number of the two nuclear arguments of a transitive verb. Several descriptive studies of such systems have already been published (\citealt{delancey81direction} on Situ,  \citealt{jackson02rentongdengdi} on Tshobdun,  \citealt{jacques10inverse} on Japhug, among others). Here we present data from \citealt{gongxun14agreement} on Zbu Rgyalrong.

Amid the overwhelming crosslinguistic diversity of direct/inverse systems, Zbu Rgyalrong stands out as having one of the most symmetrical ones. Table \ref{tab:zbu.tr} presents the non-past paradigm of a dialect of that language.  

%The four Rgyalrong languages (Situ, Japhug, Tshobdun and Zbu) all have a quasi-canonical  direct / inverse system, illustrated in Table \ref{tab:zbu.tr} by data from \citet{gongxun14agreement}.\footnote{description of direct / inverse paradigms in other Rgyalrong languages also include \citet{delancey81direction}, \citet{jackson02rentongdengdi} and \citet{jacques10inverse}.} Unlike the  Algonquian conjunct order, the Rgyalrong agreement system combines prefixes (for the inverse and second person forms) and suffixes (for first person and number).

The \ipa{wə--} inverse prefix occurs in the local (2$\rightarrow$1), inverse mixed (3$\rightarrow$1, 3$\rightarrow$2) and non-local (3$\rightarrow$3' with proximate agent and obviative patient) forms. Unlike Algonquian languages, there is no overt marker in most direct forms (which are thus identical to the corresponding intransitive ones) except in the \textsc{1sg,2sg,3sg$\rightarrow$3}    non-past   (imperfective, factual, testimonial) forms, where transitive verbs undergo various types of stem alternation. The two stems that appear in this paradigm are represented by the symbols   \ra{} and \rc{} respectively.\footnote{These symbols stand for `stem 1' and `stem 3'. There is also a stem 2 occurring in some TAM categories such as the perfective, but it is irrelevant to the present discussion.}

 

\begin{table}[h]
\caption{Zbu Rgyalrong transitive and intransitive paradigms (data adapted from \citealt{gongxun14agreement})}\label{tab:zbu.tr}
\resizebox{\columnwidth}{!}{
\begin{tabular}{l|l|l|l|l|l|l|l|l|l|l}
\textsc{} & 	\textsc{1sg} & 	  \textsc{1du} & 	\textsc{1pl} & 	\textsc{2sg} & 	\textsc{2du} & 	\textsc{2pl} & 	\textsc{3sg} & 	\textsc{3du} & 	\textsc{3pl} & 	\textsc{3'} \\ 	
\hline
\textsc{1sg} & \multicolumn{3}{c|}{\grise{}} &	\ipa{} & 	\ipa{} & 	\ipa{} &\cellcolor[wave]{600} 	\ipa{\rc{}-ŋ}   & 	\cellcolor[wave]{600} \ipa{\rc{}-ŋ-ndʑə} & 	\cellcolor[wave]{600} \ipa{\rc{}-ŋ-ɲə} & 	\grise{} \\	
\cline{8-10}
\textsc{1du} & 	\multicolumn{3}{c|}{\grise{}} &	\ipa{tɐ-\ra{}} & 	\ipa{tɐ-\ra{}-ndʑə} & 	\ipa{tɐ-\ra{}-ɲə} & 	\multicolumn{3}{c|}{ \ipa{\ra{}-tɕə}}  & 	\grise{} \\	
\cline{8-10}
\textsc{1pl} & 	\multicolumn{3}{c|}{\grise{}} & 	  & 	&  & 	\multicolumn{3}{c|}{ \ipa{\ra{}-jə}}  & 	\grise{} \\	
\cline{1-10}
\textsc{2sg} & 	\cellcolor[wave]{500}\ipa{tə-wə-\ra{}-ŋ} & 	\cellcolor[wave]{500} & 	\cellcolor[wave]{500} & 	\multicolumn{3}{c|}{\grise{}}&	\multicolumn{3}{c|}{\cellcolor[wave]{600}\ipa{tə-\rc{}}} & 	\grise{} \\	
\cline{2-2}
\cline{8-10}
\textsc{2du} & \cellcolor[wave]{500}	\ipa{tə-wə-\ra{}-ŋ-ndʑə} & \cellcolor[wave]{500}	\ipa{tə-wə-\ra{}-tɕə} & 	\cellcolor[wave]{500}\ipa{tə-wə-\ra{}-jə} & 	\multicolumn{3}{c|}{\grise{}} &	\multicolumn{3}{c|}{\ipa{tə-\ra{}-ndʑə}} & 	\grise{} \\	
\cline{2-2}
\cline{8-10}
\textsc{2pl} &\cellcolor[wave]{500} 	\ipa{tə-wə-\ra{}-ŋ-ɲə} & 	\cellcolor[wave]{500} & \cellcolor[wave]{500} & 	\multicolumn{3}{c|}{\grise{}}&	\multicolumn{3}{c|}{\ipa{tə-\ra{}-ɲə}} & 	\grise{} \\	
\hline
\textsc{3sg} & \cellcolor[wave]{500} 	\ipa{wə-\ra{}-ŋ} & 	\cellcolor[wave]{500} & 	\cellcolor[wave]{500} & 	\cellcolor[wave]{500} & 	\cellcolor[wave]{500} & 	\cellcolor[wave]{500} & \multicolumn{3}{c|}{\grise{}} &	\cellcolor[wave]{600}\ipa{\rc{}} \\ 	
\cline{2-2}
\cline{11-11}
\textsc{3du} &  \cellcolor[wave]{500}	\ipa{wə-\ra{}-ŋ-ndʑə} & 	\cellcolor[wave]{500} \ipa{wə-\ra{}-tɕə} & \cellcolor[wave]{500}		\ipa{wə-\ra{}-jə} & \cellcolor[wave]{500}	\ipa{tə-wə-\ra{}} &\cellcolor[wave]{500}	\ipa{tə-wə-\ra{}-ndʑə} & 	\cellcolor[wave]{500}\ipa{tə-wə-\ra{}-ɲə} & 	\multicolumn{3}{c|}{\grise{}} &	\ipa{\ra{}-ndʑə} \\ 
\cline{2-2}	
\cline{11-11}
\textsc{3pl} &  \cellcolor[wave]{500}	\ipa{wə-\ra{}-ŋ-ɲə} & 	\cellcolor[wave]{500} & \cellcolor[wave]{500} & 	\cellcolor[wave]{500} & 	\cellcolor[wave]{500} & 	\cellcolor[wave]{500} & \multicolumn{3}{c|}{\grise{}} &	\ipa{\ra{}-ɲə} \\ 	
\hline
\textsc{3'} & 	\multicolumn{6}{c|}{\grise{}} &\cellcolor[wave]{500}	\ipa{wə-\ra{}} & 	\cellcolor[wave]{500}\ipa{wə-\ra{}-ndʑə} & \cellcolor[wave]{500}	\ipa{wə-\ra{}-ɲə} & 	\grise{} \\	
	\hline	\hline
\textsc{intr}&\ipa{\ra{}-ŋ}&\ipa{\ra{}-tɕə}&\ipa{\ra{}-jə}&\ipa{tə-\ra{}}&\ipa{tə-\ra{}-ndʑə}&\ipa{tə-\ra{}-ɲə}&\ipa{\ra{}}&\ipa{\ra{}-ndʑə} &\ipa{\ra{}-ɲə}& 	\grise{} \\	
	\hline
\end{tabular}}
\end{table}

 Mixed and non-local scenarios (except for the stem alternations) exhibit perfect symmetry between direct and inverse forms, distinguished only by the presence or absence of the inverse prefix \ipa{wə-}. Indeed, plural mixed and non-local direct transitive forms are identical to the corresponding intransitive forms; the only difference between the two types of forms lies in the stem alternations found in the singular. In fact, the local scenario 1$\rightarrow$2 forms are the only ones which are clearly distinct from all the rest, with a synchronically opaque portmanteau \ipa{tɐ-} 1$\rightarrow$2 prefix: if the system were perfectly symmetrical, 1$\rightarrow$2 forms such as *\ipa{tə-\ra{}-ŋ} would be expected.

Rgyalrong verbal morphology does present irregular forms, but these are restricted to stem alternations (including in \textsc{1sg,2sg,3sg$\rightarrow$3} forms), first singular stems, and the second person person forms of some existential verbs. The overall structure of the transitive affixal paradigm remains the same for all verbs, even for the   highly irregular ones.

There is only one type of conjugation in Rgyalrong: there is no equivalent of the independent order vs. conjunct order of Algonquian languages.




\subsubsection{Bantawa}

While the direct/inverse system in Zbu Rgyalrong is quite close to what a canonical direct/inverse system would look like, the systems attested in the Kiranti branch of Sino-Tibetan languages are at least in part quite opaque. We present here data from \cite{doornenbal09} on Bantawa.

Despite some minor deviations from the symmetry and regularity of Zbu Rgyalrong, Bantawa presents a near-canonical direct/inverse system in the non-past affirmative indicative paradigm of transitive verbs with overall straightforwardly segmentable direction markers. This is illustrated in Table \vref{tab:bantawapos}, where the inverse marker---which occupies the slot immediately preceding the verb stem (\ro{})---is present in all relevant person configurations, excluding those where it cannot be realized due to phonological reasons, i.e., whenever there is a vowel-final person prefix or preverbal element. This prevents us from ascertaining the existence of a \textsc{sap} hierarchy in the \textsc{local} scenario and accordingly the presence or absence of the inverse marker there, but not from positing its underlying presence---with a single form (in red) calling for an alternative explanation---in all the other relevant cases (i.e., green-colored cells 
of Table \ref{tab:bantawapos}) based upon its presence in the 3\textsc{sg}/\textsc{du}$\rightarrow$ 1\textsc{sg} \textsc{mixed} scenario. Interestingly, in the \textsc{non-local} scenarios we find the direct marker instead of the inverse one in 3\textsc{sg} $\rightarrow$ 3, while the 3\textsc{du} $\rightarrow$ 3 and 3\textsc{pl} $\rightarrow$ 3\textsc{non-sg} exhibit both markers.


\begin{table}[H]
\caption{The Bantawa non-past affirmative transitive paradigm (\citealt[145-8]{doornenbal09})}\label{tab:bantawapos}
\resizebox{\textwidth}{!}{
\begin{tabular}{llllllllllll}
%\cline{1-12}
\toprule
\backslashbox{A}{P}  & 	\textsc{1sg} & 	\textsc{1di} & 	\textsc{1de} & 	\textsc{1pi} & 	\textsc{1pe} & 	\textsc{2sg} & 	\textsc{2du} & 	\textsc{2pl} & 	\textsc{3sg} & 	\textsc{3du} & \textsc{3pl}\\
% \cline{1-1}
% \cline{7-12}
\midrule
\textsc{1sg} &  \multicolumn{5}{c}{\grise{}}	 	&	\ipa{\ro{}-na} & 	\ipa{\ro{}-naci} & 	\ipa{\ro{}-nanin}& 	\ipa{\ro{}-uŋ}\cellcolor[wave]{600} & 	\multicolumn{2}{c}{\ipa{\ro{}-uŋcɨŋ}\cellcolor[wave]{600}} 	\\
\textsc{1di} & \multicolumn{8}{c}{\grise{}}	 		&	\ipa{\ro{}-cu}\cellcolor[wave]{600} & 	\multicolumn{2}{c}{\ipa{\ro{}-cuci}\cellcolor[wave]{600}}	\\
\textsc{1de} & 	 \multicolumn{5}{c}{\grise{}}	&	 	 \multicolumn{3}{c}{\ipa{\ro{}-ni}}     & 	\ipa{\ro{}-cuʔa}\cellcolor[wave]{600} & 	\multicolumn{2}{c}{\ipa{\ro{}-cuciʔa}\cellcolor[wave]{600}} 	\\
\textsc{1pi} & 	 \multicolumn{8}{c}{\grise{}}	&	\ipa{\ro{}-um}\cellcolor[wave]{600} & \multicolumn{2}{c}{	\ipa{\ro{}-umcɨm}\cellcolor[wave]{600}} 	\\
\textsc{1pe} & 	 \multicolumn{5}{c}{\grise{}}&		 \multicolumn{3}{c}{\ipa{\ro{}-ni}} & 	\ipa{\ro{}-umka}\cellcolor[wave]{600} & 	\multicolumn{2}{c}{\ipa{\ro{}-umcɨmka}\cellcolor[wave]{600}} 	\\
%\cline{10-12}
\textsc{2sg} & 	\ipa{tɨ-\ro{}-ŋa} & 	\grise{}&	\ipa{} & \grise{}	&	\ipa{} & 	 \multicolumn{3}{c}{\grise{}}	&	\ipa{tɨ-\ro{}-u}\cellcolor[wave]{600} & 	\multicolumn{2}{c}{\ipa{tɨ-\ro{}-uci}\cellcolor[wave]{600}} \\
\textsc{2du} & 	\ipa{tɨ-\ro{}-ŋaŋcɨŋ} & \grise{}&	\ipa{tɨ-\ro{}-ni(n)} & \grise{}	&	\ipa{tɨ-\ro{}-ni(n)} & 	 \multicolumn{3}{c}{\grise{}}	&	\ipa{tɨ-\ro{}-cu}\cellcolor[wave]{600} & 	\multicolumn{2}{c}{\ipa{tɨ-\ro{}-cuci}\cellcolor[wave]{600}}\\
\textsc{2pl} & 	\ipa{tɨ-\ro{}-ŋaŋnɨŋ} & \grise{}	&	\ipa{} & \grise{}	&	\ipa{} & 	 \multicolumn{3}{c}{\grise{}}&	\ipa{tɨ-\ro{}-um}\cellcolor[wave]{600} & 	\multicolumn{2}{c}{\ipa{tɨ-\ro{}-umcum}\cellcolor[wave]{600}} \\
%\cline{2-12}
\textsc{3sg} &\cellcolor[wave]{500}	 	\ipa{ɨ-\ro{}-ŋa} & 	 \cellcolor[wave]{500}	 & 	\cellcolor[wave]{500}	\ipa{(n)ɨ-\ro{}-aciʔa} &    \cellcolor{red}	& \cellcolor[wave]{500}		\ipa{(n)ɨ-\ro{}-inka} & 	\cellcolor[wave]{500}	 & 	\cellcolor[wave]{500}	& 	\cellcolor[wave]{500}	& 	\ipa{\ro{}-u}\cellcolor[wave]{600} & 	\multicolumn{2}{c}{\ipa{\ro{}-uci}\cellcolor[wave]{600}} \\
\textsc{3du} &\ipa{ɨ-\ro{}-ŋaŋcɨŋ}\cellcolor[wave]{500} & 	  \ipa{nɨ-\ro{}-ci}\cellcolor[wave]{500} 	& 	\cellcolor[wave]{500}{\multirow{2}{*}{\ipa{nɨ-\ro{}-aciʔa}}}	 & 	 \ipa{mɨ-\ro{}}\cellcolor{red} 	 & \multirow{2}{*}{\ipa{nɨ-\ro{}-inka}\cellcolor[wave]{500}} & 	\cellcolor[wave]{500}	\ipa{nɨ-\ro{}} & \ipa{nɨ-\ro{}-ci}\cellcolor[wave]{500} & 	\ipa{nɨ-\ro{}-in}\cellcolor[wave]{500} & \ipa{ɨ-\ro{}-cu} \cellcolor[wave]{550}& \multicolumn{2}{c}{\ipa{ɨ-\ro{}-cuci}\cellcolor[wave]{550}}	\\
\textsc{3pl} &	 \ipa{nɨ-\ro{}-ŋa}\cellcolor[wave]{500} & \cellcolor[wave]{500}	 &  \multirow{-2}{*}{\ipa{nɨ-\ro{}-aciʔa}\cellcolor[wave]{500}}	  & 	\cellcolor{red} &  \multirow{-2}{*}{\ipa{nɨ-\ro{}-inka}\cellcolor[wave]{500}}	 &   \cellcolor[wave]{500}		  & 	 \cellcolor[wave]{500}	  & \cellcolor[wave]{500}	   & 	\ipa{ɨ-\ro{}} \cellcolor[wave]{500}	& \multicolumn{2}{c}{\ipa{mɨ-\ro{}-uci}\cellcolor[wave]{550}} 	\\
%\cline{1-12}
\textsc{intr}	&\ipa{\ro{}-ŋa}&\ipa{\ro{}-ci}&\ipa{\ro{}-ca}&\ipa{\ro{}-in}&\ipa{\ro{}-inka}&\ipa{tɨ-\ro{}}& \ipa{tɨ-\ro{}-ci}& \ipa{tɨ-\ro{}-in}& \ipa{\ro{}}  & \ipa{\ro{}-ci} &\ipa{mɨ-\ro{}} \\
\bottomrule
\end{tabular}}
\end{table}


However, a quick look at the non-past \textit{negative} paradigm in Bantawa (cf. Table \vref{tab:bantawaneg}) reveals a different, and far more complicated picture, in which the inverse marker appears in all (phonologically licensed) person configurations, thus precluding an analysis in terms of direction marking (see especially the red-coloured cells which, whatever assumption we choose to base our analysis upon, cannot be argued to contain an inverse marker) and prompting one connecting it to the expression of negative polarity.


\begin{table}[H]
\caption{The Bantawa non-past negative transitive paradigm (\citealt[145-8]{doornenbal09})}\label{tab:bantawaneg}
\resizebox{\textwidth}{!}{
\begin{tabular}{llllllllllll}
\toprule
\backslashbox{A}{P}  & 	\textsc{1sg} & 	\textsc{1di} & 	\textsc{1de} & 	\textsc{1pi} & 	\textsc{1pe} & 	\textsc{2sg} & 	\textsc{2du} & 	\textsc{2pl} & 	\textsc{3sg} & 	\textsc{3du} & \textsc{3pl}	\\
%\cline{1-1}
%\cline{7-12}
\midrule
\textsc{1sg} &  \multicolumn{5}{c}{\grise{}}	 	&	{\ipa{ɨ-\ro{}-nan}} & 	\ipa{ɨ-\ro{}-nancin} & 	\ipa{ɨ-\ro{}-naminin}& 	\cellcolor{red}\ipa{ɨ-\ro{}-nɨŋ}& 	\multicolumn{2}{c}{\ipa{ɨ-\ro{}-nɨŋcɨŋ}\cellcolor{red}} 	\\
\textsc{1di} & \multicolumn{8}{c}{\grise{}}	 		&	\cellcolor{red}\ipa{ɨ-\ro{}-cun} & 	\multicolumn{2}{c}{\ipa{ɨ-\ro{}-cuncin}\cellcolor{red}}	\\
\textsc{1de} & 	 \multicolumn{5}{c}{\grise{}}	&	 	 \multicolumn{3}{c}{\ipa{ɨ-\ro{}-nin}}     & 	\cellcolor{red}\ipa{ɨ-\ro{}-cunka}& 	\multicolumn{2}{c}{\ipa{ɨ-\ro{}-cuncinka}\cellcolor{red}} 	\\
\textsc{1pi} & 	 \multicolumn{8}{c}{\grise{}}	&	\cellcolor{red}\ipa{ɨ-\ro{}-imin}& \multicolumn{2}{c}{\ipa{ɨ-\ro{}-imincin}\cellcolor{red}} 	\\
\textsc{1pe} & 	 \multicolumn{5}{c}{\grise{}}&		 \multicolumn{3}{c}{\ipa{ɨ-\ro{}-nin}} & 	\cellcolor{red}\ipa{ɨ-\ro{}-iminka} & 	\multicolumn{2}{c}{\ipa{ɨ-\ro{}-imincinka}\cellcolor{red}} 	\\
%\cline{10-12}
\textsc{2sg} & 	\ipa{tɨ-\ro{}-nɨŋ} & 	\grise{}&	\ipa{} & \grise{}	&	\ipa{} & 	 \multicolumn{3}{c}{\grise{}}	&	\ipa{tɨ-\ro{}-nan} & 	\multicolumn{2}{c}{\ipa{tɨ-\ro{}-nancin}} \\
\textsc{2du} & 	\ipa{tɨ-\ro{}-ŋɨŋcɨŋ} & \grise{}&	\ipa{tɨ-\ro{}-niminin} & \grise{}	&	\ipa{tɨ-\ro{}-niminin} & 	 \multicolumn{3}{c}{\grise{}}	&	\ipa{tɨ-\ro{}-nancin} & 	\multicolumn{2}{c}{\ipa{tɨ-\ro{}-nancinan}}\\
\textsc{2pl} & 	\ipa{tɨ-\ro{}-ŋɨŋmɨnɨŋ} & \grise{}	&	\ipa{} & \grise{}	&	\ipa{} & 	 \multicolumn{3}{c}{\grise{}}&	\ipa{tɨ-\ro{}-naminin}& 	\multicolumn{2}{c}{\ipa{tɨ-\ro{}-nannimincin}} \\
%\cline{2-12}
\textsc{3sg} & \ipa{ɨ-\ro{}-nɨŋ} & 	& &  &  & & & & \ipa{ɨ-\ro{}-un} & 	\multicolumn{2}{c}{\ipa{ɨ-\ro{}-uncin}} \\
\textsc{3du} & \ipa{ɨ-\ro{}-ŋɨŋcɨŋ} &   \ipa{nɨ-\ro{}-cin} 	& 	 \ipa{nɨ-\ro{}-cinka}	 & 	 \ipa{mɨ-\ro{}-nin} 	 &	\ipa{nɨ-\ro{}-iminka} & 	\ipa{nɨ-\ro{}-nan} & 	\ipa{nɨ-\ro{}-nancin} & 		\ipa{nɨ-\ro{}-naminin} & \ipa{ɨ-\ro{}-cun}& \multicolumn{2}{c}{\ipa{ɨ-\ro{}-cuncin}}	\\
\textsc{3pl} & 	\ipa{nɨ-\ro{}-nɨŋ} &  	  & 	  & 	 & 	 & 	  & 	 	  & 	   & 	\ipa{nɨ-\ro{}-un} 	& \multicolumn{2}{c}{\ipa{nɨ-\ro{}-uncin}} 	\\
%\cline{1-12}
\textsc{intr}	&\cellcolor{red}\ipa{ɨ-\ro{}-nɨŋ}&\cellcolor{red}\ipa{ɨ-\ro{}-cin}&\cellcolor{red}\ipa{ɨ-\ro{}-cinka}&\cellcolor{red}\ipa{ɨ-\ro{}-imin}&\cellcolor{red}\ipa{ɨ-\ro{}-iminka}&\ipa{tɨ-\ro{}-nan}& \ipa{tɨ-\ro{}-nanci}& \ipa{tɨ-\ro{}-naminin}& \cellcolor{red}\ipa{ɨ-\ro{}-nin}  & \cellcolor{red}\ipa{ɨ-\ro{}-cin} &\ipa{nɨ-\ro{}-nin} \\
\bottomrule
\end{tabular}}
\end{table}

While it is clearly not the case that negative polarity is expressed solely, or indeed mainly, by what in the affirmative appears as a well-behaved inverse marker (non-past negative verb forms also evince a complex affixal exponent of negation), a unified analysis of this marker's apparent polyfunctionality in synchrony is problematic. In order to account for it, a diachronic scenario on the basis of comparison with other Kiranti languages (especially Puma) seems superior to an analysis invoking mere homonymy or any type of grammatical polysemy, for that matter.

\subsection{A comparative perspective on Rgyalrong and Kiranti}

While all specialists of Sino-Tibetan languages agree that the Rgyalrong and Kiranti verbal systems are at least partially cognate (\citealt{lapolla03}, \citealt{delancey10agreement} and \citealt{jacques12agreement}),\footnote{There authors differ in their interpretation of the data: LaPolla considers the Rgyalrong / Kiranti commonalities to be common innovations, while DeLancey and Jacques think that they go back to proto-Sino-Tibetan. This controversy will not be dealt with in this paper.} there is not consensus as to exactly which type of system should be reconstructed for the common ancestor of Rgyalrong and Kiranti.

Since Rgyalrong and Kiranti languages, unlike Cree, lack ancient attestations,\footnote{There are some ancient texts in Tibetan script in Situ Rgyalrong, some of them dating back from the eighteenth century (\citealt{ngagdbang10gtamdpe}), but these texts are difficult to date, not fully understood and contain few   conjugated verbal forms; until a systematic study of verbal morphology in these texts has been undertaken, historical studies on Rgyalrong languages will have to be exclusively based on the comparative method.} former stages of these languages are only recoverable by reconstruction. The historical phonology of these languages are still imperfectly understood (see \citealt{jacques04these} and \citealt{opgenort05jero}), and the reconstruction of morphology in the Sino-Tibetan family is still in infancy -- this state of affair is due to the fact that these languages have not been described in detail until recently.

Given the absence of historical data, all logical possibilities will have to be explored to explain the commonalities and differences between the Bantawa and the Zbu verbal systems, taking also into account other Kiranti languages.




\subsubsection{Commonalities between Rgyalrong and Kiranti: non-local forms}

It is known that  Rgyalrong languages, including Zbu, are   phonologically conservative as far as syllable onsets and prefixes are concerned generally in comparison to other languages. For instance, while it has been known since \citet{conrady1896} that most if not all Sino-Tibetan languages have  traces of a causative \ipa{s--} prefix, this causative prefix only remains fully productive in Rgyalrong languages (where it can be applied to recent loanwords from Chinese). While some degree of paradigm levelling has to be posited in Rgyalrong languages in any case, otherwise the person agreement morphology would be highly irregular, it is possible that these languages have better preserved the prefixes of the proto-language than other branches.

This section focuses on the   prefixes; although most personal suffixes in Rgyalrong and Kiranti appear to be cognate, their similarity to pronouns, especially in Rgyalrong, raises the suspicion that they might have been at least in part recently  grammaticalized from pronouns, while the prefixes in general bear little resemblance to independent pronouns, and are unlikely to be recent innovations (\citealt{jacques12agreement}).   For ease of exposition, only data from Zbu and Bantawa are presented in this section. A more detailed study of comparative Rgyalrong / Kiranti verbal morphology goes beyond the scope of this paper.

Zbu Rgyalrong only counts three prefixes in its personal agreement system:   inverse   \ipa{wə--},   second person   \ipa{tə--}  and the portmanteau 1$\rightarrow$2   \ipa{tɐ--}. Bantawa has four distinct prefixes in the positive paradigm:   second person \ipa{tɨ--},   3$\rightarrow$SAP \ipa{nɨ--}, 3$\rightarrow$\textsc{1pi} and \textsc{3pl} \ipa{mɨ--} and 3$\rightarrow$1 / \textsc{3du/pl}$\rightarrow$3 \ipa{ɨ}. 


Although no rigorous phonological reconstruction of the common ancestor of Rgyalrong and Kiranti languages is yet possible,  it can be shown that  the Zbu inverse  \ipa{wə--}, and the  second person   \ipa{tə--}  are matches for Bantawa \ipa{tɨ--},  and \ipa{ɨ--} (\citealt{jacques12agreement}). There are no equivalent for the other prefixes in Zbu or in other Rgyalrong languages.

The non-local paradigms in Zbu and Bantawa are compared in Table \ref{tab:non.loc}. Zbu, like all Rgyalrong languages, has a contrast between direct and inverse forms in non-local scenarios, whose syntactic and pragmatic functions will not be described here (see \citealt{jacques10inverse} and \citealt{gongxun14agreement} for more details), but do present commonalities with the use of direct / inverse with proximate vs. obviative agent and patients in Algonquian languages. In non-local forms, the verb agrees in number with the agent in direct forms and with the patient when the inverse is present.

By contrast, no such opposition is observed in Bantawa: the marker  \ipa{ɨ--} does not have a clearly statable  morphosyntactic function in this language, but in the non-local affirmative paradigm, it only marks non-singular agent number.

\begin{table}
\caption{Comparison of non-local forms in Zbu and Bantawa}\label{tab:non.loc} \centering
\begin{tabular}{l|llll|lllllll}
\toprule
&\multicolumn{4}{c}{Zbu} & \multicolumn{3}{c}{Bantawa} \\
&\textsc{3s} & \textsc{3d} & \textsc{3p} &3' &\textsc{3s} & \textsc{3d} & \textsc{3p} \\
\hline
\textsc{3s} &\grise{} &\grise{} &\grise{} &\rc{}& \ro{}-\ipa{u} & \multicolumn{2}{c}{\ro{}-\ipa{uci}} \\ 
\textsc{3d} &\grise{} &\grise{} &\grise{} &\ro{}-\ipa{ndʑi} & \ipa{ɨ}-\ro{}-\ipa{cu}\cellcolor{green} & \multicolumn{2}{c}{\ipa{ɨ}-\ro{}-\ipa{cuci}\cellcolor{green}} \\ 
\textsc{3p} &\grise{} &\grise{} &\grise{} &\ro{}-\ipa{nɯ} & \ipa{ɨ}-\ro{}\cellcolor{green} & \multicolumn{2}{c}{\ipa{mɨ}-\ro{}-\ipa{uci}} \\ 
3' & \ipa{wɣɯ}-\ro{} \cellcolor{green}& \ipa{wɣɯ}-\ro{}-\ipa{ndʑi}  \cellcolor{green}& \ipa{wɣɯ}-\ro{}-\ipa{nɯ} \cellcolor{green}&\grise{} &\grise{} &\grise{} &\grise{} \\
\hline
\textsc{intr} & \ro{} & \ro{}-\ipa{ndʑi}  & \ro{}-\ipa{nɯ}  &\grise{}& \ro{} & \ro{}-\ipa{ci}  &\ipa{mɨ}-\ro{} \\
\bottomrule
\end{tabular}
\end{table}

It is easy to imagine a historical pathway whereby a Rgyalrong-type proximate / obviative contrast in non-local forms could be transformed into  a system like that of Bantawa. First, inverse forms are reinterpreted as non-singular agent non-local forms, a reinterpretation which is possible because the number of the agent is not specified in inverse forms the original system. As a result, the original non-singular agent direct forms are lost. Finally, the \textsc{3sg$\rightarrow$3du/pl} is built by analogy by adding the non-singular patient suffix \ipa{--uci} to the inherited direct the \textsc{3sg$\rightarrow$3'} form. 




It is also possible that the Rgyalrong and the Bantawa systems come from a third type of paradigm, for instance that inverse forms originally were indefinite third person agent forms, a hypothesis that would account for the fact that in Japhug inverse forms are also used for generic human agents. 

On the other hand, it by no means obvious to conceive a scenario explaining the Rgyalrong system from the Bantawa one: the proximate / obviative contrast in non-local forms cannot have been created out of a   number marking system. 
 
In the following, we thus assume that Bantawa non-local forms are innovative, an assumption that is confirmed by the the comparison of non-local forms in other Kiranti languages, for instance Khaling, Dumi, Thulung and  Limbu where only  forms corresponding to the Rgyalrong direct paradigm are found in non-local scenarios. Only the closely related languages Puma  and Chamling (see \citealt{bickel07puma}).

It is possible to hypothesize that proto-Kiranti still had a system with either a Rgyalrong-like direct-inverse system in the non-local forms or at least an indefinite agent marker ancestral of Bantawa \ipa{ɨ--} (and distinct from \ipa{mɨ--}, which indicates plural third person).

\subsubsection{Inverse forms}
While in the case of non-local forms we can confidently assume that Bantawa, Puma and Chamling are innovative, the same is not necessarily true of the mixed and non-local forms. The model provided by Cree and Arapaho shows that languages with a direct-inverse contrast in non-local forms can generalize it and simply reshape direct and inverse forms by combining the third person direct and inverse forms with corresponding intransitive affixes. Since in the intransitive paradigms of both Rgyalrong and nearly all Kiranti language, intransitive third person has zero marking, this reinterpretation is even easier than in Proto-Algonquian, where the third singular of the VAI paradigm was marked by *\ipa{--t--} or *\ipa{--k--}.

The regularity of the Rgyalrong paradigm somehow militates against it being entirely inherited (although this remains a distinct possibility), and suggests that it may have been remade in the same way as the Arapaho conjunct order. 

The complex and opaque Bantawa system is certainly not original and   has undergone multiple reshapings. In particular, the presence of the prefix \ipa{mɨ--} in inverse \textsc{1pi} forms derives from the third person plural \ipa{mɨ--}, which itself may derive from the Sino-Tibetan etymon for `man' (Tibetan \ipa{mi}, Japhug \ipa{tɯrme}), a development identical to that of \ipa{on} in French from Latin \textsc{homo}: `human' $\rightarrow$ generic $\rightarrow$ first plural. However, this is not to say that Bantawa cannot preserve archaisms in comparison with Rgyalrong languages.

It is possible that the ancestral language of Rgyalrong and Kiranti had a partly tripartite system like the Proto-Algonquian conjunct order, which was later further opacified in various ways in Kiranti and regularized in Rgyalrong.

Only by taking into account data from as many Sino-Tibetan languages as possible, and paying special attention to irregular morphology, can future research hope to solve this question; a  full reconstruction of the Rgyalrong and Kiranti paradigms in any case will remain impossible until the historical phonologies of these branches have been fully worked out.
 

\section{Conclusion}


 \bibliographystyle{linquiry2}
 \bibliography{biblioinverse}

\end{document}