
\documentclass{article} 
\usepackage{fontspec}
\usepackage{natbib}
\usepackage{booktabs}
\usepackage{xltxtra} 
\usepackage{polyglossia} 
 \usepackage{geometry}
 \geometry{
 a4paper,
 total={210mm,297mm},
 left=25mm,
 right=25mm,
 top=20mm,
 bottom=20mm,
 }
\usepackage[table]{xcolor}
\usepackage{gb4e} 
\usepackage{multicol}
\usepackage{graphicx}
\usepackage{float}
\usepackage{hyperref} 
\hypersetup{bookmarks=false,bookmarksnumbered,bookmarksopenlevel=5,bookmarksdepth=5,xetex,colorlinks=true,linkcolor=blue,citecolor=blue}
\usepackage[all]{hypcap}
\usepackage{memhfixc}
\usepackage{lscape}
\usepackage{amssymb}
 
%\setmainfont[Mapping=tex-text,Numbers=OldStyle,Ligatures=Common]{Charis SIL} 
\newfontfamily\phon[Mapping=tex-text,Ligatures=Common,Scale=MatchLowercase]{Charis SIL} 
\newcommand{\ipa}[1]{{\phon\textit{#1}}} 
\newcommand{\grise}[1]{\cellcolor{lightgray}\textbf{#1}}
\newfontfamily\cn[Mapping=tex-text,Ligatures=Common,Scale=MatchUppercase]{SimSun}%pour le chinois
\newcommand{\zh}[1]{{\cn #1}}
\newcommand{\Y}{\Checkmark} 
\newcommand{\N}{} 
\newcommand{\jpg}[2]{\ipa{#1} `#2'}  
\newcommand{\refb}[1]{(\ref{#1})}
\newcommand{\tld}{\textasciitilde{}}
\newcommand{\zhc}[2]{\zh{#1} \ipa{#2}} 
\bibpunct[: ]{(}{)}{,}{a}{}{,}
 \begin{document} 
\title{Apprehensive and precautioning constructions in Japhug}
\author{Guillaume Jacques\\ CNRS-CRLAO-INALCO}
\maketitle


Unique among languages of Sichuan, Japhug has an apprehensive marker, the prefix \ipa{ɕɯ-} (see example \ref{ex:CWmaqhu}), which despite its rarity in texts, presents some unusual morphological properties (such as a specific direct 3$\rightarrow$3' form). Unlike many languages with apprehensive morphology (\citealt{lichtenberg95apprehensional}, \citealt{francois03predicat}, \citealt{verstrate05mood}, \citealt{overall09linking}, \citealt{vuillermet18fear}), it is hardly if ever used in precautioning constructions. 

\begin{exe}
\ex \label{ex:CWmaqhu}
\gll  \ipa{ɕɯ-maqhu-a}  \ipa{kɯ} 	\ipa{ʑo} 	\ipa{ɲɤ-sɯso-nɯ} 	\ipa{tɕe} 	\ipa{rcanɯ,} 	\ipa{pɕoʁʑi} 	\ipa{kɯβde} 	\ipa{ʑo} 	\ipa{jo-ɣi-nɯ.}  \\
\textsc{apprehensive}-be.after-\textsc{1sg} \textsc{sfp} \textsc{emph} \textsc{ifr}-think \textsc{lnk} unexpectedly corner four \textsc{emph} \textsc{ifr}-come-\textsc{pl} \\
\glt `Fearing of being late, they came from the four corners of the world.' (150906 toutao, 19)
 \end{exe}
%争先恐后

There are two main precautioning constructions in Japhug: finite precautioning clauses and purposive clauses.

The former lacks dedicated morphology, and consists of either a clause in the irrealis, or a clause introduced by the linker \ipa{ma} `because' with a verb in the factual non-past, often with the adverb \ipa{tʰa} or \ipa{tɕetʰa} `later' as in (\ref{ex:tha.tWnAndzxo}) (see \citealt[308]{jacques14linking}), a typologically frequent type of construction (see for instance  \citealt{angelo16beware}).

\begin{exe}
\ex \label{ex:tha.tWnAndzxo}
\gll \ipa{nɤ-ŋga} 	\ipa{nɤki,} 	\ipa{mɤʑɯ} 	\ipa{tɤ-ndɤm} 	\ipa{tɕe} 	\ipa{a-tɤ-ɤjɤzjɯ} 	\ipa{ma} 	\ipa{tha} 	\ipa{tɯ-nɤndʐo} \\
\textsc{2sg}.\textsc{poss}-clothes filler again \textsc{imp}-take[III] \textsc{lnk} \textsc{irr}-\textsc{pfv}-add.some.more \textsc{lnk} later 2-feel.cold:\textsc{fact} \\
\glt `Put some more clothes on, add some, so that you don't get cold.'  
\end{exe}

%lu-zɣɯt ɕɯŋgɯ tɕe nɯtɕu ku-ɣɤrat-a tɕetha a-mɤ-lɤ-zɣɯt 
% 25-kAmYW-XpAltCin, 62

Finite precautioning constructions are only used to express evaluation by the speaker. Evaluation by a discourse participant other than the speaker are nevertheless possible by embedding the precautioning clause in a reported speech complement with a verb of cognition such as \ipa{sɯso} `think', as in (\ref{ex:GWznAndAGnW}).

\begin{exe}
\ex \label{ex:GWznAndAGnW}
\gll
\ipa{na-phɯt-nɯ} 	\ipa{kɯnɤ} 	\ipa{chɯ-βde-nɯ} 	\ipa{ɕti.}  \ipa{tɕe} 	\ipa{pjɯ-rɤtɕɯmtɕaʁ-nɯ} 	\ipa{ma} \ipa{tɯrme} 	\ipa{ra} 	\ipa{kɯ} 	\ipa{a-mɤ-tɤ-ndo-nɯ} 	\ipa{ma} 	\ipa{ɣɯ-znɤndɤɣ-nɯ} 	\ipa{ɲɯ-sɯso-nɯ} \\
\textsc{pfv}:3$\rightarrow$3'-pluck-\textsc{pl} also \textsc{ipfv}-throw.away-\textsc{pl} be.\textsc{affirm}:\textsc{fact} \textsc{lnk} \textsc{ipfv}-trample-\textsc{pl} \textsc{lnk} people \textsc{pl} \textsc{erg} \textsc{irr-neg-pfv-take}-\textsc{pl} \textsc{lnk} \textsc{inv-caus}-be.poisoned\textsc{:fact}-\textsc{pl} \textsc{sens}-think-\textsc{pl} \\
\glt `Even if they pluck (this mushroom), they throw it away and trample on it, so that people don't take it and don't get poisoned.' (23-grWBgrWBftsa, 27)
\end{exe}

%wortɕhi wojɤr ʑo tɕhi maʁ nɤ a-ɣi ra nɯ-phe ɕɯ-rɤfɕɤt-tɕi ma tɕetha ɣɯ-nɯzdɯɣ-a-nɯ
%
%kha ɯ-pɕi nɯ ʁmaʁ χsɯ-tɤ-xɯr pa-sɯ-lɤt ma "tha a-mɤ-jɤ-nɯɣi ma, mɲika" na-sɯso
% 2003qachga, 145
 
 
The negative purposive converb can be also be used to express precautioning meaning as in (\ref{ex:WmAYWsAjmWjmWt}). This construction is considerably rarer than the finite precautioning clauses, but can express evaluation by the subject of the main clause.
 
\begin{exe}
\ex \label{ex:WmAYWsAjmWjmWt}
\gll
[\ipa{kɯ-lɤɣ}   	\ipa{acɤβ}   	\ipa{nɯ}   	\ipa{kɯ}   	\ipa{\textbf{ɯ-mɤ-sɤ-jmɯ\textasciitilde{}jmɯt}},]   	\ipa{ɯ-pʰɯŋgɯ}   	\ipa{nɯ}   	\ipa{tɕu}   	\ipa{rdɤstaʁ-pɯpɯ}   	\ipa{tɕʰɯrdu}   	\ipa{ci}  \ipa{ɲɤ-rku,}\\
 \textsc{nmlz}:S/A-herd Askyabs \textsc{dem} \textsc{erg}  \textsc{3sg-neg-purp:conv}-forget \textsc{3sg.poss}-inside.clothes \textsc{dem} \textsc{loc} stone-little pebble \textsc{indef}
 \textsc{evd}-put.in\\
\glt `The shepherd Askyabs put a little pebble inside his clothes so that he would not forget it'. (qaCpa2002, 166)
\end{exe}

%tɕe nɯ ɯ-pa nɯnɯ li khɤxtu nɯnɯ, tɯci, tɯftsaʁ kɯ pjɯ-sɯspoʁ ŋgrɤl tɕe,
%tɕe ɯ-mɤ-pjɯ-sɤ-sɯspoʁ, nɯnɯ tɕu 
%nɤki, tɤrɤm kɯ-fse ɲɯ́-wɣ-ta nɯ maʁ nɤ cɯpa kɯ-fse ɲɯ́-wɣ-ta tɕe,

%\begin{exe}
%\ex \label{ex:jinbala}
%\gll 
%\ipa{tɕe}  	[\ipa{rɟɤlpu}  	\ipa{nɯ}  	\ipa{nɯ-rga}]  	\ipa{jinbala}  	\ipa{zɯ,}  	\ipa{`e,}  	\ipa{a-tɕɯ}  	\ipa{ki}  	\ipa{stɤβtsʰɤt}  	\ipa{mɯ-ɕɯ-cʰa}  	\ipa{kɯ}  	\ipa{ɲɤ-sɯso}  \\
%\textsc{lnk} king \textsc{dem} \textsc{pfv}-be.happy although \textsc{loc} \textsc{interj} \textsc{1sg.poss}-son \textsc{dem:prox} contest \textsc{neg-apprehensive}-can:\textsc{fact} \textsc{possibility} \textsc{evd}-think \\
%\glt `Although the king was pleased, he thought `Ah, I fear that my son will not succeed in this contest.'' (sras2003, 91-92)
% \end{exe}
 

%\begin{exe}
%\ex \label{ex:CWNGrW}
%\gll \ipa{nɯ} 	\ipa{fse} 	\ipa{laχtɕha} 	\ipa{ɯ-ŋgɯ} 	\ipa{chɯ-tɯ-rke} 	\ipa{tɕe} 	\ipa{ɕɯ-ɴɢrɯ} 	\ipa{kɯ!}   \\
%dem be.like:fact thing 3sg.poss-inside ipfv:downstream-2-put.in[III] lnk \textsc{apprehensive}-\textsc{anticaus}:break:\textsc{fact} \textsc{sfp} \\
%\glt `(If) you put things like that into it, it (may) break.' (elicited)
% \end{exe}

\bibliographystyle{unified}
\bibliography{bibliogj}
 \end{document}
 