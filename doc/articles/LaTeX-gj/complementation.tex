%aʑo me, nɤʑo me.

\documentclass[oldfontcommands,oneside,a4paper,11pt]{article} 
\usepackage{fontspec}
\usepackage{natbib}
\usepackage{booktabs}
\usepackage{xltxtra} 
\usepackage{longtable}
\usepackage{polyglossia} 
%\usepackage[table]{xcolor}
\usepackage{gb4e} 
\usepackage{multicol}
\usepackage{graphicx}
\usepackage{float}
\usepackage{lineno}
\usepackage{textcomp}
\usepackage{hyperref} 
\hypersetup{bookmarks=false,bookmarksnumbered,bookmarksopenlevel=5,bookmarksdepth=5,xetex,colorlinks=true,linkcolor=blue,citecolor=blue}
\usepackage[all]{hypcap}
\usepackage{memhfixc}
\usepackage{lscape}
 

\setmainfont[Mapping=tex-text,Numbers=OldStyle,Ligatures=Common]{Charis SIL} \newfontfamily\phon[Mapping=tex-text,Ligatures=Common,Scale=MatchLowercase,FakeSlant=0.3]{Charis SIL} 
\newcommand{\ipa}[1]{{\phon #1}} %API tjs en italique
 
\newcommand{\grise}[1]{\cellcolor{lightgray}\textbf{#1}}
\newfontfamily\cn[Mapping=tex-text,Ligatures=Common,Scale=MatchUppercase]{MingLiU}%pour le chinois
\newcommand{\zh}[1]{{\cn #1}}


\XeTeXlinebreaklocale "zh" %使用中文换行
\XeTeXlinebreakskip = 0pt plus 1pt %
 %CIRCG
 


\begin{document} 

\title{Complementation in Japhug Rgyalrong\footnote{
The glosses follow the Leipzig glossing rules. Other abbreviations used here are: \textsc{appl} applicative, \textsc{antipass} antipassive,\textsc{const} constative, \textsc{coord} coordinator, \textsc{dem} demonstrative, \textsc{dist} distal, \textsc{emph} emphatic, \textsc{indef} indefinite, \textsc{inv} inverse,  \textsc{pfv} perfective, \textsc{poss} possessor. %\textsc{trop} tropative, 
Acknowledgements will be added after editorial decision.  
} }
\author{Guillaume Jacques}
\maketitle
\linenumbers
 
 \citet[9]{dixon06complementation}
 
\section{Types of complement clauses} 


\subsection{Ergative marking}
paʁ ra kɯ kɤ-ndza wuma ʑo rga-nɯ
 \subsection{Stative verbs}
 
 \begin{exe}
\ex \label{ex:pjWnArte}
\gll
 \ipa{koŋla} 	\ipa{pjɯ-nɤrte} 	\ipa{ʑo} 	\ipa{kɯ-rtaʁ} 	\ipa{ʑo} 	\ipa{kɯ-wxti} 	\ipa{ɲɯ-βze} 	\ipa{ŋgrɤl} 	\ipa{ma} \\
 really \textsc{ipfv}-wear.on.the.head \textsc{emph} \textsc{inf:stat}-enough \textsc{emph}  \textsc{inf:stat}-big \textsc{ipfv}-grow[III] \textsc{n.pst}:be.usually.the.case lnk \\
 \glt It grows big enough to serve as a hat. (Burdock, 35)

\end{exe}


in particular derived tropative and deexperiencer verbs

 \section{Complementation strategies} 
\section{Primary type} 
(NP or complement clause)
 With verbs of this type, ambiguity with relative clauses
\subsection{Attention} 
 (perception)
 see, hear, notice, smell, show 
\ipa{mto}, \ipa{ru}, \ipa{nɤmnɤm}, \ipa{mtshɤm}, \ipa{sɤŋo}


ɯ-qhu kɤ-ndza ɯ-spa ɲɯ-sɯɣrom-nɯ tɕe ɲɯ-ɕkho-nɯ nɯra pɯ-mto-t-a ma
nɯ ma aʑo kɤ-ndza mɯ-pɯ-rɲo-t-a.
ʑmbrijmɤɣ, 184

kɯ-ɤrmbat kɤ-mto mɯ-pɯ-rɲo-t-a ri,
ɲɯ-nɯqambɯmbjom nɯ pɯ-mto-t-a.

ma sla tɤ-tu tɕe, tɕe ɯʑo (qarma) ju-nɯɕe ɲɯ-mtɤm tɕe, 
有月亮的时候,马鸡看得见飞走的方向(知道应该往哪里飞)

		 \begin{exe}
\ex \label{ex:mtsham2}
 \gll   	\ipa{tu-ɣɤwu}    	\ipa{ kɤ-mtshɤm}   	\ipa{pɯ-rɲo-t-a}     	 \\
 \textsc{ipf}-cry   \textsc{inf}-hear \textsc{aor}-experience-\textsc{pst}-\textsc{1sg} \\
 \glt I have already heard a wolf howling. (Wolf, 9)
\end{exe} 
 		

tɕeri mbrɤʑɯm kɯ pjɤ́-wɣ-znɤndɤɣ kɤ-ti ɯ-mtshɤm mɯ-pɯ́-wɣ-rɲo. 

 kɤndza ɲɯmɯm ɕi mɯ́jmɯm ɲɯ-sɤŋo
 nɤŋga ɲɯrtaʁ ɕi mɯ́jrtaʁ kɯ nɯ-sɤŋo
 
 
tɕe nɯ kɯ-fse tu-ste ɲɯ-ra tɕe tɕe,
kɤsɯfse kɤ-χsu tɕhi tu-ste kɯ-nɯ-ŋu maʁ mɯ-kú-wɣ-rtoʁ ma  
 
recognize, discover,  
%\ipa{nɤfse}, \ipa{sɯχsɤl}
tɕe nɯnɯ tɕe ɯ-se la-mɯmtsrɯɣ nɯ ɲɯ-saχsɤl ma,
ŋotɕu nɯ ɯku ŋu, ŋotɕu ɯjme ŋu, mɯ́jsaχsɤl

strategy:
ɕ-tɤ-ru ri,
mɤŋi kɤ-kɯ-ɣe χpɯn thɯ-kɯ-rgɤz ci pjɤ-tu tɕe, nɯ pjɤ-ŋu

		 \begin{exe}
\ex \label{ex:sANo3}
\gll \ipa{ɲɯ-kɯ-sɤŋo}   	\ipa{tɕe}   	\ipa{zɯrzɯrzɯr}   	\ipa{tu-ti}   	\ipa{qhe}   	\ipa{tɕendɤre}   	\ipa{tɤ-ndɤr}   	\ipa{ɲɯ-ɬoʁ}   	\ipa{ɕti.}\\
\textsc{ipf-genr:S/O}-listen \textsc{coord}  fuzzy.feeling \textsc{ipf}-say \textsc{coord} \textsc{coord} \textsc{indef.poss}-pimple \textsc{ipf}-appear be:\textsc{assert} \\
 \glt  One has a fuzzy feeling and a pimple  appears. (Pimples, 3)
\end{exe} 
  	
  	kú-wɣ-rtoʁ qhe ɲɯ-nɯqambɯmbjom cha,
nɯ maʁ ko-zo qhe, tɕe thɯci βʑɯ ɲɯ-fse kɯ-fse ri,

  \subsection{thinking} 	

(a) think (of/about/over), consider, imagine, dream (of/about)
sɯso, ɣɤjmŋo, nɯjmŋo
(b) assume, suppose ʁnɯ
(c) remember, forget jmɯt, ɕɯftaʁ
(d) know, understand tso, sɯz, mɤxsi
(e) believe, suspect nɤstu, 

hesitate nɯsɯmʁɲɯz

ji-kɯjŋu tsaʁ ɲɯ-fɕaʁ-a ɲɯ-sɯsam-a ŋu


ɯʑɤɣ nɯ kɤ-nɯ-thu na-nɯ-jmɯt ɲɯ-ŋu,
a-mgɯr	tɤ-ŋkhɯt	kɤ-lɤt	mɯ-ɲɤ-tɯ-nɯ-jmɯt	ɯ́-ŋu,
D	37	要给我锤背,你没忘吗?(给孙子说的)


ŋotɕu tu-rma-nɯ ɲɯŋu nɯ ku-ɕɯftaʁ-nɯ
nɯ-rga-j ritɕi, ŋotɕu jɤ-nɯɬoʁ tɕi mɯ-pjɤ-mto-j,
ŋotɕu jɤ-ɕqhlɯt tɕi kɤ-ɕɯftaʁ mɯ-pjɤ-cha-j

kupa kɯ, nɤkinɯ, khu a-pɯ́-wɣ-ɣɤjmŋo ŋu tɕe, nɤkinɯ,
a-pɯ́-wɣ-ɣɤjmŋo tɕe, χsɯ-xpa nɯ kɯ-mɤɕi tu-ti-nɯ ɲɯ-ŋu

A 118 tɕendɤre nɯtɕu ku-rɤʑi ŋu pjɤ-sɯz-nɯ tɕe
  \subsection{LIKING} 

(a) like, love, prefer, regret
rga
(b) fear
mu nɯɣmu
(c) enjoy
be worried 

nɯzdɯɣ


  \subsection{SPEAKING} 
  
  
(a) say, inform, tell (one sense)
(b) report
(c) describe, refer to
(d) promise, threaten
(e) order, command, persuade, tell (one sense)
 prepare
 %Secondary type: require complement clause

kɯ-rɤβzjoz tɤ-βgoz-i
 ndʑi-stɯnmɯ nɯ phama pɯ-βgoz pɯ-ŋu ma ʑɤni nanɯtɯɣndʑi pɯmaʁ
 
 
 \section{Secondary A} 
  The Secondary concept provides no addition to the semantic roles associated with the verb to which it is related.
 ɬoʁ
ŋgrɤl
zgɤt
jɤɣ sɯɣjɤɣ
cha sɯɣcha
khɯ ɣɤkhɯ
tʂaŋ
ra
ʁzi
ntshi
spa
nɤz
phot
mna

tɕe thɯci tsuku ʑo chɯ-βzu-nɯ ɲɯ-khɯ
tɕe nɯnɯ ɕom nɯ wuma ʑo ʁzi

nɯ ɯ-mdoʁ nɯ ɤj kɤ-ti mɯ́j-spe-a ma arŋi ɯ-ŋgɯz kɯnɤ pɣi kɯ-fse
tɕaɣi nɯ pjɯ́-wɣ-sɯxɕɤt tɕe, tɯrme ɯ-skɤt tu-βze spe tu-ti-nɯ ŋgrɤl.



kɯwxti nɯra mɤkɯsi chɯnɯmɟe mɯ́jnɤz
tɕhemɤpɯ pɯɕtia qhe tɤtɕɯ ra nɯɕki ku-rɤʑi-a mɯ́j-naz-a qhe
kɤ-ʑɣɤɕɯfka mɯ-pɯ-naz-a

 \begin{itemize}
\item negator
\item modal: ‘can’, ‘should’, ‘must’, ‘might’. + dare?
\item phasal: begin, start, continue, finish
\item try, attempt,  
\end{itemize}

phasal: ʑa, sɤʑa, sthɯt ʂɤtɕɯtʂi

attempt
tshɤt,  nɯftɕɤka
  \section{Secondary B} 
 ‘want’, ‘wish (for)’, ‘hope (for)’, ‘intend’, ‘plan (for)’, ‘pretend’.
sɯso, nɯʁjɯβtshɤt, ʁmɯɣ, nɯʁlɯmbɯɣ, nɯɕpɯz, ʑɣɤpa, nɯmga, ɯ-ʁjiz ɣi

yangyu cho tɯ-ci ʁɟa ʑo nɯ kɯ-fse tu-ndze.
ma nɯ ma tu-ndze ɯ-ʁjiz mɤ-ɣi ŋgrɤl.
kɤ-nɤma tɯ-ʁjɯz maka mɤ-ɣi.


qachɣa kɯ-mbri kɯ-fse, nɯra tu-nɯɕpɯz ɲɯ-spe. 

  \section{Secondary C} 
‘make’, ‘cause’, ‘force’, ‘let’, and ‘help’.
sɤβzu
>increase in valency


purposive:
kɯ-ɣɤrʁaʁ ra kɯ tu-ndo-nɯ.
nɯ-kɯ-qur tu-ndo-nɯ pjɤ-ŋgrɤl.

  \
 
 


\bibliographystyle{linquiry2}
\bibliography{bibliogj}
\end{document}