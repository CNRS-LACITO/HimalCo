%aʑo me, nɤʑo me.

\documentclass[oldfontcommands,oneside,a4paper,11pt]{article} 
\usepackage{fontspec}
\usepackage{natbib}
\usepackage{booktabs}
\usepackage{xltxtra} 
\usepackage{longtable}
\usepackage{polyglossia} 
%\usepackage[table]{xcolor}
\usepackage{gb4e} 
\usepackage{multicol}
\usepackage{graphicx}
\usepackage{float}
\usepackage{lineno}
\usepackage{textcomp}
\usepackage{hyperref} 
\hypersetup{bookmarks=false,bookmarksnumbered,bookmarksopenlevel=5,bookmarksdepth=5,xetex,colorlinks=true,linkcolor=blue,citecolor=blue}
\usepackage[all]{hypcap}
\usepackage{memhfixc}
\usepackage{lscape}
 

\setmainfont[Mapping=tex-text,Numbers=OldStyle,Ligatures=Common]{Charis SIL} \newfontfamily\phon[Mapping=tex-text,Ligatures=Common,Scale=MatchLowercase,FakeSlant=0.3]{Charis SIL} 
\newcommand{\ipa}[1]{{\phon #1}} %API tjs en italique
 
\newcommand{\grise}[1]{\cellcolor{lightgray}\textbf{#1}}
\newfontfamily\cn[Mapping=tex-text,Ligatures=Common,Scale=MatchUppercase]{MingLiU}%pour le chinois
\newcommand{\zh}[1]{{\cn #1}}


\XeTeXlinebreaklocale "zh" %使用中文换行
\XeTeXlinebreakskip = 0pt plus 1pt %
 %CIRCG
 


\begin{document} 

\title{Complementation in Japhug Rgyalrong\footnote{
The glosses follow the Leipzig glossing rules. Other abbreviations used here are: \textsc{appl} applicative, \textsc{antipass} antipassive,\textsc{const} constative, \textsc{coord} coordinator, \textsc{dem} demonstrative, \textsc{dist} distal, \textsc{emph} emphatic, \textsc{indef} indefinite, \textsc{inv} inverse,  \textsc{pfv} perfective, \textsc{poss} possessor. %\textsc{trop} tropative, 
Acknowledgements will be added after editorial decision.  
} }
\author{Guillaume Jacques}
\maketitle
\linenumbers
 
 \citet[9]{dixon06complementation}
 \citet{sun12complementation}
 \citet{jacques08}
 
\section{Types of complement clauses} 


\subsection{Ergative marking}
paʁ ra kɯ kɤ-ndza wuma ʑo rga-nɯ
 \subsection{Stative verbs}
 
 \begin{exe}
\ex \label{ex:pjWnArte}
\gll
 \ipa{koŋla} 	\ipa{pjɯ-nɤrte} 	\ipa{ʑo} 	\ipa{kɯ-rtaʁ} 	\ipa{ʑo} 	\ipa{kɯ-wxti} 	\ipa{ɲɯ-βze} 	\ipa{ŋgrɤl} 	\ipa{ma} \\
 really \textsc{ipfv}-wear.on.the.head \textsc{emph} \textsc{inf:stat}-enough \textsc{emph}  \textsc{inf:stat}-big \textsc{ipfv}-grow[III] \textsc{n.pst}:be.usually.the.case lnk \\
 \glt It grows big enough to serve as a hat. (Burdock, 35)

\end{exe}


in particular derived tropative and deexperiencer verbs

 \section{Complementation strategies} 
\section{Primary type} 
(NP or complement clause)
 With verbs of this type, ambiguity with relative clauses
\subsection{Attention} 
 (perception)
 see, hear, notice, smell, show 
\ipa{mto}, \ipa{ru}, \ipa{nɤmnɤm}, \ipa{mtshɤm}, \ipa{sɤŋo} nɤmɲo nɤrtoχpkɤt

\subsubsection{mto}

ɯ-qhu kɤ-ndza ɯ-spa ɲɯ-sɯɣrom-nɯ tɕe ɲɯ-ɕkho-nɯ nɯra pɯ-mto-t-a ma
nɯ ma aʑo kɤ-ndza mɯ-pɯ-rɲo-t-a.
ʑmbrijmɤɣ, 184

kɯ-ɤrmbat kɤ-mto mɯ-pɯ-rɲo-t-a ri,
ɲɯ-nɯqambɯmbjom nɯ pɯ-mto-t-a.

ma sla tɤ-tu tɕe, tɕe ɯʑo (qarma) ju-nɯɕe ɲɯ-mtɤm tɕe, 
有月亮的时候,马鸡看得见飞走的方向(知道应该往哪里飞)

  \subsubsection{rtoʁ}
tɕe nɯ kɯ-fse tu-ste ɲɯ-ra tɕe tɕe,
kɤsɯfse kɤ-χsu tɕhi tu-ste kɯ-nɯ-ŋu maʁ mɯ-kú-wɣ-rtoʁ ma  
 
 
 strategy   	kú-wɣ-rtoʁ qhe ɲɯ-nɯqambɯmbjom cha,
nɯ maʁ ko-zo qhe, tɕe thɯci βʑɯ ɲɯ-fse kɯ-fse ri,

\subsubsection{ru}
strategy:
ɕ-tɤ-ru ri,
mɤŋi kɤ-kɯ-ɣe χpɯn thɯ-kɯ-rgɤz ci pjɤ-tu tɕe, nɯ pjɤ-ŋu




\subsubsection{mtsʰɤm}
		 \begin{exe}
\ex \label{ex:mtsham2}
 \gll   	\ipa{tu-ɣɤwu}    	\ipa{ kɤ-mtshɤm}   	\ipa{pɯ-rɲo-t-a}     	 \\
 \textsc{ipf}-cry   \textsc{inf}-hear \textsc{aor}-experience-\textsc{pst}-\textsc{1sg} \\
 \glt I have already heard a wolf howling. (Wolf, 9)
\end{exe} 
 		

tɕeri mbrɤʑɯm kɯ pjɤ́-wɣ-znɤndɤɣ kɤ-ti ɯ-mtshɤm mɯ-pɯ́-wɣ-rɲo. 

\subsubsection{sɤŋo}
 kɤndza ɲɯmɯm ɕi mɯ́jmɯm ɲɯ-sɤŋo
 nɤŋga ɲɯrtaʁ ɕi mɯ́jrtaʁ kɯ nɯ-sɤŋo
 
strategy 
 		 \begin{exe}
\ex \label{ex:sANo3}
\gll \ipa{ɲɯ-kɯ-sɤŋo}   	\ipa{tɕe}   	\ipa{zɯrzɯrzɯr}   	\ipa{tu-ti}   	\ipa{qhe}   	\ipa{tɕendɤre}   	\ipa{tɤ-ndɤr}   	\ipa{ɲɯ-ɬoʁ}   	\ipa{ɕti.}\\
\textsc{ipf-genr:S/O}-listen \textsc{coord}  fuzzy.feeling \textsc{ipf}-say \textsc{coord} \textsc{coord} \textsc{indef.poss}-pimple \textsc{ipf}-appear be:\textsc{assert} \\
 \glt  One has a fuzzy feeling and a pimple  appears. (Pimples, 3)
\end{exe} 
 

% \subsubsection{nɤmnɤm}

 \subsubsection{saχsɤl}
recognize, discover,  
%\ipa{nɤfse}, \ipa{sɯχsɤl}
tɕe nɯnɯ tɕe ɯ-se la-mɯmtsrɯɣ nɯ ɲɯ-saχsɤl ma,
ŋotɕu nɯ ɯku ŋu, ŋotɕu ɯjme ŋu, mɯ́jsaχsɤl

 
% \subsubsection{nɤrtoχpjɤt}
 
  	
 

  \subsection{Thought} 	

(a) think (of/about/over), consider, imagine, dream (of/about)
sɯso, ɣɤjmŋo, nɯjmŋo
(b) assume, suppose ʁnɯ
(c) remember, forget jmɯt, ɕɯftaʁ
(d) know, understand tso, sɯz, mɤxsi
(e) believe, suspect nɤstu, 
hesitate nɯsɯmʁɲɯz


\subsubsection{sɯso}
ji-kɯjŋu tsaʁ ɲɯ-fɕaʁ-a ɲɯ-sɯsam-a ŋu

\subsubsection{ɣɤjmŋo}
kupa kɯ, nɤkinɯ, khu a-pɯ́-wɣ-ɣɤjmŋo ŋu tɕe, nɤkinɯ,
a-pɯ́-wɣ-ɣɤjmŋo tɕe, χsɯ-xpa nɯ kɯ-mɤɕi tu-ti-nɯ ɲɯ-ŋu
%\subsubsection{ʁnɯ}
\subsubsection{jmɯt}

ɯʑɤɣ nɯ kɤ-nɯ-thu na-nɯ-jmɯt ɲɯ-ŋu,
a-mgɯr	tɤ-ŋkhɯt	kɤ-lɤt	mɯ-ɲɤ-tɯ-nɯ-jmɯt	ɯ́-ŋu,
D	37	要给我锤背,你没忘吗?(给孙子说的)

\subsubsection{ɕɯftaʁ}
ŋotɕu tu-rma-nɯ ɲɯŋu nɯ ku-ɕɯftaʁ-nɯ
nɯ-rga-j ritɕi, ŋotɕu jɤ-nɯɬoʁ tɕi mɯ-pjɤ-mto-j,
ŋotɕu jɤ-ɕqhlɯt tɕi kɤ-ɕɯftaʁ mɯ-pjɤ-cha-j

\subsubsection{tso}
tɕhi skɤt ŋu, ɯ-ɲɯ́-tɯ-tso?
nɯnɯ kɤ-ndza ɣɤʑu ɲɯ-ŋu ɲɯ-tso-nɯ qhe, tɕe
tu-nɯstɤrɟɯɣ-nɯ ʑo ju-nɯɣi-nɯ ɕti

\subsubsection{sɯz}

A 118 tɕendɤre nɯtɕu ku-rɤʑi ŋu pjɤ-sɯz-nɯ tɕe
mɤxsi

%\subsubsection{nɤstu}
\subsubsection{nɯsɯmʁɲɯz}
strategy
  \subsection{Liking} 

(a) like, love, prefer, regret
rga
(b) fear
mu nɯɣmu
(c) enjoy
be worried 
nɯzdɯɣ
ɲɟɤt


\citealt{jacques13tropative}
\subsubsection{rga}
rga + nɯrga

tshɤt cho qaʑo kɤlɤt wuma ʑo pjɤrga
kɤʑɣɤnɯkhramba mɯ́jrga 他不喜欢被骗

ɯ-smɯmba nɯ tu ma khro sthɯci mɤ-sɤɕke qhe,
tɕe nɯnɯ mɤ-aɣɯxɕɤt qhe,
nɯnɯ kɤ-nɯ-βlɯ kɯ-rga me,
khɯrtshɤz nɯnɯ paʁ ra wuma ʑo kɤ-ndza rga-nɯ
ɣʑo kɯnɤ nɯ tɕu wuma ʑo, nɯnɯ tɕe, kɯ-chi ɕɯ-kɤ-tɕɤt ɲɯ-rga tɕe,

finite
nɯnɯ rcanɯ, paʁ ra kɯ tu-ndza-nɯ rga-nɯ,
ɴqiaβ, praʁku nɯra tu-ɬoʁ rga tɕe,

strategy
tɯrme kɯ tú-wɣ-rqoʁ tɕe ɲɯ-rga

χpi kɤfɕɤt ɲɯ-rga: 喜欢讲
χpi kɤfɕɤt ɲɯ-nɯrge: 又喜欢讲,又喜欢听

\subsubsection{qha}

hate
tshɤt qaʑo kɤ-lɤɣ ɲɯ-qha-nɯ ma

χpi kɤfɕɤt ɲɯ-qhe ri, kɤsɤŋo ɲɯrga


%\subsubsection{ɲɟɤt}
%\subsubsection{mu}
%\subsubsection{nɯzdɯɣ}


  \subsection{Speaking} 
  
  
(a) say, inform, tell (one sense)
(b) report
(c) describe, refer to
(d) promise, threaten
(e) order, command, persuade, tell (one sense)
 prepare
 %Secondary type: require complement clause
 
  \subsubsection{ti}
  
  indirect speech
 \subsubsection{nɤla}
 
\subsubsection{βgoz}
kɯ-rɤβzjoz tɤ-βgoz-i
 ndʑi-stɯnmɯ nɯ phama pɯ-βgoz pɯ-ŋu ma ʑɤni nanɯtɯɣndʑi pɯmaʁ
 
\subsubsection{sɯxɕɤt} 
teach

 nɤ-mu kɯ tɤ-rɟɯt kɤ-nɤmɤle ra   ɯ-tɤ-tɯ́-wɣ-sɯxɕɤt?
\subsection{Deixis}
 stu
 
 \section{Motion verbs}
 
 \citealt{jacques13harmonization}
 ɕe, ɣi rɟɯɣ nɯqambɯbjom
 \subsubsection{sɯxɕe}
 aɣi ra kɯ kɯrɤβzjoz jɤ́wɣsɤɣrianɯ
 \section{Secondary A} 
  The Secondary concept provides no addition to the semantic roles associated with the verb to which it is related.
   \begin{itemize}
\item negator
\item modal: ‘can’, ‘should’, ‘must’, ‘might’. + dare?
\item phasal: begin, start, continue, finish
\item try, attempt,  
\end{itemize}

  \subsection{Negation}
  me
  
  ndza mɤ-ndza me-a
  
  
  \subsection{Modal verbs}
 ɬoʁ
ŋgrɤl
zgɤt
jɤɣ sɯɣjɤɣ
cha sɯɣcha
khɯ ɣɤkhɯ
tʂaŋ
ra
ʁzi
ntshi
spa

 
mna

tɕe thɯci tsuku ʑo chɯ-βzu-nɯ ɲɯ-khɯ
tɕe nɯnɯ ɕom nɯ wuma ʑo ʁzi

nɯ ɯ-mdoʁ nɯ ɤj kɤ-ti mɯ́j-spe-a ma arŋi ɯ-ŋgɯz kɯnɤ pɣi kɯ-fse
tɕaɣi nɯ pjɯ́-wɣ-sɯxɕɤt tɕe, tɯrme ɯ-skɤt tu-βze spe tu-ti-nɯ ŋgrɤl.


 \subsubsection{jɤɣ}
 +sɯɣjɤɣ
 ɲɤʑɣɤsɤphɤr tɕe arca tutɯɣi mɤjɤɣ toti.

atsatsa matɯti ra ti ɲɯŋu
tɕendɤre tutatsɯm jɤɣ" toti,

ji-wa kɯ mbroχpa ɲɯ-ɕar mɤ-jɤɣ ɲɯ-ti tɕe,

"api tɕaʁmbɯm wortɕhi tukɯznɯkhoa ɯjɤɣ"

tɕe nɯnɯ ɕɯkɯnɯmbjɯm tɯʑɤsɯso mɯpɯjɤɣ

tɕe nɯnɯ cɤmtsho ʁo yiji dongwu ɲɯ-ŋu tɕe, nɯnɯ
baohu kɯ-ra tɕe pjɯ́-wɣ-sat mɤ-kɯ-jɤɣ ŋu ri

\\
tɯ-ji nɯnɯ lú-wɣ-sɯɣjɤɣ, kɤ-ɕlu lú-wɣ-sthɯt, tɕe nɯ ɯ-qhu tɕe,
\subsection{Phasal}
ʑa, sɤʑa, sthɯt ʂɤtɕɯtʂi mda rɤŋgat rɲo

\subsubsection{ʑa}
 ɯdi tɯ-mnɤm ta-ʑa tɕe cha to-rɤru tutinɯ ŋu tɕe
tɕe pɤjkhu pjɯ-si ɕɯŋgɯ ʑo ɯ-ɕa ɯ-ndza tu-ʑa-nɯ ɕti.
kumpɣa phu nɯ tɯ-mbri ta-ʑa ɯ-qhu, 
sqamŋɤ-rʑaʁ jamar ri tɕe tɕe tɯtɯrca nɯ-kɯ-ʁaʁ kumpɣamu nɯ tɯ-rɤŋgɯm chɯ-ʑe ɲɯ-ŋu.

strategy:
tɕe βzɯr ri tɕe chɯ́-wɣ-mphɯr chɯ́-wɣ-ʑa tɕe 从正方形布料(mboʁ)的其中一角开始,一直到对着的另一角卷起来
mɤpɕoʁ cho βzɯr nɯ-ɕki mɤɕtʂa chɯ́-wɣ-mphɯr.

A	100	ɬamu	kɯ	kɯtɕu	ndi	zɯ	ko-ʑa	tɕe	kɯchu	ko-tɕɤt
ɯ-fso qidian mɤɕtʂa nɯ mɯ́j-ʑi
tɕe nɯtɕu tɕe kɯ-xtɕɯ-xtɕi tɯ-ʑi ɲɯ-ʑe ɲɯ-ŋu tɕe

ɯ-phoŋbu ɯ-stɤt chɯ-ʑe tɕe ɯ-smɤt mɤɕtʂa nɯ,
chɯ-kɯ-ɤɕɯɕrɤz ɣɤʑu.
\subsubsection{sɤʑa}
kɯrcɤpɤrme (basui) tɕe kɤrɤβzjos tɤsɤʑata.

tɕe kɤ-thu [to-s...] to-sɤʑá-nɯ ɲɯ-ŋu tɕe,
\subsubsection{sthɯt}
skɤβzaŋ tshɯraŋ rmi, gaozhong pasthɯt tɕe, athi ɯpɕi ri 
chengdu ri diannao ci ɕpaβzjoz

smi kɯ ɯrkɯ ɯsi nɯra kɤtɕɤβ pjɤsthɯt tɕɤn,
tɕendɤre thɯɣej ri tʂu kɤβzu ɲɤsthɯtnɯ.

A	25	kɤ-ti	tɤ-asthɯt	tɕe,	jo-nɯɕe.		
 
χpi kɤfɕɤt pasɯɣjɤɣ/pasthɯt
χpi pafɕɤt pasthɯt
\subsubsection{jɤɣ}
aβɣa pɯjɤɣ, akɤɣndʑɯr pɯjɤɣ: kɤɣndʑɯr ɯspa tharɕo

\subsubsection{ŋgrɯ}

tɕe pjɯ-tɯ-mnɤt mɤ-jɤɣ ɲɯ-ti tɕe, 
tɕe kɤ-mnɤt mɯ-pɯ-ŋgrɯ,
\subsubsection{rɤŋgat}
tɤ-tɕɯ nɯ kɯ-si to-rɤŋgat.

\subsection{Attempt}
tshɤt,  nɯftɕɤka
nɤz
phot

\subsubsection{tshɤt}
tshɤt, rɤtshɤt

tɕe ɲɯ-βʑoʁ-nɯ ɯkhɯkha qraʁ nɯnɯ tu-tshɤt-nɯ tɕe
ɯɲɯ-ɤ́thoʁmphrɤt kɯ nɯnɯ tutshɤtnɯ ntsɯ ra
他们一边削木头一边试一下木头适不适合(铧里面的洞)

pjɯ́-wɣ-ɣɤla tɕe tɯ-ɕɣa kɯ-mŋɤm phɤn tu-ti-nɯ,
ŋu tɕe ku-nɯ-phɤn ɕima,  (210)
aʑo kɤ-tshɤt mɯ-pɯ-rɲo-t-a ma,
\subsubsection{nɤz}
kɯwxti nɯra mɤkɯsi chɯnɯmɟe mɯ́jnɤz
tɕhemɤpɯ pɯɕtia qhe tɤtɕɯ ra nɯɕki ku-rɤʑi-a mɯ́j-naz-a qhe
kɤ-ʑɣɤɕɯfka mɯ-pɯ-naz-a

 
  \section{Secondary B} 
 ‘want’, ‘wish (for)’, ‘hope (for)’, ‘intend’, ‘plan (for)’, ‘pretend’.
sɯso, nɯʁjɯβtshɤt, ʁmɯɣ, nɯʁlɯmbɯɣ, nɯɕpɯz, ʑɣɤpa, nɯmga, ɯ-ʁjiz ɣi, ɯsɯm ɕe

yangyu cho tɯ-ci ʁɟa ʑo nɯ kɯ-fse tu-ndze.
ma nɯ ma tu-ndze ɯ-ʁjiz mɤ-ɣi ŋgrɤl.
kɤ-nɤma tɯ-ʁjɯz maka mɤ-ɣi.


qachɣa kɯ-mbri kɯ-fse, nɯra tu-nɯɕpɯz ɲɯ-spe. 

  \section{Secondary C} 
‘make’, ‘cause’, ‘force’, ‘let’, and ‘help’.
sɤβzu
>increase in valency


purposive:
kɯ-ɣɤrʁaʁ ra kɯ tu-ndo-nɯ.
nɯ-kɯ-qur tu-ndo-nɯ pjɤ-ŋgrɤl.

  
 


\bibliographystyle{linquiry2}
\bibliography{bibliogj}
\end{document}