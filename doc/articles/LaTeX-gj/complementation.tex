%aʑo me, nɤʑo me.

\documentclass[oldfontcommands,oneside,a4paper,11pt]{article} 
\usepackage{fontspec}
\usepackage{natbib}
\usepackage{booktabs}
\usepackage{xltxtra} 
\usepackage{longtable}
\usepackage{polyglossia} 
%\usepackage[table]{xcolor}
\usepackage{gb4e} 
\usepackage{multicol}
\usepackage{graphicx}
\usepackage{float}
\usepackage{lineno}
\usepackage{textcomp}
\usepackage{hyperref} 
\hypersetup{bookmarks=false,bookmarksnumbered,bookmarksopenlevel=5,bookmarksdepth=5,xetex,colorlinks=true,linkcolor=blue,citecolor=blue}
\usepackage[all]{hypcap}
\usepackage{memhfixc}
\usepackage{lscape}
 

\setmainfont[Mapping=tex-text,Numbers=OldStyle,Ligatures=Common]{Charis SIL} \newfontfamily\phon[Mapping=tex-text,Ligatures=Common,Scale=MatchLowercase,FakeSlant=0.3]{Charis SIL} 
\newcommand{\ipa}[1]{{\phon #1}} %API tjs en italique
 
\newcommand{\grise}[1]{\cellcolor{lightgray}\textbf{#1}}
\newfontfamily\cn[Mapping=tex-text,Ligatures=Common,Scale=MatchUppercase]{MingLiU}%pour le chinois
\newcommand{\zh}[1]{{\cn #1}}


\XeTeXlinebreaklocale "zh" %使用中文换行
\XeTeXlinebreakskip = 0pt plus 1pt %
 %CIRCG
 


\begin{document} 

\title{Complementation in Japhug Rgyalrong\footnote{
The glosses follow the Leipzig glossing rules. Other abbreviations used here are: \textsc{appl} applicative, \textsc{antipass} antipassive,\textsc{const} constative, \textsc{coord} coordinator, \textsc{dem} demonstrative, \textsc{dist} distal, \textsc{emph} emphatic, \textsc{indef} indefinite, \textsc{inv} inverse,  \textsc{pfv} perfective, \textsc{poss} possessor. %\textsc{trop} tropative, 
Acknowledgements will be added after editorial decision.  
} }
\author{Guillaume Jacques}
\maketitle
\linenumbers
 
 
 \section{Stative verbs}
 
 \begin{exe}
\ex \label{ex:pjWnArte}
\gll
 \ipa{koŋla} 	\ipa{pjɯ-nɤrte} 	\ipa{ʑo} 	\ipa{kɯ-rtaʁ} 	\ipa{ʑo} 	\ipa{kɯ-wxti} 	\ipa{ɲɯ-βze} 	\ipa{ŋgrɤl} 	\ipa{ma} \\
 really \textsc{ipfv}-wear.on.the.head \textsc{emph} \textsc{inf:stat}-enough \textsc{emph}  \textsc{inf:stat}-big \textsc{ipfv}-grow[III] \textsc{n.pst}:be.usually.the.case lnk \\
 \glt It grows big enough to serve as a hat. (Burdock, 35)

\end{exe}

ergatif et transitivité 

paʁ ra kɯ kɤ-ndza wuma ʑo rga-nɯ


\section{Purposive}
kɯki tɤ-pɤtso ki kɯ-rɤβzjoz tɤ-βgoz-i %revérifier)
\bibliographystyle{linquiry2}
\bibliography{bibliogj}
\end{document}