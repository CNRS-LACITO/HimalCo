\documentclass[oldfontcommands,oneside,a4paper,11pt]{article} 
\usepackage{fontspec}
\usepackage{natbib}
\usepackage{booktabs}
\usepackage{xltxtra} 
\usepackage{polyglossia} 
\usepackage[table]{xcolor}
\usepackage{gb4e} 
\usepackage{multicol}
\usepackage{graphicx}
\usepackage{float}
\usepackage{hyperref} 
\hypersetup{bookmarks=false,bookmarksnumbered,bookmarksopenlevel=5,bookmarksdepth=5,xetex,colorlinks=true,linkcolor=blue,citecolor=blue}
\usepackage[all]{hypcap}
\usepackage{memhfixc}
\usepackage{lscape}

 
\newfontfamily\phon[Mapping=tex-text,Ligatures=Common,Scale=MatchLowercase,FakeSlant=0.3]{Charis SIL} 
\newcommand{\ipa}[1]{{\phon \mbox{#1}}} %API tjs en italique
\newcommand{\ipab}[1]{{\scriptsize \phon#1}} 

\newcommand{\grise}[1]{\cellcolor{lightgray}\textbf{#1}}
\newfontfamily\cn[Mapping=tex-text,Ligatures=Common,Scale=MatchUppercase]{MingLiU}%pour le chinois
\newcommand{\zh}[1]{{\cn #1}}
\newcommand{\refb}[1]{(\ref{#1})}


\XeTeXlinebreaklocale 'zh' %使用中文换行
\XeTeXlinebreakskip = 0pt plus 1pt %
 %CIRCG
 


\begin{document} 
{\flushright{Paris, \makeatletter
\@date
\makeatother}
\flushleft{Guillaume JACQUES\\
Senior Researcher\\
CNRS-CRLAO\\
INALCO\\
2, rue de Lille\\
75007 Paris\\
France  \\
rgyalrongskad@gmail.com
}}

\section*{Recommendation letter for Gao Jiayin (Assistant  Professor  position  in  Chinese  Phonetics/Phonology)}


\begin{center}

 Dear Sir or Madam, 
\end{center}  

It is my unusual pleasure to recommend \textsc{Gao} Jiayin for the position of Assistant Professor in Chinese Phonetics and Phonology at your university.

\textsc{Gao} Jiayin is a brilliant researcher who combines expertise in several fields, and is enthusiastic about both teaching and research.

Her PhD research focused on the interaction of pitch, segments and phonetion types in Shanghaiese, a study of unprecedented depth on a topic not only of crucial interest for Acoustic Phonetics, but also for the modelling of tones in Theoretic Phonology and  for the study of Tonogenesis and of the evolution of tonal systems across the world's language. In particular, this study demonstrates that contrastive features (in this case voicing) may become secondary phonetic cues, a fact that is still poorly known even among specialists of historical phonology; I recently mentioned this PhD to a highly-cited phonologist who was not aware of examples of this type.

Several articles based on this PhD are under review or revision in peer-reviewed journals, including one for the prestigious \textit{Journal of Phonetics} (IF: 1.227), one of the four major journals in Acoustic Phonetics.

Her research also included a sociolinguistic perspective, and her data confirmed the well-known observation that within the same age group, males are generally more conservative than females; these research results could easily be turned into another article for a sociolinguistics journal. 

During her doctoral research, she also learned the field of Chinese Historical Phonology, which she used to some extent in her dissertation, the knowledge of which makes her able to efficiently undertake research on any Chinese dialect.

After her PhD she turned her attention to Tamang, a Sino-Tibetan language of Nepal, which like Shanghaiese presents a correlation between voice quality and tonal contrasts. She started learning Nepali and did a fieldwork trip in Nepal under difficult condition, during which she collected high quality acoustic and Electroglottograph (EGG) data. These data are exploitable for future publications in high-profile venues.

\textsc{Gao} Jiayin has a large array of research interests and technical expertise,  and I have no doubt that she can adapt to new research and teaching environments, join existing research projects and contribute to the scholarly excellence of your department.

I remain available for any additional query regarding her research profile.\\

Best regards,\\

Guillaume \textsc{Jacques}

\end{document}