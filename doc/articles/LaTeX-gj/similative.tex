\documentclass[oneside,a4paper,11pt]{article} 
\usepackage{fontspec}
\usepackage{natbib}
\usepackage{booktabs}
\usepackage{xltxtra} 
\usepackage{polyglossia} 
\usepackage[table]{xcolor}
\usepackage{gb4e} 
\usepackage{multicol}
\usepackage{graphicx}
\usepackage{float}
\usepackage{lineno}
\usepackage{hyperref} 
\hypersetup{bookmarks=false,bookmarksnumbered,bookmarksopenlevel=5,bookmarksdepth=5,xetex,colorlinks=true,linkcolor=blue,citecolor=blue}
%\usepackage[all]{hypcap}
\usepackage{memhfixc}
\usepackage{lscape}
 

%\setmainfont[Mapping=tex-text,Numbers=OldStyle,Ligatures=Common]{Times New Roman} 
\newfontfamily\phon[Mapping=tex-text,Ligatures=Common,Scale=MatchLowercase]{Charis SIL} 
\newcommand{\ipa}[1]{{\phon\textbf{#1}}} %API tjs en italique
 
\newcommand{\grise}[1]{\cellcolor{lightgray}\textbf{#1}}
\newfontfamily\cn[Mapping=tex-text,Ligatures=Common,Scale=MatchUppercase]{SimSun}%pour le chinois
\newcommand{\zh}[1]{{\cn#1}}
\newcommand{\topic}{\textsc{dem}}
\newcommand{\tete}{\textsuperscript{\textsc{head}}}
\newcommand{\rc}{\textsubscript{\textsc{rc}}}
\XeTeXlinebreaklocale 'zh' %使用中文换行
\XeTeXlinebreakskip = 0pt plus 1pt %
 %CIRCG
\newcommand{\ro}{$\Sigma$}
\newcommand{\siga}{$\Sigma_1$} 
\newcommand{\sigc}{$\Sigma_3$}   
\newcommand{\refb}[1]{(\ref{#1})}
\newcommand{\factual}[1]{\textsc{:fact}}
\newcommand{\forme}[2]{\ipa{#1} `#2'}  
\newcommand{\rdp}{\textasciitilde{}}
\begin{document} 

\title{Equative constructions in Japhug\footnote{Acknowledgments to be added after editorial decision. Glosses follow the Leipzig glossing rules. Other abbreviations used here include: \textsc{auto} spontaneous-autobenefactive, \textsc{cisloc} cislocative,  \textsc{fact} factual/assumptive, \textsc{genr} generic, \textsc{ifr} inferential evidential,  \textsc{inv} inverse, \textsc{lnk} linker,  \textsc{sens} sensory  evidential, \textsc{sfr} sentence final particle, \textsc{transloc} translocative, \textsc{trop} tropative.  Chinese borrowings in Japhug are indicated in pinyin between chevrons. The examples are taken from a corpus that is progressively being made available on the Pangloss archive (\citealt{michailovsky14pangloss}). This research was funded by the HimalCo project (ANR-12-CORP-0006) and is related to the research strand LR-4.11 ‘‘Automatic Paradigm Generation and Language Description’’ of the Labex EFL (funded by the ANR/CGI) }}
\author{Guillaume Jacques}
\maketitle
\linenumbers

\textbf{Abstract}: This paper documents equative, similative, comparative and superlative constructions on the basis of a corpus of narratives. It reveals a previously unsuspected wealth of constructions: no less than three main types of superlatives, and four types of equative are attested, some including additional subtypes.

Most of the constructions described in this paper belong to categories previously identified by typologists, but some constructions, such as the denominal equative adjectives, appear to be crosslinguistically rarer.

\textbf{Keywords}: Japhug, equative, similative, superlative, comparative, denominal derivation, relativization
\section*{Introduction}
This paper deals with equative and similative constructions in Japhug. It comprises five sections.

 First, I present general information on the Japhug language and its morphosyntax. Second, I provide an account of degree, comparative and superlative constructions, which have similarities, and even overlap with, equative constructions. Third, I describe the four types of equative constructions in Japhug. Fourth, I show some data on parameter equative constructions, which are not normally used in Japhug but have been collected using a novel method of elicitation.  Fifth, I discuss similative constructions and how they relate to the equative construction presented in section 3.

\section{Background information} \label{sec:background}
Japhug is a Gyalrong language (Trans-Himalayan, Gyalrongic; see \citealt{jackson00sidaba} and \citealt{jacques.michaud11naish} for more information on the classification of this language) spoken in Mbarkham county, Rngaba prefecture, Sichuan province (China), by less than 10000 speakers.\footnote{Previous work on this language includes a grammar (\citealt{jacques08}), a dictionary (\citealt{jacques15japhug}), and a series of articles on specific grammatical topics (see for instance \citealt{jacques13harmonization, jacques14linking, jacques16relatives}). } 


Unlike most Trans-Himalayan languages, Japhug and other Gyalrongic languages are polysynthetic, with a very rich and irregular morphology, and are highly head-marking (\citealt{jacques13harmonization, jackson14morpho}).

In this section, I discuss four topics of Japhug morphosyntax that are relevant to the description of the constructions studied in the paper: the definition of the word class `adjective' in Japhug, general information on grammatical relations, orientation prefixes and possessive prefixes.


\subsection{Adjectives}
In Japhug, adjectives are a sub-class of stative verbs. They are conjugated and take TAM and person indexes for one argument. They can be distinguished from other stative verbs, like copula, existential verbs and some modal auxiliaries by the fact that the tropative derivation can be applied to them (\citealt{jacques13tropative}).

In Japhug, it is possible to build a transitive verb meaning `to find X, to consider as X' out of any adjective by means of the \ipa{nɤ-} prefix, as in the following examples:

\begin{itemize}
\item \ipa{mpɕɤr} `be beautiful' $\rightarrow$ \ipa{nɤ-mpɕɤr} `find beautiful'
\item \ipa{ɕqraʁ} `be intelligent' $\rightarrow$ \ipa{nɤ-ɕqraʁ} `find intelligent'
\end{itemize}

This derivation cannot be applied to copulas or existential verbs.\footnote{The verb \ipa{maʁ} `not be' has a tropative form \ipa{nɤɣ-maʁ} `consider to be unjustified', which however derives from its secondary meaning `be incorrect'.} 

\subsection{Flagging and person indexation}

A conjugated verb form without overt arguments is the minimal complete sentence in Japhug, and grammatical relations are mainly expressed by person indexation, which includes up to two arguments following a direct/inverse system (on which see \citealt{jackson02rentongdengdi, jacques10inverse, gongxun14agreement}).

Overt noun phrases take case markers such as the ergative/instrument \ipa{kɯ}, the genitive \ipa{ɣɯ} and the dative \ipa{ɯ-ɕki}. There one no prepositions, but comitative adverbs are build by means of a prefix and are in the process of being grammaticalized as a quasi-case marker (\citealt{jacques16comitative}).

\subsection{Orientation prefixes} \label{sec:orientation}
All finite verb forms, except the factual non-past, require an orientation prefix (Table \ref{tab:orientation}). Motion verbs and concrete action verb are compatible with all prefixes, but most verbs can only take one or two orientation prefixes. For those verbs, the possible orientations are lexically specified; for instance \ipa{ndza} `eat' and \ipa{tsʰi} `drink' take the `up' and `towards east' orientations respectively; `eat' can also appear with the `downstream' orientation in the case of meat-eating animals.

\begin{table}[H]
\caption{Orientation prefixes in Japhug Rgyalrong} \label{tab:orientation}
\resizebox{\columnwidth}{!}{
\begin{tabular}{llllll}
\toprule
   &  	perfective  (A) &  	imperfective  (B)  &  	perfective 3$\rightarrow$3' (C)  &  	evidential  (D) \\  	
   \midrule
up   &  	\ipa{tɤ--}   &  	\ipa{tu--}   &  	\ipa{ta--}   &  	\ipa{to--}   \\  	
down   &  	\ipa{pɯ--}   &  	\ipa{pjɯ--}   &  	\ipa{pa--}   &  	\ipa{pjɤ--}   \\  	
upstream   &  	\ipa{lɤ--}   &  	\ipa{lu--}   &  	\ipa{la--}   &  	\ipa{lo--}   \\  	
downstream   &  	\ipa{tʰɯ--}   &  	\ipa{cʰɯ--}   &  	\ipa{tʰa--}   &  	\ipa{cʰɤ--}   \\  	
east   &  	\ipa{kɤ--}   &  	\ipa{ku--}   &  	\ipa{ka--}   &  	\ipa{ko--}   \\  	
west   &  	\ipa{nɯ--}   &  	\ipa{ɲɯ--}   &  	\ipa{na--}   &  	\ipa{ɲɤ--}   \\  	
no direction &\ipa{jɤ--}   &  	\ipa{ju--}   &  	\ipa{ja--}   &  	\ipa{jo--}   \\  	
\bottomrule
\end{tabular}}
\end{table}

Some particular constructions however can override the lexically selected orientation and impose a specific one; this is the case of one of the superlative constructions described in this paper (section \ref{sec:relative.superlative}).

\subsection{Possessive prefixes} \label{sec:possessive}
Nouns and nominalized verb forms can take a series of possessive prefixes related to the pronouns, as indicated in Table \ref{tab:pronoun}. 


\begin{table}[H] \centering
\caption{Pronouns and possessive prefixes }\label{tab:pronoun}
\begin{tabular}{lllllllll} 
\toprule
 Free pronoun & Prefix & Person\\
\midrule
 \ipa{aʑo},    \ipa{aj} &	\ipa{a--}  &		1\textsc{sg} \\
\ipa{nɤʑo},  \ipa{nɤj} &	\ipa{nɤ--}  &			2\textsc{sg}\\
\ipa{ɯʑo}  &	\ipa{ɯ--}  &			3\textsc{sg}\\
\midrule
\ipa{tɕiʑo}  &	\ipa{tɕi--}  &			1\textsc{du} \\
\ipa{ndʑiʑo}  &	\ipa{ndʑi--}  &		2\textsc{du} \\	
\ipa{ʑɤni}  &	\ipa{ndʑi--}  &		3\textsc{du} \\	
\midrule
\ipa{iʑo}, \ipa{iʑora},   \ipa{iʑɤra}   &	\ipa{i--}  &			1\textsc{pl} \\
\ipa{nɯʑo}, \ipa{nɯʑora},   \ipa{nɯʑɤra}  &	\ipa{nɯ--}  &			2\textsc{pl} \\
\ipa{ʑara}  &	\ipa{nɯ--}  &			3\textsc{pl} \\
\midrule
&  \ipa{tɯ--},  \ipa{tɤ--} & indefinite \\
\ipa{tɯʑo} & \ipa{tɯ--}   &  generic\\
\bottomrule
\end{tabular}
\end{table}

The degree nominals, which are used in many of the constructions described in this paper (sections \ref{sec:NDC}, \ref{sec:NDC.equative} and \ref{sec:pred.equative}) are derived from adjectives by prefixing the nominalizing prefix \ipa{tɯ-} (also used to make action nominal) and an obligatory possessive prefix coreferent with the subject.

\section{Related constructions}
Before presenting equative constructions, I provide a brief account of three types of related constructions: degree, comparative and superlative, some of which present commonalities with the constructions described in section \ref{sec:arg.equative}.



\subsection{Degree construction}
The degree of an adjectival predicate can be expressed in two ways, either with a degree adverb, or using the nominalized degree construction.

\subsubsection{Degree adverb} \label{sec:wuma}
The degree adverb construction is very familiar; it involves the adverb \forme{wuma}{really, very}, which can appear either close to the adjective (as in \ref{ex:tCur}) or separated from the verb by a noun phrase as in (\ref{ex:pWdAn}).

\begin{exe}
\ex \label{ex:tCur}
\gll \ipa{tɕe} 	\ipa{nɯnɯ} 	\ipa{wuma} 	\ipa{ʑo} 	\ipa{tɕur} 	\ipa{ri} \\
\textsc{lnk} \textsc{dem} really  \textsc{emph} be.sour:\textsc{fact} but \\
\glt `It is very sour.' (09 mi, 67)
\end{exe}

\begin{exe}
\ex \label{ex:pWdAn}
\gll
\ipa{nɤʑo} 	\ipa{nɯ} 	\ipa{wuma} 	\ipa{ʑo} 	\ipa{nɤ-ma} 	\ipa{pɯ-dɤn} 	\ipa{ɯ́-ŋu?}  \\
\textsc{2sg} \textsc{dem} really \textsc{emph} \textsc{2sg.poss}-work \textsc{pst.ipfv}-be.many \textsc{qu}-be \\
\glt `You had a lot of work, didn't you?' (conversation, 2015)
\end{exe}

Although \forme{wuma}{really, very} is borrowed from Tibetan \ipa{ŋo.ma}, this word is not used in this way in Tibetan languages as far as I know, and despite the deep typological and lexical influence of Tibetan on Japhug, the expression of degree in Tibetan uses unrelated constructions (\citealt{simon15evaluative}).

\subsubsection{Nominalized degree construction} \label{sec:NDC}
A more elaborated construction to express degree in Japhug involves nominalizing the adjectives by means of the nominalization prefix \ipa{tɯ-} and adding a possessive prefix coreferent with the subject (see section \ref{sec:possessive}), as in the form \forme{ɯ-tɯ-tɕur}{its (degree of) sourness} in example (\ref{ex:WtWCur}), followed by a predicate expressing the degree such as  \forme{saχaʁ}{be extremely} in this example. Other possible predicates include \forme{tɕʰom}{be too much} or \forme{naχtɕɯɣ}{be identical}; in the latter case it becomes an equative construction (see section \ref{sec:NDC.equative}).  As shown in (\ref{ex:YWsWxtCur}), the degree nominal (\ipa{ɯ-tɯ-tɕur} `its (degree of) sourness') can be followed by the ergative \ipa{kɯ} and a full clause describing the degree of the property described by the adjective (`so X that Y').

\begin{exe}
\ex \label{ex:WtWCur}
\gll 
\ipa{mtɕʰi}  	\ipa{ɯ-mat}  	\ipa{rca}  	\ipa{ɯ-tɯ-tɕur}  	\ipa{saχaʁ.}  	   \\
sea.buckthorn \textsc{3sg.poss}-fruit \textsc{unexpected} \textsc{3sg-nmlz:degree}-be.sour be.extremely:\textsc{fact}   \\
\glt `The fruit of the sea-buckthorn is very sour,' (`The degree of sourness of the fruit of the sea-buckthorn is extreme', 09 mi, 65)
\end{exe}

\begin{exe}
\ex \label{ex:YWsWxtCur}
\gll 
 	[\ipa{ɯ-tɯ-tɕur}]  	\ipa{\textbf{kɯ}}  	[\ipa{tɯ-kɯr}  	\ipa{ɯ-ŋgɯ}  	\ipa{lú-wɣ-rku}  	\ipa{qʰe}  	\ipa{maka}  	\ipa{ɲɯ-sɯ-ɤmɯzɣɯt}  	\ipa{qʰe,}  	\ipa{tɯ-pʰoŋbu}  	\ipa{ra}  	\ipa{kɯnɤ}  	\ipa{ɲɯ-sɯx-tɕur}  	\ipa{kɯ-fse}  	\ipa{ɕti}]  \\
  \textsc{3sg-nmlz:degree}-be.sour \textsc{erg} \textsc{indef:poss}-mouth \textsc{3sg}-inside \textsc{ipfv:upstream-inv}-put.in \textsc{lnk} at.all \textsc{ipfv-caus}-be.evenly.distributed \textsc{lnk} \textsc{indef:poss}-body \textsc{pl} also \textsc{ipfv-caus}-be.sour \textsc{nmlz:S/A}-be.like be:\textsc{affirm}:\textsc{fact} \\
\glt `(The fruit of the sea-buckthorn) is so sour that when one puts it in one's mouth, it makes it completely (sour), and it is as if one's (whole) body became sour.' (09 mi, 66)
\end{exe}

This construction is common in Japhug (\citealt[8]{jacques16comparative}) and attested in other Rgyalrong languages such as Tshobdun (\citealt[911]{jackson06guanxiju}).

\subsection{Comparative}
The comparative construction in Japhug can be illustrated by example (\ref{ex:comp1}): the  standard is marked by the postposition \ipa{sɤz} `than' specifically used in this construction, and the comparee is marked by the ergative  \ipa{kɯ}. In comparative constructions, it is common for ergative or instrumental markers to be used with the standard, but this use on the comparee is unexpected (\citealt{jacques16comparative}).

\begin{exe}
\ex \label{ex:comp1}
\gll  \ipa{ɯ-ʁi}   	\ipa{sɤz}   	[\ipa{ɯ-pi}   	\ipa{nɯ}]   	\ipa{\textbf{kɯ}}   	\ipa{mpɕɤr}     \\
\textsc{3sg.poss}-younger.sibling \textsc{comparative} \textsc{3sg.poss}-elder.sibling \textsc{dem} \textsc{erg}  be.beautiful:\textsc{fact} \\
\glt `The elder one is more beautiful than the young one.' (elicited)
\end{exe}

\subsection{Superlative}
There are no less than three constructions expressing superlative meaning in Japhug: a degree adverb meaning `most', a possessed subject participle (`Y is the X one of ...') and a relative clause with a negative existential verb (`There is no X one like Y').

 \subsubsection{Degree adverb}
 The degree adverb superlative is a familiar construction, illustrated by example (\ref{ex:stu1}) with an adjective in finite form (factual non-past).  
 
\begin{exe}
\ex \label{ex:stu1}
\gll 
\ipa{nɯ} 	\ipa{pɣɤtɕɯ} 	\ipa{nɯ-ŋgɯz} 	\ipa{stu} 	\ipa{xtɕi} 	\ipa{lo} \\
\textsc{dem} bird \textsc{3pl.poss-}among most be.small:\textsc{fact} \textsc{sfp} \\
\glt `It is the smallest of all birds.' (hist-24-ZmbrWpGa, 126)
\end{exe}

The form \ipa{-ŋgɯz} is the irregular fusion of the relator noun \ipa{-ŋgɯ} `inside' and the locative \ipa{zɯ} (it is thus a particular case of locative superlative construction, also found in many languages, eg Kambaata in this volume, \citealt{treis17comparison}). 

Most examples of this construction appear however with adjectives in subject participle form,\footnote{For an account of participial forms and a definition of subject and objects  in Rgyalrong languages, see \citet{jackson03caodeng, jackson14morpho, jacques16relatives}.} prefixed with \ipa{kɯ-} as in (\ref{ex:stu2}).

\begin{exe}
\ex \label{ex:stu2}
\gll \ipa{kɯɕɯŋgɯ} 	\ipa{tɕe} 	<aizheng> 	\ipa{kɤ-ti} 	\ipa{pɯ-me} 	\ipa{tɕe,} 	\ipa{kɤ-kɯ-nɤndza} 	\ipa{nɯ} 	\ipa{stu} 	\ipa{ʑo} 	\ipa{kɯ-ŋɤn} 	\ipa{kɤ-pa} 	\ipa{pɯ-ŋu} \\
long.ago \textsc{lnk} cancer \textsc{nmlz}:P-say \textsc{pst.ipfv}-not.exist \textsc{lnk} \textsc{pfv}-\textsc{nmlz}:S/A-have.leprosy \textsc{dem} most \textsc{emph} \textsc{nmlz}:S/A-be.evil \textsc{nmlz}:P-do \textsc{pst.ipfv}-be \\
\glt `In former times, nobody talked about cancer, and leprosy was considered to be the most terrible (of all diseases).' (hist-25-khArWm, 35)
\end{exe}

It is also possible to find this construction with oblique participles, as in (\ref{ex:stu3}), the only such example in the corpus.

\begin{exe}
\ex \label{ex:stu3}
\gll 
\ipa{stu} 	\ipa{ɯ-sɤ-dɤn} 	\ipa{nɯ} 	\ipa{stɤmku} 	\ipa{nɯra} 	\ipa{ŋu-nɯ} \\
most \textsc{3sg.poss-nmlz:oblique}-be.many \textsc{dem} prairie \textsc{dem:pl} be:\textsc{fact-pl} \\
\glt `The places where most of them are are the prairies.' (hist-19-qachGa mWntoR, 24)
\end{exe}

 \subsubsection{Possessed participle}
Another possibility to express superlative is with an adjective in participial form  with a third plural possessive marker, as in (\ref{ex:super2}). 
 
 \begin{exe}
\ex \label{ex:super2}
\gll 
\ipa{tɕe} 	\ipa{pɣa} 	\ipa{tʰamtɕɤt} 	\ipa{ɣɯ} 	\ipa{nɯ-kɯ-mpɕɤr} 	\ipa{nɯ} 	\ipa{rmɤβja} 	\ipa{ɲɯ-ŋu} \\
\textsc{lnk} bird all \textsc{gen} \textsc{3pl.poss-nmlz}:S/A-be.beautiful \textsc{dem} peacock \textsc{sens}-be \\
\glt  `The peacock is the most beautiful of all birds.' (24-ZmbrWpGa, 84)
\end{exe}

This construction is less common, and mainly occurs with the adjectives \forme{mpɕɤr}{be beautiful} and \forme{mna}{be well}.

 \subsubsection{Relative clause} \label{sec:relative.superlative}
 A more idiomatic way of expressing superlative meaning in Japhug is by means of negative existential verb combined with a relative clause (indicated between squared brackets in the following examples) and a adjunct with the participial form of \forme{fse}{be in this way, be like}, as in (\ref{ex:kWfse.me}). This construction is a particular use of the equative construction described in section \ref{sec:fse}.\footnote{No construction exactly identical to the Japhug one is found in Gorshenin's (\citeyear{gorshenin12superlative}) survey of superlatives, though it is close to the type described in his section 3.2.5.}
 
 \begin{exe}
\ex \label{ex:kWfse.me}
\gll \ipa{ama} 	\ipa{a-pi} 	\ipa{kʰu} 	\ipa{nɯ} 	\ipa{tɕʰindʐa} 	\ipa{ku-tɯ-nɤpʰɯpʰɣo} 	\ipa{tɕe}  \ipa{nɤʑo} 	\ipa{kɯ-fse} 	[\ipa{kɯ-sɤɣmu}] 	\ipa{me} 	\\
\textsc{surprise} \textsc{1sg.poss}-elder.sibling tiger \textsc{dem} why \textsc{prs.ego}-2-flee.here.and.there \textsc{lnk} \textsc{2sg}  \textsc{nmlz}:S/A-be.like   \textsc{nmlz}:S/A-be.dreadful  not.exist:\textsc{fact} \\
\glt `Brother tiger, why are you running away like that, you are the most dreadful (animal).' (literally `There is no one dreadful like you') (2005khu, 25)
\end{exe}

This construction is potentially ambiguous, and when the relative clause contains a finite main verb (when the relativized element is the object, the semi-object or the goal see \citealt{jacques16relatives}), it is possible in some cases to use orientation prefixes to disambiguate. In example (\ref{ex:tusWza}), the verbs \forme{tso}{understand} and \forme{sɯz}{know} in the superlative construction take the `up' prefix instead of the expected `towards east' (\ipa{ku-}) and `down' (\ipa{pjɯ-}) prefixes that they respectively select to build most tenses (see section \ref{sec:orientation}).

With the `up' prefix \ipa{tu-} as in (\ref{ex:tusWza}), the adjunct \ipa{nɯ} \ipa{kɯ-fse} `like that' is outside of the scope of the negation, and the negation applies exclusively to the minimal relative clauses \ipa{tu-tso-a} `(that) I understand' and \ipa{tu-sɯz-a} `(that) I know' (`[there is nothing that I understand/know] like that').\footnote{The \textsc{1sg} pronoun \ipa{aʑo} is also }

With the lexically selected `down' prefix \ipa{pjɯ-} on \forme{sɯz}{know} as in (\ref{ex:pjWsWza}), the scope of the negation is different: it applies to the whole constituent indicated between square brackets (`there is nothing like that that I know').
 
\begin{exe}
\ex \label{ex:tusWza}
\gll \ipa{aʑo} 	\ipa{nɯ} 	\ipa{kɯ-fse} 	\ipa{ʑo} 	\ipa{maka} [\ipa{tu-tso-a}] 	\ipa{me,} [\ipa{tu-sɯz-a}] 	\ipa{me} \\
\textsc{1sg} \textsc{dem} \textsc{nmlz}:S/A-be.like \textsc{emph} at.all \textsc{ipfv:up}-understand-\textsc{1sg} not.exist:\textsc{fact} \textsc{ipfv:up}-know-\textsc{1sg} not.exist:\textsc{fact} \\
\glt `This is what I know best.' (`There is nothing that I understand, that I know better than that.' 140519 yeying, 62)
\end{exe}

\begin{exe}
\ex \label{ex:pjWsWza}
\gll [\ipa{aʑo} 	\ipa{nɯ} 	\ipa{kɯ-fse} 	\ipa{pjɯ-sɯz-a}]	\ipa{me} \\
\textsc{1sg} \textsc{dem} \textsc{nmlz}:S/A-be.like \textsc{ipfv}-know-\textsc{1sg} not.exist:\textsc{fact} \\
\glt `I know of no such thing.'
\end{exe}

This contrast cannot however be generalized to all verbs; more research is necessary to ascertain the extent, and the functional explanation for this puzzling phenomenon.

\section{Entity equative} \label{sec:arg.equative}
This section discusses the entity equative construction, that constructions expressing that two entities have a property in equal degree (X is as Y as Z). It differs from parameter equative, treated in section \ref{sec:pred.equative}, expressing that the same entity has two properties in equal degree (X is a Y as he is Z). In the following, I adopt the terminology proposed by \citet{haspelmath08equative}, as illustrated by the English example (\ref{ex:equative.eng}). 

\begin{exe}
\ex \label{ex:equative.eng}
\gll  John is as intelligent as Paul \\
\textsc{comparee} { } \textsc{parameter.marker} \textsc{parameter} \textsc{standard.marker} \textsc{standard}  \\
\end{exe}

There are no less than four distinct constructions expressing argument equative meaning in Japhug. 

\subsection{\forme{fse}{be like}} \label{sec:fse}
One equative construction is built with the deixis verb \forme{fse}{be like (this)} (or more rarely the verb \forme{afsuja}{be of the same size}) as the \textsc{standard marker}.

Both  the \textsc{standard marker} \forme{fse}{be like} and the \textsc{parameter} can appear in finite form, sharing TAM and person marking as in (\ref{ex:fsWfse}). This is in fact a particular case of the serial verb construction used to express similative (see section \ref{sec:similative}). Such examples with finite verb forms are rare in the corpus.

\begin{exe}
\ex \label{ex:fsWfse}
\glll
\ipa{nɯ} 	\ipa{li} 	\ipa{ɯ-wa} 	\ipa{fsɯfse} 	\ipa{ʑo} 	\ipa{pjɤ-fse} 	\ipa{pjɤ-sɤjloʁ} \\
\textsc{dem} again \textsc{3sg.poss}-father completely.like \textsc{emph} \textsc{ifr.ipfv}-be.like \textsc{ifr.ipfv}-be.ugly \\
\textsc{comparee} { } \textsc{standard} \textsc{parameter.marker} { } \textsc{standard.marker} \textsc{parameter} \\
\glt `(The frog son) was as ugly as his father. ' (hist150818 muzhi guniang, 100)
\end{exe}

The equative in \forme{fse}{be like (this)} is more commonly used in attributive equative clauses. Both \forme{fse}{be like} and the adjective(the \textsc{parameter}) are in participial form in  example (\ref{ex:kWfse.kWchWcha}), forming a relative clause with the comparee as the relativized element. The superlative construction studied in section \ref{sec:relative.superlative} is essentially a particular use of such relativized equative sentences.

\begin{exe}
\ex \label{ex:kWfse.kWchWcha}
\glll \ipa{aʑo} 	\ipa{kɯ-fse} 	\ipa{kɯ-cʰɯ\rdp{}cʰa} 	\ipa{ʑo} 	\ipa{ʁʑɯnɯ} 	\ipa{ɣurʑa} 	\ipa{kɯrcat} 	\ipa{ra}  \\
\textsc{1sg} \textsc{nmlz}:S/A-be.like \textsc{nmlz}:S/A-\textsc{emph}\rdp{}can \textsc{emph} young.man hundred eight have.to:\textsc{fact} \\
\textsc{standard} \textsc{standard.marker} \textsc{parameter} { } \textsc{comparee} \\
\glt `I need a hundred and eight young men as able as I am.' (x1-sloXpWn, 17)
\end{exe}

These constructions do not require an \textsc{parameter marker}, though adverbs \forme{fsɯfse}{completely identical} can serve as redundant parameter marker, as in (\ref{ex:fsWfse}).

A similar equative construction in attested in the Kyomkyo dialect of Situ Rgyalrong (\citealt[238]{prins11kyomkyo}), though with the \textsc{mark} expressed by the participial form of the Tibetan loanword \forme{ndʐa}{be like} instead of the native root corresponding to Japhug \forme{fse}{be like}.

This equative construction corresponds to Haspelmath's (\citeyear{haspelmath17equative}) type 1 (Only equative standard-marker)

\subsection{Nominalized degree construction} \label{sec:NDC.equative}
This construction is a particular case of the Nominalized Degree Construction presented in section \ref{sec:NDC}. It is by far the most common way of expressing equative meaning in Japhug. It has three slightly different variants. 

In the first construction (corresponding Haspelmath's (\citeyear{haspelmath17equative}) type 5 -- Primary reach equative unified), the \textsc{comparee}and the \textsc{standard} are included in a noun phrase, with the comitative marker \ipa{cʰo} (and its longer variant \ipa{cʰondɤre}) serving as the \textsc{mark} after one of them. This noun phrase is followed by an adjective (the \textsc{parameter}) in degree nominal form (prefixed with \ipa{tɯ-} and a possessive prefix (in dual or plural) coreferent with the preceding noun phrase. This nominalized verb and the preceding noun phrase form a larger noun phrase that is the subject of the adjective \forme{naχtɕɯɣ}{be identical} (the \textsc{parameter marker})\footnote{The adjective \forme{naχtɕɯɣ}{be identical} is a denominal verb derived from the a non-attested form *\ipa{χtɕɯɣ} borrowed from the Tibetan numeral \forme{gtɕig}{one}. } in finite form, as in examples (\ref{ex:ndZitWwxti}) and (\ref{ex:ndZitWmbro}).

\begin{exe}
\ex \label{ex:ndZitWwxti}
\glll
\ipa{qalekɯtsʰi} 	\ipa{nɯnɯ} 	\ipa{cʰondɤre} 	\ipa{βʑar} 	\ipa{nɯ} 	\ipa{ndʑi-tɯ-wxti} 	\ipa{naχtɕɯɣ} \\
bird.sp \textsc{dem} \textsc{comit} bird.sp \textsc{dem} \textsc{3du.poss-nmlz:degree}-be.big be.identical:\textsc{fact} \\
{\textsc{standard}} { } \textsc{standard.marker} {\textsc{comparee}} { } \textsc{parameter} \textsc{parameter.marker} \\
\glt The \ipa{qalekɯtsʰi} bird is as big as the \ipa{βʑar} bird. (hist-23-RmWrcWftsa, 34)
\end{exe}

In example (\ref{ex:ndZitWwxti}) note that although the bird \ipa{qalekɯtsʰi}, since it is followed by the comitative postposition, is syntactically the \textsc{standard}, it is obvious from the context that the other bird (\ipa{βʑar}) is the semantic standard since this sentence is taken from a text describing the \ipa{qalekɯtsʰi} bird. This illustrates the fact that in equative constructions, since standard and comparee are identical as regards to a particular parameter, exchanging their order has not impact on the truth value of the sentence, unlike in the case of other comparative constructions.

\begin{exe}
\ex \label{ex:ndZitWmbro}
\gll 
\ipa{mɤ-mbro} 	\ipa{tɤru} 	\ipa{cʰo} 	\ipa{ndʑi-tɯ-mbro} 	\ipa{naχtɕɯɣ} \\
\textsc{neg}-be.high:\textsc{fact} tree.sp \textsc{comit} \textsc{3du.poss-nmlz:degree}-be.high be.identical:\textsc{fact} \\
\glt `It is not high, it (grows) as high as the \ipa{tɤru} tree.' (hist-17-xCAj, 56)
\end{exe}

In the second construction, the nominalized \textsc{parameter} takes a possessive prefix only coreferent with the \textsc{comparee}, and the \textsc{standard}  together with the comitative (the \textsc{standard marker}) follows the \textsc{parameter}, as in (\ref{ex:WtWwxti}).

\begin{exe}
\ex \label{ex:WtWwxti}
\glll
\ipa{qaliaʁ} 	\ipa{nɯ} 	\ipa{ɯ-tɯ-wxti} 	\ipa{nɯ} 	\ipa{qandʑɣi} 	\ipa{cʰo} 	\ipa{naχtɕɯɣ} 	\ipa{tsa} 	\\
eagle \textsc{dem} \textsc{3sg.poss-nmlz:degree}-be.big \textsc{dem} hawk \textsc{comit} be.identical:\textsc{fact} a.little  \\
{\textsc{comparee}} { } \textsc{parameter} { } {\textsc{standard}} \textsc{standard.marker} \textsc{parameter.marker}  \textsc{parameter.marker} \\
\glt `The eagle is about as big as the hawk.' (19-qandZGi, 36)
\end{exe}

In the third construction, the parameter takes a third person singular possessive prefix, and the two comparees are indicated by person indexation on the verb. In (\ref{ex:WtWmWCtaR}), the standard and the comparee are the speaker and the addressee; they are not expressed by overt pronouns, but are rather indexed on the verb by the suffix \ipa{-tɕi}.

\begin{exe}
\ex \label{ex:WtWmWCtaR}
\glll
\ipa{tɕe} 	\ipa{ɯ-tɯ-mɯɕtaʁ} 	\ipa{ɲɯ-naχtɕɯɣ-tɕi} 	\ipa{tɕe,} 	\ipa{qʰe} 	\ipa{nɯ-tɤjpa} 	\ipa{ɲɯ-rkɯn} 	\ipa{ma} \\
\textsc{lnk} \textsc{3sg.poss-nmlz:degree}-be.cold \textsc{sens}-be.identical-\textsc{1du} \textsc{lnk} \textsc{lnk} \textsc{2pl.poss}-snow \textsc{sens}-be.few \textsc{sfr} \\
{ } \textsc{parameter} \textsc{parameter.marker-comparee+standard} \\
\glt `It is as cold here as it is in your place, you don't have a lot of snow.' (`You and I are identical as to coldness'; conversation, 2014/11)
\end{exe}



\subsection{Possessed noun} \label{sec:Wfsu}
The possessed noun\footnote{In Japhug, possessed nouns obligatorily take a possessive prefix (see Table \ref{tab:pronoun}, section \ref{sec:possessive}), here the third singular \ipa{ɯ-}. } \forme{ɯ-fsu}{of the same size as} can be used as the \textsc{standard marker} in a construction of \citet{haspelmath17equative}'s type 1 like the one discussed in section \ref{sec:fse}.\footnote{Incidentally, note that \forme{fse}{be like}, the verb which serves as the \textsc{mark} in the construction described in section \ref{sec:fse}, is etymologically related to \forme{ɯ-fsu}{of the same size as}.}

\begin{exe}
\ex \label{ex:Wfsu}
\gll \ipa{tu-mbro} 	\ipa{tɕe,} 	\ipa{tɯrme} 	\ipa{ɯ-fsu} 	\ipa{jamar} 	\ipa{tu-βze} 	\ipa{cʰa} \\
\textsc{ipfv}-be.high \textsc{lnk} man \textsc{3sg.poss}-of.the.same.size.as about \textsc{ipfv}-grow can \\
\glt `It grows high, about as high as man.' (12-ndZiNgri, 4)
\end{exe}

This construction is relatively marginal in Japhug. In the Cogtse dialect of Situ Rgyalrong, a similar construction is reported (\citealt[377]{linxr93jiarong}).

\subsection{Denominal adjectives} \label{sec:denominal}
Japhug has a denominal prefix \ipa{arɯ-} deriving adjectives meaning `like X' out of nouns. I have no examples from narratives, but it spontaneously occurs in conversation, as for instance (\ref{ex:arWsWjno}), which I heard as I was correcting the transcription of a story with my main informant. The more elaborated sentence (\ref{ex:YArWsWjno}) was given as an explanation for (\ref{ex:arWsWjno}).

\begin{exe}
\ex \label{ex:arWsWjno}
\gll \ipa{ɯ-tɯ-ɤrɯsɯjno} 	\ipa{nɯ!}  \\
\textsc{3sg.poss-nmlz:degree}-be.like.grass \textsc{sfp} \\
\glt `(She cuts their head as easily) as if it were grass.' (heard in context)
\end{exe}

\begin{exe}
\ex \label{ex:arWmWntoR}
\gll \ipa{ɯ-tɯ-ɤrɯmɯntoʁ} 	\ipa{nɯ!}  \\
\textsc{3sg.poss-nmlz:degree}-be.like.a.flower \textsc{sfp} \\
\glt `It is as (worthless) as a flower!'
\end{exe}


\begin{exe}
\ex \label{ex:YArWsWjno}
\gll  
\ipa{kɤ-pʰɯt} 	\ipa{ɯ-tɯ-mbat} 	\ipa{kɯ} 	\ipa{ɲɯ-ɤrɯsɯjno} 	\ipa{ʑo} \\
\textsc{inf}-cut  \textsc{3sg.poss-nmlz:degree}-easy \textsc{erg} \textsc{sens}-be.like.grass \\
\glt `It is as easy to cut as if it were grass.' 
\end{exe}

This unusual equative construction is productive, since it can be applied to nouns from Tibetan (\ref{ex:arWmWntoR}) or Chinese. I does not fit in any of Haspelmath's (\citeyear{haspelmath17equative}) six types of equative construction, and appear to be crosslinguistically rare. It resembles the `similative adjective' derivation in \ipa{-lágan} in Saami (\citealt[5.1]{ylikovski17similarity}). There are three main differences between the Japhug construction and its Saami equivalent.

\begin{enumerate}
\item The denominal adjectives in Japhug are a sub-class of stative verbs, rather than being noun-like as in Saami.
\item The suffix \ipa{-lágan} in Saami, like the corresponding equative postposition \ipa{láhkai} are historically related to the noun \ipa{láhki} ‘mood, manner’, whereas the prefix \ipa{arɯ-} appears to be a combination of the passive \ipa{a-} with the denominal \ipa{rɯ-} prefix.
\item The Saami suffix is mainly used in attributive equative constructions, while the Japhug prefix occurs mainly in the degree nominal form illustrated by example (\ref{ex:arWsWjno}).
\end{enumerate}



\section{Parameter equative} \label{sec:pred.equative}
Parameter equative constructions (`X is as Y as he is Z') do not occur in the Japhug corpus. This meaning can however be expressed in this language. In order to limit the effect of elicitation, the following procedure was undertaken. I first wrote in Japhug a translation of Perrault's story `Riquet à la Houppe' which contains many examples of Predicate equative sentences. The translation was then corrected with my main informant sentence by sentence. Then, she was asked to retell the story (in six episodes of 3 to 5 minutes) using her own words.

Parameter equatives, for instance `X is as stupid as s/he is beautiful' (a sentence occurring several times in the story) can be expressed in Japhug in three distinct ways.

First, the possessed noun \forme{ɯ-fsu}{of the same size as} (used in the argument equative construction, section \ref{sec:Wfsu}), follows a degree nominal derived from the first adjective; the second adjective takes a finite form.

\begin{exe}
\ex 
\gll 
\ipa{ɯ-tɯ-mpɕɤr} \ipa{ɣɯ} 	\ipa{ɯ-fsu} 	\ipa{jamar} 	\ipa{ci} 	\ipa{ɲɯ-kʰe} 	\ipa{ɕti} \\
\textsc{3sg.poss-nmlz:degree}-be.beautiful \textsc{gen} \textsc{3sg.poss}-of.the.same.size.as about \textsc{indef} \textsc{sens}-be.stupid be.\textsc{affirm:fact} \\
\glt `S/he is stupid to the extent of his/her beauty.'
\end{exe}

Second, two adjectives in degree nominal form, the first followed by the comitative postposition \ipa{cʰo}, are subject of the verb \forme{afsuja}{be of the same size}. This construction is the equivalent of the argument equative construction in section \ref{sec:NDC.equative}.

\begin{exe}
\ex \label{ex:YAfsuja}
\gll 
\ipa{ɯ-tɯ-mpɕɤr} 	\ipa{cʰo} 	\ipa{ɯ-tɯ-kʰe} 	\ipa{nɯ} 	\ipa{ɲɯ-ɤfsuja} 	\ipa{ɕti} \\
\textsc{3sg.poss-nmlz:degree}-be.beautiful \textsc{comit} \textsc{3sg.poss-nmlz:degree}-be.stupid \textsc{dem} sens-be.of.the.same.size be.\textsc{affirm:fact} \\
\glt `His/Her beauty and his/her stupidity are equal.'
\end{exe}

Third, it is possible to express the same meaning with a correlative construction, as in \ref{ex:correlative}. This construction however may be a calque from Chinese, and is of lesser interest to the study of Japhug grammar.

\begin{exe}
\ex \label{ex:correlative}
\gll 
\ipa{tɕʰi} 	\ipa{jamar} 	\ipa{kɯ-mpɕɤr} 	\ipa{nɯ,} 	\ipa{nɯ} 	\ipa{jamar} 	\ipa{ci} 	\ipa{ɲɯ-kʰe} 	\ipa{ɕti} \\
what about \textsc{nmlz}:S/A-be.beautiful \textsc{dem} \textsc{dem} about \textsc{indef} \textsc{sens}-be.stupid be.\textsc{affirm:fact} \\
\glt `A much as s/he is beautiful, s/she is stupid.'
\end{exe}

\section{Similative} \label{sec:similative}
While equative constructions express equal degree as regards to a parameter, similative constructions express similarity in the manner in which an action is performed (\citealt{haspelmath08equative}).

In Japhug, the main similative construction involves the action deixis verbs \ipa{fse} `be like (this)' (intransitive stative) and \ipa{stu} `do (this) way, do like (this)' (transitive).

These verbs occur in a serial verb construction, having the same core arguments and TAM values as the main verb, as illustrated by examples (\ref{ex:kuWGstuanW}) (TAM: imperfective; Person:\textsc{3pl$\rightarrow$1sg}) and (\ref{ex:ki.fsea}) (TAM: factual non-past; Person:\textsc{1sg}). \ipa{stu} `do (this) way, do like (this)' is used respectively when the main verb is transitive, while \ipa{fse} `be like (this)' is used when the main verb is intransitive.

It is possible to insert a linker between the two verbs of the serial construction, as in example (\ref{ex:kuWGstuanW}). 

\begin{exe}
\ex \label{ex:kuWGstuanW}
\gll 	
 \ipa{aʑo} 	\ipa{kɯki} 	\ipa{ntsɯ} 	\ipa{kú-wɣ-stu-a-nɯ} 	\ipa{tɕe,} 	\ipa{kú-wɣ-znɯkʰrɯm-a-nɯ} \\
 \textsc{1sg} \textsc{dem:prox} always \textsc{ipfv-inv}-do.like-\textsc{1sg-pl} \textsc{lnk} \textsc{ipfv-inv}-punish-\textsc{1sg-pl} \\
 \glt `They punished me like this.' (Gesar, 278)
\end{exe}	


\begin{exe}
\ex \label{ex:ki.fsea}
\gll \ipa{aʑo} 	\ipa{nɯ} 	\ipa{sŋiɕɤr} 	\ipa{ʑo} 	\ipa{kutɕu} 	\ipa{ki} 	\ipa{fse-a} 	\ipa{ndzur-a} 	\ipa{ntsɯ} 	\ipa{ɲɯ-ra} 	\ipa{tɕe,} \\
\textsc{1sg} \textsc{dem} night.and.day \textsc{emph} here \textsc{dem:prox} be.like:\textsc{fact-1sg} stand:\textsc{fact-1sg} always \textsc{sens}-have.to like \\
\glt `I have to stand like this night and day.' (The divination, 2002, 44)
\end{exe}

The standard (the demonstrative pronouns \ipa{kɯki} in \ref{ex:kuWGstuanW}, and \ipa{ki} in \ref{ex:ki.fsea}) are syntactically treated as theme and semi-object respectively, and cannot be indexed on the verb.

The equative construction in \ipa{fse} `be like (this)' (section \ref{sec:fse}) is a particular case of this serial verb construction, when the main verb is an adjective.

\section*{Conclusion}
This paper documents various constructions in Japhug, some of which had never been described previously. Japhug presents a rich array of equative constructions, some of the garden variety type, but others, like the relative clause superlative (in particular the use of the `up' orientation prefix, see section \ref{sec:relative.superlative}) and the denominal equative (section \ref{sec:denominal}), appear quite unusual and isolated at least areally.

Despite the lexical influence of Tibetan languages on Japhug, and the fact that some of the constructions described in this paper involve Tibetan borrowings (see sections \ref{sec:wuma} and \ref{sec:NDC.equative}), none of them appear to be calqued from their Tibetan equivalents.

\bibliographystyle{unified}
\bibliography{bibliogj}
\end{document}