\documentclass[oneside,a4paper,11pt]{article} 
\usepackage{fontspec}
\usepackage{natbib}
\usepackage{booktabs}
\usepackage{xltxtra} 
\usepackage{polyglossia} 
\usepackage[table]{xcolor}
\usepackage{gb4e} 
\usepackage{multicol}
\usepackage{graphicx}
\usepackage{float}
\usepackage{lineno}
\usepackage{hyperref} 
\hypersetup{bookmarks=false,bookmarksnumbered,bookmarksopenlevel=5,bookmarksdepth=5,xetex,colorlinks=true,linkcolor=blue,citecolor=blue}
%\usepackage[all]{hypcap}
\usepackage{memhfixc}
\usepackage{lscape}
 

%\setmainfont[Mapping=tex-text,Numbers=OldStyle,Ligatures=Common]{Times New Roman} 
\newfontfamily\phon[Mapping=tex-text,Ligatures=Common,Scale=MatchLowercase]{Charis SIL} 
\newcommand{\ipa}[1]{{\phon\textbf{#1}}} %API tjs en italique
 
\newcommand{\grise}[1]{\cellcolor{lightgray}\textbf{#1}}
\newfontfamily\cn[Mapping=tex-text,Ligatures=Common,Scale=MatchUppercase]{SimSun}%pour le chinois
\newcommand{\zh}[1]{{\cn#1}}
\newcommand{\topic}{\textsc{dem}}
\newcommand{\tete}{\textsuperscript{\textsc{head}}}
\newcommand{\rc}{\textsubscript{\textsc{rc}}}
\XeTeXlinebreaklocale 'zh' %使用中文换行
\XeTeXlinebreakskip = 0pt plus 1pt %
 %CIRCG
\newcommand{\ro}{$\Sigma$}
\newcommand{\siga}{$\Sigma_1$} 
\newcommand{\sigc}{$\Sigma_3$}   
\newcommand{\refb}[1]{(\ref{#1})}
\newcommand{\factual}[1]{\textsc{:fact}}
\newcommand{\forme}[2]{\ipa{#1} `#2'}  
\newcommand{\rdp}{\textasciitilde{}}
\begin{document} 

\title{Similative constructions in Japhug}
\author{Guillaume Jacques}
\maketitle
\linenumbers

\section*{Introduction}

\section{Related constructions}
\subsection{Degree construction}
The degree of an adjectival predicate can be expressed in two ways, either with a degree adverb, or using the nominalized degree construction.

\subsubsection{Degree adverb}
The degree adverb construction is very familiar; it involves the adverb \forme{wuma}{really, very}, which can appear either close to the verb (as in \ref{ex:tCur}) or separated from the verb by a noun phrase as in (\ref{ex:pWdAn}).

\begin{exe}
\ex \label{ex:tCur}
\gll \ipa{tɕe} 	\ipa{nɯnɯ} 	\ipa{wuma} 	\ipa{ʑo} 	\ipa{tɕur} 	\ipa{ri} \\
\textsc{lnk} \textsc{dem} really  \textsc{emph} be.sour:\textsc{fact} but \\
\glt `It is very sour.' (09 mi, 67)
\end{exe}

\begin{exe}
\ex \label{ex:pWdAn}
\gll
\ipa{nɤʑo} 	\ipa{nɯ} 	\ipa{wuma} 	\ipa{ʑo} 	\ipa{nɤ-ma} 	\ipa{pɯ-dɤn} 	\ipa{ɯ́-ŋu?}  \\
\textsc{2sg} \textsc{dem} really \textsc{emph} \textsc{2sg.poss}-work \textsc{pst.ipfv}-be.many \textsc{qu}-be \\
\glt `You had a lot of work, didn't you?' (conversation, 2015)
\end{exe}

Although \forme{wuma}{really, very} is borrowed from Tibetan \ipa{ŋo.ma}, this word is not used in this way in Tibetan languages as far as I know, and despite the deep typological and lexical influence of Tibetan on Japhug, the expression of degree in Tibetan uses unrelated constructions (\citealt{simon15evaluative}).

\subsubsection{Nominalized degree construction} \label{sec:NDC}
A more elaborated construction to express degree in Japhug involves nominalizing the adjectives by means of the action nominal prefix \ipa{tɯ-} and adding a possessive prefix before it coreferent with the subject, as in the form \forme{ɯ-tɯ-tɕur}{its sourness} in example (\ref{ex:WtWCur}), followed by a predicate expressing the degree such as  \forme{saχaʁ}{be extremely} in this example. Other possible predicates include \forme{tɕʰom}{be too much} or \forme{naχtɕɯɣ}{be identical}; in the latter case it becomes a similative construction (see section \ref{sec:NDC.similative}).  As shown in (\ref{ex:YWsWxtCur}), the degree nominal can be followed by the ergative \ipa{kɯ} and a full clause describing the degree of the property described by the adjective (`so X that Y').

\begin{exe}
\ex \label{ex:WtWCur}
\gll 
\ipa{mtɕʰi}  	\ipa{ɯ-mat}  	\ipa{rca}  	\ipa{ɯ-tɯ-tɕur}  	\ipa{saχaʁ.}  	   \\
sea.buckthorn \textsc{3sg.poss}-fruit \textsc{unexpected} \textsc{3sg-nmlz:degree}-be.sour be.extremely:\textsc{fact}   \\
\glt `The fruit of the sea-buckthorn is very sour,' (09 mi, 65)
\end{exe}

\begin{exe}
\ex \label{ex:YWsWxtCur}
\gll 
 	[\ipa{ɯ-tɯ-tɕur}]  	\ipa{\textbf{kɯ}}  	[\ipa{tɯ-kɯr}  	\ipa{ɯ-ŋgɯ}  	\ipa{lú-wɣ-rku}  	\ipa{qʰe}  	\ipa{maka}  	\ipa{ɲɯ-sɯ-ɤmɯzɣɯt}  	\ipa{qʰe,}  	\ipa{tɯ-pʰoŋbu}  	\ipa{ra}  	\ipa{kɯnɤ}  	\ipa{ɲɯ-sɯx-tɕur}  	\ipa{kɯ-fse}  	\ipa{ɕti}]  \\
  \textsc{3sg-nmlz:degree}-be.sour \textsc{erg} \textsc{indef:poss}-mouth \textsc{3sg}-inside \textsc{ipfv:upstream-inv}-put.in \textsc{lnk} at.all \textsc{ipfv-caus}-be.evenly.distributed \textsc{lnk} \textsc{indef:poss}-body \textsc{pl} also \textsc{ipfv-caus}-be.sour \textsc{nmlz:S/A}-be.like be:\textsc{affirm}:\textsc{fact} \\
\glt `(The fruit of the sea-buckthorn) is so sour that when one puts it in one's mouth, it makes it completely (sour), and it is as if one's (whole) body became sour.' (09 mi, 66)
\end{exe}

This construction is common in Japhug (\citealt[8]{jacques16comparative}) and attested in other Rgyalrong languages such as Tshobdun (\citealt[911]{jackson06guanxiju}).

\subsection{Comparative}
The comparative construction in Japhug can be illustrated by example (\ref{ex:comp1}): the  standard is marked by the postposition \ipa{sɤz} `than' specifically used in this construction, and the comparee is marked by the ergative  \ipa{kɯ}. In comparative constructions, it is common for ergative or instrumental markers to be used with the standard, but this use on the comparee is unexpected (\citealt{jacques16comparative}).

\begin{exe}
\ex \label{ex:comp1}
\gll  \ipa{ɯ-ʁi}   	\ipa{sɤz}   	[\ipa{ɯ-pi}   	\ipa{nɯ}]   	\ipa{\textbf{kɯ}}   	\ipa{mpɕɤr}     \\
\textsc{3sg.poss}-younger.sibling \textsc{comparative} \textsc{3sg.poss}-elder.sibling \textsc{dem} \textsc{erg}  be.beautiful:\textsc{fact} \\
\glt `The elder one is more beautiful than the young one.' (elicited)
\end{exe}

\subsection{Superlative}
There are no less than three constructions expressing superlative meaning in Japhug: a degree adverb meaning `most', a possessed subject participle (`Y is the X one of ...') and a relative clause with a negative existential verb (`There is no X one like Y').

 \subsubsection{Degree adverb}
 The degree adverb superlative is a familiar construction, illustrated by example (\ref{ex:stu1}) with an adjective in finite small (factual non-past).
 
\begin{exe}
\ex \label{ex:stu1}
\gll 
\ipa{nɯ} 	\ipa{pɣɤtɕɯ} 	\ipa{nɯ-ŋgɯz} 	\ipa{stu} 	\ipa{xtɕi} 	\ipa{lo} \\
\textsc{dem} bird \textsc{3pl.poss-}among most be.small:\textsc{fact} \textsc{sfp} \\
\glt `It is the smallest of all birds.' (hist-24-ZmbrWpGa, 126)
\end{exe}

Most examples of this construction appear however with adjectives in subject participle form,\footnote{For an account of participial forms and a definition of subject and objects  in Rgyalrong languages, see \citet{jackson03caodeng, jackson14morpho, jacques16relatives}.} prefixed with \ipa{kɯ-} as in (\ref{ex:stu2}).

\begin{exe}
\ex \label{ex:stu2}
\gll \ipa{kɯɕɯŋgɯ} 	\ipa{tɕe} 	<aizheng> 	\ipa{kɤ-ti} 	\ipa{pɯ-me} 	\ipa{tɕe,} 	\ipa{kɤ-kɯ-nɤndza} 	\ipa{nɯ} 	\ipa{stu} 	\ipa{ʑo} 	\ipa{kɯ-ŋɤn} 	\ipa{kɤ-pa} 	\ipa{pɯ-ŋu} \\
long.ago \textsc{lnk} cancer \textsc{nmlz}:P-say \textsc{pst.ipfv}-not.exist \textsc{lnk} \textsc{pfv}-\textsc{nmlz}:S/A-have.leprosy \textsc{dem} most \textsc{emph} \textsc{nmlz}:S/A-be.evil \textsc{nmlz}:P-do \textsc{pst.ipfv}-be \\
\glt `In former times, nobody talked about cancer, and leprosy was considered to be the most terrible (of all diseases).' (hist-25-khArWm, 35)
\end{exe}

It is also possible to find this construction with oblique participles, as in (\ref{ex:stu3}), the only such example in the corpus.

\begin{exe}
\ex \label{ex:stu3}
\gll 
\ipa{stu} 	\ipa{ɯ-sɤ-dɤn} 	\ipa{nɯ} 	\ipa{stɤmku} 	\ipa{nɯra} 	\ipa{ŋu-nɯ} \\
most \textsc{3sg.poss-nmlz:oblique}-be.many \textsc{dem} prairie \textsc{dem:pl} be:\textsc{fact-pl} \\
\glt `The places where most of them are are the prairies.' (hist-19-qachGa mWntoR, 24)
\end{exe}

 \subsubsection{Possessed participle}
Another possibility to express superlative is with an adjective in participial form  with a third plural possessive marker, as in (\ref{ex:super2}).
 
 \begin{exe}
\ex \label{ex:super2}
\gll 
\ipa{tɕe} 	\ipa{pɣa} 	\ipa{tʰamtɕɤt} 	\ipa{ɣɯ} 	\ipa{nɯ-kɯ-mpɕɤr} 	\ipa{nɯ} 	\ipa{rmɤβja} 	\ipa{ɲɯ-ŋu} \\
\textsc{lnk} bird all \textsc{gen} \textsc{3pl.poss-nmlz}:S/A-be.beautiful \textsc{dem} peacock \textsc{sens}-be \\
\glt  `The peacock is the most beautiful of all birds.' (24-ZmbrWpGa, 84)
\end{exe}

This construction is less flexible, and mainly occurs with the adjectives \forme{mpɕɤr}{be beautiful} and \forme{mna}{be well}.

 \subsubsection{Relative} \label{sec:relative.superlative}
 A more idiomatic way of expressing superlative meaning in Japhug is by means of negative existential verb combined with a relative and a adjunct with the participial form of \forme{fse}{be in this way, be like}, as in (\ref{ex:kWfse.me}).
 
 \begin{exe}
\ex \label{ex:kWfse.me}
\gll \ipa{ama} 	\ipa{a-pi} 	\ipa{kʰu} 	\ipa{nɯ} 	\ipa{tɕʰindʐa} 	\ipa{ku-tɯ-nɤpʰɯpʰɣo} 	\ipa{tɕe}  \ipa{nɤʑo} 	\ipa{kɯ-fse} 	\ipa{kɯ-sɤɣmu} 	\ipa{me} 	\\
\textsc{surprise} \textsc{1sg.poss}-elder.sibling tiger \textsc{dem} why \textsc{prs.ego}-2-flee.here.and.there \textsc{lnk} \textsc{2sg}  \textsc{nmlz}:S/A-be.like   \textsc{nmlz}:S/A-be.dreadful  not.exist:\textsc{fact} \\
\glt `Brother tiger, why are you running away like that, you are the most dreadful (animal).' (literally `There is no one dreadful like you') (2005khu, 25)
\end{exe}

This construction is potentially ambiguous, and when the relative has a finite main verb (when the relativized element is the object, see \citealt{jacques16relatives}), it is possible in some cases, to use orientation prefixes to disambiguate. In example (\ref{ex:tusWza}), the verb \forme{sɯz}{know} in the superlative construction takes the `up' prefix instead of the expected `down' prefix that it normally selects to build most tenses. With the  `down' prefix, scope of the negation is different, a smenatic difference that may have to do with syntactic boundaries; in (\ref{ex:tusWza}), the relative is limited to a single word \ipa{tu-sɯz-a} `(that) I know' and the negation applies exclusively to it, whereas in (\ref{ex:pjWsWza}) the negation applies to the whole constituent indicated between square brackets.
 

\begin{exe}
\ex \label{ex:tusWza}
\gll \ipa{aʑo} 	\ipa{nɯ} 	\ipa{kɯ-fse} 	[\ipa{tu-sɯz-a}] 	\ipa{me} \\
\textsc{1sg} \textsc{dem} \textsc{nmlz}:S/A-be.like \textsc{ipfv:up}-know-\textsc{1sg} not.exist:\textsc{fact} \\
\glt `There is nothing I know better than that.'
\end{exe}

\begin{exe}
\ex \label{ex:pjWsWza}
\gll [\ipa{aʑo} 	\ipa{nɯ} 	\ipa{kɯ-fse} 	\ipa{pjɯ-sɯz-a}]	\ipa{me} \\
\textsc{1sg} \textsc{dem} \textsc{nmlz}:S/A-be.like \textsc{ipfv}-know-\textsc{1sg} not.exist:\textsc{fact} \\
\glt `I know of no such thing.'
\end{exe}

This contrast cannot however be generalized to all verbs and it is unclear for now to what extent this is a widespread phenomenon in Japhug grammar.

\section{Argument similative} \label{sec:arg.similative}
This section discusses the argument similative construction, that constructions expressing that two entities have a property in equal degree (X is as Y as Z). It differs from predicate similative, treated in section \ref{sec:pred.similative}, expressing that the same entity has two properties in equal degree (X is a Y as he is Z). In the following, I adopt the generally accepted terminology used in the description of comparative construction (\citealt{dixon08comparative}), as illustrated by the English example (\ref{ex:similative.eng}). Similative construction can be considered to be a special case of comparatives. The main difference with other comparative construction is that since the comparee and the standard are identical as regards to the parameter, they can be considered to be interchangeable, and will be referred to instead as \textsc{comparee} 1 and 2 in the following discussion.

\begin{exe}
\ex \label{ex:similative.eng}
\gll  John is as intelligent as Paul \\
\textsc{comparee 1} { } \textsc{index} \textsc{parameter} \textsc{mark} \textsc{standard/comparee 2}  \\
\end{exe}

There are no less than four distinct constructions expressing argument similative meaning in Japhug. 

\subsection{\forme{fse}{be like}}
One similative construction is built with the verb \forme{fse}{be like} as the \textsc{mark} on the comparee 2.

It is most commonly in participial form (\ipa{kɯ-fse}) as in  example (\ref{ex:kWfse.kWchWcha}). The main verb (the \textsc{parameter}) is also nearly always in participial form, and the construction is mainly used in relative clauses with one of the comparees as the relativized element.

\begin{exe}
\ex \label{ex:kWfse.kWchWcha}
\gll \ipa{aʑo} 	\ipa{kɯ-fse} 	\ipa{kɯ-cʰɯ\rdp{}cʰa} 	\ipa{ʑo} 	\ipa{ʁʑɯnɯ} 	\ipa{ɣurʑa} 	\ipa{kɯrcat} 	\ipa{ra}  \\
\textsc{1sg} \textsc{nmlz}:S/A-be.like \textsc{nmlz}:S/A-\textsc{emph}\rdp{}can \textsc{emph} young.man hundred eight have.to:\textsc{fact} \\
\glt `I need a hundred and eight young men as able as I am.' (x1-sloXpWn, 17)
\end{exe}

Alternatively, it is also possible to use a finite form for both the \textsc{mark} \forme{fse}{be like} and the \textsc{parameter} as in (\ref{ex:fsWfse}), though such examples are rare in the corpus.

\begin{exe}
\ex \label{ex:fsWfse}
\gll
\ipa{nɯ} 	\ipa{li} 	\ipa{ɯ-wa} 	\ipa{fsɯfse} 	\ipa{ʑo} 	\ipa{pjɤ-fse} 	\ipa{pjɤ-sɤjloʁ} \\
\textsc{dem} again \textsc{3sg.poss}-father completely.like \textsc{emph} \textsc{ifr.ipfv}-be.like \textsc{ifr.ipfv}-be.ugly \\
\glt `(The frog son) was as ugly as his father. ' (hist150818 muzhi guniang, 100)
\end{exe}

These constructions do not require an \textsc{index}, though adverbs \forme{fsɯfse}{completely identical} can serve as redundant indexes, as in (\ref{ex:fsWfse}).

A similar similative construction in attested in the Kyomkyo dialect of Situ Rgyalrong (\citealt[238]{prins11kyomkyo}), though with the \textsc{mark} expressed by the participial form of the Tibetan loanword \forme{ndʐa}{be like} instead of the native root corresponding to Japhug \forme{fse}{be like}.

\subsection{Nominalized degree construction} \label{sec:NDC.similative}
This construction is a particular case of the Nominalized Degree Construction presented in section \ref{sec:NDC}. It has three slightly different variants.

In the first construction, the two \textsc{comparees} are included in a noun phrase, with the comitative marker \ipa{cʰo} (and its longer variant \ipa{cʰondɤre}) serving as the \textsc{mark} after one of them. This noun phrase is followed by an adjective (the \textsc{parameter}) in degree nominal form (prefixed with \ipa{tɯ-} and a possessive prefix (in dual or plural) coreferent with the preceding noun phrase. This nominalized verb and the preceding noun phrase form a noun phrase that is the subject of the adjective \forme{naχtɕɯɣ}{be identical} (the \textsc{index})\footnote{The adjective \forme{naχtɕɯɣ}{be identical} is a denominal verb derived from the a non-attested form *\ipa{χtɕɯɣ} borrowed from the Tibetan numeral \forme{gtɕig}{one}. } in finite form, as in examples (\ref{ex:ndZitWwxti}) and (\ref{ex:ndZitWmbro}).

\begin{exe}
\ex \label{ex:ndZitWwxti}
\glll
\ipa{qalekɯtsʰi} 	\ipa{nɯnɯ} 	\ipa{cʰondɤre} 	\ipa{βʑar} 	\ipa{nɯ} 	\ipa{ndʑi-tɯ-wxti} 	\ipa{naχtɕɯɣ} \\
bird.sp \textsc{dem} \textsc{comit} bird.sp \textsc{dem} \textsc{3du.poss-nmlz:degree}-be.big be.identical:\textsc{fact} \\
{\textsc{comparee} 1} { } \textsc{mark} {\textsc{comparee} 2} { } \textsc{parameter} \textsc{index} \\
\glt The \ipa{qalekɯtsʰi} bird is as big as the \ipa{βʑar} bird. (hist-23-RmWrcWftsa, 34)
\end{exe}

\begin{exe}
\ex \label{ex:ndZitWmbro}
\gll 
\ipa{mɤ-mbro} 	\ipa{tɤru} 	\ipa{cʰo} 	\ipa{ndʑi-tɯ-mbro} 	\ipa{naχtɕɯɣ} \\
\textsc{neg}-be.high:\textsc{fact} tree.sp \textsc{comit} \textsc{3du.poss-nmlz:degree}-be.high be.identical:\textsc{fact} \\
\glt `It is not high, it (grows) as high as the \ipa{tɤru} tree.' (hist-17-xCAj, 56)
\end{exe}

In the second construction, the nominalized \textsc{parameter} takes a possessive prefix only coreferent with the \textsc{comparee} 1, and the \textsc{comparee} 2 together with the comitative (the \textsc{mark}) follows the parameter, as in (\ref{ex:WtWwxti}).

\begin{exe}
\ex \label{ex:WtWwxti}
\glll
\ipa{qaliaʁ} 	\ipa{nɯ} 	\ipa{ɯ-tɯ-wxti} 	\ipa{nɯ} 	\ipa{qandʑɣi} 	\ipa{cʰo} 	\ipa{naχtɕɯɣ} 	\ipa{tsa} 	\\
eagle \textsc{dem} \textsc{3sg.poss-nmlz:degree}-be.big \textsc{dem} hawk \textsc{comit} be.identical:\textsc{fact} a.little  \\
{\textsc{comparee} 1} { } \textsc{parameter} { } {\textsc{comparee} 2} \textsc{mark} \textsc{index}  \textsc{index} \\
\glt `The eagle is about as big as the hawk.' (19-qandZGi, 36)
\end{exe}

In the third construction, the parameter takes a third person singular possessive prefix, and the two comparees are indicated by person indexation on the verb. In (\ref{ex:WtWmWCtaR}), the speaker and the addressee are XXXX

\begin{exe}
\ex \label{ex:WtWmWCtaR}
\glll
\ipa{tɕe} 	\ipa{ɯ-tɯ-mɯɕtaʁ} 	\ipa{ɲɯ-naχtɕɯɣ-tɕi} 	\ipa{tɕe,} 	\ipa{qʰe} 	\ipa{nɯ-tɤjpa} 	\ipa{ɲɯ-rkɯn} 	\ipa{ma} \\
\textsc{lnk} \textsc{3sg.poss-nmlz:degree}-be.cold \textsc{sens}-be.identical-\textsc{1du} \textsc{lnk} \textsc{2pl.poss}-snow \textsc{sens}-be.few \textsc{sfr} \\
{ } \textsc{parameter} \textsc{index-comparee} \\
\glt `It is as cold here as it is in your place, you don't have a lot of snow.' (`You and I are identical as to coldness'; conversation, 2014/11)
\end{exe}

\subsection{Possessed noun}
\ipa{tu-mbro} 	\ipa{tɕe,} 	\ipa{tɯrme} 	\ipa{ɯ-fsu} 	\ipa{jamar} 	\ipa{tu-βze} 	\ipa{cha} 
12-ndZiNgri, 4


\citet[377]{linxr93jiarong}

\subsection{Denominal verb}

\subsubsection{As ...}
\begin{exe}
\ex 
\gll \ipa{ɯ-tɯ-ɤrɯmɯntoʁ} 	\ipa{nɯ!}  \\
\textsc{3sg.poss-nmlz:degree}-be.like.a.flower \textsc{sfp} \\
\glt `It is as (worthless) as a flower!'
\end{exe}

\begin{exe}
\ex 
\gll \ipa{ɯ-tɯ-ɤrɯsɯjno} 	\ipa{nɯ!}  \\
\textsc{3sg.poss-nmlz:degree}-be.like.grass \textsc{sfp} \\
\glt `(She cuts their head as easily) as if it were grass.'
\end{exe}

\ipa{kɤ-pʰɯt} 	\ipa{ɯ-tɯ-mbat} 	\ipa{kɯ} 	\ipa{ɲɯ-ɤrɯsɯjno} 	\ipa{ʑo} 

\subsubsection{Of the same size}
\ipa{afsujɯja} `be of same size'

\section{Predicate similative} \label{sec:pred.similative}
as X as he is Y


\section*{Conclusion}


\bibliographystyle{unified}
\bibliography{bibliogj}
\end{document}