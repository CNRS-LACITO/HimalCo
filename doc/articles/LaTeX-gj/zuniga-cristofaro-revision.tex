\documentclass[twoside,a4paper,11pt]{article} 
\usepackage{polyglossia}
\usepackage{natbib}
\usepackage{booktabs}
\usepackage{xltxtra} 
\usepackage{geometry}
\usepackage[usenames,dvipsnames,svgnames,table]{xcolor}
\usepackage{multirow,slashbox}
\usepackage{gb4e} 
\usepackage{multicol}
\usepackage{graphicx}
\usepackage{float}
\usepackage{varioref,hyperref} 
\hypersetup{colorlinks=true,linkcolor=blue,citecolor=blue}
\usepackage{memhfixc}
\usepackage{lscape}
\usepackage[footnotesize,bf]{caption}

 
\setmainfont[Mapping=tex-text,Numbers=OldStyle,Ligatures=Common]{Linux Libertine O} 
 
\newfontfamily\phon[Mapping=tex-text,Ligatures=Common,Scale=MatchLowercase]{Charis SIL} 
\newcommand{\ipa}[1]{{\phon\textit{#1}}} 
\newcommand{\ipab}[1]{{\phon #1}}
\newcommand{\ipapl}[1]{{\phondroit #1}}
\newcommand{\captionft}[1]{{\captionfont #1}} 
\newfontfamily\cn[Mapping=tex-text,Scale=MatchUppercase]{IPAGothic}%pour le chinois
\newcommand{\zh}[1]{{\cn #1}}
\newcommand{\tgf}[1]{\mo{#1}}
\newfontfamily\mleccha[Mapping=tex-text,Scale=MatchLowercase]{Galatia SIL}%pour le grec

\newcommand{\sg}{\textsc{sg}}
\newcommand{\pl}{\textsc{pl}}
\newcommand{\grise}[1]{\cellcolor{lightgray}\textbf{#1}} 
\newcommand{\Σ}{\greek{Σ}}
\newcommand{\ro}{$\Sigma$}
\newcommand{\ra}{$\Sigma_1$} 
\newcommand{\rc}{$\Sigma_3$}  
 
\newcommand{\abs}{\textsc{abs}}
\newcommand{\acc}{\textsc{acc}}
\newcommand{\adess}{\textsc{adess}}
\newcommand{\agent}{\textsc{a}}
\newcommand{\anim}{\textsc{anim}}
\newcommand{\antierg}{\textsc{antierg}}
\newcommand{\allat}{\textsc{all}}
\newcommand{\aor}{\textsc{aor}}
\newcommand{\assert}{\textsc{assert}}
\newcommand{\assoc}{\textsc{assoc}}
\newcommand{\auto}{\textsc{auto}}
\newcommand{\caus}{\textsc{caus}}
\newcommand{\cis}{\textsc{cis}}
\newcommand{\classif}{\textsc{class}}
\newcommand{\concessif}{\textsc{concsf}}
\newcommand{\comit}{\textsc{comit}}
\newcommand{\conj}{\textsc{conj}}
\newcommand{\cnj}{\textsc{cnj}}
\newcommand{\conv}{\textsc{conv}}
\newcommand{\cop}{\textsc{cop}}
\newcommand{\dat}{\textsc{dat}}
\newcommand{\due}{\textsc{de}}
\newcommand{\dem}{\textsc{dem}}
\newcommand{\detm}{\textsc{det}}
\newcommand{\dui}{\textsc{di}}
\newcommand{\dir}{\textsc{dir}}
\newcommand{\du}{\textsc{du}}
\newcommand{\duposs}{\textsc{du.poss}}
\newcommand{\dur}{\textsc{dur}}
\newcommand{\dyn}{\textsc{dyn}}
\newcommand{\emphat}{\textsc{emph}}
\newcommand{\erg}{\textsc{erg}}
\newcommand{\fut}{\textsc{fut}}
\newcommand{\gen}{\textsc{gen}}
\newcommand{\hypot}{\textsc{hyp}}
\newcommand{\ideo}{\textsc{ideo}}
\newcommand{\imp}{\textsc{imp}}
\newcommand{\inan}{\textsc{inan}}
\newcommand{\indep}{\textsc{indep}}
\newcommand{\infin}{\textsc{inf}}
\newcommand{\init}{\textsc{init}}
\newcommand{\ipf}{\textsc{ipfv}}
\newcommand{\instr}{\textsc{instr}}
\newcommand{\intens}{\textsc{intens}}
\newcommand{\intr}{\textsc{intr}}
\newcommand{\intrg}{\textsc{intrg}}
\newcommand{\inv}{\textsc{inv}}
\newcommand{\irreel}{\textsc{irr}}
\newcommand{\loc}{\textsc{loc}}
\newcommand{\masc}{\textsc{m}}
\newcommand{\med}{\textsc{med}}
\newcommand{\nanim}{\textsc{na}}
\newcommand{\ninan}{\textsc{ni}}
\newcommand{\negat}{\textsc{neg}}
\newcommand{\neu}{\textsc{neu}}
\newcommand{\nmlz}{\textsc{nmlz}}
\newcommand{\nom}{\textsc{nom}}
\newcommand{\nonps}{\textsc{n.pst}}
\newcommand{\obj}{\textsc{o}}
\newcommand{\obv}{\textsc{obv}}
\newcommand{\opt}{\textsc{dir2}}
\newcommand{\perf}{\textsc{pfv}}
\newcommand{\pli}{\textsc{pi}}
\newcommand{\pe}{\textsc{pe}}
\newcommand{\plposs}{\textsc{pl.poss}}
\newcommand{\poss}{\textsc{poss}}
\newcommand{\pot}{\textsc{pot}}
\newcommand{\pret}{\textsc{pret}}
\newcommand{\prohib}{\textsc{prohib}}
\newcommand{ \prox}{\textsc{prox}}
\newcommand{\prs}{\textsc{prs}}
\newcommand{\pst}{\textsc{pst}}
\newcommand{\purp}{\textsc{purp}}
\newcommand{\recip}{\textsc{recip}}
\newcommand{\redp}{\textsc{redp}}
\newcommand{\refl}{\textsc{refl}}
\newcommand{\sgposs}{\textsc{sg.poss}}
\newcommand{\stat}{\textsc{stat}}
\newcommand{\subj}{\textsc{s}}
\newcommand{\theme}{\textsc{th}}
\newcommand{\topic}{\textsc{top}}
\newcommand{\tr}{\textsc{tr}}
\newcommand{\unspec}{\textsc{unspec}}
\newcommand{\volit}{\textsc{vol}}


\let\eachwordone=\it

\begin{document}
\title{The direction(s) of analogical change in direct/inverse systems \footnote{We would like to thank Denis Creissels, Sonia Cristofaro, Scott DeLancey, Ives Goddard, Will Oxford, Fernando Zúñiga and two anonymous reviewers for valuable comments on earlier versions of this paper. We are responsible for any remaining errors. This research was funded by the HimalCo project (ANR-12-CORP-0006) and is related to the research strand LR-4.11 `Automatic paradigm generation and language description' of the Labex EFL (funded by the ANR/CGI).We follow the Leipzig glossing rules, to which the following are added: \textsc{cnj} conjunct, \init\ initial change, \textsc{inv} inverse, \nanim\ animate noun, \ninan\ inanimate noun, PA Proto-Alqonguian, VII Intransitive inanimate verb, VAI Intransitive animate verb, VTA Transitive animate verb, VTI Transitive inanimate verb.  } }

\author{Guillaume JACQUES, Anton ANTONOV\\ CNRS-INALCO-EHESS, CRLAO}
%\date{}
\maketitle

\textbf{Abstract}: In this paper, we extract general principles of language change from the study of the evolution of the conjunct order in various Algonquian languages, and propose four generalizations concerning the directionality of the spread of analogy in these systems. These generalizations are expected to bring insights on the analysis of data from other language families with direct/inverse marking but insufficient philological record, such as for instance Sino-Tibetan.

\textbf{Keywords}: Analogy, Direct/Inverse, Hierarchical Agreement, Algonquian, Arapaho, Cree, Ojibwe, conjunct order

 
\section{Introduction}
In families without recorded history the comparative method, combined with internal reconstruction, is the only way to reconstruct unattested stages. Still, when applying the comparative method it is important to understand the directionality of analogical levelling.  Indeed, morphological systems are affected not only by regular sound changes, but are also subject to analogical changes which make them more regular, either by undoing the effects of sound change or by removing opaque morphemes.

Algonquian is the only family with direct/inverse morphology whose verbal proto-system can be reconstructed without sparking controversy. This is due to the combination of three factors. First, the sound laws of Algonquian languages are perfectly understood (except for Blackfoot). Second, some languages, in particular Fox and Miami-Illinois, are very conservative, and preserve the proto-system in an almost pristine way. Third, records dating back to the seventeenth century for some languages provide information on the intermediate stages between the proto-language and the modern forms.
 
For other families with direct/inverse systems, no such diachronic information is available, due to the absence of ancient attestations and/or the fact that many of these languages are either isolates or else belong to very small language families. Hence, it is easier at the present stage to observe the attested history of Algonquian languages and deduce from it a series of principles, which can then be tentatively applied to languages with direct/inverse systems for which such detailed information is not available.

In this paper, we will limit ourselves to formulating  four generalizations concerning the directionality of analogical change in direct/inverse systems based on data from Algonquian by way of several case studies on Cree, Ojibwe, Mi'gmaq and Arapaho.

\section{Some terminological preliminaries}

Algonquian languages present multiple challenges to the unprepared some of which (especially those pertaining to the verbal domain which is the main topic of this paper) we will try to explain in this short introduction (partly based on the more detailed discussion in \citealp{jacques14inverse}).

\subsection{Verb classes and animacy}

Algonquian verbs are traditionally classified into four big classes, according to the animacy of the S/P argument. There is thus a major distinction between animate (\nanim) and inanimate (\ninan) nouns. It is important to note that the criteria used to ascribe animate or inanimate gender to a given referent do not always coincide with those familiar from European languages: `sock(s)' and `rock(s)', for instance, are animate in Cree.

The four classes are the following: VII (intransitive verbs with an inanimate actor), VAI (intransitive verbs with an animate actor), VTI (transitive verbs with an inanimate patient) and VTA (transitive verbs with an animate patient). The last two classes also have an animate actor. In fact, there are also several subclasses of `deponent' VAI and VTI verbs whose syntactic behaviour does not match their morphological makeup (cf. Table \ref{tab:verbclass}). These are usually either not specifically signalled or else termed VAI-T and VTI-I. Here we will call them VAI$_\tr$ and VTI$_\intr$, respectively.

\begin{table}[htbp]
\caption{The four verb classes in Algonquian exemplified by Plains Cree} \label{tab:verbclass} \centering
\begin{tabular}{llll}
\toprule
Verb class & S, A, P [±\anim] & Cree & meaning\\
\midrule
VII & S=\inan &\ipa{wâpiskâ--} & `be white'\\
& & \ipa{miywâsi--} & `be good' \\
& &\ipa{wâpa--} & `be dawn'\\
VAI & S=\anim & \ipa{wâpiskisi--} & `be white'\\
& &\ipa{miywâsisi--} & `be good'\\
& & \ipa{pimipahtâ--} & `run'\\
VAI $_\tr$ & A=\anim$+$P=±\anim & \ipa{mêki-} & `give (out) s.o. or sth'\\
& A=\anim$+$P=\inan & \ipa{âpacihtâ-} &`use sth'\\
VTI & A=\anim$+$ P=\inan & \ipa{wâpaht--} & `see sth'\\
VTI$_\intr$& S=\anim & \ipa{mâham} & `canoe downriver'\\
VTA & A=\anim$+$ P=\anim & \ipa{wâpam--}& `see s.o.'\\
\bottomrule
\end{tabular}
\end{table}

\subsection{Direct/inverse and obviation}

It is important to observe that in spite of the existence of syntactically transitive deponent verbs, the only verbs that index both of their participants as long as they are not third person are the VTA (transitive animate) ones. The resulting complex forms reference their participants using S, A, P-neutral affixes. This, in turn, calls for the use of a special `direction' marker (traditionally called a `theme sign') in order to indicate the `direct' vs `inverse' direction of the action. The use of one or the other reflects the position of the agent on the following hierarchy (valid for Plains Cree):
\begin{exe}
\ex \label{ex:empathy.cree}
\glt SAP > animate proximate > animate obviative > inanimate
\end{exe}

If it is higher than the patient the verb shows direct marking, but if it is lower then the verb receives inverse marking.\footnote{It is generally considered that the second person outranks the first person (2 > 1) in Algonquian languages, but this refers to a distinct hierarchy related to the slot accessibility of person prefixes, not the distribution of direct and inverse forms. Concerning obviative inanimates, see a recent study by \citet{muehlbauer12obviation}.} Thus, we observe a tripartite distinction between proximate animate, obviative animate and inanimate referents.

Obviation is an ubiquitous feature in Algonquian which is reflected both in verbal and nominal morphology. Its basic function is to distinguish two or more third-person participants within a given sentence or stretch of discourse. Thus, in oral narratives, the \textit{obviative} (\obv, -\textit{(w)a} in Cree) is used to introduce a hitherto unknown participant by contrast with the unmarked form which is called the \textit{proximate} (\prox). There can be  at most one proximate participant within a given clause. Later on, the interplay between the two helps the listener to keep track of who does what to whom. Except if s/he is a persistent topic, no participant is inherently tied to a proximate or obviative status solely by virtue of their inherent semantic features. The obviative must also be used on the possessee, and on the verb whose argument the possessee is, whenever the possessor is third-person (cf. ex. \ref{ex:proxobvcree2} below and ex. \ref{ex:creeverbs} in section \ref{subsec:creeparadigm}).

\begin{exe}
\ex \label{ex:proxobvcree2}
\gll \ipa{pêyak}  \ipa{piko}	\ipa{nipah-êyiwa}  \ipa{o-mis-a} \ipa{wâposw-a}  \\
one just kill-3'$\rightarrow$3'  3\textsc{poss}-older\_sister-\textsc{obv} rabbit-\textsc{obv}\\
\glt  `His sister had killed but one rabbit.'  (\citealp[p. 401]{wolfart96sketch})
\end{exe}

Example \ref{ex:proxobvcree2} also illustrates the so-called \textit{further obviative} form, which is often abbreviated as 3'$\rightarrow$3'' (cf. section \ref{subsec:visual}), with the verb \ipa{nipah--} `to kill'. 

\subsection{Independent vs. Conjunct order}
The inflectional paradigms of the Algonquian verb classes have further been organized in five sets (called `orders') in Proto-Algonquian, of which most modern languages preserve only three, ie. the Independent, the Conjunct and the Imperative, having discarded the other two, ie. the Interrogative and the Prohibitive. While the imperative order is self-explanatory (and won't be dealt with in this paper), the \textbf{independent} (which will be discussed only in passim) and the \textbf{conjunct} roughly correspond to verb forms used in main and subordinate clauses, respectively (for the actual forms cf. Tables \ref{tab:paradigmcreeindep} and \ref{tab:paradigmcreeconj}). Put differently, conjunct order forms are non-finite, whereas independent order ones are finite. It is important to stress that \textit{wh-} clauses, those with focalized constituents or under the scope of (clausal) negation require the use of the conjunct order, since these `de-subordinated' clauses are underlyingly (or rather, historically) non-finite.

\subsection{Visualizing complex participant configurations}
\label{subsec:visual}
It is customary to represent systems indexing more than one argument (usually two) such as those found in Algonquian languages in tabular format as in Table \ref{tab:domain}, where rows indicate agent and columns patient. The different transitive configurations are symbolically represented by using an arrow, with the agent on its left and the patient on its right, both abbreviated as 1, 2, 3 for first, second and third person respectively. In the case of third person arguments 3 indicates \textbf{proximate} and 3' \textbf{obviative} referents. In intransitive forms, by contrast, the abbreviation refers to the sole argument of the verb. They are systematically included for reference.

The cells corresponding to the 1$\rightarrow$1 and 2$\rightarrow$2 configurations are semantically reflexive and are thus filled in grey, since in most languages they tend to be expressed by an intransitive construction\footnote{The same applies, in languages with clusivity (a distinction between first person inclusive [1\pli] vs exclusive [1\pe]), such as the Algonquian languages, to the combination of first inclusive and second person.}. The 3$\rightarrow$3 cell, on the other hand, is not since the corresponding configuration is not necessarily reflexive.

\begin{table}[H] 
\caption{The three domains of the transitive paradigm} 
 \centering \label{tab:domain}
\begin{tabular}{l|lllll} 
\toprule
&1 & 2 &3\\
\hline
1 &\grise{} &1$\rightarrow$2\cellcolor[wave]{465} & 1$\rightarrow$3 \cellcolor[wave]{520} \\
2&2$\rightarrow$1\cellcolor[wave]{465}&\grise{}&2$\rightarrow$3 \cellcolor[wave]{520} \\
3&3$\rightarrow$1 \cellcolor[wave]{520}&3$\rightarrow$2 \cellcolor[wave]{520}&3$\rightarrow$3\cellcolor[wave]{650}\\
\hline
\textsc{intr}&1&2&3\\
\bottomrule
\end{tabular}
\end{table}
 
It is convenient to separate the transitive paradigm into three \textsc{domains} (\citealt[47-54]{zuniga06}), represented in Table  \ref{tab:domain} by different colours. First, the \textsc{local} domain (in blue) comprises the forms 1$\rightarrow$2 and 2$\rightarrow$1, where both arguments are SAPs. Second, the \textsc{non-local} domain (in red) refers to the cases where both arguments are third person. Third, the \textsc{mixed} domain (in green) includes all the forms with a SAP argument and a third person (1$\rightarrow$3, 2$\rightarrow$3, 3$\rightarrow$1, 3$\rightarrow$2). 


\subsection{Plains Cree paradigms}
\label{subsec:creeparadigm}
We can now give the full paradigms for the main four classes using some of the verbs from Table \ref{tab:verbclass}. Table \ref{tab:paradigmcreeindep} presents the independent order while Table \ref{tab:paradigmcreeconj} shows the conjunct order, whose diachronic evolution will be at the centre of subsequent discussion.

%\begin{landscape}

\begin{table}[H]
\caption{Plains Cree Independent Order paradigms of VTA \ipa{wâpam--} ``see s.o.", VTI \ipa{wâpaht--} ``see sth", VAI \ipa{wâpiskisi--} ``be white (\textsc{+anim})", \ipa{pimipahtâ--}``run", VII \ipa{wâpiskâ--} ``be white (\textsc{-anim})", \ipa{miywâsin} ``be good", \ipa{wâpan} ``be dawn" (based on \citealp{wolfart96sketch})}
\label{tab:paradigmcreeindep}
%\centering
\resizebox{16cm}{!}{
%\begin{scriptsize}

\begin{tabular}{ccccccccccc}
\toprule
 \multirow{12}{*}{\textbf{VTA}} &\backslashbox{A}{P}  & 	1\sg  & 1\textsc{pi} & 1\textsc{pe} &  2\sg & 2\pl  &  3\sg & 3\pl &	3'\sg & 3'\pl \\ 
\midrule
& 1\sg   & 	\grise{}   & 	\grise{} &  \grise{} &	\ipa{ki-wâpam-iti-n}  & \ipa{ki-wâpam-iti-nâw-âw}	& \cellcolor{Dandelion}\ipa{ni-wâpam-â-w}   & 	\cellcolor{Dandelion}\ipa{ni-wâpam-â-w-ak}  & \multicolumn{2}{c}{\cellcolor{Dandelion}	\ipa{ni-wâpam-im-â-w-a}}   \\ 
& 1\textsc{pi} & \grise{}   &\grise{} & \grise{} & \multicolumn{2}{c}{\grise{}}  & \cellcolor{Dandelion}\ipa{ki-wâpam-â-naw} & \cellcolor{Dandelion}\ipa{ki-wâpam-â-na-w-ak}  & \multicolumn{2}{c}{\cellcolor{Dandelion}	\ipa{ki-wâpam-im-â-na-w-a} }  \\ 
& 1\textsc{pe} & \grise{}   &\grise{} & \grise{} & \multicolumn{2}{c}{\ipa{kiwâpamitinân}}   & \cellcolor{Dandelion}\ipa{ni-wâpam-â-nân} & \cellcolor{Dandelion}\ipa{ni-wâpam-â-nân-ak}  & \multicolumn{2}{c}{\cellcolor{Dandelion}	\ipa{ni-wâpam-im-â-nân-a}}   \\ 
& 2\sg   & 	\ipa{ki-wâpam-in}   & \grise{}& \multirow{2}{*}{\ipa{ki-wâpam-inân}}	&	\grise{}   &  \grise{} & \cellcolor{Dandelion}\ipa{ki-wâpam-âw}  & \cellcolor{Dandelion}\ipa{ki-wâpam-â-wak} &\multicolumn{2}{c}{\cellcolor{Dandelion} 	\ipa{ki-wâpam-im-â-wa}}\\ 
& 2\pl  & 	\ipa{ki-wâpam-in-âwâw} & \grise{}& \multirow{-2}{*}{ } & \grise{}  & 	\grise{}   & 	\cellcolor{Dandelion}\ipa{ki-wâpam-â-wâw}  & \cellcolor{Dandelion}\ipa{ki-wâpam-â-wâw-ak} &\multicolumn{2}{c}{\cellcolor{Dandelion} \ipa{ki-wâpam-im-â-wâw-a}} \\
& 3\sg   & 	\cellcolor{green}\ipa{ni-wâpam-ik}   & \cellcolor{green}\ipa{ki-wâpam-iko-n-aw} & \cellcolor{green}\ipa{ni-wâpam-iko-nân} & \cellcolor{green}	\ipa{ki-wâpam-ik}  & \cellcolor{green}	\ipa{ki-wâpam-iko-wâw} & \cellcolor{Dandelion}	\grise{}  & \grise{}	 & \multicolumn{2}{c}{\cellcolor{Dandelion}	\ipa{wâpam-(im)-ê-w}}\\ 
& 3\pl   & 	\cellcolor{green}\ipa{ni-wâpam-ikw-ak}&  \cellcolor{green}\ipa{ki-wâpam-iko-n-aw-ak} & \cellcolor{green}\ipa{ni-wâpam-iko-nân-ak}   & \cellcolor{green}	\ipa{ki-wâpam-ikw-ak}   & \cellcolor{green}	\ipa{ki-wâpam-iko-wâw-ak} & \cellcolor{Dandelion}	\grise{} &	\grise{}  & \multicolumn{2}{c}{\cellcolor{Dandelion}	\ipa{wâpam-(im)-ê-wak} }\\ 
& \multirow{2}{*}{3'}   & \multirow{2}{*}{\cellcolor{green}}  &  \multirow{2}{*}{\cellcolor{green}}  & \multirow{2}{*}{\cellcolor{green}} &\cellcolor{green} &  \multirow{2}{*}{\cellcolor{green}}  &\multirow{2}{*}{\cellcolor{green}}   & \multirow{2}{*}{\cellcolor{green}} & \multicolumn{2}{c}{\cellcolor{Dandelion} \ipa{wâpam-ê-yi-wa}} \\ 
 & \multirow{-2}{*}{} & \multirow{-2}{*}{\cellcolor{green}\ipa{ni-wâpam-iko-yi-wa}} & \multirow{-2}{*}{\cellcolor{green}\ipa{ki-wâpam-iko-nawa}}   &  \multirow{-2}{*}{\cellcolor{green}\ipa{ni-wâpam-iko-nâna}} &  \multirow{-2}{*}{\cellcolor{green}\ipa{ki-wâpam-iko-yi-wa}} &  \multirow{-2}{*}{\cellcolor{green}\ipa{ki-wâpam-iko-wâwa}}& \multirow{-2}{*}{\cellcolor{green}\ipa{wâpamik}}  & \multirow{-2}{*}{\cellcolor{green}\ipa{wâpam-ikw-ak}} & \multicolumn{2}{c}{\cellcolor{green} \ipa{wâpam-iko-yi-wa} }\\ 
\bottomrule
 \multirow{12}{*}{\textbf{VTI}} &\backslashbox{A}{P}  & 	1\sg  & 1\textsc{pi} & 1\textsc{pe} &  2\sg & 2\pl  &  3\sg & 3\pl &	3'\sg & 3'\pl \\ 
\midrule
& 1\sg   & 	\grise{}   & 	\grise{} &  \grise{} &	\grise{}  & \grise{} 	& \multicolumn{4}{c}{\ipa{n-iwâpaht-ê-n} } \\ 
& 1\textsc{pi} & \grise{}   &\grise{} & \grise{} & \multicolumn{2}{c}{\grise{}}  & \multicolumn{4}{c}{\ipa{ki-wâpaht-ê-(n-â)n-aw} } \\ 
& 1\textsc{pe} & \grise{}   &\grise{} & \grise{} & \multicolumn{2}{c}{\grise{}}   & \multicolumn{4}{c}{\ipa{ki-wâpaht-ê-n-ân} }\\ 
& 2\sg   & 	\grise{}  & \grise{}& \multirow{2}{*}{\grise{}}	&	\grise{}   &  \grise{} & \multicolumn{4}{c}{\ipa{ki-wâpaht-ê-n} }\\ 
& 2\pl  & 	\grise{}  & \grise{} & \multirow{-2}{*}{\grise{} } & \grise{}  & 	\grise{}   & 	\multicolumn{4}{c}{\ipa{ki-wâpaht-ê-n-âwâw} } \\
& 3\sg   & 	\grise{}  & \grise{} & \grise{} &\grise{}  & \grise{}  & \multicolumn{4}{c}{\ipa{wâpaht-am} } \\ 
& 3\pl   & \grise{} & \grise{} & \grise{}  & \grise{}  & \grise{}  & \multicolumn{4}{c}{\ipa{wâpaht-am-w-ak} }  \\ 
& 3'  &\grise{}  & \grise{}  & \grise{}  &\grise{}  &  \grise{}  &\multicolumn{4}{c}{\ipa{wâpaht-am-iyi-w-a} }\\ 
\bottomrule
\multirow{2}{*}{\textbf{VAI}} &  & \ipa{ni-wâpiskisi-n} & \ipa{ ki-wâpiskisi-(nâ)naw} & \ipa{ni-wâpiskisi-nân} &\ipa{ki-wâpiskisi-n} &\ipa{ ki-wâpiskisi-nâwâw} & \ipa{wâpiskisi-w} & \ipa{wâpiskisi-wak} & \multicolumn{2}{c}{\ipa{wâpiskisi-yi-wa}} \\
&  & \ipa{ni-pimipahtâ-n} & \ipa{ ki-pimipahtâ-(nâ)naw} & \ipa{ni-pimipahtâ-nân} &\ipa{ki-pimipahtâ-n} &\ipa{ ki-pimipahtâ-nâwâw} & \ipa{pimipahtâ-w} & \ipa{pimipahtâ-wak} & \multicolumn{2}{c}{\ipa{pimipahtâ-yi-wa}} \\
\bottomrule
\multirow{2}{*}{\textbf{VII}} &  &  \grise{}  &  \grise{}  &  \grise{} & \grise{}  & \grise{}  & \ipa{wâpiskâ-w} & \ipa{wâpiskâ-w-a} & \ipa{wâpiskâ-yi-w} & \ipa{wâpiskâ-yi-w-a}\\
  & &  \grise{}  &  \grise{}  &  \grise{} & \grise{}  & \grise{}  & \ipa{miywâsin} & \ipa{miywâsin-w-a} & \ipa{miywâsin-iyi-w} & \ipa{miywâsin-iyi-w-a}\\
 & &  \grise{}  &  \grise{}  &  \grise{} & \grise{}  & \grise{}  & \ipa{wâpan} &\grise{} & \ipa{wâpan-iyi-w} & \grise{}\\
\bottomrule
\end{tabular}

%\end{scriptsize}
}
\end{table}
%\end{landscape}

%\begin{landscape}
\begin{table}[H]
\caption{Plains Cree Conjunct Order paradigms of VTA \ipa{wâpam--} ``see s.o.", VTI \ipa{wâpaht--} ``see sth", VAI \ipa{wâpiskisi--} ``be white (\textsc{+anim})", \ipa{pimipahtâ--}``run", VII \ipa{wâpiskâ--} ``be white (\textsc{-anim})", \ipa{miywâsin} ``be good", \ipa{wâpan} ``be dawn" (based on \citealp{wolfart96sketch})}
\label{tab:paradigmcreeconj}
\centering
\resizebox{16cm}{!}{
\begin{tabular}{ccccccccccc}
\toprule
\multirow{12}{*}{\textbf{VTA}} & \backslashbox{A}{P}  & 	1\sg  & 1\textsc{pi} & 1\textsc{pe} &  2\sg & 2\pl  &  3\sg & 3\pl &	3'\sg & 3'\pl \\ 
\midrule
& 1\sg   & 	\grise{}   & 	\grise{} &  \grise{} &	\cellcolor{pink}\ipa{ê-wâpam-it-ân}  & \cellcolor{pink}\ipa{ê-wâpam-it-ako-k}	& \cellcolor{Turquoise}\ipa{ê-wâpam-ak}   & 	\cellcolor{Turquoise}\ipa{ê-wâpam-ak-ik}  & \multicolumn{2}{c}{\cellcolor{Turquoise}	\ipa{ê-wâpam-im-ak}}\\ 
 & 1\textsc{pi} & \grise{}   &\grise{} & \grise{} & \multicolumn{2}{c}{\grise{}}  & \cellcolor{Dandelion}\ipa{ê-wâpam-â-yahk} & \cellcolor{Dandelion}\ipa{ê-wâpam-â-yahko-k}  & \multicolumn{2}{c}{\cellcolor{Dandelion}	\ipa{ê-wâpam-im-â-yahk}}\\ 
& 1\textsc{pe} & \grise{}   &\grise{} & \grise{} & \multicolumn{2}{c}{\cellcolor{pink}\ipa{ê-wâpam-it-âhk}}   & \cellcolor{Dandelion}\ipa{ê-wâpam-â-yâhk} & \cellcolor{Dandelion}\ipa{ê-wâpam-â-yâhk-ik}  & \multicolumn{2}{c}{\cellcolor{Dandelion}	\ipa{ê-wâpam-im-â-yâhk}}\\ 
& 2\sg   & 	\cellcolor{cyan}\ipa{ê-wâpam-i-yan}   & \grise{}& \multirow{2}{*}{\cellcolor{cyan}}	&	\grise{}   &  \grise{} & \cellcolor{Aquamarine}\ipa{ê-wâpam-at}  & \cellcolor{Aquamarine}\ipa{ê-wâpam-ač-ik} &\multicolumn{2}{c}{\cellcolor{Aquamarine}\ipa{ê-wâpam-im-at}} \\ 
& 2\pl  & 	\cellcolor{cyan}\ipa{ê-wâpam-i-yêk} & \grise{}& \multirow{-2}{*}{\cellcolor{cyan} \ipa{ê-wâpam-i-yâhk}} & \grise{}  & 	\grise{}   & 	\cellcolor{Dandelion}\ipa{ê-wâpam-â-yêk}  & \cellcolor{Dandelion}\ipa{ê-wâpam-â-yêko-k} &\multicolumn{2}{c}{\cellcolor{Dandelion}\ipa{ê-wâpam-im-â-yêk}}\\
& 3\sg   & 	\cellcolor{cyan}\ipa{ê-wâpam-i-t}   & \cellcolor{green}\ipa{ê-wâpam-iko-yahk} & \cellcolor{green}\ipa{ê-wâpam-iko-yâhk} & \cellcolor{SkyBlue}\ipa{ê-wâpam-isk}  & \cellcolor{green}	\ipa{ê-wâpam-iko-yêk} & \cellcolor{Dandelion}	\grise{}  & \grise{}	 & \multicolumn{2}{c}{\cellcolor{Dandelion}	\ipa{ê-wâpam-(im)-â-t} }\\ 
& 3\pl   & 	\cellcolor{cyan}\ipa{ê-wâpam-i-č-ik}&  \cellcolor{green}\ipa{ê-wâpam-iko-yahko-k} & \cellcolor{green}\ipa{ê-wâpam-iko-yâhk-ik}   & \cellcolor{SkyBlue}	\ipa{ê-wâpam-isk-ik}   & \cellcolor{green}	\ipa{ê-wâpam-iko-yêko-k} & \cellcolor{Dandelion}	\grise{} &	\grise{}  & \multicolumn{2}{c}{\cellcolor{Dandelion}\ipa{ê-wâpam-(im)-â-č-ik}}\\ 
& \multirow{2}{*}{3'}   & \multirow{2}{*}{\cellcolor{cyan}}  &  \multirow{2}{*}{\cellcolor{green}}  & \multirow{2}{*}{\cellcolor{green}} &\cellcolor{SkyBlue} &  \multirow{2}{*}{\cellcolor{green}}  &\multirow{2}{*}{\cellcolor{green}}   & \multirow{2}{*}{\cellcolor{green}} & \multicolumn{2}{c}{\cellcolor{Dandelion} \ipa{ê-wâpam-â-yi-t} }\\ 
&  \multirow{-2}{*}{} & \multirow{-2}{*}{\cellcolor{cyan}\ipa{ê-wâpam-i-yi-t}} & \multirow{-2}{*}{\cellcolor{green}\ipa{ê-wâpam-iko-wâ-yahk}}   &  \multirow{-2}{*}{\cellcolor{green}\ipa{ê-wâpam-iko-wâ-yâhk}} &  \multirow{-2}{*}{\cellcolor{SkyBlue}\ipa{ê-wâpam-iy-isk}} &  \multirow{-2}{*}{\cellcolor{green}\ipa{ê-wâpam-iko-wâ-yêk}}& \multirow{-2}{*}{\cellcolor{green}\ipa{ê-wâpam-iko-t}}  & \multirow{-2}{*}{\cellcolor{green}\ipa{ê-wâpam-iko-č-ik}} & \multicolumn{2}{c}{\cellcolor{green} \ipa{ê-wâpam-iko-yi-t}}\\ 
\bottomrule
 \multirow{12}{*}{\textbf{VTI}} &\backslashbox{A}{P}  & 	1\sg  & 1\textsc{pi} & 1\textsc{pe} &  2\sg & 2\pl  &  3\sg & 3\pl &	3'\sg & 3'\pl \\ 
\midrule
& 1\sg   & 	\grise{}   & 	\grise{} &  \grise{} &	\grise{}  & \grise{} 	& \multicolumn{4}{c}{\ipa{ê-wâpaht-am-ân} } \\ 
& 1\textsc{pi} & \grise{}   &\grise{} & \grise{} & \multicolumn{2}{c}{\grise{}}  & \multicolumn{4}{c}{\ipa{ê-wâpahtamahk} } \\ 
& 1\textsc{pe} & \grise{}   &\grise{} & \grise{} & \multicolumn{2}{c}{\grise{}}   & \multicolumn{4}{c}{\ipa{ê-wâpaht-am-âhk} }\\ 
& 2\sg   & 	\grise{}  & \grise{}& \multirow{2}{*}{\grise{}}	&	\grise{}   &  \grise{} & \multicolumn{4}{c}{\ipa{ê-wâpaht-am-an} }\\ 
& 2\pl  & 	\grise{}  & \grise{} & \multirow{-2}{*}{\grise{} } & \grise{}  & 	\grise{}   & 	\multicolumn{4}{c}{\ipa{ê-wâpaht-am-êk} } \\
& 3\sg   & 	\grise{}  & \grise{} & \grise{} &\grise{}  & \grise{}  & \multicolumn{4}{c}{\ipa{ê-wâpaht-ah-k} } \\ 
& 3\pl   & \grise{} & \grise{} & \grise{}  & \grise{}  & \grise{}  & \multicolumn{4}{c}{\ipa{ê-wâpaht-ahk-ik} }  \\ 
& 3'  &\grise{}  & \grise{}  & \grise{}  &\grise{}  &  \grise{}  &\multicolumn{4}{c}{\ipa{ê-wâpaht-am-iyi-t} }\\ 
\bottomrule
\multirow{2}{*}{\textbf{VAI}}  &  & \ipa{ê-wâpiskisi-yân} & \ipa{ê-wâpiskisi-yahk} & \ipa{ê-wâpiskisi-yâhk} &\ipa{ê-wâpiskisi-yan} &\ipa{ê-wâpiskisi-yêk} & \ipa{ê-wâpiskisi-t} & \ipa{ê-wâpiskisi-č-ik} & \multicolumn{2}{c}{\ipa{ê-wâpiskisi-yi-t}} \\
& & \ipa{ê-pimipahtâ-yân} & \ipa{ ê-pimipahtâ-yahk} & \ipa{ê-pimipahtâ-yâhk} &\ipa{ ê-pimipahtâ-yan} &\ipa{ ê-pimipahtâ-yêk} & \ipa{ê-pimipahtâ-t} & \ipa{ê-pimipahtâ-č-ik} & \multicolumn{2}{c}{\ipa{ê-pimipahtâ-yi-t}}\\
\bottomrule
\multirow{2}{*}{\textbf{VII}} &  &  \grise{}  &  \grise{}  &  \grise{} & \grise{}  & \grise{}  & \ipa{ê-wâpiskâ-k} & \ipa{ê-wâpiskâ-k-i} & \ipa{ê-wâpiskâ-yi-k} & \ipa{ê-wâpiskâ-yi-k-i}\\
%&  &  \grise{}  &  \grise{}  &  \grise{} & \grise{}  & \grise{}  & \ipa{ê-misâ-k} & \ipa{ê-misâ-k-i} & \ipa{ê-misâ-yi-k} & \ipa{ê-misâ-yi-k-i}\\
 & &  \grise{}  &  \grise{}  &  \grise{} & \grise{}  & \grise{}  & \ipa{ê-miywâsih-k} & \ipa{ê-miywâsih-k-i}& \ipa{ê-miywâsin-iyi-k} & \ipa{ê-miywâsin-iyi-k-i}\\
  & &  \grise{}  &  \grise{}  &  \grise{} & \grise{}  & \grise{}  & \ipa{ê-wâpah-k} &\grise{} & \ipa{ê-wâpan-iyi-k} & \grise{}\\
 \bottomrule
\end{tabular}
}
\end{table}

%\end{landscape}

The following example from Plains Cree will serve as an illustration of the actual use of these verb classes and the two main orders.

\begin{exe}
\ex \label{ex:creeverbs}
 \gll â, êwak ôma \underline{kâ-wî-tâhkôt-am-ân}\textsf{\textsuperscript{1}}, {matwân cî} kwayask \underline{ni-ka-kî-isi-tâhkôt-ên}\textsf{\textsuperscript{2}} tânis \underline{ê-kî-itâcimostaw-it}\textsf{\textsuperscript{3}} \underline{kâ-kî-oyôhtâwî-yân}\textsf{\textsuperscript{4}}, ôm îta \underline{kâ-pakosêyim-ikawi-yân}\textsf{\textsuperscript{5}} \underline{ka-kî-tâhkôt-am-ân}\textsf{\textsuperscript{6}}, êwak ôm `ôskiciy' \underline{k-êsiyîhkâtê-k}\textsf{\textsuperscript{7}}; ât[a] âni mitoni kwayask \underline{ni-kî-wîhtamâ-ko-h}\textsf{\textsuperscript{8}} mîna n-ôhcâwîs, ita \underline{ê-kî-kanawêyiht-ah-k}\textsf{\textsuperscript{9}} êwak ôma, ita o-mosôm-a \underline{kâ-kî-ohtaskat-am-iyit}\textsf{\textsuperscript{10}} êwak ôma\\
well {\dem} {\dem{:}\inan} {\nmlz-\fut{:}\prox-discuss(VTI)-\theme-1\sg{:}\cnj} {I\_wonder} properly {1-\fut-\pst-thus-discuss(VTI)-1\sg:\indep} how {\cnj-\pst-tell\_about(VTA)-3\sg$\rightarrow$1\sg{:}\cnj} {\nmlz-\pst-have\_as\_father(VAI [\tr])-1\sg{:}\cnj} {\dem{:}\inan} here {\nmlz-expect(VTA)-\unspec-1\sg{:}\cnj} {\fut-\pst-discuss(VTI)-\theme-1\sg{:}\cnj} {\dem} {\dem{:}\inan} {pipestem(NI)} {\nmlz-be\_called(VII)-3\sg{:}\cnj} although then really properly {1-\pst-tell\_about(VTA)-\inv-\pret} {also} {1-father's\_brother} there {\cnj-\pst-keep(VTI)-\theme-3\sg{:}\cnj} {\dem} {\dem{:}\inan} there {3-grandfather-\obv} {\nmlz-\pst-leave(VTI)-\theme-3'{:}\cnj} {\dem} {\dem{:}\inan}\\
\glt `Well, this which I am about to discuss, I wonder if I will be able to discuss it with proper faithfulness, just as my late father had told me the story about it, here [at the Saskatchewan Indian Languages Insititute] where I should (be able) to discuss it, this `pipestem' as it is called; although I had most properly been told about it also by my father's brother, where he had kept this, where his grandfather had left this pipestem behind.' (\citealp[p. 107]{counselling})
\end{exe}

Verb forms (1) and (6) illustrate the use of the conjunct order while verb form (2) illustrates the use of the independent order of the TI verb \textit{tâhkôt-} `discuss sth, discourse upon sth', respectively. The verb stands in the conjunct order in (1) because it acts as a (nominalized) relative clause modifying \ipa{êwak ôma} `that one' and is thus non-finite: `that one (ie. subject) which I am going to discourse upon'; in (6) it is in a complement clause with a deontic meaning: `(it is expected of me) that I should discuss it'. In (3) we see the TA verb \ipa{itâcimostaw-} `tell s.o. thus about it' used in the conjunct order since it appears in a \textit{wh-} clause headed by \ipa{tânis} `how'. The verb form is furthermore inverse since the narrator was told about it by his father, and so we have a case where the patient (or semantically speaking, the addressee in this case) is higher than the agent on the hierarchy in \ref{ex:empathy.cree}. In (4) we see another example of \ipa{kâ-} (\nmlz) used this time as a headless relative clause built from the transitive (sic!) AI verb \ipa{oyôhtâwî-} `have s.o. as one's father' which as such appears in the conjunct order: `(litt.) the one I had as (my) father'. (5) is an example of a TA verb \ipa{pakosêyim-} `expect sth from s.o.' with the unspecified actor suffix \ipa{-ikawi-} (\unspec) used in the conjunct because it modifies \ipa{ita} `there (where)': `(litt.) there where it is expected of me'. In (7) we have the conjunct order form of the II verb \ipa{isiyîhkâtê-} `be called thus' used as a relative clause modifying \ipa{êwak ôma} `that one' (referring to  \ipa{oskiciy-} \ninan\ `pipestem'): `(litt.) that one which is called thus'. In (8) we find another TA verb \ipa{wîhtamaw-}`tell s.o. about sth/s.o.' appearing in the inverse since once again the narrator (1\sg) has been told about the pipestem by his uncle (3\sg). And finally, (9) \ipa{kanawêyiht-} `keep it' and (10) \ipa{ohtaskat-} `leave it (suddenly)' are both TI verbs appearing in the conjunct order, both of them having \ipa{oskiciy-} \ninan\ `pipestem' as their object and modifying once again \ipa{ita} `there (where)'. Observe that (10) shows obviative morphology as well since it has to agree with its subject \ipa{omosôma} `his grandfather' which as explained earlier must bear obviative marking (\ipa{-a}) as its possessor is third person.


\section{The reshaping of the conjunct order in Algonquian}

Algonquian languages share complex verbal paradigms that are mostly inherited from their common ancestor. Even languages, such as Arapaho and Cheyenne, which have undergone some drastic sound changes largely preserve the Proto-Algonquian paradigms albeit with some interesting reshaping.

The present section focuses on two particular paradigms: the conjunct order indicative intransitive animate (VAI) and transitive animate (VTA) conjugations. 

This choice is determined by the fact that the Algonquian conjunct order paradigms constitute the only case in the languages of the world where the creation of a direct/inverse system from a non-hierarchical system can be observed. While the Proto-Algonquian conjunct order paradigm was partly accusative and partly tripartite, some languages, in particular Plains Cree, varieties of Nishnaabemwin, Mi'gmaq and  Arapaho have reshaped it towards a direct/inverse system. In the case of Cree and Ojibwe, historical documents even attest intermediate stages showing how the morphological reshapings came about.

In this section, we first describe the Proto-Algonquian conjunct order conjugation, then present Plains Cree, Nishnaabemwin, Mi'gmaq and Arapaho data, and finally propose a series of generalizations based on these observations.

\subsection{Proto-Algonquian}
The reconstruction of the conjunct order paradigm of Proto-Algonquian is uncontroversial. Table \ref{tab:protoalg.conj} (based on \citealt{bloomfield46proto} and \citealt{goddard00cheyenne}) presents the indicative mode forms of that order, which are directly attested as such in Fox (Kickapoo) and Miami (\citealt{costa03miami}).

The final *\ipa{--i} in the singular direct and inverse forms is the indicative mode suffix. In the subjunctive and participle forms the suffix is *\ipa{--e} and *\ipa{--a}, respectively.\footnote{The participle also presents a different set of endings for the plural forms, which will not be discussed here.} Note that the indicative mode suffix palatalizes an earlier **\ipa{--t--} in *\ipa{--č--} contrary to the subjunctive and participle forms which preserve the non-palatalized **\ipa{--t--}. Thus, the 2\sg$\rightarrow$3 participle form is *\ipa{-ata} while the indicative one is *\ipa{--aci}. As we will see, most of the languages in which the final vowel of the verb form is lost have generalized the non-palatalized forms in the indicative mode of the conjunct order by analogy with the subjunctive and participle forms.


\begin{table}[H]
\caption{Proto-Algonquian conjunct order indicative paradigm, VAI and VTA }
\label{tab:protoalg.conj} \centering
\resizebox{\textwidth}{!}{
\begin{tabular}{lllllllll}
\toprule
 \backslashbox{A}{P}  & 	1\sg  & 1\pli & 1\pe &  2\sg & 2\pl  &  3\sg & 3\pl &	3' \\ 
\midrule
1\sg& 	\grise{}   & 	\grise{} &  \grise{} & 	\ipa{-eθ-ân-i} & 	\ipa{-eθ-akokw-e} & 	\ipa{-ak-i} & 	\ipa{-ak-wâw-i} & 	\ipa{-em-ak-i} \\ 	
1\pli & \grise{}   & 	\grise{} &  \grise{}	 & \grise{}  & \grise{} 	 & 	\ipa{-ankw-e} & 	\ipa{} & 	\ipa{-em-ankw-e} \\ 	
1\pe & 	\grise{}   & 	\grise{} &  \grise{}	 & 	\multicolumn{2}{c}{\ipa{-eθ-ânk-e}} \ipa{} & 	\ipa{-akenč-i} & 	\ipa{} & 	\ipa{-em-akenč-i} \\ 	
2\sg & 	\ipa{-iy-an-i} & \grise{} 	 & 	\multirow{2}{*}{\ipa{-iy-ânk-e}} & 	\grise{}  & \grise{} 	 & 	\ipa{-ač-i} & 	\ipa{-at-wâw-i} & 	\ipa{-em-ač-i} \\ 	
2\pl & 	\ipa{-iy-êkw-e} & \grise{} 	 & \multirow{-2}{*}{\ipa{}}  & \grise{} 	 & \grise{} 	 & 	\ipa{-êkw-e} & 	\ipa{} & 	\ipa{-em-êkw-e} \\ 	
3\sg & 	\ipa{-i-č-i} & 	\multirow{2}{*}{\ipa{-eθ-ankw-e}} & 	\multirow{2}{*}{\ipa{-iy-amenč-i}} & 	\ipa{-eθ-k-i} & 	\multirow{2}{*}{\ipa{-eθ-âkw-e}} & \grise{} & \grise{}  & 		\cellcolor{Dandelion}\ipa{-â-č-i} \\ 	
3\pl & 	\ipa{-i-wâ-č-i} & \multirow{-2}{*}{\ipa{}} & \multirow{-2}{*}{\ipa{}} & 	\ipa{-eθ-k-wâw-i} & \multirow{-2}{*}{\ipa{}} & \grise{}  & 	\grise{}  & 		\cellcolor{Dandelion}\ipa{-â-wâ-č-i} \\ 	
3' & 	\ipa{-i-ri-č-i} & 	\ipa{} & 	\ipa{} & 	\ipa{-em-eθ-k-i} & 	\ipa{} & 	\cellcolor{green}\ipa{-ekw-eč-i} & 	\cellcolor{green}	\ipa{-eko-wâ-č-i} & 	\ipa{} \\\bottomrule 	
\intr & \ipa{-ân-i}	& \ipa{-ankw-e}	 & \ipa{-ânk-e}	 & \ipa{-an-i}	 & \ipa{-êkw-e} & \ipa{-č-i} / \ipa{-k-i}	 & \ipa{-wâ-č-i}	 &  \ipa{-ri-č-i}\\ 
\bottomrule
\end{tabular}
}
\end{table}	

The proto-Algonquian system is clearly not a direct/inverse one, except for the non-local scenarios (3$\rightarrow$3' and 3'$\rightarrow$3) where what will later become the direct (\ipa{--â--}) and inverse (\ipa{--ekw--}) markers can be seen. As for the rest, some parts of the system are tripartite, in particular the first and second singular and the first person plural exclusive forms. For instance, intransitive 1\pe *\ipa{--ânk-e} and transitive 1\pe$\rightarrow$3 *\ipa{--akenč-i},  3$\rightarrow$1\pe *\ipa{--iy-amenč-i} are all marked by unrelated  morphemes ($S \ne A \ne P$).

Other forms present accusative alignment; for instance, the second plural has \ipa{--êkw-e} in both intransitive and direct forms, but *\ipa{--âkw-e}  in inverse ones ($S = A \ne P$). In all inverse and local forms, there are specific markers for first person (*\ipa{--i(y)--}) and second person (*\ipa{--eθ--}) patients. The first person inclusive, which represents the association of the speaker(s), ie. a first person, with the hearer, i.e. a second person, also shows the second person patient marker (*\ipa{--eθ--}) on top of its corresponding direct marker (*\ipa{--ankw-}) in inverse forms. Incidentally, this is one of two suffixes neutral as to the syntactic roles in the system, alongside third person *\ipa{--č-i/k-i} (cf. Table \ref{tab:protoalg.align}).


\begin{table}[H]
\caption{The alignment of PA indicative personal verb suffixes}
\centering \label{tab:protoalg.align}
\begin{tabular}{llll}
\toprule
& S & A & P\\
1\sg & \multicolumn{2}{c}{*\ipa{--âni} ($\rightarrow$2\sg)}\hspace{1.5cm} & *\ipa{--i}\\
 & & *\ipa{--akokw-e} ($\rightarrow$2\pl) &\\
& & *\ipa{--ak-i} ($\rightarrow$3) &\\
1\pli & \multicolumn{2}{c}{*\ipa{--ankw-e}}\hspace{2.2cm} & *\ipa{--eθ-ankw-e}\\
1\pe & \multicolumn{2}{c}{*\ipa{--ânk-e}  ($\rightarrow$2)} \hspace{1.5cm} & *\ipa{--iy-ânk-e} (2$\rightarrow$)\\
& & *\ipa{--akenč-i}  ($\rightarrow$3)& *\ipa{--iy-amenč-i} (3$\rightarrow$) \\
\midrule
2\sg & \multicolumn{2}{c}{*\ipa{--an-i} ($\rightarrow$1\sg)}\hspace{1.5cm} & *\ipa{--eθ}\\
&&*\ipa{--ač-i} ($\rightarrow$3) &\\
2\pl & \multicolumn{2}{c}{*\ipa{--êkw-e}}\hspace{2.5cm} & *\ipa{--eθ-akokw-e}  (1\sg$\rightarrow$) \\
& & & *\ipa{--eθ-âkw-e}  (3\sg$\rightarrow$)\\
\midrule
3\sg & \multicolumn{2}{c}{*\ipa{--č-i}/*\ipa{--k-i} ($\rightarrow$1, 2\sg, 3')}\hspace{0.5cm} & *\ipa{--č-i}/*\ipa{--k-i}\\
3\pl & *\ipa{--wâ-č-i} & \ipa{--k-wâw-i} & *\ipa{--wâw-i}/*\ipa{--wâ-č-i}  \\
\bottomrule
\end{tabular}
\end{table}


The following sections show how such a non-hierarchical system was independently reshaped as a (partial) direct/inverse system in several Algonquian languages by ousting the opaque forms and replacing them with (more) transparent ones. 
 
 

\subsection{Plains Cree}
Table \ref{tab:cree.conj} presents the conjunct order paradigm of Modern Plains Cree while Table \ref{tab:creedia.conj} presents the earliest attested stage in the conjunct order paradigm of Plains Cree.

\begin{table}[h]
\caption{Plains Cree Conjunct Order indicative paradigms.  (\citealp{wolfart96sketch})}
\label{tab:cree.conj} \centering
\resizebox{\textwidth}{!}{
\begin{tabular}{lllllllll}
\toprule
 \backslashbox{A}{P}  & 	1\sg  & 1\pli & 1\pe &  2\sg & 2\pl  &  3\sg & 3\pl &	3' \\ 
\midrule
1\sg   & 	\grise{}   & 	\grise{} &  \grise{} &	\ipa{-it-ân}  & \ipa{-it-ako-k}	& \ipa{-ak}   & 	\ipa{-ak-ik}  & 	\ipa{-im-ak}   \\ 
1\pli & \grise{}   &\grise{} & \grise{} & \multicolumn{2}{c}{\grise{}}  & \ipa{-â-yahk} & \ipa{-â-yahko-k}  & 	\ipa{-im-â-yahk}   \\ 
1\pe & \grise{}   &\grise{} & \grise{} & \multicolumn{2}{c}{\ipa{-it-âhk}}   & \ipa{-â-yâhk} & \ipa{-â-yâhk-ik}  & 	\ipa{-im-â-yâhk}   \\ 
2\sg   & 	\ipa{-i-yan}   & \grise{}& \multirow{2}{*}{}	&	\grise{}   &  \grise{} & \ipa{-at}  & \ipa{-ač-ik} &\ipa{-im-at}   \\ 
2\pl  & 	\ipa{-i-yêk} & \grise{}& \multirow{-2}{*}{ \ipa{-i-yâhk}} & \grise{}  & 	\grise{}   & 	\ipa{-â-yêk}  & \ipa{-â-yêko-k} & 	\ipa{-im-â-yêk}   \\
3\sg   & 	\ipa{-i-t}   & \ipa{-iko-yahk} & \ipa{-iko-yâhk} & \ipa{-isk}  & 	\ipa{-iko-yêk} & 	\grise{}  & \grise{}	 & 	\ipa{-(im)-â-t}   \\ 
3\pl   & 	\ipa{-i-č-ik}&  \ipa{-iko-yahko-k} & \ipa{-iko-yâhk-ik}   & 	\ipa{-isk-ik}   & 	\ipa{-iko-yêko-k} & 	\grise{} &	\grise{}  & 	\ipa{-(im)-â-č-ik}   \\ 
\multirow{2}{*}{3'}   & \multirow{2}{*}{}  &  \multirow{2}{*}{}  & \multirow{2}{*}{} & &  \multirow{2}{*}{}  &\multirow{2}{*}{}   & \multirow{2}{*}{} &  \ipa{-â-yi-t} \\ 
 \multirow{-2}{*}{} & \multirow{-2}{*}{\ipa{-i-yi-t}} & \multirow{-2}{*}{\ipa{-ikow-â-yahk}}   &  \multirow{-2}{*}{\ipa{-ikow-â-yâhk}} &  \multirow{-2}{*}{\ipa{-iy-isk}} &  \multirow{-2}{*}{\ipa{-ikow-â-yêk}}& \multirow{-2}{*}{\ipa{-iko-t}}  & \multirow{-2}{*}{\ipa{-iko-č-ik}} &  \ipa{-iko-yi-t}  \\ 
\bottomrule
\textsc{intr} & \ipa{-yân} & \ipa{ -yahk} & \ipa{-yâhk} &\ipa{ -yan} &\ipa{ -yêk} & \ipa{-t} & \ipa{-č-ik} & \ipa{-yi-t} \\
\bottomrule
\end{tabular}
}
\end{table}
 
Comparing  Table \ref{tab:cree.conj} with Table \ref{tab:creedia.conj} we can easily see that the direct forms and the inverse ones, bearing the so-called `theme signs' \textit{-â-} (direct) vs. \textit{-ikw-} (inverse), originally present only in non-local (3$\rightarrow$3' and 3'$\rightarrow$3, respectively) scenarios have been generalized to other parts of the paradigm at the expense of older and less easily segmentable ones.

\begin{table}[H]
\caption{19\textsuperscript{th} century Plains Cree Conjunct Order indicative paradigms (based on \citealp{dahlstrom89change})}
\label{tab:creedia.conj} \centering
\resizebox{\textwidth}{!}{
\begin{tabular}{lllllllll}
\toprule
 \backslashbox{A}{P}  & 	1\sg  & 1\pli & 1\pe &  2\sg & 2\pl  &  3\sg & 3\pl &	3' \\ 
\midrule
1\sg   & 	\grise{}   & 	\grise{} &  \grise{} &	\ipa{-it-ân}  & \ipa{-it-ako-k}	& \ipa{-ak}   & 	\ipa{-ak-ik}  & 	\ipa{-im-ak}   \\ 
1\pli & \grise{}   &\grise{} & \grise{} & \multicolumn{2}{c}{\grise{}}  &  \ipa{-ahk} & \ipa{-ahko-k} & 	\ipa{-im-â-yahk}   \\ 
1\pe & \grise{}   &\grise{} & \grise{} & \multicolumn{2}{c}{\ipa{-it-âhk}}   &  \ipa{-ak-iht} &  \ipa{-ak-ihč-ik}   & 	\ipa{-im-â-yâhk}   \\ 
2\sg   & 	\ipa{-i-yan}   & \grise{}& \multirow{2}{*}{}	&	\grise{}   &  \grise{} & \ipa{-at}  & \ipa{-ač-ik} &\ipa{-im-at}   \\ 
2\pl  & 	\ipa{-i-yêk} & \grise{}& \multirow{-2}{*}{ \ipa{-i-yâhk}} & \grise{}  & 	\grise{}   & 	\ipa{-êk}  & \ipa{-êko-k} & 	\ipa{-im-â-yêk}   \\
3\sg   & 	\ipa{-i-t}   & \ipa{-it-ahk} & \ipa{-i-yam-iht}  & \ipa{-isk}  & 	\ipa{-it-êk} & 	\grise{}  & \grise{}	 & 	\ipa{-(im)-â-t}   \\ 
3\pl   & 	\ipa{-i-č-ik}&  \ipa{-it-ahko-k} & \ipa{-i-yam-ihč-ik}   & 	\ipa{-isk-ik}   & 	\ipa{-it-êko-k} & 	\grise{} &	\grise{}  & 	\ipa{-(im)-â-č-ik}   \\ 
\multirow{2}{*}{3'}   & \multirow{2}{*}{}  &  \multirow{2}{*}{}  & \multirow{2}{*}{} & &  \multirow{2}{*}{}  &\multirow{2}{*}{}   & \multirow{2}{*}{} &  \ipa{-â-yi-t} \\ 
 \multirow{-2}{*}{} & \multirow{-2}{*}{\ipa{-i-yi-t}} & \multirow{-2}{*}{\ipa{-ikow-â-yahk}}   &  \multirow{-2}{*}{\ipa{-ikow-â-yâhk}} &  \multirow{-2}{*}{\ipa{-iy-isk}} &  \multirow{-2}{*}{\ipa{-ikow-â-yêk}}& \multirow{-2}{*}{\ipa{-iko-t}}  & \multirow{-2}{*}{\ipa{-iko-č-ik}} &  \ipa{-iko-yi-t}  \\ 
\bottomrule
\textsc{intr} & \ipa{-yân} & \ipa{ -yahk} & \ipa{-yâhk} &\ipa{ -yan} &\ipa{ -yêk} & \ipa{-t} & \ipa{-č-ik} & \ipa{-yi-t} \\
\bottomrule
\end{tabular}
}
\end{table}

According to \cite{dahlstrom89change}, the change proceeded in two steps. First, as shown in table \ref{tab:cree.vta.innov.inv}, the relevant inverse forms were innovated,\footnote{Here and afterward innovative forms are shown in grey.} based upon the generalized use of the inverse marker in the independent order and by analogy with the inanimate actor forms which had the inverse marker already in both orders as a result of an earlier and non-documented similar analogical process. This change was completed by the end of the 19th century. 

\begin{table}[H]
\caption{Innovative inverse forms in the Plains Cree conjunct order VTA paradigm}
\centering \label{tab:cree.vta.innov.inv}
\resizebox{\textwidth}{!}{
\begin{tabular}{llllll}
\toprule
& Innovative  & Inanimate actor& PA paradigm & Conservative VTA & PA paradigm\\
&VTA paradigm & forms & (inanimate actor) &paradigm (19th century) &  (VTA)\\
\midrule
%\textsc{3$\rightarrow$1s} &\ipa{--igo-yaan} \grise{}& 	\ipa{--id} & \ipa{--itʃ} &	 *\ipa{--iti} & 		\\
3\sg$\rightarrow$1\pe & \ipa{--iko-yâhk} \grise{}& \multirow{2}{*}{\ipa{--iko-yâhk}} &	  \multirow{2}{*}{*\ipa{--iy-amenki}}&\ipa{--iy-amiht} &  \multirow{2}{*}{*\ipa{--iy-amenči}}\\
3\pl$\rightarrow$1\pe & 	\ipa{--iko-yâhk-ik} \grise{}& \multirow{-2}{*}{} &   \multirow{-2}{*}{}	& \ipa{--iy-amihč-ik}\grise{} & \multirow{-2}{*}{}\\
3\sg$\rightarrow$1\pli & 	\ipa{--iko-yahk} \grise{}& \multirow{2}{*}{\ipa{--iko-yahk}} & \multirow{2}{*}{ *\ipa{--eθ-ankwe}}	&\ipa{--it-ahk}  &\multirow{2}{*}{ *\ipa{--eθ-ankwe}} \\
3\pl$\rightarrow$1\pli & 	\ipa{--iko-yahko-k} \grise{}& \multirow{-2}{*}{}& \multirow{-2}{*}{}	& \ipa{--it-ahko-k}\grise{}  &\multirow{-2}{*}{}\\
\midrule
%\textsc{3$\rightarrow$2s} & 	\ipa{--igo-yan} \grise{}& 	\ipa{--ik} &	\ipa{--ik} &  *\ipa{--eθki} & 		\\
3\sg$\rightarrow$2\pl & \ipa{--iko-yêk} \grise{}&  \multirow{2}{*}{\ipa{--iko-yêk}} & \multirow{2}{*}{*\ipa{--eθ-âkwe}}	&\ipa{--it-êk} & \multirow{2}{*}{*\ipa{--eθ-âkwe}}\\
3\pl$\rightarrow$2\pl & \ipa{--iko-yêko-k} \grise{}& \multirow{-2}{*}{}	& \multirow{-2}{*}{}& \ipa{--it-êko-k}\grise{} & \multirow{-2}{*}{}\\

%\midrule
%3'\textsc{$\rightarrow$3s} & \ipa{--igod} & 	\ipa{--igod} &		\ipa{--igotʃ} &*\ipa{--ekweti} & 		\\
%3'\textsc{$\rightarrow$3p} & \ipa{--igodwaa}   & 	\ipa{--igodwaa}  \grise{}&\ipa{--igowaatʃ} &*\ipa{--ekowaati} & 		\\
\bottomrule
\end{tabular}}
\end{table}

Then, possibly in an effort to rationalize the system and make it more coherent, the direct forms followed suit, and the modern system is attested as such at the very beginning of the 20th century (cf. Table \ref{tab:cree.vta.innov.dir}).

\begin{table}[H]
\caption{The Plains Cree VTA paradigm innovative conjunct order direct forms}
\centering \label{tab:cree.vta.innov.dir}
\begin{tabular}{llll}
\toprule
& Innovative & Conservative VTA & Proto-Algonquian \\
& VTA paradigm & paradigm (19th century) &\\
\midrule
%\textsc{3$\rightarrow$1s} &\ipa{--igo-yaan} \grise{}& 	\ipa{--id} & \ipa{--itʃ} &	 *\ipa{--iti} & 		\\
1\pe$\rightarrow$3\sg & 	\ipa{--â-yâhk} \grise{}& 	\ipa{--akiht} & \multirow{2}{*}{*\ipa{--akenči}} \\
1\pe$\rightarrow$3\pl & 	\ipa{--â-yâhk-ik} \grise{}& 	\ipa{--akihcik} \grise{} & \multirow{-2}{*}{}\\
%\textsc{3p$\rightarrow$1\pe} & 	\ipa{--iko-yâhkok} \grise{}& 	\ipa{--iyamihtik} &  *\ipa{--iyamenti} & 		\\
1\pli$\rightarrow$3\sg & 	\ipa{--â-yahk} \grise{}& 	\ipa{--ahk}  &\multirow{2}{*}{*\ipa{--ankw-e}} \\
1\pli$\rightarrow$3\pl & 	\ipa{--â-yahko-k} \grise{}& 	\ipa{--ahko-k} \grise{} & \multirow{-2}{*}{} \\
%\textsc{3p$\rightarrow$1\pli} & 	\ipa{--iko-yahk} \grise{}& 	\ipa{--ul-gw}  &*\ipa{--itahk} & 		\\
\midrule
%\textsc{3$\rightarrow$2s} & 	\ipa{--igo-yan} \grise{}& 	\ipa{--ik} &	\ipa{--ik} &  *\ipa{--eθki} & 		\\
2\pl$\rightarrow$3\sg & \ipa{--â-yêk} \grise{} & \ipa{--êk} &\multirow{2}{*}{*\ipa{--êkw-e}}\\
2\pl$\rightarrow$3\pl & \ipa{--â-yêko-k} \grise{} & \ipa{--êko-k}\grise{} & \multirow{-2}{*}{}\\
%\midrule
%3'\textsc{$\rightarrow$3s} & \ipa{--igod} & 	\ipa{--igod} &		\ipa{--igotʃ} &*\ipa{--ekweti} & 		\\
%3'\textsc{$\rightarrow$3p} & \ipa{--igodwaa}   & 	\ipa{--igodwaa}  \grise{}&\ipa{--igowaatʃ} &*\ipa{--ekowaati} & 		\\
\bottomrule
\end{tabular}
\end{table}

Following are some of the examples  \citealt{dahlstrom89change} gives to illustrate the change. They come from the 1855 translation of the Gospel according to St. John and the First Epistle General of John compared to a 1904 edition of the New Testament. We can see that the older forms still in use in the former two have been replaced by the innovative ones in the latter.

In ex. \ref{ex:oldercreeahk} we see an example of the direct vs inverse mixed scenario archaic forms 1\pli$\rightarrow$3\sg\ (\ipa{--aht}) and 3\sg$\rightarrow$1\pli\ (\ipa{--itahk}), respectively, which are replaced by the innovative ones, viz. \ipa{--â-yahk} and \ipa{--iko-yahk} in ex. \ref{ex:newcreeahk}.
%âêî
\begin{exe}
\ex 
\begin{xlist}
\ex \label{ex:oldercreeahk}
\gll namawiya kiyânaw, ê-kîh-sâkih-ahk Manitôw, mâka wiya ê-kîh-sâkih-itahk.\\
{\negat} {1\pli} {\cnj-\pst-love(VTA)-1\pli$\rightarrow$3\sg:\cnj} God {but} {3\sg} {\cnj-\pst-love(VTA)-3\pli$\rightarrow$1\pli:\cnj}\\
\glt `...not that we loved God, but that he loved us, ...' (First Epistle General John 4.10 (1855), \citealt[p. 3]{dahlstrom89change})


\ex \label{ex:newcreeahk}
\gll namawiya kiyânaw, ê-kîh-sâkih-â-yahk Manitôw, mâka wiya ê-kîh-sâkih-iko-yahk.\\
{\negat} {1\pli} {\cnj-\pst-love(VTA)-\dir-1\pli$\rightarrow$3\sg:\cnj} God {but} {3\sg} {\cnj-\pst-love(VTA)-\inv-3\sg$\rightarrow$1\pli:\cnj}\\
\glt `...not that we loved God, but that he loved us, ...' (First Epistle General, John 4.10 (1904), \citealt[p. 3]{dahlstrom89change})
\end{xlist}
\end{exe}

Examples like this where both the direct and the inverse forms show the archaic suffixes in the 1855 translation are less common than those where only the direct forms are archaic. Indeed, the change was already well under way in the inverse configurations, as only one third of the inverse forms documented in this translation show the relevant archaic suffixes, while the remaining two thirds had already been inovated (\citealt[p. 3]{dahlstrom89change}). Compare ex. \ref{ex:oldercreeakiht} and \ref{ex:newcreeakiht} with an example of the shift from an archaic to an innovative form in the case of a direct scenario (ie. 1\pl$\rightarrow$3) and ex. \ref{ex:oldikoyaahk} and \ref{ex:newikoyaahk} in the case of the corresponding inverse scenario (ie. 3$\rightarrow$1\pl) where the innovative form is already in use in the older version.

\begin{exe}
\ex 
\begin{xlist}
\ex \label{ex:oldercreeakiht}
 \gll kita nipah-akiht\\
{for} kill(VTA)-1\pe$\rightarrow$3\sg:\cnj\\
\glt `...for us to kill him..' (John 18.31 (1855),  \citealt[p. 3]{dahlstrom89change})

\ex \label{ex:newcreeakiht}
 \gll kita nipah-â-yâhk\\
{for} kill(VTA)-\dir-1\pe\\
\glt `...for us to kill him..' (John 18.31 (1904),  \citealt[p. 3]{dahlstrom89change})
\end{xlist}
\end{exe}
%âêî

\begin{exe}
\ex 
\begin{xlist}
\ex \label{ex:oldikoyaahk}
 \gll kâ-kîh-is-itisahw-iko-yâhk-ik\\
\nmlz-\pst-thus-send(VTA)-\inv-1\pl-3\pl\\
\glt `...them that sent us...' (John 1.22 (1855),  \citealt[p. 3]{dahlstrom89change})

\ex \label{ex:newikoyaahk}
 \gll kâ-kî-pê-itisahw-iko-yâhk-ik\\
\nmlz-\pst-thus-send(VTA)-\inv-1\pl-3\pl\\
\glt `...them that sent us...' (John 1.22 (1855),  \citealt[p. 3]{dahlstrom89change})
\end{xlist}
\end{exe}

% \begin{exe}
% \ex
% \begin{xlist}
% \ex \label{ex:oldercree2pl}
%  \gll ê-kîh-kiskêyim-êk\\
% \cnj-kill(VTA)-\dir-2\pl\\
% \glt `...because ye have known [him]..' (John 2.13 (1855),  \citealt[p. 4]{dahlstrom89change})
% 
% \ex \label{ex:newcree2pl}
%  \gll ê-kiskêyim-â-yek\\
% \cnj-kill(VTA)-\dir-2\pl\\
% \glt `...for us to kill him..' (John 2.13 (1904),  \citealt[p. 4]{dahlstrom89change})
% \end{xlist}
% \end{exe}

This reshaping of the system has thus taken place some time between the 19\textsuperscript{th} and the beginning of the 20\textsuperscript{th} centuries. It is particularly noteworthy that it has affected only mixed scenarios with plural speech act participants and has been completed only in the Plains Cree dialect. 

Indeed, other dialects such as Woods Cree, for instance, still use the archaic forms, at least those of the direct set. Ex. \ref{ex:woodsakiht} shows an archaic direct 1\pl$\rightarrow$3\sg\ form (\ipa{--akiht}), while ex. \ref{ex:woodsiyamiht} illustrates the corresponding inverse configuration with 3\sg$\rightarrow$1\pl\ and the archaic \ipa{--iyamiht}.

\begin{exe}
\ex \label{ex:woodsakiht}
 \gll îkosi â-kî-isi-kiskinawhamâ-kawi-yâ ta-pamih-akiht isa kisî-aya.\\
thus {\cnj-\pst-thus-teach(VTA)-\unspec-1\pl} {\purp-look\_after(VTA)-1\pl$\rightarrow$3\sg:\cnj} {you\_know} {old-person}\\
\glt `that's how we were taught to look after an elder, you know.' (\citealp[p. 275]{castel})
\end{exe}

\begin{exe}
\ex \label{ex:woodsiyamiht}
 \gll akwâni îkosi â-kî-isi-pimâcih-iyamiht.\\
then thus \cnj-\pst-thus-bring\_up(VTA)-3\sg$\rightarrow$1\pl:\cnj\\
\glt `...and that's how he (=my father)  brought us up.' (\citealp[p. 182]{castel})
\end{exe}

These archaic forms are used alongside the innovative forms (\ipa{--â-yâ} and \ipa{--ikow-â}, respectively), and in the case of the inverse scenario the above cited example is only one of two attested in more than 560 pages of transcribed oral corpus comprising spontaneous narratives from dozens of speakers. This and the fact that the innovative forms are the only ones attested in the direct 1\pli/2\pl$\rightarrow$3 (\ipa{--â-ya}/\ipa{--â-yîk}) and the corresponding inverse 3$\rightarrow$1\pli/2\pl\ (\ipa{--ikow-a}/\ipa{--ikow-îk}) scenarios, show that a similar analogical process is under way in the Woods Cree dialect as well, and we think it can be expected to reach the same levelling result.


\subsection{Ojibwe} \label{subsec:ojibwe}
Some Nishnaabemwin (Ojibwe) dialects  present innovations similar to those observed in Plains Cree, but limited to the inverse forms. Table \ref{tab:ojbw.conj}, based on data from \citet[295]{valentine01grammar}, presents the Nishnaabemwin conservative paradigm. The suffixes with capital \ipa{-I-} appear with the palatalized allomorphs of \ipa{s/sh--} and \ipa{n/zh--} alternating verbs. For instance `give' \ipa{miin--} / \ipa{miizh--} has \ipa{miin-inaan} 1\sg{}$\rightarrow$2\sg{} with non-palatalizing \ipa{i} (from PA *\ipa{e}) and \ipa{miizh-id} 3\sg{}$\rightarrow$1\sg{} with palatalizing \ipa{i} (from PA *\ipa{i}, the first person patient theme sign).

As in Cree, Nishnaabemwin has generalized the non-palatalized allomorphs of second and third person conjunct order suffixes: We thus find 2\sg$\rightarrow$3\sg \ipa{--ad} corresponding to proto-Algonquian *\ipa{--ači} < **\ipa{--ati} in the indicative conjunct order instead of expected *\ipa{--aj}. This is because the subjunctive and participle forms, which  were *\ipa{--ate} and *\ipa{--ata}, respectively, were not palatalized, and were continued by the non-patalized form \ipa{--ad}, which was then generalized to the indicative mode of the conjunct order after the loss of final vowels. This development is not shared by all Ojibwe dialects: The Algonquin Ojibwe dialect described by \citet{cuoq1866}, for instance, has instead generalized the palatalized form (see \citealt[101]{bloomfield46proto}).
 
\begin{table}[htbp]
\caption{The conservative Ojibwe VTA and VAi paradigms }
\label{tab:ojbw.conj} \centering
\resizebox{\textwidth}{!}{
\begin{tabular}{llllllllll}
\toprule
 \backslashbox{A}{P}  & 	1\sg  & 1\pli & 1\pe &  2\sg & 2\pl  &  3\sg & 3\pl &	3'   \\ 
\midrule
1\sg   & 	\grise{}   & 	\grise{} &  \grise{} &	\ipa{-inaan}  & \ipa{-inagog}	& \ipa{-ag}   & 	\ipa{-agwaa}    \\ 
1\pli & \grise{}   &\grise{} & \grise{} & \multicolumn{2}{c}{\grise{}}  &  \ipa{-ang} & \ipa{-ang-waa} &   \\ 
1\pe & \grise{}   &\grise{} & \grise{} & \multicolumn{2}{c}{\ipa{-inaang}}   &  \ipa{-angid} &  \ipa{-angidwaa}     \\ 
2\sg   & 	 \ipa{-Iyan}   & \grise{}& \multirow{2}{*}{}	&	\grise{}   &  \grise{} & \ipa{-ad}  & \ipa{-adwaa}     \\ 
2\pl  & 	 \ipa{-Iyeg} & \grise{}& \multirow{-2}{*}{  \ipa{-Iyaang}} & \grise{}  & 	\grise{}   & 	\ipa{-eg}  & \ipa{-egwaa}     \\
3\sg   & 	 \ipa{-Id}   & \ipa{-inang} & \ipa{-Iyangid}  & \ipa{-ik}  & 	\ipa{-ineg} & 	\grise{}  & \grise{}	 & 	\ipa{-aad}     \\ 
3\pl   & 	 \ipa{-Iwaad}&  \ipa{-inangwaa} & \ipa{-Iyangidwaa} & 	\ipa{-ikwaa}   & 	\ipa{-inegwaa} & 	\grise{} &	\grise{}  & \ipa{-aawaad}  \\ 
 3' & & & & & &   \ipa{-igod} &  \ipa{-igowaad} \\
\bottomrule
\textsc{intr} & \ipa{-yaan} & \ipa{-yang} & \ipa{-yaang} &\ipa{-yan} &\ipa{-yeg} & \ipa{-d} / \ipa{-g} & \ipa{-waad} & \ipa{-nid} \\
\bottomrule
\end{tabular}
}
\end{table}

 Table \ref{tab:ojibwe.vta.2} (see \citealt[178-9]{valentine01grammar})  shows that some dialects of Nishnaabemwin,  such as Parry Island, have developed  innovative forms combining \ipa{--igo--} with the VAI endings as optional variants of the conservative suffixes. The conservative forms themselves have been reshaped in comparison with the paradigm recorded in the 19th century. This includes the introduction of the 2\pl\ suffix \ipa{--eg} in the inverse 3$\rightarrow$2\pl\ form from the direct 2\pl$\rightarrow$3 form together with the doubling of the second person theme sign \ipa{-in} (from *--eθ--), and the replacement of the 3$\rightarrow$1\pe\  \ipa{--iyamintʃ}  by an analysable form created by combining the direct \ipa{--angid} and the first object theme sign \ipa{--i}. For the sake of comparison,  Table \ref{tab:ojibwe.vta.2} also shows the 19th century Algonquin forms from \citet[51]{cuoq1866}, which are directly inherited from proto-Algonquian.

\begin{table}[H]
\caption{The Ojibwe VTA paradigm inverse forms and their PA origins}
\centering \label{tab:ojibwe.vta.2}
\begin{tabular}{llllllll}
\toprule
& Innovative & Conservative & 19th century Nipissing Ojibwe & Proto-Algonquian \\
&paradigm & paradigm&\\
\midrule
3$\rightarrow$1\sg &\ipa{--igo-yaan} \grise{}& 	\ipa{--id} & \ipa{--itʃ} &	 *\ipa{--iči} & 		\\
3$\rightarrow$1\pli & 	\ipa{--igo-yang} \grise{}& 	\ipa{--inang} &  	\ipa{--inang}  	 &*\ipa{--eθankwe} & 		\\
3$\rightarrow$1\pe & 	\ipa{--igo-yaang} \grise{}& 	\ipa{--iyangid} \grise{}&	\ipa{--iyamintʃ}  &  *\ipa{--iyamenči} & 		\\
\midrule
3$\rightarrow$2\sg & 	\ipa{--igo-yan} \grise{}& 	\ipa{--ik} &	\ipa{--ik} &  *\ipa{--eθki} & 		\\
3$\rightarrow$2\pl & \ipa{--igo-yeg} \grise{}& 	\ipa{--ineg}  \grise{} & \ipa{--inaak}  & *\ipa{--eθâkwe} & 		\\
\midrule
3'$\rightarrow$3\sg & \ipa{--igod} & 	\ipa{--igod} &		\ipa{--igotʃ} &*\ipa{--ekweči} & 		\\
3'$\rightarrow$3\pl & \ipa{--igodwaa}   & 	\ipa{--igodwaa}  \grise{}&\ipa{--igowaatʃ} &*\ipa{--ekowaači} & 		\\
\bottomrule
\end{tabular}
\end{table}

This dialect of Nishnaabemwin goes further than Plains Cree as far as inverse forms are concerned, since the analogy has affected not only plural forms, but also singular ones. It is noteworthy that direct forms, on the other hand, have remained unchanged.

\subsection{Mi'gmaq}

The Listuguj (or Restigouche) dialect of Mi'kmaq (or Mi'gmaq in Listuguj orthography), an Eastern Algonquian language spoken in Quebec, shows a number of interesting innovations in its verbal system. The discussion here is based on \cite{Quinn12}. %which make it the most divergent variety of the language. 

\begin{table}[htbp]
\caption{Mi'gmaq independent order (< PA conjunct order participle) indicative paradigm}
\label{tab:migmaqvta1.ind} \centering
\resizebox{\textwidth}{!}{
\begin{tabular}{lllllllll}
\toprule
 \backslashbox{A}{P}  & 	1\sg & 1\pe & 1\pli&  2\sg & 2\pl  &  3\sg & 3\pl & 3' \sg\\ 
\midrule
1\sg   & 	\grise{}   & 	\grise{} &  \grise{} & \ipa{--ul}  & \ipa{-ulnoq}  & \ipa{--(V)'g}  & \ipa{--(V)'gig}\\ 
1\pe & \grise{}   &\grise{} & \grise{} & \multicolumn{2}{c}{\ipa{--ulneg}}  & \ipa{--(Ve)g't}  & \ipa{--(Ve)g'jig} \\ 
1\pli & \grise{}   &\grise{} & \grise{} & \multicolumn{2}{c}{\grise{}}  & \ipa{--ugg} & \ipa{--uggwig} \\ 
2\sg   & 	\ipa{--i'lin}   & \multirow{2}{*}{} & \grise{} &	\grise{}   &  \grise{} & \ipa{--(V)'t}  & \ipa{--(V)'jig}  \\  
2\pl  & 	\ipa{--i'lioq} &\multirow{-2}{*}{ \ipa{ --i'lieg}} &  \grise{} & \grise{}  & 	\grise{}   & 	\ipa{--(V)oq}  & \ipa{--(V)oqig}\\
3\sg   & 	\ipa{--i'lit}   &  \multirow{2}{*}{\ipa{--ugsieg}} & \ipa{--ugsi'gw} & \ipa{--(V)'sg}  &  \multirow{2}{*}{\ipa{--ugsioq}} & 	\grise{}  & \grise{} & \ipa{-a-t'l} \\ 
3\pl   & 	\ipa{--i\textquotesingle lijig}&   \multirow{-2}{*}{\ipa{}} & \ipa{--ugsi'gwig}   & 	\ipa{--(V)'sgig}   & 	 \multirow{-2}{*}{\ipa{}} & 	\grise{} & \grise{}\\ 
3'\sg &   \multicolumn{5}{c}{\grise{}} & \ipa{–t'l}  &  \multicolumn{2}{c}{\grise{}}\\\bottomrule
\end{tabular}
}
\end{table}

One such innovation concerns the transitive animate paradigm. While it has replaced, along with all Mi'gmaq dialects, the PA independent order forms by the conjunct order ones (cf. Table \ref{tab:migmaqvta1.ind}), Listuguj has departed from the other dialects' more direct PA reflexes based on local person `theme signs', still present at earlier attested stages of the dialect (cf. Table \ref{tab:hewsonfrancis}) by innovating the TA morphology for the mixed 3$\rightarrow$1/2\pl\ scenario  (cf. Table \ref{tab:migmaq.vta.innov}). According to \cite{Quinn12}, the innovation consists in a combination of the inverse suffix (\ipa{--ug--} < PA *\ipa{--ekw--}) and the reflexive one (\ipa{--si--} < PA *\ipa{--esi--}). This hypothesis is subject to debate (Will Oxford, p.c.).

\begin{table}[htbp]
\caption{Early 20th century Mi'gmaq VTA indicative independent order paradigm of \ipa{nemi-} `to see' (based on \citealp{hewsonfrancis})}\label{tab:hewsonfrancis}
\resizebox{\textwidth}{!}{
\begin{tabular}{llllllllll}
\toprule
\backslashbox{A}{P} & 1\sg & 1\pe & 1\pli & 2\sg & 2\pl & 3\sg & 3\pl & 3'\sg & 3'\pl \\ 
\midrule
1\sg  & \multicolumn{3}{c}{\grise{}} & \ipa{nemi'l}  & \ipa{nemi'–l–oq}  & \ipa{nemi'–g}  & \ipa{nemi'–g–jig}  & \multicolumn{2}{c}{\grise{}}  \\ 
1\pe &   \multicolumn{3}{c}{\grise{}} & \multicolumn{2}{c}{\ipa{nemi'–l–eg}}  & \ipa{nemi'–gət}  & \ipa{nemi'–gə–jig}  &   \multicolumn{2}{c}{\grise{}}\\ 
1\pli &   \multicolumn{5}{c}{\grise{}}  & \ipa{nemi'–gw}  & \ipa{nemi'–gw–jig}  &   \multicolumn{2}{c}{\grise{}}\\ 
2\sg & \ipa{nemi'–n}  &  \multirow{2}{*}{\ipa{nemi'–eg}}  &   \multicolumn{3}{c}{\grise{}}& \ipa{nemi'–t}  & \ipa{nemi'–jig}  &   \multicolumn{2}{c}{\grise{}} \\ 
2\pl & \ipa{nemi'–oq}  & \multirow{-2}{*}{\ipa{}}  &  \multicolumn{3}{c}{\grise{}} & \ipa{nemi'–oq}  & \ipa{nemi'-oq}  &   \multicolumn{2}{c}{\grise{}} \\ 
3\sg & \ipa{nemi'–t}  & \ipa{nemi'-namə–t}  & \ipa{nemi'–l–g}  & \ipa{nemi'-sg}  & \multirow{2}{*}{\ipa{nemi'–l–oq}}  &  \multicolumn{2}{c}{\grise{}}& \ipa{nemi'–a–jl}  & \ipa{nemi'–a–ji}  \\ 
3\pl & \ipa{nemi'–jig}  & \ipa{nemi'-namə–jig}  & \ipa{nemi'–l–gw–jig}  & \ipa{nemi'sg–jig}  &  \multirow{-2}{*}{\ipa{}}  &   \multicolumn{2}{c}{\grise{}} & \ipa{nemi'–a–ti–jl}  & \ipa{nemi'–a–ti–ji}\\ 
3'\sg &   \multicolumn{5}{c}{\grise{}} & \ipa{nemi'–a–li–jl}  & \ipa{nemi'–a–li–ji}  &  \multicolumn{2}{c}{\grise{}}\\ 
3'\pl &  \multicolumn{5}{c}{\grise{}} & \ipa{nemi'–a–ti–li–jl}  & \ipa{nemi'–a–ti–li–ji}  &   \multicolumn{2}{c}{\grise{}}\\  \bottomrule
\end{tabular}}
\label{migmaq.vta.hewson}
\end{table}

This development is comparable though only partially cognate to the development in the local scenario in Parry Island Nishnaabemwin (cf. section \ref{subsec:ojibwe}), but is also (partially) attested in Wampanoag (\citealp[556]{bragdon}).

\begin{table}[H]
\caption{The Mi'gmaq VTA paradigm innovative inverse forms}
\centering \label{tab:migmaq.vta.innov}
\begin{tabular}{lllllll}
\toprule
& Innovative & Conservative & Proto-Algonquian \\
&paradigm (Listuguj) & paradigm (other dialects) &\\
\midrule
%\textsc{3$\rightarrow$1s} &\ipa{--igo-yaan} \grise{}& 	\ipa{--id} & \ipa{--itʃ} &	 *\ipa{--ici} & 		\\
3$\rightarrow$1\pe & 	\ipa{--ugsi-eg} \grise{}& 	\ipa{--i-nam't} &  *\ipa{--iyamenči} & 		\\
\textsc{3$\rightarrow$1\pli} & 	\ipa{--ugsi-gw} \grise{}& 	\ipa{--ul-gw}  &*\ipa{--eθankwe} & 		\\
\midrule
%\textsc{3$\rightarrow$2s} & 	\ipa{--igo-yan} \grise{}& 	\ipa{--ik} &	\ipa{--ik} &  *\ipa{--eθki} & 		\\
3$\rightarrow$2\pl & \ipa{--ugsi-oq} \grise{}& 	\ipa{--ul-oq} & *\ipa{--eθâkwe} & 		\\
%\midrule
%3'\textsc{$\rightarrow$3s} & \ipa{--igod} & 	\ipa{--igod} &		\ipa{--igotʃ} &*\ipa{--ekweci} & 		\\
%3'\textsc{$\rightarrow$3p} & \ipa{--igodwaa}   & 	\ipa{--igodwaa}  \grise{}&\ipa{--igowaatʃ} &*\ipa{--ekowaati} & 		\\
\bottomrule
\end{tabular}
\end{table}

Listuguj Mi'gmaq also shows an innovative reshaping of the sequence of a TA stem ending in final \ipa{--i} and a following 1\sg\ patient theme sign \ipa{--i} as \ipa{--i'li--}. The origin of this extra \ipa{--l--} is unclear but according to \cite{Quinn12} we may be dealing with either the VTA abstract final \ipa{--l} (with no particular semantic import), or else the \ipa{--l--} may have come about due to some sort of paradigmatic analogy with the 2\sg\ patient suffix \ipa{--ul}. The regular (inherited) endings were then added after a replication of the 1\sg\ patient suffix \ipa{--i}. We think that it is possible to suggest one more solution to this problem: the \ipa{--li--} element may be related to the obviative suffix appearing in inverse non-local scenarios 3'$\rightarrow$3 in other dialects which goes back to PA *\ipa{--ri--} .

\begin{table}[H]
\caption{The Mi'gmaq VTA paradigm innovative 1\sg\ patient forms}
\centering \label{tab:migmaq.vta.innov.1s}
\begin{tabular}{lllllll}
\toprule
& Innovative & Conservative & Proto-Algonquian \\
&paradigm (Listuguj) & paradigm (other dialects) &\\
\midrule
2\sg$\rightarrow$1\sg &\ipa{--i'-li-n} \grise{}& 	\ipa{--i'-n} & *\ipa{--i-yana}\\
2$\rightarrow$1\sg/\pl &\ipa{--i'-li-eg} \grise{}& 	\ipa{--i'-eg} & *\ipa{--i-yêkwa} (2p$\rightarrow$1s)\\
\midrule
3\sg$\rightarrow$1\sg & 	\ipa{--i'-li-t} \grise{}& 	\ipa{--i'-t} &  *\ipa{--i-ta} \\
3\pl$\rightarrow$1\sg & 	\ipa{--i'-li-jig} \grise{}& 	\ipa{--i'jig}  &*\ipa{--i-ciki} \\
%\midrule
%\textsc{3$\rightarrow$2s} & 	\ipa{--igo-yan} \grise{}& 	\ipa{--ik} &	\ipa{--ik} &  *\ipa{--eθki} & 		\\
%\textsc{3$\rightarrow$2p} & \ipa{--ugsi-oq} \grise{}& 	\ipa{--ul-oq} & *\ipa{--eθâkwe} \\
%\midrule
%3'\textsc{$\rightarrow$3s} & \ipa{--igod} & 	\ipa{--igod} &		\ipa{--igotʃ} &*\ipa{--ekweti} & 		\\
%3'\textsc{$\rightarrow$3p} & \ipa{--igodwaa}   & 	\ipa{--igodwaa}  \grise{}&\ipa{--igowaatʃ} &*\ipa{--ekowaati} & 		\\
\bottomrule
\end{tabular}
\end{table}


\subsection{Arapaho}

The paradigm reshaping that has occurred in Cree, Nishnaabemwin and Mi'gmaq is not isolated. Among Algonquian languages, Arapaho provides an example of a language which has reshaped the conjunct order even further. Before discussing the Arapaho VTA paradigm, we provide some information on the VAI   paradigm, which is necessary for understanding the changes in the VTA. %Proto-Algonquian reconstructions are systematically given in this section, as 
We must warn the reader that the drastic sound changes in Arapaho (see \citealt{goddard74arapaho}) have rendered the cognate forms barely recognizable. We cannot provide here a detailed account of Arapaho historical phonology, and defer the reader to Goddard's works for an in-depth presentation of this topic. Arapaho data used in this section is taken from \citet{salzmann67arapaho.verb} and \citet{cowell06arapaho}.

The Arapaho VAI conjunct order paradigm, as shown by \citet[16-7]{goddard65arapaho}, regularly derives from the proto-Algonquian conjunct order participle (for the SAP forms, it could also originate from the corresponding indicative forms). Had it originated from the indicative conjunct order forms, the third person forms would have been different: the third singular suffix, in particular, would have been **\ipa{--θ} < *\ipa{--či}.

Table \ref{tab:arapaho.vai} shows the main  allomorphs for the conjunct order suffixes in Arapaho and their Proto-Algonquian origins. The first plural exclusive \ipa{--'} originates from the indefinite third person form *\ipa{--nki} (\citealt{goddard98morphology.arapaho}), replacing the inherited 1\pe{} ending, which would have been homophonous with that of the first singular.\footnote{The following sound laws apply here: *\ipa{-y-} $\rightarrow$ \ipa{-n-}, *\ipa{a} $\rightarrow$ \ipa{o}, *\ipa{k} $\rightarrow \emptyset $, *\ipa{nk} $\rightarrow$ \ipa{'}, *\ipa{c} $\rightarrow$ \ipa{θ},  *\ipa{r} $\rightarrow$ \ipa{n}; final vowels are always lost. In some cases, two final syllables can be lost, if they follow the pattern *--(V${_1}$)C(y,w)V${_2}$, where C is any of *\ipa{n}, *\ipa{m}, *\ipa{r}, *\ipa{y}, *\ipa{w} and V${_1}$ is a short vowel. }


\begin{table}[H]
\caption{The Arapaho VAI paradigms and its proto-Algonquian origin}
\centering \label{tab:arapaho.vai}
\begin{tabular}{lllllll}
\toprule
Person &   Arapaho    & Expected Arapaho &Proto-Algonquian\\
\midrule
1\sg{}& 	\ipa{--noo} & & 	*\ipa{--yân--} & 	\\	
1\pe{} & 	\ipa{--ni'} /  	\ipa{--'} \grise{} & **\ipa{--noo}	&	 *\ipa{--yânk--}	 \\	
1\pli{} & 	\ipa{--no'} & 	 	&	*\ipa{--yankw--} & 	\\	
\midrule
2\sg{}& 	\ipa{--n} & 	 &	*\ipa{--yan--} & 	\\	
2\pl{}& 	\ipa{--nee} & 	 & 		*\ipa{--yêkw--} & 	\\	
\midrule
3\sg{} & 	\ipa{--t} /	\ipa{--'} & 	&	*\ipa{--ta} / \ipa{--ka}& 	\\	
3'\sg{} & 	\ipa{--níθ} &  	&	*\ipa{--ričiri} & 	\\	
3\pl{}& 	\ipa{--θi'} &  	&	*\ipa{--čiki} 	\\	
3'\pl{}& 	\ipa{--níθi} & 	 &		*\ipa{--ričihi} 	\\	
\bottomrule
\end{tabular}
\end{table}

In comparison with the VAI paradigm, which is almost entirely inherited from proto-Algonquian, the VTA paradigm presents considerable reshaping. The account proposed here as well as the Proto-Algonquian reconstructions are largely based on  \citet[19-24]{goddard65arapaho} (in combination with  \citealt{goddard00cheyenne} for some details of the Proto-Algonquian paradigms). Table \ref{tab:arapaho.vta}   presents the regular endings of the VTA paradigm in Arapaho, taken from  \citet[487-490]{cowell06arapaho} and \citet[448]{cowell05hinono}. The  further obviative 3'$\rightarrow$3' direct and inverse forms are not included.

\begin{table}[H]
\caption{The Arapaho VTA paradigm}
\centering \label{tab:arapaho.vta}
\begin{tabular}{llllllllllll}
\toprule
 & 	1\sg{}& 	1\pli{} & 	1\pe{} & 	2\sg{}& 	2\pl{}& 	3\sg{} & 	3\pl{} & 	3' & 	\\
1\sg{}& \grise{} & 	\grise{} & 	\grise{} & 	\ipa{--éθen} & 	\ipa{--eθénee} & 	\ipa{--o'} &\ipa{--óú'u}  	 & 	 & 	\\
1\pli{} & 	\grise{} & 	\grise{} & 	\grise{} & 	\grise{} & 	\grise{} & 	\ipa{--óóno'} & 	 & 	 & 	\\
1\pe{} & 	\grise{} & 	\grise{} & 	\grise{} & 	\ipa{--een} & 	\ipa{--eenee} & 	\ipa{--éét} & 	\ipa{--ééθi'}  & 	 & 	\\
2\sg{}& 	\ipa{--ín} / \ipa{--ún}& 	\grise{} & \ipa{--ínee} /	\ipa{--únee} & 	\grise{} & 	\grise{} & 	\ipa{--ót} & 	 \ipa{--óti(i)}& 	 & 	\\
2\pl{}& 	\ipa{--éi'een} & 	\grise{} & 	\ipa{--éi'éénee} & 	\grise{} & 	\grise{} & 	\ipa{--óónee} & 	 & 	 & 	\\
3\sg{} & 	\ipa{--éínoo} & 	\ipa{--éíno'} & 	\ipa{--éi'éét} & 	\ipa{--éín} & 	\ipa{--éínee} & 	\grise{} & 	\grise{} & 	\ipa{--oot} & 	\\
3\pl{}& \ipa{--iθi'} /	\ipa{--uθi'} & 	 & 	\ipa{--éi'ééθí'}  & 	\ipa{--eínóni(i)}  & 	 & 	\grise{} & 	\grise{} & 	\ipa{--óóθi'} & 	\\
3' & 	 & 	 & 	 & 	 & 	 & 	\ipa{--éít} & 	\ipa{--éíθi'} &   & 	\\
\bottomrule
\end{tabular}
\end{table}

Given the complexity of the paradigm in Table \ref{tab:arapaho.vta}, we shall split the discussion  in three parts, analyzing the direct, inverse and local forms separately. The SAP$\rightarrow$3\pl{} and 3\pl{}$\rightarrow$SAP are only discussed in the case of the suffix 3\pl{}$\rightarrow$1\sg{} \ipa{--iθi'}), since they otherwise follow the same patterns of refection as the  corresponding SAP$\rightarrow$3\sg{} and 3\sg{}$\rightarrow$SAP forms.

The direct forms of the VTA paradigm are compared with the corresponding reconstructed Proto-Algonquian forms in Table \ref{tab:arapaho.vta.1}, in which the Arapaho forms that do not continue Proto-Algonquian ones are indicated in grey. This table shows that as in Plains Cree, while the singular direct forms are inherited, the SAP plural ones are reshaped by reanalyzing the third person ending \ipa{--oot} as \ipa{--oo-} + the VAI ending \ipa{--t} and generalizing this structure to the first and second person plural: \ipa{--óó-no'} 1\pli{} and \ipa{--óó-nee} 2\pl{} are built by combining the direct marker \ipa{--oo--} with the regular VAI endings.

The 1\pe{} \ipa{--éét} probably does not originate from inherited  *\ipa{--akenta}. This form should have yielded  either *\ipa{--ooot} or *\ipa{--eeet}. While it is not entirely impossible that vowel shortening would have happened, it is more satisfying to derive  \ipa{--éét}  from the unspecified form of the conjunct participle  *\ipa{--enta} (\citealt[4]{goddard98morphology.arapaho}, see the X-3 form of the TA direct paradigm).

\begin{table}[H]
\caption{The Arapaho VTA paradigm direct forms and their PA origins}
\centering \label{tab:arapaho.vta.1}
\begin{tabular}{lllll}
\toprule
Form& Arapaho & Expected Arapaho & Proto-Algonquian \\
\midrule
 1\sg{}$\rightarrow$3\sg{} & 	\ipa{--o'} & 	 & 	*\ipa{--aka} & 		\\		
1\pe{}$\rightarrow$3\sg{} & 	\ipa{--éét}\grise{} & 	**\ipa{--eeet}&  *\ipa{--akenta} & 		\\		
1\pli{}$\rightarrow$3\sg{} & 	\ipa{--óó-no'}\grise{} & 	**\ipa{--o'}& *\ipa{--ankwa} & 		\\		
\midrule
2\sg{}$\rightarrow$3\sg{} & 	\ipa{--ót} && 	*\ipa{--ata} & 		\\		
2\pl{}$\rightarrow$3\sg{} & 	\ipa{--óó-nee} \grise{}& 	**\ipa{--ee}& *\ipa{--êkwa} & 		\\		
\midrule
3\sg{}$\rightarrow$3' & 	\ipa{--oot} & 	&*\ipa{--âta} & 		\\		
3\pl{}$\rightarrow$3' & 	\ipa{--óóθi'} & &	*\ipa{--âčiki} & 		\\		
\bottomrule
\end{tabular}
\end{table}

By contrast with the direct paradigm, the inverse VTA paradigm is almost entirely remade, as in Parry Island Nishnaabemwin: only the third person forms are inherited, as can be seen in Table \ref{tab:arapaho.vta.2}. As in the direct paradigm, the third person ending \ipa{--éít} was reanalyzed as \ipa{--ei--} + the VAI ending \ipa{--t} and all other forms were rebuilt on that model, replacing the inherited forms.\footnote{Arapaho \ipa{--ei--} regularly derives from *\ipa{--ekwe--}; *\ipa{k} $\rightarrow \emptyset $ and *\ipa{we} $\rightarrow $ *\ipa{o} $\rightarrow $ \ipa{i}. } All inverse forms follow this pattern, except the 3$\rightarrow$1\pe{} suffix, where *\ipa{--éi'} would have been been obtained if \ipa{--ei} had been combined with tha VAI 1\pe{} ending \ipa{--'}. The attested 3$\rightarrow$1\pe{} form \ipa{--éi-'-éét} combines the expected form *\ipa{--éi'} with the direct ending \ipa{--éét}.

The 3\pl{}$\rightarrow$1\sg{} suffix	 \ipa{--iθi'} /	\ipa{--uθi'} is the only  suffix in the inverse configurations involving a SAP which was not renewed. It is all the more remarkable that the corresponding 3\sg{}$\rightarrow$1\sg{} form is remade.

\begin{table}[H]
\caption{The Arapaho VTA paradigm inverse forms and their PA origins}
\centering \label{tab:arapaho.vta.2}
\begin{tabular}{lllll}
\toprule
Person & Arapaho & Expected Arapaho&PA Conjunct    \\
\midrule
3\sg{}$\rightarrow$1\sg{} & 	\ipa{--éí-noo}\grise{} &   **\ipa{--it}&	*\ipa{--ita} & 		\\
3\sg{}$\rightarrow$1\pe{} & 	\ipa{--éi-'-éét} \grise{}&**\ipa{--inobeet}& *\ipa{--iyamenta} & 		\\
3\sg{}$\rightarrow$1\pli{} & 	\ipa{--éí-no'} \grise{}& 	**\ipa{--eθo'}&*\ipa{--eθankwa} & 		\\
\midrule
3\pl{}$\rightarrow$1\sg{} & 	 \ipa{--iθi'} /	\ipa{--uθi'} &    &	*\ipa{--ičiki} & 		\\
\midrule
3\sg{}$\rightarrow$2\sg{} & 	\ipa{--éí-n} \grise{}&**\ipa{--es}& *\ipa{--eθki} & 		\\
3\sg{}$\rightarrow$2\pl{} & 	\ipa{--éí-nee} \grise{}& **\ipa{--eθoo}&*\ipa{--eθâkwa} & 		\\
\midrule
3'$\rightarrow$3\sg{} & 	\ipa{--éít} & &	*\ipa{--ekweta} & 		\\
3'$\rightarrow$3\pl{} & 	\ipa{--éíθi'} & &	*\ipa{--ekočiki} & 		\\
\bottomrule
\end{tabular}
\end{table}

Just as the inverse paradigm, the local paradigm has also undergone considerable analogical reshaping with only the 2\sg{}$\rightarrow$1\sg{} and 2\pl{}$\rightarrow$1\sg{} being inherited. 

\begin{table}[H]
\caption{The Arapaho VTA paradigm local forms and their PA origins}
\centering \label{tab:vta.3}
\begin{tabular}{lllll}
\toprule
Person & Arapaho & Expected Arapaho&PA Conjunct    \\
\midrule
1\sg$\rightarrow$2\sg & \ipa{--éθen} \grise{}& **\ipa{--eθoo} &*\ipa{--eθâni}   \\
1\sg$\rightarrow$2\pl &\ipa{--eθénee} \grise{}& **\ipa{--eθou} &*\ipa{--eθakokwe} & \\
1\pe$\rightarrow$2\sg &\ipa{--één} \grise{}& **\ipa{--eθoo} &*\ipa{--eθânke} &   \\
1\pe$\rightarrow$2\pl &\ipa{--eenee} \grise{}& **\ipa{--eθoo} &*\ipa{--eθânke} &   \\
\midrule 
2\sg$\rightarrow$1\sg &\ipa{--ún} / \ipa{--ín} &   & *\ipa{--iyani}     \\
2\sg$\rightarrow$1\pe & \ipa{--éi'één}\grise{}& **\ipa{--inoo}&*\ipa{--iyânkwe} &  \\
2\pl$\rightarrow$1\sg &\ipa{--únee} / \ipa{--ínee} &  &*\ipa{--iyêkwe} &   \\
2\pl$\rightarrow$1\pe &\ipa{--éi'eenee}\grise{} & **\ipa{--inoo}&*\ipa{--iyânkwe} &   \\
\bottomrule
\end{tabular}
\end{table}

\citet[23]{goddard65arapaho} explains the forms 1\pe$\rightarrow$2\sg \ipa{--één} and 3$\rightarrow$1\pe \ipa{--éi-'-één} by proportional analogy, after the reshaping of the inverse paradigm had taken place: As direct and inverse forms were rebuilt by adding VAI endings to the first part of the third person endings \ipa{--oo--} and \ipa{--ei--} reanalyzed as direction markers, the final consonants \ipa{--t} and \ipa{--n} became   respectively 3\sg\ and 2\sg\ markers not only for  S, but also for P.

After that, even in forms where the \ipa{--t} was not a third person marker, in particular  1\pe$\rightarrow$3   \ipa{--éét} and   3$\rightarrow$1\pe\   \ipa{--éi'éét}, it became reanalyzed as such and the forms  1\pe$\rightarrow$2   \ipa{--één} and   2$\rightarrow$1\pe\   \ipa{--éi'één} were built by changing the final \ipa{--t} to \ipa{--n} on the model of the VAI and VTA inverse forms (see Table  \ref{tab:arapaho.analogy.local}).

\begin{table}[H]
\caption{Proportional analogy in the Arapaho local forms}
\centering \label{tab:arapaho.analogy.local}
\begin{tabular}{lllll}
\toprule
 Person &  Form &  Person &  Form\\
\midrule 
 VAI 3\sg & \ipa{--t} &  VAI 2\sg & \ipa{--n} \\
  3'$\rightarrow$3\sg & \ipa{--éí-\textbf{t}} &   3$\rightarrow$2\sg & \ipa{--éí-\textbf{n}} \\
  \midrule 
    1\pe$\rightarrow$3 & \ipa{--éé-\textbf{t}} & 1\pe$\rightarrow$2\sg &  \grise{}\ipa{--éé-\textbf{n}} \\
  3$\rightarrow$1\pe & \ipa{--éi'éé-\textbf{t}} & 2\sg$\rightarrow$1\pe &  \grise{}\ipa{--éi'éé-\textbf{n}} \\
\bottomrule
\end{tabular}
\end{table}

From there, the 1\sg$\rightarrow$2\sg\  \ipa{--éθen} (instead of expected *\ipa{eθoo}) is likely to have originated from the independent order 1\sg$\rightarrow$2\sg\ ending \ipa{--éθ} < *\ipa{--eθe} to which the second person suffix \ipa{--n} from the VAI paradigm was added. 

The second plural forms 1\sg$\rightarrow$2\pl\ \ipa{--eθénee}, 1\pe$\rightarrow$2\pl\ \ipa{--eenee}  and 2\pl$\rightarrow$1\pe \ipa{--éi'eenee} were built from the corresponding second singular forms by replacing the 2\sg\  \ipa{--n} marker with the 2\pl\ one \ipa{--nee}, as shown in Table \ref{tab:arapaho.analogy.local2}.
 
 
 \begin{table}[H]
\caption{Proportional analogy in the Arapaho local forms -- second plural}
\centering \label{tab:arapaho.analogy.local2}
\begin{tabular}{lllll}
\toprule
 Person &  Form &  Person &  Form\\
\midrule 
 VAI 2\sg & \ipa{--n} &  VAI 2\pl & \ipa{--nee} \\
  3$\rightarrow$2\sg & \ipa{--éí-\textbf{n}} &   3$\rightarrow$2\pl & \ipa{--éí-\textbf{nee}} \\
2\sg$\rightarrow$1\sg &  \ipa{--í-\textbf{n}} & 2\pl$\rightarrow$1\sg &  \ipa{--í-\textbf{nee}} \\
   \midrule 
    1\sg$\rightarrow$2\sg& \ipa{--éθe-\textbf{n}} & 1\sg$\rightarrow$2\pl &\ipa{--eθé-\textbf{nee}} \grise{} \\
    1\pe$\rightarrow$2\sg&\ipa{--ee-\textbf{n}} & 1\pe$\rightarrow$2\pl &\ipa{--ee-\textbf{nee}}\grise{} \\
    2\sg$\rightarrow$1\pe& \ipa{--éi'ee-\textbf{n}} & 2\pl$\rightarrow$1\pe & \ipa{--éi'ee-\textbf{nee}}\grise{}\\
\bottomrule
\end{tabular}
\end{table}
 
The restructuring that took place in the Arapaho conjunct order goes one step further than that observed in the Cree paradigms: While the extent of reshaping in the (mixed) direct paradigm is comparable, all inverse and local forms, except 2\sg{}$\rightarrow$1\sg{}, have been remade. The direct \ipa{--oo--} and inverse \ipa{--éí--} theme signs, which originally were restricted to non-local forms, were generalized to nearly direct and all inverse forms in the mixed scenarios (only the 1\sg{}$\rightarrow$3\sg{}, 2\sg{}$\rightarrow$3\sg{} and 3\pl{}$\rightarrow$1\sg{} endings remained unaffected by analogy), and the inverse one was even extended to the local 2$\rightarrow$1\pe\ forms.

Arapaho thus proves that a language can develop a near-canonical direct/inverse system from a partly accusative, partly tripartite one by generalizing the direct and inverse markers of the non-local forms to the mixed and local ones. 

\subsection{The VTA conjunct order and its relationship to other paradigms}
In the sections above, we have studied the effects of analogy in the VAI and VTA conjunct order paradigms largely in isolation from other paradigms. However, it is likely that some analogical patterns, in particular the innovative direct and inverse forms built by combining the direct or inverse theme signs with the VAI endings, are structurally modelled after forms from other more transparent paradigms. Indeed, the (perceived) identity of final  \ipa{--t}  in 3$\rightarrow$3' *\ipa{--ât--} and 3'$\rightarrow$3  *\ipa{--ekwet--} forms with the VAI third person \ipa{--t} could have prompted the reanalysis of the preceding segment *\ipa{--â--} and *\ipa{--ekwe--} as a direction marker which was then productively combined with the corresponding VAI endings in order to obtain the direct and inverse forms in the rest of the paradigm.

Another potential model, in the case of inverse configurations especially, is the unspecified actor paradigm of the conjunct order. While in PA this paradigm had a special set of endings, (\citealt[88]{goddard79comparative}, \citealt[156-7]{oxford14microparameters}), in Ojibwe and Cree, even in the most conservative dialects (and in nearly all Algonquian languages except Kickapoo, Maliseet and Miami), the forms are built by combining the theme sign \ipa{--igoo--} with the VAI person markers, except in the third person, where the inherited suffix \ipa{--ind} (Ojibwe)/\ipa{--iht} (Cree) < *\ipa{--enta} is still preserved (cf. Table \ref{tab:unspec}).

\begin{table}[htbp]
\caption{The conjunct order of the unspecified actor paradigm in Cree and Ojibwe} \label{tab:unspec} \centering
\begin{tabular}{lllllll}
\toprule
Person &   Cree & Ojibwe      &Proto-Algonquian\\
\midrule
X$\rightarrow$1\sg& \ipa{--ikawi-yân} \grise{} &\ipa{--igoo-yaan} \grise{}  & *\ipa{--i<n>ki} \\
X$\rightarrow$1\pe & \ipa{--ikawi-yâhk} \grise{}   &\ipa{--igoo-yaang} \grise{}  & *\ipa{--i<n>amenki} \\
X$\rightarrow$1\pli & \ipa{--ikawi-yahk} \grise{}  &\ipa{--igoo-yang} \grise{}  & *\ipa{--eθ<en>ankwi} \\
\midrule
X$\rightarrow$2\sg& \ipa{--ikawi-yan} \grise{} &\ipa{--igoo-yan} \grise{}  & *\ipa{--eθ<en>ki} \\
X$\rightarrow$2\pl& \ipa{--ikawi-yêk} \grise{} &\ipa{--igoo-yeg} \grise{}  & *\ipa{--eθ<en>âkwi}  \\
\midrule
X$\rightarrow$3\sg& \ipa{--iht}   &\ipa{--ind}  & *\ipa{--e<n>ta} \\
\bottomrule
\end{tabular}
\end{table}

In Cree and Ojibwe texts, we find numerous examples where the unspecified actor forms is used alongside a 3$\rightarrow$SAP form in the same sentence, with the unspecified actor corresponding to the same referent as  the definite third person agent of the 3$\rightarrow$SAP verb (see examples \ref{ex:creeunspec} and \ref{ex:woodscreeunspec} for Cree and ex. \ref{ex:anoozhid} for Ojibwe).


\begin{exe}
 \ex \label{ex:creeunspec}
\gll ``kîkwây ôm?'' îtêw êkwa awa ni-kisêyinîm; ``aya ôm'', itik, {\rm``this is three times stronger than beer,''} k-êt-ikawi-yâhk, k-êt-iko-yâhk êkwa awa.\\
what {\dem{:}\inan} {tell(VTA).3\sg$\rightarrow$3'} then {\dem{:}\anim} {1\poss-old\_man{:}\poss} well {\dem{:}\inan} {tell(VTA).3'$\rightarrow$3\sg} {} {\nmlz-tell(VTA)-\unspec-1\pl} {\nmlz-tell(VTA)-\inv-1\pl} then {\dem{:}\anim}\\
\glt ```What is this?'' my husband said to him; ``Oh this,'' the other replied to him, ``this is three times stronger than beer,'' we were told, he then said to us.' (\citealp[p. 57]{bothsides})
\end{exe}


\begin{exe}
 \ex \label{ex:woodscreeunspec}

\gll Akwa kayâs îy mistik â-wâpam-at awa pikwîta kî-ohtinam-wak kisî-ayak â-kî-ohci-ntawih-ikawi-yâ. {Isa piko} nîsta kîyâpic ôma â-pimâtisi-yân â-kî-ntawihikawiyân.\\
and long\_ago look! tree(NA) {\cnj-see(VTA)-2\sg$\rightarrow$3\sg:\cnj} {\dem{:}\anim} wherever {\pst-take(VTI)-3\pl} old-person {\cnj-\pst-with\_it-cure(VTA)-\unspec-1\pl} just {1\sg{:}\emphat} yet {\dem{:}\inan} {\cnj-live(VAI)-1\sg} {\cnj-\pst-cure(VTA)-\unspec-1\sg}\\
\glt `And long ago, when you saw a tree anywhere, the elders took it and used it to cure [us]. Even myself, in my lifetime, they cured me.' (\citealp[p. 9]{speakingtothefuture})
\end{exe}

%âêîô
\begin{exe}
\ex \label{ex:anoozhid}
\gll Miish gaa-izhi-i-goo-yaan ingoji naawakwe-g, n-ookomis gaa-izhi-anoozh-id. \\
then \pst:IC-thus-say(VTA)--X-1\sg:\cnj{} approximately be.noon(VII)-\textsc{inan.sg:cnj} 1\poss-grandmother  \pst:IC-thus-commission.to.do(VTA)-3$\rightarrow$1\sg:\cnj{} \\
\glt Around noon, I was told, I was told by my grandmother to get something. (\citealt[96]{kegg93portage})
\end{exe}

 It is thus possible that such constructions, rather than the VTA independent order, provided the model on which to shape the innovative inverse scenario forms by combining the inverse theme sign with the VAI endings as in Plains Cree and Parry island Ojibwe.

\begin{table}[htbp]
\caption{The conjunct order of the inanimate actor paradigm in Cree and Ojibwe} \label{tab:inan} \centering
\begin{tabular}{lllllll}
\toprule
Person &   Cree & Ojibwe      &Proto-Algonquian\\
\midrule
X$\rightarrow$1\sg& \ipa{--iko-yân} \grise{} &\ipa{--igo-yaan} \grise{}  & *\ipa{--i-k-i} \\
X$\rightarrow$1\pe & \ipa{--iko-yâhk} \grise{}   &\ipa{--igo-yaang} \grise{}  & *\ipa{--iy-amenk-i} \\
X$\rightarrow$1\pli & \ipa{--iko-yahk} \grise{}  &\ipa{--igo-yang} \grise{}  & *\ipa{--eθ-ankw-i} \\
\midrule
X$\rightarrow$2\sg& \ipa{--iko-yan} \grise{} &\ipa{--igo-yan} \grise{}  & *\ipa{--eθ-k-i} \\
X$\rightarrow$2\pl& \ipa{--iko-yêk} \grise{} &\ipa{--igo-yeg} \grise{}  & *\ipa{--eθ-âkw-i}  \\
\midrule
X$\rightarrow$3\sg& \ipa{--iko-t}   &\ipa{--igo-d}  & *\ipa{--ekw-eč-i} \\
\bottomrule
\end{tabular}
\end{table}
 

\section{The directionality of analogy in polypersonal systems}

The five cases studied above allow us to propose four generalizations concerning the directionality of analogy in polypersonal systems with a proximate/obviative distinction in the non-local forms.
 
First, analogy operates from 3'$\rightarrow$3 to all inverse forms and from 3$\rightarrow$3' to all direct forms. This is a particular case of   Watkins's law (\citealt{watkins62celtic}): Analogy starts out from the third person and extends to the other forms through a reanalysis of the third person ending as part of the verb stem.
 
Second, analogy can apply from direct forms to inverse and  local ones (as shown by the reshaping of 3$\rightarrow$1\pe\ and 3$\rightarrow$2\pl\ in Nishnaabemwin).

Third,  analogy first applies to plural SAP forms before influencing singular SAP forms, both in the case of direct and inverse paradigms. There is no evidence of a hierarchy between third singular and third plural, as we saw that the 3\pl{}$\rightarrow$1\sg{} resisted analogy in Arapaho while its singular counterpart 3\sg{}$\rightarrow$1\sg{} was remade.

Fourth, analogy first applies   to inverse forms before affecting direct forms. There appears  to be no hierarchy between inverse and local forms as to their sensitivity to analogy.

%Whether these four generalizations have a validity in language families other than Algonquian remains to be demonstrated, but we believe that they may be used as a heuristic principle for diachronic studies on languages whose history is less well documented.  

\section{Conclusion}

On the basis of the attested evolutions of the conjunct order paradigms in Algonquian languages, we have proposed several generalizations on the directionality of analogical levelling in polypersonal systems with proximate/obviative contrast in non-local scenarios. 

The generalizations proposed in this paper must be thought of as heuristic principles, to be tested against data from other language families with direct/inverse systems. Future studies on language families such as Sino-Tibetan, in particular on Rgyalrong and Kiranti languages which have fully functional direct/inverse systems but no historical attestations (\citealp{delancey81direction, jackson02rentongdengdi, gongxun14agreement} and \citealp{lai14person}), should make it possible to evaluate whether they remain valid when tested on a larger body of data.


The ultimate explanation for these generalizations may lie in the relative frequency of the forms: less frequent forms are more prone to undergo analogy on the model of more frequent forms. 

Independently of any language, third person forms are generally more frequent in corpora than SAP forms for most verbs (see however \citealt{jacques16ebde} for a marginal counterexample).  This observation may contribute to explain why mixed domain forms (including 3$\rightarrow$SAP and SAP$\rightarrow$3) can be renewed on the model of non-local domain person forms (3'$\rightarrow$3 and 3$\rightarrow$3' respectively) while the opposite is not attested.
 
Likewise, inverse forms are markedly less common in corpora than direct ones, at least in some languages such as Japhug (\citealt{jacques10inverse}), and it is therefore expected that inverse form are modelled on the basis of either direct or intransitive forms, and not in the opposite direction.

 The shaping of a direct/inverse system from an opaque person indexation system like that of the proto-Algonquian conjunct order can thus be accounted for without any recourse to the notion of person hierarchy. This confirms the idea that while person hierarchies may be convenient ways of describing direct/inverse system, they `simply capture the outputs of independent diachronic processes' as pointed out by Cristofaro and Zúniga (this volume), and should not be misunderstood as providing \textit{explanations} for the linguistic patterns they intend to describe.

 \bibliographystyle{unified}
 \bibliography{bibliogj}

\end{document}