\documentclass[oneside,a4paper,11pt]{article} 
\usepackage{fontspec}
\usepackage{natbib}
\usepackage{booktabs}
\usepackage{xltxtra} 
\usepackage{lineno}
\usepackage{polyglossia}
\usepackage[table]{xcolor}
\usepackage{gb4e} 
\usepackage{multicol,multirow}
\usepackage{graphicx}
\usepackage{float}
\usepackage{slashbox} 
\usepackage{rotating}
\usepackage{hyperref} 
\hypersetup{bookmarksnumbered,bookmarksopenlevel=5,bookmarksdepth=5,colorlinks=true,linkcolor=blue,citecolor=blue}
\usepackage[all]{hypcap}
\usepackage{memhfixc}
\usepackage{lscape}
\bibpunct[: ]{(}{)}{,}{a}{}{,}
\usepackage{tangutex2}

\newfontfamily\phon[Mapping=tex-text,Ligatures=Common,Scale=MatchLowercase]{Charis SIL} 
\newcommand{\ipa}[1]{{\phon#1}} %API tjs en italique
\newcommand{\ipapl}[1]{{\phon#1}} %API tjs en italique
 
\newcommand{\grise}[1]{\cellcolor{lightgray}\textbf{#1}}
\newfontfamily\cn[Mapping=tex-text,Scale=MatchUppercase]{SimSun}%pour le chinois
\newcommand{\zh}[1]{{\cn#1}}

\newcommand{\jg}[1]{\ipa{#1}\index{Japhug #1}}
\newcommand{\wav}[1]{#1.wav}
\newcommand{\tgf}[1]{\begin{tabular}{l}\mo{#1}\\{\tiny #1}\end{tabular}}

\XeTeXlinebreaklocale 'zh' %使用中文换行
\XeTeXlinebreakskip = 0pt plus 1pt %
 \newcommand{\ra}{$\Sigma_1$} 
\newcommand{\rc}{$\Sigma_3$} 
\newcommand{\ro}{$\Sigma$} 


 \sloppy
\begin{document} 


\title{Tangut morphology in comparative perspective}
\author{Guillaume Jacques}
\maketitle

\section*{Introduction}
 \citet{jacques14esquisse}

\section{Case markers and postpositions}
Tangut has a large number of postpositions, of which  a considerable number have cognates in modern West Gyalrongic languages, as shown in Table \ref{tab:postpositions} (\citealt{jacques17stau}). Kepping's (\citeyear[144-164]{kepping85}) description of the uses of these postpositions is still unsurpassed, but our improved knowledge of Gyalrongic languages allows new insights on the history of the system. 

\begin{table}[H]
\caption{Case markers and postpositions in Stau, Khroskyabs and Tangut}\label{tab:postpositions} \centering
\begin{tabular}{ll|ll|llllll}
\toprule
Stau && Khroskyabs && Tangut & \\
\hline
\ipa{-w} & \textsc{erg} &&& \mo{5880} \ipa{ŋwu²} & instrumental \\
\ipa{-j} & \textsc{gen} &\ipa{-ji} &\textsc{gen} &\mo{1139} \ipa{.jij¹} & accusative, genitive, dative\\
\ipa{-ʁa} & \textsc{all} & \ipa{-ʁɑ} & \textsc{loc} & \mo{5856} \ipa{ɣa²} & locative \\
\ipa{-tɕʰa} & \textsc{loc} &&& \mo{0089} \ipa{tɕʰjaa¹}  & on \\
\ipa{-kʰa} & \textsc{instr} &&& \mo{5993} \ipa{kʰa¹}  &in the middle of \\
\toprule
\end{tabular}
\end{table}

Two postpositions are used for core arguments: the agentive \mo{5604}\mo{5113} \ipa{dʑjɨ.wji¹} and the highly polyfunctional marker \mo{1139} \ipa{.jij¹}.

The agentive in Tangut is obviously grammaticalized from a phrase meaning `doing the action' (\citealt[145]{kepping85}, \citealt{jacques14ergative}). Examples such as \refb{ex:tg:pilier} are very rare in texts. It has no cognates in the modern languages; Gyalrongic languages either have an ergative marker borrowed from Tibetan (on which see \citealt{jacques16comparative}), or a suffix cognate with the instrumental \mo{5880} \ipa{ŋwu²} discussed below.

 
\begin{exe}
\ex \label{ex:tg:pilier}   
\glt \tgf{2736} 	\tgf{3628} \tgf{5604} 	\tgf{5113} 	\tgf{4880} 	\tgf{4399} 	\tgf{2590} 	\tgf{5755} 	\tgf{0749} \\
\gll   \ipa{biaa²ɣjwã¹}	\ipa{dʑjɨ.wji¹}	\ipa{rər²}	\ipa{dzjị²}	\ipa{.wjɨ²-.jar¹-phji¹} \\
		Ma.Yuan	\textsc{agentive}		bronze	pillar	\textsc{pfv}-stand.up-\textsc{caus}[A] \\
\glt `Ma Yuan had bronze pillars erected.' (Leilin 04.34B.7)
\end{exe}

The postposition  \mo{1139} \ipa{.jij¹} has no less than three main functions (\citealt[145-7]{kepping85}).

Genitive Object Recipient

\citet[147]{kepping85} also notices a pair of examples such as \ref{ex:jij.subj} where 
\mo{1139} \ipa{.jij¹} appears on a pronoun or a noun phrase referring to the subject of an intransitive verb.

 \begin{exe}
\ex \label{ex:jij.subj}
\glt \tgf{4028} 	\tgf{1139} 	\tgf{4274} 	\tgf{4250} 	\tgf{0151} 	\tgf{5399} 	\tgf{3099} 	\tgf{4508} 	\tgf{5981} 	\tgf{4670} 	\tgf{2590} 	\tgf{4658} 	\tgf{0749} 	\tgf{2098} 	
\gll\ipa{nji²} 	\ipa{.jij¹} 	\ipa{sow¹-sji¹} 	\ipa{ɕjuu¹} 	\ipa{khju¹} 	\ipa{dʑjiij¹} 	\ipa{tjị} 	\ipa{.a-tsjwu¹} 	\ipa{.wjɨ²-thji¹-phji¹-ŋa²} 	\\
\textsc{2sg:hon} \textsc{gen} mulberry-tree shady.place under remain food one-\textsc{cl} \textsc{imp}-ingest-\textsc{caus-1sg} \\
\glt `You, who are sitting in the shade of this mulberry, give me some food to eat.' (Leilin 06.01B.4)
\end{exe}
 
 \begin{exe}
\ex 
\glt  \tgf{3951} 	\tgf{2946} 	\tgf{5037} 	\tgf{5880} 	\tgf{0764} 	\tgf{3990} 	\tgf{4342} 	\tgf{0390} 	\tgf{5113} 
\gll \ipa{thu¹ko¹} 	\ipa{bjɨr¹-ŋwu²} 	\ipa{rjijr¹} 	\ipa{khjɨ¹} 	\ipa{dja²-khjwɨ¹-.wji¹}  \\
Tugang sword-\textsc{instr} horse foot \textsc{pfv}-cut-do[A] \\
\glt `Tugang chopped the horse's foot with his sword.' (Leilin 03.24A.6)
\end{exe}


 \section{Orientation prefixes}
 
 

 \section{Person indexation}

\subsection{Person suffixes}

\citet{jacques16th}
\subsection{Stem alternation}

\citet{gong01huying}
\citet{jacques09tangutverb}
\citet{gong16stems}


 

\begin{exe}
\ex 
\glt \tgf{2975}	\tgf{3819}	\tgf{4342}	\tgf{1105}	\tgf{3621}	\tgf{0481}	\tgf{4517}	\tgf{2104}	\tgf{1473}	\tgf{5981}	\tgf{3506}	\tgf{3621}
\gll \ipa{tsjiir¹.lu²}	\ipa{dja²-khjow¹-.wjo²}	\ipa{.jir¹}	\ipa{dzji¹}	\ipa{ɕji¹}	\ipa{su¹}	\ipa{.a-bjịj¹-.wjo¹} \\
rank \textsc{imp}-give-do[B] emolument eat before \textsc{comp} \textsc{imp}-raise-do[B] \\
\glt `Give (Mengchangjun) back his rang, and raise his emolument higher than before!" (\citealt[40;189]{solonin95})
\end{exe}

 \section{Modality and evidentiality}
 
 \citet{jacques09tangutverb}
 \ipa{ə-tɕə-jə-rə-mɔ}
 

 \section{Traces of derivational morphology}

\subsection{Anticausative}

\subsection{Sibilant causative}

\subsection{Labial causative}

\subsection{Abilitative}

\subsection{Nominalization}

\phon
\bibliographystyle{unified}
\bibliography{bibliogj}
\end{document}
