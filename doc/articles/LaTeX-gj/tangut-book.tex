\documentclass[oneside,a4paper,11pt]{article} 
\usepackage{fontspec}
\usepackage[CJK, overlap]{ruby}
\usepackage{natbib}
\usepackage{booktabs}
\usepackage{xltxtra} 
\usepackage{lineno}
\usepackage{polyglossia}
\usepackage[table]{xcolor}
\usepackage{gb4e} 
\usepackage{multicol,multirow}
\usepackage{graphicx}
\usepackage{float}
\usepackage{slashbox} 
\usepackage{rotating}
\usepackage{hyperref} 
\hypersetup{bookmarksnumbered,bookmarksopenlevel=5,bookmarksdepth=5,colorlinks=true,linkcolor=blue,citecolor=blue}
\usepackage[all]{hypcap}
\usepackage{memhfixc}
%\usepackage{lscape}
%\bibpunct[: ]{(}{)}{,}{a}{}{,}
\usepackage{tangutex2}
\usepackage{lineno}

\newfontfamily\phon[Mapping=tex-text,Ligatures=Common,Scale=MatchLowercase]{Charis SIL} 
\newcommand{\ipa}[1]{{\phon#1}} %API tjs en italique
\newcommand{\ipapl}[1]{{\phon#1}} %API tjs en italique
 
\newcommand{\grise}[1]{\cellcolor{lightgray}\textbf{#1}}
\newfontfamily\cn[Mapping=tex-text,Scale=MatchUppercase]{SimSun}%pour le chinois
\newcommand{\zh}[1]{{\cn#1}}
\newcommand{\refb}[1]{(\ref{#1})}
\newcommand{\jg}[1]{\ipa{#1}\index{Japhug #1}}
\newcommand{\wav}[1]{#1.wav}
%\newcommand{\tgf}[1]{\begin{tabular}{l}\mo{#1}\\{\tiny #1}\end{tabular}}
\newcommand{\tgf}[1]{\ruby{{ \mo{#1}}}{#1}}
\XeTeXlinebreaklocale 'zh' %使用中文换行
\XeTeXlinebreakskip = 0pt plus 1pt %
 \newcommand{\ra}{$\Sigma_1$} 
\newcommand{\rc}{$\Sigma_3$} 
\newcommand{\ro}{$\Sigma$} 
 \newcommand{\bleu}[1]{{\color{blue}#1}}
\newcommand{\rouge}[1]{{\color{red}#1}} 
\renewcommand{\rubysep}{0.1ex}
 \sloppy
\begin{document} 
%\linenumbers

\title{Tangut morphology}
\author{Guillaume Jacques}
\maketitle

\section*{Introduction}
 \citet{jacques14esquisse}

\section{Case markers and postpositions}
Tangut has a large number of postpositions, of which  a considerable number have cognates in modern West Gyalrongic languages, as shown in Table \ref{tab:postpositions} (\citealt{jacques17stau}). Kepping's (\citeyear[144-164]{kepping85}) description of the uses of these postpositions is still unsurpassed, but our improved knowledge of Gyalrongic languages allows new insights on the history of the system. 

\begin{table}[H]
\caption{Case markers and postpositions in Stau, Khroskyabs and Tangut}\label{tab:postpositions} \centering
\begin{tabular}{ll|ll|llllll}
\toprule
Stau && Khroskyabs && Tangut & \\
\hline
\ipa{-w} & \textsc{erg} &&& \mo{5880} \ipa{ŋwu²} & instrumental \\
\ipa{-j} & \textsc{gen} &\ipa{-ji} &\textsc{gen} &\mo{1139} \ipa{.jij¹} & accusative, genitive, dative\\
\ipa{-ʁa} & \textsc{all} & \ipa{-ʁɑ} & \textsc{loc} & \mo{5856} \ipa{ɣa²} & locative \\
\ipa{-tɕʰa} & \textsc{loc} &&& \mo{0089} \ipa{tśʰjaa¹}  & on \\
\ipa{-kʰa} & \textsc{instr} &&& \mo{5993} \ipa{kʰa¹}  &in the middle of \\
\toprule
\end{tabular}
\end{table}


\subsection{Core arguments}
Two postpositions are used for core arguments: the agentive \mo{5604}\mo{5113} \ipa{dʑjɨ.wji¹} and the highly polyfunctional marker \mo{1139} \ipa{.jij¹}.

The agentive in Tangut is obviously grammaticalized from a phrase meaning `doing the action' (\citealt[145]{kepping85}, \citealt{jacques14ergative}). Examples such as \refb{ex:pilier} or \refb{ex:sha2} are very uncommon in texts. It has no cognates in the modern languages; Gyalrongic languages either have an ergative marker borrowed from Tibetan (on which see \citealt{jacques16comparative}), or a suffix cognate with the instrumental \mo{5880} \ipa{ŋwu²} discussed below.


\begin{exe}
\ex \label{ex:pilier}   
\glt \tgf{2736} \tgf{3628} \tgf{5604} \tgf{5113} \tgf{4880} \tgf{4399} \tgf{2590} \tgf{5755} \tgf{0749} 
\gll   \ipa{biaa²ɣjwã¹}	\ipa{dʑjɨ.wji¹}	\ipa{rər²}	\ipa{dzjị²}	\ipa{.wjɨ²-.jar¹-phji¹} \\
		Ma.Yuan	\textsc{agentive}		bronze	pillar	\textsc{pfv}-stand.up-\textsc{caus}[A] \\
\glt `Ma Yuan had bronze pillars erected.' (Leilin 04.34B.7)
\end{exe}

The postposition  \mo{1139} \ipa{.jij¹} has no less than three main functions (\citealt[145-7]{kepping85}): it marks possession (genitive), direct objects and recipients of ditransitive verbs. 

Both nouns (example \ref{ex:bigan}) and pronouns (\ref{ex:tji.sja}) use the postposition  \mo{1139} \ipa{.jij¹} to express possession. The same is true in modern languages such as Stau, where (with the exception of \textsc{1sg} and \textsc{2sg}) the genitive of all pronouns is made by regularly adding the \ipa{-j} to the base as with nouns (\citealt{jacques17stau}).

\begin{exe}
\ex \label{ex:bigan}   
\glt  	\tgf{2228}	\tgf{0404}	\tgf{5646}	\tgf{3830}	\tgf{1139}	\tgf{5305}	\tgf{1543}	\tgf{0508}
\gll \ipa{pji¹kã¹}	\ipa{tśhjiw²}	\ipa{njij²}	\ipa{.jij¹}	\ipa{.wjij²}	\ipa{mjor¹}	\ipa{ŋwu²} \\
Bigan  Zhou king \textsc{gen/acc} uncle real be \\
\glt `Bigan was Zhou's paternal uncle.' (Leilin 03.14B.2)
\end{exe}

\begin{exe}
\ex \label{ex:tji.sja}   
\glt  \tgf{2248}	\tgf{2065}	\tgf{1139}	\tgf{2455}	\tgf{2129}	\tgf{0724}	\tgf{1734}	\tgf{4225}	\tgf{5113}	\tgf{4884}
\gll \ipa{gjɨ²mji²}	\ipa{.jij¹}	\ipa{gji²bjij²}	\ipa{njɨ²}	\ipa{tji¹-sja¹-.wji¹-nji²} \\
\textsc{1pl} \textsc{gen/acc} wife \textsc{pl} \textsc{prohib}-kill-do-\textsc{1/2pl} \\
\glt `Don't kill our wives!' (Leilin 04.08B.2)
\end{exe}

Example \refb{ex:sha} illustrates the use of \mo{1139} \ipa{.jij¹} to mark recipients (with the verb `give' \mo{1105}/\mo{5644} \ipa{khjow¹}/\ipa{khjɨj¹}) and direct objects (of the verb \mo{4225} \ipa{sja¹} `kill'). While this postposition is obligatory for recipients, it is optional for objects, as shown by example \refb{ex:sha2}, where the object of the verb \mo{4225} \ipa{sja¹} `kill' does not take case marking. 

The exact conditioning for the presence of \mo{1139} \ipa{.jij¹} on objects has not yet been investigated in detail, and a typologically-informed perspective taking into account recent work on Differential Object Marking is a desideratum for future research on Tangut morphosyntax. 

\begin{exe}
\ex \label{ex:sha}   
\glt  \tgf{3508}	\tgf{1139}	\tgf{2806}	\tgf{5205}	\tgf{0804}	\tgf{5644}	\tgf{1746}	\tgf{3508}	\tgf{2541}	\tgf{0448}	\tgf{1139}	\tgf{4225}	\tgf{4481}	\tgf{2098}
\gll \ipa{bji²}	\ipa{.jij¹}	\ipa{zur²-ɣạ¹}	\ipa{djɨ²-khjɨj¹}	\ipa{ljoor¹-bji²}	\ipa{dzjwo²} \ipa{gjɨ²}	\ipa{.jij¹}	\ipa{sja¹}	\ipa{ɕjɨ¹-ŋa²} \\
official \textsc{gen/acc} edict-sword \textsc{imp}-give[B] treacherous-official man one \textsc{gen/acc} kill go-\textsc{1sg} \\
\glt `Give to your official (me) the imperial sword, I will go to kill a treacherous official.' (03.07A.5)
\end{exe}

\begin{exe}
\ex \label{ex:sha2}   
\glt \tgf{5306}	\tgf{3894}	\tgf{0510}	\tgf{5549}	\tgf{2221}	\tgf{1449}	\tgf{5604}	\tgf{5113}	\tgf{2440}	\tgf{1421}	\tgf{2541}	\tgf{0448}	\tgf{4342}	\tgf{4225}	\tgf{0433}	\tgf{0510}	\tgf{5549}	\tgf{2862}	\tgf{5447}	\tgf{1326}	\tgf{0989}	
\gll \ipa{dzjwɨ¹}	\ipa{njɨ¹}	\ipa{ŋwər¹tśhjiw²}	\ipa{.wə̣¹}	\ipa{tśhjwor²}	\ipa{dʑjɨ.wji¹}	\ipa{njɨɨ²zjɨ̣¹}	\ipa{dzjwo²}	\ipa{gjɨ²}	\ipa{dja²-sja¹}	\ipa{bju¹}	\ipa{ŋwər¹tśhjiw²} \ipa{nji¹-do²}	\ipa{kjɨ¹-piəj²}	 \\
emperor aunt princess belonging servant \textsc{agent} daytime man one \textsc{pfv}-kill \textsc{conj} princess house-\textsc{all} \textsc{pfv}-flee \\
\glt `A servant of the emperor's aunt, the princess, had killed a man in daytime, and had fled and sought refuge at her home.' (03.08A.3)
\end{exe}
%05.19B.6


\citet[147]{kepping85} also noticed a pair of examples such as \refb{ex:jij.subj} where \mo{1139} \ipa{.jij¹} appears on a pronoun or a noun phrase referring to the subject of an intransitive verb. Other examples such as \refb{ex:meesjij} are attested in the corpus. Such examples are rare, and are all found in reported speech.

 \begin{exe}
\ex \label{ex:jij.subj}
\glt \tgf{4028} 	\tgf{1139} 	\tgf{4274} 	\tgf{4250} 	\tgf{0151} 	\tgf{5399} 	\tgf{3099} 	\tgf{4508} 	\tgf{5981} 	\tgf{4670} 	\tgf{2590} 	\tgf{4658} 	\tgf{0749} 	\tgf{2098} 	
\gll\ipa{nji²} 	\ipa{.jij¹} 	\ipa{sow¹-sji¹} 	\ipa{ɕjuu¹} 	\ipa{khju¹} 	\ipa{dʑjiij¹} 	\ipa{tjị} 	\ipa{.a-tsjwu¹} 	\ipa{.wjɨ²-thji¹-phji¹-ŋa²} 	\\
\textsc{2sg:hon} \textsc{gen} mulberry-tree shady.place under remain food one-\textsc{cl} \textsc{imp}-ingest-\textsc{caus-1sg} \\
\glt `You, who are sitting in the shade of this mulberry, give me some food to eat.' (Leilin 06.01B.4)
\end{exe}

 \begin{exe}
\ex \label{ex:meesjij}
\glt \tgf{2019}	\tgf{0433}	\tgf{0510}	\tgf{5306}	\tgf{1139}	\tgf{3294}	\tgf{1771}	\tgf{0508}	\tgf{4601}	\tgf{1416}	\tgf{2098}
\gll \ipa{thja¹}	\ipa{bju¹}	\ipa{ŋwər¹dzjwɨ¹}	\ipa{.jij¹}	\ipa{mee²sjịj²}	\ipa{ŋwu²-nja²}	\ipa{djwu²-ŋa²} \\
this because emperor \textsc{gen/acc} wise.man be-\textsc{2sg} know-1sg \\
\glt `For this reason, I know that you, (my Emperor), are a wise man.' (Leilin 03.11A.1)
\end{exe}

 \subsection{Instrument}
  \citealt[161-2]{kepping85}
 \refb{ex:bjir.ngwu} \refb{ex:hongnong}
 \begin{exe}
\ex \label{ex:bjir.ngwu}
\glt  \tgf{3951} 	\tgf{2946} 	\tgf{5037} 	\tgf{5880} 	\tgf{0764} 	\tgf{3990} 	\tgf{4342} 	\tgf{0390} 	\tgf{5113} 
\gll \ipa{thu¹ko¹} 	\ipa{bjɨr¹-ŋwu²} 	\ipa{rjijr¹} 	\ipa{khjɨ¹} 	\ipa{dja²-khjwɨ¹-.wji¹}  \\
Tugang sword-\textsc{instr} horse foot \textsc{pfv}-cut-do[A] \\
\glt `Tugang chopped the horse's foot with his sword.' (Leilin 03.24A.6)
\end{exe}

 \subsection{Spatial postpositions}
  \mo{5856} \ipa{ɣa²}   \citealt[157-9]{kepping85}
  \mo{0089} \ipa{tśʰjaa¹}    \citealt[155-6]{kepping85}
 \mo{5993} \ipa{kʰa¹}   \citealt[156-7]{kepping85}
 
 
 \section{Orientation prefixes}
 \begin{table}[H]
\caption{Orientation prefixes in Tangut}\label{tab:orientation} \centering
\begin{tabular}{lllllllll} 
\toprule
\multicolumn{3}{c}{Series A} &\multicolumn{3}{c}{Series B} & orientation \\
\midrule
 \tgf{5981} & \ipa{.a} & &  \tgf{3989} & \ipa{.jij} & 1.36&up \\
\tgf{1452} & \ipa{nja} & 1.20&  \tgf{3846} & \ipa{njij} & 2.33&down \\
 \tgf{1326} & \ipa{kjɨ} & 1.30&  \tgf{2219} & \ipa{kjij} & 1.36&cislocative\\
 \tgf{2590} & \ipa{.wjɨ} & 2.27&  \tgf{2536} & \ipa{.wjij} & 2.32&translocative\\
 \tgf{4342} & \ipa{dja} & 2.17&  \tgf{4841} & \ipa{djij} & 2.33&loss \\
 \tgf{0804} & \ipa{djɨ} & 2.28&  \tgf{4841} & \ipa{djij} & 2.33&gain \\
 \tgf{0795} & \ipa{rjɨr} & 2.77&  \tgf{3706} & \ipa{rjijr} & 2.68& neutral \\
\bottomrule
\end{tabular}
\end{table}

\citet[216]{kepping85} et \citet[94]{lifw99bijiao} 
 

 \section{Person indexation}
Person indexation in Tangut is marked by a combination of suffixes and verb stem alternation. The suffixal system was first discovered by  \citet{kepping75agreement, kepping85} while that of stem alternation, while suggested by \citet{nishida75}, was only correctly described by \citet{gong01huying}.

\subsection{Person suffixes}
As shown by \citet{ kepping85}, Tangut has three person indexation suffixes, which resemble some of the pronouns (see Table \ref{tab:pronoms.suffixes}).  

\begin{table}[H]
\caption{Pronouns and person suffixes in Tangut}\label{tab:pronoms.suffixes} \centering
\begin{tabular}{llllll} 
\toprule
\multicolumn{3}{c}{Pronoun} &\multicolumn{3}{c}{Suffix} \\
\midrule
\tgf{2098} & \ipa{ŋa²}  & 1\textsc{sg} & \tgf{2098} & \ipa{ŋa²}  &1\textsc{sg} \\
\tgf{3926} & \ipa{nja²} & 2\textsc{sg} & \tgf{4601} & \ipa{nja²} &2\textsc{sg} \\
\tgf{4028} &  \ipa{nji²} & 2\textsc{sg}  honorific or 2\textsc{pl} & \tgf{4884} & \ipa{nji²} & 1\textsc{pl} and 2\textsc{pl} \\
\bottomrule
\end{tabular}
\end{table}

The correspondence  between suffixes and pronouns is however not one-to-one, since on the one hand \mo{2098} \ipa{ŋa²}  is not the only \textsc{1sg} pronoun (we also find \mo{0261} \ipa{mjo²}, as in \ref{ex:mjo}), and on the other hand \mo{4028} \ipa{nji²}  is used with the suffix \mo{4601} \ipa{-nja²} rather than with \mo{4884} \ipa{-nji²} (example \ref{ex:mjiij.nja}).

\begin{exe} 
\ex \label{ex:mjo}
\glt	\tgf{0261} \tgf{1326} 	\tgf{4941} \tgf{2098} 
\gll   \ipa{mjo²}	\ipa{kjɨ¹-tjọ¹-ŋa²} \\
		\textsc{1sg}	\textsc{pfv}-ferment[B]-\textsc{1sg} \\
\glt `It is I who fermented (the alcohol).' (Cixiaozhuan 5.1, \citealt[18]{jacques07textes})
\end{exe}
 
\begin{exe}
\ex \label{ex:mjiij.nja}  
\glt \tgf{4028}  \tgf{2219} \tgf{1374} \tgf{0330} \tgf{4601}  
\gll   \ipa{nji²} \ipa{kjij¹-tśhjɨ¹-mjiij¹-nja²}  \\
		\textsc{2sg:hon} \textsc{irr-opt}-dream-\textsc{2sg} \\
\glt  `Did you have a dream?' (Leilin 6.16B.4)
\end{exe}
 
\subsection{Stem alternation and the indexation system}
A closed class of transitive verbs in Tangut have two stems, called A and B after \citet{gong01huying}. 

\begin{exe}
\ex \label{ex:eat.21sg}  
\glt	\tgf{2447} \tgf{1519} 	\tgf{5165} 	\tgf{2590} \tgf{4517} 	\tgf{2098} 
\gll   \ipa{ljo²}	\ipa{ɣu¹twụ¹}	\ipa{wjɨ²-dzji¹-ŋa²} \\
		elder.brother instead \textsc{imp}-eat[A]-\textsc{1sg} \\
\glt `Eat me instead of my brother!' (Cixiaozhuan 17.7, \citealt[55-6]{jacques07textes})
\end{exe}

\citet{jacques09tangutverb}
\citet{gong16stems}

\begin{table}[H]
\caption{Attested forms of the transitive paradigm in Tangut}\centering  \label{tab:paradigm}
\begin{tabular}{lllll}
\toprule
	&	\textsc{1sg}	&	\textsc{2sg}	&	1/\textsc{2pl}	&	3	\\
	\midrule
\textsc{1sg}&	?	&	A-\ipa{nja²}	&	?	&	 B-\ipa{ŋa²}	\\
\textsc{2sg}		&	A-\ipa{ŋa²}	&	B-\ipa{nja²}	&	A-\ipa{nji²}	&	 B-\ipa{nja²}	\\
1/\textsc{2pl}	&	 A-\ipa{ŋa²}	& ?	&	?	&	A-\ipa{nji²}	\\
3	&	A-\ipa{ŋa²}	&	A-\ipa{nja²}	&	?	&	A 	\\
\bottomrule
\end{tabular}
\end{table}

\subsection{Is person indexation optional?}

\citet{jacques16th}
\refb{ex:dja.khjow}
\refb{ex:sha} above (reproduced here as \ref{ex:sha3}).
 \begin{exe}
\ex \label{ex:sha3}   
\glt  \tgf{3508}	\tgf{1139}	\tgf{2806}	\tgf{5205}	\tgf{0804}	\tgf{5644}	
\gll \ipa{bji²}	\ipa{.jij¹}	\ipa{zur²-ɣạ¹}	\ipa{djɨ²-khjɨj¹}	\\
 official \textsc{gen/acc} edict-sword \textsc{imp}-give[B] \\
\glt `Give to your official (me) the imperial sword.' (03.07A.5)
\end{exe}


\begin{exe}
\ex \label{ex:dja.khjow}
\glt \tgf{3101} \tgf{2975}	\tgf{3819}	\tgf{4342}	\tgf{1105}	\tgf{3621}	\tgf{0481}	\tgf{4517}	\tgf{2104}	\tgf{1473}	\tgf{5981}	\tgf{3506}	\tgf{3621} 
\gll \ipa{.jị²} \ipa{tsjiir¹.lu²}	\ipa{dja²-khjow¹-.wjo¹}	\ipa{.jir¹}	\ipa{dzji¹}	\ipa{ɕji¹}	\ipa{su¹}	\ipa{.a-bjịj¹-.wjo¹} \\
again rank \textsc{imp}-give-do[B] emolument eat before \textsc{comp} \textsc{imp}-raise-do[B] \\
\glt `Give (Mengchangjun) back his rang, and raise his emolument higher than before!" (\citealt[40;189]{solonin95})
\end{exe}



\begin{exe}
\ex \label{ex:hongnong}
\glt \tgf{4028} 	\tgf{2104} 	\tgf{2780}\tgf{5267}\tgf{3831} 	\tgf{5113} 	\tgf{4861} 	\tgf{2302}-\tgf{5880} 	\tgf{4408} 	\tgf{4663}\tgf{0749} 	\tgf{2503} 	\tgf{1402}\tgf{3567} 	\tgf{0289}\tgf{3266} 	\tgf{0795}\tgf{5113} 	\tgf{1477} 	\tgf{4039} 	\tgf{5880} 	\tgf{4342}\tgf{2511}\tgf{4601}\tgf{3916} 	\tgf{3583} 	\tgf{5688} 	\tgf{0290} 
\gll \ipa{nji²} 	\ipa{ɕji¹} 	\ipa{kiow²ljɨj¹}\ipa{ljij²} 	\ipa{.wji¹} 	\ipa{zjọ²} 	\ipa{ljɨ¹}-\ipa{ŋwu²} 	\ipa{məə¹} 	\ipa{lha¹}-\ipa{phji¹} 	\ipa{kụ¹} 	\ipa{xũ¹}\ipa{luu²} 	\ipa{.we²}-\ipa{dzju²} 	\ipa{rjɨr²}-\ipa{.wji¹} 	\ipa{le²} 	\ipa{0} 	\ipa{ŋwu²} 	\ipa{dja²}-\ipa{rjɨr²}-\ipa{nja²}-\ipa{sji²} 	\ipa{tja¹} 	\ipa{.wa²} 	\ipa{sju²}  \\
\textsc{2sg} first Jiangling.Ling \rouge{make[A]} time wind-\textsc{instr} fire extinguish-\rouge{cause[A]} after Hongnong city-lord \textsc{pfv}-\rouge{make[A]} tiger \rouge{ride} \textsc{instr} \bleu{\textsc{pfv}-leave-\textsc{2sg-ifr}} \textsc{top} what be.like \\
\glt `First, when you were Jiangling Ling, you extinguished the fire with the wind, later, when you became Taishou of Hongnong, you went away riding a tiger. How did (you do that)?' (Leilin, 04.13B.4)
\end{exe} %卿初为江陵令,以风灭火;后为弘农太守,乘虎而去者,是因何德也?

 \section{Modality and evidentiality}
 
 \citet{jacques09tangutverb}
 \ipa{ətɕə} \ipa{jə-rə-mɔ}
 

 \section{Traces of derivational morphology}

\subsection{Anticausative}

\subsection{Sibilant causative}

\subsection{Labial causative}

\subsection{Abilitative}

\subsection{Nominalization}

\phon
\bibliographystyle{unified}
\bibliography{bibliogj}
\end{document}
