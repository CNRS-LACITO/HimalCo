\documentclass[oldfontcommands,oneside,a4paper,11pt]{article} 
\usepackage{fontspec}
\usepackage{natbib}
\usepackage{bibentry}
\usepackage{booktabs}
\usepackage{xltxtra} 
\usepackage{polyglossia} 
\setdefaultlanguage{french} 
\usepackage[table]{xcolor}
\usepackage{gb4e} 
\usepackage{graphicx}
\usepackage{float}
\usepackage{hyperref} 
\hypersetup{bookmarks=false,bookmarksnumbered,bookmarksopenlevel=5,bookmarksdepth=5,xetex,colorlinks=true,linkcolor=blue,citecolor=blue}
\usepackage[all]{hypcap}
\usepackage{memhfixc}

\bibpunct[: ]{(}{)}{,}{a}{}{,}
 
\setmainfont[Mapping=tex-text,Numbers=OldStyle,Ligatures=Common]{Charis SIL}  

\newfontfamily\phon[Mapping=tex-text,Ligatures=Common,Scale=MatchLowercase,FakeSlant=0.3]{Charis SIL} 
\newcommand{\ipa}[1]{{\phon #1}} %API tjs en italique
 \newcommand{\ipapl}[1]{{\phon #1}} %API tjs en italique
\newcommand{\grise}[1]{\cellcolor{lightgray}\textbf{#1}}
\newfontfamily\cn[Mapping=tex-text,Ligatures=Common,Scale=MatchUppercase]{MingLiU}%pour le chinois
\newcommand{\zh}[1]{{\cn #1}}




 
\begin{document}


  \title{Rapport d'activité (vague)}
 
\author{Guillaume Jacques, CNRS-INALCO-EHESS (CRLAO)}
\maketitle
\sloppy
\bibliographystyle{unified}
\nobibliography{bibliogj.bib}

\section{Curriculum vitae}
\begin{itemize}
\item 2015 Directeur de recherches (DR2), CNRS, Centre de recherches linguistiques sur l'Asie Orientale (UMR 8563), section 34.
\item 2014 Habilitation à diriger les recherches: \textit{Esquisse de phonologie et de morphologie historique du tangoute}, INALCO  (directeur: Laurent Sagart; jury: Alexander Vovin, Konstantin Pozdniakov, Boyd Michailovsky, Nathan W. Hill, Gilles Authier)
\item Depuis 2009 Chargé de recherche au CNRS (CR1), Centre de recherches linguistiques sur l'Asie Orientale (UMR 8563), section 34.
\item 2005-2009 Maître de conférences, Université Paris Descartes, département de sciences du langage.
\item 2004-2005 Postdoctorant, Fondation Chiang Ching-Kuo.
\item 2001-2004 Doctorat de linguistique: \textit{Phonologie et morphologie du japhug (rGyalrong)} Université Paris 7 (directeur: Marie-Claude Paris; jury: Laurent Sagart, Boyd Michailovsky, Jackson T.-S. Sun, Nicolas Tournadre)
\end{itemize}
\section{Recherche scientifique}

Mes recherches des cinq derniers semestres ont porté sur quatre thématiques principales, intimement liées les unes aux autres:

\begin{itemize}
\item La description de plusieurs langues à tradition orale de la famille sino-tibétaines: le japhug (Chine, Sichuan, Mbarkham), le stau (Chine, Sichuan, Rtau) et le khaling (Népal, Solukhumbu). 
\item Typologie morphosyntaxique
\item Linguistique historique et principes généraux des changements linguistiques (sino-tibétain, sioux, algonquien)
\item Modélisation informatique de la morphologie.
\end{itemize}

Dans cette section, je présente les publications parues et en cours et les résultats liées à chacune de ces thématiques, une liste complète des publications parues durant la période 2013-2017, une présentation des corpus de données sur la base desquelles mes recherches sont menées, et enfin les distinctions que j'ai reçues durant cette période.

\subsection{Description des langues}
De 2013 à 2017, J'ai effectué les séjours de terrain suivants:

\begin{itemize}
%\item février-mars 2012: Népal (khaling)
%\item juillet-août 2012: Chine, Sichuan (japhug et situ)
\item février 2013: Népal (khaling)
\item août 2014: Chine, Sichuan (japhug et situ)
\item février 2015: Népal (khaling)
\item août-septembre 2015: Chine, Sichuan (japhug)
\item août 2016: Chine, Sichuan (japhug)
\end{itemize}

Par ailleurs, j'ai travaillé sur le stau avec un locuteur habitant à Paris, en collaboration avec Lai Yunfan (un doctorant) et Anton Antonov.


En outre, j'ai effectué des recherches philologique sur le tangoute, langue ancienne de la Chine du nord disparu depuis le 16ème siècle.

\subsubsection{Japhug} \label{sec:japhug}
La description du japhug comprend l'essentiel de mon temps de recherche. Les résultats principaux de ma recherche dans ce domaine se trouvent dans les dix articles suivants, parus en partie dans des revues aréales (LTBA) mais également dans des revues de linguistique plus généralistes (en particulier \textit{Lingua}, \textit{Linguistic Typology}, \textit{Diachronica}). Chacune de ces publications a deux objectifs. D'une part, elles apportent de nouvelles données originales et pourront, en rajoutant des données complémentaires, être réutilisées pour la rédaction de ma grammaire du japhug en cours (\citet{jacques14linking} notamment est très riche en données et fait plus de 60 pages). D'autre part, elles constituent chacune une contribution distincte à la typologie morphosyntaxique (voir section \ref{sec:typologie}).

L'essentiel de la grammaire du japhug est presque entièrement couverte par ces articles; en plus des articles ci-dessous, un article sur l'évidentialité (\citealt{jacques19egophoric}), un autre sur le verb bipartites (\citealt{jacques18bipartite}) et une illustration of the IPA (\citealt{jacques18ipa}) sont en cours de parution.

\begin{enumerate}
\item {\bibentry{jacques17comitative}}
\item {\bibentry{jacques17num}}
\item {\bibentry{jacques16complementation}}
\item {\bibentry{jacques16comparative}}
\item {\bibentry{jacques16relatives}}
\item {\bibentry{jacques15spontaneous}}
\item {\bibentry{jacques15causative}}
\item {\bibentry{jacques14linking}}
\item {\bibentry{jacques14antipassive}}
\item {\bibentry{japhug14ideophones}}
\item  \bibentry{jacques13tropative}
\item {\bibentry{jacques13harmonization}}   
%\item{ \bibentry{jacques12incorp}}
%\item \bibentry{jacques12agreement}  
\end{enumerate} 

Ces articles ne représentent que la partie émergée de l'iceberg: l'essentiel de mon travail sur le japhug porte sur l'analyse et la mise en forme de mon corpus de textes (dont déjà 40h ont été archivées sur Pangloss) et aussi la compilation d'un dictionnaire japhug-chinois-français parlant (avec fichier sons). 

Le corpus de textes est mis progressivement en ligne, et le dictionnaire est accessibles en plusieurs versions (PDF, HTML, android) sur himalco.huma-num.fr et sur l'archive Pangloss.


\subsubsection{Khaling} \label{sec:khaling}
L'étude du khaling, commencée seulement en 2011 en collaboration avec Aimée Lahaussois et Boyd Michailovsky, a donné lieu pour le moment à plusieurs articles, dont les suivants depuis 2013:
 
\begin{enumerate}
\item {\bibentry{jacques16tonogenesis}}
\item {\bibentry{jacques16si}}
\item {\bibentry{jacques15derivational.khaling}}
\item {\bibentry{jacques14auditory}}
% \item  {\bibentry{jacques12khaling}}
\end{enumerate} 

Un dictionnaire des verbes khaling (khaling-népali-anglais) accompagné de fichiers sons, avec les paradigmes automatiquement générés par un script que j'ai écrit, a été publié sur le site  himalco.huma-num.fr en 2015, une version papier a été publiée au Népal en 2016, et une application android du dictionnaire accompagnée des fichiers son est sur google play.

\subsubsection{Stau}
L'étude du stau, en collaboration avec Anton Antonov et Lai Yunfan, a donné lieu à un article et un chapitre de livre:

\begin{enumerate}
\item {\bibentry{jacques17stau}}
\item {\bibentry{jacques14rtau}}
\end{enumerate} 

Un glossaire et une douzaine d'histoires ont été recueillies.

\subsubsection{Tangoute}
Mon étude du tangoute a donné lieu à un livre (\citealt{jacques14esquisse}) sur la phonologie historique, un article (\citealt{jacques14ergative}) sur le fonctionnement de l'ergatif dans cette langue et son origine historique (voir section \ref{sec:linghist}) et un autre sur l'indexation personnelle  (\citealt{jacques16th}).

\begin{enumerate}
 \item  \bibentry{jacques16th}
 \item  \bibentry{jacques14esquisse}
 \item {\bibentry{jacques14ergative}}
\end{enumerate} 
 

\subsubsection{Autres familles de langue}
Bien que mes travaux de terrain portent exclusivement sur des langues de la famille sino-tibétaine, j'ai également un intérêt pour les familles sioux (omaha, lakota), algonquiennes (ojibwe, arapaho) et indo-européennes (sanskrit)  et effectue des recherches typologiques utilisant les données de ces langues sur la base d'une étude des corpus de textes, et pas uniquement en utilisant des données de seconde main.

\subsection{Typologie} \label{sec:typologie}

Mon travail en typologie a porté sur trois thématiques principales: indexation de personne et voix, évidentialité et universaux.

\subsubsection{Indexation, voix et alignement}
Le japhug, langue polysynthétique avec accord polypersonnel, très riche en marque de voix, m'a permis d'écrire des articles contribuant à la typologie morphosyntaxique sur les systèmes direct-inverses (\citealt{jacques14inverse}), l'antipassif (\citealt{jacques14antipassive}), l'applicatif et le tropatif (\citealt{jacques13tropative}). Ces travaux contribuent également à la linguistique historique dans une perspective générale (voir section  \ref{sec:linghist}).

A ces articles doivent s'ajouter \citet{jacques15causative} et  \citet{jacques15derivational.khaling} sur le causatif et l'abilitatif en japhug et en sino-tibétain d'une part et l'applicatif/causatif en khaling d'autre part.

\begin{enumerate}
\item  \bibentry{jacques15causative}
\item  \bibentry{jacques15derivational.khaling}
\end{enumerate} 

Dans \citet{jacques16relatives}, je me suis intéressé aux pivots syntaxiques en japhug, et montre que l'on trouve aussi bien des pivots accusatifs, ergatifs, neutres et d'autres types moins facilement catégorisables dans cette langue.

\subsubsection{Evidentialité}\label{sec:evd}
Mon intérêt pour l'évidentialité a été éveillée par la découverte du démonstratif audition en khaling (\citealt{jacques14auditory}, voir section \ref{sec:khaling}, suite auquel j'ai écrit un chapitre d'encyclopédie (\citealt{jacques18nonpropositional}).

\begin{enumerate}
\item  \bibentry{jacques18nonpropositional}
\item  \bibentry{jacques14auditory}
\end{enumerate} 


J'ai également écrit un article sur l'évidentialité en japhug à paraître en 2018, où je montre qu'il s'agit d'une des seule langues au monde à avoir un système égophorique combiné avec un système d'indexation polypersonnel. Les données du japhug permettent d'éclairer le fonctionnement des systèmes égophoriques dans les autres langues sans marquage personnel, telles que les langues tibétaines, l'akhvakh ou les langues barbacoanes, en particulier dans le discours rapporté hybride.  

\subsubsection{Universaux}
Dans les articles \citet{jacques13harmonization} et \citet{antonov14need}, j'ai démontré l'invalidité de deux universaux, l'un portant sur la corrélation entre ordre des mots et affixation, l'autre sur la corrélation supposée entre la structure argumentale dans les  constructions possessives et les constructions modales exprimant la nécessité.

Je compte dans des travaux futurs continuer à travailler sur la question de l'harmonie trans-catégorielle, en utilisant les données du sioux, du chinois et de l'indo-iranien, qui présentent des types de disharmonisation distincts de ceux observés en japhug.


\subsection{Modélisation} \label{sec:modelisation}
Suite à l'article \citet{jacques12khaling}, article où j'avais moi-même mis au point un générateur de paradigmes, j'ai décidé de collaborer avec des linguistes informaticiens et morphologues à la modélisation informatique des paradigmes polypersonnels, commençant par celui du khaling. Dans \citet{walther14compactness}, nous avons comparé la compacité relative de deux analyses possible du système verbal du khaling, et montré qu'un analyse morphomique basée sur la notion de `direct/inverse' permettait une économie descriptive considérable.

\begin{enumerate}
 \item  \bibentry{walther14compactness}
  \item  \bibentry{jacques12khaling}
\end{enumerate}

\subsection{Linguistique historique} \label{sec:linghist}
Ma perspective d'étude en linguistique historique est panchronique; je m'intéresse à la fois à des cas particuliers d'évolutions phonétiques ou grammaticales, mais aussi et surtout aux principes généraux des changements linguistiques. 


\subsubsection{Sino-tibétain}
Toutes les langues sur lesquelles j'effectue du terrain et sur lesquelles je dispose de données de première main appartiennent à la famille sino-tibétaine,  l'une des familles de langue du monde ayant la plus grande diversité typologique.

Mes travaux dans ce domaine depuis 2013 ont porté sur la morphologie comparée (\citealt{jacques15causative, jacques16th}), la reconstruction du chinois archaïque (\citealt{jacques16ssuffixes, jacques17buyang}) et la phylogénie de la famille (\citealt{jacques17genetic}); voir la section \ref{sec:lois.phonetiques} sur les lois phonétiques nouvelles proposées  concernant des langues de la famille sino-tibétaine.
 
\begin{enumerate}
 \item {\bibentry{jacques17genetic}}
 \item {\bibentry{jacques17buyang}}
 \item {\bibentry{jacques16ssuffixes}}
 \item {\bibentry{jacques16th}}
  \item {\bibentry{jacques15causative}}
\end{enumerate}
 
\subsubsection{Lois phonétiques} \label{sec:lois.phonetiques}
Dans le domaine de l'étude des lois phonétiques, je me suis intéressé à un grand nombre de familles différentes. Ma contribution la plus importante dans ce domaine est mon ouvrage sur le tangoute (\citealt{jacques14esquisse}), où je propose plus d'une vingtaine de lois phonétiques nouvelles, en comparant le tangoute et les langues gyalronguiques modernes. Cet ouvrage fait référence aux textes originaux en tangoute et aux corpus de textes en japhug et en pumi que j'ai collecté, et propose des étymologies sur la base de l'usage des mots dans les textes, et non pas seulement de leur traduction dans les dictionnaires.  

Un article assez fourni (\citealt{jacques17pkiranti}) propose une nouvelle reconstruction du proto-kiranti en prenant en compte la morphologie.

J'ai également écrit des articles circonscrits sur des lois phonétiques particulières dans d'autres langues de la famille sino-tibétaine, en particulier le tibétain et le chinois archaïque (\citealt{jacques15sr}, \citealt{jacques14snom} et \citealt{jacques13yod}).

\begin{enumerate}
 \item {\bibentry{jacques17pkiranti}}
 \item {\bibentry{jacques15sr}}
 \item {\bibentry{jacques14snom}}
 \item {\bibentry{jacques14esquisse}}
 \item {\bibentry{jacques13yod}}
\end{enumerate}

Par ailleurs, j'ai un intérêt pour les changements linguistiques dans l'ensemble des langues du monde. J'ai écrit \citet{jacques13arapaho} qui présente une perspective panchronique pour expliquer le changement de *s à n en arapaho, ainsi que quelques nouvelles étymologies basées sur la comparaison de l'arapaho avec l'ojibwe.

\begin{enumerate}
 \item {\bibentry{jacques13arapaho}}
\end{enumerate}

Je compte continuer l'étude des lois phonétiques du groupe gyalronguique, en collaboration avec Gong Xun et Lai Yunfan (doctorants), et continuer d'écrire des articles ponctuels sur la phonologie/morphologie historique de diverses langues sino-tibétaines, en particulier le chinois et le tibétain.

Par ailleurs, j'ai le projet d'étudier certains changements typologiquement inhabituels, comme le passage des vélaires aux labiales dans un sous-groupe du salish, changement controversé sur lequel une étude plus détaillée est nécessaire.
         
\subsubsection{Grammaticalisation}
Dans \citet{jacques14antipassive} et \citet{jacques15causative}, je montre que les marques de voix en japhug proviennent de la réanalyse de préfixes dénominaux, et documente une origine jusqu'alors ignorée des marques de voix (en particulier l'antipassif, le causatif et l'applicatif), à savoir une grammaticalisation en deux étapes. Tout d'abord, le verbe est nominalisé et devoir un nom d'action. Ensuite, ce nom d'action subit une dérivation dénominale qui lui alloue une nouvelle valeur de transitivité. Bien que des cas partiellement similaires (provenant de prédicats complexes) aient déjà été mentionnés dans la littérature, le mécanisme attesté en japhug ne semble pas avoir d'équivalents exacts correctement documentés.

Dans \citet{jacques13harmonization}, je montre que les préfixes de mouvement associé proviennent de verbes de mouvement, mais que la construction de mouvement associé prend son origine non pas dans la construction à complément de but (`aller faire'), mais d'une construction en série ou paratactique (`aller et faire').

Je me suis également intéressé aux marques d'ergatif dans une perspective diachronique. Dans \citet{jacques14ergative}, je montre que la marque d'ergatif en tangoute vient de la forme converbiale du verbe "faire", une origine qui n'est pas mentionnée dans les travaux comparatifs sur la question. Dans \citet{jacques16comparative}, je présente les différentes grammaticalisations de la marque d'ergatif en japhug (empruntée au tibétain), notamment son usage étrange dans la construciton comparative, où elle sert à marque non pas le standard de comparaison (comme c'est habituel pour les marques ergatives), mais le comparé. Je propose un chemin de grammaticalisaiton pour expliquer cet usage très inhabituel.

\begin{enumerate}
 \item  \bibentry{jacques16comparative}
 \item {\bibentry{jacques14ergative}}
\end{enumerate}

\subsubsection{Contact de langue}
Dans l'article `Verbal valency and Japhug / Tibetan language contact', à paraître en 2019 au \textit{Journal of language contact}, j'étudie les correspondances entre les structures argumentales des verbes tibétains et ceux de leurs équivalents empruntés en japhug. Il s'agit d'un cas intéressant où la langue donneuse est essentiellement à marquage sur les dépendants tandis que la langue emprunteuse est à marquage sur la tête. Cette étude révèle qu'il est possible dans la majorité des cas de prédire la structure argumentale d'un verbe japhug emprunté au tibétain à partir de celle du verbe originel.

\subsubsection{Analogie}
Avec la régularité des changements phonétique, l'analogie est un des concepts fondamentaux de la méthode comparative. Occulté par la tradition de phonologie générative, il ne reste pas moins indispensable pour expliquer l'évolution des paradigmes grammaticaux. Néanmoins, les principes généraux des phénomènes analogiques sont mal connus, et les études classiques sur lesujet (Kurylowicz, Manczak) sont peu utile d'un point de vue pratique pour évaluer une hypothèse historique par rapport à un autre. 

Dans une série de travaux en cours de publication, dont \citet{jacques18directionality} et \citet{jacques16ebde}, je propose les règles suivantes:

\begin{enumerate}
\item L'analogie s'applique toujours suivant la hiérarchie 3>1>2 (les formes de troisième personne peuvent être la base de l'analogie pour celles de première et de deuxième personne, et les formes de première pour celles de deuxième), sauf dans descas où, pour certains verbes, la forme de 1er personne est plus courante (de face démontrable par une étude de corpus).
\item L'analogie affecte les formes du pluriel (et du duel) avant celles du singulier.
\item Les formes inverses (3>1, 3>2) peuvent être basées analogiquement sur les formes directes (1>3, 2>3), mais l'analogie ne fonctionne pas dans l'autre sens.
\end{enumerate}

\begin{enumerate}
 \item  \bibentry{jacques18directionality}
 \item {\bibentry{jacques16ebde}}
\end{enumerate}

Je compte poursuivre des travaux sur ce sujet, et essayer d'étudier les contre-exemples apparent aux principe n°1 ci-dessus (j'ai détecté des cas potentiels en yuto-aztèque et en indo-iranien), et à étudier la directionalité de l'analogie entre les catégories de TAM.

\subsection{Liste des publications principales}
Dans cette section, j'inclus les publications parues durant les cinq derniers semestres (soient celles de 2013 à 2017), excluant les publications acceptées, qui sont cependant incluses dans la liste complète ci-jointe et mentionnées ci-dessus.
%
%Je décris brièvement l'apport de chaque publication, en la replaçant dans le contexte plus général de mes recherches.



\subsubsection{Ouvrage de recherche}
\begin{enumerate}
 \item  \bibentry{jacques14esquisse}     
\end{enumerate}
  \subsubsection{Articles dans des revues à comité de lecture}
\begin{enumerate}
\item {\bibentry{jacques17buyang}}
\item {\bibentry{jacques17pkiranti}}
\item {\bibentry{jacques16comparative}}
\item {\bibentry{jacques16complementation}}
\item {\bibentry{jacques16ebde}}
\item {\bibentry{jacques16relatives}}
\item {\bibentry{jacques16th}}
\item {\bibentry{jacques15causative}}
\item {\bibentry{jacques15derivational.khaling}}
\item {\bibentry{jacques15spontaneous}}
\item {\bibentry{jacques15sr}}
\item {\bibentry{jacques14ergative}}
\item {\bibentry{jacques14rtau}}
\item {\bibentry{jacques14linking}}
\item {\bibentry{jacques14antipassive}}
\item {\bibentry{jacques14auditory}}
\item {\bibentry{antonov14need}}
\item \bibentry{jacques14inverse}  
\item {\bibentry{japhug14ideophones}}
\item  \bibentry{jacques13tropative}
\item {\bibentry{jacques13harmonization}}
\item   \bibentry{jacques13arapaho}   
\item   \bibentry{jacques13yod}   
%\item{ \bibentry{jacques12incorp}}
%\item \bibentry{jacques12agreement}  
%\item \bibentry{jacques12internal}  
%  \item  \bibentry{rg-gj12yod}
%\item  {\bibentry{michaud-jacques12nasalite}}
% \item  {\bibentry{jacques12khaling}}
 \end{enumerate}
 
   \subsubsection{Chapitres}
\begin{enumerate}
\item   \bibentry{jacques17genetic}  
\item   \bibentry{jacques17rgy}  
\item   \bibentry{jacques17traditional}  
\item   \bibentry{jacques17sketch}  
\item   \bibentry{jacques17comitative}   
\item   \bibentry{jacques17num}   
\item   \bibentry{jacques16tonogenesis}   
\item   \bibentry{jacques14cone}   
\item   \bibentry{jacques14snom}   
  \end{enumerate}
  \subsubsection{Présentations invitées}
\begin{enumerate}
\item {\bibentry{jacques15grammat}}
\item {\bibentry{jacques14converb}}
\item {\bibentry{jacques13alternations}}
\end{enumerate} 
\subsection{Corpus de données principaux}
\begin{itemize}
\item \textbf{Japhug}: 80 heures de textes (depuis 2002), dont 60 transcrites et 8 traduites (sans compter les phrases et les listes de mots). Dictionnaire de plus de 7000 entrées. Plus de 40h de textes archivés dans l'archive PANGLOSS.
\item \textbf{Khaling}: 2 heures de textes transcrits et traduits (depuis 2011). Dictionnaire de verbes (653 racines), avec conjugueur automatique.
\item \textbf{Stau}: 40 minutes de textes transcrits et traduits (depuis 2012). Dictionnaire de 1007 entrées.
  \end{itemize}
  
\subsection{Distinctions reçues}  
\begin{itemize}
\item 2015 Médaille de bronze du CNRS, section 34.
\end{itemize}
  
\section{Enseignement}
\begin{itemize}
\item 2014-présent Chargé de cours (INALCO): typologie et description morphosyntaxique.
\item 2014 école d'été du LACITO, Roscoff (\url{http://lacito.vjf.cnrs.fr/colloque/methodes/index\_en.htm})
\end{itemize}

\section{Transfert technologique, relations industrielles et valorisation}
2013-2015: Porteur du projet ANR Corpus \textbf{HimalCo} (en collaboration avec Alexis Michaud, Aimée Lahaussois et Séverine Guillaume) (\url{http://himalco.hypotheses.org/})

Ce projet a permis de développer:

\begin{enumerate}
\item Une librairie (Python) permettant la conversion du format MDF de dictionnaire vers d'autres formats (XML, HTML, \LaTeX, docx)
\item Une interface pour dictionnaires parlants en ligne; la première version de cette interface sera rendue publique en octobre 2015. Elle comprend pour le moment les dictionnairesde trois langues, japhug (G. Jacques), khaling (G. Jacques et A. Lahaussois) et na (A. Michaud).
\item Une interface de textes parallèles, testées sur les textes mythologiques kiranti (khaling, thulung et koyi).
\end{enumerate}

En outre, une quantité importante de fichiers sons d'histoires pourvus de transcriptions, sur toutes les langues étudiées par les membres du projets, sont rendus disponibles sur le site Pangloss.

\section{Encadrement, animation et management de la recherche}
\subsection{Encadrement d'étudiants}
Deux de mes doctorants ont soutenu leurs thèses:
\begin{itemize}
\item 2007-2015, doctorat de Gao Yang, EHESS, \textit{Description de la langue muya} (Sino-tibétain, birmo-qianguique, en co-direction avec Laurent Sagart).
\item 2013-2017, doctorat de Lai Yunfan, Paris 3, \textit{Grammaire du Kroskyabs} (Sino-tibétain, birmo-qianguique, en co-direction avec Pollet Samvelian)
\end{itemize}

\begin{itemize}
\item 2013-présent, doctorat de Gong Xun, Normale supérieure-INALCO, \textit{Etude descriptive et historique de la langue zbu} (Sino-tibétain, birmo-qianguique, en co-direction avec Laurent Sagart)
\item 2016-présent, thèse de Zhang Shufa sur le dialecte situ de Bragdbar., Inalco (en codirection avec Christine Lamarre)
\item 2016-présent, thèse de Julie Marsault sur l'omaha, Paris 3 (en co-direction avec Pollet Samvelian)
\item 2016-présent, thèse de Jesse Gates sur le Stau, EHESS (en co-direction avec Alexander Vovin)
\end{itemize}

Certains de mes étudiants ont déjà plusieurs publications, dont certaines sont déjà citées:

\begin{enumerate}
  \item {\bibentry{gong16amdo}}
  \item {\bibentry{gong16ld}}
  \item {\bibentry{gong16stems}}
  \item {\bibentry{gongxun14agreement}}
    \item {\bibentry{gong17xingtaixue}}
    \item {\bibentry{gong17clusters}}
 \item  \bibentry{lai16caus}
 \item  \bibentry{lai15person}
 \item {\bibentry{lai13fuyin}} 
\end{enumerate}

Mon interaction avec les étudiants ne s'est pas limité à une aide au choix du sujet, à la rédaction des mémoires et des articles. Dans le cas de Gong Xun et de Zhang Shuya, je les ai accompagnés sur le terrain et les ai aidés à trouver des informateurs.

\subsection{Animation de la recherche}
\begin{itemize}
\item 2009-présent: Directeur de l'équipe de recherche \textit{Linguistique descriptive des langues d’Asie, phonologie, morphosyntaxe et comparatisme} au sein du CRLAO.
\item 2010-présent: Responsable de l'opération PPC2-\textit{Evolutionary approaches to phonology} et de l'opération LR4.11-\textit{Automatic paradigm generation and language description} dans le cadre du Labex \textit{Empirical Foundations of Linguistics}.(\url{http://www.labex-efl.org/?q=fr/recherche/axe6})
\item conférences organisées:  3rd Workshop on Sino-Tibetan Languages of Sichuan (septembre 2013, EHESS).
\item  2014-présent: Responsable de l'opération LR2 de l'axe 6 du Labex \textit{Empirical Foundations of Linguistics} visant à développer un logiciel remplaçant le programme Toolbox du SIL.
\end{itemize}


\subsection{Activités éditoriales}
\begin{itemize}
\item Rédacteur en chef de la revue \textit{Cahiers de Linguistique -- Asie orientale} (Brill, depuis 2013) et responsable de la linguistique historique (\textit{Areal editor}) pour la revue \textit{Linguistic Vanguard} (Mouton de Gruyter)
\item membre du comité éditorial de \textit{Diachronica} (Benjamins, depuis 2008) et de \textit{Linguistics of the Tibeto-Burman Area}  (Benjamins, depuis 2014).
\item Relecteur pour les revues suivantes (en plus des précédentes): \textit{Lingua}, \textit{Studies in Language}, \textit{Folia Linguistica}, \textit{Journal of the International Phonetic Alphabet},  \textit{Language and Linguistics}, \textit{Transactions of the Philological Society}, \textit{Journal of Chinese Linguistics}, \textit{SKY journal of linguistics}, \textit{Langages}, \textit{Yuyanxue luncong}, \textit{Yuyan yanjiu}
\end{itemize}
%\section{Objectifs / Projet de recherche}



\end{document}