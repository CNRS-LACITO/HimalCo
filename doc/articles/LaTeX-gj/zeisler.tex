\documentclass[oldfontcommands,oneside,a4paper,11pt]{article} 
\usepackage{fontspec}
\usepackage{natbib}
\usepackage{booktabs}
\usepackage{xltxtra} 
\usepackage{longtable}
\usepackage{polyglossia} 
\usepackage[table]{xcolor}
\usepackage{gb4e} 
\usepackage{tangutex2} 
\usepackage{tangutex4}
\usepackage{multicol}
\usepackage{graphicx}
\usepackage{float}
\usepackage{hyperref} 
\usepackage{lineno}
\hypersetup{bookmarks=false,bookmarksnumbered,bookmarksopenlevel=5,bookmarksdepth=5,xetex,colorlinks=true,linkcolor=blue,citecolor=blue}
\usepackage[all]{hypcap}
\usepackage{memhfixc}
\usepackage{lscape}

\bibpunct[: ]{(}{)}{,}{a}{}{,}

\setmainfont[Mapping=tex-text,Numbers=OldStyle,Ligatures=Common]{Charis SIL} 
\newfontfamily\phon[Mapping=tex-text,Ligatures=Common,Scale=MatchLowercase,FakeSlant=0.3]{Charis SIL} 
\newcommand{\ipa}[1]{{\phon \mbox{#1}}} %API tjs en italique


\newcommand{\grise}[1]{\cellcolor{lightgray}\textbf{#1}}
\newfontfamily\cn[Mapping=tex-text,Ligatures=Common,Scale=MatchUppercase]{MingLiU}%pour le chinois
\newcommand{\zh}[1]{{\cn #1}}
\newcommand{\refb}[1]{(\ref{#1})}
\newcommand{\factual}[1]{\textsc{:fact}}

\XeTeXlinebreaklocale 'zh' %使用中文换行
\XeTeXlinebreakskip = 0pt plus 1pt %
 %CIRCG
 \newcommand{\bleu}[1]{{\color{blue}#1}}
\newcommand{\rouge}[1]{{\color{red}#1}} 
\newcommand{\tgf}[1]{\mo{#1}}
\newcommand{\tinynb}[1]{\tiny#1}

\begin{document} 
\title{Tangut, Gyalrongic and the antiquity of person indexation in Sino-Tibetan/Trans-Himalayan}
\author{Guillaume Jacques}
\maketitle
\linenumbers

\section{Introduction}



\citealt{jacques12internal}

\citet{hill14derivational}

\citealt{zeisler15eat}

\citet{zeisler04}


Proto-Trans-Himalayan Person Indexation System Hypothesis (henceforth $PI$ hypothesis) and those who argue that indexation systems are later innovations (the $\overline{PI}$ hypothesis)


\citet{delancey89agreement}
\citet{lapolla92}


\section{The `Umbrella' fallacy}
Although analogical reasoning may prove useful for pedagogical purposes, in most cases it is a tool that may induce more confusion that clarity.

In the beginning of her article, B. Zeisler presents the following `umbrella' analogy to ridicule the views of proponents of the $PI$ hypothesis, shown here with added references to the $PI$ vs $\overline{PI}$ debate.

\begin{itemize}
\item Premiss 1: Some members of my family have umbrellas, some don’t. = \textit{Some ST languages have person indexation, some don't.}
\item Premiss 2: It is well-known that people who have umbrellas loose [sic] (or get rid of) them easily. = \textit{it is well known that languages easily lose person indexation.}
\item Conclusion 1: Ergo, those members of my family who have no umbrellas must have lost (or got rid of) them. = \textit{ergo, the ST languages without person indexation must have lost it.}
\item Conclusion 2: Ergo, all members of my family must have had umbrellas in the first place. \textit{ergo, all ST languages must have had person indexation in the first place.}
\end{itemize}

No proponent of the $PI$ hypothesis would subscribe to the views here caricatured by Zeisler's metaphor, and in any cases there are several reasons why this analogy is inappropriate. 

First, umbrellas, in human societies, are \textit{alienable possessions} that one can own or dispose of at will, and that are not normally transmitted to one's descendants.

On the contrary, person indexation is a fundamental morphological feature which can be lost or innovated, but which is generally inherited from the ancestral language. A better human analogy for person indexation would be an \textit{inalienable possession}.

Second, the $PI$ hypothesis, as any serious hypothesis in historical linguistics, is not simply based on the presence vs absence of a particular feature, but on systemic commonalities, especially synchronically unexplainable common irregularities, a fact completely neglected by Zeisler.

Third, Zeisler's doubts concerning the genetic relationship of Tibetan to the rest of the Sino-Tibetan is \textit{non sequitur}, and hardly deserves a detailed discussion.


\section{The `Pronominalizing languages' fallacy}
The term `pronominalizing languages' that Zeisler keeps on using in her paper is a remnant of Hodgson's antiquated view of person indexation as resulting from the accretion of pronouns to the verb stem, a view which was perfectly legitimate in the mid-nineteenth century and that Bopp entertained also for Indo-European.\footnote{In Indo-European studies however, this view has long been discredited. }

Using the the term `pronominalized' to refer to ST languages with person indexation is inadequate for three reasons.

First, whatever the actual antiquity of the person indexation systems in ST, it is clear that an important part of person markers on verbs have been grammaticalized from sources that have nothing to do with pronouns, in particular nominalized forms (see for instance \citealt{jacques15generic} on the origin of the portmanteau 2$\rightarrow$1 and 1$\rightarrow$2 prefixes in Gyalrong languages).

Second, although in some ST languages such as Kuki-Chin productive person markers are identical (and probably derive from) possessive prefixes or pronouns, it is not the case everywhere, in particular in Kiranti and Gyalrong languages where, in particular, the second person prefixes cannot be shown to derive from pronouns (\citealt{jacques12agreement}, \citealt{delancey11prefixes} and \citealt{delancey14second}).

Third, the hypothesis that, whenever pronouns, possessive markers and person indexation markers present resemblances, the latter must be derived from the former is not necessarily true: the opposite direction of grammaticalization is also attested. In some languages, pronouns are derived from possessive prefixes rather than the other way round as in Japhug, where as can be seen in Table \ref{tab:pronoun} below, pronouns are built by combining possessive prefixes with the root \ipa{--ʑo} `oneself'.\footnote{This absence of `real' ancient pronouns is reminiscent of Algonquian, where pronouns are also the forms of a possessed `dummy' nominal stem, as in Ojibwe \ipa{n-iin} `I', \ipa{g-iin} `you' and \ipa{w-iin} `he', see \citet{valentine01grammar}. }

 \begin{table}[H] \centering
\caption{Pronouns and possessive prefixes in Japhug}\label{tab:pronoun}
\begin{tabular}{lllllllll} 
\toprule
 Free pronoun & Prefix & Person\\
\midrule
 \ipa{a-ʑo}  &	\ipa{a--}  &		1\textsc{sg} \\
\ipa{nɤ-ʑo}  &	\ipa{nɤ--}  &			2\textsc{sg}\\
\ipa{ɯ-ʑo}  &	\ipa{ɯ--}  &			3\textsc{sg}\\
\midrule
\ipa{tɕi-ʑo}  &	\ipa{tɕi--}  &			1\textsc{du} \\
\ipa{ndʑi-ʑo}  &	\ipa{ndʑi--}  &		2\textsc{du} \\	
\midrule
\ipa{i-ʑo}    &	\ipa{i--}  &			1\textsc{pl} \\
\ipa{nɯ-ʑo}   &	\ipa{nɯ--}  &			2\textsc{pl} \\
\midrule
\ipa{tɯ-ʑo} & \ipa{tɯ--}   &  generic\\
\bottomrule
\end{tabular}
\end{table}

Cases of pronouns derived from personal affixes are also found, especially in polysynthetic languages. For instance, in Ainu, the \textsc{1sg} pronoun \ipa{kuani} is in fact etymologically the nominalization of the \textsc{1sg} form of the existential copula (\ipa{ku-an-i} \textsc{1sg}-exist-\textsc{nmlz}, \citealt[31]{shibatani90japan}). Likewise, in Lakota the pronouns \textsc{1sg} \ipa{miyé}, \textsc{2sg} \ipa{niyé} and \textsc{3sg} \ipa{iyé} are in fact conjugated verb forms meaning `it is him, it is you, it is him' (\citealt[707;754]{ullrich08}).

Grammaticalization pathways exist between person indexation markers and pronouns, but these pathways are by no means as linear and unidirectional as the use of a term such as `pronominalizing languages' does suggest. 

\section{Person indexation in Tangut}
Given \citet{kepping94conjugation}'s lucid response to \citet{lapolla92} published more than twenty years ago, it is surprising that Zeisler still quotes the conclusions of the latter article concerning Tangut.

\citet{lapolla92}, without any reference to actual Tangut texts, claims that (1) there is a one-to-one relationship between the pronouns and the suffixes and that (2) person agreement in Tangut is optional. These two claims are evaluated here on the basis of a corpus comprising the most important narrative texts (see \citealt[8-9]{jacques14esquisse}).

%\begin{tabular}{lllllllllll}
%	\tgf{4028}&	\tgf{3986}&	\tgf{4893}&	\tgf{1139}&	\tgf{1526}&	\tgf{5880}&	\tgf{0524}&	\tgf{2590}&	\tgf{5591}&	\tgf{4601}&\tgf{3916}\\
%\tinynb{4028}&	\tinynb{3986}&	\tinynb{4893}&	\tinynb{1139}&	\tinynb{1526}&	\tinynb{5880}&	\tinynb{0524}&	\tinynb{2590}&	\tinynb{5591}&	\tinynb{4601}&\tinynb{3916}\\
%\end{tabular}

\subsection{Person indexation suffixes}
As pointed out by \citet{kepping94conjugation}, while agreement suffixes in Tangut do present  resemblances with pronouns, as shown in Table \ref{tab:pronoms.suffixes}, pronouns and agreement markers cannot be equated. 

\begin{table}[H]
\caption{Pronouns and person suffixes in Tangut (\citealt{kepping75agreement, kepping85})}\label{tab:pronoms.suffixes} \centering
\begin{tabular}{llllll} 
\toprule
\multicolumn{3}{c}{Pronoun} &\multicolumn{3}{c}{Suffix} \\
\midrule
\mo{2098} & \ipa{ŋa²}  & 1\textsc{sg} & \mo{2098} & \ipa{ŋa²}  &1\textsc{sg} \\
\mo{3926} & \ipa{nja²} & 2\textsc{sg} & \mo{4601} & \ipa{nja²} &2\textsc{sg} \\
\mo{4028} &  \ipa{nji²} & 2\textsc{sg}  honorific or 2\textsc{pl} & \mo{4884} & \ipa{nji²} & 1\textsc{pl} and 2\textsc{pl} \\
\bottomrule
\end{tabular}
\end{table}

First, for the first person singular, while the pronoun \mo{2098} \ipa{ŋa²} is given the same reading as the \textsc{1sg} suffix, a second pronoun  the pronoun \tgf{0261} \ipa{mjo²} is also very common. In the chapters 1-6 of Leilin, there are 31 occurrences of the pronoun \mo{2098} \ipa{ŋa²}, 60 occurrences of \mo{2098} \ipa{ŋa²} as a \textsc{1sg} suffix, and 20 occurrences of  \tgf{0261} \ipa{mjo²}. This latter pronoun always trigger agreement with  the \mo{2098} \ipa{ŋa²} suffix when occurring as a core argument, as in example \ref{ex:tg:mjo}. Thus, despite the homophony and homography of the \textsc{1sg} pronoun and the \textsc{1sg} suffix,\footnote{On this topic, it should be noted that it is common in Tangut to write unrelated but homophonous (or perhaps, near homophonous) morphemes with the same character, see several examples in \citet{jacques11tangut.verb}.} it is a fact of the synchronic grammar of Tangut that the two are distinct -- otherwise, we would not expect agreement between \tgf{0261} \ipa{mjo²} and \mo{2098} \ipa{ŋa²}.

\begin{exe}
\ex \label{ex:tg:mjo}  
\glll 
\tgf{0261} 	\tgf{2541} 	\tgf{1139} 	\tgf{3104} 	\tgf{0046}\tgf{0749} 	\tgf{2620}\tgf{2098} \\
\ipa{mjo²} 	\ipa{dzjwo²} 	\ipa{.jij¹} 	? 	\ipa{ljij²-phji¹} 	\ipa{njwi²-ŋa¹} \\
\textsc{1sg} man \textsc{antierg} ghost see[A]-cause[A] can-\textsc{1sg} \\
\glt I can make people see ghosts (Leilin 05.21B.4)
\end{exe}.

It is however with the honorific pronoun \tgf{4028} \ipa{nji²} and the SAP plural suffix \mo{4884} \ipa{nji²} that the difference is most telling. \tgf{4028} \ipa{nji²} is singular, and when serving as core argument, the verb takes the \textsc{2sg} \tgf{4601} \ipa{--nja²} suffix, as in \ref{ex:tg:toi}. No example of  \tgf{4028} \ipa{nji²} used with \mo{4884} \ipa{nji²} has been found in the corpus under study.
 
\begin{exe}
\ex \label{ex:tg:toi}  
\glll   \tgf{4028} 	\tgf{3986}\tgf{4893} 	\tgf{1139} 	\tgf{1526} 	\tgf{5880} 	\tgf{0524} \tgf{2590}\tgf{5591}\tgf{4601}\tgf{3916} \\
\ipa{nji²}	\ipa{njɨ¹.wjɨ¹}	\ipa{.jij¹}	\ipa{tshji²}	\ipa{ŋwu²}	\ipa{dzju¹}	\ipa{.wjɨ²-lhjị²-nja²-sji²} \\
\textsc{2sg:hon} mother.in.law  \textsc{antierg} serve \textsc{instr} order \textsc{pfv}-receive[B]-\textsc{2sg-pfv} \\
\glt `It is you_{sg} who served (our) mother-in-law and received her instructions.' (Cixiaozhuan, 33.4, \citealt{jacques07textes})
\end{exe}

The suffix \tgf{4884} \ipa{nji²}, on the other hand, appears with first or second person plural, as in \ref{ex:tg:1pi}.

\begin{exe}
\ex \label{ex:tg:1pi}  
\glll 
\tgf{2248}\tgf{2065} 	\tgf{1413}\tgf{5970} 	\tgf{1139} 	\tgf{1567}\tgf{0239} 	\tgf{0508}\tgf{4884} \\
\ipa{gjɨ²mji²} 	\ipa{tej¹pie¹} 	\ipa{.jij¹} 	\ipa{gji²lhjɨ¹} 	\ipa{ŋwu²-nji²} \\
\textsc{1pi} Taibo \textsc{gen} grandchildren be-\textsc{1/2pl} \\
\glt We are the descendants of Taibo. (Leilin 04.33A.4)
\end{exe}

Despite the fact that  \tgf{4028} \ipa{nji²} and \tgf{4884} \ipa{nji²} have identical readings, it is not obvious that the latter is derived from the former; 

XXX tbl with japhug, stau etc

\subsection{Stem Alternations}
The most compelling proof that person indexation in Tangut is not recent, however, does not come from the pronouns. As I have argued in \citet{jacques09tangutverb} and \citet{jacques14tangoute}, Stem Alternations xxx

%\begin{tabular}{llllllll}
%	\tgf{1542}&	\tgf{3508}&	\tgf{0100}&	\tgf{2798}&	\tgf{2987}&	\tgf{5481}&	\tgf{4547}&	\tgf{2098}\\
%	\tinynb{1542}&	\tinynb{3508}&	\tinynb{0100}&	\tinynb{2798}&	\tinynb{2987}&	\tinynb{5481}&	\tinynb{4547}&	\tinynb{2098}\\
%\end{tabular}
%\begin{exe}
%\ex \label{ex:tg:manger.b.1sg.3}  \vspace{-8pt}
%\gll   \ipa{ku¹}	\ipa{bji²}	\ipa{lew¹}	\ipa{.jir²}	\ipa{lhjɨ̣¹}	\ipa{bo²}	\ipa{dzjo¹-ŋa²} \\
%	alors sujet un cent coup bâton manger[B]-1sg{} \\
%\glt Alors, moi, votre sujet, je recevrai cent coups de bâtons (Leilin 06.13A.5)
%\end{exe}


%\begin{tabular}{llllll}
%	\tgf{2447}&	\tgf{1519}&	\tgf{5165}&	\tgf{2590}&	\tgf{4517}&	\tgf{2098}\\
%	\tinynb{2447}&	\tinynb{1519}&	\tinynb{5165}&	\tinynb{2590}&	\tinynb{4517}&	\tinynb{2098}\\
%\end{tabular}
%\begin{exe}
%\ex \label{ex:tg:manger.a.2.1sg}  \vspace{-8pt}
%\gll   \ipa{ljo²}	\ipa{ɣu¹twụ¹}	\ipa{wjɨ²-dzji¹-ŋa²} \\
%		frère.aîné à.la.place \dir{}-manger[A]-1\sg{} \\
%\glt Mangez-moi à la place de mon frère! (Cixiaozhuan 17.7, \citealt[55-6]{jacques07textes})
%\end{exe}
%\begin{tabular}{llllllllll}
%	\tgf{4689}&	\tgf{1531}&	\tgf{0866}&	\tgf{1139}&	\tgf{2393}&	\tgf{3456}&	\tgf{0705}&	\tgf{2219}&	\tgf{0046}&	\tgf{2098}\\
%	\tinynb{4689}&	\tinynb{1531}&	\tinynb{0866}&	\tinynb{1139}&	\tinynb{2393}&	\tinynb{3456}&	\tinynb{0705}&	\tinynb{2219}&	\tinynb{0046}&	\tinynb{2098}\\
%\end{tabular}
%%\begin{exe}
%%\ex \label{ex:tg:voir.a.3.1sg}  \vspace{-8pt}
%%\gll   \ipa{.jwar¹}	\ipa{gja¹}	\ipa{ɣu¹}	\ipa{.jij¹}	\ipa{ljiij²}	\ipa{lja¹}	\ipa{zjịj¹}	\ipa{kjij¹-ljij²-ŋa²} \\
%%		Yue armée Wu \antierg{} détruire venir quand \opt{}-voir[A]-1\sg{} \\
%%\glt Quand (les soldats) de l'armée de Yue viendront détruire Wu, ils me verront (Leilin 03.21B.4-5)
%%\end{exe}
%\begin{tabular}{llllllllll} 
%	\tgf{3133}&	\tgf{0261}&	\tgf{1531}&	\tgf{1139}&	\tgf{0795}&	\tgf{0676}&	\tgf{0046}&	\tgf{5643}&	\tgf{3092}&	\tgf{2912}\\
%	\tinynb{3133}&	\tinynb{0261}&	\tinynb{1531}&	\tinynb{1139}&	\tinynb{0795}&	\tinynb{0676}&	\tinynb{0046}&	\tinynb{5643}&	\tinynb{3092}&	\tinynb{2912}\\
%\tgf{3092}&	\tgf{5643}&	\tgf{1374}&	\tgf{4803}&	\tgf{2098}&&&&&\\
%\tinynb{3092}&	\tinynb{5643}&	\tinynb{1374}&	\tinynb{4803}&	\tinynb{2098}&&&&&\\
%\end{tabular}
%\begin{exe}
%\ex   \vspace{-8pt}
%\gll   \ipa{sjij¹}	\ipa{mjo²}	\ipa{gja¹}	\ipa{.jij¹}	\ipa{rjɨr²-.wjij¹}	\ipa{ljij²} \ipa{mjɨ¹djij²}	\ipa{lhjwo¹-djij²}	\ipa{mjɨ¹-tɕhjɨ¹-lji²-ŋa²} \\
%		aujourd'hui moi armée \antierg{} \dir{}-partir voir[A] à.part revenir-\dur{} \negat{}-\pot{}-voir[B]-1\sg{} \\
%\glt  Aujourd'hui je vois l'armée partir, mais je ne verrai pas son retour. (Leilin 3.16B.6-7)
%\end{exe}


At that time when LaPolla wrote his article, Tangut texts were not easily accessible for independent study, and stem alternations had not yet been fully described. Although \citet{nishida75} first suggested the their existence, this discovery was not taken into account by other scholars (even Kepping), and it was not until \citet{gong01huying} that it became widely known. 
 


\section{Syntactic pivots in Gyalrong and Kiranti}

\section{Conclusion}

We lack at the present moment the data to settle the $PI$ vs $\overline{PI}$ debate. Until Kiranti and Gyalrongic languages have been fully documented, the proto-languages of these groups reconstructed and the mass of data fully digested by the community of ST linguists, it is too early for a rigorous discussion based on an exhaustive account of the data. 

Some researchers participating in the debate choose data from second of third hand sources in piece-meal fashion, rather than investigating all languages on the basis of text corpora, as is done in Indo-European studies, and as should be done for any language family. What we most need at the present moment are good-quality descriptions of the languages most relevant to the debate, large corpora, and Neogrammarian-based historical linguistics.

\bibliographystyle{unified}
\bibliography{bibliogj}

\end{document}