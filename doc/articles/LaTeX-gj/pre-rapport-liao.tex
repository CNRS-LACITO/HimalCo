\documentclass[oldfontcommands,oneside,a4paper,11pt]{article} 
\usepackage{fontspec}
\usepackage{natbib}
\usepackage{booktabs}
\usepackage{xltxtra} 
\usepackage{polyglossia} 
\setdefaultlanguage{french} 
\usepackage[table]{xcolor}
\usepackage{gb4e} 
\usepackage{multicol}
\usepackage{graphicx}
\usepackage{float}
\usepackage{hyperref} 
\hypersetup{bookmarks=false,bookmarksnumbered,bookmarksopenlevel=5,bookmarksdepth=5,xetex,colorlinks=true,linkcolor=blue,citecolor=blue}
\usepackage[all]{hypcap}
\usepackage{memhfixc}
\usepackage{lscape}

\bibpunct[: ]{(}{)}{,}{a}{}{,}

%\setmainfont[Mapping=tex-text,Numbers=OldStyle,Ligatures=Common]{Charis SIL} 
\newfontfamily\phon[Mapping=tex-text,Ligatures=Common,Scale=MatchLowercase,FakeSlant=0.3]{Charis SIL} 
\newcommand{\ipa}[1]{{\phon \mbox{#1}}} %API tjs en italique
\newcommand{\ipab}[1]{{\scriptsize \phon#1}} 

\newcommand{\grise}[1]{\cellcolor{lightgray}\textbf{#1}}
\newfontfamily\cn[Mapping=tex-text,Ligatures=Common,Scale=MatchUppercase]{MingLiU}%pour le chinois
\newcommand{\zh}[1]{{\cn #1}}
\newcommand{\refb}[1]{(\ref{#1})}

\newcommand{\ra}{$\Sigma_1$} 
\newcommand{\rc}{$\Sigma_3$} 
\newcommand{\ro}{$\Sigma$} 

\XeTeXlinebreaklocale 'zh' %使用中文换行
\XeTeXlinebreakskip = 0pt plus 1pt %
 %CIRCG
 
\sloppy

\begin{document} 
\title{Pré-rapport de la thèse présentée par M Liao Shueying \textit{L'usage de la figure rythmique dans l'analyse du procédé incitatif (Xing): une méthode de lecture expérimentale pour le Shijing} en vue de l'obtention du doctorat }
\author{Guillaume Jacques, chargé de recherches HDR, CNRS\\rgyalrongskad@gmail.com}
\maketitle

La thèse de Liao Shueying propose une nouvelle méthode pour analyser la structure des poèmes du Shijing en calculant la fréquence relative des caractères dans chaque position de l'hémistiche. Ce travail, basé sur des scripts en python mis au point par l'auteur et accompagné d'une annexe copieuse, se situe à la croisée de trois disciplines, la philologie chinoise, la linguistique historique et la linguistique computationnelle. C'est une œuvre la fois cohérente et originale, mais elle souffre d'une certaine idiosyncrasie au niveau de la terminologie et de la méthode qui rendent son accès difficile. 

La partie informatique de la thèse est réduite (usage d'expressions régulières et traitement de fichiers textes) mais j'ai pu m'assurer que les scripts écrits par l'auteur fonctionnent effectivement tels qu'ils sont décrits dans la thèse. Même s'il ne s'agit pas à proprement parler d'une contribution à l'informatique, le développement et l'usage de scripts pour résoudre les problèmes que l'auteur s'était posés sont tout à fait appropriés, et lui ont permis de traiter de façon aisément reproductible une quantité importante de données, et de fournir une méthode potentiellement applicable à d'autres corpus.

La linguistique est le grand absent de cette thèse, et cette absence se fait sentir aussi bien au niveau de la terminologie que de la problématique, qui sont traités ici l'un après l'autre.

En ce qui concerne la terminologie utilisée dans thèse, on peut distinguer deux types de problèmes. 

Premièrement, l'auteur fait usage de néologismes regrettables tels que `mot-césure' pour traduire \zh{虚词}, et `sino-syllabe' pour faire référence au caractères, désignation problématique pour le Shijing, texte d'une époque où un caractère ne correspondait pas nécessairement à une syllable (comme le reconnaît l'auteur lui-même p.97 n.199). 

Deuxièmement, il utilise des termes tels que `rythme', `syntaxe' et `accentuation' dans un sens très différent de leur usage habituel en linguistique, ce qui demande un effort considérable de la part du lecteur, voire un véritable travail de traduction, pour comprendre le propos. On remarque aussi le terme `contrapuntique' (p.251) pour désigner deux hémistiches dont la structure `accentuelle' (selon la terminologie de l'auteur) est opposée, un sens sans rapport avec le contrepoint au sens musical, qui désignerait la superposition temporelle de deux motifs.

Du point de vue de la problématique de la thèse, on regrette l'absence totale de références aux nombreux travaux de phonologie (aussi bien structuraliste que générative) sur la métrique, ainsi que trois points principaux.

Premièrement, le Shijing contient de nombreuses syllabes redoublées, auxquelles l'auteur consacre plusieurs pages (p.89 par exemple); une discussion plus générale mettant en regard les propriétés des idéophones dans les langues du monde avec les caractéristiques spécifiques du chinois archaïque auraient permis d'agrandir le lectorat potentiel de ce travail.

Deuxièmement, cette thèse traite les données linguistiques d'une façon assez naïve; en particulier, elle néglige la fonction syntaxique des mots, et ne distingue pas les mots homophones écrits par le même caractère, même s'ils appartiennent à des parties du discours différentes (un exemple caractéristique est \zh{之} *\ipa{tə}, qui peut être un démonstratif, un pronom ou le verbe `aller'). Les scripts traitent ailleurs différemment des constructions quasi-identiques si elles n'ont pas le même nombre de syllabes (voir p. 99, n. 203). Ce travail omet aussi d'utiliser les corpus étiquetés du chinois, qui auraient pourtant permis une analyse plus fine des données. On ne trouve nul part mentionné que les fréquences observées des caractères dans des positions différentes sont le sous-produit de règles syntaxiques relativement bien décrites.


Troisièmement, même s'il l'on doit reconnaître que l'auteur fait référence à quelques travaux sur la phonologie historique du chinois, comme ceux de Baxter et Sagart, il néglige pour l'essentiel l'aspect phonétique de la versification.

Enfin, l'idée qui sous-tend l'ensemble de la thèse, à savoir que l'on peut définir une `accentuation' de chaque syllabe en fonction de la fréquence du caractère dans la position où il se trouve de l'hémistiche me semble très contestable. Elle n'a jamais été appliquée à aucune autre langue. N'aurait-il pas mieux valu la tester sur une langue vivante bien comprise, avant de l'utiliser sur un corpus restreint et sous certains aspects bien obscur tel que le Shijing?


Toutefois, malgré ces réserves, il est indéniable que l'auteur a démontré sa capacité à  développer une idée originale et  à se donner les moyens intellectuels pour mener une recherche à complétion. Les maladresses de ce travail pionnier, dues à l'isolation intellectuelle relative dans laquelle cette thèse a été rédigée et probablement inévitables, ne doivent pas en masquer les qualités réelles. Les critiques présentées ci-dessus sont remédiables, et l'auteur pourra tirer de cette thèse des articles avec quelques efforts d'adaptation terminologique. Les spécialistes combinant une expertise authentique sur les textes chinois antiques avec des connaissances en informatique sont extrêmement rares, et le travail de M. Liao  contribue à désenclaver les études philologiques et à combler le gouffre qui sépare encore les humanités de l'informatique.


Il ne fait donc aucun doute que cette thèse mérite d'être soutenue.

%\bibliographystyle{unified}
%\bibliography{bibliogj}
\end{document}