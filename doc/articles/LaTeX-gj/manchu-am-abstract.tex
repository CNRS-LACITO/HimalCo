\documentclass{article} 
\usepackage{fontspec}
\usepackage{natbib}
\usepackage{booktabs}
\usepackage{xltxtra} 
\usepackage{polyglossia} 
\usepackage[table]{xcolor}
\usepackage{gb4e} 
%\usepackage{multicol}
\usepackage{hyperref} 
\hypersetup{bookmarks=false,bookmarksnumbered,bookmarksopenlevel=5,bookmarksdepth=5,xetex,colorlinks=true,linkcolor=blue,citecolor=blue}
\usepackage[all]{hypcap}
\usepackage{memhfixc}

 
%\setmainfont[Mapping=tex-text,Numbers=OldStyle,Ligatures=Common]{Charis SIL} 
\newfontfamily\phon[Mapping=tex-text,Ligatures=Common,Scale=MatchLowercase]{Charis SIL} 
\newcommand{\ipa}[1]{\textbf{{\phon\mbox{#1}}}} %API tjs en italique
 \newcommand{\ipab}[1]{{\phon \mbox{#1}}} %API tjs en italique
\newcommand{\grise}[1]{\cellcolor{lightgray}\textbf{#1}}
\newfontfamily\cn[Mapping=tex-text,Ligatures=Common,Scale=MatchUppercase]{SimSun}%pour le chinois
\newcommand{\zh}[1]{{\cn#1}}

 
\begin{document} 
\title{Associated motion in Manchu in typological perspective}
%\author{José Andrés Alonso de la Fuente\\Guillaume Jacques}
\maketitle 
%acknowledgements: John Cowan, Antoine Guillaume, Sven Osterkamp, Brigitte Pakendorff, Daniel Ross, Alexander Vovin, Mikhail Zhivlov, Moïse-Fauré, Claire
%chen13conditional
\textbf{Abstract}: The present paper presents a detailed description of the Associated Motion system of Classical Manchu, on the basis of original texts from the 17-18th centuries. It shows that despite superficial similarities, Classical Manchu differs in many ways from previously described AM systems only comprising translocative vs cislocative makers, such as that of Japhug. This paper provides a basic framework for further research on the typology of simple AM systems.

\textbf{Keywords}: Associated motion, Tungusic, Manchu, Motion verbs, Andative, Venitive, Translocative, Cislocative, Grammaticalization

 \end{document}
 