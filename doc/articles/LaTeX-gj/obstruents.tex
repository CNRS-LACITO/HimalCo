\documentclass[oldfontcommands,oneside,a4paper,11pt]{article} 
\usepackage{fontspec}
\usepackage{natbib}
\usepackage{booktabs}
\usepackage{xltxtra} 
\usepackage{longtable}
\usepackage{polyglossia} 
\usepackage[table]{xcolor}
\usepackage{lineno}
\usepackage{gb4e} 
\usepackage{multicol}
\usepackage{graphicx}
\usepackage{float}
\usepackage{hyperref} 
\hypersetup{bookmarks=false,bookmarksnumbered,bookmarksopenlevel=5,bookmarksdepth=5,xetex,colorlinks=true,linkcolor=blue,citecolor=blue}
\usepackage[all]{hypcap}
\usepackage{memhfixc}
\usepackage{lscape}
\usepackage{amssymb}
\bibpunct[: ]{(}{)}{,}{a}{}{,}
%%%%%%%%%quelques options de style%%%%%%%%
%\setsecheadstyle{\SingleSpacing\LARGE\scshape\raggedright\MakeLowercase}
%\setsubsecheadstyle{\SingleSpacing\Large\itshape\raggedright}
%\setsubsubsecheadstyle{\SingleSpacing\itshape\raggedright}
%\chapterstyle{veelo}
%\setsecnumdepth{subsubsection}
%%%%%%%%%%%%%%%%%%%%%%%%%%%%%%%
\setmainfont[Mapping=tex-text,Numbers=OldStyle,Ligatures=Common]{Charis SIL} 
\newfontfamily\phon[Mapping=tex-text,Ligatures=Common,Scale=MatchLowercase,FakeSlant=0.3]{Charis SIL} 
\newcommand{\ipa}[1]{{\phon \mbox{#1}}} %API tjs en italique
 
 
 
\newcommand{\grise}[1]{\cellcolor{lightgray}\textbf{#1}}
\newfontfamily\cn[Mapping=tex-text,Ligatures=Common,Scale=MatchUppercase]{MingLiU}%pour le chinois
\newcommand{\zh}[1]{{\cn #1}}

\newcommand{\jg}[1]{\ipa{#1}\index{Japhug #1}}
\newcommand{\wav}[1]{#1.wav}
\newcommand{\tgz}[1]{\mo{#1} \tg{#1}}

\XeTeXlinebreaklocale 'zh' %使用中文换行
\XeTeXlinebreakskip = 0pt plus 1pt %
 %CIRCG
\begin{document} 

\title{ On context-free changes of place of articulation in unvoiced obstruents }
\author{Guillaume Jacques }
\maketitle

 \section{Debuccalisation}  \label{sec:debucc}
 
 
 \subsection{\ipa{k} $\rightarrow$ \ipa{ʔ} / $\varnothing$} \label{sec:k.glottal}
Debuccalisation of unvoiced stops widely attested     as a contextual change, either occurringword-initially, in clusters, between vowels or word-finally (\citealt[107]{kuemmel07wandel}). Table \ref{tab:debucc.context} summarizes attested patterns of debuccalisation in word-initial and coda position. 


There is a marked tendency for velars stops to be more prone to debuccalization word-finally than coronals and   labials, as captured by hierarchy (\ref{ex:hierarchy.coda}). 

\begin{exe}
\ex \label{ex:hierarchy.coda}
\glt \ipa{--p} >  \ipa{--t} >  \ipa{--k}
\end{exe}
Exception to this hierarchy are rare, but do exist, for instance the Sinitic Nancheng language, where *--\ipa{p} and *--\ipa{k} merge to \ipa{ʔ} while the coronal remains an oral segment.
\begin{table}[h]
\caption{Examples of contextual debuccalization of stops} \label{tab:debucc.context}
\begin{tabular}{lllllll}
\toprule
*p & *t & *k & Context & Language    & Reference \\
\midrule
$\varnothing$ \grise{}&$\varnothing$ \grise{}& $\varnothing$ \grise{}&Initial& Arrernte &\citet{koch06arandic}\\
\ipa{ʔ} \grise{}&\ipa{ʔ} \grise{}& \ipa{kh} &Initial& Yamphu  & \citet[12-3]{opgenort05jero}\\
\ipa{ʔ} \grise{}&\ipa{th} & \ipa{kh} & Initial&Yakkha   & \\
\ipa{p} or $\varnothing$ &\ipa{t}  & $\varnothing$ \grise{}&Initial& Cheyenne &\citet{goddard88cheyenne.y}\\
\midrule
\ipa{p} &\ipa{t} & \ipa{ʔ}\grise{} & Coda &Malay   &   \\
\ipa{t} &\ipa{t} & \ipa{ʔ}\grise{} & Coda &Chinese Gan languages  &  \citet[123]{chinese.atlas08}  \\
\ipa{p} &\ipa{ʔ} \grise{}& \ipa{ʔ}\grise{} & Coda&Lhasa Tibetan   & \\
\ipa{ʔ} \grise{}&\ipa{ʔ} \grise{}& \ipa{ʔ}\grise{} & Coda& Burmese  & \\
\ipa{ʔ} \grise{}&\ipa{lʔ}& \ipa{ʔ}\grise{} & Coda& Nancheng Chinese& \citet[123]{chinese.atlas08}\\
\bottomrule
\end{tabular}
\end{table}



%\ipa{*p, *t} $\rightarrow$  \ipa{ʔ} but \ipa{*k} $\rightarrow$  \ipa{kh} in Yamphu (\citealt[12-3]{opgenort05jero})

Examples of context-free or quasi-context-free changes from oral stop to glottal stop are rare, and they only involve dorsal (velar or uvular) stops. For coronal and labial stops, debuccalisation is only attested as a restricted contextual change; context-free debuccalization and loss of \ipa{p} is attested, but through a  fricative stage and it is treated in section \ref{sec:fric.h}. 


Context-free changes from \ipa{q} $\rightarrow$  \ipa{ʔ} are relatively common for language families with uvulars. It appears for instance in Maltese (\citealt[107]{kuemmel07wandel}) or in Junin Quechua (where it later change to $\varnothing$, see \citealt[202]{adelaar04andes}). It does not entail loss of all dorsal stops, since in   languages where it occurred, the ancient \ipa{k} remains untouched.

Context-free changes from \ipa{k} $\rightarrow$  \ipa{ʔ} or $\varnothing$ are rarer, and leads to the loss of a dorsal stop series at least during a short stage. Only two sets of examples are documented, in Polynesian languages (\citealt{blust04tk})\footnote{The sound change \ipa{k} $\rightarrow$  \ipa{ʔ}  is also found in various other Austronesian languages, especially in Melanesia, but it is either sporadic or in specific contexts and will not interest us here.} and in Arapahoan (Algonquian, \citealt{goddard74arapaho}).\footnote{Cheyenne also undergoes debuccalization of *\ipa{k} and *\ipa{p}, but since the conditioning is unclear, it cannot count as a context-free change and will not be discussed here (see \citet{goddard88cheyenne.y}). } It is significant that this sound change is attested only in languages with one series of stops. 

In   Polynesian languages  where the context-free sound change \ipa{k} $\rightarrow$  \ipa{ʔ} occurred, two situations are observed. First, it leads to a system without velar stops (as in formal Samoan or in Tahitian), thus as pointed out by \citet[370-371]{blust04tk}), the sequence tVt tends to dissimilate to kVt in rapid speech. Second, it is followed by a context-free change \ipa{t} $\rightarrow$  \ipa{k} in a chain shift (see \ref{sec:tk}).

In Arapahoan (Arapaho and Atsina), proto-Algonquian *\ipa{k} changes to $\varnothing$ in all contexts (\citealt[107]{goddard74arapaho}), as in \ipa{niii--}  'to camp' $\leftarrow$< *\ipa{wiik-i-}.\footnote{This verb comprises the possessed noun *\ipa{w-iik-i} "his house" plus the *\ipa{-i} animate intransitive final.} It is unclear whether an intermediate stage *\ipa{k} $\rightarrow$  *\ipa{ʔ} $\rightarrow \varnothing$ with a glottal stop must be postulated, since for instance proto-Algonquian *--\ipa{sk}-- changes to --\ipa{ʔ}--.\footnote{For instance *\ipa{me-skač-i} $\rightarrow$ \ipa{wó-ʔooθ} `his foot'.}

 
 
 

  \subsection{fricative  $\rightarrow$  \ipa{h}}  \label{sec:fric.h}
\citet[102-106]{kuemmel07wandel}

\ipa{p} $\rightarrow$ \ipa{ɸ} $\rightarrow$  \ipa{h}   $\rightarrow \varnothing $
 
sauf \citet{thurneysen21pl} 
 
 
 
  \subsection{Rebuccalisation}   \label{sec:rebucc}
Only possible as a contextual change

\citet{michaud06neutralisation}


 \section{Dorsalization}  \label{sec:dorsal}

\subsection{\ipa{t} $\rightarrow$ \ipa{k}}  \label{sec:tk}
\citet{blust04tk}

\citet{donohue06tk}

\citet[393]{blust04tk} minor exceptions to k > ʔ, limited to a few items

Entirely different from \citet{haas68chipewyan}
and Kansa

\subsection{\ipa{p} $\rightarrow$ \ipa{k}} \label{sec:pk}

\citet{goddard74arapaho}


\subsection{Fricatives} \label{sec:fric.change}


θ > f (latin)

s > θ (Shawnee, 


ɕ > ʂ > x (\citet{jacques11lingua})


ofo s > f, x >s?





\bibliographystyle{Linquiry2}
\bibliography{bibliogj}
\end{document}
