\documentclass[oldfontcommands,oneside,a4paper,11pt]{article} 
\usepackage{fontspec}
\usepackage{natbib}
\usepackage{booktabs}
\usepackage{xltxtra} 
\usepackage{longtable}
\usepackage{polyglossia} 
\usepackage[table]{xcolor}
\usepackage{lineno}
\usepackage{gb4e} 
\usepackage{multicol}
\usepackage{graphicx}
\usepackage{float}
\usepackage{hyperref} 
\hypersetup{bookmarks=false,bookmarksnumbered,bookmarksopenlevel=5,bookmarksdepth=5,xetex,colorlinks=true,linkcolor=blue,citecolor=blue}
\usepackage[all]{hypcap}
\usepackage{memhfixc}
\usepackage{lscape}
\usepackage{lineno}
\bibpunct[: ]{(}{)}{,}{a}{}{,}
%%%%%%%%%quelques options de style%%%%%%%%
%\setsecheadstyle{\SingleSpacing\LARGE\scshape\raggedright\MakeLowercase}
%\setsubsecheadstyle{\SingleSpacing\Large\itshape\raggedright}
%\setsubsubsecheadstyle{\SingleSpacing\itshape\raggedright}
%\chapterstyle{veelo}
%\setsecnumdepth{subsubsection}
%%%%%%%%%%%%%%%%%%%%%%%%%%%%%%%
\setmainfont[Mapping=tex-text,Numbers=OldStyle,Ligatures=Common]{Charis SIL} 
\newfontfamily\phon[Mapping=tex-text,Ligatures=Common,Scale=MatchLowercase,FakeSlant=0.3]{Charis SIL} 
\newcommand{\ipa}[1]{{\phon \mbox{#1}}} %API tjs en italique
 
 
 
\newcommand{\grise}[1]{\cellcolor{lightgray}\textbf{#1}}
\newfontfamily\cn[Mapping=tex-text,Ligatures=Common,Scale=MatchUppercase]{MingLiU}%pour le chinois
\newcommand{\zh}[1]{{\cn #1}}

\newcommand{\jg}[1]{\ipa{#1}\index{Japhug #1}}
\newcommand{\wav}[1]{#1.wav}
\newcommand{\tgz}[1]{\mo{#1} \tg{#1}}

\XeTeXlinebreaklocale 'zh' %使用中文换行
\XeTeXlinebreakskip = 0pt plus 1pt %
 %CIRCG
\begin{document} 

\title{ Tonogenesis and tonal alternations in Khaling }
\author{Guillaume Jacques }
\maketitle
\linenumbers
 
 \section{Introduction}
The Kiranti languages are almost unique in the Sino-Tibetan family for their intricate and typologically unusual verbal morphology (\citealt{bickel07chintang}, \citealt{jacques12agreement}). While this morphology involves the addition affixes as well as non-concatenative phenomena such as consonant and vowel alternations, tonal alternations are only attested in one Kiranti language Khaling.

Khaling relatively recently\footnote{Tonogenesis and most vowel changes postdate the start of Nepali influence in the eighteenth century, as will be shown in section \ref{sec:obstruents}.} underwent considerable phonological changes, and innovated a two-tone system. The origin of these tonal contrast is relatively transparent, and thus makes Khaling a valuable case study for investigating the creation of tonal alternations. 

In this paper, we first present a description account of Khaling synchronic phonology. Then, we show how falling tones developped from final stops and from the reduction of disyllables, using only examples from nouns, which unlike verbs do not have complex alternation. Then, we use this knowledge of tonogenesis to analyze the synchronic tonal alternations in the Khaling verbal system. Finally, we show the existence of a residue of forms that cannot be straightforwardly explained by the known tonogenetic processes, and which require either revision of the reconstruction or explanations involving analogical leveling.

\section{Synchronic phonology} \label{sec:synchr}
As shown in \citet[1098]{jacques12khaling}, Khaling has eighteen vowel (Table \ref{tab:vowels}) phonemes and 27 consonantal phonemes (Table \ref{tab:consonants}). The only word-initial clusters allowed are labial or velar stop + \ipa{l} and \ipa{r}. 

The consonant \ipa{ç} only appears as the first element on a word-internal cluster (as in \ipa{seçki} `we kill it'), never word-initially or word-finally. Only unvoiced obstruents and sonorant \ipa{--p}, \ipa{--t} \ipa{--k}, \ipa{--m}, \ipa{--n}, \ipa{--ŋ}, \ipa{--r}, \ipa{--l}, \ipa{--s} and \ipa{--j} can occur as codas. Clusters involving two unvoiced stops or affricates, including geminates (\ipa{tt}, \ipa{pp}, \ipa{kk} and \ipa{ʦʦ}), are realized with preaspiration when the preceding vowel is short.

Unique among Sino-Tibetan languages, Khaling has a contrast between simple stop codas and geminated stop codas, realized as preaspirated final consonants as in \ipa{pɛpp} [pʲɛʰp] `father'  or \ipa{pʰɛtt} [pʰʲɛʰt] `egg'.

\begin{table}[H]
\caption{List of Khaling vowel phonemes} \label{tab:vowels}\centering
\begin{tabular}{llllll}
\ipa{i iː} & \ipa{ʉ ʉː} & &&\ipa{u uː} \\
\ipa{e eː} & \ipa{ɵ ɵː} & &&\ipa{o oː} \\
\ipa{ɛ ɛː} &   & &\ipa{ʌ} &  \ipa{oɔ} \\
&&\ipa{a aː}\\
\end{tabular}
\end{table}

\begin{table}[H]
\caption{List of Khaling consonantal phonemes} \label{tab:consonants}\centering
\begin{tabular}{llllll}
\ipa{p} & \ipa{t} &&\ipa{ʦ}  & \ipa{k}&\ipa{ʔ}\\
\ipa{pʰ} & \ipa{tʰ} &&\ipa{ʦʰ}  & \ipa{kʰ}&\\
\ipa{b} & \ipa{d} &&\ipa{ʣ}  & \ipa{g}&\\
\ipa{bʰ} & \ipa{dʰ} &&\ipa{ʣʰ}  & \ipa{gʰ}&\\
\ipa{m} & \ipa{n} && & \ipa{ŋ}&\\
  & \ipa{s} && \ipa{ç}& &\ipa{ɦ}\\
  \ipa{w} & \ipa{l} &\ipa{r}&\ipa{j}  & &\\
\end{tabular}
\end{table}

Khaling has a two-way tonal contrast on open syllables with long vowels or closed syllables with a sonorant coda; There is no contrast on short vowels and on closed syllables with an obstruent coda. Table \ref{tab:minimal.pairs} illustrates possible tonal alternation on monosyllables.

\begin{table}[H]
\caption{Minimal pairs} \label{tab:minimal.pairs}\centering
\begin{tabular}{llllll}
\toprule
Form & Meaning\\
\midrule
\ipa{mɛ̄m} & there\\
\ipa{mɛ̂m} & mother\\
\ipa{mɛ̄ː} & there\\
\ipa{mɛ̂ː} & (ideophone) \\
\ipa{mɛ} & that\\
\bottomrule
\end{tabular}
\end{table}

A third tone is attested in the purposive  constructions with the locative suffix \ipa{-bi} followed by a motion verb. When the element suffixed by \ipa{-bi} is a verb  whose root ends in a sonorant consonant \ipa{m}, \ipa{n}, \ipa{ŋ}, \ipa{l} or \ipa{r},\footnote{The only verbal form allowed in the purposive construction is the bare infinitive, which has not TAM or person markers and no suffix other than \ipa{-bi}; the final root consonant \ipa{n} changes to \ipa{j} in this context.} the verb has regular high tone, as in example \ref{ex:cow}.

\begin{exe}
\ex \label{ex:cow}
\gll \ipa{bʌ̂j} \ipa{ʔu-gʰas} \ipa{kɛ̄m-bi} \ipa{kʰɵs-tɛ}  \\
cow \textsc{3sg.poss}-grass chew-\textsc{loc} go-\textsc{pst:3sg} \\
\glt `The cow went to chew the grass.'
\end{exe}

Noun  and verb forms with falling tone\footnote{The only bare infinitive forms with a  falling tone are those derived from verbal roots in \ipa{t} and \ipa{k}.} remain unchanged as in \ref{ex:elk}: the noun `elk' \ipa{kɛ̂m} has the same form as a free word and with the locative suffix.

\begin{exe}
\ex \label{ex:elk}
\gll   \ipa{kɛ̂m-bi} \ipa{kʰɵs-tɛ}  \\
elk-\textsc{loc} go-\textsc{pst:3sg} \\
\glt `He went (to hunt) for the elk.'
\end{exe}


On the other hand, monosyllabic nouns with high tone develop a low tone in the purposive construction when suffixed by \ipa{-bi}, as \ipa{kɛ̄m}  `work'\footnote{Borrowed from Nepali \ipa{kam}.} in example \ref{ex:work}.
\begin{exe}
\ex \label{ex:work}
\gll   \ipa{kɛ̀m-bi} \ipa{kʰɵs-tɛ}  \\
work-\textsc{loc} go-\textsc{pst:3sg} \\
\glt `He went for his work.'
\end{exe}


Given the fact that this third tonal contrast is fairly restricted, and that it does not appear in finite verbal forms, we will not take it into consideration in the rest of this paper.


In the following, we show the origin of the tonal and length contrasts in Khaling. We point of two origins for the falling tone: loss of obstruent codas and syllable reducation.

\section{Short and long vowels}
%distinguish pre-Khaling 1 and pre-khaling 2, too early to provide proto-kiranti or even proto-khaling-dumi reconstructions
The complex vowel system of Khaling was innovated from a simpler   vowel system, as can be ascertained from the vowel alternations in the verbal system and the complementary distributions between vowels and codas (\citealt{jacques12khaling}). Previous scholarship on proto-Kiranti (\citealt{driem90r}, \citealt{michailovsky94stops}, \citealt{starostin94kiranti}, \citealt{opgenort05jero}) has focused on the reconstruction of proto-Kiranti consonants.

As shown in \citet{jacques12khaling} and \citet{michailovsky12dumi}, in both Khaling and Dumi, no more than five vowels have to be postulated in verbal roots. Complex morphophonological alternations yield all 18 vowels in different contexts. Table \ref{tab:basic.alternations} illustrates  the most important alternations: closed syllables verb stems can be classified into two major categories, weak and strong. Weak stems are found mainly in forms with vowel initial suffixes (the exceptions are shown in Tables \ref{tab:intrans.paradigm}, \ref{tab:trans.paradigm} and \ref{tab:trans.cvct.paradigm} and discussed in section \ref{sec:tonogenesis.verb}), and strong stems are   found with consonant-initial suffixes except in some parts of the paradigm of CVCt roots (see Table \ref{tab:trans.cvct.paradigm} and the discussion in section \ref{sec:cvct}).

\begin{table}[H]
\caption{Basic vowel alternations in the Khaling verb } \label{tab:basic.alternations} \centering
\begin{tabular}{lllllllll}
 root vowel & 	open   & 	velar  & 	velar  & 	non-velar    & 	non-velar   \\
& syllable&(strong) &(weak) &(strong) &(weak) \\
\ipa{a}  & 	\ipa{ɛ}  & 	\ipa{a}  & 	\ipa{ʌ}  & 	\ipa{ɛ}  & 	\ipa{ɛ}  \\ 	
\ipa{e}  & 	\ipa{e}  & 	\ipa{e}  & 	\ipa{e}  & 	\ipa{e}  & 	\ipa{e}  \\ 	
\ipa{i}  & 	\ipa{e}  & 	\ipa{ʌ}  & 	\ipa{i}  & 	\ipa{ʌ}  & 	\ipa{i}  \\ 	
\ipa{o}  & 	\ipa{ɵ}  & 	\ipa{o}  & 	\ipa{ɵ}  & 	\ipa{oɔ}  & 	\ipa{ɵ}  \\ 	
\ipa{u}  & 	\ipa{ʉ}  & 	\ipa{u}  & 	\ipa{ʉ}  & 	\ipa{ʌ}  & 	\ipa{ʉ}  \\ 	
\end{tabular}
\end{table}

Some stem forms (\textsc{3sg:n.pst$\rightarrow$3sg} and \textsc{3sg:n.pst$\rightarrow$3sg}) have a lengthened vowel (see Table \ref{tab:basic.alternations.ex}), which we interpret as reflecting radical stress (rather than suffixal stress) in pre-Khaling.

\begin{table}[H]
\caption{Examples of basic vowel alternations in the Khaling verb } \label{tab:basic.alternations.ex} \centering
\resizebox{\columnwidth}{!}{
\begin{tabular}{lllllllll}
\toprule
Root& Meaning &   weak stem& weak stem (lengthened)& strong stem  &\\
&(\textsc{1di:n.pst$\rightarrow$3sg})&(\textsc{3sg:n.pst$\rightarrow$3sg}) &\textsc{1pi:n.pst$\rightarrow$3sg}\\
\midrule
 |\ipa{pʰrok}|  & untie &\ipa{pʰrɵk-i}& \ipa{pʰrɵ̄ːg-ʉ} & \ipa{pʰrok-ki} \\
 |\ipa{lom}|  & search &\ipa{lɵp-i}& \ipa{lɵ̄ːm-ʉ} & \ipa{loɔ̄m-ki} \\
 |\ipa{lop}| & catch &\ipa{lɵp-i} &  \ipa{lɵ̄ːb-ʉ} & \ipa{loɔ̄p-ki} \\
 \bottomrule
\end{tabular}}
\end{table}


The tables \ref{tab:intrans.paradigm}, \ref{tab:trans.paradigm} and  \ref{tab:trans.cvct.paradigm} illustrate the distribution of the stems in the Khaling verbal paradigms. Irregular cases, where a weak stems is found with a consonant-initial suffix, are shaded in grey. These tables only represent   the stem vowel alternations, and do not include information on   consonant shifts, which are not the focus of the present paper.

\begin{table}
\caption{Distribution of the stems in the intransitive paradigm } \label{tab:intrans.paradigm} \centering
\begin{tabular}{lllll}
\toprule
S & non-past & past & imperative\\
\midrule
\textsc{1s} & strong-\ipa{ŋʌ} & weak-\ipa{ʌtʌ} \\
\textsc{1di} & weak-\ipa{i} & weak-\ipa{iti} \\
\textsc{1de} & weak-\ipa{u} & weak-\ipa{utu} \\
\textsc{1pi} & strong-\ipa{ki} & strong-\ipa{tiki} \\
\textsc{1pe} & strong-\ipa{kʌ} & strong-\ipa{tʌkʌ} \\
\textsc{2s} & \ipa{i}-strong & \ipa{i}-weak-\ipa{tɛ} \grise{} &weak-\ipa{je}\grise{}\\
\textsc{2d} & \ipa{i}-weak-\ipa{i} & \ipa{i}-weak-\ipa{iti} &weak-\ipa{ije} \\
\textsc{2p} & \ipa{i}-strong-\ipa{ni} & \ipa{i}-weak-\ipa{tɛnu}\grise{} &weak-\ipa{nuje}\grise{}\\
\textsc{3s} & strong & weak-\ipa{tɛ} \grise{}\\
\textsc{3d} & weak-\ipa{i} & weak-\ipa{iti} \\
\textsc{3p} & strong-\ipa{nu} & weak-\ipa{tɛnu}\grise{} \\
\bottomrule
\end{tabular}
\end{table}

\begin{table}
\caption{Distribution of the stems in the CVC transitive paradigm } \label{tab:trans.paradigm} \centering
\begin{tabular}{lllll}
\toprule
A$\rightarrow$P& non-past & past &imperative\\
\midrule
\textsc{1s}$\rightarrow$3 & weak-\ipa{u} & weak-\ipa{utʌ} \\
\textsc{1di}$\rightarrow$3 & weak-\ipa{i} & weak-\ipa{iti} \\
\textsc{1de}$\rightarrow$3 & weak-\ipa{u} & weak-\ipa{utu} \\
\textsc{1pi}$\rightarrow$3 & strong-\ipa{ki} & strong-\ipa{tiki} \\
\textsc{1pe}$\rightarrow$3 & strong-\ipa{kʌ} & strong-\ipa{tʌkʌ} \\
\textsc{2s}$\rightarrow$3 & \ipa{i}-weak.length-\ipa{ʉ} & \ipa{i}-weak.length-\ipa{tɛ} \grise{} &weak.length-\ipa{e}\\
\textsc{2d}$\rightarrow$3 & \ipa{i}-weak-\ipa{i} & \ipa{i}-weak-\ipa{iti} &weak-\ipa{ije}\\
\textsc{2p}$\rightarrow$3 & \ipa{i}-strong-\ipa{ni} & \ipa{i}-weak-\ipa{tɛnu}\grise{} &weak-\ipa{nuje}\grise{}\\
\textsc{3s}$\rightarrow$3 & weak.length-\ipa{ʉ} & weak.length-\ipa{tɛ} \grise{}\\
\textsc{3d}$\rightarrow$3 & weak-\ipa{su} & weak.length-\ipa{tɛsu} \grise{}  \\
\textsc{3p}$\rightarrow$3 & weak-\ipa{nu} & weak.length-\ipa{tɛnu}\grise{} \\
\midrule
\textsc{1s}$\rightarrow$2 & strong-\ipa{nɛ} & strong-\ipa{tɛni}\\
\textsc{2/3s$\rightarrow$1s} & \ipa{i}-strong-\ipa{ŋʌ} & \ipa{i}-weak-\ipa{ʌtʌ} & weak-\ipa{ʌje}\\
\textsc{3s$\rightarrow$1pi} & \ipa{i}-strong-\ipa{ki} & \ipa{i}-strong-\ipa{tiki} & strong-\ipa{kʌje}\\
\bottomrule
\end{tabular}
\end{table}


\begin{table}
\caption{Distribution of the stems in the CVCt transitive paradigm } \label{tab:trans.cvct.paradigm} \centering
\begin{tabular}{lllll}
\toprule
A$\rightarrow$P& non-past & past &imperative\\
\midrule
\textsc{1s}$\rightarrow$3 & strong-\ipa{u}\grise{}  & strong-\ipa{tʌ} \\
\textsc{1di}$\rightarrow$3 & weak-\ipa{i} & weak-\ipa{iti} \\
\textsc{1de}$\rightarrow$3 & weak-\ipa{u} & weak-\ipa{utu} \\
\textsc{1pi}$\rightarrow$3 & strong-\ipa{ki} & strong-\ipa{tiki} \\
\textsc{1pe}$\rightarrow$3 & strong-\ipa{kʌ} & strong-\ipa{tʌkʌ} \\
\textsc{2s}$\rightarrow$3 & \ipa{i}-strong-\ipa{ʉ} \grise{} & \ipa{i}-strong-\ipa{tɛ}  &strong-\ipa{e} \grise{} \\
\textsc{2d}$\rightarrow$3 & \ipa{i}-weak-\ipa{i} & \ipa{i}-weak-\ipa{iti} &weak-\ipa{ije}\\
\textsc{2p}$\rightarrow$3 & \ipa{i}-strong-\ipa{ni} & \ipa{i}-weak-\ipa{tɛnu}\grise{} &weak-\ipa{nuje}\grise{}\\
\textsc{3s}$\rightarrow$3 & strong-\ipa{ʉ} \grise{} & strong-\ipa{tɛ}\\
\textsc{3d}$\rightarrow$3 & weak-\ipa{su} & strong-\ipa{tɛsu}   \\
\textsc{3p}$\rightarrow$3 & strong-\ipa{nu} & strong-\ipa{tɛnu}\\
\midrule
\textsc{1s}$\rightarrow$2 & strong-\ipa{nɛ} & strong-\ipa{tɛni}\\
\textsc{2/3s$\rightarrow$1s} & \ipa{i}-strong-\ipa{ŋʌ} & \ipa{i}-weak-\ipa{ʌtʌ} & weak-\ipa{ʌje}\\
\textsc{3s$\rightarrow$1pi} & \ipa{i}-strong-\ipa{ki} & \ipa{i}-strong-\ipa{tiki} & strong-\ipa{kʌje}\\
\bottomrule
\end{tabular}
\end{table}

The vowels alternations in Tables \ref{tab:basic.alternations} and \ref{tab:basic.alternations.ex} suggest the existence of   three (historical) vowel shifts   in Khaling: fronting, lowering and backing. We provide an account of the Khaling vowel shifts on the basis of CVC intransitive and transitive paradigms in addition with some comparative data. Then, we discuss the case of the CVCt root verbs where additional minor sound laws have to be proposed. Finally, we tackle some additional issues concerning the Khaling vowel system in historical perspective.

\subsection{Fronting} \label{sec:fronting}
The non-front vowels *\ipa{a}, *\ipa{o} and *\ipa{u} in pre-Khaling are fronted to \ipa{ɛ}, \ipa{ɵ} and \ipa{ʉ} respectively in open syllables. This shift occurred in   open syllable roots, but also in weak stems, where (in pre-Khaling) suffixes are vowel-initial and the coda is resyllabified as the onset of the next syllable. 

Thus for instance  the proto-form *\ipa{lóm-u} (search-\textsc{3sg:n.pst$\rightarrow$3sg}) is resyllabified as *\ipa{ló.mu} and undergoes the shifts *\ipa{o} $\rightarrow$ \ipa{ɵ} and *\ipa{u} $\rightarrow$ \ipa{ʉ} to  *\ipa{lɵ́.mʉ} and the stress on the first syllable causes vowel lengthening to the attested form \ipa{lɵ̄ːmʉ} `he searches for it'.

This shift does not occur in regular velar final verb with |a| vocalism such as the transitive verb |kak| `peel', which has a weak stem in \ipa{ʌ} (\ipa{kʌg-u} `I peel it') or \ipa{aː}  (\ipa{kāːg-ʉ} `I peel it') instead of expected \ipa{ɛ} and \ipa{ɛː}.  Thus, the fronting shift was limited to rounded vowels when occurring before velar.\footnote{The intransitive   verb |\ipa{bʰak}| `go (honorific)' (\citealt[1115]{jacques12khaling}) presents the stem  /\ipa{bʰɛ}/ in dual forms (\ipa{bʰɛ-ji} go-\textsc{n.pst:1pi}) with irregular loss of  the final \ipa{k} showing the \ipa{a} $\rightarrow$ \ipa{ɛ} shift (*\ipa{bʰʌk-i} would be expected for the \textsc{n.pst:1pi}). While the reason for the coda loss in this form is unclear, it this verbs confirms the conditioning of the *\ipa{a} $\rightarrow$ \ipa{ɛ} proposed here.}

 Only one verb  does not fit the pattern in Table \ref{tab:basic.alternations} :  |\ipa{jal}|  `strike', which displays \ipa{a} / \ipa{ʌ} alternation like roots with velar codas (\ipa{jʌl-u} \textsc{n.pst:1sg$\rightarrow$3sg}, \ipa{ja‍̄l-ki} \textsc{n.pst:1pi$\rightarrow$3sg}). This exception is due to the fact that this verb was borrowed from Thulung (a neighbouring Kiranti language) after the vowel shift took place.

The fronting shift as defined here is confirmed   by the fact that some borrowings from Nepali do present the sound shifts that have been postulated on the exclusive basis of internal reconstruction.

\begin{table}
\caption{Borrowings from Nepali exhibiting the fronting vowel shift} \label{tab:nep.fronting} \centering
\begin{tabular}{llll}
\toprule
Nepali & Meaning & Khaling \\
\midrule
\ipa{kam} & work & \ipa{kɛ̄m} \\
\ipa{khorsani} & chilli&\ipa{khɵrsɛ̂i}\\
\ipa{buɖʰi} & wife&\ipa{bʉri}\\
\ipa{dudʰ} & milk&\ipa{dʉt}\\
%pumpkin & \ipa{pʰɵrsi} \\
\bottomrule
\end{tabular}
\end{table}


Predicts no syllables like
\subsection{Lowering} \label{sec:lowering}
The high vowels *\ipa{i}, *\ipa{u}   in pre-Khaling are lowered to \ipa{ʌ},  and   *\ipa{o} changes to \ipa{oɔ} in closed syllables with a non-velar coda. Closed syllable occur in forms with consonant-initial suffixes, where resyllabification is not possible,\footnote{See section \ref{sec:cvct} for the case of CVCt verb roots.} For instance, in the pre-Khaling form *\ipa{lom-ki} (search-\textsc{npst:1pi$\rightarrow$3s}), the coda *\ipa{m} cannot be resyllabified, and since the verb stem remains a closed syllable, its vowel cannot undergo the fronting shift. This forms fulfills the conditions for the lowering to occur and *\ipa{o} regularly changes to the diphthong \ipa{oɔ} in this context, yielding the attested form \ipa{loɔ̄m-ki}.


This sound change predicts that there should not be in Khaling any word with the vowels \ipa{i}, \ipa{u}, \ipa{o} and their fronted equivalents \ipa{ʉ} and \ipa{ɵ} in closed syllables.

\subsection{Backing} \label{sec:backing}

\subsection{The case of CVCt roots} \label{sec:cvct}
 
\subsection{The origin of Khaling long vowels} \label{sec:long.vowel}


\subsection{Problematic vowel correspondences} \label{sec:vowel.correspondences}
a : o
a :ɛ

two a-like vowels in pre-proto-khaling *a and *ɑ
bhɵ̄l	strength
 
bhɵri	full

no claim of being ancient
\section{Falling tone from lost obstruents} \label{sec:obstruents}
First, they comes from the loss or nasalization of final stops *\ipa{-k}, *\ipa{-p} and *\ipa{-t}, as can be assessed using comparative data from Limbu and Dumi (\citealt{driem93dumi}, \citealt{michailovsky02dico}).

\begin{table}[H]
\caption{Examples of falling tones originating from final stops} \centering
\begin{tabular}{lllll}
\toprule
Pre-Khaling	&Limbu	&Dumi	&Khaling	&Meaning\\
\midrule
\ipa{*bit}	& \ipa{pit} &	\ipa{bhiʔi}	 & \ipa{bʌ̂j} &	cow\\
\ipa{*met} &	\ipa{met}	& \ipa{meːʔe} &	\ipa{mêj} &	wife\\
\ipa{*rak}	& \ipa{yak}	& &	\ipa{rôː}	& cliff \\
\ipa{*pak} &	\ipa{phak}	& \ipa{poʔo}	& \ipa{pôː}	& pig\\
\ipa{*ʔik}	& \ipa{ik}	& &	\ipa{ʔûː}	& field\\
\bottomrule
\end{tabular}
\end{table}
In monosyllables, *-t changes to –j preceded by falling tone, while *-k generates a falling tone and lengthened vowel. *-p remains a stop in Khaling as a coda, but can be nasalized to –m with falling tone if followed by a nasal-initial suffix as in the infinitive and various verbals forms:
\begin{table}[H]
\caption{Examples of falling tones originating from the nasalization of obstruents} \centering
\begin{tabular}{lllll}
\toprule
Pre-Khaling	&Limbu	&Dumi	&Khaling	&Meaning\\
\midrule
\ipa{*lop-na}	& & \ipa{lopnɨ}	 & \ipa{loɔ̂m-nɛ}	&to catch\\
\ipa{*rep-na}	& & \ipa{repnɨ	}& \ipa{rêm-nɛ}	&to stand\\
\bottomrule
\end{tabular}
\end{table}

\section{Falling tone from the reduction of disyllables} \label{sec:disyll}
Another origin for the falling tone comes from the reduction of disyllables: in inherited disyllables, the vowel and coda of the second syllable are always lost, and either the coda of the first syllable or the onset of the second syllable becomes the coda.
\begin{table}[H]
\caption{Examples of falling tone originating from the simplification of a disyllable} \centering
\begin{tabular}{llllll}
\toprule
Pre-Khaling	&Limbu	&Dumi	&Khaling	&Meaning\\
\midrule
*kɑ-mit	& khaˀmit	& kɨhɨm	&	kɵ̂m		&cloud\\
*mama	&ma		& mama	&	mɛ̂m		&mother\\
*khaliŋ				&&&	kʰɛ‍̂l	&	Khaling\\
*sili			&&	tsili	&	sîl	&	anger\\
*nam-ni	&&		naːmnɨ	&	nɛ̂m		in two days\\
*meri		&&		miri	&	mêr	&	tail\\
*nini			&&	nini	&	nîn		&aunt (FZ)\\
\bottomrule
\end{tabular}
\end{table}

\section{Tonogenesis and verbal morphology} \label{sec:tonogenesis.verb}
These origins can explain most of the tonal alternations observed in the verbal system (see \citealt{jacques12khaling}). The first origin accounts for the alternations between obstruent codas and falling tones as in Table \ref{tab:falling.verb}.

\begin{table}[H]
\caption{Falling tones from lost stops in the Khaling verbal system} \centering \label{tab:falling.verb}
\begin{tabular}{llllll}
\toprule
Verb root	&Meaning	&Non-Past \textsc{3sg} & Past \textsc{3sg}\\
\midrule
|ʔɛt|	&	say			&\ipa{ʔɛ̂j} &\ipa{ʔɛs-tɛ} \\
|tot|	&	be visible			&\ipa{toɔ̂j} &\ipa{tɵs-tɛ} \\
|ɦuk|	&	bark			&\ipa{ɦûː} &\ipa{ɦʉk-tɛ} \\
\bottomrule
\end{tabular}
\end{table}
 
 The second origin explains   alternations such as those illustrated in Table  \ref{tab:falling.verb2}, where comparative evidence shows that some syllables have been lost.
\begin{table}[H]
\caption{Falling tones from lost syllables in the Khaling verbal system} \centering \label{tab:falling.verb2}
\begin{tabular}{llllll}
\toprule
Verb root	&Meaning	&Non-Past \textsc{1sg} & Non-Past \textsc{3pl} \\
\midrule
|lom|	&	look for		&\ipa{lom-u} &\ipa{lɵ̂m-nu} $\leftarrow$ *lɵ́mʉnu\\
|pʰrok|	&	untie		&\ipa{pʰrog-u} &\ipa{pʰrɵ̂ːk-nu}$\leftarrow$ *pʰrɵ́gʉnu \\
|bʰert|	&	cause to fly			&\ipa{bʰerd-u} &\ipa{bhêr-nu} $\leftarrow$ *bhérdʉnu \\
\bottomrule
\end{tabular}
\end{table}

strong stem vs weak stem: consonant-initial vs vowel initial suffixes. (except for Ct verbs, where weak stems are only present in dual forms where the t is deleted)
strong stem, also with zero suffix

Exceptions: lɵ̂ːp-su/nu lɵ̂ːptɛ ʔilɵptɛ lɵpnuje npst:3>3dp pst:2/3sp>3 3/1>2sp imp.2p>3
sɵptɛ, sɵpje, sɵpnuje pst:2/3sp, imp2sp

\section{Unexplanable alternations}
Yet, there remain a residue of forms with falling tones that cannot be accounted for by either scenario, in particular the falling tone in past second and third singular forms. The present paper evaluates several hypothesis, including analogy, alternative proto-forms or alternative phonetic laws to explain these exceptions.


\bibliographystyle{Linquiry2}
\bibliography{bibliogj}
\end{document}
