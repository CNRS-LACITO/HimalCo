\documentclass[oneside,a4paper,11pt]{article} 
\usepackage{fontspec}
\usepackage{natbib}
\usepackage{booktabs}
\usepackage{xltxtra} 
\usepackage{polyglossia} 
\usepackage[table]{xcolor}
\usepackage{tikz}
\usetikzlibrary{trees}
\usepackage{gb4e} 
\usepackage{multicol}
\usepackage{graphicx}
\usepackage{float}
\usepackage{hyperref} 
\hypersetup{bookmarks=false,bookmarksnumbered,bookmarksopenlevel=5,bookmarksdepth=5,xetex,colorlinks=true,linkcolor=blue,citecolor=blue}
\usepackage[all]{hypcap}
\usepackage{memhfixc}
\usepackage{lscape}
\usepackage{bbding}
 
%\setmainfont[Mapping=tex-text,Numbers=OldStyle,Ligatures=Common]{Charis SIL} 
\newfontfamily\phon[Mapping=tex-text,Ligatures=Common,Scale=MatchLowercase]{Charis SIL} 
\newcommand{\ipa}[1]{{\phon\textbf{#1}}} 
\newcommand{\grise}[1]{\cellcolor{lightgray}\textbf{#1}}
\newfontfamily\cn[Mapping=tex-text,Ligatures=Common,Scale=MatchUppercase]{SimSun}%pour le chinois
\newcommand{\zh}[1]{{\cn #1}}
\newcommand{\Y}{\Checkmark} 
\newcommand{\N}{} 
\newcommand{\dhatu}[2]{|\ipa{#1}| `#2'}
\newcommand{\jpg}[2]{\ipa{#1} `#2'}  
\newcommand{\refb}[1]{(\ref{#1})}
\newcommand{\tld}{\textasciitilde{}}

 \begin{document} 
\title{The obstruentization of laterals in Jinghpo}
\author{Guillaume Jacques\\ CNRS-CRLAO-INALCO}
\maketitle

\section*{Introduction}
In Jinghpo, a handful of etyma have a dental \ipa{t} corresponding to a lateral in the rest of the family, as illustrated by the data in Table  \ref{tab:mada} with Tibetan comparanda (data from \citealt{xu83jingpo}). This correspondence, already pointed out by \citet[50-63]{matisoff03}, has not yet received a satisfactory explanation.
 

\begin{table}[H]
\caption{Correspondence of Jinghpo \ipa{t} to laterals onsets in Tibetan} \centering \label{tab:mada}
\begin{tabular}{lllllllllll}
\toprule
Jinghpo & Meaning & Tibetan \\
\midrule
\ipa{mă³¹ta̱ʔ⁵⁵} &lick& \ipa{ldag} $\leftarrow$ \ipa{*N-lag} \\
\ipa{ʃă³ta̱³³}	&moon&	\ipa{zla} \\
\ipa{lă³¹ta̱ʔ⁵⁵}	&hand&	\ipa{lag.pa} \\
\ipa{ʃă³¹tai³³}	&navel&	\ipa{lte.ba} $\leftarrow$ \ipa{*t-le} \\
\bottomrule
\end{tabular}
\end{table}

In this paper, I first propose an account for this correspondence based on typological parallels, then present a list of exceptions to the hypothesized sound law, and finally discuss several hypotheses to account for these exceptions.

\section{The obstruentization of laterals}
Fortition of lateral into dental stops is a common sound change in the Trans-Himalayan family. In this section, I present several attested examples, and show how these can be used to interpret the correspondence in Table \ref{tab:mada}.

\subsection{Stau}
While Stau has three lateral phonemes \ipa{l}, \ipa{ɮd} and \ipa{ɬ} (\citealt{jacques17stau}), only the latter two are found in clusters. 

Morphology provides irrefutable evidence that this distribution is due to a general sound change \ipa{*Cl} $\rightarrow$ \ipa{Cɮd-}. For instance, the regular \ipa{s-} causative of \ipa{l-} initial verbs is \ipa{zɮd-} (\ipa{lə} `boil (intr)' $\rightarrow$ \ipa{zɮdə} `boil (tr)'), attesting the sound change \ipa{*sl-} $\rightarrow$ \ipa{zɮd-} in this language.

Given the absence of any such alternation in other Gyalrongic languages, despites their highly complex morphophonology (\citealt{jackson07shangzhai}, \citealt{jacques15causative}, \citealt{lai16caus}), it is clear that this is a Stau-specific recent sound change.

\subsection{Pre-Tibetan}
The fate of proto-Bodic lateral into Tibetan is very complicated (\citealt{hill11laws, hill13laterals}) and in part controversial. 

The sound changes \ipa{*ml-} $\rightarrow$ \ipa{md-} and \ipa{*nl-} $\rightarrow$ \ipa{ld-} are however well attested by morphological alternations (\ipa{loŋ} `be blind', \ipa{mdoŋs.pa} `blind', \ipa{ldoŋ} `go blind') and external comparison (see for instance Tibetan \ipa{mda} `arrow' vs Burmese \ipa{mla³}), and constitute a unique definitory feature of Tibetan, not shared with its closest relatives (see \citealt{jacques04thimphu}).

\subsection{Central Tibetan}
Classical Tibetan \ipa{zl-} corresponds to \ipa{ld-} (or onsets derived thereof) in Central Tibetan dialects. As pointed out by \citet{gong16ld}, this correspondence can be accounted for by the series of intermediate sound changes in (\ref{ex:zl}):

\begin{exe}
\ex \label{ex:zl}
\glt \ipa{zl-} $\rightarrow$ \ipa{*zɮ} $\rightarrow$ \ipa{*zɮd-} $\rightarrow$ \ipa{*zld-} $\rightarrow$ \ipa{ld-}
\end{exe}

\subsection{Jinghpo} \label{sec:jinghpo.Cl}

\begin{exe}
\ex \label{ex:Cl}
\glt \ipa{*Cl-} $\rightarrow$ \ipa{*Cɮd-} $\rightarrow$ \ipa{*Cd-} $\rightarrow$ \ipa{*Cə-d-} $\rightarrow$ \ipa{Că.t-}
\end{exe}






\section{Exceptions}

\subsection{Prefixes}

\subsubsection{Causative \ipa{ʃă-}}
The causative prefix \ipa{ʃă-} (and its various allomorphs) is discussed in detail by \citet[72-3]{dai90yufa} and \citet[88-90]{kurabe16jinghpo}. As shown by Table \ref{tab:sha.caus}, verbs in \ipa{l} initial do not change to \ipa{t} following the causative \ipa{ʃă-}, as could have been expected based on the proposed sound law in \ref{sec:jinghpo.Cl}.

\begin{table}[H]
\caption{Absence of obstruentization in \ipa{ʃă-} prefixed verbs in Jinghpo} \centering \label{tab:sha.caus} 
\begin{tabular}{lllll}
\toprule
Base verb & Derived noun \\
\midrule
\ipa{lai³¹} `pass, go over' (\zh{过;过头}) & \ipa{ʃă³¹lai³¹} `cause to pass' (\zh{使……过去}) \\
\ipa{let⁵⁵} `be intelligent' (\zh{聪明}) & \ipa{ʃă³¹let⁵⁵} `cause to become intelligent' \\
&(\zh{使……聪明}) \\
\ipa{lim⁵⁵} `be covered,  (\zh{覆盖;淹没}) & \ipa{ʃă³¹lim⁵⁵} `cover, submerge'  \\
be submerged'& (\zh{使……淹没;使……覆盖})\\
\bottomrule
\end{tabular}
\end{table}

\subsubsection{Nominalization \ipa{mă-}}
While existing grammars of Jinghpo do not recognize the existence of a nominalization prefix \ipa{mă-}, \citet[115]{dai90yufa} suggests the derivation \ipa{kun³³} `carry on the back' (\zh{背;担任;携带}) $\rightarrow$ \ipa{mă³¹kun³³} `(one) burden' (classifier). A few more examples of this prefix are shown in Table \ref{tab:ma.nmlz}; we see that here again \ipa{l-} initial words do not undergo obstruentization.

\begin{table}[H]
\caption{Evidence for a \ipa{mă-} nominalization prefix in Jinghpo} \centering \label{tab:ma.nmlz}
\begin{tabular}{lllll}
\toprule
Base verb & Derived noun \\
\midrule
\ipa{juʔ⁵⁵} `come down' (\zh{下}) & \ipa{mă³¹juʔ⁵⁵} `downslope' (\zh{下坡}) \\
\ipa{ra³¹} `lack, need' (\zh{差;需要;差错}) &\ipa{mă³¹ra³¹} `error' (\zh{错误;罪}) \\
\ipa{raʔ³¹} `like' (\zh{爱}) &\ipa{mă³¹raʔ³¹} `wish' (\zh{意愿}) \\
\midrule
\ipa{lai⁵⁵} `exchange' (\zh{换}) &\ipa{mă³¹lai⁵⁵} `substitute' (\zh{替身;代用品}) \\
\ipa{lun⁵⁵} `put above' (\zh{往高处放东西}) &\ipa{mă³¹lun⁵⁵} `uphill road' (\zh{上坡路}) \\
\bottomrule
\end{tabular}
\end{table}




\subsubsection{Causative \ipa{mă-}}
There is some evidence for a causative \ipa{mă-} prefix in the word pair \ipa{lum³³} `warm' $\rightarrow$ \ipa{mă³¹lum³³} `cook (porridge)'. In this example again, no obstruentization takes place in the prefixed verb form.

\subsection{Nouns}
Exceptions to obstruentization are not restricted to prefixed verb forms. Table \ref{tab:pala} presents a few examples which have \ipa{l} in Jinghpo with \ipa{mă-} and \ipa{pă-} presyllables, and have thus escaped this sound change.

\begin{table}[H]
\caption{Non-obstruentized laterals in noun roots} \label{tab:pala} \centering
\begin{tabular}{llll}
\toprule
Jinghpo & Tibetan   \\
\midrule
\ipa{pă̱⁵⁵la⁵⁵} `arrow' &\ipa{mda} $\leftarrow$ \ipa{*mla}   \\
\ipa{mă³¹la³¹} `soul' &\ipa{bla}  \\
\bottomrule
\end{tabular}
\end{table}


\section{Are the exceptions phonologically conditionned or analogical?}


Evidence for productivity: 
\ipa{taŋ⁵⁵} `be blocked' (\zh{挡住}) $\rightarrow$ \ipa{ʃă³¹taŋ⁵⁵} `block' (\zh{使……挡住})

\citet{dai90yufa} \citet{kurabe16jinghpo}


\section*{Conclusion}
\bibliographystyle{unified}
\bibliography{bibliogj}

 \end{document}
 