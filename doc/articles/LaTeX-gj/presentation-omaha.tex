\documentclass[xcolor=table]{beamer}
\usepackage{fontspec}
\usepackage{natbib}
\usepackage{booktabs}
\usepackage{xltxtra} 

\usepackage{polyglossia} 
\usepackage[table]{xcolor}
\usepackage{color}
\usepackage{gb4e} 

\usepackage{multicol}
\usepackage{graphicx}
\usepackage{float}
\usepackage{hyperref} 
\hypersetup{bookmarks=false,bookmarksnumbered,bookmarksopenlevel=5,bookmarksdepth=5,xetex,colorlinks=true,linkcolor=blue,citecolor=blue}
\usepackage{memhfixc} 

\newcommand{\tld}{\textasciitilde{}}

\newfontfamily\gender[Mapping=tex-text,Numbers=OldStyle,Ligatures=Common]{Times New Roman} 
\newfontfamily\phon[Mapping=tex-text,Ligatures=Common,Scale=MatchLowercase]{Charis SIL} 
\newcommand{\ipa}[1]{{\phon #1}} %API tjs en italique
 

\newfontfamily\cn[Mapping=tex-text,Ligatures=Common,Scale=MatchUppercase]{SimSun}%pour le chinois
\newcommand{\zh}[1]{{\cn #1}}
\XeTeXlinebreaklocale "zh" %使用中文换行
\XeTeXlinebreakskip = 0pt plus 1pt %
 \newcommand{\bleu}[1]{{\color{blue}#1}}
\newcommand{\rouge}[1]{{\color{red}#1}} 

\begin{document} 
  \begin{frame} 

 
\title{Le système de parenté omaha en japhug}
\author{Guillaume Jacques\\ CNRS-CRLAO-INALCO}
\date{10 décembre 2016}
\maketitle
\end{frame}   

  \begin{frame} 
 \frametitle{Le japhug} 
 \includegraphics{mapKamnyu.pdf}
  \end{frame} 
  
  \begin{frame} 
 \frametitle{Abbréviations} 
 
\begin{tabular}{ll|ll|ll|ll}
\toprule
F & père & B & frère & S & fils & H & mari \\
M & mère & Z & soeur & D & fille & W & épouse \\
\bottomrule
\end{tabular}

\begin{itemize}
\item MB = frère de la mère
\item FZ = soeur du père
\end{itemize}
\end{frame}

  \begin{frame} 
 \frametitle{Système soudanais} 
\includegraphics[width=\textwidth]{kinchart-sudan.pdf}
\end{frame}

  \begin{frame} 
 \frametitle{Système iroquois} 
\includegraphics[width=\textwidth]{kinchart-iroquois.pdf}
\end{frame}

  \begin{frame} 
 \frametitle{Système eskimo} 
\includegraphics[width=\textwidth]{kinchart-eskimo.pdf}
\end{frame}

  \begin{frame} 
 \frametitle{Système hawaiien} 
\includegraphics[width=\textwidth]{kinchart-hawaii.pdf}
\end{frame}

  \begin{frame} 
 \frametitle{Système japhug} 
\includegraphics[width=\textwidth]{kinchart-japhug-c1.pdf}
\end{frame}

  \begin{frame} 
 \frametitle{Système japhug} 
\includegraphics[width=\textwidth]{kinchart-japhug-c1-2.pdf}
\end{frame}
  \begin{frame} 
 \frametitle{Préfixes possessifs en japhug} 
 \begin{tabular}{lllllllll} 
\toprule
 Préfixe & Personne & Exemple\\
\midrule
\ipa{a--}  & 1\textsc{sg} & \ipa{a-rpɯ} `mon oncle' \\
\ipa{nɤ--}  & 2\textsc{sg} & \ipa{nɤ-ɬaʁ} `ta tante' \\
\ipa{ɯ-}		& 3\textsc{sg} & \ipa{ɯ-ftsa} `son neveu'\\
\midrule
\ipa{tɕi--}  &	 1\textsc{du}  \\
\ipa{ndʑi--}  & 2\textsc{du} \\	
\ipa{ndʑi--}  & 3\textsc{du} \\	
\midrule
\ipa{i--}  & 1\textsc{pl} \\
\ipa{nɯ--}  & 2\textsc{pl} \\
\ipa{nɯ--}  & 3\textsc{pl} \\
\midrule
\ipa{tɯ--},  \ipa{tɤ--} & indéfini & \ipa{tɤ-rpɯ} `l'oncle'\\
\ipa{tɯ--}   &  générique & \ipa{tɯ-rpɯ} `son (propre) oncle'\\
\bottomrule
\end{tabular}
\end{frame}

  \begin{frame} 
 \frametitle{Texte 1} 
 \begin{exe}
\ex 
\gll 
 \ipa{tɯ-rpɯ} 	\ipa{ɯ-rɟit} 	\ipa{ɯ-ɕki} 	\ipa{tɕe} 	\ipa{tɕe} 	\ipa{a-rpɯ} \ipa{a-ɬaʁ} 	\ipa{tu-kɯ-ti} 	\ipa{ŋu}  \\
 \textsc{genr.poss}-MB \textsc{3sg.poss}-enfant \textsc{3sg-dat} \textsc{lnk}  \textsc{lnk}  \textsc{1sg.poss}-MB \textsc{1sg.poss}-MZ \textsc{ipfv-genr}-dire être:\textsc{fact} \\
\glt `On dit `mon oncle, ma tante maternelle' aux enfants de son oncle maternel.'
\end{exe}

\end{frame}

  \begin{frame} 
 \frametitle{Texte 1} 
\begin{exe}
\ex 
\gll 
\ipa{tɤ-pɤtso,} 	\ipa{tɯrme} 	\ipa{ɯ-lɯz} 	\ipa{pɯ-nɯ-xtɕi,} 	\ipa{pɯ-nɯ-wxti} 	\ipa{nɯnɯ} 	\ipa{tɤ-rpɯ} 	\ipa{ɣɯ} 	\ipa{ɯ-rɟit} 	\ipa{nɯnɯ} 	\ipa{qhe} 	\ipa{pɯ\tld{}pɯ-kɯ-ɤmphɯmphri} 	\ipa{tɤ-rpɯ} 	\ipa{ʁɟa,} 	\ipa{tɤ-ɬaʁ} 	\ipa{ʁɟa} 	\ipa{ɕti}   \\
 \textsc{indef.poss}-enfant homme  \textsc{3sg.poss}-âge \textsc{pst.ipfv-auto}-être.petit \textsc{pst.ipfv-auto}-être.grand \textsc{dem}    \textsc{indef.poss}-MB \textsc{gen} \textsc{3sg.poss}-enfant \textsc{dem} \textsc{lnk} \textsc{total\tld{}pst-nmlz:S/A}-de.génération.en.génération \textsc{indef.poss}-MB complètement \textsc{indef.poss}-MZ complètement être:\textsc{assert}:\textsc{fact} \\
\glt `Quel que soit leur âge, les enfants de l'oncle maternel sont tous appelés `oncle maternel' ou   tante maternelle' de génération en génération. 
\end{exe}
\end{frame}

  \begin{frame} 
 \frametitle{Texte 1} 
 \begin{exe}
\ex 
\gll 
\ipa{tɯ-rpɯ} 	\ipa{ɣɯ} 	\ipa{ɯ-rɟit} 	\ipa{nɯ} 	\ipa{ɣɯ} 	\ipa{ɯ-rɟit} 	\ipa{ɣɯ} 	\ipa{ɯ-rɟit} 	\ipa{kɯ-fse,}  \ipa{tɯ-rpɯ} 	\ipa{ɣɯ} 	\ipa{ɯ-rɟit} 	\ipa{ɯ-ɕki} 	\ipa{tɕe} 	\ipa{a-rpɯ} 	\ipa{tu-kɯ-ti,} \\
\textsc{genr.poss}-MB \textsc{gen} \textsc{3sg.poss}-enfant \textsc{dem} \textsc{gen} \textsc{3sg.poss}-enfant \textsc{gen} \textsc{3sg.poss}-enfant \textsc{nmlz}:S/A-être.ainsi \textsc{genr.poss}-MB  \textsc{gen} \textsc{3sg.poss}-enfant \textsc{3sg-dat} \textsc{lnk} \textsc{1sg.poss}-MB \textsc{ipfv-genr}-dire \\
\glt `En ce qui concerne les enfants des enfants des enfants de son oncle maternel, on appelle les enfants de son oncle maternel `mon oncle maternel',
\end{exe}
\end{frame}

  \begin{frame} 
 \frametitle{Texte 1} 
 \begin{exe}
\ex 
\gll 
\ipa{ɯ-ɣe} 	\ipa{ɯ-ɕki} 	\ipa{tɕe} 	\ipa{a-rpɯ} 	\ipa{tu-kɯ-ti} 	\ipa{kɯ-ŋgrɤl} 	\ipa{ɲɯ-ŋu,} \ipa{iʑora} 	\ipa{kɯrɯ} 	\ipa{kɯ} 	\ipa{tɕe.} \\
\textsc{3sg.poss}-petit-fils \textsc{3sg-dat} \textsc{lnk} \textsc{1sg.poss}-MB \textsc{ipfv-genr}-dire \textsc{inf:stat}-être.habituellement.le.cas \textsc{sens}-être \textsc{1pl} tibétain \textsc{erg} \textsc{lnk} \\
\glt et l'on appelle ses petits-fils `mon oncle maternel', nous les tibétains.'
\end{exe}
\end{frame}

  \begin{frame} 
 \frametitle{Texte 1} 
 \begin{exe}
\ex 
\gll 
\ipa{tɯ-ɲi} 	\ipa{ɣɯ} 	\ipa{chonɤ} 	\ipa{tɯ-βɣo} 	\ipa{ɣɯ} 	\ipa{nɯra} 	\ipa{ri} 	\ipa{tɕe} 	\ipa{tɕe} 	\ipa{nɯra} 	\ipa{kɤ-ti} 	\ipa{maŋe.} 	\\
\textsc{genr.poss}-FZ \textsc{gen} \textsc{comit} \textsc{genr.poss}-FB \textsc{gen} \textsc{dem:pl} \textsc{loc} \textsc{lnk}  \textsc{lnk}  \textsc{dem:pl} \textsc{nmlz:P}-dire  ne.pas.exister:\textsc{sens} \\
\glt `On ne le dit pas à ceux (les enfants) de ses tantes et oncles paternels.'
\end{exe}
\end{frame}

  \begin{frame} 
 \frametitle{Equivalences} 
 * = étoile de Kleene
\begin{exe}
\ex 
\glt MB = MBS = MBSS \bleu{[MM*BS*]} \rouge{\ipa{tɤ-rpɯ}}
\glt MZ = MBD = MBSD = MBSSD \bleu{[MM*(BS*D|Z)]} \rouge{\ipa{tɤ-ɬaʁ}}
\ex 
\glt Z(S|D) = FZ(S|D) = FFZ(S|D) \bleu{[F*ZD*(S|D)]} \rouge{\ipa{tɤ-ftsa}}
\end{exe}
\end{frame}

  \begin{frame} 
 \frametitle{Termes de parenté de base} 
\includegraphics[width=\textwidth]{kinchart-c1.pdf}
\end{frame}

  \begin{frame} 
 \frametitle{De l'embarras d'être deux générations au-dessus de son MM*BS* dans un système omaha, Texte 2} 
 \begin{exe}
\ex 
\gll 
\ipa{χpɤltɕin} 	\ipa{tɕe} 	\ipa{iɕqha,} 	\ipa{tshapa} 	\ipa{tɯrme} 	\ipa{nɯra} 	\ipa{ɣɯ} 	\ipa{nɯtɕu,} \ipa{ɯ-wi,} 	\ipa{ɯ-wi} 	\ipa{ɣɯ} 	\ipa{ɯ-wi} 	\ipa{nɯnɯ} 	\ipa{tshapa} 	\ipa{nɯtɕu} 	\ipa{nɯ-kɯ-ɣe} 	\ipa{pjɤ-ŋu} 	\ipa{tɕe}  \\
Dpalcan \textsc{lnk} ci.dessus Tshapa gens  \textsc{dem:pl} \textsc{gen}  \textsc{dem:loc} \textsc{3sg.poss}-grand.mère  \textsc{3sg.poss}-grand.mère  \textsc{gen} gen \textsc{dem} Tshapa \textsc{dem:loc} \textsc{pfv}-\textsc{nmlz}:S/A-venir[II] \textsc{ipfv.ifr}-être \textsc{lnk}  \\
\glt `Dpalcan, la grand-mère de sa grand mère était quelqu'un venue de Tshapa, 
\end{exe}
\end{frame}

  \begin{frame} 
 \frametitle{De l'embarras d'être deux générations au-dessus de son MM*BS* dans un système omaha, Texte 2} 
\begin{exe}
\ex 
\gll 
\ipa{tɕe} 	\ipa{nɯ} 	\ipa{ɣɯ} 	\ipa{ɯ-wi} 	\ipa{ɣɯ} 	\ipa{ɯ-wɤmɯ} 	\ipa{nɯnɯ} 	\ipa{ɣɯ} 	\ipa{ɯ-rɟit} 	\ipa{ɣɯ} 	\ipa{ɯ-rɟit} 	\ipa{ɣɯ} 	\ipa{ɯ-rɟit} 	\ipa{ɣɯ} 	\ipa{ɯ-rɟit} 	\ipa{kɯ-fse} 	\ipa{nɯnɯra} 	\ipa{nɯ-ɕki} 	\ipa{tɕe} 	\ipa{tɕe} 	\ipa{a-rpɯ} \ipa{nɤ} \ipa{a-ɬaʁ} 	\ipa{ntsɯ} 	\ipa{tu-ti} 	\ipa{pjɤ-ra.} \\
\textsc{lnk} \textsc{dem} \textsc{gen} \textsc{3sg.poss}-grand.mère gen \textsc{3sg.poss}-frère \textsc{dem} \textsc{gen} \textsc{3sg.poss}-enfant \textsc{gen} \textsc{3sg.poss}-enfant \textsc{gen} \textsc{3sg.poss}-enfant \textsc{gen} \textsc{3sg.poss}-enfant \textsc{nmlz}:S/A-être.ainsi \textsc{dem:pl} \textsc{3pl-dat} \textsc{lnk}  \textsc{lnk} \textsc{1sg.poss}-MB \textsc{lnk} \textsc{1sg.poss}-MZ toujours \textsc{ipfv}-dire \textsc{ifr.ipfv}-devoir \\
\glt et les enfants des enfants des enfants des enfants du frère de sa grand-mère, il devait les appeller `mon oncle' ou `ma tante'.'
\end{exe}
\end{frame}

  \begin{frame} 
 \frametitle{De l'embarras d'être deux générations au-dessus de son MM*BS* dans un système omaha, Texte 3} 
 \begin{exe}
\ex 
\gll 
\ipa{aʑo} 	\ipa{tɤ-nɯkoŋtso-a} 	\ipa{ɯ-qhu} 	\ipa{tɕe,} 	\ipa{tshapa} 	\ipa{ju-ɕe-a} 	\ipa{tɕe,} 	\ipa{tshapa} 	\ipa{nɯ} 	\ipa{ji-kɯmdza} 	\ipa{ʁɟa} 	\ipa{ɲɯ-ŋu} 	\ipa{tɕe,} 	[...] \ipa{lonba} 	\ipa{iʑo} 	\ipa{ji-kɯmdza} 	\ipa{ɲɯ-ŋu.}  \\
\textsc{1sg} \textsc{pfv}-travailler-\textsc{1sg} \textsc{3sg}-après \textsc{lnk} Tshapa \textsc{ipfv}-aller-\textsc{1sg} \textsc{lnk} Tshapa \textsc{dem} \textsc{1pl.poss}-parents  complètement \textsc{sens}-être \textsc{lnk} { } tout \textsc{1pl} \textsc{1pl.poss}-parents  \textsc{sens}-être \\
\glt `Après que j'ai commencé à travailler, quand j'allais à Tshapa, là-bas c'étaient tous des membres de notre famille,
\end{exe}
\end{frame}


  \begin{frame} 
 \frametitle{De l'embarras d'être deux générations au-dessus de son MM*BS* dans un système omaha, Texte 3} 
 \begin{exe}
\ex 
\gll 
\ipa{tɕe} 	`\ipa{a-rpɯ} 	\ipa{a-ɬaʁ}' 	\ipa{ntsɯ} 	\ipa{tu-kɯ-ti} 	\ipa{ɲɯ-ra} 	\ipa{ma} 	\ipa{nɯ-}beifen 	\ipa{ɲɯ-mbro,} 	\ipa{nɯnɯra} 	\ipa{kɯ,} 	\ipa{nɤkinɯ} 	\ipa{iʑora} 	\ipa{ji-sɤ-ɣi} 	\ipa{nɯtɕu} 	\ipa{pjɤ-ŋu} 	\ipa{tɕe}  \ipa{tɕendɤre} 	\ipa{tɤ-pɤtso} 	\ipa{kɯ-xtɕɯ\tld{}xtɕi} 	\ipa{ra} 	\ipa{kɯ} 	\ipa{nɯ} 	\ipa{a-phe} `\ipa{a-ftsa} 	\ipa{a-ftsa}' 	\ipa{tu-ti-nɯ}\\
\textsc{lnk} \textsc{1sg.poss}-MB  \textsc{1sg.poss}-MZ toujours \textsc{ipfv-genr}-dire \textsc{sens}-devoir car \textsc{3pl.poss}-génération \textsc{sens}-être.haut \textsc{dem:pl} \textsc{erg} ceci \textsc{1pl} \textsc{1pl.poss-nmlz:oblique}-venir \textsc{dem:loc} \textsc{ipfv:ifr}-être \textsc{lnk}  \textsc{lnk} \textsc{indef.poss}-enfant  \textsc{nmlz:S/A-emph}\tld{}être.petit \textsc{pl} \textsc{erg} \textsc{1sg-dat} \textsc{dem} \textsc{1sg.poss}-ZS \textsc{1sg.poss}-ZS \textsc{ipfv}-dire-\textsc{pl} \\ 
%\glt il fallait les appeller `mon oncle, ma tante' car ils étaient (considérés) d'une génération plus haute, le lieu d'où nous venons est là-bas, et eux, (même) de petits enfants m'appellaient  `mon neveu',
\end{exe}
\end{frame}


  \begin{frame} 
 \frametitle{De l'embarras d'être deux générations au-dessus de son MM*BS* dans un système omaha, Texte 3} 
  \begin{exe}
\ex 
\gll 
\ipa{a-rpɯ} 	\ipa{a-ɬaʁ} 	\ipa{kɤ-ti} 	\ipa{ɲɯ-ra,} 	\ipa{tɕendɤre} 	\ipa{tu-ti-a} 	\ipa{kɯ-zgɤt} 	\ipa{ɲɯ-ɕti} 	\ipa{ri} 	\ipa{nɯnɯ} 	\ipa{aʑo} 	\ipa{mɯ́j-nɤxtʂaŋ-a} \\
\textsc{1sg.poss}-MB \textsc{1sg.poss}-MZ \textsc{inf}-dire \textsc{sens}-devoir \textsc{lnk} \textsc{ipfv}-dire-\textsc{1sg} \textsc{inf:stat}-falloir \textsc{sens}-être:\textsc{assert} mais \textsc{dem} \textsc{1sg} \textsc{neg:sens}-trouver.juste-\textsc{1sg} \\
\glt Il fallait les appeler  mon oncle, ma tante'. Bien que je doive le dire, je trouvait cela injuste,
\end{exe}
\end{frame}

  \begin{frame} 
 \frametitle{De l'embarras d'être deux générations au-dessus de son MM*BS* dans un système omaha, Texte 3} 
 
 \begin{exe}
\ex 
\gll 
\ipa{tɕendɤre} 	``\ipa{nɯ} 	\ipa{ɯ-qhu} 	\ipa{aʑo} 	\ipa{nɯtɕu} 	\ipa{tshapa} 	\ipa{mɤ-ɕe-a} 	\ipa{ma} 	\ipa{a-rpɯ} 	\ipa{a-ɬaʁ} 	\ipa{ntsɯ} 	\ipa{kɤ-ti} 	\ipa{ɲɯ-ra"} 	\ipa{tu-ti-a} 	\ipa{pɯ-ŋu,} 	\ipa{ɲɯ́-wɣ-nɤre-a-nɯ} 	\ipa{pɯ-ŋu} \\
\textsc{lnk} \textsc{dem} \textsc{3sg}-après \textsc{1sg} \textsc{dem:loc} Tshapa \textsc{neg}-aller:\textsc{fact-1sg} car \textsc{1sg.poss}-MB \textsc{1sg.poss}-MZ toujours \textsc{inf}-dire \textsc{sens}-devoir \textsc{ipfv}-dire-\textsc{1sg} \textsc{pst.ipfv}-être \textsc{ipfv-inv}-se.mosquer.de-\textsc{1sg-pl} \textsc{pst.ipfv}-être \\
\glt Alors j'ai dit `A l'avenir je n'irai plus à Tshapa, car je dois leur dire `mon oncle, ma tante', et ils se moquaient de moi.'
\end{exe}
\end{frame}

  \begin{frame} 
 \frametitle{Lignée patrilinéaire} 
 \includegraphics[height=3in]{kinchart-rpW.pdf}  
\end{frame}
\end{document}