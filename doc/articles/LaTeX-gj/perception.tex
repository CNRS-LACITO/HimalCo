\documentclass[oldfontcommands,oneside,a4paper,11pt]{article} 
%\usepackage{xunicode}%packages de base pour utiliser xetex
\usepackage{tikz}
\usepackage{fontspec}
\usepackage{natbib}
\usepackage{booktabs}
\usepackage{xltxtra} 
\usepackage{longtable}
\usepackage{polyglossia} 
%\usepackage[table]{xcolor}
\usepackage{gb4e} 
\usepackage{multicol}
\usepackage{graphicx}
\usepackage{float}
\usepackage{textcomp}
\usepackage{hyperref} 
\hypersetup{bookmarks=false,bookmarksnumbered,bookmarksopenlevel=5,bookmarksdepth=5,xetex,colorlinks=true,linkcolor=blue,citecolor=blue}
\usepackage[all]{hypcap}
\usepackage{memhfixc}
\usepackage{lscape}
 
%\bibpunct[: ]{(}{)}{,}{a}{}{,}
%%%%%%%%%quelques options de style%%%%%%%%
%\setsecheadstyle{\SingleSpacing\LARGE\scshape\raggedright\MakeLowercase}
%\setsubsecheadstyle{\SingleSpacing\Large\itshape\raggedright}
%\setsubsubsecheadstyle{\SingleSpacing\itshape\raggedright}
%\chapterstyle{veelo}
%\setsecnumdepth{subsubsection}
%%%%%%%%%%%%%%%%%%%%%%%%%%%%%%%
\setmainfont[Mapping=tex-text,Numbers=OldStyle,Ligatures=Common]{Charis SIL} %ici on définit la police par défaut du texte
\renewcommand \thesection {\arabic{section}.}
\renewcommand \thesubsection {\arabic{section}.\arabic{subsection}.}
\newfontfamily\phon[Mapping=tex-text,Ligatures=Common,Scale=MatchLowercase,FakeSlant=0.3]{Charis SIL} 
\newcommand{\ipa}[1]{{\phon #1}} %API tjs en italique
 
\newcommand{\grise}[1]{\cellcolor{lightgray}\textbf{#1}}
\newfontfamily\cn[Mapping=tex-text,Ligatures=Common,Scale=MatchUppercase]{MingLiU}%pour le chinois
\newcommand{\zh}[1]{{\cn #1}}




\newcommand{\acc}{\textsc{acc}}
 \newcommand{\acaus}{\textsc{acaus}}
 \newcommand{\advers}{\textsc{advers}}
\newcommand{\apass}{\textsc{apass}}
\newcommand{\appl}{\textsc{appl}}
\newcommand{\allat}{\textsc{all}}
\newcommand{\aor}{\textsc{aor}}
\newcommand{\assert}{\textsc{assert}}
\newcommand{\auto}{\textsc{autoben}}
\newcommand{\caus}{\textsc{caus}}
\newcommand{\cl}{\textsc{cl}}
\newcommand{\cisl}{\textsc{cisl}}
\newcommand{\classif}{\textsc{class}}
\newcommand{\concsv}{\textsc{concsv}}
\newcommand{\comit}{\textsc{comit}}
\newcommand{\compl}{\textsc{compl}} %complementizer
\newcommand{\comptv}{\textsc{comptv}} %comparative
\newcommand{\cond}{\textsc{cond}}
\newcommand{\conj}{\textsc{conj}}
\newcommand{\coord}{\textsc{coord}}
\newcommand{\const}{\textsc{const}}
\newcommand{\conv}{\textsc{conv}}
\newcommand{\cop}{\textsc{cop}}
\newcommand{\dat}{\textsc{dat}}
\newcommand{\dem}{\textsc{dem}}
\newcommand{\degr}{\textsc{degr}}
\newcommand{\deexp}{\textsc{deexp}}
\newcommand{\distal}{\textsc{dist}}
\newcommand{\du}{\textsc{du}}
\newcommand{\duposs}{\textsc{du.poss}}
\newcommand{\dur}{\textsc{dur}}
\newcommand{\erg}{\textsc{erg}}
\newcommand{\emphat}{\textsc{emph}}
\newcommand{\evd}{\textsc{evd}}
\newcommand{\fut}{\textsc{fut}}
\newcommand{\gen}{\textsc{gen}}
\newcommand{\genr}{\textsc{genr}}
\newcommand{\hort}{\textsc{hort}}
\newcommand{\hypot}{\textsc{hyp}}
\newcommand{\ideo}{\textsc{ideo}}
\newcommand{\imp}{\textsc{imp}}
\newcommand{\indef}{\textsc{indef}}
\newcommand{\inftv}{\textsc{inf}}
\newcommand{\instr}{\textsc{instr}}
\newcommand{\intens}{\textsc{intens}}
\newcommand{\intrg}{\textsc{intrg}}
\newcommand{\inv}{\textsc{inv}}
\newcommand{\ipf}{\textsc{ipf}}
\newcommand{\irr}{\textsc{irr}}
\newcommand{\loc}{\textsc{loc}}
\newcommand{\med}{\textsc{med}}
\newcommand{\mir}{\textsc{mir}}
\newcommand{\negat}{\textsc{neg}}
\newcommand{\neu}{\textsc{neu}}
\newcommand{\nmlz}{\textsc{nmlz}}
\newcommand{\npst}{\textsc{n.pst}}
\newcommand{\pfv}{\textsc{pfv}}
\newcommand{\pl}{\textsc{pl}}
\newcommand{\plposs}{\textsc{pl.poss}}
\newcommand{\pass}{\textsc{pass}}
\newcommand{\poss}{\textsc{poss}}
\newcommand{\pot}{\textsc{pot}}
\newcommand{\pres}{\textsc{pres}}
\newcommand{\prohib}{\textsc{prohib}}
\newcommand{\prox}{\textsc{prox}}
\newcommand{\pst}{\textsc{pst}}
\newcommand{\qu}{\textsc{qu}}
\newcommand{\recip}{\textsc{recip}}
\newcommand{\redp}{\textsc{redp}}
\newcommand{\refl}{\textsc{refl}}
\newcommand{\sg}{\textsc{sg}}
\newcommand{\sgposs}{\textsc{sg.poss}}
\newcommand{\stat}{\textsc{stat}}
\newcommand{\topic}{\textsc{top}}
\newcommand{\volit}{\textsc{vol}}
\newcommand{\transloc}{\textsc{transl}}
\newcommand{\cisloc}{\textsc{cisl}}
\newcommand{\quind}{\textsc{qu.ind}} %revoir glose
\newcommand{\trop}{\textsc{trop}} 
 \newcommand{\abil}{\textsc{abil}}  
 \newcommand{\facil}{\textsc{facil}}  
  

 

\XeTeXlinebreaklocale "zh" %使用中文换行
\XeTeXlinebreakskip = 0pt plus 1pt %
 %CIRCG
 
 


\usetikzlibrary{arrows,chains,matrix,positioning,scopes}

\tikzset{>=stealth',every on chain/.append style={join},
         every join/.style={->}}
\begin{document} 

\title{Perception verbs and tropative derivation in Japhug Rgyalrong: is there a sensory hierarchy?}
\author{Guillaume Jacques}
\maketitle

\section{Introduction}
For grammar writers and typologists alike, perception verbs constitute an important topic of study for at least two reasons. First, the way the two arguments of perception verb (experiencer and stimulus) are encoded vary considerably across languages, and single languages often show item-specific constructions. Second, perception verbs, as a way of expressing information source, can be studied in parallel to sensory evidentials (see for instance \citealt{aikhenvald13knowledge}), as proposed universal principle applying to the former should also be valid for the latter.

In this paper, I document the system of perception and experiencer verbs in Japhug, and then study this system from a typological perspective. First, I present the use non-derived perception verbs. Second, I describe the tropative derivation, which changes a intransitive verb into a transitive  one by adding an experiencer (encoded as A). Third, I evaluate to what extent the Japhug data confirm or contradict \citet{tsunoda85tr} or \citet{malchukov05competition}'s verb type hierarchies. Fourth, I discuss   whether the grammatical encoding of distinct sensory informations (visual, auditory, olfactive etc) can be described in terms of a hierarchy.


\section{Background information}
\section{Perception verbs} \label{sec:perception}
 In this section, we present the different types of grammatical encoding of experiencer and stimuli in Japhug, as well as the precise meanings of each of these verbs. A typological perspective on these data appears in \ref{sec:encoding}.


Japhug has only two types of non-derived perception verbs: visual vs. non-visual perception. There are three visual perception verb: \ipa{mto}	``to see", \ipa{rtoʁ} ``to look at, to inspect" and \ipa{ru}	 ``to look at, to glance at".

\ipa{mto}	``to see", the non-volitional visual perception verb, is a morphologically transitive verb. Example \ref{ex:mto1} illustrates the double agreement with both arguments, and shows that the experiencer is coded as A and the stimulus as O.

 \begin{exe}
\ex \label{ex:mto1}
\gll \ipa{aʑo} 	\ipa{qhe} 	\ipa{tɕe,} 	\ipa{pjɯ-kɯ-mto-a} 	\ipa{nɯ} 	\ipa{ɕti} 	\ipa{qhe,} 	\ipa{nɯ} 	\ipa{kɯ-fse} 	\ipa{ŋu.} \\
I \textsc{coord} \textsc{coord} \textsc{ipf}-2>1-see-\textsc{1sg} \textsc{dem} be\textsc{:assertive} \textsc{coord} \textsc{dem} \textsc{nmlz:S-}be.like \textsc{npst:}be \\
 \glt  As for me, you see me, it is like that. (Brothers and sisters, 372)
\end{exe} 



% \begin{exe}
%\ex
%\gll \ipa{ŋgoŋpu-ɴɢoɕna} 	\ipa{nɯ} 	\ipa{a-mɤ-pɯ́-wɣ-mto} 	\ipa{ra} 	\ipa{ma} 	\ipa{kɯ-z-rɯŋgoŋpu} \\
%disaster-spider \textsc{dem} \textsc{irr-neg-pfv-genr:A}-see \textsc{npst:}have.to otherwise \textsc{genr:S/O-caus}-damage.things \\
% \glt One should not see a ``disaster-spider", otherwise it would cause one to damage things. (Spider, 131)
%\end{exe} 


 \ipa{mto} ``see" not only allows noun phrases as its O, but also entire finite propositions (cf example \ref{ex:mto3}).

 \begin{exe}
\ex \label{ex:mto3}
\gll [\ipa{nɤmkha} 	\ipa{ɲɯ-nɯqambɯmbjom}] 	\ipa{kɤ-mto} 	\ipa{pɯ-rɲo-t-a} \\
sky \textsc{ipf}-fly  \textsc{inf}-see \textsc{aor}-experience-\textsc{pst-1sg} \\
\glt I have already seen (geese) flying in the sky. (Goose, 5)
\end{exe} 

 There are two volitional  visual perception verbs, \ipa{rtoʁ} and \ipa{ru}.  \ipa{ru}	 ``to look at, to glance at" is morphologically intransitive, as shown by example \ref{ex:ru1}, where the vowel alternation \ipa{u} > \ipa{e} would be expected if the verb were transitive (we should have *\ipa{a-tɤ-tɯ-re}). 
 
 \begin{exe}
\ex \label{ex:ru1}
\gll \ipa{nɤʑo} 	\ipa{a-tɤ-tɯ-ru} 	\ipa{tɕe} 	\ipa{tu} 	\ipa{zdɯm} 	\ipa{lu-ɣi} 	\ipa{ŋu} 	\ipa{tɕe,} \\
you \textsc{irr-pfv:up}-2-look \textsc{coord} up cloud \textsc{ipf:upstream}-come \textsc{npst}:be \textsc{coord}\\
 \glt You will look up and (you will see) a cloud coming up. (Stealing the water1, 34)
\end{exe} 

The S is the experiencer and is indiciated on the verb, and does not receive ergative case.


The stimulus is not and either appears without any formal mark or with the dative \ipa{--ɕki}.

 \begin{exe}
\ex
\gll \ipa{ɯ-stɤt} 	\ipa{lu-kɯ-ru} 	\ipa{nɯnɯ} 	\ipa{pjɯ-kɯ-ɤzgɯr} 	\ipa{nɯ} 	\ipa{sŋɤrɯ} 	\ipa{ɲɯ-rmi.} \\
\textsc{3sg.poss}-upper.part \textsc{ipf:upstream-nmlz:S/A-}look.at \textsc{dem}  \textsc{ipf-nmlz:S/A}-curved \textsc{dem} pummel \textsc{const}-call \\
 \glt The one turned towards the front, the curved one, it is called ``pummel". (The saddle, 38)
\end{exe} 

 


\ipa{rtoʁ} ``to look at, to inspect", the other volitional visual perception verb, is transitive unlike   \ipa{ru}. It can be used to indicate a detailed observation, but can also be used for more casual perception as in example \ref{ex:rtoR}.

 \begin{exe}
\ex \label{ex:rtoR}
\gll \ipa{tɕe} 	\ipa{kɯrtsɤɣ} 	\ipa{nɯnɯ} 	\ipa{ɯ-ku} 	\ipa{nɯnɯ} 	\ipa{kú-wɣ-rtoʁ} 	\ipa{tɕe,} 	\ipa{lɯlu} 	\ipa{tsa} 	\ipa{ɯ-tshɯɣa} 	\ipa{fse,}  \\
\textsc{coord} panther \textsc{dem} \textsc{3sg.poss}-head  \textsc{dem}  \textsc{ipf-gen:A}-look \textsc{coord} cat a.little \textsc{3sg.poss}-appearance \textsc{npst}:be.like \\
 \glt When one looks at a panther's head, it resembles that of a cat. (Panther, 132)
\end{exe} 

There are two auditory perception verbs:  the non-volitional \ipa{mtshɤm}	``to hear" and the volitional \ipa{sɤŋo} ``to listen".

Like \ipa{mto} ``to see", \ipa{mtshɤm}	``to hear" is a transitive verb, whose experiencer is coded as the A. In example \ref{ex:mtsham1}, the generic agent is indicated by the inverse marker \ipa{--wɣ--} (on the use of the inverse as a generic A marker in Japhug, see \citealt{jacques12demotion}). As \ipa{mto}, it accepts finite complement clauses (exemple \ref{ex:mtsham2}).

 
 \begin{exe}
\ex \label{ex:mtsham1}
\gll \ipa{kɯmu} 	\ipa{ɯ-skɤt} 	\ipa{nɯ} 	\ipa{a-pɯ́-wɣ-mtshɤm} 	\ipa{tɕe} 	\ipa{phɤn} 	\ipa{tu-ti-nɯ} 	\ipa{ɲɯ-ŋu}  \\
 snowcock \textsc{3sg.poss}-voice \textsc{dem} \textsc{irr-pfv-genr:A}-hear \textsc{coord} \textsc{npst:}effective \textsc{ipf}-say-\textsc{pl} \textsc{ipf}-be-\textsc{pl}\\
 \glt It one hears the Tibetan snowcock's voice, it is effective (to cure rabbies), they say. (The Tibetan snowcock, 29)
\end{exe} 

		
		 \begin{exe}
\ex \label{ex:mtsham2}
 \gll   	\ipa{tu-ɣɤwu}    	\ipa{ kɤ-mtshɤm}   	\ipa{pɯ-rɲo-t-a}     	 \\
 \textsc{ipf}-cry   \textsc{inf}-hear \textsc{aor}-experience-\textsc{pst}-\textsc{1sg} \\
 \glt I have already heard a wolf howling. (Wolf, 9)
\end{exe} 
 		

%		ɯ-di pjɯ-mtshɤm qhe, ju-ɣi qhe, ɯ-taʁ ri ɯ-qe ku-lɤt qhe,
%qhe ɯ-qe ku-lɤt ɲɯ-ŋu ri, nɯnɯ ɯ-pɯ ɲɯ-βze ɲɯ-ɕti.
%βɣaza

\ipa{mtshɤm} is however not restricted to auditory perception, but can also be used for all non-visual senses, including smell (example \ref{ex:mtsham3}), pain or vibration of an earthquake.

 \begin{exe}
\ex \label{ex:mtsham3}
\gll \ipa{tɤ-di}   	\ipa{ci}   	\ipa{kɯ-mɯ\tilderedp{}mɯm}   	\ipa{ʑo}   	\ipa{pɯ-mtsham-a}   	 \\
\textsc{indef.poss}-smell \textsc{indef} \textsc{nmlz:S}-hear-\textsc{1sg} \\
 \glt I smelled a nice smell. (The lotus, 2)
\end{exe} 

 The corresponding volitional verb \ipa{sɤŋo}	  is labile.\footnote{As all labile verbs in Japhug, it presents agent-preserving lability (S=A), as shown in \citet{jacques12demotion}. } Conjugated intransitively like \ipa{ru} ``to look at", it means ``to listen" (example \ref{ex:sANo1}; the morphological intransivity of this verb is visible by the absence of stem alternation, otherwise *\ipa{ɕ-pɯ-sɤŋɤm} would be expected), but conjugated transitively it means ``to obey" (cf example \ref{ex:sANo2}).
 
   \begin{exe}
\ex \label{ex:sANo1}
\gll  \ipa{aki}	\ipa{ɕ-pɯ-sɤŋo}  \\ 
down \textsc{transloc-imp:down}-listen \\
 \glt  Go to listen down there! (Smanmi1, 72)
\end{exe} 
% 
%  \begin{exe}
%\ex 
%\gll \ipa{pɣɤɲaʁ}  	\ipa{nɯnɯ}  	\ipa{ɯ-skɤt}  	\ipa{ɲɯ-kɯ-sɤŋo}  	\ipa{tɕe}  	\ipa{saχsɤl}  	\ipa{ma}  	\ipa{kɤ-mto}  	\ipa{rkɯn}   \\ 
%pheasant \textsc{dem} \textsc{3sg.poss}-voice \textsc{ipf-genr:S}-listen \textsc{coord} \textsc{npst}:be.clear a.part.from \textsc{nmlz:O-}see \textsc{npst}:be.rare\\
% \glt )The pheasant, when one listens to its voice  (The pheasant, 34)
%\end{exe} 
% 
		 \begin{exe}
\ex \label{ex:sANo2}
\gll  \ipa{tɕe}   	\ipa{a-mu}   	\ipa{a-wa}   	\ipa{ra}   	\ipa{ka-sɤŋo}   	\ipa{tɕe}   	\ipa{lɤ-ari}   	\ipa{tɕe,}   	 \\
\textsc{coord} \textsc{1sg.poss}-mother  \textsc{1sg.poss}-father \textsc{pl} \textsc{aor.3>3}-obey \textsc{coord}  \textsc{aor:upstream}-go[2] \textsc{coord}\\
 \glt He obeyed my parents and went there. (Brothers and sister, 216)
\end{exe} 

Like \ipa{mtshɤm} ``to hear" however, it is not strictly an auditory perception verb. It cannot be used for smell (the volitional perception verb for smell, \ipa{nɤmnɤm}, is studied below), but can be applied to all feelings which are expressed by combination of an ideophone with the verb \ipa{ti} ``to say" as a light verb, as in example \ref{ex:sANo3}.

		 \begin{exe}
\ex \label{ex:sANo3}
\gll \ipa{ɲɯ-kɯ-sɤŋo}   	\ipa{tɕe}   	\ipa{zɯrzɯrzɯr}   	\ipa{tu-ti}   	\ipa{qhe}   	\ipa{tɕendɤre}   	\ipa{tɤ-ndɤr}   	\ipa{ɲɯ-ɬoʁ}   	\ipa{ɕti.}\\
\textsc{ipf-genr:S/O}-listen \textsc{coord}  fuzzy.feeling \textsc{ipf}-say \textsc{coord} \textsc{coord} \textsc{indef.poss}-pimple \textsc{ipf}-appear be:\textsc{assert} \\
 \glt  One has a fuzzy feeling and a pimple  appears. (Pimples, 3)
\end{exe} 
  	
While there is in effect no specific perception verb of audition in Japhug, there is a volitional transitive verb  	\ipa{nɤmnɤm} ``to smell" (example \ref{ex:namnam}), which cannot be applied to any other sense, and whose morphological structure will be studied in more detail in section \ref{sec:tropative}.
 
 \begin{exe}
\ex \label{ex:namnam}
\gll \ipa{tɯɣ}  	\ipa{kɯ-fse}  	\ipa{kɯ-tu}  	\ipa{nɯra}  	\ipa{tu-nɤmnɤm}  	\ipa{tɕe}  	\ipa{ɯ-kɯ-sɯχsɤl,}  	\ipa{nɯnɯ}  	\ipa{ɯ-kɯ-sɯχpjɤt}  	\ipa{ɲɯ-ŋu}   \\
poison  \textsc{nmlz:stat}-be.like  \textsc{nmlz:stat}-exist \textsc{dem:pl} \textsc{ipf}-smell \textsc{coord} 3sg-\textsc{nmlz:A/S}-recognize \textsc{dem} \textsc{nmlz:A/S}-perceive \textsc{const}-be \\
\glt The poisonous things, it is able to recognize them when it smells them. (the buzzard, 34)
\end{exe}



For other senses, there is no specific perception verb, and \ipa{mtshɤm} ``to hear" is generally used for all non-volitional perceptions. For touch, one can use the verbs \ipa{nɤmɤle}	and \ipa{nɤmɯma} which both mean ``to touch, to stroke" but can be used in the sense ``to feel by touch" as in \ref{ex:namale} and \ref{ex:namwma}.

 \begin{exe}
\ex \label{ex:namale}
\gll \ipa{ɲɯ́-wɣ-nɤmɤle}   	\ipa{qhe}   	\ipa{ɲɯ-rʁom}   	\ipa{kɯ-fse}   	\\
\textsc{ipf-genr:A}-touch \textsc{coord}  \textsc{const}-coarse  \textsc{nmlz:stat}-be.like\\
 \glt  It is coarse to the touch. (Pimples, 14)
\end{exe} 
 

 \begin{exe}
\ex \label{ex:namwma}
\gll \ipa{ɯ-ɕna}  	\ipa{nɯnɯ}  	\ipa{paʁ}  	\ipa{ɯ-ɕna}  	\ipa{tsa}  	\ipa{fse}  ...	\ipa{ɲɯ́-wɣ-nɤmɯma}  	\ipa{tɕe}  	\ipa{rko}  	\ipa{ʑo}   	\\
\textsc{3sg.poss}-nose \textsc{dem} pig \textsc{3sg.poss}-nose  a.bit \textsc{npst}:be.like ... \textsc{ipf-genr:A}-touch \textsc{coord} \textsc{npst}:be.hard \textsc{intens} \\
 \glt  Its nose looks like that of a pig, but is hard to the touch. (Mole, 151)
\end{exe} 



% ɣɤjmŋo nɯjmŋo



\section{Tropative} \label{sec:tropative}
The transitive verb \ipa{nɤmnɤm} ``to smell" mentioned in the previous section differs from all the other perception verbs in that it is not a primary verb, but is derived by means of a valency-increasing prefix \ipa{nɤ-} , the tropative. The base verb \ipa{mnɤm} ``to have a smell" is intransitive and normally has the inalienably possessed noun \ipa{--di} as its S (example \ref{ex:mnam}).
 \begin{exe}
\ex \label{ex:mnam}
\gll \ipa{ɯ-di}   	\ipa{tɯ-mnɤm}   	\ipa{ta-ʑa}   	\ipa{tɕe}   	\ipa{cha}   	\ipa{to-rɤru}   	\ipa{tu-ti-nɯ}   	\ipa{ŋu}   \\
\textsc{3sg.poss}-smell \textsc{nmlz:action}-smell \textsc{pfv:3>3}-begin \textsc{conj} alcohol \textsc{evd}-get.up \textsc{ipfv}-say-\textsc{pl} \textsc{npst}:be \\
\glt When it starts to smell, people say ``the alcohol has fermented" (Alcohol, 77)
\end{exe} 

\ipa{nɤmnɤm} ``to smell" is not the only perception verb derivable with \ipa{nɤ}--. Another example of such a verb (though pain is not traditionally considered to be a distinct sense) is \ipa{nɤmŋɤm} ``to feel pain", derived from \ipa{mŋɤm} ``to ache (it)".

 \begin{exe}
\ex \label{ex:namNam2}
\gll  \ipa{ɯ-xtu} \ipa{ɲɯ-nɤ-mŋɤm}  \\
\textsc{3.sg.poss}-belly \textsc{const}-\textsc{tropative}-ache \\
\glt  He feels pain in his belly (elicited, Chenzhen)
\end{exe}

Thus, no account of perception verbs in Japhug would be complete without a description of the tropative derivation.
\subsection{Morphosyntactic properties}
The tropative \ipa{nɤ-}  is a very productive derivation that can be applied to most stative verbs, having the meaning ``to consider to be X''. \ipa{nɤmnɤm} ``to smell" is the only specifically volitional tropative verb; all other verbs derived by \ipa{nɤ}-- can be used in non-volitional contexts.

As in a causative derivation (\citealt[45]{dixon00causative}), the S of the original verb becomes the O of the derived transitive verb, while the added argument (the experiencer) becomes the A of the derived verb. For instance, the stative verb  \textit{mpɕɤr} ``be beautiful'' have the derived transitive verb \textit{nɤ-mpɕɤr} ``consider to be beautiful'':


 \begin{exe}
\ex
\gll  \ipa{ɯ-mdoʁ} 	\ipa{maka} 	\ipa{mɯ́j-nɤsci} 	\ipa{tɕe,} 	\ipa{nɯ} 	\ipa{ni} 	\ipa{stu} 	\ipa{nɯ-kɤ-nɤ-mpɕɤr} 	\ipa{ɲɯ-ŋu} 
\\
3\sg{}.\poss{}-colour at.all \negat{}:\const{}-change \coord{} \distal{}.\dem{} \du{} most 3\pl{}-\nmlz{}:O-\textsc{tropative}-beautiful \ipf{}-be \\
 \glt  Its colour does not change, and these two are the ones that they consider the most beautiful. (Coloured belts, 85)
\end{exe} 

As such, the tropative \ipa{nɤ}-- is clearly  distinct from the applicative \ipa{nɯ}-- / \ipa{nɯɣ}--, which presents a different redistribution of syntactic roles: the S of the base verb becomes the A, and an adjunct is promoted to become O, as in \ipa{bɯɣ} ``to miss home (it)" vs. \ipa{nɯɣ-bɯɣ} ``to miss someone (vt)".\footnote{The two derivations, however, are potentially historically related.}



Table \ref{tab:tropative}  shows additional examples of tropative verbs in Japhug.
\begin{table}[H]
\caption{Examples of the \ipa{nɤ}- tropative prefix in Japhug}\label{tab:tropative}
\begin{tabular}{lllllllll} \toprule
basic verb  & &derived  verb &\\
\midrule
 \ipa{wxti} & be big & \ipa{nɤ-wxti} & consider to be too big \\
 \ipa{zri} & be long & \ipa{nɤ-zri} & consider to be too long \\
       \midrule
  \ipa{chi} &be sweet & \ipa{nɤx-chi}  &consider to be too sweet \\
    %\ipa{mnɤm} & have an odour & \ipa{nɤ-mnɤm} & smell (tr.) \\
  \ipa{maʁ} & not be & \ipa{nɤɣ-maʁ} & consider to not to be right \\
  \ipa{mbat} & be easy & \ipa{nɤɣ-mbat} & finish easily \\
\bottomrule
\end{tabular}
\end{table}
The examples show that the morphophonology and semantic derivation of this prefix is not entirely straightforward.

Aside from the regular \ipa{nɤ-} allomorph, one also find a \ipa{nɤɣ-} / \ipa{nɤx-} allomorph on a few verbs. A similar allomorphy is found with the causative prefix (which has \ipa{sɯɣ-} and \ipa{ɕɯɣ-} allomorphs alongside \textit{sɯ-}) and the applicative \ipa{nɯ-} (which has the allomorph \ipa{nɯɣ-} in a few examples). The  \ipa{sɯ-} vs. \ipa{sɯɣ-} allomorphy is still productive: the latter allomorph is found when the original verb was intransitive, without an initial consonant cluster and without initial velar or uvular. It is possible that a similar distribution existed at a former stage for the \ipa{nɤ-} / \ipa{nɤɣ-} allomorphs, but the data at hand do not permit a firm conclusion.

Moreover, the semantics of the derived verb is not always simply ``to consider to be X''. In the case of stative verbs whose meaning is neutral (not explicitly positive like ``beautiful''), the tropative generally has the meaning ``to consider to be too X''. Second, in the case of \ipa{nɤɣ-maʁ} ``consider to not to be right'', it seems that of of the original meanings of   \ipa{maʁ}  was ``not to be right'', as the nominalized form \textit{kɯ-maʁ} can mean ``something which is not right''. Here the tropative preserved the original meaning of the verb, and the base verb underwent an independent semantic change.

\subsection{The tropative and other derivations}

The tropative is compatible with other derivation prefixes, in particular the deexperiencer \ipa{sɤ-} (see \citealt{jacques12demotion}). For instance, we observe the derivation chain:

 \begin{exe}
\ex
 \glt  \ipa{scit} ``happy'' (of a person)
 \glt Deexperiencer: \ipa{sɤ-scit} ``funny, nice''. The literal meaning of this stative verb is in fact ``to be such that people are happy''. 
 \glt Tropative: \ipa{nɤ-sɤ-scit} ``to consider to be nice''
\end{exe} 
The doubly derived verb can be illustrated by the following example:
 \begin{exe}
\ex
\gll   \ipa{nɯ-sɤ-rma} 	\ipa{pjɯ-ɣɤrcoʁ} 	\ipa{pjɤ-ra} 	\ipa{ma} 	\ipa{kɯ-zbaʁ} 	\ipa{nɯ} 	\ipa{mɯ́j-nɤ-sɤ-scit-nɯ} 	\ipa{ɲɯ-ŋgrɤl.} \\
 3\textsc{pl.poss}-\textsc{nmlz:oblique}-live \textsc{ipf}-be.muddy \textsc{evd.ipf}-have.to because \textsc{nmlz:stat}-dry \textsc{top} \textsc{neg:const}-\textsc{tropative}-\textsc{deexperiencer}-happy-\textsc{pl} \textsc{ipf}-be.usually.the.case \\
 \glt   (The frogs)  had to live in a muddy place, because they did not like dry (places).  (Aesop adaptation, the frogs)
\end{exe} 
  
 It is however impossible to combine the tropative with the reflexive \ipa{ʑɣɤ-}. To express the meaning ``to consider oneself X'', one uses a different construction, with the complex prefix \ipa{znɤ-} and reduplication of the verb stem:
 
 
   \begin{exe}
\ex
 \glt   \ipa{ɕqraʁ} ``be intelligent'' > \ipa{znɤ-ɕqraʁ-ɕqraʁ} ``consider onself intelligent''.
\end{exe} 
 
   \begin{exe}
\ex
\gll \ipa{tɯ-znɤɕqraʁɕqraʁ} 	\ipa{nɯɣe,} 	\ipa{kɯmtɕhɯ} 	\ipa{ra} 	\ipa{mɯ́j-ra} 	\ipa{ɣe}  \\
 2-\npst{}:\trop{}:\refl{}:intelligent isn't.it toy \pl{} \negat{}:\const{}-have.to isn't.it \\
  \glt   You think you are so smart, you don't need toys, don't you? (Conversation 2003, 68)
   \end{exe}
 
  
 
We also find several examples of causative prefix \ipa{sɯ-} / \ipa{sɯɣ-} / \ipa{z-} whose semantics remind of the tropative: \ipa{znɤja} ``consider to be a shame'', \ipa{sɯpa} ``regard as'' and \jg{znɤkɤro} ``consider to be acceptable''.

The intransitive verb \ipa{nɤja} means ``to be a shame, to be a pity''.
  \begin{exe}
\ex
\gll \ipa{iɕqha} 	\ipa{laχtɕha} 	\ipa{pjɤ-ɴɢrɯ,} 	\ipa{pɯ-nɤja} \\
the.aforementioned thing \evd{}-\acaus{}:break \aor{}-be.a.shame \\
  \glt  That thing broke, what a shame!
   \end{exe}

   The transitive \ipa{z-nɤja}, rather than meaning ``to to be a shame'' as expected regularly, rather means ``to regret, be reluctant'', in other words ``to consider something to be a pity'':
   
     \begin{exe}
\ex 
\gll \ipa{wuma} 	\ipa{ʑo} 	\ipa{pɯ-znɤja-t-a}  \\
very \emphat{} \aor{}-regret-\pst{}-1\sg{} \\
  \glt  I regretted it very much (a lost cellphone cover, Dpalcan, conversation, 2010)
   \end{exe}
Another verb having unpredictable semantics with the prefix \ipa{sɯ-} is the transitive verb \ipa{sɯ-pa} ``to consider, to regard as''. The original verb here is \ipa{pa} ``to close'', etymologically ``to do'':

     \begin{exe}
\ex 
\gll \ipa{tɤkhe-pɣɤtɕɯ} 	\ipa{nɯ} 	\ipa{ɯʑo} 	\ipa{pɣɤtɕɯ} 	\ipa{nɯ} 	\ipa{kɯ-khe} 	\ipa{tu-sɯpa-nɯ} \\
stupid-bird \topic{} he bird \topic{} \nmlz{}:\stat{}-stupid \ipf{}-consider-\pl{} \\
 \glt The \ipa{tɤkhe-pɣɤtɕɯ} is considered to be a stupid bird (the buzzard, 13)
   \end{exe}
 
 This construction with  \ipa{sɯ-pa} allows to express the same meaning as the tropative derivation but with a complement clause, in particular in the case of verbs for which this derivation is not appropriate.
 
 
 \subsection{The tropative as a cross-linguistic category}
   The Japhug tropative is not unique among the world's language. In Classical Arabic, the form X of the verb can in some cases have a tropative meaning, in examples such as \textit{ħasuna} ``be good'' > \textit{istaħsana} ``deem to be good'' (see for instance \citet{larcher96}). We actually borrow the term ``tropative'' from the Arabic linguistics tradition, though ``estimative'' is also found.
   
   Besides Arabic, one finds in Turkish the suffix --\textit{(I)msA} \citet[56]{goksel05grammar} or \textit{--sA}, which can regularly derive verbs from adjective with the same meaning as in Japhug:
   
        \begin{exe}
\ex 
 \glt   \textit{büyük} ``big'' > \textit{büyük-se--} ``overestimate''
 \glt \textit{kötü } ``bad'' > \textit{ kötü-mse--} ``think ill of''
   \end{exe}
   However it differs from Japhug in that it can also be applied to a pronoun in the example \textit{ben} ``I'' > \textit{ben-imse--} ``adopt, embrace (= consider to be one's own)''.
   
   Crosslinguistically, few languages however have a special derivation specialized restricted to the tropative. Common ways of expressing the same meaning include a construction with verbs such as ``think" or ``consider" and a complement clause as the \ipa{sɯpa} construction above in Japhug, or use of causatives derivation with a tropative meaning.
    

\subsection{The grammatical encoding of experiencers and stimuli in Japhug } \label{sec:encoding}
In an oft-cited article, \citet{tsunoda85tr}  proposed a hierarchy which classifies predicates into seven classes based upon the affectedness of the participants (Table \ref{tab:tsunoda}). Predicates that are higher on this hierarchy tend to be coded as prototypically transitive verbs, while those on the lower end tend to be expressed as intransitive predicates whereby one of the participants receives oblique case.


\begin{table}[H]
\caption{\citet{tsunoda85tr}'s verb type hierarchy}\label{tab:tsunoda}
\resizebox{\columnwidth}{!}{
\begin{tabular}{lllllllll}
 \toprule
\multicolumn{2}{l}{1}  & 	\multicolumn{2}{l}{2} & 	3 & 4 & 5 & 6 & 7 \\ \midrule
\multicolumn{2}{l}{effective action} & \multicolumn{2}{l}{perception} & pursuit & knowledge & 	feeling & 	relation & ability \\ 
resultative&non-resultative&non-volitional&volitional\\
\midrule
\textit{kill}, \textit{break} &\textit{hit}, \textit{eat}& \textit{see, hear}&\textit{listen, look} & \textit{search, look for} & \textit{know, understand} & \textit{love, like} & \textit{have}, \textit{lack} & \textit{be able to, capable} \\
%\textit{ shoot} & \textit{find, look} & \textit{wait} & \textit{remember} & \textit{want, fear} & \textit{resemble} & \textit{proficient} \\
%\textit{eat} & \textit{listen, smell} & \textit{} & \textit{forget} & \textit{need} & \textit{correspond} & \textit{good} \\
\bottomrule
\end{tabular}}                        
\end{table}

\citet{malchukov05competition} proposed to refine Tsunoda's hierarchy to explain numerous counterexamples by splitting it into two sub-hierarchies (Figure \ref{fig:malchukov}). This forked hierarchy reflects the fact that the encoding of experiencers in not necessarily correlated with that of motion-like predicates crosslinguistically.
 
 \begin{figure}[H]
\caption{Malchukov's hierarchy} \label{fig:malchukov}
 \begin{tikzpicture}
  \matrix (m) [matrix of math nodes, row sep=3em, column sep=3em]
  {
 & contact &pursuit &motion\\
action&\\
 & perception &emotion &sensation\\ };
  { [start chain] \chainin (m-2-1);
    \chainin (m-1-2);}
    { [start chain] \chainin (m-1-2);
    \chainin (m-1-3); }
    { [start chain] \chainin (m-1-3);
    \chainin (m-1-4);}
  { [start chain] \chainin (m-2-1);
    \chainin (m-3-2);}
    { [start chain] \chainin (m-3-2);
    \chainin (m-3-3); }
    { [start chain] \chainin (m-3-3);
    \chainin (m-3-4);}
\end{tikzpicture}

\end{figure}
 These hierarchies do reflect some observable tendencies, some of which are applicable to the case of Japhug. In the data studied in section \ref{sec:perception} for instance, the fact that non-volitional perception verbs are encoded like effective action verbs, while volitional perception verbs are morphologically intransitive, correlates with the relative place of these predicates in Tsunoda's hierarchy.
 
However,  the tropative derivation in Japhug obviously generates transitive verbs of perception, emotion, feeling and  ability conjugated like fully transitive verbs of effective action, with the experiencer encoded as the A. The existence of such a derivation makes both hierarchies inapplicable to this language, unless one decides that only primary verbs (those consisting of a single root without valency-changing morphology) should be taken into account.

 \section{Is there a sensory hierarchy?} \label{sec:sensory}
Several types of evidence converge on indicating  that vision tends to dominate audition and all other senses. In particular,   ventriloquism  and  the effect of vision on speech perception (\citealt{mcgurk76}) show that visual information can influence auditory perception; reverse influence from audition on vision is possible, but only when visual localization is poor (\citealt{alais04ventriloq}). There is also clear evidence for dominance of vision over touch (\citealt{tastevin37aristote}) and over smell and taste (\citealt{morrot01odors}). Sensory dominance relationship between non-visual senses  have also been investigated; there are reports of dominance of audition over touch (\citealt{jousmaki98parchment}), though this dominance is less clear than that of vision over other senses, as tactile stimulation has also been shown to influence the perceived location of sounds (\citealt{caclin02tactile}). There is also evidence that touch dominates taste (\citealt{todrank91taste}).


In view of these data, it could be tempting to propose the following hierarchy (\ref{ex:hierarchy}) as a heuristic for the study of the grammatical expression of sensory perception cross-linguistically.


\begin{exe}
\ex \label{ex:hierarchy}
\glt \textsc{vision} > \textsc{audition} > \textsc{touch} > \textsc{smell}, \textsc{taste}
\end{exe}

This hierarchy, if valid, makes the following predictions:

\begin{itemize}
\item If a language has a perception verb specialized in a particular sense, it should also have specialized perception verbs for all senses higher on the hierarchy.
\item If a language has grammaticalized a particular sense-specific evidential, if should also have sense-specific evidentials for all senses higher on the hierarchy.
\end{itemize}
In this paper, we only focus on the first prediction, and defer the evaluation of the second one to another research.

Hierarchy \ref{ex:hierarchy} implies that if a language only has one sense-specific perception verb, it must be visual (``see", ``look"). 


 \section{Conclusion}
 
\bibliographystyle{linquiry2}
\bibliography{bibliogj}
\end{document}
