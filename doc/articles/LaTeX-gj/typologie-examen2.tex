\documentclass[oldfontcommands,twoside,a4paper,12pt]{article} 
\usepackage{fontspec}
\usepackage{natbib}
\usepackage{booktabs}
\usepackage{xltxtra} 
\usepackage{polyglossia} 
\usepackage[table]{xcolor}
\usepackage{color}
\usepackage{multirow}
\usepackage{gb4e} 
\usepackage{multicol}
\usepackage{graphicx}
\usepackage{float}
\usepackage{hyperref} 
\hypersetup{bookmarks=false,bookmarksnumbered,bookmarksopenlevel=5,bookmarksdepth=5,xetex,colorlinks=true,linkcolor=blue,citecolor=blue}
\usepackage{memhfixc}
\usepackage{lscape}
\usepackage[footnotesize,bf]{caption}
\usepackage{multicol}
 
 
\setmainfont[Mapping=tex-text,Numbers=OldStyle,Ligatures=Common]{Charis SIL} 
\setsansfont[Mapping=tex-text,Ligatures=Common,Mapping=tex-text,Ligatures=Common,Scale=MatchLowercase]{Lucida Sans Unicode} 
 


\newfontfamily\phon[Mapping=tex-text,Ligatures=Common,Scale=MatchLowercase,FakeSlant=0.3]{Charis SIL} 
\newcommand{\ipa}[1]{{\phon\textbf{#1}}} 

\begin{document}
\title{Typologie : examen du 22 janvier 2015}
\date{}
\maketitle
 

\section*{Exercice 1 (Arapaho, données de Cowell and Moss 2006}
\begin{enumerate}
\item Placez les formes verbales des exemples ci-dessous dans le tableau du paradigme transitif.
\item Décrivez l'indexation et le marquage dans cette langue.
\item Traduisez la phrase `Tu m'as tué' en arapaho.
\item Glosez et analysez la phrase 4.
\end{enumerate}

\begin{multicols}{2}
\begin{exe}
\ex 
\glt beniinin
\glt You_s gave it to me
\end{exe} 

\begin{exe}
\ex 
\glt beniineθen
\glt I gave it to you _s
\end{exe} 

 \begin{exe}
\ex 
\glt ísei  nihnoohowoot inenin
\glt The woman saw the man.
\end{exe} 

 \begin{exe}
\ex 
\glt ísei  nihnoohoweit inenin
\glt The man saw the woman.
\end{exe} 

 \begin{exe}
\ex 
\glt neh’einoo inen
\glt The man killed me.
\end{exe} 

 \begin{exe}
\ex 
\glt neh’o'
\glt I killed him.
\end{exe} 

 \begin{exe}
\ex  \label{pass:arapaho}
\glt neh’eet
\glt He was killed.
\end{exe} 

\end{multicols}

\section*{Exercice 2 (Khroskyabs, données de Lai Yunfan 2013,14}
\begin{enumerate}
\item Placez les formes verbales des exemples ci-dessous dans le tableau du paradigme transitif.
\item Décrivez l'indexation et le marquage dans cette langue.
\item Traduire en khroskyabs   `il t'a vu' et `tu es monté'.
\end{enumerate}

\begin{multicols}{2}
\begin{exe}
\ex 
\glt \ipa{ŋô} \ipa{ævʊ̂ŋ} 
\glt Je suis monté.
\end{exe} 
 
 \begin{exe}
\ex 
\glt \ipa{ævə̂} 
\glt Il est monté.
\end{exe} 
 
 \begin{exe}
\ex  
\glt \ipa{ŋô} (\ipa{ɣə})  \ipa{ætə̂} \ipa{nævdɑ́ŋ.} 
\glt Je l'ai vu.
\end{exe} 

 \begin{exe}
\ex  
\glt  \ipa{ætə̂}  \ipa{ɣə} 	 \ipa{nuvdé.} 
\glt  Il l'a vu.
\end{exe} 

 \begin{exe}
\ex  
\glt \ipa{nû} (\ipa{ɣə}) 	 \ipa{nævdén.} 
\glt  Tu l'as vu.
\end{exe} 

 \begin{exe}
\ex  \label{nuvdaN}
\glt \ipa{ætə̂} 	\ipa{ɣə} 	\ipa{ŋô} 	\ipa{nuvdɑ́ŋ} 	
\glt Il m'a vu.
\end{exe} 

 \begin{exe}
\ex   
\glt \ipa{nû} (\ipa{ɣə}) 	\ipa{ŋô} 	\ipa{nuvdɑ́ŋ} 	
\glt Tu m'as vu.
\end{exe} 

 \begin{exe}
\ex   
\glt \ipa{nû} 	(\ipa{ɣə}) 	\ipa{ŋɑ̂kʰe} 	\ipa{kɑpə̂} 	\ipa{rɑ̂ɣ} 	\ipa{nusŋɑ̂ŋ} 
\glt Tu m'as prêté un livre.
 \end{exe} 
  \begin{exe}
\ex   
\glt  \ipa{nû} 	(\ipa{ɣə})  	\ipa{ŋɑ̂kʰe} 	\ipa{kɑpə̂} 	\ipa{rɑɣ} 	\ipa{nəkʰɑ́n} 
 \glt Tu m'as donné un livre.
  \end{exe} 
  
 
 
 \begin{exe}
\ex   \label{pass:khroskyabs}
\glt \ipa{ŋô} 	\ipa{næʁvdɑ́ŋ.} 
\glt J'ai été vu.
\end{exe} 


\end{multicols}

\section*{Question de cours}
Commentez les exemples \ref{pass:arapaho} et \ref{pass:khroskyabs}. Quel procédé morphosyntaxique étudié en cours illustrent-ils?

\end{document}

