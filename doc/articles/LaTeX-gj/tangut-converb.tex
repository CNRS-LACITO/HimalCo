\documentclass[oldfontcommands,twoside,a4paper,12pt]{article} 
\usepackage{fontspec}
\usepackage{natbib}
\usepackage{booktabs}
\usepackage{xltxtra} 
\usepackage{longtable}
\usepackage{tangutex2} 
%\usepackage{tangutex4} 
\usepackage{polyglossia} 
\usepackage[table]{xcolor}
\usepackage{color}
\usepackage{multirow}
\usepackage{gb4e} 
\usepackage{multicol}
\usepackage{graphicx}
\usepackage{float}
\usepackage{hyperref} 
\hypersetup{bookmarks=false,bookmarksnumbered,bookmarksopenlevel=5,bookmarksdepth=5,xetex,colorlinks=true,linkcolor=blue,citecolor=blue}
\usepackage{memhfixc}
\usepackage{lscape}
\usepackage[footnotesize,bf]{caption}


%%%%%%%%%%%%%%%%%%%%%%%%%%%%%%%
\setmainfont[Mapping=tex-text,Numbers=OldStyle,Ligatures=Common]{Charis SIL} 
\setsansfont[Mapping=tex-text,Ligatures=Common,Mapping=tex-text,Ligatures=Common,Scale=MatchLowercase]{Lucida Sans Unicode} 
 


\newfontfamily\phon[Mapping=tex-text,Ligatures=Common,Scale=MatchLowercase,FakeSlant=0.3]{Charis SIL} 
\newfontfamily\phondroit[Mapping=tex-text,Ligatures=Common,Scale=MatchLowercase]{Doulos SIL} 
\newcommand{\ipa}[1]{{\phon\textbf{#1}}} 
\newcommand{\ipab}[1]{{\phon #1}}
\newcommand{\ipapl}[1]{{\phondroit #1}} 
\newcommand{\captionft}[1]{{\captionfont #1}} 
\newfontfamily\cn[Mapping=tex-text,Ligatures=Common,Scale=MatchUppercase]{MingLiU}%pour le chinois
\newcommand{\zh}[1]{{\cn #1}}
\newcommand{\tgf}[1]{{\large\mo{#1}}}

\newcommand{\acc}{\textsc{acc}}
\newcommand{\antierg}{\textsc{antierg}}
\newcommand{\allat}{\textsc{all}}
\newcommand{\aor}{\textsc{aor}}
\newcommand{\assert}{\textsc{assert}}
\newcommand{\auto}{\textsc{auto}}
\newcommand{\caus}{\textsc{caus}}
\newcommand{\classif}{\textsc{class}}
\newcommand{\concessif}{\textsc{concsf}}
\newcommand{\comit}{\textsc{comit}}
\newcommand{\conj}{\textsc{conj}}
\newcommand{\const}{\textsc{const}}
\newcommand{\conv}{\textsc{conv}}
\newcommand{\cop}{\textsc{cop}}
\newcommand{\dat}{\textsc{dat}}
\newcommand{\dem}{\textsc{dem}}
\newcommand{\detm}{\textsc{det}}
\newcommand{\dir}{\textsc{dir1}}
\newcommand{\du}{\textsc{du}}
\newcommand{\duposs}{\textsc{du.poss}}
\newcommand{\dur}{\textsc{dur}}
\newcommand{\erg}{\textsc{erg}}
\newcommand{\evd}{\textsc{evd}}
\newcommand{\fut}{\textsc{fut}}
\newcommand{\gen}{\textsc{gen}}
\newcommand{\hypot}{\textsc{hyp}}
\newcommand{\ideo}{\textsc{ideo}}
\newcommand{\imp}{\textsc{imp}}
\newcommand{\impf}{\textsc{ipfv}}
\newcommand{\instr}{\textsc{instr}}
\newcommand{\intens}{\textsc{intens}}
\newcommand{\intrg}{\textsc{intrg}}
\newcommand{\inv}{\textsc{inv}}
\newcommand{\irreel}{\textsc{irr}}
\newcommand{\loc}{\textsc{loc}}
\newcommand{\med}{\textsc{med}}
\newcommand{\negat}{\textsc{neg}}
\newcommand{\neu}{\textsc{neu}}
\newcommand{\nmlz}{\textsc{nmlz}}
\newcommand{\nonps}{\textsc{n.pst}}
\newcommand{\opt}{\textsc{dir2}}
\newcommand{\perf}{\textsc{pfv}}
\newcommand{\pl}{\textsc{pl}}
\newcommand{\plposs}{\textsc{pl.poss}}
\newcommand{\poss}{\textsc{poss}}
\newcommand{\pot}{\textsc{pot}}
\newcommand{\prohib}{\textsc{prohib}}
\newcommand{\pst}{\textsc{pst}}
\newcommand{\recip}{\textsc{recip}}
\newcommand{\redp}{\textsc{redp}}
\newcommand{\refl}{\textsc{refl}}
\newcommand{\sg}{\textsc{sg}}
\newcommand{\sgposs}{\textsc{sg.poss}}
\newcommand{\stat}{\textsc{stat}}
\newcommand{\topic}{\textsc{top}}
\newcommand{\volit}{\textsc{vol}}

\newcommand{\racine}[1]{\begin{math}\sqrt{#1}\end{math}} 
\newcommand{\grise}[1]{\cellcolor{lightgray}\textbf{#1}} 
\newcommand{\tinynb}[1]{\tiny#1}
\begin{document}

\title{Bare converbs in Tangut}
\author{Guillaume Jacques}
\maketitle

\section{Converbs in Rgyalrongic languages}

In Japhug and other Rgyalrong languages use infinitive forms (in  \ipa{kɤ--} or \ipa{kɯ--} as converbs, to express manner or background information. (\citealt{jacques14linking}). These forms can take negation, TAM directional prefixes and even possessive prefixes, but no person marking and no stem 3 alternation.

\begin{exe}
\ex \label{ex:kANke.jari}
\gll
[\ipa{\textbf{kɤ-ŋke}}] 	\ipa{jɤ-ari} 	\ipa{pɯ-ra} \\
\textsc{inf}-walk \textsc{pfv}-go[II] \textsc{pst.ipfv}-have.to \\
\glt He had to go on foot. (elicited)
\end{exe}

 
\begin{exe}
\ex \label{ex:nWmAkAsWz}
\gll
[\ipa{ɯ-ɣi}   	\ipa{ra}   	\ipa{\textbf{nɯ-mɤ-kɤ-sɯz}}]   	\ipa{nɯ}   	\ipa{rŋɯl}   	\ipa{nɯ}   	\ipa{ɲɤ-mbi.}\\
\textsc{3sg.poss}-relative \textsc{pl} \textsc{3pl-neg-inf}-know \textsc{dem} silver \textsc{dem} \textsc{evd}-give\\
\glt  She gave him silver without her relatives knowing. (The Raven4, 161)
\end{exe}

\section{Person marking in Tangut}
Tangut, like Rgyalrongic languages, has a person marking system involving suffixes and stem alternation (\citealt{kepping75agreement, gong01huying, jacques09tangutverb}).

\begin{table}[H]
\caption{Pronouns and personal suffixes}  \centering
\begin{tabular}{llllll} 
\multicolumn{3}{c}{Pronoun} &\multicolumn{3}{c}{Suffix} \\
\tgf{2098} & \ipa{ŋa²}  & 1\sg{} & \tgf{2098} & \ipa{ŋa²}  &1\sg{} \\
\tgf{3926} & \ipa{nja²} & 2\sg{} & \tgf{4601} & \ipa{nja²} &2\sg{} \\
\tgf{4028} &  \ipa{nji²} & 2\sg{}  honorifique ou 2\pl{} & \tgf{4884} & \ipa{nji²} & 1\pl{} et 2\pl{} \\
\end{tabular}
\end{table}
\begin{table}[H]
\caption{Alternating verbs}
\resizebox{\columnwidth}{!}{
\begin{tabular}{lllllllll} 
\toprule
\multicolumn{4}{c}{ A} &\multicolumn{4}{c}{ B} & Meaning \\
\midrule
4517& \tgf{4517} & \ipa{dzji } &1.1		&	4547& \tgf{4547} & \ipa{dzjo } &1.51		&	eat	\\
749& \tgf{0749} & \ipa{phji } &1.11		&	4568& \tgf{4568} & \ipa{phjo } &2.44		&	cause	\\
5026& \tgf{5026} & \ipa{mji } &1.11		&	4894& \tgf{4894} & \ipa{mjo } &1.51		&	hear	\\
731& \tgf{0731} & \ipa{lju } &2.02		&	3189& \tgf{3189} & \ipa{ljo } &2.44		&	throw 	\\
5522& \tgf{5522} & \ipa{ljiij } &2.35		&	5293& \tgf{5293} & \ipa{ljii } &2.12		&	wait 	\\
46& \tgf{0046} & \ipa{ljij } &2.33		&	4803& \tgf{4803} & \ipa{lji } &2.09		&	see	\\
\bottomrule
\end{tabular}}
\end{table}


\begin{table}[H]
\caption{Attested forms in the transitive paradigm}\label{tab:paradigme.atteste} \centering
\begin{tabular}{lllll}
	&	1\sg{}	&	2\sg{}	&	1/2\pl{}	&	3	\\
1\sg{}	&	?	&	A-\ipa{nja²}	&	?	&	 B-\ipa{ŋa²}	\\
2\sg{}	&	A-\ipa{ŋa²}	&	B-\ipa{nja²}	&	A-\ipa{nji²}	&	 B-\ipa{nja²}	\\
1/2\pl{}	&	 A-\ipa{ŋa²}	& ?	&	?	&	A-\ipa{nji²}	\\
3	&	A-\ipa{ŋa²}	&	A-\ipa{nja²}	&	?	&	A 	\\
\end{tabular}
\end{table}

 
\section{Evidence for bare converbs}

In Tangut texts, we observe examples with ``missing'' person markers and stem A regardless of the person, followed by a fully inflected verb.
\newline
\linebreak
\begin{tabular}{llllllllll}
	\tgf{4546}&	\tgf{0716}&	\tgf{2346}&	\tgf{5868}&	\tgf{4851}&	\tgf{5113}&	\tgf{3589}&	\tgf{0705}&	\tgf{5522}&	\tgf{4601}\\
	\tinynb{4546}&	\tinynb{0716}&	\tinynb{2346}&	\tinynb{5868}&	\tinynb{4851}&	\tinynb{5113}&	\tinynb{3589}&	\tinynb{0705}&	\tinynb{5522}&	\tinynb{4601}\\
\end{tabular}
\begin{exe}
\ex \label{ex:tg:attendre.a.1sg.2sg}  \vspace{-8pt}
\gll   \ipa{.jaar²}	\ipa{ɕjii¹}	\ipa{dzjụ²}	\ipa{khie²}	\ipa{bia²}	\ipa{.wji¹}	\ipa{dzjɨj¹zjịj¹}	\ipa{ljiij²-nja²} \\
		poulet tuer[A] riz gruau riz  faire[A] quand attendre[A]-2\sg{} \\
\glt Ayant tué un poulet et préparé du gruau de riz, je t'attendrai. (Leilin 03.04B.3; Typo in \citealt[221]{jacques14esquisse})
\end{exe}

\newline
\linebreak
\begin{tabular}{llllllllll}
 \tgf{1796} & 	\tgf{3830} & 	\tgf{2612} & 	\tgf{2226} & 	\tgf{3583} & 	\tgf{1204} & 	\tgf{3791} & 	\tgf{3349} & 	\tgf{1326} & 	\tgf{0134} \\
\tinynb{1796} & 	\tinynb{3830} & 	\tinynb{2612} & 	\tinynb{2226} & 	\tinynb{3583} & 	\tinynb{1204} & 	\tinynb{3791} & 	\tinynb{3349} & 	\tinynb{1326} & 	\tinynb{0134} \\
\tgf{3830} & 	\tgf{3926} & 	\tgf{3791} & 	\tgf{2226} & 	\tgf{1204} & 	\tgf{4931} & 	\tgf{3513} & 	\tgf{3349} & 	\tgf{1326} & 	\tgf{0134} \\
\tinynb{3830} & 	\tinynb{3926} & 	\tinynb{3791} & 	\tinynb{2226} & 	\tinynb{1204} & 	\tinynb{4931} & 	\tinynb{3513} & 	\tinynb{3349} & 	\tinynb{1326} & 	\tinynb{0134} \\
\tgf{3513} & 	\tgf{0645} & 	\tgf{4601} & 	\tgf{1101} &\tgf{3916} & \\
\tinynb{3513} & 	\tinynb{0645} & 	\tinynb{4601} & 	\tinynb{1101} &\tinynb{3916} & \\
\end{tabular}


\begin{exe}
\ex    \vspace{-8pt}
\gll    \ipa{tɕhjụ¹} 	\ipa{njij²} 	\ipa{phju²} 	\ipa{.we²} 	\ipa{tja¹} 	\ipa{njijr²} 	\ipa{bji²} 	\ipa{rjijr²} 	\ipa{kjɨ¹-.ju¹} 	\ipa{njij²} 	\ipa{nja²} 	\ipa{bji²} 	\ipa{.we²} 	\ipa{njijr²} 	\ipa{dʑji°} 	\ipa{mə¹} 	\ipa{rjijr²} 	\ipa{kjɨ¹-.ju¹} 	\ipa{mə¹} 	\ipa{.wu²-nja²-.jij¹-sji²}  \\ 
Chu roi haut faire \topic{} visage bas côté \dir{}-regarder roi toi  bas faire visage tendre ciel  côté \dir{}-regarde ciel aider-2\sg{}-\fut{}-\perf{} \\
\glt  Le roi de Chu était en haut et s'est tourné vers le bas, tandis que toi, ô Roi, tu étais en bas, et t'es tourné vers le ciel, (cela signifie que) le ciel t'aidera. (Leilin 06.15B.7)
\end{exe}
\newline
\linebreak
\begin{tabular}{llllllllll}
\tgf{3133} & 	\tgf{3830} & 	\tgf{3193} & 	\tgf{2937} & 	\tgf{4481} & 	\tgf{1542} & 	\tgf{1326} & 	\tgf{2833} & 	\tgf{1918} & 	\tgf{2912} & 	\tgf{4601} & 
\tinyb{3133} & 	\tinyb{3830} & 	\tinyb{3193} & 	\tinyb{2937} & 	\tinyb{4481} & 	\tinyb{1542} & 	\tinyb{1326} & 	\tinyb{2833} & 	\tinyb{1918} & 	\tinyb{2912} & 	\tinyb{4601} & 
\end{tabular}
\begin{exe}
\ex    \vspace{-8pt}
\gll \ipa{sjij¹} \ipa{njij²} \ipa{tshjĩ¹} \ipa{lhjịj} \ipa{ɕjɨ¹} \ipa{ku¹} \ipa{kjɨ¹djɨj²} \ipa{mji¹-lhjwo¹-nja²} \\
aujourd'hui roi Qin pays aller alors certainement \textsc{neg}-revenir-\textsc{2sg} \\
\glt   Si vous allez au pays de Qin, il est certains que vous ne reviendrez pas. (Leilin 03.22A36-7)
\end{exe}

Choice between two constructions: example (\ref{ex:agjwi1}) with a bare converb vs example (\ref{ex:agjwi2}) with two inflected verbs.
\newline
\linebreak
\begin{tabular}{llllllllll}
\tgf{5981} & 	\tgf{3195} & 	\tgf{1012} & 	\tgf{5612} & 	\tgf{0322} & 	\tgf{4841} & 	\tgf{5918} & 	\tgf{2098} & \\
\tinynb{5981} & 	\tinynb{3195} & 	\tinynb{1012} & 	\tinynb{5612} & 	\tinynb{0322} & 	\tinynb{4841} & 	\tinynb{5918} & 	\tinynb{2098} & \\
\end{tabular}

\begin{exe}
\ex    \vspace{-8pt} \label{ex:agjwi1}
\gll    \ipa{.a-gjwi²}	\ipa{zjịj¹}	\ipa{tshjiij¹}	\ipa{tɕhjwo¹}	\ipa{djij²-sjɨ¹-ŋa²} \\
une-parole quelques dire[A] alors \textsc{dir2}-mourir-\textsc{1sg}   \\
\glt  Que (l'on me permette) de dire quelques mots avant de mourir! (Leilin 03.06A.6)
\end{exe}
\newline
\linebreak
\begin{tabular}{llllllllllll}
\tgf{5981} & 	\tgf{3195} & 	\tgf{0795} & 	\tgf{5612} & 	\tgf{0749} & 	\tgf{2098} & 	\tgf{0705} & 	\tgf{5918} & 	\tgf{4841} & 	\tgf{4279} & 	\tgf{2098} & \\
\tinyb{5981} & 	\tinyb{3195} & 	\tinyb{0795} & 	\tinyb{5612} & 	\tinyb{0749} & 	\tinyb{2098} & 	\tinyb{0705} & 	\tinyb{5918} & 	\tinyb{4841} & 	\tinyb{4279} & 	\tinyb{2098} & \\
\end{tabular}

\begin{exe}
\ex    \vspace{-8pt} \label{ex:agjwi2}
\gll 
\ipa{.a-gjwi²}	\ipa{rjɨr²-tshjiij¹-phji¹-ŋa²}	\ipa{zjịj¹}	\ipa{sjɨ¹}	\ipa{djij²-.wəə¹-ŋa²} \\
une-parole \textsc{dir1}-dire[A]-causer[A]-\textsc{1sg} quand mourir \textsc{dir2}-soumis-\textsc{1sg} \\
\glt  Si vous me laissez dire une parole, et je mourrai résigné à mon sort.  (Leilin 03.08B.1)
\end{exe}

\bibliographystyle{Linquiry2}
\bibliography{bibliogj}
\end{document}