\documentclass[oneside,a4paper,11pt]{article} 
\usepackage{fontspec}
\usepackage{natbib}
\usepackage{booktabs}
\usepackage{xltxtra} 
\usepackage{polyglossia} 
\usepackage[table]{xcolor}
\usepackage{gb4e} 
\usepackage{multicol}
\usepackage{graphicx}
\usepackage{float}
\usepackage{hyperref} 
\hypersetup{bookmarks=false,bookmarksnumbered,bookmarksopenlevel=5,bookmarksdepth=5,xetex,colorlinks=true,linkcolor=blue,citecolor=blue}
\usepackage[all]{hypcap}
\usepackage{memhfixc}
\usepackage{lscape}
\usepackage{amssymb}
 
%\setmainfont[Mapping=tex-text,Numbers=OldStyle,Ligatures=Common]{Charis SIL} 
\newfontfamily\phon[Mapping=tex-text,Ligatures=Common,Scale=MatchLowercase]{Charis SIL} 
\newcommand{\ipa}[1]{{\phon\textbf{#1}}} 
\newcommand{\grise}[1]{\cellcolor{lightgray}\textbf{#1}}
\newfontfamily\cn[Mapping=tex-text,Ligatures=Common,Scale=MatchUppercase]{SimSun}%pour le chinois
\newcommand{\zh}[1]{{\cn #1}}
\newcommand{\Y}{\Checkmark} 
\newcommand{\N}{} 
\newcommand{\dhatu}[2]{|\ipa{#1}| `#2'}
\newcommand{\jpg}[2]{\ipa{#1} `#2'}  
\newcommand{\refb}[1]{(\ref{#1})}
\newcommand{\tld}{\textasciitilde{}}
\newcommand{\zhc}[2]{\zh{#1} \ipa{#2}} 

 \begin{document} 
\title{Antipassive derivations in Sino-Tibetan/Trans-Himalayan and their sources\footnote{The Japhug examples are taken from a corpus that is progressively being made available on the Pangloss archive (\citealt{michailovsky14pangloss},  
 \url{http://lacito.vjf.cnrs.fr/pangloss/corpus/list\textunderscore rsc.php?lg=Japhug}). This research was funded by the HimalCo project (ANR-12-CORP-0006) and  the Labex Empirical Foundations of Linguistics (ANR/CGI). I would like to thank Linda Konnerth, Ma Kun, Randy LaPolla, Mark W. Post and Willem de Reuse for comments on this paper.}}
\author{Guillaume Jacques\\ CNRS-CRLAO-INALCO}
\maketitle

\sloppy

\section*{Introduction}
Although the existence of antipassive constructions has been mentioned in several Sino-Tibetan languages (\citealt[225-7]{doornenbal09}, \citealt{jacques14antipassive}, \citealt{bickel15antipassive}), this topic has not yet received as much attention as other voice constructions such as passive or causative.

This paper is a survey of antipassive constructions in the Sino-Tibetan family. Since all of these constructions are historically transparent, they are classified by their diachronic source. Recent work on diachronic typology (\citealt[235]{janic.these},  \citealt{jacques14antipassive}, \citealt{sanso17antipassive}) has shown that antipassive constructions have four major sources in the world's languages:
\begin{itemize}
\item Agent nominalizations. (`he is the hitter' $\rightarrow$  `he hits (intr)')
\item Generic nouns/Indefinite pronouns in object position  (`he hits things/stuff' $\rightarrow$  `he hits (intr)')
\item Action nominalization + light verb (\citealt{creissels12antip}) / denominal verbalizer (\citealt{jacques14antipassive}) (`he does hitting' $\rightarrow$  `he hits (intr)')
\item Reflexive (with an intermediate stage as `co-participation' \citealt{creissels08coparticipation}) (`they hit themselves/each other' $\rightarrow$  `they partake in hitting actions' $\rightarrow$  `they hit (intr)')
\end{itemize}

In this paper, I first present a definition of antipassive and discuss related antipassive-like constructions in several languages of the ST family. Then, I provide evidence of antipassive derivations originating from three out of the four main attested sources: action nominalization, generic nouns and reflexives. These derivations are all of recent origin, but some are argued to be reconstructible to lower branches of the family. Finally, I present an overview of the distribution of antipassive construction throughout ST.

\section{Antipassive and indefinite objects}
Since transitivity is overtly (and often redundantly) marked in the morphology-rich languages of the ST family, I propose for this paper the following definition of antipassive (closely based on \citealt[146]{dixon94erg}):

\begin{exe}
\ex \label{ex:def}
\glt An antipassive construction is an overtly-marked flexion, derivation or periphrastic construction which (possibly among other functions) turns a transitive verb into an intransitive one. The agent-like argument of the base verb becomes the sole core argument of the intransitive verb, and has the same morphosyntactic properties as the sole arguments of underived intransitive verbs, while the patient-like argument is either deleted or demoted to non-core argument function.
\end{exe}

This definition excludes  (i) agent-preserving lability (since even if one could argue that the intransitive use of the verb is derived from the transitive one, it would be a zero derivation), (ii) constructions where the verb remains morphologically transitive, or maintaining an obligatory ergative marker on the A and (iii) other detransitivizing constructions such as passive, anticausative, reciprocal or reflexive. It can be applied to languages without morphological marking of transitivity if explicit criteria to distinguish transitive from intransitive construction are provided, though in the case of the Sino-Tibetan family, antipassive constructions are clearly attested only in with complex morphology. 

Although this definition is independent of the alignment of the case marking of indexation systems, antipassive constructions are more easily detectable in languages with ergatively-aligned case marking, as the agent-like argument of the transitive base verb and the sole argument of the intransitive derived verb receive different case marking in antipassive constructions. In languages with accusative alignment in case marking, case marking cannot be used as a criterion to define antipassivization.

No language of the Sino-Tibetan family has syntactic ergativity of the Dyirbal type, requiring for instance the use of the antipassive to convert the A  of a transitive verb to S status to allow relativization (\citealt[170]{dixon94erg}). Antipassive constructions in the Sino-Tibetan family are mainly used to express indefiniteness.

In language with polypersonal indexation and/or obligatory marking of transitivity, non-overt arguments are understood as definite. For instance, a Japhug sentence like (\ref{ex:tAχtWta}), with the transitive verb \ipa{χtɯ} `buy' (note the unambiguous past transitive \ipa{-t-} suffix), can only be interpreted as meaning `I bought it' with a definite (and previously mentioned) object.

\begin{exe}
\ex \label{ex:tAχtWta}
\gll \ipa{tɤ-χtɯ-t-a} \\
\textsc{pfv}-buy-\textsc{tr:pst-1sg} \\
\glt `I bought it.' (Japhug)
\end{exe}

In order to express an indefinite object, it is therefore not an option to simply leave the object position empty. Antipassive, as in (\ref{ex:tAraχtWa}; note the absence of transitive \ipa{-t-} suffix), is one way to express indefiniteness. 

\begin{exe}
\ex \label{ex:tAraχtWa}
\gll \ipa{tɤ-ra-χtɯ-a} \\
\textsc{pfv}-\textsc{antipass}-buy-\textsc{1sg} \\
\glt `I bought things.' (Japhug)
\end{exe}

Other strategies are however possible; in this section, I present four competing constructions used to to express indefinite objects in ST, which should not be confused with antipassive: lability, indefinite objects, light verbs and incorporation.  

\subsection{Agent-preserving lability} \label{sec:labile}
ST languages with polypersonal indexation all present some degree of lability, ie constructions where the same verb root can be conjugated either transitively or intransitively, with effect on case marking on the arguments. The intransitive use of the verb can be patient-preserving (the sole argument of the construction corresponding to the patient-like argument of the transitive construction), or agent-preserving (when it corresponds to the agent-like argument).  Limbu can be used to illustrate these constructions, which are attested with a few verbs such as  \ipa{kʰutt} `steal' (\citealt[527]{driem91tangut}), which can be conjugated transitively (\ref{ex:khuttang}) or intransitively with preservation of the patient (\ref{ex:sapla}) or the agent (\ref{ex:andzumin}).

\begin{exe}
\ex \label{ex:khuttang}
\gll \ipa{A-ndzum-ille}	\ipa{sapla}	\ipa{khutt-aŋ} \\
\textsc{1sg.poss}-friend-\textsc{erg} book steal-\textsc{1sg.P.pst} \\
\glt `My friend robbed me of my book.' (Limbu)
\end{exe}

\begin{exe}
\ex \label{ex:sapla}
\gll 
\ipa{Sapla}	\ipa{khutt-ɛ} \\
book steal-\textsc{pst:intr} \\
\glt `The book was stolen.' (Limbu)
\end{exe}

\begin{exe}
\ex \label{ex:andzumin}
\gll 
\ipa{A-ndzum-in}	\ipa{khutt-ɛ} \\
\textsc{1sg.poss}-friend-\textsc{def} steal-\textsc{pst:intr} \\
\glt `My friend committed a theft.' (Limbu)
\end{exe}
In addition to effect on verbal morphology and person indexation, lability also affects case marking: thus, in the case of agent-preserving lability, the agent-like argument receives ergative case in the transitive construction (\ref{ex:khuttang}), and absolutive case in the intransitive one (\ref{ex:andzumin}). Not all ST languages allow both types of lability; in Japhug, only agent-preserving lability is attested (\citealt[218]{jacques12demotion}).

 While some scholars such as \citet[359]{schackow15yakkha} use the term `antipassive' to refer to patient-preserving lability, in the more restricted definition proposed in (\ref{ex:def}), a detransitivizing construction without overt mark cannot be referred to as antipassive.

Agent preserving lability is a marginal phenomenon in languages such as Limbu or Japhug (where it concerns a restricted list of verbs, see \citealt[218]{jacques12demotion}), but it is quite productive and prominent in some Kiranti languages, such as Puma (the $\varnothing$-detransitive construction described in \citealt[9]{bickel07puma}; see \citealt{bickel11multivariate} for an examination of the various potential analyses of this construction).

In Hakha Lai, a Kuki-Chin language, \citet{kathol01alternations} have proposed to analyze as antipassive the alternation between stem I and stem II with transitive verbs.  Hakha Lai verbs have two stems (I and II); stem I is obligatory with negative and interrogative markers, stem II obligatorily occurs in some subordinate clause, but in affirmative indicative main clauses, stem alternation is determined by transitivity: intransitive verbs have stem I, while transitive verbs have stem II when the A takes the ergative marker \ipa{=niʔ}, as in example (\ref{ex:abaq}) (\citealt[413]{peterson03hakha})

\begin{exe}
\ex \label{ex:abaq}
\gll \ipa{paalaw=niʔ} \ipa{thil} \ipa{khaaʔ} \ipa{ʔa-baʔ} \\
p.n=\textsc{erg} clothes \textsc{dem} \textsc{3sg}-hang.up:II \\
\glt `Paalaw hung up the clothes.' (Hakha Lai)
\end{exe}

Transitive verbs can also be used in affirmative independent clauses in Stem I, as in example (\ref{ex:abat}). In this case, the A does not take ergative case. This is the construction which \citet{kathol01alternations} analyze as antipassive.

\begin{exe}
\ex \label{ex:abat}
\gll \ipa{paalaw} \ipa{khaaʔ}  \ipa{thil} \ipa{ʔa-bat} \\
p.n \textsc{dem} clothes \textsc{3sg}-hang.up:I \\
\glt `Paalaw hangs up/hung up the clothes.' (Hakha Lai)
\end{exe}

In this construction, stem alternation is not by itself a mark of voice derivation. Since intransitive verbs occur with stem I in affirmative independent clauses, stem alternation between examples (\ref{ex:abaq}) and (\ref{ex:abat}) rather reflects the same verb stem conjugated transitively and intransitively respectively, ie agent-preserving lability, and thus not antipassive proper according to the definition proposed in this paper.\footnote{Note also that the object of the transitive construction is not demoted to oblique status in the detransitive construction in (\ref{ex:abat}), an observation that \citet[413]{peterson03hakha} uses as argument against the antipassive analysis. \citet[37]{peterson07appl} explicitly states that `Hakha Lai has no valence-affecting constructions which target objects, such as passive or antipassive.'} 
 
\subsection{Indefinite/generic objects} \label{sec:indef}
Indefinite patient-like arguments can be expressed by indefinite pronouns in object position, or in some languages by an indefinite/generic marker on the verb, as illustrated by the Japhug examples (\ref{ex:thWci}) and (\ref{ex:YWkWnWGmu}).

\begin{exe}
\ex  \label{ex:thWci}
\gll \ipa{ɯ-jaʁ} \ipa{nɯtɕu} \ipa{tʰɯci} \ipa{ɲɤ-kʰo} \ipa{tɕe} \\
\textsc{3sg.poss}-hand \textsc{dem:loc} something \textsc{ifr}-give \textsc{lnk} \\
\glt `(Smanmi) gave him something in his hand.' (2011-4-smanmi, 105) (Japhug)
\end{exe}

\begin{exe}
\ex  \label{ex:YWkWnWGmu}
\gll
\ipa{nɯnɯ} 	\ipa{kɯ} 	\ipa{tɯrme} 	\ipa{wuma} 	\ipa{ʑo} 	\ipa{ɲɯ-kɯ-nɯɣ-mu.} \\
\textsc{dem} \textsc{erg} people really \textsc{emph} \textsc{ipfv-genr:S/P-appl}-be.afraid \\
\glt `That (bird) is very afraid of people.' (hist-24-ZmbrWpGa, 26) (Japhug)
\end{exe}

In both of these examples, the verb remains transitive, the patient-like argument is still overt (in the case of the generic construction in \ref{ex:YWkWnWGmu}, only the noun \ipa{tɯrme} `people' or the generic pronoun \ipa{tɯʑo} `oneself' can be used with a verb taking the \ipa{kɯ-} prefix)  and the agent-like argument takes the ergative marker.

However, some languages present constructions intermediate between fully transitive constructions as in (\ref{ex:thWci}) and (\ref{ex:YWkWnWGmu}) and canonical antipassives. 

In Bantawa, \citet[226;335]{doornenbal09} refers to the construction illustrated by example (\ref{ex:hityang}) as an `explicit antipassive'. In this construction, the verb conjugated intransitively (\ipa{hɨtt} `burn'), the agent-like argument is marked with the ergative and indexed one the verb with the same marking as an intransitive subject, and the indefinite \ipa{kʰa} `something' is obligatorily present in object position.

\begin{exe}
\ex \label{ex:hityang}
\gll 
\ipa{nam-ʔa} \ipa{kʰa} \ipa{hɨt-yaŋ} \\
sun-\textsc{erg} something scorch-\textsc{3sg:intr:prog} \\
\glt ‘The sun is scorching.’ (Bantawa)
\end{exe}

While this construction is certainly the source for the antipassive constructions found in Puma (see section \ref{sec:kha}), the presence of the ergative on the agent-like argument precludes from treating it as a canonical antipassive in the sense given in (\ref{ex:def}) above.

\subsection{Light verb construction} \label{sec:light}
An alternative construction used by some languages to avoid an explicit patient-like argument is to replace the transitive verb by a construction combining a nominal form derived from the transitive verb and a light verb. This construction is illustrated by Japhug (\ref{ex:tWtsGe}), with the nominal \ipa{tɯtsɣe} related to the verb \ipa{ntsɣe} `sell' of the simple transitive construction in (\ref{ex:YWntsGe}).\footnote{The irregular correspondence between \ipa{tɯtsɣe} `commerce' and \ipa{ntsɣe} `sell' is explained in \citet{jacques14antipassive}. }

\begin{exe}
\ex \label{ex:YWntsGe}
\gll \ipa{ɯ-me} 	\ipa{nɯ} 	\ipa{kɯ} 	\ipa{andi} 	\ipa{paχɕa} 	\ipa{ɲɯ-ntsɣe} 	\ipa{ŋu} \\
\textsc{3sg.poss}-daughter \textsc{dem} \textsc{erg} west pork \textsc{ipfv}-sell be:\textsc{fact} \\
\glt `Her daughter sells pork there.' (hist-17-lhazgron, 118) (Japhug)
\end{exe}


\begin{exe}
\ex \label{ex:tWtsGe}
\gll <ali> 	\ipa{kɯ-rmi} 	\ipa{nɯnɯ} 	\ipa{kɯ,} 	\ipa{tɯtsɣe} 	\ipa{tu-βze} 	\ipa{tɕe} 	\ipa{nɯ} 	\ipa{kɯ-fse} 	\ipa{ku-rɤʑi} 	\ipa{pjɤ-ŋu.}  \\
Ali \textsc{nmlz}:S/A-be.called \textsc{dem} \textsc{erg} commerce \textsc{ipfv}-do[III] \textsc{lnk} \textsc{dem} \textsc{nmlz}:S/A-be.like \textsc{ipfv}-stay \textsc{ifr.ipfv}-be \\
\glt `The person called Ali did commerce and lived like that.' (hist140516 yiguan ganlan, 4) (Japhug)
\end{exe}

Although the construction in (\ref{ex:tWtsGe}) removes the patient-like argument, it cannot be considered to be an analytic antipassive, as the main verb of the construction \ipa{βzu} is still transitive, and the agent-like argument takes the ergative \ipa{kɯ}. 


\subsection{Noun incorporation} \label{sec:incorp}
Noun incorporation can affect verbal transitivity. We commonly find examples of incorporation in which a transitive verb becomes intransitive, and the incorporated noun corresponds to the patient-like argument of the base verb and saturates its place in the argument structure.

In Japhug for instance, the intransitive incorporating verb \ipa{ɣɯ-sɯ-pʰɯt} `chop firewood' derives from the transitive verb \ipa{pʰɯt} `cut, chop' and the noun  \ipa{si} `wood, tree' (incorporated in \textit{status constructus} form \ipa{sɯ-} with the denominal prefix \ipa{ɣɯ-}, see  \citealt{jacques12incorp}). Example (\ref{ex:kosWphWt}) shows the transitive construction, with the subject taking the ergative \ipa{kɯ} and the verb with the progressive prefix \ipa{asɯ-/ɤsɯ-/osɯ-} which only appears on transitive verbs, while (\ref{ex:kuGWsWphWt}) show the corresponding incoporating construction, without ergative marking on the subject and progressive on the verb.

 
\begin{exe}
\ex \label{ex:kosWphWt}
\gll \ipa{a-wa} \ipa{kɯ} \ipa{si} \ipa{ku-osɯ-pʰɯt} \\
\textsc{1sg.poss}-father \textsc{erg} tree \textsc{egoph.pres-prog}-fell \\
\glt `My father is felling trees/the tree.' (Japhug)
\end{exe}	 

\begin{exe}
\ex   \label{ex:kuGWsWphWt}
\gll \ipa{a-wa} \ipa{ku-ɣɯ-sɯ-pʰɯt} \\
\textsc{1sg.poss}-father \textsc{egoph.pres-denom}-tree-fell \\
\glt `My father is felling trees.' (Japhug)
\end{exe}	 

Constructions of the type illustrated by example (\ref{ex:kuGWsWphWt}) have been referred to as antipassive (\citealt[47-48]{say09antipassive}) and indeed fulfil the definition proposed in (\ref{ex:def}). Note the parallelism between (\ref{ex:kuGWsWphWt}) and the antipassive (\ref{ex:pjArAtsxWB}) below.

However, a full examination of antipassive-like incorporation in ST is not possible until a survey of incorporation in the family has been undertaken, and has therefore to be deferred to future research. In particular, the presence of noun incorporation in Kiranti languages such as Puma or Chintang crucially depends on one's analysis of the zero detransitive construction (\citealt{bickel11multivariate}).

 
\section{Incorporation of generic noun / indefinite element} \label{sec:kha}
Puma has an antipassive \ipa{kʰa-} prefix whose function can be illustrated by examples (\ref{ex:ennung}) and (\ref{ex:khaennga}) taken from \citet[7-9]{bickel07puma}). The base verb \ipa{enn-} `hear' in (\ref{ex:ennung}) is transitive; it indexes both subject and object, and the subject takes the ergative suffix \ipa{-a}.

\begin{exe}
\ex \label{ex:ennung}
\gll 
\ipa{ŋa-a} \ipa{kho-lai} \ipa{enn-u-ŋ} \\
\textsc{1sg-erg} \textsc{3sg-dat} hear.\textsc{n.pst}-\textsc{3sg:P-1sg:A} \\
\glt `I hear him/her.' (Puma)
\end{exe}

The corresponding form with prefixed \ipa{kʰa-} in (\ref{ex:khaennga}) is morphologically intransitive, only indexes one argument, and the only argument (\textsc{1sg}) is in the absolutive. 

\begin{exe}
\ex \label{ex:khaennga}
\gll 
\ipa{ŋa} \ipa{kʰa-en-ŋa} \\
\textsc{1sg} \textsc{antipass}-hear-\textsc{1sg}:S/P \\
\glt `I hear someone/people.' (not `I hear something') (Puma)
\end{exe}
 
The demoted object argument cannot be relativized (\citealt[10]{bickel07puma}), while the subject presents all the properties of a intransitive subject; this construction unambiguously fulfils all criteria of a canonical antipassive (\ref{ex:def}). 

A particularity of the Puma antipassive is that the demoted object can only refer to humans; to refer to indefinite non-human, a labile construction (the $\varnothing$-detransitive) is used instead.

The Puma antipassive prefix \ipa{kʰa-} is obviously related to the `antipassive' construction (\citealt[226;335]{doornenbal09}) with the indefinite \ipa{kʰa} `something' mentioned in section \ref{sec:indef}. The Bantawa and the Puma constructions differ in several regards:
\begin{itemize}
\item In Bantawa the agent-like argument is marked with the ergative (resulting in a mismatch between case marking and indexation, since the subject is indexed as the sole argument of an intransitive verb), while in Puma it is in the absolutive.
\item In Puma, the demoted object is necessarily interpreted as human, while no such constraint exists in Bantawa.
\item The element \ipa{kʰa} is phonologically less integrated into the verbal word in Bantawa than in Puma.
\end{itemize}

The etymology of the indefinite element \ipa{kʰa} still deserves additional discussion (\citealt[67]{bickel15antipassive} argue that the Puma antipassive is related to the etymon reflected as Khaling \ipa{kʰɵle} `all', proto-Kiranti \ipa{*kʰɑle} in Jacques' \citeyear{jacques17pkiranti} system). In any case, within South Kiranti (the branch to which Bantawa and Puma belong), the following stages can be postulated:

\begin{enumerate}
\item X-\textsc{erg} {   } \textsc{indefinite}:\textsc{abs} {   } V:X$\rightarrow$\textsc{3sg} (fully transitive construction)
\item X-\textsc{erg} {   } \textsc{indefinite}:\textsc{abs}=V:X:\textsc{intr} (Bantawa)
\item X:\textsc{abs}  {   } \textsc{antipass}-V:X\textsc{intr} (Puma)
\end{enumerate}

While a canonical antipassive in \ipa{kʰa-} is only attested in Puma, \citet{bickel15antipassive} point out that the first inclusive object marker \ipa{kʰa-} in Chamling and Western Chintang is historically related, and that an intermediate stage as an antipassive could be postulated. The Western Chintang 2/3$\rightarrow$\textsc{1n.sg} forms are in particular exactly identical the corresponding second or third person intransitive forms with the addition of the \ipa{kʰa-} prefix. Since however no mention is made of a constraint against ergative marking on the subject with these verb forms, its is likely that a Bantawa-like construction (stage 2) rather than a full-grown antipassive as in Puma  has to be postulated.

 
\section{Action nominalization + denominal verbalization} \label{sec:ra}
The Northern Gyalrong languages, Tshobdun (\citealt{jackson06paisheng, jackson14morpho} ), Japhug  (\citealt{, jacques12demotion, jacques14antipassive}) and Zbu, have a pair of antipassive prefixes \ipa{rɐ-} and \ipa{sɐ-} (in Tshobdun) and \ipa{rɤ-/ra-} and \ipa{sɤ-/sa-} (in Japhug), respectively used to indicate non-human and human indefinite patient. No cognate antipassive prefixes have been reported in the closely related languages Situ (\citealt[98]{zhang16bragdbar}), Khroskyabs (\citealt{lai13affixale}) and Stau (\citealt{jacques17stau}), and they could be a northern Gyalrong innovation.

The following examples illustrate the use of the antipassive prefix \ipa{rɤ-} in Japhug; the base verb \ipa{tʂɯβ} `sew' requires the subject to take the ergative \ipa{kɯ}, and has to take the transitive progressive prefix \ipa{asɯ-/ɤsɯ-} to be used in inferential imperfective form (\ref{ex:pjAkAsWtsxWBci}) (see \citealt{jacques17sketch} on this restriction), while the derived intransitive verb  \ipa{rɤ-tʂɯβ} `sew things; do sewing' cannot take an overt patient, does not select the ergative on the subject and cannot take the progressive prefix \ipa{asɯ-/ɤsɯ-}.\footnote{Note that in the text corpus at my disposal, antipassive verb forms are mainly attested in either imperfective finite forms or nominalized forms. Although perfective forms of these verbs can be elicited (see example \ref{ex:tAraχtWa} above), they are not commonly employed (on the interaction of antipassivization and aspect, see in particular \citealt{cooreman94antipassive}).} 

%\begin{exe}
%\ex \label{ex:chAtsxWB}
%\gll 
%\ipa{stu}	\ipa{kɯ-xtɕi}	\ipa{nɯ}	\ipa{kɯ}	\ipa{nɯŋa}	\ipa{ɣɯ}	\ipa{ɯ-ndʐi}	\ipa{nɯnɯ}	\ipa{cʰɤ-pɣaʁ}	\ipa{tɕendɤre}	\ipa{cʰɤ-tʂɯβ,} \\
%most \textsc{nmlz:S/A}-be.small \textsc{dem} \textsc{erg} cow \textsc{gen} \textsc{3sg.poss}-skin \textsc{dem} \textsc{ifr}-turn \textsc{ifr}-sew \\
%\glt `The youngest (brother) turned the cow's hide over and sewed it.' (hist-02-deluge2012, 24-5)
%\end{exe}

\begin{exe}
\ex \label{ex:pjAkAsWtsxWBci}
\gll 
\ipa{rgɤnmɯ}	\ipa{nɯ}	\ipa{kɯ}	\ipa{li}	\ipa{iɕqʰa}	<yuwang>	\ipa{nɯ}	\ipa{pjɤ-k-ɤsɯ-tʂɯβ-ci} \\
old.woman \textsc{dem} \textsc{erg} again the.aforementioned fish.net \textsc{dem} \textsc{ifr.ipfv-evd-prog}-sew-\textsc{evd} \\
\glt `The old woman was sewing the fish nets.' (hist140430 yufu he tade qizi, 297) (Japhug)
\end{exe}

\begin{exe}
\ex \label{ex:pjArAtsxWB}
\gll \ipa{iɕqʰa}	\ipa{kɯ-rɤ-tʂɯβ}	\ipa{nɯ}	\ipa{pɤjkʰu}	\ipa{pjɤ-rɤ-tʂɯβ}	\ipa{ɕti.} \\
the.aforementioned \textsc{nmlz:S/A-antipass}-sew  \textsc{dem} already  \textsc{ifr:ipfv-antipass}-sew be:\textsc{affirmative:fact} \\
\glt `(Very early in the morning), the tailor was already sewing.' (hist140512 alibaba, 151) (Japhug)
\end{exe}

\citet{jacques14antipassive} accounts for the \ipa{rɤ-} prefix as originating from the reanalysis of the intransitive denominal \ipa{rɤ-/rɯ-} prefix. This reanalysis took place in two steps. 

First, an action or patient nominal is derived from the intransitive verb (for instance, \ipa{ɕpʰɤt} `patch (transitive)' $\rightarrow$  \ipa{tɤ-ɕpʰɤt} `a patch (noun)'). Such nominals take either a nominalization \ipa{tɯ-} prefix or combine the bare verb root with a possessive prefix (which can be either a definite possessive such as \ipa{ɯ-} `his/her/its' or an indefinite possessor \ipa{tɤ-/ta-} as in the example `patch' above). This nominalization neutralizes the valency of the base verb.

Second, this nominal undergoes denominal verbalizing derivation by means of the prefix \ipa{rɤ-}. The possessive or nominalization prefixes are removed during this derivation, as is the case with nouns that are not derived from verbs, as in the following examples:
\begin{itemize}
\item \ipa{ta-ma} `work (noun)'\footnote{The prefix \ipa{ta-} here is the indefinite possessor prefix, as it is an inalienably possessed noun, which requires a possessive prefix. }  $\rightarrow$  \ipa{rɤ-ma} `work (intransitive)'
\item \ipa{tɯ-krɤz} `discussion'\footnote{The prefixal element \ipa{tɯ-} is here synchronically unanalyzable, but could be a fossilized indefinite possessor. The root \ipa{-krɤz} is borrowed from the Tibetan noun \ipa{gros} `discussion'. } $\rightarrow$ \ipa{rɤ-krɤz} `discuss (intransitive)'
\end{itemize}

The second stage of the derivation \ipa{tɤ-ɕpʰɤt} `a patch (noun)' $\rightarrow$ \ipa{rɤ-ɕpʰɤt} `patch, do patching (intransitive)') is thus still transparent; \ipa{rɤ-ɕpʰɤt} is synchronically ambiguous between a denominal derivation from the noun `patch' and an antipassive derivation of the base verb  patch `transitive'. The intermediate noun is however not clearly attested for all verbs, and the antipassive \ipa{rɤ-} is synchronically a distinct morpheme from the denominal \ipa{rɤ-}.

Note that the antipassive is not isolated among voice derivations in Gyalrong languages to originate from a denominal prefix; the same source has been proposed for causative, applicative and passive prefixes (see \citealt{jacques15causative}, \citealt{lai18denom}).

The antipassive in \ipa{rɤ-} is semantically very close to the light verb construction mentioned in \ref{sec:light}, with the verb \ipa{ntsɣe} `sell' and the nominal \ipa{tɯ-tsɣe} `commerce'. Note that the antipassive \ipa{rɤ-tsɣe} `do commerce, sell things' is irregular in that its root \ipa{tsɣe} slightly differs from that of the base verb \ipa{ntsɣe} `sell', an irregularity shared with the action nominal \ipa{tɯ-tsɣe} `commerce'. This common irregularity is a further clue that the antipassive in \ipa{rɤ-} diachronically comes through a action nominal stage.


\section{Reflexive/Middle}
One of the most common sources of antipassive constructions, in particular in languages with accusatively aligned case marking, are reflexive/middle markers (\citealt{janic16antipassif}). 

Most of the morphology-rich branches of the family, including Kiranti, Thangmi, Dulong-Rawang, Kham and West-Himalayish (but not Gyalrongic), share a reflexive suffix with a dental fricative followed by a high fronted vowel (Limbu \ipa{-siŋ}, Khaling \ipa{-si}, Kham \ipa{-si}, Rawang \ipa{-shì} etc), which is likely to be reconstructible to proto-ST (\citealt[94]{bauman75}, \citealt[320]{driem93agreement}, \citealt{jacques16ssuffixes}).

There is some diffuse evidence for antipassive-like use of these suffixes in some ST languages.

\subsection{Kiranti} \label{sec:kiranti.si}
In Kiranti, we find a few lexicalized examples of antipassive-like use of the reflexive in Khaling, Thulung and Limbu.

In Khaling (\citealt{jacques16si}), the \ipa{-si} derivation in Khaling, alongside reflexive, reciprocal, autobenefactive and generic subject, also has an antipassive value when applied to transitive verbs expressing a feeling (whose A and P are experiencers and stimuli respectively). As shown by examples (\ref{ex:ghryamt})  and (\ref{ex:ghryamtsi}), the \ipa{-si} derivation removes the P (the stimulus) and changes the A of the base verb into an S. The stimulus is still recoverable, but must be assigned oblique case (the ablative \ipa{-kʌ}).  

\begin{exe}
\ex \label{ex:ghryamt} 
\gll 
  	\ipa{lokpei}  	\ipa{ghrɛ̄md-u.}  \\
leech  be.disgusted.by-\textsc{1sg$\rightarrow$3} \\
 \glt  I am disgusted by leeches. (Khaling)
\end{exe}

\begin{exe}
\ex \label{ex:ghryamtsi} 
\gll \ipa{gʰrɛ̄m-si-ŋʌ}\\
 be.disgusted.by-\textsc{refl-1sg:S/P} \\
\glt  I feel disgust. (Khaling)
\end{exe}

Another example of antipassive in Khaling is the verb \ipa{|mim-si|} `think', derived from \ipa{|mimt|} `think about'.

The same examples are also found in Thulung, where the cognate reflexive verbs \ipa{gʰram-si-} `be disgusted' and \ipa{mim-si-} `think' also have an antipassive reading (\citealt[56]{lahaussois16reflexive}).

In Limbu,  the transitive \ipa{khɛtt-} `chase' has a reflexive form \ipa{khɛt-chiŋ-} whose meaning is `run'; \citet[87]{driem87} points out that the relationship between the reflexive verb and its base verb are not felt by native speakers. Here the patient of the base verb is semantically completely deleted in the reflexive form, unlike what is observed in Khaling.

%\begin{exe}
%\ex 
%\gll \ipa{àng} \ipa{vhø̄-shì-ē} \\
%\textsc{3sg} laugh-\textsc{refl-n.pst} \\
%\glt `He is laughing.'
%\end{exe}
In  Kiranti languages other than Khaling, Thulung and Limbu, no clear example of antipassive use of the reflexive/middle suffix have been found, for instance in Wambule (\citealt[305-6]{opgenort04wambule}), Kulung (\citealt[61-2]{tolsma06kulung}), Yakkha (\citealt[307-9]{schackow15yakkha}) and Chintang (\citealt{schikowski15flexible}). Dumi has one example that could be interpreted as a frozen antipassive:  \ipa{waːt-nsi} `put on jewelry' (\citealt[125-9]{driem93dumi}), which derives from the verb \ipa{waːt} `bear (children)' (which probably formerly also meant `put on (clothes)', as its Limbu cognate \ipa{waːt-} `wear').

\subsection{Dulong-Rawang} \label{sec:rawang}
Dulong and Rawang have cognate reflexive suffixes (respectively \ipa{-ɕɯ̌} and \ipa{-shì}, see \citealt{lapolla04reflexive}). Rawang shows a few examples of the use of the reflexive/middle \ipa{-shì} as an antipassive marker, when applied to transitive experiencer verbs. The transitive construction in (\ref{ex:shvngo}) has agentive marking on the subject, and third person object \ipa{-ò} on the verb, while the reflexive/middle construction in (\ref{ex:shvngoshi}) has the subject in the absolutive and complete deletion of the stimulus, without reflexive, reciprocal or autobenefactive meaning.

\begin{exe}
\ex  \label{ex:shvngo}
\gll \ipa{à:ng-i} \ipa{àng-sv̀ng} \ipa{shvngō-ò-ē} \\
3sg-\textsc{agt} 3sg-\textsc{loc} hate-\textsc{3:tr.n.pst-n.pst} \\
\glt `He hates him.' (\citealt[294]{lapolla01valency}) (Rawang)
\end{exe}

\begin{exe}
\ex  \label{ex:shvngoshi}
\gll  \ipa{àng} \ipa{nø̄} \ipa{shvngō-shì-ē} \\
3sg \textsc{top} hate-\textsc{refl-n.pst} \\
\glt `He's hateful.' (\citealt[294]{lapolla01valency}) (Rawang)
 \end{exe}
 
 \subsection{Kuki-Chin} \label{sec:kc}
While Kuki-Chin languages do not appear to preserve the Reflexive/Middle \ipa{-si} suffix, most languages of this group have a detransitive \ipa{ŋə-} prefix with passive, reciprocal and reflexive functions (see for instance \citealt[203-209]{hartmann09grammar} on Daai Chin). This prefix is related to the \ipa{a-} ($\leftarrow$ \ipa{*ŋa-}) passive/reciprocal prefix in Japhug, the \ipa{ʁ-} passive prefix in Khroskyabs and the \ipa{ŋə-}reciprocal prefix in Tangkhul (\citealt[904-5]{jacques07passif}) and is possibly ultimately  of denominal origin (see \citealt{lai18denom}).

In K'cho, \citet[57]{mang06kcho} describes, in addition to the passive, reflexive and reciprocal functions, an antipassive use of the \ipa{ŋ-} prefix (orthographic \ipa{ng-}) in examples such as (\ref{ex:pyeinci}) and (\ref{ex:ngpyeinci}).

\begin{exe}
\ex  \label{ex:pyeinci}
\gll  \ipa{Páihtiim} \ipa{noh} \ipa{a} \ipa{pó} \ipa{pyéin-ci}. \\
Paihtiim \textsc{erg} \textsc{3sg.poss} friend tell.I-\textsc{non.future} \\
\glt `Paihtiim gossiped about her friend.' (K'cho)
\end{exe}

\begin{exe}
\ex  \label{ex:ngpyeinci}
\gll  \ipa{Páithiim} \ipa{ng-pyéin-ci} \\
Paithiim \textsc{detransitive}-tell.I-\textsc{non.future} \\
\glt `Paithiim gossips.' (K'cho)
\end{exe}

Given the fact that in this language the same prefix also has a productive reflexive and reciprocal functions (\citealt[55-6]{mang06kcho}), it is likely that the antipassive use also derives from them; one could conceive an intermediate reciprocal stage *`gossip about each other', then reinterpreted as meaning `gossip' when used with a singular subject.

\subsection{Old Chinese} \label{sec:oc}
Old Chinese has several examples of the departing tone derivation which can be interpreted as antipassive, as indicated in Table \ref{tab:oc} (data from \citealt[287-288]{downer59}).

\begin{table}[H]
\caption{Antipassive derivation in Old Chinese} \label{tab:oc} \centering
\begin{tabular}{llllll}
\toprule
Base verb & Meaning & Derived verb & Meaning \\
\midrule
\zhc{覺}{kæwk} & `be conscious of' &  \zhc{覺}{kæwH}& `awake' \\
\zhc{知}{ʈje} & `know' &  \zhc{知智}{ʈjeH}& `be wise' \\
\zhc{射}{ʑek} & `shoot at' &  \zhc{射}{ʑæH}& `practise archery' \\
\zhc{勝}{ɕiŋ} & `overcome' &  \zhc{勝}{ɕiŋH}& `be victorious' \\
\bottomrule
\end{tabular}
\end{table}

Unlike the other languages discussed in this paper, Old Chinese lacks person indexation morphology and transitivity marking (or at least no observable trace of it subsists in the material at hand). The transitivity of a verb can only be determined by its ability to take an overt object (since Old Chinese has SVO basic word order except in very specific constructions, the object follows the verb).

As examples of antipassive verbs in Old Chinese, compare for instance the transitive verbs \zhc{射}{ʑek} `shoot at'  and \zhc{知}{ʈje}  `know'  (examples \ref{ex:zyek} and \ref{ex:trje}) with their intransitive equivalents \zhc{射}{ʑæH}  `practice archery' and  \zhc{知}{ʈjeH} `be wise' (examples \ref{ex:zyaeH} and \ref{ex:trjeH}) in the departing tone.\footnote{The readings of the examples are given in Middle Chinese (in an IPA-based version of Baxter's \citeyear{baxter92} transcription) rather than Old Chinese, since Middle Chinese is the earliest stage of Chinese whose phonological system is completely understood. }

\begin{exe}
\ex \label{ex:zyek}
\glt
\glt \zh{祝聃射王中肩,王亦能軍}  %射王 食亦反 中肩 丁仲反
\glll \zh{祝} \zh{聃} \zh{射} \zh{王} \zh{中} \zh{肩} \zh{王} \zh{亦} \zh{能} \zh{軍} \\
\ipa{tɕuk} \ipa{tham} \ipa{ʑek} \ipa{hjwaŋ} \ipa{ʈjuŋH} \ipa{ken} \ipa{hjwaŋ}  \ipa{jek}   \ipa{noŋ} \ipa{kjun}     \\
p.n. p.n. shoot king hit shoulder king also can army \\
\glt `Zhu Dan shot at the king and hit his shoulder, but the king was still able to lead his army.'  (Zhuozhuan, Huan 5) (Old Chinese)
\end{exe}

\begin{exe}
\ex \label{ex:trje}
\glt \zh{秦晉圍鄭,鄭既知亡矣}
\glll \zh{秦} \zh{晉} \zh{圍} \zh{鄭} \zh{鄭} \zh{既} \zh{知} \zh{亡} \zh{矣} \\ 
\ipa{dzin} \ipa{tsinH} \ipa{hjwɨj} \ipa{ɖjeŋH} \ipa{ɖjeŋH} \ipa{kjɨjH}  \ipa{ʈje}  \ipa{mjaŋ} \ipa{hi}  \\
Qin Jin encircle Zheng Zheng already know disappear particle \\
\glt `The country of Zheng is besieged by Qin and Jin, and already knows that it will perish.'  (Zhuozhuan, Xi 30) (Old Chinese)
\end{exe}

\begin{exe}
\ex \label{ex:zyaeH}
\glt \zh{君使士射,不能,則辭以疾;言曰:「某有負薪之憂。」}
\glll \zh{君} \zh{使} \zh{士} \zh{射} \zh{不} \zh{能} \zh{則} \zh{辭} \zh{以} \zh{疾} 
\zh{言} \zh{曰} \zh{某} \zh{有} \zh{負} \zh{薪} \zh{之} \zh{憂} \\
\ipa{kjun} \ipa{ʂiX} \ipa{dʐiX} \ipa{ʑæH} \ipa{pjuwX} \ipa{noŋ}  \ipa{tsok}   \ipa{zi} \ipa{jiX}  \ipa{dzit} \ipa{ŋjon} \ipa{hjwot} \ipa{muwX} \ipa{hjuwX} \ipa{bjuwX} \ipa{sin} \ipa{tɕi} \ipa{ʔjuw}  \\
ruler cause officer practice.archery \textsc{neg} can then decline because ill word say some have carry firewood \textsc{gen} worry\\
\glt `When a ruler wishes an officer to take a place at an archery meeting,
and he is unable to do so, he should decline on the ground of being ill, and say, `I am suffering from carrying firewood.' (Liji, translation by Legge) (Old Chinese)
\end{exe}

\begin{exe}
\ex \label{ex:trjeH}
\glt \zh{失其所與,不知}
\glll \zh{失} \zh{其} \zh{所} \zh{與} \zh{不} \zh{知} \\ %音智
\ipa{ɕit} \ipa{gi} \ipa{ʂjoX} \ipa{joX} \ipa{pjuwX} \ipa{ʈjeH}  \\
lose \textsc{3:poss} \textsc{nmlz:oblique} be.allied \textsc{neg} be.wise \\
\glt `Loosing an ally is not wise.'  (Zhuozhuan, Xi 30) (Old Chinese)
\end{exe}

 The departing tone derivation (which has many other functions, see \citealt{downer59}) is known to originate from an \ipa{*-s} suffix (\citealt{haudricourt54chinois}) For instance, the pair of verbs \zhc{射}{ʑek}  `shoot at'  $\rightarrow$ \zhc{射}{ʑæH} `practise archery'  is reconstructed as  \ipa{*Cə-lAk} $\rightarrow$ \ipa{*Cə-lAk-s} by \citealt{bs14oc})). 
 
 \citet{jacques16ssuffixes} proposes that the diverse functions of the departing tone derivation can be accounted for by assuming that it originates from several unrelated suffixes, and hypothesizes that the antipassive and passive functions of this derivation are remnants of the reflexive/middle \ipa{-si} (described in the previous sections on Kiranti and Dulong-Rawang) in Old Chinese. Even if this historical interpretation is not accepted, the direction of derivation and its meaning are not in doubt.

  \subsection{Other languages} \label{sec:other.refl}
Evidence for the antipassive use of reflexive/middle markers is less clear in other languages of the Sino-Tibetan family.
  
 In Kham and Thangmi, despite the existence of detailed descriptions of the function of the reflexive/detransitive \ipa{-si} suffix, no evidence of antipassive use are found in \citet[105;240-7]{watters02grammar} and \citet[372-6]{turin12thangmi}.
 
 In West Himalayish, we find one example in \citet[452;466]{widmer14bunan} of the reflexive/middle \ipa{-s} suffix: \ipa{broŋ-} `to make fun off' $\rightarrow$ \ipa{broŋ-s-} `to prance' (one of two verbs with \ipa{-s} and simple intransitive, rather than reflexive conjugation).
 
%\ipa{broŋ-s-men} “to prance” < \ipa{broŋ-tɕ-um} “to make fun off”
%%broŋ-s-ɕ-um “to banter”
%\ipa{dziŋ-s-men} “to quarrel” < \ipa{dziŋ-tɕ-um} “to scold”
%%dziŋ-s-ɕ-um “to quarrel”

\section{Conclusion}
This survey has only found evidence for antipassive constructions in a few subgroups of Sino-Tibetan, indicated in Table \ref{tab:summary}; languages with productive antipassive constructions are indicated in bold.

\begin{table}[h]
\caption{Antipassive constructions in ST} \label{tab:summary} \centering
\begin{tabular}{llllll}
\toprule
Branch & Language & Type & Section \\
\midrule
Kiranti & \textbf{Puma} & Indefinite & \ref{sec:kha} \\
&Limbu & Reflexive/Middle & \ref{sec:kiranti.si} \\
&Khaling & Reflexive/Middle & \ref{sec:kiranti.si} \\
&Thulung & Reflexive/Middle & \ref{sec:kiranti.si} \\
Gyalrongic & \textbf{Tshobdun} & Nominalization + verbalization & \ref{sec:ra} \\
  & \textbf{Japhug} & Nominalization + verbalization&  \ref{sec:ra}   \\
  & \textbf{Zbu} & Nominalization + verbalization&  \ref{sec:ra}   \\
  Nungish & Rawang & Reflexive/Middle & \ref{sec:rawang} \\
  West Himalayish & Bunan & Reflexive/Middle  & \ref{sec:other.refl} \\
Kuki-Chin & K'cho  & Reflexive/Reciprocal &\ref{sec:kc}\\
Sinitic & Old Chinese  & Reflexive/Middle? & \ref{sec:oc} \\
\bottomrule
\end{tabular}
\end{table} 

However, few grammars (see \citealt[83]{tournadre96erg}, \citealt[108]{genetti07grammar} for instance) explicitly indicate the \textit{absence} of detransitivizing constructions. It is possible that constructions analyzable as antipassive in other languages of the ST family have been overlooked by the present work.%\footnote{Although \citet{polinsky11antipassive} cites Hakha Lai as having an antipassive, her source \citet[37]{peterson07appl} explicitly says that `Hakha Lai has no valence-affecting constructions which target objects, such as passive or antipassive.'}

Antipassive constructions in ST are all of relatively recent origin. The \ipa{rɤ-} antipassive in Gyalrong is restricted to the three northern Gyalrong languages (Tshobdun, Japhug and Zbu), and probably a local innovation. The \ipa{kʰa-} antipassive in Puma is a language-specific innovation, not even shared with its closest relatives Bantawa and Chamling (within the South Kiranti group). The antipassive uses of the \ipa{-si} reflexive suffixes are always limited and restricted to a few lexicalized examples, and never became productive antipassive constructions. It is also clear that this antipassive use of \ipa{-si} results from parallel development in all the languages that have it, since no cognate antipassive verbs are found between even closely related languages.

Apart from Old Chinese, all languages with antipassive derivations in Sino-Tibetan also have ergative or agentive case marking.

Despite their rarity, antipassive constructions in ST are highly diverse, and exemplify three out of the four main sources of antipassives (\citealt{sanso17antipassive}). The languages having antipassive constructions in ST are not in contact with each other or with languages from other language families with antipassives. The presence of antipassive in the languages studied in this paper cannot be attributed to contact; rather, it is a spontaneous language-internal development.

 

\bibliographystyle{unified}
\bibliography{bibliogj}

 \end{document}
 