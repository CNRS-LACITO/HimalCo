\documentclass[oneside,a4paper,11pt]{article} 
\usepackage{fontspec}
\usepackage{natbib}
\usepackage{booktabs}
\usepackage{xltxtra} 
\usepackage{polyglossia} 
\usepackage[table]{xcolor}
\usepackage{tikz}
\usetikzlibrary{trees}
\usepackage{gb4e} 
\usepackage{multicol}
\usepackage{graphicx}
\usepackage{float}
\usepackage{hyperref} 
\hypersetup{bookmarks=false,bookmarksnumbered,bookmarksopenlevel=5,bookmarksdepth=5,xetex,colorlinks=true,linkcolor=blue,citecolor=blue}
\usepackage[all]{hypcap}
\usepackage{memhfixc}
\usepackage{lscape}
\usepackage{amssymb}
 
%\setmainfont[Mapping=tex-text,Numbers=OldStyle,Ligatures=Common]{Charis SIL} 
\newfontfamily\phon[Mapping=tex-text,Ligatures=Common,Scale=MatchLowercase]{Charis SIL} 
\newcommand{\ipa}[1]{{\phon\textbf{#1}}} 
\newcommand{\grise}[1]{\cellcolor{lightgray}\textbf{#1}}
\newfontfamily\cn[Mapping=tex-text,Ligatures=Common,Scale=MatchUppercase]{SimSun}%pour le chinois
\newcommand{\zh}[1]{{\cn #1}}
\newcommand{\Y}{\Checkmark} 
\newcommand{\N}{} 
\newcommand{\dhatu}[2]{|\ipa{#1}| `#2'}
\newcommand{\jpg}[2]{\ipa{#1} `#2'}  
\newcommand{\refb}[1]{(\ref{#1})}
\newcommand{\tld}{\textasciitilde{}}

 \begin{document} 
\title{Antipassive derivations in Sino-Tibetan/Trans-Himalayan and their origin}
\author{Guillaume Jacques\\ CNRS-CRLAO-INALCO}
\maketitle

\section*{Introduction}
Although the existence of antipassive constructions has been mentioned in several Sino-Tibetan languages (\citealt[225-7]{doornenbal09}, \citealt{jacques14antipassive}, \citealt{bickel15antipassive}), this topic has no yet received as much attention as other voice constructions such as passive or causative.

This paper is a survey of antipassive constructions in the Sino-Tibetan family. Since all of these constructions are historically transparent, they are classified by their diachronic source. Recent work on diachronic typology (\citealt[235]{janic.these},  \citealt{jacques14antipassive}, \citealt{sanso17antipassive}) has shown that antipassive constructions have four major sources in the world's languages:
\begin{itemize}
\item Agent nominalizations. (`he is the hitter' $\rightarrow$  `he hits (intr)')
\item Generic nouns/Indefinite pronouns in object position  (`he hits things/stuff' $\rightarrow$  `he hits (intr)')
\item Action nominalization + light verb (\citealt{creissels12antip}) / denominal verbalizer (\citealt{jacques14antipassive}) (`he does hitting' $\rightarrow$  `he hits (intr)')
\item Reflexive (with an intermediate stage as `co-participation' \citet{creissels08coparticipation}) (`they hit themselves/each other' $\rightarrow$  `they partake in hitting actions' $\rightarrow$  `they hit (intr)')
\end{itemize}

In this paper, I first present a definition of antipassive and discuss related antipassive-like constructions in several languages of the ST family. Then, I provide evidence of antipassive derivations originating from three out of the four main attested sources: action nominalization, generic nouns and reflexives. These derivations are all of recent origin, but some are argued to be reconstructible to lower branches of the family. Finally, I present an overview of the distribution of antipassive construction throughout ST.

\section{Antipassive and indefinite objects}
Since transitivity is overtly (and often redundantly) marked in the morphology-rich languages of the ST family, I propose for this paper the following definition of antipassive specifically tailored to this family:

\begin{exe}
\ex \label{ex:def}
\glt An antipassive construction is an overtly-marked derivation or periphrastic construction which (possibly among other functions) turns a transitive verb into an intransitive one. The agent-like argument of the base verb becomes the sole argument of the intransitive verb, and has the same morphosyntactic properties as an intransitive subject, while the patient-like argument is either deleted or demoted to non-core argument function.
\end{exe}

This definition excludes  (i) agent-preserving lability (since even if one could argue that the intransitive use of the verb is derived from the transitive one, it would be a zero derivation) and (ii) constructions where the verb remains morphologically transitive, or maintaining an obligatory ergative marker on the A. It can be applied to languages without morphological marking of transitivity if explicit criteria to distinguish transitive from intransitive construction are provided, though in the case of the Sino-Tibetan family, antipassive constructions appear to be absent from languages without such marking.

In language with polypersonal indexation and/or obligatory marking of transitivity, non-overt arguments are understood as definite. For instance, a Japhug sentence like (\ref{ex:tAχtWta}), with the transitive verb \ipa{χtɯ} `buy' (note the unambiguous past transitive \ipa{-t-} suffix), can only be interpreted as meaning `I bought it' with a definite (and previously mentioned) object.

\begin{exe}
\ex \label{ex:tAχtWta}
\gll \ipa{tɤ-χtɯ-t-a} \\
\textsc{pfv}-buy-\textsc{tr:pst-1sg} \\
\glt `I bought it.'
\end{exe}

In order to express an indefinite object, it is therefore not an option to simply leave the object position empty. Antipassive, as in (\ref{ex:tAraχtWa}; note the absence of transitive \ipa{-t-} suffix), is one way to express indefiniteness. 

\begin{exe}
\ex \label{ex:tAraχtWa}
\gll \ipa{tɤ-ra-χtɯ-a} \\
\textsc{pfv}-\textsc{antipass}-buy-\textsc{1sg} \\
\glt `I bought things.'
\end{exe}

Other strategies are however possible; in this section, I present three competing types of construction to express indefinite objects in ST, which should not be confused with antipassive: lability, indefinite objects and incorporation.  

\subsection{Agent-preserving lability} \label{sec:labile}
ST languages with polypersonal indexation all present some degree of lability, ie constructions where the same verb root can be conjugated either transitively or intransitively, with effect on case marking on the arguments. The intransitive use of the verb can be patient-preserving (the sole argument of the construction corresponding to the patient-like argument of the transitive construction), or agent-preserving (when it corresponds to the agent-like argument).  Limbu can be used to illustrate these constructions, which are attested with a few verbs such as  \ipa{kʰutt} `steal' (\citealt[527]{driem91tangut}), which can be conjugated transitively (\ref{ex:khuttang}) or intransitively with preservation of the patient (\ref{ex:sapla}) or the agent (\ref{ex:andzumin}).

\begin{exe}
\ex \label{ex:khuttang}
\gll \ipa{A-ndzum-ille}	\ipa{sapla}	\ipa{khutt-aŋ} \\
\textsc{1sg.poss}-friend-\textsc{erg} book steal-\textsc{1sg.P.pst} \\
\glt `My friend robbed me of my book.'
\end{exe}

\begin{exe}
\ex \label{ex:sapla}
\gll 
\ipa{Sapla}	\ipa{khutt-ɛ} \\
book steal-\textsc{pst:intr} \\
\glt `The book was stolen.'
\end{exe}

\begin{exe}
\ex \label{ex:andzumin}
\gll 
\ipa{A-ndzum-in}	\ipa{khutt-ɛ} \\
\textsc{1sg.poss}-friend-\textsc{def} steal-\textsc{pst:intr} \\
\glt `My friend committed a theft.'
\end{exe}
In addition to effect on verbal morphology and person indexation, lability also affects case marking: thus, in the case of agent-preserving lability, the agent-like argument receives ergative case in the transitive construction (\ref{ex:khuttang}), and absolutive case in the intransitive one (\ref{ex:andzumin}). Not all ST languages allow both types of lability; in Japhug, only agent-preserving lability is attested (\citealt[218]{jacques12demotion}).

 While some scholars such as \citet[359]{schackow15yakkha} use the term `antipassive' to refer to patient-preserving lability, in the more restricted definition proposed in (\ref{ex:def}), a detransitivizing construction without overt mark cannot be referred to as antipassive.

Agent preserving lability is a marginal phenomenon in languages such as Limbu or Japhug (where it concerns a restricted list of verb, see \citealt[218]{jacques12demotion}), but it is quite productive and prominent in some Kiranti languages, such as Puma (the $\varnothing$-detransitive construction described in \citealt[9]{bickel07puma}).
 
 
 

\subsection{Indefinite objects} \label{sec:indef}
In Bantawa, \citet[226;335]{doornenbal09} refers to constructions such as (\ref{ex:hityang}), with a verb conjugated intransitively (\ipa{hɨtt} `burn'), the agent marked with the ergative and the indefinite \ipa{kʰa} `something' as its object as an antipassive construction.

\begin{exe}
\ex \label{ex:hityang}
\gll 
\ipa{nam-ʔa} \ipa{kʰa} \ipa{hɨt-yaŋ} \\
sun-\textsc{erg} something scorch-\textsc{prog} \\
\glt ‘The sun is scorching.’
\end{exe}

While this construction is certainly the source for the antipassive constructions found in Puma or Chintang (see section \ref{sec:kha}), the presence of the ergative on the agent-like argument precludes from treating it as a full-fledge antipassive in the sense given in (\ref{ex:def}) above.


\subsection{Incorporation} \label{sec:incorp}

\ipa{phɯt} `cut, chop' \ipa{si} `wood, tree'
\ipa{ɣɯ-sɯ-phɯt} `chop firewood'
 \citet{jacques12incorp}
 
 incorporated noun, type of indexation
 
\section{Action nominal + verbalizer} \label{sec:ra}
\citet{creissels12antip}
\citet{jacques14antipassive}

\section{Generic noun} \label{sec:kha}
\citet[9]{bickel07puma}
\citet{bickel15antipassive}

\section{Reflexive}
\citet{janic.these}
\citet{jacques16si}
%\citet[290]{lapolla01valency}
%\begin{exe}
%\ex 
%\gll \ipa{àng} \ipa{vhø̄-shì-ē} \\
%\textsc{3sg} laugh-\textsc{refl-n.pst} \\
%\glt `He is laughing.'
%\end{exe}
\citet[294]{lapolla01valency}
\begin{exe}
\ex 
\gll \ipa{à:ngi} \ipa{àngsv̀ng} \ipa{shvngō-ò-ē} \\
3sg-\textsc{agt} 3sg-\textsc{loc} hate-\textsc{3:tr.n.pst-n.pst} \\
\glt `He hates him'
\end{exe}
\begin{exe}
\ex 
\gll  \ipa{àng} \ipa{nø̄} \ipa{shvngō-shì-ē} \\
3sg \textsc{top} hate-\textsc{refl-n.pst} \\
\glt `He's hateful'
 \end{exe}
\citet[452;466]{widmer14bunan}
\ipa{broŋ-s-men} “to prance” < \ipa{broŋ-tɕ-um} “to make fun off”
%broŋ-s-ɕ-um “to banter”
\ipa{dziŋ-s-men} “to quarrel” < \ipa{dziŋ-tɕ-um} “to scold”
%dziŋ-s-ɕ-um “to quarrel”

\section{Distribution of antipassive morphology in Sino-Tibetan}
Although \citet{polinsky11antipassive} cites Hakha Lai as having an antipassive, her source \citet[37]{peterson07appl} explicitly says that `Hakha Lai has no valence-affecting constructions which target objects, such as passive or antipassive.'


\section*{Conclusion}

\bibliographystyle{unified}
\bibliography{bibliogj}

 \end{document}
 