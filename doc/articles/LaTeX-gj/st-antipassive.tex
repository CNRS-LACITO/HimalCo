\documentclass[oneside,a4paper,11pt]{article} 
\usepackage{fontspec}
\usepackage{natbib}
\usepackage{booktabs}
\usepackage{xltxtra} 
\usepackage{polyglossia} 
\usepackage[table]{xcolor}
\usepackage{tikz}
\usetikzlibrary{trees}
\usepackage{gb4e} 
\usepackage{multicol}
\usepackage{graphicx}
\usepackage{float}
\usepackage{hyperref} 
\hypersetup{bookmarks=false,bookmarksnumbered,bookmarksopenlevel=5,bookmarksdepth=5,xetex,colorlinks=true,linkcolor=blue,citecolor=blue}
\usepackage[all]{hypcap}
\usepackage{memhfixc}
\usepackage{lscape}
\usepackage{bbding}
 
%\setmainfont[Mapping=tex-text,Numbers=OldStyle,Ligatures=Common]{Charis SIL} 
\newfontfamily\phon[Mapping=tex-text,Ligatures=Common,Scale=MatchLowercase]{Charis SIL} 
\newcommand{\ipa}[1]{{\phon\textbf{#1}}} 
\newcommand{\grise}[1]{\cellcolor{lightgray}\textbf{#1}}
\newfontfamily\cn[Mapping=tex-text,Ligatures=Common,Scale=MatchUppercase]{SimSun}%pour le chinois
\newcommand{\zh}[1]{{\cn #1}}
\newcommand{\Y}{\Checkmark} 
\newcommand{\N}{} 
\newcommand{\dhatu}[2]{|\ipa{#1}| `#2'}
\newcommand{\jpg}[2]{\ipa{#1} `#2'}  
\newcommand{\refb}[1]{(\ref{#1})}
\newcommand{\tld}{\textasciitilde{}}

 \begin{document} 
\title{Antipassive derivations in Sino-Tibetan/Trans-Himalayan and their origin}
\author{Guillaume Jacques\\ CNRS-CRLAO-INALCO}
\maketitle

\section*{Introduction}
Although the existence of antipassive constructions has been mentioned in several Sino-Tibetan languages (\citealt[225-7]{doornenbal09}, \citealt{jacques14antipassive}, \citealt{bickel15antipassive}), this topic has no yet received as much attention as other voice constructions such as passive or causative.

This paper is a survey of antipassive constructions in the Sino-Tibetan family. Since all of these constructions are historically transparent, they are classified by their diachronic source. Recent work on diachronic typology (\citealt[235]{janic.these},  \citealt{jacques14antipassive}, \citealt{sanso17antipassive}) has shown that antipassive constructions have four major sources in the world's languages:
\begin{itemize}
\item Agent nominalizations. (`he is the hitter' $\rightarrow$  `he hits (intr)')
\item Generic nouns/Indefinite pronouns in object position  (`he hits things/stuff' $\rightarrow$  `he hits (intr)')
\item Action nominalization + light verb (\citealt{creissels12antip}) / denominal verbalizer (\citealt{jacques14antipassive}) (`he does hitting' $\rightarrow$  `he hits (intr)')
\item Reflexive (with an intermediate stage as `co-participation' \citet{creissels08coparticipation}) (`they hit themselves/each other' $\rightarrow$  `they partake in hitting actions' $\rightarrow$  `they hit (intr)')
\end{itemize}

In this paper, I first present a definition of antipassive and discuss related antipassive-like constructions in several languages of the ST family. Then, I provide evidence of antipassive derivations originating from three out of the four main attested sources: action nominalization, generic nouns and reflexives. These derivations are all of recent origin, but some are argued to be reconstructible to lower branches of the family.

\section{Antipassive}
Since transitivity is overtly marked in the morphology-rich languages of the ST family, I propose for this paper the following definition of antipassive specifically tailored to this family:

\begin{exe}
\ex 
\glt An antipassive construction is an overtly-marked derivation or periphrastic construction which turns a transitive verb into an intransitive one, with the only argument of the intransitive verb corresponding to the agent-like argument of the base verb.
\end{exe}

This definition excludes  (i) agent-preserving lability (since even if one could argue that the intransitive use of the verb is derived from the transitive one, it would be a zero derivation) and (ii) constructions where the verb remains morphologically transitive, or maintaining an obligatory ergative marker on the A. 

Examples of these constructions are discussed below, followed by an overview of the distribution of antipassive constructions in the ST family.

\subsection{Agent-preserving lability}

ST languages with polypersonal indexation all present some 
\citet{bickel07puma}
\citet[527]{driem91tangut}
\ipa{kʰutt} `steal'
\citet{jacques12demotion}

\subsection{Indefinite objects}
\citet[335]{doornenbal09}

\subsection{Distribution of antipassive morphology}
Although \citet{polinsky11antipassive} cites Hakha Lai as having an antipassive, her source \citet[37]{peterson07appl} explicitly says that `Hakha Lai has no valence-affecting constructions which target objects, such as passive or antipassive.'

\section{Action nominal + verbalizer}
\citet{creissels12antip}
\citet{jacques14antipassive}

\section{Generic noun}

\citet{bickel15antipassive}

\section{Reflexive}
\citet{janic.these}
\citet{jacques16si}
%\citet[290]{lapolla01valency}
%\begin{exe}
%\ex 
%\gll \ipa{àng} \ipa{vhø̄-shì-ē} \\
%\textsc{3sg} laugh-\textsc{refl-n.pst} \\
%\glt `He is laughing.'
%\end{exe}
\citet[294]{lapolla01valency}
\begin{exe}
\ex 
\gll \ipa{à:ngi} \ipa{àngsv̀ng} \ipa{shvngō-ò-ē} \\
3sg-\textsc{agt} 3sg-\textsc{loc} hate-\textsc{3:tr.n.pst-n.pst} \\
\glt `He hates him'
\end{exe}
\begin{exe}
\ex 
\gll  \ipa{àng} \ipa{nø̄} \ipa{shvngō-shì-ē} \\
3sg \textsc{top} hate-\textsc{refl-n.pst} \\
\glt `He's hateful'
 \end{exe}
\citet[452;466]{widmer14bunan}
\ipa{broŋ-s-men} “to prance” < \ipa{broŋ-tɕ-um} “to make fun off”
%broŋ-s-ɕ-um “to banter”
\ipa{dziŋ-s-men} “to quarrel” < \ipa{dziŋ-tɕ-um} “to scold”
%dziŋ-s-ɕ-um “to quarrel”
\section*{Conclusion}

\bibliographystyle{unified}
\bibliography{bibliogj}

 \end{document}
 