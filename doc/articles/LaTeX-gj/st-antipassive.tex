\documentclass[oneside,a4paper,11pt]{article} 
\usepackage{fontspec}
\usepackage{natbib}
\usepackage{booktabs}
\usepackage{xltxtra} 
\usepackage{polyglossia} 
\usepackage[table]{xcolor}
\usepackage{tikz}
\usetikzlibrary{trees}
\usepackage{gb4e} 
\usepackage{multicol}
\usepackage{graphicx}
\usepackage{float}
\usepackage{hyperref} 
\hypersetup{bookmarks=false,bookmarksnumbered,bookmarksopenlevel=5,bookmarksdepth=5,xetex,colorlinks=true,linkcolor=blue,citecolor=blue}
\usepackage[all]{hypcap}
\usepackage{memhfixc}
\usepackage{lscape}
\usepackage{bbding}
 
%\setmainfont[Mapping=tex-text,Numbers=OldStyle,Ligatures=Common]{Charis SIL} 
\newfontfamily\phon[Mapping=tex-text,Ligatures=Common,Scale=MatchLowercase]{Charis SIL} 
\newcommand{\ipa}[1]{{\phon\textbf{#1}}} 
\newcommand{\grise}[1]{\cellcolor{lightgray}\textbf{#1}}
\newfontfamily\cn[Mapping=tex-text,Ligatures=Common,Scale=MatchUppercase]{SimSun}%pour le chinois
\newcommand{\zh}[1]{{\cn #1}}
\newcommand{\Y}{\Checkmark} 
\newcommand{\N}{} 
\newcommand{\dhatu}[2]{|\ipa{#1}| `#2'}
\newcommand{\jpg}[2]{\ipa{#1} `#2'}  
\newcommand{\refb}[1]{(\ref{#1})}
\newcommand{\tld}{\textasciitilde{}}

 \begin{document} 
\title{Antipassive derivations in Sino-Tibetan/Trans-Himalayan}
\author{Guillaume Jacques\\ CNRS-CRLAO-INALCO}
\maketitle

\section*{Introduction}

\citet{sanso17antipassive}
Three out of the four attestion origins of antipassive markers

\section{Lability}
\citet{bickel07puma}
\citet{jacques12demotion}

\section{Denominal derivation of action nominal}
\citet{creissels12antip}
\citet{jacques14antipassive}

\section{Noun}

\citet{bickel15antipassive}

\section{Reflexive}
\citet{janic.these}
\citet{jacques16si}

\citet[452;466]{widmer14bunan}
broŋ-s-men “to prance” < broŋ-tɕ-um “to make fun off”
broŋ-s-ɕ-um “to banter”
dziŋ-s-men “to quarrel” < dziŋ-tɕ-um “to scold”
dziŋ-s-ɕ-um “to quarrel”
\section*{Conclusion}

\bibliographystyle{unified}
\bibliography{bibliogj}

 \end{document}
 