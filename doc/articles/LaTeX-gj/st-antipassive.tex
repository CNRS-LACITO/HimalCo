\documentclass[oneside,a4paper,11pt]{article} 
\usepackage{fontspec}
\usepackage{natbib}
\usepackage{booktabs}
\usepackage{xltxtra} 
\usepackage{polyglossia} 
\usepackage[table]{xcolor}
\usepackage{tikz}
\usetikzlibrary{trees}
\usepackage{gb4e} 
\usepackage{multicol}
\usepackage{graphicx}
\usepackage{float}
\usepackage{hyperref} 
\hypersetup{bookmarks=false,bookmarksnumbered,bookmarksopenlevel=5,bookmarksdepth=5,xetex,colorlinks=true,linkcolor=blue,citecolor=blue}
\usepackage[all]{hypcap}
\usepackage{memhfixc}
\usepackage{lscape}
\usepackage{amssymb}
 
%\setmainfont[Mapping=tex-text,Numbers=OldStyle,Ligatures=Common]{Charis SIL} 
\newfontfamily\phon[Mapping=tex-text,Ligatures=Common,Scale=MatchLowercase]{Charis SIL} 
\newcommand{\ipa}[1]{{\phon\textbf{#1}}} 
\newcommand{\grise}[1]{\cellcolor{lightgray}\textbf{#1}}
\newfontfamily\cn[Mapping=tex-text,Ligatures=Common,Scale=MatchUppercase]{SimSun}%pour le chinois
\newcommand{\zh}[1]{{\cn #1}}
\newcommand{\Y}{\Checkmark} 
\newcommand{\N}{} 
\newcommand{\dhatu}[2]{|\ipa{#1}| `#2'}
\newcommand{\jpg}[2]{\ipa{#1} `#2'}  
\newcommand{\refb}[1]{(\ref{#1})}
\newcommand{\tld}{\textasciitilde{}}

 \begin{document} 
\title{Antipassive derivations in Sino-Tibetan/Trans-Himalayan and their origin}
\author{Guillaume Jacques\\ CNRS-CRLAO-INALCO}
\maketitle

\section*{Introduction}
Although the existence of antipassive constructions has been mentioned in several Sino-Tibetan languages (\citealt[225-7]{doornenbal09}, \citealt{jacques14antipassive}, \citealt{bickel15antipassive}), this topic has no yet received as much attention as other voice constructions such as passive or causative.

This paper is a survey of antipassive constructions in the Sino-Tibetan family. Since all of these constructions are historically transparent, they are classified by their diachronic source. Recent work on diachronic typology (\citealt[235]{janic.these},  \citealt{jacques14antipassive}, \citealt{sanso17antipassive}) has shown that antipassive constructions have four major sources in the world's languages:
\begin{itemize}
\item Agent nominalizations. (`he is the hitter' $\rightarrow$  `he hits (intr)')
\item Generic nouns/Indefinite pronouns in object position  (`he hits things/stuff' $\rightarrow$  `he hits (intr)')
\item Action nominalization + light verb (\citealt{creissels12antip}) / denominal verbalizer (\citealt{jacques14antipassive}) (`he does hitting' $\rightarrow$  `he hits (intr)')
\item Reflexive (with an intermediate stage as `co-participation' \citet{creissels08coparticipation}) (`they hit themselves/each other' $\rightarrow$  `they partake in hitting actions' $\rightarrow$  `they hit (intr)')
\end{itemize}

In this paper, I first present a definition of antipassive and discuss related antipassive-like constructions in several languages of the ST family. Then, I provide evidence of antipassive derivations originating from three out of the four main attested sources: action nominalization, generic nouns and reflexives. These derivations are all of recent origin, but some are argued to be reconstructible to lower branches of the family. Finally, I present an overview of the distribution of antipassive construction throughout ST.

\section{Antipassive and indefinite objects}
Since transitivity is overtly (and often redundantly) marked in the morphology-rich languages of the ST family, I propose for this paper the following definition of antipassive:

\begin{exe}
\ex \label{ex:def}
\glt An antipassive construction is an overtly-marked flexion, derivation or periphrastic construction which (possibly among other functions) turns a transitive verb into an intransitive one. The agent-like argument of the base verb becomes the sole core argument of the intransitive verb, and has the same morphosyntactic properties as the sole arguments of underived intransitive verbs, while the patient-like argument is either deleted or demoted to non-core argument function.
\end{exe}

This definition excludes  (i) agent-preserving lability (since even if one could argue that the intransitive use of the verb is derived from the transitive one, it would be a zero derivation) and (ii) constructions where the verb remains morphologically transitive, or maintaining an obligatory ergative marker on the A. It can be applied to languages without morphological marking of transitivity if explicit criteria to distinguish transitive from intransitive construction are provided, though in the case of the Sino-Tibetan family, antipassive constructions appear to be absent from languages without such marking.

In language with polypersonal indexation and/or obligatory marking of transitivity, non-overt arguments are understood as definite. For instance, a Japhug sentence like (\ref{ex:tAχtWta}), with the transitive verb \ipa{χtɯ} `buy' (note the unambiguous past transitive \ipa{-t-} suffix), can only be interpreted as meaning `I bought it' with a definite (and previously mentioned) object.

\begin{exe}
\ex \label{ex:tAχtWta}
\gll \ipa{tɤ-χtɯ-t-a} \\
\textsc{pfv}-buy-\textsc{tr:pst-1sg} \\
\glt `I bought it.'
\end{exe}

In order to express an indefinite object, it is therefore not an option to simply leave the object position empty. Antipassive, as in (\ref{ex:tAraχtWa}; note the absence of transitive \ipa{-t-} suffix), is one way to express indefiniteness. 

\begin{exe}
\ex \label{ex:tAraχtWa}
\gll \ipa{tɤ-ra-χtɯ-a} \\
\textsc{pfv}-\textsc{antipass}-buy-\textsc{1sg} \\
\glt `I bought things.'
\end{exe}

Other strategies are however possible; in this section, I present three competing types of construction to express indefinite objects in ST, which should not be confused with antipassive: lability, indefinite objects and incorporation.  

\subsection{Agent-preserving lability} \label{sec:labile}
ST languages with polypersonal indexation all present some degree of lability, ie constructions where the same verb root can be conjugated either transitively or intransitively, with effect on case marking on the arguments. The intransitive use of the verb can be patient-preserving (the sole argument of the construction corresponding to the patient-like argument of the transitive construction), or agent-preserving (when it corresponds to the agent-like argument).  Limbu can be used to illustrate these constructions, which are attested with a few verbs such as  \ipa{kʰutt} `steal' (\citealt[527]{driem91tangut}), which can be conjugated transitively (\ref{ex:khuttang}) or intransitively with preservation of the patient (\ref{ex:sapla}) or the agent (\ref{ex:andzumin}).

\begin{exe}
\ex \label{ex:khuttang}
\gll \ipa{A-ndzum-ille}	\ipa{sapla}	\ipa{khutt-aŋ} \\
\textsc{1sg.poss}-friend-\textsc{erg} book steal-\textsc{1sg.P.pst} \\
\glt `My friend robbed me of my book.'
\end{exe}

\begin{exe}
\ex \label{ex:sapla}
\gll 
\ipa{Sapla}	\ipa{khutt-ɛ} \\
book steal-\textsc{pst:intr} \\
\glt `The book was stolen.'
\end{exe}

\begin{exe}
\ex \label{ex:andzumin}
\gll 
\ipa{A-ndzum-in}	\ipa{khutt-ɛ} \\
\textsc{1sg.poss}-friend-\textsc{def} steal-\textsc{pst:intr} \\
\glt `My friend committed a theft.'
\end{exe}
In addition to effect on verbal morphology and person indexation, lability also affects case marking: thus, in the case of agent-preserving lability, the agent-like argument receives ergative case in the transitive construction (\ref{ex:khuttang}), and absolutive case in the intransitive one (\ref{ex:andzumin}). Not all ST languages allow both types of lability; in Japhug, only agent-preserving lability is attested (\citealt[218]{jacques12demotion}).

 While some scholars such as \citet[359]{schackow15yakkha} use the term `antipassive' to refer to patient-preserving lability, in the more restricted definition proposed in (\ref{ex:def}), a detransitivizing construction without overt mark cannot be referred to as antipassive.

Agent preserving lability is a marginal phenomenon in languages such as Limbu or Japhug (where it concerns a restricted list of verb, see \citealt[218]{jacques12demotion}), but it is quite productive and prominent in some Kiranti languages, such as Puma (the $\varnothing$-detransitive construction described in \citealt[9]{bickel07puma}; see \citealt{bickel11multivariate} for an examination of the various potential analyses of this construction).
 
\subsection{Indefinite objects} \label{sec:indef}
In Bantawa, \citet[226;335]{doornenbal09} refers to constructions such as (\ref{ex:hityang}) as `explicit antipassive'. In this construction, the verb conjugated intransitively (\ipa{hɨtt} `burn'), the agent-like argument is marked with the ergative and indexed one the verb with the same marking as an intransitive subject, and the indefinite \ipa{kʰa} `something' is obligatorily present in object position.

\begin{exe}
\ex \label{ex:hityang}
\gll 
\ipa{nam-ʔa} \ipa{kʰa} \ipa{hɨt-yaŋ} \\
sun-\textsc{erg} something scorch-\textsc{prog} \\
\glt ‘The sun is scorching.’
\end{exe}

While this construction is certainly the source for the antipassive constructions found in Puma or Chintang (see section \ref{sec:kha}), the presence of the ergative on the agent-like argument precludes from treating it as a canonical antipassive in the sense given in (\ref{ex:def}) above.

\subsection{Noun incorporation} \label{sec:incorp}
Noun incorporation can affect verbal transitivity. We commonly find examples of incorporation in which a transitive verb becomes intransitive, and the incorporated noun corresponds to the patient-like argument of the base verb and saturates its place in the argument structure.

In Japhug for instance, the intransitive incorporating verb \ipa{ɣɯ-sɯ-pʰɯt} `chop firewood' derives from the transitive verb \ipa{pʰɯt} `cut, chop' and the noun  \ipa{si} `wood, tree' (incorporated in \textit{status constructus} form \ipa{sɯ-} with the denominal prefix \ipa{ɣɯ-}, see  \citealt{jacques12incorp}). 

Constructions of this type have been referred to as antipassive (\citealt[47-48]{say09antipassive}) and indeed fulfil the definition proposed in (\ref{ex:def}). However, a full examination of this antipassive-like incorporation in ST is not possible until a survey of incorporation in the family has been undertaken, and has therefore to be deferred to future research.

 
\section{Incorporation of generic noun / indefinite element} \label{sec:kha}
Puma has an antipassive \ipa{kʰa-} prefix whose function can be illustrated by examples (\ref{ex:ennung}) and (\ref{ex:khaennga}) taken from \citet[7-9]{bickel07puma}). The base verb \ipa{enn-} `hear' in (\ref{ex:ennung}) is transitive; it indexes both subject and object, and the subject takes the ergative suffix \ipa{-a}.

\begin{exe}
\ex \label{ex:ennung}
\gll 
\ipa{ŋa-a} \ipa{kho-lai} \ipa{enn-u-ŋ} \\
\textsc{1sg-erg} \textsc{3sg-dat} hear.\textsc{n.pst}-\textsc{3sg:P-1sg:A} \\
\glt `I hear him/her.'
\end{exe}

The corresponding form with prefixed \ipa{kʰa-} in (\ref{ex:khaennga}) is morphologically intransitive, only indexes one argument, and the only argument (\textsc{1sg}) is in the absolutive. 

\begin{exe}
\ex \label{ex:khaennga}
\gll 
\ipa{ŋa} \ipa{kʰa-en-ŋa} \\
\textsc{1sg} \textsc{antipass}-hear-\textsc{1sg}:S/P \\
\glt `I hear someone/people.' (not `I hear something')
\end{exe}
 
The demoted object argument cannot be relativized (\citealt[10]{bickel07puma}), while the subject presents all the properties of a intransitive subject; this construction unambiguously fulfils all criteria of a canonical antipassive (\ref{ex:def}). 

A particularity of the Puma antipassive is that the demoted object can only refer to humans; to refer to indefinite non-human, a labile construction (the $\varnothing$-detransitive) is used instead.

The Puma antipassive prefix \ipa{kʰa-} is obviously related to the `antipassive' construction (\citealt[226;335]{doornenbal09}) with the indefinite \ipa{kʰa} `something' mentioned in section \ref{sec:indef}. The Bantawa and the Puma constructions differ in several regards:
\begin{itemize}
\item In Bantawa the agent-like argument is marked with the ergative (resulting in a mismatch between case marking and indexation, since the subject is indexed as the sole argument of an intransitive verb), while in Puma it is in the absolutive.
\item In Puma, the demoted object is necessarily interpreted as human, while no such constraint exists in Bantawa.
\item The element \ipa{kʰa} is phonologically less integrated into the verbal word in Bantawa than in Puma.
\end{itemize}

The etymology of the indefinite element \ipa{kʰa} still deserves additional discussion (\citealt[67]{bickel15antipassive} argue that the Puma antipassive is related to the etymon reflected as Khaling \ipa{kʰɵle} `all', proto-Kiranti \ipa{*kʰɑle} in Jacques' \citeyear{jacques17pkiranti} system). In any case, within South Kiranti (the branch to which Bantawa and Puma belong), the following stages can be postulated:

\begin{enumerate}
\item X-\textsc{erg} {   } \textsc{indefinite}:\textsc{abs} {   } V:X$\rightarrow$\textsc{3sg} (fully transitive construction)
\item X-\textsc{erg} {   } \textsc{indefinite}:\textsc{abs}=V:X:\textsc{intr} (Bantawa)
\item X:\textsc{abs} \textsc{antipass}-V:X\textsc{intr} (Puma)
\end{enumerate}

While a canonical antipassive in \ipa{kʰa-} is only attested in Puma, \citet{bickel15antipassive} point out that the first inclusive object marker \ipa{kʰa-} in Chamling and Western Chintang is historically related, and that an intermediate stage as an antipassive could be postulated. The Western Chintang 2/3$\rightarrow$\textsc{1n.sg} forms are in particular exactly identical the corresponding second or third person intransitive forms with the addition of the \ipa{kʰa-} prefix. Since however no mention is made of a constraint against ergative marking on the subject with these verb forms, its is likely that a Bantawa-like construction (stage 2) rather than a full-grown antipassive as in Puma  has to be postulated.

 
\section{Action nominal + denominal verbalizer} \label{sec:ra}
The Northern Gyalrong languages, Tshobdun (\citealt{jackson06paisheng, jackson14morpho} ), Japhug  (\citealt{, jacques12demotion, jacques14antipassive}) and Zbu, have a pair of antipassive prefixes \ipa{rɐ-} and \ipa{sɐ-} (in Tshobdun) and \ipa{rɤ-/ra-} and \ipa{sɤ-/sa-} (in Japhug), respectively used to indicate non-human and human indefinite patient. No cognate antipassive prefixes have been reported in the closely related languages Situ (\citealt[98]{zhang16bragdbar}), Khroskyabs (\citealt{lai13affixale}) and Stau (\citealt{jacques17stau}), and they could be a northern Gyalrong innovation.

The following examples illustrate the use of the antipassive prefix \ipa{rɤ-} in Japhug; the base verb \ipa{tʂɯβ} `sew' requires the subject to take the ergative \ipa{kɯ}, and has to take the transitive progressive prefix \ipa{asɯ-/ɤsɯ-} to be used in inferential imperfective form (\ref{ex:pjAkAsWtsxWBci}) (see \citealt{jacques17sketch} on this restriction), while the derived intransitive verb  \ipa{rɤ-tʂɯβ} `sew things; do sewing' cannot take an overt patient, does not select the ergative on the subject and cannot take the progressive prefix \ipa{asɯ-/ɤsɯ-}.\footnote{Note that in the text corpus at my disposal, antipassive verb forms are mainly attested in either imperfective finite forms or nominalized forms. Although perfective forms of these verbs can be elicited (see example \ref{ex:tAraχtWa} above), they are not commonly employed (on the interaction of antipassivization and aspect, see in particular \citealt{cooreman94antipassive}).} 

%\begin{exe}
%\ex \label{ex:chAtsxWB}
%\gll 
%\ipa{stu}	\ipa{kɯ-xtɕi}	\ipa{nɯ}	\ipa{kɯ}	\ipa{nɯŋa}	\ipa{ɣɯ}	\ipa{ɯ-ndʐi}	\ipa{nɯnɯ}	\ipa{cʰɤ-pɣaʁ}	\ipa{tɕendɤre}	\ipa{cʰɤ-tʂɯβ,} \\
%most \textsc{nmlz:S/A}-be.small \textsc{dem} \textsc{erg} cow \textsc{gen} \textsc{3sg.poss}-skin \textsc{dem} \textsc{ifr}-turn \textsc{ifr}-sew \\
%\glt `The youngest (brother) turned the cow's hide over and sewed it.' (hist-02-deluge2012, 24-5)
%\end{exe}

\begin{exe}
\ex \label{ex:pjAkAsWtsxWBci}
\gll 
\ipa{rgɤnmɯ}	\ipa{nɯ}	\ipa{kɯ}	\ipa{li}	\ipa{iɕqʰa}	<yuwang>	\ipa{nɯ}	\ipa{pjɤ-k-ɤsɯ-tʂɯβ-ci} \\
old.woman \textsc{dem} \textsc{erg} again the.aforementioned fish.net \textsc{dem} \textsc{ifr.ipfv-evd-prog}-sew-\textsc{evd} \\
\glt `The old woman was sewing the fish nets.' (hist140430 yufu he tade qizi, 297)
\end{exe}

\begin{exe}
\ex \label{ex:pjArAtsxWB}
\gll \ipa{iɕqʰa}	\ipa{kɯ-rɤ-tsɯβ}	\ipa{nɯ}	\ipa{pɤjkʰu}	\ipa{pjɤ-rɤ-tʂɯβ}	\ipa{ɕti.} \\
the.aforementioned \textsc{nmlz:S/A-antipass}-sew  \textsc{dem} already  \textsc{ifr:ipfv-antipass}-sew be:\textsc{affirmative:fact} \\
\glt `(Very early in the morning), the tailor was already sewing.' (hist140512 alibaba, 151)
\end{exe}

\citet{jacques14antipassive} accounts for the \ipa{rɤ-} prefix as originating from the reanalysis of the intransitive denominal \ipa{rɤ-/rɯ-} prefix. This reanalysis took place in two steps. 

First, an action or patient nominal is derived from the intransitive verb (for instance, \ipa{ɕpʰɤt} `patch (transitive)' $\rightarrow$  \ipa{tɤ-ɕpʰɤt} `a patch (noun)'). Such nominals take either a nominalization \ipa{tɯ-} prefix or combine the bare verb root with a possessive prefix (which can be either a definite possessive such as \ipa{ɯ-} `his/her/its' or an indefinite possessor \ipa{tɤ-/ta-} as in the example `patch' above). This nominalization neutralizes the valency of the base verb.

Second, this nominal undergoes denominal verbalizing derivation by means of the prefix \ipa{rɤ-}. The possessive or nominalization prefixes are removed during this derivation, as is the case with nouns that are not derived from verbs, as in the following examples:
\begin{itemize}
\item \ipa{ta-ma} `work (noun)'\footnote{The prefix \ipa{ta-} here is the indefinite possessor prefix, as it is an inalienably possessed noun, which requires a possessive prefix. }  $\rightarrow$  \ipa{rɤ-ma} `work (intransitive)'
\item \ipa{tɯ-krɤz} `discussion'\footnote{The prefixal element \ipa{tɯ-} is here synchronically unanalyzable, but could be a fossilized indefinite possessor. The root \ipa{-krɤz} is borrowed from the Tibetan noun \ipa{gros} `discussion'. } $\rightarrow$ \ipa{rɤ-krɤz} `discuss (intransitive)'
\end{itemize}

The second stage of the derivation \ipa{tɤ-ɕpʰɤt} `a patch (noun)' $\rightarrow$ \ipa{rɤ-ɕpʰɤt} `patch, do patching (intransitive)') is thus still transparent; \ipa{rɤ-ɕpʰɤt} is synchronically ambiguous between a denominal derivation from the noun `patch' and an antipassive derivation of the base verb  patch `transitive'. The intermediate noun is however not clearly attested for all verbs, and the antipassive \ipa{rɤ-} is synchronically a distinct morpheme from the denominal \ipa{rɤ-}.

Note that the antipassive is not isolated among voice derivations in Gyalrong languages to originate from a denominal prefix; the same source has been proposed for causative, applicative and passive prefixes (see \citealt{jacques15causative}, \citealt{lai18denom}).


\section{Reflexive}
One of the most common sources of antipassive constructions, in particular in languages with accusatively aligned case marking, are reflexive/middle markers (\citealt{janic16antipassif}). There is some diffuse evidence for antipassive-like use of reflexives in some ST languages.

Most of the morphology-rich branches of the family, including Kiranti, Dulong-Rawang, Kham and West-Himalayish (but not Gyalrongic), share a reflexive suffix with a dental fricative followed by a high fronted vowel (Limbu \ipa{-siŋ}, Khaling \ipa{-si}), Rawang \ipa{-shì} etc), which is likely to be reconstructible to proto-ST (\citealt[94]{bauman75}, \citealt[320]{driem93agreement}, \citealt{jacques16ssuffixes}).

In Kiranti, we find a few lexicalized examples of antipassive-like use of the reflexive in Khaling and Limbu.

In Khaling (\citealt{jacques16si}), the \ipa{-si} derivation in Khaling, alongside reflexive, reciprocal, autobenefactive and generic subject, also has an antipassive value when applied to transitive verbs expressing a feeling (whose A and P are experiencers and stimuli respectively). As shown by examples (\ref{ex:ghryamt})  and (\ref{ex:ghryamtsi}), the \ipa{-si} derivation removes the P (the stimulus) and changes the A of the base verb into an S. The stimulus is still recoverable, but must be assigned oblique case (the ablative \ipa{-kʌ}).  

\begin{exe}
\ex \label{ex:ghryamt} 
\gll 
  	\ipa{lokpei}  	\ipa{ghrɛ̄md-u.}  \\
leech  be.disgusted.by-\textsc{1sg$\rightarrow$3} \\
 \glt  I am disgusted by leeches.
\end{exe}

\begin{exe}
\ex \label{ex:ghryamtsi} 
\gll \ipa{gʰrɛ̄m-si-ŋʌ}\\
 be.disgusted.by-\textsc{refl-1sg:S/P} \\
\glt  I feel disgust.
\end{exe}

In Limbu,  the transitive \ipa{khɛtt-} `chase' has a reflexive form \ipa{khɛt-chiŋ-} whose meaning is `run'; \citet[87]{driem87} points out that the relationship between the reflexive verb and its base verb are not felt by native speakers. Here the patient of the base verb is semantically completely deleted in the reflexive form, unlike what is observed in Khaling.



%\begin{exe}
%\ex 
%\gll \ipa{àng} \ipa{vhø̄-shì-ē} \\
%\textsc{3sg} laugh-\textsc{refl-n.pst} \\
%\glt `He is laughing.'
%\end{exe}
\citet[294]{lapolla01valency}
\begin{exe}
\ex 
\gll \ipa{à:ngi} \ipa{àngsv̀ng} \ipa{shvngō-ò-ē} \\
3sg-\textsc{agt} 3sg-\textsc{loc} hate-\textsc{3:tr.n.pst-n.pst} \\
\glt `He hates him'
\end{exe}
\begin{exe}
\ex 
\gll  \ipa{àng} \ipa{nø̄} \ipa{shvngō-shì-ē} \\
3sg \textsc{top} hate-\textsc{refl-n.pst} \\
\glt `He's hateful'
 \end{exe}
\citet[452;466]{widmer14bunan}
\ipa{broŋ-s-men} “to prance” < \ipa{broŋ-tɕ-um} “to make fun off”
%broŋ-s-ɕ-um “to banter”
\ipa{dziŋ-s-men} “to quarrel” < \ipa{dziŋ-tɕ-um} “to scold”
%dziŋ-s-ɕ-um “to quarrel”

\section{Distribution of antipassive morphology in Sino-Tibetan}
All types except agent nominal

Although \citet{polinsky11antipassive} cites Hakha Lai as having an antipassive, her source \citet[37]{peterson07appl} explicitly says that `Hakha Lai has no valence-affecting constructions which target objects, such as passive or antipassive.'


\section*{Conclusion}

\bibliographystyle{unified}
\bibliography{bibliogj}

 \end{document}
 