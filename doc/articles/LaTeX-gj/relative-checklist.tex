\documentclass[oldfontcommands,oneside,a4paper,11pt]{article} 
\usepackage{fontspec}
\usepackage{natbib}
\usepackage{booktabs}
\usepackage{xltxtra} 
\usepackage{polyglossia} 
\usepackage[table]{xcolor}
\usepackage{gb4e} 
\usepackage{multicol}
\usepackage{graphicx}
\usepackage{float}
\usepackage{textcomp}
\usepackage{hyperref} 
\hypersetup{bookmarks=false,bookmarksnumbered,bookmarksopenlevel=5,bookmarksdepth=5,xetex,colorlinks=true,linkcolor=blue,citecolor=blue}
\usepackage[all]{hypcap}
\usepackage{memhfixc}
\usepackage{lscape}
 

%\setmainfont[Mapping=tex-text,Numbers=OldStyle,Ligatures=Common]{Charis SIL} 
\newfontfamily\phon[Mapping=tex-text,Ligatures=Common,Scale=MatchLowercase,FakeSlant=0.3]{Charis SIL} 
\newcommand{\ipa}[1]{{\phon #1}} %API tjs en italique
 
\newcommand{\grise}[1]{\cellcolor{lightgray}\textbf{#1}}
\newcommand{\bleute}[1]{\cellcolor{green}\textbf{#1}}
\newcommand{\rouge}[1]{\cellcolor{red}\textbf{#1}}
\newfontfamily\cn[Mapping=tex-text,Ligatures=Common,Scale=MatchUppercase]{SimSun}%pour le chinois
\newcommand{\zh}[1]{{\cn #1}}
\newcommand{\topic}{\textsc{dem}}
\newcommand{\tete}{\textsuperscript{\textsc{head}}}
\newcommand{\rc}{\textsubscript{\textsc{rc}}}
\newcommand{\refb}[1]{(\ref{#1})}
\XeTeXlinebreaklocale 'zh' %使用中文换行
\XeTeXlinebreakskip = 0pt plus 1pt %
 
 


\begin{document} 
\title{Checklist for fieldwork on relative clauses}
\author{Guillaume Jacques}
\maketitle

\textbf{General references on relative clauses}: \citet{comrie81relative}, \citet{lehmann86relatives}, \citet[205-250]{creissels06sgit2}, \citet{andrews07relatives}, \citet[313-369]{dixon10basic2}

\section{Morpho-syntactic properties of relative clauses}




\begin{itemize}
\item Position (Pre-nominal, Post-nominal, Head-Internal, Headless). In verb-final languages, be careful to distinguish post-nominal from head-internal relatives (if they both exist), and in verb initial languages, pre-nominal relatives from head-internal relatives.
\item Are there relative pronouns, or complementizers? Be careful to distinguish demonstrative elements from real complementizers.
\item Is there a limit on the size of relatives?
\item Are there correlatives?
\item Are there morphosyntactic differences between restrictive, non-restrictive and maximalzing relatives (\citealt{grosu98maximalizing})?
\item Can complement clauses headed by a noun be meaningfully distinguished from relative clauses in the target language (examples like French \textit{l'idée qu'il puisse venir m'inquiète}).
\item How do relative clauses differ from the corresponding independent clause? Are these difference specific to relatives, or are they attested with other subordinate clauses?
\begin{enumerate}
\item Is the verb in a non-finite form (participial relative)?
\item Word order (eg: SOV order in German)
\item Is there a resumptive pronominal element in the relative that is not found in the indepedent clause (\citealt{comrie81relative})?
\item Are there restrictions on Tense/Aspect/Modality/Evidentiality marking in relatives? In particular, is evidential marking neutralized in relatives (\citealt[253-6]{aikhenvald06}, \citealt{jacques16relatives})?
\item Does the use of the indefinite possessive marker  in relatives differ from its use in independent clauses (as in Japhug, cf \citealt{jacques16relatives})?
\item Do ideophones have a special status in relatives (as in Japhug, cf \citealt[275]{japhug14ideophones})?
\end{enumerate}
\end{itemize}
 \section{Inventory of relative clauses by syntactic function of the relativized element} \label{sec:inventory}
\citet{keenan77accessibility}'s Accessibility Hierarchy

Recheck all these functions, combined with various TAME categories, definiteness, referentiality etc. Which combinations are impossible?

\begin{itemize}
\item Core arguments (S, A, P)
\item R and T (check secundative vs indirective ditransitive verbs)
\item Possessor of an argument
\item Instrument
\item Place adjunct (with/without motion, with various cases/adpositions)
\item Time adjunct
\item Comitative (and all other possible cases)
\item Standard of the comparative construction (the lowest on the accessibility hierarchy)
\item Element in a complement clause embedded within the relative (eg, French \textit{la personne dont je sais qu'elle est partie pour la France}).

\end{itemize}


\section{Particular uses of the relative clauses}
\begin{itemize}
\item Attributive adjectives. In languages where adjective are a subclass of stative verbs, are NP containing attributive adjective necessarily relative clauses?
\item Indefinite (example \ref{ex:nAkWnWGmu}, Japhug). Is this the only construction to express indefiniteness in the language? Are there restrictions on this construction?

\begin{exe}
   \ex  \label{ex:nAkWnWGmu}
\gll   
\ipa{nɤʑo}  	\ipa{nɯ-nɯ-ɣɤwu}  	\ipa{ma,}  	\ipa{nɤ-kɯ-nɯɣ-mu}  	\ipa{me}  	\ipa{ma}  	\ipa{mɤ-ta-mbi}  \\
you \textsc{imp-auto}-cry because \textsc{2sg-nmlz:S-appl}-be.afraid not.exist:\textsc{fact} because \textsc{neg-1$\rightarrow$2-}give:\textsc{fact} \\
\glt Cry as you wish, nobody is afraid of you (there is no one who is afraid of you), I will not give her to you.  (The frog, 38)
\end{exe}
\item Focalization: are cleft and  pseudo-clef sentences possible, and are they really used to focalize NPs?
\end{itemize}


\section{Syntactic pivots}
Use of restrictive neutralization in relative constructions (\citealt[275]{vanvalin97syntax}) to study syntactic pivots.

Use the data collected in section \ref{sec:inventory} to a Table like Table \ref{tab:summary} (\citealt{jacques16relatives}; HI stands for \textit{head-internal} relative and PN for \textit{prenominal} relative).

\begin{table}[H]
\caption{Summary of relatives in Japhug } \label{tab:summary}
\resizebox{\columnwidth}{!}{
\begin{tabular}{l|ccc|ccc}
\toprule
&\multicolumn{3}{c}{Participial Relative Clause} & \multicolumn{2}{c}{Finite Relative Clause} \\
Function & \ipa{kɯ-}  & \ipa{kɤ-}  & \ipa{sɤ-}  & Simple  & Relator noun \\
\midrule
S	& HI, PN \bleute{}& &&&& \\
A & HI, PN \bleute{}& &&&& \\
\hline
possessor & PN&&&&& \\
\hline
P & & HI, PN \rouge{}&& HI, PN \rouge{} &\\
semi-object & & HI, PN \rouge{}&& HI, PN \rouge{} &\\
T & & HI, PN \rouge{} && HI, PN \rouge{}&\\
R (secundative) & & HI, PN \rouge{} && HI, PN \rouge{}&\\
\hline 
goal & & &HI, PN  & HI, PN  &\\
\hline
R (indirective) & &&HI, PN\\
comitative & &&HI, PN\\
instrumental adjunct  & &&HI, PN \\
\hline
time adjunct  & &&HI, PN&&PN\\
place adjunct  &&&HI, PN&&PN\\ 
\bottomrule
\end{tabular}}
\end{table}

From such a Table (combined with the study of case marking, person indexation, complementation etc), derive a Table like \ref{tab:japhug.pivot} summarizing the syntactic pivots  attested in the target language (data from Japhug, \citealt{jacques16relatives}; The symbol P' is used for the semi-object of semi-transitive verbs, and T_1 / R_1 vs T_2 / R_2 for the arguments of secundative vs indirective transitive verbs).

\begin{table}[H]
\caption{Syntactic pivots in Japhug} \label{tab:japhug.pivot} \centering
\begin{tabular}{llllll}
\toprule
Pivot & Construction \\
\midrule
\{S, A\}  & prenominal relativization with \ipa{kɯ-} participle  \\
 (subject)&(possessive prefix) on \ipa{kɤ-} participles in relatives \\
 \midrule
\{P, P', R_1, T\}  &relativization with \ipa{kɤ-} participle \\
(object)  & \\
\midrule
\{P, P', R_1, T, goal\} &
relatives with a finite main verb \\
(extended object) &(without relator noun)\\
\midrule
 \{S, P, R_1, T_2\} & generic person marking\\
 (absolutive argument)\\
 \midrule
  \{S, A, P, P', R_1, T\} & control constructions (\ipa{rga}  `like') \\
 (core argument)\\
\bottomrule
\end{tabular}
\end{table}



\bibliographystyle{unified}
\bibliography{bibliogj}
\end{document}