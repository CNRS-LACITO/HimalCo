\documentclass[xcolor=table]{beamer}
\usepackage{xunicode}
\usepackage{fontspec}
\usepackage{gb4e} 
\usepackage{natbib}
%\usepackage[table]{xcolor}
\usepackage{color}
\usepackage{graphicx}
 \newcommand{\bleu}[1]{{\color{blue}#1}}
\newcommand{\rouge}[1]{{\color{red}#1}} 
 \setmainfont[Mapping=tex-text]{Charis SIL}
\let\sfdefault\rmdefault
\newcommand{\racine}[1]{\begin{math}\sqrt{#1}\end{math}} 
\newcommand{\grise}[1]{\cellcolor{lightgray}\textbf{#1}} 
\newcommand{\gray}[1]{\cellcolor{lightgray!30}{#1}}
\newcommand{\lightgray}[1]{\cellcolor{lightgray!10}{#1}}
  \usepackage{amssymb}
  \newfontfamily\phon[Mapping=tex-text,Ligatures=Common,Scale=MatchLowercase]{Charis SIL} 
\newcommand{\ipa}[1]{{\phon\textit{#1}}} 
 \newcommand{\dhatu}[2]{|\ipa{#1}| `#2'}
 \newfontfamily\cn[Mapping=tex-text,Ligatures=Common,Scale=MatchUppercase]{STHeiti}%pour le chinois
\newcommand{\zh}[1]{{\cn #1}}
    \newenvironment{ssize}{\footnotesize}{}{}
     \newcommand{\quo}[1]{“#1”}
     \newcommand{\sens}[1]{‘#1’}




 \begin{document}

 \title{Associated motion in Sino-Tibetan/Trans-Himalayan}
 \author{Guillaume Jacques (\textsc{Cnrs-Crlao})\\ Aimée Lahaussois (\textsc{Htl-Cnrs}) \\ Zhang Shuya (\textsc{Inalco-Crlao})}
 \date{}
 \maketitle
  
 \begin{frame} 
\frametitle{Language groups considered in this study }
\begin{enumerate}
\item Gyalrongic
\item Kiranti
\item Sinitic
\item Other (Karbi, Burmese, Tangsa)
\end{enumerate}
 \end{frame}  
  
\begin{frame} 
\frametitle{Typological parameters of AM in ST}
Based on: \citealt{koch84associated.motion}, \citealt{wilkins91associated.motion}, \citealt{guillaume16am}
\begin{enumerate}
\item Temporal relation (prior, concurrent, subsequent)
\item Deixis (circum-, cis-, translocative)
\item Argument of motion (S/A, P)
\item Non-AM meanings (orientation without motion, aspect, person configuration, voice)
\item Vertical dimension
\end{enumerate}
One additional parameter that is relevant to the description of AM in ST:  \bleu{Mono- vs pluriactionality}
 \end{frame}  

\begin{frame} 
\frametitle{Gyalrongic}
 \framesubtitle{Main characteristics of AM}
\begin{enumerate}
\item Devoted AM prefixes
\item AM prefixes and orientation prefixes co-occur (different templatic slots, ex. \ref{ex:CpWru})
\item Echo phenomena
\item Mono-actionality
\end{enumerate}

\begin{exe}
\ex \label{ex:CpWru}
\glll \ipa{\rouge{ɕ}-\bleu{pɯ}-ru} \\
 \rouge{\textsc{go\&do}}-\bleu{\textsc{down}:\textsc{pfv}}-look \\
\rouge{AM}-\bleu{orientation}-V \\
\glt `He \rouge{went and} looked \bleu{down}.' (Japhug)
\end{exe}

 \end{frame}  
 
\begin{frame} 
\frametitle{Gyalrongic}
 \framesubtitle{Echo phenomena (multiple marking for a single motion event)}

\begin{exe}
\ex \label{ex:CtAru}
\gll \ipa{tɕʰi} \ipa{ɯ-taʁ} \ipa{to-\rouge{ɕe}} \ipa{tɕe} \ipa{\rouge{ɕ}-tɤ-ru}   \\
stairs \textsc{3sg}.\textsc{poss}-on \textsc{ifr}:\textsc{up}-\rouge{go} \textsc{lnk}  \rouge{\textsc{go\&do}}-\textsc{up}:\textsc{pfv}-look \\
\glt `He \rouge{went} up the stairs and looked up.' (Japhug)
\end{exe}
\begin{exe}
\ex \label{ex:GWtaBzu}
\gll  \ipa{tɕe} 	\ipa{a-kʰa} 	\ipa{ra} 	\ipa{\rouge{ɣɯ}-ta-rɤroʁrɯz,} 	\ipa{a-mgo} 	\ipa{ra} 	\ipa{\rouge{ɣɯ}-ta-βzu} 	\ipa{ŋu} 	\ipa{ɕi} 
\\
\textsc{lnk} \textsc{1sg}.\textsc{poss}-house \textsc{pl} \rouge{\textsc{come\&do}}-\textsc{pfv}:3$\rightarrow$3'-tidy 
 \textsc{1sg}.\textsc{poss}-food \textsc{pl} \rouge{\textsc{come\&do}}-\textsc{pfv}:3$\rightarrow$3'-make be:\textsc{fact} \textsc{qu} \\ 
\glt `Is it (the neighbour's wife who took pity on me), and \rouge{came} to tidy my house and make food for me?' (Japhug)
\end{exe}
 \end{frame}   
 
 \begin{frame} 
\frametitle{Gyalrongic}
 \framesubtitle{Mono-actionality}
 Tests used:
\begin{enumerate}
\item Concessives and negation (\citealt{jacques13harmonization})
\item Interrogatives
\item Protasis of conditionals
\item Scope of complement-taking verbs
 \end{enumerate}
 \end{frame}  
 
  \begin{frame} 
\frametitle{Gyalrongic}
 \framesubtitle{Mono-actionality in interrogatives}
 Motion verb construction (\ref{ex:tChi.WkWndza}) vs AM (\ref{ex:tChi.GWtAtWndzat}):
\begin{exe}
\ex \label{ex:tChi.WkWndza}
\gll tɕʰi ɯ-kɯ-pa jɤ-tɯ-\bleu{ɣe}? \\
what \textsc{3sg.poss}-do \textsc{pfv}-2-\bleu{come[II]} \\
\glt `What have you \bleu{come} to do?' (Japhug)
\end{exe}

\begin{exe}
\ex \label{ex:tChi.GWtAtWndzat}
\gll tɕʰi \rouge{ɣɯ}-tɤ-tɯ-pa-t \\
what \rouge{\textsc{come\&do}}-\textsc{pfv}-2-do-\textsc{pst:tr}    \\
\glt `What did you do upon \rouge{coming} here?' (Japhug)
\end{exe} 
  \end{frame}  

  \begin{frame} 
\frametitle{Gyalrongic}
 \framesubtitle{Mono-actionality (complement clauses)}
\begin{exe}
\ex \label{ex:CWkAmtChot}
\gll tɯrsa \rouge{ɕɯ}-kɤ-mtɕʰot to-mda ɲɯ-ŋu \\
grave \rouge{\textsc{transloc}}-\textsc{inf}-make.offerings \textsc{ifr}-be.time.to \textsc{sens}-be \\
\glt `It was the time to (\rouge{go} and) make offerings for the graves.' (160630 abao-zh, 70)
 \end{exe} 

 \begin{exe}
\ex \label{ex:kWrAma.kACe}
\gll   kɯ-rɤma kɤ-\bleu{ɕe} mda mɤ-mda nɯtɕu ɕ-tu-kɯ-ru pɯ-ŋgrɤl. \\
 \textsc{nmlz}:S/A-work \textsc{inf}-\bleu{go} be.time:\textsc{fact} \textsc{neg}-be.time:\textsc{fact} \textsc{dem}:\textsc{loc} \textsc{transloc}-\textsc{ipfv}:up-\textsc{genr}:S/P-look \textsc{pst}.\textsc{ipfv}-be.usually \\
 \glt  `(people used to watch when (these stars) came out or disappeared to know) whether it was time to \bleu{go} to work.' (29-LAntshAm, 66)
  \end{exe}
    \end{frame}  
    \begin{frame} 
    
    
\frametitle{Gyalrongic}
 \framesubtitle{Volitionality}
MVC implies only a volitional verbal action, while AM can be also used with non-volitional verbs, such as \quo{See \textsc{vs.} Happen upon; come across}  as in \ref{ex:CpjAmto}.

  \begin{exe}
\ex  \label{ex:CpjAmto}
\gll  nɯɕɯmɯma ʑo tɯ-ci ɯ-ŋgɯ pjɤ-\bleu{ɕe} qʰe iɕqʰa tɤɕime kɯ ɯ-sɤcɯ pɯ-kɤ-nɯ-ɕlɯɣ nɯ \rouge{ɕ}-pjɤ-mto. \\
immediately \textsc{emph} \textsc{indef}.\textsc{poss}-water \textsc{3sg}.\textsc{poss}-inside \textsc{ifr}:\textsc{down}-go \textsc{lnk} the.aforementioned lady \textsc{erg} \textsc{3sg}.\textsc{poss}-key \textsc{pfv}:\textsc{down}-\textsc{nmlz}:P-\textsc{auto}-drop \textsc{dem} \rouge{\textsc{transloc}}-\textsc{ifr}-see \\
\glt `He \bleu{went} immediately into the water and \rouge{found there} the key that the lady had dropped by mistake.' (140510 fengwang, 118)
\end{exe}
  \end{frame}  
  
  
\begin{frame} 
\frametitle{Kiranti}
 \framesubtitle{Main characteristics of AM}
\begin{enumerate}
\item All Kiranti languages have compound verbs (V\textsubscript{1}+V\textsubscript{2}), and some V\textsubscript{2}s come from motion verbs. The motion V\textsubscript{2}s are often grammaticalized as aspect markers (eg \citealt{bickel96aspect}), but retain in some languages the ability to express a motion event.
\item Vertical dimension in AM markers (eg \ref{ex:khiptyanta})
\item Argument of motion: can be the subject (S/A) or both subject and object (S/A+P) with a labile AM marker (\rouge{\textsc{do\&bring}} (\ref{ex:khiptyanta}), \rouge{\textsc{do\&take}} etc)
\begin{exe}
\ex \label{ex:khiptyanta}
 \gll \ipa{tukkâ-m}	\ipa{ʦulo-tʉ}	\ipa{kʰir}	\ipa{kʰip-\rouge{tɛn}-tʌ} \\
  up.there.\textsc{distal}-\textsc{rel} hearth-\textsc{loc}:\textsc{up} porridge cook-\rouge{\textsc{do\&bring.down}}-\textsc{1sg}.\textsc{pst} \\
  \glt `I cooked porridge on the stove up there (on the roof) and \rouge{brought it down}.' (Khaling)
 \end{exe}
 \item Pluri-actionality
\end{enumerate}
\end{frame}  

\begin{frame} 
\frametitle{Kiranti}
 \framesubtitle{Pluri-actionality (seen through the scope of negation)}

\begin{exe}  
\ex \label{ex:hungkhondu}
 \gll  \ipa{ʔīn} \ipa{kʰɵs-tʰer-e} \ipa{uŋʌ} \ipa{ʔʌ-dʌrʌm} \ipa{ɦûŋ-\rouge{kʰond}-u} \\
 2sg go-\textsc{habit}-\textsc{imp}:\textsc{2sg} \textsc{1sg}.\textsc{erg} \textsc{1sg}.\textsc{poss}-friend wait-\rouge{\textsc{do\&come.up}}-\textsc{1sg.A.n.pst} \\
\glt `You keep going, I will wait for my friends and \rouge{come up} then.' (Khaling)
\end{exe}

\begin{exe}
\ex \label{ex:mahunkhonya}
 \gll
\ipa{ʔuŋʌ} \ipa{ʔʌ̄m} \ipa{mʌ-ɦû-n-\rouge{kʰōː}-nɛ-ʔɛ} \ipa{ʔu-nûː} \ipa{ŋes-tɛ} \\
\textsc{1sg}.\textsc{erg} \textsc{3sg} \textsc{neg}-wait-\textsc{inf}-\rouge{\textsc{do\&come.up}}-\textsc{inf}-\textsc{erg} \textsc{3sg}.\textsc{poss}-mind hurt-2/3:\textsc{pst} \\
\glt `I did not wait for him \rouge{before going up} and he was sad. (I was already gone up by the time he arrived at the waiting place).' (Khaling)
\end{exe}
\end{frame}  

\begin{frame} 
\frametitle{Kiranti}
 \framesubtitle{The Khaling AM system} 
 
\resizebox{\columnwidth}{!}{
\begin{tabular}{lllllll}
Source & AM (V\textsubscript{2}) & Temp.& Deixis & Verticality & Argument \\
?& \dhatu{-le(t)-}{go around doing X} &C& \textsc{circum} & $\varnothing$ & S/A \\
\dhatu{kʰot}{go} &\dhatu{-kʰo(t)-}{do X and go} &S& \textsc{trans} & $\varnothing$ &  S/A(+P) \\
?&\dhatu{-pɛ(t)-}{go and do X} &P& \textsc{trans} & $\varnothing$ & S/A \\
\dhatu{ɦo}{come} &\dhatu{-ɦo(t)-}{do X and come} &S& \textsc{cis} & $\varnothing$ & S/A(+P) \\
\dhatu{kʰoŋ}{come up}&\dhatu{-kʰoŋ-}{do X and come up} &S& \textsc{cis} &up & S/A \\
 \dhatu{pi}{come (same level)}&\dhatu{-pi(t)-}{come (same level)} &S/P& \textsc{cis} &same level & S/A(+P) \\
\dhatu{tɛn}{fall} & \dhatu{-tɛ(nt)-}{do X and come/bring down} &S/P & \textsc{cis} &down & S/A(+P) \\
\end{tabular}}
  \end{frame}  

 \begin{frame} 
\frametitle{Sinitic}
\begin{itemize}
\item An AM approach for the post-VP \zh{去 \ipa{tɕʰy}}/\zh{来 \ipa{lai}} in the purposive constructions [VP+ \zh{去/来}] 
\begin{itemize}
\item Phonetic weakening
\item Loss of verbality, showing evidence of grammaticalization
\item Described with \textsc{motion} and \textsc{purpose}
\end{itemize}
\item Chinese translation: \zh{关联位移} \ipa{guānlián wèiyí}
\end{itemize}

\begin{exe}
\ex AM, Standard Mandarin \citetext{\citealp[]{lu1985vpqu}; \citealp[]{lamarre17deictic}} \label{distribution2}
\glll
\zh{喝} \zh{点儿} \zh{水}=\rouge{\zh{去}} \\
\ipa{Hē} \ipa{diǎnr} \ipa{shuǐ}=\rouge{\ipa{qu}} \\
drink  a.little water=\textsc{go}$\&$ \\
\glt \sens{Go and drink some water}.
\end{exe}

\begin{exe}
\ex AM, Standard Mandarin \citep[479]{chao68chinese} 
\glll
\zh{换} \zh{了} \zh{衣裳}  \zh{打球}=\rouge{\zh{来}} \\
\ipa{Huàn} \ipa{le} \ipa{yīshang} \ipa{dǎqi\'{u}}=\rouge{\ipa{lai}}  \\
change \textsc{asp} clothes play.ball=\textsc{come}$\&$ \\
\glt \sens{Change one's clothes to play ball.}
\end{exe}

\end{frame}

\begin{frame}{Sinitic}


\begin{exe}
\ex MVC, Standard Mandarin \citep{lu1985vpqu} \label{inventory2}
\glll
\bleu{\zh{去}} \zh{喝} \zh{点儿} \zh{水} \\
\bleu{\ipa{Qù}} \ipa{hē} \ipa{diǎnr} \ipa{shuǐ} \\
go drink a.little water \\
\glt \sens{Go to drink some water.}
\end{exe}

\begin{exe}
\ex MVC, Standard Mandarin \citep[479]{chao68chinese} 
\glll
 \bleu{\zh{来}} \zh{打球}  \\
\bleu{\ipa{Lái}} \ipa{dǎqi\'{u}}   \\
come play.ball \\
 \glt \sens{come to play ball}
\end{exe}

\end{frame}

\begin{frame}{Sinitic}
\begin{table} [H]
\caption{\textsc{am} marker in Standard Mandarin} \centering
\begin{tabular}{llllllllll}
\hline
& \zh{去} & \zh{来}  \\
\hline
Source verb & \zh{去} \ipa{qù} (\ipa{tɕʰy\textsuperscript{51}}) &  \zh{来} \ipa{lái} (\ipa{lai\textsuperscript{35}}) \\
\textsc{am} marker form & \zh{去} \ipa{qu} (\ipa{tɕʰy}) &  \zh{来} \ipa{lai} (\ipa{lai}) \\
Meaning & to go  & to come  \\
Gloss & \textsc{go}$\&$ & \textsc{come}$\&$ \\
Temporality  &  Prior to action & Prior to action \\
Deixis & Translocative & Cislocative  \\
Vertical dimension & None & None \\
Actionality & ? & ?  \\
\hline
\end{tabular}
\end{table}

\begin{itemize}
\item Presence \textsc{vs.} absence
\item Unbalanced distribution of \zh{去} and \zh{来}
\end{itemize}

\end{frame}



\begin{frame}{Sinitic}

\resizebox*{!}{\dimexpr\textheight-2\baselineskip\relax}{%
\begin{tabular}{lllllllll}
\hline
& Language branch & Dialect & \textsc{am} markers   \\
\hline
\grise{I}. & \grise{Mandarin \zh{官话}} & \grise{}&  \grise{}  \\
 \grise{}& \grise{Beijing \zh{北京}}  &  \grise{Beijing \zh{北京}} & \grise{\zh{去} \zh{来} } \\
 \grise{}& \grise{Northwestern (Lanyin \zh{兰银})} & \grise{Yinchuan \zh{银川}} &\grise{\zh{去} \zh{来}} \\
  \grise{}& \grise{} & \grise{Dachuan \zh{达川}} & \grise{\zh{去} \zh{来}}\\
 \grise{} &  \grise{Central plains (Zhongyuan \zh{中原})} & \grise{Xi'an \zh{西安} }  &  \grise{\zh{去} \zh{来} } \\
  \gray{} &  \gray{Southwestern \zh{西南}}  & \gray{Chengdu \zh{成都}} & \gray{\zh{去?} \zh{来?} } \\
   \gray{} &  \gray{}  & \gray{Wuhan \zh{武汉}} &  \gray{\zh{去?} \zh{来?} }  \\
    \gray{} & \gray{}  & \gray{Yuxi \zh{玉溪}} &  \gray{\zh{去?} }\\
  \gray{} & \gray{Jianghuai \zh{江淮} (Southern)}  & \gray{Anqing \zh{安庆}}& \gray{\zh{去?} \zh{来?} }  \\
\grise{II.} & \grise{Jin \zh{晋}} & \grise{Shenmu \zh{神木}} &  \grise{\zh{去} \zh{来} } \\
\grise{} & \grise{} & \grise{Datong \zh{大同}} &  \grise{\zh{去} \zh{来} }  \\
\grise{} & \grise{} & \grise{Linfen \zh{临汾}} & \grise{\zh{去} \zh{来}} \\
\gray{III.} &\gray{Xiang \zh{湘}} & \gray{Changsha \zh{长沙}} &   \gray{\zh{去?} \zh{来?}}\\
\gray{} &\gray{} & \gray{Changde \zh{常德}} & \gray{\zh{去?} \zh{来}?}   \\
\gray{IV.} & \gray{Wu \zh{吴}} &  \gray{Shanghai \zh{上海}} & \gray{\zh{去?} \zh{来?}} \\
\gray{} & \gray{} &  \gray{Hangzhou \zh{杭州}} &  \gray{\zh{去?}}   \\
\gray{} & \gray{} &  \gray{Ningbo \zh{宁波}} &  \gray{\zh{去?}} \\
\gray{} & \gray{} &  \gray{Suzhou \zh{苏州}} & \gray{None} \\
\gray{V.} & \gray{Hui \zh{徽}} &  \gray{Huizhou \zh{徽州}} & \gray{\zh{去?}}   \\
\lightgray{VI.} & \lightgray{Gan \zh{赣}} & \gray{Ji'an \zh{吉安}} & \gray{\zh{去} \zh{来}}  \\
 & &  Fuzhou \zh{抚州} & None \\
VII. & Min  \zh{闽} & Chaozhou \zh{潮州}  & None  \\
VIII. & Kejia \zh{客家} (Hakka)  & Rucheng \zh{汝城} & None \\
IX. & Yue \zh{粤} (Cantonese) & Guangzhou \zh{广州} & None   \\
\hline
\end{tabular}}

\end{frame}

\begin{frame}{Sinitic}
 \begin{minipage}[c]{.46\linewidth}
\includegraphics[scale=0.3]{../map.pdf} 
   \end{minipage} \hfill
   \begin{minipage}[c]{.46\linewidth}
   
   \begin{itemize}
\item Northern China: \textsc{am} encoding tends to be present.
\item Southern China: \textsc{am} encoding tends to be absent.
\item Central China: data unclear
\end{itemize}
\end{minipage}
\end{frame}


\begin{frame} 
\frametitle{Other languages}
\begin{enumerate}
\item<1-> Karbi (NE India, \citealt{konnerth15cisloc}): prior cislocative motion (\rouge{nang}= \textsc{come\&do})
\begin{exe}
\ex 
\gll alàng-lì=ke là-tūm a-hēm=si \rouge{nang}=vùr-si sá aját \rouge{nang}=jùn-lò \\
 3-\textsc{hon}=\textsc{top} this-\textsc{pl} \textsc{poss}-house=\textsc{foc} \rouge{\textsc{cis}}=drop.in-\textsc{nf}:\textsc{rl} tea \textsc{genex} \rouge{\textsc{cis}}=drink-\textsc{rl} \\
 \glt `...it was him, at their house we stopped by and had tea [\rouge{come} and drink tea] and everything.'
\end{exe}
 \item<2-> Tangsa (NE India, \citealt[311-312]{boro17hakhun})
 \begin{exe}
\ex 
\gll   inɤ́ ʒuk \rouge{kà} l-oʔ \\
there drink \rouge{go} \textsc{imp}-\textsc{2sg} \\
\glt ‘\rouge{Go} and drink (tea) there.’
\end{exe}
  \item<3->   \citet[388]{Jenny16grammar}
 \begin{exe}
\ex 
\gll   θu-dó tʰəmìn sà-\rouge{la}-dɛ \\
3-\textsc{pl} cooked.rice eat-\rouge{come}-\textsc{nfut}  \\
\glt ‘They have eaten (\rouge{before coming here}).'
\end{exe}
   \end{enumerate}
\end{frame}  

\begin{frame} 
\frametitle{Summary of AM systems in ST}
\resizebox{\columnwidth}{!}{
\begin{tabular}{llllllllll}
Subgroup&Language  &Number &Temp. & Deixis & Vertical & Other & Mono- &Arg.\\
&&&relation &&dimension&functions& actional & \\
Gyalrong&Japhug & 2&P & Cis/Trans & $\varnothing$ &  $\varnothing$ & Y &S/A \\
&Situ &2&P & Cis/Trans & $\varnothing$ & Aspect&N &S/A \\
Kiranti &Khaling & 7&P, C, S &Cis/Trans & Y & Aspect,  &  N  &S/A, S/A+P \\
&&&&&&Orientation&&\\
&Dumi & 4? &P, C, S &Cis/Trans &  $\varnothing$ & Aspect,  &  ?&S/A, S/A+P \\
  &&&&&&Orientation&&\\
&Thulung & 1 & C  &Circum &  $\varnothing$ & $\varnothing$ &?  &S/A  \\
&Wambule & 5 & P  &Cis/Trans &  Y & Aspect,  &?  &S/A, S/A+P \\
     &&&&&&Orientation&&\\
%    &Jero & 2 & C, P?  & Trans? &  \Y & ?  &?  &S/A  \\
&Yamphu & 7 & P, C, S  & Cis/Trans &  Y & Aspect,  &?  &S/A, S/A+P \\
        &&&&&&Orientation&&\\
&Yakkha & 7? & P, C, S  & Cis/Trans &  Y & Aspect,  &?  &S/A, S/A+P \\
        &&&&&&Orientation&&\\
&Belhare & 3? & P, C, S  & Cis/Trans &  Y & Aspect,  &?  &S/A, S/A+P \\
        &&&&&&Orientation&&\\
&Hayu & 1  & P  & Trans &  N & Aspect,  &?  &S/A  \\
        &&&&&&Voice&&\\
Sinitic & Mandarin &2& P & Cis/Trans & $\varnothing$ &Aspect,  Modality& N   &S/A  \\
        &&&&&&Orientation&&\\
Karbi&&1&S&Cis&  $\varnothing$ &Person, Orientation&?  &S/A+P \\
Sal&Tangsa&1&P&Trans&  $\varnothing$ & ?&?  &S/A+P  \\
LB&Burmese&6&P, C, S&Cis/Trans&  $\varnothing$ &Aspect, &?  &S/A+P  \\
        &&&&&&Orientation&&\\
 \end{tabular}}
 
\end{frame}  
 
\begin{frame} 
\frametitle{Suggestions for fieldworkers}

\begin{enumerate}
\item Illustrate AM with verbs that do not imply intrinsic motion (motion verbs, verbs of manipulation) -- otherwise, impossible to distinguish between AM and orientation.
\item Test for mono- vs pluri-actionality (negation, concessives, interrogatives, conditions, complement clauses)
\item For labile AM markers, specify whether P is an argument of motion or not.
\end{enumerate}

\end{frame}  

\begin{frame} 
\frametitle{Conclusion}
\begin{enumerate}
\item<1->  AM is found in most major branches of ST, and widely distributed geographically.
\item<2->  Independent grammaticalization from motion verbs (except perhaps in Karbi).
\item<3->  Highly diverse systems  (from 1 to 7 markers). The most grammaticalized systems are found in the branches with the most complex morphology.
\end{enumerate}
\end{frame}  

\begin{frame} 
\frametitle{References} 
\tiny
\bibliographystyle{linquiry2}
\bibliography{bibliogj}
\end{frame}  

\end{document}