\documentclass[xcolor=table]{beamer}
\usepackage{xunicode}
\usepackage{fontspec}
\usepackage{gb4e} 
\usepackage{natbib}
%\usepackage[table]{xcolor}
\usepackage{color}
\usepackage{graphicx}
 \newcommand{\bleu}[1]{{\color{blue}#1}}
\newcommand{\rouge}[1]{{\color{red}#1}} 
 \setmainfont[Mapping=tex-text]{Charis SIL}
\let\sfdefault\rmdefault
\newcommand{\racine}[1]{\begin{math}\sqrt{#1}\end{math}} 
\newcommand{\grise}[1]{\cellcolor{lightgray}\textbf{#1}} 
  \usepackage{amssymb}
  \newfontfamily\phon[Mapping=tex-text,Ligatures=Common,Scale=MatchLowercase]{Charis SIL} 
\newcommand{\ipa}[1]{{\phon\textit{#1}}} 
 \newcommand{\dhatu}[2]{|\ipa{#1}| `#2'}
 
 \begin{document}

 \title{Associated motion in Sino-Tibetan/Trans-Himalayan}
 \author{Guillaume Jacques\\ Aimée Lahaussois \\ Zhang Shuya}
 \date{}
 \maketitle
  
 \begin{frame} 
\frametitle{Language groups considered in this study }
\begin{enumerate}
\item Gyalrongic
\item Kiranti
\item Sinitic
\item Other (Karbi, Burmese, Tangsa)
\end{enumerate}
 \end{frame}  
  
\begin{frame} 
\frametitle{Typological parameters of AM in ST}
Based on: \citealt{koch84associated.motion}, \citealt{wilkins91associated.motion}, \citealt{guillaume16am}
\begin{enumerate}
\item Temporal relation (prior, concurrent, subsequent)
\item Deixis (circum-, cis-, translocative)
\item Argument of motion (S/A, P)
\item Non-AM meanings (orientation without motion, aspect, person configuration, voice)
\item Vertical dimension
\end{enumerate}
One additional parameter that is relevant to the description of AM in ST:  \bleu{Mono- vs pluriactionality}
 \end{frame}  

\begin{frame} 
\frametitle{Gyalrongic}
 \framesubtitle{Main characteristics of AM}
\begin{enumerate}
\item Devoted AM prefixes
\item AM prefixes and orientation prefixes co-occur (different templatic slots, ex. \ref{ex:CpWru})
\item Echo phenomena
\item Mono-actionality
\end{enumerate}

\begin{exe}
\ex \label{ex:CpWru}
\glll \ipa{\rouge{ɕ}-\bleu{pɯ}-ru} \\
 \rouge{\textsc{go\&do}}-\bleu{\textsc{down}:\textsc{pfv}}-look \\
\rouge{AM}-\bleu{orientation}-V \\
\glt `He \rouge{went and} looked \bleu{down}.' (Japhug)
\end{exe}

 \end{frame}  
 
\begin{frame} 
\frametitle{Gyalrongic}
 \framesubtitle{Echo phenomena (multiple marking for a single motion event)}

\begin{exe}
\ex \label{ex:CtAru}
\gll \ipa{tɕʰi} \ipa{ɯ-taʁ} \ipa{to-\rouge{ɕe}} \ipa{tɕe} \ipa{\rouge{ɕ}-tɤ-ru}   \\
stairs \textsc{3sg}.\textsc{poss}-on \textsc{ifr}:\textsc{up}-\rouge{go} \textsc{lnk}  \rouge{\textsc{go\&do}}-\textsc{up}:\textsc{pfv}-look \\
\glt `He \rouge{went} up the stairs and looked up.' (Japhug)
\end{exe}
\begin{exe}
\ex \label{ex:GWtaBzu}
\gll  \ipa{tɕe} 	\ipa{a-kʰa} 	\ipa{ra} 	\ipa{\rouge{ɣɯ}-ta-rɤroʁrɯz,} 	\ipa{a-mgo} 	\ipa{ra} 	\ipa{\rouge{ɣɯ}-ta-βzu} 	\ipa{ŋu} 	\ipa{ɕi} 
\\
\textsc{lnk} \textsc{1sg}.\textsc{poss}-house \textsc{pl} \rouge{\textsc{come\&do}}-\textsc{pfv}:3$\rightarrow$3'-tidy 
 \textsc{1sg}.\textsc{poss}-food \textsc{pl} \rouge{\textsc{come\&do}}-\textsc{pfv}:3$\rightarrow$3'-make be:\textsc{fact} \textsc{qu} \\ 
\glt `Is it (the neighbour's wife who took pity on me), and \rouge{came} to tidy my house and make food for me?' (Japhug)
\end{exe}
 \end{frame}   
 
 \begin{frame} 
\frametitle{Gyalrongic}
 \framesubtitle{Mono-actionality}
 Tests used:
\begin{enumerate}
\item<1-> Concessives and negation (\citealt{jacques13harmonization})
\item<2-> Interrogatives
\item<3-> Protasis of conditionals
\item<3-> Scope of complement-taking verbs
 \end{enumerate}
 \end{frame}  
 
  \begin{frame} 
\frametitle{Gyalrongic}
 \framesubtitle{Mono-actionality in interrogatives}
 Motion verb construction (\ref{ex:tChi.WkWndza}) vs AM (\ref{ex:tChi.GWtAtWndzat}):
\begin{exe}
\ex \label{ex:tChi.WkWndza}
\gll tɕʰi ɯ-kɯ-pa jɤ-tɯ-ɣe? \\
what \textsc{3sg.poss}-do \textsc{pfv}-2-come[II] \\
\glt `What have you come to do?' (Japhug)
\end{exe}

\begin{exe}
\ex \label{ex:tChi.GWtAtWndzat}
\gll tɕʰi \rouge{ɣɯ}-tɤ-tɯ-pa-t \\
what \rouge{\textsc{come\&do}}-\textsc{pfv}-2-do-\textsc{pst:tr}    \\
\glt `What did you do, after you \rouge{came} here?' (Japhug)
\end{exe} 
  \end{frame}  
  
\begin{frame} 
\frametitle{Kiranti}
 \framesubtitle{Main characteristics of AM}
\begin{enumerate}
\item<1-> All Kiranti languages have compound verbs (V_1+V_2), and some V_2s come from motion verbs. The motion V_2s are often grammaticalized as aspect markers (eg \citealt{bickel96aspect}), but retain in some languages the ability to express a motion event.
\item<2-> Vertical dimension in AM markers (eg \ref{ex:khiptyanta})
\item<3-> Argument of motion: can be the subject (S/A) or both subject and object (S/A+P) with a labile AM marker (\rouge{\textsc{do\&bring}} (\ref{ex:khiptyanta}), \rouge{\textsc{do\&take}} etc)
\begin{exe}
\ex \label{ex:khiptyanta}
 \gll \ipa{tukkâ-m}	\ipa{ʦulo-tʉ}	\ipa{kʰir}	\ipa{kʰip-\rouge{tɛn}-tʌ} \\
  up.there.\textsc{distal}-\textsc{rel} hearth-\textsc{loc}:\textsc{up} porridge cook-\rouge{\textsc{do\&bring.down}}-\textsc{1sg}.\textsc{pst} \\
  \glt `I cooked porridge on the stove up there (on the roof) and \rouge{brought it down}.' (Khaling)
 \end{exe}
 \item<4-> Pluri-actionality
\end{enumerate}
\end{frame}  

\begin{frame} 
\frametitle{Kiranti}
 \framesubtitle{Pluri-actionality (seen through the scope of negation)}

\begin{exe}  
\ex \label{ex:hungkhondu}
 \gll  \ipa{ʔīn} \ipa{kʰɵs-tʰer-e} \ipa{uŋʌ} \ipa{ʔʌ-dʌrʌm} \ipa{ɦûŋ-\rouge{kʰond}-u} \\
 2sg go-\textsc{habit}-\textsc{imp}:\textsc{2sg} \textsc{1sg}.\textsc{erg} \textsc{1sg}.\textsc{poss}-friend wait-\rouge{\textsc{do\&come.up}}-\textsc{1sg.A.n.pst} \\
\glt `You keep going, I will wait for my friends and \rouge{come up} then.' (Khaling)
\end{exe}

\begin{exe}
\ex \label{ex:mahunkhonya}
 \gll
\ipa{ʔuŋʌ} \ipa{ʔʌ̄m} \ipa{mʌ-ɦû-n-\rouge{kʰōː}-nɛ-ʔɛ} \ipa{ʔu-nûː} \ipa{ŋes-tɛ} \\
\textsc{1sg}.\textsc{erg} \textsc{3sg} \textsc{neg}-wait-\textsc{inf}-\rouge{\textsc{do\&come.up}}-\textsc{inf}-\textsc{erg} \textsc{3sg}.\textsc{poss}-mind hurt-2/3:\textsc{pst} \\
\glt `I did not wait for him \rouge{before going up} and he was sad. (I was already gone up by the time he arrived at the waiting place).' (Khaling)
\end{exe}
\end{frame}  

\begin{frame} 
\frametitle{Kiranti}
 \framesubtitle{The Khaling AM system} 
 
\resizebox{\columnwidth}{!}{
\begin{tabular}{lllllll}
Source &V_2 & Temp.& Deixis & Verticality & Argument \\
?& \dhatu{-le(t)-}{go around doing X} &C& \textsc{circum} & $\varnothing$ & S/A \\
\dhatu{kʰot}{go} &\dhatu{-kʰo(t)-}{do X and go} &S& \textsc{trans} & $\varnothing$ &  S/A(+P) \\
?&\dhatu{-pɛ(t)-}{go and do X} &P& \textsc{trans} & $\varnothing$ & S/A \\
\dhatu{ɦo}{come} &\dhatu{-ɦo(t)-}{do X and come} &S& \textsc{cis} & $\varnothing$ & S/A(+P) \\
\dhatu{kʰoŋ}{come up}&\dhatu{-kʰoŋ-}{do X and come up} &S& \textsc{cis} &up & S/A \\
 \dhatu{pi}{come (same level)}&\dhatu{-pi(t)-}{come (same level)} &S/P& \textsc{cis} &same level & S/A(+P) \\
\dhatu{tɛn}{fall} & \dhatu{-tɛ(nt)-}{do X and come/bring down} &S/P & \textsc{cis} &down & S/A(+P) \\
\end{tabular}}
  \end{frame}  

 \begin{frame} 
\frametitle{Sinitic}


 \end{frame}  


\begin{frame} 
\frametitle{Other languages}
\begin{enumerate}
\item<1-> Karbi (NE India, \citealt{konnerth15cisloc}): prior cislocative motion (\rouge{nang}= \textsc{come\&do})
\begin{exe}
\ex 
\gll alàng-lì=ke là-tūm a-hēm=si \rouge{nang}=vùr-si sá aját \rouge{nang}=jùn-lò \\
 3-\textsc{hon}=\textsc{top} this-\textsc{pl} \textsc{poss}-house=\textsc{foc} \rouge{\textsc{cis}}=drop.in-\textsc{nf}:\textsc{rl} tea \textsc{genex} \rouge{\textsc{cis}}=drink-\textsc{rl} \\
 \glt `...it was him, at their house we stopped by and had tea [\rouge{come} and drink tea] and everything.'
\end{exe}
 \item<2-> Tangsa (NE India, \citealt[311-312]{boro17hakhun})
 \begin{exe}
\ex 
\gll   inɤ́ ʒuk \rouge{kà} l-oʔ \\
there drink \rouge{go} \textsc{imp}-\textsc{2sg} \\
\glt ‘\rouge{Go} and drink (tea) there.’
\end{exe}
  \item<3->   \citet{Jenny16grammar}
   \end{enumerate}
\end{frame}  

\begin{frame} 
\frametitle{Summary of AM systems in ST}
\resizebox{\columnwidth}{!}{
\begin{tabular}{llllllllll}
Subgroup&Language  &Number &Temp. & Deixis & Vertical & Other & Mono- &Arg.\\
&&&relation &&dimension&functions& actional & \\
Gyalrong&Japhug & 2&P & Cis/Trans & $\varnothing$ &  $\varnothing$ & Y &S/A \\
&Situ &2&P & Cis/Trans & $\varnothing$ & Aspect&N &S/A \\
Kiranti &Khaling & 7&P, C, S &Cis/Trans & Y & Aspect,  &  N  &S/A, S/A+P \\
&&&&&&Orientation&&\\
&Dumi & 4? &P, C, S &Cis/Trans &  $\varnothing$ & Aspect,  &  ?&S/A, S/A+P \\
  &&&&&&Orientation&&\\
&Thulung & 1 & C  &Circum &  $\varnothing$ & $\varnothing$ &?  &S/A  \\
&Wambule & 5 & P  &Cis/Trans &  Y & Aspect,  &?  &S/A, S/A+P \\
     &&&&&&Orientation&&\\
%    &Jero & 2 & C, P?  & Trans? &  \Y & ?  &?  &S/A  \\
&Yamphu & 7 & P, C, S  & Cis/Trans &  Y & Aspect,  &?  &S/A, S/A+P \\
        &&&&&&Orientation&&\\
&Yakkha & 7? & P, C, S  & Cis/Trans &  Y & Aspect,  &?  &S/A, S/A+P \\
        &&&&&&Orientation&&\\
&Belhare & 3? & P, C, S  & Cis/Trans &  Y & Aspect,  &?  &S/A, S/A+P \\
        &&&&&&Orientation&&\\
&Hayu & 1  & P  & Trans &  N & Aspect,  &?  &S/A  \\
        &&&&&&Voice&&\\
Sinitic & Mandarin &2& P & Cis/Trans & $\varnothing$ &Aspect,  Modality& N   &S/A  \\
        &&&&&&Orientation&&\\
Karbi&&1&S&Cis&  $\varnothing$ &Person, Orientation&?  &S/A+P \\
Sal&Tangsa&1&P&Trans&  $\varnothing$ & ?&?  &S/A+P  \\
 \end{tabular}}
 
\end{frame}  
 
\begin{frame} 
\frametitle{Suggestions for fieldworkers}

\begin{enumerate}
\item Illustrate AM with verbs that do not imply intrinsic motion (motion verbs, verbs of manipulation) -- otherwise, impossible to distinguish between AM and orientation.
\item Test for mono- vs pluri-actionality (negation, concessives, interrogatives, conditions, complement clauses)
\item For labile AM markers, specify whether P is an argument of motion or not.
\end{enumerate}

\end{frame}  

\begin{frame} 
\frametitle{Conclusion}
\begin{enumerate}
\item<1->  AM is found in most major branches of ST, and widely distributed geographically.
\item<2->  Independent grammaticalization from motion verbs (except perhaps in Karbi).
\item<3->  Highly diverse systems  (from 1 to 7 markers). The most grammaticalized systems are found in the branches with the most complex morphology.
\end{enumerate}
\end{frame}  

\begin{frame} 
\frametitle{References} 
\tiny
\bibliographystyle{unified}
\bibliography{bibliogj}
\end{frame}  

\end{document}