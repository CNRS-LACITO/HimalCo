\documentclass[oneside,a4paper,11pt]{article} 
\usepackage{fontspec}
\usepackage{natbib}
\usepackage{booktabs}
\usepackage{xltxtra} 
\usepackage{polyglossia} 
\usepackage[table]{xcolor}
\usepackage{gb4e} 
\usepackage{multicol}
\usepackage{graphicx}
\usepackage{float}
\usepackage{hyperref} 
\hypersetup{bookmarksnumbered,bookmarksopenlevel=5,bookmarksdepth=5,colorlinks=true,linkcolor=blue,citecolor=blue}
\usepackage[all]{hypcap}
\usepackage{memhfixc}
\usepackage{lscape}
\usepackage{amssymb}
 
%\bibpunct[: ]{(}{)}{,}{a}{}{,}

%\setmainfont[Mapping=tex-text,Numbers=OldStyle,Ligatures=Common]{Charis SIL} 
\newfontfamily\phon[Mapping=tex-text,Scale=MatchLowercase]{Charis SIL} 
\newcommand{\ipa}[1]{\textbf{{\phon\mbox{#1}}}} %API tjs en italique
%\newcommand{\ipab}[1]{{\scriptsize \phon#1}} 

\newcommand{\grise}[1]{\cellcolor{lightgray}\textbf{#1}}
\newfontfamily\cn[Mapping=tex-text,Ligatures=Common,Scale=MatchUppercase]{SimSun}%pour le chinois
\newcommand{\zh}[1]{{\cn #1}}

\XeTeXlinebreakskip = 0pt plus 1pt %
 %CIRCG
 
\begin{document}

\title{Stem alternations in Kuki-Chin}
\author{Guillaume Jacques}
\maketitle

\section*{Introduction}

\section{Function}
Hakha Lai verbs have two stems (I and II); stem I is obligatory with negative and interrogative markers, stem II obligatorily occurs in some subordinate clause, but in affirmative indicative main clauses, stem alternation is determined by transitivity: intransitive verbs have stem I, while transitive verbs have stem II when the A takes the ergative marker \ipa{=niʔ}, as in example (\ref{ex:abaq}) (\citealt[413]{peterson03hakha})

\begin{exe}
\ex \label{ex:abaq}
\gll \ipa{paalaw=niʔ} \ipa{thil} \ipa{khaaʔ} \ipa{ʔa-baʔ} \\
p.n=\textsc{erg} clothes \textsc{dem} \textsc{3sg}-hang.up:II \\
\glt `Paalaw hung up the clothes.' (Hakha Lai)
\end{exe}

Transitive verbs can also be used in affirmative independent clauses in Stem I, as in example (\ref{ex:abat}). In this case, the A does not take ergative case. This is the construction which \citet{kathol01alternations} analyze as antipassive.

\begin{exe}
\ex \label{ex:abat}
\gll \ipa{paalaw} \ipa{khaaʔ}  \ipa{thil} \ipa{ʔa-bat} \\
p.n \textsc{dem} clothes \textsc{3sg}-hang.up:I \\
\glt `Paalaw hangs up/hung up the clothes.' (Hakha Lai)
\end{exe}

In this construction, stem alternation is not by itself a mark of voice derivation. Since intransitive verbs occur with stem I in affirmative independent clauses, stem alternation between examples (\ref{ex:abaq}) and (\ref{ex:abat}) rather reflects the same verb stem conjugated transitively and intransitively respectively, ie agent-preserving lability, and thus not antipassive proper according to the definition proposed in this paper.\footnote{Note also that the object of the transitive construction is not demoted to oblique status in the detransitive construction in (\ref{ex:abat}), an observation that \citet[413]{peterson03hakha} uses as argument against the antipassive analysis. \citet[37]{peterson07appl} explicitly states that `Hakha Lai has no valence-affecting constructions which target objects, such as passive or antipassive.'} 

\section{Form}

\citet{vanbik09pkc}
\citet{chelliah07lamkang}
\citet{hartmann09grammar}
\citet{mang06kcho}

\section{Possible origins}

\subsection{Stem alternation}
\citet{jackson00sidaba}, \citet{jackson00puxi}, \citet{jackson04showu}

\citet{jacques08}

\citet{lai17khroskyabs}

\citet{sun16cigan}


\citet{jacques12khaling}

\subsection{Nominalization}

\citet[625]{hill14voicing}
\citet{jacques16ssuffixes}

\subsection{Tangsa}

\citet{morey17tangsa}

\section*{Conclusion}

\bibliographystyle{unified}
\bibliography{bibliogj}
\end{document}
