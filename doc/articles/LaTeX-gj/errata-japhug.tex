\documentclass[oldfontcommands,oneside,a4paper,11pt]{article} 
\usepackage{fontspec}
\usepackage{natbib}
\usepackage{booktabs}
\usepackage{xltxtra} 
\usepackage{polyglossia} 
\usepackage[table]{xcolor}
\usepackage{gb4e} 
\usepackage{multicol}
\usepackage{graphicx}
\usepackage{float}
\usepackage{hyperref} 
\hypersetup{bookmarks=false,bookmarksnumbered,bookmarksopenlevel=5,bookmarksdepth=5,xetex,colorlinks=true,linkcolor=blue,citecolor=blue}
\usepackage[all]{hypcap}
\usepackage{memhfixc}
\usepackage{lscape}

\bibpunct[: ]{(}{)}{,}{a}{}{,}

%\setmainfont[Mapping=tex-text,Numbers=OldStyle,Ligatures=Common]{Charis SIL} 
\newfontfamily\phon[Mapping=tex-text,Ligatures=Common,Scale=MatchLowercase,FakeSlant=0.3]{Charis SIL} 
\newcommand{\ipa}[1]{{\phon \mbox{#1}}} %API tjs en italique
\newcommand{\ipab}[1]{{\scriptsize \phon#1}} 

\newcommand{\grise}[1]{\cellcolor{lightgray}\textbf{#1}}
\newfontfamily\cn[Mapping=tex-text,Ligatures=Common,Scale=MatchUppercase]{MingLiU}%pour le chinois
\newcommand{\zh}[1]{{\cn #1}}
\newcommand{\refb}[1]{(\ref{#1})}

\newcommand{\ra}{$\Sigma_1$} 
\newcommand{\rc}{$\Sigma_3$} 
\newcommand{\ro}{$\Sigma$} 

\XeTeXlinebreaklocale 'zh' %使用中文换行
\XeTeXlinebreakskip = 0pt plus 1pt %
 %CIRCG
 
\sloppy

\begin{document} 
\title{Errata of articles on Japhug grammar}
\author{Guillaume Jacques}
\maketitle

\section{\citet{jacques13harmonization}, Verbal template and associated motion}
\begin{enumerate}
\item \citet[199]{jacques13harmonization}: `Second, in the suffixal chain, the 1sg occurs closer to the stem than the past tense marker.' : Please disregard this sentence, actually the \textsc{1sg} is closer to the verb stem (I probably had in mind instead the evidential \ipa{--ci} suffix, which occurs after all person markers, as in \ipa{pjɤ-k-ɤsɯ-ndza-nɯ-ci} \textsc{ifr-evd-prog}-eat-\textsc{pl-evd} `they were eating it'. 

\end{enumerate}

\section{\citet{japhug14ideophones}, Ideophones}
\begin{enumerate}
\item \citet[273, ex. 27]{japhug14ideophones} \ipa{ɬɯɣnɤlɯɣ} $\Rightarrow$ \ipa{ɬɯɣnɤɬɯɣ}
\begin{exe}
\ex \label{ex:lhWG.nA.lhWG}
\gll
\ipa{ɯ-sŋɯro} 	\ipa{lu-lɤt} 	\ipa{tɕe} 	\ipa{ɬɯɣnɤɬɯɣ,} 	\ipa{\textbf{ɬɯɣnɤɬɯɣ}} 	\textsc{\ipa{tu-ste}} 	\ipa{ɲɯ-ŋu} \\
\textsc{3sg.poss}-breath \textsc{ipfv:upstream}-throw \textsc{lnk} \textsc{ideo:dyn}:breathing.movement \textsc{ideo:dyn}:breathing.movement \textsc{ipfv}-do.like[III] \textsc{testim}-be \\
\glt `When it breathes, (one can see its body) expanding and retracting with each breath.' (Frog, 3)
\end{exe}

\item p. 287. 

Caodang   $\Rightarrow$ Caodeng


Shidanlua   $\Rightarrow$ Shidanluo
\end{enumerate}
\section{\citet{jacques14linking}, Clause linking}
\begin{enumerate}


\item \citet[284, ex. 33]{jacques14linking}; The word \ipa{ɯ-pɤl}  its palm' was wrongly translated as  handle'; in fact, it refers to the blade of this implement, the part used to break earth clots.


\begin{exe}
\ex \label{ex:CkrAz.tAme}
\gll 
   	\ipa{sɯmpʰɯ}  	\ipa{ɯ-pɤl,}  	\ipa{ɕkrɤz}  	\ipa{tɤ-me}  	\ipa{tɕe}  	\ipa{nɯnɯ}  	\ipa{xɕɤj}  	\ipa{ɲɯ-βzu-nɯ}  	\ipa{sna.}  \\
   	tool.for.breaking.earth.clods \textsc{3sg.poss}-palm oak \textsc{pfv}-not.exist \textsc{lnk} \textsc{dem} tree.species \textsc{ipfv}-make-\textsc{pl} be.appropriate:\textsc{fact} \\
\glt The blade of the earth clod breaker, when/if there is no oak wood, people can also make it using the \ipa{xɕɤj} wood (\ipa{xɕɤj}, 44).
\end{exe}
\end{enumerate}


\bibliographystyle{unified}
\bibliography{bibliogj}
\end{document}