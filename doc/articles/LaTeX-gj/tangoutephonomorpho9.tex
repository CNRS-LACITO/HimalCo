\documentclass[oldfontcommands,twoside,a4paper,11pt,draft]{memoir} 
%\usepackage{xunicode}%packages de base pour utiliser xetex
%\usepackage{fancyhdr}
%\usepackage[top=2in, bottom=1.5in, left=1in, right=1in]{geometry}
%\usepackage[a4paper,text={110mm,190mm},includehead,centering]{geometry}

\usepackage{fontspec}
\usepackage{natbib}
\usepackage{booktabs}
\usepackage{xltxtra} 
\usepackage{longtable}
\usepackage{tangutex2} 
\usepackage{tangutex4} 
\usepackage{polyglossia} 
\setdefaultlanguage{french} 
\usepackage[table]{xcolor}
\usepackage{newgb4e} 
\usepackage{multicol}
\usepackage{graphicx}
\usepackage{float}
%\usepackage{hyperref} 
 \usepackage{idxlayout}
\usepackage{lscape}
%\hypersetup{bookmarks=false,bookmarksnumbered,bookmarksopenlevel=5,bookmarksdepth=5,xetex,colorlinks=true,linkcolor=blue,citecolor=blue}
%\usepackage[all]{hypcap}
\usepackage{memhfixc}
\usepackage{newicktree}
\bibpunct[~: ]{(}{)}{,}{a}{}{,}

%\pagestyle{empty}
%\fancyhead{}
%\fancyfoot{}
%\fancyhead[LE,RO]{\thepage}
%\fancyhead[CO]{\leftmark}
%\fancyhead[CE]{\textsc{Esquisse de phonologie et de morphologie}}
%\fancyhead[C]{\chaptermark}
%\fancyhead[CO]{Esquisse}
%\lfoot{}
%\cfoot{\thepage}
%\rfoot{}
%\renewcommand{\headrulewidth}{0pt}
%\setvalue[section][indentnext=no]

\makeoddhead{headings}{}{\small\leftmark}{\small\thepage}
\makeevenhead{headings}{\small\thepage}{\textsc{esquisse de phonologie et de morphologie}}{}
\addtopsmarks{headings}{}{\createmark{chapter}{left}{nonumber}{}{}}
\pagestyle{headings}

\makeoddfoot{plain}{}{}{}
\makeevenfoot{plain}{}{}{}

\setsecheadstyle{\SingleSpacing\LARGE\scshape\raggedright\MakeLowercase}
\setsubsecheadstyle{\SingleSpacing\Large\itshape\raggedright}
\setsubsubsecheadstyle{\SingleSpacing\itshape\raggedright}
\setsecnumdepth{subsubsection}
%\settypeblocksize{190mm}{110mm}{*}
  \renewcommand{\thetable}{ \arabic{table}}
\sloppy

\makechapterstyle{global}{%
\setlength{\beforechapskip}{0pt}
\renewcommand*{\printchaptername}{\centering}
\renewcommand*{\chapnumfont}{\Large}
\renewcommand*{\chaptitlefont}{\LARGE\scshape}
\renewcommand{\printchaptertitle}[1]{%
\centerline{\chaptitlefont\MakeUppercase{##1}}}%
}

\chapterstyle{global}

\setmainfont[Mapping=tex-text,Numbers=OldStyle,Ligatures=Common]{Charis SIL} 

\SetSymbolFont{letters}{normal}{\encodingdefault}{\rmdefault}{m}{rm}
\setmathrm{Charis SIL}
\newfontfamily\phon[Mapping=tex-text,Ligatures=Common,Scale=MatchLowercase,FakeSlant=0.3]{Charis SIL} 
\newfontfamily\phondroit[Mapping=tex-text,Ligatures=Common,Scale=MatchLowercase]{Charis SIL} 
\newcommand{\ipa}[1]{{\phon #1}} %API tjs en italique
\newcommand{\ipapl}[1]{{\phondroit #1}} 
\newfontfamily\cn[Mapping=tex-text,Ligatures=Common,Scale=MatchUppercase]{MingLiU}%pour le chinois
\newcommand{\zh}[1]{{\cn #1}}
\newcommand{\captionb}[1]{\caption{\textsc{#1}}} 
\newcommand{\racine}[1]{\begin{math}\sqrt{#1}\end{math}} 
\newcommand{\petit}[1]{\scriptsize  #1}
%\newcommand{\ipab}[1]{{\tiny \phon#1}} 
\newcommand{\grise}[1]{\cellcolor{lightgray}\textbf{#1}}
\newfontfamily\mleccha[Mapping=tex-text,Ligatures=Common,Scale=MatchLowercase]{Galatia SIL}%pour le grec
\newcommand{\grec}[1]{{\mleccha #1}}
\newcommand{\tgz}[1]{#1 \mo{#1} \tg{#1}}
\newcommand{\indextg}[1]{\index{Tangoute!\tge{#1}@\mo{#1} \tg{#1}}}
\newcommand{\tgb}[1]{\tgz{#1}\indextg{#1}}
\newcommand{\tgc}[1]{\tg{#1} #1\indextg{#1}}
\newcommand{\tgd}[1]{\tge{#1}\indextg{#1}}
\newcommand{\tgf}[1]{\mo{#1}\indextg{#1}}
\newcommand{\tinynb}[1]{\tiny#1}

\newcommand{\plb}[2]{proto-LB *#1 #2\index{LB!#1}\index{LB!#2 #1}}
\newcommand{\tib}[1]{tibétain \ipa{#1}\index{Tibétain!#1}}
\newcommand{\jpg}[1]{japhug \ipa{#1}\index{Japhug!#1}}
\newcommand{\situ}[1]{situ \ipa{#1}\index{Situ!#1}}
\newcommand{\pumi}[1]{pumi \ipa{#1}\index{Pumi!#1}}
\newcommand{\naxi}[1]{naxi \ipa{#1}\index{Naxi!#1}}
\newcommand{\nayn}[1]{na \ipa{#1}\index{Na!#1}}
\newcommand{\laze}[1]{lazé \ipa{#1}\index{Lazé!#1}}
\newcommand{\protona}[1]{proto-na *#1\index{Proto-Na!#1}}
\newcommand{\bir}[1]{birman \ipa{#1}\index{Birman!#1}}
\newcommand{\ptang}[2]{(pré-tangoute *#1) ``#2''\index{*#1 ``#2''}}

\XeTeXlinebreaklocale "zh" %使用中文换行 
\XeTeXlinebreakskip = 0pt plus 1pt % 

\newfontfamily\qidan{QIDAN}
\newcommand{\khitan}[1]{{\huge \qidan #1}} 
\newcommand{\acc}{\textsc{acc}}
\newcommand{\antierg}{\textsc{gen}}
\newcommand{\allat}{\textsc{all}}
\newcommand{\aor}{\textsc{aor}}
\newcommand{\assert}{\textsc{assert}}
\newcommand{\auto}{\textsc{auto}}
\newcommand{\caus}{\textsc{caus}}
\newcommand{\classif}{\textsc{class}}
\newcommand{\concessif}{\textsc{concsf}}
\newcommand{\comit}{\textsc{comit}}
\newcommand{\conj}{\textsc{conj}}
\newcommand{\const}{\textsc{const}}
\newcommand{\conv}{\textsc{conv}}
\newcommand{\cop}{\textsc{cop}}
\newcommand{\cisl}{\textsc{cisl}}
\newcommand{\dat}{\textsc{dat}}
\newcommand{\dem}{\textsc{dem}}
\newcommand{\dir}{\textsc{dir1}}
\newcommand{\du}{\textsc{du}}
\newcommand{\duposs}{\textsc{du.poss}}
\newcommand{\dur}{\textsc{dur}}
\newcommand{\erg}{\textsc{erg}}
\newcommand{\fut}{\textsc{fut}}
\newcommand{\gen}{\textsc{gen}}
\newcommand{\hypot}{\textsc{hyp}}
\newcommand{\ideo}{\textsc{ideo}}
\newcommand{\imp}{\textsc{imp}}
\newcommand{\impf}{\textsc{impf}}
\newcommand{\instr}{\textsc{instr}}
\newcommand{\intens}{\textsc{intens}}
\newcommand{\intrg}{\textsc{intrg}}
\newcommand{\inv}{\textsc{inv}}
\newcommand{\irreel}{\textsc{irr}}
\newcommand{\loc}{\textsc{loc}}
\newcommand{\med}{\textsc{med}}
\newcommand{\negat}{\textsc{neg}}
\newcommand{\neu}{\textsc{indef.poss}}
\newcommand{\nmls}{\textsc{nmls}}
\newcommand{\nonps}{\textsc{n.ps}}
\newcommand{\opt}{\textsc{dir2}}
\newcommand{\pass}{\textsc{pass}}
\newcommand{\perf}{\textsc{prf}}
\newcommand{\pl}{\textsc{pl}}
\newcommand{\plposs}{\textsc{pl.poss}}
\newcommand{\poss}{\textsc{poss}}
\newcommand{\pot}{\textsc{pot}}
\newcommand{\prohib}{\textsc{prohib}}
\newcommand{\ps}{\textsc{pst}}
\newcommand{\recip}{\textsc{recip}}
\newcommand{\redp}{\textsc{redp}}
\newcommand{\refl}{\textsc{refl}}
\newcommand{\sg}{\textsc{sg}}
\newcommand{\sgposs}{\textsc{sg.poss}}
\newcommand{\stat}{\textsc{stat}}
\newcommand{\topic}{\textsc{top}}
\newcommand{\transl}{\textsc{transl}}
\newcommand{\volit}{\textsc{vol}}

\makeindex 

%\setcounter{tocdepth}{3}
%\setcounter{secnumdepth}{3}

\newlength{\malongueur}
\setlength{\malongueur}{-2\baselineskip}
\addtolength{\malongueur}{-7mm}
\addtolength{\malongueur}{190mm}

\settypeblocksize{\malongueur}{110mm}{*}
\setheadfoot{\baselineskip}{\baselineskip}
\setlrmargins{*}{*}{1}
\setheaderspaces{*}{7mm}{*}
\setlength{\textwidth}{110mm}
%\setulmargins{*}{*}{0.7}
\checkandfixthelayout
\begin{document}

\pagenumbering{roman}
\setcounter{page}{5}
\thispagestyle{empty}
 \tableofcontents
\chapter*{Remerciements}
\thispagestyle{empty}
Ce travail n'aurait jamais vu le jour sans l'aide de nombreux collègues et amis dont l'assistance et le soutien ont été précieux: 

 Chenzhen et Ngag-dbang, qui m'ont appris le japhug et le pumi respectivement.
 
  Nie Hongyin, Vyacheslav Zaytsev, Marc Miyake, Lin Ying-chin et Kirill Solonin avec lesquels j'ai entretenu de fructueux échanges épistolaires concernant la philologie tangoute.

  David Bradley, qui a relu intégralement ce travail et m'a suggéré de nombreuses idées.

  Juha Janhunen, Evgeni Kychanov et Arakawa Shintaro  qui m'ont gracieusement offert plusieurs ouvrages et articles de tangoutologie.
  
   Sasha Vovin, qui a accepté cet ouvrage dans la collection ``Languages of Asia" et dont les travaux sur les langues ``altaïques", en particulier le khitan, ont nourri ma réflexion sur les contacts de langues en Haute Asie.
 
  Boyd Michailovsky, Martine Mazaudon, Nicolas Tournadre,  Jim Matisoff, Scott DeLancey, Stephen Morey, Mark Post, George van Driem, Randy LaPolla, Katia Chirkova, Nathan Hill, Lin Youjing et Jackson T.-S. Sun, dont j'ai bénéficié des conseils éclairés dans le domaine de la linguistique comparée et descriptive des langues sino-tibétaines.
 
   Thomas Pellard, qui  a conçu un système pour traiter le tangoute sous \LaTeX{}.
 
  Alexis Michaud et Anton Antonov, avec lesquels j'ai collaboré à plusieurs articles communs, qui  m'ont aidé à corriger   ce travail en détail et dont les commentaires ont enrichi ma réflexion.

   Laurent Sagart, grâce auquel j'ai pu apprendre la phonologie historique du chinois qui est à la base de la reconstruction du tangoute,  qui m'a guidé tout au long de ces années dans mon étude de la linguistique historique des langues sino-tibétaines et qui a attentivement relu le manuscrit à la base de cet ouvrage.
\section*{Liste des abréviations}
 Liste des abréviations:
 \begin{multicols}{2}
\begin{itemize}
 \item \acc{} : accusatif
\item \antierg{} : antiergatif
\item \allat{} : allatif
\item \aor{} : aoriste
\item \assert{} : assertif
\item \auto{} : autobénéfactif / spontanné
\item \caus{} : causatif
\item \classif{} : classificateur
\item \concessif{} : concessif
\item \comit{} : comitatif
\item \conj{} : conjonctiong
\item \const{} : constatif
\item \conv{} : converb
\item \cop{} : copule
\item \cisl{} : cislocatif
\item \dat{} : datif
\item \dem{} : démonstratif
\item \dir{} : directionnel
\item \du{} : duel
\item \dur{} : duratif
\item \erg{} : ergatif
\item \fut{} : futur
\item \gen{} : génitif
\item \hypot{} : hypothétique
\item \ideo{} : idéophone
\item \imp{} : impératif
\item \impf{} : imperfectif
\item \instr{} : instrumental
\item \intens{} : intensif
\item \intrg{} : interrogatif
\item \inv{} : inverse
\item \irreel{} : irréel
\item \loc{} : locatif
\item \med{} : médiatif
\item \negat{} : négation
\item \neu{} : possesseur indéfini/générique
\item \nmls{} : nominalisation
\item \nonps{} : non passé
\item \opt{} : optatif
\item \pass{} : passif
\item \perf{} : perfectif
\item \pl{} : pluriel
\item \poss{} : possessif
\item \pot{} : potentiel
\item \prohib{} : prohibitif
\item \ps{} : passé
\item \recip{} : réciproque
\item \redp{} : réduplication
\item \refl{} : réfléchi
\item \sg{} : singulier
\item \stat{} : statif
\item \topic{} : topique
\item \transl{} : translocatif
\item \volit{} : volitionnel
\end{itemize}
  \end{multicols}


\chapter{Introduction} 
\thispagestyle{empty}
\pagenumbering{arabic}
 \setcounter{page}{1}
\section{La place du tangoute dans la famille sino-tibétaine}
Le tangoute est l'une des langues les plus anciennement attestées de la famille sino-tibétaine: il   nous est connu par des textes datant du onzième au seizième siècle. Bien que cette langue ait disparu depuis plusieurs centaines d'années, elle est suffisamment bien documentée pour être utilisable aussi bien par la linguistique historique que par la typologie.

Il est généralement accepté par tous les spécialistes de la question que le tangoute appartient à la famille sino-tibétaine, et plus précisément à la sous-branche `qianguique'. C'est notamment le point de vue présenté dans \citet{lifw99bijiao}. 

L'hypothèse de l'existence d'une branche qianguique n'est pas universellement acceptée: aucune preuve formelle, basée sur des innovations communes, n'a pour le moment été présentée, et la place de la langue qiang est particulièrement difficile à préciser, car sa phonologie historique est mal connue. Nous montrerons toutefois dans la conclusion de cet ouvrage  que l'on peut trouver des innovations communes au rgyalrong, au tangoute et à quelques autres langues qui justifient de les placer dans un même groupe. Etant donné toutefois que le qiang lui-même est peu conservateur et que sa place exacte est incertaine, nous suggérons de la renommer `macro-rgyalronguique'. Les savants tibétains pourraient mieux accepter un tel terme que  `qianguique' qui cause beaucoup de crispation. 


%
\begin{newicktree}
  \small \label{fig:stammbaum}
  \setunitlength{15cm} \righttree \nobranchlengths \nodelabelformat{}
\drawtree{(((((Rtau:0.07,Dge-bshes:0.07,Stod-sde:0.07):0.15[Trehor],(Thugs-chen:0.07,Njo-rogs:0.07):0.15[Lavrung],(Japhug:0.07, Tshobdun:0.07, Showu:0.07,(Situ du N.:0.02,Situ du S.:0.02):0.05[Situ]):0.15[Rgyalrong]):0.12[Rgy.ique],(Tangoute:0.2,Pumi:0.2):0.14[PT],(Qiang du N.:0.07,Qiang du S.:0.07):0.27[Qiang],Queyu:0.34,Zhaba:0.34,Muya:0.34):0.08[MR],((Naxi:0.07,Na:0.07,Laze:0.07):0.1[Naish],Shixing:0.17,Namuyi:0.17):0.25[Naïque],(Ersu:0.07,Lizu:0.07,Tosu:0.07):0.35[Ersuique]):0.05,Lolo-birman:0.05):0.05[BQ];}
  \par %\scalebar[0.1]
\end{newicktree}

Nous définissons cette famille macro-rgyalronguique comme l'ancêtre commun du rgyalrong et du tangoute.

La figure \ref{fig:stammbaum} présente un possible Stammbaum pour le macro-rgyalronguique (noté MR pour raisons de place) au sein du groupe birmo-qianguique (ici BQ); les innovations communes détaillant certains des clades apparaissant dans ce schéma seront exposées au chapitre \ref{chap:classification}.

Comme la plupart des langues macro-rgyalronguiques actuelles,  le tangoute a subi une évolution phonologique radicale qui obscurcit considérablement les traces de morphologie ancienne. Une étude qui entreprendrait de retracer la genèse de la morphologie tangoute sans prendre en compte la phonologie historique serait donc vouée à l'échec. C'est pourquoi le présent travail propose tout d'abord un modèle de pré-tangoute basé sur la comparaison avec les langues les plus conservatrices de la famille, en particulier le tibétain et le rgyalrong japhug. Le choix de ces deux langues est partiellement arbitraire, puisqu'il s'agit des langues de la famille sino-tibétaine avec lesquelles l'auteur de ces lignes a le plus de familiarité, mais il est toutefois justifié. Le tibétain est la grande langue littéraire de la famille, son vocabulaire, extrêmement étendu, est bien connu par rapport à celui des autres langues ST. Même si le tibétain est relativement éloigné du tangoute, le rapport historique entre Tangoutes et Tibétains n'est plus à démontrer (\citealt[137-61]{kychanov68}), et le tangoute connait quelques emprunts au tibétain, qui seront mentionnés tout au long de ce travail.


En ce qui concerne les langues rgyalrongs, ce choix est motivé par le conservatisme de ces langues du point de vue phonologique et morphologique. Le conservatisme phonologique du groupe rgyalrong est évident, puisque ce sont les seules langues de la région (si l'on exclut le tibétain ancien) qui préservent la quasi-totalité des consonnes finales ainsi que les groupes occlusive+r/l. En ce qui concerne la morphologie, leur conservatisme est un sujet plus polémique (voir \citealt{jacques12agreement} et \citealt{delancey10agreement} sur cette question), mais certains archaïsmes sont indéniables, en particulier la préservation de la productivité du préfixe causatif \ipa{sɯ}-- et de la prénasalisation anticausative (voir les sections \ref{subsec:causatif} et \ref{subsec:anticausatif}).

Comme il est quasiment certain que le tangoute appartient au macro-rgyalronguique (ce sujet fera l'objet du dernier chapitre de cet ouvrage), les langues rgyalrong sont donc le choix privilégié pour le reconstruire. Naturellement, ce travail serait plus complet avec des données sur l'ensemble des quatre langues rgyalrong (zbu, situ, tshobdun, japhug), et non uniquement sur deux d'entre elles (japhug et situ), mais les données sur ces langues sont encore trop peu accessibles pour permettre ce type de recherche.\footnote{Pour le situ, on dispose de nombreux travaux, dont la grammaire de \citet{linxr93jiarong}, le dictionnaire de \citet{huangsun02} et l'article de \citet{youjing03zhuokeji}. Le zbu (ribu, showu) et le tshobdun (caodeng) font l'objet de travaux importants en syntaxe par Jackson Sun, en particulier \citet{jackson98morphology, jackson00sidaba,jackson00puxi,jackson02rentongdengdi,jackson04showu,jackson06guanxiju,jackson06paisheng,jackson07shangzhai,jackson07irrealis,sun12complementation,jackson13morpho}. Nous ne pouvons pas nous servir de façon systématique de ces données dans la reconstruction de la phonologie étant donné que les dictionnaires de zbu et de tshobdun ne sont pas encore parus, mais ils nous seront d'une grande utilité dans le second chapitre sur la morphologie.}


Outre ces deux langues principales, nous ferons usage d'autres langues proches, telles que le pumi (\citealt{lusz01pumi} et notes de terrains), les autres langues macro-rgyalronguiques sur lesquelles on dispose de données fiables, les langues Na (\citealt{boydalexis06}, \citealt{michaud06neutralisation}, \citealt{michaud08yn}, \citealt{jacques.michaud11naish}) et le proto-lolo-birman tel qu'il est reconstruit par Matisoff et Bradley; on le citera ici essentiellement en suivant \citet{bradley79}, ouvrage qui est d'un usage commode. Il n'a pas été possible d'intégrer les résultats de la reconstruction du proto-ersu (\citealt{yu12ersuic}), car notre travail était déjà quasiment terminé lorsque cette thèse a été soutenue.


 Par ailleurs, nous mentionnerons les cognats dans deux langues plus éloignées, le chinois (dans la reconstruction \citealt{bs13oc})\footnote{Le chinois moyen est noté dans la transcription de \citet{baxter92} avec des modifications de \citealt{bs13oc}.} et occasionnellement le jingpho, en s'appuyant sur des comparaisons tirées de \citet{matisoff03} et \citet{peiros96st} (voir \citealt{xu83jingpo}), afin de bénéficier d'une perspective plus large sur l'histoire de la langue tangoute.

%\section{Sources textuelles}
La phonologie synchronique du tangoute des XI-XII^{e} siècles a fait l'objet de nombreux travaux depuis le début du vingtième siècle. Ce n'est pas là l'objet principal de notre recherche, mais celle-ci s'inscrit néanmoins dans cette tradition. On dispose de cinq sources de données principales:
\begin{enumerate}
\item Les transcriptions chinoises du tangoute dans le \textit{Fānhàn héshí zhǎngzhōngzhū} \zh{番漢合時掌中珠} `La perle dans la paume' (ZZZ), un dictionnaire bilingue tangoute-chinois qui indique la prononciation du tangoute au moyen de caractères chinois. Outre le ZZZ, on trouve également des transcriptions isolées mais plus anciennes dans certains textes historiques tels que le \textit{xù zīzhì tōngjiàn} \zh{續資治通鑒}. Les données du ZZZ sont présentées dans un format accessible dans l'ouvrage de \citet{lifw94zzz}.
\item Les transcriptions tangoutes du chinois. L'ensemble du vocabulaire chinois du ZZZ est transcrit en tangoute, mais l'on retrouve également  dans la quasi-totalité des textes tangoutes traduits du chinois des transcriptions  de noms de personnes, de lieux, voire de certains mots difficiles à traduire.
\item Les transcriptions tibétaines du tangoute (\citealt{nevskij26}, \citealt{nie86qianjiazi}, \citealt{arakawa99}, \citealt{tai08duiyin}). On peut inclure dans ces données certains noms tangoutes apparaissant dans des textes purement tibétains (Jacques 2008b).
\item Les transcriptions tangoutes du sanskrit, principalement dans les textes bouddhiques (\citealt{arakawa97}).
\item Les \ipa{fǎnqiè} \zh{反切} (voir section \ref{sec:notions} pour l'explication de ce terme) des dictionnaires de rime tangoutes. Ces dictionnaires de rimes sont habituellement connus dans un nom retraduit en chinois: le \textit{Wénhǎi} \zh{文海} `La mer des caractères', le \textit{Tóngyīn} \zh{同音} `Les homophones', le \textit{Wénhǎi zálèi} \zh{文海雜類} `Les catégories mixtes' ainsi qu'un dictionnaire sans titre (\citealt[85-89]{sofronov68a}). A cela, on peut ajouter les rimes dans la poésie en tangoute, qui apporte quelques indices importants sur la proximité phonétique de certaines rimes (\citealt{arakawa01}).
\end{enumerate}
Plusieurs reconstructions du tangoute ont été proposées, basées en premier lieu sur le système des \ipa{fǎnqiè} et les transcriptions chinoises:
\begin{enumerate}
\item \citet{hashimoto63tongju}
\item Nishida Tatsuo (1964)
\item \citet{sofronov68a}
\item Gong Hwangcherng (de nombreux articles regroupés dans son ouvrage \citealt{gong02a})
\item Li Fanwen (1986, \citealt{lifw94zzz}, \citealt{lifw97})
\item Arakawa Shintarô (\citealt{arakawa01}, 2006)
\end{enumerate}
Ces différents systèmes présentent certaines divergences, mais sont fondamentalement compatibles les uns avec les autres, étant basés sont le même ensemble de données. Comme certaines distinctions dans les \ipa{fǎnqiè} ne sont pas représentées dans toutes ces reconstructions,\footnote{Par exemple, Gong Hwangcherng ne distingue pas dans sa reconstruction la différence entre les rimes 1 et 4, 2 et 3, 6 et 7, 10 et 11, 19 et 20, 21 et 24, 30 et 31, 36 et 37, 46 et 47. Voir le tableau en annexe.}  il est d'usage de citer également le numéro de la rime du caractère en question. A partir de ce numéro, il est possible de retranscrire mécaniquement d'un système à l'autre. Le problème est beaucoup plus délicat avec le système des consonnes initiales, car les différents systèmes n'analysent pas les caractères notant l'initiale du \textit{fǎnqiè} de la même façon. 

	Le présent travail fera usage de la reconstruction de Gong Hwangcherng. Cette reconstruction est ici employée davantage comme une représentation alphabétique des informations contenues dans les \ipa{fǎnqiè}. Quels que soient les mérites des divers systèmes de reconstruction, aucun ne peut prétendre reproduire la prononciation exacte du tangoute du douzième siècle. Néanmoins, du point de vue de la linguistique historique, à partir du moment où toutes les distinctions présentes dans les \ipa{fǎnqiè} sont authentiques, il est possible de les utiliser pour établir des correspondances phonétiques avec d'autres langues, même si certains aspects du détail phonétique restent mystérieux.
	
	La comparaison du tangoute avec d'autres langues sino-tibétaines est une sixième source de données importante qui peut permettre de reconstruire ``de l'amont'' la phonologie de cette langue et compléter les cinq sources mentionnées ci-dessus. Néanmoins, nous ne proposerons pas un nouveau système de reconstruction du tangoute: nous servirons simplement d'autres systèmes basés sur les données des transcriptions et des \ipa{fǎnqiè} ; ainsi, le risque de la circularité (modifier une reconstruction pour faciliter les comparaisons externes) peut être évité. 
Les essais de comparaison ont commencé dès les années trente du vingtième siècle, avec les travaux de Wang Jingru et le dictionnaire de Nicolas Nevsky (\citealt{nevskij60}), mais à présent que les langues modernes de la famille macro-rgyalronguique sont mieux connues, en particulier les langues rgyalronguiques, il est devenu moins difficile d'étudier de façon détaillée la phonologie du tangoute dans une perspective comparative. 


Dans ce travail, nous avons tenté d'être aussi exhaustif que possible en incluant toutes les formes comparables que nous avons pu rassembler. Naturellement, les quelques centaines de groupes de cognats que nous avons regroupés ne forment qu'une partie des étymons étymologisables\footnote{En dehors des emprunts chinois, qui sont le plus souvent transparents.}. Nous espérons que des travaux futurs contribueront à enrichir le nombre de comparaisons entre le tangoute et d'autres langues ST. Dans nos comparaisons, nous présentons dans la mesure du possible des exemples textuels tangoutes pour tous les étymons tangoutes comparés, surtout les verbes, ceci afin de nous assurer du sens réel de ces mots, qui peut être obscurci par une définition de dictionnaire. C'est principalement du recueil d'histoires ``La forêt des catégories'' que nos exemples sont tirés, (en chinois \zh{類林} \textit{Lèilín}, en tangoute \tgf{3017}\tgf{3890} \ipa{djịj¹bo¹} ; voir \citealt{kepping83} et \citealt{leilin90})  mais également des textes suivants:

\begin{enumerate}

\item  ``L'art de la guerre de Sunzi'' \zh{孫子兵法} \textit{Sūnzǐ Bīngfǎ}: \citet{lin94sunzi}
\item ``Nouveau recueil sur l'amour parental et la piété filiale'' \zh{新集慈孝傳} \textit{Cíxiàozhuàn} (\tgf{3457}\tgf{0478}\tgf{1483}\tgf{2323}\tgf{5404}\tgf{4625}\tgf{5302} \ipa{sjiw¹ɕioo¹	njij²wə¹la¹	mjiij² ?}): \citet{kepping90cixiaozhuan}, \citet{jacques07textes}
\item ``les Douzes Royaumes''  \zh{十二國} \textit{Shí'èrguó} (\tgf{1084}\tgf{4027}\tgf{2937} \ipa{ɣạ²njɨɨ¹lhjịj}): \citet{solonin95}, \citet{nie02shierguo}, \citet{sunyx03}
\item recueil de proverbes tangoutes: \citet{kychanov74}
\item ``Mer des significations établie par les saints'' \zh{聖立義海} \textit{Shènglì yìhǎi} (\tgf{2544}\tgf{0678}\tgf{3183}\tgf{0661} \ipa{\ipa{ɕjɨj²gu¹.wo²ŋjow²}}), une encyclopédie qui préserve des conceptions et des mythes indigènes: \citet{kychanov97}, \citet{kychanov95}
\end{enumerate}
On notera dans le corpus étudié la quasi-absence de textes bouddhiques, dont l'intérêt pour l'étude de la grammaire du tangoute est moins grand que celui des textes laïcs.

\section{Notions de phonologie  tangoute} \label{sec:notions}
Comme nous l'avons expliqué dans la section précédente, nous acceptons ici la reconstruction du Gong Hwangcherng, et nous ne proposerons pas une nouvelle reconstruction du tangoute. Toutefois, nous serons obligé dans ce travail de faire usage de termes techniques spécifiques à la phonologie historique du chinois et du tangoute, qui peuvent dérouter les spécialistes d'autres familles de langues.

Comme le chinois, le tangoute est écrit dans un système d'écriture qui révèle peu d'informations sur la prononciation effective des mots. La prononciation doit donc être \textit{reconstruite} au moyen des sources de données mentionnées dans la section précédente. Les sources externes (gloses chinoises et tibétaines) apportent des informations utiles mais peu systématiques et il est impossible de baser  le système de reconstruction sur ces données exclusivement.

Les Tangoutes ont imité les Chinois en compilant des dictionnaires de rimes tels que le \textit{Wénhǎi}, où chaque caractère a sa prononciation glosée au moyen de deux caractères, selon le système appelé \zh{反切} \textit{fǎnqiè}. En tangoute comme en chinois médiéval, à chaque caractère correspondait une syllabe, et aucun groupe de consonnes initial ne subsistait. Le premier caractère du \textit{fǎnqiè} indique la consonne initiale et le second la rime. Voici un exemple du fonctionnement des \textit{fǎnqiè} en tangoute:

 \tgz{5612} ``parler'' est glosé par \tgz{5870} ``réciter'' et \tgz{0218} ``plat''. On remarque qu'il partage la même consonne initiale \textit{tsh}-- avec le premier caractère, et la même rime -jiij avec le second. Ici 1.39 indique le numéro de la rime: 1 représente le ton, et 39 indique qu'il s'agit de la trente-neuvième rime au ton 1 selon l'ordre des dictionnaires tangoutes.
 
 L'information contenue dans un \textit{fǎnqiè} est partiellement redondante avec l'emplacement du caractère dans le dictionnaire: en effet, les caractères sont classés par rime, puis à l'intérieur de chaque rime par consonne initiale. Or, les consonnes initiales ne sont pas indiquées explicitement dans les dictionnaires, et sans les \textit{fǎnqiè}, on ne pourrait donc reconstruire  avec certitude que les catégories de rime.
 
En effet, si le dictionnaire indique la liste des rimes, on ne dispose pas de liste fiable des consonnes initiales (on ne trouve qu'un classement des initiales en neuf catégories). Par ailleurs, le problème majeur est qu'une même consonne initiale peut être représentée par plusieurs caractères, comme en chinois. Il faut donc trouver un moyen de déterminer, parmi tous les caractères qui servent à noter les consonnes initiales, lesquels représentent la même consonne et lesquels représentent une consonne différente.

 Pour cela, on étudie les chaînes de \textit{fǎnqiè}, selon une méthode également employée en chinois, et connue sous le nom de \ipa{xìliánfǎ} \zh{系聯法}. Ces chaînes peuvent s'appliquer aussi bien aux caractères indiquant les initiales qu'à ceux indiquant les rimes. Observons quelques exemples, en commençant par les indicateurs de rime.

On recherche le caractère indicateur de rime de \tgz{0218}, puis à nouveau le caractère indicateur de rime de celui-ci récursivement jusqu'à fermer la boucle\footnote{On n'indique pas ici le sens de ces caractères, qui n'a aucune importance pour expliquer le fonctionnement du système de notation.}:
\begin{enumerate}
\item \tgz{0218} > \tgz{3606} + \tgz{1398}
\item \tgz{1398} > \tgz{1447} + \tgz{2493}
\item \tgz{2493} > \tgz{0349} + \tgz{2135}
\item \tgz{2135} > \tgz{3606} + \tgz{1398}
\end{enumerate}
En à peine trois itérations, on retrouve le caractère \tgz{1398} qui avait déjà servi à gloser la rime de \tgz{0218}, son homophone. Appliquons la même procédure à la consonne initiale de  \tgz{5612}:
\begin{enumerate}
\item \tgz{5612} > \tgz{5870} + \tgz{0218}
 \item \tgz{5870} > \tgz{3974} + \tgz{1696}
\end{enumerate}
Ici, la chaîne est coupée car on ne dispose que de la première moitié du dictionnaire \textit{Wénhǎi}: la section des mots au ton 2 est perdue et il est nécessaire dans ces cas de compléter avec un autre dictionnaire. Cette procédure complexe ne sera pas présentée ici. Observons maintenant une chaîne que l'on peut boucler:
\begin{enumerate}
\item \tgz{5890} > \tgz{1073} + \tgz{0866}
 \item \tgz{1073} > \tgz{2599} + \tgz{0515}
  \item \tgz{2599} > \tgz{3732} + \tgz{0866}
   \item \tgz{3732} > \tgz{2599} + \tgz{0515}
\end{enumerate}
Cette chaîne donne la garantie que \tgz{5890}, \tgz{1073},  \tgz{2599} et \tgz{3732} ont la même consonne initiale, et que tous les caractères dont ils servent de premier caractère de \textit{fǎnqiè} aussi partagent cette initiale.

\citet{sofronov68b} a appliqué systématiquement cette méthode à tous les caractères, et présente une liste complète des chaînes de \textit{fǎnqiè} initiaux, ainsi qu'une analyse détaillée du système phonologique. La reconstruction de Sofronov, dont les résultats principaux sont acceptés par tous les chercheurs, résout presque entièrement le problème de l'inventaire consonantique du tangoute. La seule question qui n'est toujours pas résolue concerne le nombre de consonnes latérales: Sofronov en reconstruit trois, tandis que d'autres auteurs tels que Gong Hwangcherng n'en reconstruisent que deux: on ne parvient pas à relier clairement les chaînes d'initiales pour cette catégorie.

En effectuant cette recherche, \citet[106-110, 134-137]{sofronov68a} s'est aperçu que les caractères servant à noter les consonnes initiales sont séparés en plusieurs classes selon la \textit{rime du caractère}. C'est une propriété importante des  \textit{fǎnqiè} tangoutes qui n'a pas vraiment d'équivalent en chinois.

Pour comprendre le principe séparation des caractères qui notent les consonnes initiales, il convient tout d'abord d'observer que les rimes sont classées en quatre groupes:\footnote{Les caractéristiques phonologiques de ces quatre groupes ne peuvent pas se déduire à partir des données internes au tangoute; elles seront abordées plus en détail en sections \ref{subsubsec:rpreinitiale}, \ref{subsubsec:preinitialec}.}
\begin{enumerate}
\item Le  cycle majeur (большой цикл): Rimes 1 à 60.
\item Le premier cycle mineur (I малый цикл): Rimes 61 à 76.
\item Le second cycle mineur (II малый цикл): Rimes 77 à 98.
\item Le troisième cycle mineur  (III малый цикл): Rimes 99 à 105.
\end{enumerate}
Ces quatre groupes n'apparaissent pas de façon explicite dans les dictionnaires, mais leur existence est suggérée par deux faits importants. Premièrement, à l'intérieur de chaque cycle (sauf le dernier), on observe que les voyelles (telles qu'on peut les reconstruire à partir des comparaisons avec le chinois et le tibétain) suivent l'ordre général suivant:
\begin{enumerate}
\item \textit{u}
\item \textit{i/e}
\item \textit{a}
\item \textit{ə/ɨ}
\item \textit{ij/ej}
\item \textit{əj/ɨj}
\item \textit{iw/ew}
\item \textit{o}
\end{enumerate}
On observe quelques variations à cet ordre général, mais les dernières rimes des trois premiers cycles ont toujours un vocalisme \textit{o} (voir l'annexe pour s'en convaincre). A cela s'ajoute le fait que mis à part les   caractères du troisième cycle mineur, les caractères initiaux de  \textit{fǎnqiè}  ont une rime qui appartient presque toujours au même cycle que celle du caractère qu'ils glosent. Observons des chaines de \textit{fǎnqiè} d'initiales de caractères du premier  cycle mineur (notée avec un point sous la voyelle dans la reconstruction de Gong Hwangcherng):

\begin{enumerate}
\item \tgz{5449} > \tgz{4603} + \tgz{5212}
\item \tgz{4603} > \tgz{3735} + \tgz{0500}
\item \tgz{3735} > \tgz{4603} + \tgz{1321}
\end{enumerate}

Voici maintenant un exemple avec des caractères du second cycle mineur (notée avec un --r dans la reconstruction de Gong Hwangcherng):
\begin{enumerate}
\item \tgz{0808} > \tgz{0937} + \tgz{2714}
\item \tgz{0937} > \tgz{3925} + \tgz{0681}
 \item \tgz{3925} > \tgz{0937} + \tgz{2714}
\end{enumerate}

La séparation entre les trois catégories n'est pas absolue, surtout pour le second cycle mineur qui compte moins de caractères. On trouve des exceptions telles que:

\tgz{0739} > \tgz{5449} + \tgz{5564}


où un caractère du premier cycle mineur sert à représenter l'initiale d'un caractère du second cycle mineur.


La signification de cette distribution singulière des  caractères de \textit{fǎnqiè} pour la reconstruction du tangoute n'est pas entièrement claire, mais elle implique que les consonnes initiales des caractères du premier et du second cycles mineurs avaient des propriétés phonétiques différentes de celles du cycle majeur. \citet[109-110]{sofronov68a} relie cette observation aux transcriptions tibétaines, où les syllabes du premier cycle mineur correspondent parfois à des syllabes en \textit{h}-- et où les syllabes du second cycle mineur correspondent en général à des syllabes à \textit{r}-- préinitiale. Il n'est pas certain toutefois que l'on puisse en conclure que le tangoute avait encore des groupes de consonnes au moment où le dictionnaire \textit{Wénhǎi} a été compilé,\footnote{On considère qu'il s'agissait de voyelles tendues et rhotacisées respectivement, et que la tension et la rhotacisation étaient des traits suprasegmentaux influençant la consonne initiale.} mais ces indices seront utiles pour reconstruire le pré-tangoute.

Outre les concepts de  \textit{fǎnqiè} et de cycle que nous venons d'expliquer, nous ferons usage dans ce travail des termes suivants, tirés de la phonologie traditionnelle du chinois:

\begin{enumerate}
\item \zh{開口} \textit{kāikǒu} : il s'agit respectivement des syllabes sans médiane --\textit{w}--.
\item \zh{合口} \textit{hékǒu} : syllabes ayant une  médiane --\textit{w}--.
\item \zh{攝} \textit{shè} : groupement de rimes ayant la même catégorie de voyelle et la même consonne finale.\footnote{Le tangoute n'ayant aucune consonne finale, seuls les traits vocaliques définissent ces \textit{shè} pour cette langue. }

\end{enumerate}

\section{Systèmes phonologiques des langues étudiées}
Avant d'aborder l'étude diachronique des langues macro-rgyalronguiques, nous présentons ici les informations de base sur le système phonologique des deux langues principales qui vont nous intéresser: le japhug et le tangoute.

 \subsection{Système phonologique du japhug}
Le  système phonologique du japhug a déjà été décrit en détail dans des publications précédentes, telles que \citet{jacques08}, et nous nous limiterons dans cette section aux faits les plus importants.

La structure syllabique du japhug, comme celle de la plupart des langues rgyalronguiques, est (CC)C(C^m)V(C). On trouve des groupes de trois phonèmes consonantiques en début de syllabe, comme \textit{βzɟɯr} ``changer'' ou \textit{ʁmbɣi} ``soleil'', et jamais plus d'une consonne finale. 

L'inventaire vocalique est limité à huit voyelles: \textit{a e i o u ɤ ɯ y}. \textit{y} n'est attestée que dans un seul étymon natif, \textit{qaɟy} ``poisson''. Les voyelles \textit{ɤ} et \textit{ɯ} sont majoritaires dans les préfixes et les syllabes affaiblies. \textit{ɤ} et \textit{a} ne sont que médiocrement distinctives lorsqu'elles sont suivies d'une uvulaire.

L'inventaire consonantique du japhug est relativement complexe: on compte 49 phonèmes (tableau \ref{tab:inventaire-jpg}).

\begin{table}
\captionb{Inventaire des consonnes du japhug}\label{tab:inventaire-jpg}
\begin{tabular}{lllllllllllll} 
p	&	t	&		&		&	c	&	k	&	q	\\
pʰ	&	tʰ	&		&		&	cʰ	&	kʰ	&	qʰ	\\
b	&	d	&		&		&	ɟ	&	g	&		\\
mb	&	nd	&		&		&	ɲɟ	&	ŋg	&	ɴɢ	\\
m	&	n	&		&		&	ɲ	&	ŋ	&		\\
	&		&		&		&		&		&		\\
	&	ts	&	tɕ	&	tʂ	&		&		&		\\
	&	tsʰ	&	tɕʰ	&	tʂʰ	&		&		&		\\
	&	dz	&	dʑ	&	dʐ	&		&		&		\\
	&	ndz	&	ndʑ	&	ndʐ	&		&		&		\\
	&		&		&		&		&		&		\\
	&	s	&	ɕ	&	ʂ	&		&	x	&	χ	\\
	&	z	&	ʑ	&		&		&	ɣ	&	ʁ	\\
w	&	l	&		&	r	&	j	&		&		\\
	&	ɬ	&		&		&		&		&		\\
\end{tabular}
\end{table}
Le phonème /w/ présente un allophone [β] dans certains contextes que nous distinguons dans notre transcription.

Six de ces phonèmes, \textit{r l j w ɣ ʁ}, peuvent apparaître en position médiane (C^m) entre la consonne principale et la voyelle (comme le \textit{ɣ} de \textit{ʁmb\textbf{ɣ}i} ``soleil''). 

Seuls dix d'entre eux sont admis en coda: \textit{--w --t --ɣ --ʁ --z --r --l --m --n --ŋ}; l'opposition de voisement y est neutralisée, et les consones finales, y compris les sonantes, sont assourdies en fin d'énoncé ou en isolation.  Les fricatives et sonantes finales --\textit{w}, --\textit{ɣ} et --\textit{ʁ} viennent de *--p, *--k, *--q respectivement.

Les mots japhugs sont majoritairement polysyllabiques, mais la plupart des racines verbales sont monosyllabiques. On trouve en revanche de nombreuses racines nominales polysyllabiques.

\subsection{Système phonologique du tangoute}
Le tangoute étant une langue morte, nous connaissons mal son système phonologique, et encore moins la réalisation des phonèmes.

La syllabe du tangoute est généralement reconstruite comme C(j,i,w)V(j,w); plus aucun auteur ne semble favoriser l'hypothèse que des groupes de consonnes existaient encore à l'époque du dictionnaire \textit{Wénhǎi}.   L'inventaire du système consonantique tangoute tel que le reconstruit Gong Hwangcherng est indiqué dans le tableau \ref{tab:inventaire-tangut}.

\begin{table}
\captionb{Inventaire des consonnes du tangoute}\label{tab:inventaire-tangut}
\begin{tabular}{lllllllllllll} 
p	&	t	&		&			k	&	\\
pʰ	&	tʰ	&		&			kʰ		\\
mb	&	nd	&		&			ŋg	&		\\
m	&	n	&		&			ŋ	&		\\
	&		&		&				&		\\
	&	ts	&	tɕ	&			&		\\
	&	tsʰ	&	tɕʰ		&		&		&		\\
	&	ndz	&	ndʑ		&		&		&		\\
	&		&		&		&		&		&		\\
	&	s	&	ɕ	&				x	&		\\
	&	z	&	ʑ	&			ɣ	&		\\
w	&	l	&			j&	&		&		&		\\
	&	ɬ	&		&		&		&		&		\\
	&r\\
\end{tabular}
\end{table}
Deux détails sont encore débattus: la nécessité ou non de reconstruire une troisième latérale /ld/, et la nature phonétique des prénasalisées; Gong Hwangcherng les reconstruit comme des voisées et les note comme telles.

Si la reconstruction du système consonantique est relativement bien acceptée, celle du système vocalique est beaucoup plus problématique. Bien que l'on connaisse les catégories distinctives, il est très difficile de retrouver la valeur phonétique de ces oppositions. Le tangoute, comme le muya et le pumi, avait un système vocalique très riche, auquel les transcriptions chinoises et tibétaines ne rendent pas justice.

Le système de Gong Hwangcherng a l'avantage d'être plus cohérent et plus lisible que ceux proposés par les autres chercheurs, mais il ne reflète qu'une possibilité parmi de nombreuses autres.

Gong reconstruit un système à sept voyelles \textit{u}, \textit{o} \textit{i}, \textit{e},  \textit{a},  \textit{ɨ} et \textit{ə}, dont les deux dernières sont en distribution complémentaire par rapport à la présence de médiane --j--. Ces voyelles ont par ailleurs une opposition de longueur et une triple opposition  relâchée / tendue / rhotacisée (correspondant aux trois premiers des quatre cycles décrits dans la section précédente). Les seules consonnes finales possibles sont les semi-voyelles --\textit{j} et --\textit{w}. A cela, il faut ajouter les voyelles nasales \textit{ã}, \textit{ĩ}, \textit{ẽ} et \textit{ũ} qui n'apparaissent que dans les emprunts chinois.



\chapter{Phonologie comparée} \label{phono}
 \thispagestyle{empty}
La méthode employée dans ce travail pour étudier la phonologie historique du tangoute se base sur la méthode comparative. Toutefois, le pré-tangoute que nous reconstruisons ici n'est pas l'ancêtre commun à plusieurs langues attestées, puisque qu'une seule variété de tangoute nous est connue par les textes existants. C'est plus exactement une projection du tangoute dans le passé basée sur la comparaison avec les autres langues sino-tibétaines, et principalement les langues macro-rgyalronguiques.

Il est impossible actuellement de proposer une reconstruction rigoureuse du proto-rgyalrong, et à plus forte raison du proto-macro-rgyalronguique ou du proto-sino-tibétain. Nous ne pouvons donc pas réaliser une reconstruction depuis la proto-langue vers la langue fille (``top-down''), partant du proto-macro-rgyalronguique vers le tangoute. La reconstruction du proto-macro-rgyalronguique est certes un but plus réalisable qu'une reconstruction complète du proto-sino-tibétain. Néanmoins, étant donné que la majorité des langues sont imparfaitement connues et très sommairement décrites, une reconstruction basée sur les documents de seconde main actuellement disponibles serait d'une fiabilité douteuse. Par ailleurs, utiliser le tangoute directement est extrêmement difficile pour le comparatiste, car il a subi une quantité importante de changements phonétiques  qui le rendent radicalement différent des langues plus conservatrices comme le japhug ou le tibétain.

Nous proposons donc dans ce travail d'effectuer un premier pas vers la reconstruction du proto-macro-rgyalronguique: par la comparaison avec les langues plus conservatrices qui ont préservé les consonnes finales et les groupes initiaux, proposer un modèle du pré-tangoute qui éclaircira l'analyse de la morphologie et facilitera la comparaison avec les langues proches. Ce travail s'inscrit donc dans une perspective plus néo-grammairienne que structuraliste: nous n'avons pas la prétention de reconstruire un état de langue réel intermédiaire entre le proto-macro-rgyalronguique et le tangoute, mais simplement un outil explicatif pour analyser l'histoire du tangoute et des autres langues macro-rgyalronguiques. La reconstruction de la succession des états synchroniques entre le proto-macro-rgyalronguique et le tangoute ne pourra se concevoir que lorsque l'ensemble des changements phonétiques et morphologiques sera connu et que leur chronologie relative aura été établie. 

Cette reconstruction a trois objectifs. Premièrement, c'est une première étape vers la reconstruction du proto-macro-rgyalronguique, qui présente de façon explicite certains changements phonétiques qui ont dû avoir lieu entre le proto-macro-rgyalronguique et l'état attesté du tangoute. Deuxièmement, c'est un outil heuristique qui rend plus facile la  découverte de nouveau cognats dans les autres langues sino-tibétaines. Troisièmement, c'est un modèle qui permet de donner une explication à certaines alternances morphologiques opaques du tangoute (voir en particulier la section \ref{subsubsec:origine.alternances} p.\pageref{subsubsec:origine.alternances}).

\section{Consonnes simples et groupes de consonnes} \label{sec:initiales}
Comme nous l'avons expliqué dans le chapitre précédent, le tangoute est quasi-universellement reconstruit sans groupes de consonnes. \citet[112]{sofronov68a} et Nie Hongyin (1986) ont montré que les préinitiales b--, d--, g-- dans les transcriptions tibétaines sont en fait des marques des consonnes médianes. Par ailleurs, bien que plusieurs caractères soient parfois employés dans le ZZZ pour transcrire un caractère tangoute, la quasi-totalité des spécialistes admet qu'il n'y a pas lieu de reconstruire d'authentiques groupes de consonnes dans ces cas;\footnote{Le traitement de ces données par \citet[85-114]{lifw94zzz} est particulièrement clair et détaillé.} le ou les caractères chinois supplémentaires servent seulement d'indicateurs auxiliaires de prononciation.

De même, il est peu vraisemblable que le tangoute ait préservé des consonnes finales, même si l'opinion des spécialistes varie sur ce point. \citet{sofronov68a}, en particulier, reconstruit deux états du tangoute, dont le premier a des consonnes finales non-nasales, qui ont disparues dans le second. Par ailleurs, tous les systèmes reconstruisent certaines voyelles nasales. Dans celui de Gong Hwangcherng, les voyelles nasales sont limitées à des rimes dans lesquelles n'apparaissent  que des emprunts chinois, tandis que dans la plupart des autres systèmes, certaines rimes dans lesquelles apparaissent aussi des mots natifs sont reconstruites avec des voyelles nasalisées. On reconstruit, d'après Nishida, deux qualités de voix en tangoute: les voyelles ``tendues'' (notées avec un point souscrit ou avec un --\textit{q} dans la reconstruction d'Arakawa) qui se trouvent dans les rimes appartenant au premier cercle mineur de \ipa{fǎnqiè}, et les voyelles ``rhotacisées'', notées par un --\textit{r}, dans les rimes du second cercle mineur.	On prendra ici ces reconstructions comme point de départ de notre étude comparative.
\subsection{Reconstruction des préinitiales} \label{subsec:preinitiales}
Dans les langues conservatrices de la famille sino-tibétaine, l'attaque peut comporter trois, voire quatre consonnes distinctes. Outre la consonne de base, on observe les consonnes médianes ([\ipapl{--r--, --y--, --w--, --l--}] en tibétain et [\ipapl{r, j w, l, ɣ, ʁ}] en japhug) qui apparaissent entre la consonne principale et la voyelle, et les consonnes préinitiales qui apparaissent avant la consonne initiale. Le tableau \ref{tab:syllabetibetain} indique quelques exemples en tibétain.

\begin{table}
\captionb{Exemples de structures syllabiques en tibétain}\label{tab:syllabetibetain}
\resizebox{\columnwidth}{!}{
\begin{tabular}{llllll} 
forme	&	préinitiale	&	consonne 	&	médiane	&	rime	\\
&&principale\\
kʰa ``amer''	&		&	k	&		&	a	\\
kʰra ``faucon''	&			&	k	&	r	&	a	\\
bka ``parole''	&	b	&	k	&		&	a	\\
bkra ``décoré''	&	b	&	k	&	r	&	a	\\
\end{tabular}}
\end{table}
En tibétain, on peut avoir au plus deux préinitiales, et parfois deux médianes (dans un mot comme \tib{grʷa} « côté »). En japhug et dans les autres langues rgyalrong, une structure similaire peut s'observer (tableau \ref{tab:syllabejaphug}).

\begin{table}
\captionb{Exemples de structures syllabiques en japhug}\label{tab:syllabejaphug}
\centering
\resizebox{\columnwidth}{!}{
\begin{tabular}{lllll} 
	forme	&	préinitiale	&	consonne 	&	médiane	&	rime	\\
	&&principale\\
\ipa{	ʑaka} ``chacun''	&		&	k	&		&	a	\\
\ipa{	kra} ``faire tomber''	&		&	k	&	r	&	a	\\
\ipa{	ɕkala} ``boîteux''	&	\ipapl{ɕ}	&	k	&		&	a	\\
\ipa{	skra} ``sale et malpoli''	&	s	&	k	&	r	&	a	\\
\end{tabular}}
\end{table}

En japhug, on peut également trouver des cas de doubles préinitiales, mais jamais de double médianes.
L'état du japhug et du tibétain ne reflète pas nécessairement celui du proto-sino-tibétain. Il est en effet probable qu'au moins une partie des préinitiales de ces langues proviennent de présyllabes à voyelle réduite, comme le suggère la comparaison avec d'autres langues conservatrices telles que le dulong ou le jingpo, comme le montre le tableau \ref{tab:syllabejaphug}.\footnote{Ces comparaisons sont tirées de \citealt[64]{dai90zangmian}.} 

\begin{table}
\captionb{préinitiales du tibétain correspondant à des présyllabes à voyelle réduite en dulong et en jingpo}\label{tab:preinitialesjaphug}
\begin{tabular}{llll} 
	tibétain	&	dulong	&	jingpo	&	sens	\\	
\ipa{	mnam}	&	\ipa{pɯ̆³¹nam⁵⁵}	&	\ipa{mă³¹nam⁵⁵	}&	sentir	\\	
\ipa{	sram}	&	\ipa{sɯ̆³¹ɹɑm⁵⁵}	&	\ipa{ʃă³¹ʒam³³	}&	loutre	\\	
\ipa{	dgu}	&	\ipa{dɯ̆³¹gɯ⁵³}	&	\ipa{tʃă³¹khu³¹	}&	9	\\	
\end{tabular}
\end{table}

Néanmoins, pour les besoins pratiques de la reconstruction de langues telles que le tangoute qui ont subi une monosyllabisation massive, il est secondaire de savoir si l'on reconstruit une préinitiale ou une présyllabe: seules en restent des traces phonétiques indirectes. Par la comparaison du tangoute avec les langues macro-rgyalronguiques modernes, il est possible de reconstruire quatre préinitiales. 

La première est une préinitiale à lieu d'articulation indéterminé *S--, mais qui devait au moins comprendre les *s-- anciens, plus éventuellement des *l-- et des *\ipapl{ɕ--}. \citet{gong99jinyuanyin} a suggéré de reconstruire une telle préinitiale  à un stade pré-tangoute dans les syllabes dont les rimes appartiennent au premier cycle mineur (rimes 61-75) en tangoute sur la base de comparaisons avec le pumi. Gong reconstruit les rimes du premier cycle mineur avec une voix tendue en tangoute, notée par un point en dessous de la voyelle. Même si la réalité phonétique exacte des rimes appartenant à ce groupe nous échappe, la démonstration par la comparaison avec le pumi effectuée par Gong est concluante et nous l'acceptons. Nous reconstruisons donc une préinitiale *S-- pour toutes les syllabes de ce groupe.


La seconde est *r--, reconstruite pour certaines syllabes du second et du troisième cycle mineurs, et peut-être exceptionnellement aussi dans quelques autres exemples.
La troisième est la préinitiale *p--, qui devient une médiane --w-- en tangoute.
La quatrième est la préinitiale notée *C-, probablement une occlusive *k ou *t, qui peut se reconstruire dans le cas d'initiales lénifiées.

\subsubsection{Préinitiale *r-} \label{subsubsec:rpreinitiale}

Le cas des syllabes des second et troisième cycles mineurs, reconstruites par Nishida comme des voyelles rhotacisées, est plus épineux, comme le montre le tableau \ref{tab:cycle2r}. Dans certains cas, la voyelle rhotacisée correspond clairement à une préinitiale r-. \footnote{Dans ce tableau, ainsi que dans tous ceux qui apparaîtront dans ce travail, les formes rgyalrong sont tirées du japhug. Dans certains cas, nous avons pris des formes du Situ du dictionnaire de \citet{huangsun02}, indiquées par (situ). Le sens indiqué est celui du rgyalrong, sauf lorsqu'il n'est pas attesté, auquel cas on choisit le sens du tangoute. L'analyse plus détaillée de chaque comparaison sera présentée dans la section sur les voyelles; il serait redondant de la répéter ici. Les emprunts potentiels au tibétain sont indiqués par une astérisque.}


\begin{table}
\captionb{Syllabes de second cycle mineur ayant une préinitiale *r-- en pré-tangoute.}\label{tab:cycle2r}
\resizebox{\columnwidth}{!}{
\begin{tabular}{lllllllllllll}
\toprule
&\multicolumn{2}{c}{tangoute} && japhug & sens & tibétain & pré-tangoute \\
\midrule
5528& \tgf{5528} & \ipa{bar} &1.80& \ipa{tɤ-rmbɣo} & tambour && \ipa{*rmb-}\\ 
3688&\tgf{3688} & \ipa{gjwɨr} &1.86& \ipa{nɯrŋgɯ} & s'allonger && \ipa{*rŋg-}\\ 
1909&\tgf{1909} & \ipa{gur} &1.75& \ipa{kərgú (situ)} & boeuf && \ipa{*rŋg-}\\ 
5755&\tgf{5755} & \ipa{.jar} &1.82& \ipa{rjap (situ)} & être debout && \ipa{*rj-}\\ 
4602&\tgf{4602} & \ipa{.jar} &1.82& \ipa{kɯrcat} & huit &brgyad <*p-rjat& \ipa{*rj-}\\
1894&\tgf{1894} & \ipa{.jar} &1.82& \ipa{tɤ-rʑaβ} & épouse && \ipa{*rj-}\\ 
811&\tgf{0811} & \ipa{.jaar} &2.75& \ipa{tɤ-rʑaʁ} & nuit && \ipa{*rj-}\\ 
2798&\tgf{2798} & \ipa{.jir} &2.72& \ipa{ɣurʑa} & cent &brgya < *p-rja& \ipa{*rj-}\\ 
5396&\tgf{5396} & \ipa{kjur} &1.76& \ipa{rku} & mettre dans && \ipa{*rk-}\\ 
5817&\tgf{5817} & \ipa{kjwɨɨr} &1.92& \ipa{mɯrkɯ} & voler &rku& \ipa{*rk-}\\ 
458&\tgf{0458} & \ipa{kor} &1.89& \ipa{tɯ-rqo} & gorge && \ipa{*rk-}\\ 
4032&\tgf{4032} & \ipa{mur} &2.69& \ipa{tə-rmô (situ)} & grêle && \ipa{*rm-}\\ 
2600&\tgf{2600} & \ipa{mjar} &1.82& \ipa{tɤ-rme} & poil && \ipa{*rm-}\\ 
3818&\tgf{3818} & \ipa{mjijr} &2.68& \ipa{tɯ-rme} & homme && \ipa{*rm-}\\ 
306&\tgf{0306} & \ipa{njir} &2.72& \ipa{} & emprunter &rnya < *rŋja& \ipa{*rŋj-}\\ 
1228&\tgf{1228} & \ipa{ŋur} &1.75& \ipa{rŋu} & frire &rŋo& \ipa{*rŋ-}\\ 
654&\tgf{0654} & \ipa{ŋwər} &2.76& \ipa{arŋi} & bleu && \ipa{*rŋw-}\\ 
2005&\tgf{2005} & \ipa{tɕior} &1.90& \ipa{tɤ-rcoʁ} & saleté, boue && \ipa{*rtɕ-}\\ 
4739&\tgf{4739} & \ipa{tsewr} &1.87& \ipa{tɯ-rtsɤɣ} & section &tshigs& \ipa{*rts-}\\ 
1464&\tgf{1464} & \ipa{tsur} &1.75& \ipa{tɯ-qartsɯ} & coup de pied && \ipa{*rts-}\\ 
1490&\tgf{1490} & \ipa{tsur} &1.75& \ipa{qartsɯ} & hiver && \ipa{*rts-}\\ 


\bottomrule
\end{tabular}}
\end{table}


La reconstruction d'une préinitiale *r-- est donc une des origines possibles en pré-tangoute des voyelles reconstruites comme rhotacisées en tangoute. Naturellement, les groupes *rC-- du tangoute ne correspondent pas nécessairement à ceux du rgyalrong ou du tibétain. Observons le tableau \ref{tab:sans.preinitiale.r:preinitiale.r}.
\begin{table}
\captionb{Syllabes de cycle majeur et de premier cycle mineur correspondant à des cognats en japhug à préinitiale r--.}\label{tab:sans.preinitiale.r:preinitiale.r} 
\resizebox{\columnwidth}{!}{
\begin{tabular}{lllllllllllll}
\toprule
&\multicolumn{2}{c}{tangoute} && japhug & sens & tibétain & pré-tangoute \\
\midrule
2200& \tgf{2200} & \ipa{ba} &1.17 & \ipa{ɣɤrʁaʁ} &chasser& & \ipa{*mb--}\\
2737& \tgf{2737} & \ipa{ljɨɨ} &1.32 & \ipa{rʑi} &lourd&ldʑid-po & \ipa{*l--}\\
2639& \tgf{2639} & \ipa{mjiij} &2.35 & \ipa{tɤ-rmi} &nom&miŋ & \ipa{*m--}\\
4693& \tgf{4693} & \ipa{na} &1.17 & \ipa{rnaʁ} &profond& & \ipa{*(r)n--}\\
1671& \tgf{1671} & \ipa{njij} &1.36 & \ipa{ɣɯrni} &rouge& & \ipa{*(r)n--}\\
2915& \tgf{2915} & \ipa{no} &1.49 & \ipa{tɯ-rnom} &côte& & \ipa{*(r)n-- }\\
118& \tgf{0118} & \ipa{no} &2.42 & \ipa{tɯ-rnoʁ} &cerveau& & \ipa{*(r)n-}\\
\midrule
5509& \tgf{5509} & \ipa{bjị} &1.67 & \ipa{tɤ-rmbi} &urine& & \ipa{*Smb--}\\
5203& \tgf{5203} & \ipa{wjị} &1.67 & \ipa{tɯ-rpa} &hache& & \ipa{*C--Sp--}\\
5105& \tgf{5105} & \ipa{tsə̣} &1.68 & \ipa{tɯ-rtshɤz} &poumon& & \ipa{*Sts--}\\
1098& \tgf{1098} & \ipa{tsjụ} &1.59 & \ipa{tɯ-qartsɯ} &coup de pied& & \ipa{*Sts--}\\
3439& \tgf{3439} & \ipa{pjịj} &1.61 & \ipa{tɤ-rpi} &sūtra& & \ipa{*Sp-}\\
5170& \tgf{5170} & \ipa{.wạ} &1.63 & \ipa{tɯ-rpaʁ} &épaule&phrag & \ipa{*C--Sp--}\\

\bottomrule
\end{tabular}}
\end{table}



Des étymons à préinitiale r- en japhug peuvent correspondre à des syllabes du cycle majeur (sans préinitiale *r-- ou *S--) ou à des syllabes du premier cycle mineur (pré-initiale *S--). Dans ces cas, on remarque l'absence de préinitiale \textit{r}-- dans les mots tibétains apparentés. On n'a d'autre choix ici que de reconstruire soit une absence de préinitiale, soit une préinitiale *S--. La différence avec le rgyalrong n'est pas explicable dans l'état actuel de nos connaissances, mais suggère que certains de ces éléments préinitiaux ont eu un rôle morphologique, et que leur présence en rgyalrong est une innovation. Une autre possibilité serait que le *r-- disparaît devant certaines rimes en pré-tangoute ou certaines initiales en tangoute. Ainsi, on trouve plusieurs exemples de \textit{rn}-- en japhug ou en tibétain correspondant à n-- avec une rime du cycle majeur en tangoute, mais aucun exemple de \textit{rn}- correspondant à \textit{n}-- avec une rime du second cycle mineur. Il est donc très vraisemblable que la préinitiale *r-- ait chuté tôt devant *n--, si bien que les mots à *rn-- ont acquis des rimes du cycle majeur comme ceux à initiale *n-- en pré-tangoute.


Dans le cas des mots du premier cycle mineur, une possibilité serait de reconstruire deux préinitiales *Sr-- qui se simplifieraient en *S-- par la suite, le *r-- ne laissant pas de trace visible.

Par ailleurs, la préinitiale *r-- n'est peut-être pas la seule origine de ces syllabes. De nombreuses syllabes à consonne finale –r en rgyalrong ou en tibétain correspondent à des syllabes de deuxième ou troisième cycles mineurs en tangoute, comme on peut l'observer dans le tableau \ref{tab:cycle2finalerjaphug}.

\begin{table}
\captionb{Syllabes de second cycle mineur correspondant à des étymons à finale –r en rgyalrong.}\label{tab:cycle2finalerjaphug}
\resizebox{\columnwidth}{!}{
\begin{tabular}{lllllllllllll}
\toprule
&\multicolumn{2}{c}{tangoute} && japhug & sens & tibétain & pré-tangoute \\
\midrule
4543& \tgf{4543} & \ipa{mər} &1.84 & \ipa{tɯ-ɣmɤr} &bouche &&*m--\\
1254& \tgf{1254} & \ipa{dʑjwɨr} &1.86 & \ipa{ɣndʑɯr} &moudre && *p-dʑ--\\
5037& \tgf{5037} & \ipa{bjɨr} &1.86 & \ipa{mbrɯtɕɯ}  &couteau&&*mb-- \\
65& \tgf{0065} & \ipa{gjwɨr} &2.77 & \ipa{tɯ-mgɯr} & dos & &*ŋg-\\
2739& \tgf{2739} & \ipa{tɕhjwɨr} &2.77 & \ipa{tɕur} &acide&skʲur.mo &*tɕʰ--\\
5682, 5592 & \tgf{5682},\tgf{5592} & \ipa{kaar} &1.83 & \ipa{skɤr} & peser (emprunt?) &skar &*k--\\
2464& \tgf{2464} & \ipa{tswər} &1.84 & \ipa{ftsɯr} &essorer&btsir *pts--\\
\bottomrule
\end{tabular}}
\end{table}



Dans \ipa{mbrɯtɕɯ}, la syllabe du japhug vient par métathèse de \ipapl{*mbɯr} (voir \citealt[278]{jacques04these})

Toutefois, on trouve une quantité non-négligeable de contre-exemples. Même si une partie des contre-exemples rassemblés dans le tableau suivant  pourraient s'expliquer d'une façon alternative\footnote{En particulier, dans \ipa{kjạ}, la voix tendue a eu l'ascendant sur la voix rhotacisée, comme nous l'avons suggéré pour certains exemples du tableau \ref{tab:sans.preinitiale.r:preinitiale.r}}, il est difficile d'accepter sans réserves que la finale *--r du pré-tangoute donne des syllabes du second cycle mineur. En particulier, on peut affirmer que cette règle ne s'applique pas aux syllabes dont la voyelle principale est --a (voir section \ref{subsubsec:correspondance:a:ar} p.\pageref{subsubsec:correspondance:a:ar}). 

\begin{table}
\captionb{Syllabes du cycle majeur correspondant à des étymons en –r final en rgyalrong.}\label{tab:noncycle2finalerjaphug}
\begin{tabular}{llllll} \toprule
\multicolumn{4}{c}{tangoute} & japhug & sens   \\
\midrule
2539& \tgf{2539} & \ipa{kjạ} &1.64 & \ipa{nɤscɤr} &être effrayé\\
4526& \tgf{4526} & \ipa{ma} &2.14 & \ipa{tamar} &beurre\\
1530& \tgf{1530} & \ipa{mja} &1.20 & \ipa{smar} &fleuve\\
284& \tgf{0284} & \ipa{ɕjwo} &1.48 & \ipa{ɕɤr} &soir\\
\bottomrule
\end{tabular}
\end{table}
Néanmoins, dans le présent travail, cette hypothèse sera retenue de façon provisoire. 


Outre les préinitiales et les finales *r, on trouve quelques cas qui pourraient suggérer que la médiane *--r-- du pré-tangoute elle peut causer la rhotacisation dans les syllabes du second cycle mineur (tableau \ref{tab:cycle2japhugrmediane}).

\begin{table}
\captionb{Syllabes du second et du troisième cycles mineurs correspondant à des syllabes à médiane --r-- en rgyalrong.}\label{tab:cycle2japhugrmediane}
\resizebox{\columnwidth}{!}{
\begin{tabular}{lllllll} \toprule
\multicolumn{4}{c}{tangoute} & japhug & sens & tibétain  \\
\midrule
1298& \tgf{1298} & \ipa{kjiwr} &1.79 & \ipa{tɯ-zgrɯ} &coude&gru.mo\\
860& \tgf{0860} & \ipa{kwər} &1.84 & \ipa{tɯ-skhrɯ} &corps&sku\\
2768& \tgf{2768} & \ipa{kjiwr} &1.88 & \ipa{qro} &fourmi&grog.ma\\
5921& \tgf{5921} & \ipa{zar} &2.73 & \ipa{nɤzraʁ} &avoir honte& \\
2858& \tgf{2858} & \ipa{zjir} &2.72 & \ipa{zri} &long& riŋ-po\\
4796& \tgf{4796} & \ipa{zjɨr} &1.86 & \ipa{zrɯ} &adret& \\
3582& \tgf{3582} & \ipa{kjɨɨr} &2.85 & \ipa{tɯ-ɕkrɯt} &bile&mkʰris.pa\\
\bottomrule
\end{tabular}}
\end{table}


Ici, il convient probablement d'analyser les exemples au cas par cas. Trois d'entre eux (``coude'', ``corps'' et ``bile'') ont comme caractéristique d'avoir à la fois une préinitiale \ipa{s--/ɕ--} en japhug et une médiane --r--. On peut donc proposer que les syllabes du type *S-CrV passent à CVr en tangoute.\label{analyse:S-Cr} Cette règle suppose toutefois que le pré-tangoute était ici très proche du japhug, et différait du tibétain, car la même structure ne se retrouve pas en tibétain. \footnote{Une complication est apportée par le mot ``coude'', car cette étymologie n'est pas certaine: voir p.\pageref{analyse:coude}.} 

Pour le mot « fourmi », \label{analyse:fourmi} on doit prendre en compte le fait que la vélaire est un préfixe, qui a pu rester une présyllabe pendant davantage de temps que la plupart des autres préfixes et que la rhotacisation est une trace d'un état où *r était encore une initiale. \footnote{Toutes les syllabes à initiale *r-- appartiennent au second cycle mineur.}

Pour les trois exemples \tgf{2858} \ipa{zjir²} ``long'', \tgf{4796} \ipa{zjɨr¹} ``sud'' et \tgf{5921} \ipa{zar²} ``avoir honte'', \label{analyse:sr} le problème est plus complexe. Il faut prendre en compte le fait qu'en tangoute, on ne trouve quasiment pas de rimes de second cycle mineur avec l'initiale s--, mis à part 2967  \tgf{2967} \ipa{sar} 2.73 emprunté au chinois \zh{撒} (chinois moyen \textit{sat}), ainsi que 2585 \tgf{2585} \ipa{saar} 2.83 ``trembler'',  3131 \tgf{3131} \ipa{saar} 2.83 et 1554 \tgf{1554} \ipa{sjɨr} 1.86 ``être rempli''. 



Cette distribution lacunaire mérite une explication. Nous proposons la solution suivante: Les groupes *sr-- et *rs-- du pré-tangoute donnent en tangoute une initiale z-- avec une voyelle rhotacisée. Ainsi, \tgf{2858} \ipa{zjir²} vient d'un pré-tangoute *\ipapl{s--rje} (< \ipapl{s--rjeN}) et \tgf{5921} \ipa{zar²} de *\ipapl{s--rak}. La sonorisation du s- dans ces mots s'est également produite en japhug de façon indépendante. Dans d'autres langues ST, on retrouve des traces d'une préinitiale *s-- dans les cognats de \tgf{2858} \ipa{zjir²}, voir \citet[277]{matisoff03}.

Enfin, on trouve le cas d'étymons du tangoute du second ou du troisième cycle mineur qui correspondent à des mots rgyalrong ou tibétains sans r- préinitial, médian ou final (tableau \ref{tab:cycle2japhugsansr}).


\begin{table}
\captionb{Syllabes du second et troisième cycle mineurs correspondant à des étymons rgyalrong ou tibétain sans r.}\label{tab:cycle2japhugsansr}
\resizebox{\columnwidth}{!}{
\begin{tabular}{llllllll} \toprule
\multicolumn{4}{c}{tangoute} & japhug & sens & tibétain & pré-tangoute \\ 
\midrule
3925& \tgf{3925} & \ipa{mur} &1.75 &   &ténèbres&mun.pa & \ipa{*rm-}\\
5957& \tgf{5957} & \ipa{tser} &1.77 & \ipa{ntsɣe} &vendre& & \ipa{*rts-}\\
2547& \tgf{2547} & \ipa{tɕier} &1.78 & \ipa{χcha} &droite& & \ipa{*rtɕ-}\\
2462& \tgf{2462} & \ipa{bowr} &1.91 &  &abeille&bung.ba & \ipa{*rmb-}\\
2205& \tgf{2205} & \ipa{ljɨɨr} &1.92 & \ipa{kɯβde} &quatre&bzhi & \ipa{*rl-}\\
973& \tgf{0973} & \ipa{.wjijr} &2.68 & \ipa{βɣa} &moulin& & \ipa{*C-rp-}\\
4480& \tgf{4480} & \ipa{kar} &2.73 & \ipa{qɤt} &séparer& & \ipa{*rk-}\\
\bottomrule
\end{tabular}}
\end{table}

Dans ces exemples, on reconstruira une préinitiale malgré l'absence de r-- en japhug ou en tibétain. Le caractère préfixal secondaire de ces *r-- reconstruits est suggéré par le fait que certains des sept mots tangoutes cités ci-dessus ont des variantes n'appartenant pas au second cycle: 
 
1162  \tgf{1162} \ipa{.wjịj} 2.60 ``moulin'' et 1160  \tgf{1160} \ipa{ka} 2.14 ``séparer''. Ces alternances seront discutées plus en détail dans la section sur les voyelles.



\subsubsection{Préinitiale *p-} \label{subsubsec:preinitialep}

Outre les préinitiales *S-- et *r-- reconstruites plus haut, une troisième préinitiale peut être reconstruite: *p--. Cette préinitiale devient en tangoute la médiane --w--: il s'opère une métathèse entre préinitiale et médiane. Les exemples qui suggèrent cette reconstruction viennent du japhug:

\begin{table}
\captionb{Exemples de préinitiale *p-- en tangoute.}\label{tab:preinitialep}

\resizebox{\columnwidth}{!}{
\begin{tabular}{lllllll} \toprule
\multicolumn{4}{c}{tangoute} & japhug & sens &  pré-tangoute \\
\midrule
3869& \tgf{3869} & \ipa{kjwị} &1.67 & \ipa{fka < *pəka} &être rassasié & \ipa{*S--pk--}\\
1670& \tgf{1670} & \ipa{sjwij} &1.36 & \ipa{fse} &aiguiser & \ipa{*p--s--}\\
3929& \tgf{3929} & \ipa{tɕhjwi} &1.10 & \ipa{ftʂi} &faire fondre& \ipa{*p--tr--}\\
5120 & \tgf{5120}& \ipa{swew} &	1.43	& \ipa{fsoʁ} &clair, lumineux &\ipa{*p--s--}\\
2464& \tgf{2464} & \ipa{tswər} &1.84 & \ipa{ftsɯr} &essorer & \ipa{*p--s--}\\
2134& \tgf{2134} & \ipa{zjwị} &1.67 & \ipa{tɯ-ftsa} &neveu & \ipa{*C-S-pts--}\\
\bottomrule
\end{tabular}}
\end{table}



En japhug, les exemples ci-dessus ont un f-- préinitial (variante du phonème /w/), mais la comparaison avec le situ montre qu'il provient d'un *p--. Parmi ces exemples, le verbe ``essorer'' du rgyalrong et son cognat tangoute sont potentiellement des emprunts à la forme passée du \tib{btsir} (présent \ipa{ɴtsʰir}, auquel cas la préinitiale serait le préfixe de passé du tibétain). Cette hypothèse est toutefois douteuse pour une raison sémantique. En tibétain, \ipa{btsir} s'emploie pour ``presser (de l'huile)'', ou dans le sens d'interroger un suspect,\footnote{Ce sont là les sens donnés par  \citet{bodrgya}. Ce dictionnaire cite la phrase \ipa{til.ma btsir-nas snum ɴdon-pa} « On presse le sésame et l'huile sort ».}  tandis qu'en tangoute, le sens est celui de ``traire une vache''. Le sens du tangoute est partagé par le cognat pumi \ipa{\ipa{tsə́}}:

\begin{exe}
\ex \label{ex:pu:traire} 
\gll
		\ipa{ruɜ̂-ɬǐ}	\ipa{hõbǎ}	\ipa{ruɜ̂}	\ipa{tsə́-rə-khiɛ}	\ipa{ʑɛ̂-mədərə} \\
		yak-faire.paître	fille	yak	traire-\dur{}-\conv{}	être.là-\med{} \\
\trans Il y avait là une bergère en train de traire ses yaks. (Le mendiant, 4)
\end{exe}


Le tibétain a quant à lui développé un verbe dénominal \ipa{ɴdʑo, bʑos} ``traire'' tiré du nom \ipa{ʑo} < *ljo ``yaourt'' dont le sens originel est certainement ``lait'', comme le montre le cognat japhug \ipa{\ipa{tɤ-lu}} <*lo ``lait''. En voici un exemple:

\begin{exe}
\ex \label{ex:tib:traire} 
\gll
		a.ma	kʰʲed-kʲis	ɴbri	rgan  	bʑos-ɕog \\
		mère	vous-\erg{}	yak.femelle	vieux	traire.\ps{}-aller.\imp{} \\
\trans Mère, allez traire la vieille femelle de yak.  (Gesar, Ndanyul, 1)
\end{exe}

Si les étymons japhug et tangoute étaient empruntés au tibétain, on s'attendrait à une plus grande similitude sémantique. On en conclut donc qu'il est plus probable d'y voir un authentique cognat entre japhug, tangoute et tibétain, dont le sens de ``presser'' est probablement le plus ancien. La préinitiale *p-- n'a donc rien à voir avec le préfixe de passé du tibétain.

	Dans des syllabes avec u comme voyelle principale, la présence de la préinitiale *p- est invisible, mais on peut supposer son existence lorsque les données comparatives le suggèrent (tableau \ref{tab:prototgps}).


\begin{table}
\captionb{pré-tangoute *ps-- devant u.}\label{tab:prototgps}
\resizebox{\columnwidth}{!}{
\begin{tabular}{lllllllllllll}
\toprule
&\multicolumn{2}{c}{tangoute} && japhug & sens &  pré-tangoute \\
\midrule
	2999	& \tgf{2999}	&\ipa{swu}	&2.01	& \ipa{ɯ-fsu}	& semblable &	*(p)s-- \\
	290	& \tgf{0290}	&\ipa{sju}	& 2.03	& \ipa{fse < *psoj (ST phsó)}	& ressembler  &	*(p)s-- \\
\bottomrule
\end{tabular}}
\end{table}
La métathèse *pC-- > Cw-- n'est pas un phénomène entièrement inconnu dans les langues de la région, puisque dans certains dialectes tibétains de l'Amdo, comme celui de Rngaba, un changement de ce type est attesté avec les vélaires. Par exemple, le \tib{bka} ``ordre, parole (honorifique)'' y devient \ipa{pk͡wa} avec coarticulation labiovélaire. Il s'agit d'une étape intermédiaire dans le changement pk > pk͡w > kʷ.

\subsubsection{Préinitiale *C-} \label{subsubsec:preinitialec}

Une caractéristique frappante des initiales du tangoute est que les initiales occlusives du rgyalrong et du tibétain correspondent parfois à des sonantes w--, \ipapl{ɣ--}, l ou à la fricative z--. Pour rendre compte de ces correspondances, plusieurs hypothèses seraient a priori possibles. 

La première consisterait à proposer qu'il s'agisse là des correspondances caractéristiques des sonores simples du pré-tangoute (les autres étant des prénasalisées). Dans cette hypothèse, les sonores simples subiraient une lénition à l'initiale. 

L'autre hypothèse serait de poser qu'il convient de reconstruire des occlusives sourdes précédées d'une préinitiale *C--, qui pourrait avoir été *k-- ou *t-- avec une voyelle réduite. L'occlusive initiale, compressée par la présyllabe, aurait alors subi une lénition.Cette hypothèse s'inspire de la spirantisation en vietnamien, où les présyllabes ont causé un effet proche de celui qui semble s'être produit en tangoute (\citealt{ferlus82spirantisation}). Un processus du même type a conduit les groupes dp-- du tibétain ancien à évoluer en \ipa{χw--} en amdoskad, et les anciens db-- à devenir w-- (ton haut) dans le dialecte de Lhasa.

\begin{table}
\captionb{Deux hypothèses pour rendre compte de la lénition des occlusives en tangoute.}\label{tab:hypotheseslenition}
\begin{tabular}{llllll} \toprule
	 Japhug	&tangoute&	pré-tangoute 1	& pré-tangoute 2 \\
\midrule
	p  	&	w--			&*b 		&	*C--p \\
	k 	&	\ipapl{ɣ--}	&*g	 	&*C--k \\
	ts 	&	z-- 			&*dz		& *C--ts \\
	t 	&	l--	 		&*d		& *C--t \\
\bottomrule
\end{tabular}
\end{table}

Dans le cadre de ce travail, on privilégiera la seconde hypothèse. Sa faiblesse est que l'on ne reconstruit *C-- que devant les initiales occlusives, jamais devant les nasales ou les sonantes (toutefois, on trouve des cas de *t--r-- et de *k--r--, et l'on a un exemple de *\ipapl{t-ɕ-}, voir p.\pageref{analyse:viande}). Sa force est de pouvoir mieux rendre compte des alternances morphologiques, comme celle entre ``fumer'' et ``fumée''.

D'une façon indépendante de l'auteur de ces lignes, \citet{miyake06} a fait l'hypothèse d'un mécanisme de lénition en pré-tangoute quasiment identique à celui que nous avons conçu. Ses idées ont été publiées de 2007 à 2009 sur son blog.\footnote{Voir en particulier: \newline 
 http://amritas.com/091121.htm\#11160306 \newline 
 http:/amritas.com/071124.htm\#11192357 \newline 
 http:/amritas.com/071201.htm\#11290329 }
Elles étaient inspirées à la fois du modèle du vietnamien et de celui du coréen. Le fait que la même idée ait ainsi été formulée de façon indépendante suggère qu'elle mérite tout au moins d'être sérieusement considérée.


\begin{table}

\captionb{pré-tangoute *C-p.}\label{tab:prototgCp} 
\resizebox{\columnwidth}{!}{
\begin{tabular}{lllllll} \toprule

& \multicolumn{2}{c}{tangoute}& &  japhug & sens  &tibétain\\
\midrule
294& \tgf{0294} & \ipa{.wa} &1.17 & 		\ipa{paʁ} &cochon&pʰag\\
5170& \tgf{5170} & \ipa{.wạ} &1.63 & 	\ipa{tɯ-rpaʁ} &épaule&pʰrag\\
5134& \tgf{5134} & \ipa{.we} &1.08 & &oiseau &bʲa\\
923& \tgf{0923} & \ipa{.wə̣} &1.68 & 		\ipa{mbe} &vieux&\\
320& \tgf{0320} & \ipa{.wəə} &1.31 & 	\ipa{mpɯ} &mou&\\
4585& \tgf{4585} & \ipa{.wja} &1.19 & \ipa{mɯjphɤt} &vomir&\\
2467& \tgf{2467} & \ipa{.wjạ} &1.64 & \ipa{qarɣɤ-pɤt} & espèce de fleur&\\
527& \tgf{0527} & \ipa{.wjạ} &1.64 & 		\ipa{zbɤβ} &goître&\\
5113& \tgf{5113} & \ipa{.wji} &1.10 & 	\ipa{pa} &faire > fermer&bʲed, bʲas\\
2625& \tgf{2625} & \ipa{.wji} &1.10 & 	\ipa{tɯ-pi} (G)&hôte&\\
2712& \tgf{2712} & \ipa{.wji} &1.10 & 	\ipa{tɯ-xpa} &année&\\
385& \tgf{0385} & \ipa{.wjị} &2.60 & 		\ipa{spa} &pouvoir&\\
4091& \tgf{4091} & \ipa{.wjị} &1.67 & 	\ipa{tɤ-jpa} &neige&\\
5203& \tgf{5203} & \ipa{.wjị} &1.67 & 	\ipa{tɯ-rpa} &hache&\\
5974& \tgf{5974} & \ipa{.wjịj} &2.55 &  &envoyer&ɴpʰaŋs\\
973& \tgf{0973} & \ipa{.wjijr} &2.68 & 		\ipa{βɣa (*<kpa)} &moulin&\\
2926& \tgf{2926} & \ipa{.wju} &2.02 & 	\ipa{tɯ-pu} &intestin&pʰo\\
1805& \tgf{1805} & \ipa{.wọ} &1.70 & 	\ipa{jpum} &épais&sbom-po \\
4053& \tgf{4053} & \ipa{.wọ} &1.70 & 	\ipa{tɤ-jpɣom} &glace & \\
\bottomrule
\end{tabular}}
\end{table}

Comme l'illustre le tableau (\ref{tab:prototgCp}), les étymons à occlusives bilabiales en japhug et en tibétain correspondent plus souvent à des formes à w-- en tangoute (vingt exemples) qu'à des formes en p-- (huit exemples). Conformément au principe présenté ci-dessus, on reconstruit ici l'initiale *C-p. Il est possible qu'aient existé des aspirées à préinitiales *C-ph, mais les données internes au tangoute ne permettent pas de  trancher. Sur les vingt exemples du tableau ci-dessus, seuls trois   (``vomir'', ``vieux'' et ``goître'') ont une initiale autre que p en japhug ; toutefois, nous ne reconstruirons pas  d'initiale spéciale en pré-tangoute sur la foi de ces données externes.


Le nombre important de syllabes à préinitiale *C-- reconstruites ici ne doit pas occulter le fait que la majorité des formes fléchies en japhug ont des préfixes à vélaire, uvulaire ou dentale, en particulier dans le cas des formes verbales. La reconstruction de préinitiales dans ces mots ne va donc pas (sauf pour ``cochon'' et ``goître'') à l'encontre des formes attestées du rgyalrong.

\begin{table}
\captionb{Pré-tangoute *C-k.}\label{tab:prototgCk}
\resizebox{\columnwidth}{!}{
\begin{tabular}{lllllll} \toprule
\multicolumn{4}{c}{tangoute} & japhug & sens & tibétain \\
\midrule
4935& \tgf{4935} & \ipa{ɣa} &1.17 & \ipa{taqaβ} &aiguille&kʰab\\
1084& \tgf{1084} & \ipa{ɣạ} &2.56 & \ipa{sqi} &dix&\\
3008& \tgf{3008} & \ipa{ɣja} &2.16 & \ipa{fkaβ} &couvrir&bkab\\
597& \tgf{0597} & \ipa{ɣjɨ} &1.29 & \ipa{ta-kû (situ)} &oncle maternel&a.kʰu\\
4629& \tgf{4629} & \ipa{ɣjii} &1.14 & \ipa{sqa} &cuire&\\
5497& \tgf{5497} & \ipa{ɣjɨj} &1.42 & \ipa{tɤ-mkɯm} &oreiller&\\
3600& \tgf{3600} & \ipa{ɣju} &1.03 & \ipa{akhu} &appeler&\\
3673& \tgf{3673} & \ipa{ɣju} &1.03 & \ipa{tɤ-khɯ} &fumée&\\
2750& \tgf{2750} & \ipa{ɣu} &1.04 & \ipa{tɯ-ku} &tête&mgo\\
\bottomrule
\end{tabular}}
\end{table}


Le pré-tangoute *C--k est moins courant que l'initiale *k seule qui donne k-- en tangoute. Parmi les formes non-verbales du tableau ci-dessus,  seul le numéral ``dix'' n'a pas de préinitiale dentale ou dorsale en rgyalrong. Or, il n'est pas déraisonnable de reconstruire une préinitiale *C   dans ce mot également (*C-SkaP), car il peut s'agir de la généralisation du préfixe numéral à ``dix'' (le même phénomène s'applique au numéral ``un''). 
\begin{table}
\captionb{Pré-tangoute *C-t.}\label{tab:prototgCt}
\begin{tabular}{lllllll} \toprule
\multicolumn{4}{c}{tangoute} & japhug & sens & tibétain \\
\midrule
630	& \tgf{0630}	& \ipa{la}	&1.17	& \ipa{taʁ}	&tisser	&ɴtʰag, btags\\
5845 &\tgf{5845}&\ipa{lwə}&2.25& \ipa{χtɯ}&  acheter &\\
100	& \tgf{0100}	& \ipa{lew}&	1.43	& \ipa{tɤɣ}&	un	& gtɕig\\
\bottomrule
\end{tabular}
\end{table}


Les exemples de *C-t sont beaucoup plus rares que pour *C-p et *C-k. Pour ``un'', on n'observe pas de préinitiale en japhug, mais la forme tangoute pourrait avoir subi la généralisation de la présyllabe des numéraux à ``un'', comme c'est probablement le cas aussi pour ``dix'' (voir ci-dessus).

\begin{table}
\captionb{ɓré-tangoute *C-ts.}\label{tab:prototgCts}
\begin{tabular}{lllllll} \toprule
\multicolumn{4}{c}{tangoute} & japhug & sens & tibétain \\
\midrule
45	& \tgf{0045}&	\ipa{zar}	&1.80	& \ipa{mɤrtsaβ}&piquant&	\\
4209	& \tgf{4209}& \ipa{zew}& 2.38&  \ipa{tɯ-rtsɤɣ} & une section & tsʰigs \\
5480	& \tgf{5480}&	\ipa{zewr}	&2.78	& \ipa{kɯrtsɤɣ}&	panthère&	gzig\\
1321&	\tgf{1321}&	\ipa{zjị}&	1.67&	\ipa{tɯ-xtsa}&	chaussure&	\\
2134	& \tgf{2134}&	\ipa{zjwị} &1.67	&\ipa{tɯ-ftsa}	&neveu	\\
\bottomrule
\end{tabular}
\end{table}


Le groupe *C-ts n'est attesté que par un nombre limité d'exemples, dont certains présentent des initiales complexes: panthère serait *C-r-tsek (une forme remarquablement proche de celle du rgyalrong), ``neveu'' serait *C-S-ptsja.

Outre ces quatre cas distincts de préinitiale *C-- devant occlusive, on doit noter que devant *r--, il est parfois possible de reconstruire certaines préinitiales *t-- et *k--. Les groupes *tr-- seront abordés dans la section suivante, et un exemple possible de groupe k--r-- a été mentionné à propos du mot ``fourmi'' p. \pageref{analyse:fourmi}.

\subsubsection{Conclusion}

Dans cette section, on a reconstruit les changements phonétiques suivants:
\begin{enumerate}
\item *SCV > \ipapl{CṾ} (loi de Gong) 
\item 	*rCV > CVr (loi de Nishida)
\item 	*pCV > CwV
\item 	*C--p > w-- 
\item 	*C--k > \ipapl{ɣ--}
\item 	*C--t > l--
\item 	*C--ts- > z--
\item 	*rn-- > *n--
\item 	*SCrV > CVr
\item 	*CVr > CVr
\item 	*srV, *rsV > zVr
\end{enumerate}
Les changements phonétiques 4 à 11 ne sont attestés que par un faible nombre  d'exemples.

\subsection{Reconstruction des initiales} \label{subsec:initialestg}
Dans cette section, nous étudierons les origines de chacune des initiales du tangoute. Pour la plupart de ces initiales, on ne reconstruira qu'une consonne simple en pré-tangoute, car les consonnes médianes telles que le *--r-- ou le *--w-- se manifestent avant tout sur la rime. 
Les correspondances de mode d'articulation entre le japhug et le tangoute sont régulières en ce qui concerne les nasales (une nasale correspond toujours à une autre nasale à une exception près) et les prénasalisées (bien qu'on observe quelques exceptions qui seront analysées ci-dessous), mais pour les sourdes et les aspirées, nous n'avons pas été en mesure de trouver un schéma explicatif pour rendre compte des correspondances. On ne cherchera donc pas à résoudre ce problème dans le présent travail. Pour les prénasalisées, on consacrera une section au problème des exceptions  à la correspondance générale.

A la suite de la section sur les prénasalisées, nous aborderons l'un après l'autre les lieux d'articulation des initiales du tangoute ; en effet, il est nécessaire de reconstruire un nombre important de lois phonétiques pour expliquer certaines correspondances complexes entre le tangoute et les autres langues macro-rgyalronguiques.

\subsubsection{Reconstruction des prénasalisées} \label{subsubsec:prenasaliseestg}
Les prénasalisées du tangoute\footnote{Reconstruites comme des voisées par Gong Hwangcherng dans la reconstruction que nous employons.} b--, d--, dz--, \ipapl{dʑ--} et g-- correspondent de façon très régulière à des prénasalisées en japhug, et inversement. Les contre-exemples sont peu nombreux: sur 65 comparaisons, on ne trouve que 13 exceptions (soit des sourdes du tangoute correspondant à des sonores en japhug, soit l'inverse), dont 5 sont explicables.
Quatre d'entre elles sont explicables par la phonologie historique du rgyalrong (tableau \ref{tab:exceptionsprenasaliseesjaphug}).

\begin{table}
\captionb{Correspondances irrégulières des voisées et sourdes non-aspirées du tangoute.}\label{tab:exceptionsprenasaliseesjaphug}
\resizebox{\columnwidth}{!}{
\begin{tabular}{lllllll} \toprule
\multicolumn{4}{c}{tangoute} & japhug & sens & tibétain \\
\midrule
1298	&\tgf{1298}& \ipa{kjiwr}&	1.79	&\ipa{tɯ-zgrɯ}	&coude	&gru.mo\\
5149	&\tgf{5149} &\ipa{duu}&	1.05	&\ipa{ajtɯ}	&s'accumuler	&\\
2200	&\tgf{2200} &\ipa{ba}&	1.17	&\ipa{ɣɤrʁaʁ}	&chasser	&\\
4567	&\tgf{4567} &\ipa{bạ}&	2.56	&\ipa{tɤ-jwaʁ}&	feuille	&\\
\bottomrule
\end{tabular}}
\end{table}


Concernant le verbe \ipa{ajtɯ} ``s'accumuler", il est peut-être préférable de rapprocher la forme tangoute du mot japhug apparenté \ipa{ndɯ} ``s'accumuler en une seule fois''.

Pour l'étymologie de ``coude'', voir p.\pageref{analyse:coude}.

En ce qui concerne les deux derniers mots, on reconstruit les groupes *rb-- et *lb-- en proto-japhug (\citealt[322, 329-30]{jacques04these}). Ces exemples ne sont en aucun cas des contre-exemples à la loi de correspondance.

En ce concerne les vraies exceptions à la correspondance, on peut en distinguer trois groupes. 

Premièrement, un exemple où une prénasalisée du japhug correspond à une sourde du tangoute: 5890 \tgf{5890} \ipa{ku¹} ``relâché'' se compare au japhug \ipa{ɴɢu} de même sens (voir p.\pageref{ex:tg:relache} ).

Deuxièmement, un exemple où une nasale du tangoute correspond à une prénasalisée en japhug: 2047 \tgf{2047} mjii 1.14 ``donner'' correspondant à \ipa{mbi} en japhug. Cette comparaison n'est pas certaine.


Troisièmement, les exemples où les prénasalisées du tangoute correspondent à une sourde ou à une aspirée du japhug (voir tableau \ref{tab:prenasaliseestgaspireesjpg}; la dernière comparaison est problématique).

\begin{table}
\captionb{Exemples de prénasalisées en tangoute correspondant à des sourdes en japhug.}\label{tab:prenasaliseestgaspireesjpg}
\begin{tabular}{lllllll} \toprule
\multicolumn{4}{c}{tangoute} & japhug & sens &  pré-tangoute \\
\midrule
2144	&\tgf{2144}	&\ipa{gie}&	1.09	&\ipa{ɴqa}&	difficile	&\ipa{*ŋg-}\\
4052& \tgf{4052}&	\ipa{dạ}	&2.56	& \ipa{mɯɕtaʁ}	&froid&\ipa{*S-nd-}\\
4399	&\tgf{4399}	&\ipa{dzjị}&	2.60	&\ipa{tɤ-jtsi}& 	pilier	&\ipa{*S-ndz-}\\
2664	&\tgf{2664}	&\ipa{dze}&	1.08	&\ipa{tɯ-tsi}&	vie	&\ipa{*ndz-}\\
2451	&\tgf{2451}	&\ipa{bọ}&	2.62	&\ipa{phɣo}&	fuir	&\ipa{*S-mb-}\\
5436	&\tgf{5436}	&\ipa{dʑjị}&	1.67		&		\ipa{tɯ-ɲcɣa} & serpe !& \ipa{*S-ndʑ-}\\
\bottomrule
\end{tabular}
\end{table}


On reconstruira en pré-tangoute des initiales prénasalisées pour toutes les formes ci-dessus. Pour certains de ces exemples   (en particulier  ``difficile'' et peut-être ``serpe''), il s'agit cependant selon toute probabilité de cas où une préinitiale nasale a voisé l'initiale sourde en pré-tangoute. On peut poser *Nka > *ŋga > \ipa{gie} pour ``difficile''.

\subsubsection{Reconstruction des vélaires} \label{subsubsec:velairestg}
Les caractères transcrivant les consonnes initiales (\ipa{fǎnqiè shàngzì}) pour les vélaires (groupe V) comportent 20 chaînes de \ipa{fǎnqiè} selon \citet[76-80]{sofronov68a}. Selon Gong, les chaînes 14 et 20 correspondent à l'initiale ŋ-- du tangoute, tandis que les chaînes 15 à 19 correspondent à l'initiale g-; ces initiales sont mal distinguées par les transcriptions chinoises du ZZZ, mais les transcriptions tibétaines permettent de les distinguer (voir \citealt[182-7]{tai08duiyin}). 

En pré-tangoute, le statut de la médiane --w-- est problématique. Elle provient en partie de la préinitiale *p-- (voir section \ref{subsubsec:preinitialep}) et de la diphtongaison de certaines rimes (voir section \ref{subsubsec:correspondance:eu:u}). Toutefois, on trouve certains exemples où une vélaire suivie d'une médiane --w-- correspond à une labiovélaire dans une autre langue, comme en zbu ou en proto-tibétain (en japhug et en situ, les labiovélaires se sont confondues avec les vélaires). Etant donné le peu d'exemples, on ne reconstruira pas une série de labiovélaires en pré-tangoute (on doit reconstruire une médiane --w-- devant d'autres catégories d'initiales). Pour le mot ``genou'', la médiane est attestée en muya, où l'on trouve la forme \ipa{ɴuə̠²⁴pə⁵⁵lɑ³³} ainsi que dans les langues Na (Yongning \ipa{ŋwɤ˧ku˥}, lazé \ipa{ŋwɑ˩tu˥}).


\begin{table}
\captionb{Traces d'anciennes labiovélaires en tangoute.}\label{tab:labiovelairesptg}
\resizebox{\columnwidth}{!}{
\begin{tabular}{lllllllllllll}
\toprule
\multicolumn{4}{c}{tangoute} & japhug & sens &  zbu & tibétain \\
\midrule
1752	& \tgf{1752}	&\ipa{kwạ} &	2_56	&\ipa{qaʁ} &	houe	&\ipa{qwɐ̂χ}	& \\
74	& \tgf{0074}	&\ipa{khwə} &	1.27	&\ipa{ɯ-qiɯ} &	moitié	&	& \\
1200	& \tgf{1200}	&\ipa{khjwɨ} &	1.30	&\ipa{khɯna} &	chien		&&kʰʲi < *kwi \\
4906	& \tgf{4906}	&\ipa{gjwi} &	2.10	&\ipa{ŋga} &	s'habiller	&\ipa{ngwêt}&	bgo < *p-gʷa\\
333	& \tgf{0333}	&\ipa{ŋwer} &	2.71	&\ipa{təmŋá}  (situ)	&genou		&& \\
257	& \tgf{0257}	&\ipa{ŋwər} &	1.84	&\ipa{arŋi} &	bleu	&\ipa{rŋwiʔ}	& sŋo\\
395	& \tgf{0395}	&\ipa{ŋwe} &	2.07	&\ipa{nɯŋa} &	vache	&\ipa{ŋwəleʔ}	& \\
\bottomrule
\end{tabular}}
\end{table}



Outre les vélaires du tangoute, d'autres initiales proviennent également de vélaires en pré-tangoute. 


Premièrement, dans la catégorie des gutturales (groupe VIII), l'initiale reconstruite *\ipapl{ɣ-} par Gong Hwangcherng provient de *C-k (voir p.\pageref{tab:prototgCk}). Cette initiale correspond aux chaînes de \ipa{fǎnqiè} 1 et 7, et se distingue des autres initiales gutturales (initiale *zéro et *x-) par sa transcription tibétaine, qui est le plus souvent g- avec diverses préinitiales (\citealt[194-197]{tai08duiyin}).


Deuxièmement, dans la catégorie des dentales, il semble que l'initiale n- provienne dans certains cas de ŋ- devant voyelle antérieure. En effet, si l'on examine le tableau des compatibilités des rimes du groupe de rimes n°2 (section \ref{rimes:02:1:dent}) et du groupe  n°7 (section \ref{rimes:07:1:dent}) avec les initiales (Li 1986:195-199), on s'aperçoit que ces rimes n'apparaissent pas avec l'initiale ŋ- du tangoute, à part quatre exceptions:
\begin{enumerate}
\item Les syllabes avec médiane --w-- \ipa{ŋwe} (rime 8, 1.8/2.7), \ipa{ŋwej} (rime 34, 1.33/2.30) et \ipa{ŋwer} (rime 82, 1.77/2.71)
\item Deux exemples de la rime 13, dont 5608 \tgz{5608} ``timide"
\item Un exemple de la rime 84, le caractère 2444 \tgz{2444} ``catastrophe"
\item Un exemple de la rime 38, le caractère 914 \tgz{0914} (caractère de transcription)
\end{enumerate}

Or, pour les exceptions 2) à 4), si l'on se base sur la reconstruction de Gong Hwangcherng, on reconstruira ici une initiale g-, car le caractère qui transcrit leur consonne initiale appartient respectivement aux chaînes 17 (groupe 2 et 4) et 19 (groupes 3). 5608 \tgf{5608} est donc reconstruit \textit{giee}, 914 \tgf{0914} \textit{gieej} et 2444 \tgf{2444} \textit{gjir}. Il est possible donc d'affirmer qu'avec la reconstruction de Gong Hwangcherng, qui restreint la reconstruction de l'initiale ŋ- aux chaînes de \ipa{fǎnqiè} 14 et 20, les rimes des \ipa{shè} 2 et 7 sont incompatibles avec l'initiale ŋ, à l'exception des rimes \textit{hékǒu} avec la voyelle e. Cette distribution ne peut pas être due au hasard, et doit être le fait d'une évolution phonétique qui a supprimé les syllabes de type *ŋi et *ŋe: on suppose que le ŋ s'est palatalisé devant une voyelle antérieure et s'est confondu avec le phonème n dans ce contexte. La seule exception à cette règle porte sur les syllabes ŋwe / ŋwej / ŋwer; la médiane --w-- a bloqué la palatalisation.

Cette loi phonétique nous permet donc de comparer des étymons à initiales ŋ-- dans d'autres langues sino-tibétaines avec le tangoute \ipa{n}--. Le tableau \ref{tab:ngpalatalisation} indique les exemples les plus probants du changement *ŋ > nj /\underline{  ~~  }  [+antérieure,-arrondi].

\begin{table}
\captionb{Exemples de la palatalisation du *ŋ devant voyelle antérieure en tangoute}\label{tab:ngpalatalisation}
\resizebox{\columnwidth}{!}{
\begin{tabular}{lllllll} \toprule
\multicolumn{4}{c}{tangoute} & japhug & sens &   tibétain \\
\midrule
306	&\tgf{0306} & \ipa{njir} &	2.72	&&	prêter&	rɲa (*rŋja)\\
5554	&\tgf{5554} & \ipa{njij} &	2.33	& \ipa{sɤŋo < *ŋaŋ} &	écouter	&\\
317	&\tgf{0317} & \ipa{nẹj} &	2.53	& \ipa{rŋil} &	faner	&\\
1204	&\tgf{1204} & \ipa{njijr} &	2.68	& \ipa{tɯ-rŋa} &	visage	&ŋo ?\\
\bottomrule
\end{tabular}}
\end{table}


On reconstruira donc pour les exemples du tableau ci-dessus une initiale *ŋ au lieu d'une initiale *n. On remarque que ce changement s'est produit après l'évolution *rn-- > *n--, sans quoi on ne devrait pas retrouver une rime de second cycle mineur dans les mots \tgf{1204} \ipa{njijr²} et \tgf{0306} \ipa{njir²}. En effet, si c'était le cas, *ŋje > *nje aurait nourri *rn-- > *n-- et *rŋja aurait donné *rŋje > *rnje > *nje >nji. En fait, tous les exemples de n-- avec une rime du second cycle mineur doivent venir soit d'emprunts, soit de mots à initiale *rŋ-- en pré-tangoute.

\section{Reconstruction des rimes} \label{sec:rimes}

Dans l'étude de la phonologique historique du tangoute, comme dans nombre d'autres langues  de cette région, il est rarement possible de distinguer la voyelle et la consonne finale de façon triviale. Une même voyelle peut en effet subir des évolutions radicalement différentes selon la consonne qui la suit voire même celle qui la précède. Dans ces conditions, on doit étudier le système de façon méticuleuse, et ne pas s'interdire de postuler des règles phonétiques complexes.

Dans ce chapitre, nous partirons tout d'abord des rimes du japhug, regroupées en groupes (\ipa{shè} \zh{攝}), pour étudier les origines possibles de chacune d'entre elles. Après un passage en revue de chaque rime, nous conclurons par une synthèse dans laquelle on décrira l'évolution dans le sens inverse, du pré-tangoute au tangoute attesté.

	On suivra ici les douze \ipa{shè} regroupés par Gong Hwangcherng. D'autres auteurs, tels qu'Arakawa, adoptent une distribution légèrement différente mais pas incompatible. Parmi ces douze groupes, trois (les n°3, n°5 et n°12) n'apparaissent qu'avec des emprunts chinois. Nous ne les prendrons donc pas en compte dans cette étude, et nous nous limiterons aux neuf autres. 
	
A l'intérieur de chaque \ipa{shè} , on classera les exemples par groupe de correspondance, et à l'intérieur de chaque groupe de correspondance, par consonne initiale. L'ordre suivra d'abord le lieu d'articulation (labial, dental, palatal, vélaire), puis le mode d'articulation (occlusive sourde, aspirée, sonore prénasalisée, affriquées, nasales, autres sonantes).

\subsection{Voyelle u} \label{subsec:voyelle.u}
Les rimes du \ipa{shè} n°1 sont reconstruites par tous les auteurs avec une voyelle postérieure arrondie fermée. La différence entre les rimes 1 et 4, 2 et 3  et 6 et 7 n'est pas reconstruite par tous les auteurs, et les données transcriptionnelles ne permettent pas de reconstituer avec certitude les détails phonétiques qui ont poussé les savants tangoutes à distinguer ces rimes.

\begin{table}
\captionb{Reconstructions du \ipa{shè} n°1}\label{tab:she1}
\resizebox{\columnwidth}{!}{
\begin{tabular}{lllllllll} \toprule
rime&ton 1&ton 2&Sofronov1&Sofronov2&Nishida&Li&Gong&Arakawa\\
1&1.01&2.01& \ipa{u} & \ipa{u} & \ipa{u} & \ipa{u} & \ipa{u} & \ipa{u}\\
2&1.02&2.02& \ipa{i̯u} & \ipa{i̯u} & \ipa{ǐu} & \ipa{ǐu} & \ipa{ju} & \ipa{yu}\\
3&1.03&2.03& \ipa{Y} & \ipa{i̯u} & \ipa{ǐuɦ} & \ipa{ǐu̠} & \ipa{ju} & \ipa{yu}\\
4&1.04&2.04& \ipa{u+C} & \ipa{u} & \ipa{uɦ} & \ipa{uo} & \ipa{u} & \ipa{uː}\\
5&1.05&2.05& \ipa{un} & \ipa{u} & \ipa{ʊ} & \ipa{ʊ} & \ipa{uu} & \ipa{u'}\\
6&1.06&& \ipa{ûn} & \ipa{û} & \ipa{ʊɦ} & \ipa{ǐʊ} & \ipa{juu} & \ipa{yu'}\\
7&1.07&2.06& \ipa{i̯un} & \ipa{i̯u} & \ipa{ǐʊɦ} & \ipa{ǐʊ̠} & \ipa{juu} & \ipa{uː'}\\
61&1.58&2.51& \ipa{ụ} & \ipa{ụ} & \ipa{ụ} & \ipa{ụ} & \ipa{ụ} & \ipa{uq}\\
62&1.59&2.52& \ipa{i̯ụ} & \ipa{i̯ụ} & \ipa{ǐụ} & \ipa{ǐu̠̣} & \ipa{jụ} & \ipa{yuq}\\
80&1.75&2.69& \ipa{ụ} & \ipa{ụ} & \ipa{ur} & \ipa{ʊ̣} & \ipa{ur} & \ipa{ur}\\
81&1.76&2.7& \ipa{i̯ụ} & \ipa{i̯ụ} & \ipa{ǐur} & \ipa{ǐʊ̠̣} & \ipa{jur} & \ipa{yur}\\
\bottomrule
\end{tabular}}
\end{table}

Gong marque une distinction entre syllabes sans médiane --w-- (\textit{kāikǒu} \zh{開口}) et syllabes à médiane --w-- \textit{hékǒu} \zh{合口}  dans ces rimes à voyelle arrondie. La réalité  de cette distinction est contestable,  et nous ne la reconstruirons pas en pré-tangoute.
	Dans cette section, l'étymologie de chaque comparaison sera détaillée. Les comparaisons pouvant être des emprunts sont indiquée par un astérisque *, et celles qui sont douteuses par point d'exclamation. Le tableau \ref{tab:comparaisons:u} rassemble des groupes de cognats proposés entre tangoute, rgyalrong et tibétain. Il contient également des mots ayant des cognats en lolo-birman ou en naish mais sans cognats en rgyalrong ou en tibétain.

 
\begin{longtable}{lllllll}	
\captionb{Comparaisons des étymons en --u du tangoute avec le japhug et le tibétain.}\label{tab:comparaisons:u} \\
\toprule
\multicolumn{4}{c}{tangoute} & sens & japhug & tibétain  \\
\midrule
\tinynb{3003}& \tgf{3003} & \ipa{.ju} &\tinynb{1.02}&fantôme & \ipa{tɯ-ɟo} &\\
\tinynb{2926}& \tgf{2926} & \ipa{.wju} &\tinynb{2.02}&intestin & \ipa{tɯ-pu} &pʰo.ba\\
\tinynb{5146}& \tgf{5146}& \ipa{bju} &\tinynb{2.03}&appeler&&ɴbod\\
\tinynb{4425}& \tgf{4425} & \ipa{ɕju} &\tinynb{1.02}&frais & \ipa{ɣɤɕu} &\\
\tinynb{930}& \tgf{0930} & \ipa{dju} &\tinynb{1.03}&y avoir & \ipa{tu} &\\
\tinynb{3053}& \tgf{3053} & \ipa{gju} &\tinynb{1.03} &traverser  & \ipa{ʑŋgu} & \\
&&&&une rivière\\
\tinynb{1907}& \tgf{1907} & \ipa{gju} &\tinynb{2.03}&tendon & \ipa{tɯ-ŋgru} &rgʲus.pa\\
\tinynb{5102}& \tgf{5102} & \ipa{gjur} &\tinynb{1.76} &rein & \ipa{tə-rgó} (situ) & \\
\tinynb{3600}& \tgf{3600} & \ipa{ɣju} &\tinynb{1.03}&appeler & \ipa{akhu} &\\
\tinynb{2750}& \tgf{2750} & \ipa{ɣu} &\tinynb{1.04}&tête & \ipa{tɯ-ku} &mgo\\
\tinynb{4040}& \tgf{4040} & \ipa{khjuu} &\tinynb{1.06}&inviter & \ipa{qru} &\\
\tinynb{2278}& \tgf{2278} & \ipa{kjụ} &\tinynb{1.59}&oignon & \ipa{ɕku} &sgog.pa\\
\tinynb{5396}& \tgf{5396} & \ipa{kjur} &\tinynb{1.76}&mettre dans & \ipa{rku} &\\
\tinynb{5890}& \tgf{5890} & \ipa{ku} &\tinynb{1.04}&relâché & \ipa{ɴɢu} &\\
\tinynb{0328}& \tgf{0328} & \ipa{ku} &\tinynb{2.04}&aveugle & \ipa{ɕqwa} &\\
\tinynb{2503}& \tgf{2503} & \ipa{kụ} &\tinynb{1.58}&après & \ipa{ɯ-qhu} &\\
\tinynb{3065}& \tgf{3065} & \ipa{lhju} &\tinynb{1.03}&lait & \ipa{tɤ-lu} &ʑo\\
\tinynb{4726}& \tgf{4726} & \ipa{lhjụ} &\tinynb{2.52}&bambou & \ipa{ɟu} &\\
\tinynb{2801}& \tgf{2801} & \ipa{lhuu} &\tinynb{1.05}&moelle & \ipa{tɯ-pju} &\\
\tinynb{475}& \tgf{0475} & \ipa{lju} &\tinynb{1.03}&matelas & \ipa{tɤ-βɟu} &\\
\tinynb{4174}& \tgf{4174} & \ipa{mju} &\tinynb{2.03}&bouger & \ipa{mɯnmu} & \\
\tinynb{3306}& \tgf{3306} & \ipa{mju} &\tinynb{2.03}&danse* & & bro \\
\tinynb{4032}& \tgf{4032} & \ipa{mur} &\tinynb{2.69}&grêle & \ipa{tə rmô (situ)} & \\
\tinynb{4785	}&	\tgf{4785} &\ipa{njụ} &	\tinynb{2.52}	& bon &	\ipa{mnu} \\
\tinynb{1228}& \tgf{1228} & \ipa{ŋur} &\tinynb{1.75}&frire & \ipa{rŋu} &rŋo\\
\tinynb{508}& \tgf{0508} & \ipa{ŋwu} &\tinynb{2.01}&être & \ipa{ŋu} &\\
\tinynb{4600}& \tgf{4600} & \ipa{ŋwụ} &\tinynb{1.58}&promesse & \ipa{kɯjŋu} &\\
\tinynb{4413}& \tgf{4413} & \ipa{pju} &\tinynb{2.03}&chauffer & \ipa{pu} &\\
\tinynb{2795 }& \tgf{2795} & \ipa{rur} &\tinynb{1.75} & pâturage & \ipa{rɯŋgu} \\
\tinynb{5736 }& \tgf{5736} & \ipa{rjur} &\tinynb{2.70} & se lever  & \ipa{rɤru} \\
&&&&> ferment\\
\tinynb{290}& \tgf{0290} & \ipa{sju} &\tinynb{2.03}&semblable & \ipa{fse (ST phsó)} &\\
\tinynb{487}& \tgf{0487} & \ipa{sjwụ} &\tinynb{2.52}&vivant & \ipa{sɯsu} &\\
\tinynb{984}& \tgf{0984} & \ipa{tshwu} &\tinynb{1.01}&gras & \ipa{tshu} &tsʰo-po\\
\tinynb{4615}& \tgf{4615} & \ipa{tsụ} &\tinynb{2.51}&tousser &  &\\
\tinynb{3266}& \tgf{3266} & \ipa{dzju} &\tinynb{2.03}&maître &  &\\
\tinynb{2196}& \tgf{2196}& \ipa{tu} &\tinynb{1.01}& épais &\\
\tinynb{1569}& \tgf{1569} & \ipa{twụ} &\tinynb{1.58}&droit & \ipa{astu} &\\
\tinynb{5165}& \tgf{5165} & \ipa{twụ} &\tinynb{1.58}&endroit & \ipa{ɯ-stu} &\\
\tinynb{0369}& \tgf{0369} & \ipa{thjuu} &\tinynb{1.07}&inspecter  & \ipa{thu} &\\
&&&&> demander\\
\tinynb{4557}& \tgf{4557} & \ipa{zjur} &\tinynb{2.70}&torche & \ipa{tɤtʂu} &\\
\tinynb{2101}& \tgf{2101} & \ipa{bur} &\tinynb{1.75}&entasser & \ipa{rmbɯ} &\\
\tinynb{355}& \tgf{0355} & \ipa{mju} &\tinynb{1.03}&frère  & \ipa{tɤ-wɤmɯ} &\\
\tinynb{5149}& \tgf{5149} & \ipa{duu} &\tinynb{1.05}&s'accumuler & \ipa{ajtɯ} &\\
\tinynb{1464}& \tgf{1464} & \ipa{tsur} &\tinynb{1.75}&coup de pied & \ipa{tɯ-qartsɯ} &\\
\tinynb{2396}& \tgf{2396} & \ipa{dzuu} &\tinynb{2.05}&s'asseoir & \ipa{amdzɯ} &\\
\tinynb{3673}& \tgf{3673} & \ipa{ɣju} &\tinynb{1.03}&fumée & \ipa{tɤ-khɯ} &\\
\tinynb{1136}& \tgf{1136} & \ipa{gu} &\tinynb{2.01}&intérieur & \ipa{ɯ-ŋgɯ} &\\
\tinynb{1909}& \tgf{1909} & \ipa{gur} &\tinynb{1.75}&boeuf & \ipa{kərgú (situ)} &\\
\tinynb{4189}& \tgf{4189} & \ipa{khu} &\tinynb{1.04}&bol & \ipa{khɯtsa} &\\
\tinynb{5621}& \tgf{5621} & \ipa{lhu} &\tinynb{1.01}&ajouter & \ipa{ɣɤjɯ} &\\
\tinynb{124}& \tgf{0124} & \ipa{ljụ} &\tinynb{2.52}&tête & & \\
\tinynb{4506}& \tgf{4506} & \ipa{lu} &\tinynb{2.01}&allumer un feu & \ipa{βlɯ} &\\
\tinynb{1304}& \tgf{1304} & \ipa{lụ} &\tinynb{1.58}&ver & \ipa{qajɯ} & klu \\
\tinynb{1784}& \tgf{1784} & \ipa{lụ} &\tinynb{1.58}&homme &  &\\
\tinynb{2273}& \tgf{2273} & \ipa{lụ} &\tinynb{2.51}&branche & \ipa{wu-lû} (situ) &\\
\tinynb{4614}& \tgf{4614} & \ipa{nju} &\tinynb{2.03}&sein & \ipa{tɯ-nɯ} &nu.ma\\
\tinynb{3388}& \tgf{3388} & \ipa{ŋwu} &\tinynb{2.01}&pleurer & \ipa{ɣɤwu}! &ŋu\\
\tinynb{5814}& \tgf{5814} & \ipa{phu} &\tinynb{2.01}&arbre & \ipa{tɯ-phɯ} &\\
\tinynb{3176}& \tgf{3176} & \ipa{ɕjụ} &\tinynb{1.59}&lente& \ipa{ndʑrɯ} & sro.ma\\
\tinynb{1490}& \tgf{1490} & \ipa{tsur} &\tinynb{1.75}&hiver & \ipa{qartsɯ} &\\
\tinynb{2385}& \tgf{2385} & \ipa{ɕju} &\tinynb{2.02}&viande& & sha\\
\tinynb{4681}& \tgf{4681} & \ipa{nju} &\tinynb{1.03}&oreille & \ipa{tɯ-rna} & rna\\
\tinynb{1338}& \tgf{1338} & \ipa{dzu} & \tinynb{1.01} & amical & &mdza \\
\tinynb{1899}& \tgf{1899} & \ipa{tju} &\tinynb{2.03}&frapper! & \ipa{rtɤβ} &\\
\tinynb{3925}& \tgf{3925} & \ipa{mur} &\tinynb{1.75}&ténèbres !& & mun.pa\\
\tinynb{4789}& \tgf{4789} & \ipa{nju} &\tinynb{1.03}&légume! & \ipa{sɯjno} &\\
\tinynb{5716}& \tgf{5716} & \ipa{thu} &\tinynb{2.01}&mouton* & \ipa{thoɲa} &\\
\bottomrule
\end{longtable}

On observe dans le tableau ci-dessus deux grandes correspondances principales. Les rimes en u du tangoute correspondent d'une part à tibétain \textit{u}:: japhug \ipa{ɯ}, d'autre part à tibétain \textit{o}:: japhug \textit{u}. On trouve aussi un résidu de quelques formes isolées. Les étymologies présentées ici seront étudiées plus en détail en trois temps. Premièrement, la correspondance avec le tibétain \textit{o}; deuxièmement, celle avec le tibétain \textit{u}; et enfin toutes les correspondances marginales.

\subsubsection{Tangoute --u :: tibétain --o :: japhug --u} \label{subsubsec:correspondance:u:o:u}
Cette correspondance est attestée par un grand nombre d'exemples. Le japhug --\textit{u} provient d'un proto-japhug *--o.
\newline

A. Initiales labiales \label{rimes:01:1:lab}
\newline

\begin{enumerate}


\item 4413 \tgf{4413} \ipa{pju} 2.03 \ptang{pjo}{chauffer, brûler} peut se comparer avec le  \jpg{pu} ``chauffer dans la braise'':

\begin{exe}
\ex \label{ex:jpg:cuirebraise} 
\gll	\ipa{jaŋjy}	\ipa{thɯ-kɤ-pu}	\ipa{ɲɯ-mɯm} \\
		pomme.de.terre	\aor{}-\nmls{}:O-cuire	\const{}-bon.à.manger \\
\trans Les pommes de terre cuites dans les braises sont bonnes à manger (dictionnaire Japhug)
\end{exe}

On peut en rapprocher aussi le   \plb{pút}{0642} ``roast in fire''.

Ce verbe présente une alternance --ju/--jo dans sa conjugaison (voir la discussion en \ref{subsec:personne} ), son thème B\footnote{Voir la section \ref{sec:morpho.verbale.flex} pour plus de détail sur les alternances vocaliques en tangoute.} étant 604  \tgf{0604} \ipa{pjo} 1.51. Ce verbe s'emploie en tangoute à propos des  aliments, mais aussi des métaux. Il peut simplement se traduire par  ``brûler'' dans certains contextes:
\newline
\linebreak
\begin{tabular}{llllllllll}
 \tgf{4408}	& \tgf{5604}&	 \tgf{5113}	& \tgf{2098}	& \tgf{4945}	& \tgf{1326}	& \tgf{4413}&	 \tgf{5880}&	 \tgf{4342}	& \tgf{3363} \\
\tinynb{4408}	& \tinynb{5604}&	 \tinynb{5113}	& \tinynb{2098}	& \tinynb{4945}	& \tinynb{1326}	& \tinynb{4413}&	 \tinynb{5880}&	 \tinynb{4342}	& \tinynb{3363} \\
\end{tabular}
\begin{exe}
\ex \label{ex:tg:brulerpju}  \vspace{-8pt}
\gll 	\ipa{məə¹}	\ipa{dʑjɨ.wji¹}	\ipa{ŋa²}      \ipa{.wə¹}	\ipa{kjɨ¹-pju²}	\ipa{ŋwu²}	\ipa{dja²-sji²} \\
		feu	\erg{}	1sg	mari	\dir{}-brûler[A]	\conj{}	\dir{}-mourir[A] \\
\glt Le feu a brûlé mon mari et il en est mort (Leilin, 4.11A.5).
\end{exe}
Il est probable que le japhug a restreint le sens de cette racine, qui autrefois devait être plus large comme en tangoute. 

On peut peut-être rapprocher de \tgf{4413} \ipa{pju²} le verbe 4311 \tgf{4311} \ipa{.wju} 1.02 ``carboniser, brûler'', qui viendrait d'un pré-tangoute *C-pjo.



\item 2926  \tgf{2926} \ipa{.wju} 2.02 \ptang{C-pjo}{intestin} peut se comparer avec le  \jpg{tɯ-pu} ``intestin'', le \plb{ʔu¹}{0146} et plus lointainement avec le \tib{pʰo.ba} ``estomac''.
\newline
\linebreak
\begin{tabular}{llllllllll}
\tgf{1245}&	\tgf{1204}&	\tgf{1153}&	\tgf{0323}&	\tgf{4684}&	\tgf{5671}&	\tgf{3900}&	\tgf{1954}&	\tgf{2926}&	\tgf{2797}\\
\tinynb{1245}&	\tinynb{1204}&	\tinynb{1153}&	\tinynb{0323}&	\tinynb{4684}&	\tinynb{5671}&	\tinynb{3900}&	\tinynb{1954}&	\tinynb{2926}&	\tinynb{2797}\\

\end{tabular}
\begin{exe}
\ex \label{ex:tg:intestin}  \vspace{-8pt}
\gll 	\ipa{jij¹}	\ipa{njijr²}	\ipa{dʑji¹}	\ipa{ɕjar¹}	\ipa{mej¹}	\ipa{thjɨ¹}	\ipa{.o¹}	\ipa{.wjar¹}	\ipa{.wju²}	\ipa{lho} \\
		soi-même	face	peau	écorcher	œil	arracher estomac ouvrir	intestin	sortir	\\
\trans Il s'écorcha la peau du visage, s'arracha les yeux, et s'ouvrit l'estomac et les intestins. (Cixiaozhuan 23.1, Jacques 2007: 71).
\end{exe}


\item 5146  \tgf{5146} \ipa{bju} 2.03 \ptang{mbjo}{appeler} est comparable au \tib{ɴbod, bos} ``appeler". En tangoute, ce verbe s'emploie surtout dans le sens de ``convoquer'' (chinois \zh{召} \textit{zhào}). 
\newline
\linebreak
\begin{tabular}{lllllllllll}
\tgf{2219}&	\tgf{3738}&	\tgf{2857}&	\tgf{0091}&	\tgf{1234}&	\tgf{0546}&	\tgf{1139}&	\tgf{5981}&	\tgf{5146}&	\tgf{4342} \\
\tinynb{2219}&	\tinynb{3738}&	\tinynb{2857}&	\tinynb{0091}&	\tinynb{1234}&	\tinynb{0546}&	\tinynb{1139}&	\tinynb{5981}&	\tinynb{5146}&	\tinynb{4342} \\
\tgf{2082} \\
\tinynb{2082}\\
\end{tabular}
\begin{exe}
\ex \label{ex:tg:appeler}  \vspace{-8pt}
\gll	\ipa{kjij¹kow¹}	\ipa{ŋo²}	\ipa{thew²}	\ipa{thjɨj¹.u²}	\ipa{.jij¹}	\ipa{.a-bju²}	\ipa{dja²-.jɨr¹} \\
		Jing.Gong	maladie	attraper	Tian.Wu	\antierg{}	\dir{}-appeler	\dir{}-demander \\
\glt Jing Gong (\zh{景公}) tomba malade. Il fit appeler Tian Wu (\zh{田巫}) et lui en demanda (la raison). (Leilin 6.7B.1-2)
\end{exe}



\item 355 \tgf{0355} \ipa{mju} 1.03 \ptang{mjo}{frère} peut être rapproché du  \jpg{tɤ-wɤmɯ} ``frère". Dans ces deux langues, il a   le sens spécifique de frère d'une femme. Cette comparaison est discutée dans \citet{jacques11kinship} et ne sera pas examinée ici en détail, même si elle jouera un rôle dans notre discussion sur la place du tangoute dans la famille sino-tibétaine. 



\item 4174 \tgf{4174} \ipa{mju} 2.03 \ptang{mjoN > mjo}{bouger} est apparenté au  \jpg{mɯnmu} ``bouger". Ce verbe est conjugué comme intransitif, et l'on doit lui adjoindre le préfixe causatif z- (\ipa{zmɯnmu}) pour le rendre transitif ``déplacer qqch". 
\begin{exe}
\ex \label{ex:jpg:bouger} 
\gll		\ipa{βʑɯ}	\ipa{nɯ}	\ipa{kɯ}	\ipa{ɯ-mtɕhi}	\ipa{ɲɯ-tɯ-mɯnmu}	\ipa{tɕe, }	``\ipa{aʑo}	\ipa{tɤ-nɯ-nɯrdóʁ-a}	\ipa{ɕti"}	\ipa{to-ti,} \\
			souris	\dem{}	\erg{}	3\sgposs{}-bouche	\impf{}-\conv{}-bouger	\conj{}	je \aor{}-\auto{}-ramasser.les.morceaux-1\sg{}	être	\med{}-dire \\
\glt La souris, dès que sa bouche bougeait, disait: « C'est moi qui les ai ramassés ! » (La souris, le moineau et le chat, 69-70).
\end{exe}
En tangoute, ce verbe s'emploie aussi normalement comme un intransitif, et doit être accompagné de l'auxiliaire 749 \tgf{0749} \ipa{phji} 1.11 pour s'employer transitivement.
\newline
\linebreak
\begin{tabular} {lllllll}
	\tgf{2302}&	\tgf{4456}&	\tgf{4342}&	\tgf{4174}&	\tgf{0991}&	\tgf{1259}&	\tgf{1941}\\
	\tinynb{2302}&	\tinynb{4456}&	\tinynb{4342}&	\tinynb{4174}&	\tinynb{0991}&	\tinynb{1259}&	\tinynb{1941}\\
\end{tabular}
\begin{exe}
\ex \label{ex:tg:bouger}  \vspace{-8pt}
\gll	\ipa{ljɨ¹}	\ipa{ljịj²}	\ipa{dja²-mju²}	\ipa{le²}	\ipa{tɕiow¹dzjɨ̣²} \\
		vent	grand	\dir{}-bouger	brume	se.rassembler \\
\glt Un grand vent souffla, et les nuages se rassemblèrent (Leilin, 03.33B.3)
\end{exe}
Un autre verbe tangoute est apparenté à \tgf{4174} \ipa{mju²}, 4450 \ipa{mjuu} 2.06  \tgf{4450}. Les exemples sont peu nombreux, mais ils sont intransitifs également. La différence entre \ipa{mju²} et \ipa{mjuu²} méritera une étude plus détaillée lorsque davantage de textes tangoutes seront analysés. Dans l'exemple ci-dessous, le causatif \tgz{0749}  est employé pour le rendre transitif:
\newline
\linebreak
\begin{tabular} {llllllll}
	\tgf{0509}&	\tgf{4450}&	\tgf{0749}&	\tgf{5880}&	\tgf{3900}&	\tgf{3791}&	\tgf{1452}&	\tgf{4469}\\
	\tinynb{0509}&	\tinynb{4450}&	\tinynb{0749}&	\tinynb{5880}&	\tinynb{3900}&	\tinynb{3791}&	\tinynb{1452}&	\tinynb{4469}\\
\end{tabular}
\begin{exe}
\ex \label{ex:tg:bouger2}  \vspace{-8pt}
\gll	\ipa{thjowr²}	\ipa{mjuu²}	\ipa{phji¹}	\ipa{ŋwu²}	\ipa{.o¹}	\ipa{bji²}	\ipa{nja¹-ɕji²} \\
		bouger	bouger	causer[A]	\conj{}	estomac	bas	\dir{}-aller[A] \\
\glt Il fit bouger (le médicament dans la bouche d'un malade) et (le médicament) descendit dans le ventre (du malade) (Leilin, 06.11A.5)
\end{exe}
Le limbu  \racine{munt} ``to shake" (voir \citealt{michailovsky02dico}) suggère qu'il convient de reconstruire une nasale finale *--n final à un stade antérieur au pré-tangoute; cette nasale ne laisse pas de trace observable en rgyalrong et en tangoute.


\item 4032 \tgf{4032} \ipa{mur} 2.69 \ptang{rmo}{grêle} se compare avec le rgyalrong \situ{tə-rmô}. Le japhug a ici perdu cette racine et l'a remplacée par un emprunt tibétain \ipa{sɤrwa} (de \tib{ser.ba}).


\end{enumerate}

 Aux exemples ci-dessus, on peut ajouter le verbe 3306 \tgf{3306} \ipa{mju} 2.03 ``danse" qui peut être soit un cognat du \tib{bro}, soit un emprunt ancien au chinois moyen \zh{舞} mjuX.  Pour servir de prédicat, ce nom doit s'employer avec le verbe ``faire'' \tgf{5113} \ipa{.wji¹}, comme dans l'exemple suivant:
\newline
\linebreak
\begin{tabular} {lll}
		\tgf{3306}&	\tgf{5113}&	\tgf{1338}\\
		\tinynb{3306}&	\tinynb{5113}&	\tinynb{1338}\\
\end{tabular}
\begin{exe}
\ex \label{ex:tg:danser}  \vspace{-8pt}
\gll		\ipa{mju²}	\ipa{.wji¹}	\ipa{dzu¹} \\
			danse	faire[A]	aimer \\
\glt Ils aiment danser (Leilin, 04.32A.7).
\end{exe}

	B.	Initiales dentales \label{rimes:01:dent}
\newline

\begin{enumerate}


\item 1569 \tgf{1569} \ipa{twụ} 1.58, aussi écrit 5128 \tgf{5128} \ptang{S-to}{droit} correspond au  \jpg{astu} (<*ŋasto) ``droit (au sens propre, non courbé)". Un lien avec le LB *(C)-dwaŋ¹ en revanche semble impossible.

Dans les textes tangoutes, \tgf{1569} \ipa{twụ¹} apparaît surtout dans le sens dérivé de ``loyal, honnête, correct'':
\newline
\linebreak
\begin{tabular} {lllllll}
		\tgf{1013}&	\tgf{3119}&	\tgf{1139}&	\tgf{1045}&	\tgf{3583}&	\tgf{1569}&	\tgf{5285}  \\
		\tinynb{1013}&	\tinynb{3119}&	\tinynb{1139}&	\tinynb{1045}&	\tinynb{3583}&	\tinynb{1569}&	\tinynb{5285}  \\
\end{tabular}
\begin{exe}
\ex \label{ex:tg:droit}  \vspace{-8pt}
\gll	\ipa{tɕju¹}	\ipa{.ji¹}	\ipa{.jij¹}	\ipa{dạ²}	\ipa{tja¹}	\ipa{twụ¹}	\ipa{ljɨ¹} \\
		Zhu Yun	\antierg{}	parole	\topic{}	droit	\cop{} \\
\glt C'est par loyauté que Zhu Yun (\zh{朱雲}) a dit ces paroles. (Leilin, 03.07B.4)
\end{exe}
Le japhug a toutefois également un verbe statif dérivé \ipa{rɤstu } ``honnête''. En tangoute, un autre adjectif apparenté à \tgf{1569} \ipa{twụ¹}, 5127 \tgf{5127} \ipa{twụ} 2.51, est attesté dans le sens de ``direct'':
\newline
\linebreak
\begin{tabular} {llllll}
		\tgf{5127}&	\tgf{5127}&	\tgf{0959}&	\tgf{3554}&	\tgf{0795}&	\tgf{4469} \\
		\tinynb{5127}&	\tinynb{5127}&	\tinynb{0959}&	\tinynb{3554}&	\tinynb{0795}&	\tinynb{4469} \\
\end{tabular}
\begin{exe}
\ex \label{ex:tg:direct}  \vspace{-8pt}
\gll 	\ipa{twụ² twụ²}	\ipa{thej¹ljow¹}	\ipa{rjɨr¹-ɕji²} \\
		direct	Daliang	\dir{}-aller[A] \\
\glt Il alla directement à Daliang (Sunzi, 55A.3)
\end{exe}
Il s'agit toutefois clairement de sens dérivés de la signification originelle ``droit'' que le japhug a ici préservée. L'alternance tonale entre \tgf{1569} \ipa{twụ¹} et \tgf{5127} \ipa{twụ²} n'a pas d'explication dans l'état actuel de nos connaissances.


\item 5165 \tgf{5165} \ipa{twụ} 1.58 \ptang{S-to}{endroit} est comparable avec le  \jpg{ɯ-stu} de même sens. Il s'agit dans les deux langues d'un dérivé irrégulier de la racine ``y avoir, être là'' (``l'endroit où l'on est" > ``l'endroit''). En tangoute l'irrégularité porte sur le fait que le verbe originel \tgz{0930} ``y avoir'' a une initiale prénasalisée, tandis qu'en japhug c'est l'absence de voyelle du préfixe de nominalisation  (la forme attendue serait *sɤ-tu avec le préfixe \ipa{sɤ}--).

\item 0369 \tgf{0369}  \ipa{thjuu} 1.07 \ptang{thjoo}{inspecter} pourrait potentiellement être rapproché du  \jpg{thu} ``demander''.

\item 930 \tgf{0930} \ipa{dju} 1.03 \ptang{ndjo}{y avoir} ainsi que 2495 \tgf{2495} \ipa{du} 1.04 peut se comparer au  \jpg{tu} ``y avoir, être là'', étymon qui apparaît avec une prénasalisée dans d'autres dialectes rgyalrongs tels que le  \situ{ndó}. Un rapprochement avec le \tib{ɴdug} semble très improbable.  Voir \ref{sec:cas} pour plus de détail sur la construction possessive en tangoute.


\item 4785	\tgf{4785} \ipa{njụ} 2.52 \ptang{S-njo}{`bon} peut se comparer au  \jpg{mnu} ``lisse'' et au \plb{C-nu²}{0528} ``soft''. 

\item 984 \tgf{0984} \ipa{tshwu} 1.01 \ptang{tsho}{gras} peut être rapproché du  \jpg{tshu} et du \tib{tsʰo-po, tsʰon-po}. Un cognat se retrouve dans les langues lolo-birmanes \plb{tsu¹}{0532}, ce qui suggère fortement qu'il ne s'agit pas d'un emprunt tibétain comme on pourrait le penser au premier abord. En tangoute, cet adjectif s'applique aussi bien aux animaux qu'aux hommes:
\newline
\linebreak
\begin{tabular}{lllllll}
	\tgf{1208}&	\tgf{3583}&	\tgf{1176}&	\tgf{5285}&	\tgf{0683}&	\tgf{4305}&	\tgf{0984} \\
	\tinynb{1208}&	\tinynb{3583}&	\tinynb{1176}&	\tinynb{5285}&	\tinynb{0683}&	\tinynb{4305}&	\tinynb{0984} \\
\end{tabular}
\begin{exe}
\ex \label{ex:tg:gras}  \vspace{-8pt}
\gll 	\ipa{tɕia¹}	\ipa{tja¹}	\ipa{gaa²}	\ipa{ljɨ¹}	\ipa{xia¹tsũ¹}	\ipa{tshwu¹} \\
		Li	\topic{}	maigre	\cop{}	Xiaozong	gras \\
\glt Li (\zh{禮})\footnote{Le caractère \mo{1208} \ipa{tɕia¹} a une prononciation qui ne correspond pas à celle du nom de ce frère dans le texte chinois \zh{禮} liX. Il s'agit peut-être d'une confusion avec le caractère \zh{札} zha dont la forme ressemble au caractère simplifié  \zh{礼}, forme vulgaire déjà en usage à l'époque Song.} est maigre, tandis que moi, Xiaozong (\zh{孝宗}), je suis gras (Leilin 03.30B.3)
\end{exe}

\item 4615 \tgf{4615}  \ipa{tsụ} 2.51 \ptang{S-tso}{`tousser} ne se retrouve ni en rgyalrong ni en tibétain, mais s'apparente au lolo-birman \plb{tso²}{0576}. Un lien avec \tgz{5105} ``poumon" est vraisemblable.


\item  290 \tgf{0290} \ipa{sju} 2.03 \ptang{p-sjo}{comme, être semblable à}\footnote{Pour la reconstruction du *p- dans ce mot, voir p.\pageref{tab:prototgps} }  est comparable au   \jpg{fse} ``être semblable, être ainsi''. On trouve également des cognats dans les langues lolo-birmanes (voir \citealt[180-1]{matisoff03}). Le vocalisme –e du japhug est dû à la fusion avec un suffixe *--j. le vocalisme originel est préservé en situ  \ipa{phsó}. Le vocalisme *o est préservé dans le nom apparenté  \jpg{ɯ-fsu} ``égal, semblable''. En japhug, le sens de cette racine est avant tout ``être d'une certaine façon'', et ``être semblable'' en est un cas particulier. 

Le caractère de verbe statif et non d'adverbe en tangoute se manifeste par le fait qu'il peut s'adjoindre de suffixes d'accord  personnels:
\newline
\linebreak
\begin{tabular}{llllllll}
		\tgf{0510}&	\tgf{5306}&	\tgf{4005}&	\tgf{5646}&	\tgf{5306}&	\tgf{0290}&	\tgf{4601}&	\tgf{5285} \\
		\tinynb{0510}&	\tinynb{5306}&	\tinynb{4005}&	\tinynb{5646}&	\tinynb{5306}&	\tinynb{0290}&	\tinynb{4601}&	\tinynb{5285} \\
\end{tabular}
\begin{exe}
\ex \label{ex:tg:semblable}  \vspace{-8pt}
\gll 	\ipa{ŋwər¹dzjwɨ¹}	\ipa{kha²}	\ipa{tɕhjiw²}	\ipa{dzjwɨ¹}	\ipa{sju²-nja²}	\ipa{ljɨ¹} \\
		empereur	Jie	Zhou	seigneur	semblable-2\sg{}	\cop{} \\
\glt Mon empereur, vous êtes semblable aux rois Jie (\zh{桀}) et Zhou (\zh{紂}) (Leilin, 03.07A.1-2)
\end{exe}
\tgf{2999} \ipa{swu²} est un autre mot apparenté à \tgf{0290} \ipa{sju²}, qui s'emploie aussi comme verbe statif:
\newline
\linebreak
\begin{tabular} {lllll}
	\tgf{1801}&	\tgf{5814}&	\tgf{4950}&	\tgf{1918}&	\tgf{2999} \\
	\tinynb{1801}&	\tinynb{5814}&	\tinynb{4950}&	\tinynb{1918}&	\tinynb{2999} \\
\end{tabular}
\begin{exe}
\ex \label{ex:tg:semblable2}  \vspace{-8pt}
\gll  \ipa{dzji²}	\ipa{phu²}	\ipa{rjir²}	\ipa{mji¹-swu²} \\
		reste	arbre	\comit{}	\negat{}-semblable \\
\glt A la différence des autres arbres (Leilin, 03.30A.6)
\end{exe}
La différence entre \tgf{0290} \ipa{sju²} et \tgf{2999} \ipa{swu²} n'est pas claire, mais nous n'avons pas trouvé d'exemples du second pourvu de suffixes personnels.



\item 2048 \tgf{2048} \ipa {sjwụ} 2.5, aussi écrit 487 \tgf{0487} \ptang{S-sjo}{vivant} peut être rapproché du  \jpg{sɯsu} ``vivant". En japhug, ce verbe statif est rédupliqué ; le *S- reconstruit en pré-tangoute est peut-être la trace exceptionnelle de cette réduplication, et non pas un préfixe. Si cette hypothèse venait à se confirmer, on pourrait reconstruire une racine rédupliquée dans ce mot en proto-macro-rgyalronguique. 

La graphie \tgf{2048} \ipa{sjwụ²} s'emploie surtout avec le préfixe directionnel perfectivant \tgf{5981} \ipa{.a} dans le sens de ``revivre'':
\newline
\linebreak
\begin{tabular}{llll}
	 \tgf{0824} &\tgf{5417} &\tgf{5981} &\tgf{2048} \\
	 \tinynb{0824} &\tinynb{5417} &\tinynb{5981} &\tinynb{2048} \\
\end{tabular}
\begin{exe}
\ex \label{ex:tg:vivre}  \vspace{-8pt}
\gll \ipa{tɕhjɨ²rjar²} \ipa{.a-sjwụ²} \\
     immédiatement  \dir-vivre \\
\trans `Il ressuscita immédiament (suite de l'exemple (9), Leilin 06.11A.6)'
\end{exe}

Dans le texte chinois reconstruit par Sun et al. (1990:289) \tgf{2048} \ipa{sjwụ²} correspond au chinois \zh{甦} su, mais la similitude phonétique entre les deux est fortuite et il n'y a pas lieu de croire qu'il s'agit d'un emprunt du chinois, même si cette similarité phonétique a justement pu motiver les tangoutes à créer deux caractères pour le même verbe. La graphie \tgf{0487} \ipa{sjwụ²} correspond quant à elle dans le sens de ``vivant" (chinois \zh{活} \textit{hwat}), et ne s'emploie jamais avec un préfixe directionnel. Ces deux caractères notent le même mot; la motivation pour créer deux caractères distincts est probablement l


\item 4557 \tgf{4557}  \ipa{zjur} 2.70 \ptang{srjo}{torche} peut se comparer potentiellement au \jpg{tɤtʂu} < *-tro. La correspondance de l'initiale n'est pas complètement régulière, et il faut supposer des préfixes distincts dans les deux langues à partir d'une racine *ro.

\item 3065 \tgf{3065} \ipa{lhju} 1.03 \ptang{lhjo}{lait} est apparenté au  \jpg{tɤ-lu} ``lait'' et au \tib{ʑo} ``yaourt''.\footnote{Concenant la forme tibétaine, voir la discussion p.\pageref{ex:tib:traire} }  
\newline
\linebreak
\begin{tabular}{llllllllllllll}
\tgf{4778}&	\tgf{3305}&	\tgf{2226}&	\tgf{3916}&	\tgf{0960}&	\tgf{1139}&	\tgf{3065}&	\tgf{5880}&	\tgf{1074}&	\tgf{2541} \\	
\tinynb{4778}&	\tinynb{3305}&	\tinynb{2226}&	\tinynb{3916}&	\tinynb{0960}&	\tinynb{1139}&	\tinynb{3065}&	\tinynb{5880}&	\tinynb{1074}&	\tinynb{2541} \\	
\tgf{4582}&	\tgf{1542}&	\tgf{1326}&	\tgf{2833}&	\tgf{5834} & &  &   &\\
\tinynb{4582}&	\tinynb{1542}&	\tinynb{1326}&	\tinynb{2833}&	\tinynb{5834} & &  &   &\\
\end{tabular}
\begin{exe}
\ex \label{ex:tg:lait}  \vspace{-8pt}
\gll \ipa{ɕjạ¹}	\ipa{kjiw¹}	\ipa{.we²-sji²}	\ipa{mjịj¹}	\ipa{.jij¹}	\ipa{lhju¹}	\ipa{ŋwu²}	\ipa{lụ¹}	\ipa{dzjwo²}	\ipa{tjị¹}	\ipa{ku¹}	\ipa{kjɨ¹djɨj¹}	\ipa{lej²} \\
	 sept	an	devenir-\nmls{}	fille	\antierg{}	lait	\conj{}	pierre	homme nourrir[A]	alors	certainement	se.transformer	\\
\glt Si on le nourrit avec le lait d'une fille de sept ans, l'homme de pierre se transformera certainement (Leilin, 06.28B.4-5)
\end{exe}
Voir aussi la discussion sur le verbe ``téter'' p.\pageref{ex:tg:teter}.



\item  4726 \tgf{4726} \ipa{lhjụ} 2.52 \ptang{S-lhjo}{bambou} peut être rapproché du  \jpg{ɟu} (< *tljo). Cet étymon apparaît également dans le nom du bananier \tgf{3369}\tgf{4726} \ipa{mja¹ lhjụ²} (3369) attesté dans l'inscription de la pagode de Ganying. Pour un exemple textuel, voir p.\pageref{ex:tg:mettre.dans}. Le nom ``flûte de bambou'' 4194 \tgf{4194} \ipa{lju} 1.02 y est apparenté (cf. japhug \ipa{ɟuli} ``flûte'').


\item 2801 \tgf{2801} \ipa{lhuu} 1.05 \ptang{lhoo}{moelle} est apparenté au  \jpg{tɯ-pju} (< *pə-ljo).
\newline
\linebreak
\begin{tabular}{llllllll}
		\tgf{5479}&	\tgf{2857}&	\tgf{2778}&	\tgf{2801}&	\tgf{5993}&	\tgf{1326}&	\tgf{1616}&	\tgf{3916} \\
		\tinynb{5479}&	\tinynb{2857}&	\tinynb{2778}&	\tinynb{2801}&	\tinynb{5993}&	\tinynb{1326}&	\tinynb{1616}&	\tinynb{3916} \\
\end{tabular}
\begin{exe}
\ex \label{ex:tg:moelle}  \vspace{-8pt}
\gll  \ipa{rjar¹ŋo²}	\ipa{rjɨr¹}	\ipa{lhjuu²}	\ipa{kha¹}	\ipa{kjɨ¹-.o²-sji²} \\
		maladie	os	moelle	intérieur 	\dir{}-entrer-\perf{} \\
\glt  La maladie est déjà entrée dans ses os et sa moelle. (Leilin 06.08A.7).
\end{exe}



\item 475 \tgf{0475} \ipa{lju} 1.03 ``matelas, couche" ainsi que 922 \tgf{0922} \ipa{ljuu} 2.06 peuvent être comparés au  \jpg{tɤ-βɟu} (< *ptljo).
\newline
\linebreak
\begin{tabular}{lllllll}
	\tgf{3058}&	\tgf{1204}&	\tgf{0089}&	\tgf{0475}&	\tgf{5449}&	\tgf{2590}&	\tgf{2396} \\
	\tinynb{3058}&	\tinynb{1204}&	\tinynb{0089}&	\tinynb{0475}&	\tinynb{5449}&	\tinynb{2590}&	\tinynb{2396} \\
\end{tabular}
\begin{exe}
\ex \label{ex:tg:matelas}  \vspace{-8pt}
\gll   \ipa{zjɨɨr²}	\ipa{njijr²}	\ipa{tɕhjaa¹}	\ipa{lju¹}	\ipa{tjị¹}	\ipa{.wjɨ²-dzuu²} \\
		eau	face	sur	matelas	mettre[A]	\dir{}-s'asseoir \\
\glt  Il mit un matelas sur l'eau et s'assit dessus (Leilin 05.23B.1)
\end{exe}
\end{enumerate}

C. Initiales palatales \label{rimes:01:1:pal}

\begin{enumerate}


\item 3003 \tgf{3003} \ipa{.ju} 1.02 \ptang{jo}{fantôme} correspond à l'étymon du tshobdun emprunté en  \jpg{tɯ-ɟo} dans certaines de nos histoires.\footnote{Le japhug emploie normalement l'emprunt tibétain \ipa{ɬandʐi} < lha.ɴdre.}  Il faut reconstruire ici un proto-rgyalrong *jo identique à la forme du pré-tangoute.
\newline
\linebreak
\begin{tabular}{lllllllllll}
		\tgf{4978}&	\tgf{3003}&	\tgf{2546}&	\tgf{1943}&	\tgf{1542}&	\tgf{1326}&	\tgf{2833}&	\tgf{1666}&	\tgf{1870}&	\tgf{0508}\\
		\tinynb{4978}&	\tinynb{3003}&	\tinynb{2546}&	\tinynb{1943}&	\tinynb{1542}&	\tinynb{1326}&	\tinynb{2833}&	\tinynb{1666}&	\tinynb{1870}&	\tinynb{0508} \\
		\tgf{4601} \\
		\tinynb{4601} \\
\end{tabular}
\begin{exe}
\ex \label{ex:tg:fantome}  \vspace{-8pt}
\gll 
		\ipa{tjij¹}	\ipa{.ju¹}	\ipa{njạ¹}	\ipa{nja²}	\ipa{ku¹}	\ipa{kjɨ¹djɨj²}	\ipa{nədʑiə¹}	\ipa{ŋwu²-nja¹} \\
		si	démon	dieu	ne.pas.être	alors	certainement renard	être-2\sg{} \\
\glt  Si tu n'es pas un démon ou un dieu, alors tu es certainement un renard. (Leilin 06.30B.1)
\end{exe}
Le verbe  \tgz{3446}  qui semble vouloir dire ``faire tomber malade'' est apparenté à ce nom, même si la direction de la dérivation n'est pas claire. Le nom \tgz{2314} ``mort'' y est vraisemblablement relié.



\item 4425 \tgf{4425} \ipa{ɕju} 1.02 \ptang{ɕo}{frais} et la forme apparentée 151 \tgf{0151} \ipa{ɕjuu} 1.07 peuvent se comparer au  \jpg{ɣɤɕu} ``frais''.
\newline
\linebreak
\begin{tabular}{llllll}
		\tgf{4871}&	\tgf{1067}&	\tgf{3068}&	\tgf{4425}&	\tgf{0702}&	\tgf{5346} \\
		\tinynb{4871}&	\tinynb{1067}&	\tinynb{3068}&	\tinynb{4425}&	\tinynb{0702}&	\tinynb{5346} \\
\end{tabular}
\begin{exe}
\ex \label{ex:tg:frais}  \vspace{-8pt}
\gll  \ipa{ŋər¹}	\ipa{rar²}	\ipa{njɨ̣j²}	\ipa{ɕju¹}	\ipa{tjoo¹.ju²} \\
		montagne	ombre	sombre	frais	chercher[A] \\
\glt  Il cherchait des endroits sombres et frais dans la montagne (Leilin 03.28B.7)
\end{exe}



\item 2795  \tgf{2795}  \ipa{rur} 1.75 \ptang{ro}{pâturage}  peut se comparer à la première syllable du \jpg{rɯŋgu} de même sens. Le nom \jpg{rɯ} ``lieu d'habitation temporaire dans la montagne" pourrait être aussi apparenté, s'il ne s'agit pas d'un emprunt au \tib{ri} ``montagne".


\item 5736  \tgf{5736}  \ipa{rjur} 2.70 \ptang{rjo}{ferment} est dérivé d'une racine verbale *rjo ``se lever'' cognate du \jpg{rɤru} ``se lever''. Ce verbe peut s'employer en japhug à propos du vin, ce qui justifie cette étymologie:


\begin{exe}
\ex \label{ex:jpg:selever} 
\gll  \ipa{cha} \ipa{tɤ-rɤru} \\
vin \aor{}-se.lever\\
\glt  Le vin a fermenté.
\end{exe}
 
\end{enumerate}

D.	Initiales vélaires \label{rimes:01:1:vel}

\begin{enumerate}


\item 5890 \tgf{5890} \ipa{ku} 1.04 \ptang{ko}{relâché} correspond au  \jpg{ɴɢu} ``relâché''. En japhug, ce verbe peut s'employer à propos d'une ceinture, mais aussi à propos de la terre:
\begin{exe}
\ex \label{ex:jpg:relache} 
\gll \ipa{li}	\ipa{lú-wɣ-ɕlu}	\ipa{tɕɤn}	\ipa{nɤki,}	\ipa{thɤlwa}	\ipa{kɯ-ɴɢɯ-ɴɢu}	\ipa{ʑo}	\ipa{ɟɯɣɟɯɣ}	\ipa{tú-wɣ-βzu}	\ipa{tɕe} \\
		encore	\impf{}-\inv{}-labourer	\conj{}	\conj{}	terre	\nmls{}-\redp{}-relâché	\intens{} \ideo{}:2:meuble	\impf{}-\inv{}-faire	\conj{} \\
\glt  On le laboure encore, et on rend la terre meuble et aérée. (rtsampa, 5)
\end{exe}

En tangoute, nous n'avons pas encore trouvé d'exemples textuels de \tgf{5890} \ipa{ku¹} en dehors de ceux des dictionnaires, mais le verbe causatif dérivé 2668 \tgf{2668} \ipa{kụ} 1.58 est quant à lui bien attesté. Il peut s'employer dans le sens de ``desserrer une ceinture'',  mais il apparaît aussi dans des sens figurés, comme dans l'exemple suivant où il s'oppose à \tgf{2320} \ipa{ɣar¹} ``presser, forcer''.
\newline
\linebreak
\begin{tabular}{llllllllll}
		\tgf{4342}&	\tgf{3811}&	\tgf{0089}&	\tgf{2171}&	\tgf{2090}&	\tgf{1943}&	\tgf{2668}&	\tgf{4884}&	\tgf{1542}&	\tgf{4478}\\
				\tinynb{4342}&	\tinynb{3811}&	\tinynb{0089}&	\tinynb{2171}&	\tinynb{2090}&	\tinynb{1943}&	\tinynb{2668}&	\tinynb{4884}&	\tinynb{1542}&	\tinynb{4478}\\
		\tgf{1906}&	\tgf{2590}&	\tgf{1918}&	\tgf{0134}&	\tgf{4342}&	\tgf{2320}&	\tgf{4884}&	\tgf{1542}&	\tgf{1906}&	\tgf{2912}\\
		\tinynb{1906}&	\tinynb{2590}&	\tinynb{1918}&	\tinynb{0134}&	\tinynb{4342}&	\tinynb{2320}&	\tinynb{4884}&	\tinynb{1542}&	\tinynb{1906}&	\tinynb{2912}\\
		\tgf{3614}&	\tgf{4370}&	\tgf{5918}&	\tgf{4916}&	\tgf{5113}& &&&& \\
		\tinynb{3614}&	\tinynb{4370}&	\tinynb{5918}&	\tinynb{4916}&	\tinynb{5113}& &&&& \\
\end{tabular}
\begin{exe}
\ex \label{ex:tg:relache}  \vspace{-8pt}
\gll  \ipa{dja²-.war¹}	\ipa{tɕhjaa¹}	\ipa{tha}	\ipa{lew²}	\ipa{nja²}	\ipa{kụ¹-nji²}	\ipa{ku¹}	\ipa{ta¹}	\ipa{nioow¹}	\ipa{.wjɨ²-mji¹-.ju¹}	\ipa{dja²-ɣar¹-nji²}	\ipa{ku¹}	\ipa{nioow¹}	\ipa{lhjwo¹}	\ipa{kạ¹}	\ipa{ljuu¹}	\ipa{sjɨ¹}	\ipa{ɣwej¹}	\ipa{.wji¹} \\
\dir{}-désespéré	sur	acculer	nom	ne.pas.être	relâcher-2\pl{} alors	fuir	après	\dir{}-\negat{}-regarder \dir{}-presser-2\pl{}	alors	après	retourner vie	parier[A]	mourir[B]	combat	faire[A] \\
\glt  On n'accule pas celui qui est désespéré. Si vous relâchez la pression sur lui, il fuira sans regarder en arrière. Si vous augmentez la pression, il se retournera et combattra à mort. (Sunzi, 14A.1-2, voir \citealt[n.164,516]{lin94sunzi})
\end{exe}
La forme \tgf{2668} \ipa{kụ¹} est dérivée de \tgf{5890} \ipa{ku¹}  par le préfixe causatif *S-. La différence de voisement entre le japhug et le tangoute n'est pas explicable pour le moment.


\item 0328  \tgf{0328}   \ipa{ku}  2.04 \ptang{ko}{aveugle} est apparenté à la racine que l'on retrouve dans le \jpg{ɕqwa} ``aveugle'' et le \jpg{ɕqudoŋ} ``aveugle (insulte)'' (voir aussi peut-être le pumi \ipa{ɲɜkuə̌} ``aveugle''). Etant donnée la forme japhug, on attendrait en tangoute une syllabe du premier cycle mineur.

\item 2278 \tgf{2278} \ipa{kjụ} 1.59 ``oignon" (*PT S-kjo) se rapproche du   \jpg{ɕku} ``oignon" et du \tib{sgog.pa} ``ail''. La correspondance entre japhug et tibétain est irrégulière, et il doit s'agir d'un Wanderwort. Toutefois, il est possible que cet étymon soit reconstructible au moins au niveau du proto-macro-rgyalronguique, car des formes similaires se retrouvent dans les autres langues de cette région, par exemple en pumi.



\item 5396 \tgf{5396} \ipa{kjur} 1.76 \ptang{rkjo}{mettre, placer dans} est comparable au  \jpg{rku} ``mettre dans'', verbe très courant:
\begin{exe}
\ex \label{ex:jpg:mettre.dans} 
\gll  	\ipa{tɕendɤre }	\ipa{mar-rgɤm }	\ipa{ɯ-ŋgɯ }	\ipa{zɯ }	\ipa{χwɤr }	\ipa{ra }	\ipa{ɣɯ }	\ipa{nɯ-rɟaβlun}	\ipa{thamtɕɤt }	\ipa{ɣɯ }	\ipa{nɯ-ku }	\ipa{nɯ }	\ipa{pjɤ-rku } \\
		\conj{} 	beurre-boîte 	3\sgposs{}-intérieur 	\loc{} 	Hor 	\pl{} 	\gen{} 	3\plposs{}-ministre tout 	\gen{} 	3\plposs{}-tête 	cela 	\med{}:bas-mettre.dans  \\
\glt Il mit toutes les têtes des ministres de Hor (qu'il venait de massacrer) dans une boîte à beurre. (Gesar, 372).
\end{exe}
La forme passive \jpg{arku} ``être placé  dans'' est également d'un usage fréquent. Il est probable que le \plb{ʔ-kun³/²}{0607} ``put in'' soit apparenté. Il s'agirait donc d'un des exemples d'étymons en *--n final, consonne qui a disparu sans laisser de traces claires en rgyalrong et en tangoute.

En tangoute, le sens de ce verbe est identique à celui du japhug:
\newline
\linebreak
\begin{tabular}{lllll}
		\tgf{4726}&	\tgf{4177}&	\tgf{2983}&	\tgf{5868}&	\tgf{5396} \\
		\tinynb{4726}&	\tinynb{4177}&	\tinynb{2983}&	\tinynb{5868}&	\tinynb{5396} \\
\end{tabular}
\begin{exe}
\ex \label{ex:tg:mettre.dans}  \vspace{-8pt}
\gll 	\ipa{lhjụ²}	\ipa{dụ¹}	\ipa{.u²}	\ipa{khie²}	\ipa{kjur¹} \\
		bambou	tube	intérieur	riz	mettre.dans[A] \\
\glt Ils mettaient du riz dans un tube de bambou (Leilin 03.23A.2)
\end{exe}

Ce verbe a un thème B inattendu: 3675 \tgf{3675} \ipa{kjọ} 1.72. On aurait attendu plutôt la rime 1.90 du second cycle mineur *kjor. Toutefois, on doit tenir compte du fait que la syllabe *kjor n'existe pas en tangoute parmi les mots connus; il s'agit donc peut-être d'un cas spécial d'évolution de la trace laissée par la préinitiale *r-. Ce thème B n'est pas attesté dans les textes que nous connaissons.


\item  2503 \tgf{2503} \ipa{kụ} 1.58  \ptang{S-ko}{après}  peut être rapproché du  \jpg{ɯ-qhu} ``après, arrière". Il s'agit fondamentalement d'un nom dans les deux langues, mais il s'emploie comme une conjonction. En tangoute, il sert aussi à qualifier d'autres noms, par exemple pour traduire l'adjectif ``postérieur'' dans \zh{後漢} Hòu Hàn ``Han postérieurs''. 


\item 4040 \tgf{4040} \ipa{khjuu} 1.06 \ptang{khjoo}{inviter} est comparable au  \jpg{qru} ``inviter'' et au \bir{krui²}. Un trait intéressant commun à ces verbes en japhug et en tangoute est qu'il peuvent s'employer dans le sens d'``épouser" (avec un agent masculin). 
\begin{exe}
\ex \label{ex:jpg:inviter} 
\gll 	\ipa{lɤβzaŋ}	\ipa{ɣɯ}	\ipa{ɯ-mu}	\ipa{kɯ-maqhu}	\ipa{kɤ-kɤ-qru} \\
		Blobzang 	\gen{} 3\sgposs{}-mère	\nmls{}:\textsc{stat}-après	\aor{}-\nmls{}:O-inviter \\
\glt La (belle-)mère de Blobzang, celle que (son père) avait épousée après, (Blobzang 16)
\end{exe}

Voici un exemple équivalent en tangoute:
\newline
\linebreak
\begin{tabular}{llllllll}
		\tgf{3554}&	\tgf{3830}&	\tgf{5026}&	\tgf{1906}&	\tgf{2541}&	\tgf{5791}&	\tgf{4040}&	\tgf{4469} \\
		\tinynb{3554}&	\tinynb{3830}&	\tinynb{5026}&	\tinynb{1906}&	\tinynb{2541}&	\tinynb{5791}&	\tinynb{4040}&	\tinynb{4469} \\
\end{tabular}
\begin{exe}
\ex \label{ex:tg:inviter}  \vspace{-8pt}
\gll 	\ipa{ljow¹}	\ipa{njij²}	\ipa{mji¹}	\ipa{nioow¹}	\ipa{dzjwo²}	\ipa{wjạ²}	\ipa{khjuu¹}	\ipa{ɕji²} \\
		Liang	roi	entendre[A]	après	homme	envoyer	inviter	aller[A] \\
\glt Le roi de Liang, ayant entendu cette histoire, envoya des gens pour lui proposer de l'épouser. (Cixiaozhuan, 30.7-8, Jacques 2007: 93-4).
\end{exe}
Outre ce sens, ils peuvent s'employer dans leur sens premier d'inviter ou d'accueillir un hôte. \tgf{4040} \ipa{khjuu¹} est apparenté également à \tgf{2791} \ipa{khju²} ``appeler, inviter'' et son dérivé nominal \tgf{3254} \ipa{khju²} ``édit impérial''.


\item 3600 \tgf{3600} \ipa{ɣju} 1.03 \ptang{C-kjo}{appeler} est apparenté au  \jpg{akhu}. En japhug, ce verbe est intransitif, et signifie plutôt ``crier'', mais la racine en proto-rgyalrong était probablement transitive, car le préfixe a- est la marque du passif.  Cette forme se retrouve dans le \plb{ku/aw¹}{0665} (\bir{khaw²})(\citealt[225]{matisoff03}).\footnote{Matisoff cite aussi le jingpo \ipa{gāu}, mais ce mot signifie \zh{告狀} ``intenter un procès'' selon \citet{xu83jingpo}.}  Le caractère transitif de ce verbe est prouvé par l'exemple suivant, où l'agent est marqué à l'ergatif:
\newline
\linebreak
\begin{tabular}{llllll}
		\tgf{3118}&	\tgf{2219}&	\tgf{5604}&	\tgf{5113}&	\tgf{0804}&	\tgf{3600} \\
		\tinynb{3118}&	\tinynb{2219}&	\tinynb{5604}&	\tinynb{5113}&	\tinynb{0804}&	\tinynb{3600} \\
\end{tabular}
\begin{exe}
\ex \label{ex:tg:appeler2}  \vspace{-8pt}
\gll    \ipa{xu¹kjij¹}	\ipa{dʑjɨ.wji¹}	\ipa{djɨ²-ɣju¹} \\
		Fu.Jian	\erg{}   	\dir{}-appeler \\
\glt Fujian (\zh{苻堅}) l'invita. (Lielin, 03.35B.5)
\end{exe}

\item 3053 \tgf{3053}  \ipa{gju} 1.03 \ptang{ŋgjo}{traverser une rivière} est apparenté au \jpg{ʑŋgu} ``traverser une rivière en bateau'', même si on attendrait plutôt une rime du premier cycle mineur. Une comparaison avec le \tib{gru} ``bateau'' est plus difficile: il faudrait supposer l'existence d'un infixe dans cette langue.

\item 1907 \tgf{1907} \ipa{gju} 2.03 \ptang{ŋgjo}{tendon} peut être comparé  au  \jpg{tɯ-ŋgru} ``tendon". Un rapport avec le \tib{rgʲus.pa} est envisageable, mais difficile à soutenir avec certitude.


\item 5102 \tgf{5102} \ipa{gjur} 1.76 \ptang{r-ŋgo}{rein} est apparenté au  \situ{tə-rgó} ``testicule''.


\item 2750 \tgf{2750} \ipa{ɣu} 1.04 \ptang{C-ko}{tête} est cognat avec le  \jpg{tɯ-ku} ``tête" et le \tib{mgo}. Cette racine est très répandue en ST; le \plb{u²}{0088} pourrait y être apparenté, ce qui supposerait une lénition de l'occlusive en pré-proto-LB (mais une étymologie avec le \tib{dbu} proposée par Benedict est aussi vraisemblable). En japhug comme en tangoute, ce mot peut servir dans le sens secondaire de ``partie supérieure'', et apparaît dans un nombre important de composés. \footnote{Par exemple le  \jpg{sɯku} ``cime d'un arbre'', dont le premier élément est l'état construit de \jpg{si} ``arbre''.} L'exemple suivant illustre un emploi métaphorique de \tgf{2750} \ipa{ɣu¹}:
\newline
\linebreak
\begin{tabular}{lllllllll}
		\tgf{0243}&	\tgf{2541}&	\tgf{2590}&	\tgf{5671}&	\tgf{0289}&	\tgf{2750}&	\tgf{0089}&	\tgf{3678}&	\tgf{0749} \\
		\tinynb{0243}&	\tinynb{2541}&	\tinynb{2590}&	\tinynb{5671}&	\tinynb{0289}&	\tinynb{2750}&	\tinynb{0089}&	\tinynb{3678}&	\tinynb{0749} \\
\end{tabular}
\begin{exe}
\ex \label{ex:tg:tete}  \vspace{-8pt}
\gll  	\ipa{sji²dzjwo¹}	\ipa{.wjɨ²-thjɨ¹}	\ipa{.we²}	\ipa{ɣu¹}	\ipa{tɕhjaa¹} \ipa{to²}	\ipa{phji¹} \\
		femme	\dir{}-renvoyer	muraille	tête	sur apparaître	causer[A] \\
\glt Il envoya des femmes pour qu'elles soient visibles sur le haut des murailles de la ville. (Leilin 04.08B.1)
\end{exe} 

Le mot \tgf{2750} \ipa{ɣu¹} a un quasi-synonyme : 124 \tgf{0124} \ipa{ljụ} 2.52 ``tête". Ce mot est d'un usage moins courant, et n'apparaît que pour désigner la partie du corps, jamais pour désigner la partie supérieure d'un objet. Les noms 5416 \tgf{5416} \ipa{ɣwə} 2.25 ``devant'' et 2103  \tgf{2103} \ipa{ɣu} 1.04 ``devant'' lui sont aussi apparentés. On peut en rapprocher également la seconde syllabe de \tgf{3150}\tgf{2813} \ipa{thjɨ²ɣu²} ``empereur'' et \tgf{3435} \ipa{ɣu¹} ``un type de divinité''.


\item 508 \tgf{0508} \ipa{ŋwu} 2.01 \ptang{ŋo}{être} est comparable au  \jpg{ŋu} et au \plb{ŋwa¹}{0698a}	``be the case''. On retrouve un étymon similaire dans les autres langues macro-rgyalronguiques (sauf le pumi), ainsi que le \nayn{ŋi˩}. Si la correspondance entre japhug et tangoute est parfaite, il n'en va pas de même avec le lolo-birman et le na, dont les formes sont irréconciliables. Le \nayn{ŋi˩} représente une forme à ablaut qui rappelle le verbe irrégulier \jpg{maŋe} ``ne pas y avoir'', qui semble être formé d'un préfixe de négation suivi du thème ablauté du verbe ``être''. Le \plb{ŋwa¹}{0698a} est plus difficile à expliquer. Les formes japhug et tangoutes feraient attendre une racine *ŋo¹ en pré-lolo-birman; *ŋwa¹ pourrait résulter de la fusion de ce *ŋo¹ avec un suffixe de TAM, ou être la trace d'un ablaut.


Il est notable qu'en japhug ce verbe a spécifiquement le sens ``être correct, être adéquat'' dans certains contextes, même s'il est par ailleurs la copule neutre. Ainsi on trouve le verbe dénominal \jpg{rɯkɯŋu} ``être attentionné envers sa famille, utiliser tout son argent pour la maison", qui dérive du nom déverbal \jpg{kɯ-ŋu} ``correct, vrai".
 
En tangoute, cette copule s'emploie avec les suffixes d'accord personnel et peut s'adjoindre de la négation \tgf{1918} \ipa{mji¹}, mais aucun exemple avec des préfixes directionnels n'a été découvert pour le moment.
\newline
\linebreak
\begin{tabular}{lllllllll}
		\tgf{4797}&	\tgf{3543}&	\tgf{1918}&	\tgf{3320}&	\tgf{1542}&	\tgf{2541}&	\tgf{1918}&	\tgf{0508}&	\tgf{5285} \\
		\tinynb{4797}&	\tinynb{3543}&	\tinynb{1918}&	\tinynb{3320}&	\tinynb{1542}&	\tinynb{2541}&	\tinynb{1918}&	\tinynb{0508}&	\tinynb{5285} \\
\end{tabular}
\begin{exe}
\ex \label{ex:tg:etre}  \vspace{-8pt}
\gll  \ipa{.jwɨr²dzwə¹}	\ipa{mji¹-ɣiew¹}	\ipa{ku¹}	\ipa{dzjwo²}	\ipa{mji¹-ŋwu²}	\ipa{ljɨ¹} \\
		texte	\negat{}-étudier	alors	homme	\negat{}-être	\cop{} \\
\glt Celui qui n'étudie pas les textes n'est pas un homme. (Leilin 05.19B.4)
\end{exe} 
Le pré-tangoute *ŋo est probablement lié à la racine nominale \racine{ngo} que l'on retrouve dans une famille de mot en tibétain:
\begin{enumerate}
\item \tib{ŋo.bo} `essence, entité' (sanskrit \textit{bhāva}-, \textit{rūpa}-)
\item \tib{ŋo.ma} ‘vrai’
\item \tib{dŋos} ‘partie principale, vrai’
\item forme au cas terminatif \textit{dŋos-su}  `explicitement, évident’ (sanskrit \textit{sākṣāt})
\item \textit{dngos.po} ‘bien, possession’
\end{enumerate}
Pour expliquer ces formes, on peut conjecturer l'existence d'un verbe `être vrai' *ŋo en proto-tibétain, dont la forme nominalisée ancienne attendue est *k-ŋo-s > \textit{dŋos}, tandis que \textit{ŋo.bo} et \textit{ŋo.ma} représentent des nominalisations récentes.

Pour la dérivation de `être vrai' à `bon' et `bien, possession', on peut tout simplement citer la racine indo-européenne *h₁es-, d'où l'on tire un nom neutre *h₁ós-u `bien, bon' (attesté par le hittite \textit{aššu}) et un adjectif *h₁és-u-s `bon' (grec \grec{ἐΰς} `bon').

L'usage de cette racine comme copule exclusive est potentiellement une des innovations communes aux langues birmo-qianguiques.

\item 4600  \tgf{4600} \ipa{ŋwụ} 1.58 \ptang{S-ŋo}{promesse} peut être rapproché du  \jpg{kɯjŋu} ``promesse". En japhug, pour dire ``promettre'', on doit utiliser le verbe ``lever'' \ipa{joʁ} ou le verbe ``poser'' \ipa{ta}.
\begin{exe}
\ex \label{ex:jpg:promesse} 
\gll 
		\ipa{tɕheme}	\ipa{nɯ}	\ipa{kɯjŋu}	\ipa{kɯ-wxtɯ-wxti}	\ipa{ʑo}	\ipa{pa-sɯ-tá-ndʑi}	\ipa{nɤ,} \\
		fille	\dem{}	promesse	\nmls{}:\stat{}-\redp{}-grand	\intens{} \aor{}:3>3-\caus{}-poser-\du{}	\conj{} \\
\glt Ils forcèrent tous les deux la fille à faire une promesse solennelle. (le renard,142)
\end{exe} 
En tangoute, on emploie le verbe ``faire" \tgz{5113}  pour exprimer ce sens :
\newline
\linebreak
\begin{tabular}{lllllll}
		\tgf{4600}&	\tgf{0795}&	\tgf{5113}&	\tgf{1918}&	\tgf{2904}&	\tgf{4481}&	\tgf{2098} \\
				\tinynb{4600}&	\tinynb{0795}&	\tinynb{5113}&	\tinynb{1918}&	\tinynb{2904}&	\tinynb{4481}&	\tinynb{2098} \\
\end{tabular}
\begin{exe}
\ex \label{ex:tg:promesse}  \vspace{-8pt}
\gll  \ipa{ŋwụ¹}	\ipa{rjɨr²-.wji¹}	\ipa{mji¹-.jar²-ɕjɨ¹-ŋa²}  \\
		promesse	\dir{}-faire[A]	\negat{}-se.marier-aller-1\sg{} \\
\glt Elle jura ``Je ne me (re)marierai pas''. (Leilin 06.06A.4)
\end{exe} 

Ce mot est peut-être apparenté à la forme 4847 \tgf{4847} \ipa{ŋwe} 2.07 qui semble être un verbe signifiant ``jurer'', mais dont aucun exemple textuel n'a encore été mis au jour. Il faudrait toutefois postuler une alternance vocalique\ipa{ --ụ/--e} inhabituelle : \citet{gong88alternations} n'en mentionne aucune de ce type.



\item 1302 \tgf{1302} \ipa{ŋur} 1.75  aussi écrit 1228 \tgf{1228}  \ptang{rŋo}{crêpe, gâteau} se compare avec le  \jpg{rŋu} ``griller (la tsampa)'' et le \tib{rŋo} ``griller, frire''. Il est difficile d'être absolument certain que le verbe japhug n'est pas un emprunt au tibétain, mais l'hypothèse qu'il est cognat est plus vraisemblable. En tangoute, ce mot ne semble pas pouvoir être employé comme verbe. Il est attesté dans le ZZZ dans l'expression \tgf{1302} \tgf{4619} \ipa{ŋur¹ ŋiạ²} (4619) traduite en chinois par \zh{胡餅} \textit{húbǐng}, un type de crêpe ou de gâteau. La définition du dictionnaire \textit{Wénhǎi} montre bien qu'il s'agit de grains grillés :
\newline
\linebreak
\begin{tabular}{llllllllll}
		\tgf{1302}&	\tgf{3583}&	\tgf{0106}&	\tgf{5905}&	\tgf{2104}&	\tgf{1326}&	\tgf{4413}&	\tgf{1302}&	\tgf{4342}&	\tgf{0632} \\
		\tinynb{1302}&	\tinynb{3583}&	\tinynb{0106}&	\tinynb{5905}&	\tinynb{2104}&	\tinynb{1326}&	\tinynb{4413}&	\tinynb{1302}&	\tinynb{4342}&	\tinynb{0632} \\
		\tgf{1139}&	\tgf{2639}&	\tgf{0508}&	\tgf{5285}& & & & & & \\
		\tinynb{1139}&	\tinynb{2639}&	\tinynb{0508}&	\tinynb{5285}& & & & & & \\
\end{tabular}
\begin{exe}
\ex \label{ex:tg:frire}  \vspace{-8pt}
\gll  \ipa{ŋur¹}	\ipa{tja¹}	\ipa{ɕioow¹}	\ipa{sji²}	\ipa{ɕji¹}	\ipa{kjɨ¹-pju¹}	\ipa{ŋur¹}	\ipa{dja²-.we¹}	\ipa{.jij¹}	\ipa{mjiij²}	\ipa{ŋwu²}	\ipa{ljɨ¹} \\
		crêpe	\topic{}	grain	grain	avant	\dir{}-cuire crêpe	\dir{}-cuit	\antierg{}	nom	être	\cop{} \\
\glt Ce qu'on appelle crêpe (peut se décrire de la façon suivante).Tout d'abord, on fait griller les grains, puis la crêpe est cuite. (Wenhai).
\end{exe} 
La nature exacte du gâteau ou de la crêpe ainsi désigné(e) est malheureusement difficile à déterminer avec plus de précision sur la base des textes tangoutes. Le tangoute n'a ici préservé que le dérivé nominal du verbe ``griller'', qui a disparu en tant que tel. Le japhug et le tibétain, mais aussi d'autres langues encore plus éloignées comme le chinois (\zh{熬} \ipapl{ŋaw < *ŋŋaw}) ou le jingpo (\ipa{gəngāu}), préservent le verbe d'où ce nom est tiré.

\end{enumerate}



\subsubsection{Tangoute --u :: tibétain --u :: japhug \ipapl{--ɯ}}	\label{subsubsec:correspondance:u:u:w}
Cette correspondance est également bien attestée. Le japhug \ipapl{-ɯ} provient du proto-japhug *--u (situ –u). Cette rime comporte certaines formes à nasales finales, comme l'atteste la comparaison avec le tibétain, le pumi et le LB (elle correspond à *-oŋ). Toutefois, nous ne distinguerons pas  les rimes en *--o de celles venant de *--oŋ en pré-tangoute.
\newline
A. Initiales labiales \label{rimes:01:2:lab}
\newline

\begin{enumerate}

\item 5814  \tgf{5814} \ipa{phu} 2.01 \ptang{phoN > pho}{arbre (classificateur)} correspond au  \jpg{tɯ-phɯ} ``arbre (classificateur)". En japhug, c'est un classificateur pour les arbres (\ipa{tɯ-} est le préfixe du numéral ; les formes \ipa{ʁnɯ-phɯ} ``deux arbres'', \ipa{χsɯ-phɯ} ``trois arbres'' etc sont également possibles). Une relation avec le LB *baŋ² est exclue étant donnée la différence des finales.

\begin{exe}
\ex \label{ex:jpg:arbreclassificateur} 
\gll 	\ipa{tʂu}	\ipa{ɯ-rkɯ}	\ipa{zɯ}	\ipa{si}	\ipa{tɯ-phɯ}	\ipa{pjɤ-tu,} \\
		chemin	3\sgposs{}-bord	\loc{}	arbre	un-\textit{cl}	\med{}\impf{}-avoir \\
\glt Au bord du chemin, il y avait un arbre.	(divination, 9)
\end{exe}
La rime de ce mot avait peut-être une nasale finale, si le classificateur du  \pumi{sbõ¹} (Lanping, \citet[52]{lusz01pumi}) y est apparenté. En qiang du nord, on trouve une forme \ipa{səɸ} ``arbre'' qui est clairement la contraction du nom ``arbre'' ( \jpg{si}, \tib{ɕiŋ}) avec cet ancien classificateur. En tangoute, ce mot peut s'employer comme nom simple sans numéral dans le sens d'arbre (voir exemple (17) p.25). Toutefois, il s'emploie aussi comme classificateur. Avec \tgz{5814}, en effet, le préfixe numéral \tgz{5981}  ``un'' spécifique aux classificateurs est employé (avec les numéraux plus élevé, il n'y a pas de différence):\footnote{On trouve aussi la forme \tgf{0100}\tgf{5814} lew¹ phu² ``un arbre''.}
\newline
\linebreak
\begin{tabular}{llllllll}
		\tgf{2004}&	\tgf{2983}&	\tgf{4274}&	\tgf{5981}&	\tgf{5814}&	\tgf{2590}&	\tgf{3678}&	\tgf{3916} \\
		\tinynb{2004}&	\tinynb{2983}&	\tinynb{4274}&	\tinynb{5981}&	\tinynb{5814}&	\tinynb{2590}&	\tinynb{3678}&	\tinynb{3916} \\
\end{tabular}
\begin{exe}
\ex \label{ex:tg:arbreclassificateur}  \vspace{-8pt}
\gll
		\ipa{khja²}	\ipa{.u²}	\ipa{sow¹}	\ipa{.a-phu²}	\ipa{.wjɨ²-to²-sji²} \\
		puits	intérieur	mûrier	un-arbre	\dir{}-apparaître-\perf{} \\
\glt Un mûrier était apparu dans le puits. (Leilin, 06.19A.1)
\end{exe}

Enfin, la forme \tgf{4250} \tgf{5814} \ipa{sji¹phu²} ``arbre'' (Leilin 05.23B.2) mérite d'être notée: c'est le correspondant presque exact du \jpg{si} \ipa{tɯ-phɯ} et du qiang \ipa{səɸ}. Il s'agit probablement toutefois d'une évolution parallèle, et ces formes ne nous autorisent pas à envisager la reconstruction d'une collocation en proto-macro-rgyalronguique.


\item 2101  \tgf{2101} \ipa{bur} 1.75 \ptang{rmbo}{empiler, accumuler (?)} peut être rapproché du verbe  \jpg{rmbɯ} ``empiler". Ce verbe tangoute n'est malheureusement pas attesté dans les textes que nous avons étudiés jusqu'ici, et son sens est donc incertain. Toutefois, il est glosé dans le dictionnaire \textit{Wénhǎi} par 1305   \tgf{1305} \ipa{twe} 1.08  ``empiler'', un emprunt au chinois \zh{堆} ``empiler''. D'autre part, le caractère \tgf{2101} est analysé comme étant composé de 2063  \tgf{2063} \ipa{dzjiw} 1.45 pour sa partie gauche\footnote{En fait, la partie de ``l'homme'' \tgf{2541}, l'un des éléments les plus courants du système graphique tangoute. Le choix du verbe ``accumuler'' pour cette analyse est donc clairement utilisable comme une glose sémantique de ce verbe.}  et 5820  \tgf{5820} \ipa{.ụ} 1.58 pour sa partie droite. Or, \tgf{2063} \ipa{dzjiw¹} est clairement attesté dans le sens ``d'accumuler" (Cixiaozhuan, 28.1). 

Malgré l'absence d'attestations textuelles hors du dictionnaire, il est donc raisonnable de comparer ce verbe à celui du japhug, étant donné la régularité des correspondances phonétiques. Le témoignage du pumi (Lanping \pumi{sbu¹}, Shuiluo \pumi{bú}) montre que cet étymon n'avait pas de nasale finale en proto-macro-rgyalronguique, et qu'une comparaison avec le \tib{pʰuŋ-po} ``tas'' ou le \bir{puṃ²} n'est pas envisageable.
\end{enumerate}


B. initiales dentales  \label{rimes:01:2:dent}
\newline

\begin{enumerate}

\item 2196 \tgf{2196} \ipa{tu} 1.01 ``épais" pourrait être apparenté au \plb{tu¹}{0531} ``épais'' et reconstruit *to en pré-tangoute.  \citet{lifw97} propose toutefois qu'il s'agirait d'un emprunt au chinois \zh{篤} twok ``sincère''. En l'absence d'attestation en dehors des dictionnaires, il est difficile de décider laquelle de ces étymologies est correcte.


\item 5149   \tgf{5149} \ipa{duu} 1.05 \ptang{ndoo}{accumuler} peut se comparer au  \jpg{ajtɯ} ``s'accumuler''. Le verbe japhug est intransitif, mais le préfixe a- est un ancien détransitivant, ce qui suggère que la forme non-préfixée devait être transitive. On a également en japhug la forme apparentée \jpg{ndɯ} ``s'accumuler en une seule fois'', qui correspond peut-être mieux au tangoute. En tangoute, il n'est pas entièrement simple de déterminer si ce verbe est vraiment transitif. Observons l'exemple suivant:
\newline
\linebreak
\begin{tabular} {llllllllll}
		\tgf{0102}	&\tgf{4408}	&\tgf{1918}	&\tgf{4663}	&\tgf{0322}&	\tgf{4456}&	\tgf{5655}&	\tgf{0159}&	\tgf{5149} &\tgf{0429}\\
		\tinynb{0102}	&\tinynb{4408}	&\tinynb{1918}	&\tinynb{4663}	&\tinynb{0322}&	\tinynb{4456}&	\tinynb{5655}&	\tinynb{0159}&	\tinynb{5149} &\tinynb{0429}\\
		\tgf{0100}&	\tgf{1388}&	\tgf{2194}&	\tgf{3133}&	\tgf{1999}&	\tgf{4729}&	\tgf{3126}&&&\\
		\tinynb{0100}&	\tinynb{1388}&	\tinynb{2194}&	\tinynb{3133}&	\tinynb{1999}&	\tinynb{4729}&	\tinynb{3126}&&&\\
\end{tabular}
\begin{exe}
\ex \label{ex:tg:accumuler}  \vspace{-8pt}
\gll \ipa{gjɨ²} \ipa{məə¹} \ipa{mji¹-lha¹} \ipa{tɕhjwo¹}	\ipa{ljịj²}	\ipa{ljɨ̣¹}	\ipa{no²}	\ipa{duu¹} \ipa{njwo²}	\ipa{lew¹}	\ipa{ljii¹}	\ipa{mjij¹}	\ipa{sjij¹}	\ipa{ŋwə¹}	\ipa{lhwu¹}	\ipa{dʑjij²}\\
		nuit feu \negat{}-éteindre donc	grand	trésor	richesse	accumuler autrefois un pantalon ne.pas.avoir aujourd'hui cinq habit avoir\\
\glt Il ne faisait pas éteindre le feu la nuit, et c'est ainsi qu'ils accumulèrent de grandes richesses, autrefois (ces gens) n'avaient même pas un pantalon, maintenant ils ont cinq costumes. (Leilin, 04.19A.1)
\end{exe}



Cet exemple pourrait également s'interpréter comme ``de grandes richesses se sont accumulées''. Seul un exemple avec l'agent marqué à l'ergatif pourrait permettre de déterminer si oui ou non ce verbe est transitif. 

On peut rapprocher aussi le verbe \tgz{5418} ``conserver''.


\item 1490 \tgf{1490} \ipa{tsur} 1.75 \ptang{r-tso}{hiver} est cognat avec le  \jpg{qartsɯ} ``hiver''. Ce mot se retrouve en LB et en naish: \bir{choŋ³}, \naxi{mɯ˧tsʰɯ˧},	 \nayn{tsʰi˥},	\laze{mu˧tsʰy˧bie˧},	\protona{tsʰu} voir \citet{jacques.michaud11naish}.



\item 1464   \tgf{1464} \ipa{dzur/tsur} 1.75 \ptang{rtso}{donner un coup de pied} et la forme apparentée 1098 \tgf{1098} \ipa{tsjụ} 1.59 \ptang{S-rtso}{donner un coup} se comparent au  \jpg{tɯ-qartsɯ} ``coup de pied". La reconstruction de \tgf{1464} pose problème. En effet, il est placé dans la section des voisées dans le Wenhai zalei, mais son \ipa{fǎnqiè} suggère une initiale sourde : [3641  \tgf{3641} \ipa{tsjir} 1.79] + [2795   \tgf{2795} \ipa{rur} 1.75]. On privilégiera ici la lecture de son \ipa{fǎnqiè} \ipa{tsur}, qui correspond mieux à la comparaison avec le japhug. Le caractère verbal de ce mot en tangoute, contrairement au japhug, est prouvé par un exemple tel que :
\newline
\linebreak
\begin{tabular} {lllllllll}
		\tgf{5108}&	\tgf{3601}&	\tgf{1326}&	\tgf{1464}&	\tgf{5113}&	\tgf{5509}&	\tgf{5643}&	\tgf{1374}&	\tgf{1839} \\
		\tinynb{5108}&	\tinynb{3601}&	\tinynb{1326}&	\tinynb{1464}&	\tinynb{5113}&	\tinynb{5509}&	\tinynb{5643}&	\tinynb{1374}&	\tinynb{1839} \\
\end{tabular}
\begin{exe}
\ex \label{ex:tg:coupdepied}  \vspace{-8pt}
\gll \ipa{sjo²khew²}	\ipa{kjɨ¹-tsur¹}	\ipa{.wji¹}	\ipa{bjị¹}	\ipa{mjɨ¹-tɕhjɨ¹-lhjo¹} \\
		vessie	\dir{}-donner.un.coup.de.pied	faire[A]	urine  \negat{}-\pot{}-perdre \\
\glt Lorsque l'on reçoit un coup de pied à la vessie, il est impossible de ne pas perdre son urine. (\citealt[178]{kychanov74})
\end{exe} 

Le japhug n'a préservé ici que le dérivé nominal, qui apparaît toujours avec un préfixe numéral \textit{tɯ}--.


\item 2396   \tgf{2396} \ipa{dzuu} 2.05 \ptang{ndzoo}{s'asseoir} se compare au  \jpg{amdzɯ} (attesté avec une variante \ipa{amdzɯt}) ``s'asseoir" et au naish \naxi{ndzɯ˩},	\nayn{dzi˩},	\laze{dzy˥},	\protona{ndzu}. Cet étymon est d'une distribution très large en ST, puisqu'il se retrouve même en chinois \zh{坐} \ipapl{dzwaX < *dzˁojʔ}. Un dérivé en tangoute est le nom du ``siège'' : 2397  \tgf{2397} \ipa{dʑjwi} 2.09.  L'alternance des consonnes initiales et des voyelles dans cette paire, toutefois, ne trouve de parallèle nulle part ailleurs dans la langue. Outre ``s'asseoir", \tgf{2396} \ipa{dzuu²} a de nombreux usages dérivés, dont ``prendre un bateau":
\newline
\linebreak
\begin{tabular} {llll}
		\tgf{3070}&	\tgf{3798}&	\tgf{0089}&	\tgf{2396} \\
		\tinynb{3070}&	\tinynb{3798}&	\tinynb{0089}&	\tinynb{2396} \\
\end{tabular}
\begin{exe}
\ex \label{ex:tg:asseoir}  \vspace{-8pt}
\gll  \ipa{dzjwɨ̣¹}	\ipa{tsəj¹}	\ipa{tɕhjaa¹}	\ipa{dzuu²} \\
		bateau	petit	sur	s'asseoir \\
\glt Il prenait un petit bateau (Leilin 03.35A.5)
\end{exe} 


\item 3266	\tgf{3266} \ipa{dzju}	2.03 ``maître'' n'a pas de cognat en rgyalrong ou en tibétain, mais peut se comparer au \plb{m-dzo²}{0365}. \citet[123]{matisoff03} propose une comparaison plutôt avec \tgz{5306} ``empereur'', mais nous pensons que ce dernier mot doit dériver de \tgz{5043} ``juger'' avec une alternance entre tons 2 et 1, comme nous l'avons mentionné dans \citet{jacques10imperial}.

\item 4614   \tgf{4614} \ipa{nju} 2.03 \ptang{njo}{téter (à propos d'un nourisson)}  et 4834  \tgf{4834} \ipa{njụ} 2.52 \ptang{S-njo}{donner le sein, allaiter} peuvent être rapproché du  \jpg{tɯ-nɯ} ``sein'' et   surtout de son dérivé verbal \jpg{nɯnɯ} ``téter''. On trouve également la forme 53  \tgf{0053} \ipa{njọ} 2.64 qui semble être le radical 2 de \tgf{4614} \ipa{njụ²} mais on manque d'exemples textuels clairs. 


Le nom ``sein'' en revanche se dit 2123  \tgf{2123} \ipa{new} 1.43 (*PT nok) en tangoute, forme que l'on rapprochera plutôt du zbu \ipa{tə-nôx} avec une finale vélaire. Les formes à vélaires (*nok) et à syllabe ouverte (*njo) étaient probablement distinctes déjà en proto-macro-rgyalronguique, mais pourraient avoir un rapport étymologique plus lointain. La forme 5275   \tgf{5275} \ipa{ner} 2.71 ``sein'' serait quant à elle selon \citet{lifw97} un emprunt au chinois \zh{奶} \ipapl{nɛɨX} ; c'est une hypothèse toutefois problématique dans la mesure où aucun autre emprunt chinois n'a la rime 2.71 (Gong 1981b). La rime \ipapl{ɛɨ} du chinois correspond normalement à la rime \ipapl{-iej} 35 (1.34-2.31) du tangoute dans les emprunts (Gong [1981b]2002:421). Toutefois, s'il ne s'agit pas d'un mot chinois, nous ne sommes pas en mesure de proposer une reconstruction en pré-tangoute.

La racine se retrouve dans le \tib{nu.ma} ``sein'' et le verbe dénominal correspondant au \tib{nu, nus} ``téter'' et au causatif \tib{snun, bsnun} ``allaiter''. Le \plb{no³}{0119a} et le chinois \zh{乳} \ipapl{nyuX} y sont également apparentés. Les deux verbes tangoutes sont très clairement attestés :
\newline
\linebreak
\begin{tabular}{lllllll}
		\tgf{2541}&	\tgf{3300}&	\tgf{1906}&	\tgf{2365}&	\tgf{0276}&	\tgf{5872}&	\tgf{4834} \\
		\tinynb{2541}&	\tinynb{3300}&	\tinynb{1906}&	\tinynb{2365}&	\tinynb{0276}&	\tinynb{5872}&	\tinynb{4834} \\
\end{tabular}
\begin{exe}
\ex \label{ex:tg:allaiter}  \vspace{-8pt}
\gll   \ipa{dzjwo²}	\ipa{dju}	\ipa{nioow¹}	\ipa{pha¹}	\ipa{no²ɕjọ²}	\ipa{njụ²}  \\
 		homme	naître	après	ailleurs	nourrice	allaiter \\
 \glt Lorsqu'un être humain naît,  une nourrice l'allaite.   (Shengliyihai, 32a.1, \citealt[162,264]{kychanov97})\footnote{Le terme ``nourrice'' \tgf{0276}\tgf{5872} \ipa{no²ɕjọ²} n'est pas mentionné par les dictionnaires, mais se retrouve dans le chapitre sur les degrés de deuil du code de loi tangoute.}
\end{exe}


\begin{tabular}{llllllllll}
		\tgf{2541}&	\tgf{2590}&	\tgf{3300}&	\tgf{1906}&	\tgf{0377}&	\tgf{1319}&	\tgf{0092}&	\tgf{1139}&	\tgf{2801}&	\tgf{5555} \\
		\tinynb{2541}&	\tinynb{2590}&	\tinynb{3300}&	\tinynb{1906}&	\tinynb{0377}&	\tinynb{1319}&	\tinynb{0092}&	\tinynb{1139}&	\tinynb{2801}&	\tinynb{5555} \\
		\tgf{3065}&	\tgf{3966}&	\tgf{4614}&	\tgf{5285} \\
		\tinynb{3065}&	\tinynb{3966}&	\tinynb{4614}&	\tinynb{5285} \\
\end{tabular}
\begin{exe}
\ex \label{ex:tg:teter}  \vspace{-8pt}
\gll  \ipa{dzjwo²}	\ipa{.wjɨ²-dju}	\ipa{nioow¹}	\ipa{njwij¹}	\ipa{tshji¹}	\ipa{mja¹}	\ipa{.jij¹}	\ipa{lhjuu²}	\ipa{njiij¹}	\ipa{lhju}	\ipa{.wjị¹}	\ipa{nju²}  \\
		homme	\dir{}-naître	après	nourriture	important mère	\antierg{}	moelle	centre	lait	goût	 téter \\
 \glt Quand un humain naît, il est important de le nourrir. Il tète le lait délicieux venu de la moelle de sa mère, (Shengliyihai, 32a.3, \citealt[162,264]{kychanov97}) 
\end{exe}

La forme causative \tgf{4834} \ipa{njụ²} est une des seules formes tangoutes à apparaître dans les textes tibétains, comme nous l'avons montré dans \citet{jacques08debther}. Dans le \textit{Deb.ther dmar-po} ``Le livre rouge'', on trouve en effet le passage suivant à propos du roi du Mi.nyag :
\begin{exe}
\ex \label{ex:tib:voma}
\gll
		rus.pa ŋo.snu'i ʑes-pa Bod-skad-du bsgʲur-na / ba-la ɦo.ma ɴtʰuŋ-ba ʑes zer \\
		os ? appeler-\nmls{} Tibet-langue-\textsc{terminatif} changer.\ps{}-si / vache-\dat{} lait boire-\nmls{} appeler dire \\
\glt		Son nom de clan fut Ngo.snu'i, ce qui en tibétain se traduit par ``celui qui boit le lait de la vache''.
\end{exe}

Nous avons proposé d'interpréter Ngo.snu'i, nom opaque en tibétain, comme la transcription de *\tgf{0395}\tgf{4834} \ipa{ŋwe² njụ²} qui pourrait se traduire comme ``La vache allaite'', comme nous l'avons montré dans \citet{jacques11ngwemi}.


\item 5621 \tgf{5621} \ipa{lhu} 1.01 \ptang{lho}{ajouter, rajouter} est comparable au  \jpg{ɣɤjɯ} ``rajouter" (proto-japhug *--lju). A l'intérieur du tangoute, on trouve d'autres formes apparentées : 1538  \tgf{1538} \ipa{lhju} 1.03, 1763 \tgf{1763} \ipa{lhu}, 2190 \tgf{2190} \ipa{lhjɨ} 1.30. Dans le Wenhai, ces caractères se servent mutuellement de glose ; cela montre que leur sens est proche, mais est insuffisant pour comprendre la différence d'usage entre les quatre et les alternances morphologiques en jeu dans cette famille de mots. Seul \tgf{5621} \textit{lhu¹} est attesté dans un usage verbal dans les textes à notre disposition:
\newline
\linebreak
\begin{tabular}{llllllllll}
		\tgf{3045}&	\tgf{5964}&	\tgf{0824}&	\tgf{5417}&	\tgf{3196}&	\tgf{2403}&	\tgf{2750}&	\tgf{0089}&	\tgf{2449}&	\tgf{2403}  \\
		\tinynb{3045}&	\tinynb{5964}&	\tinynb{0824}&	\tinynb{5417}&	\tinynb{3196}&	\tinynb{2403}&	\tinynb{2750}&	\tinynb{0089}&	\tinynb{2449}&	\tinynb{2403}  \\
		\tgf{1452}&	\tgf{5621}&	\tgf{5113}&	\tgf{5285}& &&&&&  \\
		\tinynb{1452}&	\tinynb{5621}&	\tinynb{5113}&	\tinynb{5285}& &&&&&  \\
\end{tabular}
\begin{exe}
\ex \label{ex:tg:ajouter}  \vspace{-8pt}
\gll   \ipa{tshew¹tsha²}	\ipa{tɕhjɨ²rjar²}	\ipa{ljɨ²}	\ipa{dji²}	\ipa{ɣu¹}	\ipa{tɕhjaa¹}	\ipa{be²}	\ipa{dji²}	\ipa{nja¹-lhu¹}	\ipa{.wji¹} \\
		Cao.Cao	immédiatement	`li'	caractère	tête	sur soleil	caractère	\dir{}-ajouter	faire[A] \\
\glt Cao Cao (\zh{曹操}) ajouta immédiatement le caractère ``soleil'' \zh{日} \textit{rì} sur le caractère ``se tenir debout'' \zh{立} \textit{lì} (Leilin 06.22A.3)
\end{exe} 

Le caractère \tgf{1763} \ipa{lhu} a quelques attestations connues en dehors du Wenhai, comme le ZZZ (113), où il apparaît dans le composé \tgf{1763} \tgf{2814} \ipa{lhu lhjị²} (2814) ``mois intercalaire'' (chinois \zh{閏月}).


\item 4506 \tgf{5621} \ipa{lu} 2.01 \ptang{lo}{allumer un feu} peut être rapproché du  \jpg{βlɯ} ``brûler, allumer un feu'' (*plu). Une comparaison avec le \plb{ʔ-loŋ¹}{0517} ``hot'' est envisageable.

En japhug, ce verbe désigne spécifiquement un feu utilisé pour se réchauffer à la maison (il peut désigner également le fait de brûler des ronces ou de brûler vifs des criminels). C'est un verbe transitif, dont l'objet est habituellement (mais pas exclusivement) \ipa{smi} ``feu'' :
\begin{exe}
\ex \label{ex:jpg:bruler} 
\gll 
		\ipa{tɕendɤre}	\ipa{jɤ-azɣɯt}	\ipa{ndɤre,}	\ipa{ɯ-wa}	\ipa{cho}	\ipa{nɤ}	\ipa{iɕqha}	\ipa{jamɤn}	\ipa{ni}	\ipa{kɯ}	\ipa{smi}	\ipa{ta-βlɯ́-ndʑi}	\ipa{ɲɯ-ŋu} \\
		\conj{}	\aor{}-arriver	\conj{}	3\sgposs{}-père	et	\conj{} à.l'instant	Yamon	\du{}	\erg{}	feu	\aor{}:3>3-brûler-\du{}	\impf{}-être \\
\glt Il arriva (à la maison de son père), et son père et Yamon allumèrent un feu. (Blobzang, 76)
\end{exe} 
En tangoute, il peut s'employer pour désigner une brûlure sur le corps (Leilin 04.23A.4) ou un incendie :
\newline
\linebreak
\begin{tabular} {llllll}
		\tgf{1374}&	\tgf{0705}&	\tgf{4408}&	\tgf{5791}&	\tgf{2862}&	\tgf{4506} \\
				\tinynb{1374}&	\tinynb{0705}&	\tinynb{4408}&	\tinynb{5791}&	\tinynb{2862}&	\tinynb{4506} \\
\end{tabular}
\begin{exe}
\ex \label{ex:tg:brulerlu}  \vspace{-8pt}
\gll   \ipa{tɕhjɨ¹}	\ipa{zjịj¹}	\ipa{məə¹}	\ipa{.wjạ²}	\ipa{nji¹}	\ipa{lu²} \\
		ce	temps	feu	lâcher	maison	brûler \\
\glt A ce moment, elle mit le feu et brûla sa maison (Leilin 04.11A.4)
\end{exe} 
Le sens de ce verbe en tangoute ne recouvre donc pas exactement celui du japhug. Parmi les verbes signifiant ``brûler'' en tangoute, outre \tgf{4506} \ipa{lu²}, on trouve \tgf{4413} \ipa{pju²} (voir p.\pageref{ex:tg:brulerpju}) ``brûler, cuire'' et \tgf{5192} \ipa{njwɨ̣²}  ``brûler du bois, allumer une lampe/une bougie'' qui ont aussi des cognats japhug. 


\item 1304  \tgf{1304} \ipa{lụ} 1.58 \ptang{S-lo}{ver} est comparable au  \jpg{qajɯ} ``ver'' et probablement au \tib{klu} ``nāga"'. Ce caractère ne forme pas un mot à lui tout seul, et apparaît toujours dans les composés \tgf{1888}\tgf{1304} \ipa{bə²lụ¹} (1888) ou \tgf{5270}\tgf{1304} \ipa{bəə¹lụ¹} (5270). Un exemple textuel de ce mot apparaît p.\pageref{ex:tg:bouillir}. \label{analyse:ver}


\item 2273  \tgf{2273} \ipa{lụ} 2.51 \ptang{S-lo}{tige} peut être comparé au  \situ{wu-lû} ; ``poignée, tige''. Le  \jpg{ɯ-jɯ} ``tige, poignée" est probablement un emprunt au \tib{ju.ba} ``poignée'', qui est quant à lui sans relation étymologique avec le tangoute \tgf{2273} \ipa{lụ²}.


\item 124  \tgf{0124} \ipa{ljụ} 2.52 \ptang{S-ljo}{tête} n'a de cognats ni en rgyalrong ni en tibétain, mais peut être rapproché du chinois \zh{首} \ipapl{*l̻uʔ}. C'est la seule langue macro-rgyalronguique à préserver ce cognat. Ce mot est d'un usage moins courant que \tgf{2750} \ipa{ɣu¹} (p.29), et à la différence de ce dernier il n'est pas utilisé métaphoriquement dans le sens de ``partie supérieure, dessus''.
\newline
\linebreak
\begin{tabular} {llllll}
		\tgf{0124}&	\tgf{2061}&	\tgf{0074}&	\tgf{1572}&	\tgf{4342}&	\tgf{2226} \\
		\tinynb{0124}&	\tinynb{2061}&	\tinynb{0074}&	\tinynb{1572}&	\tinynb{4342}&	\tinynb{2226} \\
\end{tabular}
\begin{exe}
\ex \label{ex:tg:tete2}  \vspace{-8pt}
\gll  \ipa{ljụ²}	\ipa{piə̣j²}	\ipa{khwə¹}	\ipa{phiow¹}	\ipa{dja²-.we²} \\
		tête	poil	moitié	blanc		\dir{}-devenir \\
 \glt Ses cheveux étaient à moitié blancs (Leilin, 03.12B.5-6)
\end{exe}
Toutefois, on remarque un emploi particulier de \tgf{0124} \ipa{ljụ²} pour traduire ``ciel'' dans ``empereur céleste'' \zh{天帝}:
\newline
\linebreak
\begin{tabular} {llllllllll}
		\tgf{0124}&	\tgf{5306}&	\tgf{1045}&	\tgf{2978}&	\tgf{3639}&	\tgf{5285} &\tgf{1278}\\
				\tinynb{0124}&	\tinynb{5306}&	\tinynb{1045}&	\tinynb{2978}&	\tinynb{3639}&	\tinynb{5285} &\tinynb{1278}\\
\end{tabular}
\begin{exe}
\ex \label{ex:tg:tete2.2}  \vspace{-8pt}
\gll 		 \ipa{ljụ²}	\ipa{dzjwɨ¹}	\ipa{dạ²}	\ipa{tjij²rjijr²}	\ipa{ljɨ¹} \ipa{.jɨ²} \\
			tête	seigneur	dire	bien	\cop{} dire\\
 \glt L'empereur céleste dit : ``C'est bien." (Leilin 08.09A.4).
\end{exe}

\item 1784  \tgf{1784} \ipa{lụ} 1.58 \ptang{S-lo}{homme} pourrait être apparenté au \bir{lu²} ``homme''. Il apparaît dans la Grande Ode, ligne 2.6.

\item 3176 \tgf{3176}  \ipa{ɕjụ} 1.59 ``lente'' ressemble au  \jpg{ndʑrɯ} et au \tib{sro.ma}, mais une proto-forme *srjo devrait donner plutôt *zjur. Il pourrait s'agir d'un emprunt au tibétain (d'un dialecte où sr-- > ʂ--), et dans l'incertitude nous ne donnerons pas de reconstruction.

\end{enumerate}


C. Initiales vélaires  \label{rimes:01:2:vel}
\newline

\begin{enumerate}

\item 4189 \tgf{4189} \ipa{khu} 1.04 \ptang{kho}{bol} peut se comparer au  \jpg{khɯtsa} ``bol''.



\item 3673 \tgf{3673} \ipa{ɣju} 1.03  \ptang{C-kjo}{fumée} peut être rapproché du  \jpg{tɤ-khɯ} ``fumée" et du \plb{ko²}{0333}. Le verbe ``fumer'' 4323  \tgf{4323} \ipa{kjur} 1.76  (PT*r-kjo) en est un dérivé ; il est attesté comme verbe transitif dans un exemple   cité dans Li (1997):
\newline
\linebreak
\begin{tabular} {lllll}
		\tgf{2541}&	\tgf{1139}&	\tgf{4684}&	\tgf{1139}&	\tgf{4323} \\
		\tinynb{2541}&	\tinynb{1139}&	\tinynb{4684}&	\tinynb{1139}&	\tinynb{4323} \\
\end{tabular}
\begin{exe}
\ex \label{ex:tg:fumer}  \vspace{-8pt}
\gll 	\ipa{dzjwo²}	\ipa{.jij¹}	\ipa{mej¹}	\ipa{.jij¹}	\ipa{kjur¹} \\
		homme	\antierg{}	œil	\antierg{}	fumer \\
 \glt (Il) enfume les yeux de l'homme.\footnote{Traduction effectuée sans tenir compte du contexte d'où est tirée cette phrase, car nous n'avons pas accès au texte complet.}
\end{exe}


\item 1136 \tgf{1136} \ipa{gu} 2.01 \ptang{ŋgo}{intérieur} se compare avec le   \jpg{ŋgɯ} de même sens. Dans les deux langues, ce nom tend à devenir une marque de locatif. Le \plb{koŋ¹}{0812} est un cognat potentiel.



\item 1909 \tgf{1909} \ipa{gur} 1.75 \ptang{r-ŋgo}{bœuf}peut se comparer au  \situ{kərgú} ; ce mot a disparu en japhug.
\newline
\linebreak
\begin{tabular} {llll}
		\tgf{0261}&	\tgf{1139}&	\tgf{1909}&	\tgf{1943} \\
		\tinynb{0261}&	\tinynb{1139}&	\tinynb{1909}&	\tinynb{1943} \\
\end{tabular}
\begin{exe}
\ex \label{ex:tg:boeuf}  \vspace{-8pt}
\gll 		\ipa{mjo²}	\ipa{.jij¹}	\ipa{gur¹}	\ipa{nja²} \\
			moi	\antierg{}	bœuf	ne.pas.être \\
\glt Ce n'est pas mon bœuf (Leilin 04.14B.5)
\end{exe}

\tgf{1909} \ipa{gur¹}  désigne les bœufs en tant qu'espèce (et peut apparaître dans des composés), tandis que 395 \tgf{0395} \ipa{ŋwe} 2.07 signifie plutôt spécifiquement la ``vache''.


\item 3388 \tgf{3388} \ipa{ŋwu} 2.01 \ptang{ŋo}{pleurer} est apparenté au  \tib{ŋu} ``pleurer'' et au \plb{ŋo¹}{0670}. La comparaison avec le  \jpg{ɣɤwu} proposée dans \citet{jacques06comparaison} paraît à la réflexion très problématique, et ne sera pas suivie dans le présent travail. Ce verbe apparaît le plus souvent dans les composés  \tgf{3388} \tgf{5614} \ipa{ŋwu² lwụ¹} (5614) et \tgf{3388} \tgf{1029} \ipa{ŋwu² kwar¹} (1029).
\end{enumerate}

\subsubsection{Autres correspondances} \label{subsubsec:correspondance:u:autre}


Outre les deux correspondances principales présentées ci-dessus, on en observe également d'autres, plus limitées. Certaines portent sur des exemples qui sont soit des emprunts au tibétain, soit des comparaisons peu probables, mais qui méritent néanmoins d'être mentionnées.

\begin{enumerate}


\item --ju :: --a \label{rimes:01:3:ju/a}


On rencontre trois cas de –ju en tangoute correspondant à –a en japhug ou en tibétain. 


2385 \tgf{2385} \ipa{ɕju} 2.02 ``viande, chair" pourrait correspondre au \tib{ɕa} \plb{xa²}{0135}  de même sens. Toutefois, un autre mot pour ``viande'' (bien mieux attesté dans les textes) serait éventuellement comparable à \tib{ɕa} : 3465 \tgf{3465} \ipa{tɕhji} 1.10 . La rime de ce mot correspond bien à pré-tangoute *--ja. Concernant le problème de l'initiale \ipapl{tɕh-}, voir p. \pageref{analyse:viande}.


Le meilleur exemple pour soutenir la correspondance --ju :: --a est 4681  \tgf{4681} \ipa{nju} 1.03 ``oreille", correspondant au  \jpg{tɯ-rna}, \tib{rna}, au \plb{(C)-na²}{0102} et au naish \naxi{he˥tsɯ˩},	\nayn{ɬi.pi^L},	\laze{ɬie˧tu˥},	\protona{l ̥a}. Une forme pré-tangoute \ipapl{*rn(j)a} devrait donner la forme non-attestée *ne ou *nji (la préinitiale *r-- chute devant *n-- en pré-tangoute, voir p.\pageref{tab:sans.preinitiale.r:preinitiale.r}).

Finalement, le verbe 1338 \tgf{1338} \ipa{dzu} 1.01 ``aimer'' pourrait être comparé au \tib{mdza} ``proche, amical'' si la correspondance  --ju :: --a est confirmée. Ce verbe a un thème B irrégulier 4973 \tgf{4973} \ipa{dzju} 1.02, qui est attesté avec les suffixes de 1\sg{} et 2\sg{} :
\newline
\linebreak
\begin{tabular}{llll}
	\tgf{0558}&	\tgf{3844}&	\tgf{4973}&	\tgf{2098}\\
\tinynb{0558}&	\tinynb{3844}&	\tinynb{4973}&	\tinynb{2098}\\
\end{tabular}
\begin{exe}
\ex \label{ex:tg:aimer.b}  \vspace{-8pt}
\gll   \ipa{njijr¹}	\ipa{dʑjịj¹}	\ipa{dzju¹-ŋa²} \\
		gibier aller aimer[B]-1\sg{} \\
\glt Je souhaite aller chasser. (Les douze royaumes N°133.1.4, \citealt[54,133]{solonin95})
\end{exe}


Parmi ces trois exemples, les deux derniers ont des cognats en chinois avec la rime \zh{之} zhī du chinois archaïque, à savoir \zh{耳} *\ipapl{nɨʔ} et \zh{慈} *\ipapl{dzɨ}.\footnote{\citet[61-2]{sagart99roc} propose de reconstruire un *--\ipapl{ŋ} final en chinois archaïque dans le mot oreille. Si cette consonne finale a effectivement existé en chinois archaïque dans ce mot, il doit s'agir d'une innovation propre au chinois.} Il semble que le tangoute préserve la distinction  entre *ə et *a qui se retrouve en chinois, contrairement à la plupart des autres langues qui les confondent en *a. Le témoignage du tangoute est crucial pour le sino-tibétain: il prouve que les langues non-chinoises n'ont pas l'innovation commune consistant à confondre *a et *ə.  On proposera ici un \ptang{njə}{oreille} et \ptang{ndzə}{aimer}.

La distinction entre *a et *ə a toutefois été perdue en tangoute en syllabe ouverte et précédée d'une initiale labiovélaire, comme dans le nom de la vache \tgz{0395}.

Etant donné le fait que le verbe \tgz{1338} présente une alternance irrégulière, il n'est pas injustifié de poser \tgf{1338} \ipa{dzu¹} < *ndzə et  \tgf{4973} \ipa{dzju¹} < *ndzə-u en pré-tangoute, même si l'alternance entre rimes 1.01 et 1.02 est assez inattendue.

\item --un en tibétain

Cette correspondance est en fait du même type que pour \tgz{5396} et \tgz{4174} étudiés plus haut, qui correspondaient à des étymons en *-un en proto-LB ou en limbu. On peut y rajouter la famille de mots suivante:


3925 \tgf{3925} \ipa{mur} 1.75 ``sombre", 2727 \tgf{2727} \ipa{mur} 1.75 ``troubler, égarer'' et 2764 \tgf{2764} \ipa{mur} 1.75 ``homme du commun, stupide" pourraient être rapproché du \tib{mun.pa} ``ténèbres", \tib{mun-po} ``sombre" et \tib{dmun-pa} ``stupide" ainsi que du \bir{hmun²} et du chinois \zh{昏} \ipapl{xwon (<*m̻ˁən)}. Une comparaison avec \tib{smug-po} ``violet, sombre" semble en revanche impossible du point de vue phonétique. Il pourrait s'agir d'un emprunt au tibétain, ou bien d'un cognat. Une autre possibilité serait un rapprochement avec la racine \ipapl{-mɯr} ``soir'' qui apparaît dans \jpg{jɯɣmɯr} ``ce soir'' et \jpg{mɯrkɯrku} ``tous les soirs'', mais elle est peut probable, car un pré-tangoute *mur donnerait *\ipapl{mər} et non \ipapl{mur}, voir \ref{subsubsec:correspondance:eu:vr}. 

On trouve également une forme rédupliquée 5078 \tgf{5078}\tgf{3925} \ipa{mjɨɨ¹mur¹}  ``soir''.


\item --u :: \ipapl{--ɤβ (<*--ɔp)} \label{rimes:01:3:u:vw:}


1899 \tgf{1899} \ipa{tju} 2.03 ``frapper" et 3679 \tgf{3679} \ipa{tjwɨ} 1.30 sont peut-être comparables au  \jpg{rtɤβ} ``frapper'' (\situ{tóp}). 
\newline
\linebreak
\begin{tabular} {lllllll}
		\tgf{5970}&	\tgf{0546}&	\tgf{2872}&	\tgf{2649}&	\tgf{1139}&	\tgf{2590}&	\tgf{1899} \\
		\tinynb{5970}&	\tinynb{0546}&	\tinynb{2872}&	\tinynb{2649}&	\tinynb{1139}&	\tinynb{2590}&	\tinynb{1899} \\
\end{tabular}
\begin{exe}
\ex \label{ex:tg:frapper} 
\gll 	\ipa{pie¹.u²}	\ipa{.wẽ¹tɕjow¹}	\ipa{.jij¹}	\ipa{.wjɨ²-tju²} \\
		Bowu	Wenzhang	\antierg{}	\dir{}-frapper \\
\glt Bowu frappa Wenzhang. (Leilin 07.15A.5)
\end{exe}

Toutefois, cette comparaison est douteuse si l'on considère les correspondances du proto-rgyalrong \ipapl{*--ɔp} avec --ew (voir \ref{subsec:voyelle.ew}).

\item --u :: \ipapl{--o (<*--aŋ)} \label{rimes:01:3:u:ang}

4789 \tgf{4789} \ipa{nju} 1.03 ``un type de légume" est peut-être apparenté au  \jpg{sɯjno} ``légume" et 5716  \tgf{5716} \ipa{thu} 2.01 ``mouton" peut peut-être être rapproché du  \jpg{thoɲa}. Ces deux exemples sont toutefois assez douteux, et il peut s'agir soit de Wanderwörter, soit de coïncidences.

\item Emprunts tibétains



A ces exemples, on doit ajouter plusieurs emprunts au tibétain.
\newline

1546 \tgf{1546} \ipa{ljụ} 2.52 ``corps" pourrait être emprunté au \tib{lus} ``corps''. il ne peut s'agir d'un cognat, car le sens premier de cette racine en tibétain est ``rester'' et le sens de corps en est dérivé, lus-po, avec suffixe nominalisateur, signifiant ``ce qui est laissé'', la forme nominale lus monosyllabique étant simplement l'abréviation de lus-po. Néanmoins, on doit noter que dans les langues du Sichuan qui empruntent ce mot au tibétain, il l'est normalement sous la forme dissyllabique \textit{lus-po}. Il pourrait donc s'agir d'une coïncidence. Un pré-tangoute *l(j)us devrait de toute façon donner *\ipapl{ljwɨ} ou *\ipapl{lwə} en tangoute.
\newline

3140 \tgf{3140} \ipa{lhjụ} 2.52 ``chanson" est probablement un emprunt au \tib{glu} ``chanson''. Aucun cognat de glu n'est en effet connu en dehors des langues bodiques, et il est inenvisageable que ce soit un cognat. Ce pourrait être aussi une coïncidence.
\newline

1175 \tgf{1175} \ipa{ljwu} 1.03 ``tromper" est vraisemblablement un emprunt au verbe \tib{slu} ``tromper'', car ce verbe n'a pas de cognats externes. 
\newline

5217 \tgf{5217} \ipa{kuu} 1.05 signifie à la fois ``creuser (la terre)" et ``graver des caractères" (\tgf{0489}\tgf{5217} dji²kuu¹), tout comme le \tib{rko}. Le fait qu'il partage avec le tibétain le sens secondaire de ``graver" suggère qu'il s'agirait plutôt d'un emprunt. Dans les langues rgyalronguiques, ce verbe a également été emprunté (\jpg{rkɤz} <\ipapl{*rkɔs}).
\newline

2498 \tgf{2498} \ipa{dzuu} 2.05 ``planter, être planté" peut se comparer au \tib{ɴdzugs, btsugs} ``planter, installer", un verbe transitif. En tangoute, ce verbe a toutefois plutôt le sens ``être planté'' à propos d'un arbre, et nous n'avons trouvé qu'un seul emploi transitif :
\newline
\linebreak
\begin{tabular} {llllllllll}
		\tgf{4935}&	\tgf{5880}&	\tgf{1413}&	\tgf{3045}&	\tgf{1139}&	\tgf{1084}&	\tgf{3484}&	\tgf{5422}&	\tgf{5399} & \tgf{2498} \\
		\tinynb{4935}&	\tinynb{5880}&	\tinynb{1413}&	\tinynb{3045}&	\tinynb{1139}&	\tinynb{1084}&	\tinynb{3484}&	\tinynb{5422}&	\tinynb{5399} & \tinynb{2498} \\
	\tgf{5113}&&&&&&&&& \\
	\tinynb{5113}&&&&&&&&& \\
\end{tabular}
\begin{exe}
\ex \label{ex:tg:planter}  \vspace{-8pt}
\gll 	\ipa{ɣa¹}	\ipa{ŋwu²}	\ipa{tej¹tshew¹}	\ipa{.jij¹}	\ipa{ɣạ²}	\ipa{nər²}	\ipa{dzjị¹}	\ipa{khju¹}	\ipa{dzuu²}	\ipa{.wji¹} \\
		aiguille	\conj{}	Dai.Jiu	\antierg{}	dix	doigt	ongle	sous	planter	faire[A] \\
\glt Il planta des aiguilles sous les dix doigts de Dai Jiu (\zh{戴就}) (Leilin 04.23A.2) 
\end{exe}

\tib{ɴdzugs} peut aussi s'employer à propos d'une épée (\textit{gri ɴdzugs}). Il est donc indéniable que le sens du tangoute \tgf{2498} \textit{dzuu²} recouvre celui du \tib{ɴdzugs}. S'il s'agissait d'un cognat, une forme *ndzuk/ndzjuk devrait donner *dzew  ou *dzjiw en tangoute. Il est donc plus vraisemblable qu'il s'agisse d'un emprunt.
\newline

1930 \tgf{1930} \ipa{tju} 1.03 ``allumer (un feu)" peut être rapproché du dérivé nominal en \tib{dugs} ``moxibustion'' et au verbe attesté en tibétain occidental \textit{dugs} ``chauffer, allumer'' (forme citée par Jaeschke). Toutefois, étant donné la médiocre attestation de ce verbe en tibétain, cette comparaison est douteuse. 
\end{enumerate}

\subsection{Voyelles e et i} \label{subsec:voyelle.e.i}

Les rimes du \ipa{shè} n°2 sont reconstruites par Gong Hwangcherng avec les voyelles e et i ; il est parallèle au \ipa{shè} n°7, dont les rimes ont –ij ou –ej (voir \ref{subsec:voyelle.ej}). Tous les spécialistes s'accordent pour reconstruire des voyelles antérieures e, i ou \ipapl{ɪ}, sauf la rime 99 qui est reconstruite avec une rime arrondie par Arakawa et Nishida, et les rimes 68 à 70 que Sofronov reconstruit comme des diphtongues. 
\begin{table}
\captionb{Reconstructions du \ipa{shè} n°2}\label{tab:she2}
\resizebox{\columnwidth}{!}{
\begin{tabular}{lllllllll} \toprule
rime&	ton 1&	ton 2&	Sofronov1&	Sofronov2&	Nishida&	Li&	Gong&	Arakawa \\
8&	1.8&	2.7&	\ipa{e}&	\ipa{e}&	\ipa{ɪ ʷɪ}&	\ipa{e ʊe}&	\ipa{e}&	\ipa{i} \\
9&	1.9&	2.8&	\ipa{ê}&	\ipa{ê}&	\ipa{ɪě}&	\ipa{e̠}&	\ipa{ie}&	\ipa{yi} \\
10&	1.10&	2.9&	\ipa{i̯e}&	\ipa{i̯e}&	\ipa{i}&	\ipa{ie}&	\ipa{ji}&	\ipa{iː} \\
11&	1.11&	2.1&	\ipa{i}&	\ipa{i}&	\ipa{iɦ ʷiɦ}&	\ipa{i}&	\ipa{ji}&	\ipa{iː} \\
12&	1.12&	2.11&	\ipa{e+C}&	\ipa{e}&	\ipa{ʷɪɦ}&	\ipa{uɪ}&	\ipa{ee}&	\ipa{i'} \\
13&	1.13&	&	\ipa{ê+C}&	\ipa{ê}&	\ipa{ʷɪɦ²}&	\ipa{ue}&	\ipa{iee}&	\ipa{yi'} \\
14&	1.14&	2.12&	\ipa{i̯e+C}&	\ipa{i̯e}&	\ipa{ɪɦ}&	\ipa{ǐei}&	\ipa{jii}&	\ipa{iː'} \\
68&	1.65&	2.58&	\ipa{ại}&	\ipa{ẹi}&	\ipa{ɪ̣}&	\ipa{ẹ uẹ}&	\ipa{ẹ}&	\ipa{iq} \\
69&	1.66&	2.59&	\ipa{ại}&	\ipa{ẹi}&	\ipa{ɪ̣ě}&	\ipa{e̠̣}&	\ipa{iẹ}&	\ipa{yiq} \\
70&	1.67&	2.60&	\ipa{i̯ại}&	\ipa{i̯ẹi}&	\ipa{ị ʷị}&	\ipa{ǐẹi}&	\ipa{jị}&	\ipa{iːq} \\
82&	1.77&	2.71&	\ipa{ẹ}&	\ipa{ẹ}&	\ipa{ɪr}&	\ipa{ẹ}&	\ipa{er}&	\ipa{ir} \\
83&	1.78&	&	\ipa{}&	\ipa{}&	\ipa{iěr}&	\ipa{ǐẹ}&	\ipa{ier}&	\ipa{yir} \\
84&	1.79&	2.72&	\ipa{i̯ẹ+C}&	\ipa{i̯ẹ}&	\ipa{ir}&	\ipa{ǐe̠̣}&	\ipa{jir}&	\ipa{iːr} \\
99&	&	2.84&	\ipa{ẹ}&	\ipa{ẹ}&	\ipa{ʷɔr}&	\ipa{ǐẹ}&	\ipa{eer}&	\ipa{ywor} \\
101&	1.93&	2.86&	\ipa{}&	\ipa{Ị}&	\ipa{ǐə̣r}&	\ipa{ǐẹ̃}&	\ipa{jiir}&	\ipa{yer2} \\
\bottomrule
\end{tabular}}
\end{table}



Il est important de noter que les rimes 10 et 11 sont en distribution complémentaire par rapport à leurs consonnes initiales (\citealt{gong89reconstruction}, \citealt[92]{gong02a}) : la rime 11 apparaît avec les bilabiales, les dentales et les vélaires, et la rime 10 avec les autres initiales (Gong rend compte des quelques exceptions apparentes). Du point de vue comparatif, il est donc justifié, comme le fait Gong Hwangcherng, de ne reconstruire qu'une seule rime.
Dans la suite de cette section, nous allons traiter séparément les voyelles reconstruites respectivement *e et *i par Gong. Il convient toutefois de noter que l'opposition que Gong effectue entre *e et *i ne correspond pas à une catégorie à part entière dans les autres systèmes de reconstruction.

\begin{longtable} {lllllll}
\captionb{Comparaison  des étymons en --i du tangoute avec le japhug et le tibétain.}\label{tab:comparaisons:i} \\
\toprule
&\multicolumn{2}{c}{tangoute}& &  japhug & sens &tibétain  \\
\midrule
\endfirsthead
\tinynb{2798}&	\tgf{2798}&	\ipa{.jir}&	\tinynb{2.72}&	\ipa{ɣurʑa}&	cent&	brgʲa\\
\tinynb{5113}&	\tgf{5113}&	\ipa{.wji}&	\tinynb{1.10}&	\ipa{pa}&	faire&	bʲed bʲas\\
\tinynb{2712}&	\tgf{2712}&	\ipa{.wji}&	\tinynb{1.10}&	\ipa{tɯ-xpa}&	année&	\\
\tinynb{385}&	\tgf{0385}&	\ipa{.wjị}&	\tinynb{2.60}&	\ipa{spa}&	pouvoir&	\\
\tinynb{900	}&\tgf{0900} &	\ipa{.wji}	&	\tinynb{1.09}& \ipa{ɯ-pa}& bas & \\
\tinynb{4091}&	\tgf{4091}&	\ipa{.wjị}&	\tinynb{1.67}&	\ipa{tɤ-jpa}&	neige&	\\
\tinynb{5203}&	\tgf{5203}&	\ipa{.wjị}&	\tinynb{1.67}&	\ipa{tɯ-rpa}&	hache&	\\
\tinynb{1475}&	\tgf{1475}&	\ipa{bji}&	\tinynb{1.11}&	\ipa{mba}&	fin&	\\
\tinynb{716}&	\tgf{0716}&	\ipa{ɕjii}&	\tinynb{1.14}&	\ipa{ntɕha}&	tuer&	bɕa\\
\tinynb{169}&	\tgf{0169}&	\ipa{ɕjwi}&	\tinynb{1.1&}	\ipa{tɯ-ɕɣa}&	dent&	so\\
\tinynb{4517}&	\tgf{4517}&	\ipa{dzji}&	\tinynb{1.10}.&	\ipa{ndza}&	manger&	za\\
\tinynb{5436	}&\tgf{5436}	&\ipa{dʑjị}&	\tinynb{1.67}		&		\ipa{tɯ-ɲcɣa} & serpe !& \\
\tinynb{4906}&	\tgf{4906}&	\ipa{gjwi}&	\tinynb{2.10}&	\ipa{ŋga}&	s'habiller&	bgo\\
\tinynb{3869}&	\tgf{3869}&	\ipa{kjwị}&	\tinynb{1.67}&	\ipa{fka}&	rassasié&	\\
\tinynb{4807 }&	\tgf{4807}&	\ipa{khji}&	\tinynb{1.11}&	\ipa{kra}&	jeter&	\\
\tinynb{1388}&	\tgf{1388}&	\ipa{ljii}&	\tinynb{1.14}&	 &	pantalon&	\\
\tinynb{1036}&	\tgf{1036}&	\ipa{lhji}&	\tinynb{2.10}&	\ipa{ɣɤla}&	humide&	bʑa\\
\tinynb{2814}&	\tgf{2814}&	\ipa{lhjị}&	\tinynb{2.6&}	\ipa{sla}&	lune&	zla\\
\tinynb{1892}&	\tgf{1892}&	\ipa{mjii}&	\tinynb{1.11}&	\ipa{rɤrma}&	habiter&	\\
\tinynb{5700}&	\tgf{5700}&	\ipa{njii}&	\tinynb{2.12}&	\ipa{tɯ-ɕna}&	nez&	sna\\
\tinynb{306}&	\tgf{0306}&	\ipa{njir}&	\tinynb{2.72}&	&	emprunter&	rɲa\\
\tinynb{749}&	\tgf{0749}&	\ipa{phji}&	\tinynb{1.11}&	\ipa{ɣɤxpra}&	envoyer&	\\
\tinynb{1599}& \tgf{1599}& \ipa{rjir} & 	\tinynb{1.79}& &obtenir&	\\
\tinynb{5449}&	\tgf{5449}&	\ipa{tjị}&	\tinynb{1.67}&	\ipa{ta}&	mettre&	\\
\tinynb{1321}&	\tgf{1321}&	\ipa{zjị}&	\tinynb{1.67}&	\ipa{tɯ-xtsa}&	chaussure&	\\
\tinynb{2134}&	\tgf{2134}&	\ipa{zjwị}&	\tinynb{1.67}&	\ipa{tɯ-ftsa}&	neveu&	\\
\midrule
\tinynb{2625}&	\tgf{2625}&	\ipa{.wji}&	\tinynb{1.10}&	\ipa{tɯ-pi (gsar)}&	hôte&	\\
\tinynb{4519}&	\tgf{4519}&	\ipa{bji}&	\tinynb{2.10}&	\ipa{mbri}&	crier&	\\
\tinynb{5509}&	\tgf{5509}&	\ipa{bjị}&	\tinynb{1.67}&	\ipa{tɤ-rmbi}&	urine&	\\
\tinynb{4399}&	\tgf{4399}&	\ipa{dzjị}&	\tinynb{2.6&}	\ipa{tɤ-jtsi}&	colonne&	\\
\tinynb{1638}&	\tgf{1638}&	\ipa{gji}&	\tinynb{1.11}&	\ipa{mgri}&	claire (eau)&	\\
\tinynb{2059}&	\tgf{2059}&	\ipa{lhjị}&	\tinynb{2.6&}	\ipa{tɯ-ɣli}&	excrément&	ltɕi ba\\
\tinynb{3668}&	\tgf{3668}&	\ipa{ljị}&	\tinynb{1.67}&	\ipa{ji}&	planter&	\\
\tinynb{4518}&	\tgf{4518}&	\ipa{lji}&	\tinynb{2.09}&	\ipa{ɯ-di}&	odeur&	dri ma\\
\tinynb{2047}&	\tgf{2047}&	\ipa{mjii}&	\tinynb{1.14}&	\ipa{mbi}&	donner&	sbʲin, bʲin\\
\tinynb{2537}&	\tgf{2537}&	\ipa{rjir}&	\tinynb{2.72}&	\ipa{ri}&	rester&	\\
\tinynb{567}&	\tgf{0567}&	\ipa{rjir}&	\tinynb{2.72}&	\ipa{ɯ-ʁɤri}&	devant&	\\
\tinynb{3072}&	\tgf{3072}&	\ipa{sji}&	\tinynb{2.10}&	\ipa{si}&	mourir&	ɴtɕʰi, ɕi\\
\tinynb{3929}&	\tgf{3929}&	\ipa{tɕhjwi}&	\tinynb{1.10}&	\ipa{ftʂi}&	faire fondre&	\\
\tinynb{4658}&	\tgf{4658}&	\ipa{thji}&	\tinynb{1.11}&	\ipa{tshi}&	boire&	\\
\tinynb{1319}&	\tgf{1319}&	\ipa{tshji}&	\tinynb{1.11}&	\ipa{nɤntshi}&	aimer&	\\
\tinynb{2858}&	\tgf{2858}&	\ipa{zjir}&	\tinynb{2.72}&	\ipa{zri}&	long&	\\
\tinynb{251}&	\tgf{0251}&	\ipa{bji}&	\tinynb{2.10}&	\ipa{tɯ-mbri}&	corde&	ɴbreŋ \\
\tinynb{4250}&	\tgf{4250}&	\ipa{sji}&	\tinynb{1.11}&	\ipa{si}&	arbre&	shing\\
\tinynb{5273}&	\tgf{5273}&	\ipa{sji}&	\tinynb{2.10}&	\ipa{tɯ-mtshi}&	foie&	mtɕʰin-pa\\
\tinynb{4469}&	\tgf{4469}&	\ipa{ɕji}&	\tinynb{2.09}&	\ipa{ɕe}&	aller&	\\
\bottomrule
\end{longtable}




Comme le montre le tableau ci-dessus, la rime i du tangoute correspond massivement soit au japhug –\textit{i}, soit au japhug –\textit{a}.\footnote{Le nom \tgf{1388} \ipa{ljii¹} ``pantalon'' et le verbe \tgf{1599} \ipa{rjir¹} obtenir'' n'ont pas de cognats en japhug, mais on en trouve dans les langues lolo-birmanes et en na.} Parmi les correspondances au japhug –\textit{i}, la plupart correspondent au tibétain –\textit{i} également, mais on observe quelques exemples correspondant à une rime en nasale. Ces tendances resteraient valides même si les mots à –\textit{jii}, qui sont reconstruits avec une voyelle différente dans certains systèmes, sont enlevés.
A part ces deux groupes majeurs de correspondances, on peut distinguer un exemple correspondant au japhug –\textit{e}, un un autre au tibétain –\textit{ur}.


\begin{longtable} {lllllll}
\captionb{Comparaison  des étymons en --e  du tangoute avec le japhug et le tibétain.}\label{tab:comparaisons:e} \\
\toprule
&\multicolumn{2}{c}{tangoute}& &  japhug & sens &tibétain  \\
\midrule
\endfirsthead
\tinynb{5134}&	\tgf{5134}&	\ipa{.we}&	\tinynb{1.08}&	&	oiseau&bʲa	\\
\tinynb{4966}&	\tgf{4966}&	\ipa{.wẹ}&	\tinynb{1.65}&	\ipa{sɣa}&	rouille&	\\
\tinynb{3531}&	\tgf{3531}&	\ipa{dze}&	\tinynb{1.08}&	\ipa{--nɤndza}&	lèpre&	mdze.nad\\
\tinynb{2144}&	\tgf{2144}&	\ipa{gie}&	\tinynb{1.09}&	\ipa{ɴqa}&	difficile&	\\
\tinynb{439}&	\tgf{0439}&	\ipa{ɣiẹ}&	\tinynb{1.66}&	\ipa{sqa}&	cuire&	\\
\tinynb{3596}&	\tgf{3596}&	\ipa{ɣiwe}&	\tinynb{1.09}&	\ipa{βʁa}&	gagner&	\\
\tinynb{4046}&	\tgf{4046}&	\ipa{khie}&	\tinynb{1.09}&	\ipa{}&	amer&	kʰa-ba\\
\tinynb{4092}&	\tgf{4092}&	\ipa{khie}&	\tinynb{1.09}&	\ipa{qha}&	détester&	\\
\tinynb{1195}& \tgf{1195}& \ipa{khie}&\tinynb{2.08} &\ipa{qra} & yak femelle &\\
\tinynb{927}&	\tgf{0927}&	\ipa{le}&	\tinynb{2.07}&	\ipa{ala}&	bouillant&	\\
\tinynb{333}&	\tgf{0333}&	\ipa{ŋwer}&	\tinynb{2.71}&	\ipa{təmŋá (situ)}&	genou&	\\
\tinynb{395}&	\tgf{0395}&	\ipa{ŋwe}&	\tinynb{2.07}&	\ipa{nɯŋa}&	vache&	\\
\tinynb{1634}&	\tgf{1634}&	\ipa{rer}&	\tinynb{2.71}&	 &	filet&	dra.ba\\
\tinynb{2456}&	\tgf{2456}&	\ipa{se}&	\tinynb{1.08}&	\ipa{tasa}&	chanvre&	\\
\tinynb{2547}&	\tgf{2547}&	\ipa{tɕier}&	\tinynb{1.78}&	\ipa{χcha}&	droite&	\\
\tinynb{499}&	\tgf{0499}&	\ipa{piẹ}&	\tinynb{1.66}&	\ipa{qaɕpa}&	grenouille&	sbal\\
\midrule
\tinynb{2449}&	\tgf{2449}&	\ipa{be}&	\tinynb{2.07}&	\ipa{ʁmbɣi}&	soleil&	\\
\tinynb{2878}&	\tgf{2878}&	\ipa{biẹ}&	\tinynb{1.66}&	\ipa{ʑmbri}&	saule&	\\
\tinynb{5390}& \tgf{5390} & \ipa{phie}& \tinynb{2.08} & & détacher& \\
\tinynb{2664}&	\tgf{2664}&	\ipa{dze}&	\tinynb{1.08}&	\ipa{tɯ-tsi}&	vie&	tshe\\
\tinynb{5868}&	\tgf{5868}&	\ipa{khie}&	\tinynb{2.08}&	\ipa{smaikhrí (situ)}&	millet&	kʰre\\
\tinynb{4825}&	\tgf{4825}&	\ipa{me}&	\tinynb{2.07}&	&	dormir&	rmi.lam\\
\tinynb{0809}&	\tgf{0809}&	\ipa{rer}&	\tinynb{1.77}&	\ipa{tɤri} &	corde, fil&	 \\
\tinynb{5957}&	\tgf{5957}&	\ipa{tser}&	\tinynb{1.77}&	\ipa{ntsɣe}&	vendre&	\\

\tinynb{800}&	\tgf{0800}&	\ipa{dzeej}&	\tinynb{1.37}&	\ipa{}&	combattre&	ɴdziŋ\\

\bottomrule
\end{longtable}


Avec les mots à voyelle \textit{e}, on retrouve les mêmes grandes correspondances : des exemples correspondants à --\textit{a} en japhug et en tibétain, d'autres correspondant à --\textit{i} en japhug.\footnote{Le verbe \tgf{5390} \ipa{phie²} n'a pas de cognat en japhug, mais correspond au \bir{phre²}.} Toutefois, on observe également un nombre plus large de formes à correspondance unique, dont certaines impliquent des rimes à occlusives finales.

La correspondances des rimes du \textit{shè} n°2 à des voyelles ouvertes dans d'autres langues a déjà été remarquée par  \citet{matisoff04brightening}. Dans cette section, nous allons tenter d'en rendre compte d'une façon systématique.
%ne pas oublier *jij-w  XXXXXXX
\subsubsection{Tangoute --i :: tibétain --a :: japhug --a}	\label{subsubsec:correspondance:i:a:a}

De toutes les correspondances attestées pour les rimes du \ipa{shè} n°2, celle du tangoute --\textit{i} à --\textit{a} en tibétain et en japhug est attestée par le plus grand nombre d'exemples. On reconstruira *--ja en pré-tangoute pour cette rime (*--a simple donnant --\textit{e}).
\newline
\linebreak
A. Initiales labiales \label{rimes:02:1:lab}
\newline
\begin{enumerate}


\item 749 \tgf{0749} \ipa{phji} 1.11 \ptang{phja}{envoyer} est un des verbes tangoutes les plus courants, qui sert à former le causatif synthétique. Ce verbe présente une alternance vocalique, son thème B étant 4568 \tgf{4568} \ipa{phjo} 2.44 (*phja-w) avec une alternance tonale irrégulière. Une comparaison possible serait le \jpg{ɣɤxpra} qui signifie ``ordonner". Cette correspondance pourrait toutefois être fortuite.



\item 1475 \tgf{1475} \ipa{bji} 1.11 \ptang{mbja}{fin, mince} correspond parfaitement au  \jpg{mba}, au \plb{ba²}{0533} et au langues naish  \naxi{mbe˧}, \nayn{bi˥}, \laze{tʰɑ˧ bie˥}, \protona{mba}. En japhug, \ipa{mba} ``mince'' s'emploie à propos d'objets plats (feuille, planche etc; il correspond au sens du chinois \zh{薄} \textit{báo}) et s'oppose à \ipa{xtshɯm}, qui s'emploie pour des objets longs (comme le chinois \zh{細} \textit{xì}). Le\jpg{xtshɯm} a également un cognat tangoute 1861 \tgf{1861} \ipa{tshjɨj} 1.42 ``fin''. Les exemples textuels de \tgf{1861} \ipa{tshjɨj¹} cités dans Li (1997) montrent que cet adjectif peut s'employer à propos du cou ou des sourcils, ce qui correspondrait à des objets allongés. Cela suggère que l'opposition sémantique entre \tgf{1861} \ipa{tshjɨj¹} et \tgf{1475} \ipa{bji¹} en tangoute est proche de celle de \ipa{xtshɯm} et \ipa{mba} en japhug. \label{analyse:fin}


\item 2712 \tgf{2712} \ipa{.wji} 1.10 \ptang{C-pja}{année} peut se comparer au  \jpg{tɯ-xpa} ``année". En tangoute, il existe deux noms pour ``année'', l'autre, plus courant dans les textes,  est 3305  \tgf{3305} \ipa{kjiw} 1.45. \tgf{2712} \ipa{.wji¹} apparaît surtout dans les expressions ``l'année dernière, cette année, l'année prochaine'' (attestées dans le ZZZ) tandis que  \tgf{3305} \ipa{kjiw¹} se retrouve dans la plupart des autres contextes, en particulier en association avec un numéral, où il fonctionne comme un classificateur. Les propriétés morphologiques de ces deux racines seront examinées en \ref{subsec:num}.

Il est possible que le   \jpg{tɯ-xpa} soit apparenté au nom \jpg{taχpa} ``récolte''.  Si c'est le cas, on peut déterminer qu'il s'agit d'une innovation lexicale commune au tangoute et aux langues rgyalrongs (ainsi qu'aux langues nas, comme nous le verrons en \ref{subsec:num}).


\item 5113 \tgf{5113} \ipa{.wji} 1.10 \ptang{C-pja}{faire} correspond au  \jpg{pa} ``fermer (originellement ``faire'')'', au \tib{bʲed, bʲas} (racine \racine{bʲa}) ``faire" et au naish  \naxi{be˧},	\nayn{i˥},	\laze{vie˧},	\protona{Cba}. 3621 \tgf{3621} \ipa{.wjo} 1.51 (PT \ipapl{*pja-w}) est le thème B de ce verbe. Il est aussi employé dans le sens de ``servir comme":
\newline
\linebreak
\begin{tabular}{lllllll}
		\tgf{1319}&	\tgf{1643}&	\tgf{3738}&	\tgf{5447}&	\tgf{5212}&	\tgf{0760}&	\tgf{5113} \\
		\tinynb{1319}&	\tinynb{1643}&	\tinynb{3738}&	\tinynb{5447}&	\tinynb{5212}&	\tinynb{0760}&	\tinynb{5113} \\
\end{tabular}
\begin{exe}
\ex \label{ex:tg:faire1}  \vspace{-8pt}
\gll   \ipa{tshji¹}	\ipa{xwã¹}	\ipa{kow¹}	\ipa{do²}	\ipa{phjị¹dzjɨj²}	\ipa{.wji¹} \\
		Qi	Huan	Gong	\loc{}	premier.ministre	faire[A] \\
\glt Il servait comme premier ministre de Qi Huan Gong (\zh{齊桓公}) (Leilin, 04.01A.7)
\end{exe}  
Au perfectif avec le préfixe \tgf{0795} \ipa{rjɨr²} il peut aussi avoir le sens de ``devenir'', comme le passif japhug \ipa{apa}. Le verbe \tgf{2226} \ipa{.we²} qui signifie également ``faire" est en revanche sans relation avec \tgf{5113}\ipa{.wji¹}, c'est un emprunt au chinois \zh{為} \textit{wéi}.

Il est possible que 1498 \tgf{1498} \ipa{.wjị} 2.60 \ptang{C-S-pja}{faire semblant}, voir l'exemple p.\pageref{ex:tg:se.coucher}) soit originellement un dérivé de la racine ``faire''. On retrouve un sens proche avec le \jpg{ʑɣɤpa} ``faire semblant''  une forme réfléchie du verbe ``faire''. La nature exacte de ce préfixe *S-- en pré-tangoute est toutefois encore difficile à déterminer: il ne peut pas s'agir d'un cognat du japhug \ipapl{ʑɣɤ--}, qui est une innovation propre aux langues rgyalrong (voir \citealt{jacques10refl}).

Le verbe suivant  \tgz{0385} est aussi un dérivé de la même racine.


\item 385  \tgf{0385} \ipa{.wjị} 2.60 \ptang{C-S-pja}{pouvoir} est comparable au  \jpg{spa} ``savoir, pouvoir". Son thème B est 832  \tgf{0832} \ipa{wjọ} 2.64 (*C-S-pja-w). C'est le seul exemple d'un verbe pourvu du préfixe abilitatif *S- en  tangoute (voir section \ref{subsec:abilit}).

C'est un verbe modal, dont la négation est \tgf{5643} \ipa{mjɨ¹} (voir par exemple Leilin 04.24A.4). En japhug, le sens de ce verbe est ``savoir faire, connaître'' peut s'employer avec un nom objet ou un verbe complément (Jacques 2008 :343-344). Un usage similaire existe en tangoute:
\newline
\linebreak
\begin{tabular}{llllll}
		\tgf{2546}&	\tgf{1139}&	\tgf{1451}&	\tgf{0467}&	\tgf{0385}&	\tgf{0433} \\
		\tinynb{2546}&	\tinynb{1139}&	\tinynb{1451}&	\tinynb{0467}&	\tinynb{0385}&	\tinynb{0433} \\
\end{tabular}
\begin{exe}
\ex \label{ex:tg:pouvoir}  \vspace{-8pt}
\gll   \ipa{njạ¹}	\ipa{.jij¹}	\ipa{dzji¹tsjiir¹}	\ipa{.wjị²}	\ipa{bju¹} \\
		divin	\antierg{}	magie	savoir[A]	\instr{} \\
\glt Comme il connaissait la magie divine (Leilin 05.23B.6)
\end{exe}
Toutefois, avec un verbe complément,  \tgf{0385} \ipa{.wjị²} peut simplement avoir le sens de ``pouvoir'' sans que l'idée d'un apprentissage nécessaire soit impliqué.



\item 900	\tgf{0900} \ipa{.wji}	1.09 \ptang{C-pja}{fond} est cognat du  \jpg{ɯ-pa} ``bas''. En japhug, ce nom a été grammaticalisé pour former le préfixe directionnel ``vers le bas'' (\ipapl{pɤ-, pjɯ-}).
%attestation Leilin 8.8A.7 XXX
\newline
\linebreak
\begin{tabular}{llllllllll}
	\tgf{5916}&	\tgf{0493}&	\tgf{1139}&	\tgf{3990}&	\tgf{0900}&	\tgf{5399}&	\tgf{2590}&	\tgf{0013}&	\tgf{4444}&	\tgf{1906}\\
		\tinynb{5916}&	\tinynb{0493}&	\tinynb{1139}&	\tinynb{3990}&	\tinynb{0900}&	\tinynb{5399}&	\tinynb{2590}&	\tinynb{0013}&	\tinynb{4444}&	\tinynb{1906}\\
	\tgf{5815}&	\tgf{5414}&	\tgf{2514}&	\tgf{5921}&	\tgf{2590}&	\tgf{3159}&	\tgf{0749}&&&\\
	\tinynb{5815}&	\tinynb{5414}&	\tinynb{2514}&	\tinynb{5921}&	\tinynb{2590}&	\tinynb{3159}&	\tinynb{0749}&&&\\
\end{tabular}
\begin{exe}
\ex \label{ex:tg:fond}  \vspace{-8pt}
\gll   \ipa{xã¹sjɨ²}	\ipa{.jij¹}	\ipa{khjɨ¹}	\ipa{.wji¹}	\ipa{khju¹}	\ipa{.wjɨ²-zar²}	\ipa{ljɨ̣¹}	\ipa{nioow¹}	\ipa{tsjɨ¹}	\ipa{rejr²}	\ipa{ju¹zar²}	\ipa{.wjɨ²-lhjịj²-phji¹} \\
	Han.Xin \antierg{} pied fond sous \dir{}-passer \conj{} après aussi beaucoup honte \dir{}-recevoir-causer[A] \\
\glt Ils firent passer Han Xin sous leurs pieds, et lui firent subir de nombreuses humiliations. (Leilin 8.14A.5-6)
\end{exe}


\item 4091  \tgf{4091} \ipa{.wjị} 1.67 \ptang{C-S-pja}{neige} peut se comparer au  \jpg{tɤ-jpa} de même sens. Ce mot se retrouve dans les langues lolo-birmanes, macro-rgyalronguiques et naish \naxi{mbe˧},	\nayn{bi˥},	\laze{vie˧},	\protona{Smba}.


\item 5203  \tgf{5203} \ipa{.wjị} 1.67 \ptang{C-S-pja}{hache} est apparenté au  \jpg{tɯ-rpa} et au Na  \naxi{lɑ˩mbe˧},  \nayn{bi.mi^L}, \protona{(S)mba} On retrouve même un cognat en chinois  \zh{斧} \ipapl{*paʔ}. 


\item 1892  \tgf{1892} \ipa{mjii} 1.11 \ptang{mjaa}{maison} peut potentiellement être rapproché du  \jpg{rma} ``habiter chez quelqu'un". Il s'agirait dans cette hypothèse d'un nom déverbal. Toutefois, on attendrait a priori une forme *mjir si le pré-tangoute était *rmja ; la correspondance est imparfaite, et la comparaison est peut-être incorrecte.

\end{enumerate}
B. Initiales dentales \label{rimes:02:1:dent}

\begin{enumerate}



\item 5449  \tgf{5449} \ipa{tjị} 1.67 \ptang{S-tja}{mettre} est apparenté au  \jpg{ta} ``mettre" et au \plb{ʔ-ta²}{0762}. La préinitiale *S- du tangoute n'a pas d'équivalent en japhug, et il s'agit selon toute vraisemblance d'un préfixe réinterprété comme partie de la racine verbale. Ce verbe a pour thème 2 la forme 5633  \tgf{5633} \ipa{tjọ} 1.72. La racine ``poser" a un emploi spécial dans le sens de ``faire fermenter" pour lequel les caractères spéciaux suivants ont été créés en rajoutant un élément supérieur : \tgf{4907} 4907 et \tgf{4941} 4941. En voici un exemple textuel: 
\newline
\linebreak
\begin{tabular}{llll}
		\tgf{0261}&	\tgf{1326}&	\tgf{4941}&	\tgf{2098}\\
				\tinynb{0261}&	\tinynb{1326}&	\tinynb{4941}&	\tinynb{2098}\\
\end{tabular}
\begin{exe}
\ex \label{ex:tg:fermenter}  \vspace{-8pt}
\gll   \ipa{mjo²}	\ipa{kjɨ¹-tjọ¹-ŋa²} \\
		moi	\dir{}-fermenter[B]-1\sg{} \\
\glt C’est moi qui ai préparé (l’alcool) (Cixiaozhuan 5.1, Jacques 2007 :18)
\end{exe}
Le nom ``endroit'' 5645  \tgf{5645} \ipa{tjị} 2.60 est dérivé de \tgf{5449} \ipa{tjị¹} par alternance tonale. On retrouve une dérivation similaire en japhug, où le mot pour ``endroit'' \ipa{ɯ-sta} est dérivé du verbe ``poser'' \ipa{ta} par l'addition d'un préfixe de nominalisation oblique \ipa{sɤ-}, lequel s'est fossilisé, d'où la perte irrégulière de la voyelle. La proto-forme devait avoir un préfixe *S-, qui peut soit correspondre au préfixe intégré à la racine du verbe, soit au préfixe dérivationnel du japhug.


\item 4517  \tgf{4517} \ipa{dzji} 1.10 \ptang{ndzja}{manger} est comparable au  \jpg{ndza} `` manger", au \plb{dza²}{0629} et au \tib{za} de même sens. Ce verbe a une forme alternante 4547   \tgf{4547} \ipa{dzjo} 1.51 (*ndzja-w) et présente également un dérivé nominal 4513   \tgf{4513} \ipa{dzji} 2.10 ``nourriture" par alternance tonale, exactement comme le \plb{dza¹}{0274} ``nourriture''. Outre le sens de ``manger", ce verbe a des sens dérivés, dont ``recevoir un coup":
\newline
\linebreak
\begin{tabular}{llllllll}
		\tgf{1542}&	\tgf{3508}&	\tgf{0100}&	\tgf{2798}&	\tgf{2987}&	\tgf{5481}&	\tgf{4547}&	\tgf{2098}\\
	\tinynb{1542}&	\tinynb{3508}&	\tinynb{0100}&	\tinynb{2798}&	\tinynb{2987}&	\tinynb{5481}&	\tinynb{4547}&	\tinynb{2098}\\
\end{tabular}
\begin{exe}
\ex \label{ex:tg:manger}  \vspace{-8pt}
\gll   \ipa{ku¹}	\ipa{bji²}	\ipa{lew¹}	\ipa{.jir²}	\ipa{lhjɨ̣¹}	\ipa{bo²}	\ipa{dzjo¹-ŋa²} \\
		alors	sujet	un	cent	coup	bâton	manger[B]-1\sg{} \\
\glt Alors moi, votre sujet, je recevrai cent coups de bâton (Leilin 06.13A.5)
\end{exe}
Le sens de ``manger'' est souvent exprimé par 4658  \tgf{4658} \ipa{thji} 1.11 ``boire'', notamment avec 4508   \tgf{4508} \ipa{tjị¹} ``repas'' comme objet ; Dans les chapitres 3 à 6 de Leilin,   \tgf{4517} \ipa{dzji¹} apparaît huit fois, tandis que   \tgf{4658} \ipa{thji¹} est attesté onze fois dans le sens de manger (et quinze dans le sens de ``boire'').

En naish, on retrouve le dérivé \naxi{dze˧},	\nayn{dze.lɯ^M},	\laze{dze˥},	\protona{dza}
``nourriture'', tandis que le verbe ``manger'' doit être reconstruit \protona{ndzi}; il s'agit d'une trace d'alternance vocalique en naish, peut-être liée au suffixe *-jə du proto-rgyalrong et au suffixe \tgz{1101}.

\item 5700   \tgf{5700} \ipa{njii} 2.12 \ptang{njaa}{nez} est apparenté au japhug \ipa{tɯ-ɕna}, le \plb{s-na¹}{0093} le \tib{sna}. Les correspondances ne sont pas idéales, on attendrait PTg *S-nja > \ipapl{*njị}. La forme pré-tangoute n'avait plus de présyllabe, pour une raison inconnue.


\item 306   \tgf{0306} \ipa{njir} 2.72 ``prêter" est l'exemple principal du changement *ŋ > nj /\underline{  ~~  }  [+antérieure,-arrondi] postulé p.\pageref{tab:ngpalatalisation}. Il se compare avec le \tib{rnʲa} ``emprunter, prêter" (*<rŋja), le \plb{s-ŋa²}{0600} ``emprunter" et le naish \naxi{ŋi˧},	\nayn{ŋi˥},	\laze{ŋi˧},	\protona{ŋi/a}. On peut proposer ici une reconstruction \ptang{rŋja}{emprunter} au lieu de *rnja. En tangoute, ce verbe signifie ``prêter'', pas ``emprunter'' :
\newline
\linebreak
\begin{tabular}{lllllll}
		\tgf{5306}&	\tgf{1918}&	\tgf{3575}&	\tgf{0824}&	\tgf{5417}&	\tgf{0020}&	\tgf{0306}\\
		\tinynb{5306}&	\tinynb{1918}&	\tinynb{3575}&	\tinynb{0824}&	\tinynb{5417}&	\tinynb{0020}&	\tinynb{0306}\\
\end{tabular}
\begin{exe}
\ex \label{ex:tg:preter}  \vspace{-8pt}
\gll   \ipa{dzjwɨ¹}	\ipa{mji¹-nji²}	\ipa{tɕhjɨ²rjar²}	\ipa{tɕja¹}	\ipa{njir²} \\
		seigneur	\negat{}-écouter	immédiatement	route	prêter \\
\glt Le seigneur ne l'écouta pas, et laissa passer (l'armée de Jin) (Leilin, 03.16A.6)
\end{exe}
Pour exprimer le sens ``emprunter", il convient d'ajouter le verbe ``demander'' \tgf{0147} \ipa{ɕjij²}:
\newline
\linebreak
\begin{tabular}{lllllllll}
		\tgf{3926}&	\tgf{2104}&	\tgf{3535}&	\tgf{5981}&	\tgf{5878}&	\tgf{0804}&	\tgf{0306}&	\tgf{0147}&	\tgf{4601}\\
		\tinynb{3926}&	\tinynb{2104}&	\tinynb{3535}&	\tinynb{5981}&	\tinynb{5878}&	\tinynb{0804}&	\tinynb{0306}&	\tinynb{0147}&	\tinynb{4601}\\
\end{tabular}
\begin{exe}
\ex \label{ex:tg:emprunter}  \vspace{-8pt}
\gll   \ipa{nja²}	\ipa{ɕji¹}	\ipa{ŋwu²}	\ipa{.a-bie.j¹}	\ipa{djɨ²-njir²-ɕjij²-nja²} \\
		toi	autrefois	pinceau	un-\classif{}	\dir{}-prêter-demander-2\sg{} \\
\glt Il y a longtemps, tu as emprunté un pinceau (Leilin, 06.18A.2)
\end{exe}




\item 1321	\tgf{1321}	\ipa{zjị}	1.67 \ptang{C-S-tsja}{chaussure} est cognat du japhug \ipa{tɯ-xtsa} ``chaussure''.  Un verbe dérivé au ton shang 4956	 \tgf{4956}	\ipa{zjị} 2.60 ``porter une chaussure'' est également attesté.
	

\item 2134   \tgf{2134} \ipa{zjwị} 1.67  \ptang{C-S-ptsja}{neveu} est apparenté au  \jpg{tɯ-ftsa} ``neveu" et plus lointainement, au naish \naxi{dze˧ɯ˧}	\nayn{ze.v̩^L}	\laze{ze˧}	\protona{Cdza} et au \tib{tsʰa.bo} ``neveu, petit-fils". Comme les autres termes de parenté, ce mot est étudié dans \citet{jacques11kinship} et la discussion ne sera pas répétée ici.


\item 2814   \tgf{2814} \ipa{lhjị} 2.60 \ptang{S-lhja}{lune} peut se comparer au  \jpg{sla} ``lune'', au \plb{bəla³}{0318}, au naish \naxi{le˩},	\nayn{ɬi.mi^M},	\laze{ɬie˧mie˧},	\protona{Sla} et au \tib{zla} (aussi écrit sla en tibétain ancien). Comme ses équivalents tibétains et japhug, il peut également désigner le mois.


\item 1036   \tgf{1036} \ipa{lhji} 2.10 \ptang{lhja}{humide} est comparable au  \jpg{ɣɤla} ``humide" et au \tib{bʑa} (<*p-lja) et \tib{rlan} ``humide''. Voici un exemple d'emploi de ce mot, où il semble servir plutôt comme nom que comme adjectif:
\newline
\linebreak
\begin{tabular}{llllllllll}
		\tgf{0322}&	\tgf{2627}&	\tgf{1890}&	\tgf{0089}&	\tgf{0394}&	\tgf{3031}&	\tgf{1036}&	\tgf{2194}&	\tgf{1542}&	\tgf{5479}\\
	\tinynb{0322}&	\tinynb{2627}&	\tinynb{1890}&	\tinynb{0089}&	\tinynb{0394}&	\tinynb{3031}&	\tinynb{1036}&	\tinynb{2194}&	\tinynb{1542}&	\tinynb{5479}\\
		\tgf{2857}&	\tgf{1918}&	\tgf{5065}&	\tgf{1326}&	\tgf{2833}&	\tgf{4916}&	\tgf{0206}&&&\\
		\tinynb{2857}&	\tinynb{1918}&	\tinynb{5065}&	\tinynb{1326}&	\tinynb{2833}&	\tinynb{4916}&	\tinynb{0206}&&&\\
\end{tabular}
\begin{exe}
\ex \label{ex:tg:humide}  \vspace{-8pt}
\gll   \ipa{tɕhjwo¹}	\ipa{ljɨ̣²}	\ipa{bjij²}	\ipa{tɕhjaa¹}	\ipa{lhjị²}	\ipa{tsji²}	\ipa{lhji²}	\ipa{mjij¹}	\ipa{ku¹}	\ipa{rjar¹ŋo²}	\ipa{mji¹-ljị¹}	\ipa{kjɨ¹djɨj²}	\ipa{ɣwej¹}	\ipa{buu²} \\
		alors	endroit	haut	 sur	s'installer	humide	humide	ne.pas.avoir donc	maladie	\negat{}-tomber.dans	certainement	guerre	vaincre \\
\glt Donc si l'on s'installe sur un endroit élevé, où il n'y a pas d'humidité, on ne subira pas de maladie et on remportera la victoire à la guerre. (Sunzi, 21A-7b-6, \citealt[240]{lin94sunzi})
\end{exe}




\item 1388 \tgf{1388} \ipa{ljii} 1.14 \ptang{ljaa}{pantalon} n'a pas de cognats en japhug ou en tibétain, mais on retrouve une forme apparentée en lolo-birman \plb{ʔ-/k-la²}{0228} et en naish \naxi{le˧},	\nayn{ɬi.qʰwɤ^L},	\laze{ɬie˥kʰwɤ˥},	\protona{Sla}. Voir l'exemple p.\pageref{ex:tg:accumuler}.

Le verbe 4991 \tgf{4991} \ipa{ljii}  2.12 ``porter un pantalon'' en est dérivé par alternance tonale (voir \citealt{gong88alternations}, \citealt[56]{gong02a}). Ce verbe a la forme alternante 5381 \tgf{5381} \ipa{ljoo} 2.46, mais aucune des deux formes verbales n'est attestée dans les textes.
\end{enumerate}

C. Initiales palatales et r-\label{rimes:02:1:pal}
\begin{enumerate}


\item 5436	\tgf{5436} \ipa{dʑjị} 1.67 \ptang{S-ndʑa}{serpe}	est comparable au japhug \ipa{tɯ-ɲcɣa} ``serpe''. Ce rapprochement pose deux difficultés: la correspondance de l'initiale et l'absence de médiane *--w-- en tangoute.

\item 716   \tgf{0716} \ipa{ɕjii} 1.14 \ptang{ɕaa}{tuer (animal)} peut être rapproché du japhug \ipa{ntɕha} ``tuer un animal, découper'' et du \tib{bɕa} de même sens. Il convient toutefois de noter que le japhug est peut-être emprunté à une forme de présent de \textit{bɕa} dans une forme non-standard de tibétain *ɴtɕʰa.\footnote{Communication personnelle de Nathan Hill, 2007.}  D'autre part, à l'intérieur du tibétain, ce verbe est clairement dérivé du nom \tib{ɕa} ``viande".\footnote{On applique au verbe le préfixe b-- du passé comme dans \tib{btɕu} ``puiser de l’eau" et \tib{bʑo} ``traire".}  Si la forme tangoute est effectivement comparable à celle du tibétain, c'est donc aussi un verbe dérivé d'un nom signifiant à l'origine ``viande".  Le nom 3465 \tgf{3465} \ipa{tɕhji} 1.10 ``viande'' est très probablement la forme d'où le verbe  \tgf{0716} \ipa{ɕjii¹}  est dérivé, et nous proposons de reconstruire une forme préfixée *\ipapl{t-ɕa} en pré-tangoute. Elle n'est pas directement comparable au japhug \ipa{tɯ-ɕa}, car cette forme japhug est selon toute vraisemblance un emprunt au tibétain; toutefois, la préfixation en \ipa{tɯ-} montre que la proto-forme postulée *\ipapl{t-ɕa} est possible. \label{analyse:viande}
\newline
On a proposé p.\pageref{rimes:01:3:ju/a} une correspondance du \tib{ɕa} avec 2385   \tgf{2385} \ipa{ɕju} 2.02 ``viande". Cette hypothèse est clairement  incompatible avec l'étymologie présentée ici, mais les données manquent encore pour pouvoir trancher.


  \tgf{0716} \ipa{ɕjii¹} a normalement pour objet un animal (sauf dans un exemple de Leilin 04.21A.5):
\newline
\linebreak
\begin{tabular}{llllllllll}
			\tgf{4978}&	\tgf{1531}&	\tgf{1183}&	\tgf{0930}&	\tgf{4978}&	\tgf{1909}&	\tgf{0716}&	\tgf{5880}&	\tgf{3513}&	\tgf{0078}\\
	\tinynb{4978}&	\tinynb{1531}&	\tinynb{1183}&	\tinynb{0930}&	\tinynb{4978}&	\tinynb{1909}&	\tinynb{0716}&	\tinynb{5880}&	\tinynb{3513}&	\tinynb{0078}\\
\end{tabular}
\begin{exe}
\ex \label{ex:tg:tuer.animal}  \vspace{-8pt}
\gll   \ipa{tjij¹}	\ipa{gja¹}	\ipa{dạ²}	\ipa{dju¹}	\ipa{tjij¹}	\ipa{gur¹}	\ipa{ɕjii¹}	\ipa{ŋwu²}	\ipa{mə¹}	\ipa{tjị¹} \\
		si	armée	affaire	y.avoir	si	bœuf	tuer[A] \conj{}	ciel	sacrifier \\
\glt En cas de guerre, ils tuent un\footnote{Le caractère \tgf{4978} est probablement une erreur pour 4981   \tgf{4981} \ipa{tɕhioow} 1.57 ``un certain".} bœuf comme sacrifice au ciel (Leilin 04.31B.3-4)
\end{exe}
Le verbe  \tgf{0716} \ipa{ɕjii¹} présente un thème B 4571  \tgf{4571} \ipa{ɕjoo} 1.53 (*ɕaa-w).


\item 169   \tgf{0169} \ipa{ɕjwi} 1.10 \ptang{ɕwa}{dent} correspond au \plb{swa²}{0096}, au naish \naxi{hɯ˧},	\nayn{hi˥},	\laze{i˧tʰu˧},	\protona{Swa}, au \tib{so} (<*swa, d'après la loi de Laufer, cf \citet{jacques09wazur} et \citet{hill11laws}) et au  \jpg{tɯ-ɕɣa} (\ipapl{<*ɕwa}). Ce nom se retrouve dans la plupart des langues ST.


\item 2798  \tgf{2798} \ipa{.jir} 2.72 \ptang{r-ja}{cent} se rapproche du  \jpg{ɣurʑa} (\ipapl{<*wə-rj}a), du \plb{C-ra¹}{0488} et du \tib{brgʲa} (<*p-rja, par la deuxième loi de Li Fang-kuei selon la terminologie de \citet{hill11laws}; voir aussi le numéral ``huit'' p.\pageref{analyse:huit}). Noter que pré-tangoute *rja donnerait rjir. \label{analyse:cent}



\item 1599 \tgf{1599} \ipa{rjir} 1.79 ainsi que 1700  \tgf{1700} \ipa{rjir} 2.72 \ptang{rja}{obtenir} peuvent se comparer au \bir{ra¹} ``obtenir''\footnote{Concernant le verbe ``devoir'' qui est peut-être apparenté, voir la discussion en  \ref{subsubsec:correspondance:a:a:a}.} La différence entre les deux formes \tgf{1599} \ipa{rjir¹} et  \tgf{1700} \ipa{rjir²} n'est pas claire dans les textes, mais la première est de loin la plus courante. Leurs formes alternantes qui s'emploient avec un agent 1sg ou 2sg et un patient à la troisième personne sont 23 \tgf{0023} \ipa{rjor} 1.09 et 98 \tgf{0098} \ipa{rjor}   2.81 respectivement (*rja-u) :
\newline
\linebreak
\begin{tabular}{llllllllll}
	\tgf{0726}&	\tgf{5399}&	\tgf{4456}&	\tgf{4871}&	\tgf{1736}&	\tgf{1465}&	\tgf{4921}&	\tgf{0403}&	\tgf{1139}&	\tgf{3354}\\
\tinynb{0726}&	\tinynb{5399}&	\tinynb{4456}&	\tinynb{4871}&	\tinynb{1736}&	\tinynb{1465}&	\tinynb{4921}&	\tinynb{0403}&	\tinynb{1139}&	\tinynb{3354}\\
\tgf{0433}&	\tgf{0140}&	\tgf{0098}&	\tgf{2098}& &&&&&\\
\tinynb{0433}&	\tinynb{0140}&	\tinynb{0098}&	\tinynb{2098}& &&&&&\\
\end{tabular}
\begin{exe}
\ex \label{ex:tg:obtenir}  \vspace{-8pt}
\gll   \ipa{djị¹}	\ipa{khju¹}	\ipa{ljịj²}	\ipa{ŋər¹}	\ipa{tɕju¹-ljịj¹}	\ipa{swẽ¹.wow¹}	\ipa{.jij¹}	\ipa{ɣie¹}	\ipa{bju¹}	\ipa{rejr²}	\ipa{rjor²-ŋa²} \\
	enfer sous grande montagne affaire-servir Sun.E \antierg{} force \instr{} bonheur obtenir[B]-1\sg{} \\
\glt En enfer, grâce aux efforts de Sun E (\zh{孫阿}), l'intendant (\zh{錄事}) de Taishan, j'ai obtenu la paix. (Leilin 06.23A.4-5)
\end{exe}


\end{enumerate}

D Initiales vélaires \label{rimes:02:1:vel}

\begin{enumerate}


\item 3869   \tgf{3869} \ipa{kjwị} 1.67 ``être rassasié" (*S-p-kja) est apparenté au  \jpg{fka} de même sens.
\newline
\linebreak
\begin{tabular}{llllllll}
		\tgf{4024}&	\tgf{4027}&	\tgf{4342}&	\tgf{3547}&	\tgf{4342}&	\tgf{3869}&	\tgf{1906}&	\tgf{3349}\\
	\tinynb{4024}&	\tinynb{4027}&	\tinynb{4342}&	\tinynb{3547}&	\tinynb{4342}&	\tinynb{3869}&	\tinynb{1906}&	\tinynb{3349}\\
\end{tabular}
\begin{exe}
\ex \label{ex:tg:rassasie} 
\gll   \ipa{zjɨ̣²}	\ipa{njɨɨ¹}	\ipa{dja²-lia²}	\ipa{dja²-kjwị¹}	\ipa{nioow¹}	\ipa{rjijr²}\\
		deux	deux	\dir{}-saoul	\dir{}-rassasié	après	\loc{} \\
\glt Après qu'ils furent saouls et rassasiés tous les deux, (Leilin 05.23A.7)
\end{exe}
896  \tgf{0896} \ipa{gjwij} 1.36 ``rassasié" est peut-être aussi apparenté, bien que sa relation avec \ipa{kjwị¹} soit peu claire.



\item 4807 \tgf{4807} \ipa{khji} 1.11 \ptang{khja}{jeter} peut se comparer au  \jpg{kra} ``faire tomber''. Dans les deux langues, on trouve également un verbe apparenté par dérivation anticausative (voir p.\pageref{subsec:anticausatif}) : 4930 \tgf{4930} \ipa{gji} 1.11 ``tomber'' et \ipa{ŋgra} de même sens. \label{analyse:faire.tomber}




\item 4906   \tgf{4906} \ipa{gjwi} 2.10  \ptang{ŋgwja}{s'habiller} correspond au  \jpg{ŋga}, au \plb{wát}{0681} et au \tib{bgo} (<*p-gwa) de même sens. Ce verbe a pour forme alternante 3686   \tgf{3686} \ipa{gjwo} 2.44 (*gwja-w) ainsi que le causatif 3146  \tgf{3146} \ipa{gjwị} 1.67 \ptang{S-ŋgwja}{habiller}, forme alternante 539   \tgf{0539} \ipa{gjwọ} 1.72 < *S-ŋgwja-w. Malgré l'existence de cette forme causative, il est possible d'utiliser aussi l'auxiliaire causatif:
\newline
\linebreak
\begin{tabular}{llllllllll}
	\tgf{2019}&	\tgf{1085}&	\tgf{4950}&	\tgf{4971}&	\tgf{1737}&	\tgf{5525}&	\tgf{1241}&	\tgf{3852}&	\tgf{3370}&\tgf{5981}\\
\tinynb{2019}&	\tinynb{1085}&	\tinynb{4950}&	\tinynb{4971}&	\tinynb{1737}&	\tinynb{5525}&	\tinynb{1241}&	\tinynb{3852}&	\tinynb{3370}&\tinynb{5981}\\
		\tgf{1910}&	\tgf{4729}&	\tgf{1105}&	\tgf{4906}&	\tgf{0749}&&&&&\\
	\tinynb{1910}&	\tinynb{4729}&	\tinynb{1105}&	\tinynb{4906}&	\tinynb{0749}&&&&&\\
\end{tabular}
\begin{exe}
\ex \label{ex:tg:habiller} 
\gll   \ipa{thja¹}	\ipa{zji¹}	\ipa{rjir²}	\ipa{ɕjwi¹}	\ipa{ka¹}	\ipa{zjɨ̣¹lji²}	\ipa{dʑjij¹}	\ipa{gu²}	\ipa{.a-tjɨ̣j²}	\ipa{lhwu¹}	\ipa{khjow¹}	\ipa{gjwi²-phji} \\
		ce	enfant	\comit{}	année	égal	enfant	aller	ensemble un-manière	habit	donner[A]	s'habiller[A]-causer[A] \\
\glt Il donna cet enfant et à d'autres enfants du même âge des habits du même type, et les fit s'en vêtir (Leilin 04.10A.4-5)
\end{exe}

Le verbe   \tgf{4906} \ipa{gjwi²} a pour dérivés nominaux 5598   \tgf{5598} \ipa{gjwi} 2.10 ``vêtement" et  1212 \tgf{1212} \ipa{gjwi} 1.11 (qui apparaît dans  \tgf{1153}\tgf{1212} \ipa{dʑjɨ¹gjwi¹} ``veste de cuir''). L'alternance tonale ici est peut-être un phénomène d'assimilation tonale plutôt qu'une dérivation.
\end{enumerate}

\subsubsection{Tangoute --i :: tibétain --i :: japhug --i/--e} \label{subsubsec:correspondance:i:i:i}

On reconstruira *--je en pré-tangoute pour la correspondance du tangoute --i avec un --i en tibétain et en japhug. On placera dans la même correspondance l'exemple avec --ji :: --e en japhug. Les formes correspondant à des syllabes en nasales finales en tibétain seront également inclues ici, mais on proposera aussi une reconstruction *--je pour celles-ci.
\newline


A. Initiales labiales \label{rimes:02:2:lab}
\begin{enumerate}



\item 2625   \tgf{2625} \ipa{.wji} 1.10 \ptang{C-pje}{hôte, invité} ainsi que 3107  \tgf{3107} \ipa{.wji} 2.10 ``hôte, amphitryon" s'apparentent au \jpg{tɯ-pi} (Gsardzong). On trouve en naish \naxi{bə˞˧}	\nayn{hĩ.bæ^H}	\protona{briN}, formes qui pourraient aussi être apparentées.


La direction de la dérivation, ainsi que la raison pour l'alternance entre  \tgf{2625} \ipa{.wji¹} et   \tgf{3107} \ipa{.wji²} reste inexpliquée. Toutefois, la comparaison avec le japhug suggère que la forme signifiant ``invité" est originelle et l'autre dérivée.
\newline
\linebreak
\begin{tabular}{llllllllll}
		\tgf{5093}&	\tgf{3640}&	\tgf{1906}&	\tgf{4989}&	\tgf{4889}&	\tgf{4884}&	\tgf{5447}&	\tgf{0795}&	\tgf{4469}&	\tgf{2625}\\
		\tinynb{5093}&	\tinynb{3640}&	\tinynb{1906}&	\tinynb{4989}&	\tinynb{4889}&	\tinynb{4884}&	\tinynb{5447}&	\tinynb{0795}&	\tinynb{4469}&	\tinynb{2625}\\
		\tgf{5113}&&&&&&&&&\\
		\tinynb{5113}&&&&&&&&&\\
\end{tabular}
\begin{exe}
\ex \label{ex:tg:hote}  \vspace{-8pt}
\gll   \ipa{tɕhjiw¹tha²}	\ipa{nioow¹}	\ipa{.wjɨ̣¹dʑjwɨ¹}	\ipa{nji²}	\ipa{do²}	\ipa{rjɨr²-ɕji²}	\ipa{.wji¹}	\ipa{.wji¹} \\
		Zhao.Dao	après	ami	\pl{}	\allat{}	\dir{}-aller[A] hôte	faire[A] \\
\glt Zhao Dao alla chez des amis comme invité.  (Leilin 05.24B.1)
\end{exe}
Il existe aussi un nom 4015   \tgf{4015} \ipa{bjiij} 2.35 ``hôte (invité)", dont voici un exemple d'emploi, où il s'oppose à  \tgf{3107} \ipa{.wji²}:
\newline
\linebreak
\begin{tabular}{llllllllll}
	\tgf{2104}&	\tgf{4174}&	\tgf{3583}&	\tgf{4015}&	\tgf{2226}&	\tgf{2503}&	\tgf{4174}&	\tgf{3583}&	\tgf{3107}&	\tgf{2226}\\
	\tinynb{2104}&	\tinynb{4174}&	\tinynb{3583}&	\tinynb{4015}&	\tinynb{2226}&	\tinynb{2503}&	\tinynb{4174}&	\tinynb{3583}&	\tinynb{3107}&	\tinynb{2226}\\
\end{tabular}
\begin{exe}
\ex \label{ex:tg:hote2}  \vspace{-8pt}
\gll   \ipa{ɕji¹}	\ipa{mju²}	\ipa{tja¹}	\ipa{bjiij²}	\ipa{.we²}	\ipa{kụ¹}	\ipa{mju²}	\ipa{tja¹}	\ipa{.wji²}	\ipa{.we²} \\
		avant	bouger	\topic{}	hôte	devenir	après	bouger	\topic{} amphitryon	devenir \\
\glt Celui qui bouge en premier est l’invité, celui qui bouge après est l'hôte. (Sunzi, 7A-6b-3), dans l’original chinois \zh{先動為客後動為主}.
\end{exe}
Ce nom, malgré la proximité phonétique superficielle avec   \tgf{3107} \ipa{.wji²} et    \tgf{2625} \ipa{.wji¹}, ne leur est pas apparenté. En effet   \tgf{4015} \ipa{bjiij²} signifie également ``voyager" (\tgf{3852}\tgf{4015} \ipa{dʑjij¹bjiij²}), et le sens ``hôte" en est un dérivé.


\item 4519   \tgf{4519} \ipa{bji} 2.10 \ptang{mbje}{crier, appeler} se compare au  \jpg{mbri} ``crier (animal), fort (bruit)''. Ce verbe s'applique au cri des animaux également en tangoute:
\newline
\linebreak
\begin{tabular}{lllll}
			\tgf{5142}&	\tgf{0491}&	\tgf{3349}&	\tgf{4519}&	\tgf{3092}\\
		\tinynb{5142}&	\tinynb{0491}&	\tinynb{3349}&	\tinynb{4519}&	\tinynb{3092}\\
\end{tabular}
\begin{exe}
\ex \label{ex:tg:crier:animal}  \vspace{-8pt}
\gll  \ipa{khia²}	\ipa{ljọ²}	\ipa{rjijr²}	\ipa{bji²-djij²}\\
		pie	où	\loc{}	chanter-\dur{} \\
\glt Où la pie chante-t-elle? (Leilin 04.04B.6)
\end{exe}


\item 251	\tgf{0251}	\ipa{bji}	2.10  \ptang{mbjeN > *mbje}{corde} est comparable au 	\jpg{tɯ-mbri} et au \tib{ɴbreŋ} de même sens. Le chinois \zh{繩} zying <\ipapl{*Cə-mrəŋ} est également potentiellement comparable, même si la correspondance vocalique est irrégulière. La forme tibétaine montre qu'il faut  reconstruire ici une nasale finale en pré-pré-tangoute.



\item 5509   \tgf{5509} \ipa{bjị} 1.67 \ptang{S-mbjeN > *S-mbje}{urine}, aussi écrit 3142  \tgf{3142}, peut être rapproché du  \jpg{tɤ-rmbi} ``urine". Le cognat  \pumi{bĩ̌} avec une voyelle nasale permet de montrer qu'une nasale a disparu ici en tangoute. Le naish \naxi{mbi˧ } \protona{mbi} est ici ambigu quant à la présence de nasale.


\end{enumerate}

A ces exemples, il est peut-être possible de rajouter le verbe 2047 \tgf{2047} \ipa{mjii} 1.14 ``donner" comparé au  \jpg{mbi}, au \plb{be²}{0605} et au \tib{sbʲin, bʲin}. Cette comparaison supposerait d'admettre une correspondance entre prénasalisée du japhug et nasale du tangoute qui ne s'observe pas habituellement, comme nous l'avons montré en \ref{subsubsec:prenasaliseestg}. Elle est donc très problématique. La finale --n du tibétain est une innovation de cette langue, on ne la retrouve pas habituellement dans les autres langues ST qui contrairement au tangoute et au rgyalrong, préservent le --n final. Par exemple, la forme birmane est \ipa{pe³}.
\newline
\linebreak
B. Initiales dentales \label{rimes:02:2:dent}

\begin{enumerate}

\item 4658	\tgf{4658} \ipa{thji}	1.11	\ptang{thjeN > *thje}{boire} peut se comparer au  \jpg{tshi} ``boire".\footnote{La forme \ipa{tshi} est propre au dialecte de Kamnyu qui a subi le changement *thi > tshi, dans les autres dialectes japhug on trouve \ipa{thi}.} Le cognat  \pumi{thĩ̌, khə-thĩ̂} ``boire" indique la présence d'une nasale en pré-pré-tangoute.
Comme nous l'avons mentionné p.\pageref{ex:tg:manger}, outre son sens de ``boire", ce verbe s'emploie souvent dans la sens de ``manger".


\item 1319	\tgf{1319}	\ipa{tshji} 1.11 \ptang{tshje}{aimer} est potentiellement comparable au  \jpg{nɤntshi}	``aimer ". En japhug, ce verbe est la forme tropative (préfixe \ipa{nɤ}--, qui dérive un verbe transitif signifiant ``considérer comme'' d'un intransitif), du \jpg{ntshi} ``falloir". Le verbe  \ipa{nɤntshi} signifie donc littéralement ``considérer comme important''.
 
 
Le sens précis de \tgf{1319} \ipa{tshji¹} mérite une discussion détaillée, car son emploi dans les textes est difficile à ramener à un sens unique. Le sens d'``aimer" est clairement attesté dans le Wenhai, où \tgf{1319}	\ipa{tshji¹} est glosé par \tgf{1338} \ipa{dzu¹} ``aimer". Toutefois, nous n'avons jusqu'ici trouvé aucun exemple textuel où \tgf{1319}	\ipa{tshji¹} a ce sens. Il sert habituellement à traduire le chinois \zh{要} \textit{yào} ``important, être nécessaire". Ainsi, dans la version de Sunzi, il apparaît dans l'expression \tgf{1319}\tgf{2963} \ipa{tshji¹no²} ``point vital" (chinois \zh{要害} \textit{yàohài}). On le trouve également de le sens de ``falloir, être nécessaire":
\newline
\linebreak
\begin{tabular}{llllllllll}
	\tgf{5167}&	\tgf{5754}&	\tgf{3357}&	\tgf{0705}&	\tgf{5167}&	\tgf{3099}&	\tgf{5645}&	\tgf{2699}&	\tgf{2541}&	\tgf{1319}\\
		\tinynb{5167}&	\tinynb{5754}&	\tinynb{3357}&	\tinynb{0705}&	\tinynb{5167}&	\tinynb{3099}&	\tinynb{5645}&	\tinynb{2699}&	\tinynb{2541}&	\tinynb{1319}\\
\end{tabular}
\begin{exe}
\ex \label{ex:tg:important1}  \vspace{-8pt}
\gll   \ipa{lha¹}	\ipa{lju²}	\ipa{kiẹj²}	\ipa{zjịj¹}	\ipa{lha¹}	\ipa{dʑjiij¹}	\ipa{tjị²}	\ipa{nwə¹}	\ipa{dzjwo²}	\ipa{tshji¹} \\
		cerf	attraper[A]	vouloir	temps	cerf	habiter	endroit	savoir	homme	falloir \\
\glt Lorsque l'on veut attraper un cerf, il faut un homme qui connaisse l'endroit où habite le cerf. (Sunzi 6A.24.3, voir \citet[72]{lin94sunzi}
\end{exe}
Ce verbe apparaît aussi suivi d'une copule, dans un usage manifestement nominalisé  ``affaire importante":
\newline
\linebreak
\begin{tabular}{llllll}
	\tgf{1531}&	\tgf{2805}&	\tgf{1139}&	\tgf{1183}&	\tgf{1319}&	\tgf{0508}\\
\tinynb{1531}&	\tinynb{2805}&	\tinynb{1139}&	\tinynb{1183}&	\tinynb{1319}&	\tinynb{0508}\\
\end{tabular}
\begin{exe}
\ex \label{ex:tg:important2}  \vspace{-8pt}
\gll   \ipa{gja¹bjuu²}	\ipa{.jij¹}	\ipa{dạ²}	\ipa{tshji¹}	\ipa{ŋwu²} \\
		général	\antierg{} affaire important être \\
\glt  Les affaires qui concernent les généraux sont d'une importance capitale, Sunzi 29-6-11. \footnote{Notre interprétation de cette phrase diffèrent ici légèrement de celle de \citet{lin94sunzi}, qui considère \tgf{1183}\tgf{1319} \ipa{dạ²}	\ipa{tshji¹} comme formant un syntagme.}
\end{exe}
Enfin, le verbe \tgf{1319}	\ipa{tshji¹} apparaît transitivement avec comme objet \tgf{3195} \ipa{gjwi²}	"parole" dans le sens de ``confier (une tâche), recommander, exhorter" (chinois \zh{囑} \textit{zhǔ}), comme on peut le constater dans l'exemple suivant:
\newline
\linebreak
\begin{tabular}{llllllllll}
	\tgf{3735}&	\tgf{0092}&	\tgf{4028}&	\tgf{4921}&	\tgf{0403}&	\tgf{1139}&	\tgf{3195}&	\tgf{1326}&	\tgf{1319}&	\tgf{4601}\\
		\tinynb{3735}&	\tinynb{0092}&	\tinynb{4028}&	\tinynb{4921}&	\tinynb{0403}&	\tinynb{1139}&	\tinynb{3195}&	\tinynb{1326}&	\tinynb{1319}&	\tinynb{4601}\\
		\tgf{2098}&	\tgf{1139}&	\tgf{0248}&	\tgf{0140}&	\tgf{4841}&	\tgf{1599}&	\tgf{4568}&&&\\
		\tinynb{2098}&	\tinynb{1139}&	\tinynb{0248}&	\tinynb{0140}&	\tinynb{4841}&	\tinynb{1599}&	\tinynb{4568}&&&\\
\end{tabular}
\begin{exe}
\ex \label{ex:tg:important3}  \vspace{-8pt}
\gll   \ipa{tji.¹}	\ipa{mja¹}	\ipa{nji²}	\ipa{swẽ¹}	\ipa{.wow¹}	\ipa{.jij¹}	\ipa{gjwi²}	\ipa{kjɨ¹-tshji¹-nja²}	\ipa{ŋa²}	\ipa{.jij¹}	\ipa{no²}	\ipa{rejr²}	\ipa{djij²-rjir¹-phjo²}\\
		souhaiter	mère	toi	Sun	E	\antierg{}	parole	\dir{}-confier-2\sg{}	je	\antierg{}	paix	bonheur	
		\opt{}-obtenir[A]-causer[B] \\
\glt Mère, je t'en prie, demande à Sun E (d'intervenir en ma faveur), afin de me permettre d'obtenir la paix.\footnote{Dans cette phrase, le thème B du verbe implique que l'agent de \tgf{4568} \ipa{phjo²} est à la seconde personne du singulier.} (Leilin, 06.22B.4)
\end{exe}
Le lien sémantique entre ``être important" et ``exhorter, confier" peut s'expliquer si l'on considère cet usage transitif comme un causatif ``rendre important (une parole) aux yeux de quelqu'un".
\newline
\tgf{1319} \ipa{tshji¹} a selon \citet{gong01huying} une forme alternante 5610 \tgf{5610} \ipa{tshjɨ¹} ``aimer", mais nous ne disposons pas d'exemples textuels de ce mot. Il est certain en tout cas qu'il ne s'agit pas d'une opposition de personne, car on aurait attendu sinon la forme  \tgf{5610} \ipa{tshjɨ¹} dans l'exemple \ref{ex:tg:important3}, puisqu'il est suivi du suffixe de seconde personne \tgf{4601} \textit{nja²}.




\item 4399	\tgf{4399}	\ipa{dzjị}	2.60 \ptang{S-ndzje}{colonne} est comparable au  \jpg{tɤ-jtsi} ``pilier, poteau''. Comme nous l'avons mentionné en \ref{subsubsec:prenasaliseestg}, ce mot fait partie des exceptions pour lesquelles une prénasalisée en tangoute correspond à une sourde en japhug.
\newline
\linebreak
\begin{tabular}{lllllllll}
	\tgf{2736}&	\tgf{3628}&	\tgf{5604}&	\tgf{5113}&	\tgf{4880}&	\tgf{4399}&	\tgf{2590}&	\tgf{5755}&	\tgf{0749}\\
	\tinynb{2736}&	\tinynb{3628}&	\tinynb{5604}&	\tinynb{5113}&	\tinynb{4880}&	\tinynb{4399}&	\tinynb{2590}&	\tinynb{5755}&	\tinynb{0749}\\
\end{tabular}
\begin{exe}
\ex \label{ex:tg:pilier}  \vspace{-8pt}
\gll   \ipa{biaa²ɣjwã¹}	\ipa{dʑjɨ.wji¹}	\ipa{rər²}	\ipa{dzji.²}	\ipa{.wjɨ²-.jar¹-phji¹} \\
		Ma.Yuan	\erg{}		bronze	pilier	\dir{}-se.lever-causer[A] \\
\glt Ma Yuan y fit ériger des piliers de bronze. (Leilin 04.34B.7)
\end{exe}
On retrouve une forme \pumi{stã¹} (Lanping) ``pilier'' correspondant à \ipa{sɛ̃́} dans le dialecte de Shuiluo. Si ces noms sont apparentés aux formes japhug et tangoute, il faudrait normalement reconstruire ici une voyelle nasale, mais la correspondance ne semble pas régulière, car on attendrait plutôt la rime \ipapl{-ĩ}.


\item 4250	\tgf{4250}	\ipa{sji}	1.11 \ptang{sjeN > *sje}{bois, arbre} correspond au  \jpg{si}, au naish \naxi{si˥},	\nayn{si˧},	\laze{siN},	\protona{mbi} et au \tib{ɕiŋ} de même sens; en LB on trouve une forme en occlusive finale \plb{sík}{0303}. Ce nom est répandu dans l'ensemble de la famille sino-tibétaine, et correspond en particulier au chinois \zh{薪} \textit{xīn}. Voir la discussion p.\pageref{ex:tg:arbreclassificateur} à propos de son emploi avec le classificateur \tgf{5814} \ipa{phu²}. 


\item 5273	\tgf{5273}	\ipa{sji}	2.10 \ptang{sjeN > *sje}{foie} peut être rapproché du  \jpg{tɯ-mtshi}, du \plb{(ʃ)-sin²}{0143}, du naish \naxi{sə˞˥}	\nayn{si˧}	\laze{si˧}	\protona{siN}  et du \tib{mtɕʰin.pa} ``foie''. En japhug, on trouve dans certains dialectes japhug tels que celui de Gsardzong la forme \ipa{tɯ-fsi}. En situ également, on observe une forme sans nasale \ipa{tə-pɕé}. On postule donc en proto-japhug deux reconstructions *m-sij et *p--sij (\citealt[290]{jacques04these}): l'affriquée provient ici d'une fricative (le groupe ms-- est interdit par la phonotactique de la langue). En tibétain, un phénomène similaire s'est produit: d'après la seconde loi de Li Fang-kuei, les fricatives deviennent des affriquées lorsqu'elles sont précédées de nasales en proto-tibétain. La forme mchin- provient donc d'un proto-tibétain \ipapl{*m-ɕin < **m-sin}. Voir \citet[277]{matisoff03} pour une liste de cognats dans d'autres langues ST.




\item 3072	\tgf{3072}	\ipa{sji}	2.10 \ptang{sje}{mourir}, aussi écrit 3363 \tgf{3363} est cognat du  \jpg{si}, du \plb{ʃe²}{0599}, du naish \naxi{ʂɯ˧}	\nayn{ʂɯ˧}	\laze{sɯ˩}	\protona{rsi} et du \tib{ɴtɕʰi, ɕi} ``mourir''. Ce mot pan-sino-tibétain se retrouve dans la majorité des langues de la famille. Il présente une forme alternante 5918 \tgf{5918} \ipa{sjɨ} 1.30 qui, comme nous l'avons montré dans \citet{jacques09tangutverb}, n'est pas liée à une alternance de personne. La fonction morphologique de cette alternance est encore inconnue, voir \ref{sec:morpho.verbale.flex}.


\item 2858	\tgf{2858} \ipa{zjir}	2.72 \ptang{s-rjeN > *s-rje}{long} correspond au japhug \ipa{zri} ``long'', au \plb{s/m-riŋ¹}{0754}, au naish \naxi{ʂə˞˩},	\nayn{ʂæ˧},	\protona{ɕiN}  et au \tib{riŋ-po}. Concernant la reconstruction de ce mot, consulter la discussion p.\pageref{analyse:sr}.



\item 2059	\tgf{2059} \ipa{lhjị} 2.60 \ptang{S-hlje}{excrément} peut être rapproché du  \jpg{tɯ-ɣli} ``engrais, purin" (<*kli), du \plb{ʔ/k(l)e²}{0150} et du \tib{ltɕi.ba} (<*lhji) de même sens. Le nom 3499 \tgf{3499} \ipa{ljɨ} 1.29 ``bouse'' y est également apparenté, même si l'alternance morphologique entre ces deux formes est difficile à expliquer.


\item 3668	\tgf{3668} \ipa{ljị} 1.67 \ptang{S-lje}{planter}  peut être rapproché du  \jpg{ji} (<*lji) ``planter''.\footnote{Ce verbe japhug n'est pas apparenté, malgré les apparences, au nom \ipa{tɯ-ji} ``champs'' cognat avec le \tib{ʑiŋ}. Concernant le changement *lj > j en rgyalrong, voir \citet[366-367]{jacques08}}.



\item 4518 \tgf{4518} \ipa{lji} 2.09 \ptang{lje}{odeur} peut se comparer au  \jpg{ɯ-di} (<\ipapl{*tli}). Une relation plus lointaine avec le \tib{dri.ma} ``odeur'' n'est pas à exclure. Ce nom apparaît dans les textes bouddhiques, par exemple pour traduire le nom des \textit{gandharva-} \tgf{4518}\tgf{4517} \ipa{lji²dzji¹} ``mangeur d'odeur" dans \citet[266, ligne 461]{lin06manjusri}. Ce nom est parallèle au \tib{dri.za} ``gandharva''; toutefois, comme ce composé est attesté aussi dans les textes bouddhiques chinois (\zh{食香} \textit{shíxiāng}), il est difficile de savoir si cette traduction s'est faite à partir du tibétain ou du chinois, bien que la première option semble plus probable. Ce nom est écrit 4647 \tgf{4647} dans le composé \tgf{4647}\tgf{5018} \ipa{lji²naa²} ``herbe odorante''.

A ce nom, on doit relier le verbe ``sentir'' 5242 \tgf{5242} \ipa{ljii} 1.14. 
\newline
\linebreak
\begin{tabular}{lllllll}
	\tgf{2154}&	\tgf{2983}&	\tgf{0795}&	\tgf{5449}&	\tgf{1909}&	\tgf{1918}& \tgf{5242} \\
	\tinynb{2154}&	\tinynb{2983}&	\tinynb{0795}&	\tinynb{5449}&	\tinynb{1909}&	\tinynb{1918}& \tinynb{5242} \\
\end{tabular}
\begin{exe}
\ex \label{ex:tg:sentir}  \vspace{-8pt}
\gll   \ipa{rewr¹}	\ipa{.u¹}	\ipa{rjɨr¹-tjị¹}	\ipa{gur¹}	\ipa{mji¹-ljii¹} \\
		fleuve	intérieur		\dir{}-mettre[A] bœuf \negat{}-sentir \\
\glt Si on les met dans le fleuve, le bœuf ne les sentira pas. (\citealt[192]{kychanov74}).
\end{exe}
Comme on ne retrouve en rgyalrong et dans les autres langues ST que le cognat du nom ``odeur'' et pas celui du verbe sentir, on peut en conclure qu'il est plus que probable que ce verbe est un dérivé du nom, et qu'il s'agit d'une innovation propre au tangoute. 


\end{enumerate}


C. Initiales palatales \label{rimes:02:2:pal}

\begin{enumerate}


\item 3929 \tgf{3929}	\ipa{tɕhjwi} 1.10 ``faire fondre'' (PT*p-thrje) est cognat du	 \jpg{ftʂi} de même sens. Dans les deux langues, il s'emploie pour désigner la fonte des métaux:
\newline
\linebreak
\begin{tabular}{lllllllll}
	\tgf{3058}&	\tgf{1254}&	\tgf{0152}&	\tgf{5671}&	\tgf{1074}&	\tgf{3929}&	\tgf{3572}&	\tgf{4880}&	\tgf{5113}\\
	\tinynb{3058}&	\tinynb{1254}&	\tinynb{0152}&	\tinynb{5671}&	\tinynb{1074}&	\tinynb{3929}&	\tinynb{3572}&	\tinynb{4880}&	\tinynb{5113}\\
\end{tabular}
\begin{exe}
\ex \label{ex:tg:fondre}  \vspace{-8pt}
\gll   \ipa{zjɨɨr²}	\ipa{dʑjwɨr¹}	\ipa{kiẹ¹}	\ipa{thjɨ¹}	\ipa{lụ¹}	\ipa{tɕhjwi¹}	\ipa{ŋwo²}	\ipa{rər²}	\ipa{.wji¹} \\
		eau	moudre	or	rejetter	pierre fondre argent bronze faire[A] \\
\glt L'eau érode (les roches) et rejette de l'or. On fait fondre les pierres et on en fait de l'argent et du bronze. (Shengliyihai, 4b:3, \citealt[115,305]{kychanov97})
\end{exe}
Le verbe intransitif 3956 \tgf{3956} \ipa{dʑji} 1.10 \ptang{ndrje}{fondre} en est le dérivé par prénasalisation anticausative (voir section \ref{subsec:anticausatif}). Il correspond exactement au  \jpg{ndʐi} ``fondre''. Une reconstruction \ipa{dʑjwi} serait également possible, auquel cas il faudrait reconstruire *p-ndrje avec la même préinitiale *p- que pour la forme transitive.



\item 4469	\tgf{4469}	\ipa{ɕji} 2.09 ``aller'' (PT \ipapl{*ɕe}) peut se comparer au  \jpg{ɕe}. Un rapport  avec le \tib{mtɕʰi, mtɕʰis} ``aller'' serait également possible, car cette forme pourrait théoriquement provenir d'un \ipapl{*m-ɕi} par la seconde loi de Li Fang-kuei. 	\tgf{4469}	\ipa{ɕji²} a une forme alternante 4481 \tgf{4481} \ipa{ɕjɨ} 1.29. Cette alternance n'est pas conditionnée par la personne, comme nous l'avons montré dans \citet{jacques09tangutverb}. Parmi les verbes apparentés, on note également la forme 5424 \tgf{5424} \ipa{ɕjwɨ} 1.29 (voir \citealt{gong88alternations}, \citealt[45]{gong02a}), qui n'est attestée que dans la forme \tgf{5424}\tgf{5065} \ipa{ɕjwɨ¹ljị²} ``aller et venir, avoir des relations avec'':
\newline
\linebreak
\begin{tabular}{llllllllll}
	\tgf{0153}&	\tgf{0547}&	\tgf{3590}&	\tgf{1936}&	\tgf{3118}&	\tgf{0724}&	\tgf{3370}&	\tgf{0243}&	\tgf{2541}&	\tgf{0683}\\
\tinynb{0153}&	\tinynb{0547}&	\tinynb{3590}&	\tinynb{1936}&	\tinynb{3118}&	\tinynb{0724}&	\tinynb{3370}&	\tinynb{0243}&	\tinynb{2541}&	\tinynb{0683}\\
	\tgf{5157}&	\tgf{5447}&	\tgf{5424}&	\tgf{5065}&&&&&&\\
	\tinynb{5157}&	\tinynb{5447}&	\tinynb{5424}&	\tinynb{5065}&&&&&&\\
\end{tabular}
\begin{exe}
\ex \label{ex:tg:alleretvenir}  \vspace{-8pt}
\gll   \ipa{khu¹njij²}	\ipa{gjii¹xiəj²xu¹}	\ipa{njɨ²}	\ipa{gu²}	\ipa{sji²dzjwo²}	\ipa{xia¹kji¹}	\ipa{do²}	\ipa{ɕjwɨ¹ljị¹} \\
		Kong.Ning Yi.Xingfu \pl{} ensemble dame Xia.Ji \antierg{} avoir.des.relations \\
\glt Kong Ning (\zh{孔寧}) et Yi Xingfu  (\zh{儀行父}) avaient tous deux des relations avec l'épouse Xia Ji. (Leilin 03.17A.4)
\end{exe}

\end{enumerate}

D. Initiales r-- \label{rimes:02:2:r}

\begin{enumerate}

\item 2537	\tgf{2537} \ipa{rjir} 2.72 \ptang{rje}{rester} se compare au  \jpg{ri} ``rester'':
\newline
\linebreak
\begin{tabular}{lllllll}
	\tgf{4408}&	\tgf{4115}&	\tgf{0558}&	\tgf{4413}&	\tgf{2537}&	\tgf{2090}&	\tgf{2194}\\
	\tinynb{4408}&	\tinynb{4115}&	\tinynb{0558}&	\tinynb{4413}&	\tinynb{2537}&	\tinynb{2090}&	\tinynb{2194}\\
\end{tabular}
\begin{exe}
\ex \label{ex:tg:rester}  \vspace{-8pt}
\gll   \ipa{məə¹}	\ipa{gjij²}	\ipa{njijr¹}	\ipa{pju²}	\ipa{rjir²}	\ipa{lew²}	\ipa{mjij¹} \\
		feu campagne animaux.sauvages brûler[A] rester \nmls{} ne.pas.avoir \\
\glt Le feu a brûlé la campagne et les animaux, et il ne reste rien. (Leilin 05.20B.3-4)
\end{exe}
On retrouve aussi \tgf{2537} \ipa{rjir²} dans la forme \tgf{1906}\tgf{2537} \ipa{nioow¹rjir²} ``rester en arrière, arriver en dernier'' (Lelin 04.12B.2).


\item 567	\tgf{0567}	\ipa{rjir}	2.72 \ptang{rje}{devant} est apparenté à la dernière syllabe du nom  \jpg{ɯ-ʁɤri} ``avant''. La même racine se retrouve dans d'autres langues macro-rgyalronguiques telles que le  \pumi{rə³}. En tangoute, \tgf{0567}	\ipa{rjir²} n'est pas normalement attesté seul, il apparaît dans les composés \tgf{1778}\tgf{0567} \ipa{.ju²rjir²}  et  \tgf{5416}\tgf{0567} \ipa{ɣwə²rjir²} .
\end{enumerate}

D. Initiales vélaires \label{rimes:02:2:vel}
\begin{enumerate}


\item 1638	\tgf{1638}	\ipa{gji}	1.11 \ptang{ŋgje}{clair, propre} peut être rapproché du  \jpg{mgri} ``claire (eau)''. 
\newline
\linebreak
\begin{tabular}{lllllllll}
	\tgf{2019}&	\tgf{3058}&	\tgf{0856}&	\tgf{1245}&	\tgf{1638}&	\tgf{3133}&	\tgf{0441}&	\tgf{4672}&	\tgf{0749}\\
	\tinynb{2019}&	\tinynb{3058}&	\tinynb{0856}&	\tinynb{1245}&	\tinynb{1638}&	\tinynb{3133}&	\tinynb{0441}&	\tinynb{4672}&	\tinynb{0749}\\
\end{tabular}
\begin{exe}
\ex \label{ex:tg:claire}  \vspace{-8pt}
\gll   \ipa{thja1}	\ipa{zjɨɨr²}	\ipa{mər²}	\ipa{.jij¹}	\ipa{gji¹}	\ipa{sjij¹}	\ipa{sjwɨ¹}	\ipa{niəj¹}	\ipa{phji¹} \\
		cette	eau	origine soi-même	clair aujourd'hui	qui trouble causer[A] \\
\glt Son eau était naturellement propre. Qui l'a rendue trouble telle qu'elle est aujourd'hui ? (Leilin 03.25A.1-2)
\end{exe}

\end{enumerate}




\subsubsection{Tangoute --e :: tibétain --a :: japhug --a}	\label{subsubsec:correspondance:e:a:a}
On reconstruira *a en pré-tangoute pour la correspondance du tangoute --e et --ie avec a en japhug ou en tibétain. En effet, --e et --ie sont en distribution complémentaire sauf devant les labiales: --ie apparait avec les vélaires (sauf la nasale) et les palatales, --e avec les dentales et la nasale vélaire. Devant les vélaires, on verra à la sous-section suivante que l'opposition entre --e et --ie reflète une distinction réelle de la proto-langue lorsque cette voyelle du tangoute correspond à un --i ou à un --e en japhug ou en tibétain *--e contre *--re). En revanche, dans la correspondance --e/--ie :: --a, nous n'avons pas de paire minimal et aucune preuve de la nécessité de reconstruire cette opposition en pré-tangoute. 
\newline
\linebreak
A. Initiales labiales, dentales et palatales \label{rimes:02:3:lab}
Les exemples de mots avec ces initiales sont peu nombreux, et nous les regroupons donc dans la même sous-section.
\begin{enumerate}

\item 5134 \tgf{5134} \ipa{.we} 1.08 (PT *C-pa) peut être rapproché du  \tib{bʲa} ``oiseau''. Le  \jpg{pɣa} ``oiseau'' en revanche n'est pas apparenté, malgré les apparences, car il provient d'un <*pka, comme le montre le situ \ipa{pká}, voir \citet[373]{jacques08}.


\item 4966	\tgf{4966}	\ipa{.wẹ}	1.65 \ptang{S-wa}{rouille} peut se comparer au  \jpg{sɣa} ``rouille'' (<*swa).


\item 499 \tgf{0499} \ipa{piẹ} 1.66   \ptang{S-pa}{grenouille} est cognat avec  \jpg{qaɕpa} ``grenouille'', au \plb{k-ʔ-pa²}{0066}  et au \tib{sbal}. En tangoute comme en rgyalrong et en LB, le --l a chuté après le *--a à époque ancienne. \label{analyse:grenouille}


\item 3531	\tgf{3531}	\ipa{dze}	1.08 \ptang{ndza}{lèpre} peut être rapproché du  \jpg{kɤkɯnɤndza} et du \tib{mdze}. Un rapport étymologique avec la racine du verbe ``manger'' \tgf{4517} \ipa{dzji¹} est fort probable (la maladie qui ronge), mais cette dérivation a eu lieu à une période ancienne et n'est de toute façon pas pertinente pour l'étude du tangoute.

\item 1634	\tgf{1634}	\ipa{rer} 	2.71 \ptang{ra}{filet} se compare au	\tib{dra.ba} et peut-être aussi au chinois \zh{羅}  *rˁaj de même sens. Il faut sans doute partir d'une forme plus ancienne *ral, avec chute du *--l en birmo-qianguique commun comme dans ``grenouille''.

\item 2456 \tgf{2456} \ipa{se} 1.08 ``chanvre'' peut se comparer au  \jpg{tasa} de même sens. Il s'agit d'un Wanderwort qu'on retrouve dans les langues naish comme le \naxi{sɑ}.



\item 927 	\tgf{0927}	\ipa{le}  2.07 \ptang{la}{bouillant} peut être rapproché du  \jpg{ala} ``bouillant''. Ce verbe statif a une forme causative 4664 \tgf{4664} \ipa{lẹ} 1.65 \ptang{S-la}{faire bouillir}:
\newline
\linebreak
\begin{tabular}{llllllllll}
	\tgf{2019}&	\tgf{0433}&	\tgf{3612}&	\tgf{4664}&	\tgf{4342}&	\tgf{4582}&	\tgf{0824}&	\tgf{5417}&	\tgf{5865}&	\tgf{4100}\\
	\tinynb{2019}&	\tinynb{0433}&	\tinynb{3612}&	\tinynb{4664}&	\tinynb{4342}&	\tinynb{4582}&	\tinynb{0824}&	\tinynb{5417}&	\tinynb{5865}&	\tinynb{4100}\\
	\tgf{1888}&	\tgf{1304}&	\tgf{5981}&	\tgf{4585}&&&&&&\\
	\tinynb{1888}&	\tinynb{1304}&	\tinynb{5981}&	\tinynb{4585}&&&&&&\\
\end{tabular}
\begin{exe}
\ex \label{ex:tg:bouillir}  \vspace{-8pt}
\gll   \ipa{thja¹}	\ipa{bju¹}	\ipa{tsə̣¹}	\ipa{lẹ¹}	\ipa{dja²-tjị²}	\ipa{tɕhjɨ²rjar²}	\ipa{sọ¹-ɕjɨj²}	\ipa{bə²lụ¹}	\ipa{.a-.wja¹} \\
		cela	\instr{} médicament bouillir \dir{}-donner.à.boire immédiatement trois-pinte ver \dir{}-vomir \\
\glt Ainsi, il lui prépara son médicament, le lui donna à boire et il vomit trois pintes de vers (Leilin 06.12A.2)
\end{exe}


\item 2547	\tgf{2547} \ipa{tɕier} 1.78 \ptang{r-tɕa}{droite} est comparable au  \jpg{χcha} ``droite''. Ce mot apparaît souvent en coordination avec 2920 \tgf{2920} \ipa{ʑjɨ̣} 1.69 ``gauche'' qui par contre n'a pas de cognats en japhug ni dans aucune langue ST connue.

\end{enumerate}
B. Initiales vélaires \label{rimes:02:3:vel}
\begin{enumerate}


\item 439  \tgf{0439} \ipa{ɣiẹ} 1.66 \ptang{C-S-qa}{cuire} peut se comparer au  \jpg{sqa} ``cuire". C'est un verbe transitif qui implique un agent humain : 
\newline
\linebreak
\begin{tabular}{llllllll}
	\tgf{5817}&	\tgf{5604}&	\tgf{5113}&	\tgf{0804}&	\tgf{5754}&	\tgf{0439}&	\tgf{1101}&	\tgf{0705}\\
	\tinynb{5817}&	\tinynb{5604}&	\tinynb{5113}&	\tinynb{0804}&	\tinynb{5754}&	\tinynb{0439}&	\tinynb{1101}&	\tinynb{0705}\\
\end{tabular}
\begin{exe}
\ex \label{ex:tg:cuire}  \vspace{-8pt}
\gll   \ipa{kjwɨɨr¹}	\ipa{dʑjɨ.wji¹}	\ipa{djɨ¹-lju²}	\ipa{.ɣiẹ¹-.jij¹}	\ipa{zjịj¹} \\
		brigand	\erg{}	\dir{}-attraper[A]	cuire-\fut{}	quand \\
\glt Les brigands l'avaient attrapé, et alors qu'ils s'apprêtaient à le cuire,  (Leilin 03.30B.2)
\end{exe}
Ce verbe est peut-être apparenté à 4629   \tgf{4629} \ipa{ɣjii} 1.14 \ptang{C-kjaa}{cuire} non attesté dans les textes que nous avons examinés, bien que la relation de dérivation qui les relie est complètement opaque. 

\item 3596	\tgf{3596}	\ipa{ɣiwe}	1.09 \ptang{C-p-ka}{puissance}	est apparenté au \jpg{βʁa} ``gagner, être le meilleur (à un concours)'' < *p-ka. Il s'agit vraisemblablement d'un dérivé nominalisé du type \jpg{kɯ-βʁa} ``noble < celui qui est le meilleur''. Les formes naish \naxi{ŋgɑ˧}, \nayn{ʁɑ˥}, \protona{ŋga/aC1} y sont apparentées. 

Le nom \tgz{3354} \ptang{C-ka}{force} est vraisemblablement la racine originelle d'où *C-p-ka est dérivé, bien que l'on ne retrouve pas d'équivalent à ce préfixe *p-- dans les autres langues. On retrouve des cognats en naish \naxi{kɑ˧tɯ˥}, \nayn{ʁɑ˥}, \laze{ʁɑ˩zi˩}, \protona{Nka/aC1}.

\item 4046 \tgf{4046} \ipa{khie} 1.09 ``amer'' peut se comparer au \tib{kʰa-ba}. Cet étymon a des cognats dans la plupart des langues sino-tibétaines, y compris le chinois \zh{苦} khuX < *\ipapl{khˁaʔ}. Contrairement à ce que proposait \citet{gong95st}, cette forme sino-tibétaine est en revanche sans relation avec la première syllabe de \tgf{5803}\tgf{5774} \ipa{kha²rewr¹} attesté dans le ZZZ (151) où il est traduit par le chinois \zh{苦蕖} \textit{khǔqú} ``laitue amère''. Ce dissyllabe est attesté dans les proverbes tangoutes à plusieurs reprises, mais le premier caractère \tgf{5803} n'apparaît jamais seul, et rien n'autorise à penser que son sens propre est ``amer''.





\item 4092 \tgf{4092} \ipa{khie} 1.09 \ptang{kha}{détester, haïr} correspond au  \jpg{qha} ``détester, s'énerver''. 
\begin{exe}
\ex \label{ex:jpg:detester}  
\gll   \ipa{tɕheme}	\ipa{nɯ}	\ipa{kɯ}	\ipa{wuma}	\ipa{ʑo}	\ipa{pjɤ-qha} \\
		fille cette \erg{} très \intens{} \med{}-détester \\
\glt La fille le détestait énormément (La grenouille, 19)
\end{exe}
Une comparaison alternative avec le  \jpg{nɤkhe} ``maltraiter'' serait possible également, mais semble moins convaincante, car le sens du verbe  \tgf{4092} \ipa{khie} est identique au japhug \ipa{qha} :
\newline
\linebreak
\begin{tabular}{lllllllll}
	\tgf{5243}&	\tgf{4172}&	\tgf{5447}&	\tgf{1677}&	\tgf{0502}&	\tgf{0433}&	\tgf{4574}&	\tgf{0010}&	\tgf{4092}\\
	\tinynb{5243}&	\tinynb{4172}&	\tinynb{5447}&	\tinynb{1677}&	\tinynb{0502}&	\tinynb{0433}&	\tinynb{4574}&	\tinynb{0010}&	\tinynb{4092}\\
\end{tabular}
\begin{exe}
\ex \label{ex:tg:detester}  \vspace{-8pt}
\gll   \ipa{sjij².ju²}	\ipa{do²}	\ipa{ljị²}	\ipa{zjɨɨr¹}	\ipa{bju¹}	\ipa{mjɨ¹}	\ipa{zji²}	\ipa{khie¹} \\
		peuple \allat{} bienfait peu \instr{} autre tous détester[A] \\
\glt Comme il n'apportait au peuple que peu de bienfaits, tous le détestaient (Leilin 04.22A.3)
\end{exe}
La forme alternante 1525 \tgf{1525} \ipa{khio} 1.50 pourrait suggérer la présence d'une médiane *--r-- à cause de son  --i--, mais il semble préférable d'expliquer celui-ci par l'analogie avec le thème 1. On aurait eu une alternance \ipa{khie¹} < (*kha) / *kho (<*kha-u) régularisée en \ipa{khie¹} /\ipa{khio¹}. La forme \tgf{1525} \ipa{khio¹} s'emploie avec un agent à la première et à la seconde personnes du singulier, comme le montre l'exemple \ref{ex:tg:craindre2} p.\pageref{ex:tg:craindre2}.


La paire de formes alternantes à initiales non-aspirées au ton 2, 4041 \tgf{4041} \ipa{kie} 2.08 et 1524 \tgf{1524} \ipa{kio} 2.43 signifiant aussi ``détester'' sont apparentées à \tgf{4092} \ipa{khie¹}. La raison de cette alternance d'aspiration et de tons est inexplicable pour le moment.


Il convient de noter que le caractère \tgf{4092} s'emploie parfois de façon pour transcrire l'homophone \tgf{4046} \ipa{khie¹}, comme dans la phrase \tgf{3966}\tgf{4092} \ipa{.wjị¹khie¹} `son goût était amer'' (Leilin 04.07B.4-5) à la place de  \tgf{3966}\tgf{4046}. L'identité de prononciation et la similitude graphique entre ces deux caractères a facilité cet emprunt phonétique (\ipa{jiǎjiè}). 


\item 1195 \tgf{1195} \ipa{khie} 2.08 \ptang{kha}{yak} est comparable au  \jpg{qra} ``yak femelle''. Ce mot ne se retrouve pas dans les autres langues ST, mais il est d'une grande importance, car il permet de montrer qu'au moins deux racines distinctes pour ``yak'' existaient en proto-macro-rgyalronguique.


\item 2144 \tgf{2144} \ipa{gie}   1.09 \ptang{ŋga}{difficile} peut se comparer au  \jpg{ɴqa} ``difficile'' et au \tib{dka}. La correspondance de l'initiale est irrégulière (voir \ref{subsubsec:prenasaliseestg}); la préinitiale nasale a pu exceptionnellement sonoriser l'initiale ici. En tangoute comme en japhug, ce verbe statif peut avoir des compléments verbaux:
\newline
\linebreak
\begin{tabular}{llll}
	\tgf{5354}&	\tgf{2857}&	\tgf{3832}&	\tgf{2144}\\
	\tinynb{5354}&	\tinynb{2857}&	\tinynb{3832}&	\tinynb{2144}\\
\end{tabular}
\begin{exe}
\ex \label{ex:tg:difficile}  \vspace{-8pt}
\gll   \ipa{thjɨ²}	\ipa{ŋo²}	\ipa{djị²}	\ipa{gie¹} \\
		cette	maladie guérir[A] difficile \\
\glt Cette maladie est difficile à soigner. (Leilin 06.10A.34)
\end{exe}


\item 333 \tgf{0333} \ipa{ŋwer} 2.71 \ptang{r-ŋwa}{genou} peut être rapproché du  \situ{təmŋá}. Le autres langues rgyalrong, telles que le japhug, ont une autre racine \jpg{tɯ-χpɯm}, mais l'on retrouve des cognats de \tgf{0333} \ipa{ŋwer²} en Rtau (\ipa{rŋə}), en Queyu  (\ipa{ʂŋi⁵⁵}) et en Muya (voir p.\pageref{tab:labiovelairesptg}). Cette racine se retrouve aussi dans les langues naish \nayn{ŋwɤ.ko^H},	\laze{ŋwɑ˩tu˥},	\protona{ŋwa}.


\item 395	\tgf{0395}	\ipa{ŋwe}	2.07 \ptang{ŋwa}{vache} \label{analyse:vache} est apparenté au  \jpg{nɯŋa} ``vache''. On retrouve des cognats dans de nombreuses langues ST, y compris le chinois \zh{牛} ngjuw < *\ipapl{ŋʷɨ} (\citealt[192]{sagart99roc}). Ce mot apparaît en transcription tibétaine dans le nom du roi Minyag  \textit{Ngo.snu'i}, voir la discussion p.\pageref{ex:tib:voma}.



\end{enumerate}

\subsubsection{Tangoute --e :: tibétain  --e :: japhug --i/--e}	\label{subsubsec:correspondance:e:a:a}
Nous reconstruisons *e en pré-tangoute pour les mots appartenant à cette correspondance. L'opposition entre --ie et --e n'est pertinente que devant les labiales, et nous proposons de reconstruire une médiane *--r-- dans les mots à rime --ie devant labiale de cette correspondance, car les exemples dont nous disposons suggèrent cette distribution.
\newline
\linebreak
A. Initiales labiales \label{rimes:02:4:lab}
\begin{enumerate}

\item 2449	\tgf{2449}	\ipa{be}	2.07 \ptang{mbe}{soleil} correspond au  \jpg{ʁmbɣi} ``soleil'' et en naish  \naxi{bi˧}, \protona{bi}. En japhug, ce nom n'est jamais utilisé dans le sens de ``jour'', pour lequel on emploie \jpg{tɯ-sŋi}. On ne peut peut pas lui adjoindre de préfixe numéral. En tangoute, en revanche, il peut signifier ``jour'' en plus de ``soleil'', mais contrairement à \tgf{2440} \ipa{njɨɨ²}, est très rarement précédé par un numéral. Dans les deux langues, c'est ce nom qui sert dans l'expression ``éclipse de soleil'' (\jpg{ʁmbɣɯzɯn}):\footnote{En japhug, la deuxième partie de ce composé est un emprunt au \tib{zin}.}
\newline
\linebreak
\begin{tabular}{llll}
	\tgf{2449}&	\tgf{0153}&	\tgf{1326}&	\tgf{2357}\\
	\tinynb{2449}&	\tinynb{0153}&	\tinynb{1326}&	\tinynb{2357}\\
\end{tabular}
\begin{exe}
\ex \label{ex:tg:soleil}  \vspace{-8pt}
\gll   \ipa{be²}	\ipa{khu¹}	\ipa{kjɨ¹-gjị²} \\
	soleil éclipse \dir{}-se.produire(éclipse) \\
\glt Il y eut une éclipse (Leilin 04.29B.7)
\end{exe}



\item 2878	\tgf{2878} \ipa{biẹ}	1.66 \ptang{S-mbre}{saule} se compare au  \jpg{ʑmbri} ``saule''. La forme 4252 \tgf{4252} \ipa{biə} 1.28 ``saule'' y est également apparentée.
\newline
\linebreak
\begin{tabular}{lllllll}
	\tgf{2878}&	\tgf{4070}&	\tgf{1918}&	\tgf{5625}&	\tgf{2302}&	\tgf{0535}&	\tgf{2064}\\
	\tinynb{2878}&	\tinynb{4070}&	\tinynb{1918}&	\tinynb{5625}&	\tinynb{2302}&	\tinynb{0535}&	\tinynb{2064}\\
\end{tabular}
\begin{exe}
\ex \label{ex:tg:saule}  \vspace{-8pt}
\gll   \ipa{biẹ¹}	\ipa{mej²}	\ipa{mji¹-thwuu¹}	\ipa{ljɨ¹}	\ipa{ɕjij¹}	\ipa{.wor¹} \\
		saule chaton \negat{}-semblable vent en.suivant se.lever \\
\glt (La neige) n'est pas comme les chatons de saule, qui se relèvent avec le vent (Leilin 05.20A.5)
\end{exe}



\item 5390 \tgf{5390} \ipa{phie} 2.08 \ptang{phre}{détacher, libérer} n'a pas de cognat en japhug, mais peut se comparer au \plb{pre¹}{0709} ``détacher''. En tangoute, ce verbe transitif (il apparait avec l'ergatif en \ref{ex:tg:detacher:expliquer}) peut s'employer pour détacher un nœud, libérer (d'une prison) ou, dans un sens dérivé, ``expliquer'':
\newline
\linebreak
\begin{tabular}{llllllll}
	\tgf{0004}&	\tgf{0092}&	\tgf{5604}&	\tgf{5113}&	\tgf{2019}&	\tgf{3183}&	\tgf{5390}&	\tgf{5113}\\
	\tinynb{0004}&	\tinynb{0092}&	\tinynb{5604}&	\tinynb{5113}&	\tinynb{2019}&	\tinynb{3183}&	\tinynb{5390}&	\tinynb{5113}\\
\end{tabular}
\begin{exe}
\ex \label{ex:tg:detacher:expliquer}  \vspace{-8pt}
\gll   \ipa{la²mja¹}	\ipa{dʑjɨ.wji¹}	\ipa{thja¹}	\ipa{.wo²}	\ipa{phie²}	\ipa{.wji¹} \\
		tante \erg{} ce sens expliquer faire[A] \\
\glt Sa tante lui en expliqua le sens. (Leilin 06.25B.4-5)
\end{exe}
En tangoute comme en birman, le verbe \tgf{5390} \ipa{phie²} a un dérivé anticausatif par prénasalisation 2162 \tgf{2162} \ipa{bie} 2.08 (pré-tangoute *mbre, \bir{pre²} ``être détaché, s'ouvrir''):\footnote{Voir \citet{gong88alternations}.}
\newline
\linebreak
\begin{tabular}{llllllllll}
	\tgf{2019}&	\tgf{0433}&	\tgf{2627}&	\tgf{2889}&	\tgf{5689}&	\tgf{2162}&	\tgf{3951}&	\tgf{2049}&	\tgf{2590}&	\tgf{2797}\\
	\tinynb{2019}&	\tinynb{0433}&	\tinynb{2627}&	\tinynb{2889}&	\tinynb{5689}&	\tinynb{2162}&	\tinynb{3951}&	\tinynb{2049}&	\tinynb{2590}&	\tinynb{2797}\\
\end{tabular}
\begin{exe}
\ex \label{ex:tg:detache:anticaus}  \vspace{-8pt}
\gll   \ipa{thja¹}	\ipa{bju¹}	\ipa{ljɨ̣²}	\ipa{meej²}	\ipa{ɣa¹}	\ipa{bie²}	\ipa{thu¹-sjã²}	\ipa{.wjɨ²-lho} \\
		cela \instr{} terre grotte porte s'ouvrir Du.Xie \dir{}-sortir \\
\glt Ainsi, la porte de la grotte souterraine s'ouvrit, et Du Xie (\zh{杜燮}) put sortir (Leilin 06.11B.2-3)
\end{exe}


\item 4825	\tgf{4825} \ipa{me}  2.07 \ptang{me}{dormir} est peut-être apparenté avec la première syllabe du composé \tib{rmi.lam} ``rêve''. Cette comparaison est toutefois incertaine. Une connexion avec le \jpg{rma} ``passer une nuit chez quelqu'un" est également envisageable.

\end{enumerate}

B. Initiales dentales et vélaires \label{rimes:02:4:dent}


\begin{enumerate}

\item 5957 \tgf{5957} \ipa{tser} 1.77 \ptang{r-tse}{vendre} est apparenté au  \jpg{ntsɣe} ``vendre''.
\newline
\linebreak
\begin{tabular}{lllllllll}
	\tgf{1675}&	\tgf{2541}&	\tgf{4574}&	\tgf{5447}&	\tgf{3354}&	\tgf{5957}&	\tgf{3818}&	\tgf{5113}&	\tgf{5993}\\
	\tinynb{1675}&	\tinynb{2541}&	\tinynb{4574}&	\tinynb{5447}&	\tinynb{3354}&	\tinynb{5957}&	\tinynb{3818}&	\tinynb{5113}&	\tinynb{5993}\\
\end{tabular}
\begin{exe}
\ex \label{ex:tg:vendre}  \vspace{-8pt}
\gll  \ipa{tɕjo¹} \ipa{dzjwo²}	\ipa{mjɨ¹}	\ipa{do²}	\ipa{ɣie¹}	\ipa{tser¹}	\ipa{mjijr²}	\ipa{wji¹}	\ipa{kha¹} \\
	indiquer homme autre \allat{} force vendre \nmls{}:A faire[A] milieu \\
\glt Cet homme, bien qu'il travaille pour les autres (litt. : bien qu'il vende sa force à d'autres gens) (Cixiaozhuan, 26.2, \citealt[80]{jacques07textes})
\end{exe}


\item 2664	\tgf{2664}	\ipa{dze} 1.08 \ptang{ndze}{vie} est comparable au  \jpg{tɯ-tsi} ``vie, période d'une vie'' et au \tib{tsʰe}  ``temps, moment''. La correspondance des initiales est irrégulière  et inexplicable.

\item  0809 \tgf{0809}  	\ipa{rer} 	1.77 \ptang{re}{fil, corde} est cognat du 	\jpg{tɤ-ri} ``fil''. En tangoute, ce nom désigne en particulier la corde d'un arc ou d'un instrument de musique.


\item 5868 \tgf{5868} \ipa{khie} 2.08 \ptang{khe}{riz}\footnote{On ne peut pas reconstruire de *r ici avec certitude car l'opposition entre --ie et --e, comme nous l'avons mentionné plus haut, n'est pas distinctive devant les vélaires.} peut être rapproché du  \situ{smaikhrí} ``millet'' et du \tib{kʰre} ``millet''. En tangoute, ce mot désigne le riz cuit, comme le montre l'exemple \ref{ex:tg:mettre.dans} p.\pageref{ex:tg:mettre.dans}. Voir \citet{sagart03cereals} pour une analyse de cet étymon dans une perspective sino-tibétaine.
\end{enumerate}

\subsubsection{Autres correspondances}	\label{subsubsec:correspondance:e:autres}
On trouve également un certain nombre de correspondances isolées qui pourraient toutefois refléter des cognats authentiques ou des emprunts.
\begin{enumerate}


\item 800	\tgf{0800}	\ipa{dzeej}	1.37	 ``combattre, se disputer qqch'' est comparable au	\tib{ɴdziŋ} ``combattre''. Il peut s'agir d'un emprunt comme d'un cognat, mais la première hypothèse semble plus probable, car ce verbe n'est visiblement pas attesté ailleurs en macro-rgyalronguique, et le --ing du tibétain correspond habituellement à --jij.

\item 2620 \tgf{2620} \ipa{njwi} 2.10 ``pouvoir'' pourrait être rapproché du \jpg{nɤz} et du \tib{nus} `oser', mais on attendrait plutôt une rime --jwɨ ou --wə.

\item 1327	\tgf{1327}	\ipa{phjii}	1.14 ``voler'' pourrait potentiellement être apparenté au	\tib{ɴpʰur}, mais la correspondance vocalique attendue serait avec les rimes --(j)ur ou --wər / --jwɨr.
\end{enumerate}
Les formes suivantes ont été comparées entre le rgyalrong et le tangoute  dans des travaux antérieurs, mais les correspondances sont insatisfaisantes, car la rime --e du tangoute ne correspond normalement pas à des rimes à finales occlusives en japhug. Il s'agit vraisemblablement de ressemblances fortuites, mais nous les présentons ici tout de même dans l'éventualité où d'autres exemples du même type seraient mis au jour:
\begin{itemize}

\item \tgz{1921}  et	\jpg{asɯɣ} ``serré''

\item	\tgz{0250} et	\jpg{qambɯt} ``sable''
\item	\tgz{0383} et	\jpg{alɯlɤt} ``se battre''

\item	\tgz{1697} et	\jpg{tɯ-ʁar} ``aile''

\end{itemize}
\subsection{Voyelle a} \label{subsec:voyelle.a}

Les rimes du \ipa{shè} n°4\footnote{Le \ipa{shè} n°3 n'est pas inclus ici, car il ne comprend que des emprunts au chinois.} sont reconstruites par Gong Hwangcherng avec les voyelles a. Comme le montre le tableau suivant, tous les spécialistes du tangoute reconstruisent cette voyelle pour toutes les rimes appartenant à ce \ipa{shè} (voir \ref{tab:she4}).

\begin{table}
\captionb{Reconstructions du \ipa{shè} n°4}\label{tab:she4}
\resizebox{\columnwidth}{!}{
\begin{tabular}{lllllllll} \toprule
rime&ton 1&ton 2&Sofronov1&Sofronov2&Nishida&Li&Gong&Arakawa\\
17&	1.17&	2.14&	\ipa{a}&	\ipa{a}&	\ipa{ɑɦ}&	\ipa{a̠}&	\ipa{a}&	\ipa{a}\\	
18&	1.18&	2.15&	\ipa{â}&	\ipa{â}&	\ipa{a}&	\ipa{ǐa}&	\ipa{ia}&	\ipa{ya}\\	
19&	1.19&	2.16&	\ipa{i̯a}&	\ipa{i̯a}&	\ipa{ǐa}&	\ipa{ǐa̠}&	\ipa{ja}&	\ipa{aː}\\	
20&	1.20&	2.17&	\ipa{a+C}&	\ipa{a}&	\ipa{aɦ}&	\ipa{ɑ̠}&	\ipa{ja}&	\ipa{aː}\\	
21&	1.21&	2.18&	\ipa{i̯a+C}&	\ipa{i̯aɯ}&	\ipa{ǐaɦ}&	\ipa{ǐɑ}&	\ipa{jaa}&	\ipa{yaː}\\	
22&	1.22&	2.19&	\ipa{a+C}&	\ipa{aɯ}&	\ipa{ɑɯ}&	\ipa{ǔɑ}&	\ipa{aa}&	\ipa{a’}\\	
23&	&	2.20&	\ipa{â+C}&	\ipa{âɯ}&	\ipa{aɯ}&	\ipa{ɑr}&	\ipa{iaa}&	\ipa{ya’}\\	
24&	1.23&	2.21&	\ipa{i̯a+C}&	\ipa{i̯aɯ}&	\ipa{ɑ}&	\ipa{ɑ}&	\ipa{jaa}&	\ipa{aː’}\\	
66&	1.63&	2.56&	\ipa{ạ}&	\ipa{ạ}&	\ipa{ɑ̣}&	\ipa{a̠̣}&	\ipa{ạ}&	\ipa{aq}\\
67&	1.64&	2.57&	\ipa{i̯ạ}&	\ipa{i̯ạ}&	\ipa{ạ}&	\ipa{ǐa̠̣}&	\ipa{jạ}&	\ipa{aːq}\\
85&	1.80&	2.73&	\ipa{ạ}&	\ipa{ạ}&	\ipa{ɑr}&	\ipa{ɑ̣}&	\ipa{ar}&	\ipa{ar}\\
86&	1.81&	&	\ipa{â}&	\ipa{â}&	\ipa{ǐɑr}&	\ipa{ar}&	\ipa{iar}&	\ipa{yar}\\
87&	1.82&	2.74&	\ipa{i̯ạ}&	\ipa{i̯ạ}&	\ipa{ar}&	\ipa{ǎr}&	\ipa{jwar}&	\ipa{aːr}\\
88&	1.83&	&	\ipa{ạ+C}&	\ipa{ạɯ}&	\ipa{ar²}&	\ipa{a̠̣}&	\ipa{aar}&	\ipa{ar’}\\
89&	&	2.75&	\ipa{i̯ạ+C}&	\ipa{i̯ạɯ}&	\ipa{ǐar}&	\ipa{ạu}&	\ipa{jaar}&	\ipa{yar’}\\
\bottomrule
\end{tabular}}
\end{table}
Gong Hwangcherng reconstruit ici de la même façon les rimes 19 et 20 d'une part, et 21 et 24 d'autre part, tandis que les autres spécialistes ont choisi des solutions différentes. La reconstruction de Gong est motivée par le fait que ces rimes sont en distribution complémentaire par rapport aux initiales (\citealt{gong94length}, \citealt[147]{gong02a}).

{\small
\begin{longtable} {lllllll}
\captionb{Comparaison  des étymons en --a du tangoute avec le japhug et le tibétain.}\label{tab:comparaisons:a} \\
\toprule
&\multicolumn{2}{c}{tangoute}& &  japhug & sens  &tibétain\\
\midrule
\endfirsthead
\tinynb{527}& \tgf{0527}&	\ipa{.wjạ}	&\tinynb{1.64}	& \ipa{zbɤβ}	&goitre&\\
\tinynb{1894}& \tgf{1894}&	\ipa{.jar}	&\tinynb{1.82}	& \ipa{tɤ-rʑaβ}	&épouse&\\
\tinynb{5755}& \tgf{5755}&	\ipa{.jar}	&\tinynb{1.82}	& \ipa{rjap} (situ)	&être debout&\\
\tinynb{4935}& \tgf{4935}&	\ipa{ɣa}	&\tinynb{1.17}	& \ipa{taqaβ}&aiguille	&kʰab\\
\tinynb{1084}& \tgf{1084}&	\ipa{ɣạ}	&\tinynb{2.56}	& \ipa{sqi}	&dix&\\
\tinynb{3008}& \tgf{3008}&	\ipa{ɣja}	&\tinynb{2.16}	& \ipa{fkaβ}&couvrir	&bkab\\
\tinynb{4003}& \tgf{4003}&	\ipa{khja}	&\tinynb{2.17}	& \ipa{kaβ}	&puiser de l'eau&\\
\tinynb{2171}& \tgf{2171}&	\ipa{tha}	& 	&  & forcer, pousser à bout&ɴtʰab \\
\tinynb{5731}& \tgf{5731}&	\ipa{nạ}	&\tinynb{1.63}	& \ipa{tɯ-ɕnaβ}	&mucus nasal&snabs\\
\tinynb{5440}& \tgf{5440}&	\ipa{tɕhja}	&\tinynb{1.19}	& \ipa{tʂaβ}	&faire rouler&\\
\tinynb{45 }& \tgf{0045} & \ipa{zar} &\tinynb{1.80} & \ipa{mɤrtsaβ} &piquant &\\
\midrule
\tinynb{4602}& \tgf{4602}&	\ipa{.jar}	&\tinynb{1.82}	& \ipa{kɯrcat}	&huit&brgʲad\\
\tinynb{4585}& \tgf{4585}&	\ipa{.wja}	&\tinynb{1.19}	& \ipa{mɯjphɤt}	&vomir&\\
\tinynb{2467}& \tgf{2467}&	\ipa{.wjạ}	&\tinynb{1.64}	& \ipa{ɣɤwɤt}	&fleurir&\\
\tinynb{596}& \tgf{0596}&	\ipa{dzja}	&\tinynb{1.20}	& \ipa{ndzɤt}	&grandir&\\
\tinynb{4480}& \tgf{4480}&	\ipa{kar}	&\tinynb{2.73}	& \ipa{qɤt}	&séparer&\\
\tinynb{2436}& \tgf{2436}&	\ipa{mjaa}	&\tinynb{1.23}	& \ipa{sɯmat}	&fruit&\\
\tinynb{2546}& \tgf{2546}&	\ipa{njạ}	&\tinynb{1.64}	& 	&divinité&\\
\tinynb{3136 }&\tgf{3136}	&\ipa{pjạ} &\tinynb{1.64} &\ipa{ɕphɤt} &rafistoler & \\
\tinynb{2475}& \tgf{2475}&	\ipa{phia}	&\tinynb{1.18}	& \ipa{prɤt}	&couper, casser&\\
\tinynb{1715}& \tgf{1715}&	\ipa{rjar}	&\tinynb{1.82}	& \ipa{rɤt}	&écrire&ɴbrad\\
\tinynb{4225}& \tgf{4225}&	\ipa{sja}	&\tinynb{1.20}	& \ipa{sat}	&tuer&gsod bsad\\
\midrule
\tinynb{811}& \tgf{0811}&	\ipa{.jaar}	&\tinynb{2.75}	& \ipa{rʑaʁ}	&jour&ʑag\\
\tinynb{4546}& \tgf{4546}&	\ipa{.jaar}	&\tinynb{2.75}	& 	&poulet& \\
\tinynb{294}& \tgf{0294}&	\ipa{.wa}	&\tinynb{1.17}	& \ipa{paʁ}	&cochon&pʰag\\
\tinynb{5170}& \tgf{5170}&	\ipa{.wạ}	&\tinynb{1.63}	& \ipa{tɯ-rpaʁ}	&épaule&pʰrag \\
\tinynb{1360	}& \tgf{1360}&	\ipa{.wa}	&\tinynb{1.17}	& \ipa{anbaʁ}	&se cacher& \\
\tinynb{2200}& \tgf{2200}&	\ipa{ba}	&\tinynb{1.17}	& \ipa{ɣɤrʁaʁ}	&chasser&\\
\tinynb{4567}& \tgf{4567}&	\ipa{bạ}	&\tinynb{2.56}	& \ipa{tɤ-jwaʁ}	&feuille&\\
\tinynb{4052}& \tgf{4052}&	\ipa{dạ}	&\tinynb{2.56}	& \ipa{mɯɕtaʁ}	&froid&\\
\tinynb{3544}& \tgf{3544}&	\ipa{dʑiaa}	&\tinynb{2.21}	& \ipa{ndʑaʁ}	&nager&\\
\tinynb{4680}& \tgf{4680}&	\ipa{khia}	&\tinynb{2.15}	& \ipa{qraʁ}	&soc&\\
\tinynb{1752}& \tgf{1752}&	\ipa{kwạ}	&\tinynb{2.56}	& \ipa{qaʁ}	&houe&\\
\tinynb{630}& \tgf{0630}&	\ipa{la}	&\tinynb{1.17}	& \ipa{taʁ}	&tisser&ɴtʰag btags\\
\tinynb{4}& \tgf{0004}&	\ipa{la}	&\tinynb{2.14}	& \ipa{tɤ-ɬaʁ} &tante	&\\
\tinynb{301}& \tgf{0301}&	\ipa{lạ}	&\tinynb{1.63}	& \ipa{qaliaʁ}	&aigle&\\
\tinynb{3485}& \tgf{3485}&	\ipa{lạ}	&\tinynb{1.63}	& \ipa{tɯ-jaʁ}	&main&lag\\
\tinynb{3192}& \tgf{3192}&	\ipa{laa}	&\tinynb{1.21}	& \ipa{jaʁ}	&épais&\\
\tinynb{4663}& \tgf{4663}&	\ipa{lha}	&\tinynb{1.17}	& \ipa{}	&éteindre&brlag \\
\tinynb{4820}& \tgf{4820}&	\ipa{mạ}	&\tinynb{1.63}	& \ipa{tɯ-nmaʁ}&gendre	&mag.pa\\
\tinynb{4693}& \tgf{4693}&	\ipa{na}	&\tinynb{1.17}	& \ipa{rnaʁ}	&profond&\\
\tinynb{176}& \tgf{0176}&	\ipa{njaa}	&\tinynb{1.21}	& \ipa{ɲaʁ}	&noir&nag.po\\
\tinynb{4532}& \tgf{4532}&	\ipa{pạ}	&\tinynb{2.56}	& \ipa{ɕpaʁ}	&avoir soif&\\
\tinynb{3708 }&\tgf{3708} &\ipa{phja}	&\tinynb{1.20}& \ipa{phaʁ}	&casser&\\
\tinynb{3936}& \tgf{3936}&	\ipa{pha}	&\tinynb{1.17}	& \ipa{ɯ-phaʁ}	&moitié&\\
\tinynb{5921 }&\tgf{5921} &\ipa{zar} &\tinynb{2.73} & \ipa{nɤzraʁ} &avoir honte&\\
\midrule
\tinynb{791}& \tgf{0791}&	\ipa{ljạ}	&\tinynb{1.64}	& \ipa{tɯ-laz}	&front&\\
\tinynb{5702}& \tgf{5702}&	\ipa{mjaa}	&\tinynb{1.23}	& \ipa{tɯ-ɣmaz}	&blessure&rma\\
\tinynb{5181}& \tgf{5181}&	\ipa{tsar}	&\tinynb{1.80}	& \ipa{qartshaz}	&cerf&\\
\tinynb{2539}& \tgf{2539}&	\ipa{kjạ}	&\tinynb{1.64}	& \ipa{nɤscɤr}	&être saisi de frayeur&\\
\tinynb{4526}& \tgf{4526} & \ipa{ma} &\tinynb{2.14} & \ipa{tamar} &beurre&mar\\
\tinynb{1530}& \tgf{1530}&	\ipa{mja}	&\tinynb{1.20}	& \ipa{smar}	&fleuve&\\
\tinynb{5370}& \tgf{5370}&	\ipa{pjạ}	&\tinynb{1.64}	& \ipa{}	&paume&sbar.mo\\
\midrule
\tinynb{1391}& \tgf{1391}&	\ipa{ba}	&\tinynb{1.17}	& \ipa{tɤ-mbɣo}	&sourd&\\
\tinynb{5528}& \tgf{5528}&	\ipa{bar}	&\tinynb{1.80}	& \ipa{tɤ-rmbɣo}	&tambour&\\
\tinynb{975 }& \tgf{0975}&	\ipa{par}	&\tinynb{1.80}	& \ipa{jpɣom}	&geler&\\
\midrule
\tinynb{573}& \tgf{0573}&	\ipa{na}	&\tinynb{1.17}	&\ipa{khɯna}	& chien&\\
\tinynb{2098}& \tgf{2098}&	\ipa{ŋa}	&\tinynb{2.14}	& \ipa{aʑo}	&moi&ŋa\\
\tinynb{5523}& \tgf{5523}&	\ipa{rjar}	&\tinynb{1.82}	& \ipa{ra}	&devoir&\\
\tinynb{5049}& \tgf{5049}&	\ipa{.wja}	&\tinynb{1.19}	& \ipa{a-wa}	&père &a.pʰa\\
\bottomrule
\end{longtable}
}



On observe dans le tableau ci dessus que les rimes en --a du tangoute correspondent très régulièrement aux rimes --aC du tibétain ou du rgyalrong, où --C représente une occlusive (*--p, *--t, *--k/q) un *--s ou un *--r. Nous avons expliqué dans la section précédente que les anciens *--a et *--ja du pré-tangoute étaient devenus respectivement --e et --ji en tangoute attesté. Ainsi, il s'est produit un changement en chaîne :
\newline

\begin{tabular} {lll} 
*--a &> &--e\\
*--aC &> &--a\\
\end{tabular}
\linebreak

Le *--r final du pré-tangoute ne laisse pas de trace dans les mots en --a, contrairement à d'autres rimes, où il est indirectement préservé (voir p.\pageref{tab:cycle2finalerjaphug}).


La rime --ia du tangoute correspond régulièrement à *--raC ou *\ipapl{--rɐC} en proto-rgyalrong, comme le montrent les exemples  \tgf{2475} \ipa{phia¹} :: \ipa{prɤt} ``couper'', \tgf{4680} \ipa{khia²}	:: \ipa{qraʁ} ``soc'' ainsi que l'emprunt au tibétain ancien 4851 \tgf{4851} \ipa{bia²} ``riz cuit, gruau de riz'' (\tib{ɴbras}). On reconstruira donc une médiane *--r-- ici en pré-tangoute (voir également p.\pageref{rimes:02:4:lab}).

On observe par ailleurs d'autres correspondances, dont l'une avec le proto-japhug *--\ipapl{aˠŋ}, l'autre avec le japhug --a. Comme le *--a du proto-japhug est normalement devenu --e, ces cas où le --a du tangoute correspond à --a dans les autres langues demandent des explications alternatives.
\subsubsection{Tangoute --a :: tibétain  --ab :: japhug \ipapl{--aβ/--ɤβ}}	\label{subsubsec:correspondance:a:ap}
On reconstruira *--ap en pré-tangoute pour les mots appartenant à cette correspondance.
\newline
\linebreak
A. Initiales labiales et dentales \label{rimes:04:1:lab}
\newline
\begin{enumerate}

\item 527 \tgf{0527} \ipa{.wjạ}	1.64	 \ptang{C-S-pjap}{tumeur} est comparable au  \jpg{zbɤβ}	``goître'', \jpg{tɯ-ɣmbɤβ} ``ulcère purulent'' et le dérivé verbal de ce dernier \ipa{nɯɣmbɤβ}  ``enfler''. Le préfixe *S-- du pré-tangoute, comme le z-- en japhug, doit être un ancien préfixe de nominalisation oblique. Il faut partir d'un verbe statif ``enfler'' (attesté en  \situ{mbóp}), auquel on adjoint le préfixe de nominalisation oblique qui signifie alors ``l'endroit enflé''. 

\item  2171 \tgf{2171} \ipa{tha} \ptang{thap}{forcer, pousser à bout} et 1394 \tgf{1394} \ipa{tha} 1.22 ``oppresser'' peuvent potentiellement être rapprochés du  \tib{ɴtʰab} ``combattre'', même si la sémantique n'est pas parfaite. Si ces formes sont cognates, le sens du tibétain doit dériver de ``presser''.


\begin{tabular}{llllllllll}
\tgf{3627} & 	\tgf{3627} & 	\tgf{1796} & 	\tgf{3119} & 	\tgf{5871} & 	\tgf{5093} & 	\tgf{4633} & 	\tgf{1139} & 	\tgf{2862} & 	\tgf{5447}  \\
\tinynb{3627} & 	\tinynb{3627} & 	\tinynb{1796} & 	\tinynb{3119} & 	\tinynb{5871} & 	\tinynb{5093} & 	\tinynb{4633} & 	\tinynb{1139} & 	\tinynb{2862} & 	\tinynb{5447}  \\
\tgf{1394} &\tgf{4225} & 	\tgf{4481} & 	\tgf{0749} \\
\tinynb{1394} &\tinynb{4225} & 	\tinynb{4481} & 	\tinynb{0749} \\
\end{tabular}
\begin{exe}
\ex \label{ex:tg:faire.tomber}  \vspace{-8pt}
\gll  \ipa{nji²nji²} 	\ipa{tɕhjụ¹.ji¹} 	\ipa{zeew²} 	\ipa{tɕhjiw¹thwər¹} 	\ipa{.jij¹} 	\ipa{nji¹} 	\ipa{do²} 	\ipa{tha¹} 	\ipa{sja¹} 	\ipa{ɕjɨ¹-phji¹}  \\
		 en.secret Chu.Ni envoyer Zhao.Dun \gen{} maison \textit{all} presser tuer aller[B]-causer[A]\\
\glt Il envoya en secret Chu Ni pour tuer Zhao Dun. (Leilin, 03.95A.6-7)
\end{exe}

\item 5731 \tgf{5731} \ipa{nạ} 1.63 \ptang{S-nap}{mucus nasal}  est cognat du  \jpg{tɯ-ɕnaβ} et du  \tib{snabs} de même sens. Il est probable que ce mot est un composé comprenant le nom ``nez" (\jpg{tɯ-ɕna}, \tib{sna}), mais sa formation remonte au moins à l'ancêtre commun du birmo-qianguique et du tibétain.


\item 45 \tgf{0045} \ipa{zar} 1.80 \ptang{C-r-tsap}{amer}  est apparenté au  \jpg{mɤrtsaβ} ``piquant''. Ce verbe statif a peut-être une forme dérivée 512 \tgf{0512} \ipa{tsar} \ptang{r-tsap}{xanthoxyle}; toutefois, ce nom pourrait aussi être à rapprocher du  \jpg{tɕɣom} ``xanthoxyle'' et reconstruit *\ipapl{r-tsaˠm} (voir section \ref{subsubsec:correspondance:a:agg}).
		
\end{enumerate}
B. Initiales palatales \label{rimes:04:1:pal}
\newline
\begin{enumerate}



\item 5440 \tgf{5440} \ipa{tɕhja} 1.19	 \ptang{thrap}{détruire, faire tomber} correspond au  \jpg{tʂaβ} ``faire tomber''. Les sens du verbe tangoute et du verbe japhug sont très proches: les exemples suivants montrent qu'ils peuvent tous deux s'employer pour désigner l'action de faire tomber un objet long dressé:
\begin{exe}
\ex \label{ex:jpg:faire.tomber}  \vspace{-8pt}
\gll   \ipa{qale}	\ipa{ta-βzu}	\ipa{tɕe}	\ipa{chɯ-tʂaβ}	\ipa{ŋu}	\ipa{tɕe} \\
		vent \aor{}-faire \conj{} \impf{}:aval-faire.tomber \nonps{}:être \conj{} \\
\glt S'il y a du vent, il les fait se coucher (à propos des épis d'orge)  (rtsampa 72-73)
\end{exe}
L'exemple tangoute suivant est tiré des phrases citées dans \citet{lifw97}:
\newline
\linebreak
\begin{tabular}{lllllll}
	\tgf{2191}&	\tgf{0290}&	\tgf{2302}&	\tgf{2466}&	\tgf{5814}&	\tgf{4456}&	\tgf{5440}\\
	\tinynb{2191}&	\tinynb{0290}&	\tinynb{2302}&	\tinynb{2466}&	\tinynb{5814}&	\tinynb{4456}&	\tinynb{5440}\\
\end{tabular}
\begin{exe}
\ex \label{ex:tg:faire.tomber}  \vspace{-8pt}
\gll   \ipa{dzjọ¹}	\ipa{sju²}	\ipa{ljɨ¹}	\ipa{lhjwɨ²}	\ipa{phu²}	\ipa{ljịj²}	\ipa{tɕhja¹} \\
		comparaison être.semblable vent brusquement arbre grand faire.tomber \\
\glt Tel le vent qui fait s'abattre violemment les grands arbres (Suvarna 10.346)
\end{exe}
En japhug, ont trouve également une forme anticausative \jpg{ndʐaβ} qui n'a pas d'équivalent en tangoute (on attendrait *\ipapl{dʑja}). En revanche, on trouve deux verbes à initiale non aspirée qui sont apparentés à \tgf{5440} \ipa{tɕhja¹} : 1208 \tgf{1208} \ipa{tɕia} 1.18 et 5375  \tgf{5375} \ipa{tɕja} 1.19. Ce sont des verbes intransitifs signifiant ``s'écrouler, tomber'' comme l'illustre l'exemple suivant:
\newline
\linebreak
\begin{tabular}{llllllllll}
	\tgf{1326}&	\tgf{0330}&	\tgf{1796}&	\tgf{3830}&	\tgf{4950}&	\tgf{5981}&	\tgf{3355}&	\tgf{0383}&	\tgf{1452}&	\tgf{5375}\\
\tinynb{1326}&	\tinynb{0330}&	\tinynb{1796}&	\tinynb{3830}&	\tinynb{4950}&	\tinynb{5981}&	\tinynb{3355}&	\tinynb{0383}&	\tinynb{1452}&	\tinynb{5375}\\
\end{tabular}
\begin{exe}
\ex \label{ex:tg:tomber}  \vspace{-8pt}
\gll   \ipa{kjɨ¹-mjiij¹}	\ipa{tɕhjụ¹}	\ipa{njij²}	\ipa{rjir²}	\ipa{.a-lə²le²}	\ipa{nja¹-tɕja¹} \\
		\dir{}-rêver Chu roi \comit{} \dir{}-se.battre \dir{}-tomber \\
\glt (Le prince de Jin) rêva qu'il s'était battu avec le roi de Chu, et qu'il (le prince de Jin) était tombé (Leilin 06.15B.4-5)
\end{exe}
On reconstruirait *trap en pré-tangoute pour ces deux verbes; ici le passage à la forme intransitive s'accompagne d'une perte d'aspiration au lieu d'une prénasalisation. La différence entre les rimes 1.18 et 1.19 n'est pas explicable, mais on peut douter que ces deux rimes étaient distinctives devant les palatales. La médiane en pré-tangoute *--r-- peut difficilement à la fois affriquer l'occlusive initiale et contribuer à la formation d'une médiane --i-- dans  \tgf{1208} \ipa{tɕia¹}.



\item 1894 \tgf{1894} \ipa{.jar}	1.82 \ptang{r-jap}{bru, belle-fille} est apparenté au \jpg{tɤ-rʑaβ} ``épouse''. Ce nom a également un dérivé 2904 \tgf{2904} \ipa{.jar} 2.74 ``se marier (à propos d'une femme: devenir une belle-fille)'' qui apparaît le plus souvent avec un verbe de mouvement:
\newline
\linebreak
\begin{tabular}{llllllllll}
	\tgf{2019}&	\tgf{2455}&	\tgf{2129}&	\tgf{5545}&	\tgf{4401}&	\tgf{1918}&	\tgf{2904}&	\tgf{5049}&	\tgf{0092}&	\tgf{5604}\\
	\tinynb{2019}&	\tinynb{2455}&	\tinynb{2129}&	\tinynb{5545}&	\tinynb{4401}&	\tinynb{1918}&	\tinynb{2904}&	\tinynb{5049}&	\tinynb{0092}&	\tinynb{5604}\\
	\tgf{5113}&	\tgf{3510}&	\tgf{5880}&	\tgf{2904}&	\tgf{4481}&	\tgf{0749}&	\tgf{3357}&&&\\
	\tinynb{5113}&	\tinynb{3510}&	\tinynb{5880}&	\tinynb{2904}&	\tinynb{4481}&	\tinynb{0749}&	\tinynb{3357}&&&\\
\end{tabular}
\begin{exe}
\ex \label{ex:tg:se.marier}  \vspace{-8pt}
\gll   \ipa{thja¹}	\ipa{gji²bjij²}	\ipa{kjur²}	\ipa{zow²}	\ipa{mji¹-.jar²}	\ipa{.wja¹}	\ipa{mja¹}	\ipa{dʑjɨ.wji¹}	\ipa{.jijr¹}	\ipa{ŋwu²}	\ipa{.jar²}	\ipa{ɕjɨ¹}	\ipa{phji¹}	\ipa{kiẹj²} \\
	cette épouse volonté tenir \negat{}-se.marier père mère  \erg{} forcer \conj{} se.marier aller[B] causer[A] vouloir \\
\glt Son épouse décida de ne jamais se remarier. Les parents de celle-ci voulurent la forcer à se marier, (Leilin 06.06A.3)
\end{exe}
L'exemple ci-dessus montre que \tgf{2904} \ipa{.jar²} peut servir tout seul de prédicat et être précédé d'une négation, ce qui prouve qu'il s'agit d'un verbe. Cet exemple est toutefois exceptionnel, et \tgf{2904} \ipa{.jar} apparaît normalement suivi du verbe ``aller''.


\item 5755 \tgf{5755} \ipa{.jar} 1.82 ``être debout'' (PT *r-jap) correspond au  \situ{rjap} ``être debout, se lever''. Cette racine verbale est très répandue dans la famille ST, voir \citet[339]{matisoff03} pour une liste de cognats.
\end{enumerate}

B. Initiales vélaires \label{rimes:04:1:vel}
\newline
\begin{enumerate}


\item 4003 \tgf{4003} \ipa{khja} 2.17	\ptang{khjap}{puiser de l'eau} peut se comparer au \jpg{kaβ} ``puiser de l'eau'', au \bir{khap} et au chinois \zh{汲} kip < *\ipa{kɨp}. C'est un verbe transitif, qui peut apparaître avec \tgf{3058} \ipa{zjɨɨr²} pour objet. Ce verbe a pour dérivé 2004 \tgf{2004} \ipa{khja} 2.17 ``puits'' (comparer au \jpg{sakaβ}).
\newline
\linebreak
\begin{tabular}{llllll}
	\tgf{5442}&	\tgf{2513}&	\tgf{1245}&	\tgf{2455}&	\tgf{4003}&	\tgf{0749}\\
	\tinynb{5442}&	\tinynb{2513}&	\tinynb{1245}&	\tinynb{2455}&	\tinynb{4003}&	\tinynb{0749}\\
\end{tabular}
\begin{exe}
\ex \label{ex:tg:puiser}  \vspace{-8pt}
\gll   \ipa{ɕiə¹}	\ipa{ju²}	\ipa{jij¹}	\ipa{gji²}	\ipa{kha²}	\ipa{phji¹} \\
		Shi souvent soi-même épouse puiser envoyer[A] \\
\glt Shi (\zh{詩}) y envoyait souvent son épouse pour puiser (de l'eau) (Cixiaozhuan 2.6, \citealt[11]{jacques07textes})
\end{exe}


\item 4935 \tgf{4935} \ipa{ɣa} 1.17 \ptang{C-kap}{aiguille} correspond au \jpg{taqaβ}, au naish \naxi{ko˩},	\nayn{ʁu˧},	\laze{u˩},	\protona{NkaC} et au \tib{kʰab} de même sens. Cet étymon se retrouve dans un nombre très large de langues, dont le \bir{ʔap} (\plb{g-ràp}{0382}) et le chinois \zh{針} \textit{tsyim}. Un exemple textuel est disponible p.\pageref{ex:tg:planter}.



\item 3008 \tgf{3008}	\ipa{ɣja} 2.16 \ptang{C-kjap}{couvrir} peut se comparer au \jpg{fkaβ} ``couvrir'' et au \tib{ɴgebs, bkab} (la forme japhug pourrait être un emprunt au tibétain). Ce verbe s'emploie notamment à propos des habits, comme en chinois:
\newline
\linebreak
\begin{tabular}{llllllllll}
	\tgf{3045}&	\tgf{5964}&	\tgf{0824}&	\tgf{5417}&	\tgf{3444}&	\tgf{1746}&	\tgf{5880}&	\tgf{1326}&	\tgf{2321}&	\tgf{5037}\\
	\tinynb{3045}&	\tinynb{5964}&	\tinynb{0824}&	\tinynb{5417}&	\tinynb{3444}&	\tinynb{1746}&	\tinynb{5880}&	\tinynb{1326}&	\tinynb{2321}&	\tinynb{5037}\\
	\tgf{4401}&	\tgf{4729}&	\tgf{3008}&	\tgf{3818}&	\tgf{1139}&	\tgf{1196}&&&&\\
	\tinynb{4401}&	\tinynb{4729}&	\tinynb{3008}&	\tinynb{3818}&	\tinynb{1139}&	\tinynb{1196}&&&&\\
\end{tabular}
\begin{exe}
\ex \label{ex:tg:couvrir}  \vspace{-8pt}
\gll   \ipa{tshew¹tsha²}	\ipa{tɕhjɨ²rjar²}	\ipa{dzew²ljoor¹}	\ipa{ŋwu²}	\ipa{kjɨ¹-tɕjɨɨr²}	\ipa{bjɨr¹}	\ipa{zow²}	\ipa{lhwu¹}	\ipa{ɣja²}	\ipa{mjijr²}	\ipa{.jij¹}	\ipa{ɕjwo¹} \\
	Cao.Cao immédiatement tromper \conj{} \dir{}-surpris couteau tenir habit couvrir \nmls{}:A \antierg{} chasser \\
\glt Cao Cao fit immédiatement semblant d'être réveillé en sursaut, se saisit de son couteau et chassa ceux qui l'avaient couvert d'habits. (Leilin, 04.03A.7-B.1)
\end{exe}
Le dérivé nominal 3401	\tgf{3401}	\ipa{ɣja} 2.16 ``bouchon'' y est apparenté, même si la dérivation ne s'accompagne pas de changement tonal ou consonantique. 


\item 1084 \tgf{1084}	\ipa{ɣạ}	2.56 \ptang{C-S-kap}{dix} peut se comparer au \jpg{sqi}  ``dix''. La forme libre du japhug semble sans rapport avec le tangoute, mais les dizaines présentent une variante sqap-- (\citealt[188-9]{jacques08}): \ipa{sqaptɯɣ} ``onze'', \ipa{sqaprɤɣ} ``seize'' etc. Une variation similaire s'observe dans les autres langues rgyalrong. Même si la forme libre du \jpg{sqi} ne correspond pas au tangoute, on peut considérer la forme composée comme étant celle de base, ce qui nous permet de reconstruire une rime *--ap en pré-tangoute.
\newline
Pour exprimer les dizaines, on observe également une forme alternative 1040 \tgf{1040} \ipa{ɕja¹} toujours suivie d'un autre numéral qui ne semble pas être liée à \tgf{1084}	\ipa{ɣạ²}.

\end{enumerate}

\subsubsection{Tangoute --a :: tibétain  --ad/--as :: japhug \ipapl{--at/--ɤt/--az/--ɤz}}	\label{subsubsec:correspondance:a:at}
On reconstruira *--at en pré-tangoute pour les mots appartenant à cette correspondance.
\newline
\linebreak
A. Initiales labiales \label{rimes:04:2:lab}
\newline
\begin{enumerate}

\item 4585 \tgf{4585}	\ipa{.wja} 1.19 \ptang{C-pjat}{vomir} correspond au \jpg{mɯjphɤt}	``vomir'' et au \plb{C-pàt}{0577}. Voir l'exemple textuel \ref{ex:tg:bouillir} p.\pageref{ex:tg:bouillir}.


\item 2467 \tgf{2467}	\ipa{.wjạ} 1.64 \ptang{C-pjat}{fleur} est apparenté au verbe \jpg{ɣɤwɤt / ɣɤpɤt}	``fleurir'',  au  \situ{tapát} ``fleur'' et au \plb{k-wát}{0301}. On retrouve aussi cette racine nominale en \bir{pan³} `fleur'' avec une finale nasale.

\item 3136 \tgf{3136}	\ipa{pjạ} 1.64 \ptang{S-pjat}{rapiécer} correspond au \jpg{ɕphɤt} ``rafistoler'', verbe transitif qui s'emploie à propos des habits. On le trouve dans un dérivé nominal \jpg{tʂɤɕphɤt} ``plantain'' avec l'état construit de \jpg{tʂu} ``chemin''. 



\item 2475 \tgf{2475} \ipa{phia} 1.18 \ptang{phrat}{casser} peut être rapproché du \jpg{prɤt}	``couper, casser (à propos d'un fil, d'une corde)''. Ce verbe a une forme anticausative 4314 \tgf{4314} \ipa{bia} 1.18  \ptang{mbrat}{se casser} correspondant au \jpg{mbrɤt}. L'exemple japhug suivant illustre l'opposition entre les formes transitive et anticausative:
\begin{exe}
\ex \label{ex:tg:appeler2}  \vspace{-8pt}
\gll   \ipa{ɯ-mkɤɣɯr}	\ipa{tʂhɯβ}	\ipa{ʑo}	\ipa{pa-prɤt}	\ipa{ɲɯ-ŋu,}	``\ipa{wo}	\ipa{a-ʑi}	\ipa{ra}	\ipa{nɯ-mkɤɣɯr}	\ipa{pɯ-mbrɤt}'' \\
	3\sgposs{}-collier interjection \intens{} \aor{}:3>3-casser \impf{}-être Oh 1\sgposs{}-dame \pl{} 2\plposs{}-collier \aor{}-se.casser \\
\glt Elle cassa son collier (et ils dirent): ``Oh, ma dame, votre collier s'est cassé'' (Kunbzang 214)
\end{exe}
Dans le contexte où apparaît cette phrase, les serviteurs qui prononcent les paroles rapportées pensent que le collier s'est cassé spontanément (ils ignorent que la reine l'a cassé exprès).
\newline
En tangoute, on ne dispose que d'exemples textuels de l'anticausatif, comme le suivant cité dans \citet{lifw97}, où il est rendu transitif au moyen de l'auxiliaire causatif \tgf{0749} \ipa{phji¹}:
\newline
\linebreak
\begin{tabular}{llllllll}
	\tgf{3133}&	\tgf{2098}&	\tgf{1139}&	\tgf{1546}&	\tgf{2518}&	\tgf{1734}&	\tgf{4314}&	\tgf{0749}\\
	\tinynb{3133}&	\tinynb{2098}&	\tinynb{1139}&	\tinynb{1546}&	\tinynb{2518}&	\tinynb{1734}&	\tinynb{4314}&	\tinynb{0749}\\
\end{tabular}
\begin{exe}
\ex \label{ex:tg:casser:mbrat}  \vspace{-8pt}
\gll   \ipa{sjij¹}	\ipa{ŋa²}	\ipa{.jij¹}	\ipa{ljụ²}	\ipa{njiij¹}	\ipa{tji¹-bia¹-phji¹} \\
		aujourd'hui moi \antierg{} corps cœur \prohib{}-briser-causer[A] \\
\glt Ne me brise pas le cœur. (Suvarna, 10.344)  %Wang (1933:XXX)) %vérifier la trad
\end{exe}
Cet exemple prouve que le sens de ce verbe diffère considérablement de son équivalent japhug. Du point de vue sémantique, une comparaison avec le \jpg{phaʁ} serait meilleure, mais on a préféré une autre solution (voir p.\pageref{ex:tg:couper3}).
\newline
Il faut également rapprocher de \tgf{2475} \ipa{phia¹} le nom dérivé par allongement vocalique et alternance tonale 4008 \tgf{4008} \ipa{phiaa} 2.20 \ptang{phraat}{partie}, qui s'emploie en particulier pour exprimer les fractions, comme dans l'exemple \tgf{5865} \tgf{4008} \tgf{5993} \tgf{4027} \tgf{4008} \ipa{sọ¹phiaa²kha¹njɨɨ¹phiaa²} ``deux parties parmi trois parties", c'est à dire ``deux tiers'' (Sunzi 5A.1, \citealt{lin94sunzi}).


\item 2436 \tgf{2436} \ipa{mjaa} 1.23 \ptang{mjaat}{fruit} peut se comparer à la seconde syllabe du mot \jpg{sɯmat} ``fruit''. En japhug, la première syllabe de ce composé est l'état construit du nom \jpg{si} ``arbre''. Il est remarquable de noter que le même composé est attesté en tangoute: \tgf{4250}\tgf{2436} \ipa{sji¹mjaa¹} `fruit'' (voir par exemple Leilin 05.17B.5).




\item 5702 \tgf{5702} \ipa{mjaa} 1.23   \ptang{mjaat}{blessure} ainsi que son verbe dérivé 5628 \tgf{5628} \ipa{mjạ} 1.64 \ptang{S-mjat}{blesser} se comparent au \jpg{tɯ-ɣmaz} et au \tib{rma} ``blessure''. 


\end{enumerate}


B. Initiales dentales et rétroflexes \label{rimes:04:2:dent}
\newline
\begin{enumerate}

\item 2546 \tgf{2546} \ipa{njạ} 1.64 \ptang{S-njat}{divinité} correspond au \bir{nat} ``divinité''. C'est le nom de divinité le plus courant en tangoute. Voir les exemples d'emploi p.\pageref{ex:tg:fantome} et p.\pageref{ex:tg:pouvoir}. La forme 584 \tgf{0584} \ipa{na}  1.17 est peut-être apparentée \ptang{nat}{divinité}.

\item 5181 \tgf{5181}	\ipa{tsar}	1.80	\ptang{r-tsat}{animal sauvage} correspond au \jpg{qartshaz}	``cerf''. Le sens précis de ce nom en tangoute est mal connu.

\item 596 \tgf{0596}	\ipa{dzja}	1.20 \ptang{ndzjat}{grandir} est cognat avec le \jpg{ndzɤt}	``grandir''. En tangoute comme en japhug, ce verbe s'emploie à propos des êtres humains:
\newline
\linebreak
\begin{tabular}{llllllllll}
	\tgf{0596}&	\tgf{2503}&	\tgf{1999}&	\tgf{5938}&	\tgf{3183}&	\tgf{5981}&	\tgf{3574}&	\tgf{4444}&	\tgf{1918}&	\tgf{0929}\\
	\tinynb{0596}&	\tinynb{2503}&	\tinynb{1999}&	\tinynb{5938}&	\tinynb{3183}&	\tinynb{5981}&	\tinynb{3574}&	\tinynb{4444}&	\tinynb{1918}&	\tinynb{0929}\\
	\tgf{1906}& &&&&&&&&\\
	\tinynb{1906}& &&&&&&&&\\
\end{tabular}
\begin{exe}
\ex \label{ex:tg:grandir}  \vspace{-8pt}
\gll   \ipa{dzja¹}	\ipa{kụ¹}	\ipa{ŋwə¹} \ipa{gjij²}	\ipa{.wo²}	\ipa{.a-tsjij²}	\ipa{ljɨ̣¹}	\ipa{mji¹-dʑjij¹}	\ipa{nioow¹} \\
		grandir après cinq classiques sens \dir{}-comprendre[A] \conj{} \negat{}-pur après \\
\glt Après qu'il eut grandi, non seulement il comprenait le sens des cinq classiques (Leilin 04.25B.2-3)
\end{exe}


\item 4225 \tgf{4225} \ipa{sja}	1.20 \ptang{sjat}{tuer} est apparenté au \jpg{sat} et au \tib{gsod, bsad} de même sens. Cette racine verbale se retrouve dans toute la famille sino-tibétaine, voir les cognats cités dans \citet[330]{matisoff03}.


\item 791 \tgf{0791}	\ipa{ljạ} 1.64	 \ptang{S-ljat}{front} est apparenté au \jpg{tɯ-laz} ``front'' et au  \pumi{ɬɛpĩ́}.



\item 4602 \tgf{4602}	\ipa{.jar}	 1.82	\ptang{r-jat}{huit} est cognat du \jpg{kɯrcat}, du \plb{C-yèt}{0485} et du \tib{brgʲad} (par la seconde loi de Li <*p-rjat comme avec le numéral ``cent'' p.\pageref{analyse:cent} ). La forme japhug est inexplicable (on attendrait *\ipapl{rʑat}), mais la correspondance est régulière avec les autres langues.\footnote{La forme chinoise pour ``huit'' \zh{八} \ipapl{pɛt} < *pret est proche de celle observée en proto-tibétain. Les données du rgyalrong et du tangoute suggèrent que le *e du chinois archaïque est en fait secondaire, résultant de la fusion d'un plus ancien *ja, car la vraie initiale de ce mot est la semi-voyelle palatale et non pas la labiale. La même remarque s'applique au numéral ``cent'', voir p.\pageref{analyse:cent}.} \label{analyse:huit}



\item 1715 \tgf{1715} \ipa{rjar}	1.82 \ptang{rjat}{écrire} est comparable au \jpg{rɤt} ``écrire'' et au \tib{ɴbrad}  ``gratter''. C'est le tibétain qui préserve ici le sens originel; le tangoute et le rgyalrong ont innové le sens d'écrire, comme le latin \ipa{scribō} de l'indo-européen *\ipapl{skrei̯bʰ-} ``kratzen, ritzen'' (\citealt[562]{liv}). Il est difficile de savoir si l'évolution sémantique s'est faite de façon indépendante ou si elle reflète une innovation commune entre le rgyalrong et le tangoute. En effet, on peut mentionner le cas du \tib{ɴbri}\footnote{En fait \textit{ɴdri} < *N-ri, comme l'a montré \citet{hill05vbri}.} ``écrire'' et le \bir{re³} ``écrire''. Ces deux formes sont cognates, mais il est invraisemblable que le sens ``écrire'' ait existé à l'époque de l'ancêtre commun du tibétain et du birman, car il faut remonter à une période où l'écriture n'existait pas encore dans cette région du monde. Ainsi, on doit admettre un changement sémantique indépendant de ``dessiner'' à ``écrire'' en tibétain et en lolo-birman (le sens de ``dessiner'' est attesté en tibétain et en birman).

Ce verbe a comme dérivé nominal par alternance tonale 575 \tgf{0575} \ipa{rjar} 2.74 ``trace, marque''. Ce nom a été probablement dérivé avant le changement sémantique de ``gratter'' à ``écrire''.\label{analyse:ecrire}

C. Initiales vélaires \label{rimes:04:3:vel}



\item 2539 \tgf{2539}	\ipa{kjạ}	1.64 ``avoir peur, être surpris'' (*S-kjar) est apparenté au \jpg{nɤscɤr}	``être saisi de frayeur'' (verbe intransitif). En tangoute, ce verbe peut s'employer sous une forme nominalisée ``danger, péril'' (voir l'exemple \ref{ex:tg:separer1}  p.\pageref{ex:tg:separer1}), mais il s'emploie également comme verbe transitif ``craindre'':
\newline
\linebreak
\begin{tabular}{llllllllll}
	\tgf{2098}&	\tgf{2539}&	\tgf{1542}&	\tgf{2694}&	\tgf{3935}&	\tgf{1918}&	\tgf{2539}&	\tgf{2694}&	\tgf{3935}&	\tgf{2539}\\
\tinynb{2098}&	\tinynb{2539}&	\tinynb{1542}&	\tinynb{2694}&	\tinynb{3935}&	\tinynb{1918}&	\tinynb{2539}&	\tinynb{2694}&	\tinynb{3935}&	\tinynb{2539}\\
		\tgf{1542}&	\tgf{2098}&	\tgf{1918}&	\tgf{2539}&&&&&&\\
		\tinynb{1542}&	\tinynb{2098}&	\tinynb{1918}&	\tinynb{2539}&&&&&&\\
\end{tabular}
\begin{exe}
\ex \label{ex:tg:craindre}  \vspace{-8pt}
\gll   \ipa{ŋa²}	\ipa{kjạ¹}	\ipa{ku¹}	\ipa{wjij²zjɨ̣¹}	\ipa{mji¹-kjạ¹}	\ipa{wjij²zjɨ̣¹}	\ipa{kjạ¹}	\ipa{ku¹}	\ipa{ŋa²}	\ipa{mji¹-kjạ¹} \\
		moi craindre[A] \conj{} ennemi \negat{}-craindre[A] ennemi craindre[A] \conj{} moi \negat{}-craindre[A] \\
\glt Si l'on a peur de moi, alors on ne craindra pas l'ennemi, et si l'on a peur de l'ennemi alors on ne me craindra pas. (Sunzi, 33B.6)
\end{exe}
Ce verbe a pour forme alternante 1252 \tgf{1252} \ipa{kjɨ̣} 1.69 \ptang{S-kjor ou plutôt*S-kjar-u}{effrayer}, qui apparaît avec un agent à la première ou à la seconde personne du singulier, comme l'illustre l'exemple suivant:
\newline
\linebreak
\begin{tabular}{llllllllll}
\tgf{0243}&	\tgf{2867}&	\tgf{2038}&	\tgf{3926}&	\tgf{5815}&	\tgf{1525}&	\tgf{4601}&	\tgf{2098}&	\tgf{5815}&	\tgf{1525}\\
\tinynb{0243}&	\tinynb{2867}&	\tinynb{2038}&	\tinynb{3926}&	\tinynb{5815}&	\tinynb{1525}&	\tinynb{4601}&	\tinynb{2098}&	\tinynb{5815}&	\tinynb{1525}\\
\tgf{2098}&	\tgf{1093}&	\tgf{5757}&	\tgf{0080}&	\tgf{3926}&	\tgf{5815}&	\tgf{1252}&	\tgf{4601}&	\tgf{2098}&	\tgf{5815}\\
\tinynb{2098}&	\tinynb{1093}&	\tinynb{5757}&	\tinynb{0080}&	\tinynb{3926}&	\tinynb{5815}&	\tinynb{1252}&	\tinynb{4601}&	\tinynb{2098}&	\tinynb{5815}\\
	\tgf{1252}&\tgf{2098} &&&&&&&& \\
	\tinynb{1252}&\tinynb{2098} &&&&&&&& \\
\end{tabular}
\begin{exe}
\ex \label{ex:tg:craindre2}  \vspace{-8pt}
\gll  \ipa{sji²kiej²}	\ipa{sew¹}	\ipa{nja²}	\ipa{tsjɨ¹}	\ipa{khio¹-nja²}	\ipa{ŋa²}	\ipa{tsjɨ¹}	\ipa{khio¹-ŋa²}
 \ipa{rar¹}	\ipa{ʑjạ¹}	\ipa{phio²}	\ipa{nja²}	\ipa{tsjɨ¹}	\ipa{kjɨ̣¹-nja²}	\ipa{ŋa²}	\ipa{tsjɨ¹}	\ipa{kjɨ̣¹-ŋa²} \\
 	femme jalouse toi aussi détester[B]-2\sg{} moi aussi détester[B]-1\sg{} 
	montagne entre serpent toi aussi craindre[B]-2\sg{} moi aussi craindre[B]-1\sg{} \\
\glt Les femmes jalouses, tu les détestes et je les déteste. Les serpents des montagnes, tu en as peur et j'en ai peur aussi. (\citealt[187]{kychanov74})
\end{exe}




\item 4480 \tgf{4480}	\ipa{kar} 2.73	\ptang{r-kat}{séparer} ainsi que 1160 \tgf{1160} \ipa{ka} 2.14 \ptang{kat}{séparer} sont comparables au \jpg{qɤt}	``séparer'', bien qu'une comparaison avec \jpg{qar} ``choisir'' ne soit pas impossible pour le premier.
\tgf{1160} \ipa{ka²} apparaît le plus souvent dans le composé \tgf{5163}\tgf{1160} \ipa{dʑjow¹ka²} ``se séparer (de quelqu'un)'', mais il est également attesté dans le sens d'éviter:
\newline
\linebreak
\begin{tabular}{llllll}
	\tgf{0322}&	\tgf{2019}&	\tgf{2539}&	\tgf{4950}&	\tgf{2590}&	\tgf{1160}\\
	\tinynb{0322}&	\tinynb{2019}&	\tinynb{2539}&	\tinynb{4950}&	\tinynb{2590}&	\tinynb{1160}\\
\end{tabular}
\begin{exe}
\ex \label{ex:tg:separer1}  \vspace{-8pt}
\gll   \ipa{tɕhjwo¹}	\ipa{thja¹}	\ipa{kjạ¹}	\ipa{rjir²}	\ipa{.wjɨ²-ka²} \\
	\conj{} \dem{} danger \comit{} \dir{}-séparer \\
\glt Il put ainsi éviter ce danger (Leilin 06.26B.7-27A.1)
\end{exe}
\tgf{4480}	\ipa{kar²} a quant à lui des attestations comme verbe transitif signifiant ``séparer, distribuer, donner'':
\newline
\linebreak
\begin{tabular}{lllllll}
	\tgf{3909}&	\tgf{2152}&	\tgf{5127}&	\tgf{3454}&	\tgf{5553}&	\tgf{2798}&	\tgf{4480}\\
	\tinynb{3909}&	\tinynb{2152}&	\tinynb{5127}&	\tinynb{3454}&	\tinynb{5553}&	\tinynb{2798}&	\tinynb{4480}\\
\end{tabular}
\begin{exe}
\ex \label{ex:tg:séparer2}  \vspace{-8pt}
\gll   \ipa{pu¹ɕji¹}	\ipa{twụ²}	\ipa{tshjij²}	\ipa{gji²}	\ipa{.jir²}	\ipa{kar²} \\
		Bu.Shi vraiment ovin un cent séparer \\
\glt Bu Shi (\zh{卜式}) lui donna cent moutons. (Cixiaozhuan 16.4, \citet[52]{jacques07textes}
\end{exe}
En japhug, \ipa{qɤt} a un dérivé anticausatif \ipa{nɯɴɢɤt} ``se séparer'' qui n'a pas d'équivalent en tangoute.


\end{enumerate}

Finalement, 3235  \tgf{3235}	\ipa{kjạ} 2.57 ``chanson'' pourrait être apparenté au \tib{skad} ``parole'', auquel cas on proposera une reconstruction *S-kjat. Il pourrait toutefois s'agir aussi d'un emprunt au chinois \zh{歌} \textit{ka} comme le propose \citet{lifw97}.
\subsubsection{Tangoute --a :: tibétain  --ag :: japhug \ipapl{--aʁ}}	\label{subsubsec:correspondance:a:ak}
On reconstruira *--ak en pré-tangoute pour les mots appartenant à cette correspondance.
\newline
\linebreak
A. Initiales labiales \label{rimes:04:3:lab}
\newline
\begin{enumerate}


\item 4532 \tgf{4532}	\ipa{pạ}	2.56 \ptang{S-pak}{avoir soif} est cognat avec le \jpg{ɕpaʁ}	``avoir soif''.
\newline
\linebreak
\begin{tabular}{llllllllll}
	\tgf{4534}&	\tgf{0705}&	\tgf{3452}&	\tgf{3465}&	\tgf{4517}&	\tgf{4532}&	\tgf{3589}&	\tgf{3452}&	\tgf{3065}&	\tgf{4658}\\
	\tinynb{4534}&	\tinynb{0705}&	\tinynb{3452}&	\tinynb{3465}&	\tinynb{4517}&	\tinynb{4532}&	\tinynb{3589}&	\tinynb{3452}&	\tinynb{3065}&	\tinynb{4658}\\
\end{tabular}
\begin{exe}
\ex \label{ex:tg:soif}  \vspace{-8pt}
\gll   \ipa{dʑjwiw²}	\ipa{zjịj¹}	\ipa{.jij²}	\ipa{tɕhji¹}	\ipa{dzji¹}	\ipa{pạ²}	\ipa{dzjɨj¹}	\ipa{.jij²}	\ipa{lhju¹}	\ipa{thji¹} \\
		avoir.faim temps mouton viande manger[A] avoir.soif temps mouton lait boire \\
\glt Quand il avait faim, il mangeait de la viande de mouton, et quand il avait soif, buvait du lait de mouton. (Leilin, 03.11B.7-12A.1)
\end{exe}



\item 3708 \tgf{3708} \ipa{phja} 1.20 ``couper'' est apparenté au \jpg{phaʁ} ``casser, faire s'écrouler'':
\begin{exe}
\ex \label{ex:jpg:casser}  \vspace{-8pt}
\gll   \ipa{li}	\ipa{nɯ-kha}	\ipa{nɯ}	\ipa{ɯ-χcɤl}	\ipa{ʑo}	\ipa{ɲɤ-phaʁ} \\
		encore 3\plposs{}-maison \dem{} 3\sgposs{}-milieu  \intens{} \med{}-casser \\
\glt Et il fit s'écrouler le milieu de leur maison (La grenouille, 61)
\end{exe}
Un autre cognat potentiel aurait été \ipa{prɤt} ``casser, couper'', mais comme le *--r-- du pré-tangoute est préservé devant *a, donnant la rime --ia, on a jugé meilleur du point de vue phonétique de comparer \ipa{prɤt} à \tgf{2475} \ipa{phia¹}, voir p.\pageref{ex:tg:casser:mbrat}. Toutefois, le sens de \tgf{3708} \ipa{phja¹} est plus proche du \jpg{prɤt} que celui de  \tgf{2475} \ipa{phia¹}. \tgf{3708} \ipa{phja¹} apparaît dans les sens dérivés de ``décider'' ou d'``interdire'':
\newline
\linebreak
\begin{tabular}{llllllllll}
	\tgf{5297}&	\tgf{2627}&	\tgf{5258}&	\tgf{3830}&	\tgf{5604}&	\tgf{5113}&	\tgf{0510}&	\tgf{5399}&	\tgf{2045}&	\tgf{5981}\\
	\tinynb{5297}&	\tinynb{2627}&	\tinynb{5258}&	\tinynb{3830}&	\tinynb{5604}&	\tinynb{5113}&	\tinynb{0510}&	\tinynb{5399}&	\tinynb{2045}&	\tinynb{5981}\\
	\tgf{3708}&	\tgf{5993}&&&&&&&&\\
	\tinynb{3708}&	\tinynb{5993}&&&&&&&&\\
\end{tabular}
\begin{exe}
\ex \label{ex:tg:couper3}  \vspace{-8pt}
\gll   \ipa{ɕioow¹}	\ipa{ljɨ̣².iọ¹}	\ipa{njij²}	\ipa{dʑjɨ.wji¹}	\ipa{ŋwər¹}	\ipa{khju¹}	\ipa{.o²}	\ipa{.a-phja¹}	\ipa{kha¹} \\
		Shu endroit roi \erg{} ciel sous alcool \dir{}-couper intérieur \\
\glt Alors que le roi de Shu avait interdit l'alcool dans le monde (Leilin 04.06B.5-6)
\end{exe}
On trouve un verbe apparenté 1527 \tgf{1527} \ipa{phjaa} 1.23 \ptang{phjaak}{interdire}:
\newline
\linebreak
\begin{tabular}{lllllllll}
	\tgf{4456}&	\tgf{5464}&	\tgf{2983}&	\tgf{1136}&	\tgf{4719}&	\tgf{2045}&	\tgf{1527}&	\tgf{0467}&	\tgf{3844}\\
	\tinynb{4456}&	\tinynb{5464}&	\tinynb{2983}&	\tinynb{1136}&	\tinynb{4719}&	\tinynb{2045}&	\tinynb{1527}&	\tinynb{0467}&	\tinynb{3844}\\
\end{tabular}
\begin{exe}
\ex \label{ex:tg:interdire}  \vspace{-8pt}
\gll   	\ipa{ljịj²}	\ipa{ʑiejr²}	\ipa{u²}	\ipa{gu²}	\ipa{kiẹj²}	\ipa{.o²}	\ipa{phjaa¹}	\ipa{tsjiir¹}	\ipa{dʑjịj¹} \\
		grand habiter dans milieu capitale alcool interdire loi être.appliqué  \\
\glt Durant l'ère Tai'an, une loi de prohibition de l’alcool était appliquée dans la capitale. (Cixiaozhuan 4.7, \citealt[17]{jacques07textes})
\end{exe}



\tgf{3708} \ipa{phja¹} a un dérivé anticausatif 4459 \tgf{4459} \ipa{bja} 2.17 \ptang{mbjak}{se couper, se casser} qui a également un équivalent \jpg{mbaʁ} ``se casser (s'emploie par exemple à propos du bambou)'', tandis que \tgf{1527} \ipa{phjaa¹} a pour dérivé 2350 \tgf{2350} \ipa{bjaa} 1.23 ``se finir''. Dans l'exemple suivant, il est clair d'après le contexte que l'action de casser s'effectue spontanément.
\newline
\linebreak
\begin{tabular}{lllllll}
	\tgf{1993}&	\tgf{0186}&	\tgf{5981}&	\tgf{4459}&	\tgf{0294}&	\tgf{4342}&	\tgf{4478}\\
	\tinynb{1993}&	\tinynb{0186}&	\tinynb{5981}&	\tinynb{4459}&	\tinynb{0294}&	\tinynb{4342}&	\tinynb{4478}\\
\end{tabular}
\begin{exe}
\ex \label{ex:tg:couper4}  \vspace{-8pt}
\gll   \ipa{kwə²lu²}	\ipa{.a-bja²}	\ipa{.wa¹}	\ipa{dja²-ta¹} \\
		corde \dir{}-se.casser cochon \dir{}-s'enfuir \\
\glt La corde se cassa et le cochon s'enfuit (Leilin 06.09A.6)
\end{exe}
Par ailleurs, tout comme son équivalent chinois \zh{斷} \textit{duàn}, \tgf{4459} \ipa{bja²} s'emploie dans le sens de ``s'arrêter'' (habituellement au négatif \tgf{1918}\tgf{4459} \ipa{mji¹bja²} ``sans arrêt''.
\newline 
Le verbe 4007 \tgf{4007} \ipa{pha} 1.17 \ptang{phak}{être perdu, mourir, vaincre, être vaincu} est également apparenté à \tgf{3708} \ipa{phja¹}. C'est un verbe labile, qui peut s'employer transitivement ou intransitivement.
\newline
\linebreak
\begin{tabular}{lllllllll}
	\tgf{2547}&	\tgf{1531}&	\tgf{4342}&	\tgf{4007}&	\tgf{1542}&	\tgf{3119}&	\tgf{2541}&	\tgf{2518}&	\tgf{1068}\\
	\tinynb{2547}&	\tinynb{1531}&	\tinynb{4342}&	\tinynb{4007}&	\tinynb{1542}&	\tinynb{3119}&	\tinynb{2541}&	\tinynb{2518}&	\tinynb{1068}\\
\end{tabular}
\begin{exe}
\ex \label{ex:tg:etrevaincu}  \vspace{-8pt}
\gll   \ipa{tɕier¹}	\ipa{gja¹}	\ipa{dja²-pha¹}	\ipa{ku¹}	\ipa{.ji¹}	\ipa{dzjwo²}	\ipa{njiij¹}	\ipa{ljɨ¹}\\
		droit armée \dir{}-être.vaincu \conj{} nombreux homme cœur tomber \\
\glt Si l'armée de droite est vaincue, la plupart des hommes perdront espoir. (Sunzi 12B.1-2, \citealt{lin94sunzi}).
\end{exe}





\item 3936 \tgf{3936}	\ipa{pha} 1.17	 \ptang{phak}{côté, moitié} ainsi que 335 \tgf{0335} \ipa{phja} 1.20   \ptang{phjak}{bord, côté} sont cognats du \jpg{ɯ-phaʁ} ``côté, moitié''. L'adjectif et adverbe 2365 \tgf{2365} \ipa{pha} 1.17 ``différent, séparément'' est peut-être également à rapprocher de  \tgf{3936} \ipa{pha¹}.


\item 294 \tgf{0294}	\ipa{.wa} 1.17 \ptang{C-pak}{cochon} est cognat avec le \jpg{paʁ} et le \tib{pʰag} de même sens. 


\item 1360 \tgf{1360}	\ipa{.wa} 1.17	 \ptang{C-pak}{se cacher} est comparable au \jpg{anbaʁ} ``se cacher''. On peut également comparer le \bir{hwak} ``cacher'', \plb{wàk}{0771}.


\item 5170 \tgf{5170}	\ipa{.wạ} 1.63	\ptang{C-S-pak}{épaule} est apparenté au \jpg{tɯ-rpaʁ} et au \tib{pʰrag} qui signifient également ``épaule''. Ce nom a un dérivé verbal \ipa{.wạ} 2.56 ``porter à l'épaule, porter sur le dos'' qui peut s'écrire au moyen de trois caractères différents: 2021 \tgf{2021}, 4111 \tgf{4111} et 4432 \tgf{4432}. Ce verbe dérivé existe aussi en  \plb{bak^L}{661b} ``porter à l'épaule''.


\item 2200 \tgf{2200} \ipa{ba}	1.17	\ptang{mbak}{chasser} peut être rapproché du \jpg{ɣɤrʁaʁ}	``chasser'' (<*rbaq). On ne dispose pas d'exemples textuels de ce verbe.


\item 4567 \tgf{4567}	\ipa{bạ}	2.56	\ptang{S-mbak}{feuille} correspond au \jpg{tɤ-jwaʁ} ``feuille'' (<*lbaq). \citet[317,321]{matisoff03} mentionne deux formes birmanes potentiellement apparentées: \ipa{phak} et \ipa{rwak} signifiant toutes deux ``feuille'', \plb{C-pàk}{0305}


\item 4820 \tgf{4820}	\ipa{mạ}	1.63	\ptang{S-mak}{gendre} peut se comparer au \jpg{tɯ-nmaʁ} et au \tib{mag.pa} de même sens. Comme les autres termes de parenté, ce nom est traité dans \citet{jacques11kinship}.

\end{enumerate}

B. Initiales dentales \label{rimes:04:3:dent}
\newline
\begin{enumerate}

\item 4052 \tgf{4052}	\ipa{dạ}	2.56	\ptang{S-ndak}{froid} est comparable au \jpg{mɯɕtaʁ}	``froid''. On ne connaît pas d'exemples textuels de ce mot. La prénasalisation est irrégulière (voir \ref{subsubsec:prenasaliseestg} p.\pageref{subsubsec:prenasaliseestg}), et peut-être due à une ancienne présyllabe nasale (correspondant au Japhug \ipa{--mɯ--}) qui aurait sonorisé l'initiale.


\item 4693 \tgf{4693} \ipa{na}	1.17	\ptang{nak}{profond} peut être rapproché du \jpg{rnaʁ} ``profond'', du naish \naxi{ho˥},	\nayn{ɬo˥},	\laze{hɑ˩},	\protona{l̥aC1/l̥aC2/SnaC}	
et du \bir{nak} de même sens. En vertu de la loi phonétique selon laquelle la préinitiale *r-- tombe devant l'initiale *n-- (voir \ref{tab:sans.preinitiale.r:preinitiale.r} p.\pageref{tab:sans.preinitiale.r:preinitiale.r}). Une reconstruction alternative *r-nak en pré-tangoute serait également possible.



\item 176 \tgf{0176}	\ipa{njaa}	1.21 \ptang{njaak}{noir} peut se comparer au \jpg{ɲaʁ} ainsi qu'au \tib{nag.po} de même sens. La présence d'une médiane --j-- (où d'une initiale nasale palatale) dans ce mot semble une caractéristique uniquement partagée par les langues macro-rgyalronguiques: pumi de Shuiluo \pumi{ɲɛ̌}, queyu \ipa{ɲe⁵⁵ɲe³³}, muya \ipa{ɲi⁵⁵ɲi³³}, qiang \ipa{ɲɪχ}. Le  \situ{nâk} ``noir'' est plus vraisemblablement un emprunt au tibétain. En naish \nayn{nɑ˩}	\laze{nɑ˥}	\protona{naC1} et en \plb{C-nák}{0503} on ne retrouve pas cette palatalisation.
\newline
En tangoute, on trouve néanmoins une forme apparentée qui n'a pas la médiane --j--: 2491 \tgf{2491} \ipa{na} 1.17  \ptang{nak}{nuit}.



\item 5921 \tgf{5921} \ipa{zar} 2.73, aussi écrit 1193 \tgf{1193}  \ptang{srak}{avoir honte} (voir p.\pageref{analyse:sr}) est apparenté au \jpg{nɤzraʁ} ``avoir honte'', \plb{s-ràk}{0520}. Le \tib{ɴtɕʰags, bɕags} ``se confesser'' est en revanche sans lien avec cette racine malgré le fait que le pré-tibétain *sr-- donne sh-: ce verbe signifiait plutôt ``déclarer'' en tibétain ancien (voir l'inscription bilingue de 822, \citealt[40,80]{licoblin87}), et il n'y a pas lieu de supposer que le sens de ``confesser'' vient d'``avoir honte''.



\item 630 \tgf{0630}	\ipa{la}	1.17  et 2497 \tgf{2497}	\ipa{la} 2.14 \ptang{C-tak}{tisser} correspondent au \jpg{taʁ}, au naish \nayn{dɑ˩},	\laze{dɑ˩},	\protona{daC1}	et au \tib{ɴtʰag, btags} ``tisser''. Comme le tangoute, le LB a ici une initiale lénifiée \plb{ràk}{0712}.
\newline
\linebreak
\begin{tabular}{llllllllll}
	\tgf{4274}&	\tgf{3668}&	\tgf{1888}&	\tgf{3042}&	\tgf{5364}&	\tgf{3872}&	\tgf{1600}&	\tgf{0630}&	\tgf{5880}&	\tgf{4869}\\
	\tinynb{4274}&	\tinynb{3668}&	\tinynb{1888}&	\tinynb{3042}&	\tinynb{5364}&	\tinynb{3872}&	\tinynb{1600}&	\tinynb{0630}&	\tinynb{5880}&	\tinynb{4869}\\
	\tgf{0154}&	\tgf{3986}&	\tgf{4893}&	\tgf{1139}&	\tgf{0105}&	\tgf{4887}&	\tgf{5113}&&&\\
		\tinynb{0154}&	\tinynb{3986}&	\tinynb{4893}&	\tinynb{1139}&	\tinynb{0105}&	\tinynb{4887}&	\tinynb{5113}&&&\\
\end{tabular}
\begin{exe}
\ex \label{ex:tg:tisser}  \vspace{-8pt}
\gll   \ipa{sow¹}	\ipa{ljị¹}	\ipa{bə²}	\ipa{.jur¹}	\ipa{lew²kjir¹}	\ipa{thu¹}	\ipa{la¹}	\ipa{ŋwu²}	\ipa{kjɨ¹.o¹}	\ipa{njɨ¹.wjɨ¹}	\ipa{.jij¹}	\ipa{kjụ¹tshwew¹}	\ipa{.wji¹} \\
		mûrier cultiver ver élever tissu tisser tisser \conj{} beau-père belle-mère \antierg{} donner.en.offrande faire[A] \\
\glt Je cultive des mûriers, élève des vers et tisse des habits afin de subvenir aux besoins de mes beaux-parents (Leilin 06.01B.6-7)
\end{exe}


\item 4 \tgf{0004}	\ipa{la}	2.14 \ptang{lak}{tante maternelle, épouse de l'oncle paternel} peut être rapproché du \jpg{tɤ-ɬaʁ} ``tante maternelle''. Ce mot est étudié plus en détail dans \citet{jacques11kinship}



\item 301 \tgf{0301} \ipa{lạ} 1.63 \ptang{S-lak}{aigle} peut se comparer avec le \jpg{qaliaʁ} ``aigle'', \tib{glag}. Il n'est pas facile de déterminer précisément de quelle espèce de rapace il s'agit en tangoute. Dans le ZZZ (166), on trouve l'équivalence \tgf{5134} \tgf{0301} \ipa{we¹lạ¹} = \zh{鷹雕} \textit{yīngdiāo} ``aigle''.


\item 3485 \tgf{3485} \ipa{lạ} 1.63  \ptang{S-lak}{main} est apparenté au \jpg{tɯ-jaʁ}, au naish \naxi{lɑ˩o˧},	\nayn{lo.qʰwɤ^L^M},	\laze{lɑ˩pʰie˩},	\protona{laC1/2}, au \plb{làk}{0111} et au \tib{lag} ``main''. La présence d'une présyllabe *S-- est inexpliquée, car on ne retrouve rien de tel dans ce mot parmi les langues qui préservent les groupes de consonnes initiaux. Il pourrait s'agir du préfixe non-aliénable.


\item 3192 \tgf{3192} 	\ipa{laa} 1.21 \ptang{laak}{épais} ainsi que 2700  \tgf{2700} 	\ipa{lạ} 1.63 \ptang{S-lak}{épais} sont apparentés au \jpg{jaʁ} ``épais'' et au naish \naxi{lɑ˥}	,\nayn{lo˥},	\laze{ɑ˥pɤ˥lu˧},	\protona{laC1/laC}. En japhug, ce verbe statif s'oppose à \ipa{mba} ``fin'' (voir p.\pageref{analyse:fin}) et s'applique aux objets plats, ou aux couches de neige. Il s'oppose en cela à \ipa{jpum}, qui désigne la largeur d'objets  longs (bâtons, arbres).
\begin{exe}
\ex \label{ex:jpg:epais}  \vspace{-8pt}
\gll   \ipa{tɕendɤre}	\ipa{tɤjpa}	\ipa{kɯngɯ-rtsɤɣ}	\ipa{kɯ-jaʁ}	\ipa{ko-sɯ-lɤt}	\ipa{qhe}  \\
		\conj{} neige neuf-étage \nmls{}:\stat{}-épais \med{}-\caus{}-jeter \conj{} \\
\glt Il fit tomber une neige épaisse de neuf étages (Gesar 149)
\end{exe}
\tgf{2700} 	\ipa{lạ¹} a pour antonyme \tgf{1475} \ipa{bji¹} (cognat du \jpg{mba}), comme on peut le constater dans le ZZZ (263), où le chinois \zh{厚薄} \textit{hòu báo} est rendu par \tgf{2700}\tgf{1475} \ipa{lạ¹bji¹}. L'autre paire d'antonymes qui désignent quant à eux l'épaisseur des objets longs (\tgf{1861} \ipa{tshjɨj¹} et \tgf{1805} \ipa{.wọ¹}) a également des cognats exacts en japhug (\ipa{xtshɯm} et \ipa{jpum}).


\item 4663 \tgf{4663}	\ipa{lha} 1.17 ``éteindre'' est peut-être comparable au \tib{brlag} ``perdre'' ou au \tib{rlog, brlag} ``détruire''. Si cette comparaison est correcte, une reconstruction \ptang{lhak}{éteindre} peut être proposée.  Le verbe \tib{rlog, brlag} est attesté sous la forme \textit{brlhag} en tibétain ancien:
\begin{exe}
\ex \label{ex:tib:perdre}  \vspace{-8pt}
\gll   kʰer-gʲi srïd brlhag-pa-r gnam-gʲis tɕïɦu-la bkaɦ stsald-pa bʲas-so\\
		Jie-\gen{} gouvernement détruire.\fut{}-\nmls{}-\dat{} ciel-\erg{} Zhou-\dat{} ordre donner.\ps{}-\nmls{} faire.\ps{}-\assert{} \\
\glt Le ciel a donné l'ordre à Zhou de détruire le gouvernement de Jie. (PT 986.12, voir \citealt{huangbf81shangshu} et \citealt{coblin91shangshu1})
\end{exe}
Sémantiquement, les verbes signifiant ``éteindre'' proviennent parfois de verbes signifiant ``tuer, détruire'' (en tibétain,``éteindre'' se dit \textit{me gsod} ``tuer le feu''), ou bien le changement inverse est également attesté, le verbe ``éteindre'' développant le sens ``d'anéantir'' (chinois \zh{滅} \textit{miè}).
\end{enumerate}
C. Initiales palatales et vélaires \label{rimes:04:3:vel}
\newline
\begin{enumerate}
\item 3544 \tgf{3544} \ipa{dʑiaa}	2.21	\ptang{ndʑaak}{traverser une rivière} est cognat du \jpg{ndʑaʁ}	``traverser une rivière à la nage''. Ce mot n'est pas limité aux langues macro-rgyalronguiques, puisqu'on le retrouve en limbou  \racine{ca:k}  ``traverser une rivière à la nage''.

\item 811 \tgf{0811}	\ipa{.jaar}	2.75 \ptang{r-jaak}{jour} correspond au \jpg{tɤ-rʑaʁ} ``une nuit'' et au \tib{ʑag} ``24 heures''. En japhug, on trouve également un verbe \ipa{rʑaʁ} ``passer un certain nombre de jours'':
\begin{exe}
\ex \label{ex:jpg:passer.jour}  \vspace{-8pt}
\gll   \ipa{tɤ-rɟit}	\ipa{χsɯm}	\ipa{nɯ}	\ipa{ma}	\ipa{mɯ-to-rʑaʁ}	\ipa{ri} \\
		\neu{}-enfant trois \dem{} à.part \negat{}-\med{}-passer.jours mais \\
\glt Bien que le garçon eût à peine trois jours (Gesar 78)
\end{exe}
La forme \ipa{tɤ-rʑaʁ} ``une nuit'' est donc un dérivé nominal de ce verbe. Le verbe \ipa{rʑaʁ} est quant à lui vraisemblablement un verbe dénominal dérivé par un préfixe r-- de la racine correspondant au \tib{ʑag}. La forme tangoute  \tgf{0811} \ipa{.jaar²}, comme le \jpg{tɤ-rʑaʁ}, est une forme déverbale, mais le verbe ne semble pas avoir été préservé en tangoute.


\item 4546 \tgf{4546}	\ipa{.jaar}	2.75 \ptang{r-jaak}{poulet} est comparable au \bir{krak} ``poulet'', \plb{k-ràk}{0050}. \citet[138]{matisoff03} apporte des preuves crédibles selon lesquelles le k-- initial en birman serait préfixal, et non une partie de la racine. Une étymologie interne au tangoute est toutefois possible également : ce nom pourrait être un dérivé du verbe \tgf{4521} \ipa{.jaar} 2.75 ``crier'', désignant originellement le ``coq'' < le ``crieur''.


\item 4680 \tgf{4680}	\ipa{khia}	2.15 \ptang{khrak}{soc} peut être rapproché du \jpg{qraʁ} ``soc''. Dans le ZZZ (271), ce nom est glosé comme le chinois \zh{鏵} \textit{huá} ``soc''.



\item 1752 \tgf{1752}	\ipa{kwạ}	  2.56 \ptang{S-kwak}{houe} peut se comparer au \jpg{qaʁ}	``houe''.  \tgf{1752}	\ipa{kwạ²} glose le chinois \zh{鋤} \textit{chú} dans le ZZZ (366).
\end{enumerate}

\subsubsection{Tangoute --a :: tibétain  --ar :: japhug \ipapl{--ar, --ɤr}}	\label{subsubsec:correspondance:a:ar}
On reconstruira *--ar en pré-tangoute pour les mots appartenant à ces deux correspondances. On a noté plus haut (p.\pageref{tab:noncycle2finalerjaphug}) que les finales *--ar du pré-tangoute n'évoluaient pas en syllabe du deuxième cycle mineur.
\begin{enumerate}


\item 5370 \tgf{5370} \ipa{pjạ}	1.64	\ptang{S-pjar}{paume} est cognat avec \tib{sbar.mo} ``paume''.  Le \jpg{ɯ-pɤl} ``paume'', bien que superficiellement similaire, ne doit pas être apparenté; ce serait le seul cas de --r final en tibétain correspondant à --l en rgyalrong. Par ailleurs, les langues rgyalrong ne préservent pas le --l final ancien, et l'on ne devrait pas à en trouver dans un mot cognat avec le tibétain.



\item 4526  \tgf{4526}	\ipa{ma}	2.14 \ptang{mar}{huileux} peut être rapproché du   \jpg{tamar} ``beurre'' (et de \ipa{nɤmar} ``huileux'') et du \tib{mar} ``beurre''. Il est intéressant de noter l'existence du verbe homophone 4737 \tgf{4737}	\ipa{ma²} ``enduire, oindre''\footnote{\citet{lifw97} rapproche ce verbe du chinois \zh{抹} \textit{mat}, ce qui est aussi une possibilité.} dérivé de  \tgf{4526}	\ipa{ma²}. En japhug, on trouve de la même façon à côté de la forme \ipa{tamar} ``beurre'' un verbe transitif \ipa{mar} ``oindre''. Le nom en japhug est dérivé de ce verbe. Le sens originel de cette racine serait ``enduire'', d'où l'on tire le nom du beurre ``ce dont on enduit'' et un adjectif ``huileux''.


\item 1530 \tgf{1530}	\ipa{mja}	1.20 ``fleuve'' pourrait se comparer au \jpg{smar} ``fleuve''. Une reconstruction *mjar serait envisageable, mais l'absence de préinitiale *S-- en pré-tangoute est irrégulière. \citet[365-7]{kepping94phiow} pense que  \tgf{1530} \ipa{mja¹} est en fait le nom du Fleuve Jaune, et suggère une relation étymologique avec son homonyme \tgf{0092} \ipa{mja¹} ``mère''. Si l'hypothèse de Kepping est correcte, notre étymologie doit être abandonnée. On pourrait également proposer qu'il s'agit d'un emprunt au nom tibétain du fleuve jaune, Rma-chu ``le fleuve du paon''.

Toutefois, il n'est pas certain que l'idée de Kepping soit valide, car \tgf{1530} \ipa{mja¹} est clairement attesté dans le sens de ``fleuve'' :
\newline
\linebreak
\begin{tabular}{llllllllll}
	\tgf{2669}&	\tgf{2042}&	\tgf{1530}&	\tgf{0661}&	\tgf{4950}&	\tgf{5163}&	\tgf{1160}&	\tgf{0807}&	\tgf{3798}&	\tgf{2983}\\
\tinynb{2669}&	\tinynb{2042}&	\tinynb{1530}&	\tinynb{0661}&	\tinynb{4950}&	\tinynb{5163}&	\tinynb{1160}&	\tinynb{0807}&	\tinynb{3798}&	\tinynb{2983}\\
\tgf{3456}& &&&&&&&&\\
\tinynb{3456}& &&&&&&&&\\

\end{tabular}
\begin{exe}
\ex \label{ex:tg:appeler2}  \vspace{-8pt}
\gll   \ipa{dze¹kia²}	\ipa{mja¹}	\ipa{ŋjow²}	\ipa{rjir²}	\ipa{dʑjow¹ka²}	\ipa{ɕjwa¹}	\ipa{tsəj¹}	\ipa{.u²}	\ipa{lja¹} \\
		oie fleuve mer \comit{} séparer rivière petit dans venir \\
\glt Lorsque les oies quittent les fleuves et la mer, et viennent dans une petite rivière (Les douze royaumes 133.7.4, \citealt[56,127]{solonin95})
\end{exe}

\end{enumerate}
\subsubsection{Tangoute --a :: japhug \ipapl{--ɣoN}}	\label{subsubsec:correspondance:a:agg}
On trouve trois exemples où le tangoute --a correspond à des rimes du japhug \ipapl{--ɣom} ou \ipapl{--ɣoŋ}. On a montré dans \citet[380-1]{jacques08} qu'il convenait de reconstruire des rimes à voyelle vélarisée en proto-japhug *\ipapl{aˠm et aˠŋ}. On adoptera la même solution en tangoute.

\begin{enumerate}


\item 1391 \tgf{1391}	\ipa{ba} 1.17 \ptang{mbaˠŋ}{sourd} est apparenté au \jpg{tɤ-mbɣo} ``sourd'' et au \plb{baŋ²}{0573} . Voir une liste de cognats dans d'autres langues citées dans \citet[267]{matisoff03}.


\item 5528 \tgf{5528}	\ipa{bar}	1.80	\ptang{r-mbaˠŋ}{tambour} peut se comparer au \jpg{tɤ-rmbɣo}	``tambour'' (<*\ipapl{r-mbaˠŋ}).


\item 975 \tgf{0975}	\ipa{par} 1.80 ``coaguler (sang)'' est peut-être comparable au \jpg{jpɣom} ``geler''  (proto-japhug <*\ipapl{lpaˠm}). Si cette comparaison est correcte, on pourra proposer une reconstruction \ptang{r-paˠm}{se coaguler} en pré-tangoute. Toutefois, elle n'est pas sans difficulté car le nom ``glace''  4053	\tgf{4053} \ipa{.wọ} 1.70 est également potentiellement apparenté au dérivé nominal de  \ipa{jpɣom} ``geler'' ,  \ipa{tɤ-jpɣom} ``glace'' . Ces deux étymologies sont mutuellement incompatibles.
\newline
On retrouve des formes comparables en  \pumi{bubõ̌}, en queyu \ipa{ʂpo⁵⁵} et même peut-être en chinois \zh{冰} *p.rəŋ.
\end{enumerate}

A ces trois exemples, on peut peut-être rajouter le 512  \tgf{0512} \ipa{tsar} 1.80 \ptang{r-tsaˠm}{xanthoxyle} que l'on comparerait avec le \jpg{tɕɣom} ``xanthoxyle''. La véracité de cette hypothèse est soumise à la condition que l'on puisse reconstruire \ipa{tɕɣom} *\ipapl{tɕaˠm} en proto-japhug, ce qui est incertain dû au manque de données comparatives. Une autre étymologie possible pour \tgf{0512} \ipa{tsar¹} serait de le considérer comme un dérivé de \tgf{0045} \ipa{zar¹} ``amer'', voir p.\pageref{subsubsec:correspondance:a:ap}.
	
	
\subsubsection{Tangoute --a :: japhug \ipapl{--a}}	\label{subsubsec:correspondance:a:a:a}
Quatre exemples présentent une correspondance inattendue entre le tangoute --a et --a en japhug ou en tibétain. Nous avons montré en \ref{subsubsec:correspondance:i:a:a} et en \ref{subsubsec:correspondance:e:a:a} que les *a du pré-tangoute devenaient des voyelles antérieures. Une correspondance du tangoute --\textit{a} à --\textit{a} dans d'autres langues nécessite donc une explication alternative. Certains des mots présentant cette correspondance sont des emprunts au tibétain qui seront traités en \ref{subsubsec:correspondance:a:autres}. Les autres doivent être étudiés au cas par cas:
\begin{enumerate}
 
\item 2098 \tgf{2098}	\ipa{ŋa}	2.14	``pronom de 1sg'' est comparable au \jpg{aʑo}	(<*ŋa-jaŋ) et au \tib{ŋa}. Une forme pré-tangoute *\ipapl{ŋa} devrait donner *ne (voir la discussion p.\pageref{tab:ngpalatalisation}). Nous avons proposé dans \citet[130-1]{jacques06comparaison} que le vocalisme du pronom \tgf{2098} \ipa{ŋa²} en tangoute est dû à l'analogie avec les pronoms de seconde    \tgf{3926} \ipa{nja²} et de troisième personne \tgf{2019} \ipa{thja¹}. Il pourrait s'agir également de la fusion avec un autre morphème, comme nous le proposons en \ref{subsec:personne}. Par conséquent, il est impossible de reconstruire ce pronom en pré-tangoute. \label{analyse:1sg}



\item 5523 \tgf{5523}	\ipa{rjar}	1.82	correspond au \jpg{ra}	``devoir'' et au \bir{ra¹} ``devoir''. La forme birmane est habituellement considérée comme la grammaticalisation du verbe homophone ``obtenir'', mais son existence en proto-\plb{(k)-ra¹}{0787} va à l'encontre de cette idée. Le verbe ``obtenir'' \tgf{1599} \ipa{rjir} et les formes apparentées ont été traitées p.\pageref{ex:tg:obtenir}; nous laissons ouverte la possibilité d'un lien entre ``obtenir'' et ``devoir''. Pour expliquer la forme \tgf{5523}	\ipa{rjar¹}, on est obligé de postuler un suffixe verbal en pré-tangoute. L'hypothèse la plus probable serait de reconstruire une forme pourvue du suffixe de perfectif *--s, d'où une reconstruction *\ipapl{rja--s}. Ce n'est toutefois pas la seule solution possible. 

Il faut aussi considérer le verbe rare \tgz{0197} ``pouvoir'' qui pourrait provenir d'un \ptang{ra}{pouvoir}. Il s'agit peut-être d'une forme non-suffixée de la même racine, mais il faudrait tout d'abord trouver des exemples de ce verbe dans les textes.

\item 5049 \tgf{5049}	\ipa{.wja}	1.19	``père'' ressemble fortement au \jpg{tɤ-wa} et au \tib{a.pʰa}. Toutefois, il est loin d'être sûr que ces mots sont hérités d'un étymon commun; de même que les formes enfantines du français \textit{papa} et le russe \textit{папа} sont sans rapport étymologique malgré leur similitude phonique, il est vraisemblable que ces termes de parentés, originellement des formes enfantines expressives, ne peuvent servir à la reconstruction de la proto-langue. 
\newline
Un autre terme provenant également du langage enfantin pour désigner le père en tangoute est  \tgf{3425}\tgf{3391} \ipa{pja¹pjɨ¹} attesté dans le ZZZ.


\item 573 \tgf{0573}	\ipa{na} 1.17 ``branche terrestre du chien'' ressemble au \jpg{khɯna}	``chien'', ainsi qu'à une forme Gurung \ipa{nakyu}, %demander à Martine XXX
comme \citet{nevskij60} lui-même l'avait déjà remarqué. On doit mentionner également la première syllabe du composé \tgf{5147}\tgf{5507} \ipa{njij²ɣa¹} ``chien''. Ces deux formes sont inexplicables, et on ne proposera pas de reconstructions en pré-tangoute.
\end{enumerate}
\subsubsection{Autres correspondances}	\label{subsubsec:correspondance:a:autres}
Dans cette section, nous mentionnerons tout d'abord quelques cognats potentiels qui présentent des correspondances sans équivalents ailleurs dans la langue, puis quelques mots empruntés au tibétain. 
\begin{enumerate} 

\item 4778 \tgf{4778}	\ipa{ɕjạ}	1.64	``sept'' est une forme problématique. Le numéral ``sept''  est en \jpg{kɯɕnɯz} ``sept'' (<*\ipapl{kəɕnis}), mais une forme *S-nis en pré-tangoute devrait donner *\ipa{njɨ̣}; les autres langues macro-rgyalronguiques et birmo-qianguiques présentent des formes similaires au japhug. Seul le shixing \ipa{ʂa⁵⁵} semble comparable au tangoute (voir \citet{chirkova09grammar}; mais la phonologie historique du shixing est encore inconnue, ce qui ne nous permet aucune certitude).


\item 2718 \tgf{2718}	\ipa{khiwa} 1.18 ``rein'' est peut-être comparable au \tib{mkʰal.ma} ``rein''. La médiane --w-- pourrait être la trace d'une préinitiale bilabiale (voir \ref{subsubsec:preinitialep} p.\pageref{tab:preinitialep}). En revanche, la rime --al du tibétain devrait correspondre au pré-tangoute *--a et au tangoute --e ou --ji (voir le mot pour ``grenouille'' p.\pageref{analyse:grenouille}).


\item 3190 \tgf{3190} \ipa{lhjwa} 1.20 ``langue'' partage avec le \tib{ltɕe} ``langue'' une initiale latérale, mais la correspondance de la rime ne convient pas. L'honorifique \tib{ldʑags} (dérivé du verbe \tib{ldag} ``lécher'' avec l'infixe honorifique de Gong --j--\footnote{Voir \citet{gong77}.} et le suffixe de nominalisation --s) ne correspond pas non plus, car pour expliquer la médiane --w-- il faudrait admettre une préinitiale *p--: il ne peut donc pas s'agir d'un emprunt tibétain. ɕe mot pourrait être apparenté au \plb{ʔl(y)a¹}{0097}.

\item Emprunts tibétains
\newline

3261 \tgf{3261}	\ipa{gia}	2.21 ``heureux'' est emprunté au \tib{dga} ``être heureux, aimer'', comme le montre la correspondance vocalique.\footnote{Les langues rgyalrong  ont également emprunté ce verbe tibétain, comme en \jpg{rga} ``aimer''.} On manque toutefois d'exemples textuels, si bien que cette comparaison reste incertaine.




909	\tgf{0909}	\ipa{kạ} 2.56, 5592  \tgf{5592}/5682	\tgf{5682}\ipa{kaar} 1.83 ``mesurer, peser''	sont vraisemblablement empruntés au \tib{skar} ``peser'' (comme le \jpg{skɤr} ``peser''). En effet, le sens originel de ce verbe tibétain est ``accrocher'' (voir \tib{ɴkʰar} ``accrocher'').


4851 \tgf{4851}	\ipa{bia} 2.15 ``riz, gruau de riz'' est un emprunt ancien au \tib{ɴbras}, qui reflète la médiane *--r--. On peut reconstruire un pré-tangoute *mbras littéralement identique à la forme tibétaine. Ce nom a également été emprunté en rgyalrong (\jpg{mbrɤz}). La région d'origine des tangoutes, Zungchu (Songpan) à Rngaba, est une région où le riz ne peut pas être cultivé.




1829 \tgf{1829}	\ipa{tshja} 1.20 ``brûler''	est vraisemblablement un emprunt du \tib{tsʰa} ``brûlant''. Cette racine se retrouve en \plb{tsa¹}{0517b}, mais la correspondence vocalique exclut a priori un rapport direct,\footnote{On attendrait *tshji en tangoute.} à moins qu'on postule un suffixe consonnantique *tshja-C en pré-tangoute. Les formes 1825  \tgf{1825}	\ipa{tshjwa} 1.20 ``brûler'' et  618  \tgf{0618}	\ipa{tsja} 1.26 ``chaud'' en sont dérivées par des processus morphologiques internes au tangoute.

\end{enumerate}

\subsection{Voyelles \ipapl{ə} et \ipapl{ɨ}} \label{subsec:voyelle.eu}

Les rimes du \ipa{shè} n°6\footnote{Le \ipa{shè} n°5 n'est pas inclus ici, car il ne comprend que des emprunts au chinois.} sont reconstruites par Gong Hwangcherng avec les voyelles centrales \ipapl{ə} et \ipapl{(j)ɨ}. Les autres spécialistes du tangoute reconstruisent aussi des voyelles centrales, mais également dans certains cas des voyelles antérieures (voir \ref{tab:she6}).

\begin{table}
\captionb{Reconstructions du \ipa{shè} n°6}\label{tab:she6}
\resizebox{\columnwidth}{!}{
\begin{tabular}{lllllllll} \toprule
rime&ton 1&ton 2&Sofronov1&Sofronov2&Nishida&Li&Gong&Arakawa\\
28&	1.27&	2.25&	\ipa{ə}&	\ipa{ə}&	\ipa{ʉɦ}&	\ipa{ə̠}&	\ipa{ə}&	\ipa{I}\\	
29&	1.28&	2.26&	\ipa{ə̂}&	\ipa{ə̂}&	\ipa{ʊ}&	\ipa{ə}&	\ipa{iə}&	\ipa{yI}\\	
30&	1.29&	2.27&	\ipa{i̯ə}&	\ipa{i̯ə}&	\ipa{ɨɦ}&	\ipa{ɪ̠}&	\ipa{jɨ}&	\ipa{Iː}\\	
31&	1.30&	2.28&	\ipa{I}&	\ipa{I}&	\ipa{ɨ ʷɨ}&	\ipa{ɪ̠ uɪ̠}&	\ipa{jɨ}&	\ipa{Iː}\\	
32&	1.31&	&	\ipa{ə+C}&	\ipa{}&	\ipa{ʉɴ}&	\ipa{ə̠̂ uə̠̂}&	\ipa{əə}&	\ipa{I’}\\	
33&	1.32&	2.29&	\ipa{i̯ə+C}&	\ipa{i̯ə}&	\ipa{ɨɴ}&	\ipa{ǐə̠}&	\ipa{jɨɨ}&	\ipa{yI’}\\	
71&	1.68&	&	\ipa{?}&	\ipa{ə̣}&	\ipa{ʉ̣}&	\ipa{ə̠̣}&	\ipa{ə̣}&	\ipa{iq’}\\
72&	1.69&	2.61&	\ipa{?}&	\ipa{i̯ə̣}&	\ipa{ɨ̣}&	\ipa{ǐə̠̣}&	\ipa{jɨ̣}&	\ipa{iːq’}\\
90&	1.84&	2.76&	\ipa{}&	\ipa{}&	\ipa{ʉr}&	\ipa{uə̠̣}&	\ipa{ər}&	\ipa{Ir}\\
91&	1.85&	&	\ipa{ə̣}&	\ipa{ə̣}&	\ipa{ər}&	\ipa{ue̠̣}&	\ipa{iər}&	\ipa{yIr}\\
92&	1.86&	2.77&	\ipa{i̯ə̣}&	\ipa{i̯ə̣}&	\ipa{ɨr}&	\ipa{er}&	\ipa{jɨr}&	\ipa{Iːr}\\
100&	1.92&	2.85&	\ipa{i̯ẹ}&	\ipa{i̯ẹ}&	\ipa{ɨr}&	\ipa{ǐe̠̣}&	\ipa{jɨɨr}&	\ipa{yIr}\\
\bottomrule
\end{tabular}}
\end{table}
Les rimes 30 et 31 étant en distribution complémentaire, Gong les reconstruit de la même façon. Comme nous le mentionnons p.\pageref{analyse:rime.1.92}, il semble vraisemblable que la rime 1.92/2.85 n'avait pas de voyelle rhotacisée, mais plutôt une voyelle tendue. On la traitera comme telle dans la reconstruction, et on reconstruira une préinitiale *S-- plutôt que *r--.            

Parmi les mots tangoutes de ce \ipa{shè}, certains correspondent à des rimes en syllabe fermées (finale --r, --s, --t et --p) en japhug ou en tibétain:


\begin{longtable} {lllllll}
\captionb{Comparaison des étymons en \ipapl{--ə} et \ipapl{--ɨ} du tangoute avec des formes à syllabes fermées en japhug et en tibétain.}\label{tab:comparaisons:eu:VC} \\
\toprule
&\multicolumn{2}{c}{tangoute}& &  japhug & sens  &tibétain\\
\midrule
\endfirsthead
\tinynb{5136}&\tgf{5136}&\ipa{.jɨ}&\tinynb{2.28}&\ipa{nɯʑɯβ}&   dormir &\\
\tinynb{4662}&\tgf{4662}&\ipa{dʑjɨ̣}&\tinynb{1.67}&\ipa{ndʑɤβ}&   brûler &\\
\tinynb{2568}&\tgf{2568}&\ipa{rər}&\tinynb{1.84}&\ipa{tʂɯβ}&  coudre  &ɴdrub\\
\midrule
\tinynb{5037}&\tgf{5037}&\ipa{bjɨr}&\tinynb{1.86}&\ipa{mbrɯtɕɯ}& couteau   &\\
\tinynb{1254}&\tgf{1254}&\ipa{dʑjwɨr}&\tinynb{1.86}&\ipa{ɣndʑɯr}&   meule &\\
\tinynb{65}&\tgf{0065}&\ipa{gjwɨr}&\tinynb{2.77}&\ipa{tɯ-mgɯr}&   dos &\\
\tinynb{4543}&\tgf{4543}&\ipa{mər}&\tinynb{1.84}&\ipa{tɯ-ɣmɤr}&  bouche  &\\
\tinynb{2739}&\tgf{2739}&\ipa{tɕhjwɨr}&\tinynb{2.77}&\ipa{tɕur}&   acide &skʲur.mo\\
\tinynb{2464}&\tgf{2464}&\ipa{tswər}&\tinynb{1.84}&\ipa{ftsɯr}&    traire&\\
\midrule
\tinynb{3582}&\tgf{3582}&\ipa{kjɨɨr}&\tinynb{2.85}&\ipa{tɯ-ɕkrɯt}&  bile  &mkʰris pa\\
\tinynb{5932}&\tgf{5932}&\ipa{mə}&\tinynb{2.25}&\ipa{(ʁnɯz)-mɯz}&   espèce &\\
\tinynb{4027}&\tgf{4027}&\ipa{njɨɨ}&\tinynb{1.32}&\ipa{ʁnɯz}& deux   &gɲis\\
\tinynb{5570 & \tgf{5570}&\ipa{ŋwə}&\tinynb{1.27}&\ipa{nɯndzɯlŋɯz}& s'assoupir   & \\
\tinynb{2778}&\tgf{2778}&\ipa{rjɨr}&\tinynb{1.86}&\ipa{ɕɤrɯ}&   os &rus\\
\tinynb{5105}&\tgf{5105}&\ipa{tsə̣}&\tinynb{1.68}&\ipa{tɯ-rtshɤz}& poumon   &\\
\midrule
\tinynb{2472}&\tgf{2472}&\ipa{gjwɨ}&\tinynb{1.30}&\ipa{ngɯt}& solide   &\\
\tinynb{2128}&\tgf{2128}&\ipa{məə}&\tinynb{1.31}&\ipa{ɣɤmɯt}& souffler   &\\
\tinynb{2325}&\tgf{2325}&\ipa{mjɨ̣}&\tinynb{2.61}&\ipa{jmɯt}&   oublier &\\
\tinynb{5192}&\tgf{5192}&\ipa{njwɨ̣}&\tinynb{2.61}&\ipa{nɯt}&   brûler &\\
\tinynb{2552}&\tgf{2552}&\ipa{tɕhiə}&\tinynb{1.28}&\ipa{tɕɤt}&   retirer &\\
\tinynb{99}&\tgf{0099}&\ipa{thjwɨ}&\tinynb{1.30}&\ipa{sthɯt}&   finir &\\
\tinynb{2367}}&\tgf{2367}&\ipa{tshjɨ}&\tinynb{1.30}&\ipa{tshɤt}&  chèvre  &\\
\bottomrule
\end{longtable}

Dans la reconstruction de Gong Hwangcherng, les symboles \textit{ə} et \textit{ɨ} sont en distribution complémentaire, et nous les considèrerons comme une seule entité.

En tangoute, les rimes à syllabe  fermée  ayant une finale autre que *--k et dont la voyelle principale n'était pas *a se confondent. Seule la voyelle *u reste distincte (devant les consonnes initiales non labiales) en donnant une syllabe à médiane --w--, comme \tgf{5192} \ipa{njwɨ̣} ``brûler'' du pré-tangoute *S-nut. On proposera dans le cadre de ce travail des reconstructions où seule la consonne finale est reconstruite, et la voyelle simplement indiquée par un \textit{v} (où v signifie ``voyelle autre que *a ou *u).

On trouve aussi des cas où les mots à rimes du \ipa{shè} 6 correspondent à des formes à syllabe ouverte avec des voyelles fermées en japhug ou en tangoute:




\begin{longtable} {lllllll}
\captionb{Comparaisons des étymons en \ipapl{--ə} et \ipapl{--ɨ} du tangoute avec des formes à syllabes ouvertes en japhug et en tibétain.}\label{tab:comparaisons:eu:V} \\
\toprule
&\multicolumn{2}{c}{tangoute}& &  japhug & sens  &tibétain\\
\midrule
\endfirsthead

\tinynb{320}&\tgf{0320}&\ipa{.wəə}&\tinynb{1.31}&\ipa{mpɯ}&  mou  &\\
\tinynb{1888}&\tgf{1888}&\ipa{bə}&\tinynb{2.25}& &   ver &ɴbu\\
\tinynb{597}&\tgf{0597}&\ipa{ɣjɨ}&\tinynb{1.29} &  \ipa{ta-kû} (situ) &oncle maternel &a.kʰu\\
\tinynb{3113}&\tgf{3113}&\ipa{gjɨɨ}&\tinynb{1.32}&\ipa{kɯngɯt}& neuf   &dgu\\
\tinynb{3688}&\tgf{3688}&\ipa{gjwɨr}&\tinynb{1.86}&\ipa{nɯrŋgɯ}&  s'allonger  &\\
\tinynb{3517}&\tgf{3517}&\ipa{khiwə}&\tinynb{1.28}&\ipa{taʁrɯ}&   corne &ru\\
\tinynb{74}&\tgf{0074}&\ipa{khwə}&\tinynb{1.27}&\ipa{ɯ-qiɯ}&   moitié &\\
\tinynb{5817}&\tgf{5817}&\ipa{kjwɨɨr}&\tinynb{1.92}&\ipa{mɯrkɯ}&  voler  &rku\\
\tinynb{860}&\tgf{0860}&\ipa{kwər}&\tinynb{1.84}&\ipa{tɯ-skhrɯ}&  corps  &sku\\
\tinynb{5845 }&\tgf{5845}&\ipa{lwə}&\tinynb{2.25}& \ipa{χtɯ}&  acheter &\\
\tinynb{3513}&\tgf{3513}&\ipa{mə}&\tinynb{1.27}&\ipa{tɯ-mɯ}&   ciel &\\
\tinynb{1999}&\tgf{1999}&\ipa{ŋwə}&\tinynb{1.27}&\ipa{kɯmŋu}&cinq    &lŋa\\
\tinynb{5274}&\tgf{5274}&\ipa{pə̣}&\tinynb{1.68}&\ipa{tɤ-spɯ}&  pus  &\\
\tinynb{5950}&\tgf{5950}&\ipa{phə}&\tinynb{1.27}&\ipa{ɯ-phɯ}&   prix &\\
\tinynb{3234}&\tgf{3234}&\ipa{phə}&\tinynb{1.27}& &  aîné & pʰu-bo\\
\tinynb{432}&\tgf{0432}&\ipa{sjwɨ}&\tinynb{2.28}&\ipa{ɕu}&  qui  &su\\
\tinynb{70}&\tgf{0070}&\ipa{thjwɨ}&\tinynb{1.30}&\ipa{cɯ}&  ouvrir  &\\
\tinynb{5518}&\tgf{5518}&\ipa{thwɨ̣}&\tinynb{1.69}&\ipa{tɯ-ɕtɯ}& vagin   &stu\\
\tinynb{4796}& \tgf{4796} & \ipa{zjɨr} &\tinynb{1.86} & \ipa{zrɯ} &adret& \\
\midrule
\tinynb{923}&\tgf{0923}&\ipa{.wə̣}&\tinynb{1.68}&\ipa{mbe}&  vieux  &\\
\tinynb{1153}&\tgf{1153}&\ipa{dʑjɨ}&\tinynb{1.3&}\ipa{tɯ-ndʐi}&   peau &\\
\tinynb{1200}&\tgf{1200}&\ipa{khjwɨ}&\tinynb{1.3&}\ipa{khɯna}&   chien &kʰʲi\\
\tinynb{4565}&\tgf{4565}&\ipa{lə}&\tinynb{2.25}& \ipa{mdzadi}& puce   &ldʑi.ba\\
\tinynb{5667}&\tgf{5667}&\ipa{lhjɨ̣}&\tinynb{1.69}&\ipa{tɯ-di, zdi}&   flèche &\\
\tinynb{2302}&\tgf{2302}&\ipa{ljɨ}&\tinynb{1.29}&\ipa{qale}&  vent  &\\
\tinynb{2737}&\tgf{2737}&\ipa{ljɨɨ}&\tinynb{1.32}&\ipa{rʑi}&   lourd &ldʑid po\\
\tinynb{2205}&\tgf{2205}&\ipa{ljɨɨr}&\tinynb{1.92}&\ipa{kɯβde}&  quatre  &bʑi\\
\tinynb{4408}&\tgf{4408}&\ipa{məə}&\tinynb{1.31}&\ipa{smi}&   feu &me\\
\tinynb{4574}&\tgf{4574}&\ipa{mjɨ}&\tinynb{1.30}&\ipa{tɯ-rme}&  homme  &mi\\
\tinynb{5382}&\tgf{5382}&\ipa{mjɨ}&\tinynb{2.28} &\ipa{tɯ-mi}&  jambe  & \\
\tinynb{3894}&\tgf{3894}&\ipa{njɨ}&\tinynb{1.30}&\ipa{tɤ-ɲi}&   tante paternelle &\\
\tinynb{2440}&\tgf{2440}&\ipa{njɨɨ}&\tinynb{2.29}&\ipa{}& jour   &ɲi.ma\\
\tinynb{654}&\tgf{0654}&\ipa{ŋwər}&\tinynb{2.76}&\ipa{arŋi}&   bleu &sŋo\\
\tinynb{4880}&\tgf{4880}&\ipa{rər}&\tinynb{2.76}& &   cuivre & gri\\
\tinynb{2511}&\tgf{2511}&\ipa{rjɨr}&\tinynb{2.77}&\ipa{ari}&   aller &\\
\tinynb{4273}&\tgf{4273}&\ipa{tsə̣}&\tinynb{1.68}&\ipa{}&   médicament &rtsi ?\\
\tinynb{3158 }& \tgf{3158}&\ipa{ŋwər}&\tinynb{2.76}&\ipa{tɯ-rŋa}&   visage &ŋo ?\\
\midrule
\tinynb{5168}&\tgf{5168}&\ipa{.jɨ}&\tinynb{2.28}&\ipa{japa / jɯfɕɯr}&  hier  &\\
\tinynb{5186}&\tgf{5186}&\ipa{tshjɨ}&\tinynb{2.28}&\ipa{}&   sel &tsʰa\\
\bottomrule
\end{longtable}




Pour ces cas, on reconstruira des voyelles fermées *u et *i en pré-tangoute. Un changement en chaîne massif a eu lieu en tangoute, les anciennes voyelles fermées devenant des schwa, et les voyelles mi-ouvertes *o et *e prenant leur place (voir la reconstruction des  \ipa{shè} 1 et 2 respectivement):
\newline
\begin{tabular} {lll} 
\ipapl{*--ji} &> &\ipapl{--jɨ}\\
\ipapl{*--je} &> &\ipapl{--ji}\\
\ipapl{*--ju} &> &\ipapl{--jwɨ}\\
\ipapl{*--jo} &> &\ipapl{--ju} \\
\end{tabular}
\linebreak
Les voyelles pré-tangoutes *u et *i restent toutefois distincts, car le *u donne une syllabe à médiane --w--, tandis que cette médiane n'apparaît pas dans les syllabes dont la rime provient de *i, à moins qu'elle soit dérivée d'un  autre   segment en proto-tangoute (labiovélaire, préinitiale *p--).

On reconstruira une médiane *--r-- en pré-tangoute pour les syllabes de la rime 29 \ipapl{--iə} devant les initiales vélaires et peut-être labiales. Le seul exemple probant de cette règle pour le moment est \tgf{3517} \ipa{khiwə¹} (<*khru) ``corne'' comparé au \jpg{ta-ʁrɯ} ``corne'' (<*qru).

\subsubsection{Tangoute \ipapl{--ə/--jɨ} :: tibétain  --ub, --ib :: japhug \ipapl{--ɯβ/--ɤβ}}	\label{subsubsec:correspondance:eu:vp}
On reconstruira *--vp en pré-tangoute pour les mots appartenant à cette correspondance. Noter que la rime *--op du pré-tangoute donne --ew et ne se confond pas avec les autres voyelles dans cette rime (voir \ref{subsec:voyelle.ew}).
\begin{enumerate}


\item 5136 \tgf{5136} \ipa{.jɨ} 2.28 \ptang{jvp}{dormir} est cognat avec le \jpg{nɯʑɯβ} ``dormir'' et le birman  \ipa{ip} (\plb{yìp}{0735}) de même sens (voir \citet[354]{matisoff03}; cet auteur cite le \tib{jib} ``se cacher'' comme apparenté à la forme birmane, mais ce rapprochement est très peu plausible).


\item 4662 \tgf{4662} \ipa{dʑjɨ̣} 1.67 \ptang{S-ndʑvp}{brûler} se compare au \jpg{ndʑɤβ}  (*<\ipapl{ndʑɔp}). En japhug, ce verbe est la forme anticausative de \jpg{tɕɤβ} ``mettre le feu à''\footnote{En japhug, ce verbe peut s'employer aussi bien à propos d'un champs que d'un habit.}. En tangoute, la forme transitive correspondante est 5983 \tgf{5983} \ipa{tɕhjɨ̣} 1.69 \ptang{S-tɕhvp}{brûler}.


\item 2568 \tgf{2568} \ipa{rər} 1.84 \ptang{rvp}{coudre} est apparenté au \jpg{tʂɯβ}, au \tib{ɴdrub}, au naish \naxi{ʐv̩˧},	\nayn{ʐv̩˥},	\laze{ʐv̩˩},	\protona{*C-ru} et au \bir{khyup} (reconstruit en \plb{gyùp}{0680}) de même sens (voir \citet[369-70]{matisoff03}.

\end{enumerate}
\subsubsection{Tangoute \ipapl{--ə/--jɨ} :: tibétain  --ud, --id :: japhug \ipapl{--ɯt/--ɤt}}	\label{subsubsec:correspondance:eu:vt}
On reconstruira *--vt ou *--ut en pré-tangoute pour les mots appartenant à cette correspondance.
\begin{enumerate}


\item 2128 \tgf{2128} \ipa{məə} 1.31 \ptang{mvvt}{souffler} est cognat avec le \jpg{ɣɤmɯt} ``souffler'' (<*mot, cf. situ \ipa{mót} ``boire'') et le \plb{s-mút}{0690} (\bir{hmut}). Une relation avec le \tib{ɴbud, bus} ``souffler'' est en revanche, contrairement à la proposition de \citet[364]{matisoff03}, peu probable car ni l'initiale ni la finale ne correspondent. Le --\textit{d} dans \ipa{ɴbud} est en effet le suffixe du présent et non une partie de la racine.


\item 2325 \tgf{2325} \ipa{mjɨ̣} 2.61 \ptang{S-mjvt}{oublier}  correspond au \jpg{jmɯt} ``oublier'' (<*lmit), au naish \naxi{le˧mi˥}	\nayn{mv̩.pʰæ^L^+^M^H}	\laze{læ˥mv̩˩}	\protona{mi}
et au \plb{ʔ-me³}{0591}.



\item 99 \tgf{0099} \ipa{thjwɨ} 1.30 \ptang{thut}{finir}  se compare au \jpg{sthɯt}  ``finir''. C'est un verbe transitif aussi bien en japhug  (\citealt[352]{jacques08}) qu'en tangoute, où il peut apparaître avec un objet nominal dans le sens d'``achever'':
\newline
\linebreak
\begin{tabular}{llllllllll}
	\tgf{5925}&	\tgf{4620}&	\tgf{4971}&	\tgf{1084}&	\tgf{4027}&	\tgf{2226}&	\tgf{3535}&	\tgf{1770}&	\tgf{2412}&	\tgf{5981}\\
\tinynb{5925}&	\tinynb{4620}&	\tinynb{4971}&	\tinynb{1084}&	\tinynb{4027}&	\tinynb{2226}&	\tinynb{3535}&	\tinynb{1770}&	\tinynb{2412}&	\tinynb{5981}\\
\tgf{0099}& &&&&&&&&\\
\tinynb{0099}& &&&&&&&&\\
\end{tabular}
\begin{exe}
\ex \label{ex:tg:finir}  \vspace{-8pt}
\gll   \ipa{tsə¹kja¹}	\ipa{ɕjwi¹}	\ipa{ɣạ²njɨɨ ¹}	\ipa{.we²}	\ipa{ŋwu²}	\ipa{lhjwi¹}	\ipa{dzjo²}	\ipa{.a-thjwɨ¹} \\
		Zijian année douze devenir pinceau prendre poème \dir{}-finir[A] \\
\glt Zijian, qui avait douze ans, se saisit d'un pinceau et réalisa un poème. (Leilin 07.02B.7)
\end{exe}
La forme alternante 48 \tgf{0048} \ipa{thjwu} 2.03 n'est pas attestée dans des textes, mais est associée à  \tgf{0099} \ipa{thjwɨ¹}  dans le Tongyin. Il doit s'agir du thème B. On peut le reconstruire *\ipapl{thut-u} avec le suffixe de troisième personne objet. (voir \ref{subsubsec:origine.alternances}).


\item 2367 \tgf{2367} \ipa{tshjɨ} 1.30 \ptang{tshjvt}{chèvre} peut se comparer \jpg{tshɤt} ``chèvre'' et au \plb{(k)-cìt}{0004}, \bir{chit} (voir \citealt[350]{matisoff03}).


\item 5192 \tgf{5192} \ipa{njwɨ̣} 2.61 \ptang{S-nut}{brûler}  est cognat avec le \jpg{nɯt} ``brûler (intr.)'' (<*not, cf. zbu \ipa{snôt}, voir \citealt[252]{jacques04these}). La forme tangoute a le préfixe causatif *S- comme la forme du zbu, et il s'agit contrairement au japhug d'un verbe transitif:
\newline
\linebreak
\begin{tabular}{llllllllll}
	\tgf{4342}&	\tgf{5834}&	\tgf{5814}&	\tgf{1417}&	\tgf{0448}&	\tgf{2226}&	\tgf{4342}&	\tgf{0390}&	\tgf{1326}&	\tgf{5192}\\
\tinynb{4342}&	\tinynb{5834}&	\tinynb{5814}&	\tinynb{1417}&	\tinynb{0448}&	\tinynb{2226}&	\tinynb{4342}&	\tinynb{0390}&	\tinynb{1326}&	\tinynb{5192}\\
\tgf{2019}&	\tgf{1087}&	\tgf{1567}&	\tgf{0089}&	\tgf{1326}&	\tgf{5692}& &&&\\
\tinynb{2019}&	\tinynb{1087}&	\tinynb{1567}&	\tinynb{0089}&	\tinynb{1326}&	\tinynb{5692}& &&&\\
\end{tabular}
\begin{exe}
\ex \label{ex:tg:allumer}  \vspace{-8pt}
\gll   \ipa{dja²-lej²}	\ipa{phu²}	\ipa{rowr¹}	\ipa{gjɨ²}	\ipa{.we²}	\ipa{dja²-khjwɨ¹}	\ipa{kjɨ¹-njwɨ̣²}	\ipa{thja¹}	\ipa{dzjiij²gji²}	\ipa{tɕhjaa¹}	\ipa{kjɨ¹-swew¹} \\
	\dir{}-changer arbre desséché un devenir \dir{}-couper \dir{}-brûler ce lettré sur \dir{}-éclairer \\
\glt Il se changea en un arbre desséché, et on le coupa et le fit brûler pour éclairer ce lettré. (Leilin 06.30B.6-7)
\end{exe}


\item 2552 \tgf{2552} \ipa{tɕhiə} 1.28 \ptang{tɕhvt}{retirer} peut se comparer au \jpg{tɕɤt} ``enlever, retirer''. On ne dispose malheureusement pas d'exemple textuels en tangoute.


\item 2472 \tgf{2472} \ipa{gjwɨ} 1.30 \ptang{ŋgut}{solide} correspond au \jpg{ngɯt} ``solide''. 
\newline
\linebreak
\begin{tabular}{llllllllll}
\tgf{1531} & 	\tgf{5993} & 	\tgf{2541} & 	\tgf{0448} & 	\tgf{5604} & 	\tgf{5113} & 	\tgf{2472} & 	\tgf{4898} & 	\tgf{4342} & 	\tgf{2393} \\
\tinynb{1531} & 	\tinynb{5993} & 	\tinynb{2541} & 	\tinynb{0448} & 	\tinynb{5604} & 	\tinynb{5113} & 	\tinynb{2472} & 	\tinynb{4898} & 	\tinynb{4342} & 	\tinynb{2393} \\
\tgf{5113} & 	\tgf{3213} & 	\tgf{1531} & 	\tgf{4342} & 	\tgf{1508} & 	 \\
\tinynb{5113} & 	\tinynb{3213} & 	\tinynb{1531} & 	\tinynb{4342} & 	\tinynb{1508} & 	 \\
\end{tabular}
\begin{exe}
\ex \label{ex:tg:solide}  \vspace{-8pt}
\gll  \ipa{gja¹} 	\ipa{kha¹} 	\ipa{dzjwo²} 	\ipa{gjɨ²} 	\ipa{dʑjɨ.wji¹} 	\ipa{gjwɨ¹} 	\ipa{zjir²} 	\ipa{dja²-ljiij²-.wji¹} 	\ipa{tsjĩ¹} 	\ipa{gja¹} 	\ipa{dja²-bej¹} 	 \\
armée intérieur homme un \erg{} solide cuirasse \dir{}-détruire-faire[A] Jin armée \dir{}-vaincre \\
\glt Dans l'armée (de Chu), un homme détruit sa solide cuirasse (s'élança à l'assaut sans égard pour sa vie) et vainquit l'armée de Jin (Leilin 07.22B.2-3)
\end{exe}



\end{enumerate}




\subsubsection{Tangoute \ipapl{--ə/--jɨ} :: tibétain  --us, --is :: japhug \ipapl{--ɯz/--ɤz}}	\label{subsubsec:correspondance:eu:vs}
On reconstruira *--vt en pré-tangoute pour les mots appartenant à cette correspondance. En effet, on ne tentera pas de reconstruire une finale *--s distincte de  *--t car si elles ont un jour été distinctes, les rimes qui pourraient avoir un *--s final ont toutes les mêmes correspondances que celles à *--t final, et ceci quelle que soit la voyelle.
\begin{enumerate}


\item 5932 \tgf{5932} \ipa{mə} 2.25 \ptang{mvt}{espèce, sorte} peut se comparer avec la seconde syllabe de l'expression \jpg{ʁnɯz-mɯz}  ``deux sortes'' qui apparaît dans certaines histoires traditionnelles\footnote{L'exemple \ref{ex:jpg:espece} n'est pas en dialecte japhug pur. Dans le conte d'où elle est tirée, il s'agit des paroles prononcées par un oiseau. Le verbe ``savoir'' \ipa{mɤnɯɕaŋ} a une forme inanalysable (on attendrait plutôt \ipa{mɤxsi} ou \ipa{mɤnɯxsi}). Elle ressemble à une forme de première personne singulier, provenant peut-être d'un dialecte tel que le tshobdun.}:
\begin{exe}
\ex \label{ex:jpg:espece}  \vspace{-8pt}
\gll \ipa{tɕɯtɕɯ}	\ipa{kɯ-lɤɣ} \ipa{acɤβ},	\ipa{nɤ-rɟɤlpu}	\ipa{khe}	\ipa{nɤ}	\ipa{mɤ-nɯ-ɕaŋ}	\ipa{ɕqraʁ}	\ipa{nɤ}	\ipa{mɤ-nɯ-ɕaŋ},	\ipa{khɤdi}	\ipa{ʁnɯzmɯz}	\ipa{nɯ-antsɤndu}	\ipa{mɤ-kɯ-nɯ-sɯ-rtoʁ}	\\
   gazouillement \nmls{}:A-faire.paître Askyabs 2\sgposs{}-roi \nonps{}:stupide \conj{} \negat{}-\auto{}-savoir \nonps{}:intelligent  \conj{} \negat{}-\auto{}-savoir place.de.la.maîtresse.de.maison deux.sorte \aor{}-être.échangé.par.erreur \negat{}-\nmls{}:A-\auto{}-\textsc{abilitatif}-s'apercevoir \\
\glt Piou piou, pâtre Askyabs, je ne sais si ton roi est stupide ou intelligent, mais il ne s'est pas rendu compte qu'à la place réservée à son épouse, deux personnes ont été échangées. (La grenouille, 154-5)
\end{exe}
On retrouve le correspondant exact de \ipa{ʁnɯz-mɯz} en tangoute, \tgf{4027}\tgf{5932}\ipa{njɨɨ¹mə²}	 comme dans l'exemple suivant:
\newline
\linebreak
\begin{tabular}{llllllll}
	\tgf{5354}&	\tgf{4027}&	\tgf{5932}&	\tgf{0981}&	\tgf{0083}&	\tgf{2539}&	\tgf{2090}&	\tgf{0508}\\
\tinynb{5354}&	\tinynb{4027}&	\tinynb{5932}&	\tinynb{0981}&	\tinynb{0083}&	\tinynb{2539}&	\tinynb{2090}&	\tinynb{0508}\\
\end{tabular}
\begin{exe}
\ex \label{ex:tg:espece}  \vspace{-8pt}
\gll   \ipa{thjɨ²}	\ipa{njɨɨ¹}	\ipa{mə²}	\ipa{.war²}	\ipa{.we¹}	\ipa{kjạ¹}	\ipa{lew²}	\ipa{ŋwu²} \\
		ces deux sorte objet dragon effrayer[A] \nmls{} être \\
\glt Ces deux types d'objets sont des choses qui effraient le dragon. (Leilin 03.23A.7)
\end{exe}


\item 5105 \tgf{5105} \ipa{tsə̣} 1.68 \ptang{S-tsvt}{poumon} peut se comparer au \jpg{tɯ-rtshɤz} ``poumon''  (<*\ipapl{rtshos}). On retrouve des cognats dans les langues naish \naxi{ʈʂʰʷə˞˥},	\nayn{ʈʂʰɻ˧˥},	\laze{tsʰv̩˩},	\protona{*rtsʰU} et en LB où deux reconstructions \plb{tsi²}{0141a} et \plb{tsút}{0141a} sont proposées. Un rapport avec \tgz{4615} ``tousser'' semble probable, peut-être par nominalisation *S-tso-s > *S-tsvt ``celui qui tousse''.


\item 4027 \tgf{4027} \ipa{njɨɨ} 1.32 \ptang{njvvt}{deux} est le numéral commun aux langues ST. On le retrouve en \jpg{ʁnɯz} et en \tib{gɲis}.


\item 2778 \tgf{2778} \ipa{rjɨr} 1.86 \ptang{rjvt}{os} peut être rapproché du \tib{rus} ``os'' et d'autres formes apparentées dans de nombreuses langues, dont le \plb{ʃə-ro²}{0136} (voir les cognats cités dans \citealt[435]{matisoff03}). Le \jpg{ɕɤrɯ} est peut-être un emprunt au \tib{ɕa.rus} ``viande et os'', peut-être un cognat, mais en tous les cas l'absence de finale --s est inexplicable.


\item 3582 \tgf{3582} \ipa{kjɨɨr} 2.85 \ptang{S-krjvvt}{bile}, voir la discussion p.\pageref{tab:cycle2japhugrmediane}) peut se comparer au \jpg{tɯ-ɕkrɯt} et au \tib{mkʰris.pa} de même sens. Le -t final du japhug provient parfois d'un proto-japhug *--s, mais le conditionnement n'a pas encore été élucidé (\citealt[376]{jacques08}). \citet[453]{matisoff03} cite des langues où des cognats potentiels du \tib{mkʰris.pa} et du tangoute \tgf{3582} \ipa{kjɨɨr²} n'ont pas de trace de --s final.

\item 5570  \tgf{5570} \ipa{ŋwə} 1.27 \ptang{ŋut}{s'assoupir} est comparable au \jpg{nɯndzɯlŋɯz} ``s'assoupir''. Ce mot ne semble pas avoir de cognats ailleurs, et n'est pas attesté dans les textes.


\end{enumerate}
\subsubsection{Tangoute \ipapl{--ə/--jɨ} :: tibétain  --ur, --ir :: japhug \ipapl{--ɯr/--ɤr}}	\label{subsubsec:correspondance:eu:vr}
Tous les exemples appartenant à cette correspondance sont du deuxième cycle mineur. On reconstruira *--vr ou *--ur en pré-tangoute pour les autres mots appartenant à cette correspondance.
\begin{enumerate}


\item 5037 \tgf{5037} \ipa{bjɨr} 1.86 \ptang{mbjvr}{couteau} est comparable à la première syllabe du \jpg{mbrɯtɕɯ} ``couteau'' (<**\ipapl{mbɯr}, voir \citealt[:278]{jacques04these}).



\item 4543 \tgf{4543} \ipa{mər} 1.84 \ptang{mvr}{bouche} peut se comparer au \jpg{tɯ-ɣmɤr} ``bouche''. \citet[397]{matisoff03} propose une liste de cognats potentiels, dont les formes tibétaines mur ``mâcher'' et mur.gong ``tempes''; il est encore incertain qu'elles soient réellement apparentées aux noms japhug et tangoute. La forme \tgf{0730} \ipa{mu¹} ``bouche'' est probablement apparentée à \tgf{4543} \ipa{mər¹}, mais le procédé morphologique qui les relie est opaque.



\item 2464 \tgf{2464} \ipa{tswər} 1.84 \ptang{p-tsvr}{traire} est cognat avec le \jpg{ftsɯr} ``essorer''. Voir la discussion p.\pageref{ex:pu:traire}. La forme 3353  \tgf{3353} \ipa{dzər}  2.76, qui signifie aussi ``traire'' en est un dérivé, mais en l'absence d'exemple (hormis ceux des dictionnaires), il est difficile de s'assurer de la différence précise entre ces deux formes.



\item 1254 \tgf{1254} \ipa{dʑjwɨr} 1.86 \ptang{ndʑur}{moulin, meule} correspond au \jpg{ɣndʑɯr} ``moudre'', \situ{ta-ndzór} ``moulin'') Le verbe 617  \tgf{0617} \ipa{dʑiər} 1.85 ``écraser, broyer'' est peut-être apparenté, mais l'absence de médiane --w-- s'explique mal.
\newline
\linebreak
\begin{tabular}{llllll}
	\tgf{1254}&	\tgf{0089}&	\tgf{4176}&	\tgf{4176}&	\tgf{3559}&	\tgf{4916}\\
	\tinynb{1254}&	\tinynb{0089}&	\tinynb{4176}&	\tinynb{4176}&	\tinynb{3559}&	\tinynb{4916}\\
\end{tabular}
\begin{exe}
\ex \label{ex:tg:moudre}  \vspace{-8pt}
\gll   \ipa{dʑjwɨr¹}	\ipa{tɕhjaa¹}	\ipa{tju¹tju¹}	\ipa{ɣwə¹ɣwej¹} \\
		moulin sur tourterelle \recip{}:battre \\
\glt  Des tourterelles se battaient sur le moulin. (Leilin 06.08B.3)
\end{exe}


\item 2739 \tgf{2739} \ipa{tɕhjwɨr} 2.77 \ptang{tɕhur}{acide} peut se comparer au \jpg{tɕur} et au \tib{skʲur-mo} ``acide, aigre''. Le \plb{ʔ-kyin¹}{0549} n'est probablement pas apparenté du fait de la divergence de voyelle.


\item 65 \tgf{0065} \ipa{gjwɨr} 2.77  \ptang{ŋgur}{dos} peut être rapproché du \jpg{tɯ-mgɯr} ``dos''.
\end{enumerate}

\subsubsection{Tangoute \ipapl{--wə/--jwɨ} :: tibétain  --u :: japhug \ipapl{--ɯ/--u}}	\label{subsubsec:correspondance:eu:u}
On reconstruira *u en pré-tangoute pour les rimes des mots appartenant à cette correspondance. Devant les initiales labiales, la médiane labiale n'apparaît pas (l'opposition \textit{hékǒu} / \textit{kāikǒu} est neutralisée), et seule la comparaison avec les autres langues permet de distinguer les *u des *i. 

Deux mots à initiales vélaires (\tgz{3113} ``neuf'' et \tgz{0597}  ``oncle maternel'' ) appartenant à ce groupe ont une syllabe sans médiane --w--; on ne proposera pas de reconstruction pour ces formes inexplicables.
A. Initiales labiales \label{rimes:06:4:1:lab}
\begin{enumerate}


\item 5274 \tgf{5274} \ipa{pə̣} 1.68 \ptang{S-pu}{pus} peut se comparer au \jpg{tɤ-spɯ} ``pus''. On retrouve des cognats dans d'autres langues macro-rgyalronguiques, mais pas en LB ou en naish, où on trouve un étymon \plb{m-bliŋ¹}{0154}, \naxi{mbə˞˩}	\nayn{bæ˩˧}	\laze{bæ˩}	\protona{priN} qui ressemble superficiellement à \jpg{tɤ-spɯ} dans certaines langues.


\item 5950 \tgf{5950} \ipa{phə} 1.27 \ptang{phu}{prix} peut être rapproché du \jpg{ɯ-phɯ} ``prix'', du \plb{po²}{0421} et naish \naxi{kɑ˧pʰv̩˧},	\nayn{ʁɑ.pʰv̩^H},	\laze{ʁɑ˧pʰv̩˧},	\protona{pʰu}. 

Le nom 2984  \tgf{2984} \ipa{pjụ} 1.59  correspond parfois au nom chinois ``prix'' \zh{價} \textit{jià} dans les texteset semblerait être comparé également au \jpg{ɯ-phɯ}. Toutefois, il semble que c'est à l'origine un verbe signifiant ``estimer'' ou ``examiner'':
\newline
\linebreak
\begin{tabular}{lllllllll}
	\tgf{5354}&	\tgf{2205}&	\tgf{2403}&	\tgf{0010}&	\tgf{5882}&	\tgf{2403}&	\tgf{0433}&	\tgf{2984}&	\tgf{0113}\\
	\tinynb{5354}&	\tinynb{2205}&	\tinynb{2403}&	\tinynb{0010}&	\tinynb{5882}&	\tinynb{2403}&	\tinynb{0433}&	\tinynb{2984}&	\tinynb{0113}\\
\end{tabular}
\begin{exe}
\ex \label{ex:tg:estimer}  \vspace{-8pt}
\gll   \ipa{thjɨ²}	\ipa{ljɨɨr¹}	\ipa{dji²}	\ipa{zji²}	\ipa{zar¹}	\ipa{dji²}	\ipa{bju¹}	\ipa{pjụ¹}	\ipa{ɕjɨj¹} \\
		cela quatre caractère tout chinois caractère \instr{} examiner être.formé \\
\glt Ces quatre caractères sont tous formés sur  la base (d'autres) caractères chinois. (Leilin 04.29B.3)
\end{exe}
Ainsi, l'étymologie avec \tgz{5950} semble plus probable.


\item 3234 \tgf{3234} \ipa{phə} 1.27 \ptang{phu}{aîné} peut se comparer avec le \tib{pʰu-bo} ``aîné, grand frère''. Le \tib{nu-bo} ``petit frère'' à laquelle \ipa{pʰu-bo} s'oppose ne semble pas en revanche avoir de cognat en tangoute.



\item 1888 \tgf{1888} \ipa{bə} 2.25 ``ver''\ptang{mbu}{ver} ainsi que la forme à voyelle longue  5270 \tgf{5270} \ipa{bəə} 1.31 \ptang{mbuu}{ver} sont comparables au \plb{bu¹/²}{0080} et au \tib{ɴbu} ``ver''. Consulter la discussion p.\pageref{analyse:ver} et les exemples p.\pageref{ex:tg:bouillir} et p.\pageref{ex:tg:tisser}.



\item 320 \tgf{0320} \ipa{.wəə} 1.31 \ptang{C-puu}{mou, faible} est cognat avec le \jpg{mpɯ}  ``mou''. La première syllabe du composé \tgf{5042}\tgf{4074} \ipa{bə¹mej²}  ``mou, moëlleux, doux'' pourrait aussi y être apparentée; on proposerait dans ce cas un pré-tangoute *mbu.


\item 3513 \tgf{3513} \ipa{mə} 1.27 \ptang{mu}{ciel} est comparable au \jpg{tɯ-mɯ} ``ciel, pluie'' et au \bir{mui³}. Contrairement au birman et au japhug, le nom tangoute \tgf{3513} \ipa{mə¹} ne peut pas être employé dans le sens de ``pluie''. En tangoute, le mot ciel est également écrit 1977  \tgf{1977} \ipa{mə} 1.27 dans le sens de ``divinité''. Ce nom de divinité rappelle naturellement celui des \textit{Dmu} en tibétain ancien (voir \citealt{stein51minyag}). Il est difficile d'avoir une certitude sur la question, mais on ne peut exclure l'idée, suggérée par Stein, que ce nom de divinité en tibétain ancien soit un emprunt au tangoute ou tout au moins à une langue macro-rgyalronguique. Toutefois, il est possible aussi que \textit{Dmu} soit un cognat réel, hérité de l'ancien nom du ``ciel''. La phonologie historique ne permet pas de trancher en faveur de l'une ou l'autre de ces deux hypothèses à l'heure actuelle.


\end{enumerate}
B. Initiales dentales \label{rimes:06:4:1:dent}
\begin{enumerate}

\item 70 \tgf{0070} \ipa{thjwɨ} 1.30 \ptang{thju}{ouvrir}  peut se comparer au \jpg{cɯ} ``ouvrir'' (\situ{tû}, zbu \ipa{tɣwɐʔ}). Le sens précis du verbe tangoute et de son équivalent japhug ne sont toutefois pas entièrement compatibles. En japhug, ce mot s'emploie pour ouvrir une porte, alors qu'en tangoute c'est plutôt le verbe \tgf{5390} \ipa{phie²}  (voir p.\pageref{ex:tg:detacher:expliquer}) qui a ce sens. \tgf{0070} \ipa{thjwɨ¹} est utilisé dans le sens d'``ouvrir un chemin, ouvrir une brèche''. Il est attesté en particulier avec comme objet le nom 20 \tgf{0020} \ipa{tɕja} 1.19 ``chemin, route'' (six attestations dans Sunzi):
\newline
\linebreak
\begin{tabular}{llllll}
	\tgf{0289}&	\tgf{5218}&	\tgf{0021}&	\tgf{0070}\\
\tinynb{0289}&	\tinynb{5218}&	\tinynb{0021}&	\tinynb{0070}\\
\end{tabular}
\begin{exe}
\ex \label{ex:tg:ouvrir}  \vspace{-8pt}
\gll   \ipa{.we²}	\ipa{tɕhjwo¹}	\ipa{tɕja¹}	\ipa{thjwɨ¹} \\
		muraille traverser chemin ouvrir[A] \\
\glt Il ouvrit un chemin de passage dans la muraille. (Leilin 04.08B.6)
\end{exe}
En japhug, utilisé avec \ipa{tʂu} ``chemin'' comme objet, le verbe \ipa{cɯ} a un sens entièrement différent:
\begin{exe}
\ex \label{ex:jpg:ouvrir}  \vspace{-8pt}
\gll   \ipa{tʂu} \ipa{tɤ-cɯ-t-a} \\
		chemin \dir{}-ouvrir-\ps{}-1\sg{} \\
\glt Je lui ai laissé le chemin (je me suis écarté pour le laisser passer).
\end{exe}
La comparaison avec le japhug n'est donc pas entièrement satisfaisante du point de vue sémantique. Une autre possibilité serait le verbe \jpg{thɯ} qui peut signifier ``construire (un pont ou une route), monter une tente''.

On trouve également un exemple étrange où \tgf{0070} \ipa{thjwɨ¹} s'emploie avec \tgf{2639} \ipa{mjiij²} ``nom'' comme objet, où il signifie ``prendre le nom de, se prétendre'':
\newline
\linebreak
\begin{tabular}{llllllllll}
	\tgf{3738}&	\tgf{4921}&	\tgf{1760}&	\tgf{0151}&	\tgf{2627}&	\tgf{5258}&	\tgf{2983}&	\tgf{3444}&	\tgf{1746}&	\tgf{5880}\\
\tinynb{3738}&	\tinynb{4921}&	\tinynb{1760}&	\tinynb{0151}&	\tinynb{2627}&	\tinynb{5258}&	\tinynb{2983}&	\tinynb{3444}&	\tinynb{1746}&	\tinynb{5880}\\
\tgf{3830}&	\tgf{2639}&	\tgf{5981}&	\tgf{0070}& &&&&&\\
\tinynb{3830}&	\tinynb{2639}&	\tinynb{5981}&	\tinynb{0070}& &&&&&\\
\end{tabular}
\begin{exe}
\ex \label{ex:tg:ouvrir2}  \vspace{-8pt} 
\gll   \ipa{kow¹swẽ¹ɕjwɨ¹}	\ipa{ɕjuu¹}	\ipa{ljɨ̣².iọ¹}	\ipa{.u²}	\ipa{dzew²ljoor¹}	\ipa{ŋwu²}	\ipa{njij²}	\ipa{mjiij²}	\ipa{.a-thjwɨ¹} \\
		Gongsun.Shu Shu endroit dans tromper \instr{} roi nom \dir{}-ouvrir[A] \\
\glt Gongsun Shu (\zh{公孫述}) se prétendait frauduleusement roi de Shu (\zh{蜀}). (Leilin 03.24B.1-2)
\end{exe}


Le verbe \tgf{0070} \ipa{thjwɨ¹}  est associé avec le caractère \tgf{0018} \ipa{thjwu²} dans le Tongyin, qui n'est malheureusement pas attesté dans les textes connus. Il pourrait s'agir du thème B de \tgf{0070} \ipa{thjwɨ¹} employé avec un agent 1sg/2sg, auquel cas une reconstruction *thu-u serait possible. Dans le Tongyin, \tgf{0018} \ipa{thjwu²} est également associé au verbe \tgf{0091} \ipa{thew} 2.38 ``attraper une maladie'': le même caractère sert probablement à noter le thème 2 de deux verbes distincts (voir p.\pageref{rimes:09:dent}). Toutefois, ce problème ne pourra être résolu  que si des exemples de \tgf{0018} \ipa{thjwu²}  sont trouvés.



\item 5518 \tgf{5518} \ipa{thwɨ̣} 1.69 \ptang{S-thu}{vagin} est cognat du \jpg{tɯ-ɕtɯ} et le \tib{stu} de même sens (liste de cognats dans \citealt[247]{matisoff03}).



\item 432	\tgf{0432} \ipa{sjwɨ} 	2.28	et 441 \tgf{0441} \ipa{sjwɨ} 1.30 \ptang{su}{qui} peuvent se comparer au \jpg{ɕu} ``qui'' au \tib{su} et au \plb{ʔəsu¹}{0430}(voir \citet[180-1]{matisoff03} pour une liste de cognats dans les langues lolo-birmanes). La différence grammaticale entre \tgf{0432} \ipa{sjwɨ²} et \tgf{0441} \ipa{sjwɨ¹}  n'est pas claire; on dispose de peu d'exemples de \tgf{0432} \ipa{sjwɨ²}.


4796 \tgf{4796} \ipa{zjɨr} 1.86 \ptang{sru}{sud}, voir p.\pageref{analyse:sr}) est apparenté au \jpg{zrɯ} ``adret''. L'absence de médiane --w-- en tangoute est inattendue, mais comme aucune syllabe de type *\ipapl{zjwɨr} n'existe, il s'agit peut-être d'un contexte où cette médiane ne peut apparaître.


\item 5845 \tgf{5845} \ipa{lwə} 2.25 peut s'interpréter de deux façons. On peut soit proposer un \ptang{lu}{acheter} et le rapprocher du verbe \tib{blu} ``racheter'', soit reconstruire \ptang{C-tu}{acheter} avec lénition et le comparer avec \jpg{χtɯ} ``acheter". La seconde hypothèse semble la plus probable, car une forme  lénition s'observe également en rtau \ipa{rə} ``acheter".

\end{enumerate}
C. Initiales vélaires \label{rimes:06:4:1:vel}
\begin{enumerate}


\item 5817 \tgf{5817} \ipa{kjwɨɨr} 1.92 \ptang{r-kjuu}{voler, voleur} est apparenté au \jpg{mɯrkɯ} ``voler'', au \plb{ko²}{0615} et au \tib{rku} ``voler''. Il est employé comme nom dans le sens de ``brigand, voleur''. Ce verbe a deux formes alternantes 5904 \tgf{5904} \ipa{kjur} 2.70 et 1855 \tgf{1855} \ipa{kjir} 2.86, la dernière étant associée à \tgf{5817} \ipa{kjwɨɨr¹} dans le Tongyin. On dispose d'un exemple textuel :
\newline
\linebreak
\begin{tabular}{llllllllll}
\tgf{1999} & 	\tgf{2541} & 	\tgf{4950} & 	\tgf{0756} & 	\tgf{2104} & 	\tgf{4342} & 	\tgf{1427} & 	\tgf{0764} & 	\tgf{2019} & 	\tgf{4884} \\
gf{2104} & 	\tinynb{4342} & 	\tinynb{1427} & 	\tinynb{0764} & 	\tinynb{2019} & 	\tinynb{4884} \\
\tgf{1326} & 	\tgf{5904} & 	\tgf{2590} & 	\tgf{4517} & 	\tgf{3916} & 	\tgf{0046} & \\
\tinynb{1326} & 	\tinynb{5904} & 	\tinynb{2590} & 	\tinynb{4517} & 	\tinynb{3916} & 	\tinynb{0046} & \\
\end{tabular}
\begin{exe}
\ex \label{ex:tg:voler}  \vspace{-8pt} 
\gll   \ipa{ŋwə¹} 	\ipa{dzjwo²} 	\ipa{rjir²} 	\ipa{dʑju²} 	\ipa{ɕji¹} 	\ipa{dja²-phji²} 	\ipa{rjijr¹} 	\ipa{thja¹} 	\ipa{nji²} 	\ipa{kjɨ¹-kjur²} 	\ipa{.wjɨ²-dzji¹-sji²} 	\ipa{ljij²}  \\
cinq homme \comit{} rencontrer auparavant \dir{}-perdre[A] cheval ceci \pl{} \dir{}-voler \dir{}-manger[A]-\ps{} voir[A] \\
\glt Il rencontra cinq hommes, et vit qu'ils avaient volé puis mangé le cheval qu'il avait perdu auparavant. (Leilin 07.21B.3)
\end{exe}

L'alternance entre \tgf{5817} \ipa{kjwɨɨr¹} et \tgf{5904} \ipa{kjur²} n'est manifestement pas liée à la personne, puisque dans l'exemple ci-dessus l'agent et le patient du verbe sont tous deux à la troisième personne. Il pourrait s'agir d'une marque de TAM, puisque \tgf{5904} \ipa{kjur²} apparaît à la forme perfective.
Par ailleurs, en zbu, le verbe ``voler'' se conjugue au thème 1 \ipa{mərkəʔ} et au thème 2 \ipa{mərkhiʔ} (voir \citealt[226]{jackson00puxi} ainsi que \citealt{jackson04showu}): la forme \tgf{1855} \ipa{kjir²} pourrait provenir d'un pré-tangoute *r-kje, qui correspondrait à la forme d'aoriste \ipa{mərkhiʔ} du zbu. 

Toutefois, on ne peut pas affirmer simplement que \tgf{5904} \ipa{kjur²}  ou  \tgf{1855} \ipa{kjir²} soit la forme du perfectif, car  \tgf{5817} \ipa{kjwɨɨr¹} peut aussi apparaître avec un préfixe directionnel de la première classe :
\newline
\linebreak
\begin{tabular}{llllllllll}
\tgf{3830} & 	\tgf{2098} & 	\tgf{1139} & 	\tgf{2583} & 	\tgf{0804} & 	\tgf{5817} & 	\tgf{5113} & 	\tgf{2098} & 	\tgf{1278} & \\
\tinynb{3830} & 	\tinynb{2098} & 	\tinynb{1139} & 	\tinynb{2583} & 	\tinynb{0804} & 	\tinynb{5817} & 	\tinynb{5113} & 	\tinynb{2098} & 	\tinynb{1278} & \\
\end{tabular}
\begin{exe}
\ex \label{ex:tg:voler2}  \vspace{-8pt} 
\gll  \ipa{njij²} 	\ipa{ŋa²} 	\ipa{.jij¹} 	\ipa{nji} 	\ipa{djɨ²-kjwɨɨr¹-.wji¹-ŋa²} 	\ipa{.jɨ²} \\
roi moi \gen{} perle \dir{}-voler-faire[A]-1\sg{} dire \\
\glt  Le roi dit ``Tu as volé ma perle." (Leilin 04.01B.6-7)
\end{exe}



\item 860 \tgf{0860} \ipa{kwər} 1.84 \ptang{r-ku}{corps} (ou plutôt *S-kru, voir l'explication p.\pageref{tab:cycle2japhugrmediane}) est comparable au \jpg{tɯ-skhrɯ} ``corps'' et au \tib{sku} ``corps (honorifique)''.


\item 3517 \tgf{3517} \ipa{khiwə} 1.28  \ptang{khru}{corne} est cognat avec le \jpg{ta-ʁrɯ} ``corne'', le \plb{kro¹}{0082}  et probablement aussi le \tib{ru, rʷa} ``corne''.\footnote{Concernant l'origine de l'alternance entre --u et --wa dans ce mot, voir \citet{jacques09wazur}.} \citet[182]{matisoff03} présente des cognats de \tgf{3517} \ipa{khiwə¹}.


\item 74 \tgf{0074} \ipa{khwə} 1.27   \ptang{khu ou *khwi}{moitié} peut se comparer au \jpg{ɯ-qiɯ} ``moitié''. La forme du japhug est difficile à reconstruire, et il n'est donc pas possible de trancher entre plusieurs reconstructions possibles en pré-tangoute. Ce caractère s'écrit aussi \tgz{1705} dans un composé.


\item 3113 \tgf{3113} \ipa{gjɨɨ} 1.32 ``neuf'' correspond au \jpg{kɯngɯt} ``neuf'' et au \tib{dgu} de même sens. Le -t final en japhug est analogique de ``huit'' \ipa{kɯrcat} (\citealt[188]{jacques08}). La forme tangoute \ipa{gjɨɨ¹} ne peut pas provenir d'un *\ipapl{ŋgjuu} d'après les lois phonétiques régulières: on attendrait une forme à médiane --w-- *\ipapl{gjwɨɨ}. Il est notable qu'aucune syllabe de ce type n'existe en tangoute, mais il est impossible pour le moment d'expliquer cette irrégularité: on ne peut même pas supposer un processus analogique similaire au japhug, car un pré-tangoute *\ipapl{ŋgjuut} donnerait aussi *\ipapl{gjwɨɨ} d'après les lois phonétiques que nous avons exposées. On ne posera donc pas de proto-forme pour ce numéral.


\item 3688 \tgf{3688} \ipa{gjwɨr} 1.86 \ptang{r-ŋgu}{se coucher} est cognat du \jpg{rŋgɯ} ``s'allonger''.
\newline
\linebreak
\begin{tabular}{llllllll}
	\tgf{2440}&	\tgf{1421}&	\tgf{3688}&	\tgf{0705}&	\tgf{1498}&	\tgf{1498}&	\tgf{5880}&	\tgf{4825}\\
\tinynb{2440}&	\tinynb{1421}&	\tinynb{3688}&	\tinynb{0705}&	\tinynb{1498}&	\tinynb{1498}&	\tinynb{5880}&	\tinynb{4825}\\
\end{tabular}
\begin{exe}
\ex \label{ex:tg:se.coucher}  \vspace{-8pt}
\gll   \ipa{njɨɨ²}	\ipa{zjɨ̣¹}	\ipa{gjwɨr¹}	\ipa{zjịj¹}	\ipa{.wjị².wjị²}	\ipa{ŋwu²}	\ipa{me²} \\
		jour jour se.coucher quand faire.semblant \conj{} dormir \\
\glt Il se coucha pendant le jour, faisant semblant de dormir (Leilin 04.03A.5)
\end{exe}


\item 597 \tgf{0597} \ipa{ɣjɨ} 1.29 ``oncle maternel'' correspond au  \situ{ta-kû} ``oncle maternel'' et au \tib{a.kʰu} ``oncle paternel''. Tout comme le numéral ``neuf'', ce nom est problématique. On attendrait en effet *\ipapl{ɣjwɨ} en tangoute si le pré-tangoute était *\ipapl{C-ku} : on ne reconstruira pas ici de pré-tangoute.


\item 1999 \tgf{1999} \ipa{ŋwə} 1.27 \ptang{ŋu}{cinq} correspond au \jpg{kɯmŋu} ``cinq''. En dehors des langues macro-rgyalronguiques, ce numéral a une voyelle ouverte: \tib{lŋa}, \plb{ŋa²}{0482} (\bir{ŋa³}), chinois *\ipapl{ŋˁaʔ}. On retrouve une médiane labiale en \nayn{ŋwɤ˧} < *\ipapl{ŋwa}, mais ce numéral ne suit pas les lois phonétiques habituelles en  naish.


\end{enumerate}
\subsubsection{Tangoute \ipapl{--ə/--jɨ} :: tibétain  --i :: japhug \ipapl{--i/--e}}	\label{subsubsec:correspondance:eu:i}
On reconstruira *i en pré-tangoute pour les rimes des mots appartenant à cette correspondance, et *wi dans le cas des mots à médiane --w--.
\newline

A. Initiales labiales \label{rimes:06:4:2:lab}
\begin{enumerate}


\item 923 \tgf{0923} \ipa{.wə̣} 1.68 \ptang{S-pi ou *S-mbi}{vieux} correspond au \jpg{mbe} ``vieux''. On note également la forme apparentée 2140  \tgf{2140} \ipa{.wjɨ̣} 1.69 qui n'est pas attestée dans les textes.



\item 4408 \tgf{4408} \ipa{məə} 1.31 \ptang{mii}{feu} est comparable au \jpg{smi}, au naish \naxi{mi˧},	\nayn{mv̩˥},	\laze{mv̩˧},	\protona{mi}, au \plb{C-mi²}{0329} et au \tib{me} de même sens. On retrouve des cognats dans la plupart des langues sino-tibétaines (y compris le chinois, voir \citealt[159]{sagart99roc}). La syllabe 5193 \tgf{5193} \ipa{mjɨ} 2.28 qui apparaît dans l'expression \tgf{5193}\tgf{1930} \ipa{mjɨ²tju¹} ``allumer un feu'' est probablement apparentée: on reconstruira un pré-tangoute *mji. Il s'agit peut-être ici d'un nom incorporé (voir \citealt{jacques11tangut.verb}).



\item 4574 \tgf{4574} \ipa{mjɨ} 1.30 \ptang{mji}{autre} est potentiellement comparable au \jpg{tɯrme} ``homme'' et au \tib{mi}. Il est possible toutefois que ce nom sino-tibétain de l'homme soit plutôt reflété par l'autonyme des tangoutes 2344 \tgf{2344} \ipa{mji} 2.10 \ptang{mje}{tangoute} ou bien par le suffixe nominalisateur  \tgz{3818} (voir \ref{subsec:nmlz}). On doit mentionner aussi le \bir{mi¹mi¹} ``soi-même".\footnote{D.Bradley, communication personnelle.}



\item 5382 \tgf{5382} \ipa{mjɨ} 2.28 \ptang{mji}{(trace de) pied} est vraisemblablement apparenté au \jpg{tɯ-mi} ``jambe''.  \tgf{5382} \ipa{mjɨ²} n'apparaît pas tout seul en tangoute, il n'est attesté que dans le composé \tgf{5382}\tgf{0575} \ipa{mjɨ²rjar²} ``trace de pied''. La seconde syllabe de ce composée est analysée p.\pageref{analyse:ecrire}. Ce nom a un verbe dérivé  \tgf{5733}\tgf{0575} \ipa{mjɨ̣̣²rjar²}  ``suivre à la trace'' (\citealt[130]{gong02a}).
\end{enumerate}

B. Initiales dentales \label{rimes:06:4:2:dent}
\begin{enumerate}

\item 3612 \tgf{3612} \ipa{tsə̣} \ptang{S-tsi}{médicament} et 4273 \tgf{4273} \ipa{tsə̣} 1.68  \ptang{S-tsi}{thé} peuvent être comparés à la première syllabe du composé signifiant ``médicament'' dans les langues naish:  \nayn{ʈʂʰæ˧ɯ˧},   \laze{tsʰɯ˧fi˧} et  \naxi{ʈʂʰə˞˧ɯ˧}, \protona{rtsʰi Swa} ainsi qu'au \bir{che³} de même sens. Le \tib{rtsi} ``jus, laque'' est peut-être aussi apparenté. Le sens originel  de cette forme est probablement ``décoction''. 



\item 3894 \tgf{3894} \ipa{njɨ} 1.30 \ptang{nji}{tante paternelle} est cognat du \jpg{tɤ-ɲi} ``tante paternelle'' et peut-être également du \tib{a.ne} ``tante''.  La même racine apparaît dans la première syllabe du composé  \tgf{3986}\tgf{4893} \ipa{njɨ¹wjɨ¹} ``belle-mère''. 


\item 2440 \tgf{2440} \ipa{njɨɨ} 2.29   \ptang{njii}{jour, soleil} est apparenté au \tib{ɲi.ma} ``jour, soleil''. On retrouve ce mot dans la majorité des langues sino-tibétaines (voir \citealt[191]{matisoff03}).  Contrairement à \tgf{2449} \ipa{be²} ``soleil'' (voir p.\pageref{ex:tg:soleil}), \tgf{2440} \ipa{njɨɨ²} s'emploie comme classificateur précédé d'un numéral, et apparaît dans les expressions ``hier'' et ``aujourd'hui''. 

Nous aborderons les particularités morphologiques de ce nom en \ref{subsec:num}.


\item 4565 \tgf{4565} \ipa{lə} 2.25 \ptang{li}{puce} est comparable avec la seconde syllabe du \jpg{mdzadi} ``puce'' ainsi qu'avec le \tib{ldʑi.ba} ``puce''. \citet[192-3]{matisoff03} présente une liste de cognats dans d'autres langues.



\item 5667 \tgf{5667} \ipa{lhjɨ̣} 1.69 \ptang{lhji}{arc} est apparenté au \jpg{tɯ-di} ``arc''. Il convient peut-être d'y rapprocher le nom de la ``flèche'', en tangoute 5710 \tgf{5710} \ipa{ljị} 1.67 (*S-lje) et en \jpg{zdi}. Si le rapport entre ces deux noms est possible, le mécanisme de dérivation n'est pas clair, car on ne dispose pas d'exemple d'un tel préfixe s-- / *S-- nominal en japhug ou en tangoute.
\newline
\linebreak
\begin{tabular}{llllllllll}
	\tgf{2738}&	\tgf{0257}&	\tgf{2590}&	\tgf{4222}&	\tgf{2449}&	\tgf{1572}&	\tgf{0046}&	\tgf{0705}&	\tgf{2098}&	\tgf{5667}\\
\tinynb{2738}&	\tinynb{0257}&	\tinynb{2590}&	\tinynb{4222}&	\tinynb{2449}&	\tinynb{1572}&	\tinynb{0046}&	\tinynb{0705}&	\tinynb{2098}&	\tinynb{5667}\\
\tgf{2078}&	\tgf{2098}&	\tgf{5710}&	\tgf{4401}& &&&&&\\
\tinynb{2078}&	\tinynb{2098}&	\tinynb{5710}&	\tinynb{4401}& &&&&&\\
\end{tabular}
\begin{exe}
\ex \label{ex:tg:arc}  \vspace{-8pt}
\gll   \ipa{djɨ̣j²}	\ipa{ŋwər¹}	\ipa{.wjɨ²-pheej²}	\ipa{be²}	\ipa{phiow¹}	\ipa{ljij²}	\ipa{zjịj¹}	\ipa{ŋa²}	\ipa{lhjɨ̣¹}	\ipa{thu¹-ŋa²}	\ipa{ljị¹}	\ipa{zow²} \\
		nuage noir \dir{}-disperser soleil blanc voir quand moi arc brandir-1\sg{} flèche tenir \\
\glt  Quand les nuages sombres se seront dispersés et que le soleil blanc sera visible, je  brandirai mon arc et saisirai une flèche. (Leilin 05.16B.2)\footnote{Cette phrase pose problème, car le second verbe \tgz{4401} devrait être aussi suffixé de la marque de personne \tgz{2098}.}
\end{exe}
En dehors des langues macro-rgyalronguiques, on trouve  une forme signifiant ``arc'', comme le \plb{le²}{0264} (\bir{le³}). \citet[192-3]{matisoff03} cite des cognats dans d'autres langues lolo-birmanes, en jingpo et en karen, auxquelles on peut rajoute le chinois \zh{矢} *l̥i[j]ʔ.



\item 2302 \tgf{2302} \ipa{ljɨ} 1.29 \ptang{lji}{vent} est cognat du \jpg{qale} ``vent'' et du \plb{le¹}{0326} \bir{le²}.


\item 2737 \tgf{2737} \ipa{ljɨɨ} 1.32 \ptang{ljvvt ou *ljii}{lourd} peut être rapproché du \jpg{rʑi} et du \tib{ltɕi-po} (*lhji) et \tib{ldʑid-po} (*N-lji-t) de même sens (voir \citet[192]{matisoff03} pour les cognats dans les autres langues). Cette racine apparaît en tibétain ainsi que dans d'autres langues avec un suffixe --d adjectival, mais il est probable que la forme tangoute doit être reconstruite avec une syllabe ouverte.


\item 2205 \tgf{2205} \ipa{ljɨɨr} 1.92 \ptang{r-lii}{quatre} est cognat du \jpg{kɯβde} et du \tib{bʑi} (<*p-lji). La forme du pré-tangoute avec une présyllabe *r-- ne ressemble à aucune des langues macro-rgyalronguiques modernes, sauf peut-être le muya \ipa{rə̠--}.


\end{enumerate}

C. Autres initiales \label{rimes:06:4:2:vel}
\begin{enumerate}


\item 1153 \tgf{1153} \ipa{dʑjɨ} 1.30 \ptang{nri > *ndri}{peau} est comparable au \jpg{tɯ-ndʐi} ``peau'', au \plb{re¹}{0134} (\bir{a.re²}) et au  \pumi{rə̂}.

\item 4880 \tgf{4880} \ipa{rər} 2.76 \ptang{ri}{cuivre} peut se comparer  au \plb{gre²}{404} ``cuivre'' et peut-être au  \tib{gri} ``épée''.

\item 2511 \tgf{2511} \ipa{rjɨr} 2.77 \ptang{rji}{partir, aller} est apparenté au \jpg{ari} la forme irrégulière d'aoriste du verbe \ipa{ɕe} ``aller''  et au zbu \ipa{rî} ``aller (dans une certaine direction)'' (\citealt[274]{jackson04showu}). En tangoute, ce verbe est \textit{toujours} attesté avec un préfixe directionnel, le plus souvent \tgf{4342} \ipa{dja²--}, ce qui rappelle l'usage du verbe cognat  \ipa{rî} en zbu.


\item 1200 \tgf{1200} \ipa{khjwɨ} 1.30 \ptang{khwi plutôt que *khu}{chien} peut se comparer à la première syllabe du \jpg{khɯna} ``chien'' ainsi qu'au \tib{kʰʲi} (<*kwi).


\item 257 \tgf{0257} \ipa{ŋwər} 1.84 et sa variante à l'autre ton 654 \tgf{0654} \ipa{ŋwər} 2.76 \ptang{rŋwi ou *rŋu}{bleu, noir, vert}   sont cognats du \jpg{arŋi} ``bleu/vert''. Une relation avec le \tib{sŋo, sŋon-po} ``bleu/vert'' est également envisageable mais plus spéculative.

Un dérivé de \tgf{0257} \ipa{ŋwər¹}, 510 \tgf{0510} \ipa{ŋwər} 1.84 ``céleste, impérial'' entre dans de nombreux composés dont   \tgf{0510}\tgf{5306} \ipa{ŋwər¹dzjwɨ¹} ``empereur''.

\item 3158  \tgf{3158} \ipa{ŋwər} 2.76 ``visage'' est peut-être comparable avec le \jpg{tɯ-rŋa} et le \tib{ŋo} de même sens. Toutefois, une étymologie alternative pour ces formes japhug et tibétain existe aussi: \tgz{1204} ``visage''.


Il est toutefois aussi possible que \tgz{3158} soit un emprunt au \tib{ŋo}, car il apparaît, comme l'a suggéré \citet{pengxq09}, dans l'expression \tgf{3158}\tgf{1829} \ipa{ŋwər²tshja¹} ``avoir honte'' qui correspond exactement au \tib{ŋo.tsʰa} (littéralement, ``avoir le visage brûlant''). Comme la seconde syllabe de cette expression est certainement tibétaine du fait des correspondances phonétiques, il est fortement possible que ce soit aussi le cas de la première, même si cela explique mal la correspondance vocalique avec une rime du second cycle mineur.


\end{enumerate}

\subsubsection{Autres correspondances}	\label{subsubsec:correspondance:eu:autre}

On observe très peu d'exemples incompatibles avec les correspondances présentées ci-dessus. En premier lieu, on peut noter des préfixes tels que  \tgf{5300} \ipa{tjɨ¹--} ``un'' (voir p.\pageref{ex:tg:un.jour}) correspondant au japhug \ipapl{tɯ--} ou \tgf{5168} \ipa{.jɨ²--}, préfixe qui apparaît dans ``l'année dernière'' et ``hier'' correspondant à \ipapl{ja-} ou \ipapl{jɯ-} en japhug. Ici, on a affaire dans les deux langues à des syllabes affaiblies dont la voyelle a été remplacée par un schwa. Il n'y a pas lieu ici de chercher à reconstruire le vocalisme de ces préfixes.



Un exemple différent est celui de 5186 \tgf{5186} \ipa{tshjɨ} 2.28 ``sel'' qu'on peut rapprocher du \tib{tsʰʷa} ``sel''. La correspondance --a à  --\ipapl{jɨ} est absolument irrégulière, et il faudrait poser un pré-tangoute *\ipapl{tshji} ou *\ipapl{tshjvC}. Nous laissons la reconstruction de cet étymon à des recherches futures.




Par ailleurs, on doit noter deux mots qui n'ont de cognats qu'avec le pumi, et pour lesquels il est difficile de proposer une reconstruction. Premièrement, 5828  \tgf{5828} \ipa{tsə̣} 1.68 ``automne'' peut être rapproché du Shuiluo \pumi{tsǎ} ``automne'' et peut-être de la première syllabe du \bir{tshoŋ³ ʔu³}. Deuxièmement, 5171 \tgf{5171} \ipa{sə} 1.27 ``plein, remplir'' est comparable au  Shuiluo \pumi{suə²} ``plein''; ce verbe a une forme alternante  4734 \tgf{4734} \ipa{swu} 2.01 qui n'est toutefois pas attestée dans les textes.



Finalement, on doit mentionner deux emprunts au tibétain. 

Premièrement, la désignation tangoute des tibétains, 5233  \tgf{5233} \ipa{phə} 1.27 est clairement apparentée au nom tibétain du Tibet : Bod. Comme nous l'avons vu p.\pageref{subsubsec:correspondance:eu:vt}, on peut reconstruire *--vt en pré-tangoute pour les rimes du \ipa{shè} n°6. Ainsi, on peut proposer un pré-tangoute *phvt pour rendre compte du tangoute \tgf{5233} \ipa{phə¹}, qui aurait donc été emprunté d'un dialecte tibétain où les anciennes voisées étaient devenues sourdes aspirées.

Deuxièmement, 4607 \tgf{4607} \ipa{zər} 2.76 ``rosée'' qui pourrait provenir du \tib{zil} ``rosée''. C'est le seul exemple d'une rime à --l final en tibétain correspondant au --r du tangoute. 

\subsection{Rimes ej / ij} \label{subsec:voyelle.ej}

Les rimes du \ipa{shè} n°7 sont reconstruites par Gong Hwangcherng avec les voyelles i et e qui apparaissaient déjà dans le \ipa{shè} n°2 (voir \ref{subsec:voyelle.e.i}) , suivies d'une finale --j. Certains auteurs, tels que Sofronov, Li et Arakawa, reconstruisent ici des nasales, mais les données des transcriptions aussi bien chinoises que tibétaines ne permettent pas de conclure de façon claire quant à la présence ou non de nasales. Dans le cadre de ce travail, on admettra avec Gong qu'aucune de ces rimes n'a de voyelle nasale.

\begin{table}
\captionb{Reconstructions du \ipa{shè} n°7}\label{tab:she7}
\resizebox{\columnwidth}{!}{
\begin{tabular}{lllllllll} \toprule
rime&ton 1&ton 2&Sofronov1&Sofronov2&Nishida&Li&Gong&Arakawa\\
34&	1.33&	2.30&	\ipa{ai}&	\ipa{ei}&	\ipa{ɛ ʷɛ}&	\ipa{ɛ uɛ}&	\ipa{ej}&	\ipa{e}\\	
35&	1.34&	2.31&	\ipa{ai}&	\ipa{ei}&	\ipa{iɛ}&	\ipa{ɛǐ}&	\ipa{iej}&	\ipa{ye}\\	
36&	1.35&	2.32&	\ipa{i̯ai}&	\ipa{i̯ei}&	\ipa{ɛɦ}&	\ipa{ǐɛ̃}&	\ipa{jij}&	\ipa{eː}\\	
37&	1.36&	2.33&	\ipa{In}&	\ipa{In}&	\ipa{eɦ ʷeɦ}&	\ipa{ẽ}&	\ipa{jij}&	\ipa{eː}\\	
38&	1.37&	2.33&	\ipa{ai+C}&	\ipa{ai}&	\ipa{e ʷe}&	\ipa{ǐe}&	\ipa{eej}&	\ipa{e’}\\	
39&	1.38&	&	\ipa{ai+C}&	\ipa{ai}&	\ipa{eʸ}&	\ipa{e}&	\ipa{ieej}&	\ipa{ye’}\\	
40&	1.39&	2.35&	\ipa{i̯ai+C}&	\ipa{i̯e}&	\ipa{ǐeɦ}&	\ipa{ɪẽ}&	\ipa{jiij}&	\ipa{eː’}\\	
63&	1.60&	2.53&	\ipa{?}&	\ipa{ại}&	\ipa{ǐɛ̣}&	\ipa{ɛ̣}&	\ipa{iẹj}&	\ipa{yeq}\\
64&	1.61&	2.54&	\ipa{?}&	\ipa{i̯ẹ}&	\ipa{ɛ̣}&	\ipa{ǐɛ̣̃}&	\ipa{jịj}&	\ipa{enq}\\
77&	1.73&	2.66&	\ipa{?}&	\ipa{ại}&	\ipa{ẹ}&	\ipa{ẹ̃}&	\ipa{ejr}&	\ipa{yeq2}\\
78&	&	2.67&	\ipa{?}&	\ipa{ại}&	\ipa{ʷǐẹ}&	\ipa{uẹ}&	\ipa{iejr}&	\ipa{eq’}\\
79&	1.74&	2.68&	\ipa{?}&	\ipa{i̯e}&	\ipa{ǐẹ}&	\ipa{ǐẹ̃ uẹ̃}&	\ipa{jijr}&	\ipa{yeq’}\\
\bottomrule
\end{tabular}}
\end{table}
Gong reconstruit de façon semblable les rimes 36 et 37, car elles se trouvent en distribution complémentaire, voir \citet{gong89reconstruction} et \citet[93-95]{gong02a}.

\begin{longtable} {lllllll}
\captionb{Comparaison  des étymons en --ij du tangoute avec le japhug et le tibétain.}\label{tab:comparaisons:ij}\\
\toprule
\multicolumn{4}{c}{tangoute} & sens & japhug & tibétain  \\
\midrule
\endfirsthead
\tinynb{676}&  \tgf{0676}  &\ipa{.wjij} &\tinynb{1.35}  &partir &\ipa{ɣi}  & \\
\tinynb{5362}&  \tgf{5362}  &\ipa{bjij} &\tinynb{2.33}  &pénis &\ipa{tɯ-mbɯ}  & \\
\tinynb{2160}&  \tgf{2160}  &\ipa{ɕjij} &\tinynb{1.36}  &orge &\ipa{tɤɕi}  & \\
\tinynb{2407}&  \tgf{2407}  &\ipa{dzeej} &\tinynb{2.34}  &chevaucher &\ipa{}  & \\
\tinynb{109}&  \tgf{0109}  &\ipa{gjịj} &\tinynb{1.61}  &étoile &\ipa{ʑŋgri}  & \\
\tinynb{5143}&  \tgf{5143}  &\ipa{kiej} &\tinynb{1.34}  &insulter &\ipa{nɤmqe}  & \\
\tinynb{3361}&  \tgf{3361}  &\ipa{kiẹj} &\tinynb{1.60}  &soeur & \ipa{tɤ-sqhɤj}  & \\
\tinynb{5834 }& \tgf{5834} & \ipa{lej}  & \tinynb{2.30} & changer & & rdʑe \\ 
\tinynb{2563}&  \tgf{2563}  &\ipa{mej} &\tinynb{2.30}  &poil &\ipa{tɤ-rme}  & \\
\tinynb{5677}&  \tgf{5677}  &\ipa{mjiij} &\tinynb{1.39}  &queue &\ipa{tɤ-jme}  & \\
\tinynb{2370}&  \tgf{2370}  &\ipa{mjij} &\tinynb{2.33}  & terre&\ipa{}  & \\
\tinynb{960}&  \tgf{0960}  &\ipa{mjịj} &\tinynb{1.61}  &jeune fille &\ipa{tɯ-me}  & \\
\tinynb{2639}&  \tgf{2639}  &\ipa{mjiij} &\tinynb{2.35}  &nom &\ipa{tɤ-rmi}  &miŋ \\
\tinynb{317}&  \tgf{0317}  &\ipa{nẹj} &\tinynb{2.53}  &faner &\ipa{rŋil}  & \\
\tinynb{213}&  \tgf{0213}  &\ipa{njij} &\tinynb{1.36}  &proche &  & ɲe\\
\tinynb{1671}&  \tgf{1671}  &\ipa{njij} &\tinynb{1.36}  &rouge &\ipa{ɣɯrni}  & \\
\tinynb{2518}&  \tgf{2518}  &\ipa{njiij} &\tinynb{1.39}  &coeur &\ipa{tɯ-sni}  &sɲiŋ \\
\tinynb{3439}&  \tgf{3439}  &\ipa{pjịj} &\tinynb{1.61}  &sūtra &\ipa{tɤrpi}  & \\
\tinynb{4335}&  \tgf{4335}  &\ipa{rjijr} &\tinynb{2.68}  &rire &\ipa{nɤre}  & \\
\tinynb{3469}&  \tgf{3469}  &\ipa{sjij} &\tinynb{2.33}  &savoir &\ipa{sɯz}  &ɕes \\
\tinynb{2734}&  \tgf{2734}  &\ipa{sjij} &\tinynb{1.36}  &sang &\ipa{tɤ-se}  & \\
\tinynb{1670}&  \tgf{1670}  &\ipa{sjwij} &\tinynb{1.36}  &aiguiser &\ipa{fse}  & \\
\tinynb{5573 }&  \tgf{5573}  &\ipa{tjij} &\tinynb{2.33}  &  nombril&   &lte.ba\\
\tinynb{5356}&  \tgf{5356}  &\ipa{tjịj} &\tinynb{1.61}  &seul &\ipa{ɯ-sti}  & \\
\tinynb{5612}&  \tgf{5612}  &\ipa{tshjiij} &\tinynb{1.39}  &dire &\ipa{ti}  & \\
\midrule
\tinynb{3807}&  \tgf{3807}  &\ipa{.jij} &\tinynb{1.36}  &léger &\ipa{ʑo}  &jaŋ.po \\
\tinynb{1245}&  \tgf{1245}  &\ipa{.jij} &\tinynb{1.36}  &soi-même &\ipa{tɯ-ʑo}  &raŋ \\
\tinynb{3452}&  \tgf{3452}  &\ipa{.jij} &\tinynb{2.54} &mouton &\ipa{qaʑo}  &gjaŋ.dkar \\
\tinynb{5974}&  \tgf{5974}  &\ipa{.wjịj²} &\tinynb{2.33} &envoyer &  &ɴpʰen \\
\tinynb{1890}&  \tgf{1890}  &\ipa{bjij} &\tinynb{1.36}  &haut &\ipa{mbro}  & \\
\tinynb{735}&  \tgf{0735}  &\ipa{dʑjij} &\tinynb{1.36}  &froid &\ipa{ɣɤndʐo}  & \\
\tinynb{2478}&  \tgf{2478}  &\ipa{khjij} &\tinynb{1.36}  &étendre &\ipa{ɕkho}  & \\
\tinynb{3526}&  \tgf{3526}  &\ipa{khjij} &\tinynb{2.33}  &pigeon &\ipa{qro}  & \\
\tinynb{0046}&  \tgf{0046}  &\ipa{ljij} &\tinynb{2.33}  &regarder &\ipa{nɤmɲo}  & \\
\tinynb{5522}&  \tgf{5522}  &\ipa{ljiij} &\tinynb{2.35}  &attendre &\ipa{nɤjo}  & \\
\tinynb{330}&  \tgf{0330}  &\ipa{mjiij} &\tinynb{1.39}  &rêve &\ipa{tɯ-jmŋo}  &  \\
\tinynb{2192}&  \tgf{2192}  &\ipa{mjiij} &\tinynb{1.39}  &cadavre &  & \\
\tinynb{638}&  \tgf{0638}  &\ipa{njij} &\tinynb{2.33}  &chasser &\ipa{no}  &snaŋ \\
\tinynb{5554}&  \tgf{5554}  &\ipa{njij} &\tinynb{2.33}  &entendre &\ipa{sɤŋo}  & \\
\tinynb{764}&  \tgf{0764}  &\ipa{rjijr} &\tinynb{1.74}  &cheval &\ipa{mbro}  & \\
\tinynb{5500}&  \tgf{5500}  &\ipa{sjij} &\tinynb{1.36}  &demain &\ipa{fso}  &saŋ.ɲin \\
\tinynb{2621}& \tgf{2621}  &\ipa{sjiij} &\tinynb{2.35} &penser& \ipa{sɯso}  & \\
\tinynb{3574}&  \tgf{3574}  &\ipa{tsjij} &\tinynb{2.33}  &comprendre &\ipa{tso}  & \\
\midrule
\tinynb{5778}&  \tgf{5778}  &\ipa{khjij} &\tinynb{1.36}  &couper &\ipa{rɤkrɯ}  & \\
 
\tinynb{973}&  \tgf{0973}  &\ipa{.wjijr} &\tinynb{2.68}  &meule &\ipa{βɣa}  & \\
\tinynb{4684}&  \tgf{4684}  &\ipa{mej} &\tinynb{1.33}  &œil&\ipa{tɯ-mɲaʁ}  &mig, dmʲig \\
\bottomrule
\end{longtable}
Comme le montre le tableau ci-dessus, les rimes de ce  \ipa{shè} correspondent principalement à des rimes en --i/--e en japhug ainsi qu'au --o du japhug qui provient d'un *--\ipapl{aŋ} en proto-japhug. Pour la première correspondance, on reconstruira *--\ipapl{ej} / *--\ipapl{jej} en pré-tangoute. Pour la seconde, on reconstruira *--\ipapl{jaŋ}. On verra en \ref{rimes:10:1:dent} que *--\ipapl{aŋ} donne --ow en tangoute. La proto-rime *--\ipapl{aŋ} est donc divisée entre les  \ipa{shè} 7 et 10.

Les rimes 34 et 35 (--ej et --iej) sont en distribution complémentaire devant la majorité des initiales: 34 apparaît après dentales et glottales, et 35 après vélaires et palatales. Toutefois, les deux rimes peuvent s'opposer devant labiales et labiovélaires : on a par exemple 4916 \tgf{4916} \ipa{ɣwej} 1.33 ``combattre'' qui s'oppose à 5030 \tgf{5030} \ipa{ɣiwej} 1.34 ``saisir''. 

La seule exception à ces règles de distribution est la paire   4186  \tgf{4186} \ipa{khej} 1.33  ``luxuriant'' qui s'oppose à 2810 \tgf{2810} \ipa{khiej} 1.34 ``vent du nord'', puisque la syllabe \tgf{4186} \ipa{khej¹} combine une initiale vélaire avec la rime 34. Toutefois, comme cette exception porte sur un mot rare et sans étymologie, on la négligera dans le cadre de la reconstruction du pré-tangoute.

Devant les labiovélaires et les labiales, on poserait a priori une opposition entre *--ej et *--rej pour rendre compte de la distinction entre -ej et --iej, mais on ne dispose pas de bonnes preuves comparatives qui pourraient justifier une telle reconstruction.

A ces deux correspondances s'ajoutent quelques correspondances irrégulières. Certains des exemples présentés ci-dessous sont peut-être des comparaisons incorrectes.
\subsubsection{Tangoute \ipapl{--ej/--ij} :: tibétain --e :: japhug \ipapl{--i/--e/--ɤj}}	\label{subsubsec:correspondance:ej:i}
On reconstruira *--ej en pré-tangoute pour les mots appartenant à cette correspondance.
\newline

A. Initiales labiales \label{rimes:07:1:lab}



\begin{enumerate}


\item 3439  \tgf{3439} \ipa{pjịj} 1.61 \ptang{S-pjej}{magicien, guérisseur} peut être rapproché du \jpg{tɤrpi} ``sūtra (chanté pour guérir une maladie)'' et du proto-LB *ʔpiy¹.\footnote{D.Bradley, communication personnelle.} La forme japhug n'est pas un emprunt au \tib{dpe} contrairement aux apparences (ce mot y est emprunté comme \ipa{χpi}). Le mot tangoute est attesté dans un titre de chapitre de Leilin, \tgf{3832}\tgf{3439} \ipa{dji²pjịj¹} ``médecins et guérisseurs''.


\item 5362  \tgf{5362} \ipa{bjij} 2.33 \ptang{mbjej}{pénis} correspond au zbu \ipa{tə-mbî} et au \jpg{tɯ-mbɯ}. La correspondance entre japhug et zbu est exceptionnelle et il est impossible de donner une reconstruction en proto-rgyalrong. 


\item 676  \tgf{0676} \ipa{.wjij} 1.35 \ptang{C-pjej}{partir, s'en aller} peut se comparer au \jpg{ɣi}  (<*wi) ``venir''. En tangoute, ce verbe signifie clairement ``partir'' et non pas ``venir''. Il apparaît en binôme avec \tgf{3456} \ipa{lja¹} ``venir'' dans l'expression  \tgf{3456}\tgf{0676}   \ipa{lja¹.wjij¹} ``aller et venir'', voir \citet[269-70]{kepping85}.
\newline
\linebreak
\begin{tabular}{llllllllll} 
	\tgf{3133}&	\tgf{0261}&	\tgf{1531}&	\tgf{1139}&	\tgf{0795}&	\tgf{0676}&	\tgf{0046}&	\tgf{5643}&	\tgf{3092}&	\tgf{2912}\\
\tinynb{3133}&	\tinynb{0261}&	\tinynb{1531}&	\tinynb{1139}&	\tinynb{0795}&	\tinynb{0676}&	\tinynb{0046}&	\tinynb{5643}&	\tinynb{3092}&	\tinynb{2912}\\
\tgf{3092}&	\tgf{5643}&	\tgf{1374}&	\tgf{4803}&	\tgf{2098}&&&&&\\
\tinynb{3092}&	\tinynb{5643}&	\tinynb{1374}&	\tinynb{4803}&	\tinynb{2098}&&&&&\\
\end{tabular}
\begin{exe}
\ex \label{ex:tg:partir}  \vspace{-8pt}
\gll   \ipa{sjij¹}	\ipa{mjo²}	\ipa{gja¹}	\ipa{.jij¹}	\ipa{rjɨr²-.wjij¹}	\ipa{ljij²} \ipa{mjɨ¹djij²}	\ipa{lhjwo¹-djij²}	\ipa{mjɨ¹-tɕhjɨ¹-lji²-ŋa²} \\
		aujourd'hui moi armée \antierg{} \dir{}-partir voir[A] à.part revenir-\dur{} \negat{}-\pot{}-voir[B]-1\sg{} \\
\glt  Aujourd'hui je vois l'armée partir, mais je ne verrai pas son retour. (Leilin 3.16B.6-7)
\end{exe}


Le  verbe 5974	\tgf{5974} \ipa{.wjịj} 2.54 ``envoyer'' pourrait théoriquement être un dérivé causatif de   \tgf{0676} \ipa{.wjij¹}, bien que les tons ne conviennent pas (auquel cas on proposerait pré-tangoute *C-S-pjej). Toutefois,une comparaison avec le \tib{ɴpʰen, ɴpʰaŋs} ``jeter'' est plus probable (voir p.\pageref{ex:tg:envoyer}, comme pourrait le suggérer l'alternance en \tgf{5974} \ipa{.wjịj²} et  \tgf{5791} \ipa{.wjạ²}.



\item 2563  \tgf{2563} \ipa{mej} 2.30  \ptang{mej}{poil} peut se comparer au \jpg{tɤ-rme}. On doit également rapprocher de \tgf{2563} \ipa{mej²} la forme 2600  \tgf{2600} \ipa{mjar} 1.82 qui apparaît dans le composé   \tgf{4543}\tgf{2600} \ipa{mər¹mjar¹} ``moustaches''. La reconstruction de \tgf{2600} \ipa{mjar¹} est problématique. cette forme semble impliquer un groupe initial *r-m-- en pré-tangoute, qui correspondrait à celui du cognat japhug, mais sa rime en revanche impliquerait un pré-tangoute --aC, qui est incompatible avec le japhug. On peut également concevoir un lien avec \plb{ʔ-mwe³}{0085}, mais si c'est le cas la rime **--ul (LB --we) a eu un destin différent dans ce mot que dans ``argent'' et ``serpent''.


\item 5677  \tgf{5677} \ipa{mjiij} 1.39  \ptang{mjeej}{queue} ainsi que 4625 \tgf{4625} \ipa{mjiij} 2.35 \ptang{mjeej}{dernier}  sont apparentés au \jpg{tɤ-jme}   ``queue'' et au \plb{ʔ-mri²}{0084b} (\bir{a.mri³}).


\item 2639 \tgf{2639} \ipa{mjiij}  2.35 \ptang{mjeeN > mjeej}{nom} est vraisemblablement cognat avec le \jpg{tɤ-rmi} ``nom'', le naish \naxi{mi˩}, \nayn{mv̩.ʈʂæ^M^H}, \protona{mi}, le \plb{ʔ-m(y)iŋ¹}{0419} avec le \tib{miŋ} ``nom''. Dans le cas de ce mot, en tangoute comme d'ailleurs en tibétain, il n'est pas possible d'exclure l'hypothèse d'un emprunt d'une autre langue sino-tibétaine (en particulier en tibétain l'absence de préinitiale est troublante).


\item 2370  \tgf{2370} \ipa{mjij} 2.33  ``terre'' peut potentiellement se comparer au \plb{ʔ-mre¹}{0323} (\bir{mre²}). Toutefois, ce mot n'est pas attesté dans les textes connus hormis les dictionnaires.




\item 960  \tgf{0960} \ipa{mjịj} 1.61 \ptang{S-mjej}{jeune fille} est apparenté au \jpg{tɯ-me}  ``fille'', \plb{C-mi²}{0162} (voir aussi \citealt[187]{matisoff03}). La préinitiale *S-- pose problème, mais toute comparaison avec le \tib{smin.ma} ``jeune fille'' est impossible (voir \citealt{jacques07naksatram}). 3209 \tgf{3209} \ipa{mjɨɨ} 1.30, qui apparaît comme la première syllabe du composé \tgf{3209}\tgf{3028} \ipa{mjɨɨ.jur} 2.70 ``servante'', est peut être apparenté à \tgf{0960} \ipa{mjịj¹}; il pourrait s'agir d'une forme suffixée en *--s ou en *--t. Le caractère 3168  \tgf{3168}  se lit également \ipa{mjɨɨ} 1.30 et semble signifier ``jeune fille'', mais il n'est pas attesté dans les textes connus.

\end{enumerate}
B. Initiales dentales \label{rimes:07:1:dent}
\begin{enumerate}
\item 5573    \tgf{5573}   \ipa{tjij}  2.33  \ptang{tjej}{nombril}  peut être rapproché du   \tib{lte.ba}. Il s'agit d'une racine attestée dans l'ensemble de la famille sino-tibétain.

\item 5356  \tgf{5356} \ipa{tjịj} 1.61   \ptang{S-tjej}{seul} est cognat avec le \jpg{ɯ-sti} ``seul''. Dans les deux langues, il s'agit d'un nom: \tgf{5356} \ipa{tjịj¹} se place en effet \textit{avant}, et non \textit{après} le nom qu'il détermine.



\item 5612  \tgf{5612} \ipa{tshjiij} 1.39  \ptang{tshjeej}{dire} peut potentiellement se comparer au \jpg{ti} ``dire''. Cette comparaison est complexe dans le détail. La forme japhug est inattendue, car on retrouve une affriquée dans les autres langues rgyalrong : \situ{tsə́s}, zbu \ipa{tshəʔ}. C'est le seul exemple connu d'une occlusive simple en japhug correspondant à une affriquée dans les autres langues. Ce verbe est aussi particulier car il présente un irrégularité à l'aoriste: sa forme est \ipa{tɯt}, gardant la trace d'un ancien suffixe *--s de passé. Une comparaison avec \plb{dze²}{0663b} semble peu probable en vertu de la différence de consonne initiale.

En tangoute on trouve le thème B 3974 \tgf{3974} \ipa{tshjii} 2.12 avec un agent 1sg/2sg, pour lequel on posera un pré-tangoute *tshjeej-u. Ce verbe a trois dérivés possibles.

Premièrement, en composition, on rencontre la forme 5615 \tgf{5615} \ipa{tshjiij} 2.35 dans le binôme  \tgf{1097}\tgf{5615} \ipa{.u²tshjiij²} ``plaisanter''. Cette forme dérive de  \tgf{5612} \ipa{tshjiij¹} par alternance tonale.

Deuxièmement, on doit mentionner le dérivé nominal 5632 \tgf{5632} \ipa{tshjɨ} 1.30 qui semble signifier ``aphorisme'' (il n'est pas attesté dans les textes que nous avons étudiés). Pour rendre compte de l'alternance vocalique, une possibilité s'offre à nous: supposer un suffixe nominalisateur *--s en pré-tangoute. On aurait ainsi : *tsheej-s > *tshvs > \ipa{tshjɨ}. (

Troisièmement, de façon spéculative, le verbe 5870 \tgf{5870} \ipa{tshjɨɨ} 1.32 ``réciter'' pourrait aussi être un dérivé de  \tgf{5612} \ipa{tshjiij¹}. En effet, on trouve quelques exemples où  \tgf{5612} \ipa{tshjiij¹} semble bien avoir le sens de ``lire'' :
\newline
\linebreak
\begin{tabular}{llllllllll}
	\tgf{4889}&	\tgf{1045}&	\tgf{3926}&	\tgf{3627}&	\tgf{3627}&	\tgf{0795}&	\tgf{3974}&	\tgf{4601}&	\tgf{1278}&	\tgf{5087}\\
\tinynb{4889}&	\tinynb{1045}&	\tinynb{3926}&	\tinynb{3627}&	\tinynb{3627}&	\tinynb{0795}&	\tinynb{3974}&	\tinynb{4601}&	\tinynb{1278}&	\tinynb{5087}\\
\tgf{5964}&	\tgf{1045}&	\tgf{5354}&	\tgf{3583}&	\tgf{0502}&	\tgf{1012}&	\tgf{5870}&	\tgf{2620}&	\tgf{2098}&\\
\tinynb{5964}&	\tinynb{1045}&	\tinynb{5354}&	\tinynb{3583}&	\tinynb{0502}&	\tinynb{1012}&	\tinynb{5870}&	\tinynb{2620}&	\tinynb{2098}&\\
\end{tabular}
\begin{exe}
\ex \label{ex:tg:dire}  \vspace{-8pt}
\gll   \ipa{dʑjwɨ¹}	\ipa{dạ²}	\ipa{nja²}	\ipa{nji²nji²}	\ipa{rjɨr²-tshjii²-nja²}	\ipa{.jɨ²}	\ipa{.jow¹tsha²}	\ipa{dạ²}	\ipa{thjɨ²}	\ipa{tja¹}	\ipa{zjɨɨr¹zjịj¹}	\ipa{tshjɨɨ¹}	\ipa{njwi²-ŋa²} \\
		ami dire toi en.secret \dir{}-dire[B]-2\sg{} dire Wang.Can dire ceci \topic{} un.peu réciter pouvoir-1\sg{} \\
\glt  Son ami lui dit : ``Tu lis (cette inscription) à voix basse ?'' et Wang Can \zh{王粲} dit : ``J'ai quelques facilités pour réciter (les textes", puis il lut l'inscription à voix haute) (Leilin 04.28A.4-5)
\end{exe}
Dans l'exemple ci-dessus, le verbe \tgf{5612} \ipa{tshjiij¹} (au thème de 2sg) correspond au chinois \zh{誦} \textit{sòng} ``réciter'' dans le texte original d'où il a été traduit. Il est donc vraisemblable sémantiquement que \tgf{5870} \ipa{tshjɨɨ¹} en soit un dérivé. Toutefois, l'origine de l'alternance vocalique entre les deux formes n'est pas explicable: on pourrait invoquer un suffixe *--t ou *--s, mais on ne dispose pas de bons exemples d'un préfixe de ce type ayant la même fonction sémantique ni en rgyalrong ni dans les autres langues sino-tibétaines à morphologie conservatrice.


\item 2407  \tgf{2407} \ipa{dzeej} 2.34  ``chevaucher'' est apparenté au \plb{dzi²}{0651} (\bir{ci³}, voir aussi \citet[188]{matisoff03} pour des cognats dans d'autres langues macro-rgyalronguiques, lolo-birmanes et peut-être en tujia). Ce verbe a un dérivé nominal 520\tgf{0520} \ipa{dzeej}	1.37 ``cavalier''.


\item 317  \tgf{0317} \ipa{nẹj} 2.53  ``faner'' pourrait se comparer au \jpg{rŋil} de même sens. cette comparaison pose problème, car on ne retrouve normalement pas de --l finaux dans les mots proprement rgyalrong.

On peut envisager   une reconstruction *S-\ipapl{ŋej} en proto-macro-rgyalronguique : le *\ipapl{ŋ} s'est palatalisé (voir p.\pageref{tab:ngpalatalisation}).


\item 1671  \tgf{1671} \ipa{njij} 1.36  \ptang{(r-)njej}{rouge} est cognat avec le \jpg{ɣɯrni}  ``rouge'', le \plb{ʔ-ni¹}{0502} En pré-tangoute, la préinitiale *r-- ne laisse pas de trace lorsque l'initiale est n--, comme nous l'avons mentionné p.\pageref{tab:sans.preinitiale.r:preinitiale.r}.


\item 213	\tgf{0213} \ipa{njij} 1.36 \ptang{njej}{proche} est à rapprocher du \tib{ɲe-ba} ``proche'' et du \plb{b-ni²}{0751} (\bir{ni³}) ``proche''.


\item 2518  \tgf{2518} \ipa{njiij} 1.39  \ptang{njeeN > *njeej}{cœur} peut être rapproché du \jpg{tɯ-sni} ``cœur'' et du \tib{sɲiŋ} ``cœur''. L'absence de préinitiale en pré-tangoute est inexplicable (on attendrait *\ipapl{njịj}. On trouve en lolo-birman des   formes potentiellement apparentées, mais phonétiquement irréconciliables telles que \plb{ni³}{0142} d'après Bradley et *s-nik pour rendre compte du \bir{nhac lum³} (\citealt[480]{matisoff03}).


\item 3469  \tgf{3469} \ipa{sjij} 2.33  \ptang{sjej}{savoir, reconnaître} est apparenté au \jpg{sɯz} ``savoir'', naish \naxi{sɯ˧},\nayn{sɯ˥}, \laze{sɯ˩}, \protona{si}, \plb{si²}{0590} et au \tib{ɕes}. Ce mot n'avait pas de finale *--s en pré-tangoute; ce n'est toutefois pas un obstacle pour cette comparaison, car on retrouve des formes sans --s aussi en rgyalrong, tel que la forme impersonnelle irrégulière \jpg{mɤxsi} ``on ne sait pas''.
\newline
\linebreak
\begin{tabular}{llllllllll}
	\tgf{1906}&	\tgf{1084}&	\tgf{3305}&	\tgf{1640}&	\tgf{1326}&	\tgf{2511}&	\tgf{5815}&	\tgf{0021}&	\tgf{2983}&	\tgf{2019}\\
\tinynb{1906}&	\tinynb{1084}&	\tinynb{3305}&	\tinynb{1640}&	\tinynb{1326}&	\tinynb{2511}&	\tinynb{5815}&	\tinynb{0021}&	\tinynb{2983}&	\tinynb{2019}\\
\tgf{4233}&	\tgf{2750}&	\tgf{5460}&	\tgf{5911}&	\tgf{0046}&	\tgf{5981}&	\tgf{3469}& &&\\
\tinynb{4233}&	\tinynb{2750}&	\tinynb{5460}&	\tinynb{5911}&	\tinynb{0046}&	\tinynb{5981}&	\tinynb{3469}& &&\\
\end{tabular}
\begin{exe}
\ex \label{ex:tg:savoir}  \vspace{-8pt}
\gll   \ipa{nioow¹}	\ipa{ɣạ²}	\ipa{kjiw¹}	\ipa{dzjịj¹}	\ipa{kjɨ¹-rjɨr²}	\ipa{tsjɨ¹}	\ipa{tɕja¹}	\ipa{.u²}	\ipa{thja¹}	\ipa{ko¹}	\ipa{ɣu¹kạ¹}	\ipa{khwa¹}	\ipa{ljij²}	\ipa{.a-sjij²} \\
		après dix ans passer \dir{}-aller aussi chemin dans ce chariot garde loin voir[A] \dir{}-connaître \\
\glt  Dix ans plus tard, il vit de loin sur la route ce cocher, et il le reconnut. (Leilin 04.27B.2)
\end{exe}

On trouve également une forme rédupliquée \tgf{4993}\tgf{3469} \ipa{sjɨ²sjij²} qui semble signifier ``connaissance, personne connue''.\footnote{La présence d'une forme rédupliquée suggère qu'il s'agit en fait d'un verbe réciproque ``se connaître'' nominalisé.} Contrairement aux apparences on ne posera pas un pré-tangoute *sjvt pour \tgf{4993}\ipa{sjɨ²}  (forme qui aurait mieux convenu pour la comparaison avec le tibétain et le rgyalrong), car c'est la réduction vocalique dans les formes rédupliquées, et non une ancienne finale *--s, qui cause cette alternance vocalique, voir section \ref{subsec:redp-alt}.
\newline
\linebreak
\begin{tabular}{llllllllll}
	\tgf{2503}&	\tgf{2440}&	\tgf{4831}&	\tgf{2909}&	\tgf{2104}&	\tgf{4993}&	\tgf{3469}&	\tgf{0187}&	\tgf{2541}&	\tgf{0448}\\
\tinynb{2503}&	\tinynb{2440}&	\tinynb{4831}&	\tinynb{2909}&	\tinynb{2104}&	\tinynb{4993}&	\tinynb{3469}&	\tinynb{0187}&	\tinynb{2541}&	\tinynb{0448}\\
\tgf{2373}& &&&&&&&&\\
\tinynb{2373}& &&&&&&&&\\
\end{tabular}
\begin{exe}
\ex \label{ex:tg:connaissance}  \vspace{-8pt}
\gll  \ipa{kụ¹}	\ipa{njɨɨ²}	\ipa{mə²la²}	\ipa{ɕji¹}	\ipa{sjɨ²sjij²}	\ipa{nar²}	\ipa{dzjwo²}	\ipa{gjɨ²}	\ipa{ljịj²} \\
		après jour en.effet avant connaissance vieux homme un venir \\
\glt  Le jour suivant en effet, un vieil homme qu'il connaissait auparavant arriva. (Leilin 06.08B.6)
\end{exe}


\item 2734  \tgf{2734} \ipa{sjij} 1.36     \ptang{sjej}{sang} est cognat avec le \jpg{tɤ-se} ``sang''. Une comparaison avec le \plb{swe²}{0147} (\bir{swe³}) est en revanche plus hypothétique, car la médiane --w-- devrait être préservée aussi bien en rgyalrong qu'en tangoute.


\item 1670  \tgf{1670} \ipa{sjwij} 1.36  \ptang{p-sjej}{aiguiser} est probablement apparenté au \jpg{fse} ``aiguiser''. En tangoute, le sens de ce verbe n'est pas assuré, car il n'est pas connu dans les textes. Il pourrait signifier ``broyer, moudre'', mais également ``aiguiser'' comme le japhug. Dans le \textit{Wénhǎi}, \tgf{1670} \ipa{sjwij¹} est glosé par les caractères \tgf{4441}\tgf{0662} \ipa{sjwo¹khjɨɨ¹}, et \tgf{0662} \ipa{khjɨɨ¹} est lui-même glosé de la façon suivante :
\newline
\linebreak
\begin{tabular}{llllllllll}
	\tgf{4441} &\tgf{5856}&\tgf{0169}&\tgf{3678}&\tgf{0749} \\
	\tinynb{4441} &\tinynb{5856}&\tinynb{0169}&\tinynb{3678}&\tinynb{0749} \\
\end{tabular}
\begin{exe}
\ex \label{ex:tg:aiguiser}  \vspace{-8pt}
\gll   \ipa{sjwo¹} \ipa{ɣa²} \ipa{ɕjwi¹} \ipa{to²} \ipa{phji¹} \\
 meule sur dent sortir causer[A] \\
\end{exe}
Cette phrase pourrait bien signifier ``aiguiser sur une meule''. Par ailleurs,  \tgf{1670} \ipa{sjwij¹} est associé dans le \textit{Wénhǎi} au verbe avec 398 \tgf{0398} \ipa{sjwo} 2.44, qui, lui, apparaît  selon \citet[II:217]{nevskij60} dans l'expression \tgf{5037}\tgf{0398} \ipa{bjɨr¹sjwo²} qui ne peut signifier qu'``aiguiser un couteau''.  \tgf{1670} \ipa{sjwij¹} et \tgf{0398} \ipa{sjwo²} pourraient être deux formes alternantes d'un même verbe. Toutefois, en l'absence d'exemples textuels de \tgf{1670} \ipa{sjwij¹} hormis les dictionnaires, il est difficile de tirer des conclusions définitives.

\tgf{1670} \ipa{sjwij¹} pourrait être apparenté par alternance vocalique avec \tgf{4441} \ipa{sjwo¹}  ``meule'', un terme qui apparaît aussi dans le nom de l'encrier (chinois \zh{硯} \textit{yàn}).  



\item 5834 \tgf{5834} \ipa{lej}  2.30 \ptang{lej}{changer, se transformer (intr.)} et son dérivé causatif 5893 \tgf{5893} \ipa{lhej} 2.30 ``changer (tr.)'' (voir p.\pageref{subsec:causatif}) peuvent être rapprochés du \bir{lay³} ``changer, remplacer''  et du \tib{rdʑe} ``échanger''. L'exemple suivant permet d'illustrer l'usage de \tgf{5834} \ipa{lej²}  comme verbe intransitif :
\newline
\linebreak
\begin{tabular}{llllllllll}
\tgf{1870} & 	\tgf{0824} & 	\tgf{5417} & 	\tgf{4342} & 	\tgf{5834} & 	\tgf{4797} & 	\tgf{3320} & 	\tgf{1118} & 	\tgf{1085} & 	\tgf{5498} \\
\tinynb{1870} & 	\tinynb{0824} & 	\tinynb{5417} & 	\tinynb{4342} & 	\tinynb{5834} & 	\tinynb{4797} & 	\tinynb{3320} & 	\tinynb{1118} & 	\tinynb{1085} & 	\tinynb{5498} \\
\tgf{0448} & 	\tgf{0795} & 	\tgf{5113} & \\
\tinynb{0448} & 	\tinynb{0795} & 	\tinynb{5113} & \\
\end{tabular}
\begin{exe}
\ex \label{ex:tg:se.transformer}  \vspace{-8pt}
\gll   \ipa{dʑiə¹} 	\ipa{tɕhjɨ²rjar²} 	\ipa{dja²-lej²} 	\ipa{.jwɨr²} 	\ipa{ɣiew¹dzjɨɨ²} 	\ipa{zji¹} 	\ipa{.jij¹} 	\ipa{gjɨ²} 	\ipa{rjɨr²-.wji¹} \\
	renard immédiatement \dir{}-se.transformer lettre étudier garçon forme un \dir{}-faire[A] \\
 \glt Le renard se transforma et prit la forme d'un jeune lettré. (Leilin 06.30A.3)
\end{exe}



\end{enumerate}
B. Initiales palatales, rhotiques et vélaires \label{rimes:07:1:vel}
\begin{enumerate}

\item 2160  \tgf{2160} \ipa{ɕjij} 1.36  \ptang{ɕej}{orge} est cognat du \jpg{tɤɕi}  ``orge''. Des formes vraisemblablement apparentées se retrouvent à travers les langues macro-rgyalronguiques: queyu \ipa{ʃa⁵⁵}, muya  \ipa{ʂa²⁴}.


\item 4335  \tgf{4335} \ipa{rjijr} 2.68  \ptang{rjej}{rire} est apparenté au japhug  \ipa{nɤre}  ``rire'' et au \plb{ray¹}{0668} (\bir{ray²}). 



\item 109  \tgf{0109} \ipa{gjịj} 1.61 \ptang{S-ŋgjej}{étoile} est comparable au \jpg{ʑŋgri}  ``étoile'' et \plb{C-gray¹}{0319}. \citet{matisoff85} propose une origine possible pour ce mot d'une racine signifiant ``accrocher''. Si c'est le cas, il s'agirait d'une innovation birmo-qianguique.

On trouve également une forme rédupliquée avec une première syllabe à voyelle réduite \tgf{0108}\tgf{0109} \ipa{gjɨ̣²gjịj¹} qui signifie ``constellation''. Cette forme rédupliquée est précieuse pour la phonologie historique, car elle montre que l'on doit reconstruire *\ipapl{S-ŋgjej--S-ŋgjej} > *\ipapl{S-ŋgjə--S-ŋgjej} > \ipa{gjɨ̣²gjịj¹}: la présyllabe a chuté \textit{après} la réduction de la rime de la syllabe rédupliquée.

Rédupliqué, le nom de l'étoile devait avoir un sens de collectif, d'où la signification de ``constellation''. On retrouve peut-être un parallèle dans les langues sémitiques avec le nom de l'étoile *kabkabu > hébreu \ipa{kôk̠āb̠}, qui remonterait anciennement à une forme de collectif, comme l'a suggéré Romain Garnier (c.p.).


\item 5143  \tgf{5143} \ipa{kiej} 1.34  \ptang{kej}{insulter} peut se comparer au \jpg{nɤmqe} ``gronder, insulter''. On ne dispose pas toutefois d'exemples textuels de ce verbe.


\item 3361  \tgf{3361} \ipa{kiẹj} 1.60  \ptang{S-kej}{sœur} est apparenté au \jpg{tɤ-sqhɤj}  ``sœur''. Dans les deux langues, ce terme s'emploie spécifiquement pour la sœur d'une femme, jamais pour la sœur d'un homme.  
\end{enumerate}
\subsubsection{Tangoute \ipapl{--ej/--ij} :: tibétain --ang :: japhug \ipapl{--o}}	\label{subsubsec:correspondance:jij:o}
On reconstruira *--\ipapl{jaŋ} en pré-tangoute pour les mots appartenant à cette correspondance. Seules les rimes 36/37 --jij, 40 -jiij et 79 -jijr sont attestées parmi ces exemples. Le pré-tangoute *--\ipapl{aŋ} donne en tangoute --o ou --ow (voir section \ref{subsubsec:correspondance:o:ang} ).
\newline

A. Initiales labiales \label{rimes:07:2:lab}

\begin{enumerate}

\item 1890  \tgf{1890} \ipa{bjij} 1.36    \ptang{mbjaŋ}{haut} est cognat avec le \jpg{mbro}  ``haut'', \plb{(ʔ)-mroŋ³}{0758}. Ce verbe statif apparaît sous une forme rédupliquée  \tgf{4511}\tgf{1890} \ipa{bjɨ¹bjij¹} ``très élevé, partie la plus élevée'':
\newline
\linebreak
\begin{tabular}{llllllllll}
	\tgf{2857}&	\tgf{3583}&	\tgf{2518}&	\tgf{4511}&	\tgf{1890}&	\tgf{3349}&	\tgf{4444}&	\tgf{2518}&	\tgf{4669}&	\tgf{3791}\\
\tinynb{2857}&	\tinynb{3583}&	\tinynb{2518}&	\tinynb{4511}&	\tinynb{1890}&	\tinynb{3349}&	\tinynb{4444}&	\tinynb{2518}&	\tinynb{4669}&	\tinynb{3791}\\
\tgf{3349}&	\tgf{0724}&	\tgf{4342}&	\tgf{5449}& &&&&&\\
\tinynb{3349}&	\tinynb{0724}&	\tinynb{4342}&	\tinynb{5449}& &&&&&\\
\end{tabular}
\begin{exe}
\ex \label{ex:tg:haut}  \vspace{-8pt}
\gll   \ipa{ŋo²}	\ipa{tja¹}	\ipa{njiij¹}	\ipa{bjɨ¹bjij²}	\ipa{rjijr²}	\ipa{ljɨ̣¹}	\ipa{njiij¹}	\ipa{bjɨ¹bji²}	\ipa{rjijr²}	\ipa{njɨ²}	\ipa{dja²-tjị¹} \\
		maladie \topic{} cœur haut \loc{} \conj{} cœur bas \loc{} \pl{} \dir{}-mettre[A] \\
\glt  La maladie s'est placée dans la partie la plus haute du cœur et dans la partie la plus basse du cœur. (Leilin 06.10A.2-3)
\end{exe}
En japhug comme en tangoute, l'élément /b/ de l'initiale est en fait épenthétique : comme le montre le  \bir{mraŋ¹} ``haut'', il faut reconstruire à un stade plus ancien *\ipapl{mrjaŋ} >*\ipapl{mbrjaŋ} > *\ipapl{mbjaŋ} > bjij¹. \tgf{1890} \ipa{bjij¹} a un dérivé causatif 3506 \tgf{3506} \ipa{bjịj} 1.61 \ptang{S-mbjaŋ}{élever}, qui s'emploie surtout dans le sens de ``donner une promotion''.


\item 5974 \tgf{5974} \ipa{.wjịj} 2.54 ``envoyer'' pourrait être apparenté avec le \tib{ɴpʰen, ɴpʰangs} ``jeter, tirer à l'arc'' et son dérivé \ipa{spoŋ, spaŋs} ``abandonner''.\index{Tibétain!spoŋ, spaŋs} Une autre étymologie potentielle serait de le considérer comme le causatif du verbe \tgf{0676} \ipa{.wjij¹}  ``partir, s'en aller''  (voir p.\pageref{ex:tg:partir}). Toutefois,  \tgf{5974} \ipa{.wjịj} a une forme potentiellement apparentée 5791 \tgf{5791} \ipa{.wjạ} 2.57. Or, cette alternance entre --jij et --ja ne fait sens que si l'on reconstruit *\ipapl{--jaŋ} contre *\ipapl{--jaŋ-C} > *\ipapl{--ja-C}, où --C serait un suffixe *--t ou *--s.  \tgf{5791} \ipa{.wjạ²} a souvent le sens de ``relâcher, libérer'', mais on le trouve également utilisé  dans le sens d'``envoyer'' comme \tgf{5974} \ipa{.wjịj²} :
\newline
\linebreak
\begin{tabular}{lllllll}
	\tgf{0942}&	\tgf{3349}&	\tgf{2790}&	\tgf{3674}&	\tgf{5993}&	\tgf{0795}&	\tgf{5791}\\
\tinynb{0942}&	\tinynb{3349}&	\tinynb{2790}&	\tinynb{3674}&	\tinynb{5993}&	\tinynb{0795}&	\tinynb{5791}\\
\end{tabular}
\begin{exe}
\ex \label{ex:tg:envoyer}  \vspace{-8pt}
\gll   \ipa{ljạ¹}	\ipa{rjijr²}	\ipa{djɨ²kjwɨɨr¹}	\ipa{kha¹}	\ipa{rjɨr²-.wjạ²} \\
		nord \loc{} Xiongnu milieu \dir{}-envoyer \\
\glt  Il fut envoyé au nord, chez les Xiongnu. (Leilin 03.11A.5)
\end{exe}
Toutefois, il convient de noter que  \tgf{5791} \ipa{.wjạ²} pourrait avoir une étymologie différente, comme on le verra dans la discussion p.\pageref{ex:tg:relacher2}.


\item 330  \tgf{0330} \ipa{mjiij} 1.39  \ptang{mjaaŋ}{rêve, rêver} peut être rapproché du \jpg{tɯ-jmŋo}  ``rêve'' et du \plb{C-mák}{0586}. Des cognats apparaissent dans de nombreuses langues (voir \citealt[325]{matisoff03}). Toutefois, la correspondance inattendue entre proto-rgyalrong *-aˠŋ et le tangoute --jiij suggère qu'il pourrait aussi s'agir d'un emprunt au chinois \zh{夢} mèng.

En tangoute, il peut s'agir aussi bien d'un nom que d'un verbe:
\newline
\linebreak
\begin{tabular}{lllll}
	\tgf{4028} &\tgf{2219}&\tgf{1374}&\tgf{0330}&\tgf{4601} \\
	\tinynb{4028} &\tinynb{2219}&\tinynb{1374}&\tinynb{0330}&\tinynb{4601} \\
\end{tabular}
\begin{exe}
\ex \label{ex:tg:rever}  \vspace{-8pt}
\gll   \ipa{nji²} \ipa{kjij¹-tɕhjɨ¹-mjiij¹-nja²}  \\
		toi \opt{}-\pot{}-rêver-2\sg{} \\
\glt  Aurais-tu rêvé ? (Leilin 6.16B.4)
\end{exe}




\item 2192  \tgf{2192} \ipa{mjiij} 1.39 \ptang{mjaaŋ}{cadavre} n'a pas de cognat en japhug ou en tibétain, mais on trouve des formes apparentées dans de nombreuses langues de l'ensemble de la famille sino-tibétaine (voir \citealt[265]{matisoff03}) comme le jingpo \ipa{māng} ``cadavre''. Toutefois, il est possible que ce nom soit un dérivé du verbe ``mourir'' 781 \tgf{0781} \ipa{mjij} 2.33 (aussi écrit 788 \tgf{0788}) qui pourrait provenir d'un *\ipapl{mjaŋ}. Ce verbe pourrait être rapproché du chinois \zh{亡} *\ipapl{maŋ} ``disparaître''. Si c'est le cas, la dérivation de ``mourir'' à ``cadavre'' a dû s'opérer à une date très ancienne, avant la séparation des langues où un cognat de \tgf{2192} \ipa{mjiij¹}  apparaît.
\end{enumerate}
B. Initiales dentales \label{rimes:07:2:dent}
\begin{enumerate}



\item 3574  \tgf{3574} \ipa{tsjij} 2.33 \ptang{tsjaŋ}{comprendre} est comparable au \jpg{tso} ``comprendre''. Ce verbe a une forme alternante 3615	\tgf{3615} \ipa{tsji} 2.10. Elle n'est pas attestée dans les textes connus, mais elle apparaît dans la glose de \tgf{3574} \ipa{tsjij²} dans le Tongyin, et il semblerait possible  qu'il s'agisse du thème B. Toutefois, on trouve de nombreux cas, comme l'exemple (\ref{ex:tg:attendre}) ci-dessous, où la forme \tgf{3574} \ipa{tsjij²} s'emploie avec un agent à la première personne   du singulier, ce qui va à l'encontre de cette hypothèse.


\item 638  \tgf{0638} \ipa{njij} 2.33  \ptang{njaŋ}{chasser, forcer} est apparenté avec \ipa{no} ``chasser''. Dans les deux langues, ce verbe s'emploie notamment dans le sens de ``pousser, chasser'' à propos des animaux (le sens du chinois \zh{驅趕} \textit{qūgǎn}), comme la phrase \tgf{3452}\tgf{2770} \tgf{0638} \ipa{.jij²lo² njij²} ``pousser un troupeau de mouton'' (Sunzi, 40A.4b). 



\item 5554  \tgf{5554} \ipa{njij} 2.33  est peut-être comparable au \jpg{sɤŋo}   ``écouter'', auquel cas une reconstruction \ptang{ŋjaŋ}{écouter?} pourrait être proposée. \citet{gong01huying} pense que 3575 \tgf{3575} \ipa{nji} 2.10 ``obéir, écouter'' est le thème B de \tgf{5554} \ipa{njij²}, mais \tgf{3575} \ipa{nji²} peut apparaître avec un agent autre que 1sg/2sg, ce qui invalide cette hypothèse. \tgf{5554} \ipa{njij²} n'est pas attesté dans les textes connus, et même dans les dictionnaires, il n'apparaît que dans la forme composée \tgf{5554}\tgf{2454}\tgf{5224} \ipa{njij²phjɨ¹lụ¹} qui semble vouloir  dire ``écouter''; cette attestation limitée rend cette comparaison problématique.



\item 5500  \tgf{5500} \ipa{sjij} 1.36  \ptang{sjaŋ}{prochaine (année)} apparaît dans le composé \tgf{5500}\tgf{2712} \ipa{sjij¹.wji¹} ``l'année prochaine''. Cette syllabe peut se comparer avec  le \jpg{fso}  ``demain'' et la première syllabe de \ipa{fsaqhe} ``l'année prochaine''. On peut également comparer le \tib{saŋ.ɲin} ``demain'' et \tib{saŋ.lo} ``l'année prochaine''.

La forme 3133 \tgf{3133} \ipa{sjij} 1.36 ``aujourd'hui'' est peut-être aussi apparentée. Il faut partir du sens ``matin'' (\jpg{sos}), d'où un développement sémantique soit vers ``demain'', soit vers ``ce matin'', d'où ``aujourd'hui''.



\item 2621 \tgf{2621} \ipa{sjiij} 2.35 \ptang{sjaaŋ}{penser} est apparenté au \jpg{sɯso} ``penser, vouloir''. Le verbe japhug est une forme rédupliquée, et la racine provient d'un proto-japhug *\ipapl{saŋ}. Le chinoise \zh{想} sjaŋX < *[s]aŋʔ y est sans doute aussi apparenté.

En tangoute, ce verbe signifie ``réfléchir'' comme l'illustre l'exemple (\ref{ex:tg:attendre}), mais il peut aussi s'employer dans le sens modal de ``vouloir''. 



\item 0046  \tgf{0046} \ipa{ljij} 2.33 \ptang{ljaŋ}{voir} peut se comparer aux \jpg{nɤmɲo}  ``regarder'' et du \plb{ʔ-mraŋ¹}{0596}. Ce verbe a pour thème B \tgf{4803} \ipa{lji²} (voir un exemple p.\pageref{ex:tg:partir}).




\item 5522  \tgf{5522} \ipa{ljiij} 2.35  \ptang{ljaaŋ}{attendre} est cognat avec le \jpg{nɤjo} ``attendre''. Le \plb{C-lo}{0711} est en revanche sans lien avec cette forme. Ce verbe a un thème B 5293 \tgf{5293} \ipa{ljii} 2.12 qui apparaît avec un agent de 1sg/2sg (*ljaaŋ-u), comme l'illustre l'exemple suivant:
\newline
\linebreak
\begin{tabular}{llllllllll}
	\tgf{3592}&	\tgf{3527}&	\tgf{3574}&	\tgf{0734}&	\tgf{3926}&	\tgf{1734}&	\tgf{4841}&	\tgf{1014}&	\tgf{4601}&	\tgf{3592}\\
\tinynb{3592}&	\tinynb{3527}&	\tinynb{3574}&	\tinynb{0734}&	\tinynb{3926}&	\tinynb{1734}&	\tinynb{4841}&	\tinynb{1014}&	\tinynb{4601}&	\tinynb{3592}\\
\tgf{2621}&	\tgf{2098}&	\tgf{2590}&	\tgf{5293}& &&&&&\\
\tinynb{2621}&	\tinynb{2098}&	\tinynb{2590}&	\tinynb{5293}& &&&&&\\
\end{tabular}
\begin{exe}
\ex \label{ex:tg:attendre}  \vspace{-8pt}
\gll   \ipa{ɣjɨr¹}	\ipa{mja¹-tsjij²-mo²}	\ipa{nja²}	\ipa{tji¹-djij²-ŋwuu¹-nja²}	\ipa{ɣjɨr¹}	\ipa{sjij²-ŋa²}	\ipa{.wjɨ²-ljii²} \\
		moi \hypot{}-comprendre[A]-\hypot{} toi \prohib{}-\opt{}-parler-2\sg{} moi penser-1\sg{} \dir{}-attendre[B] \\
\glt  Je devrais comprendre, ne dis rien, attends que j'aie réfléchi. (Leilin 04.28B.7)
\end{exe}
\end{enumerate}


B. Initiales palatales, rhotiques et vélaires \label{rimes:07:2:lab}

\begin{enumerate}

\item 143 \tgf{0143} \ipa{dʑjij} 2.32  \ptang{ndraŋ}{froid, glacé} est apparenté au \jpg{ɣɤndʐo} ``froid''. Ce verbe statif peut s'écrire également 4033 \tgf{4033} \ipa{dʑjij} 2.32. Il a une forme au ton 1, 735  \tgf{0735} \ipa{dʑjij} 1.36. Une relation avec le \tib{graŋ-mo} ``froid'' est plus conjecturale mais pas impossible: il faudrait supposer que le pré-tangoute vient de *\ipapl{ndraŋ} < **\ipapl{n-raŋ}, et que la vraie racine est donc */\ipapl{raŋ}/; la forme tibétain aurait donc été formée avec un préfixe g- distinct (voir \citet[26]{sagart99roc} pour une discussion du cognat chinois de cette racine).


\item 3807  \tgf{3807} \ipa{.jij} 1.36  \ptang{jaŋ}{léger} est cognat du \jpg{ʑo} et du \tib{jaŋ.po} de même sens.


\item 1245  \tgf{1245} \ipa{.jij} 1.36  \ptang{jaŋ}{soi-même} peut être rapproché du \jpg{tɯ-ʑo} ``soi-même''. Le \tib{raŋ} ``soi-même'' est peut-être également apparenté. Nous avons montré dans \citet{jacques10refl} que le préfixe réfléchi \ipa{ʑɣɤ-} du japhug était également dérivé de la même racine. Voir \ref{subsubsec:innovations.japhug} pour une discussion plus détaillée de cette comparaison.


\item 3452  \tgf{3452} \ipa{.jij} 2.33 \ptang{jaŋ}{mouton} est comparable au \jpg{qaʑo} ``mouton'' et au \tib{gjaŋ-dkar} ``mouton''.\footnote{Le \tib{gjag} est évidemment dans rapport aucun avec cette racine, contrairement à \citet[523]{matisoff03}. }



\item 764  \tgf{0764} \ipa{rjijr} 1.74  \ptang{rjaŋ}{cheval} est apparenté au \jpg{mbro}  ainsi que le \plb{mraŋ²}{0006} (\bir{mraŋ³} ``cheval''). Dans la quasi-totalité des langues macro-rgyalronguiques et lolo-birmanes, le nom ``cheval'' et le verbe statif ``haut'' sont des quasi-homophones, distingués seulement par leur ton (en japhug, les deux se prononcent \ipa{mbro}), mais en tangoute les deux sont différents, car ``haut'' se dit \tgf{1890} \ipa{bjij¹}. On doit proposer des formes reconstruites distinctes *\ipapl{rjaŋ} et *\ipapl{mb(r)jaŋ} en pré-tangoute.

 \tgf{0764} \ipa{rjijr¹} a une forme alternante 803 \tgf{0803} \ipa{rjar} 2.74, qui a peut-être un équivalent en muya, comme l'a suggéré \citet{ikeda04}. La reconstruction exacte de cette forme pose problème. \tgf{0803} \ipa{rjar²} pourrait théoriquement provenir de *rjaC, où C représente *--p, *--t, *--k, *--s ou *--r. Pour expliquer l'alternance avec  \tgf{0764} \ipa{rjijr¹}, il faudrait admettre une loi phonétique *\ipapl{rjaŋ-C} > *\ipapl{rja-C}. Un suffixe indéterminé --C aurait fait chuter la nasale finale déjà en pré-tangoute. On observe la même alternance avec le verbe ``envoyer''  \tgf{5974} \ipa{.wjịj} p.\pageref{ex:tg:envoyer}.


\item 2478  \tgf{2478} \ipa{khjij} 1.36  \ptang{khjaŋ}{étendre} peut se comparer au \jpg{ɕkho} ``étaler, étendre''. Dans les deux langues, ces verbes sont utilisés dans le sens d'``étendre des herbes pour les faire sécher'': 
\newline
\linebreak
\begin{tabular}{llllll}
	\tgf{0683}&	\tgf{2182}&	\tgf{3070}&	\tgf{0089}&	\tgf{3612}&	\tgf{2478}\\
	\tinynb{0683}&	\tinynb{2182}&	\tinynb{3070}&	\tinynb{0089}&	\tinynb{3612}&	\tinynb{2478}\\
\end{tabular}
\begin{exe}
\ex \label{ex:tg:entendre}  \vspace{-8pt}
\gll   \ipa{xia¹thow¹}	\ipa{dzjwɨ̣¹}	\ipa{tɕhjaa¹}	\ipa{tsə̣¹}	\ipa{khjij¹} \\
		Xia.Tong bateau sur médicament étendre \\
\glt  Xia Tong (\zh{夏統}) faisait sécher des médicaments sur son bateau. (Leilin 03.32B.6)
\end{exe}


\item 3626  \tgf{3626} \ipa{khjij} 2.33 ``pigeon'' pourrait être rapproché du \jpg{qro}  ``pigeon'' et reconstruit *\ipapl{khjaŋ}. Toutefois, la forme \ipa{ɕtʂó} du situ suggère que la vélaire est ici préfixale, et sans une meilleure compréhension de l'évolution de ce mot dans les langues rgyalrong, il est difficile de juger si cette comparaison est possible.
\end{enumerate}
\subsubsection{Autres correspondances}	\label{subsubsec:correspondance:jij:autre}
Outre les deux groupes de correspondances présentés ci-dessus, on trouve quelques formes inhabituelles, pour lesquelles aucune reconstruction n'est possible pour le moment.



Deux noms présentent un --a en japhug correspondant à --jij en tangoute. 


Il s'agit tout d'abord de 1204  \tgf{1204} \ipa{njijr} 2.68  ``visage''\footnote{Noter également la forme problématique 3158  \tgf{3158} \ipa{ŋuər} 2.76 qui devrait venir d'un *\ipapl{r-ŋu}.	} correspondant au \jpg{tɯ-rŋa} de même sens. Cet exemple n'est toutefois pas certain; on pourrait suggérer une étymologie alternative par le mongol \ipa{nihur} ``visage'': il s'agirait d'un emprunt au khitan ou à une autre langue para-mongolique. On doit mentionner une étymologie alternative pour le \jpg{tɯ-rŋa}: \tgz{3158} ``visage''.


Par ailleurs, on trouve 973  \tgf{0973} \ipa{.wjijr} 2.68  ``meule'' correspondant au \jpg{βɣa} ``moulin'' (<*kpa).\footnote{Ce nom a une forme 1162 \tgf{1162} \ipa{.wjịj}  2.60} Il n'est pas possible dans l'état de nos connaissances de proposer une reconstruction de la rime de ces mots en pré-tangoute, car on attendrait *njir et *.wjir respectivement si les proto-formes étaient *\ipapl{r-ŋja} et *\ipapl{C-r-pja}.




Le verbe 5778 \tgf{5778} \ipa{khjij} 1.36  ``couper'' peut se comparer au \jpg{rɤkrɯ} ``couper (des légumes)''. Ce verbe a la forme alternante 4458	\tgf{4458} \ipa{khji} 2 .10 ainsi que le dérivé nominal 5769 \tgf{5769} \ipa{khji} 1.11 ``couteau''. Une reconstruction *khjej en pré-tangoute serait logique, mais la forme tangoute est inexplicable. Il s'agit peut-être d'un mot comme \ipa{tɯ-mbɯ} (voir p.\pageref{rimes:07:1:lab}).



4684  \tgf{4684} \ipa{mej} 1.33 ``œil'' est cognat du \jpg{tɯ-mɲaʁ} et du \tib{mig} ou \ipa{dmʲig} de même sens. Là encore, la forme tangoute est inattendue: un pré-tangoute *mjak devrait avoir donné *mja, tandis qu'un pré-tangoute *mik serait devenu *mew. \ipa{mej} n'a d'ailleurs qu'une origine possible d'après les règles que nous avons proposées : *mej ``œil'' \index{*mej ``œil''} sans occlusive finale, qui ne correspond à aucune forme connue. en sino-tibétain 

\subsection{Rimes \ipapl{--əj} / \ipapl{--ɨj}} \label{subsec:voyelle.euj}

Les rimes du \ipa{shè} n°8 sont reconstruites par Gong Hwangcherng avec les voyelles  \ipapl{ə} et \ipapl{ɨ} qui apparaissaient dans le \ipa{shè} n°6, suivies d'une finale --j. Plusieurs autres auteurs, en revanche, reconstruisent des finales nasales pour les rimes 42, 43 et 65. On verra que les données comparatives montrent que les rimes 43 et 65 viennent exclusivement de rimes à finales nasales en pré-tangoute.


\begin{table}
\captionb{Reconstructions du \ipa{shè} n°8}\label{tab:she8}
\resizebox{\columnwidth}{!}{
\begin{tabular}{lllllllll} \toprule
rime&ton 1&ton 2&Sofronov1&Sofronov2&Nishida&Li&Gong&Arakawa\\
41&	1.40&	&	\ipa{ai+C}&	\ipa{ai}&	\ipa{ǐe}&	\ipa{ɛ}&	\ipa{əj}&	\ipa{eː’}\\	
42&	1.41&	2.36&	\ipa{ai+C}&	\ipa{ai}&	\ipa{eɴ}&	\ipa{ɪɛ}&	\ipa{iəj}&	\ipa{en}\\	
43&	1.42&	2.37&	\ipa{i̯ai+C}&	\ipa{i̯e}&	\ipa{ieɴ}&	\ipa{ɪɛ̃}&	\ipa{jɨj}&	\ipa{yen}\\	
65&	1.62&	2.55&	\ipa{?}&	\ipa{Ị}&	\ipa{ɛ̣ɴ}&	\ipa{ɪẹ̃}&	\ipa{jɨ̣j}&	\ipa{yenq}\\
76&	&	2.65&	\ipa{}&	\ipa{}&	\ipa{ʷẹ}&	\ipa{ọ̃}&	\ipa{iə̣j}&	\ipa{eq2}\\
\bottomrule
\end{tabular}}
\end{table}
Le nombre d'exemples de cognats appartenant à ce  \ipa{shè} est bien moindre que pour les sections précédentes (voir tableau \ref{tab:comparaisons:euj}).



\begin{table}
\captionb{Comparaison des étymons en  --\ipapl{əj} et --\ipapl{ɨj} du tangoute avec le japhug et le tibétain.}\label{tab:comparaisons:euj}
\resizebox{\columnwidth}{!}{
\begin{tabular}{lllllllll} \toprule
\multicolumn{4}{c}{tangoute} & sens & japhug & tibétain  \\
\midrule
\tinynb{2560	}& \tgf{2560} & \ipa{.jɨj} &	\tinynb{2.37}	& maison&\ipa{		} &	kʰʲim	\\
\tinynb{3433	}& \tgf{3433} & \ipa{djɨj} &	\tinynb{1.42}	&peu profond \ipa{		} &		\\
\tinynb{2738	}& \tgf{2738} & \ipa{djɨ̣j} &	\tinynb{2.55}	& nuage&\ipa{zdɯm} &		\\
\tinynb{5497	}& \tgf{5497} & \ipa{ɣjɨj} &	\tinynb{1.42}	& oreiller&\ipa{tɤ-mkɯm	} &		\\
\tinynb{1079	}& \tgf{1079} & \ipa{ljɨ̣j} &	\tinynb{2.55}	& bon à manger&\ipa{		} &	ʑim-po	\\
\tinynb{1861	}& \tgf{1861} & \ipa{tshjɨj} &	\tinynb{1.42}	& fin&\ipa{xtshɯm} &		\\
\tinynb{3798	}& \tgf{3798} & \ipa{tsəj} &	\tinynb{1.40}	&petit &\ipa{xtɕi	} &		\\
\tinynb{5109	}& \tgf{5109} & \ipa{.wiəj} &	\tinynb{1.36}	& pet&\ipa{tɯ-phe} (Gsardzong) &		\\\bottomrule
\end{tabular}}
\end{table}
Pour les rimes 43 et 65, on a une correspondance très claire avec le proto-japhug  *--im et le tibétain --im, sauf pour le nom ``oreiller'' qui n'a pas une rime à voyelle antérieure dans les autres langues sino-tibétaines. On reconstruira *--im en pré-tangoute pour tous les étymons avec cette rime.

Pour les deux autres exemples, on observe une correspondance avec des voyelles antérieures en japhug, mais il est difficile de proposer une reconstruction basée sur si peu d'exemples. On reconstruira provisoirement *--ij.

Dans un travail antérieur (\citealt{jacques06comparaison}), nous avions comparé le tangoute 5605 \tgf{5605} \ipa{ljɨj} 2.37	``ours'' avec le \jpg{pri}. La reconstruction donnée dans \citet{lifw97} pour ce mot est \ipa{rjɨj}, mais ce caractère appartient au groupe d'homophones dans le Tongyin correspondant à la syllabe \ipa{ljɨj} 2.37 et il n'y a pas lieu de reconstruire un r-- ici, d'autant plus que les rhotiques sont incompatibles avec des rimes du cycle majeur. Cette comparaison doit donc être rejetée.

\begin{enumerate}


\item 3433	\tgf{3433} \ipa{djɨj}	1.42	\ptang{ndim}{peu profond} est comparable au \bir{tim²} (voir \citet[271]{matisoff03} qui présente de nombreux cognats en lolo-birman et en macro-rgyalronguique). 


\item 2738	\tgf{2738} \ipa{djɨ̣j} 2.55	\ptang{S-ndim}{nuage} est apparenté au \jpg{zdɯm} et au \plb{C-dim¹}{0320b} (\bir{tim²}).


\item 1861 \tgf{1861} \ipa{tshjɨj} 1.42	 \ptang{tshim}{fin}  est cognat avec le \jpg{xtshɯm}. Consulter la discussion p.\pageref{analyse:fin}.


\item 1079	\tgf{1079} \ipa{ljɨ̣j} 2.55	\ptang{lim}{bon à manger} est apparenté au \tib{ʑim-po}	(<*ljim). Ce verbe statif a une forme dérivée 4616 \tgf{4616} \ipa{ljiij} 2.35 qui quant à elle signifie ``trouver bon à manger, aimer manger''. 
\newline
\linebreak
\begin{tabular}{llllllllll}
	\tgf{1402}&	\tgf{1139}&	\tgf{0605}&	\tgf{2923}&	\tgf{0856}&	\tgf{2698}&	\tgf{5399}&	\tgf{3551}&	\tgf{2045}&	\tgf{4616}\\
\tinynb{1402}&	\tinynb{1139}&	\tinynb{0605}&	\tinynb{2923}&	\tinynb{0856}&	\tinynb{2698}&	\tinynb{5399}&	\tinynb{3551}&	\tinynb{2045}&	\tinynb{4616}\\
\end{tabular}
\begin{exe}
\ex \label{ex:tg:trouver.bon}  \vspace{-8pt}
\gll   \ipa{xũ¹}	\ipa{jij¹}	\ipa{tjọ²}	\ipa{phji¹}	\ipa{mər²}	\ipa{tsjir²}	\ipa{khju¹}	\ipa{niow²}	\ipa{.o²}	\ipa{ljiij²} \\
		Hong \antierg{} petit.frère Bi origine nature bas mauvais alcool trouver.bon \\
\glt  Bi (\zh{弼}), le petit frère de Hong (\zh{弘}) était vil et bas par nature, et aimait l'alcool. (Cixiaozhuan 20.6-7, \citealt[64]{jacques07textes})
\end{exe}
Nous ne pouvons pas proposer pour le moment une interprétation de cette alternance en pré-tangoute.


\item 2560	\tgf{2560} \ipa{.jɨj} 2.37 \ptang{jim ou *C-tɕim}{maison}    est apparenté au situ \ipa{tə-tɕîm}, au \tib{kʰʲim}, au \plb{yim¹}{0341} (\bir{ʔim²}) et au  \pumi{tɕə̃́} signifiant tous ``maison''. La reconstruction de l'initiale de ce mot pose problème. \citet[212]{sagart06review} propose de reconstruire des uvulaires lorsque l'initiale zéro du birman correspond à une vélaire en tibétain. En japhug, on trouve une affriquée palatale, mais on peut montrer que certaines palatales viennent d'uvulaires en situ (voir \citealt[359-60]{jacques08}). En pré-tangoute, on ne reconstruit pas de toute façon d'uvulaires, et la forme \tgf{2560} \ipa{.jɨj²} n'est pas facile à expliquer. L'hypothèse que nous adopterons est de supposer une initiale affriquée comme en pumi ou en situ précédée d'un préfixe *C-- lénifiant.



\item 5497 \tgf{5497} \ipa{ɣjɨj} 1.42 \ptang{C-kim}{oreiller} est  cognat avec le \jpg{tɤ-mkɯm}  ``oreiller'' et le \plb{m-gum²}{0380}. Un terme pour ``oreiller'' apparaît dans de nombreuses langues sino-tibétaines (\citealt[275]{matisoff03}), mais on retrouve un vocalisme --u-- comme dans le jingpo \ipa{hkúm} ``oreiller''. Un vocalisme de ce type est attesté également en tangoute avec la forme 1440 \tgf{1440} \ipa{ɣjow} 1.56 qui provient d'un *C-kvm. La raison pour cette alternance vocalique, ainsi que la différence sémantique entre \tgf{5497} \ipa{ɣjɨj¹} et \tgf{1440} \ipa{ɣjow¹} sont encore peu clairs.


\item 3798	\tgf{3798} \ipa{tsəj} 1.40	  \ptang{tsij}{petit} est probablement apparenté au \jpg{xtɕi} ``petit'', bien que la correspondance des initiales ne soit pas celle que l'on attendrait.



\item 5109	\tgf{5109} \ipa{.wiəj} 1.36 \ptang{C-pij ??}{pet} peut se comparer au japhug de Gsardzong \ipa{tɯ-phe} ``pet''. Il peut s'agir toutefois d'une onomatopée lexicalisée indépendamment dans les deux langues.
\end{enumerate}


A part les mots ci-dessus, on trouve parmi les mots de ce \ipa{shè} un étymon \ipa{tjɨ̣j} 2.55 ``modèle, miroir, sceau, rite''   pour lequel une étymologie khitane est envisageable. Ce mot est écrit de façon différente selon ses usages: 1904 \tgf{1904} ``carte'', 1910 \tgf{1910}  ``rite'', 1962 \tgf{1962}  ``modèle'', 2528 \tgf{2528}  ``gouvernement'', 3607 \tgf{3607} ``miroir'', 5779 \tgf{5779} ``sceau''. Selon \citet[66]{kane09kitan}, on trouve en khitan un mot  \khitan{ǿΆ} <254-257> *<d-em> signifiant ``to grant an honorary title (=\zh{封} \textit{fēng})'' qu'il rapproche du mongol \ipa{temdeg} et du manchou \ipa{temgetu} ``signe, marque''. Cette forme pourrait être l'origine de l'étymon tangoute, étant donné qu'il suit la correspondance /voyelle antérieure+m/ : --\ipapl{jɨj}. Le sens premier en tangoute serait aussi celui de ``signe''; les autres sens seraient secondaires.




\subsection{Rimes \ipapl{--ew} / \ipapl{--iw}} \label{subsec:voyelle.ew}

Les rimes du \ipa{shè} n°9 sont reconstruites par Gong Hwangcherng avec les voyelles  \ipapl{e} et \ipapl{i} qui apparaissaient dans le \ipa{shè} n°2, suivies d'une finale --w. Des solution similaires sont adoptées par les autres auteurs (Sofronov reconstruisait *--\ipapl{ɯ}), sauf pour les deux rimes 93 et 94 du second cycle mineur, que Nishida, Li et Arakawa reconstruisent sans finale --w/--u.

\begin{table}
\captionb{Reconstructions du \ipa{shè} n°9}\label{tab:she9}
\resizebox{\columnwidth}{!}{
\begin{tabular}{lllllllll} \toprule
rime&ton 1&ton 2&Sofronov1&Sofronov2&Nishida&Li&Gong&Arakawa\\
44&	1.43&	2.38&	\ipa{eɯ}&	\ipa{eɯ}&	\ipa{əw}&	\ipa{e̠}&	\ipa{ew}&	\ipa{eu}\\	
45&	1.44&	2.39&	\ipa{êɯ}&	\ipa{êɯ}&	\ipa{ew}&	\ipa{eʊ}&	\ipa{iew}&	\ipa{yeu}\\	
46&	1.45&	2.40&	\ipa{i̯eɯ}&	\ipa{i̯eɯ}&	\ipa{ǐəw}&	\ipa{ieʊ}&	\ipa{jiw}&	\ipa{euː}\\	
47&	1.46&	&	\ipa{i̯eɯ}&	\ipa{i̯eɯ}&	\ipa{ǐew}&	\ipa{ǐeʊ}&	\ipa{jiw}&	\ipa{euː}\\	
48&	&	2.41&	\ipa{əɯ}&	\ipa{əɯ}&	\ipa{ǐew}&	\ipa{eu̠}&	\ipa{eew}&	\ipa{eu’}\\	
49&	1.47&	&	\ipa{i̯əɯ}&	\ipa{i̯əɯ}&	\ipa{iw}&	\ipa{ǐeu̠}&	\ipa{jiiw}&	\ipa{yeu’}\\	
93&	1.87&	2.78&	\ipa{ẹɯ}&	\ipa{ẹɯ}&	\ipa{iə̣r}&	\ipa{ǐə̠̣}&	\ipa{ewr}&	\ipa{er}\\
94&	1.88&	2.79&	\ipa{ə̣ɯ}&	\ipa{ə̣ɯ}&	\ipa{iʉr}&	\ipa{ǐə̣}&	\ipa{jiwr}&	\ipa{yer}\\
\bottomrule
\end{tabular}}
\end{table}
Gong reconstruit les rimes 46 et 47 de façon similaire  car celles-ci se trouvent en distribution complémentaire par rapport au lieu d'articulation des consonnes initiales.

La correspondance générale des rimes du \ipa{shè} n°9 est avec des rimes à finale -g en tibétain et --\ipapl{ɣ} en japhug. On reconstruira *--vk en pré-tangoute, où *v signifie ici *i, *u ou *e; en effet, *--ak donne --a et *--ok donne --o: les deux voyelles *a et *o suivies de *--k ne se confondent pas avec les autres.





\begin{table}
\captionb{Comparaisons des étymons en  --\ipapl{ew} et --\ipapl{iw} du tangoute avec le japhug et le tibétain.}\label{tab:comparaisons:ew}
\resizebox{\columnwidth}{!}{
\begin{tabular}{lllllllll} \toprule
\multicolumn{4}{c}{tangoute} & sens & japhug & tibétain  \\
\midrule
\tinynb{5705	}& \tgf{5705}& \ipa{ɕjiw} &	\tinynb{2.40}	& pou &\ipa{zrɯɣ} &	ɕig	\\
\tinynb{3305	}& \tgf{3305}& \ipa{kjiw} &	\tinynb{1.45}	& année & &		\\
\tinynb{2768	}& \tgf{2768}& \ipa{kjiwr} &	\tinynb{1.88}	& fourmi &\ipa{qro} &	grog.ma	\\
\tinynb{1377	}& \tgf{1377}& \ipa{kjiwr} &	\tinynb{2.79}	& courber &\ipa{kɤɣ} &	gug-po	\\
\tinynb{993	}& \tgf{0993}& \ipa{lhew} &	\tinynb{1.43}	& faire paître &\ipa{lɤɣ	} &		\\
\tinynb{5120 }& \tgf{5120}& \ipa{swew} &	\tinynb{1.43}	& clair, lumineux &\ipa{fsoʁ} &		\\
\tinynb{3457	}& \tgf{3457}& \ipa{sjiw} &	\tinynb{1.46}	&nouveau & \ipa{ɕɤɣ} &		\\
\tinynb{55 }& \tgf{0055} & \ipa{tɕjiw} & \tinynb{2.40} & sommet de la tête & \ipa{kɤcɯɣ} & gtsug ? \\
\tinynb{3200	}& \tgf{3200}& \ipa{tɕhjiw} &	\tinynb{1.46}	& six & \ipa{kɯtʂɤɣ} &	drug	\\
\tinynb{91	}& \tgf{0091}& \ipa{thew} &	\tinynb{2.38}	& rencontrer &\ipa{atɯɣ	} &	tʰug	\\
\tinynb{4739	}& \tgf{4739}& \ipa{tsewr	} &	\tinynb{1.87}	&   section &\ipa{tɯ-rtsɤɣ} &	tsʰigs	\\
\tinynb{5480	}& \tgf{5480}& \ipa{zewr} &	\tinynb{2.78}	& léopard &\ipa{kɯrtsɤɣ} &	gzig	\\
\tinynb{4118	}& \tgf{4118}& \ipa{ʑjiw} &	\tinynb{1.46}	& genévrier &\ipa{ɕɤɣ} &	ɕug	\\
\midrule
\tinynb{1298	}& \tgf{1298}& \ipa{kjiwr} &	\tinynb{1.79}	& coude &\ipa{tɯ-zgrɯ} &	gru.mo	\\
\tinynb{3299	}& \tgf{3299}& \ipa{lwew} &	\tinynb{1.43}	& vapeur &\ipa{tɤ-jlɤβ} &		\\
\bottomrule
\end{tabular}}
\end{table}
Le changement de *--k à --w a dû s'effectuer par l'intermédiaire d'un *--\ipapl{ɣ}, c'est à dire l'étape observée actuellement en japhug de Kamnyu. Les emprunts au tibétain et au chinois à finale occlusive vélaire ne correspondent pas en général à cette rime, sauf peut-être 864 \tgf{0864} \ipa{dew} 2.38 ``fruit'' qui pourrait provenir de la seconde syllabe du \tib{ɕiŋ-tog} ``fruit''.


Pour la correspondance avec le japhug --\ipapl{ɤβ} dans le nom ``vapeur'', on reconstruira ici *--op, rime distincte de *--vp reconstruite  p.\pageref{subsubsec:correspondance:eu:vp} : le \jpg{tɤ-jlɤβ} vient d'un *\ipapl{jlɔp}, comme le montre la correspondance avec le situ \ipa{ta-jlôp}.


Le nom du ``coude'' 1298	\tgf{1298} \ipa{kjiwr} 1.79	\label{analyse:coude}pourrait se comparer au \jpg{tɯ-zgrɯ} et au \tib{gru.mo}	de même sens. En proto-japhug, le mot coude doit se reconstruire \ipa{*--kru} (\situ{tə-krú}, gsardzong \ipa{tə-ɣru}). ɬa forme de Kamnyu \jpg{tɯ-zgrɯ} provient d'une proto-forme \ipa{*sə-kru} dans laquelle l'occlusive s'est d'abord spirantisée devant –r, puis a subi une fortition  du fait de la présyllabe (un groupe comme zɣr- est impossible en japhug, \citealt[315]{jacques04these}). 

Une autre comparaison possible, qui rendrait mieux compte de la correspondance de la rime pourrait s'établir avec la racine du \jpg{kɤɣ} ``courber", mais le détail phonologique et morphologique est obscur.

Un pré-tangoute *S-krju donnerait toutefois *\ipapl{kjwɨr} et non \ipa{kjiwr¹} selon les lois exposées p.\pageref{analyse:S-Cr}. \ipa{kjiwr¹} pourrait venir d'une forme *r-kjvk ou *S-krjvk. L'étymologie avec le   \jpg{tɯ-zgrɯ} est donc problématique, car il faudrait supposer un suffixe *--k non motivé. Une hypothèse alternative et peut-être plus probable serait de considérer ce nom comme un dérivé de la racine ``courber'' (voir p.\pageref{ex:tg:soumettre}).

On remarque l'absence de labiales parmi les mots de cette correspondance: on ne trouve que très peu de mots à initiale labiale et à rime du \ipa{shè} n°9 en tangoute. Les seuls exemples sont toujours des caractères de transcription employés pour retranscrire  le chinois. C'est là l'effet d'un phénomène de dissimilation.

A. Initiales dentales \label{rimes:09:dent}
\begin{enumerate}


\item 91	\tgf{0091} \ipa{thew} 2.38 \ptang{thvk}{attraper (une maladie)} est comparable au \jpg{atɯɣ} ``rencontrer'', au \plb{m-tók}{0697} et au \tib{tʰug} ``rencontrer''. Ce verbe est associé dans le Tongyin avec \tgf{0018} \ipa{thjwu²} qui est peut-être son thème B (*thvk-u),\footnote{Voir aussi p.\pageref{ex:tg:ouvrir2}. } mais il est impossible d'en avoir la certitude sans exemples textuels. Voir un exemple de ce verbe p.\pageref{ex:tg:appeler}.

En japhug également il peut s'employer dans le sens d'``attraper une maladie'' comme le montre l'exemple suivant:
\begin{exe}
\ex \label{ex:jpg:recontrer}  \vspace{-8pt} 
\gll   \textit{tɕhomba} \textit{ɲɤ-k-atɯɣ-chɯ} \\
		rhume \med{}-(liaison)-rencontrer-\med{} \\
\glt Il a attrapé un rhume.
\end{exe}

\item 4739	\tgf{4739} \ipa{tsewr} 1.87	\ptang{r-tsvk}{articulation, section} est cognat avec le  \jpg{tɯ-rtsɤɣ} ``une section'' (``articulation'' se dit \ipa{tɯ-rɯ-rtsɤɣ} en ajoutant le nom ``os'' \ipa{--rɯ--}) et le \tib{tsʰigs}	 ``articulation''. La forme \tgf{4209} \ipa{zew} 2.38	 qui apparait dans \tgf{3485}\tgf{4209} \ipa{la¹zew²}  ``poignet'' en est un dérivé, et peut se reconstruire *C-tsvk ou peut-être avec une voisée *C-dzvk, ce qui correspondrait mieux au \plb{C-dzìk}{0110}  ``poignet''.



\item 5120 \tgf{5120} ou 5692 \tgf{5692} \ipa{swew} 1.43 \ptang{p-svk}{brillant} peut se comparer au \jpg{fsoʁ} ``clair, lumineux (à propos du ciel)'' et peut-être le chinois \zh{夙} sjuk < *[s]uk.\footnote{La correspondance avec le chinois a été proposée par L. Sagart (p.c.).}




\item 3457	\tgf{3457} \ipa{sjiw} 1.46 \ptang{sjvk}{nouveau} peut se comparer au \jpg{ɕɤɣ} ``nouveau'' et \plb{C-ʃik}{0536} (\bir{sac}). La forme 3460 \tgf{3460} \ipa{sjwɨ} 1.30 qui apparaît dans \tgf{1894}\tgf{3460} \ipa{.jar¹sjwɨ¹} ``nouvelle mariée'' ainsi que 1597  \tgf{1597} \ipa{sjɨ} 2.28 de \tgf{1701}\tgf{1597} \ipa{lhjwi¹sjɨ²}, expression  qui semble signifier ``se remarier'', ont été tous deux rapprochés de \tgf{3457} \ipa{sjiw¹} par \citet{gong95st}. Une relation entre ces formes n'est pas absolument in-envisageable : \ipa{sjɨ²} pourrait provenir d'un *sjvs < *sjvk-s, autrement dit la forme \tgf{3457} \ipa{sjiw¹} suffixée. 



\item 5480	\tgf{5480} ou 5768 \tgf{5768} \ipa{zewr} 2.78	\ptang{C-r-tsvk}{léopard} peut être rapproché du \jpg{kɯrtsɤɣ}, du \plb{k-zìk}{0014} et du \tib{gzig}	de même sens.



\item 993	\tgf{0993}  ou 507 \tgf{0507} \ipa{lhew} 1.43 \ptang{lhvk}{faire paître} peut être rapproché du \jpg{lɤɣ} ``faire paître'' (<*\ipapl{lɔk}). On trouve un nom dérivé \tgz{0507} ``fourrage'' qui n'a pas d'équivalent en rgyalrong.


\end{enumerate}

B. Initiales palatales \label{rimes:09:pal}
\begin{enumerate}

\item 55  \tgf{0055} \ipa{tɕjiw}  2.40 \ptang{tɕvk}{sommet de la tête} est apparenté à la seconde syllabe du  \jpg{tɯ-kɤcɯɣ} ``fontanelle'' (la première syllabe est l'état construit de \ipa{tɯ-ku} ``tête''). Une comparaison avec le \tib{gtsug} ``sommet de la tête'' (sanskrit \ipa{kapāla-}) peut être envisagée. \tgf{0055} \ipa{tɕjiw²}  est un mot très employé dans les textes bouddhiques qui correspond au chinois \zh{頂} \textit{dǐng} et au \tib{gtsug}.



\item 3200	\tgf{3200} \ipa{tɕhjiw}	1.46	\ptang{thrjvk}{six} est cognat avec le \jpg{kɯtʂɤɣ} et le \tib{drug}	de même sens. Dans d'autres langues ST, on trouve un groupe en vélaire, comme en \plb{C-kròk}.


\item 5705	\tgf{5705} \ipa{ɕjiw} 2.40	\ptang{ɕvk}{pou} est comparable avec le \jpg{zrɯɣ} et le \tib{ɕig}	``pou''. En tibétain, on a toutefois, un changement phonétique *sr-- > sh--, et on attendrait une forme tangoute *srik > *srjvk > *zjiwr. On ne peut pas exclure la possibilité qu'il s'agisse donc d'un emprunt au tibétain, effectué à époque ancienne avant la lénition du *--k final.


\item 4118	\tgf{4118} \ipa{ʑjiw} 1.46	``genévrier'' est apparenté au \jpg{ɕɤɣ} et au \tib{ɕug} de même sens. Le voisement de l'initiale est inexplicable. Un pré-tangoute *\ipapl{ɕvk} devrait donner *\ipapl{ɕjiw}. 
\end{enumerate}
C. Initiales vélaires \label{rimes:09:vel}
\begin{enumerate}

\item 3305	\tgf{3305} \ipa{kjiw} 1.45	``année'' n'a de cognats ni en japhug, ni en tibétain ni en birman, mais c'est le réflexe d'un étymon que l'on retrouve en lolo-birman, reconstruit \plb{C-kòk} dans Bradley et *C-kuk^L dans \citet[357-8]{matisoff03}. On proposera donc *kjvk en pré-tangoute pour cette forme.


\item 2768	\tgf{2768} \ipa{kjiwr} 1.88	\ptang{k-rjvk}{fourmi} est comparable au \jpg{qro} (l'absence de finale est irrégulière au sein même des dialectes japhug), le \plb{p-rwàk}{0073} et le \tib{grog.ma} (< *k-rok). Comme nous l'avons mentionné p.\pageref{analyse:fourmi}, on ne reconstruira pas ici *r-kjvk (la reconstruction mécanique) car les données comparatives montrent que l'initiale vélaire est un préfixe, et que la vraie initiale de cette racine était un *r--.



\item 1377	\tgf{1377} \ipa{kjiwr} 2.79	\ptang{r-kjvk}{mauvais, penché} peut être rapproché d'une importante famille d'étymons comprenant le \jpg{kɤɣ} ``courber'' (et son dérivé anticausatif \jpg{ŋgɤɣ} ``être courbé'') et le \tib{gug-po}	 ``courbé'' et \tib{ɴgugs, bkug} ``courber, rassembler (les forces armées)'', voir aussi \plb{səkòk}{0774}. On attendrait a priori une initiale prénasalisée en japhug si \tgf{1377} \ipa{kjiwr²} est un verbe statif, mais comme ce verbe n'est pas attesté hors des dictionnaires, son sens exact est incertain.

Il faut également rapprocher de la même racine le verbe 3351	\tgf{3351} \ipa{ɣjiw} 1.46	(*C-kjvk) ``soumettre, vaincre". Une évolution sémantique du sens de ``courber (tr.)'' à ``soumettre'' n'est guère problématique.
\newline
\linebreak
\begin{tabular}{llllllllll}
	\tgf{0320}&	\tgf{5880}&	\tgf{3510}&	\tgf{3351}&	\tgf{1542}&	\tgf{1326}&	\tgf{2833}&	\tgf{5377}&	\tgf{2393}&	\tgf{5285}\\
\tinynb{0320}&	\tinynb{5880}&	\tinynb{3510}&	\tinynb{3351}&	\tinynb{1542}&	\tinynb{1326}&	\tinynb{2833}&	\tinynb{5377}&	\tinynb{2393}&	\tinynb{5285}\\
\end{tabular}
\begin{exe}
\ex \label{ex:tg:soumettre}  \vspace{-8pt}
\gll   \ipa{.wəə¹}	\ipa{ŋwu²}	\ipa{.jijr¹}	\ipa{ɣjiw¹}	\ipa{ku¹}	\ipa{kjɨ¹djɨj²}	\ipa{ŋwo²ljiij²}	\ipa{ljɨ¹} \\
		faible \conj{} fort soumettre alors certainement périr \cop{} \\
\glt  Si, étant faible, on tente de soumettre quelqu'un de plus fort, on est certain de périr. (Leilin 03.27A.1-2)
\end{exe}
\end{enumerate}

\subsection{Rimes en \ipapl{--o} } \label{subsec:voyelle.o}
Nous traiterons conjointement dans cette section des rimes du  \ipa{shè} n°10 et du \ipa{shè} n°11, car leur reconstruction ne peut pas s'effectuer de façon séparée. Gong Hwangcherng reconstruit les rimes du  \ipa{shè} n°10  avec une voyelle principale --o, et celles du \ipa{shè} n°11 comme --ow avec une finale --w. On remarque dans le tableau \ref{tab:she10} que le \ipa{shè} n°11 n'a pas de rime de premier cycle mineur (*--\ipapl{ọw}). Ce détail sera d'une grand importance pour notre reconstruction.

Si tous les auteurs reconstruisent une voyelle principale /o/ pour ces rimes, aucun ne partage avec Gong Hwangcherng l'idée d'une finale --w aux rimes du \ipa{shè} n°11. 

\begin{table}
\captionb{Reconstructions des \ipa{shè} n°10 et n°11}\label{tab:she10}
\resizebox{\columnwidth}{!}{
\begin{tabular}{lllllllll} \toprule
rime&ton 1&ton 2&Sofronov1&Sofronov2&Nishida&Li&Gong&Arakawa\\
50&	1.48&	&	\ipa{?}&	\ipa{i̯o}&	\ipa{oɦ}&	\ipa{ǐõ}&	\ipa{jwo}&	\ipa{o}\\	
51&	1.49&	2.42&	\ipa{o}&	\ipa{o}&	\ipa{ɔɦ}&	\ipa{o}&	\ipa{o}&	\ipa{o}\\	
52&	1.50&	2.43&	\ipa{o}&	\ipa{o}&	\ipa{ǐou}&	\ipa{ou}&	\ipa{io}&	\ipa{yo}\\	
53&	1.51&	2.44&	\ipa{i̯o}&	\ipa{i̯o}&	\ipa{ǐɔɦ}&	\ipa{ǐo}&	\ipa{jo}&	\ipa{oː}\\	
54&	1.52&	2.45&	\ipa{o+C}&	\ipa{oɯ}&	\ipa{ɔw}&	\ipa{ɔ}&	\ipa{oo}&	\ipa{o’}\\	
55&	1.53&	2.46&	\ipa{i̯o+C}&	\ipa{i̯oɯ}&	\ipa{ow}&	\ipa{uɔ}&	\ipa{ioo}&	\ipa{yo’}\\	
56&	1.54&	2.47&	\ipa{on}&	\ipa{on}&	\ipa{oɴ}&	\ipa{ɔ̃}&	\ipa{ow}&	\ipa{on}\\	
57&	1.55&	2.48&	\ipa{on}&	\ipa{on}&	\ipa{ǐoɴ}&	\ipa{iɔ̃}&	\ipa{iow}&	\ipa{yon}\\	
58&	1.56&	2.49&	\ipa{i̯on}&	\ipa{i̯on}&	\ipa{ǐʷoɴ}&	\ipa{uɔ̃}&	\ipa{jow}&	\ipa{oːn}\\	
59&	1.57&	&	\ipa{i̯uo}&	\ipa{i̯uo}&	\ipa{ǐʷo}&	\ipa{iɔ}&	\ipa{ioow}&	\ipa{o’’}\\	
60&	&	2.50&	\ipa{i̯uo+C}&	\ipa{i̯uo}&	\ipa{ǐʷo}&	\ipa{ǐɔ̠}&	\ipa{joow}&	\ipa{yo’’}\\	
73&	1.70&	2.62&	\ipa{ọ}&	\ipa{ọ}&	\ipa{ɔ̣}&	\ipa{ọ}&	\ipa{ọ}&	\ipa{oq}\\
74&	1.71&	2.63&	\ipa{ọn}&	\ipa{ọn}&	\ipa{ọɴ}&	\ipa{uọ}&	\ipa{iọ}&	\ipa{onq}\\
75&	1.72&	2.64&	\ipa{i̯ọn}&	\ipa{i̯ọn}&	\ipa{ǐoɴ}&	\ipa{ǐọ̃}&	\ipa{jọ}&	\ipa{yonq}\\
95&	1.89&	2.80&	\ipa{ụo}&	\ipa{ụo}&	\ipa{or}&	\ipa{uɔ̣}&	\ipa{or}&	\ipa{or}\\
96&	1.90&	2.81&	\ipa{i̯ụo}&	\ipa{i̯ụo}&	\ipa{ǐor}&	\ipa{iɔu}&	\ipa{jor/ior}&	\ipa{yor}\\
97&	1.91&	2.82&	\ipa{ụo+C}&	\ipa{ụo}&	\ipa{ɔr}&	\ipa{ɔ̣}&	\ipa{owr}&	\ipa{oːr}\\
98&	&	2.83&	\ipa{i̯ụo+C}&	\ipa{i̯ụo}&	\ipa{wor}&	\ipa{ǐɔ̣}&	\ipa{jowr}&	\ipa{wor}\\
102&	1.94&	&	\ipa{ọ}&	\ipa{ọ}&	\ipa{ʷọ}&	\ipa{ɑ̣}&	\ipa{oor}&	\ipa{woq2}\\
103&	1.95&	&	\ipa{i̯ọ}&	\ipa{i̯ọ}&	\ipa{ǐɑɴ}&	\ipa{ǐɑ̣̃}&	\ipa{joor}&	\ipa{yaːn}\\
\bottomrule
\end{tabular}}
\end{table}
On observe trois groupes de correspondances récurrentes. Premièrement, avec les rimes à *--m final (autres que *--im), que l'on reconstruira *--vm, v signifiant ici ``voyelle autre que i''. Deuxièmement, avec la rime --o du japhug ou --ang du tibétain, que l'on reconstruira *--\ipapl{aŋ}. Troisièmement, avec les rimes --\ipapl{oʁ}  ou --\ipapl{ɤɣ} du japhug ou le --og du tibétain, que l'on reconstruira *--ok. Outre ces trois groupes principaux, on trouve également quelques correspondances marginales qui seront traitées séparément.


\begin{longtable} {lllllll}
\captionb{Comparaison  des étymons en --o(w) du tangoute avec le japhug et le tibétain.}\label{tab:comparaisons:o}\\
\toprule
\multicolumn{4}{c}{tangoute} & sens & japhug & tibétain  \\
\midrule
\endfirsthead
\tinynb{1805}& \tgf{1805} & \ipa{.wọ} &\tinynb{1.70}& épais& \ipa{jpum} & sbom-po\\
\tinynb{4053}& \tgf{4053} & \ipa{.wọ} &\tinynb{1.70}& glace& \ipa{tɤ-jpɣom} & \\
\tinynb{4995}& \tgf{4995} & \ipa{ɕjow} &\tinynb{1.56}& fer& \ipa{ɕom} & \\
\tinynb{1099}& \tgf{1099} & \ipa{dow} &\tinynb{2.47}& ours& \ipa{} & dom\\
\tinynb{2262}& \tgf{2262} & \ipa{dʑjow} &\tinynb{1.56}& voler& \tiny \ipa{nɯqambɯmbjom} & \\
\tinynb{2584}& \tgf{2584} & \ipa{dzow} &\tinynb{1.54}& pont& \ipa{ndzom} & zam\\
\tinynb{1105}& \tgf{1105} & \ipa{khjow} &\tinynb{1.56}& donner& \ipa{kho} & \\
\tinynb{39}& \tgf{0039} & \ipa{kowr} &\tinynb{2.82}& étal& \ipa{tɤ-mɢom} & \\
\tinynb{4247}& \tgf{4247} & \ipa{lhjoor} &\tinynb{1.94}&empan  & \ipa{tɯ-ɟom} & ɴdom\\
\tinynb{4874}& \tgf{4874} & \ipa{low} &\tinynb{2.47}& large& \ipa{jom} & \\
\tinynb{115}& \tgf{0115} & \ipa{low} &\tinynb{2.47}& chaud& \ipa{} & \\
\tinynb{2290}& \tgf{2290} & \ipa{low} &\tinynb{2.47}& rond& \ipa{} & zlum po\\
\tinynb{5990}& \tgf{5990} & \ipa{niọ} &\tinynb{2.63}& épis& \ipa{kɯɕnom} & \\
\tinynb{0549}& \tgf{0549} & \ipa{niọ} &\tinynb{1.71}& sœur& \ipa{tɤ-snom} & \\
\tinynb{5668}& \tgf{5668} & \ipa{no} &\tinynb{2.42}& sentir& \ipa{nɤmnɤm} & mnam\\
\tinynb{2915}& \tgf{2915} & \ipa{no} &\tinynb{1.49}& côte& \ipa{tɯ-rnom} & \\
\tinynb{1572}& \tgf{1572} & \ipa{phiow} &\tinynb{1.55}& blanc& \ipa{wɣrum} & \\
\tinynb{3163}& \tgf{3163} & \ipa{phow} &\tinynb{1.54}& giron& \ipa{tɯ-phɯm} & \\
\tinynb{1417}& \tgf{1417} & \ipa{rowr} &\tinynb{1.91}& sec& \ipa{rom} & \\
\tinynb{5865}& \tgf{5865} & \ipa{sọ} &\tinynb{1.70}& trois& \ipa{χsɯm} & gsum\\
\tinynb{148}& \tgf{0148} & \ipa{tọ} &\tinynb{1.70}& se solidifier& \ipa{stɤm} & \\
\midrule
\tinynb{1616}& \tgf{1616} & \ipa{.o} &\tinynb{2.42}& venir& \ipa{} & joŋ\\
\tinynb{458}& \tgf{0458} & \ipa{kor} &\tinynb{1.89}& gorge& \ipa{tɯ-rqo} & \\
&&&&&< *rqaŋ & \\
\tinynb{2857}& \tgf{2857} & \ipa{ŋo} &\tinynb{2.42}& être malade& \ipa{ngo}  & \\
&&&&&< *tŋgaŋ & \\
\tinynb{3443}& \tgf{3443} & \ipa{po} &\tinynb{1.49}& oncle& \ipa{tɤ-βɣo}  & \\
&&&&&< *pkaŋ & \\
\midrule
\tinynb{8}& \tgf{0008} & \ipa{do} &\tinynb{1.49}& poison& \ipa{tɤ-ndɤɣ} & dug\\
&&&&&< *-ndɔk  & \\
\tinynb{2797}& \tgf{2797} & \ipa{lho} && sortir& \ipa{ɬoʁ}  & \\
&&&&&< *ɬoq & \\
\tinynb{118}& \tgf{0118} & \ipa{no} &\tinynb{2.42}& cerveau& \ipa{tɯ-rnoʁ}  & \\
&&&&&< *rnoq & \\
\tinynb{2005}& \tgf{2005} & \ipa{tɕior} &\tinynb{1.90}& boue& \ipa{tɤ-rcoʁ} & \\
&&&&&< *rcoq & \\
\midrule
\tinynb{3572}& \tgf{3572} & \ipa{ŋwo} &\tinynb{2.42}& argent& \ipa{} & dŋul\\
\tinynb{80}& \tgf{0080} & \ipa{phio} &\tinynb{2.43}& serpent& \ipa{qapri} & sbrul\\
\midrule
\tinynb{3678}& \tgf{3678} & \ipa{to} &\tinynb{2.42}& sortir& \ipa{} & ɴtʰon\\
\tinynb{2451}& \tgf{2451} & \ipa{bọ} &\tinynb{2.62}& fuir& \ipa{phɣo} < *pʰoˠ& \\
\tinynb{2462}& \tgf{2462} & \ipa{bowr} &\tinynb{1.91}& guêpe& \ipa{} & buŋ.ba\\
\tinynb{284}& \tgf{0284} & \ipa{ɕjwo} &\tinynb{1.48}& soir& \ipa{ɕɤr} & \\
\tinynb{212}& \tgf{0212} & \ipa{ɕioow} &\tinynb{1.57}& élever& \ipa{χsu} & \\
\bottomrule
\end{longtable}


On observe dans le tableau ci-dessus que les rimes *--\ipapl{aŋ} (correspondant au japhug --\ipa{o} et au pré-japhug *-aŋ) et *--\ipapl{ok} (japhug -\ipa{oʁ}, pré-japhug *-oq) du pré-tangoute  deviennent toujours des rimes du \ipa{shè} n°10 --o ou --io. Toutefois, la situation est plus complexe avec *--\ipapl{vm} (correspondance au japhug --\ipa{om}, --\ipa{ɤm} et --\ipa{ɯm}), qui correspond parfois aux rimes du \ipa{shè} n°10, parfois à celles du des rimes du \ipa{shè} n°11 (voir ci-dessous).

Devant les labiales, le --i-- des rimes 52 et 57 reflète un ancien *--r--, comme l'illustrent les exemples \tgf{1572} \ipa{phiow¹}  \ptang{phrvm}{blanc} et \ipa{phio} 2.43 \ptang{phroj}{serpent}.

\subsubsection{Tangoute --o(w) :: tibétain --Vm :: japhug --Vm} \label{subsubsec:correspondance:o:vm}
On reconstruit *--vm en pré-tangoute lorsque les rimes des du \ipa{shè} n°10 ou \ipa{shè} n°11 correspondent à --\ipapl{om}, --\ipapl{ɤm}, --\ipapl{ɯm} ou --\ipapl{um} en japhug et à --am, --om ou --um en tibétain. 

Dans cette correspondance, on pourrait être tenté de penser que la distribution entre --o et --ow reflète le vocalisme du pré-tangoute, mais ce n'est pas le cas. En effet, parmi les exemples correspondant au japhug --om (<*--am), certains comme \tgf{2915} \ipa{no¹} ``côte'' ont une rime --o, tandis que d'autres telles que \tgf{2584} \ipa{dzow¹} ``pont'' ont une rime --ow. La même remarque est valable pour les exemples correspondant à la rime --\ipapl{ɯm} du japhug, comme le montre la paire \tgf{5865} \ipa{sọ¹} ``trois'' et \tgf{3163} \ipa{phow} ``giron''.

On a déjà mentionné plus tôt que seule une série de rimes du premier cycle mineur existe en tangoute pour les deux \ipa{shè} 10 et 11 : on ne trouve pas de rime *--\ipapl{iọw} . Ainsi, une syllabe pré-tangoute *S-Cvm ne pourrait pas donner la forme *\ipapl{Cọw}; il est logique qu'elle devienne \ipapl{Cọ}. Cette simple observation permet de rendre compte de la majorité des formes du \ipa{shè} n°10 venant de *--vm : ce sont des rimes de premier cycle mineur --\ipapl{ọ} ou --\ipapl{iọ}. De la même façon, on peut expliquer que --ioor corresponde au japhug --om dans \tgf{4247} \ipa{lhjoor¹} ``empan'' du fait qu'il n'y a de rimes du troisième cycle mineur que dans le \ipa{shè} n°10 et non le \ipa{shè} n°11 (pas de rime de type *--oowr). 


Seuls deux exemples restent inexplicables : \tgf{5668} \ipa{no²} ``sentir'' et \tgf{2915} \ipa{no¹} ``côte''. Pour les deux premiers, on constate qu'il n'existe aucune syllabe de type *now en tangoute; rien ne s'oppose donc a supposer une règle spécifique pour les mots à initiale n-- : *nvm > no. 
\newline


A. Initiales labiales \label{rimes:10:1:lab}

\begin{enumerate}

\item 1572 \tgf{1572} \ipa{phiow} 1.55 \ptang{phrvm}{blanc} est cognat avec le \jpg{wɣrum} ``blanc''. Le \bir{phru²} ``blanc'' est en revanche sans relation aucune avec cette forme tangoute. Un pré-tangoute *ph(r)o ou *ph(r)u aurait dû donner *phu ou *\ipapl{phə}, le --r-- ne laissant pas de trace dans ce contexte.


\item 3163 \tgf{3163} \ipa{phow} 1.54 \ptang{phvm}{giron} peut être rapproché du \jpg{tɯ-phɯm} ``pan du vêtement'' et de \ipa{tɯ-phɯŋgɯ}  ``giron''.


\item 1805 \tgf{1805} \ipa{.wọ} 1.70 \ptang{C-S-pvm}{épais}  est comparable au \jpg{jpum} ``épais'' et au \tib{sbom-po} ``épais''. Consulter la discussion p.\pageref{ex:jpg:epais}.


\item 4053 \tgf{4053} \ipa{.wọ} 1.70 ``glace'' est potentiellement comparable au \jpg{tɤ-jpɣom} ``glace'', auquel cas une reconstruction *C-S-pvm pourrait être proposée. Toutefois, un autre cognat possible existe pour cet étymon japhug, voir section \ref{subsubsec:correspondance:a:agg}.
\end{enumerate}
B. Initiales dentales \label{rimes:10:1:dent}

\begin{enumerate}

\item 148 \tgf{0148} \ipa{tọ} 1.70 \ptang{S-tvm}{se solidifier, coaguler} est apparenté au \jpg{stɤm} ``se solidifier''.


\item 1099 \tgf{1099} \ipa{dow} 2.47 \ptang{ndvm}{ours} peut se comparer au \plb{k-d-wam¹}{0012}, au  \tib{dom} ``ours'' et au  \situ{tə-wám}. La forme tibétaine vient de *d-wam par la loi de Laufer. En tangoute, la présence d'une prénasalisée est inattendue, on aurait soit attendu a priori une forme non-préfixée *wvm > *.ow, soit une forme avec préfixe *t- *t-wvm > *tvm > *tow.


\item 2584 \tgf{2584} \ipa{dzow} 1.54 \ptang{ndzvm}{pont} est cognat du \jpg{ndzom}, du \plb{dzam¹}{0393}  et du \tib{zam} ``pont'' (voir \citet[254,257]{matisoff03} pour une liste de cognats dans d'autres langues).

On en dérive en tangoute le nom \tgz{3940}, qui désigne un élément de la selle.


\item 5865 \tgf{5865} \ipa{sọ} 1.70 \ptang{S-svm}{trois} a pour cognats le \jpg{χsɯm} ``trois'' et le \tib{gsum}; ce numéral est partagé par la totalité des langues sino-tibétaines. La présence de la préinitiale *S- en pré-tangoute est inattendue; il pourrait s'agir d'une forme rédupliquée *\ipapl{sə-svm}.\footnote{Le numéral tagalog \ipa{tatlo} ``trois'' est aussi rédupliqué.} Cette particularité n'est partagée par aucune langue ST connue.


\item 5990 \tgf{5990} \ipa{niọ} 2.63 \ptang{S-njvm}{épi}  peut se comparer avec le \jpg{kɯɕnom} ``épi'' et le \bir{ʔa-hnaṁ²} (voir \citet[254,257]{matisoff03} pour d'autres cognats).


\item 549 \tgf{0549} \ipa{niọ} 1.71 \ptang{S-njvm}{sœur} peut être rapproché du \jpg{tɤ-snom} ``sœur'' et du \plb{ʔəs-nam¹}{0205} ``sœur cadette''. Dans ces deux langues, ce terme désigne spécifiquement la sœur en relation par rapport  à son frère, ce en quoi il s'oppose à \tgf{3361} \ipa{kiẹj¹} ``sœur d'une femme'' en tangoute et à \ipa{tɤ-sqhɤj} en japhug.



\item 5668 \tgf{5668} \ipa{no} 2.42 \ptang{nvm}{sentir}  est apparenté au \jpg{mnɤm} ``avoir une odeur'',\footnote{``Sentir'' se dit en \jpg{nɤmnɤm} au tropatif. } au \plb{ʔ-nam²}{0513} et au \tib{mnam} de même sens. Ce mot est souvent cité comme un exemple de ``préfixe m-- intransitif", en raison de la paire \tib{snom, bsnams} ``sentir" et \ipa{mnam} ``avoir une odeur", où apparemment le verbe transitif ``sentir" a un préfixe s-- et le verbe intransitif un préfixe m--. Cette analyse résulte malheureusement d'une méconnaissance de la phonologie historique du tibétain: il faut restituer pour le verbe ``sentir" une proto-forme *s-mnam, avec simplification du groupe *smn-- en sn--. Le \jpg{ɕɯ-mnɤm} ``donner une odeur" (noter la différence de sens) avec le causatif \ipa{ɕɯ--} atteste de la structure syllabique ancienne. Le soi-disant ``préfixe m-- intransitif" est donc un mirage, car le m-- ici fait partie de la racine.


\item 2915 \tgf{2915} \ipa{no} 1.49 \ptang{(r-)nvm}{côte} est apparenté au \jpg{tɯ-rnom} ``côte'' et à la première syllabe du \bir{naṁrui³} de même sens.



\item 4874 \tgf{4874} ou 34 \tgf{0034} \ipa{low} 2.47 \ptang{lvm}{large} se compare au \jpg{jom} ``large''. On ne trouve pas de forme apparentée dans les autres langues sino-tibétaines, même lolo-birmanes.


\item 115 \tgf{0115} \ipa{low} 2.47 \ptang{lvm}{chaud}  est comparable au \plb{lum¹}{0516}, jingpo \ipa{lūm} ``tiède''.


\item 2290 \tgf{2290} \ipa{low} 2.47 \ptang{C-tvm ou *lvm}{rond} pourrait être rapproché du \jpg{tɯ-lɯm} ``taille, dimension", \tib{zlum-po} ``rond'' et du \bir{luṁ³}; on proposerait alors la reconstruction *lvm en pré-tangoute. Toutefois, une autre possibilité existe: la racine rgyalrong */tum/ qui apparaît en japhug dans le verbe statif \ipa{artɯm} ``rond'' ainsi que dans \ipa{tɯ-mbɤtɯm} ``rein''. Dans ce cas, il faudrait admettre une spirantisation de l'initiale (voir section  \ref{subsubsec:preinitialec}) et reconstruire *C-tvm.

Il faut également rapprocher le nom 2688  \tgf{2688} \ipa{ljow} 1.56 ``boule'' qui pourrait donc venir de *C-tjvm ou de *ljvm ainsi que 4676 \tgf{4676} \ipa{ljɨj} 2.37 ``boule'' qui viendrait d'un *C-tim ou d'un *lim en pré-tangoute.

Cette racine se retrouve vraisemblablement dans la seconde syllabe du nom du ``hérisson'' \tgf{4351}\tgf{2515} \ipa{ʑjuu¹low²}. La première syllabe est une graphie alternative pour  \tgz{4296}. Ce nom signifie donc littéralement ``boule d'épines''.


\item 4247 \tgf{4247} \ipa{lhjoor} 1.94 \ptang{r-lhjvvm}{empan} est apparenté au \jpg{tɯ-ɟom} et au \tib{ɴdom} (<*N-lom par la loi de Sun; le vocalisme est inexplicable) de même sens.
\end{enumerate}
C. Autres initiales \label{rimes:10:1:pal}

\begin{enumerate}


\item 4995 \tgf{4995} \ipa{ɕjow} 1.56 \ptang{ɕvm}{fer} est cognat du \jpg{ɕom} ``fer'' et du \bir{saṁ²} ``fer'', reconstruit \plb{xam¹}{0403}. Ce nom se retrouve dans les langues lolo-birmanes et macro-rgyalronguiques. \citet[200]{sagart99roc} propose qu'il s'agit d'un emprunt au sanskrit \ipa{śyāma-} ``noir'', car le fer est effectivement désigné en sanskrit comme le ``métal noir'' \ipa{kṛṣṇāyas-} ou \ipa{śyāmāyas-}. Ce n'est toutefois pas le terme le plus courant pour ``fer'' en sanskrit, et cette étymologie est incertaine.


\item 2262 \tgf{2262} \ipa{dʑjow} 1.56 ``voler'' est peut-être comparable au \jpg{nɯqambɯmbjom} ``voler'' et au \plb{(b)-yam¹}{0659}, \bir{pyaṁ²}. La rime --jow peut provenir d'un pré-tangoute *--vm, mais la correspondance de l'initiale est problématique : un pré-tangoute *mbjvm devrait donner normalement *\ipapl{bjow}. Toutefois, on remarque qu'il n'existe aucune syllabe telle que *\ipapl{bjow} en tangoute, et peu d'exemples de labiales devant --jow; il n'est donc pas inconcevable de postuler un changement *mbj-- > \ipa{dʑ--} /\underline{\hspace{10pt}}[\ipapl{*--vm, *--aŋ}]. Nous n'avons toutefois pas trouvé d'autres exemples qui pourraient le corroborer.


\item 1417 \tgf{1417} \ipa{rowr} 1.91 \ptang{rvm}{sec, desséché, asséché} est  apparenté au \jpg{rom} ``desséché''. Ce verbe statif s'emploie dans les deux langues pour désigner des plantes qui ont flétri:

\begin{exe}
\ex \label{ex:jpg:desseche}  \vspace{-8pt}
\gll  \ipa{si}	\ipa{ɯ-phaʁ}	\ipa{ntsi}	\ipa{pɯ-kɯ-rom}	 \ipa{nɯ}	\ipa{ɲɤ-kɤtɯ́ɣ-chɯ} \\
		arbre 3\sgposs{}-côté un \aor{}-\nmls{}:\stat{}-desséché \dem{} \med{}-rencontrer-\med{} \\
\glt  Il rencontra l'arbre qui était à moitié desséché. (La divination, 101)
\end{exe}
\begin{tabular}{llllllllll}
	\tgf{5815}&	\tgf{4250}&	\tgf{5814}&	\tgf{1417}&	\tgf{0497}&	\tgf{1139}&	\tgf{3229}&	\tgf{2590}&	\tgf{5870}&	\tgf{5113}\\
\tinynb{5815}&	\tinynb{4250}&	\tinynb{5814}&	\tinynb{1417}&	\tinynb{0497}&	\tinynb{1139}&	\tinynb{3229}&	\tinynb{2590}&	\tinynb{5870}&	\tinynb{5113}\\
\tgf{0705}& &&&&&&&&\\
\tinynb{0705}& &&&&&&&&\\
\end{tabular}
\begin{exe}
\ex \label{ex:tg:desseche}  \vspace{-8pt}
\gll   \ipa{tsjɨ¹}	\ipa{sji¹}	\ipa{phu²}	\ipa{rowr¹}	\ipa{ŋewr²}	\ipa{.jij¹}	\ipa{ŋwəə¹}	\ipa{.wjɨ²-tshjɨɨ¹}	\ipa{.wji¹}	\ipa{zjịj¹} \\
	aussi arbre arbre desséché nombreux \antierg{} formule.magique \dir{}-réciter faire[A] quand \\
\glt  Et lorsqu'il récita une formule magique pour ces arbres desséchés, (Leilin 05.23B.2)
\end{exe}


\item 39 \tgf{0039} \ipa{kowr} 2.82  ``dent'' (*r-kvm) peut se comparer au  \situ{tə-mkám} et au \jpg{tɤ-mɢom}  qui signifient tous deux ``étau'' (le voisement en japhug est inexpliqué), ainsi que le \bir{ʔaṁ²} ``molaire''.



\item 1105 \tgf{1105} \ipa{khjow} 1.56 ``donner'' est  un verbe particulièrement anormal en tangoute, puisque sa forme de thème B est 5644 \tgf{5644} \ipa{khjɨj} 1.42	(Nishida 2002) : c'est le seul verbe de toute la langue à présenter une telle alternance, et elle ne peut pas s'expliquer comme dans les autres cas par l'addition d'un suffixe *--u. Si l'on applique mécaniquement les lois phonétiques aux thèmes A et B de ce verbe, on obtiendrait respectivement *khjvm et *khim. Ces formes semblent potentiellement comparables au verbe \ipa{kho} du japhug. Ce verbe semblerait venir d'un proto-japhug *\ipapl{khaŋ}, mais il a une forme alternante régulière \ipa{khɤm} au thème 3, dont l'antiquité au sein des langues rgyalronguiques est garantie par la forme \ipa{khɐ̂m} du zbu qui elle n'alterne pas : le zbu a généralisé le thème 3 (un effet probablement de la fréquence de l'emploi de l'impératif). Il faut supposer qu'en japhug l'alternance --o/ --ɤm a été surgénéralisée dans ce mot, comme c'est le cas avec l'emprunt tibétain en \jpg{fkro}, thème 3 \ipa{fkrɤm}  ``mettre en ordre'' du thème de passé \tib{bkram}.

L'alternance \ipapl{--jvm} / \ipapl{--im} reste inexplicable.
\end{enumerate}
\subsubsection{Tangoute --o :: tibétain --ang :: japhug --o} \label{subsubsec:correspondance:o:ang}
On reconstruira *--\ipapl{aŋ} en pré-tangoute pour les étymons appartenant à cette correspondance. Le japhug et le situ ont subi ici un changement phonétique similaire à celui du tangoute, mais c'est là clairement une innovation indépendante, car les autres langues rgyalronguiques ont des contextes entièrement différents (--i en tshobdun, --\ipapl{ɐ} en zbu etc). Un changement identique a eu lieu dans le chinois du nord-ouest d'époque Tang et Song avec lequel le tangoute était en contact, et il pourrait s'agir d'une influence indirecte. On trouve relativement peu de mots dans cette correspondance, la plupart des étymons en *-aŋ dans les autres langues correspondant au tangoute --jij < *-jaŋ (voir \ref{subsubsec:correspondance:jij:o}).


\begin{enumerate}


\item 1616 \tgf{1616} \ipa{.o} 2.42 \ptang{aŋ}{entrer} est apparenté au \tib{joŋ / ɦoŋ} ``venir'' (ce verbe peut aussi signifier ``entrer'' dans certains contextes). La forme tibétaine provient d'un proto-tibétain * \ipapl{waŋ} avec l'effet de la loi de Laufer. L'initiale *w--, si elle a existé un jour en tangoute, n'est pas détectable.


\item 3443 \tgf{3443} \ipa{po} 1.49  \ptang{paŋ}{oncle paternel}  peut se comparer au \jpg{tɤ-βɣo} ``oncle paternel'' (<*kpaŋ). On attendrait a priori plutôt une forme **\ipapl{C-paŋ} qui devrait donner un tangoute *wo : l'absence de préinitiale n'est pas explicable.


\item 458 \tgf{0458} \ipa{kor} 1.89 \ptang{r-kaŋ}{gorge} est cognat avec le japhug  \ipa{tɯ-rqo} ``gorge'' (<*\ipapl{rqaŋ}).


\item 2857 \tgf{2857} \ipa{ŋo} 2.42 \ptang{ŋaŋ}{être malade} est apparenté au \jpg{ngo} ``être malade'' et \ipa{tɯ-ŋgo} ``maladie".\footnote{Il faut bien lire ici [ngo] et non *[ŋo]. Le verbe dérive ici probablement du nom par un préfixe dénominal \ipa{nɯ}-- avec perte irrégulière de la voyelle, d'où *nɯ-ŋgo > *nŋgo réalisé comme \ipa{ngo}. } La correspondance de ces deux mots entre le tangoute et le japhug est problématique, les prénasalisées et les nasales ne se correspondent pas normalement les unes avec les autres (voir p.\pageref{subsubsec:prenasaliseestg}). Toutefois, ce mot étant transcrit en tibétain comme <rŋo> en tibétain (\citealt[288]{tai08duiyin}), nous avons la garantie qu'il faut reconstruire une nasale en tangoute, et ceci malgré l'absence de \textit{fǎnqiè} pour ce caractère. Dans d'autres langues rgyalronguiques, telles que le  rtau, on trouve une forme à nasale vélaire \ipa{ŋɔ} ``être malade".

\end{enumerate}
\subsubsection{Tangoute --o :: tibétain --og :: japhug \ipapl{--oʁ, --ɤɣ}} \label{subsubsec:correspondance:o:ok}
\begin{enumerate}


\item 8 \tgf{0008} \ipa{do} 1.49 ``poison'' peut se comparer au \jpg{tɤ-ndɤɣ} ``poison'', \plb{dòk}{0391}, naish \naxi{ndv̩˩}, \nayn{dv̩˩}, \laze{ɖv̩˩}, \protona{ndu} et au \tib{dug}. Il n'est pas certain qu'il s'agisse d'un cognat; car la rime --\ipapl{ɤɣ} du japhug correspond normalement au tangoute --ew. Il existe une autre forme 3699 \tgf{3699} \ipa{do} 2.42	apparentée au ton 2.\label{analyse:poison}


\item 118 \tgf{0118} \ipa{no} 2.42 \ptang{(r)-nok}{cerveau}, voir p.\pageref{tab:sans.preinitiale.r:preinitiale.r})  est cognat avec le \jpg{tɯ-rnoʁ} ``cerveau'' et le \plb{(C)-nòk}{0140}.


\item 2797 \tgf{2797} \ipa{lho}   \ptang{lhok}{sortir} correspond au \jpg{ɬoʁ} ``sortir''.





\item 2005 \tgf{2005} \ipa{tɕior} 1.90   \ptang{r-tɕok}{boue, saleté} est comparable au \jpg{tɤ-rcoʁ} ``boue''. On trouve plusieurs formes apparentées : 2113 \tgf{2113} \ipa{tɕior} 2.81 ``saleté''., 3061 \tgf{3061} \ipa{tɕior} 2.81 ``terre (utilisée pour la construction de murs)'' et 5697 \tgf{5697} \ipa{tɕior} 1.90 ``sale''.

\end{enumerate}
\subsubsection{Autres correspondances} \label{subsubsec:correspondance:o:autre}
On trouve quelques exemples qui ne se peuvent se classer dans les trois catégories ci-dessus. 
\begin{enumerate}


\item Tout d'abord, deux exemples correspondent au tibétain --ul : 3572 \tgf{3572} \ipa{ŋwo} 2.42 ``argent'' (\tib{dŋul})\footnote{Pour une étude de ce mot dans un contexte plus large, voir \citet{antonov12kumush}.} et 80 \tgf{0080} \ipa{phio} 2.43 ``serpent'' (\tib{sbrul}, \jpg{qapri}). Il est peu vraisemblable qu'un *--l final ait été préservé en proto-macro-rgyalronguique, et nous ne pouvons pas projeter ici le tibétain --ul en pré-tangoute. Toutefois, cette rime devait être distincte de *--u et *--o, sans quoi elle n'aurait pu maintenir sa distinction. Nous reconstruirons donc provisoirement ici un pré-tangoute *--oj, le *--j chutant préservant le timbre de la voyelle comme avec *--ok.  \tgf{3572} \ipa{ŋwo²} et \tgf{0080} \ipa{phio²}  proviennent respectivement de \ptang{ŋoj}{argent} et \ptang{phroj}{serpent}. La médiane *--r-- est ici aussi reflétée comme un --i-- en tangoute devant labiale.




\item 3678 \tgf{3678} \ipa{to} 2.42 ``sortir''  pourrait se comparer avec le \tib{ɴtʰon} de même sens. Sans autres exemples de cognats potentiels à rime --on en tibétain, il est difficile de se prononcer sur la plausibilité de cette comparaison. Une autre possibilité serait le \plb{ʔ-dwak^h}{0656} ``sortir".\footnote{D.Bradley, communication personnelle.}


\item 2451 \tgf{2451} \ipa{bọ} 2.62 ``fuir'' pourrait être rapproché du \jpg{phɣo} ``fuir'' et du \tib{ɴbro, bros}.\footnote{Ce verbe est donné comme \textit{ɴbros, bros} dans le DCT, mais les formes que nous citons sont clairement attestées dans l'épopée de Gesar (Ndanyul 1) : \textit{bros-na ɴbro sa mi-ɴdug} ``Si je tente de fuir, je n'ai nul endroit où fuir''.} La seule hypothèse possible pour expliquer le vocalisme irrégulier est de prendre en compte le fait qu'on trouve une voyelle vélarisée dans ce mot en proto-japhug *\ipapl{phoˠ} qui cause l'apparition de la médiane \ipapl{--ɣ--} et qui bloque le passage régulier de *--o à --u. On reconstruira donc *\ipapl{S-mboˠ} en pré-tangoute.


\item 2462 \tgf{2462} \ipa{bowr} 1.91 ``guêpe'' est comparable au \tib{buŋ.ba} (ou \ipa{buŋ.gu}). Il est impossible de déterminer s'il s'agit d'un emprunt ou d'un cognat. C'est le seul cas clair de correspondance avec la rime --ung en tibétain, dont --owr est donc peut-être la correspondance régulière. D'autres exemples sont nécessaires avant de pouvoir tirer cette conclusion.


\item 284 \tgf{0284} \ipa{ɕjwo} 1.48 ``soir'' est peut-être apparenté au \jpg{ɕɤr} ``soir'' et au  \situ{swár} ``soir''.  Un pré-tangoute *\ipapl{ɕar} devrait donner *\ipapl{ɕja}, et *\ipapl{ɕwar} devrait devenir *\ipapl{ɕjwa} selon les lois phonétiques exposées ci-dessus. Cette forme est donc problématique; si elle est apparentée aux noms japhug et situ, il faut sans doute supposer l'existence d'un suffixe.


\item 212  \tgf{0212}   \ipa{ɕioow}  1.57 ``élever'' ressemble au \jpg{χsu} de même sens. Il faut peut-être supposer une forme du non-passé suffixée en --m *ɕioom pour expliquer le vocalisme irrégulier de la forme tangoute. Il pourrait aussi s'agir d'un hasard.

\end{enumerate}
Outre les formes ci-dessus, on trouve trois cas clairs d'emprunts tibétains. 


5437  \tgf{5437} \ipa{dzow} 1.54 ``intelligent'' provient du \tib{mdzaŋs-pa} ``intelligent, courageux, élevé''. S'il s'agissait d'un cognat (ou d'un emprunt suffisamment ancien), le tibétain --ang devrait correspondre à la rime --o, et non à la rime --ow. --ow sert également à transcrire la rime des mots chinois qui remontent au chinois moyen --ang.


5816 \tgf{5816} \ipa{ɕiow} 2.48 ``loutre'' doit être emprunté au \tib{sram}, car un pré-tangoute *srvm devrait devenir *zowr et non \ipa{ɕiow²}. Ce mot provient d'un dialecte tibétain où sr-- a dû passer à \ipapl{ʂ--} comme c'est le cas d'une grande partie des dialectes actuels de l'Amdo.


11  \tgf{0011} \ipa{lo}	``riche'' pourrait être un emprunt du \tib{lhug-po} ``riche''. On observe ici la même correspondance --o :: --ug entre tangoute et tibétain que pour le nom ``poison'' p.\pageref{analyse:poison}.





\subsection{Conclusion } \label{subsec:voyelle.conclusion}
Dans ce travail, nous avons reconstruit pour le pré-tangoute sept voyelles distinctes, *a *e *i *o *u plus deux voyelles vélarisées *\ipapl{aˠ}	 et *\ipapl{oˠ}	à la distribution très limitée. On peut éventuellement postuler *ə pour rendre compte de  mots tels que \tgz{1338} ``aimer" et \tgz{4681} ``oreille". Ces voyelles peuvent être accompagnées d'une médiane *--w--, *--r-- ou *--j--. La médiane *--j-- influe considérablement sur l'évolution des voyelles, et nous avons donc rajouté une série de voyelles avec cette médiane dans le tableau \ref{tab:rimes}.

On a reconstruit huit consonnes finales distinctes: les occlusives *\ipapl{--p},	*\ipapl{--t} et *\ipapl{--k} plus *\ipapl{--r},  *\ipapl{--j}	et deux nasales : *\ipapl{--m} et *\ipapl{--ŋ}. On remarque l'absence de la nasale *\ipapl{--n} et la distribution très limitée de *\ipapl{--ŋ} : les nasales autres que *\ipapl{--m} chutent sans laisser de traces dans la plupart des rimes, se confondant avec les voyelles simples ou les rimes à *\ipapl{--j}. Le *\ipapl{--m} a chuté lui aussi, mais il a dû disparaître plus tard que les autres nasales, car les correspondances des rimes en *\ipapl{--m} sont très différentes des voyelles en syllabe ouverte.

Une distinction entre *--s et *--t a dû avoir existé à une époque plus ancienne, mais elle n'est pas détectable à partir des données internes du tangoute. Les finales *--p et *--t sont elles-mêmes distinctes uniquement après les voyelles arrondies *o et *u, et ne sont pas autrement distinguables à partir des données internes.

On observe un nombre considérable de confusions dans les syllabes fermées; en particulier, *e et *i ne sont distinctes qu'en syllabe ouverte, avant  *\ipapl{--m} et avant  *\ipapl{--j}.

\begin{table}
\captionb{Tableau récapitulatif des rimes du pré-tangoute}\label{tab:rimes}
\begin{tabular}{l|lllllllll} \toprule
\ipapl{}	&	\ipapl{0}	&	\ipapl{*--p}	&	\ipapl{*--t}	&	\ipapl{*--k}	&	\ipapl{*--r}		&	\ipapl{*--j}	&	\ipapl{*--m}	&	\ipapl{*--ŋ}	\\
\midrule
\ipapl{*a}	&	\ipapl{e}	&	\ipapl{a}	&	\ipapl{a}	&	\ipapl{a}	&	\ipapl{a}	&	\ipapl{}	&	\ipapl{ow}	&	\ipapl{o}	\\
\ipapl{*o}	&	\ipapl{u}	&	\ipapl{ew}	&	\ipapl{ə}	&	\ipapl{o}	&	\ipapl{ər}	&	\ipapl{o}	&	\ipapl{ow}	&	\ipapl{}	\\
\ipapl{*u}	&	\ipapl{wə}	&	\ipapl{ə}	&	\ipapl{wə}	&	\ipapl{ew}	&	\ipapl{wər}		&	\ipapl{}	&	\ipapl{ow}	&	\ipapl{}	\\
\ipapl{*e}	&	\ipapl{e}	&	\ipapl{}	&	\ipapl{}	&	\ipapl{}	&	\ipapl{}	&		\ipapl{ej}	&	\ipapl{ow}	&	\ipapl{}	\\
\ipapl{*i}	&	\ipapl{ə}	&	\ipapl{ə}	&	\ipapl{ə}	&	\ipapl{ew}	&	\ipapl{ər}		&	\ipapl{əj}	&	\ipapl{jɨj}	&	\ipapl{}	\\
\ipapl{*ə}	 &\ipapl{u} \\
\midrule
\ipapl{*aˠ}	&	\ipapl{}	&	\ipapl{}	&	\ipapl{}	&	\ipapl{}	&	\ipapl{}	&	\ipapl{}	&	\ipapl{a}	&	\ipapl{a}	\\
\ipapl{*oˠ}	&	\ipapl{o}	&	\ipapl{}	&	\ipapl{}	&	\ipapl{}	&	\ipapl{}	&		\ipapl{}	&	\ipapl{}	&	\ipapl{}	\\
\midrule
\ipapl{*ja}	&	\ipapl{ji}	&	\ipapl{ja}	&	\ipapl{ja}	&	\ipapl{ja}	&	\ipapl{ja}		&	\ipapl{}	&	\ipapl{jow}	&	\ipapl{jij}	\\
\ipapl{*jo}	&	\ipapl{ju}	&	\ipapl{?}	&	\ipapl{jɨ}	&	\ipapl{jo}	&	\ipapl{jɨr}		&	\ipapl{}	&	\ipapl{jow}	&	\ipapl{}	\\
\ipapl{*ju}	&	\ipapl{jwɨ}	&	\ipapl{jɨ}	&	\ipapl{jwɨ}	&	\ipapl{jiw}	&	\ipapl{jwɨr}		&	\ipapl{}	&	\ipapl{jow}	&	\ipapl{}	\\
\ipapl{*je}	&	\ipapl{ji}	&	\ipapl{}	&	\ipapl{}	&	\ipapl{}	&	\ipapl{}	&	\ipapl{jij}	&	\ipapl{jow}	&	\ipapl{}	\\
\ipapl{*ji}	&	\ipapl{jɨ}	&	\ipapl{jɨ}	&	\ipapl{jɨ}	&	\ipapl{jiw}	&	\ipapl{jɨr}		&		&	\ipapl{}	&	\ipapl{}	\\
\ipapl{*ə}	 &\ipapl{ju} \\

\bottomrule
\end{tabular}
\end{table}



Aux correspondances de ce tableau, il faut rajouter celles qui sont liées à l'ajout de suffixes, qui seront exposées dans le chapitre suivant. La plupart des rimes du tangoute ont plusieurs origines possibles en pré-tangoute. Seuls --u, --ej et \ipapl{-jɨj} n'ont chacun qu'une origine claire : *--o, *--ej et *--im respectivement.

Les rimes à *--\ipapl{ŋ} et *--\ipapl{n} finaux sont difficiles à distinguer des autres. Les langues rgyalrong ne préservent pas de finales nasales dans ces rimes, et sont donc d'un intérêt médiocre pour aider à l'analyse des faits tangoutes. Ce sont donc surtout le tibétain et et le pumi qui peuvent nous guider. Avec les voyelles antérieures, on trouve deux rimes qui correspondent à --ing ou --in en tibétain (ou --\ipapl{ĩ} en pumi) : --ji et --jiij.\footnote{On doit mentionner l'exemple isolé \tgf{0800} \ipa{dzee¹} ``combattre'' qui pourrait être soit un cognat, soit un emprunt au \tib{ɴdziŋ} ``combattre''.   } Le tableau \ref{tab:ing} contient une liste de nos exemples.



\begin{table}
\captionb{Reconstruction des nasales finales après voyelles antérieures en pré-tangoute}\label{tab:ing}
\resizebox{\columnwidth}{!}{
\begin{tabular}{llllllll} \toprule
\multicolumn{4}{c}{tangoute} & sens &pré-tangoute &  tibétain  & pumi\\
\midrule
\tinynb{251	}&	\tgf{0251}	&	\ipa{bji}	&	\tinynb{2.10}	&	corde	&	*mbje	&	ɴbreŋ	&	\ipa{brĩ̆}	\\
\tinynb{5509	}&	\tgf{5509}	&	\ipa{bjị}	&	\tinynb{1.67}	&	urine	&	*S-mbje	&		&	\ipa{bĩ̌}	\\
\tinynb{4658	}&	\tgf{4658}	&	\ipa{thji}	&	\tinynb{1.11}	&	boire	&	*thje	&		&	\ipa{thĩ̌}	\\
\tinynb{4250	}&	\tgf{4250}	&	\ipa{sji}	&	\tinynb{1.11}	&	bois, arbre	&	*sje	&	ɕiŋ	&	\ipa{sĩ̌}	\\
\tinynb{2858	}&	\tgf{2858}	&	\ipa{zjir}	&	\tinynb{2.72}	&	long	&	*s-rje	&	riŋ-po	&	\ipa{ʂɛ̃́}	\\
\tinynb{5273	}&	\tgf{5273}	&	\ipa{sji}	&	\tinynb{2.10}	&	foie	&	*sje	&	mtɕʰin.pa	&	\ipa{tsyĩ̂}	\\
\tinynb{2639	}&	\tgf{2639}	&	\ipa{mjiij}	&	\tinynb{2.35}	&	nom	&	*mjeej	&	miŋ	&	\ipa{mɛ̃́}	\\
\tinynb{2518	}&	\tgf{2518}	&	\ipa{njiij}	&	\tinynb{1.39}	&	coeur	&	*njeej	&	sɲiŋ	&		\\
\bottomrule
\end{tabular}}
\end{table}
Comme les rimes à *--\ipapl{ŋ} et *--\ipapl{n} finaux ne sont pas distinguables des rimes à voyelles antérieures en syllabe ouverte, on ne tentera pas de proposer ici une reconstruction distincte en pré-tangoute. Toutefois, à un stade du pré-pré-tangoute on peut supposer les changements suivant pour rendre compte des formes du tableau \ref{tab:ing}:
\newline

*--\ipapl{jen}, *--\ipapl{jeŋ} > *--\ipapl{je} > --\ipapl{ji} 
\newline
*--\ipapl{jeen}, *--\ipapl{jeeŋ} > *--\ipapl{jeej} > --\ipapl{jiij} 
\newline


En ce qui concerne la rime *--\ipapl{uN}, du pré-pré-tangoute, les exemples sont trop peu nombreux pour être probants, et on attendra d'en rassembler davantage.


Notre pré-tangoute est un système intermédiaire entre le proto-macro-rgyalronguique et le tangoute attesté, et ne représente pas à proprement parler la reconstruction d'un état synchronique: c'est un modèle qui permet de synthétiser un certain nombre de changements phonétiques non-triviaux, et servira de base pour reconstruire par la suite le proto-macro-rgyalronguique et découvrir de nouveaux cognats entre le tangoute et les autres langues. Sur la base seule du tangoute, la quasi-totalité des rimes a plusieurs origines distinctes, particulièrement --jɨ et -ə, qui peuvent venir soit de voyelles fermées *-u *-i soit de syllabes fermées en occlusive *-t et *-p. 

D'un point de vue méthodologique, la reconstruction adoptée dans ce travail peut paraître non-orthodoxe, car il faudrait en toute rigueur reconstruire directement le proto-macro-rgyalronguique sur la base de toutes les langues, puis étudier le développement du proto-macro-rgyalronguique en tangoute. Toutefois, cette tâche paraît encore impossible, tant les correspondences phonétiques entre langues sont complexes. Même au sein des langues rgyalronguiques, les correspondances sont extrêmement irrégulières aussi bien pour les voyelles que pour les groupes initiaux: on observe de nombreuses correspondances attestées par un seul étymon. 

Il est donc contre-productif de vouloir respecter une orthodoxie méthologique illusoire: la seule priorité à l'heure actuelle est d'établir le plus de correspondances phonétiques possible et de tenter de leur donner une interprétation phonétique. Le système rigoureux de reconstruction complet viendra progressivement au fur et à mesure que les langues seront mieux décrites et que notre liste de cognats aura grandi. Comme la détection des cognats n'est pas séparable de l'établissement des lois phonétiques, une semi-reconstruction comme notre pré-tangoute nous semble une étape indispensable pour toute étude comparative des langues macro-rgyalronguiques.


\chapter{Morphologie comparée}
\thispagestyle{empty}
La nature de l'écriture tangoute, ainsi que la nature des documents tangoutes eux-mêmes, traduits ou adaptés pour l'essentiel du chinois, induisent chez le lecteur de ces textes une tentation difficile à éviter : celle de lire ces textes comme s'il s'agissait de chinois avec un ordre des mots légèrement différent, autrement dit, traiter le tangoute comme une langue isolante. 


Toutefois, l'étude de la morphologie du tangoute entreprise par K.B. Kepping sur la morphosyntaxe et par Gong Hwangcherng sur les alternances morphophonologiques a permis de montrer que le tangoute était considérablement plus proche de langues macro-rgyalronguiques telles que le pumi ou le muya que du chinois du point de vue de son type morphologique. Le verbe tangoute avait un accord personnel, des alternances vocaliques complexes constituées en classes de conjugaisons et un marquage sophistiqué du temps-aspect-mode basé sur l'usage de préfixes directionnels qui en font une langue radicalement distincte du chinois du point de vue typologique. Cette constatation est importante non seulement pour les linguistes, mais pour tous les spécialistes qui se consacrent à la lecture de textes tangoutes, qu'ils soient historiens, littéraires ou anthropologues, car une compréhension correcte de la grammaire du tangoute permet dans certains cas de parvenir à une traduction plus sûre de phrases autrement ambiguës.


Ce travail s'inscrit dans la continuité des études de Kepping et de Gong. Sans négliger l'aspect philologique, nous prendrons en compte les données des langues macro-rgyalronguiques modernes, en particulier le japhug et le pumi pour lesquels nous disposons de données de première main.


Ce chapitre comporte cinq parties. Premièrement, nous traiterons  brièvement de la morphologie nominale, car peu de formes sont directement comparables entre le tangoute et le japhug en dehors du domaine verbal: nous nous  intéressons principalement à certaines marques de cas et aux préfixes numéraux des classificateurs.

Deuxièmement, nous aborderons  la morphologie flexionnelle, en restreignant notre attention sur les formes présentant des cognats dans d'autres langues:  le marquage de la personne, les préfixes directionnels, les suffixes de TAM et les négations. 

Troisièmement, nous étudierons la morphologie dérivationnelle, en particulier  la morphologie fossile non-productive apparentée avec celle   des langues rgyalronguiques. 

Quatrièmement, nous décrirons les procédés de réduplication  et présenterons des comparaisons potentielles avec les langues rgyalrongs. 

Cinquièmement, nous comparerons la structure du complexe verbal du tangoute avec celui du japhug, afin de mettre en évidence les ressemblances et les différences aussi bien historiques que typologiques.


\section{Morphologie nominale}
La morphologie nominale du tangoute est considérablement moins élaborée que la morphologie verbale, et nous n'aborderons dans cette sections que les deux plus importantes catégories nominales: le marquage casuel et les préfixes numéraux des classificateurs.


\subsection{Marquage casuel}  \label{sec:cas}

On ne trouve qu'une seule marque casuelle apparentée entre le tangoute et les langues rgyalrongs: il s'agit du génitif / anti-ergatif  \tgz{1139}. Ce clitique marque à la fois la relation de possession entre deux groupes nominaux ou entre un pronom et un groupe nominal (\ref{ex:antierg.poss}). Il peut aussi marquer un récipientaire ou même un object direct (\ref{ex:antierg.objet}) . Il n'est obligatoire dans aucune de ces fonctions.
\newline
\linebreak
\begin{tabular} {llll}
		\tgf{0261}&	\tgf{1139}&	\tgf{1909}&	\tgf{1943} \\
		261&	1139&	1909&	1943 \\
\end{tabular}
\begin{exe}
\ex \label{ex:antierg.poss}  \vspace{-8pt}
\gll 		\ipa{mjo²}	\ipa{.jij¹}	\ipa{gur¹}	\ipa{nja²} \\
			moi	\antierg{}	bœuf	ne.pas.être \\
\glt Ce n'est pas mon bœuf (Leilin 04.14B.5)
\end{exe}

\begin{tabular}{llllllllll}
	\tgf{4689}&	\tgf{1531}&	\tgf{0866}&	\tgf{1139}&	\tgf{2393}&	\tgf{3456}&	\tgf{0705}&	\tgf{2219}&	\tgf{0046}&	\tgf{2098}\\
\tinynb{4689}&	\tinynb{1531}&	\tinynb{0866}&	\tinynb{1139}&	\tinynb{2393}&	\tinynb{3456}&	\tinynb{0705}&	\tinynb{2219}&	\tinynb{0046}&	\tinynb{2098}\\
\end{tabular}
\begin{exe}
\ex \label{ex:antierg.objet}  \vspace{-8pt}
\gll   \ipa{.jwar¹}	\ipa{gja¹}	\ipa{ɣu¹}	\ipa{.jij¹}	\ipa{ljiij²}	\ipa{lja¹}	\ipa{zjịj¹}	\ipa{kjij¹-ljij²-ŋa²} \\
		Yue armée Wu \antierg{} détruire venir quand \opt{}-voir[A]-1\sg{} \\
\glt Quand (les soldats) de l'armée de Yue viendront détruire Wu, ils me verront (Leilin 03.21B.4-5)
\end{exe}

Par ailleurs, il intervient dans la construction de prédication possessive, qui présente des similarités entre le situ et le tangoute. En situ, le possesseur est marqué avec le cas locatif –\textit{i} (\citealt[257]{linxr93jiarong})\footnote{Le japhug ne peut pas être cité ici, car la construction s'effectue avec le génitif \ipa{ɣɯ}, le locatif en –\textit{i} ayant disparu.}:
\begin{exe}
\ex \label{ex:st:avoir}  \vspace{-8pt}
\gll 	\ipa{ŋə-i}	\ipa{tapu}	\ipa{kəmŋo}	\ipa{ndo}\\
		je-\loc{}	enfant	cinq	avoir \\
\glt J'ai cinq enfants.
\end{exe}


En tangoute, le possesseur est marqué au moyen du génitif/accusatif \tgf{1139} \ipa{.jij¹}, un marqueur probablement apparenté au locatif –i du situ : 
\newline
\linebreak
\begin{tabular}{llllll}
		\tgf{3513}&	\tgf{1139}&	\tgf{2750}&	\tgf{5981}&	\tgf{1374}&	\tgf{0930}\\
		3513&	1139&	2750&	5981&	1374&	930\\
\end{tabular}
\begin{exe}
\ex \label{ex:tg:avoir}  \vspace{-8pt}
\gll 	\ipa{mə¹}	\ipa{.jij¹}	\ipa{ɣu¹}	\ipa{.a-tɕhjɨ¹-dju¹} \\
		ciel	\antierg{}	tête	\intrg{}-\pot{}-avoir[A] \\
\glt Le ciel a-t-il une tête? (Leilin 05.14A.1)
\end{exe}

La similarité frappante entre les deux constructions suggère fortement qu'il s'agit là d'une préservation, et non d'une similarité fortuite ou d'une évolution parallèle. C'est un des rares cas où l'on peut proposer la reconstruction d'une structure syntaxique en proto-macro-rgyalronguique. 



Les autres marques de cas du tangoute n'ont pas d'équivalents clairs en rgyalrong. Pour certaines d'entre elles, il est même possible de déterminer leur étymologie. Ainsi, le clitique d'ergatif  	\tgf{5604}\tgf{5113} \ipa{dʑjɨ.wji¹} est un composé de \tgz{5604} ``action" et de \tgz{5113} ``faire", probablement à une forme converbale non-marquée. Cette grammaticalisation inhabituelle est d'un grand intérêt pour la théorie de la grammaticalisation, car si il est bien connu que les marques d'ergatif et d'agentif peuvent provenir de cas locatifs (voir en particulier \citealt{agent02palancar}), le développement de tels marqueurs à partir du verbe ``faire" n'a pas été   documentée en détail.


On peut trouver des pronoms de première ou seconde personne avec la marque de l'ergatif, mais le verbe \tgf{5113} \ipa{wji¹} ``faire" ne présente jamais d'alternance vocalique:
\newline
\linebreak
\begin{tabular}{llllllllllll}
\tgf{2583} &	\tgf{3926} &	\tgf{5604} &	\tgf{5113} &	\tgf{0804} &	\tgf{1770} &	\tgf{5113} &	\tgf{5880} &	\tgf{2590} &	\tgf{4517} \\	
\tinynb{2583} &	\tinynb{3926} &	\tinynb{5604} &	\tinynb{5113} &	\tinynb{0804} &	\tinynb{1770} &	\tinynb{5113} &	\tinynb{5880} &	\tinynb{2590} &	\tinynb{4517} \\	
\tgf{1278} &	\tgf{2098} &	\\
\tinynb{1278} &	\tinynb{2098} &	\\
\end{tabular}
\begin{exe}
\ex \label{ex:erg2sg}  \vspace{-8pt}
\gll \ipa{nji}  	\ipa{nja²}  	\ipa{dʑjɨ.wji¹}  	\ipa{djɨ²-lhjwi¹}  	\ipa{.wji¹}  	\ipa{ŋwu²}  	\ipa{.wjɨ²-dzji¹} \ipa{.jɨ²-ŋa²}  	\\
 perle toi \erg{} \dir{}-prendre faire[A] et \dir{}-manger[A] dire-1\sg{} \\
\glt Je dirai que tu as pris la perle et l'as mangée. (Leilin 04.02A.1)
\end{exe}
Toutefois, il convient de noter qu'outre le verbe ``faire" inclus dans la marque d'ergatif, les verbes ``prendre" et ``manger" sont ici aussi à une forme de troisième personne, car il s'agit d'une forme de discours indirect hybride. Il est donc difficile de conclure d'un tel exemple que la forme d'ergatif ne pourrait jamais apparaître à une forme de 1\sg{} ou de 2\sg{} telle que  	*\tgf{5604}\tgf{3621}	 \ipa{dʑjɨ.wjo¹}. 

\subsection{Préfixes numéraux} \label{subsec:num} 
Les langues macro-rgyalronguiques actuelles, et en particulier le japhug, présentent toutes une ou plusieurs séries de préfixes numéraux qui s'adjoignent aux classificateurs. Le tableau \ref{tab:num} contient  le paradigme des numéraux de 1 à 10 en japhug (\citealt[191]{jacques08zh}). \index{Japhug!tɯ-xpa}
\begin{table} 
\captionb{Numéraux et préfixes numéraux en japhug.}\label{tab:num} \centering
\begin{tabular}{llll}
\toprule
forme libre & forme préfixée\\
\midrule
\ipa{ci}, \ipa{tɤɣ} & \ipa{tɯ-xpa} \\
\ipa{ʁnɯz} & \ipa{ʁnɯ-xpa} \\
\ipa{χsɯm} & \ipa{χsɯ-xpa} \\
\ipa{kɯβde} & \ipa{kɯβde-xpa} \\
\ipa{kɯmŋu} & \ipa{kɯmŋu-xpa} \\
\ipa{kɯtʂɤɣ} & \ipa{kɯtʂɤ-xpa} \\
\ipa{kɯɕnɯz} & \ipa{kɯɕnɯ-xpa} \\
\ipa{kɯrcat} & \ipa{kɯrcɤ-xpa} \\
\ipa{kɯngɯt} & \ipa{kɯngɯ-xpa} \\
\ipa{sqi} & \ipa{sqɯ-xpa} \\
\bottomrule
\end{tabular}
\end{table}
On remarque que tous les préfixes numéraux dérivent des numéraux libres par plusieurs procédés phonologiques: chute de la consonne finale et dans certains cas neutralisation de la voyelle en \ipa{ɯ} ou en \ipa{ɤ}. Les numéraux ``quatre" et ``cinq" ne subissent aucune modification entre la forme libre et la forme conjointe, ce qui trahit un développement récent de ce paradigme.

En tangoute,   l'écriture n'indique aucune différence entre numéraux libres et préfixes numéraux, à l'exception de ``un", dont la forme libre est  \tgz{0100}	 et la forme préfixée \tgz{5981}. Cette supplétion n'existe pas en japhug, mais s'observe dans une langue rgyalronguique, le rtau, où l'on trouve la forme libre \ipa{ru} ``un" (<*rik, cognat de \tgz{0100}) et la forme préfixée \ipa{e}--.\footnote{Ces données ont été collectées auprès de Lobsang Nima en décembre 2012 avec la collaboration d'Anton Antonov.} Il ne peut s'agir que d'un héritage du proto-macro-rgyalronguique. Le japhug a donc dû avoir refait son système de préfixes numéraux.


Une autre idiosyncrasie s'observe entre le tangoute et les autres langues birmo-qianguiques: la supplétion du nom de l'année. 
On observe en effet la forme \tgz{3305} avec les préfixes numéraux (comme \tgf{5981}\tgf{3305} \ipa{.a-kjiw¹} ``une année"  Leilin 04.33A.5, correspondant au rtau \ipa{e-fku}), mais \tgz{2712} dans les autres contextes.
 
  Dans les autres langues macro-rgyalronguiques et en naish, on observe également des formes de supplétisme similaire (tableau \ref{tab:suppletisme:annee}).\footnote{Les données muya sont tirées de Huang (1992), le proto-naish vient de \citet{michaud11cl}. }

\begin{table}
\captionb{Supplétisme dans les formes du nom ``année" en tangoute.}\label{tab:suppletisme:annee}
\resizebox{\columnwidth}{!}{
\begin{tabular}{llllll} \toprule
sens&	tangoute&	japhug&	pumi (Shuiluo)&	muya & proto-naish\\
l'année dernière&	\tgf{5168}\tgf{2712} \ipa{.jɨ².wji¹}&	\ipa{ja\textbf{pa}}&	\ipa{ʑɛ́\textbf{pə}}&	\ipa{jø³³zɑ²⁴}\ & *\textbf{C-ba}\\
cette année&	\tgf{0748}\tgf{2712} \ipa{pjɨ¹.wji¹}&	\ipa{ɣɯj\textbf{pa}}&	\ipa{pə\textbf{pə́}}&	\ipa{pə³³\textbf{βə⁵³}}& *\textbf{C-ba} \\
l'année prochaine&	\tgf{5500}\tgf{2712} \ipa{sjij¹.wji¹}&	\ipa{fsaqhe}&	\ipa{ʑɛkhiú}&	\ipa{sæ³³\textbf{βə⁵³}} &*\textbf{C-ba} \\
un an&	\tgf{5981}\tgf{3305} \ipa{.a-kjiw¹}&	\ipa{tɯ-xpa}&	\ipa{tɜ́-\textbf{kó}}&	\ipa{tɐ⁵⁵-\textbf{kui⁵³}} &*\textbf{kʰu}\\
deux ans&	\tgf{4027}\tgf{3305} \ipa{ njɨɨ¹-kjiw¹}&	\ipa{ʁnɯ-xpa}&	\ipa{ɲí-\textbf{kó}}&	&*\textbf{kʰu}\\
\bottomrule
\end{tabular}}
\end{table}

Ce supplétisme entre deux racines pour ``année'' ne se retrouve pas en japhug,\footnote{\jpg{fsaqhe} vient de *psaŋ-qʰo-j, un composé ancien formé de ``demain'', ``après'' suivi du suffixe locatif. \textit{-qhe} est donc sans lien ici avec \tgz{3305}.}  mais a des équivalents dans d'autres langues macro-rgyalronguiques, en particulier le pumi et le muya, où la racine apparentée à \tgf{2712} \ipa{.wji¹} se retrouve aussi dans les expressions ``l'année prochaine, l'année dernière, cette année'', et celle apparentée à \tgf{3305} \textit{kjiw¹} apparaît avec les préfixes numéraux ainsi qu'à l'état libre. Il est également important de noter que les formes préfixées  5168 \tgf{5168} \ipa{.jɨ} 2.28 ``(l'année) dernière'', 748 \tgf{0748} \ipa{pjɨ} 1.30 ``cette (année)'' et 5500 \tgf{5500} \ipa{sjij} 1.36 ``(l'année) prochaine'' sont elle aussi cognats des premières syllabes des autres langues (nous avons indiqué les formes cognats en gras), en particulier de celles que l'on trouve en  muya. C'est là un des rares cas où des dissyllabes sont potentiellement reconstructibles en proto-macro-rgyalronguique, voire  à un nœud plus élevé du Stammbaum.
 \index{Japhug!tɯ-xpa}



Aucune langue macro-rgyalronguique actuelle ne présente toutefois des formes comparables à celles du tangoute (voir le tableau \ref{tab:suppletisme:jour}).
\begin{table}
\captionb{Formes du nom ``jour" en tangoute.}\label{tab:suppletisme:jour}
\resizebox{\columnwidth}{!}{
\begin{tabular}{lllll} \toprule
sens&	tangoute&	japhug&	pumi (Shuiluo)&	muya\\
hier&	\tgf{5168}\tgf{2440} \ipa{.jɨ²njɨɨ²}&	\ipa{jɯfɕɯr}&	\ipa{ʑɜ́ṇə́}&	\ipa{pə³³sə⁵³}\\
aujourd'hui&	\tgf{0748}\tgf{2440} \ipa{pjɨ¹njɨɨ²}&	\ipa{jɯsŋi}&	\ipa{pə́ṇə́}&	\ipa{ji³³sə⁵³}\\
demain&	\tgf{3696}\tgf{5124} \ipa{na¹rar²}&	\ipa{fso}&	\ipa{ɕĩbó}&	\ipa{sæ²⁴sə³³}\\
un jour&	\tgf{5300}\tgf{2440} \ipa{tjɨ¹njɨɨ²}&	\ipa{tɯ-sŋi}&	\ipa{tɜ́ṇə́}&	\ipa{tɐ⁵⁵-si⁵³}\\
\bottomrule
\end{tabular}}
\end{table}
La forme ``un jour'' est particulièrement remarquable, c'est le seul classificateur qui a ce préfixe numéral \tgf{5300} \ipa{tjɨ¹--}, les autres apparaissant avec le préfixe \tgf{5981} \ipa{.a--}:
\newline
\linebreak
\begin{tabular}{lllllll}
	\tgf{5300}&\tgf{2440} &\tgf{0966} &\tgf{1604} &\tgf{5553} &\tgf{0481} &\tgf{4517} \\
	\tinynb{5300}&\tinynb{2440} &\tinynb{0966} &\tinynb{1604} &\tinynb{5553} &\tinynb{0481} &\tinynb{4517} \\
\end{tabular}
\begin{exe}
\ex \label{ex:tg:un.jour}  \vspace{-8pt}
\gll   \ipa{tjɨ¹-njɨɨ²} \ipa{khjɨ²} \ipa{dzjɨj¹} \ipa{gji¹} \ipa{.jir¹} \ipa{dzji¹} \\
		un-jour dix.mille argent un fortune manger[A] \\
\glt  En un jour, il mangeait pour dix milles pièces d'argent. (Leilin 08.08A.7)
\end{exe}
Le préfixe \tgf{5300} \ipa{tjɨ¹--} est apparenté à celui du japhug \ipapl{tɯ--} et du pumi \ipapl{tɜ--}. Comme dans ces langues, il a une voyelle réduite.



\section{Morphologie verbale flexionnelle} \label{sec:morpho.verbale.flex}
Dans toutes les langues sino-tibétaines non-isolantes, la verbe est la partie du discours qui présente la morphologie la plus complexe, aussi bien par le nombre d'affixes et de procédés morphologiques que par le nombre d'irrégularités. 

Comparé à des langues telles que le japhug, le verbe tangoute n'est pas particulièrement complexe en termes de nombre d'affixes. Toutefois, sa morphologie est synchroniquement d'une opacité redoutable, qui ne peut être comprise qu'avec l'aide de la phonologie historique.

La graphie tangoute, dans laquelle chaque syllabe est marquée par un caractère, empêche de concevoir aisément les divisions en mots. Toutefois, nous allons   montrer que les verbes tangoutes conjugués étaient généralement polysyllabiques (comme nous l'indiquons dans nos glosés), et que les morphèmes grammaticaux tels que les marques de direction ou de personne doivent être considérés comme des affixes et non comme des clitiques. Cette question est loin d'être triviale, car elle nous donne une idée de la datation de la grammaticalisation des morphèmes grammaticaux dans les langues macro-rgyalronguiques. 

\subsection{Marquage de la personne} \label{subsec:personne}

Le marquage de la personne en tangoute est resté ignoré jusqu'aux travaux de Kepping, publiés dans une série d'articles (\citealt{kepping75agreement, kepping75soglasovanie, kepping77soglasovanie,kepping79,kepping81agreement,kepping83agreement, kepping94conjugation}) où elle met en évidence l'existence de trois suffixes : \tgf{2098} \ipa{ŋa²} (première personne du singulier), \tgf{4601} \ipa{nja²} (deuxième personne du singulier) et \tgf{4884} \ipa{nji²} (première ou seconde personne du pluriel). Sur la base des travaux de Kepping, \citet{delancey81direction} et \citet{driem91tangut} ont proposé des comparaisons avec d'autres langues sino-tibétaines à système d'accord.


Ces suffixes sont homonymes avec certains pronoms libres du tangoute, comme on peut l'observer dans le tableau \ref{tab:pronoms.suffixes}. Cette similarité a conduit certains auteurs (\citealt{lapolla92}) à supposer une origine tardive pour le système d'accord du tangoute.

\begin{table}
\captionb{Pronoms et suffixes personnels en tangoute}\label{tab:pronoms.suffixes}
\begin{tabular}{llllll} 
\multicolumn{3}{c}{pronom} &\multicolumn{3}{c}{suffixe} \\
\tgf{2098} & \ipa{ŋa²}  & 1\sg{} & \tgf{2098} & \ipa{ŋa²}  &1\sg{} \\
\tgf{3926} & \ipa{nja²} & 2\sg{} & \tgf{4601} & \ipa{nja²} &2\sg{} \\
\tgf{4028} &  \ipa{nji²} & 2\sg{}  honorifique ou 2\pl{} & \tgf{4884} & \ipa{nji²} & 1\pl{} et 2\pl{} \\
\end{tabular}
\end{table}

On observe dans le tableau ci-dessus que les suffixes du tangoute sont effectivement phonétiquement identiques au pronoms correspondants (du moins du point de vue de la lecture des \ipa{fǎnqiè}). C'est un argument non négligeable en faveur d'une grammaticalisation récente. Toutefois, il est également manifeste que le suffixe \tgf{4884} \ipa{nji²}  ne correspond pas parfaitement au pronom \tgf{4028} \ipa{nji²}, ce qui montre que la grammaticalisation de ces marques de personne a eu lieu au moins avant la création de l'honorifique, qui est clairement secondaire. L'exemple suivant montre bien que \tgf{4028}  \ipa{nji²}  apparaît avec un verbe marqué par le suffixe du 2\sg{} \tgf{4601} \ipa{nja²} :
\newline
\linebreak
\begin{tabular}{llllllllll}
	\tgf{4028}&	\tgf{3986}&	\tgf{4893}&	\tgf{1139}&	\tgf{1526}&	\tgf{5880}&	\tgf{0524}&	\tgf{2590}&	\tgf{5591}&	\tgf{4601}\\
\tinynb{4028}&	\tinynb{3986}&	\tinynb{4893}&	\tinynb{1139}&	\tinynb{1526}&	\tinynb{5880}&	\tinynb{0524}&	\tinynb{2590}&	\tinynb{5591}&	\tinynb{4601}\\
\tgf{3916}& &&&&&&&&\\
\tinynb{3916}& &&&&&&&&\\
\end{tabular}
\begin{exe}
\ex \label{ex:tg:toi}  \vspace{-8pt}
\gll   \ipa{nji²}	\ipa{njɨ¹.wjɨ¹}	\ipa{.jij¹}	\ipa{tshji²}	\ipa{ŋwu²}	\ipa{dzju¹}	\ipa{.wjɨ²-lhjị²-nja²-sji²} \\
		toi belle-mère \antierg{} servir \conj{} ordre \dir{}-recevoir[B]-2\sg{}-\perf{} \\
\glt C'est toi qui as servi (notre) belle-mère et as reçu ses instructions. (Cixiaozhuan, 33.4)
\end{exe}
Ce type d'exemples n'est pas exceptionnel, c'est même le cas généralement observé pour le pronom \tgf{4028}  \ipa{nji²}. C'est là la preuve que la grammaticalisation des suffixes de personnes n'était pas en cours synchroniquement en tangoute (contrairement à l'opinion de \citealt{lapolla92}), mais ne permet pas d'affirmer avec certitude que celle-ci n'était pas récente malgré tout en tangoute, avant la création de l'écriture. On doit prendre en compte aussi l'aspect phonologique du problème. 

Les pronoms \tgf{2098} \ipa{ŋa²} et \tgf{3926} \ipa{nja²} ressemblent à ceux du \bir{ŋa²} et \ipa{naŋ²} par leur initiale. Toutefois, des formes pré-tangoutes *\ipapl{ŋ(j)a} et \ipapl{n(j)aŋ} aurait dû devenir en tangoute *nji/*ne ou *njij/*no respectivement. Comme nous l'avons mentionné p.\pageref{analyse:1sg}, il est notable que ces deux pronoms, ainsi que celui de troisième personne (\tgf{2019} \ipa{thja²)} partagent la même rime et le même ton.  

Il est probable que des mécanismes d'analogie et de fusion avec d'autres morphèmes ont été à l'œuvre et ont remodelé le système pronominal du tangoute. Il est donc vain de tenter une réelle reconstruction de ce système, et absolument exclu de faire remonter le pronom \tgf{2098} \ipa{ŋa²} à une proto-forme *\ipapl{ŋa} pour faciliter la comparaison avec les autres langues. La ressemblance de \tgf{2098} \ipa{ŋa²} avec le \bir{ŋa²} et le \tib{ŋa} n'est donc qu'apparente.

Quelle que soit l'origine de cette rime --a/--ja dans les pronoms tangoutes, le fait qu'elle soit commune au pronom et aux suffixes personnels implique que l'analogie ou la fusion avec un autre morphème a eu lieu \textit{avant} la grammaticalisation finale de ces suffixes. C'est donc un argument supplémentaire qui va à l'encontre de l'hypothèse d'une grammaticalisation très ancienne des pronoms.

Les preuves qui permettent de démontrer l'antiquité du système d'accord en tangoute sont d'une autre nature : l'alternance vocalique du système verbal. Ce phénomène a été découvert en premier par \citet{nishida75}, puis développé par \citet{gong01huying}. Nishida distingue deux formes alternantes qu'il nomme A et B (nous conservons cette terminologie). La forme A est la forme de base qui apparaît dans la majorité des cas, et la forme B n'est employée qu'avec un agent à la première ou à la seconde personne du singulier. Tous les verbes à alternance A/B sont transitifs. On trouve des verbes intransitifs à alternance, mais cette alternance n'est pas liée à la personne (voir \citealt[23-24]{jacques09tangutverb} et p.\pageref{tab:aller.alternances}).


\subsubsection{Formes du paradigme biactanciel} \label{subsubsec:biactanciel}
Nous allons présenter des exemples de toutes   les configurations attestées pour deux actants, en distinguant 4 formes distinctes: 1\sg{}, 2\sg{}, la forme commune de 1\pl{} et 2\pl{} et celle de troisième personne. Dans le corpus à notre disposition, certaines formes ne sont pas attestées, en particulier celle avec un patient 1/2\pl{}. Les seuls exemples connus ont été découverts par \citet[224,228]{kepping85}, et concernent la forme 2>1/2\pl{} ; l'un d'entre eux sera cité plus bas (exemple \ref{ex:tg:causer.a.2.1pl} p.\pageref{ex:tg:causer.a.2.1pl}) . L'autre forme non attestée est 1\pl>2\sg{}; on trouve en revanche des exemples de 1\sg{}>2\sg{}, dont la forme verbale serait de toute façon semblable.

Les exemples suivants sont rangés d'abord par le patient du procès, dans l'ordre : troisième personne, puis 1\sg{}, puis 2\sg{} puis 1/2\pl{}. Le tableau \ref{tab:verbes.alternants.exemples1} indique les formes A et B des verbes qui apparaissent dans les exemples suivants avec les suffixes de personne.

\begin{table}
\captionb{Exemples de verbes alternants}\label{tab:verbes.alternants.exemples1}
\resizebox{\columnwidth}{!}{
\begin{tabular}{lllllllll} 
\toprule
\multicolumn{4}{c}{forme A} &\multicolumn{4}{c}{forme B} & sens \\
\midrule
4517& \tgf{4517} & \ipa{dzji } &1.1		&	4547& \tgf{4547} & \ipa{dzjo } &1.51		&	manger	\\
749& \tgf{0749} & \ipa{phji } &1.11		&	4568& \tgf{4568} & \ipa{phjo } &2.44		&	causer	\\
5026& \tgf{5026} & \ipa{mji } &1.11		&	4894& \tgf{4894} & \ipa{mjo } &1.51		&	entendre	\\
731& \tgf{0731} & \ipa{lju } &2.02		&	3189& \tgf{3189} & \ipa{ljo } &2.44		&	jeter	\\
5522& \tgf{5522} & \ipa{ljiij } &2.35		&	5293& \tgf{5293} & \ipa{ljii } &2.12		&	attendre	\\
46& \tgf{0046} & \ipa{ljij } &2.33		&	4803& \tgf{4803} & \ipa{lji } &2.09		&	voir	\\
\bottomrule
\end{tabular}}
\end{table}


\begin{enumerate}

\item 1\sg{} agent > 3 patient : forme B


\begin{tabular}{llllllll}
	\tgf{1542}&	\tgf{3508}&	\tgf{0100}&	\tgf{2798}&	\tgf{2987}&	\tgf{5481}&	\tgf{4547}&	\tgf{2098}\\
	\tinynb{1542}&	\tinynb{3508}&	\tinynb{0100}&	\tinynb{2798}&	\tinynb{2987}&	\tinynb{5481}&	\tinynb{4547}&	\tinynb{2098}\\
\end{tabular}
\begin{exe}
\ex \label{ex:tg:manger.b.1sg.3}  \vspace{-8pt}
\gll   \ipa{ku¹}	\ipa{bji²}	\ipa{lew¹}	\ipa{.jir²}	\ipa{lhjɨ̣¹}	\ipa{bo²}	\ipa{dzjo¹-ŋa²} \\
	alors sujet un cent coup bâton manger[B]-1sg{} \\
\glt Alors, moi, votre sujet, je recevrai cent coups de bâtons (Leilin 06.13A.5)
\end{exe}

\item 2\sg{} agent > 3 patient : forme B


\begin{tabular}{llllllllll}
	\tgf{4028}&	\tgf{2620}&	\tgf{5604}&	\tgf{5688}&	\tgf{1075}&	\tgf{2513}&	\tgf{2541}&	\tgf{1139}&	\tgf{1336}&	\tgf{4568}\\
	\tinynb{4028}&	\tinynb{2620}&	\tinynb{5604}&	\tinynb{5688}&	\tinynb{1075}&	\tinynb{2513}&	\tinynb{2541}&	\tinynb{1139}&	\tinynb{1336}&	\tinynb{4568}\\
\tgf{4601}& &&&&&&&&\\
\tinynb{4601}& &&&&&&&&\\
\end{tabular}
\begin{exe}
\ex \label{ex:tg:causer.b.2sg.3}  \vspace{-8pt}
\gll   \ipa{nji²}	\ipa{njwi²dʑjɨ}	\ipa{.wa²}	\ipa{dʑjow²}	\ipa{.ju²}	\ipa{dzjwo²}	\ipa{.jij¹}	\ipa{lhạ²}	\ipa{phjo²-nja²} \\
		toi technique quoi avoir[B] commun homme \antierg{} être.perdu causer[B]-2\sg{} \\
		Quelle technique as-tu pour causer la confusion chez les hommes du commun ? (Leilin 05.21B.3-4)
\glt
\end{exe}

\item 1\pl{} ou 2\pl{} agent > 3 patient  : forme A


\begin{tabular}{llllllllll}
	\tgf{4681}&	\tgf{2983}&	\tgf{2219}&	\tgf{5026}&	\tgf{4884}&	\tgf{5815}&	\tgf{2246}&	\tgf{2983}&	\tgf{0070}&	\tgf{5612}\\
	\tinynb{4681}&	\tinynb{2983}&	\tinynb{2219}&	\tinynb{5026}&	\tinynb{4884}&	\tinynb{5815}&	\tinynb{2246}&	\tinynb{2983}&	\tinynb{0070}&	\tinynb{5612}\\
\tgf{1918}&	\tgf{1274}& &&&&&&&\\
\tinynb{1918}&	\tinynb{1274}& &&&&&&&\\
\end{tabular}
\begin{exe}
\ex \label{ex:tg:entendre.a.2pl.3}  \vspace{-8pt}
\gll  \ipa{nju¹}	\ipa{u²}	\ipa{kjij¹-mji¹-nji²}	\ipa{tsjɨ¹}	\ipa{ljaa²}	\ipa{u²}	\ipa{thjwɨ¹}	\ipa{tshjiij¹}	\ipa{mji¹-wo²} \\
		oreille dans \opt{}-entendre[A]-2\pl{} aussi bouche dans ouvrir[A] dire[A] \negat{}-devoir \\
\glt Si vous veniez à entendre de telles paroles, qu’elles ne sortent pas de votre bouche ! (Cixiaozhuan 6.7-8)
\end{exe}

\item 3 agent > 1\sg{} patient  : forme A


\begin{tabular}{llllllllll}
	\tgf{4689}&	\tgf{1531}&	\tgf{0866}&	\tgf{1139}&	\tgf{2393}&	\tgf{3456}&	\tgf{0705}&	\tgf{2219}&	\tgf{0046}&	\tgf{2098}\\
	\tinynb{4689}&	\tinynb{1531}&	\tinynb{0866}&	\tinynb{1139}&	\tinynb{2393}&	\tinynb{3456}&	\tinynb{0705}&	\tinynb{2219}&	\tinynb{0046}&	\tinynb{2098}\\
\end{tabular}
\begin{exe}
\ex \label{ex:tg:voir.a.3.1sg}  \vspace{-8pt}
\gll   \ipa{.jwar¹}	\ipa{gja¹}	\ipa{ɣu¹}	\ipa{.jij¹}	\ipa{ljiij²}	\ipa{lja¹}	\ipa{zjịj¹}	\ipa{kjij¹-ljij²-ŋa²} \\
		Yue armée Wu \antierg{} détruire venir quand \opt{}-voir[A]-1\sg{} \\
\glt Quand (les soldats) de l'armée de Yue viendront détruire Wu, ils me verront (Leilin 03.21B.4-5)
\end{exe}

\item 3 agent > 2\sg{} patient  : forme A


\begin{tabular}{llllllllll}
	\tgf{1030}&	\tgf{2460}&	\tgf{3262}&	\tgf{1906}&	\tgf{2639}&	\tgf{2476}&	\tgf{1278}&	\tgf{2019}&	\tgf{0615}&	\tgf{1542}\\
	\tinynb{1030}&	\tinynb{2460}&	\tinynb{3262}&	\tinynb{1906}&	\tinynb{2639}&	\tinynb{2476}&	\tinynb{1278}&	\tinynb{2019}&	\tinynb{0615}&	\tinynb{1542}\\
\tgf{4028}&	\tgf{1139}&	\tgf{3527}&	\tgf{4200}&	\tgf{0749}&	\tgf{4601}&	\tgf{0734}& &&\\
\tinynb{4028}&	\tinynb{1139}&	\tinynb{3527}&	\tinynb{4200}&	\tinynb{0749}&	\tinynb{4601}&	\tinynb{0734}& &&\\
\end{tabular}
\begin{exe}
\ex \label{ex:tg:causer.a.3.2sg}  \vspace{-8pt}
\gll   \ipa{tɕjow¹}	\ipa{sə¹khow¹}	\ipa{nioow¹}	\ipa{mjiij²}	\ipa{xiwa¹}	\ipa{.jɨ²}	\ipa{thja¹}	\ipa{dwewr²}	\ipa{ku¹}	\ipa{nji²}	\ipa{.jij¹}	\ipa{mja¹-dzow¹-phji¹-nja²-mo²} \\
	Zhang Sikong après nom Hua appelé cela se.rendre.compte alors toi \antierg{} \hypot{}-emprisonné-causer[A]-2sg{}-\hypot{} \\
\glt Zhang sikong (\zh{張司空}), aussi appelé Hua (\zh{華}), s'en rendra compte et il te mettra en prison (Leilin 06.29B.7-30B.1)
\end{exe}


%\item 3 agent > 1\pl{} ou 2\pl{} agent : forme A

\item 2\sg{} agent > 1\sg{} patient  : forme A

\begin{tabular}{lllllll}
	\tgf{4508}&	\tgf{5981}&	\tgf{4670}&	\tgf{2590}&	\tgf{4658}&	\tgf{0749}&	\tgf{2098}\\
\tinynb{4508}&	\tinynb{5981}&	\tinynb{4670}&	\tinynb{2590}&	\tinynb{4658}&	\tinynb{0749}&	\tinynb{2098}\\
\end{tabular}
\begin{exe}
\ex \label{ex:tg:causer.a.2.1sg}  \vspace{-8pt}
\gll   \ipa{tjị¹}	\ipa{.a-tsjwu¹}	\ipa{.wjɨ²-thji¹}	\ipa{phji¹-ŋa²} \\
		repas un-théière \dir{}-boire causer[B]-1\sg{} \\
\glt Permettez-moi de prendre un repas. (Leilin, 06.01B.4)
\end{exe}

\item 2\pl{} agent > 1\sg{} patient  : forme A


\begin{tabular}{llllll}
	\tgf{2447}&	\tgf{1519}&	\tgf{5165}&	\tgf{2590}&	\tgf{4517}&	\tgf{2098}\\
	\tinynb{2447}&	\tinynb{1519}&	\tinynb{5165}&	\tinynb{2590}&	\tinynb{4517}&	\tinynb{2098}\\
\end{tabular}
\begin{exe}
\ex \label{ex:tg:manger.a.2.1sg}  \vspace{-8pt}
\gll   \ipa{ljo²}	\ipa{ɣu¹twụ¹}	\ipa{wjɨ²-dzji¹-ŋa²} \\
		frère.aîné à.la.place \dir{}-manger[A]-1\sg{} \\
\glt Mangez-moi à la place de mon frère! (Cixiaozhuan 17.7, \citealt[55-6]{jacques07textes})
\end{exe}

\item 1\sg{} agent > 2\sg{} patient  : forme A

\begin{tabular}{llllllllll}
	\tgf{4546}&	\tgf{0716}&	\tgf{2346}&	\tgf{5868}&	\tgf{4851}&	\tgf{5113}&	\tgf{3589}&	\tgf{0705}&	\tgf{5522}&	\tgf{4601}\\
	\tinynb{4546}&	\tinynb{0716}&	\tinynb{2346}&	\tinynb{5868}&	\tinynb{4851}&	\tinynb{5113}&	\tinynb{3589}&	\tinynb{0705}&	\tinynb{5522}&	\tinynb{4601}\\
\end{tabular}
\begin{exe}
\ex \label{ex:tg:attendre.a.1sg.2sg}  \vspace{-8pt}
\gll   \ipa{.jaar²}	\ipa{ɕjii¹}	\ipa{dzjụ²}	\ipa{khie²}	\ipa{bia²}	\ipa{.wji¹}	\ipa{dzjɨj¹zjịj¹}	\ipa{ljiij²-nja²} \\
		poulet tuer[A] riz gruau riz  faire[A] quand attendre[B]-2\sg{} \\
\glt Ayant tué un poulet et préparé du gruau de riz, je t'attendrai. (Leilin 03.04B.3)
\end{exe}
Les deux premiers verbes dans la phrase ci-dessus sont à une forme converbale, où la personne n'est pas marquée, d'où l'absence de forme B et de suffixe de personne. La forme 2\sg{}>3 du même verbe avec un thème B est présente dans l'exemple \ref{ex:tg:attendre}.

\item 2 agent > 1\pl{} patient  : forme A
\newline
\linebreak
\begin{tabular}{llllll}
	\tgf{2098}&	\tgf{0724}&	\tgf{1139}&	\tgf{0046}&	\tgf{0749}&	\tgf{4884}\\
\tinynb{2098}&	\tinynb{0724}&	\tinynb{1139}&	\tinynb{0046}&	\tinynb{0749}&	\tinynb{4884}\\
\end{tabular}
\begin{exe}
\ex \label{ex:tg:causer.a.2.1pl}  \vspace{-8pt}
\gll   \ipa{ŋa²}	\ipa{njɨ²}	\ipa{.jij¹}	\ipa{ljij²}	\ipa{phji¹-nji²} \\
		moi	\pl{} \antierg{} voir[A] causer[A]-2\pl{} \\
\glt Montrez-le-nous (cité dans \citet[228]{kepping85}, provenant du dictionnaire de Nevsky).
\end{exe}
Kepping cite seulement un autre exemple du même type dans son ouvrage, et nous n'avons pas été en mesure de trouver d'autres cas où 	\tgf{4884} \ipa{nji²} marque le patient. C'est toutefois selon toute vraisemblance un effet de la nature limitée du corpus tangoute contenant des dialogues.

\end{enumerate}

Outre les cas 1\sg{}>3 et 2\sg{}>3, la seule autre forme verbale où la forme B peut également apparaître est le réfléchi :
\newline
\linebreak
\begin{tabular}{lllllll}
	\tgf{4408}&	\tgf{5993}&	\tgf{1245}&	\tgf{3189}&	\tgf{4601}&	\tgf{0582}&	\tgf{5057}\\
	\tinynb{4408}&	\tinynb{5993}&	\tinynb{1245}&	\tinynb{3189}&	\tinynb{4601}&	\tinynb{0582}&	\tinynb{5057}\\
\end{tabular}
\begin{exe}
\ex \label{ex:tg:se.jeter}  \vspace{-8pt}
\gll   \ipa{məə¹}	\ipa{kha¹}	\ipa{jij¹-ljo²-nja²}	\ipa{thjij²ɣiej¹} \\
		feu intérieur \refl{}-jeter[B]-1\sg{} pourquoi \\
\glt Pourquoi te jeter dans le feu ? (Cixiaozhuan 14.1-2, voir \citealt[44-5]{jacques07textes})
\end{exe}


Si l'on place les formes attestées dans un tableau  où les colonnes indiquent le patient et les lignes l'agent, on obtient la représentation suivante pour les verbes à alternance (A indiquant la forme A, et B la forme B). Pour les formes où les deux actants sont à la troisième personne, on obtient toujours le thème A, y compris avec les verbes réfléchis.

\begin{table}
\captionb{Formes attestées du paradigme personnel des verbes tangoutes}\label{tab:paradigme.atteste}
\begin{tabular}{lllll}
	&	1\sg{}	&	2\sg{}	&	1/2\pl{}	&	3	\\
1\sg{}	&	?	&	A-\ipa{nja²}	&	?	&	 B-\ipa{ŋa²}	\\
2\sg{}	&	A-\ipa{ŋa²}	&	B-\ipa{nja²}	&	A-\ipa{nji²}	&	 B-\ipa{nja²}	\\
1/2\pl{}	&	 A-\ipa{ŋa²}	& ?	&	?	&	A-\ipa{nji²}	\\
3	&	A-\ipa{ŋa²}	&	A-\ipa{nja²}	&	?	&	A 	\\
\end{tabular}
\end{table}
On peut extrapoler le paradigme suivant, donné ici pour le verbe \tgf{0749} \ipa{phji¹} / \tgf{4568} \ipa{phjo²} ``causer'', car c'est le seul dont la quasi-totalité des formes est effectivement attestée dans les textes (les formes de ce verbe non-attestées sont indiquées par une astérisque). Contrairement à l'usage pour la plupart des langues himalayennes où le réfléchi est traité séparément, il est inclu dans le tableau \ref{tab:paradigme.extrapolation}.

\begin{table}
\captionb{Extrapolation du paradigme verbal complet}\label{tab:paradigme.extrapolation}
\resizebox{\columnwidth}{!}{
\begin{tabular}{lllll}
	&	1\sg{}	&	2\sg{}	&	1/2\pl{}	&	3	\\
1\sg{}	&	*\tgf{4568}\tgf{2098} \ipa{phjo²-ŋa²}	&	*\tgf{0749}\tgf{4601} \ipa{phji¹-nja²}	&	*\tgf{0749}\tgf{4884} \ipa{phji¹-nji²}	&	\tgf{4568}\tgf{2098} \ipa{phjo²-ŋa²}	\\
2\sg{}	&	\tgf{0749}\tgf{2098} \ipa{phji¹-ŋa²}	&	*\tgf{4568}\tgf{4601} \ipa{phjo²-nja²}	&	\tgf{0749}\tgf{4884} \ipa{phji¹-nji²}	&	\tgf{4568}\tgf{4601} \ipa{phjo²-nja²}	\\
1/2\pl{}	&	*\tgf{0749}\tgf{2098} \ipa{phji¹-ŋa²}	&	*\tgf{0749}\tgf{4601} \ipa{phji¹-nja²}	&	*\tgf{0749}\tgf{4884} \ipa{phji¹-nji²}	&	\tgf{0749}\tgf{4884} \ipa{phji¹-nji²}	\\
3	&	\tgf{0749}\tgf{2098} \ipa{phji¹-ŋa²}	&	\tgf{0749}\tgf{4601} \ipa{phji¹-nja²}	&	*\tgf{0749}\tgf{4884} \ipa{phji¹-nji²}	&	\tgf{0749} \ipa{phji¹}	\\

\end{tabular}}
\end{table}
La règle qui gouverne le fonctionnement de l'accord en tangoute est basée tout d'abord sur la hiérarchie d'empathie (\citealt{delancey81direction}, \citealt{jackson02rentongdengdi}), d'après laquelle lorsqu'un verbe transitif a un actant  participant au discours (première ou seconde personne) et un actant non-participant (troisième personne), c'est le premier qui est marqué quel que soit son rôle sémantique (agent ou patient). 

Lorsque les deux actants sont participants au discours, c'est par contre toujours le patient qui est marqué. En cela, l'accord du tangoute suit un principe partiellement ergatif. Un système similaire se retrouve en Dargwa, une langue du Daghestan (\citealt{sumbatova11person}).

L'alternance des formes A et B est quant à elle liée   au rôle sémantique de l'argument qui est indicié par le suffixe personnel. Lorsque celui-ci est agent, on emploie la forme B, et lorsqu'il est patient la forme A. Les formes réfléchies sont importantes, car elle montrent que lorsque l'argument coréférencé est à la fois agent et patient, c'est malgré tout la forme B qui est employée.

\subsubsection{Autres emplois des suffixes d'accord} \label{subsubsec:autres.emplois} 
Outre les cas mentionnés ci-dessus, où les suffixes de personne peuvent marquer les actants d'un verbe, on trouve deux situations où ils ont une fonction  différente.


Premièrement, comme \citet{kepping85} l'avait découvert, les suffixes d'accord peuvent référencer un non-actant, tel que le possesseur d'un actant; il s'agit d'un cas de montée du possesseur, un phénomène bien connu en français:
\newline
\linebreak
\begin{tabular}{llllllllll}
	\tgf{5354}&	\tgf{3583}&	\tgf{1326}&	\tgf{2833}&	\tgf{2635}&	\tgf{5218}&	\tgf{5604}&	\tgf{5113}&	\tgf{4028}&	\tgf{1139}\\
	\tinynb{5354}&	\tinynb{3583}&	\tinynb{1326}&	\tinynb{2833}&	\tinynb{2635}&	\tinynb{5218}&	\tinynb{5604}&	\tinynb{5113}&	\tinynb{4028}&	\tinynb{1139}\\
\tgf{2455}&	\tgf{2129}&	\tgf{4342}&	\tgf{4225}&	\tgf{5113}&	\tgf{4601}&	\tgf{3916}&&&\\
\tinynb{2455}&	\tinynb{2129}&	\tinynb{4342}&	\tinynb{4225}&	\tinynb{5113}&	\tinynb{4601}&	\tinynb{3916}&&&\\
\end{tabular}
\begin{exe}
\ex \label{ex:tg:non.actant.indicie.tuer}  \vspace{-8pt}
\gll   \ipa{thjɨ²}	\ipa{tja¹}	\ipa{kjɨ¹djɨj²}	\ipa{xjow²tɕhjwo¹}	\ipa{dʑjɨ.wji¹}	\ipa{nji²}	\ipa{.jij¹}	\ipa{gji²bjij²}	\ipa{dja²-sja¹-.wji¹-nja²-sji²} \\
		cela \topic{} certainement Feng.Chang  \erg{} toi \antierg{} épouse \dir{}-tuer-faire[A]-2\sg{}-\perf{} \\
\glt Etant donné tout cela, il est certain que c'est Feng Chang (\zh{馮昌}) qui a  tué ton épouse (littéralement ``qui te l'a tué l'épouse'') (Leilin 06.17A.6)
\end{exe}
La relation entre la personne indiciée et les arguments du verbe peut dans certains cas être peu claire. Dans l'exemple suivant, la 2\sg{} n'est ni l'agent (le roi) ni le patient (la perle), ni même le possesseur de la perle. C'est sémantiquement un circonstant locatif, puisque la perle se trouve dans le corps de la personne adressée à la 2\sg{}.
\newline
\linebreak
\begin{tabular}{llllllllll}
	\tgf{3830}&	\tgf{1326}&	\tgf{2833}&	\tgf{3926}&	\tgf{1139}&	\tgf{3900}&	\tgf{1954}&	\tgf{2583}&	\tgf{5173}&	\tgf{5113}\\
	\tinynb{3830}&	\tinynb{1326}&	\tinynb{2833}&	\tinynb{3926}&	\tinynb{1139}&	\tinynb{3900}&	\tinynb{1954}&	\tinynb{2583}&	\tinynb{5173}&	\tinynb{5113}\\
\tgf{4601}& &&& &&& &&\\
\tinynb{4601}& &&& &&& &&\\
\end{tabular}
\begin{exe}
\ex \label{ex:tg:non.actant.indicie.t}  \vspace{-8pt}
\gll   \ipa{njij²}	\ipa{kjɨ¹djɨj²}	\ipa{nja²}	\ipa{.jij¹}	\ipa{.o¹}	\ipa{.wjar¹}	\ipa{nji}	\ipa{dʑjɨ̣-.wji¹-nja²} \\
		roi certainement toi \antierg{} ventre ouvrir perle extraire-faire[A]-2\sg{} \\
\glt Le roi va certainement t'ouvrir le ventre pour en extraire la perle. (Leilin 04.02A.2)
\end{exe}
Il pourrait s'agir d'un cas de remontée de la marque de personne depuis le verbe \tgz{1954} ``ouvrir''.

Ce type d'exemples prouve que l'accord en tangoute est d'un type radicalement différent de celui observé en rgyalrong, où l'accord opère strictement avec les arguments de base du verbe (voir \citealt[211-212]{jacques08}). 


Deuxièmement, les suffixes d'accord peuvent apparaître dans de très rares cas suffixés à des noms. Dans \citet{jacques08weiyu}, nous avions présenté un exemple de ce type, que nous citons à nouveau ici :
\newline
\linebreak
\begin{tabular}{llll}
	\tgf{1918}&	\tgf{1220}&	\tgf{5306}&	\tgf{4601}\\
	\tinynb{1918}&	\tinynb{1220}&	\tinynb{5306}&	\tinynb{4601}\\
\end{tabular}
\begin{exe}
\ex \label{ex:tg:suffixe2sg.nom}  \vspace{-8pt}
\gll   \ipa{mji¹-dʑjwu¹}	\ipa{dzjwɨ¹-nja²} \\
		\negat{}-bienveillant seigneur-2\sg{} \\
\glt  Tu n'es pas un seigneur bienveillant. (Leilin 03.10B.2)
\end{exe}

Nous  avons trouvé deux autres exemples du même type dans notre corpus:
\newline
\linebreak
\begin{tabular}{llllllllll}
	\tgf{4848}&	\tgf{1034}&	\tgf{2800}&	\tgf{2546}&	\tgf{0448}&	\tgf{2590}&	\tgf{3678}&	\tgf{1245}&	\tgf{1139}&	\tgf{2639}\\
	\tinynb{4848}&	\tinynb{1034}&	\tinynb{2800}&	\tinynb{2546}&	\tinynb{0448}&	\tinynb{2590}&	\tinynb{3678}&	\tinynb{1245}&	\tinynb{1139}&	\tinynb{2639}\\
\tgf{5612}&	\tgf{3628}&	\tgf{0707}&	\tgf{2098}&	\tgf{1278}& &&&&\\
\tinynb{5612}&	\tinynb{3628}&	\tinynb{0707}&	\tinynb{2098}&	\tinynb{1278}& &&&&\\
\end{tabular}
\begin{exe}
\ex \label{ex:tg:suffixe1sg.nom}  \vspace{-8pt}
\gll   \ipa{ɣã¹kwo¹xjwã¹}	\ipa{njạ¹}	\ipa{gjɨ²}	\ipa{wjɨ²-to²}	\ipa{jij¹}	\ipa{jij¹}	\ipa{mjiij²}	\ipa{tshjiij¹}	\ipa{ɣjwã¹tɕjiw¹-ŋa²} \ipa{jɨ²}\\
Anguxian dieu un \dir{}-apparaître soi-même \antierg{} nom dire[A] Yuanzhou-1\sg{} dire \\
\glt Au département d' Angu (\zh{安固}), un dieu apparut, et il dit son nom : ``Je suis Yuanzhou (\zh{袁周})'' (Leilin 10.04B.2-3)
\end{exe}

\begin{tabular}{llllllllll}
\tgf{0416} & 	\tgf{5815} & 	\tgf{4601} & 	\tgf{2098} & 	\tgf{4543} & 	\tgf{2600} & 	\tgf{4342} & 	\tgf{5670} & 	\tgf{4601} & 	\tgf{3425} \\
\tinynb{0416} & 	\tinynb{5815} & 	\tinynb{4601} & 	\tinynb{2098} & 	\tinynb{4543} & 	\tinynb{2600} & 	\tinynb{4342} & 	\tinynb{5670} & 	\tinynb{4601} & 	\tinynb{3425} \\
\tgf{3391} & 	\tgf{4601} & 	\tgf{2098} & 	\tgf{3371} & 	\tgf{1144} & 	\tgf{4342} & 	\tgf{5449} & 	\tgf{4601} & \\
\tinynb{3391} & 	\tinynb{4601} & 	\tinynb{2098} & 	\tinynb{3371} & 	\tinynb{1144} & 	\tinynb{4342} & 	\tinynb{5449} & 	\tinynb{4601} & \\
\end{tabular}
\begin{exe}
\ex \label{ex:tg:suffixe1sg.nom2}  \vspace{-8pt}
\gll   \ipa{tsja¹tsjɨ¹-nja²}  	\ipa{mər¹mjar¹}  	\ipa{dja²-.o¹-nja²}  	\ipa{pja²pjɨ¹-nja²}  	\ipa{ŋa²}  	\ipa{dzja¹djị¹}  	\ipa{dja²-tjị¹-nja²}   \\
	sœur.aînée-2\sg{}  moustache \dir{}-avoir-2\sg{} grand.père-2\sg{} moi chignon \dir{}-placer[A]-2\sg{} \\
\glt  Tu es ma sœur aînée, ma moustache est là pour toi, tu es mon grand-père, mon chignon a été placé pour toi. (Traduction incertaine, Proverbes tangoutes 14b.5, \citealt[106;179]{kychanov74}, \citealt{jacques11tangut.verb}).
\end{exe}


Un suffixe d'accord peut donc être directement suffixé à un nom en tangoute pour former un prédicat nominal. Le tangoute est toutefois radicalement différent d'une langue omniprédicative comme le nahuatl (\citealt{launey94}) car tout d'abord ces exemples sont exceptionnels et il est normalement indispensable d'avoir une copule pour former un prédicat nominal, et d'autre part les noms n'ont pas besoin d'être relativisés pour servir d'argument. 



Ces exemples sont d'une grande importance pour comprendre comment les suffixes de personnes se sont grammaticalisés. On ne trouve un tel phénomène dans aucune langue macro-rgyalronguique : les suffixes pronominaux ne peuvent être affixés à des noms dans aucun cas ni en rgyalrong, ni en pumi, ni dans aucune langue sur laquelle nous disposons de données fiables. Toutefois, il est difficile, étant donné le nombre limité d'exemples, d'apporter une interprétation fiable à ces données; il pourrait s'agir de la trace d'un stade de la grammaticalisation des suffixes personnels où l'on trouvait un système plus proche typologiquement des langues turciques, avant que la copule \tgf{0508} \ipa{ŋwu²} ne devienne (quasi-)obligatoire pour former un prédicat nominal.

\subsubsection{Origine de l'alternance vocalique} \label{subsubsec:origine.alternances}
Pour expliquer l'origine de l'alternance A/B, \citet{gong01huying} avait suggéré l'influence du suffixe sur la voyelle du radical du verbe. Cette explication, bien que possible a priori, pose toutefois le problème que les suffixes d'accord personnel sont parfois absents, alors que l'alternance A/B est toujours présente. Gong cite des exemples de ce phénomène, mais l'on peut consulter l'exemple \ref{ex:tg:attendre} p.\pageref{ex:tg:attendre}.

Nous avons proposé une hypothèse alternative que nous allons expliciter ici plus en détail.  \citet{gong01huying}, \citet[239-242]{gong02a} recense quatre types d'alternances A/B. Toutefois, son type 3 est problématique, comme nous allons le voir plus bas,   on ne conservera que ses types 1, 2 et 4, auxquels on peut rajouter deux autres catégories, voir p.\pageref{ex:tg:craindre} et \citet[63]{jacques08alternations} ainsi que \citet{shijb10}. Les  types d'alternances relevés par Gong sont présentés dans le tableau \ref{tab:verbes.alternants.exemples2}.

\begin{table}
\captionb{Les catégories d'alternances A/B}\label{tab:verbes.alternants.exemples2}
\resizebox{\columnwidth}{!}{
\begin{tabular}{lllllllllll} 
\toprule
\multicolumn{2}{c}{alternance} & \multicolumn{4}{c}{forme A} &\multicolumn{4}{c}{forme B} & sens \\
\midrule
1&\ipapl{--ji / --jo} &4517& \tgf{4517} & \ipa{dzji } &1.10		&	4547& \tgf{4547} & \ipa{dzjo } &1.51		&	manger	\\
&\ipapl{--jii / --joo} & 716  & \tgf{0716}& \ipa{ɕjii} &1.14 & 4571& \tgf{4571} & \ipa{ɕjoo} &1.53		&tuer (animal)\\
&\ipapl{--jị / --jọ} & 5449 & \tgf{5449} &\ipa{tjị} &1.67 & 5633 & \tgf{5633}& \ipa{tjọ} &1.72&mettre \\
&\ipapl{--jir / --jor} &1599& \tgf{1599} &\ipa{rjir} &1.79& 23 &\tgf{0023} &\ipa{rjor} &1.09 &obtenir \\
&\ipapl{--jij / --jo} &3126& \tgf{3126} &\ipa{dʑjij} &2.32& 1075 &\tgf{1075} &\ipa{dʑjo} &2.44 &avoir \\
2&\ipapl{--ju / --jo} &731& \tgf{0731} & \ipa{lju } &2.02		&	3189& \tgf{3189} & \ipa{ljo } &2.44		&	jeter	\\
&\ipapl{--juu / --joo} &4370& \tgf{4370} & \ipa{ljuu} &1.07		&	4248& \tgf{4248} & \ipa{ljoo} &1.53		&	parier	\\
&\ipapl{--jụ / --jọ} &4834& \tgf{4834} & \ipa{njụ} &2.52		&	53& \tgf{0053} & \ipa{njọ} &2.64	&	allaiter	\\
4&\ipapl{--jij / --ji}&46& \tgf{0046} & \ipa{ljij } &2.33		&	4803& \tgf{4803} & \ipa{lji} &2.09		&	voir	\\
&\ipapl{--jiij / --jii}&5522& \tgf{5522} & \ipa{ljiij } &2.35		&	5293& \tgf{5293} & \ipa{ljii} &2.12		&	attendre	\\
&\ipapl{--jịj / --jị}&3159& \tgf{3159} & \ipa{lhjịj} &2.54?	&	5591& \tgf{5591} & \ipa{ljị} &2.60		&	recevoir	\\
&\ipapl{--jijr / --jir}&1857& \tgf{1857} & \ipa{.jijr} &2.68	&	3764& \tgf{3764} & \ipa{.jir} &2.72	&	mettre à mort	\\
5&\ipapl{--ier / --ior}&4063&\tgf{4063} & \ipa{.wier } &1.78		&	4064& \tgf{4064} & \ipa{.wior} &1.90		&	aimer\\
&	\ipapl{--jạ / --jɨ̣}&  2539 &\tgf{2539}	&\ipa{kjạ}	&1.64&	1252& \tgf{1252} & \ipa{kjɨ̣} &1.69		&	craindre \\
\bottomrule
\end{tabular}}
\end{table}
Le type \ipapl{--ji / --jɨ} de Gong était justifié par six paires, que nous nous proposons d'analyser ici une à une pour illustrer les difficultés qui sont posées par l'hypothèse que ces alternances relèveraient de la personne.



\begin{enumerate}
\item 4469 \tgf{4469}  \ipa{ɕji} 2.09 et 4481 \tgf{4481}  \ipa{ɕjɨ} 1.29 ``aller''. Dans \citet[24]{jacques09tangutverb}, nous avons présenté le tableau suivant, où nous comptons les attestations de ces deux formes verbales en fonction de la personne de l'actant unique :
\newline
\linebreak
\begin{tabular}{lll} \label{tab:aller.alternances}
personne & \tgf{4469}  \ipa{ɕji²} & \tgf{4481}  \ipa{ɕjɨ¹} \\
1\sg{}, 2\sg{} & 2 & 2 \\
3 & 21 & 13 \\
ambigu & 3 &2 \\
\end{tabular}
\newline
Ces chiffres parlent d'eux-mêmes: aucune relation n'existe entre l'utilisation de l'une ou l'autre forme et la personne de l'actant.
\item 3072 \tgf{3072}  \ipa{sji} 2.10 et 5918 \tgf{5918}  \ipa{sjɨ} 1.30  ``mourir''. La seconde forme apparaît avec un actant à la troisième personne dans de nombreux cas, comme nous l'avons suggéré dans \citet[63]{jacques08alternations}. On peut trouver un exemple de ce type p.\pageref{ex:tg:relache}. De la même façon, \tgf{3072}  \ipa{sji²} peut être suivi de suffixe de personne :
\newline
\linebreak
\begin{tabular}{lllll}
\tgf{2104}&	\tgf{4342}&	\tgf{3363}&	\tgf{2098}&	\tgf{4861}\\
\tinynb{2104}&	\tinynb{4342}&	\tinynb{3363}&	\tinynb{2098}&	\tinynb{4861}\\
\end{tabular}
\begin{exe}
\ex \label{ex:tg:mourir.a.1sg}  \vspace{-8pt} 
\gll  \ipa{ɕji¹}	\ipa{dja²-sji²-ŋa²}	\ipa{zjọ²} \\
		avant \dir{}-mourir-1\sg{} quand \\
\glt Quand je suis mort (Leilin 06.11A.7)
\end{exe}


\item 1427 \tgf{1427}  \ipa{phji} 2.10 et 5607 \tgf{5607}  \ipa{phjɨ} 1.30 ``jeter''. La seconde forme est clairement attestée à une forme nominalisée, où il est clair que l'agent ne peut pas être à la 1\sg{}/2\sg{}, comme nous l'avons mentionné dans \citet[62]{jacques08alternations} :
\newline
\linebreak
\begin{tabular}{llllllllll}
	\tgf{5607}&	\tgf{0981}&	\tgf{1582}&	\tgf{2931}&	\tgf{5880}&	\tgf{5604}&	\tgf{5697}&\tgf{0749}&	\tgf{5645}&	\tgf{0491}\\
	\tinynb{5607}&	\tinynb{0981}&	\tinynb{1582}&	\tinynb{2931}&	\tinynb{5880}&	\tinynb{5604}&	\tinynb{5697}&\tinynb{0749}&	\tinynb{5645}&	\tinynb{0491}\\
	\tgf{2562}& &&&&&&&&\\
	\tinynb{2562}& &&&&&&&&\\
\end{tabular}
\begin{exe}
\ex \label{ex:tg:jeter}  \vspace{-8pt} 
\gll   \ipa{phjɨ¹}	\ipa{war²}	\ipa{gjịj¹}	\ipa{sej¹}	\ipa{ŋwu²}	\ipa{dʑjɨ}	\ipa{tɕior¹}	\ipa{phji¹}	\ipa{tjị²}	\ipa{ljọ²}	\ipa{wjij²} \\
		jeter objet profit calculer \conj{} action sale causer[A] endroit où y.avoir \\
\glt Ainsi, comment peut-on se salir  en voulant tirer profit d’un objet perdu (Cixiaozhuan 27.5-6)
\end{exe}
On ne trouve pas d'exemples de \tgf{1427}  \ipa{phji²} suivi des suffixes de 1\sg{} et de 2\sg{}, mais il s'agit probablement d'un effet de corpus.
\item 1319 \tgf{1319}  \ipa{tshji} 1.11 ``important'' et 5610 \tgf{5610}  \ipa{tshjɨ} 1.30 ``aimer ?''. Ces formes sont traitées p.\pageref{ex:tg:important1}. La seconde forme n'a pas d'attestation connue avec des suffixes personnels.
\item 5293 \tgf{5293}  \ipa{ljii} 2.12 ``attendre'' et 5291 \tgf{5291}  \ipa{ljɨɨ} 2.29 ``détester ?'' . La première forme, comme nous le montrons plus haut, est clairement le thème B du verbe  \tgf{5522} \ipa{ljiij²}. Voir une attestation p.\pageref{ex:tg:attendre}. La seconde forme n'est pas attestée dans des textes et son sens exact est peu clair, mais ne signifie certainement pas ``attendre''.
\item 374 \tgf{0374}  \ipa{lhjii} 1.14  et 1382 \tgf{1382}  \ipa{lhjɨɨ} 2.29 ``regretter''. Le second caractère est visiblement la première syllabe d'une forme rédupliquée (voir section \ref{subsec:redp-alt}), et l'alternance vocalique ici est sans relation avec le marquage personnel.

\end{enumerate}


L'idée d'une alternance \ipapl{--ji / --jɨ} liée à la personne attend donc encore d'être confirmée.  Toutefois, il existe une autre paire, non mentionnée par Gong, qui pourrait donner cette confirmation. Il s'agit de la forme de la paire 1628 \tgf{1628} \ipa{.wjị} 2.60 et 5089 \tgf{5089} \ipa{.wjɨ̣} 2.61, dont le sens premier semble être ``envoyer''. Ces deux formes sont associées dans le Tongyin, ce qui suggère qu'elles constituent deux thèmes du même paradigme. Le thème B est clairement attesté avec un suffixe de 1\sg{} dans l'exemple suivant, où il signifie ``épargner'' ou ``relâcher'' :
\newline
\linebreak
\begin{tabular}{llllllllll}
	\tgf{4978}&	\tgf{1183}&	\tgf{0685}&	\tgf{2331}&	\tgf{3818}&	\tgf{0930}&	\tgf{2019}&	\tgf{1139}&	\tgf{0613}&	\tgf{5341}\\
	\tinynb{4978}&	\tinynb{1183}&	\tinynb{0685}&	\tinynb{2331}&	\tinynb{3818}&	\tinynb{0930}&	\tinynb{2019}&	\tinynb{1139}&	\tinynb{0613}&	\tinynb{5341}\\
\tgf{3818}&	\tgf{0930}&	\tgf{3583}&	\tgf{5148}&	\tgf{5643}&	\tgf{1374}&	\tgf{5089}&	\tgf{2098}&&\\
\tinynb{3818}&	\tinynb{0930}&	\tinynb{3583}&	\tinynb{5148}&	\tinynb{5643}&	\tinynb{1374}&	\tinynb{5089}&	\tinynb{2098}&&\\
\end{tabular}
\begin{exe}
\ex \label{ex:tg:épargner}  \vspace{-8pt}
\gll   \ipa{tjij¹}	\ipa{dạ²}	\ipa{ŋạ²}	\ipa{tɕjwow²}	\ipa{mjijr²}	\ipa{dju¹}	\ipa{thja¹}	\ipa{.jij¹}	\ipa{?rjir²}	\ipa{mjijr²}	\ipa{dju¹}	\ipa{tja¹}	\ipa{dʑjị²}	\ipa{mjɨ¹-tɕhjɨ¹-.wjɨ̣²-ŋa²} \\
 si affaire bonne offrir \nmls{} avoir cela \antierg{} empêcher \nmls{} avoir \topic{} crime \negat{}-\pot{}-relâcher[B]-1\sg{} \\
\glt S'il y a quelqu'un qui vient m'apporter une bonne chose, et que quelqu'un (d'autre) l'en empêche, je n'épargnerai pas celui-là. (Les douze royaumes, 133.12 voir \citealt[58,120]{solonin95} et \citealt[201]{nie02shierguo})
\end{exe}

 
Toutefois, l'association de ces deux formes dans le Tongyin n'est pas une garantie absolue qu'elles soient réellement apparentées. Il est également possible que \tgf{5089} \ipa{.wjɨ̣²} soit le thème B du verbe \tgf{5791}  \ipa{.wjạ²} ``envoyer, relâcher''; Ce verbe n'apparaît jamais avec un suffixe de 1\sg{} ou de 2\sg{}, sauf dans le suivant où le suffixe réfère au patient, et où donc une forme A serait attendue :
\newline 
\linebreak
\begin{tabular}{llllllllll}
	\tgf{0998}&	\tgf{0276}&	\tgf{1857}&	\tgf{3357}&	\tgf{3583}&	\tgf{5918}&	\tgf{5148}&	\tgf{0508}&	\tgf{1734}&	\tgf{5791}\\
	\tinynb{0998}&	\tinynb{0276}&	\tinynb{1857}&	\tinynb{3357}&	\tinynb{3583}&	\tinynb{5918}&	\tinynb{5148}&	\tinynb{0508}&	\tinynb{1734}&	\tinynb{5791}\\
\tgf{4601}& &&&&&&&&\\
\tinynb{4601}& &&&&&&&&\\
\end{tabular}
\begin{exe}
\ex \label{ex:tg:relacher2}  \vspace{-8pt}
\gll   \ipa{dzjiij²no²}	\ipa{.jijr²}	\ipa{kiẹj²}	\ipa{tja¹}	\ipa{sjɨ¹}	\ipa{dʑjị²}	\ipa{ŋwu²}	\ipa{tji¹-.wjạ²-nja²} \\
		professeur tuer[A] vouloir \topic{} mourir[C] crime être \prohib{}-relâcher-2\sg{} \\
\glt Vouloir tuer mon précepteur est un crime mortel. Je ne te pardonnerai pas. (Leilin 03.07A.7)
\end{exe}

Si c'est vrai, ce verbe reflète en fait l'alternance \ipapl{--ja / --jɨ} mentionnée plus haut, et contrairement à ce que nous avons proposé p.\pageref{ex:tg:envoyer}, il n'est pas apparenté à \tgf{5974} \ipa{.wjịj²} ``envoyer''. Davantage de recherches sur les textes tangoutes seront nécessaires pour déterminer l'étymologie de ces formes.

Pour expliquer les alternances du tableau \ref{tab:verbes.alternants.exemples2}, nous avons proposé dans \citet{jacques09tangutverb} une hypothèse basée sur la comparaison avec les langues macro-rgyalronguiques modernes. En Qiang du nord tel qu'il est décrit par \citet[131-3]{huangbf06qiangyu}, on trouve le paradigme verbal indiqué dans le tableau \ref{tab:qiang.accord}.

\begin{table}
\captionb{Système d'accord en qiang du nord}\label{tab:qiang.accord}
\begin{tabular}{llllllll} \toprule
& 1P & 2P & 3P & intransitif\\
1A &\cellcolor{lightgray} & --a / --æ & --w--a / --w--æ &--a / --æ \\
 & \cellcolor{lightgray}&--\ipapl{əɻ} & --w-\ipapl{əɻ} &--\ipapl{əɻ} \\
2A& --n &\cellcolor{lightgray} & --w--\ipapl{ən} & --n\\
& --j &\cellcolor{lightgray}& --w--\ipapl{əj} & --j\\
3A &0 &0&--w & 0 \\
\bottomrule
\end{tabular}
\end{table}
Les verbes transitifs du qiang ont donc aux formes à patient de troisième personne un suffixe --w qui apparaît entre le suffixe de personne et le radical du verbe. L'existence d'un tel suffixe a déjà été mentionné à propos de nombreuses langues, notamment par \citet{delancey81direction} à propos du rgyalrong situ.

L'hypothèse que nous proposons à propos du tangoute est la suivante. En pré-tangoute, un suffixe (que nous noterons *--u) cognat avec le suffixe --w de troisième personne du qiang du nord et du rgyalrong situ, apparaissait dans la même partie du complexe verbal qu'en Qiang, à savoir entre le radical du verbe et le suffixe personnel. Toutefois, ce suffixe a subi une fusion avec le radical verbal, et ceci à une époque ancienne, à un stade du pré-tangoute antérieur au passage de *--ja à --ji. Le tableau \ref{tab:reconstruction.alternances} présente une reconstruction des fusions vocaliques d'où proviennent les alternances observées.

\begin{table}
\captionb{Reconstruction des alternances}\label{tab:reconstruction.alternances}
\begin{tabular}{lll} \toprule
pré-tangoute & tangoute attesté & page de référence\\
\ipapl{*--ja} & \ipapl{--ji} &  \pageref{subsec:voyelle.e.i}\\
\ipapl{*--ja-u} & \ipapl{--jo} &\\
\ipapl{*--jo} & \ipapl{--ju} & \pageref{subsec:voyelle.u} \\
\ipapl{*--jo-u} & \ipapl{--jo} &\\
\ipapl{*--jaŋ} & \ipapl{--jij} & \pageref{subsec:voyelle.ej}  \\
\ipapl{*--jaŋ-u} & \ipapl{--jo}& \\
\ipapl{*--jej} & \ipapl{--jij} & \pageref{subsec:voyelle.ej}  \\
\ipapl{*--jej-u} & \ipapl{--ji}& \\
\ipapl{*--ra} & \ipapl{--ie} &\pageref{subsec:voyelle.e.i}\\
\ipapl{*--ra-u} & \ipapl{--io} &\\
\ipapl{*--jaC} & \ipapl{--ja} & \pageref{subsec:voyelle.a}  \\
\ipapl{*--jaC-u} & \ipapl{--jɨ}& \\
\bottomrule
\end{tabular}
\end{table}
Ayant eu lieu avant les évolutions vocaliques expliquées dans le chapitre \ref{phono}, la fusion de la voyelle du radical verbal a empêché la voyelle du radical du thème B de suivre son évolution normale, ce qui explique la grande différence entre la prononciation des deux thèmes en tangoute attesté.


Un problème qui reste inexpliqué est le fait que les verbes à rime --jij / --jiij peuvent alterner soit avec --jo, soit avec --ji. Ceci ne semble pas pouvoir s'expliquer par la rime du verbe en pré-tangoute aussi simplement que nous l'avons présenté dans le tableau \ref{tab:reconstruction.alternances}. En effet,  \tgf{5522} \ipa{ljiij²} ``attendre'' a une rime provenant du pré-tangoute *--\ipapl{jaaŋ} tandis que celle de \tgf{5612} \ipa{tshjiij¹} ``dire'' vient de *--\ipapl{jeej}, mais leur thème B a pour rime -jii. Des processus d'analogie ont dû perturber les alternances vocaliques. L'alternance --jij / --jo est clairement résiduelle et ne concerne qu'une poignée de verbes, tandis que celle entre --jij et -ji est beaucoup plus courante. Nous suggérons donc qu'à l'origine la distribution a dû être plus fidèle à l'étymologie, mais que l'alternance plus productive --jij / --ji s'est étendue au dépens de l'alternance --jij / --jo.


D'autres alternances que celles mentionnées dans le tableau \ref{tab:verbes.alternants.exemples2} existent. Ce sont --u /--ju, --\ipapl{jow}/--\ipapl{jɨj} et \ipapl{--ạ / --ẹ}. Elles   sont chacune exemplifiées   par une seule paire :
\begin{itemize}
\item \tgf{1338} \ipa{dzu¹} / \tgf{4973} \ipa{dzju¹} ``aimer'' (voir p.\pageref{ex:tg:aimer.b}). Les exemples de cette paire  peuvent se trouver dans \citet{solonin95}.
\item \tgf{1105} \ipa{khjow¹} / \tgf{5644} \ipa{khjɨj¹} ``donner''. Cet exemple a été mentionné pour la première fois par \citet[2]{nishida86yueyue}.
\item \tgf{5077}\tgf{1817} \ipa{mjɨ¹dạ²} / \tgf{5077}\tgf{5283} \ipa{mjɨ¹dẹ²} ``ne pas savoir". Ce verbe est traité dans \citet{lin08mjida}, qui montre dans ses exemples que \tgf{5077}\tgf{5283} \ipa{mjɨ¹dẹ²} apparaît toujours avec le suffixe personnel \tgf{2098} \ipa{ŋa²}. Cette alternance remonterait à *\ipapl{--aC} / *\ipapl{--a}, autrement dit une consonne finale serait présente au thème A et absente au thème B. Ce verbe ne semble pas avoir de cognats externes connus, mais un tel processus pourrait s'expliquer par l'assimilation de la consonne finale au suffixe que le suit. Ce verbe serait donc le dernier de la langue à avoir maintenu cette alternance.
\end{itemize} 

Nous ne disposons pas encore d'une théorie adéquate pour expliquer l'origine de ces alternances. --\ipapl{jow}/--\ipapl{jɨj} pourrait venir de *\ipapl{--jvm} / *\ipapl{--im}, tandis que \ipapl{--ạ / --ẹ} pourrait avoir pour origine *\ipapl{--aC} / *\ipapl{--a}. Toutefois, ces reconstructions n'ont aucun pouvoir explicatif et sont donc de peu d'intérêt. Ces alternances restent donc un résidu inanalysable, et des recherches futures seront nécessaires pour tirer au clair leur origine.


\subsection{Les préfixes directionnels} \label{subsec:directionnels}
\citet[176-203,208-216]{kepping85} a décrit en détail le fonctionnement des préfixes directionnels en tangoute. Contrairement au marquage de la personne où la découverte des alternances vocaliques a permis une nouvelle description du système verbal, notre compréhension du système des préfixes directionnels n'a pas été révolutionnée depuis son époque. La seule innovation concerne le statut des préfixes directionnels (clitiques ou vrais préfixes), voir \citet{jacques11tangut.verb}.


Le système directionnel du tangoute compte sept directions et deux séries de préfixes, que nous appellerons par commodité A et B. La série A est de très loin la plus courante, et apparaît surtout pour exprimer une action achevée. La série B, quand à elle, marque l'optatif ou l'irréel.


\begin{table}
\captionb{Préfixes directionnels du tangoute}\label{tab:directionnels}
\resizebox{\columnwidth}{!}{
\begin{tabular}{lllllllll} 
\toprule
\multicolumn{4}{c}{Série A} &\multicolumn{4}{c}{Série B} & direction \\
\midrule
5981& \tgf{5981} & \ipa{.a} & & 3989& \tgf{3989} & \ipa{.jij} & 1.36&haut\\
1452& \tgf{1452} & \ipa{nja} & 1.20& 3846& \tgf{3846} & \ipa{njij} & 2.33&bas\\
1326& \tgf{1326} & \ipa{kjɨ} & 1.30& 2219& \tgf{2219} & \ipa{kjij} & 1.36&cislocatif\\
2590& \tgf{2590} & \ipa{.wjɨ} & 2.27& 2536& \tgf{2536} & \ipa{.wjij} & 2.32&translocatif\\
4342& \tgf{4342} & \ipa{dja} & 2.17& 4841& \tgf{4841} & \ipa{djij} & 2.33&perte\\
804& \tgf{0804} & \ipa{djɨ} & 2.28& 4841& \tgf{4841} & \ipa{djij} & 2.33&obtention\\
795& \tgf{0795} & \ipa{rjɨr} & 2.77& 3706& \tgf{3706} & \ipa{rjijr} & 2.68& neutre\\
\bottomrule
\end{tabular}}
\end{table}

\citet[216]{kepping85} et \citet[94]{lifw99bijiao} ont mentionné la similitude de ce système avec celui d'autres langues macro-rgyalronguiques. Toutefois, il est manifeste que malgré la similarité du système de préfixes directionnels entre ces langues, peu de préfixes présentent une similarité formelle qui puisse justifier l'hypothèse d'une origine commune. 

De toutes les langues macro-rgyalronguiques,  c'est visiblement le pumi qui a le système le plus similaire à celui du tangoute. Sur six préfixes, trois ont des formes qui pourraient être apparentées, celui du ``bas'', du ``cislocatif'' et du ``translocatif''. Les données ci-dessous viennent du dialect de Shuiluo, mais on trouve des formes semblables dans tous les dialectes pumi (\citealt[157]{lusz01pumi}).

\begin{table}
\captionb{Préfixes directionnels du pumi comparés à ceux du tangoute}\label{tab:directionnels.pumi}
\centering
\begin{tabular}{llll} 
\toprule
\multicolumn{2}{c}{tangoute}  & pumi & direction \\
\midrule
&&\ipa{tə́--} & haut \\
\tgf{1452} & \ipa{nja¹} &\ipa{nɜ--} &bas \\
&&\ipa{khə--} & entrée  \\
&&\ipa{hɜ--} & sortie \\
\tgf{0804} & \ipa{djɨ²}&\ipa{də--} & cislocatif\\
\tgf{4342} & \ipa{dja²}&\ipa{thɜ--} & translocatif\\
\bottomrule
\end{tabular}
\end{table}

En tangoute (pour la série A) comme en pumi, on remarque les préfixes directionnels ont deux vocalismes différents : \ipapl{--jɨ} et \ipapl{--ja} d'un côté contre \ipapl{--ə} et \ipapl{--ɜ} de l'autre, et pour les trois préfixes potentiellement comparables, on remarque une correspondance récurrente \ipapl{--jɨ}  ::  \ipapl{--ə} et \ipapl{--ja} :: \ipapl{--ɜ}, qui semble limiter la possibilité d'une ressemblance due au hasard.

Les préfixes directionnels, après s'être grammaticalisés, ont dû subir une attrition phonétique qui a limité la variété des vocalismes possibles à seulement deux. Le fait que pumi et tangoute partagent un vocalisme semblable pourrait être un argument en faveur de l'hypothèse selon laquelle la grammaticalisation de ces préfixes directionnels a eu lieu dans l'ancêtre commun du tangoute et du pumi, et qu'il ne s'agit pas d'une évolution indépendante.


Pour les autres préfixes du tangoute, il n'est pas facile de trouver des étymologies possibles. Une possibilité pour le préfixe de translocatif tangoute \tgf{2590} \ipa{.wjɨ²} serait de le rapprocher du verbe \tgf{0676} \ipa{.wjij¹} ``partir''. Un parallèle typologique est offert par le japhug (\citealt{jacques13harmonization}), dans lequel des préfixes cislocatif \ipa{ɣɯ--} et translocatif \ipa{ɕɯ--} ont été créés à partir des verbes  ``venir'' \ipa{ɣi} et ``aller'' \ipa{ɕe} respectivement. Devenant préfixes, ces anciens verbes ont subi une érosion phonologique,  centralisant leur voyelle  en \ipa{ɯ}. Un phénomène similaire a pu s'opérer en tangoute; il convient toutefois de noter que le verbe \tgf{0676} \ipa{.wjij¹}, pourtant très vraisemblablement cognat du \jpg{ɣi}, a le sens inverse de ce dernier.

Contrairement au japhug qui dispose à la fois d'un système cislocatif / translocatif et d'un système de directionnel, le tangoute aurait développé un système plus simple regroupant les deux catégories, comme c'est le cas en pumi ou en qiang.


La série B des préfixes directionnels se distingue de la série A en ce que tous les préfixes ont la rime --jij. La différence de rime entre les préfixes de la série B et ceux de la série A est vraisemblablement due au fait que les préfixes de série B représentent une forme fusionnée avec un autre préfixe à fonction modale dont la rime était telle en pré-tangoute qu'elle pouvait donner --jij en tangoute attesté (c'est à dire soit *--\ipapl{jaŋ}, soit *--\ipapl{jej}). La fusion a forcément opéré dans ce sens : si les formes de base étaient les préfixes de série A, il serait impossible d'expliquer pourquoi parmi les préfixes de série A, certains ont la rime --\ipapl{ja} et d'autres la rime --\ipapl{jɨ}. 

L'ordre relatif des préfixes directionnels avec les autres préfixes sera traité en section \ref{subsec:direct-ordre}.

\subsection{Autres affixes} \label{subsec:affixes.autre}
Dans cette section, nous étudierons les autres affixes qui apparaissent dans le système verbal : les affixes de temps/aspect/mode ainsi que les préfixes de négation. On ne traitera que des affixes qui présentent une étymologie potentielle dans d'autres langues; un traitement exhaustif de tous les affixes reviendrait à rédiger une grammaire complète du tangoute.
\subsubsection{Temps / aspect / mode} \label{subsubsec:TAM}
Le tangoute compte de nombreux affixes de temps / aspect / mode (TAM), mais seule une minorité est transparente étymologiquement: le suffixe de futur \tgf{1101} \ipa{.jij¹}, le suffixe de perfectif  \tgf{3916} \ipa{sji²} et le préfixe interrogatif \tgf{5981} \ipa{.a}.



Le suffixe 1101	\tgf{1101} \ipa{.jij} 1.36 s'emploie pour exprimer une action qui va se produire avec un certain degré de certitude, ou exprimer l'idée de ``s'apprêter à''. Ses fonctions ont déjà été traitées par  \citet[173-6]{kepping85}, qui contrairement à nous, considère ce morphème comme un ``mot outil'' (служебное слово) et non comme un suffixe. Nous le glosons comme ``futur'', et non pas comme irréel ou hypothétique, catégorie modale pour laquelle existent le suffixe \tgf{0734} \ipa{mo²} ou le circonfixe \tgf{3527} ...	\tgf{0734} \ipa{mja¹-...-mo²}.

On pourra consulter la phrase \ref{ex:tg:cuire} p.\pageref{ex:tg:cuire} comme exemple du sens de ``s'apprêter à''. Le passage suivant en revanche illustre l'emploi de \tgf{1101} \ipa{.jij¹} comme futur certain, où il s'emploie conjointement avec le suffixe de perfectif qui sera traité juste après :
\newline
\linebreak
\begin{tabular}{llllllllll}
	\tgf{0289}&	\tgf{3266}&	\tgf{3273}&	\tgf{1136}&	\tgf{1888}&	\tgf{1304}&	\tgf{4342}&	\tgf{2724}&	\tgf{1602}&	\tgf{1064}\\
	\tinynb{0289}&	\tinynb{3266}&	\tinynb{3273}&	\tinynb{1136}&	\tinynb{1888}&	\tinynb{1304}&	\tinynb{4342}&	\tinynb{2724}&	\tinynb{1602}&	\tinynb{1064}\\
\tgf{4464}&	\tgf{0113}&	\tgf{3916}&	\tgf{3092}&	\tgf{0113}&	\tgf{1101}&	\tgf{3916}&	\tgf{5354}&	\tgf{3583}&	\tgf{0981}\\
\tinynb{4464}&	\tinynb{0113}&	\tinynb{3916}&	\tinynb{3092}&	\tinynb{0113}&	\tinynb{1101}&	\tinynb{3916}&	\tinynb{5354}&	\tinynb{3583}&	\tinynb{0981}\\
\tgf{3063}&	\tgf{5604}&	\tgf{5113}&	\tgf{2484}&	\tgf{5285}& &&&&\\
\tinynb{3063}&	\tinynb{5604}&	\tinynb{5113}&	\tinynb{2484}&	\tinynb{5285}& &&&&\\
\end{tabular}
\begin{exe}
\ex \label{ex:tg:futur1}  \vspace{-8pt}
\gll  \ipa{.we²-dzju²}	\ipa{ɣar²}	\ipa{gu²}	\ipa{bə²lụ¹}	\ipa{dja²-tɕhju¹}	\ipa{ŋowr²}	\ipa{mjij²-ljɨ̣¹-ɕjɨj¹-sji²-djij²}	\ipa{ɕjɨj¹-.jij¹-sji²}	\ipa{thjɨ²}	\ipa{tja¹}	\ipa{.war²}	\ipa{sjịj²}	\ipa{dʑjɨ.wji¹}	\ipa{nioow¹}	\ipa{ljɨ¹} \\
		ville-seigneur poitrine dans ver \dir{}-avoir entièrement  \negat{}:\ps{}-\concessif{}-s'achever-\perf{}-\concessif{} s'achever-\fut{}-\perf{} \dem{} \topic{} objet puant  \erg{} cause \cop{} \\
\glt Préfet, il y a des vers dans votre poitrine; bien qu'ils ne se soient pas encore complètement développés, ils vont bientôt l'être. Cette (maladie) est due à ces choses répugnantes. (Leilin 06.11B.7-12A.1)
\end{exe}
Le suffixe \tgf{1101} \ipa{.jij¹} peut être rapproché avec un fort degré de probabilité du suffixe de non-passé --\ipa{jə} du tshobdun, décrit par \citet[496]{jackson03caodeng}, qui apparaît au non-passé des verbes transitifs à syllabe ouverte. On retrouve des traces du même suffixe en japhug, où il cause une série d'alternances vocaliques, comme nous l'avons montré dans \citet[234]{jacques08} (voir tableau \ref{tab:japhug.alternances.vocaliques}).
\begin{table}
\captionb{Origine de l'alternance vocalique du thème 3 en japhug}\label{tab:japhug.alternances.vocaliques}
\resizebox{\columnwidth}{!}{
\begin{tabular}{llllll} 
\toprule
thème 1 & thème 2 & \multicolumn{3}{c}{exemples} \\
\midrule
--a & --e & manger & \ipa{ndza} < *\ipapl{ndza} & \ipa{ndze} < *\ipapl{ndza-j} \\
--u & --e &cuire  & \ipa{pu} < *\ipapl{po} & \ipa{pe} < *\ipapl{po-j} \\
--\ipapl{ɯ} & --i  & acheter & \ipa{χtɯ} < *\ipapl{qtɯ} & \ipa{χti} < *\ipapl{qtɯ-j} \\
\bottomrule
\end{tabular}}
\end{table}


Le suffixe \tgf{3916} \ipa{sji²} marque une action achevée, et son usage fait l'objet d'une description détaillée dans \citet[171-3]{kepping85}. Le verbe apparaît toujours préfixé d'un directionnel de série A, sauf lorsqu'il est déjà suffixé de \tgf{1101} \ipa{.jij¹}  comme dans l'exemple \ref{ex:tg:futur1} ci-dessus. On peut consulter dans d'autres parties de cet ouvrage de nombreux exemples qui illustrent le fonctionnement de ce suffixe : \ref{ex:tg:arbreclassificateur} p.\pageref{ex:tg:arbreclassificateur}, \ref{ex:tg:moelle} p.\pageref{ex:tg:moelle}, \ref{ex:tg:toi} p.\pageref{ex:tg:toi}, \ref{ex:tg:non.actant.indicie.tuer} p.\pageref{ex:tg:non.actant.indicie.tuer}.

Il ne faut pas confondre le suffixe perfectif avec le nominalisateur \tgf{3916} \ipa{sji²} qui est représenté par le même caractère dans l'écriture. Un des moyen de les distinguer est précisément l'absence de préfixes directionnels dans le cas du second. Le suffixe nominalisateur sera traité en \ref{subsec:nmlz}.

Étymologiquement, on pourrait potentiellement comparer \tgf{3916} \ipa{sji²} avec le suffixe de passé --s du tibétain, le suffixe --s ou --t du japhug (\citealt[261]{jacques08}) et des autres langues rgyalronguiques.\footnote{\citet{huangbf97s.houzhui} cite des formes similaires dans d'autres langues ST, dont certaines sont potentiellement cognates.} Ce rapprochement n'est pas sans difficulté, car le *--s final du pré-tangoute devrait disparaître et ne laisser qu'une influence indirecte sur les voyelles en tangoute, comme nous l'avons vu dans le chapitre sur la phonologie. Pour que cette comparaison soit valable, il faudrait que le \tgf{3916} \ipa{sji²} du tangoute soit une forme paradoxalement plus archaïque que celles du tibétain ou du rgyalrong. Cela supposerait qu'en proto-macro-rgyalronguique (et même à un stade encore plus ancien comprenant l'ancêtre commun du macro-rgyalronguique et du tibétain), le suffixe de perfectif était une syllabe indépendante. En tibétain et en rgyalrong, elle s'est accolée au verbe et a subi une réduction, étant réduite à une consonne finale. Le tangoute, à l'inverse, aurait préservé tout le long ce suffixe comme syllabe indépendante.

L'explication inverse est moins probable, mais mérite néanmoins d'être considérée. La forme proto-macro-rgyalronguique aurait été un suffixe *--s, auquel une voyelle d'appui aurait été rajoutée à un certain stade du pré-tangoute. Ce type de phénomène n'est pas impossible en principe. En japhug, le cognat du suffixe de locatif --s du situ est un clitique \ipa{zɯ}. Cette forme vient nécessairement d'un proto-japhug *--s, car le *--s final devient phonologiquement /z/ en japhug, comme c'est le cas du tshobdun (\citealt{jackson05yingao}). En japhug, ce clitique est donc passé par un stade de consonne finale. Une évolution du même type pourrait être envisagée pour le \tgf{3916} \ipa{sji²} du tangoute.


Le préfixe interrogatif 5981 \tgf{5981} \ipa{.a} est écrit avec un caractère qui sert à transcrire aussi le numéral ``un'' des classificateurs et le préfixe directionnel de série A ``haut'' (voir \ref{subsec:directionnels}). Il exprime une question polaire, comme l'a montré \citet[284,316-7]{kepping85}. Il n'est pas facile de distinguer ce préfixe du directionnel dans certains cas. Il apparaît surtout avec des verbes modaux (ceux qui peuvent être préfixés de la négation \tgf{5643} \ipa{mjɨ¹--}, comme dans l'exemple suivant, déjà cité par Kepping :
\newline
\linebreak
\begin{tabular}{llllllllll}
	\tgf{4978}&	\tgf{1918}&	\tgf{2912}&	\tgf{2098}&	\tgf{1542}&	\tgf{4028}&	\tgf{1526}&	\tgf{1465}&	\tgf{5981}&	\tgf{0303}\\
	\tinynb{4978}&	\tinynb{1918}&	\tinynb{2912}&	\tinynb{2098}&	\tinynb{1542}&	\tinynb{4028}&	\tinynb{1526}&	\tinynb{1465}&	\tinynb{5981}&	\tinynb{0303}\\
\tgf{4601}& &&&&&&&&\\
\tinynb{4601}& &&&&&&&&\\
\end{tabular}
\begin{exe}
\ex \label{ex:tg:interrogatif}  \vspace{-8pt}
\gll  \ipa{tjij¹}	\ipa{mji¹-lhjwo¹-ŋa²}	\ipa{ku¹}	\ipa{nji²}	\ipa{tshji²ljịj¹}	\ipa{a-dʑjij-nja²} \\
		si \negat{}-revenir-1\sg{} alors toi servir \intrg{}-consentir-2\sg{} \\
\glt Si je ne reviens pas (de la guerre), consens-tu à t'occuper (de ma mère) ? (Cixiaozhuan, 3.6, \citealt[16]{jacques07textes})
\end{exe}
Voir la section \ref{subsec:position3} pour plus de détails sur la position relative de ce préfixe avec les autres préfixes du gabarit verbal.


L'interrogatif \tgf{5981} \ipa{.a} peut se comparer au \jpg{ɯ́--} \citet[292-3]{jacques08} ainsi qu'au préfixe e-- du tibétain classique. Toutefois, il n'existe aucune correspondance vocalique régulière entre les trois langues pour ce préfixe. Il est donc impossible de le reconstruire dans un état plus ancien.


\subsubsection{Négation} \label{subsubsec:negation}




Nous ne traiterons ici que des préfixes de négation, et pas des auxiliaires de négation  tels que \tgf{2194} \ipa{mjij¹} ``ne pas y avoir'' et \tgf{1943} \ipa{nja²} ``ne pas être'', qui sont des lexèmes à part entière. On trouve les quatre préfixes de négation suivants : \tgf{1918} \ipa{mji¹}, \tgf{5643} \ipa{mjɨ¹}, \tgf{1064} \ipa{mjij²} et \tgf{1734} \ipa{tji¹}. Nous allons dans cette section décrire un à un leur usage, puis proposer des étymologies sur la base de la comparaison avec les autres langues macro-rgyalronguiques.

\tgf{1918} \ipa{mji¹} est le plus courant de ces quatre préfixes. Il apparaît aussi bien avec des préfixes directionnels de la série A que de la série B. On trouve quelques exemples où il semble pouvoir s'employer avec des noms. Toutefois, ceux-ci sont alors employés soit dans un usage prédicatif, soit adverbial, comme l'illustre l'exemple suivant :
\newline
\linebreak
\begin{tabular}{llllllllll}
\tgf{5049} &	\tgf{4444} &	\tgf{2447} &	\tgf{0724} &	\tgf{1796} &	\tgf{3789} &	\tgf{3830} &	\tgf{5604} &	\tgf{5113} &	\tgf{1918} \\
\tinynb{5049} &	\tinynb{4444} &	\tinynb{2447} &	\tinynb{0724} &	\tinynb{1796} &	\tinynb{3789} &	\tinynb{3830} &	\tinynb{5604} &	\tinynb{5113} &	\tinynb{1918} \\
\tgf{3183} &	\tgf{4342} &	\tgf{1857} &&&&&&&\\
\tinynb{3183} &	\tinynb{4342} &	\tinynb{1857} &&&&&&&\\
\end{tabular}
\begin{exe}
\ex \label{ex:tg:neg1}  \vspace{-8pt}
\gll  \ipa{.wja¹}	\ipa{ljɨ̣¹}	\ipa{ljo²}	\ipa{njɨ²}
\ipa{tɕhjụ¹}	\ipa{phjij¹}	\ipa{njij²}	\ipa{dʑjɨ.wji¹}	\ipa{mji¹-.wo²}	\ipa{dja²-.jijr²} \\
	père et frère \pl{} Chu Ping roi \erg{} \negat{}-justice \dir{}-exécuter \\
\glt Le roi Ping de Chu avait exécuté son père et son frère de façon injuste (Leilin, 04.01B.4)
\end{exe}
On trouve également des calques du chinois composés de la négation suivie d'un nom. Ainsi, \tgf{1918}\tgf{3589}\ipa{mji¹dzjɨj¹}   ``à un moment inopportun'' (Leilin 06.10B.5), où la négation précède le nom \tgf{3589}\ipa{dzjɨj¹}  ``temps'' est clairement calquée sur le chinois \zh{不時} \textit{bùshí}. 

 \tgf{5643} \ipa{mjɨ¹} s'emploie quant à lui exclusivement avec les verbes modaux (\citealt[283]{kepping85}, et n'est jamais accompagné d'un préfixe directionnel. On peut dresser une liste exhaustive des verbes avec lesquels il apparaît :
 
 \begin{enumerate}
\item 2620 \tgf{2620} \ipa{njwi} 2.10 ``pouvoir'' 
\item 5694 \tgf{5694} \ipa{ljɨɨ} 1.29 ''pouvoir'' 
\item 1818 \tgf{1818} \ipa{khia} 2.15 ``être ferme, pouvoir''
\item 5176 \tgf{5176} \ipa{dʑioow} 2.50 ``être possible, être approprié'' (s'emploie en général avec une complétive nominalisée en \tgf{2090} \ipa{lew²}.
\item 0385 \tgf{0385} \ipa{.wjị} 2.60 ``être capable'' et son thème B 0832 \tgf{0832} \ipa{.wjọ} 2.64.
\item 1786  \tgf{1786} \ipa{.u}  2.01 ``être en mesure de réussir une tâche''
\item 0768 \tgf{0768} ? ``pouvoir atteindre'' . Le sens de ce verbe est assez spécifique, et mérite d'être illustré par des exemples :
\newline
\linebreak
\begin{tabular}{llllllllll}
\tgf{2546} &	\tgf{1139} &	\tgf{1451} &	\tgf{0467} &	\tgf{0385} &	\tgf{0433} &	\tgf{2236} &	\tgf{0594} &	\tgf{0123} &	\tgf{0705} \\
\tinynb{2546} &	\tinynb{1139} &	\tinynb{1451} &	\tinynb{0467} &	\tinynb{0385} &	\tinynb{0433} &	\tinynb{2236} &	\tinynb{0594} &	\tinynb{0123} &	\tinynb{0705} \\
\tgf{0764} &	\tgf{2402} &	\tgf{0089} &	\tgf{1196} &	\tgf{5643} &	\tgf{0768} & &&&\\
\tinynb{0764} &	\tinynb{2402} &	\tinynb{0089} &	\tinynb{1196} &	\tinynb{5643} &	\tinynb{0768} & &&&\\
\end{tabular}
\begin{exe}
\ex \label{ex:tg:neg.mod}  \vspace{-8pt}
\gll  \ipa{njạ¹}	\ipa{.jij¹}	\ipa{dzji¹tsjiir¹}	\ipa{.wjị²}	\ipa{bju¹}	\ipa{lhjɨ¹lhji²}	\ipa{dʑji}	\ipa{zjịj¹}	\ipa{rjijr¹}	\ipa{ljwɨ̣¹}	\ipa{tɕhjaa¹}	\ipa{ɕjwo¹}	\ipa{mjɨ¹}-? \\
 dieu \antierg{} magie être.capable[A] \instr{} lentement marcher quand cheval courir sur rattraper \negat{}-pouvoir.atteindre \\
\glt Grâce au fait qu'il était capable d'exercer une magie divine, quand il se déplaçait lentement, un cheval au galop ne pouvait le rattraper. (Leilin, 05.23B.6)
\end{exe}


\begin{tabular}{llllllllll}
\tgf{3354} &	\tgf{3668} &	\tgf{3368} &	\tgf{4950} &	\tgf{0756} &	\tgf{5643} &	\tgf{0768} &	\tgf{4274} &	\tgf{5241} &	\tgf{3419} \\
\tinynb{3354} &	\tinynb{3668} &	\tinynb{3368} &	\tinynb{4950} &	\tinynb{0756} &	\tinynb{5643} &	\tinynb{0768} &	\tinynb{4274} &	\tinynb{5241} &	\tinynb{3419} \\
\tgf{3622} &	\tgf{0756} &	\tgf{5643} &	\tgf{0768}  &&&&&&\\
\tinynb{3622} &	\tinynb{0756} &	\tinynb{5643} &	\tinynb{0768}  &&&&&&\\
\end{tabular}
\begin{exe}
\ex \label{ex:tg:neg.mod2}  \vspace{-8pt}
\gll  \ipa{ɣie¹}	\ipa{ljị¹}	\ipa{thjwị¹}	\ipa{rjir²}	\ipa{dʑju²}	\ipa{mjɨ¹}-?	\ipa{sow¹}	\ipa{kjwi}	\ipa{bjuu¹}	\ipa{kjɨɨr²}	\ipa{dʑju²}	\ipa{mjɨ¹}-? \\
 force cultiver jeune.homme \comit{} rencontrer \negat{}-pouvoir.atteindre mûrier cueillir respecté maison rencontrer \negat{}-pouvoir.atteindre \\
\glt   Si l'on cultive par la force de ses bras, on ne peut rencontrer un fort jeune homme, si l'on cueille des mûres, on ne peut pas rencontrer une maison noble (Leilin, 06.01B.5-6)
\end{exe}
\item 2699 \tgf{2699} \ipa{nwə} 1.27 ``savoir''
\item 3576 \tgf{3576} \ipa{sjwij} 1.36 ``être clair''
\item 0303 \tgf{0303} \ipa{dʑjij} ``consentir''
\item 1617 \tgf{1617} \ipa{kjir} 2.72 ``oser''
\item 1274 \tgf{1274} \ipa{.wo} 2.42 ``devoir (par obligation morale)''
\item  5869 \tgf{5869} \ipa{zeew}  2.41 ``supporter'' et sa variante graphique 5976 \tgf{5976}.
\item 1296 \tgf{1296} \ipa{.o} 2.42 ``avoir le temps''


\end{enumerate}
Parmi tous ces verbes, seuls \tgf{2699} \ipa{nwə¹} ``savoir'' et \tgf{1274} \ipa{.wo²} ``devoir'' peuvent s'employer avec les autres préfixes, comme le montre l'exemple suivant :
\newline
\linebreak
\begin{tabular}{llllllllll}
\tgf{5583} & \tgf{4368} &	\tgf{5878} &	\tgf{0448} &	\tgf{2983} &	\tgf{2931} &	\tgf{0705} &	\tgf{0968} &	\tgf{1183} &	\tgf{1918} \\
\tinynb{5583} & \tinynb{4368} &	\tinynb{5878} &	\tinynb{0448} &	\tinynb{2983} &	\tinynb{2931} &	\tinynb{0705} &	\tinynb{0968} &	\tinynb{1183} &	\tinynb{1918} \\
\tgf{2194} & \tgf{2699}&	\tgf{3583} 	 \\
\tinynb{2194} & \tinynb{2699}&	\tinynb{3583} 	 \\
\end{tabular}
\begin{exe}
\ex \label{ex:tg:pouvoir.mji}  \vspace{-8pt}
\gll   \ipa{tsə̣¹dwu²}	\ipa{biẹj¹}	\ipa{gjɨ²}	\ipa{.u²}	\ipa{sej¹}	\ipa{zjịj¹}	\ipa{rjur¹}	\ipa{dạ²}	\ipa{mji¹-nwə¹}	\ipa{tja¹}	\ipa{mjij¹} \\
 baguette classificateur un intérieur compter quand tout affaire \negat{}-savoir \topic{} ne.pas.y.avoir \\
\glt  Quand il comptait (pratiquait la divination) avec sa baguette,  il n'y avait rien qu'il ne sût pas. (Leilin, 05.24A.2-3)
\end{exe}


Par ailleurs, \tgf{5643} \ipa{mjɨ¹} se combine avec le suffixe \tgf{3092}  \ipa{djij²} pour former la postposition \tgf{5643}\tgf{3092} \ipa{mjɨ¹djij²}  ``à part'' qui peut s'employer après un nom comme après une proposition :
\newline
\linebreak
\begin{tabular}{llllllllll}
\tgf{2019} &	\tgf{5643} &	\tgf{3092} &	\tgf{5643} &	\tgf{0832} &	\tgf{2098} &\\
\tinynb{2019} &	\tinynb{5643} &	\tinynb{3092} &	\tinynb{5643} &	\tinynb{0832} &	\tinynb{2098} &\\
\end{tabular}
\begin{exe}
\ex \label{ex:tg:a.part}  \vspace{-8pt}
\gll  \ipa{thja¹} 	\ipa{mjɨ¹djij²} 	\ipa{mjɨ¹-.wjọ²-ŋa²} \\
	cela à.part \negat{}-pouvoir[B]-1\sg{} \\
\glt Je ne peux pas faire les autres. (littéralement : ``A part ceux-là, je ne peux pas." Les douze royaumes 133.19.3, \citealt[60,115]{solonin95}, \citealt[202]{nie02shierguo})
\end{exe}
\begin{tabular}{llllllllll}
	\tgf{4028}&	\tgf{1451}&	\tgf{0467}&	\tgf{0685}&	\tgf{0685}&	\tgf{0832}&	\tgf{4601}&	\tgf{2699}&	\tgf{2098}&	\tgf{0433}\\
\tinynb{4028}&	\tinynb{1451}&	\tinynb{0467}&	\tinynb{0685}&	\tinynb{0685}&	\tinynb{0832}&	\tinynb{4601}&	\tinynb{2699}&	\tinynb{2098}&	\tinynb{0433}\\
\tgf{0322}&	\tgf{3926}&	\tgf{5682}&	\tgf{4601}&	\tgf{5643}&	\tgf{3092}&	\tgf{4508}&	\tgf{0685}&	\tgf{4063}&	\tgf{1943}\\
\tinynb{0322}&	\tinynb{3926}&	\tinynb{5682}&	\tinynb{4601}&	\tinynb{5643}&	\tinynb{3092}&	\tinynb{4508}&	\tinynb{0685}&	\tinynb{4063}&	\tinynb{1943}\\
\end{tabular}
\begin{exe}
\ex \label{ex:tg:concessif}  \vspace{-8pt}
\gll   \ipa{nji²}	\ipa{dzji¹tsjiir¹}	\ipa{ŋạ²ŋạ²}	\ipa{.wjọ²-nja²}	\ipa{nwə¹-ŋa²}	\ipa{bju¹}	\ipa{tɕhjwo¹}	\ipa{nja²}	\ipa{kaar¹-nja²}	\ipa{mjɨ¹djij²}	\ipa{tjị¹}	\ipa{ŋạ²}	\ipa{.wier¹} \ipa{nja²} \\
 toi magie bien être.capable[B]-2\sg{} savoir-1\sg{}  \instr{} donc toi examiner-2\sg{} à.part nourriture bonne chérir[A] ne.pas.être \\
\glt Comme je sais que tu connais bien la magie, je t'ai testé, et ce n'est en aucun cas parce que je ne pouvais me résigner à (te donner) ces bonnes victuailles (que je les ai cachées). (Leilin 05.24B.6-7)
\end{exe}

\tgf{1064} \ipa{mjij²} s'emploie pour exprimer la négation du perfectif; il est en concurrence avec la combinaison de la négation simple \tgf{1918} \ipa{mji¹}  avec un préfixe directionnel. 
\newline
\linebreak
\begin{tabular}{llllllll}
	\tgf{5905}&	\tgf{1064}&	\tgf{0632}&	\tgf{5354}&	\tgf{3349}&	\tgf{5241}&	\tgf{5645}&	\tgf{2194}\\
\tinynb{5905}&	\tinynb{1064}&	\tinynb{0632}&	\tinynb{5354}&	\tinynb{3349}&	\tinynb{5241}&	\tinynb{5645}&	\tinynb{2194}\\
\end{tabular}
\begin{exe}
\ex \label{ex:tg:neg.mjij}  \vspace{-8pt}
\gll   \ipa{sji²}	\ipa{mjij²-.we¹}	\ipa{thjɨ²}	\ipa{rjijr²}	\ipa{kjwi}	\ipa{tjị²}	\ipa{mjij¹} \\
	céréales \negat{}:\ps{}-mûr cela \loc{} cueillir endroit ne.pas.y.avoir \\
\glt Lorsque les grains ne sont pas mûrs, il ne faut pas récolter. (Leilin 03.27A.1)
\end{exe}

\tgf{1734} \ipa{tji¹} est un préfixe prohibitif, qui peut comme \tgf{1918} \ipa{mji¹}  s'employer avec des préfixes directionnels :
\newline
\linebreak
\begin{tabular}{llllllllll}
	\tgf{4342}&	\tgf{1734}&	\tgf{0509}&	\tgf{4174}&	\tgf{4868}&	\tgf{0433}&	\tgf{4098}&	\tgf{2983}&	\tgf{2396}&	\tgf{0749}\\
\tinynb{4342}&	\tinynb{1734}&	\tinynb{0509}&	\tinynb{4174}&	\tinynb{4868}&	\tinynb{0433}&	\tinynb{4098}&	\tinynb{2983}&	\tinynb{2396}&	\tinynb{0749}\\
\tgf{1245}&	\tgf{2847}&	\tgf{5880}&	\tgf{3852}& &&&&&\\
\tinynb{1245}&	\tinynb{2847}&	\tinynb{5880}&	\tinynb{3852}& &&&&&\\
\end{tabular}
\begin{exe}
\ex \label{ex:tg:neg.prohib}  \vspace{-8pt}
\gll   \ipa{dja²-tji¹-thjowr²mju²}	\ipa{gjii²}	\ipa{bju¹}	\ipa{khu²}	\ipa{.u²}	\ipa{dzuu²}	\ipa{phji¹}	\ipa{.jij¹}	\ipa{.ụ²}	\ipa{ŋwu²}	\ipa{dʑjij¹} \\
	\dir{}-\prohib{}-bouger vouloir \instr{} hotte dans asseoir causer[A] soi-même porter \conj{} marcher \\
\glt Afin de ne pas la déranger, il la fit asseoir dans une hotte, et il marcha en la portant sur le dos lui-même. (Leilin 03.29A.1)
\end{exe}

La reconstruction de ces négations n'est pas un problème facile à résoudre.  Dans les autres langues sino-tibétaines, on trouve aussi différents morphèmes de négation  associés à différentes catégories de TAM. En tibétain, on compte \textit{ma--} (négation du perfectif) et \textit{mi--} (négation du présent/futur). En japhug, \ipa{mɤ--} (négation du non-passé), \ipa{mɯ--} (négation du perfectif), \ipa{mɯ́j-} (négation du constatif) et \ipa{ma--} prohibitif. En pumi on distingue aussi quatre morphèmes distincts, \ipa{ma--}/\ipa{miɛ--} (négation générale), \ipa{mí-} (négation du perfectif), \ipa{tiɛ̌--} (prohibitif) et \ipa{mə--} qui apparaît dans certaines formes complexes :
\begin{exe}
\ex \label{ex:pumi:prohib}  \vspace{-8pt}
\gll   \ipa{thɜʑɛ̂}	\ipa{χãɲí}	\ipa{tiɛ̌}	\ipa{pə̌-ku} \\
		s'il.te.plaît s'énerver  \prohib{} faire-\opt{} \\
\glt S'il te plaît, ne t'énerve pas! (Le mendiant).
\end{exe}
\begin{exe}
\ex \label{ex:pumi:neg.perf}  \vspace{-8pt}
\gll   \ipa{kí-wõ}	\ipa{mə́ɕ<y>ɛ́-mədərə}	\ipa{mí-tʂhuɜ́-mədərə} \\
		pâturage-\loc{} chercher<3A.\volit{}>-\textsc{narratif} \negat{}:\ps{}-trouver-\textsc{narratif} \\
\glt Elle la chercha sur les pâturages, mais ne la trouva pas. (Le mendiant).
\end{exe}


Il est difficile de comparer ces formes avec certitude. \tgf{1064} \ipa{mjij²} provient vraisemblablement de la même forme que la négation du pumi \ipa{mí-}, mais comme le pumi \ipapl{--ĩ}\footnote{L'opposition entre  voyelles fermées orales et nasales est neutralisée devant initiale nasale:  --\ipa{i}  note ici l'archiphonème de /i/ et de /ĩ/, phonétiquement [ĩ].} provient entre autres de *--\ipapl{aŋ} et que l'opposition entre voyelles fermées orales et nasales est neutralisée devant initiale nasale, la forme du pumi ne permet pas de savoir s'il est justifié  de reconstruire ici *\ipapl{mjej} plutôt que *\ipapl{mjaŋ} en pré-tangoute. Le verbe \tgf{2194} \ipa{mjij¹} ``ne pas y avoir'' pourrait être apparenté avec \tgf{1064} \ipa{mjij²}, mais rien n'exclut que ces deux formes aient des proto-formes distinctes.

La négation générale \tgf{1918} \ipa{mji¹} pourrait se reconstruire soit *mja soit *mje, les données externes ne permettent pas de trancher. Pour \tgf{5643} \ipa{mjɨ¹} , l'absence de forme exactement comparable dans les autres langues rend le choix d'une reconstruction encore plus difficile; *mji, *mju ou *mjvC seraient également plausibles.

La prohibitif \tgf{1734} \ipa{tji¹} en revanche, vient certainement de *tja, car les cognats en lolo-birman, en bodo-garo et en kiranti ont une voyelle ouverte. 
\begin{table}
\captionb{Préfixes de négation du tangoute}\label{tab:negation}
\begin{tabular}{llllll} 
\toprule
\multicolumn{4}{c}{tangoute} & fonction & Pré-tangoute\\
\midrule
1918 & \tgf{1918} & \ipa{mji} & 1.11 & neutre & *mja ou * mje\\
5643 & \tgf{5643} & \ipa{mjɨ} &1.30 & modal & *mji, *mju ou *mjvC \\
1064	& \tgf{1064} & \ipa{mjij}  & 2.33 &accompli & *\ipapl{mjej} ou *\ipapl{mjaŋ} \\
1734 & \tgf{1734} & \ipa{tji} & 1.11 &prohibitif & *tja\\
\bottomrule
\end{tabular}
\end{table}


\section{Morphologie verbale dérivationnelle} \label{sec:morpho.verbale.deriv}
On peut retrouver en tangoute de nombreuses traces de morphologie dérivationnelle communes avec le rgyalrong et le tibétain, mais du fait des changements phonétiques importants, ces traces sont difficiles à reconnaître sans une étude détaillée. Dans ce travail, nous allons passer en revue toutes les alternances lexicales du tangoute susceptibles d'être comparées avec des phénomènes attestés dans d'autres langues du groupe macro-rgyalronguique.


\subsection{Prénasalisation anticausative} \label{subsec:anticausatif}
La prénasalisation anticausative est un phénomène bien connu dans les langues sino-tibétaines, décrit dès les travaux de \citet{conrady1896}. Son sens originel exact n'est pas une simple forme ``intransitive'' ou passive comme on le décrit habituellement, mais bien un anticausatif, c'est à dire un verbe dont l'action se déroule spontanément sans agent; c'est la fonction que ce procédé morphologique présente en rgyalrong (\citealt{jacques12demotion}).


\begin{table}
\captionb{Anticausatif en japhug}\label{tab:anticausatif.japhug}
\begin{tabular} {lllllllllll}
\toprule
verbe de base & & verbe dérivé &&\\
\midrule
 \ipa{χtɤr} & éparpiller    &  \ipa{ʁndɤr} & s'éparpiller    & \\
 \ipa{prɤt} & couper    &  \ipa{mbrɤt} & se couper    & \\
  \ipa{kra} & faire tomber    &  \ipa{ŋgra} & tomber   & \\
    \ipa{ftʂi} & faire fondre    &  \ipa{ndʐi} & fondre   & \\
 \ipa{pɣaʁ} & retourner    &  \ipa{mbɣaʁ} &   être retourné   & \\
 \bottomrule
 \end{tabular}
\end{table}
Il faut distinguer l'anticausatif en japhug du vrai passif en \ipa{a-} < *ŋa-, qui implique un agent non exprimé. Ainsi \ipa{a-prɤt} signifie ``être coupé (par quelqu'un)'' par opposition à \ipa{mbrɤt} ``se couper (spontanément)''. La directionalité de la dérivation (de transitif à intransitif et non l'inverse) est garantie par la présence du verbe \ipa{χtɤr}, emprunt au tibétain \textit{gtor} dans cette liste: la forme anticausative  \ipa{ʁndɤr}, sans équivalent en tibétain, est le résultat de l'application d'une règle morphophonologique de prénasalisation et de voisement de la consonne initiale à la forme transitive \ipa{χtɤr}. Cet exemple démontre aussi que ce procédé de dérivation était encore productif lorsque les locuteurs des langues rgyalrong ont été en contact avec ceux du tibétain.




En tangoute, \citet{gong88alternations} a présenté une liste de paires de verbes transitifs et intransitifs liés par une alternance similaire. Ses données sont rapportées dans le tableau \ref{tab:anticausatif}.\footnote{Sa liste inclut aussi des paires nom/nom et nom/verbe qui ont été supprimées ici. Pour les paires de verbes qui présentent des cognats, nous indiquons la page où elles sont traitées.}


\begin{table}
\captionb{Paires anticausatives}\label{tab:anticausatif}
\resizebox{\columnwidth}{!}{
\begin{tabular} {lllllllllll}
\toprule
\multicolumn{4}{c}{verbe transitif} &\multicolumn{4}{c}{verbe intransitif} & \\
\midrule
 \tgf{5390}  & \ipa{phie}  & 2.08  & détacher, libérer   & \tgf{2162} &  \ipa{bie}  & 2.08 & être détaché & \pageref{ex:tg:detacher:expliquer}\\
 \tgf{3929}	 & \ipa{tɕhjwi}  & 1.10  & faire fondre &    \tgf{3956}	 & \ipa{dʑj(w)i}  & 1.10  & fondre & \pageref{ex:tg:fondre}\\
\tgf{4807}  & \ipa{khji}  & 1.11  & faire tomber &   \tgf{4930} &  \ipa{gji}  & 1.11  & tomber & \pageref{analyse:faire.tomber}\\
 \tgf{2475} &  \ipa{phia}  & 1.18  & casser &   \tgf{4314} &  \ipa{bia}  & 1.18 & se casser & \pageref{ex:tg:appeler2}\\
 \tgf{3708} &  \ipa{phja}  & 1.20  & couper &  \tgf{4459}  & \ipa{bja}  & 2.17  & se couper & \pageref{ex:tg:couper3}\\
 \tgf{1527}  & \ipa{phjaa}  & 1.23  & interdire & \tgf{2350} &  \ipa{bjaa} &  1.23  & se finir & \pageref{ex:tg:interdire}\\
 \tgf{0390} &  \ipa{khjwɨ}  & 1.30  & couper &  \tgf{5746} &  \ipa{gjwɨ}  & 1.30  & être coupé\\
 \tgf{2827}& \ipa{phej} & 1.33 & attacher & \tgf{5979}  & \ipa{bej}  & 1.33  & être attaché  \\
  \tgf{2478}& \ipa{khjij} & 1.36 & étendre, sécher & \tgf{0471}  & \ipa{gjij}  & 1.36  & être étendu  \\
  &&&&&&& être ouvert & \\
\tgf{4592}& \ipa{phjo} & 2.44 & distinguer & \tgf{3923} & \ipa{bjo} & 2.44 & être clair   & \\
 \bottomrule
 \end{tabular}}
\end{table}
On constate qu'au moins trois des paires de verbes tangoutes (``fondre'', ``tomber'', ``être coupé'') ont des équivalents exacts en japhug.

Il convient toutefois de noter qu'en tangoute, contrairement au japhug, il est possible que certaines de ces paires ne reflètent pas la prénasalisation anticausative, mais la préfixation du préfixe de passif (japhug \ipa{a-} < *ŋa): l'attrition phonologique en tangoute est telle que les deux procédés morphologiques auraient le même résultat.

Les exemples suivants illustrent l'emploi de \tgf{5979} \ipa{bej¹} et \tgf{1955} \ipa{dʑjwo¹} comme   verbes intransitifs:
\newline
\linebreak
\begin{tabular}{llllllllll}
\tgf{2098} & 	\tgf{0726} & 	\tgf{5399} & 	\tgf{3099} & 	\tgf{4456} & 	\tgf{4871} & 	\tgf{0760} & 	\tgf{5306} & 	\tgf{5447} & 	\tgf{5979} \\
\tinynb{2098} & 	\tinynb{0726} & 	\tinynb{5399} & 	\tinynb{3099} & 	\tinynb{4456} & 	\tinynb{4871} & 	\tinynb{0760} & 	\tinynb{5306} & 	\tinynb{5447} & 	\tinynb{5979} \\
\tgf{2221} & 	\tgf{4465} & 	\tgf{4675} & 	\tgf{1045} & 	\tgf{5612} & 	\tgf{5645} & 	\tgf{2194} & \\
\tinynb{2221} & 	\tinynb{4465} & 	\tinynb{4675} & 	\tinynb{1045} & 	\tinynb{5612} & 	\tinynb{5645} & 	\tinynb{2194} & \\
\end{tabular}
\begin{exe}
\ex \label{ex:tg:etre.attache}  \vspace{-8pt}
\gll   \ipa{ŋa²} 	\ipa{djị¹} 	\ipa{khju¹} 	\ipa{dʑjiij¹} 	\ipa{ljịj²} 	\ipa{ŋər¹} 	\ipa{dzjɨj²} \ipa{dzjwɨ¹} 	\ipa{do²} 	\ipa{bej¹} 	\ipa{.wə̣¹} 	\ipa{ljɨ̣¹rjijr²} 	\ipa{dạ²} 	\ipa{tshjiij¹} 	\ipa{tjị²} 	\ipa{mjij¹}  \\
	moi enfer sous se.trouver grand montagne juger empereur \loc{} être.attaché attribuer souffrances  parole dire[A] endroit ne.pas.y.avoir \\
\glt  Je suis en enfer, attaché auprès du juge Taishan, et je subis des souffrances indicibles. (Leilin, 06.22B.1)
\end{exe}

\begin{tabular}{llllllllll}
\tgf{2926} & 	\tgf{1452} & 	\tgf{2797} & 	\tgf{3087} & 	\tgf{5856} & 	\tgf{4896} & 	\tgf{2471} & 	\tgf{1326} & 	\tgf{1955} & 	\tgf{0585} \\
\tinynb{2926} & 	\tinynb{1452} & 	\tinynb{2797} & 	\tinynb{3087} & 	\tinynb{5856} & 	\tinynb{4896} & 	\tinynb{2471} & 	\tinynb{1326} & 	\tinynb{1955} & 	\tinynb{0585} \\
\tgf{5880} & 	\tgf{5183} & \\
\tinynb{5880} & 	\tinynb{5183} & \\
\end{tabular}
\begin{exe}
\ex \label{ex:tg:etre.perce}  \vspace{-8pt}
\gll  \ipa{.wju²} 	\ipa{nja¹-lho} 	\ipa{dʑjiw¹} 	\ipa{ɣa²} 	\ipa{tɕiə²} 	\ipa{bja²} 	\ipa{kjɨ¹-dʑjwo¹} 	\ipa{ɕjɨ²} 	\ipa{ŋwu²} 	\ipa{tjɨ̣j¹}  \\
intestin \dir{}-sortir taille \loc{} attacher ventre \dir{}-être.percé herbe \instr{} fermer \\
\glt  Lorsque l'intestin ressort, on l'attache à la taille, lorsque le ventre est percé, on ferme le trou avec de la paille. (Proverbes, 28B.7, \citealt[122,206]{kychanov74})
\end{exe}


Les paires suivantes pourraient aussi être des exemples de formes anticausatives, mais n'étant pas attestées de façon suffisamment claire  dans les textes, il n'est pas possible de l'affirmer avec certitude :
\begin{enumerate}
\item 5569 \tgf{5569} \ipa{khjiij} 1.39  et 1719 \tgf{1719} \ipa{gjiij} 1.39  ``s'amuser''. 
\item 2483  \tgf{2483} \ipa{phəj} 1.40 et 4809 \tgf{4809} \ipa{bəj}  1.40 ``large, étendu'' . 
\item 5848 \tgf{5848}  \ipa{tshjo}  1.51   et 5854 \tgf{5854} \ipa{dzjo}  1.51 ``étrangler''.
\item 4633 \tgf{4633}  \ipa{thwər} 1.84 et 4628 \tgf{4628}  \ipa{dwər}  2.76 ``brûler''. Le premier est certainement transitif, mais il manque une confirmation pour le second.
\item 861 \tgf{0861}  \ipa{tshjɨr} 1.86 et 528 \tgf{0528}  \ipa{dzjɨr}  1.86 ``couper''.
\item 2136 \tgf{2136}  \ipa{tɕhjow} 1.56 et 5163 \tgf{5163}  \ipa{dʑjow}  1.56 ``se séparer''. 

\end{enumerate}


D'autres formes citées par Gong semblent difficiles à considérer comme faisant partie du même type.\footnote{On a  exclu les formes à latérales initiales qui seront traitées dans la section sur le préfixe *S-- }

\begin{enumerate}

\item 2404 \tgf{2404} \ipa{dji} 2.10 et  4658	\tgf{4658} \ipa{thji}	1.11 ``boire". Sans exemple textuel, on ignore le sens de la forme \tgf{2404} \ipa{dji²}. 


\item 2719 \tgf{2719} \ipa{phə} 2.25  ``abandonner'' et 3594/3692 \tgf{3594}\tgf{3692}  \ipa{bə} 1.27, premier élément du composé \tgf{3594}\tgf{5586} \ipa{bə¹dʑjwo²} ``jeter''. Ces deux verbes sont transitifs, et l'on ne peut donc pas  supposer ici une dérivation anticausative.



\item 63/4762 \tgf{0063}/\tgf{4762} \ipa{tɕhjij}  1.35 ``lever'' et 3852 \tgf{3852} \ipa{dʑjij}  1.35 ``marcher''. Le premier verbe est effectivement transitif alors que le second est intransitif. Toutefois, \tgf{3852} \ipa{dʑjij¹} ne peut pas être considéré comme une forme anticausative. \tgf{4762} \ipa{tɕhjij¹} s'emploie avec l'objet \tgf{3990} \ipa{khjɨ¹} ``pied'' dans le sens de ``prendre la route, s'en aller'', et écrit avec le caractère \tgf{0063},  ce même verbe s'emploie avec diverses parties du corps et objets, tels que \tgf{4684} \ipa{mej¹} ``œil'' ou \tgf{4557} \ipa{zjur²} ``torche, bougie''. Le sens anticausatif d'un tel verbe serait ``être soulevé (spontanément)'', ce qui ne correspond pas au sens de \tgf{3852} \ipa{dʑjij¹}. Toutefois, on ne peut pas absolument exclure que le sens de ``marcher'' se soit développé secondairement en suivant le chemin  ``être soulevé'' > ``se lever'' > ``s'en aller'' > ``marcher''. Seule l'étude de cognats dans d'autres langues pourra permettre de déterminer si une relation étymologique existe entre \tgf{3852} \ipa{dʑjij¹} et \tgf{4762} \ipa{tɕhjij¹}.

\end{enumerate}


\subsection{Préfixe causatif *S--} \label{subsec:causatif}
Le préfixe causatif en sifflante, comme la prénasalisation anticausative, est l'un des rares procédés morphologiques à  avoir laissé des traces dans la quasi-totalité des langues de la famille sino-tibétaine. Les réflexes dans diverses langues ont été identifiés dans de nombreux travaux. Dans les langues conservatrices, ce préfixe est toujours prononcé comme une consonne distincte. En tibétain classique, il s'agit simplement du préfixe s-; en japhug, le préfixe cognat a trois allomorphes réguliers \ipa{sɯ--}, \ipa{sɯɣ--} et \ipa{z--}  et cinq irréguliers (\citealt[68-70]{jacques08}).

\subsubsection{Les traces du préfixe causatif en lolo-birman}
Dans la majorité des langues sino-tibétaines, toutefois, on ne peut que deviner indirectement l'existence de l'ancien préfixe à sifflante par des traces indirectes. Par exemple, en birman, on trouve de nombreuses paires telles que \ipa{mraŋ¹} ``haut'' / \ipa{hmraŋ¹} ``élever'' (\citealt[I,205-8]{okell69}): il est admis par tous les spécialistes que les formes aspirées ici avaient un préfixe *\ipapl{s--} ou *\ipapl{sə--} en proto-lolo-birman (\citealt[238]{bradley79}). 

Cette certitude n'existe qu'avec les initiales en sonante; les oppositions entre occlusive aspirée et sourde en birman peuvent être des traces de prénasalisation anticausative et ne peuvent pas être mécaniquement reconstruites avec un préfixe *s--. Par exemple, la paire citée par Okell \ipa{khya¹} ``faire tomber'' et \ipa{kya¹} correspond au tangoute \tgb{4807}  ``faire tomber'' / \tgb{4930} ``tomber'' et au \jpg{kra} ``faire tomber'' / \ipa{ŋgra} ``tomber''. Il est évident que dans ce cas, la forme non-aspirée vient d'une ancienne prénasalisée, tandis que l'aspirée vient d'une ancienne sourde. Il n'y a pas lieu de reconstruire un *s-- en proto-lolo-birman dans ``faire tomber''. Ni la forme  \jpg{kra} ``faire tomber''  ni son équivalent tangoute  \tgb{4807}   n'ont jamais eu de préfixe *s--, et la dérivation a eu lieu ici dans toutes les langues du verbe transitif au verbe intransitif.\footnote{Etant donné la faible productivité de l'anticausatif, une \textit{Rückbildung} semble très improbable.}



\subsubsection{Le préfixe causatif en tangoute}
En tangoute, les traces de ce préfixe sont plus nombreuses et plus claires qu'en lolo-birman. \citet{gong99jinyuanyin} a consacré un article aux traces du préfixe causatif à sifflante dans cette langue.  Il donne une liste de nombreuses paires de verbes présentant une alternance phonologique de voix non-tendue / tendue, dans laquelle le verbe à voix tendue est l'équivalent causatif de l'autre. Il mentionne également le fait que l'alternance de qualité de voix est parfois corrélée à une alternance d'aspiration:  forme de base à occlusive sourde aspirée s'opposant à une forme causative à occlusive sourde non-aspirée, ou forme de base à sonante  s'opposant à une forme causative à sonante sourde. Il souligne que ces phénomènes sont compatibles avec l'effet d'un préfixe *s-- sur l'initiale (on observe une alternance d'aspiration similaire en tibétain). Il mentionne quelques exceptions, dues selon lui au fait que l'alternance relâchée / tendue a été réanalysée à époque postérieure comme seul marqueur de dérivation causative.


Nous acceptons pleinement les conclusions de Gong concernant la formation du causatif en tangoute. Voici une liste de ces paires, tirée essentiellement de ses travaux: 


\begin{enumerate}
\item \tgb{4658} ``boire'' et \tgb{4582} ``donner à boire". Noter l'alternance d'aspiration.

\item \tgb{4614} ``boire du lait'' et \tgb{4834} ``donner du lait à boire''.

\item \tgb{4906} ``s'habiller'' et \tgb{3146} ``habiller quelqu'un''. Ces deux verbes ont une forme B en --jo:  \tgb{3686} et \tgb{0539}.

\item \tgb{4269} ``prendre'' et \tgb{1828} ``faire prendre''. On observe une alternance d'aspiration entre la forme de base et la forme causative. Ces deux verbes ont aussi des formes B en --jo : \tgb{0876} et \glt \tgb{4479}. La forme \tgb{4230} est peut-être aussi apparentée.

\item \tgb{5683} ``tendre'' et \tgb{5209} ``tendre (?)''. Il serait nécessaire d'avoir des attestations du deuxième verbe pour pouvoir affirmer qu'il s'agit là de l'équivalent causatif du premier. 

\item \tgb{4767} et  \tgb{4916} ``combattre'' et  \tgb{2487} ``envoyer à la guerre''\footnote{Dans l'original:  \tgf{4829}\tgf{4767}\tgf{0009}\tgf{0749}\tgd{4829}\tgd{4767}\tgd{0009}\tgd{0749}..} 

\item \tgb{4519} ``chanter, crier (oiseau)'' et  \tgb{4727} ``jouer d'un instrument à vent''.

\item \tgb{3078} ``être mélangé'' et   \tgb{4850} ``mélanger''. La forme au ton 2 \tgb{4786} ``mélanger'' y est aussi apparentée.

\item \tgb{0927} ``bouillir'' et  \tgb{4664} ``faire bouillir''.

\item \tgb{0703} ``se cacher (?)'' et \tgb{2854} ``cacher, recouvrir (?)''. Le sens de ces verbes est incertain sans meilleures attestations textuelles.

\item  \tgb{0676} ``partir, s'en aller''' a deux formes causatives possibles:  \tgb{3813} ``accompagner pour faire ses adieux'' et  \tgb{5974} ``relâcher''.

\item \tgb{1737} ``égal'' et  \tgb{1576} ``diviser à parts égales (?)''.

\item \tgb{5890} ``relâché'' et \tgb{2668} ``desserrer''.


\item \tgb{1231} ``pousser, s'étendre'' et \tgb{2893} ``grand'' ont pour équivalent causatif  \tgb{4808} ``élargir, étendre''.

 
\item \tgb{3791} ``bas'' et \tgb{1866} ``abaisser'' (\tgf{3791}\tgf{0749}  \tgd{3791}\tgd{0749}). On doit rapprocher de cette paire le nom \tgb{2666} ``le bas''.

\item \tgb{1475} ``fin (peu épais)'' et \tgb{1789} ``rendre fin (?)''. On manque d'attestations textuelles pour la forme causative.

\item \tgb{1890} ``haut'' et \tgb{3506} ``élever''.


\item \tgb{2194} ``négation'' et  \tgb{1064} ``négation'' ont peut-être une ancienne forme causative  \tgb{2376} qui signifie toutefois ``manquer (de)''.

\item \tgb{3852} ``marcher'' et \tgb{3844} ``effectuer''.

\item \tgb{0779} ``penché, courbé'' et  \tgb{0841} ``pencher, tordre'' (\tgf{1918}\tgf{1836}\tgf{0749}   \tgd{1918}\tgd{1836}\tgd{0749}).

\item \tgb{5911} ``loin'' et \tgb{1639} ``éloigner, destituer''.


\item \tgb{5292} ``complet'' et \tgb{1941} ``rassembler''.

\item  \tgb{3545}   ``laisser tomber, perdre'' et \tgb{1068}   ``tomber''. Cet exemple va à l'encontre de l'analyse de Gong Huangcheng de la rime 1.92 comme étant rhotacisée et suggère plutôt une voyelle tendue. \label{analyse:rime.1.92} On reconstruira donc *S-lii contre *li en pré-tangoute. Le préfixe *S-- cause ici à la fois une voix tendue et le dévoisement de l'initiale.

\item  \tgb{5893}  ``faire changer'' et  \tgb{5834}  ``changer'' (voir p.\pageref{ex:tg:se.transformer}). Ici la forme causative n'a pas de voix tendue.

\item  \tgb{5065}  ``tomber'' et \tgb{4101}  ``tomber, entrer''. Cet exemple est douteux, car l'alternance phonologique ne semble pas être corrélée à un sens causatif.


\end{enumerate}


\subsection{Préfixe causatif *p--} \label{subsec:p.caus}
\citet[45-6]{gong88alternations} cite un certain nombre de paires qui impliquent un verbe intransitif et son équivalent causatif. Nous ne citons ici que les paires pour lesquelles l'opposition de valence est certaine, soit par leurs occurrences textuelles, soit par leur définition dans le Wenhai, qui contient l'auxiliaire \tgf{0749} \ipa{phji¹} dans le cas des verbes transitifs.

\begin{table}
\captionb{Exemples du préfixe *p--}\label{tab:prefixep} 
\resizebox{\columnwidth}{!}{
\begin{tabular}{lllllllllll} 
\toprule
\multicolumn{4}{c}{verbe intransitif} &\multicolumn{4}{c}{verbe transitif} & \\
\midrule
\tgf{3259} &\ipa{dzji} & 1.11 & être calme & \tgf{3411} &\ipa{dzjwi} & 1.11 & calmer\\
\tgf{1829} &\ipa{tshja } & 1.20 & être chaud & \tgf{1825} &\ipa{tshjwa} & 1.20 & chauffer \\
\tgf{4033} &\ipa{dʑjij} & 2.32 & être froid & \tgf{0735} &\ipa{dʑjwij} & 1.35 & refroidir \\
\tgf{4186} &\ipa{khej} & 1.33 & être abondant & \tgf{1231} &\ipa{khwej} & 1.33 & prospérer, défricher \\
 \bottomrule
\end{tabular}}
\end{table}
Pour expliquer cette alternance, il est possible, selon les règles présentées page \pageref{subsubsec:preinitialep}, de reconstruire ici un préfixe causatif *p--. Plus spécifiquement, ce préfixe pourrait être rapproché du \jpg{ɣɤ--} (\citealt[74]{jacques08}) ou du tshobdun \ipa{wɐ--} (\citealt[10]{jackson06paisheng}), qui permettent de dériver les formes causatives des verbes statifs. Par exemple, à partir de \ipa{smi} ``cuit", \ipa{mpja} ``chaud" et \ipa{rɲɟi} ``long" on peut dériver les verbes \ipa{ɣɤsmi} ``cuire", \ipa{ɣɤmpja} ``chauffer" et \ipa{ɣɤrɲɟi} ``allonger". S'il dérive un verbe intransitif, \ipa{ɣɤ--} a le sens de ``devenir ... facilement'', comme par exemple \ipa{ɣɤrɲɟi} ``s'allonger facilement" ou  \ipa{ɣɤsɤmbrɯ} ``s'énerver facilement".

Le *p-- pré-tangoute que nous reconstruisons ici n'est qu'une formule abstraite, et  il n'est pas possible de  déterminer à partir des données internes du tangoute si le préfixe était *w- plutôt que *p-.   Il est possible que les préfixes dérivationnels aient subi une lénition en pré-proto-rgyalrong, et que la forme du préfixe soit *\ipapl{pɐ}-- ou  *\ipapl{wɐ}--.\footnote{Aucun préfixe dérivationnel hors du système de nominalisation n'a de forme à occlusive, et l'on peut supposer des changements phonétiques spécifiques pour ces préfixes.} Cette hypothèse pourrait permettre de rapprocher le préfixe causatif du rgyalrong et du tangoute avec celui d'autres langues (voir \citealt[132]{matisoff03} pour une liste de langues où des préfixes causatifs à labiale).

Ce préfixe pourrait être dû à la grammaticalisation d'un verbe, comme Benedict l'a suggéré. Il propose de rapprocher ce préfixe du verbe ``donner'' (\tib{sbʲin, bʲin}, \jpg{mbi}), mais il pourrait aussi bien s'agir du verbe ``faire'' (\tib{bʲed, bʲas}, \jpg{pa}). Toutefois, ces hypothèses sont difficilement démontrables.


\subsection{Préfixe dénominal *r--} \label{subsec:dénominal}


On peut poser l'existence de ce préfixe sur la base de paires telles que \tgf{3673} \ipa{ɣju¹} ``fumée'' < *C-kjo et \tgf{4323} \ipa{kjur¹} < *r-kjo ``fumer''.

On retrouve la même type d'alternance en rgyalrong, voir \citet[6]{jackson06paisheng} \citet[64-65]{jacques08}, par exemple \ipa{tɤ-spɯ} ``pus'' > \ipa{rɤspɯ} ``couler (pus)''.

\subsection{Préfixe dénominal *S--} \label{subsec:denominal-s}
\citet{gong99jinyuanyin} présente un nombre considérable d'exemples de *S- dénominal (tableau \ref{tab:prefixe-s-den}).
\begin{table}
\captionb{Exemples du préfixe *S- dénominal en tangoute}\label{tab:prefixe-s-den} 
\resizebox{\columnwidth}{!}{
\begin{tabular}{lllllllllll} 
\toprule
forme de base & sens & forme dérivée & sens \\
\midrule
\tgz{0797} &  idée  & \tgz{5212}    & discuter  \\		
\tgz{5702} &   blessure  & \tgz{5628}    &blesser   \\		
\tgz{1823} & lame   & \tgz{4688}    &  couper, percer, mordre \\		
\tgz{0181} &  trou  & \tgz{2658}    & enterrer  \\		
5382 0575 \tgf{5382}\tgf{0575} \ipa{mjɨ²rjar²}&  trace de pas  & 5733 0575  \tgf{5733}\tgf{0575}   \ipa{mjɨ²rjar²}  &  suivre à la trace  \\	
\tgz{1254} &meule    & \tgz{1270}    &  ??? \\		
 \bottomrule
\end{tabular}}
\end{table}
Hormis le dernier, tous ces exemples sont corrects, et peuvent se comparer au préfixe dénominatif \textit{sɯ--} du rgyalrong, que l'on retrouve par exemple dans \jpg{tɯ-jaʁndzu} ``doigt'' > \jpg{sɯ-jaʁndzu} ``montrer du doigt'' dérivant des verbes transitifs.


Le fait que ce processus s'applique aussi à un nom bisyllabique prouve qu'il était encore productif au moment où ce composé a été créé.
\subsection{Préfixe habilitatif *S--} \label{subsec:abilit}

Il existe dans les langues rgyalronguiques un préfixe `habilitatif' sɯ-/z- visiblement réservé aux verbes transitifs signifiant `être capable de'. Le tableau \ref{tab:prefixe-abil}  présente quelques exemples courants de ce préfixe.

\begin{table}
\captionb{Exemples du préfixe habilitatif en japhug}\label{tab:prefixe-abil} 
\resizebox{\columnwidth}{!}{

\begin{tabular}{lllllllllll} 
\toprule
forme de base & sens & forme dérivée & sens \\
\midrule
\ipa{nɤɕqa} & supporter & \ipa{z-nɤɕqa} & être capable de supporter \\
\ipa{rtoʁ} & voir, s'apercevoir & \ipa{sɯ-rtoʁ} & être capable de voir \\
\ipa{ndza} &manger & \ipa{sɯ-ndza} & pouvoir manger \\
 \bottomrule
\end{tabular}}
\end{table}
Ce préfixe, relativement productif, est synonyme avec le causatif pour la plupart des verbes. On trouve un exemple d'habilitatif fossilisé en tangoute, le verbe \tgz{0385} ``être capable'' et son thème B  \tgz{0832} *C-S-pja, dérivé de  ``faire'' \tgz{5113}, thème B \tgz{3621} *C-pja.

On trouve aussi en rgyalrong un cognat direct, comme le \jpg{spa}, lui aussi historiquement dérivé du verbe ``faire'', même si la relation n'est synchroniquement plus visible avec le \jpg{pa} ``ouvrir''. 

Ce verbe prouve que le préfixe habilitatif a dû exister en proto-macro-rgyalronguique, et constitue une innovation commune au tangoute et au rgyalrong.

\subsection{Préfixe *S- de nom déverbal} \label{subsec:S-deverbal}
\citet{gong99jinyuanyin} donne sept exemples de noms dérivés de verbes par le passage à une voyelle tendue, c'est à dire notre pré-tangoute *S- (voir tableau \ref{tab:prefixe-s-dev}).

\begin{table}
\captionb{Exemples du préfixe *S- déverbal en tangoute}\label{tab:prefixe-s-dev} 
\resizebox{\columnwidth}{!}{
\begin{tabular}{lllllllllll} 
\toprule
forme de base & sens & forme dérivée & sens \\
\midrule
\tgz{4658} & boire, manger   & \tgz{4508}    & nourriture  \\
\tgz{3177} &  glace? geler?  & \tgz{3358}    &  glace? \\
\tgz{0557} &  tromper  & \tgz{1673}    &appât   \\
\tgz{5898} & couvrir   & \tgz{2529}    & coiffe  \\
\tgz{4577} &   être éparpillé & \tgz{0692}    &  poudre (maquillage) \\
\tgz{3469} &  savoir  & \tgz{1771}    & sagesse   \\
\tgz{5918} &  mourir  & \tgz{1493}    &  mort \\
 \bottomrule
\end{tabular}}
\end{table}
Ces comparaisons semblent valables, à l'exception de \tgf{3177} \textit{kur²} / \tgf{3358} dont le sens exact comme la reconstruction phonétique sont problématiques. On peut rajouter aussi \tgz{5205} ``épée'', dérivé de \tgz{5653} ``découper'' (*C-S-kaC < *C-kaC).

On retrouve en rgyalrong un préfixe \ipa{sɤ}-- dérivant des noms obliques (lieu / moment /  instrument), voir \citealt[446]{jacques04these}, \citealt{jackson06paisheng}), comme \jpg{ʑmbri} ``jouer d'un instrument à vent'' > \jpg{ɯ-sɤ-ʑmbri} ``instrument à vent''.

Ce lien sémantique ne fonctionne ici correctement qu'avec ``appât'' (< ``objet qui sert à  tromper") et ``coiffe'' (< ``objet qui sert à  couvrir"), mais les autres exemples ont une sémantique de nominalisation objet, différente de celle observée en japhug.
\subsection{Alternances tonales}
\citet{gong88alternations} documente une série d'alternances tonales reliant des verbes et des noms. Il cite de nombreux exemples, dont nous ne présenterons ici qu'un choix limité dans le tableau \ref{tab:alternances-tonales}.

\begin{table}
\captionb{Alternances tonales}\label{tab:alternances-tonales} 
\resizebox{\columnwidth}{!}{
\begin{tabular}{lllllllllll} 
\toprule
nom & sens & verbe & sens \\
\midrule
\tgz{0520} & cavalier &  \tgz{2407} & monter à cheval \\
\tgz{1321}	& chaussure &  \tgz{4956} & porter une chaussure \\
\tgz{5170} & épaule& \tgz{2021} & porter à l'épaule \\
\midrule
 \tgz{4840} & armoire &\tgz{4732}& garder, conserver \\
\tgz{4513} & nourriture & \tgz{4517} & manger \\
\tgz{0497} & nombre & \tgz{2613} & compter \\
 \bottomrule
\end{tabular}}
\end{table}
D'autres exemples, tels que \textit{1212} ``veste de cuir'' et  \tgz{4906} ``porter un vêtement'', sont plus problématiques. En effet, le dérivé nominal normal du verbe \tgz{4906} est \tgz{5598} ``vêtement'', sans alternance tonale. \tgz{1212} n'apparaît que dans le dissyllabe \tgf{1153}\tgf{1212} \ipa{dʑjɨ¹gjwi¹}, et l'alternance tonale apparente pourrait être l'effet d'une assimilation avec le ton de la première syllabe.


Gong Hwangcherng n'a pas essayé de déterminer la directionnalité de la dérivation des paires présentées dans le tableau \ref{tab:alternances-tonales}. Pourtant, la comparaison avec les autres langues macro-rgyalronguiques permet de répondre partiellement à cette question. Les trois cas sur les quatre théoriquement possibles sont attestés:

\begin{enumerate}
\item \textbf{Un nom au ton 1 dérivant un verbe au ton 2}. En japhug, on trouve la paire \textit{tɯ-rpaʁ} ``épaule'' > \textit{mɤrpaʁ} ``porter à l'épaule'', dérivation basée sur le préfixe \textit{mɤ}-- non-productif. La forme \tgz{2021} ``porter à l'épaule'' doit donc dériver de \tgz{5170} ``épaule''. Cette dérivation doit être très ancienne, puisqu'on retrouve en lolo-birman le verbe dérivé \plb{bak^L}{661b} ``porter à l'épaule''. L'exemple avec ``chaussure'' est du même type.

\item \textbf{Un verbe au ton 2 dérivant un nom au ton 1}. Le verbe \tgz{2407} ``chevaucher''  se retrouve en \plb{dzi²}{651} et en \pumi{dzɛ̌i}. Le nom d'agent \tgz{0520} ``cavalier'' est isolé en tangoute et doit donc être un dérivé de ce verbe.\footnote{\tgz{5449} ``poser'', d'où dérive \tgz{5645} ``endroit'', pourrait être un exemple du même type, mais il est compliqué par la question du préfixe *S-- déverbal dans ce mot.}

\item \textbf{Un verbe au ton 1 dérivant un nom au ton 2}. Le verbe \tgz{4517} ``manger'' est quasi-universel en sino-tibétain (\jpg{ndza}, \tib{za, zos} etc). S'il forme de nombreux dérivés nominaux dans la plupart des langues de la famille, ceux-ci sont régulièrement refaits. Le nom \tgz{4513} ``nourriture" est donc certainement dérivé du verbe, et non l'inverse.

 

\end{enumerate}
On ne dispose pas de cas certains de nom au ton 2 dérivant un verbe au ton 1. On trouve aussi de nombreux cas, comme celui de ``vêtement'', où la dérivation s'effectue sans changement tonal.


L'interprétation historique de ces alternances est un problème épineux, d'autant que la tonogénèse du tangoute et des langues macro-rgyalronguiques est un domaine d'étude vierge. Deux possibilités doivent être prises en compte:

\begin{itemize}
\item Une alternance tonale causée par une présyllabe perdue, telle que le préfixe \textit{mɤ}-- que l'on observe en japhug. Reconstruire la forme du préfixe en question est impossible sur la base des données du tangoute, mais l'on peut affirmer qu'il ne s'agit ni d'un *r--, ni d'un *S--, ni d'un *p-- qui auraient laissé des traces phonétiques différentes. Cette alternance serait polaire, c'est à dire que le préfixe perdu inverserait la catégorie tonale étymologique du nom lors de sa conversion en verbe (et réciproquement).

\item Il peut également s'agir d'une authentique alternance tonale ancienne, comme la polarité tonale qui apparaît dans le système verbal de la plupart des langues rgyalronguiques (\citealt{jackson00sidaba}).

\end{itemize}

\subsection{Suffixes de nominalisation} \label{subsec:nmlz}
La nominalisation dans les langues rgyalrongs est marquées par une série de préfixes. La seule trace de ces préfixes en tangoute, comme nous l'avons montré plus haut en \ref{subsec:S-deverbal}, est le *S- déverbal, apparenté au préfixe de nominalisation \ipa{sɤ-} du japhug.


Les affixes productifs de nominalisation sont quant à eux récemment innovés, et apparaissent sous la forme de syllabes indépendantes.

En premier lieu, il faut noter le suffixe \tgz{3818} de nom d'agent.


Ce suffixe ou clitique peut apparaître dans une phrase relative dont le verbe est transitif et possède un objet exprimé par un nom, comme l'illustre l'exemple \ref{ex:tg:mjijr.relative}


\begin{tabular}{llllllllllll}
 \tgf{2541}  &	\tgf{4225}  &	\tgf{3818}  &	\tgf{5037}  &	\tgf{4401}  &	\tgf{2679}  &	\tgf{2373}  &\\
 \tinynb{2541}  &	\tinynb{4225}  &	\tinynb{3818}  &	\tinynb{5037}  &	\tinynb{4401}  &	\tinynb{2679}  &	\tinynb{2373}  &\\
\end{tabular}
\begin{exe}
\ex   \vspace{-8pt} \label{ex:tg:mjijr.relative}
\gll   \ipa{dzjwo²}  	\ipa{sja¹}  	\ipa{mjijr²}  	\ipa{bjɨr¹}  	\ipa{zow²}  	\ipa{njɨ²}  	\ipa{ljịj²}  
	   \\
 homme tuer \nmls{} couteau tenir arriver venir \\
\glt  Le bourreau arriva avec un sabre. (Leilin, 03.19B.2)
\end{exe}

avec nom antécédent:



\begin{tabular}{llllllllllll}
	 \tgf{1030}  &	\tgf{2946}  &	\tgf{2104}  &	\tgf{4342}  &	\tgf{5262}  &	\tgf{5163}  &	\tgf{1160}  &	\tgf{3818}  &	\tgf{2541}  &	\tgf{0497}  \\
\tinynb{1030}  &	\tinynb{2946}  &	\tinynb{2104}  &	\tinynb{4342}  &	\tinynb{5262}  &	\tinynb{5163}  &	\tinynb{1160}  &	\tinynb{3818}  &	\tinynb{2541}  &	\tinynb{0497}  \\
\tgf{0010}  &	\tgf{0478}  &	\tgf{1941}  &	\tgf{0749}  &\\
\tinynb{0010}  &	\tinynb{0478}  &	\tinynb{1941}  &	\tinynb{0749}  &\\
\end{tabular}
\begin{exe}
\ex   \vspace{-8pt}
\gll   \ipa{tɕjow¹ko¹}  	\ipa{ɕji¹}  	\ipa{dja²-ŋewr¹}  	\ipa{dʑjow¹ka²}  	\ipa{mjijr²}  	\ipa{dzjwo²}  	\ipa{ŋewr²}  	\ipa{zji²}  	\ipa{ɕioo¹dzjɨ̣²-phji¹}  	   \\
 Zhang.Gang auparavant \dir{}-troublé séparé \nmls{} homme nombreux tous rassembler-\caus{} \\
\glt  Zhang Gang rassembla ceux qui s’étaient dispersés durant les troubles. (Leilin, 03.10A.2)
\end{exe}

Une grammaticalisation similaire s'observe dans d'autres langues macro-rgyalronguiques. Ainsi, en pumi, un des suffixes de nominalisation, --\textit{mə}, se dérive de façon transparente du nom ``homme''  \textit{mə̂}. Il n'est pas clair s'il s'agit d'une grammaticalisation parallèle ou d'une innovation commune héritée d'un ancêtre commun.

Le tangoute connaît un autre suffixe de nominalisation, \tgz{3916}, de même forme que le suffixe de perfectif.



\subsection{Remarques finales}
Parmi les six préfixes dérivationnels reconstruits dans ce chapitre, quatre ont la forme *S-. Cette homonymie importante pourrait sembler être un artefact de notre reconstruction; toutefois, on doit noter que le japhug a encore bien plus de préfixes ayant la forme \textit{sɯ--} ou \textit{sɤ--} (par exemple l'antipassif et le déexpérienceur), et ceux-ci sont toujours productifs et bien distincts dans cette langue.

Les alternances tonales observées en tangoute  pourraient être la trace d'autres préfixes (d'une forme différente de *S--, *p-- ou *r--), mais cette hypothèse ne peut pas être testée tant que l'origine des tons dans les langues macro-rgyalronguiques n'a pas été élucidée.   

\section{Réduplication} 
La réduplication est un phénomène quasi-universel, et particulièrement développé dans toutes les langues autres macro-rgyalronguiques, et il n'est pas surprenant qu'il se retrouve également en tangoute.

 

La réduplication en tangoute, aussi bien dans le domaine verbal que nominal, présente deux grandes catégories d'alternances, que \citet{gong97chongdie} a décrites et analysées en détail: réduplication partielle par affaiblissement de la voyelle de la syllabe rédupliquée ou réduplication par ablaut.
\subsection{Affaiblissement} \label{subsec:redp-alt}
En premier lieu, la voyelle de première syllabe de la réduplication subit un affaiblissement: les voyelles antérieures notées --\textit{e}, --\textit{ji}, --\textit{ej} et --\textit{jij} (mais pas --\textit{eej}) deviennent des voyelles centrales --\textit{ə} ou --\textit{jɨ} selon les règles suivantes indiquées dans le tableau \ref{tab:reduplication-partielle1}.

\begin{table}
\captionb{Réduplication partielle: affaiblissement de la syllabe rédupliquée}\label{tab:reduplication-partielle1} 
\resizebox{\columnwidth}{!}{
\begin{tabular}{llll|lllllll} 
\toprule
rime &  \multicolumn{3}{c}{forme de base} &\multicolumn{4}{c}{syllabe rédupliquée} \\
  \midrule
 8&	1.8&	2.7&	 	\ipa{e}&	 28&	1.27&	2.25&	 	\ipa{ə}&	\\	
9&	1.9&	2.8&	  	\ipa{ie}&	 29&	1.28&	2.26&	 	\ipa{iə}&	\\	
10&	1.10&	2.9&	 	\ipa{ji}&	  30&	1.29&	2.27&	 	\ipa{jɨ}&	\\	
11&	1.11&	2.10&	 	\ipa{ji}&	 	31&	1.30&	2.28&	 	\ipa{jɨ}&	 \\	
12&	1.12&	2.11&	 	\ipa{ee}&	 32&	1.31&	&	 	\ipa{əə}&	 \\	 
14&	1.14&	2.12& 	\ipa{jii}&	 33&	1.32&	2.29&	  \ipa{jɨɨ}&	 \\	
34&	1.33&	2.30&	 	\ipa{ej}&	 28&	1.27&	2.25&	 	\ipa{ə}&	\\	
35&	1.34&	2.31&	 \ipa{iej}&	 29&	1.28&	2.26&	 	\ipa{iə}&	\\	
36&	1.35&	2.32&	 \ipa{jij}&  30&	1.29&	2.27&	 	\ipa{jɨ}&	\\	
37&	1.36&	2.33&	 	\ipa{jij}&	 31&	1.30&	2.28&	 	\ipa{jɨ}&	 \\	
40&	1.39&	2.35&	 	\ipa{jiij}&	 33&	1.32&	2.29&	  \ipa{jɨɨ}&	 \\	
 \bottomrule
\end{tabular}}
\end{table}
Ces règles fonctionnent de façon très stricte, sauf pour les alternances entre les rimes 30/31 d'une part et 10/11 et 36/37 d'autre part:  on trouve des cas où par exemple la rime 37 correspond à la rime 30 au lieu de la rime 31. Toutefois, les rimes 10, 30 et 36 ne sont pas distinctives respectivement de 11, 31 et 37: comme Gong Hwangcherng l'a montré, on constate une distribution complémentaire en fonction des initiales à l'intérieur de chacune de ces trois paires de rimes (voir sections \ref{subsec:voyelle.e.i} et \ref{subsec:voyelle.ej}).


Le tableau \ref{tab:reduplication-exemples1} illustre ces alternances sur la base d'exemples tirés de l'article de Gong Hwangcherng.


\begin{table}
\captionb{Exemples de réduplication partielle}\label{tab:reduplication-exemples1} 
\resizebox{\columnwidth}{!}{
\begin{tabular}{l|ll|lllllll} 
\toprule
rime &  \multicolumn{2}{c}{forme de base} &\multicolumn{2}{c}{syllabe rédupliquée} \\
  
  
  8&	 \tgz{0383}	& \tgz{3355}	& \tgf{3355}\tgf{0383} &se  battre\\


9&	 	\tgz{4533} &	\tgz{0223} & \tgf{0223}\tgf{4533} &appeler \\
 
11&	\tgz{0594} &	\tgz{2236} &	\tgf{2236}\tgf{0594} &    lentement \\
 
12&	 \tgz{1001} &	\tgz{4259} &	\tgf{4259}\tgf{1001} &  saleté   \\

14&	\tgz{3404} &	\tgz{1833} &	\tgf{1833}\tgf{3404} &   hésiter, douter \\
 
34&	  \tgz{4916} &\tgz{3559}	& \tgf{3559}\tgf{4916} & se  battre \\
35&	 \tgz{4708} &	\tgz{0612} &	\tgf{0612}\tgf{4708} &   guider \\   
 
37&	 \tgz{3469} &	\tgz{4993} &	\tgf{4993}\tgf{3469} &   se connaître \\

40&	 \tgz{1316} &	\tgz{4612} &	\tgf{4612}\tgf{1316} & rire   \\

 \bottomrule
\end{tabular}}
\end{table}


L'affaiblissement des rimes en voyelle centrale se retrouve en japhug (la voyelle de la syllabe rédupliquée est invariablement --\textit{ɯ}), mais dans cette langue toutes les voyelles sont susceptibles de subir cette alternance, pas seulement les  voyelles antérieures.

Les alternances décrites ci-dessus ne sont pas spécifiques toutefois à la réduplication. On trouve quelques cas où des syllabes à voyelles affaiblies apparaissent suivies d'autres morphèmes. Par exemple, \tgz{4669}, forme affaiblie de \tgz{3791}  ``le bas'', se retrouve dans 	\tgf{4669}\tgf{3791} \textit{bjɨ¹bji²} ``le bas'' (forme rédupliquée), mais aussi dans un composé tel que \tgf{4669}\tgf{2541} \textit{bjɨ¹dzjwo²} ``homme inférieur''. 

Ici, il ne s'agit pas de toute évidence de réduplication, mais d'une forme en composition à voyelle centralisée. Un phénomène similaire se retrouve en rgyalrong: l'état construit. Un certain nombre de mots composés (y compris d'ailleurs les noms incorporés, voir section \ref{subsec:incorporation}) subissent une série de changements vocaliques (voir tableau \ref{tab:status-constr}).



\begin{table}
\captionb{Etat construit en japhug}\label{tab:status-constr} 
\resizebox{\columnwidth}{!}{
\begin{tabular}{l|ll|lllllll} 
\toprule
  i > ɯ & \textit{si} & bois, arbre & \textit{sɯ-zbɤβ} & nœud sur un arbre \\
  
 &&&& (litt. ``le goitre de l'arbre'') \\
a > ɤ & \textit{tɯ-ɕna} & nez & \textit{ɕnɤ-ri} & corde attachée   \\
 &&&& au nez des animaux \\
o > ɤ & \textit{mbro} & cheval & \textit{mbrɤ-sno} & selle \\
u > ɤ & \textit{tʂu} & chemin & \textit{tʂɤ-ɕphɤt} & plantain (litt. ``(la plante)  \\
 &&&& qui rapièce le chemin") \\
om > ɤm & \textit{ɕom} & fer & \textit{ɕɤmɯɣdɯ} & fusil \\
oʁ > aʁ & \textit{stoʁ} & fève & \textit{staʁpɯ} & haricot  \\
 &&&&  (litt. ``le fils de la fève'')  \\
 \bottomrule
\end{tabular}}
\end{table}

L'alternance --\textit{ji} > --\textit{jɨ} en tangoute dans un cas tel que  \tgf{4669}\tgf{2541} \textit{bjɨ¹dzjwo²}  est un phénomène du même type, quoique beaucoup plus rare en tangoute qu'en rgyalrong. Il est   probable que l'apparition de l'état construit en rgyalronguique et en tangoute soit une évolution parallèle plutôt qu'un héritage commun.

\subsection{Ablaut} \label{subsec:redp-ablaut}
On observe un deuxième cas d'alternance en tangoute, moins trivial typologiquement mais mal attesté, dans lequel la syllabe rédupliquée subit un ablaut. \citet[612]{gong03tangut} a présenté une dizaine d'exemples, dont seulement deux peuvent être retenus (tableau \ref{tab:reduplication-partielle2} ).


\begin{table}
\captionb{Réduplication partielle: ablaut}\label{tab:reduplication-partielle2} 
\resizebox{\columnwidth}{!}{
\begin{tabular}{lllll|llllll} 
\toprule
\tgz{0161} &	\tgz{1124} &	\tgf{1124}\tgf{0161} &   se lamenter \\
\tgz{0139} &	\tgz{0248} &	\tgf{0248}\tgf{0139} &  calme  \\
 \bottomrule
\end{tabular}}
\end{table}

Les autres exemples qu'il cite, tels que  	\tgz{0141} / \tgz{0122}  ``faire du commerce'',  	\tgz{5208}\tgz{5086}  ``stupide'',
\tgz{4041} / \tgz{1524} ``détester'',  	\tgz{2757} / \tgz{2743}  ``entourer'', ne sont pas des cas de réduplications, mais   des alternances d'un autre type (en particulier, le marquage de la personne,  voir section \ref{subsec:personne}). Le seul argument qui pourrait soutenir l'hypothèse qu'il s'agit de réduplication est le fait qu'ils apparaissent ensemble dans le dictionnaire \textit{Tóngyīn}, mais c'est aussi le cas de toutes les formes des verbes à alternance. Nous n'avons conservé que les exemples de collocations attestées dans les textes.

Même \tgf{0248}\tgf{0139} \ipa{no²nej²} est problématique comme exemple de réduplication, car  \tgf{0248}  \ipa{no²} apparaît parfois seul dans les textes. 

L'alternance de voyelle antérieure à --o observée dans les deux exemples du tableau \ref{tab:reduplication-partielle2} rappelle toutefois fortement un phénomène que l'on observe en japhug: la réduplication en --\textit{oʁ} (voir tableau \ref{tab:reduplication-oR}).
\begin{table}
\captionb{Réduplication en --\textit{oʁ} en japhug}\label{tab:reduplication-oR} 
\resizebox{\columnwidth}{!}{
\begin{tabular}{llllll} 
\toprule
 \textit{βzdɯ} & ramasser & \textit{a-βzdoʁ-βzdɯ} & être en ordre \\
 \textit{βdi} & bien, beau & \textit{a-βdoʁ-βdi} & être en bonne santé, paisible \\
 \textit{mdi} & complet & \textit{mdoʁ-mdi} & entier, complet \\
 \bottomrule
\end{tabular}}
\end{table}
C'est un procédé productif dans cette langue, qui s'applique à des emprunts tibétains tels que \textit{bsdu} ou \textit{bde} dans le tableau ci-dessus. Comme le japhug --\textit{oʁ} correspond régulièrement au tangoute --\textit{o}, il est possible qu'il s'agisse là d'un phénomène hérité. Pour le prouver, il faudrait tout au moins trouver un exemple de dissyllabe rédupliqué commun aux deux langues. Si l'on pouvait confirmer l'existence de ce phénomène en tangoute et sa relation avec la réduplication partielle en japhug, il s'agirait d'une innovation commune non-triviale, car aucun phénomène de ce type ne semble documenté dans les autres branches du sino-tibétain.
\subsection{Fonction}
Les fonctions grammaticales de la réduplication (partielle ou totale) sont variées et seulement partiellement comprises. 

La fonction la plus claire pour les verbes dynamiques est le sens réciproque. Observons le verbe transitif  \tgz{4916} ``combattre'', qui se forme normalement avec la postposition comitative \tgz{4950}:
\newline
\linebreak
\begin{tabular}{llllllllll}
 \tgf{0866} &	\tgf{3830} &	\tgf{3540} &	\tgf{4579} &	\tgf{4689} &	\tgf{3830} &	\tgf{2074} &	\tgf{3277} &	\tgf{4950} &	\tgf{5981} \\
\tinynb{0866} &	\tinynb{3830} &	\tinynb{3540} &	\tinynb{4579} &	\tinynb{4689} &	\tinynb{3830} &	\tinynb{2074} &	\tinynb{3277} &	\tinynb{4950} &	\tinynb{5981} \\
\tgf{4916} &	\tgf{4861}& \tgf{4342} &	\tgf{5377} &	\tgf{5918} \\
\tinynb{4916} &	\tinynb{4861}& \tinynb{4342} &	\tinynb{5377} &	\tinynb{5918} \\
\end{tabular}
\begin{exe}
\ex \label{ex:tg:combattre}  \vspace{-8pt}
\gll   \ipa{ɣu¹}  	\ipa{njij²}  	\ipa{xa¹lju²}  	\ipa{.jwar¹}  	\ipa{njij²}  	\ipa{kew¹tshja²}  	\ipa{rjir²}  	\ipa{.a°-ɣwej¹}  \ipa{zjọ²}  \ipa{dja²-ŋwo²}  	\ipa{sjɨ¹}  \\
	Wu roi Helü  Yue roi Gou.Jian \comit{} \dir{}-combattre moment \dir{}-blessé mourir \\
\glt   Quand He Lü \zh{闔閭}, roi de Wu livra bataille avec Gou Jian \zh{勾踐}, roi de Yue, il fut blessé, et mourut (peu après).  (Leilin 03.20B.2)
\end{exe}

Ce verbe présente aussi une  forme réciproque rédupliquée \tgf{3559}\tgf{4916} 	\ipa{ɣwə¹ɣwej¹}, dans l'exemple suivant (déjà cité p.\pageref{ex:tg:moudre}):
\newline
\linebreak
\begin{tabular}{llllll}
	\tgf{1254}&	\tgf{0089}&	\tgf{4176}&	\tgf{4176}&	\tgf{3559}&	\tgf{4916}\\
	\tinynb{1254}&	\tinynb{0089}&	\tinynb{4176}&	\tinynb{4176}&	\tinynb{3559}&	\tinynb{4916}\\
\end{tabular}
\begin{exe}
\ex   \vspace{-8pt}
\gll   \ipa{dʑjwɨr¹}	\ipa{tɕhjaa¹}	\ipa{tju¹tju¹}	\ipa{ɣwə¹ɣwej¹} \\
		moulin sur tourterelle \recip{}:battre \\
\glt  Des tourterelles se battaient sur le moulin. (Leilin 06.08B.3)
\end{exe}

En japhug on observe aussi une réduplication dans la plupart des formes réciproques, mais le verbe est aussi précédé d'un préfixe \textit{a}-- intransitivant\footnote{Ce préfixe marque le passif quand il est employé seul.}, comme dans \textit{lɤt} ``jeter'' > \textit{a-lɯ-lɤt }``se battre'', \textit{ndza} ``manger'' > \textit{a-ndzɯ-ndza} ``se manger les uns les autres''.

La similarité apparente entre ces constructions ne prouve pas toutefois qu'elles sont héritées d'une forme commune.  L'emploi, presque iconique, de la réduplication pour indiquer le réciproque est si répandu dans les langues  qu'il pourrait s'agir d'une banale évolution parallèle.


\section{Structure du complexe verbal}
La morphologie verbale du tangoute ne peut être décrite de façon satisfaisante en établissant simplement l'inventaire des affixes et des alternances; il convient également d'analyser la structure du complexe verbal, autrement l'ordre relatif des affixes et les règles qui gouvernent à leurs interactions. La façon traditionnelle d'étudier le tangoute,  qui ne reconnaît pas le niveau du mot dans cette langue du fait du filtre qu'opère l'écriture, obscurcit complètement la structure du verbe. Le présent travail s'inscrit dans une optique radicalement différente, basée sur la connaissance des langues modernes du Sichuan et la typologie morphologique.


Dans \citet{jacques11tangut.verb}, nous avons   proposé une analyse de l'ordre des préfixes dans le complexe verbal. Le tableau \ref{tab:template-tang} représente la structure complète du verbe tangoute, et inclut également les suffixes, qui n'avaient pas été traités dans l'article sus-mentionné.

\begin{table}
\captionb{Structure  du verbe tangoute }\label{tab:template-tang} 
\resizebox{\columnwidth}{!}{
\begin{tabular}{lllllllllllll} 
\toprule
1  &2 &3 & 4 & 5 & 6 & 7 & 8&9&10\\
directionnels & \negat{} & modalité & nom incorporé & racine & auxiliaire & personne & futur& aspect & modalité \\
 
 \bottomrule
\end{tabular}}
\end{table}
Les affixes apparaissant dans les positions 1, 2, 7-10 ont  été étudiés dans la section précédente. La position 6 comprend des auxiliaires tels que le causatif \tgz{0749} ou le verbe  ``faire'' \tgz{5113}, dont l'alternance A/B  semble servir à distinguer les formes 2,1>3 des formes 3>2,1 pour les verbes non-alternants.

La structure verbale du tangoute est une morphologie  de type  gabaritique (\textit{templatic})\footnote{Nous employons ici ``gabarit'' comme traduction de l'anglais ``template''.} plutôt qu'une morphologie en  couche (\textit{layered}).   \cite[218]{bickel07inflectional} proposent les critères suivants pour caractériser la morphologie gabaritique:
\begin{enumerate}
\item ``There can be more than one root or head.''
\item ``Dependencies can obtain between non-adjacent formatives.''
\item ``Allomorphy of more inward formatives and the position of formatives in the string can be determined by their formal categories, or by phonological principles, rather than their syntactic or semantic functions.''
\end{enumerate}

Nous allons ici examiner un à un ces trois critères.

Premièrement, le verbe tangoute peut comprendre deux racines verbales, la racine lexicale accompagnée d'un auxiliaire comme dans l'exemple suivant:


\begin{tabular}{lllllllll}
	\tgf{2736}&	\tgf{3628}&	\tgf{5604}&	\tgf{5113}&	\tgf{4880}&	\tgf{4399}&	\tgf{2590}&	\tgf{5755}&	\tgf{0749}\\
	\tinynb{2736}&	\tinynb{3628}&	\tinynb{5604}&	\tinynb{5113}&	\tinynb{4880}&	\tinynb{4399}&	\tinynb{2590}&	\tinynb{5755}&	\tinynb{0749}\\
\end{tabular}
\begin{exe}
\ex \label{ex:tg:pilier}  \vspace{-8pt}
\gll   \ipa{biaa²ɣjwã¹}	\ipa{dʑjɨ.wji¹}	\ipa{rər²}	\ipa{dzji.²}	\ipa{.wjɨ²-.jar¹-phji¹} \\
		Ma.Yuan	\erg{}		bronze	pilier	\dir{}-se.lever-causer[A] \\
\glt Ma Yuan y fit ériger des piliers de bronze. (Leilin 04.34B.7)
\end{exe}
Ici un seul préfixe directionnel apparaît pour deux racines verbales. Si l'auxiliaire \tgz{0749} était un verbe indépendant, on s'attendrait soit à ce qu'il porte lui-même le préfixe directionnel et que le verbe complément, non fini, ne le porte pas, soit que les deux verbes en portent. Or, une structure du type DIR V1 DIR AUX ne semble pas attestée couramment. On en conclut que la marque du causatif n'est plus un verbe indépendant, mais plutôt une sorte de suffixe intégré au complexe verbal, comme cela s'observe dans les langues kiranties, comme en dumi (\citealt[224-7]{driem93dumi}). 


Deuxièmement, on observe bien des circonfixes tels que l'hypothétique \ipa{mja¹-}....\ipa{-mo²}, qui présente donc une dépendance non-adjacente:

\begin{tabular}{llllllllll}
	\tgf{1030}&	\tgf{2460}&	\tgf{3262}&	\tgf{1906}&	\tgf{2639}&	\tgf{2476}&	\tgf{1278}&	\tgf{2019}&	\tgf{0615}&	\tgf{1542}\\
\tinynb{1030}&	\tinynb{2460}&	\tinynb{3262}&	\tinynb{1906}&	\tinynb{2639}&	\tinynb{2476}&	\tinynb{1278}&	\tinynb{2019}&	\tinynb{0615}&	\tinynb{1542}\\
\tgf{4028}&	\tgf{1139}&	\tgf{3527}&	\tgf{4200}&	\tgf{0749}&	\tgf{4601}&	\tgf{0734}& &&\\
\tinynb{4028}&	\tinynb{1139}&	\tinynb{3527}&	\tinynb{4200}&	\tinynb{0749}&	\tinynb{4601}&	\tinynb{0734}& &&\\
\end{tabular}
\begin{exe}
\ex \label{ex:tg:causer.causatif}  \vspace{-8pt}
\gll   \ipa{tɕjow¹}	\ipa{sə¹khow¹}	\ipa{nioow¹}	\ipa{mjiij²}	\ipa{xiwa¹}	\ipa{.jɨ²}	\ipa{thja¹}	\ipa{dwewr²}	\ipa{ku¹}	\ipa{nji²}	\ipa{.jij¹}	\ipa{\textbf{mja¹}-dzow¹-phji¹-nja²-\textbf{mo²}} \\
	Zhang Sikong après nom Hua appelé cela se.rendre.compte alors toi \antierg{} \hypot{}-emprisonné-causer[A]-2sg{}-\hypot{} \\
\glt Zhang Sikong (\zh{張司空}), aussi appelé Hua (\zh{華}), s'en rendra compte et il te mettra en prison (Leilin 06.29B.7-30B.1)
\end{exe}
Le troisième critère ne semble en revanche pas vérifié, mais c'est peut-être un effet du filtre de l'écriture, qui masque peut-être certaines alternances vocaliques fines.

Outre les principes ci-dessus, il faut mentionner deux indices additionnels que la morphologie du tangoute est de type gabaritique plutôt qu'en couche.

  Le premier indice est l'ordre strict des affixes:  préfixes directionnels, préfixes modaux et négation apparaissent dans un ordre qui ne peut pas être changé. Cet ordre est partiellement arbitraire, mais est conforme pour l'essentiel à la \textit{Relevance Hierarchy} de \citet{bybee85morpho}. Selon cet auteur, les affixes ont tendance à suivre l'ordre suivant (qu'il soit préfixal ou suffixal) dans les  systèmes verbaux des langues du monde:
 
\begin{exe}
\ex  \label{bybee}
personne < mode  < temps < aspect < voix < radical verbal
\end{exe}
Le gabarit du tangoute ne contredit cette hiérarchie que sur un point: le fait que les suffixes de personnes (position 7) soient plus proches de la racine verbale que ceux d'aspect et de mode (positions 9 et 10). Parmi les préfixes, les préfixes de mode (position 3) sont plus proche de la racine verbale que les préfixes directionnels, mais ceux-ci encodent à la fois l'aspect et le mode, et ne constituent donc pas un contre-exemple. De même, parmi les suffixes, on observe bien que les suffixes d'aspect (position 9) sont plus proches de la racine que ceux de mode (position 10).


  
  Le second indice est l'impossibilité d'observer deux affixes appartenant à la même position dans la même forme verbale, par exemple deux préfixes directionnels, deux négations ou deux suffixes de personnes. Les apparents contre-exemples à ce principe ont été expliqués dans \citet{jacques11tangut.verb}, et nous reprenons ces arguments en \ref{subsec:direct-ordre}.
 
 Nous allons dans cette section nous intéresser à plusieurs propriétés du gabarit du tangoute, et étudier une à une les positions du complexe verbal.
 
  Notre travail s'achèvera sur une comparaison avec le gabarit des langues rgyalrongs, où nous tenterons de mettre au jour les parties de ces gabarits qui sont innovées indépendamment et celles qui proviennent de l'ancêtre commun aux deux langues.
  
 
\subsection{Préfixes directionnels et négation (positions 1 et 2)} \label{subsec:direct-ordre}
Contrairement aux langues rgyalrongs, les préfixes directionnels du tangoute se placent avant la négation: ce sont les premiers éléments du complexe verbal. Si l'on peut démontrer que ces éléments ne sont pas des clitiques mais des préfixes, alors on peut en conclure que tous les autres morphèmes qui se trouvent entre les directionnels et le radical du verbe sont des préfixes également.

L'ordre /directionnels+\negat{}/, étrange du point de vue rgyalrong, se retrouve toutefois dans d'autres langues de la région telles que le shixing (\citealt{chirkova09grammar}), et n'est donc pas en soi un argument contre l'idée que les directionnels sont des préfixes.

L'argument principal contre l'hypothèse que les directionnels ne seraient que des clitiques est la présence de deux séries d'affixes, comme nous l'avons vu en section \ref{subsec:directionnels}. L'existence de deux séries d'affixes doit s'expliquer par la fusion de la série B (celle dont toutes les rimes sont en --jij) avec un autre préfixe verbal. Il est peu probable qu'une telle fusion ait pu avoir eu lieu entre des morphèmes indépendants.

On trouve toutefois deux contre-exemples apparents à l'idée que les morphèmes directionnels sont des préfixes.

\subsubsection{Le morphème  \textit{kjɨ¹}}
\citet[8]{nishida02} et \citet[177-181]{linyc07spatial} ont noté que le caractère \tgz{1326} apparaît dans trois contextes où les préfixes directionnels ne se trouvent habituellement jamais.

Premièrement, il apparaît avec l'interrogatif \tgz{4435}, comme dans l'exemple suivant cité par  \citet[180]{linyc07spatial}:
\newline
\linebreak
\begin{tabular}{llllllllll}
 \tgf{4435} & 	\tgf{1326} & 	\tgf{3583} & 	\tgf{206} & 	\tgf{4435} & 	\tgf{1326} & 	\tgf{3583} & 	\tgf{3570} \\ 
\tinynb{4435} & 	\tinynb{1326} & 	\tinynb{3583} & 	\tinynb{206} & 	\tinynb{4435} & 	\tinynb{1326} & 	\tinynb{3583} & 	\tinynb{3570} \\ 
\end{tabular}
\begin{exe}
\ex    \vspace{-8pt}
\gll     \ipa{ljɨ̣¹} 	\ipa{kjɨ¹} 	\ipa{tja¹} 	\ipa{buu²} 	\ipa{ljɨ̣¹} 	\ipa{kjɨ¹} 	\ipa{tja¹} 	\ipa{dʑju²}  \\
quoi ??? \topic{} victorieux quoi ??? \topic{}  faible\\

\glt   Qu'est-ce que la victoire, et qu'est-ce que la faiblesse?\footnote{Nous n'avons pas pu consulter le texte original et traduisons sans prendre en compte le contexte.}
\end{exe}



On trouve également ce caractère apparaissant devant d'autres directionnels:
\newline
\linebreak
\begin{tabular}{llllllllll}
 \tgf{4435} & 	\tgf{1326} & 	\tgf{3583} & 	\tgf{206} & 	\tgf{4435} & 	\tgf{1326} & 	\tgf{3583} & 	\tgf{3570} \\ 
\tinynb{4435} & 	\tinynb{1326} & 	\tinynb{3583} & 	\tinynb{206} & 	\tinynb{4435} & 	\tinynb{1326} & 	\tinynb{3583} & 	\tinynb{3570} \\ 
\end{tabular}
\begin{exe}
\ex    \vspace{-8pt}
\gll     \ipa{ŋwə¹ɣạ²} 	\ipa{kja²} 	\ipa{tsəj¹} 	\ipa{kjɨ¹} 	\ipa{.wjɨ²-rar²} 	\ipa{djij²} \\
  cinquante	kalpa	petit	???	\dir{}-passer bien.que\\
\glt    Bien que cinquante petits kalpas soient déjà passés,
\end{exe}



Enfin, il apparaît aussi en fin de phrase:
\newline
\linebreak
\begin{tabular}{llllllllll}
 \tgf{2467} & 	\tgf{4751} & 	\tgf{3575} & 	\tgf{4481} & 	\tgf{1326} \\ 
\tinynb{2467} & 	\tinynb{4751} & 	\tinynb{3575} & 	\tinynb{4481} & 	\tinynb{1326} \\ 
\end{tabular}
\begin{exe}
\ex    \vspace{-8pt}
\gll       \ipa{.wjạ¹} 	\ipa{sej¹} 	\ipa{nji²} 	\ipa{ɕjɨ¹} 	\ipa{kjɨ¹}  \\
fleur pure écouter aller ??? \\
\glt     Va écouter le lotus!
\end{exe}

Rien n'indique toutefois qu'il s'agisse ici du préfixe directionnel. Le tangoute présente d'autres cas où un même caractère note deux morphèmes homophones mais sans relation étymologique: on peut en particulier mentionner \tgz{3916}, qui marque à la fois le suffixe perfectif et le nominalisateur, ou \tgz{1374}, qui sert à écrire un démonstratif en même temps qu'un préfixe modal. C'est une situation particulièrement habituelle pour les morphèmes grammaticaux.

Il est donc plus probable de considérer la forme \tgz{1326} dans les deux premières phrases comme un adverbe, comparable  à l'usage adverbial ``un peu" du  \jpg{ci} ``un'',\footnote{Dans cette optique, il pourrait s'agir d'une forme apparentée au numéral \tgz{0448} ``un''. } et le second comme une particule de fin de phrase de sens modal.  Par eux-même, ces exemples ne permettent pas de réfuter le modèle présenté dans le tableau \ref{tab:template-tang}.

\subsubsection{Doubles préfixes directionnels}
Dans les langues rgyalronguiques modernes, jamais plus d'un préfixe directionnel n'apparaît dans une forme verbale, et nous avons proposé la même règle en tangoute dans le modèle du tableau \ref{tab:template-tang}.

 En tangoute, on trouve toutefois des exemples qui sembleraient infirmer cette théorie. Les cas de \tgz{1326} apparaissant avant un autre préfixe directionnel comme nous l'avons présenté plus haut  peuvent être expliqués, mais on trouve également quelques cas comme l'exemple suivant:
\newline
\linebreak
\begin{tabular}{llllllllll}

\tgf{3045} & 	\tgf{5964} & 	\tgf{1139} & 	\tgf{1793} & 	\tgf{4797} & 	\tgf{795} & 	\tgf{2590} & 	\tgf{5113} & 	\tgf{2503} \\
\tinynb{3045} & 	\tinynb{5964} & 	\tinynb{1139} & 	\tinynb{1793} & 	\tinynb{4797} & 	\tinynb{795} & 	\tinynb{2590} & 	\tinynb{5113} & 	\tinynb{2503} \\

\end{tabular}
\begin{exe}
\ex   \vspace{-8pt}
\gll      \ipa{tshew¹tsha¹} 	\ipa{.jij¹} 	\ipa{lhji.jwɨr²} 	\ipa{rjɨr²-.wjɨ²-wji¹} 	\ipa{kụ¹}   \\
Cao.Cao \gen{} déclaration.de.guerre \dir{}-???-faire après\\ 
\glt     (Yuan Shao lui fit) écrire la déclaration de guerre contre Cao Cao. (Leilin 05.18A.6)
\end{exe}

Tous les exemples présentent la forme \tgf{0795}\tgf{2590}\tgf{5113} \ipa{rjɨr²-.wjɨ²-wji¹} avec le verbe \tgz{5113}  ``faire''. Le préfixe directionnel habituel de ce verbe est \tgz{0795} comme l'illustrent les exemples \ref{ex:tg:se.transformer} p.\pageref{ex:tg:se.transformer} et \ref{ex:tg:promesse} p.\pageref{ex:tg:promesse}.

C'est donc le caractère \tgz{2590} qui pose problème ici. Plutôt que de le considérer comme un préfixe directionnel, une solution alternative est possible: il pourrait s'agir tout simplement d'un verbe rédupliqué. L'alternance vocalique --\textit{ji} > --\textit{jɨ} est exactement celle que l'on attendrait étant données les règles de réduplication. Comme il n'y a pas par ailleurs de caractère spécial pour transcrire la syllabe rédupliquée du verbe ``faire'', cette hypothèse ne pose aucune difficulté.

On peut donc conclure qu'aucun exemple ne contredit le modèle du système verbal tel que nous l'avons décrit p.\pageref{tab:template-tang}.

\subsubsection{Le préfixe interrogatif}
Un dernier problème concernant les positions 1 et 2 est le statut du préfixe interrogatif \tgz{5981}. Il ne peut clairement pas se trouver dans la position 3, car il peut se combiner avec l'un des préfixes modaux de cette position:

\begin{tabular}{llllll}
		\tgf{3513}&	\tgf{1139}&	\tgf{2750}&	\tgf{5981}&	\tgf{1374}&	\tgf{0930}\\
		\tinynb{3513}&	\tinynb{1139}&	\tinynb{2750}&	\tinynb{5981}&	\tinynb{1374}&	\tinynb{0930}\\
\end{tabular}
\begin{exe}
\ex   \vspace{-8pt}
\gll 	\ipa{mə¹}	\ipa{.jij¹}	\ipa{ɣu¹}	\ipa{.a-tɕhjɨ¹-dju¹} \\
		ciel	\antierg{}	tête	\intrg{}-\pot{}-avoir[A] \\
\glt Le ciel a-t-il une tête? (Leilin 05.14A.1)
\end{exe}

Toutefois, il n'est pas certain qu'il soit nécessaire de supposer une position distincte pour ce préfixe: en effet, il n'apparaît pas, dans les exemples à notre disposition, en combinaison avec la négation ou avec les préfixes directionnels. Si un exemple de \tgz{5981} avec les directionnels ou les négatifs était découvert, un tel exemple impliquerait l'ajout d'une position supplémentaire au gabarit du tangoute.

\subsection{Préfixes modaux (Position 3)} \label{subsec:position3}
Trois préfixes modaux peuvent apparaître en position 3:
\begin{enumerate}

\item \tgz{1374} : potentiel
\item \tgz{4444} : concessif 
\item \tgz{5815}: fonction peu claire
\end{enumerate}

%\item \tgz{5981}: interrogatif
Ces préfixes apparaissent dans les combinaisons suivantes de préfixes, sur la base desquels notre modèle du gabarit verbal tangoute a été construit. Nous allons présenter dans cette section un inventaire des formes dans lesquelles ces préfixes sont attestés dans les textes à notre disposition. 

Des trois préfixes modaux, \tgz{1374} est de loin le plus courant. Il est attesté dans les trois combinaisons suivantes:
\begin{itemize}
\item \tgz{1374} peut se combiner avec des préfixes directionnels de la première catégorie, exprimant un mode irréel:


\begin{tabular}{lllll}
	\tgf{4028} &\tgf{2219}&\tgf{1374}&\tgf{0330}&\tgf{4601} \\
	\tinynb{4028} &\tinynb{2219}&\tinynb{1374}&\tinynb{0330}&\tinynb{4601} \\
\end{tabular}
\begin{exe}
\ex \label{ex:tg:rever}  \vspace{-8pt}
\gll   \ipa{nji²} \ipa{kjij¹-tɕhjɨ¹-mjiij¹-nja²}  \\
		toi \opt{}-\pot{}-rêver-2\sg{} \\
\glt  Aurais-tu rêvé ? (Leilin 6.16B.4)
\end{exe}

\item \tgf{5643}\tgf{1374} \ipa{mjɨ¹-tɕhjɨ¹}. Avec la négation, le préfixe  \ipa{mjɨ¹-} exprime l'impossibilité (\citealt[292-4]{kepping85}, \citealt[428]{jacques11tangut.verb}):

\begin{tabular}{llllllllll} 
	\tgf{3133}&	\tgf{0261}&	\tgf{1531}&	\tgf{1139}&	\tgf{0795}&	\tgf{0676}&	\tgf{0046}&	\tgf{5643}&	\tgf{3092}&	\tgf{2912}\\
	\tinynb{3133}&	\tinynb{0261}&	\tinynb{1531}&	\tinynb{1139}&	\tinynb{0795}&	\tinynb{0676}&	\tinynb{0046}&	\tinynb{5643}&	\tinynb{3092}&	\tinynb{2912}\\
\tgf{3092}&	\tgf{5643}&	\tgf{1374}&	\tgf{4803}&	\tgf{2098}&&&&&\\
\tinynb{3092}&	\tinynb{5643}&	\tinynb{1374}&	\tinynb{4803}&	\tinynb{2098}&&&&&\\
\end{tabular}
\begin{exe}
\ex   \vspace{-8pt}
\gll   \ipa{sjij¹}	\ipa{mjo²}	\ipa{gja¹}	\ipa{.jij¹}	\ipa{rjɨr²-.wjij¹}	\ipa{ljij²} \ipa{mjɨ¹djij²}	\ipa{lhjwo¹-djij²}	\ipa{mjɨ¹-tɕhjɨ¹-lji²-ŋa²} \\
		aujourd'hui moi armée \antierg{} \dir{}-partir voir[A] à.part revenir-\dur{} \negat{}-\pot{}-voir[B]-1\sg{} \\
\glt  Aujourd'hui je vois l'armée partir, mais je ne verrai pas son retour. (Leilin 3.16B.6-7)
\end{exe}

\item  \tgf{5981}\tgf{1374} \ipa{.a-tɕhjɨ¹}. Cette formation apparaît dans les questions rhétoriques (\citealt[317]{kepping85}):
\newline
\linebreak
\begin{tabular}{lllllllllll}
\tgf{5354} & 	\tgf{2262} & 	\tgf{3583} & 	\tgf{4098} & 	\tgf{5993} & 	\tgf{4226} & 	\tgf{3098} & 	\tgf{0749} & 	\tgf{5981}  \\
\tinynb{5354} & 	\tinynb{2262} & 	\tinynb{3583} & 	\tinynb{4098} & 	\tinynb{5993} & 	\tinynb{4226} & 	\tinynb{3098} & 	\tinynb{0749} & 	\tinynb{5981}  \\
	 	\tgf{1374}&\tgf{0303} & \\
 \tinynb{1374}& \tinynb{0303}\\ 

\end{tabular}
\begin{exe}
\ex \label{ex:tg:intrg.pot}  \vspace{-8pt}
\gll 	\ipa{thjɨ²}  \ipa{dʑjow¹}   \ipa{tja¹}  \ipa{khu²}  \ipa{kha¹}   \ipa{ljwị²}  \ipa{djɨj²-phji¹}  	\ipa{.a-tɕhjɨ¹-dʑjij} \\
		ce oiseau \topic{} cage intérieur tomber arrêter-causer[A]	\intrg{}-\pot{}-pouvoir \\
\glt Comment peut-on mettre cet  oiseau dans une cage? (Leilin 03.36B.5)
\end{exe} 


 \end{itemize}
 
Les exemples du concessif \tgz{4444} en combinaison avec d'autres préfixes sont plus rares, mais on trouve les cas suivants:
 \begin{itemize}
 
 \item   \tgz{4444} apparaît avec le préfixe négatif \tgz{5643} comme le préfixe \tgz{5981} ci-dessus, mais en revanche est utilisé avec les préfixes directionnels du groupe 1:

\begin{tabular}{llllllllllll}
  \tgf{795} & 	\tgf{4444} & 	\tgf{5113} & 	\tgf{5643} & 	\tgf{4444} & 	\tgf{5176} & 	\tgf{5981} & 	\tgf{4444} & 	\tgf{4859} & 	\tgf{5643}   \\
\tinynb{795} & 	\tinynb{4444} & 	\tinynb{5113} & 	\tinynb{5643} & 	\tinynb{4444} & 	\tinynb{5176} & 	\tinynb{5981} & 	\tinynb{4444} & 	\tinynb{4859} & 	\tinynb{5643}   \\
\tgf{4444} & 	\tgf{139} & \\
\tinynb{4444} & 	\tinynb{139} & \\
\end{tabular}
\begin{exe}
\ex    \vspace{-8pt}
\gll    \ipa{rjɨr²-ljɨ̣¹-.wji¹}	\ipa{mjɨ¹-ljɨ̣¹-dʑioow²}	\ipa{.a-ljɨ̣¹-to²}	\ipa{mjɨ¹-ljɨ̣¹-nej²}	   \\
 \dir{}-concessif-faire[A] \negat{}-concessif-pouvoir \dir{}-concessif-finir \negat{}-concessif-achever \\
\glt  Bien qu'on l'ait fait, ce n'est pas satisfaisant; bien qu'on l'ait fini, ce n'est pas achevé. (Traduction incertaine, Proverbes tangoutes 29a.3, \citealt[122,209]{kychanov74}) 
 \end{exe} 

  

\item  Préfixe directionnel+concessif+nom incorporé:  un seul exemple de ce type existe; il sera  traité en \ref{subsec:incorporation}:

\begin{tabular}{llllllllll}
  	 \tgf{2541} & 	\tgf{1139} & 	\tgf{2627} & 	\tgf{5258} & 	\tgf{2975} & 	\tgf{3744} & 	\tgf{5981} & 	\tgf{4444} & 	\tgf{2639} & 	\tgf{5113} \\
  	 \tinynb{2541} & 	\tinynb{1139} & 	\tinynb{2627} & 	\tinynb{5258} & 	\tinynb{2975} & 	\tinynb{3744} & 	\tinynb{5981} & 	\tinynb{4444} & 	\tinynb{2639} & 	\tinynb{5113} \\
\tgf{3092} & 	\tgf{1105} & 	\tgf{2090} & 	\tgf{1918} & 	\tgf{930} \\
\tinynb{3092} & 	\tinynb{1105} & 	\tinynb{2090} & 	\tinynb{1918} & 	\tinynb{930} \\
\end{tabular}
\begin{exe}
\ex    \vspace{-8pt}
\gll       \ipa{dzjwo²} 	\ipa{.jij¹} 	\ipa{ljɨ̣².iọ¹} 	\ipa{tsjiir¹dʑiej²} 	\ipa{.a-ljɨ̣¹-mjiij²-.wji¹} 	\ipa{djij²} 	\ipa{khjow¹-lew²} 	\ipa{mji¹-dju¹} \\
homme \gen{} région titre \dir{}-\concessif{}-nom-faire bien.que donner-\nmls{} \negat{}-avoir \\
\glt   Il a octroyé des titres  aux gens, mais n'avait pas de terres à leur donner.  (Sunzi, 7A.4b) 
 \end{exe} 

 
\end{itemize} 
 
 
 Le dernier préfixe modal \tgz{5815} est mal connu, car outre le fait que seuls trois exemples ont été découverts, ils proviennent tous du recueil de proverbes tangoutes, dont le sens est difficilement compréhensible, car nous ne disposons pas d'équivalent chinois ou tibétain qui permettrait de s'assurer de la validité des traductions proposées. 
 
 L'exemple suivant permet toutefois de confirmer qu'il apparaît bien après les préfixes directionnels:
 
\begin{tabular}{llllllllllll}
 \tgf{551} & 	\tgf{551} & 	\tgf{4889} & 	\tgf{1139} & 	\tgf{2518} & 	\tgf{1045} & 	\tgf{1734} & 	\tgf{5612} & 	\tgf{5444} & 	\tgf{2763}   \\
\tinynb{551} & 	\tinynb{551} & 	\tinynb{4889} & 	\tinynb{1139} & 	\tinynb{2518} & 	\tinynb{1045} & 	\tinynb{1734} & 	\tinynb{5612} & 	\tinynb{5444} & 	\tinynb{2763}   \\
\tgf{3696} & 	\tgf{5124} & 	\tgf{4092} & 	\tgf{4092} & 	\tgf{2541} & 	\tgf{1139} & 	\tgf{1014} & 	\tgf{1734} & 	\tgf{3551} & 	\tgf{4884}  \\
\tinynb{3696} & 	\tinynb{5124} & 	\tinynb{4092} & 	\tinynb{4092} & 	\tinynb{2541} & 	\tinynb{1139} & 	\tinynb{1014} & 	\tinynb{1734} & 	\tinynb{3551} & 	\tinynb{4884}  \\
\tgf{1906} & 	\tgf{2503} & 	\tgf{3513} & 	\tgf{2627} & 	\tgf{5981} & 	\tgf{5815} & 	\tgf{756} &   \\
\tinynb{1906} & 	\tinynb{2503} & 	\tinynb{3513} & 	\tinynb{2627} & 	\tinynb{5981} & 	\tinynb{5815} & 	\tinynb{756} &   \\

\end{tabular}
\begin{exe}
\ex    \vspace{-8pt}
\gll        \ipa{.jow².jow²}   	\ipa{dʑjwɨ¹}   	\ipa{.jij¹}   	\ipa{njiij¹}   	\ipa{dạ²}   	\ipa{tji¹-tshjiij¹}   	\ipa{lẹj²ɣu¹}   	\ipa{na¹rar²}   	\ipa{dja²-tsjɨ¹-kie²}   	\ipa{khie¹khie¹}   	\ipa{dzjwo²}   	\ipa{.jij¹}   	\ipa{ŋwuu¹}   	\ipa{tji¹-niow²-nji²}   	\ipa{nioow¹kụ¹}   	\ipa{mə¹}   	\ipa{ljɨ̣¹}   	\ipa{.a-tsjɨ¹-dʑju²} \\
parents réciproque \gen{} cœur parole \prohib{}-dire[A] soir lendemain \dir{}-???-haïr[A] détester homme \gen{} parole \prohib{}-mal-2\pl{} après ciel terre \dir{}-????-se.rencontrer \\
\glt Si l'on ne dit pas les paroles de son cœur aux gens de sa famille, le lendemain on sera détesté. Si l'on ne parle pas en mal des gens que l'on hait, le ciel et la terre se rencontreront.  (Traduction incertaine, Proverbes tangoutes 23a.1-2, \citealt[197]{kychanov74})
  \end{exe} 

 
 
\subsection{Incorporation (Position 4)} \label{subsec:incorporation}

Si l'on admet que les directionnels sont bien des préfixes et non des proclitiques, la conséquence logique est que tous les morphèmes qui apparaissent entre les directionnels et le radical verbal sont eux aussi des préfixes intégrés au complexe verbal. Or, on observe quelques exemples où un nom apparaît précisément dans cette position:
\newline
\linebreak
\begin{tabular}{llllllllll}
\tgf{1567} & 	\tgf{1139} & 	\tgf{1326} & 	\tgf{3266} & 	\tgf{3852} & 	\tgf{3045} & 	\tgf{5964} & 	\tgf{1139} \\ 
\tinynb{1567} & 	\tinynb{1139} & 	\tinynb{1326} & 	\tinynb{3266} & 	\tinynb{3852} & 	\tinynb{3045} & 	\tinynb{5964} & 	\tinynb{1139} \\ 
\end{tabular}
\begin{exe}
\ex    \vspace{-8pt}
\gll      \ipa{gji²} 	\ipa{.jij¹} 	\ipa{kjɨ¹-dzju²-phjo²-nja²} \\
fils \gen{} \dir{}-seigneur-envoyer[B]-2\sg{} \\
\glt      Tu as fait de ton fils le seigneur (de Zhongshan). (Leilin 03.10B.4)
\end{exe}
 
 
 \begin{tabular}{llllllllll}
 \tgf{3798} & 	\tgf{4861} & 	\tgf{4962} & 	\tgf{3830} & 	\tgf{3045} & 	\tgf{5964} & 	\tgf{1139} & 	\tgf{795} & 	\tgf{524} & 	\tgf{5522} \\ 
 \tinynb{3798} & 	\tinynb{4861} & 	\tinynb{4962} & 	\tinynb{3830} & 	\tinynb{3045} & 	\tinynb{5964} & 	\tinynb{1139} & 	\tinynb{795} & 	\tinynb{524} & 	\tinynb{5522} \\ 
\end{tabular}
\begin{exe}
\ex   \vspace{-8pt}
\gll       \ipa{tsəj¹} 	\ipa{zjọ²} 	\ipa{.we²} 	\ipa{njij²} 	\ipa{tshew¹tsha²} 	\ipa{.jij¹} 	\ipa{rjɨr²-dzju¹-ljiij²} \\
petit temps Wei roi Cao.Cao \gen{} \dir{}-ordre-attendre[A] \\
\glt       
\end{exe} Quand il était jeune, il servait Cao Cao, le roi de Wei. (Leilin 06.21B.5-6)


\begin{tabular}{llllllllll}
  	\tgf{685} & 	\tgf{685} & 	\tgf{4342} & 	\tgf{2518} & 	\tgf{1410} \\ 
    	\tinynb{685} & 	\tinynb{685} & 	\tinynb{4342} & 	\tinynb{2518} & 	\tinynb{1410} \\ 
\end{tabular}
\begin{exe}
\ex    \vspace{-8pt}
\gll       \ipa{ŋạ²ŋạ²} 	\ipa{dja²-njiij¹-ljɨ̣j²} \\
 bien \dir{}-cœur-content \\
\glt    Il se réjouit.   (Leilin 06.29A.1)
\end{exe}


\begin{tabular}{llllllllll}
  	 \tgf{2541} & 	\tgf{1139} & 	\tgf{2627} & 	\tgf{5258} & 	\tgf{2975} & 	\tgf{3744} & 	\tgf{5981} & 	\tgf{4444} & 	\tgf{2639} & 	\tgf{5113} \\
  	 \tinynb{2541} & 	\tinynb{1139} & 	\tinynb{2627} & 	\tinynb{5258} & 	\tinynb{2975} & 	\tinynb{3744} & 	\tinynb{5981} & 	\tinynb{4444} & 	\tinynb{2639} & 	\tinynb{5113} \\
\tgf{3092} & 	\tgf{1105} & 	\tgf{2090} & 	\tgf{1918} & 	\tgf{0930} \\
\tinynb{3092} & 	\tinynb{1105} & 	\tinynb{2090} & 	\tinynb{1918} & 	\tinynb{0930} \\
\end{tabular}
\begin{exe}
\ex    \vspace{-8pt}
\gll       \ipa{dzjwo²} 	\ipa{.jij¹} 	\ipa{ljɨ̣².iọ¹} 	\ipa{tsjiir¹dʑiej²} 	\ipa{.a-ljɨ̣¹-mjiij²-.wji¹} 	\ipa{djij²} 	\ipa{khjow¹-lew²} 	\ipa{mji¹-dju¹} \\
homme \gen{} région titre \dir{}-\concessif{}-nom-faire bien.que donner-\nmls{} \negat{}-avoir \\
\glt  Il a octroyé des titres  aux gens, mais n'avait pas de terres à leur donner.  (Sunzi, 7A.4b)

\end{exe}
Le dernier exemple est particulièrement important: il illustre le fait que le concessif \tgz{4444} apparaît entre le directionnel et le nom.

Nous interprétons ces phrases comme des exemples d'incorporation nominale en tangoute. Bien que ce phénomène soit très réduit en sino-tibétain, il est bien documenté en rgyalrong, comme le montrent les exemples du tableau \ref{tab:incorporation} tirés de \citet{jacques12incorp}.\footnote{Le dernier exemple incorpore un nom d'origine chinoise (\zh{票子} \textit{piàozi} ``billet'').}
\begin{table}
\captionb{Incorporation en japhug}\label{tab:incorporation}
\resizebox{\columnwidth}{!}{
\begin{tabular}{llllll} 
 \toprule
nom&	verbe&	verbe dérivé \\	
\midrule
\ipa{qhu} ``arrière'' &	\ipa{ru} ``regarder'' & 	\ipa{nɤ-qha-ru}	``tourner la tête vers l'arrière'' \\
\ipa{si} ``bois''	&\ipa{phɯt} ``couper''	& \ipa{ɣɯ-sɯ-phɯt}	``couper du bois'' \\
\ipa{pɕawtsɯ} ``argent''  	&\ipa{fsoʁ} ``accumuler'' & \ipa{ɣɯ-pɕawtsɯ-fsoʁ} ``gagner de l'argent'' \\
\bottomrule
\end{tabular}}
\end{table}

En japhug, le nom incorporé apparaît entre un préfixe dérivationnel {\ipa{nɤ--},} \ipa{ɣɯ--} ou \ipa{nɯ--} et la racine verbale, et subit habituellement les changements vocaliques caractéristiques de l'état construit (voir section \ref{subsec:redp-alt}).  Par ailleurs, les verbes incorporés sont en fait des verbes dénominaux formés à partir d'un nom déverbal comprenant une racine nominale suivi d'une racine verbale (par exemple \ipa{qharu} ``regard en arrière", \ipa{sɯphɯt} ``fait de couper du bois"): le japhug et les autres langues rgyalrongs illustrent d'ailleurs une origine jusqu'alors peu documentée des verbes à incorporation.

Rien de tel n'est apparent en tangoute: aucune marque extérieure n'indique que le nom est incorporé, si ce n'est sa place par rapport aux préfixes directionnels. Il n'est donc pas clair si ces phénomènes sont dus à un héritage du proto-macro-rgyalronguique ou à des évolutions indépendantes en rgyalronguique et en tangoute. Le caractère dérivé de l'incorporation en rgyalrong (provenant de la dérivation dénominale) suggère toutefois que ce procédé morphologique n'a guère d'antiquité.

Si l'incorporation est effectivement héritée, on devrait retrouver des cas d'incorporations non-productifs et opaques commun à au moins deux langues du macro-rgyalronguique. Aucun exemple de ce type n'a pour le moment été mis au jour: les quatre cas d'incorporations en tangoute présentés ci-dessus n'ont pas de cognats rgyalrongs. 

\subsection{Suffixes de TAM (positions 8 et 9)} \label{subsec:position8}

Les suffixes suivants, dont les fonctions et les étymologies potentielles ont été discutées en \ref{subsubsec:TAM}, apparaissent strictement après les suffixes de personne (position 7) dans la chaîne suffixale du gabarit tangoute:
 
\begin{enumerate}
 
\item \tgz{3092}: duratif
\item \tgz{3916}: perfectif
\item \tgz{1101}: futur 
\item \tgz{0734}: hypothétique
\end{enumerate}

L'exemple suivant, dont la forme verbale est particulièrement complexe, illustre le fait que le suffixe \tgz{3916} apparaît après  le futur \tgz{1101}, qui lui-même se trouve après le suffixe de personne:
\newline
\linebreak
\begin{tabular}{llllllllll}
 \tgf{1796} & 	\tgf{3830} & 	\tgf{2612} & 	\tgf{2226} & 	\tgf{3583} & 	\tgf{1204} & 	\tgf{3791} & 	\tgf{3349} & 	\tgf{1326} & 	\tgf{0134} \\
\tinynb{1796} & 	\tinynb{3830} & 	\tinynb{2612} & 	\tinynb{2226} & 	\tinynb{3583} & 	\tinynb{1204} & 	\tinynb{3791} & 	\tinynb{3349} & 	\tinynb{1326} & 	\tinynb{0134} \\
\tgf{3830} & 	\tgf{3926} & 	\tgf{3791} & 	\tgf{2226} & 	\tgf{1204} & 	\tgf{4931} & 	\tgf{3513} & 	\tgf{3349} & 	\tgf{1326} & 	\tgf{0134} \\
\tinynb{3830} & 	\tinynb{3926} & 	\tinynb{3791} & 	\tinynb{2226} & 	\tinynb{1204} & 	\tinynb{4931} & 	\tinynb{3513} & 	\tinynb{3349} & 	\tinynb{1326} & 	\tinynb{0134} \\
\tgf{3513} & 	\tgf{0645} & 	\tgf{4601} & 	\tgf{1101} &\tgf{3916} & \\
\tinynb{3513} & 	\tinynb{0645} & 	\tinynb{4601} & 	\tinynb{1101} &\tinynb{3916} & \\
\end{tabular}


\begin{exe}
\ex    \vspace{-8pt}
\gll    \ipa{tɕhjụ¹} 	\ipa{njij²} 	\ipa{phju²} 	\ipa{.we²} 	\ipa{tja¹} 	\ipa{njijr²} 	\ipa{bji²} 	\ipa{rjijr²} 	\ipa{kjɨ¹-.ju¹} 	\ipa{njij²} 	\ipa{nja²} 	\ipa{bji²} 	\ipa{.we²} 	\ipa{njijr²} 	\ipa{dʑji°} 	\ipa{mə¹} 	\ipa{rjijr²} 	\ipa{kjɨ¹-.ju¹} 	\ipa{mə¹} 	\ipa{.wu²-nja²-.jij¹-sji²}  \\
Chu roi haut faire \topic{} visage bas côté \dir{}-regarder roi toi  bas faire visage tendre ciel  côté \dir{}-regarde ciel aider-2\sg{}-\fut{}-\perf{} \\
\glt  Le roi de Chu était en haut et s'est tourné vers le bas, tandis que toi, ô Roi, tu étais en bas, et t'es tourné vers le ciel, (cela signifie que) le ciel t'aidera. (Leilin 06.15B.7)\footnote{L'absence de marque de personne dans le second verbe indique qu'il s'agit d'une forme converbale.}
\end{exe}
La position de l'hypothétique \tgz{0734} est incertaine; il est certain qu'il apparaît toujours après les suffixes de personne, mais nous ne disposons pas d'exemples permettant de déterminer son ordre relatif par rapport aux autres suffixes de TAM.


 Contrairement à l'approche de Kepping, qui considère ces morphèmes comme des particules clitiques, nous les analysons comme des affixes, car leur ordre relatif est  strict. Il est toutefois difficile d'avoir une certitude absolue sur le degré de cohésion de ces morphèmes avec le radical verbal, et la position de Kepping n'est pas intenable. La connaissance des langues modernes du Sichuan incite toutefois à considérer ces morphèmes comme  des suffixes. Il est probable  que tout comme en pumi (\citealt{jacques11pumi.tone}), des alternances tonales avaient lieu lors de la flexion et l'addition de préfixes et de suffixes, mais que l'écriture masque ces phénomènes.


 
\subsection{Comparaison avec la structure des langues rgyalrongs}
La structure gabaritique du verbe tangoute, que nous venons de décrire dans le chapitre précédent, n'est pas isolée parmi les langues de la famille sino-tibétaine. Parmi les autres langues à structure gabaritique, on peut compter notamment les langues kiranties (citées comme telles dans \citealt[218]{bickel07inflectional}) et les langues non-tibétaines du Sichuan qui présentent un système d'accord, en particulier rgyalronguiques.

Dans cette section, nous allons  comparer le système du tangoute avec celui d'une langue rgyalronguique, le japhug. Une analyse exhaustive de l'histoire du complexe verbal tangoute demanderait normalement de prendre en compte l'ensemble des données des langues sino-tibétaines à morphologie complexe, ou tout au moins des langues modernes du Sichuan. Nous décidons de nous limiter toutefois à la comparaison avec une seule langue, car pour la quasi-totalité des langues du Sichuan, même si des descriptions existent, elle ne présentent pas la structure du système verbal sous forme de gabarit, et il est difficile de reconstituer celui-ci sur la base des données publiées. La question qui va nous intéresser ici sera donc de savoir dans quelle mesure le gabarit du tangoute et celui du japhug sont innovés indépendamment ou présentent des propriétés communes qui suggèrent un héritage de leur ancêtre commun.

Nous allons tout d'abord présenter la structure du verbe japhug, puis effectuer une comparaison typologique et historique avec celui du tangoute.

\subsubsection{Le système verbal du japhug}

Le gabarit du verbe japhug peut se décrire de la façon suivante (\citealt{jacques12incorp}):


 
   \begin{landscape}

\begin{table}
\caption{Le gabarit verbal du japhug (les préfixes dérivationnels sont indiqués en grisé)}\label{tab:template:derivational}
   \resizebox{\columnwidth}{!}{
\begin{tabular}{llllll|llllllll|lllll} \toprule
 
\ipab{a-}  &  	\ipab{mɯ- }   &  	\ipab{ɕɯ-}   &\ipab{tɤ-} &  	\ipab{tɯ-}  &  	\ipab{wɣ-}   &

  	 \grise{\ipab{ʑɣɤ-}}  &  	\grise{\ipab{sɯ-}}  & \grise{\ipab{rɤ-}}& \grise{\ipab{nɤ-}} &   	 \grise{\ipab{a-}}   &  	\grise{\ipab{nɯ-}}  &  	\grise{\ipab{ɣɤ-}}  &  	\grise{\ipab{noun}}    &  	 \begin{math}\Sigma\end{math}    &  	\ipab{-t}  &  	\ipab{-a}  &  	\ipab{-nɯ}   &  \\
   &  	\ipab{mɤ-}   &  	\ipab{ɣɯ-}   &\ipab{pɯ-}&  	  &  	 
    & \grise{ }	  &  	 \grise{ }	  &  	  \grise{ }	  &  	   \grise{ }	&  	\grise{\ipab{sɤ-}}&  \grise{ }	 &  	\grise{\ipab{rɯ-}}  &  	 \grise{ }	  &  	  &  	  &  	  &  	\ipab{-ndʑi} &  \\
  &  	   &     &  etc.	  & & 	  &  	  &  	 & &  	  &  	 & &  etc.	  &  	  &  	  &  	  &  	  &  	  &  \\
1  &  	2  &  	3  &  	4  &  	5  &  	6  &  	7  &  	8  &  	9  &  	10  &  	11  &  	12  &  	13  &  	14  &  	15  & 16 &17&18\\
\bottomrule
\end{tabular}}
\end{table}
\begin{multicols}{2}
{\footnotesize
\begin{enumerate}



\item Irréel  \ipa{a}-- et \ipa{ɯβrɤ}--, Interrogatif \ipa{ɯ́}--, conatif \ipa{jɯ}--
\item Négation \ipa{ma}-- / \ipa{mɤ}-- / \ipa{mɯ}-- / \ipa{mɯ́j}--
\item Mouvement associé \ipa{ɕɯ}-- et \ipa{ɣɯ}--}
\item Préfixes directionnels (tɤ- pɯ- lɤ- thɯ- kɤ- nɯ- jɤ-, tu- pjɯ- lu- chɯ- ku- ɲɯ- ju-) permansif \ipa{nɯ}--, appréhensif \ipa{ɕɯ}--
\item Seconde personne (\ipa{tɯ}--, \ipa{kɯ}-- 2>1 et \ipa{ta}-- 1>2)
\item Inverse -\ipa{wɣ}- / Générique S/O  \ipa{kɯ}-. Dans certains dialectes du japhug (Gsar-rdzong, Da-tshang), le préfixe direct d'aoriste -a- apparaît dans cette position.
\item Réfléchi \ipa{ʑɣɤ}-- 
\item Causatif \ipa{sɯ}--, habilitatif \ipa{sɯ}--
\item  Antipassif  \ipa{sɤ}-- / \ipa{rɤ}--
\item Causatif sɯ-/z-/sɯɣ-/ɕɯ-/ɕ-/ɕɯɣ-/ʑ-/ɣɤ-, tropatif \ipa{nɤ}--, applicatif \ipa{nɯ}--
\item Passif \ipa{a}-- / Déexperienceur \ipa{sɤ}--
\item Autobénéfactif-spontané (apparaît dans certains cas entre les positions 6 et 7) \ipa{nɯ}--
\item Autres préfixes dérivationnels \ipa{nɯ}-- \ipa{ɣɯ}-- \ipa{rɯ}-- \ipa{nɤ}-- \ipa{ɣɤ}-- \ipa{rɤ}--
\item Nom incorporé  
\item Racine verbale
\item Suffixe de passé 1sg/2sg>3 --\ipa{t}  
\item 1sg --\ipa{a}\footnote{L'idée que la première personne constitue dans les langues rgyalrong une position morphologique distincte des autres suffixes de personne provient de \citet{gongxun12}. }
\item Autres suffixes d'accord (--\ipa{tɕi}, --\ipa{ji}, --\ipa{nɯ}, --\ipa{ndʑi})
\end{enumerate}
}\end{multicols} \end{landscape}
 

 Les cinq propriétés suivantes suggèrent que la morphologie du verbe japhug est une morphologie gabaritique plutôt qu'une morphologie en couches (\citealt{jacques13harmonization}):
 \begin{enumerate}
 
\item Le préfixe autobénéfactif/spontané  apparaît dans une position différente selon que la position 10 est ou non occupée et dans les formes à racines discontinues (voir ci-dessous).
\item La récursion des affixes est impossible (jamais deux préfixes directionnels ou deux négations).
\item L'ordre des préfixes les uns par rapport aux autres ne peut être changé pour obtenir une différence de portée sémantique.
\item On observe des relations de dépendance entre morphèmes non-contigus (par exemple, le passé -t avec les préfixes directionnels d'un certain type).
\item Incompatibilité fonctionnellement arbitraire entre préfixes appartenant à la même position (par exemple, appréhensif et préfixes directionnels).
\end{enumerate}

Nous avons déjà montré au début de ce chapitre que le système verbal du tangoute était lui aussi du type gabaritique. Des cinq critères ci-dessus, le tangoute en partage quatre (2-5) avec le japhug. Toutefois, les morphèmes en question ne sont pas tous apparentés entre les deux langues, et il s'agit d'une ressemblance typologique, qui n'est pas nécessairement corrélée à un héritage commun. Il convient donc de rechercher des propriétés typologiques plus spécifiques qui seraient exclusivement partagées par le tangoute et le rgyalrong. 

Le système verbal du japhug présente quatre propriétés typologiquement inhabituelles; nous allons ici exposer les particularités du japhug, puis effectuer une comparaison avec le tangoute.

Premièrement, alors que le japhug est une langue strictement verbe-final, le verbe est majoritairement \textit{préfixant}. On remarque treize positions pour les préfixes (dont les six premières relèvent de la morphologie flexionnelle, et les suivants de la morphologie dérivationnelle) pour seulement trois positions suffixales. Cette typologie va à l'encontre d'un des universels proposés par \citet{lehmann73structural} et  \citet{vennemann74analogy},\footnote{Mais inspiré par l'universel 27 de  \citet[93]{greenberg66}:  \textit{If a language is exclusively suffixing, it is postpositional; if it is exclusively
prefixing, it is prepositional.}   } selon lesquels  ``[...] VSO languages are characterized by prefixation and OV languages by suffixation'' (\citealt[23]{lehmann78typology})

 Selon cette idée influente, une corrélation existe entre l'ordre OV/VO et le fait qu'une langue sont majoritairement préfixante ou suffixante. Selon cette idée, les langues OV sont généralement suffixantes, avec peu ou pas de préfixes. Cette idée a été reprise par \citet[227]{hawkins88prefixing}, qui propose le principe d'ordre des têtes (\textit{Head Ordering Principle}) selon lequel:
 
\begin{exe}
\ex \label{ex:hawkins}
\glt \textbf{The Head Ordering Principle}: The affixal head of a word is ordered on the same side of its
subcategorized modifier(s) as P is ordered relative to NP within PP, and as V is ordered relative to a direct object NP.
\end{exe}

Il est manifeste que les langues ayant un ordre OV strict sont rarement préfixantes: si l'on combine les chapitres 26 (\citealt{dryer11chapter26}) et 83 (\citealt{dryer11ov}) du WALS, on observe que sur 945 langues, seules 6 sont à la fois OV et ``strongly prefixing" (fortement préfixantes).\footnote{Voir \citet{jacques13harmonization}.} Il s'agit en particulier des langues athabasques, du séri et d'une langue kuki-chin.\footnote{Cette dernière devrait plutôt se classer dans le type ``Weakly prefixing" de Dryer.} Le japhug et les autres langues rgyalrongs appartiennent également à ce type extrêmement inhabituel.

Deuxièmement, on observe quelques cas de \textit{racines discontinues}, où le préfixe autobénéfactif/spontané apparaît à l'intérieur de la racine:

\begin{exe}
\ex 
\gll \ipa{ɯʑo}	\ipa{tɤ-azgrɯ}	\ipa{nɤ},	\ipa{ɯ-taʁ}	\ipa{tɕheme}	\ipa{nɯ}	\ipa{kɤ-a<nɯ>mdzɯ}	\ipa{nɤ}\\
il \aor{}-se.courber \textsc{cnj} 3\sg{}-sur fille \topic{} \aor{}-<\textsc{autoben}>s'asseoir \textsc{conj} \\
\glt Il se courba, et la fille s'assit sur lui. (Kubzang 76)
\end{exe}

Il s'agit toujours de cas de racines bipartites dont le premier élément \textit{a-} était étymologiquement un préfixe intransitivant (correspondant au \textit{a-} passif en japhug). Synchroniquement toutefois, dans le verbe \textit{amdzɯ} ``s'asseoir'', le premier élément n'est pas analysable, et on ne trouve pas de forme *mdzɯ seule.
 
 
 Troisièmement, on observe une particularité  qui va à l'encontre de la \textit{Relevance Hierarchy} de  \citet{bybee85morpho}:
 
\begin{exe}
\ex  
\glt personne < mode  < temps < aspect < voix < radical verbal
\end{exe}
 
On remarque en effet qu'une marque personnelle (la seconde personne, position 5) apparaît plus près de la racine que des marqueurs d'aspect (position 5) ou de mode (position 1).  Le même phénomène se retrouve dans le verbe ket et athabasque (\citealt[180-245]{rice2000scope}). 

Quatrièmement, on note dans cette langue la présence d'incorporation nominale. Bien que l'incorporation soit relativement courante dans certaines régions du monde, en particulier l'Amérique de nord, ce n'est pas le cas en Asie, surtout parmi les langues sino-tibétaines. Il est clair que l'incorporation est une innovation récente en japhug (cf \citealt{jacques12incorp}), mais il est concevable qu'il s'agisse d'une innovation commune avec le tangoute.


  
 

Le tangoute ne partage pas toutes ces propriétés typologiques avec le japhug; seules les deux dernières (positions des marques de personne et incorporation) se retrouvent dans les deux langues.

Premièrement, contrairement au japhug, le verbe tangoute n'est pas majoritairement préfixant (voir tableau \ref{tab:template-tang}).


\begin{table}
\captionb{Structure  du verbe tangoute }\label{tab:template-tang} 
\resizebox{\columnwidth}{!}{
\begin{tabular}{lllllllllllll} 
\toprule
1  &2 &3 & 4 & 5 & 6 & 7 & 8&9&10\\
directionnels & \negat{} & modalité & nom incorporé & racine & auxiliaire & personne & futur& aspect & modalité \\
 
 \bottomrule
\end{tabular}}
\end{table}
Ce schéma comprend quatre positions préfixales et quatre positions suffixales. Le fait d'observer un nombre égal de préfixe et de suffixes n'est pas commun pour une langue à verbe final comme le tangoute, mais considérablement moins inhabituel que la proportion 13:3 observée en japhug. 

Deuxièmement, on ne constate pas de racines discontinues en tangoute à proprement parler, même si l'on détecte certains infixes non productifs (provenant du préfixes, tels que le causatif -w- étudié en \ref{subsec:p.caus}).


Troisièmement, on observe bien que les marques de personnes sont plus proches de la racine que les marque de TAM en japhug et en tangoute. Toutefois, dans le premier cas il s'agit de préfixes, tandis que dans le second ce sont des suffixes. La ressemblence est donc là encore purement typologique.

Quatrièmement, la présence d'incorporation dans les deux langues est une particularité inhabituelle intéressante. Pour prouver que les deux système d'incorporation sont apparentés, il faudra découvrir des verbes à incorporation cognats dans les deux langues, ayant si possible subi une évolution sémantique non triviale de façon indépendante. Malheureusement, les exemples d'incorporation sont très rares en tangoute. 

Si l'on constate donc que les systèmes verbaux du japhug et du tangoute présentent certains traits typologiques communs, on observe peu de phénomènes  suffisamment spécifiques pour ne pouvoir être analysables comme  des évolutions parallèles, en dehors des traces de morphologie non-productive décrites en \ref{sec:morpho.verbale.flex} et en \ref{sec:morpho.verbale.deriv}.

Afin d'expliquer maintenant les divergences entre les systèmes verbaux du tangoute et du japhug, nous allons nous intéresser aux traits de ces systèmes verbaux dont on peut juger qu'il s'agit d'innovations relativement récentes. Nous traiterons aussi des innovations parallèles qui ont eu lieu entre les deux langues.

\subsubsection{Innovations du japhug} \label{subsubsec:innovations.japhug}

La structure complexe du verbe japhug n'est pas héritée en ligne droite du proto-sino-tibétain, et même si certains indices laissent à penser qu'une partie de cette morphologie pourrait être ancienne, il est clair qu'une grande partie  des positions de la chaîne préfixale sont des innovations.  

On peut partiellement retracer l'origine de certains affixes du complexe verbal, en particulier quatre des positions préfixales: 2, 3, 4 et 7. Le cas des négations (position 2) et les préfixes directionnels (position 4), déjà abordé dans \ref{subsubsec:negation} et \ref{subsec:direct-ordre}. Nous aborderont spécifiquement la question de l'ancienneté de ces préfixes en \ref{subsubsec:innovations.paralleles}.

Le préfixe réfléchi \ipa{ʑɣɤ-} (position 7), qui n'a d'équivalent dans aucune langue hors des quatre langues rgyalrong,  est une innovation de ce sous-groupe. On a montré  (voir \citealt{jacques10refl}) que ce préfixe provenait d'une proto-forme *wjaŋ- ou *wjɐ- en proto-rgyalrong. Il s'agit soit du pronom de troisième personne incorporé (japhug \ipa{ɯʑo} < *ujaŋ), soit du préfixe possessif de troisième singulier associé à la racine ``soi-même'' (japhug \ipa{-ʑo} < *jaŋ) incorporé au complexe verbal. Le cognat de cette racine \tgz{1245} existe bien en tangoute, mais il s'agit dans cette langue d'un pronom réfléchi: le marqueur  	\tgz{1139}  de génitif-antiergatif peut en effet le séparer du verbe, comme l'illustre l'exemple suivant:


\begin{tabular}{llllllllllll}
 \tgf{824}   & 	\tgf{5417}   & 	\tgf{1245}   & 	\tgf{1139}   & 	\tgf{4342}   & 	\tgf{4225}   & 	\tgf{5916}   & 	\tgf{2885}   & 	\tgf{5026}  & 	\tgf{1906}   \\
 \tinynb{824}   & 	\tinynb{5417}   & 	\tinynb{1245}   & 	\tinynb{1139}   & 	\tinynb{4342}   & 	\tinynb{4225}   & 	\tinynb{5916}   & 	\tinynb{2885}   & 	\tinynb{5026}  & 	\tinynb{1906}   \\
\tgf{1245}   & 	\tgf{5815}   & 	\tgf{1245}   & 	\tgf{1139}   & 	\tgf{4342}   & 	\tgf{4225}   & \\
\tinynb{1245}   & 	\tinynb{5815}   & 	\tinynb{1245}   & 	\tinynb{1139}   & 	\tinynb{4342}   & 	\tinynb{4225}   & \\
\end{tabular}
\begin{exe}
\ex   \vspace{-8pt}
\gll   	\ipa{tɕhjɨ²rjar²}  	\ipa{.jij¹}  	\ipa{.jij¹}  	\ipa{dja²-sja¹}  	\ipa{xã¹phow¹}  	\ipa{mji¹}  	\ipa{nioow¹}  	\ipa{.jij¹}  	\ipa{tsjɨ¹}  	\ipa{.jij¹}  	\ipa{.jij¹}  	\ipa{dja²-sja¹}  \\
immédiatement soi-même \gen{} \dir{}-tuer Han.Ping entendre[A] après soi-même aussi soi-même \gen{} \dir{}-tuer \\
\glt  Il se suicida immédiatement. Han Ping, ayant entendu cela, se suicida aussi. (Leilin, 06.04B.3)
\end{exe}
Le tangoute ne partage donc pas avec les langues rgyalrongs la grammaticalisation de la marque de réfléchi comme un préfixe verbal.


Les préfixes translocatif et cislocatif des langues rgyalrongs (position 3) constituent une autre innovation de ce groupe, comme nous l'avons présenté dans \citet{jacques13harmonization}. Le  translocatif \ipa{ɕɯ-} et le cislocatif \ipa{ɣɯ-} du japhug servent à exprimer le sens ``d'aller faire'' ou de ``venir faire'', comme l'illustrent les exemples suivants:


\begin{exe}
\ex 
\gll \ipa{mphrɯmɯ}	\ipa{ɕ-pɯ-sɯ-re}	\ipa{tɕe},	\ipa{ɕ-tɤ-the}	\ipa{ra} \\
 divination \textsc{transloc}-\textsc{imp}-\textsc{caus}-regarder[III] \textsc{coord} \textsc{transloc}-\textsc{imp}-demander[III] \textsc{n.pst}:falloir \\
 \glt Il faut que tu ailles faire faire une divination, et que tu ailles lui en demander (la raison).
 
 \ex
\gll 	\ipa{ɯ-ɕki}	\ipa{zɯ}	\ipa{ɣɯ-tɤ-nɯ-thu-nɯ}	\\
3\textsc{sg}-\textsc{dat}  \textsc{loc}  \textsc{cisloc}-\textsc{imp}-\textsc{autoben}-demander-\textsc{pl} \\
\glt 	Venez la demander (en mariage). (le prince, 66)
\end{exe}

Ces préfixes proviennent de façon transparente des verbes ``aller'' \ipa{ɕe} et ``venir'' \ipa{ɣi} (<*w). Toutefois, leur position \textit{préfixale} plutôt que suffixale est paradoxale pour une langue à verbe final comme le japhug, d'autant plus qu'une construction concurrente existe, où le verbe de mouvement apparaît bien après la subordonnée de but:
 
\begin{exe}
\ex 
\gll \textit{ɯ-fso}	\textit{tɕe}	\textit{ɬamu}	\textit{línɤ}	\textit{pɯwɯ}	\textit{ɯ-kɯ-no}	\textit{pjɤ-ɕe},	 \\
 3\textsc{sg}-demain \textsc{coord} Lhamo  à.nouveau âne 3\textsc{sg}-\textsc{nmls:A}-chasser \textsc{med:bas}-aller \\
\glt Le lendemain, descendit à nouveau pour faire avancer son âne. (Le corbeau, 62)
\end{exe}

Une construction présentant le même ordre des mots se retrouve d'ailleurs en tangoute, ainsi que dans toutes les langues du Sichuan sur lesquelles nous disposons de données:

\begin{tabular}{llllllllll}
\tgf{3627} & 	\tgf{3627} & 	\tgf{1796} & 	\tgf{3119} & 	\tgf{5871} & 	\tgf{5093} & 	\tgf{4633} & 	\tgf{1139} & 	\tgf{2862} & 	\tgf{5447}  \\
\tinynb{3627} & 	\tinynb{3627} & 	\tinynb{1796} & 	\tinynb{3119} & 	\tinynb{5871} & 	\tinynb{5093} & 	\tinynb{4633} & 	\tinynb{1139} & 	\tinynb{2862} & 	\tinynb{5447}  \\
\tgf{1394} &\tgf{4225} & 	\tgf{4481} & 	\tgf{0749} \\
\tinynb{1394} &\tinynb{4225} & 	\tinynb{4481} & 	\tinynb{0749} \\
\end{tabular}
\begin{exe}
\ex   \vspace{-8pt}
\gll  \ipa{nji²nji²} 	\ipa{tɕhjụ¹.ji¹} 	\ipa{zeew²} 	\ipa{tɕhjiw¹thwər¹} 	\ipa{.jij¹} 	\ipa{nji¹} 	\ipa{do²} 	\ipa{tha¹} 	\ipa{sja¹} 	\ipa{ɕjɨ¹-phji¹}  \\
		 en.secret Chu.Ni envoyer Zhao.Dun \gen{} maison \allat{} presser tuer aller[B]-causer[A]\\
\glt Il envoya en secret Chu Ni pour tuer Zhao Dun. (Leilin, 03.95A.6-7)
\end{exe}
Pour expliquer ce paradoxe, il faut sans doute, plutôt que d'une subordonnée de but, partir d'une construction en série, du type:

\begin{exe}
\ex  
\gll *jo-ɕe to-χtɯ > ɕ-to-χtɯ \\
\textsc{med}-aller \textsc{med}-acheter > \textsc{transloc}-\textsc{med}-acheter \\
\glt Il est allé l'acheter.
\end{exe}

Il est clair que rien de tel ne se retrouve en tangoute. Si une telle construction avait existé on s'attendrait à deux situations possibles. Premièrement, les préfixes translocatifs et cislocatifs seraient devenus des préfixes *S- et *P- respectivement en pré-tangoute; dans cette hypothèse, on trouverait des  verbes dérivés de verbes simples, marqués par la voie tendue (venant de *S-) ou  la médiane -w- (venant de *P-), conjointement à une alternance sémantique qui pourrait provenir d'un translocatif ou d'un cislocatif respectivement. Aucun exemple de ce type n'a été toutefois découvert. Deuxièmement, on devrait trouver une construction telle que:

\begin{exe}
\ex   
\glt préfixe directionnel - verbe de mouvement - verbe principal
\end{exe}

Mais là encore aucun exemple de ce type n'existe dans les textes tangoutes à notre connaissance. 

%\subsubsection{Innovations du tangoute} \label{subsubsec:innovations.tang}
 
\subsubsection{Innovations parallèles entre japhug et tangoute} \label{subsubsec:innovations.paralleles}
Dans la section précédente, nous avons abordé la question des innovations du japhug qui n'ont aucun équivalent en tangoute. La présente section sera consacrée aux affixes grammaticaux cognats qui sont vraisemblablement le résultat de grammaticalisations parallèles et non d'héritage commun entre tangoute et japhug. Il s'agira des préfixes directionnels, des négations et du suffixe de futur.

Les préfixes directionnels (position 4 en japhug, position 1 en tangoute), comme nous l'avons vu en \ref{subsec:directionnels}, ont des fonctions typologiquement similaires en japhug et en tangoute, mais leur formes sont totalement divergentes et phonétiquement irréconciliables. Seuls les préfixes \tgz{1452}  et \tgz{3846} ``vers le bas'' présentent une similarité avec certaines langues rgyalrongs, comme le préfixe du tshobdun \ipa{nɐ-} (le japhug \ipa{pɯ-} à la place pour le directionnel ``vers le bas''). Le système de préfixes directionnels du tangoute est en revanche remarquablement similaire à celui du pumi. La question de savoir s'il faut postuler un clade tangoute-pumi au sein du macro-rgyalronguique, dans lequel le système de préfixes directionnels aurait déjà été formé, dépasse le cadre de ce travail, car une étude systématique des correspondances entre pumi et tangoute serait nécessaire pour cela.

Les préfixes de négation du tangoute et du japhug  (position 2 dans le complexe verbal des deux langues),  ont de toute évidence une origine commune, comme nous l'avons montré en \ref{subsubsec:negation}. Toutefois,  il n'est pas nécessaire de supposer que la grammaticalisation des particules de négation en préfixes avait déjà eu lieu dans l'ancêtre commun à ces deux langues: on ne retrouve en effet pas la moindre allomorphie commune des préfixes de négation entre tangoute et japhug. Par ailleurs, l'ordre divergent directionnel - négation en tangoute, par opposition à négation - préfixe directionnel en japhug (\ref{subsec:direct-ordre}) est un indice supplémentaire du fait que ces préfixes représentent des grammaticalisations indépendantes.

Parmi les suffixes, on trouve peu de formes communes entre japhug et tangoute. Dans la section  \ref{sec:morpho.verbale.flex}, nous n'avons mis en évidence que les suffixes d'accord personnel et le suffixe de futur 	\tgz{1101}. 

La question de l'antiquité des marques de personne est complexe. La similarité des suffixes de personne avec les pronoms est frappante et suggère une grammaticalisation tardive, mais l'alternance vocalique (\ref{subsubsec:origine.alternances}) en revanche ne peut pas être secondaire. Le système d'accord du tangoute combine des réfections analogiques tardives  avec des préservations anciennes. Le seul élément du système dont on peut être certain qu'il est hérité d'une marque d'accord dans l'ancêtre commun au tangoute et aux langues rgyalrong est le suffixe *-u de troisième personne patient que nous avons reconstruit en pré-tangoute (comparé au suffixe de même fonction -w en situ). Le statut des trois suffixes d'accord, 1\sg{}  \tgz{2098}, 2\sg{}  \tgz{4601}, 1/2\pl{}  \tgz{4884}  en revanche est plus complexe et ne pourra être traité que dans le cadre d'une étude exhaustive des marques de personnes dans les langues macro-rgyalronguiques.

Le suffixe du futur \tgz{1101} se compare au suffixe \ipa{-jə} du tshobdun et à une des alternances vocaliques du japhug (\ref{subsubsec:TAM}). Il est difficile de juger avec certitude de l'antiquité de ce morphème comme suffixe plutôt que comme particule grammaticale. L'argument principal en faveur de l'hypothèse que ce suffixe avait déjà été grammaticalisé en proto-macro-rgyalronguique est le fait qu'il apparaît non plus comme élément indépendant, mais sous la forme d'alternances vocaliques en japhug (et peut-être même en zbu), et que sa distribution est caractérisée par de nombreuses contraintes morphophonologiques arbitraire en tshobdun (voir \citealt[496]{jackson03caodeng}). Par ailleurs, dans la chaîne suffixale du tangoute, il apparaît bien plus près de la racine verbale que tous les autres suffixes de TAM, comme nous l'avons montré en \ref{subsec:position8}. 

L'argument principal en faveur de l'hypothèse d'une grammaticalisation indépendante est la divergence de position entre les langues rgyalrongs et le tangoute. Dans les premières, il est plus proche de la racine verbale (et même fusionne avec elle) que le suffixe de première personne, comme l'illustre l'exemple suivant tiré du japhug:

 
\begin{exe}
\ex 
\gll \ipa{tu-ndze-a} \\
  \impf{}-manger[III]-1\sg{} \\
\glt  Je mange.
\end{exe}

Le thème III du verbe \ipa{ndza} ``manger'' provient originellement de la combinaison de la racine nue *ndza avec le suffixe *-jə: il est clair que ce suffixe était situé entre la racine et le suffixe de personne à un stade plus ancien. En revanche, le suffixe  \tgz{1101} du tangoute apparaît \textit{après} la marque de personne, comme nous l'avons vu en \ref{subsec:position8}. 

La divergence de position relative des préfixes entre  le tangoute et le rgyalrong n'est toutefois pas un argument décisif contre l'hypothèse d'une grammaticalisation commune dans la proto-langue; on observe en effet même à l'intérieur des dialectes du japhug des divergences sur l'ordre des préfixes. Par exemple, le préfixe de médiatif \textit{-a}, qui en japhug de Kamnyu est fusionné avec le préfixe directionnel (et n'a donc pas été inclus dans le gabarit verbal comme une position distincte), apparaît après le préfixe de seconde personne dans les dialectes de Datshang et de Gsardzong (\citealt[249]{jacques08}):

\begin{exe}
\ex 
\gll \ipa{to-tɯ-ndza-t} \ipa{to-ndza} \\
  \med{}-2-manger-\textsc{pst}    \med{}-manger \\
\glt  Tu l'as mangé. Il l'a mangé. (Dialecte de Kamnyu).
\ex 
\gll \ipa{tu-tɯ-ɤ-ndza} \ipa{tu-ɤ-ndza} \\
  \impf{}-2-\med{}-manger \impf{}-\med{}-manger \\
\glt  Tu l'as mangé. Il l'a mangé. (Dialecte de Gsardzong).
\end{exe}

 En dialecte de Kamnyu, le  préfixe directionnel imperfectif et celui du médiatif ont fusionné en un seul, après un changement de position relative entre le préfixe de médiatif et celui de seconde personne, tandis qu'en Gsardzong ils sont bien distincts. Ici la troisième personne a été la forme pivot: on a refait la seconde personne sur la base de la troisième. Une explication similaire en termes d'analogie pourrait rendre compte de la différence d'ordre relatif du suffixe de futur et de celui de 1\sg{} entre le tangoute et le rgyalrong.
 
 
En conclusion, sur les treize positions préfixales du japhug (voir le tableau \ref{tab:template:derivational2}, qui reproduit le gabarit illustré en \ref{tab:template:derivational}), les positions 2, 3, 7 et 14 sont innovées.
\begin{table}
\caption{Le gabarit verbal du japhug }\label{tab:template:derivational2}
\resizebox{\columnwidth}{!}{
\begin{tabular}{llllll|llllllll|lllll} \toprule

\ipab{a-}  &  	\ipab{mɯ- }   &  	\ipab{ɕɯ-}   &\ipab{tɤ-} &  	\ipab{tɯ-}  &  	\ipab{wɣ-}   &

  	 \grise{\ipab{ʑɣɤ-}}  &  	\grise{\ipab{sɯ-}}  & \grise{\ipab{rɤ-}}& \grise{\ipab{nɤ-}} &   	 \grise{\ipab{a-}}   &  	\grise{\ipab{nɯ-}}  &  	\grise{\ipab{ɣɤ-}}  &  	\grise{\ipab{noun}}    &  	 \begin{math}\Sigma\end{math}    &  	\ipab{-t}  &  	\ipab{-a}  &  	\ipab{-nɯ}   &  \\
   &  	\ipab{mɤ-}   &  	\ipab{ɣɯ-}   &\ipab{pɯ-}&  	  &  	 
    & \grise{ }	  &  	 \grise{ }	  &  	  \grise{ }	  &  	   \grise{ }	&  	\grise{\ipab{sɤ-}}&  \grise{ }	 &  	\grise{\ipab{rɯ-}}  &  	 \grise{ }	  &  	  &  	  &  	  &  	\ipab{-ndʑi} &  \\
  &  	   &     &  etc.	  & & 	  &  	  &  	 & &  	  &  	 & &  etc.	  &  	  &  	  &  	  &  	  &  	  &  \\
1  &  	2  &  	3  &  	4  &  	5  &  	6  &  	7  &  	8  &  	9  &  	10  &  	11  &  	12  &  	13  &  	14  &  	15  & 16 &17&18\\
\bottomrule
\end{tabular}}
\end{table}
 


Seules les positions 8, 9, 13 et peut-être 10 (voir \ref{subsec:anticausatif}) ont des cognats potentiels en tangoute. Certaines des autres positions, en particulier 5 et 6, le  préfixe  de seconde personne et l'inverse, ont des cognats   en kiranti (voir \citealt{jacques12agreement}) qui ne peuvent pas s'expliquer comme des grammaticalisations indépendantes. Comme le tangoute est plus proche génétiquement des langues rgyalrongs que celles-ci ne le sont du kiranti, il est logique de conclure que ces préfixes ont dû disparaître sans laisser de trace, ce qui n'est pas surprenant étant donnée l'attrition massive des présyllabes en tangoute.

Il est manifeste qu'aucun cognat certain parmi les préfixes entre rgyalrong et tangoute n'est resté une syllabe indépendante en tangoute. Les traces de morphologie ancienne sont non-productives, et leurs traces indirectes. La morphologie qui apparaît sous forme de syllabe indépendante en tangoute est généralement innovée ou tout au moins considérablement refaite.

Seule une   minorité  des préfixes du japhug entre les positions 7 et 13 ont un cognat en tangoute. Certains de ces préfixes pourraient être des innovations propres aux langues rgyalrongs dont l'étymologie n'a pas encore été élucidée, mais il est également possible que les traces d'autres préfixes dérivationnels attendent d'être découvertes en tangoute.


\section{Conclusion}
 
L'étude de la morphologie du pré-tangoute illustre l'intérêt de notre reconstruction phonologique. D'une part, elle facilite la comparaison de la morphologie du tangoute avec celle des autres langues macro-rgyalronguiques. D'autre part,  en apportant une profondeur historique aux phénomènes synchroniques du tangoute, elle permet de résoudre certaines questions telles que la directionalité de dérivation de certains procédés morphologiques, ainsi que l'origine des alternances vocaliques et  de certaines formations fossiles. 

Le tangoute n'était pas une langue isolante comme le chinois moyen ou les langues  lolo-birmanes, mais une langue à la morphologie verbale relativement complexe comme le pumi ou le muya. Il préservait encore quelques traces de la morphologie polysynthétique des langues rgyalrongs, en particulier la présence d'incorporation et une structure élaborée du gabarit verbal.

L'approche philologique de l'étude du tangoute a permis des progrès considérables dans la compréhension des textes, mais seule la comparaison avec les langues modernes permet d'espérer pouvoir un jour écrire une véritable grammaire de référence du tangoute libérée du carcan du modèle chinois.

\chapter{Classification} \label{chap:classification}
\thispagestyle{empty}
Ce travail nous a permis d'étudier en détail l'ensemble du vocabulaire et de la morphologie étymologisable du tangoute. Nous avons proposé un nombre important de nouvelles lois phonétiques, qui permettent non seulement de trouver les cognats entre le tangoute et les autres langues macro-rgyalronguiques, mais également d'expliquer certaines alternances morphologiques.

Nous avons pu montrer, outre l'existence d'anciennes consonnes finales et de nombreuses pré-syllabes, la présence de voyelles vélarisées en pré-tangoute (voir \ref{subsubsec:correspondance:a:agg}) et une trace d'une opposition vocalique correspondant à  *a et *ə en chinois (voir p.\pageref{rimes:01:3:ju/a}), opposition que l'on considère habituellement comme perdue dans les langues non-chinoises.


Nous avons mis au jour l'existence de deux préfixes causatifs en proto-macro-rgyalronguique (\ref{subsec:causatif} et \ref{tab:prefixep}) et de plusieurs autres préfixes dérivationnels, et avons expliqué les alternances vocaliques du système d'accord, qui prouve l'antiquité de l'accord en sino-tibétain (voir \ref{subsec:personne}).

Toutefois, une grande partie des reconstructions présentées dans ce travail ne sont que d'un intérêt limité pour déterminer la place du tangoute dans le Stammbaum du sino-tibétain: il pourrait en effet s'agir de rétentions (proto-sino-tibétaines) communes au rgyalrong et au tangoute. Or, conformément au principe de \citet[VII]{leskien1876},\footnote{``Die Kriterien einer engeren Gemeinschaft können nur in positiven Uebereinstimmungen der betreffenden Sprachen, die zugleich Abweichungen von den übrigen sind, gefunden werden".} seules les innovations communes ont une valeur pour prouver que deux langues appartiennent bien à un sous-groupe commun.

Dans ce chapitre, nous allons donc essayer de déterminer les indices dans le lexique et la morphologie qui peuvent soutenir l'existence d'un groupe macro-rgyalronguique comprenant:

\begin{itemize}
\item le tangoute
\item les langues rgyalronguiques (lavrung, rtau, shangzhai, zbu, japhug, tshobdun, situ etc.)
\item le queyu
\item le muya
\item le pumi 
\end{itemize}
auxquelles il convient peut-être de rajouter  le qiang et le zhaba. Le shixing, le ersu et le naxi appartiennent probablement à un même sous-groupe du sino-tibétain, mais sont plus lointainement apparentés (voir par exemple \citealt{chirkova12qiangic}).

On s'intéressera à trois aspects: la phonologie, le lexique et la morphologie.

\section{Phonologie}

Deux caractéristiques phonologiques du tangoute ont une importance non-triviale pour la classification des langues macro-rgyalronguiques et sino-tibétaines.

Premièrement, comme nous l'avons illustré en \ref{subsubsec:correspondance:a:agg}, on observe une correspondance claire entre les rimes --\textit{a} et --\textit{ar} du tangoute et les rimes à voyelle vélarisée des langues rgyalrong, qui s'opposent aux voyelles non-vélarisées (tableau \ref{tab:velarisee.tangoute}).
\begin{table}
\captionb{Voyelles vélarisées et non-vélarisées entre japhug et tangoute} \label{tab:velarisee.tangoute}
\begin{tabular}{llllllll}
\toprule
tangoute & sens & japhug & proto-japhug\\
\midrule
\tgz{2584} &   			pont   &  	ndzom    &  	< *ndzam   \\  
\tgz{3443} &   			oncle   &  	tɤ-βɣo    &  	< *-kpaŋ   \\  
\tgz{0039} &   			dent   &  	tɤ-mɢom    &  	< *-mɴɢam   \\  
\midrule
\tgz{1391} &   			sourd   &  	tɤ-mbɣo    &  	< *-mbaˠŋ   \\  
\tgz{5528} &   			tambour   &  	tɤ-rmbɣo    &  	< *-mbaˠŋ   \\  
\tgz{0975} &   			geler   &  	jpɣom    &  	< *lpaˠm   \\  
\bottomrule
\end{tabular}
 \end{table}
Cette opposition ne semble   reflétée dans aucune autre branche du sino-tibétain que le rgyalronguique et le tangoute dans l'état actuel des connaissances. Or, si le tangoute est très érodé, les langues rgyalrongs préservent les groupes de consonnes initiaux mieux n'importe quelle autre groupe de la famille, et il est donc difficile d'expliquer cette opposition comme une innovation due à la perte d'une consonne.
 
 Il est donc  difficile d'affirmer que cette caractéristique est une innovation commune des langues macro-rgyalronguiques. Il pourrait tout aussi bien d'agir d'un archaïsme, d'une opposition disparue dans toutes les autres langues sino-tibétaines que seul le macro-rgyalronguique permettrait de reconstruire.
 
 
 Une autre caractéristique intéressante du tangoute est la  trace d'une opposition vocalique correspondant à  *a et *ə en chinois (voir tableau {tab:a.schwa.chinois}, ainsi que la discussion p.\pageref{rimes:01:3:ju/a}).
 \begin{table}
\captionb{Préservation de l'opposition correspondant à celle entre *a et *ə en chinois archaïque } \label{tab:a.schwa.chinois}
\begin{tabular}{lllllll}
\toprule
tangoute & sens & japhug & tibétain & chinois\\
\midrule
 \tgz{5203} &   			hache   &  	tɯ-rpa   &  	   &  	\zh{斧} *pˁaʔ   \\  
\tgz{4046} &   			amer   &  	   &  	kʰa   &  	\zh{苦} *kʰˁaʔ   \\  
\midrule
\tgz{4681} &   			oreille   &  	tɯ-rna   &  	rna   &  	\zh{耳} *nəʔ   \\  
\tgz{1338} &   			aimer   &  	   &  	mdza   &  	\zh{慈} *dzˁə   \\  

\bottomrule
\end{tabular}
 \end{table} 
L'opposition entre *a et *ə du chinois archaïque n'a pas d'équivalent en tangoute en revanche en syllabe fermée. Toutefois, étant donné les neutralisations massives de rime qui s'observent en tangoute entre certaines voyelles bien distinctes (comme pré-tangoute *a et *i par exemple dans certains contextes), il est tout à fait possible que cette distinction ait existé mais ait disparu en tangoute suite à la perte des consonnes finales.

La préservation de cette opposition en tangoute est d'une importance non-négligeable, car il s'agit d'une des rares innovations communes proposées pour le ``tibéto-birman" (voir par exemple \citealt{handel08st}). Le tangoute montre que certaines langues non-chinoises préservent cette opposition, ce qui suggère que l'existence supposée d'un clade ``tibéto-birman" est extrêmement douteuse. 


\section{Lexique}

Si la phonologie apporte peu d'informations positives concernant la classification, le lexique est en revanche d'une importance cruciale, et c'est dans ce domaine qu'on trouve l'essentiel des preuves de l'existence d'un groupe macro-rgyalronguique.

Dans cette section, nous discuterons d'abord des innovations lexicales claires communes aux langues macro-rgyalronguiques, puis du vocabulaire commun exclusif non-étymologisable (d'une valeur moins importante pour classer les langues).

\subsection{Innovations lexicales}
Par ``innovations lexicales" nous désignons spécifiquement le vocabulaire commun exclusif pour lequel une étymologie interne peut être proposée. Pour ces mots, nous avons en effet la certitude qu'il s'agit d'innovations et non pas de rétentions communes; on ne peut pas exclure a priori l'hypothèse de développements parallèles, mais cela  peut difficilement être le cas pour tous les exemples suivants.

On trouve en tout cinq groupes d'innovations certaines. Il s'agit soit de mots dont on peut   analyser la structure morphologique et montrer qu'ils dérivent d'une racine au sens différent, soit dont on peut  montrer que le sens est innové par rapport à d'autres langues de la famille sino-tibétaine. 

\begin{enumerate}

\item Le verbe \tgz{0385}  ``être capable'', dérivé de ``faire'' par le préfixe habilitatif (voir p.\pageref{tab:prefixe-abil}). Cet exemple est doublement important, car il prouve également l'existence de ce préfixe dans l'ancêtre commun du tangoute et du rgyalrong. On doit noter que cette dérivation a le mérite d'expliquer une anomalie typologique aussi bien en japhug qu'en tangoute: la transitivité de ce verbe. 

En effet, d'après   \citet{tsunoda85tr}, on observe la hiérarchie suivante des  types de prédicats:
\begin{exe}
\ex \label{ex:tsunoda} 
\glt Effective Action   >> 
Perception  >> 
Pursuit   >> 
Knowledge   >> 
Feeling   >> 
Relationship   >> 
Ability  
\end{exe}

Les verbes appartenant aux catégories élevées dans cette hiérarchie vont avoir tendance à travers les langues à être exprimés par des prédicats transitifs, tandis que ce ceux situés plus bas dans la hiérarchie ont tendance à être encodés comme des procès intransitifs. Or, le verbe ``être capable" en \jpg{spa} et en tangoute \tgz{0385}, s'il appartient à la catégorie ``ability" (la plus basse de la hiérarchie de Tsunoda), est un verbe transitif alors que des verbes appartenant à des catégories supérieures telles que ``relationship" ou même ``perception" sont morphologiquement intransitifs. Le japhug et le tangoute contredisent donc  tous deux  cette hiérarchie de la même façon. 

 \item Le nom \tgz{5645} ``lieu'', nom déverbal tiré de ``poser'', comme son cognat en rgyalrong \jpg{ɯ-sta}. On a le même changement sémantique ``là où il est posé" devenant le nom générique de ``l'endroit''.

 

 \item Le nom \tgz{0527} ``tumeur, goitre'' (correspondant au \jpg{zbɤβ}) dérivé d'une racine verbale ``enfler'' qui n'est pas directement attesté en tangoute, mais qui l'est en japhug (\jpg{nɯɣmbɤβ}).
 
  \item Le système de désignation des frères et sœurs, qui distingue le sexe du parent en question et de la personne par laquelle passe la relation,\footnote{Cette particularité est désignée habituellement par ``male speaking" et ``female speaking" en ethnologie, mais cette terminologie présente un problème, car elle n'est valable que lorsque le possesseur est première personne; cette opposition désigne plutôt le sexe du possesseur.} est identique entre le tangoute et le rgyalrong, et le pumi peut en être dérivé (voir \citealt{jacques11kinship}).
  
  \begin{table}
  \caption{Tableau comparatif des termes de parenté désignant les frères et les sœurs en japhug et en tangoute} \centering
\begin{tabular}{lllllll}
\toprule
sens & japhug & tangoute \\
\midrule
frère (d'un homme) & \ipa{tɤ-xtɤɣ} & \tgz{0605}\\
frère (d'une femme) & \ipa{tɤ-wɤmɯ} & \tgz{0355}\\
sœur (d'un homme) & \ipa{tɤ-snom} & \tgz{0549}\\
sœur  (d'une femme) & \ipa{tɤ-sqhɤj} & \tgz{3361}\\
\bottomrule
\end{tabular}
\end{table}
  
De ces quatre termes, les trois derniers sont cognats au japhug et au tangoute. Le premier ne peut pas l'être car, bien qu'en principe le japhug \ipa{--ɤɣ} et le tangoute \ipa{--o} puissent se correspondre, ici \ipa{--xtɤɣ} vient de *ktek, et la comparaison ne peut être acceptée. Toutefois, le fait que les trois autres termes soient cognats indique que le système existait déjà dans la proto-langue; la désignation du frère (d'un homme) a subi ensuite une innovation dans l'une des deux langues. 


  
  
  \item Le verbe ``écrire'' (\jpg{rɤt}, tangoute \tgz{1715}) vient d'une racine signifiant originellement  ``gratter", toujours attestée dans ce sens en tibétain (voir p.\pageref{analyse:ecrire}).  Il pourrait toutefois s'agir d'un emprunt ancien entre ces deux langues, car il est peu vraisemblable que   les ancêtres communs des tangoutes et des rgyalrongs connaissaient l'écriture.
 \end{enumerate}

\subsection{Vocabulaire commun}
Par ailleurs, une vingtaine d'étymons communs entre le rgyalrong et le tangoute ne se retrouvent pas en dehors du groupe macro-rgyalronguique tel qu'il est défini plus haut. Il s'agit aussi bien de verbes que de noms. Aucun d'entre eux n'appartient à la liste de Swadesh, et certains pourraient être des emprunts anciens présentant les mêmes correspondances phonétiques que les cognats.

\begin{enumerate}
 \item Parties du corps
 \begin{enumerate}


\item \tgz{0791} ``front'' 

\end{enumerate}
\item Phénomènes naturels
 \begin{enumerate}
\item \tgz{1530} ``fleuve''  

\item \tgz{4796} ``sud'' (< ``adret")

\item \tgz{3299} ``vapeur''



\item \tgz{2005} ``boue''

 
\end{enumerate}

\item Agriculture
\begin{enumerate}

\item \tgz{4680} ``soc'' 

\item \tgz{1752} ``houe''

\item  \tgz{2795} ``pâturages''

\end{enumerate}

\item Verbes de sensation corporelle
\begin{enumerate}
\item \tgz{2857} ``malade''  

\item \tgz{4532} ``avoir soif''
\item \tgz{3869} ``rassasié''

\item \tgz{0045} ``piquant''
\end{enumerate}
\item Autres
\begin{enumerate}

\item \tgz{1569} ``droit'' 
\item \tgz{1638} ``clair''
\item \tgz{0596} ``grandir''



\item \tgz{0099} ``finir''

\item \tgz{2797} ``sortir''


\item \tgz{5621} ``ajouter''

\item \tgz{3003} ``fantôme'' 


\item \tgz{4600} ``promesse''
\end{enumerate}
\end{enumerate}

 
Il n'est pas possible de montrer rigoureusement que ces étymons exclusifs sont des innovations tant que leur étymologie reste inconnue, mais il s'agit au moins de candidats à une étude plus approfondie.


Ce vocabulaire commun nous permet de proposer quelques hypothèses sur le mode vie des locuteurs de ces proto-langues afin de contribuer à localiser leur Urheimat.\ 

\begin{quote}



Les locuteurs du \textbf{proto-birmo-qianguique}  possédaient le \textbf{fer} \tgz{4995} et l'\textbf{argent} \tgz{3572}. Ils se vêtaient de pantalons \tgz{1388}. Leur organisation sociale comportait déjà des \textbf{seigneurs} ou des rois \tgz{3266}. Ils élevaient des \textbf{poulets} \tgz{4546}. Ils connaissaient aussi les ovins et les bovins, mais ce vocabulaire n'est pas exclusif au groupe birmo-qianguique. Ils \textbf{chevauchaient} \tgz{2407} des \textbf{chevaux} \tgz{0764}. Ils habitaient une région où la \textbf{neige}  \tgz{4091} tombe.


Ceux du \textbf{proto-macro-rgyalronguique}, en plus des caractéristiques mentionnées ci-dessus, pratiquaient la culture de l'\textbf{orge} \tgz{2160} sur l'\textbf{adret} des montagnes \tgz{4796} avec   des \textbf{socs} \tgz{4680} et des \textbf{houes} \tgz{1752}. 
Ils vivaient probablement dans une région  où la \textbf{grêle} \tgz{4032} tombe. Ils produisaient de la farine au \textbf{moulin} (\tgz{1254} et \tgz{0973}).  Le \textbf{xanthoxyle} \tgz{0512} poussait  aussi dans cette région.

 Ils \textbf{aiguisaient} \tgz{1670} leurs \textbf{couteaux} \tgz{5037}. Leur économie incluait également des pasteurs nomades qui faisaient \textbf{paître} leur bétail \tgz{0993} sur les \textbf{pâturages} \tgz{2795}. Outre les bœufs, les chèvres et les moutons, ils élevaient aussi des \textbf{yaks} \tgz{1195}. Ils croyaient au pouvoir des \textbf{magiciens} \tgz{3439} et des \textbf{fantômes} \tgz{3003}.

\end{quote}
Il est intéressant d'observer qu'une partie de ce vocabulaire exclusif fait référence à l'élevage des animaux et à la montagne; il est peu probable par conséquent que l'Urheimat proto-macro-rgyalronguique soit à chercher dans les plaines du Sichuan. La région d'origine de ces peuple est sans doute dans les districts tibétains de Rnga.ba et Dkar.mdzes.


\section{Morphologie}
La morphologie reconstruite en pré-tangoute ressemble sous certains aspects typologiques à celle que l'on retrouve en rgyalronguique, en pumi ou en muya: présence de préfixes directionnels marquant le TAM, de marquage personnel hiérarchique, de préfixes négatifs, d'incorporation nominale et d'alternances vocaliques et tonales. 

Il est toutefois difficile de déterminer si ces traits communs sont des rétentions, des innovations communes ou des innovations parallèles dues au contact. On s'intéressera en particulier à deux propriétés: le suffixe de nominalisation et les préfixes directionnels.

\subsection{Suffixe de nominalisation}
Comme nous l'avons montré en \ref{subsec:nmlz}, le suffixe de nominalisation agentive  \tgz{3818} est   apparenté au suffixe du même type 
 --\textit{mə} en pumi, qui   dérive de façon transparente du nom ``homme''  \textit{mə̂}. Si en tangoute l'origine de ce suffixe n'est pas synchroniquement apparente puisque le nom originel a disparu, il ne fait aucun doute que la même grammaticalisation a eu lieu dans les deux langues.
 
Il est clair qu'il ne s'agit pas là d'une innovation macro-rgyalronguique, puisque les langues rgyalrongs ont un système de nominalisation préfixale clairement apparenté avec celui des langues kiranties (bien représenté en limbou et en athpare, voir \citealt{jacques12agreement}), et dont même le tangoute préserve des traces indirectes (cf \ref{subsec:S-deverbal}).
 
Ce point commun entre le tangoute est le pumi pourrait en revanche être  une innovation exclusive de ces deux langues, qui les opposerait ainsi au reste du macro-rgyalronguique. Il est difficile toutefois de l'affirmer, car il pourrait tout aussi bien s'agir d'une innovation parallèle dans les deux langues: le nom ``homme" est souvent grammaticalisé comme nominalisateur dans les langues du monde.


\subsection{Préfixes directionnels}
Le système de préfixes directionnels, considéré par Sun Hongkai comme marque par excellence du groupe ``qianguique'', est probablement davantage un phénomène aréal qu'une innovation commune (voir par exemple \citealt{chirkova12qiang}). 

Toutefois, la grande similarité du système de directionnels du pumi et celle du tangoute (voir section \ref{subsec:directionnels}) suggère qu'il peut s'agir là d'une innovation commune à ces deux langues au sein du macro-rgyalronguique: autrement dit, qu'un clade rassemblant le pumi et le tangoute puisse être suggéré. Cette idée préliminaire est naturellement soumise à confirmation par une étude plus détaillée du vocabulaire et de la morphologie du pumi.

Un argument en faveur de l'ancienneté du système de directionnel est la présence de deux séries de préfixes directionnels, l'une avec des préfixes à voyelle centrale, et l'autre à voyelle antérieure en tangoute. Cette caractéristique est partagée par les langues Rtau, ou l'on trouve une série de préfixes accentués en à vocalisme antérieur pour les formes interrogatives.

\subsection{Autres}

Le préfixe habilitatif commun au rgyalronguique et au tangoute (voir  section \ref{subsec:abilit}) pourrait lui-même être une innovation commune au macro-rgyalronguique, si l'on peut l'expliquer comme dérivé du préfixe causatif, mais il est nécessaire avant de pouvoir en juger de retrouver des traces de ce préfixes dans les autres langues de ce groupe.

Les données morphologiques tendent à suggérer une parenté privilégiée entre tangoute et pumi, et donc l'établissement d'un clade pumi-tangoute au sein du macro-rgyalronguique. Une comparaison lexicale détaillée entre macro-rgyalronguique et pumi n'est malheureusement pas encore possible, car on ne dispose sur cette langue que de glossaires limités.


\section{Birmo-qianguique}

L'hypothèse d'un groupement birmo-qianguique, incluant les langues tangoutes, naish, lolo-birmanes et peut-être le lizu/ersu semble probable. Ce groupement est suggéré par certaines innovations phonologiques et lexicales:\footnote{Les traces de morphologie en lolo-birman sont malheureusement trop limitées pour permettre l'usage d'innovations morphologiques en l'état actuel des connaissances.}

\begin{itemize}
\item Certains changements phonétiques tels que la confusion des anciennes rimes *--a et *--al.
\item  La copule \tgz{0508} qui dérive vraisemblablement d'un verbe signifiant ``être vrai''.
\item Le supplétisme dans le nom de l'année (venant du nom de la ``récolte'', voir p.\pageref{tab:suppletisme:annee}).


\item Le nom ``poumon'' \tgz{5105} *S-tsvt < *S-tso-s venant du verbe ``tousser'' \tgz{4615} *S-tso. 

\item Une quantité appréciable d'étymons exclusivement partagés entre ces langues, analysés en partie dans \citet{jacques.michaud11naish}.

\end{itemize}

Le groupement birmo-qianguique s'oppose à l'hypothèse ``ronguique" de \citet{lapolla03}, selon lequel les langues à morphologie complexe du sino-tibétain (rgyalrong, kiranti, dulong/rawang et kham) formeraient un clade au sein de la famille. Il établit ce groupement uniquement sur la base de la morphologie, sans proposer aucune innovation lexicale ou phonologique.


Selon LaPolla, la morphologie commune à ces quatre groupes de langues serait une \textit{innovation commune} qui justifierait donc cette reconstruction. Curieusement, il en déduit ensuite que, toutes les langues à morphologie complexe appartenant à un même groupe, cette morphologie est tardive et que les langues sans système d'accord, telles que le tibétain et le lolo-birman, n'ont jamais connu de morphologie complexe. 

Ce raisonnement est toutefois dangereusement circulaire. Il n'explique pas la proximité frappante du lexique rgyalrong avec celui du tangoute et du lolo-birman démontré tout au long de ce travail, ni l'absence de vocabulaire commun exclusif entre rgyalrong et kiranti. La comparaison de l'\textit{ensemble} de notre dictionnaire des verbes khalings (qui comprend plus de 600 entrées) et de notre dictionnaire japhug (qui contient 1900 verbes) n'a révélé aucun cognat exclusif à ces langues qui ne se retrouve ni en tibétain, ni en birman, ni en chinois. 

En dépit de la ressemblance frappante de leurs systèmes d'accord, les langues kiranties et rgyalrongs n'ont que très peu de vocabulaire commun, et tous ces mots se retrouvent dans d'autres langues sino-tibétaines. 









\chapter{Conclusion}



Ce travail est la première étape vers la constitution d'un dictionnaire étymologique du tangoute et  l'établissement d'une grammaire comparée des langues macro-rgyalronguiques. Ce projet à plus long terme ne pourra toutefois être réalisé que lorsque les autres langues macro-rgyalronguiques dans leur ensemble auront été correctement décrites.

La priorité pour la linguistique historique est avant tout de mieux reconstruire la branche rgyalronguique du macro-rgyalronguique, qui est plus conservatrice que les autres au moins du point de vue du consonantisme et de la morphologie. Ces langues, comme nous l'avons montré à propos du tangoute, permettent d'interpréter des faits autrement opaques et arbitraires dans les langues moins conservatrices. Les autres branches du macro-rgyalronguique (muya, pumi, queyu, et peut-être qiang et zhaba) devront tout d'abord être comparées une à une au rgyalronguique en suivant la procédure adoptée dans notre travail et dans notre étude des langues naish \citet{jacques.michaud11naish}.

Comme le tangoute, ces langues ont en effet perdu les consonnes finales et la majorité des groupes initiaux. Or, il est pratiquement impossible de reconstituer ces éléments sur la seule base des correspondances phonétiques, sans données comparatives avec des langues plus conservatrices.

Les langues rgyalronguiques, relativement proches mais ayant mieux conservé ces traits phonétiques, offrent un point de comparaison idéal pour interpréter les correspondances complexes des voyelles et des attaques dans ces langues. Elle jouent de ce point de vue le même rôle que le birman au sein du lolo-birman ou le tibétain au sein des langues tibétiques (incluant le tamang  et le boumthang).

L'intérêt de notre travail sur le tangoute dépasse   la seule linguistique historique du sino-tibétain, mais concerne les philologues d'une part et les spécialistes généralistes de linguistique historique d'autre part.

Premièrement, l'étude de l'étymologie et de la morphologie historique   permet  d'affiner notre compréhension du sens de certains mots et de mieux analyser la structure des phrases. D'une part, notre étude des fonctions grammaticales des alternances vocaliques dans le système verbal et notre analyse de la structure du complexe verbal peuvent potentiellement contribuer à désambiguïser certaines phrases obscures. D'autre part, la comparaison avec le rgyalrong permet de mieux percevoir les nuances sémantiques des mots quasi-synonymes (voir en particulier les divers verbes qui peuvent se traduire par ``brûler''). Enfin, les lois phonétiques que nous avons mises au jour permettent désormais de savoir \textit{comment rechercher des cognats} en tibétain et en rgyalronguique. Si par exemple un mot obscur au sens mal documenté se retrouve dans les textes non-traduits du chinois, nos lois permettent d'imaginer les formes phonétiques potentielles de cognats en tibétain ou en rgyalronguique, et de rechercher si des formes apparentées existent. De même que l'étude du tocharien ou de l'anatolien ne peut pas se concevoir sans l'apport de la grammaire comparée indo-européenne, il est dommageable d'étudier le tangoute en isolation des langues auxquelles il est apparenté le plus directement.

Les recherches sur les textes tangoutes ont connu un progrès considérable durant les vingt  dernières années. Toutefois, comme la quasi-totalité des textes non-bouddhiques traduits du chinois sont désormais étudiés et traduits, il n'y a plus guère de progrès majeurs à attendre de l'étude philologique classique  basée sur la comparaison avec les textes originaux chinois: les progrès futurs viendront de la grammaire comparée des langues macro-rgyalronguiques.

Deuxièmement, les recherches sur la phonologie historique du tangoute offrent un nouvel exemple de changement typologique radical qui n'est pas sans intérêt pour la théorie générale des changements linguistiques. Le pré-tangoute que nous avons reconstruit ressemblait au rgyalronguique: c'était une langue à la structure syllabique  complexe, à la morphologie riche et essentiellement concaténative, et dont les mots était majoritairement polysyllabiques. 



Une série de changements phonétiques a bouleversé entièrement ce système. Les groupes de consonnes initiaux et les consonnes finales ont été progressivement perdus; ils devaient encore exister en partie au début de la dynastie Xixia, car les plus anciennes transcriptions chinoises suggèrent la préservation de consonnes finales. Les anciens polysyllabes sont devenus des monosyllabes, selon un schéma bien documenté dans les langues d'Asie en général (voir \citealt{michaud08monosyl}). Le tangoute diffère cependant des autres langues érodées de la famille sino-tibétaine telles que le lolo-birman, en cela que l'évolution phonétique n'a pas causé une perte radicale de la morphologie. Au contraire, l'ancienne morphologie concaténative, basée sur des préfixes et des suffixes, s'est transformée en une morphologie opaque exclusivement basée sur des alternances consonantiques et vocaliques. 

En dépit d'une phonologie très évoluée, le tangoute est donc paradoxalement une langue relativement conservatrice du point de vue morphologique. Sur de nombreux aspects, comme la flexion personnelle, elle préserve mieux les formes anciennes que le tibétain. C'est de ce point de vue une langue comparable au vieil irlandais en indo-européen, ou au blackfoot en algonquien.

L'hypothèse birmo-qianguique défendue dans ce travail implique que la morphologie complexe observée dans les langues rgyalrongs et partiellement reflétée par le tangoute est en grande partie un \textit{archaïsme}, qui a été perdu en lolo-birman et dans les langues naïques et ersu. Avec des langues peu conservatrices de ce type, il est vain d'entreprendre une reconstruction purement mécanique à partir des formes modernes. Il est plus approprié de bâtir un système de reconstruction de la phonologie et de la morphologie du birmo-qianguique basé sur les langues conservatrices (principalement les langues rgyalrongs), puis de projeter les alternances morphologiques ainsi postulées  en lolo-birman et en naish en utilisant les lois phonétiques connues. C'est cette méthode que nous avons employée tout au long de ce travail pour élucider la morphologie opaque du tangoute. 

Il ne s'agit en aucun cas d'une innovation méthodologique: les indo-européanistes et même les algonquianistes font amplement usage de la technique que \citet[5]{watkins62celtic} a appelé \textit{reconstructing forward}. Dans le cas où les changements morphologiques et phonétiques qu'a subis une langue particulière sont si considérables que les morphèmes deviennent méconnaissables, il est peu productif de projeter les formes modernes attestées dans une proto-langue, car la quantité d'information perdue est trop importante.

Il est préférable de partir de proto-formes attestées de façon indépendantes par des langues plus conservatrices, de leur appliquer mécaniquement les lois phonétiques connues étape par étape afin de déterminer un scénario probable de restructuration du système en terme de réanalyse et d'analogie. Cette technique a permis des progrès considérables dans la reconstruction du celtique, mais aussi de l'arapaho, langue algonquienne à la phonologie historique particulièrement complexe (\citealt{goddard98arapaho}).

La reconstruction du proto-birmo-qianguique ne pourra progresser qu'en combinant d'une part l'établissement progressif de lois phonétiques précises pour toutes les langues de ce groupe, et d'autre part la reconstruction du proto-rgyalronguique, qui seule permettra de guider la découverte de traces de morphologie en lolo-birman et en naïque.


%Malgré l'incertitude de nos reconstructions phonologiques du tangoute, ce travail documente un certain nombre d'évolutions phonétiques et de changements en chaîne qui peuvent contribuer à une théorie panchronique du changement linguistique. 

\fancyhead{}
\chapter{Annexes}
\thispagestyle{empty}
%⁴⁵⁶⁷⁸⁹��XXXXX
\section{Tableau comparatif des rimes du tangoute}

\begin{longtable} {lllllllll}
\captionb{cycle majeur}\label{tab:cyclemajeur} \\
\toprule
rime&	ton 1&	ton 2&	Sof.1&	Sof.2&	Ni.&	Li&	Gong&	Ar.\\	
\midrule
\endfirsthead
1&	1.1&	2.1&	\ipa{u}&	\ipa{u}&	\ipa{u}&	\ipa{u}&	\ipa{u}&	\ipa{u}\\	
2&	1.2&	2.2&	\ipa{i̯u}&	\ipa{i̯u}&	\ipa{ǐu}&	\ipa{ǐu}&	\ipa{ju}&	\ipa{yu}\\	
3&	1.3&	2.3&	\ipa{Y}&	\ipa{i̯u}&	\ipa{ǐuɦ}&	\ipa{ǐu̠}&	\ipa{ju}&	\ipa{yu}\\	
4&	1.4&	2.4&	\ipa{u+C}&	\ipa{u}&	\ipa{uɦ}&	\ipa{uo}&	\ipa{u}&	\ipa{uː}\\	
5&	1.5&	2.5&	\ipa{un}&	\ipa{u}&	\ipa{ʊ}&	\ipa{ʊ}&	\ipa{uu}&	\ipa{u’}\\	
6&	1.6&	&	\ipa{ûn}&	\ipa{û}&	\ipa{ʊɦ}&	\ipa{ǐʊ}&	\ipa{juu}&	\ipa{yu’}\\	
7&	1.7&	2.6&	\ipa{i̯un}&	\ipa{i̯u}&	\ipa{ǐʊɦ}&	\ipa{ǐʊ̠}&	\ipa{juu}&	\ipa{uː’}\\	
8&	1.8&	2.7&	\ipa{e}&	\ipa{e}&	\ipa{ɪ ʷɪ}&	\ipa{e ʊe}&	\ipa{e}&	\ipa{i}\\	
9&	1.9&	2.8&	\ipa{ê}&	\ipa{ê}&	\ipa{ɪě}&	\ipa{e̠}&	\ipa{ie}&	\ipa{yi}\\	
10&	1.10&	2.9&	\ipa{i̯e}&	\ipa{i̯e}&	\ipa{i}&	\ipa{ie}&	\ipa{ji}&	\ipa{iː}\\	
11&	1.11&	2.10&	\ipa{i}&	\ipa{i}&	\ipa{iɦ ʷiɦ}&	\ipa{i}&	\ipa{ji}&	\ipa{iː}\\	
12&	1.12&	2.11&	\ipa{e+C}&	\ipa{e}&	\ipa{ʷɪɦ}&	\ipa{uɪ}&	\ipa{ee}&	\ipa{i’}\\	
13&	1.13&	&	\ipa{ê+C}&	\ipa{ê}&	\ipa{ʷɪɦ²}&	\ipa{ue}&	\ipa{iee}&	\ipa{yi’}\\	
14&	1.14&	2.12&	\ipa{i̯e+C}&	\ipa{i̯e}&	\ipa{ɪɦ}&	\ipa{ǐei}&	\ipa{jii}&	\ipa{iː’}\\	
15&	1.15&	2.13&	\ipa{en}&	\ipa{en}&	\ipa{əɴ ʷəɴ}&	\ipa{ẽ}&	\ipa{ẽ}&	\ipa{in}\\	
16&	1.16&	&	\ipa{ên}&	\ipa{ên}&	\ipa{ǐəɴ ʷǐəɴ}&	\ipa{ǐẽ}&	\ipa{jĩ}&	\ipa{yin}\\	
17&	1.17&	2.14&	\ipa{a}&	\ipa{a}&	\ipa{ɑɦ}&	\ipa{a̠}&	\ipa{a}&	\ipa{a}\\	
18&	1.18&	2.15&	\ipa{â}&	\ipa{â}&	\ipa{a}&	\ipa{ǐa}&	\ipa{ia}&	\ipa{ya}\\	
19&	1.19&	2.16&	\ipa{i̯a}&	\ipa{i̯a}&	\ipa{ǐa}&	\ipa{ǐa̠}&	\ipa{ja}&	\ipa{aː}\\	
20&	1.20&	2.17&	\ipa{a+C}&	\ipa{a}&	\ipa{aɦ}&	\ipa{ɑ̠}&	\ipa{ja}&	\ipa{aː}\\	
21&	1.21&	2.18&	\ipa{i̯a+C}&	\ipa{i̯aɯ}&	\ipa{ǐaɦ}&	\ipa{ǐɑ}&	\ipa{jaa}&	\ipa{yaː}\\	
22&	1.22&	2.19&	\ipa{a+C}&	\ipa{aɯ}&	\ipa{ɑɯ}&	\ipa{ǔɑ}&	\ipa{aa}&	\ipa{a’}\\	
23&	&	2.20&	\ipa{â+C}&	\ipa{âɯ}&	\ipa{aɯ}&	\ipa{ɑr}&	\ipa{iaa}&	\ipa{ya’}\\	
24&	1.23&	2.21&	\ipa{i̯a+C}&	\ipa{i̯aɯ}&	\ipa{ɑ}&	\ipa{ɑ}&	\ipa{jaa}&	\ipa{aː’}\\	
25&	1.24&	2.22&	\ipa{an}&	\ipa{an}&	\ipa{ɑɴ}&	\ipa{uɑ̃}&	\ipa{ã}&	\ipa{an}\\	
26&	1.25&	2.23&	\ipa{ân}&	\ipa{ân}&	\ipa{aɴ}&	\ipa{ɑ̃}&	\ipa{iã}&	\ipa{yan}\\	
27&	1.26&	2.24&	\ipa{i̯an}&	\ipa{i̯an}&	\ipa{ǐaɴ}&	\ipa{ǐɑ̃}&	\ipa{jã}&	\ipa{aːn}\\	
28&	1.27&	2.25&	\ipa{ə}&	\ipa{ə}&	\ipa{ʉɦ}&	\ipa{ə̠}&	\ipa{ə}&	\ipa{I}\\	
29&	1.28&	2.26&	\ipa{ə̂}&	\ipa{ə̂}&	\ipa{ʊ}&	\ipa{ə}&	\ipa{jə}&	\ipa{yI}\\	
30&	1.29&	2.27&	\ipa{i̯ə}&	\ipa{i̯ə}&	\ipa{ɨɦ}&	\ipa{ɪ̠}&	\ipa{jɨ}&	\ipa{Iː}\\	
31&	1.30&	2.28&	\ipa{I}&	\ipa{I}&	\ipa{ɨ ʷɨ}&	\ipa{ɪ̠ uɪ̠}&	\ipa{jɨ}&	\ipa{Iː}\\	
32&	1.31&	&	\ipa{ə+C}&	\ipa{}&	\ipa{ʉɴ}&	\ipa{ə̠̂ uə̠̂}&	\ipa{əə}&	\ipa{I’}\\	
33&	1.32&	2.29&	\ipa{i̯ə+C}&	\ipa{i̯ə}&	\ipa{ɨɴ}&	\ipa{ǐə̠}&	\ipa{jɨɨ}&	\ipa{yI’}\\	
34&	1.33&	2.30&	\ipa{ai}&	\ipa{ei}&	\ipa{ɛ ʷɛ}&	\ipa{ɛ uɛ}&	\ipa{ej}&	\ipa{e}\\	
35&	1.34&	2.31&	\ipa{ai}&	\ipa{ei}&	\ipa{iɛ}&	\ipa{ɛǐ}&	\ipa{iej}&	\ipa{ye}\\	
36&	1.35&	2.32&	\ipa{i̯ai}&	\ipa{i̯ei}&	\ipa{ɛɦ}&	\ipa{ǐɛ̃}&	\ipa{jij}&	\ipa{eː}\\	
37&	1.36&	2.33&	\ipa{In}&	\ipa{In}&	\ipa{eɦ ʷeɦ}&	\ipa{ẽ}&	\ipa{jij}&	\ipa{eː}\\	
38&	1.37&	2.33&	\ipa{ai+C}&	\ipa{ai}&	\ipa{e ʷe}&	\ipa{ǐe}&	\ipa{eej}&	\ipa{e’}\\	
39&	1.38&	&	\ipa{ai+C}&	\ipa{ai}&	\ipa{eʸ}&	\ipa{e}&	\ipa{ieej}&	\ipa{ye’}\\	
40&	1.39&	2.35&	\ipa{i̯ai+C}&	\ipa{i̯e}&	\ipa{ǐeɦ}&	\ipa{ɪẽ}&	\ipa{jiij}&	\ipa{eː’}\\	
41&	1.40&	&	\ipa{ai+C}&	\ipa{ai}&	\ipa{ǐe}&	\ipa{ɛ}&	\ipa{əj}&	\ipa{eː’}\\	
42&	1.41&	2.36&	\ipa{ai+C}&	\ipa{ai}&	\ipa{eɴ}&	\ipa{ɪɛ}&	\ipa{iəj}&	\ipa{en}\\	
43&	1.42&	2.37&	\ipa{i̯ai+C}&	\ipa{i̯e}&	\ipa{ieɴ}&	\ipa{ɪɛ̃}&	\ipa{jɨj}&	\ipa{yen}\\	
44&	1.43&	2.38&	\ipa{eɯ}&	\ipa{eɯ}&	\ipa{əw}&	\ipa{e̠}&	\ipa{ew}&	\ipa{eu}\\	
45&	1.44&	2.39&	\ipa{êɯ}&	\ipa{êɯ}&	\ipa{ew}&	\ipa{eʊ}&	\ipa{iew}&	\ipa{yeu}\\	
46&	1.45&	2.40&	\ipa{i̯eɯ}&	\ipa{i̯eɯ}&	\ipa{ǐəw}&	\ipa{ieʊ}&	\ipa{jiw}&	\ipa{euː}\\	
47&	1.46&	&	\ipa{i̯eɯ}&	\ipa{i̯eɯ}&	\ipa{ǐew}&	\ipa{ǐeʊ}&	\ipa{jiw}&	\ipa{euː}\\	
48&	&	2.41&	\ipa{əɯ}&	\ipa{əɯ}&	\ipa{ǐew}&	\ipa{eu̠}&	\ipa{eew}&	\ipa{eu’}\\	
49&	1.47&	&	\ipa{i̯əɯ}&	\ipa{i̯əɯ}&	\ipa{iw}&	\ipa{ǐeu̠}&	\ipa{jiiw}&	\ipa{yeu’}\\	
50&	1.48&	&	\ipa{?}&	\ipa{i̯o}&	\ipa{oɦ}&	\ipa{ǐõ}&	\ipa{jwo}&	\ipa{o}\\	
51&	1.49&	2.42&	\ipa{o}&	\ipa{o}&	\ipa{ɔɦ}&	\ipa{o}&	\ipa{o}&	\ipa{o}\\	
52&	1.50&	2.43&	\ipa{o}&	\ipa{o}&	\ipa{ǐou}&	\ipa{ou}&	\ipa{io}&	\ipa{yo}\\	
53&	1.51&	2.44&	\ipa{i̯o}&	\ipa{i̯o}&	\ipa{ǐɔɦ}&	\ipa{ǐo}&	\ipa{jo}&	\ipa{oː}\\	
54&	1.52&	2.45&	\ipa{o+C}&	\ipa{oɯ}&	\ipa{ɔw}&	\ipa{ɔ}&	\ipa{oo}&	\ipa{o’}\\	
55&	1.53&	2.46&	\ipa{i̯o+C}&	\ipa{i̯oɯ}&	\ipa{ow}&	\ipa{uɔ}&	\ipa{ioo}&	\ipa{yo’}\\	
56&	1.54&	2.47&	\ipa{on}&	\ipa{on}&	\ipa{oɴ}&	\ipa{ɔ̃}&	\ipa{ow}&	\ipa{on}\\	
57&	1.55&	2.48&	\ipa{on}&	\ipa{on}&	\ipa{ǐoɴ}&	\ipa{iɔ̃}&	\ipa{iow}&	\ipa{yon}\\	
58&	1.56&	2.49&	\ipa{i̯on}&	\ipa{i̯on}&	\ipa{ǐʷoɴ}&	\ipa{uɔ̃}&	\ipa{jow}&	\ipa{oːn}\\	
59&	1.57&	&	\ipa{i̯uo}&	\ipa{i̯uo}&	\ipa{ǐʷo}&	\ipa{iɔ}&	\ipa{ioow}&	\ipa{o’’}\\	
60&	&	2.50&	\ipa{i̯uo+C}&	\ipa{i̯uo}&	\ipa{ǐʷo}&	\ipa{ǐɔ̠}&	\ipa{joow}&	\ipa{yo’’}\\	\bottomrule
\end{longtable}


\begin{table}[H]
\captionb{premier cycle mineur} %\label{tab:cyclemineur1} 
\resizebox{\columnwidth}{!}{
\begin{tabular} {lllllllll}
\toprule
rime&	ton 1&	ton 2&	Sof.1&	Sof.2&	Ni.&	Li&	Gong&	Ar.\\	
\midrule
61&	1.58&	2.51&	\ipa{ụ}&	\ipa{ụ}&	\ipa{ụ}&	\ipa{ụ}&	\ipa{ụ}&	\ipa{uq}\\
62&	1.59&	2.52&	\ipa{i̯ụ}&	\ipa{i̯ụ}&	\ipa{ǐụ}&	\ipa{ǐu̠̣}&	\ipa{jụ}&	\ipa{yuq}\\
63&	1.60&	2.53&	\ipa{?}&	\ipa{ại}&	\ipa{ǐɛ̣}&	\ipa{ɛ̣}&	\ipa{iẹj}&	\ipa{yeq}\\
64&	1.61&	2.54&	\ipa{?}&	\ipa{i̯ẹ}&	\ipa{ɛ̣}&	\ipa{ǐɛ̣̃}&	\ipa{jịj}&	\ipa{enq}\\
65&	1.62&	2.55&	\ipa{?}&	\ipa{Ị}&	\ipa{ɛ̣ɴ}&	\ipa{ɪẹ̃}&	\ipa{jɨ̣j}&	\ipa{yenq}\\
66&	1.63&	2.56&	\ipa{ạ}&	\ipa{ạ}&	\ipa{ɑ̣}&	\ipa{a̠̣}&	\ipa{ạ}&	\ipa{aq}\\
67&	1.64&	2.57&	\ipa{i̯ạ}&	\ipa{i̯ạ}&	\ipa{ạ}&	\ipa{ǐa̠̣}&	\ipa{jạ}&	\ipa{aːq}\\
68&	1.65&	2.58&	\ipa{ại}&	\ipa{ẹi}&	\ipa{ɪ̣}&	\ipa{ẹ uẹ}&	\ipa{ẹ}&	\ipa{iq}\\
69&	1.66&	2.59&	\ipa{ại}&	\ipa{ẹi}&	\ipa{ɪ̣ě}&	\ipa{e̠̣}&	\ipa{iẹ}&	\ipa{yiq}\\
70&	1.67&	2.60&	\ipa{i̯ại}&	\ipa{i̯ẹi}&	\ipa{ị ʷị}&	\ipa{ǐẹi}&	\ipa{jị}&	\ipa{iːq}\\
71&	1.68&	&	\ipa{?}&	\ipa{ə̣}&	\ipa{ʉ̣}&	\ipa{ə̠̣}&	\ipa{ə̣}&	\ipa{iq’}\\
72&	1.69&	2.61&	\ipa{?}&	\ipa{i̯ə̣}&	\ipa{ɨ̣}&	\ipa{ǐə̠̣}&	\ipa{jɨ̣}&	\ipa{iːq’}\\
73&	1.70&	2.62&	\ipa{ọ}&	\ipa{ọ}&	\ipa{ɔ̣}&	\ipa{ọ}&	\ipa{ọ}&	\ipa{oq}\\
74&	1.71&	2.63&	\ipa{ọn}&	\ipa{ọn}&	\ipa{ọɴ}&	\ipa{uọ}&	\ipa{iọ}&	\ipa{onq}\\
75&	1.72&	2.64&	\ipa{i̯ọn}&	\ipa{i̯ọn}&	\ipa{ǐoɴ}&	\ipa{ǐọ̃}&	\ipa{jọ}&	\ipa{yonq}\\
76&	&	2.65&	\ipa{}&	\ipa{}&	\ipa{ʷẹ}&	\ipa{ọ̃}&	\ipa{iə̣j}&	\ipa{eq2}\\
\bottomrule
\end{tabular}}
\end{table}

\begin{table}[H]
\captionb{second cycle mineur} %\label{tab:cyclemineur2} 
\resizebox{\columnwidth}{!}{
\begin{tabular} {lllllllll}
\toprule
rime&	ton 1&	ton 2&	Sof.1&	Sof.2&	Ni.&	Li&	Gong&	Ar.\\	
\midrule
77&	1.73&	2.66&	\ipa{?}&	\ipa{ại}&	\ipa{ẹ}&	\ipa{ẹ̃}&	\ipa{ejr}&	\ipa{yeq2}\\
78&	&	2.67&	\ipa{?}&	\ipa{ại}&	\ipa{ʷǐẹ}&	\ipa{uẹ}&	\ipa{iejr}&	\ipa{eq''}\\
79&	1.74&	2.68&	\ipa{?}&	\ipa{i̯e}&	\ipa{ǐẹ}&	\ipa{ǐẹ̃ uẹ̃}&	\ipa{jijr}&	\ipa{yeq''}\\
80&	1.75&	2.69&	\ipa{ụ}&	\ipa{ụ}&	\ipa{ur}&	\ipa{ʊ̣}&	\ipa{ur}&	\ipa{ur}\\
81&	1.76&	2.70&	\ipa{i̯ụ}&	\ipa{i̯ụ}&	\ipa{ǐur}&	\ipa{ǐʊ̠̣}&	\ipa{jur}&	\ipa{yur}\\
82&	1.77&	2.71&	\ipa{ẹ}&	\ipa{ẹ}&	\ipa{ɪr}&	\ipa{ẹ}&	\ipa{er}&	\ipa{ir}\\
83&	1.78&	&	\ipa{}&	\ipa{}&	\ipa{iěr}&	\ipa{ǐẹ}&	\ipa{ier}&	\ipa{yir}\\
84&	1.79&	2.72&	\ipa{i̯ẹ+C}&	\ipa{i̯ẹ}&	\ipa{ir}&	\ipa{ǐe̠̣}&	\ipa{jir}&	\ipa{iːr}\\
85&	1.80&	2.73&	\ipa{ạ}&	\ipa{ạ}&	\ipa{ɑr}&	\ipa{ɑ̣}&	\ipa{ar}&	\ipa{ar}\\
86&	1.81&	&	\ipa{â}&	\ipa{â}&	\ipa{ǐɑr}&	\ipa{ar}&	\ipa{iar}&	\ipa{yar}\\
87&	1.82&	2.74&	\ipa{i̯ạ}&	\ipa{i̯ạ}&	\ipa{ar}&	\ipa{ǎr}&	\ipa{jwar}&	\ipa{aːr}\\
88&	1.83&	&	\ipa{ạ+C}&	\ipa{ạɯ}&	\ipa{ar²}&	\ipa{a̠̣}&	\ipa{aar}&	\ipa{ar’}\\
89&	&	2.75&	\ipa{i̯ạ+C}&	\ipa{i̯ạɯ}&	\ipa{ǐar}&	\ipa{ạu}&	\ipa{jaar}&	\ipa{yar’}\\
90&	1.84&	2.76&	\ipa{}&	\ipa{}&	\ipa{ʉr}&	\ipa{uə̠̣}&	\ipa{ər}&	\ipa{Ir}\\
91&	1.85&	&	\ipa{ə̣}&	\ipa{ə̣}&	\ipa{ər}&	\ipa{ue̠̣}&	\ipa{iər}&	\ipa{yIr}\\
92&	1.86&	2.77&	\ipa{i̯ə̣}&	\ipa{i̯ə̣}&	\ipa{ɨr}&	\ipa{er}&	\ipa{jɨr}&	\ipa{Iːr}\\
93&	1.87&	2.78&	\ipa{ẹɯ}&	\ipa{ẹɯ}&	\ipa{iə̣r}&	\ipa{ǐə̠̣}&	\ipa{ewr}&	\ipa{er}\\
94&	1.88&	2.79&	\ipa{ə̣ɯ}&	\ipa{ə̣ɯ}&	\ipa{iʉr}&	\ipa{ǐə̣}&	\ipa{jiwr}&	\ipa{yer}\\
95&	1.89&	2.80&	\ipa{ụo}&	\ipa{ụo}&	\ipa{or}&	\ipa{uɔ̣}&	\ipa{or}&	\ipa{or}\\
96&	1.90&	2.81&	\ipa{i̯ụo}&	\ipa{i̯ụo}&	\ipa{ǐor}&	\ipa{iɔu}&	\ipa{jor/ior}&	\ipa{yor}\\
97&	1.91&	2.82&	\ipa{ụo+C}&	\ipa{ụo}&	\ipa{ɔr}&	\ipa{ɔ̣}&	\ipa{owr}&	\ipa{oːr}\\
98&	&	2.83&	\ipa{i̯ụo+C}&	\ipa{i̯ụo}&	\ipa{wor}&	\ipa{ǐɔ̣}&	\ipa{jowr}&	\ipa{wor}\\
\bottomrule
\end{tabular}}
\end{table}

\begin{table}[H]
\captionb{troisième cycle mineur} %\label{tab:cyclemineur3} \\
\resizebox{\columnwidth}{!}{
\begin{tabular} {lllllllll}
\toprule
rime&	ton 1&	ton 2&	Sof.1&	Sof.2&	Ni.&	Li&	Gong&	Ar.\\	
\midrule
99&	&	2.84&	\ipa{ẹ}&	\ipa{ẹ}&	\ipa{ʷɔr}&	\ipa{ǐẹ}&	\ipa{eer}&	\ipa{ywor}\\
100&	1.92&	2.85&	\ipa{i̯ẹ}&	\ipa{i̯ẹ}&	\ipa{ɨr}&	\ipa{ǐe̠̣}&	\ipa{jɨɨr}&	\ipa{yIr}\\
101&	1.93&	2.86&	\ipa{}&	\ipa{Ị}&	\ipa{ǐə̣r}&	\ipa{ǐẹ̃}&	\ipa{jiir}&	\ipa{yer2}\\
102&	1.94&	&	\ipa{ọ}&	\ipa{ọ}&	\ipa{ʷọ}&	\ipa{ɑ̣}&	\ipa{oor}&	\ipa{woq2}\\
103&	1.95&	&	\ipa{i̯ọ}&	\ipa{i̯ọ}&	\ipa{ǐɑɴ}&	\ipa{ǐɑ̣̃}&	\ipa{joor}&	\ipa{yaːn}\\
104&	1.96&	&	\ipa{ọn}&	\ipa{ọn}&	\ipa{}&	\ipa{ɑ̣̃}&	\ipa{ũ}&	\ipa{un}\\
105&	1.97&	&	\ipa{i̯ọn}&	\ipa{i̯ọn}&	\ipa{}&	\ipa{ǐuɑ̣̃}&	\ipa{jwar}&	\ipa{ua}\\
\bottomrule
\end{tabular}}
\end{table}





\bibliographystyle{linquiry2}
\bibliography{bibliogj}
\printindex
\end{document}
