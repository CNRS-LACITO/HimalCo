\documentclass[oneside,a4paper,11pt]{article} 
\usepackage{fontspec}
\usepackage{natbib}
\usepackage{booktabs}
\usepackage{xltxtra} 
\usepackage{polyglossia} 
\usepackage[table]{xcolor}
\usepackage{gb4e} 
\usepackage{multicol}
\usepackage{graphicx}
\usepackage{float}
\usepackage{hyperref} 
\hypersetup{bookmarksnumbered,bookmarksopenlevel=5,bookmarksdepth=5,colorlinks=true,linkcolor=blue,citecolor=blue}
\usepackage[all]{hypcap}
\usepackage{memhfixc}
\usepackage{lscape}
\usepackage{amssymb}
 
\bibpunct[: ]{(}{)}{,}{a}{}{,}

%\setmainfont[Mapping=tex-text,Numbers=OldStyle,Ligatures=Common]{Charis SIL} 
\newfontfamily\phon[Mapping=tex-text,Scale=MatchLowercase]{Charis SIL} 
\newcommand{\ipa}[1]{\textbf{{\phon\mbox{#1}}}} %API tjs en italique
%\newcommand{\ipab}[1]{{\scriptsize \phon#1}} 

\newcommand{\grise}[1]{\cellcolor{lightgray}\textbf{#1}}
\newfontfamily\cn[Mapping=tex-text,Ligatures=Common,Scale=MatchUppercase]{SimSun}%pour le chinois
\newcommand{\zh}[1]{{\cn #1}}
\newfontfamily\mleccha[Mapping=tex-text,Ligatures=Common,Scale=MatchLowercase]{Galatia SIL}%pour le grec
\newcommand{\grec}[1]{{\mleccha #1}}


\newcommand{\sg}{\textsc{sg}}
\newcommand{\pl}{\textsc{pl}}
\newcommand{\ro}{$\Sigma$}
\newcommand{\ra}{$\Sigma_1$} 
\newcommand{\rc}{$\Sigma_3$}   
\newcommand{\change}[2]{*\ipa{#1} $\rightarrow$ \ipa{#2}}
 

\XeTeXlinebreakskip = 0pt plus 1pt %
 %CIRCG
 
\newcommand{\forme}[2]{\ipa{#1} `#2'}  
\newcommand{\refb}[1]{(\ref{#1})}
\newcommand{\tld}{\textasciitilde{}}
\newcommand{\zhc}[2]{\zh{#1} \ipa{#2}} 
\newcommand{\oc}[3]{\zhc{#1}{#2} `#3'} 
\newcommand{\fanqie}[1]{(\zh{(#1切)})}
\newcommand{\mc}[1]{\zh{(#1)}}
\newcommand{\mien}[5]{\ipa{#1}^{#2} `#3' (\zh{#4}, p.#5)} 


\begin{document}

\title{The Sino-Tibetan origins of the Middle Chinese voicing alternation}
\author{Guillaume Jacques}
\maketitle


%Abstract: Among the traces of morphology that remain in Middle Chinese, one of the clearest one is the transivity-related voicing alternation (\citealt[79-80]{zhou62goucibian}, \citealt{downer73loanwords}, \citealt{sagart03prenasalized}), exemplified by verbs such as \zhc{敗}{pæjH} `defeat' and \zhc{敗}{bæjH} `be defeated'. Consensus does not yet exist on the proto-Trans-Himalayan origins of this alternation (\citealt{handel12valence}): according to the view of some scholars (for instance \citealt{mei12caus}) the intransitive voiced verb is the base form, and the transitive one is its causative (with a sigmatic causative prefix devoicing the initial), while other scholars (\citealt{sagart12sprefix}) argue that this alternation is unrelated to the causative prefix, and that the base form is the transitive verb instead.
%
%In this paper, drawing on data from more conservative and morphologically complex languages of the family such as Gyalrongic (in particular \citealt{jacques15causative, lai16caus, gong17xingtaixue}), Jinghpo and Tibetan, I show that the first view is untenable, but that the second view must be slightly revised: the voicing alternation in Old Chinese reflects the merger of the ancient anticausative derivation (which appears as prenasalization in Gyalrongic, and may be related to the autobenefactive/spontaneous prefix, see  \citealt{jacques15spontaneous}) and the passive \ipa{*ŋa-} derivation (\citealt{jacques07passif, jacques12demotion}), the latter explaining cases like \zhc{現}{ɣenH} `to appear' which make no sense as anticausatives.
%
%Following on  \citet{jacques16ssuffixes}, this paper suggests that the apparently rudimentary state of Old Chinese morphology is in fact an illusion due to the massive phonological mergers that took place, and obscured the nature of morphological alternations.  

 \section*{Introduction}
 \section{The voicing alternation in Old Chinese}
Consensus does not yet exist on the proto-Trans-Himalayan origins of this alternation (\citealt{handel12valence}): according to the view of some scholars (for instance \citealt{mei12caus}) the intransitive voiced verb is the base form, and the transitive one is its causative (with a sigmatic causative prefix devoicing the initial), while other scholars (\citealt{sagart12sprefix}) argue that this alternation is unrelated to the causative prefix, and that the base form is the transitive verb instead.
 
 
 In the following, I review several proposed functions of the voicing alternation in Old Chinese.
 
 \subsection{Transitivity alternation}
 The most discussed function of the voicing alternation in Old Chinese is the case of verb pairs, in which the unvoiced counterpart is transitive, and the voiced one intransitive. Table \ref{tab:voicing.transitivity} presents examples from several sources (\citealt[79-80; 86]{zhou62goucibian}, \citealt{downer73loanwords}, \citealt{sagart03prenasalized}). \citet[86]{zhou62goucibian} analyzes some of these examples as \zh{既事式}  `perfect', by which is presumably implied a resultative meaning.
 
\begin{table}[H]
\caption{Voicing alternation and transitivity} \label{tab:voicing.transitivity}
\begin{tabular}{llllllllll}
\toprule
Unvoiced form &Meaning & Voiced form & Meaning\\
\midrule
\zhc{别}{pjet} \mc{幫}	&separate & \zhc{别}{bjet} \mc{並} & take leave \\
\zhc{敗}{pæjH} \mc{幫}	&defeat & \zhc{敗}{bæjH} \mc{並} & be defeated \\
\midrule
\zhc{斷}{twanH} \mc{端}	&cut & \zhc{斷}{dwanH} \mc{並} & be cut, stop \\
\midrule
\zhc{折}{tɕet} \mc{章}	&break (tr) & \zhc{折}{dʑet} \mc{常} &break (it), be broken, \\
&&&die young \\
\zhc{屬}{tɕuwk} \mc{章}	&compose & \zhc{屬}{dʑuwk} \mc{常} &belong to \\
\midrule 
\zhc{見}{kenH} \mc{見}	&see (tr) & \zhc{現}{ɣenH} \mc{匣} & appear \\
\zhc{繫}{kejH} \mc{見}	&tie & \zhc{繫}{ɣejH} \mc{匣} & be tied \\
\zhc{解}{kɛɨX} \mc{見}	&untie & \zhc{解}{ɣɛɨX} \mc{匣} & be loosened \\
\zhc{會}{kwajH} \mc{見}	&gather (vt) & \zhc{會}{ɣwajH} \mc{匣} & gather (vi) \\
\zhc{壞}{kwɛjH} \mc{見}	&destroy  & \zhc{壞}{ɣwɛjH} \mc{匣} & be destroyed \\
\zhc{夾}{kɛp} \mc{見}	&hold between,   & \zhc{狹}{ɣɛp} \mc{匣} & be narrow \\
&press from both sides&&\\
\bottomrule
\end{tabular}
\end{table}

 \citet{downer73loanwords} and \citet{sagart03prenasalized} point out in addition the existence of 
a few verb pairs with voicing in Mien loanwords from Chinese, in particular:\footnote{All Mien forms in the paper are taken from Mao Zongwu's \citeyear{maozw92mien} dictionary, with page number and original gloss.} 

\begin{enumerate}
\item \mien{khoːi}{1}{open (tr)}{打开}{131}  vs \mien{goːi}{1}{open (it)}{开}{131}  (Mao Zongwu notes `\zh{多作补语祸自然开裂、开放}')
\item \mien{tshɛʔ}{7}{pull down (house)}{拆}{120} vs \mien{dzɛʔ}{7}{be cracked (of earth, of skin)}{拆开;皴}{151}
\end{enumerate}

Prenasalization in Mien is uncontroversially reconstructed as prenasalization (\citealt{ratliff10protohm}), and voiced obstruents with the low register tones (1, 3, 5, 7) originate from nasal+unvoiced stop or nasal+unvoiced aspirated stop in pre-Mien, and these Mien examples support the idea that the intransitive counterpart derives from the transitive one by addition of a nasal element. These two examples are also interesting in showing prenasalization with verbs that has aspirated unvoiced initials in Chinese; note that none of the example in Table \ref{tab:voicing.transitivity} has an aspirated initial, a fact which \citet{sagart03prenasalized} interprets as resulting from a sound change whereby prenasalization is lost in MC in the case of aspirated stops.

Transitivity alternations in Chinese loanwords in Mien between unvoiced unaspirated stops and voiced stops are also found, as shown by \mien{pɛːŋ}{1}{to stretch tight, to pull tight}{绷;拉紧;拔}{120} vs \mien{bɛːŋ}{1}{to crack}{裂开}{150-1}, a fact expected in the theory that the intransitive form derives from the transitive one, but unexplainable in the alternative theory.


\subsection{Volitional}
\citet[55;131-5]{bs14oc}
\begin{table}[H]
\caption{Voicing alternation and transitivity/volition?}
\begin{tabular}{llllllllll}
\toprule
Unvoiced form &Meaning & Voiced form & Meaning\\
\midrule
\zhc{覺}{kæwH} \mc{見}	&awaken & \zhc{學}{ɣæwk} \mc{匣} & study, imitate \\
\zhc{見}{kenH} \mc{見}	&see & \zhc{見}{ɣenH} \mc{匣} & cause to appear, introduce \\
\zhc{晶}{tsjeŋ} \mc{見}	&bright, limpid & \zhc{淨}{dzjeŋH} \mc{匣} & cleanse\\
\zhc{平}{bjæŋ} \mc{並}	&be flat, be even & \zhc{平}{bjæŋ} \mc{並}	 & make even \\
\multicolumn{2}{c}{Xiamen \ipa{pĩ²} (pMi. \ipa{*b-})} & \multicolumn{2}{c}{Xiamen  \ipa{pʰĩ²}  (pMi. \ipa{*bh-} $\leftarrow$ \ipa{*m-b-})} \\
\zhc{上}{dʑaŋX} \mc{禪}	&ascend & \zhc{上}{dʑaŋX} \mc{禪}	 & put up \\
\multicolumn{2}{c}{Xiamen \ipa{tsiũ^6}  (pMi. \ipa{*-dž-})} & \multicolumn{2}{c}{Xiamen  \ipa{tsʰiũ^6}  (pMi. \ipa{*džʰ-} $\leftarrow$ \ipa{*m-b-})} \\
\bottomrule
\end{tabular}
\end{table}

\citet[282-4]{wangyt14jingdian}
\subsection{Nominalization}
Another function of the voicing alternation proposed by \citet[55]{bs14oc} is the derivation of an `agentive/instrumental noun' from a (transitive) verb, as shown by the examples in Table \ref{tab:nmlz.voicing.oc}.


\begin{table}[H]
\caption{Voicing alternation and nominalization} \label{tab:nmlz.voicing.oc}
\begin{tabular}{llllllllll}
\toprule
Unvoiced form &Meaning & Voiced form & Meaning\\
\midrule
\zhc{包}{pæw} \mc{幫}	&wrap, bundle & \zhc{袍}{baw} \mc{並} & long robe \\
\zhc{判}{phanH} \mc{滂}	&divide & \zhc{畔}{banH} \mc{並} & bank between fields \\
\zhc{拄}{ʈjuX} \mc{知}	&prop up, support  & \zhc{柱}{ɖjuX} \mc{澄} & pillar \\
\zhc{稱}{tɕʰiŋ} \mc{昌}	&weigh; evaluate; call & \zhc{稱}{tɕʰiŋH} \mc{昌} & steelyard \\
\bottomrule
\end{tabular}
\end{table}

The last example in Table \ref{tab:nmlz.voicing.oc} does not present voicing alternation in Middle Chinese, but \citet{bs14oc} propose that a nominalization prefix \ipa{*mə-} similar to the one hypothesized in the other three examples has to be reconstructed for the noun \zhc{稱}{tɕʰiŋH} `steelyard' (for which their OC reconstruction is \ipa{*mə-tʰəŋ-s}), on the basis of Hmong-Mien forms like Jiangdi Mien \ipa{dzjaŋ^5} `steelyard' (\citealt[68]{maozw92mien}, \zh{秤}), whose initial \ipa{dz-} unproblematically originates from a prenasalized aspirated affricate. However, the base verb \ipa{dzjaŋ^1} `weigh' (\citealt[167]{maozw92mien}, \zh{称}) is also attested in Mien with a voiced initial, indicating that the Chinese donor had prenasalization in both the verb and the noun. What this pair reflects is simply a case of \textit{qùshēng} nominalization (on which see \citealt{downer59}, \citealt{jacques16ssuffixes}), and cannot be used to prove that a nasal nominalization prefix did exist in Old Chinese.


\subsection{Denominal verbs}
\citet[55]{bs14oc}



\begin{table}[H]
\caption{Voicing alternation and verbalization}
\begin{tabular}{llllllllll}
\toprule
Unvoiced form &Meaning & Voiced form & Meaning\\
\midrule
\zhc{背}{pwojH} \mc{幫}	&the back& \zhc{背}{bwojH} \mc{並} & turn the back on \\
\zhc{倉}{tsʰaŋ} \mc{清}	&granary & \zhc{藏}{dzaŋ} \mc{從} & store \\
\zhc{朝}{ʈjew} \mc{知}	&morning  & \zhc{朝}{ɖjew} \mc{澄} & go to (morning) audience at court \\
\bottomrule
\end{tabular}
\end{table}


\section{The sigmatic causative in Trans-Himalayan}

\subsection{The Tibetan model}

\subsection{Rgyalrongic}
\subsection{Tibetan}
\subsection{Jinghpo}

\subsubsection{The sigmatic prefix and its allomorphs}
\subsubsection{The causative tonal alternation}
Jinghpo also has a tonal alternation which derives transitive verbs with 55 tone out of intransitive ones with tone 33, as shown by Table \ref{tab:jinghpo.tone} (data from  \citet[78]{dai92yufa}).

\begin{table}
\caption{Examples of transitivizing tonal alternation in Jinghpo} \label{tab:jinghpo.tone} \centering
\begin{tabular}{llll}
\toprule
\ipa{rōng} &exist &\ipa{róng} &enclose (cattle) \\
\ipa{yām} &serve as a slave &\ipa{yám} &enslave \\
\ipa{nōi} &be hung &\ipa{nói} &hang \\
\ipa{mànām} &smell, have a smell &\ipa{mànám} &smell (vt) \\
\bottomrule
\end{tabular}
\end{table}
 
 
 \begin{table}
\caption{Examples of -t applicative in Jinghpo} \label{tab:jinghpo.appl} \centering
\begin{tabular}{llll}
\toprule
\ipa{màlāng} &straight &\ipa{màlán} &straighten \\
\ipa{màdī} &be wet &\ipa{màdìt} &make wet \\
\ipa{shàmū} &move &\ipa{shàmòt} &cause to move \\
\ipa{lùng} &go up&\ipa{lún} &put up (at a high place) \\
\bottomrule
\end{tabular}
\end{table}
 

Trace of the \ipa{-t} applicative (\citealt{michailovsky85dental, jacques15derivational.khaling}), not of sigmatic causative
\subsection{Lolo-Burmese}
\subsection{Old Chinese}


\section{Passive and anticausative}

\subsection{Anticausative in Rgyalrongic}
%Anticausative prenasalization:
%
%\citet{jacques15spontaneous, jacques15causative}
%
%Passive:
%
% \citet{jacques07passif, jacques12demotion}
% 
% Not s-causative:
%
%\citet{mei12caus}
%\citet{sagart12sprefix}
%\citet{jacques15causative}
%
% 

% 
% 
% \begin{exe}
%\ex 
%\glt \zh{止子路宿,殺雞為黍而食之,見其二子焉}
%\glll \zh{止} \zh{子路} \zh{宿} \zh{殺} \zh{雞} \zh{為} \zh{黍} \zh{而} \zh{食} \zh{之} \zh{見} \zh{其} \zh{二} \zh{子} \zh{焉} \\
%\ipa{tɕiX} \ipa{tsiX.luH} \ipa{sjuwk} \ipa{ʂɛt} \ipa{kej} \ipa{hjwe} \ipa{ɕoX} \ipa{ɲi} \ipa{siH}  \ipa{tɕi}  \ipa{ɣenH}  \ipa{gi}  \ipa{ɲijH}  \ipa{tsiX}   \ipa{hjen} \\
%stop Zilu stay.overnight kill chicken make rice \textsc{caus}:eat \textsc{3sg} ?\textsc{caus}:see his two son to.him \\
%\glt (The old man) invited Zilu to stay overnight, killed a chicken and made dinner to feed him, and had his two sons come out (to greet Zilu).
% \end{exe}
% 
%\textsc{causative} $\Rightarrow$  \textsc{passive} attested in Manchu (\forme{bu-}{give} $\rightarrow$ \forme{-bu-}{causative}  $\rightarrow$ \forme{-bu-}{passive})
%
%but the reverse not attested?

\subsection{Passive in Rgyalrongic}
 
 
 \subsection{Jinghpo}
 \citet[78]{dai92yufa}
\begin{itemize}
\item \ipa{byàt} \zh{解决;定价} $\leftarrow$ \ipa{hpyàt} \zh{结(案);定(价)}
\item \ipa{byá'} \zh{垮} $\leftarrow$ \ipa{hpyá'} \zh{使垮} %拆房子 剖开 抢劫
\item \ipa{byō} \zh{融化} $\leftarrow$ \ipa{hpyō} \zh{冲(易融化食物)} 
\item \ipa{gà'} \zh{裂开;破;穿过} $\leftarrow$ \ipa{hkà'} \zh{分离;离婚} 
\end{itemize}

g - k : 
  \ipa{gyìt} \zh{拴;系}  \ipa{kyít}   \zh{勒裤袋}
  \ipa{makjít} \zh{结} 
    \ipa{gyìt dún}\zh{拴着} =>  \ipa{dún}\zh{拴}
    %hkà' númrî hpún kó' gyìt dún dá ǹhtóm ńtâ wà káu dá ù' ai
    %只好把网拴在柳树上先回家去
shàngài kàrūm āi jān gàw dāi làtá' htá' hkyēng āi sūmrī gyìt nhtáwm, 
%shàngài 生  gàrūm "help" jān 已婚女子  
%hkyē "red"
so the midwife took a scarlet thread and tied it on his wrist and said, "This one came out first."
NINGPAWT NINGHPANG 38.28

not transitive vs intransitive:
 \ipa{gàwàn} \zh{绕着} $\leftarrow$ \ipa{hkáwân} \zh{绕上} %划圈儿 “你在纸上划个人圈儿吧 == 绕成一团
 \ipa{gàyòm} \zh{卷} $\leftarrow$ \ipa{hkáyôm} \zh{卷} %卷草烟 == 笋壳卷起来了 
 \ipa{gūm} \zh{齐} $\leftarrow$ \ipa{hkum} \zh{齐} %人齐了吗?
  \ipa{gùmbà'} \zh{叠;折叠} $\leftarrow$ \ipa{hkùmbà'} \zh{折} 
\ipa{pyān} \zh{开放(花);舒展}  \ipa{hpyàn} \zh{解开;解释} %incorrect comparison

\subsection{Tibetan}
\subsection{Old Chinese}
\section{Nominalization}
\subsection{Jinghpo}

 \citet[4]{dai92yufa}
gūn « porter sur le dos 背;担任;携带 » => mà-gún « (un) fardeau » (classificateur; exemple suggéré sans explication p115 de la grammaire de Dai), cognat du tibétain ɴkʰur འཁུར་
yúʔ « descendre 下 » => mà-yúʔ « contrebas 下坡 »
rà « manque, avoir de besoin de, commettre une erreur 差欠;需要;差错 » => mà-rà « erreur, faute 错误;罪 »
ràʔ « aimer 爱 » => mà-ràʔ « souhait 意愿 »
lún « placer en haut (往高处放东西) » => ma-lún « chemin en pente vers haut 上坡路 » ; noter que lún dérive de lùng « monter » par l’application de la dérivation applicative (*-t), qui se manifeste ici par un changement tonal et -ng > -n
lái « échanger 换 » => mà-lái « remplacement 替身;代用品 », cognat du tibétain brdʑe བརྗེ་
kyít makyít

\subsection{Tibetan}
tibetan:
zong < 'tshong
zer < 'tsher
\subsection{Old Chinese}
\section{Denominal verbalization}
\subsection{Rgyalrongic}
\subsection{Old Chinese}
\section{Labial causative or volitional prefixes}
\subsection{Kuki-Chin}
\subsection{Rgyalrongic}
\subsection{Jinghpo}
\subsection{Old Chinese}
\section*{Conclusion}

\bibliographystyle{unified}
\bibliography{bibliogj}
\end{document}
