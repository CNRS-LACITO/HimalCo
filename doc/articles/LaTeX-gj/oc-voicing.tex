\documentclass[oneside,a4paper,11pt]{article} 
\usepackage{fontspec}
\usepackage{natbib}
\usepackage{booktabs}
\usepackage{xltxtra} 
\usepackage{polyglossia} 
\usepackage[table]{xcolor}
\usepackage{gb4e} 
\usepackage{multicol}
\usepackage{graphicx}
\usepackage{float}
\usepackage{hyperref} 
\hypersetup{bookmarksnumbered,bookmarksopenlevel=5,bookmarksdepth=5,colorlinks=true,linkcolor=blue,citecolor=blue}
\usepackage[all]{hypcap}
\usepackage{memhfixc}
\usepackage{lscape}
\usepackage{amssymb}
 
\bibpunct[: ]{(}{)}{,}{a}{}{,}

%\setmainfont[Mapping=tex-text,Numbers=OldStyle,Ligatures=Common]{Charis SIL} 
\newfontfamily\phon[Mapping=tex-text,Scale=MatchLowercase]{Charis SIL} 
\newcommand{\ipa}[1]{\textbf{{\phon\mbox{#1}}}} %API tjs en italique
%\newcommand{\ipab}[1]{{\scriptsize \phon#1}} 

\newcommand{\grise}[1]{\cellcolor{lightgray}\textbf{#1}}
\newfontfamily\cn[Mapping=tex-text,Ligatures=Common,Scale=MatchUppercase]{SimSun}%pour le chinois
\newcommand{\zh}[1]{{\cn #1}}
\newfontfamily\mleccha[Mapping=tex-text,Ligatures=Common,Scale=MatchLowercase]{Galatia SIL}%pour le grec
\newcommand{\grec}[1]{{\mleccha #1}}


\newcommand{\sg}{\textsc{sg}}
\newcommand{\pl}{\textsc{pl}}
\newcommand{\ro}{$\Sigma$}
\newcommand{\ra}{$\Sigma_1$} 
\newcommand{\rc}{$\Sigma_3$}   
\newcommand{\change}[2]{*\ipa{#1} $\rightarrow$ \ipa{#2}}
 

\XeTeXlinebreakskip = 0pt plus 1pt %
 %CIRCG
 
\newcommand{\forme}[2]{\ipa{#1} `#2'}  
\newcommand{\refb}[1]{(\ref{#1})}
\newcommand{\tld}{\textasciitilde{}}
\newcommand{\zhc}[2]{\zh{#1} \ipa{#2}} 
\newcommand{\oc}[3]{\zhc{#1}{#2} `#3'} 
\newcommand{\fanqie}[1]{(\zh{(#1切)})}
\newcommand{\mc}[1]{\zh{(#1)}}
\begin{document}

\title{The Sino-Tibetan origins of the Middle Chinese voicing alternation}
\author{Guillaume Jacques}
\maketitle

 \section*{Introduction}
 \section{Passive and anticausative}
 \citet[86]{zhou62goucibian}
\begin{table}[H]
\begin{tabular}{llllllllll}
\toprule
Unvoiced form &Meaning & Voiced form & Meaning\\
\midrule
\zhc{别}{pjet} \mc{幫}	&separate & \zhc{别}{bjet} \mc{並} & take leave \\
\zhc{敗}{pæjH} \mc{幫}	&defeat & \zhc{敗}{bæjH} \mc{並} & be defeated \\
\midrule
\zhc{斷}{twanH} \mc{端}	&cut & \zhc{斷}{dwanH} \mc{並} & be cut, stop \\
\midrule
\zhc{折}{tɕet} \mc{章}	&break (tr) & \zhc{折}{dʑet} \mc{常} &break (it), be broken, die young \\
\midrule
\zhc{見}{kenH} \mc{見}	&see (tr) & \zhc{現}{ɣenH} \mc{匣} & appear \\
\zhc{繫}{kejH} \mc{見}	&tie & \zhc{繫}{ɣejH} \mc{匣} & be tied \\
\zhc{解}{kɛɨX} \mc{見}	&untie & \zhc{解}{ɣɛɨX} \mc{匣} & be loosened \\
\bottomrule
\end{tabular}
\end{table}

\subsection{Hmong-Mien} 
\citet{downer73loanwords, sagart03prenasalized}

\subsection{Sino-Tibetan comparison} 
Anticausative prenasalization:

\citet{jacques15spontaneous, jacques15causative}

Passive:

 \citet{jacques07passif, jacques12demotion}
 
 Not s-causative:

\citet{mei12caus}
\citet{sagart12sprefix}
\citet{jacques15causative}

 \section{Causative}
 
 
 \citet[282-4]{wangyt14jingdian}
 
 
 \begin{exe}
\ex 
\glt \zh{止子路宿,殺雞為黍而食之,見其二子焉}
\glll \zh{止} \zh{子路} \zh{宿} \zh{殺} \zh{雞} \zh{為} \zh{黍} \zh{而} \zh{食} \zh{之} \zh{見} \zh{其} \zh{二} \zh{子} \zh{焉} \\
\ipa{tɕiX} \ipa{tsiX.luH} \ipa{sjuwk} \ipa{ʂɛt} \ipa{kej} \ipa{hjwe} \ipa{ɕoX} \ipa{ɲi} \ipa{siH}  \ipa{tɕi}  \ipa{ɣenH}  \ipa{gi}  \ipa{ɲijH}  \ipa{tsiX}   \ipa{hjen} \\
stop Zilu stay.overnight kill chicken make rice \textsc{caus}:eat \textsc{3sg} ?\textsc{caus}:see his two son to.him \\
\glt (The old man) invited Zilu to stay overnight, killed a chicken and made dinner to feed him, and had his two sons come out (to greet Zilu).
 \end{exe}
 
\textsc{causative} $\Rightarrow$  \textsc{passive} attested in Manchu (\forme{bu-}{give} $\rightarrow$ \forme{-bu-}{causative}  $\rightarrow$ \forme{-bu-}{passive})

but the reverse not attested?
 
  \section{Conclusion}

\bibliographystyle{unified}
\bibliography{bibliogj}
\end{document}
