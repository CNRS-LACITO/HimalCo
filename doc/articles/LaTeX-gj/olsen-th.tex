\documentclass[oldfontcommands,oneside,a4paper,11pt]{article} 
\usepackage{fontspec}
\usepackage{natbib}
\usepackage{booktabs}
\usepackage{xltxtra} 
\usepackage{polyglossia} 
\usepackage[table]{xcolor}
\usepackage{gb4e} 
\usepackage{multicol}
\usepackage{graphicx}
\usepackage{float}
\usepackage{hyperref} 
\usepackage{lineno}
\hypersetup{bookmarks=false,bookmarksnumbered,bookmarksopenlevel=5,bookmarksdepth=5,xetex,colorlinks=true,linkcolor=blue,citecolor=blue}
\usepackage[all]{hypcap}
\usepackage{memhfixc}
\usepackage{lscape}

%\bibpunct[: ]{(}{)}{,}{a}{}{,}

%\setmainfont[Mapping=tex-text,Numbers=OldStyle,Ligatures=Common]{Charis SIL} 
\newfontfamily\phon[Mapping=tex-text,Ligatures=Common,Scale=MatchLowercase]{Charis SIL} 
\newcommand{\ipa}[1]{{\phon\textit{#1}}} %API tjs en italique
\newcommand{\ipab}[1]{{\scriptsize \phon#1}} 

\newcommand{\grise}[1]{\cellcolor{lightgray}\textbf{#1}}
\newfontfamily\cn[Mapping=tex-text,Ligatures=Common,Scale=MatchUppercase]{SimSun}%pour le chinois
\newcommand{\zh}[1]{{\cn #1}}
\newcommand{\refb}[1]{(\ref{#1})}
\newcommand{\factual}[1]{\textsc{:fact}}
\newcommand{\rdp}{\textasciitilde{}}

\XeTeXlinebreaklocale 'zh' %使用中文换行
\XeTeXlinebreakskip = 0pt plus 1pt %
 %CIRCG 
\newcommand{\ro}{$\Sigma$}

\begin{document} 
\title{La loi d'Olsen et les désinences de présent en \ipa{-th-} en Indo-Iranien}
\author{Guillaume Jacques}
 \maketitle
 
 \section*{Introduction}
\citet{olsen94laryngeal} a proposé l'existence d'une loi de date indo-européenne d'aspiration des occlusives sourdes par une laryngale *\ipa{h_1} ou *\ipa{h_2}, suivant une voyelle courte ou une consonne syllabique:

\begin{exe}
\ex \label{ex:olsen}
\glt *\ipa{-V̆h_{1/2}T-} > \ipa{-V̆tʰ-}
\end{exe}

Un des meilleurs exemples de cette loi est le Skt. \ipa{vṛthā} `à volonté' (puis dans des textes plus tardifs `en vain'), d'un *\ipa{wḷh_1-teh_2}, tiré de la racine *\ipa{welh_1} (\citealt[677-8]{liv}). 

D'autres exemples, tels que Skt. \ipa{gūtha-} m., Av. \ipa{gūθəm} `bouse' < *\ipa{gʷuh_2-to-}, auraient selon \citet{olsen94laryngeal}  une aspiration due à la loi \refb{ex:olsen}, mais auraient rétabli la voyelle longue attendue par la laryngale par analogie.

Les contre-exemples à cette loi sont naturellement légion; ils comprennent notamment tous les participes en *\ipa{-to-} de verbes dont la racine contient une laryngale finale (tels que Skt. \ipa{jātá-} `né'  < *\ipa{ǵṇh_1-tó}, qui  aurait remplacé un $\dagger$\ipa{jathá-}), mais également une myriade d'exemples dans les systèmes nominaux et verbaux, qu'il faudrait rendre compte de façon analogique.

La loi proposée par \citet{olsen94laryngeal} est jusqu'ici restée controversée (elle n'est pas mentionnée par les monographies récentes sur les laryngales telles que \citealt{mayrhofer05fortsetzung} ou \citealt{zair12celtic} par exemple). Ce travail a pour but de suggérer un ensemble d'exemples additionnels possibles de cette loi, et de montrer que son application est peut-être plus étendue qu'elle n'y paraît au premier abord.

 \section{Les désinences en *\ipa{-th-} dans le paradigme du présent actif en Indo-Iranien}

L'indo-iranien se démarque de toutes les autres langues indo-européennes par la présence d'une série de sourdes aspirées (devenues fricatives en iranien), dont il est admis qu'elles sont d'origine secondaire, provenant en particulier des groupes occlusives sourde + laryngales de l'indo-européen (voir par exemple \citealt[112-4]{mayrhofer05fortsetzung}).

Ces aspirées apparaissent dans de nombreuses désinences verbales. Dans le cas de la \textsc{2sg} du parfait, la reconstruction *\ipa{-th_2e} proposée par \citet{kurylowicz1927schwa} (en notation modernisée) est universellement admise. En revanche, certaines terminaisons du présent actif et médio-passif, présentent des aspirées inattendues au regard des autres langues indo-européennes.

Les tableaux \ref{tab:actif} et \ref{tab:medio} indiquent en grisé les formes qui nous intéressent: la \textsc{2pl} de l'actif \ipa{-tha}, et les \textsc{2du} et \textsc{3du} de l'actif (\ipa{-thas}) et du médio-passif  (\ipa{-(ā|e)the}), qui ont des aspirées à des formes où, contrairement au cas de la \textsc{2sg} du parfait, aucune trace de laryngales ne se laisse déceler ailleurs en indo-européen de façon certaine (\citealt[309-311]{burrow55skt})
Pour la seconde pluriel \ipa{-tha}, aucune trace, métrique ou segmentale, de laryngale n'a été rapportée dans les langues autres que l'indo-iranien. 

Pour celles du duel, la situation est plus compliquée en raison de la mauvaise préservation de ces formes dans les langues de la famille. Les formes du grec, quoiqu'indéniablement apparentées à celle de l'indo-iranien, sont peu informatives sur la présence ou non d'une laryngale. Certains auteurs ont proposé que le germanique aurait *\ipa{t} correspondant au sanskrit \ipa{th} au lieu de *\ipa{θ} comme on l'attendrait d'après la loi de Grimm (dossier cité dans \citealt[113]{mayrhofer05fortsetzung}), sur la base en particulier de la désinence de \textsc{2sg} du parfait \ipa{-t} (sanskrit \ipa{-tha} $\rightarrow$ *\ipa{th_2e}) et de celle de \textsc{2du} du présent  \ipa{-ts} en gothique (sanskrit \ipa{-thas} $\rightarrow$ *\ipa{th_2es}?). Cette solution n'est toutefois pas retenue de nos jours, et les formes germaniques peuvent s'expliquer de façon plus satisfaisante par extension analogique de l'allomorphe qui suivait les radicaux verbaux en dentale (\citealt[84]{hill03zusammenstoss}, \citealt[192]{ringe06PIE}). La présence ou non de laryngales ou d'une aspiration ancienne dans ces formes est donc incertaine.

Seuls le sanskrit et l'avestique seront mentionnés dans ce travail, car les formes en question ne sont pas attestées en vieux perse, et mal conservées ailleurs en indo-iranien.

On remarque immédiatement une différence importante entre sanskrit et avestique: alors que l'opposition d'aspiration distingue la deuxième de la troisième personne du duel en sanskrit,  on trouve en avestique des formes provenant d'aspirées (reflétées par la fricative \ipa{θ} $\leftarrow$ *\ipa{th}) et de sourde simples pour les troisièmes personnes du duel. \citet{martinez14avestan}, considérant la distribution du sanskrit comme originelle, interprètent les formes en \ipa{θ} de troisième duel en avestique comme une extension analogique de la deuxième duel. Toutefois, l'analogie se propage normalement de la troisième personne aux autres formes, sauf cas exceptionnels de verbes dont la première personne singulier est plus courante que la troisième personne (\citealt{jacques16ebde}), et l'hypothèse de Martínez et de Vaan est donc peu probable.

\begin{table}[H]
\caption{Paradigmes du présent actif en indo-iranien} \label{tab:actif}
\begin{tabular}{lllllll}
\toprule
 & 	Sanskrit  & 	Sanskrit & 	Avestique & 	Grec & 	\\
 &thématique&athématique&&\\
\midrule
\textsc{1sg} & 	\ipa{-āmi} & 	\ipa{-mi} & 	\ipa{-ā} & 	\ipa{-mi} & 	\\
\textsc{2sg} & 	\ipa{-asi} & 	\ipa{-si} & 	\ipa{-hī, šī} & 	\ipa{-si} & 	\\
\textsc{3sg} & 	\ipa{-ati} & 	\ipa{-ti} & 	\ipa{-tī} & 	\ipa{-ti} & 	\\
\textsc{1du} & 	\ipa{-āvas} & 	\ipa{-vas} & 	\ipa{-uuahī} & 	x & 	\\
\textsc{2du} & 	\ipa{-athas} \grise{}& 	\ipa{-thas} \grise{}& 	\ipa{x} \grise{}& 	\ipa{-ton} & 	\\
\textsc{3du} & 	\ipa{-atas} \grise{}& 	\ipa{-tas} \grise{}& 	\ipa{-tō, -θō} \grise{}& 	\ipa{-ton} & 	\\
\textsc{1pl} & 	\ipa{-āmas} & 	\ipa{-mas} & 	\ipa{-mahī} & 	\ipa{-me(n|s)} & 	\\
\textsc{2pl} & 	\ipa{-atha}\grise{} & 	\ipa{-tha} \grise{}& 	\ipa{-θā} \grise{}& 	\ipa{-tes} & 	\\
\textsc{3pl} & 	\ipa{-anti} & 	\ipa{-anti} & 	\ipa{-ṇtī, -aiṇti} & 	\ipa{-ousi} & 	\\
\bottomrule
\end{tabular}
\end{table}

\begin{table}[H]
\caption{Paradigmes du présent médio-passif  en indo-iranien}  \label{tab:medio}
\begin{tabular}{lllllll}
\toprule
 & 	Sanskrit  & 	Sanskrit & 	Avestique & 	Grec & 	\\
 &thématique&athématique&&\\
 \midrule
\textsc{1sg} & 	\ipa{-e} & 	\ipa{-̣e} & 	\ipa{-ē, -ōi} & 	\ipa{-mai} & 	\\
\textsc{2sg} & 	\ipa{-ase} & 	\ipa{-se} & 	\ipa{-hē, -ŋ́hē, -šē} & 	\ipa{-sai} & 	\\
\textsc{3sg} & 	\ipa{-ate} & 	\ipa{-te} & 	\ipa{-tē, -ē} & 	\ipa{-tai} & 	\\
\textsc{1du} & 	\ipa{-āvahe} & 	\ipa{-vahe} & 	\ipa{x} & 	x& 	\\
\textsc{2du} & 	\ipa{-ethe} \grise{}& 	\ipa{-āthe} \grise{}& 	\ipa{x} \grise{}& 		\ipa{-sthon} & 	\\
\textsc{3du} & 	\ipa{-ete}\grise{} & 	\ipa{-āte}\grise{} & 	\ipa{-aētē, -ōiθe, -āitē} \grise{}& 	\ipa{-sthon} & 	\\
\textsc{1pl} & 	\ipa{-āmahe} & 	\ipa{-mahe} & 	\ipa{-maidē} & 	\ipa{-metha}& 	\\
\textsc{2pl} & 	\ipa{-adhve} & 	\ipa{-dhve} & 	\ipa{-duiiē} & 	\ipa{-sthe} & 	\\
\textsc{3pl} & 	\ipa{-ante} & 	\ipa{-ate} & 	\ipa{-n̄tē, -aitē} & 	\ipa{-ntai} & 	\\
\bottomrule
\end{tabular}
\end{table}
 
La solution proposée ici pour rendre compte de l'aspiration dans les formes indo-iraniennes citées ci-dessus part de deux constatations:

\begin{itemize}
\item L'aspiration dans la désinence de \textsc{2pl} actif n'a pas d'équivalent ailleurs en indo-européen, et il est vraisemblable qu'il s'agisse d'une innovation indo-iranienne.
\item La distribution au duel entre \ipa{th} à la deuxième personne et \ipa{t} à la troisième est propre au sanskrit; l'avestique suggère plutôt que la troisième personne du duel avait des formes aspirées et non-aspirées selon une distribution encore mal comprise, et que la restriction à des personnes spécifiques observée en sanskrit est secondaire.
\end{itemize}

Ces observations suggèrent que les formes aspirées (\ipa{-tha}, \ipa{-thas}, \ipa{-(e)the}) pourraient être à l'origine des variantes contextuelles de formes héritées sans aspiration, généralisées analogiquement à l'ensemble du lexique (imparfaitement en avestique).

\section{Application de la loi d'Olsen}
Si la loi d'Olsen est valide, on attendrait pour les verbes athématique de classe II à laryngale finale et de classe IX les paradigmes suivants (ici présentés avec les racines *\ipa{mlewh_2} `parler' (\citealt[446]{liv} et *\ipa{ĝewH} `se dépêcher' (\citealt[166]{liv}).


\begin{table}[H]
\caption{Présent de classe II} \label{tab:II}
\begin{tabular}{lllllll}
\toprule
 & 	IE  & 	Expected proto-IIr & 	Expected Sanskrit & 	Sanskrit   	\\
\midrule
\textsc{1sg} & 	\ipa{*mléwh_2-mi} & 	\ipa{*mráwīmi} & 	\ipa{brávīmi} & 	  & 	\\
\textsc{2sg} & 	\ipa{*mléwh_2-si} & 	\ipa{*mráwīsi} & 	\ipa{brávīsi} & 	 & 	\\
\textsc{3sg} & 	\ipa{*mléwh_2-ti} & 	\ipa{*mráwīti} & 	\ipa{brávīti} &   & 	\\
\textsc{1du} & 	\ipa{*mluh_2-wé(n|s)} & 	\ipa{*mrūwás} & 	\ipa{brūvás} & 	  & 	\\
\textsc{2du} & 	\ipa{*mluh_2-tó(n|s)} & 	\ipa{*mruthás} \grise{}& 	$\dagger$\ipa{bruthás} \grise{}& 	\ipa{brūthás} & 	\\
\textsc{3du} & 	\ipa{*mluh_2-tó(n|s)} & 	\ipa{*mruthás}  \grise{} & $\dagger$\ipa{bruthás} \grise{} & 	\ipa{brūtás} & 	\\
\textsc{1pl} & 	\ipa{*mluh_2-mé(n|s)} & 	\ipa{*mrūmás} & 	\ipa{brūmás} & 	  & 	\\
\textsc{2pl} & 	\ipa{*mluh_2-té} & 	\ipa{*mruthá}  \grise{}& 	$\dagger$\ipa{bruthá}  \grise{}& 	\ipa{brūthá} & 	\\
\textsc{3pl} & 	\ipa{*mluh_2-ónti} & 	\ipa{*mruwánti} & 	\ipa{bruvánti} &  & 	\\
\bottomrule
\end{tabular}
\end{table}

\begin{table}[H]
\caption{Présent de classe IX} \label{tab:IX}
\begin{tabular}{lllllll}
\toprule
 & 	IE  & 	Expected proto-IIr & 	Expected Sanskrit & 	Sanskrit   	\\
\midrule
\textsc{1sg} & 	\ipa{*ĝew<né>H-mi} & 	\ipa{*j́unā́mi} & 	\ipa{junā́mi} & 	  & 	\\
\textsc{2sg} & 	\ipa{*ĝew<né>H-si} & 	\ipa{*j́unā́si} & 	\ipa{junā́si} & 	 & 	\\
\textsc{3sg} & 	\ipa{*ĝew<né>H-ti} & 	?\ipa{*j́unáthi} \grise{}& 	$\dagger$\ipa{junáthi} \grise{}& \ipa{junā́ti}  & 	\\
\textsc{1du} & 	\ipa{*ĝwṇH-wé(n|s)}   & 	\ipa{*j́wāwás} \grise{} &$\dagger$\ipa{jvāvás}\grise{}& 	 \ipa{junīvás}    	   & 	\\
\textsc{2du} & 	\ipa{*ĝwṇH-tó(n|s)} & 	\ipa{*j́wathás} \grise{}& 	$\dagger$\ipa{jvathás} \grise{}& 	\ipa{junīthás} & 	\\
\textsc{3du} & 	\ipa{*ĝwṇH-tó(n|s)} & 	\ipa{*j́wathás}  \grise{} & $\dagger$\ipa{jvathás} \grise{} & 	\ipa{junītás} & 	\\
\textsc{1pl} & 	\ipa{*ĝwṇH-mé(n|s)} & 	\ipa{*j́wāmás}  \grise{}& 	$\dagger$\ipa{jvāmás} \grise{} & 	   	\ipa{junīmás} \\
\textsc{2pl} & 	\ipa{*ĝwṇH-té} & 	\ipa{*j́wathá}  \grise{}& 	$\dagger$\ipa{jvāthás}  \grise{}& 	\ipa{junīthá} &  	\\
\textsc{3pl} & 	\ipa{*ĝunH-ónti} & 	\ipa{*j́unánti} & 	\ipa{junánti} &  & 	\\
\bottomrule
\end{tabular}
\end{table}


Dans le tableau \ref{tab:II}, la voyelle longue aux formes \textsc{2/3du} et \textsc{2pl} à la place d'une voyelle courte comme on l'attendrait si le traitement phonétique avait été régulier peut s'expliquer par analogie avec les formes de première personne duelle et plurielle. Les désinences, mis à part la \textsc{3du}, seraient toutes phonétiquement régulière.

Dans le tableau \ref{tab:IX}, les formes du radical sont refaites (car des formes exclusivement phonétiques se auraient créés des alternances trop complexes pour être viables), mais la loi d'Olsen s'appliquer de la même façon aux suffixes en occlusive dentale que dans le paradigme de la classe II, à l'exception de la troisième personne du singulier qui devrait aussi avoir subi ce changement.

La distribution entre les variantes aspirée et non-aspirée des deuxième et troisième personnes du duel en Sanskrit  n'est sans doute pas même indo-iranienne, puisque l'avestique préserve les deux pour la troisième personne (voir le tableau \ref{tab:actif}). Les deux allomorphes \ipa{-thas} et \ipa{-tas}, originellement variantes d'un même suffixe (l'aspirée avec certains verbes de classe II, III et IX, la non-aspirée avec les verbes thématiques et tous les autres athématiques) ne distinguant pas la seconde de la troisième personne, ont été spécialisés en Sanskrit comme une opposition de personne. Ces affixes auraient ensuite été généralisé à toutes les classes de présent, y compris thématiques.

\section*{Conclusion}
Si la loi d'Olsen reste une hypothèse controversée, le présent travail propose que l'Indo-iranien aurait préservé dans les paradigmes du présent actif des désinences issues de formes ayant subi ce changement phonétique, sont les traces auraient été anéanties dans les autres branches de la famille. Le maintien remarquable des paradigmes athématiques en Indo-iranien explique sans doute pourquoi cette branche de la famille précisément aurait préservé un archaïsme qui ne pourrait laisser de trace dans la conjugaison thématique. 

\bibliographystyle{unified}
\bibliography{bibliogj}

\end{document}