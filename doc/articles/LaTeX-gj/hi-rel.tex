\documentclass[oneside,a4paper,11pt]{article} 
\usepackage{fontspec}
\usepackage{natbib}
\usepackage{booktabs}
\usepackage{xltxtra} 
\usepackage{polyglossia} 
\usepackage[table]{xcolor}
\usepackage{multicol}
\usepackage{graphicx}
\usepackage{float}
\usepackage{hyperref} 
\hypersetup{bookmarksnumbered,bookmarksopenlevel=5,bookmarksdepth=5,colorlinks=true,linkcolor=blue,citecolor=blue}
\usepackage[all]{hypcap}
\usepackage{memhfixc}
\usepackage{lscape}
\usepackage{tikz}
\usetikzlibrary{trees}
\usepackage{gb4e} 
 
%\setmainfont[Mapping=tex-text,Numbers=OldStyle,Ligatures=Common]{Charis SIL}  
\newfontfamily\phon[Mapping=tex-text,Ligatures=Common,Scale=MatchLowercase]{Charis SIL} 
\newcommand{\ipa}[1]{\mbox{\phon\textit{#1}}} %API tjs en italique
 
\newcommand{\grise}[1]{\cellcolor{lightgray}\textbf{#1}}
\newfontfamily\cn[Mapping=tex-text,Ligatures=Common,Scale=MatchUppercase]{SimSun}%pour le chinois
\newcommand{\zh}[1]{{\cn #1}}
\newcommand{\tld}{\textasciitilde{}}


\begin{document} 
\title{Head-Internal Relative clauses\\in Sino-Tibetan and beyond}
\maketitle

\section*{Introduction}
One of the crosslinguistically uncommon features that are widespread in Sino-Tibetan is the presence of Head-Internal relatives, ie. relative clauses whose head noun is neither preposed or postposed but occupies the same position and takes the same case marking as in the corresponding indepdendent clauses, as in example (\ref{ex:nWwGmbi}) from Japhug, where the head noun \ipa{ʑmbrɯ} `the boat' occurs between the subject and the verb of the clause.

\begin{exe}
\ex \label{ex:nWwGmbi}
\gll
[\ipa{tɤ-wɯ} 	\ipa{kɯ} 	\textbf{\ipa{ʑmbrɯ}} 	\ipa{nɯ́-wɣ-mbi}] 	\ipa{nɯ} 	 	\ipa{cʰɤ-lɤt} \\
\textsc{indef.poss}-grandfather \textsc{erg} \textbf{boat} \textsc{pfv-inv}-give \textsc{dem} \textsc{ifr}-throw \\
\glt He took the boat that the old man had given him. (\citealt{jacques16relatives})
\end{exe}

While head-internal relative clauses are uncommon in languages of the world (\citealt{dryer13relative}), they are found in various branches of the Sino-Tibetan family, including Tibetan (\citealt{mazaudon78relatives}), Rgyalrongic (\citealt{jackson06guanxiju, jacques16relatives}) Dulong-Rawang (\citealt{lapolla08relative}) and Ao (\citealt{coupe07mongsen}).

All of these languages allow the three possible orders between head noun and relative clause (prenominal, postnominal, head-internal); all head-internal relatives described up to now in ST belong to Lehman's (\citeyear[109]{lehmann84relativsatz}) `circumnominal' type, and no example of HIRC with preposed head (`Mit vorangestellten Nukleus') has been described up to now. While HIRC appear to be common in some ST languages such as Japhug, they are more marginal in most of the language mentioned above. 

The present project aims at an in-depth study of HIRC in the ST family, with typological comparisons with other families where these RC are attested. The project will comprise three main 

\section{Survey of HIRC in ST}
The team will make a comprehensive survey of the published literature on HIRC in the ST family, and produce a detailed overview of their geographical distribution and morphosyntactic properties in the family.

\section{Corpus study of HIRC}
In order to avoid the detrimental effects of direct elicitation on the reliability of the data, the team will focus on text corpora. Table (\ref{tab:french}) presents the languages on which the French team has primary data on, and XXXX Singapore team.

\begin{table}[h]
\caption{Languages studied by the French team} \centering \label{tab:french}
\begin{tabular}{llllll}
\toprule
Family &Language & Researcher\\
\midrule
ST (Rgyalrongic) &Japhug & G. Jacques \\
& Zbu & Gong Xun \\
& Situ & Zhang Shuya \\
& Khroskyabs & Lai Yunfan \\
&Stau & GJ \& Lai Yunfan \\
ST (Kiranti) &Khaling & GJ \& Aimée Lahaussois \\
Siouan & Omaha & Julie Marsault \\
\bottomrule
\end{tabular}
\end{table}

Each team member will focus on a manageable portion of his/her corpus (between 1 and 5 hours), and collect a spreadsheet database of all RC, specifying for each of them the following features:

\begin{itemize}
\item Order of RC and head noun (Prenominal, Postnominal, HI, headless)
\item Syntactic role of the head noun.
\item Subtype of RC (language-specific): for instance, in Japhug, participial vs finite RC.
\end{itemize}

Note that in some languages, attributive `adjectives' have to be analyzed as short RC (as in example \ref{ex:kWmpCWmpCAr} from Japhug), and should be systematically included in the sample.

\begin{exe}
\ex \label{ex:kWmpCWmpCAr}
\gll 
\ipa{tɕʰeme} \ipa{kɯ-mpɕɯ\tld{}mpɕɤr} \\
girl \textsc{nmlz:S/A-emph}\tld{}be.beautiful \\
\glt `A beautiful girl' = `A girl who is beautiful'
\end{exe}

The main difficulty will concern distinguishing between HI and postnominal RC. Since all languages in the sample (including the non-ST language Omaha) are verb-final languages, in both HIRC and postnominal RC, the verb occurs after the head noun. In the case of short RC (as example \ref{ex:kWmpCWmpCAr}), it may not make sense to attempt at distinguishing between the two. Each team member will have to specify clearly the criteria he used to distinguish between HIRC and postnominal RC, as some of them may be language-particular.

On the basis of the database, each team member will write a chapter on the language(s) he is in charge of for a collective volume, presenting the facts following a common framework.

\section{Typology of HIRC}
Questions:

\begin{itemize}
\item Why are HIRC rare in the world's languages? (one clue: difficulty to parse / garden-path effect of HIRC)
\item Are HIRC in ST an areal phenomenon? Are all languages where these RC are attested in a contiguous zone or do we find several unconnected areas?
\item Do HIRC in ST languages present morphosynactic properties that distinguish them from HIRC in Siouan, XXXX, XXX ?
\end{itemize}

\bibliographystyle{unified}
\bibliography{bibliogj}
\end{document}
