
\documentclass{article} 
\usepackage{fontspec}
\usepackage{natbib}
\usepackage{booktabs}
\usepackage{xltxtra} 
\usepackage{polyglossia} 
 \usepackage{geometry}
 \geometry{
 a4paper,
 total={210mm,297mm},
 left=25mm,
 right=25mm,
 top=20mm,
 bottom=20mm,
 }
\usepackage[table]{xcolor}
\usepackage{gb4e} 
\usepackage{multicol}
\usepackage{graphicx}
\usepackage{float}
\usepackage{hyperref} 
\hypersetup{bookmarks=false,bookmarksnumbered,bookmarksopenlevel=5,bookmarksdepth=5,xetex,colorlinks=true,linkcolor=blue,citecolor=blue}
\usepackage[all]{hypcap}
\usepackage{memhfixc}
\usepackage{lscape}
\usepackage{amssymb}
 
%\setmainfont[Mapping=tex-text,Numbers=OldStyle,Ligatures=Common]{Charis SIL} 
\newfontfamily\phon[Mapping=tex-text,Ligatures=Common,Scale=MatchLowercase]{Charis SIL} 
\newcommand{\ipa}[1]{{\phon\textit{#1}}} 
\newcommand{\grise}[1]{\cellcolor{lightgray}\textbf{#1}}
\newfontfamily\cn[Mapping=tex-text,Ligatures=Common,Scale=MatchUppercase]{SimSun}%pour le chinois
\newcommand{\zh}[1]{{\cn #1}}
\newcommand{\Y}{\Checkmark} 
\newcommand{\N}{} 
\newcommand{\jpg}[2]{\ipa{#1} `#2'}  
\newcommand{\refb}[1]{(\ref{#1})}
\newcommand{\tld}{\textasciitilde{}}
\newcommand{\zhc}[2]{\zh{#1} \ipa{#2}} 

 \begin{document} 
\title{Pseudo-cleft constructions in Japhug}
\author{Guillaume Jacques\\ CNRS-CRLAO-INALCO}
\maketitle

Like many strictly verb-final languages, Japhug does not allow true cleft sentences. What is closest to a cleft sentence in Japhug is the \textbf{pseudo-cleft construction}, combining a headless participial relative clause in S function (generally with the determiner \ipa{nɯ}) with a nominal predicate followed by a positive (\ipa{ŋu} `be') or negative (\ipa{maʁ} `not be') copula. The basic structure of this construction is summarized in (\ref{ex:pseudo.cleft}).

\begin{exe}
\ex \label{ex:pseudo.cleft}
\glt [\textsc{nmlz}-verb] \textsc{dem} noun \textsc{copula}
\end{exe}

Pseudo-clefts in Japhug do occur in S (\ref{ex:jAkWGe.nW}), O (\ref{ex:nWkArga})  and A (\ref{ex:WkWsWmphWl}) functions, but are on the whole very rare in the corpus.\footnote{The Japhug text corpus is available on the Pangloss archive (\citealt{michailovsky14pangloss}), at the address: \url{http://lacito.vjf.cnrs.fr/pangloss/corpus/list\textunderscore rsc.php?lg=Japhug} }

\begin{exe}
\ex \label{ex:jAkWGe.nW}
\gll [\ipa{stu}	\ipa{kɯ-mɤku}	\ipa{jɤ-kɯ-ɣe}]	\ipa{nɯ}	\ipa{rɟɤlpu}	\ipa{pjɤ-ŋu.}	  \\
most \textsc{nmlz}:S/A-be.before \textsc{pfv}-\textsc{nmlz}:S/A-come[II] \textsc{dem} king \textsc{ifr}.\textsc{ipfv}-be \\
\glt `The one who came first was the king.' (140514 xizajiang he lifashi, 66)
\end{exe}

\begin{exe}
   \ex   \label{ex:nWkArga}  
\gll [\ipa{pɣa}  	\ipa{ra}  	\ipa{nɯ-kɤ-rga}]  	\ipa{nɯ}  	\ipa{qaj}  	\ipa{ntsɯ}  	\ipa{ŋu}  \\
bird \textsc{pl} \textsc{3pl-nmlz:P}-like \textsc{dem} wheat always be:\textsc{fact} \\
\glt `(The food) that birds like is always wheat (not barley).' (23 pGAYaR, 29)
\end{exe} 


\begin{exe}
\ex   \label{ex:WkWsWmphWl}  
\gll \ipa{tɕeri}	[\ipa{nɯnɯ}	\ipa{ɯ-kɯ-sɯ-mpʰɯl}]	\ipa{nɯ}	\ipa{li}	\ipa{ɯ-zrɤm}	\ipa{ɲɯ-ɕti}	\ipa{ma}	\ipa{ɯ-rɣi}	\ipa{ɲɯ-maʁ}  \\
but \textsc{dem} \textsc{3sg}.\textsc{poss}-\textsc{nmlz}:S/A-\textsc{caus}-reproduce \textsc{dem} again  \textsc{3sg}.\textsc{poss}-root \textsc{sens}-be.\textsc{affirm} \textsc{lnk} \textsc{3sg}.\textsc{poss}-seeds \textsc{sens}-not.be \\
\glt `The thing that it reproduces with is its root, not its seeds.' (11-paRzwamWntoR, 113)
\end{exe} 

The present paper comprises three parts. First, it presents a short introduction to relativisation in Japhug (based in part on \citealt{jacques16relatives}). Second, it attempts to collect and classify all instances of pseudo-cleft constructions attested in the corpus. Third, it discusses the discourse function of pseudo-clefts and the alternative means of expressing contrastive and corrective focalisation in Japhug.

   %iɕqha <yazi> tɤ-mu kɯ ku-ɕɯm nɯnɯ, <tiane> ɣɯ ɯ-ŋgɯm pjɤ-ɕti rca wo matɕi.

%kɯ-mɤku lu-kɯ-ɣi nɯ zdɯm nɯ kɯ-wɣrum ŋu,
%tɕe nɯnɯ aʑo ŋu-a
%ɯ-qhu nɯ tɕu zdɯm kɯ-ɲaʁ ci lu-ɣi ŋu tɕe,
%nɯnɯ tɕe tɕethi tɤmujku nɯ lu-ɣi ŋu tɕe

%\begin{exe}
%\ex \label{ex:mAkWnaXtCWG}
%\gll \ipa{tɕeri}	[\ipa{mɤ-kɯ-naχtɕɯɣ}]	\ipa{tɕe,}	\ipa{ɯ-ndzrɯ}	\ipa{ɣɤʑu} \\
%but \textsc{neg}-\textsc{nmlz}:S/A-be.similar \textsc{lnk} \textsc{3sg}.\textsc{poss}-claw exist:\textsc{sens} \\
%\glt `What is different is that it has claws.' (21-pri, 37)
%\end{exe}


%\begin{exe}
%\ex \label{ex:kAnWmga.nW}
%\gll [\ipa{tɕe}	\ipa{paʁ}	\ipa{ndɤre}	\ipa{stu}	\ipa{kɤ-nɯmga}]	\ipa{nɯ}	\ipa{ɯ-ɕa}	\ipa{ŋu}	\ipa{qʰe} \\
%\textsc{lnk} pig \textsc{lnk} most \textsc{nmlz}:S/A-want.for \textsc{dem} \textsc{3sg}.\textsc{poss}-meat be:\textsc{fact} \textsc{lnk} \\
%\glt `Pigs are (raised) for their meat.' (05-paR, 100)
%\end{exe}


%\begin{exe}
%\ex \label{ex:aZo.pWnWmtota}
%\gll
%\ipa{kɯki}	\ipa{tɕʰeme}	\ipa{ki}	\ipa{ndɤre}	\ipa{aʑɯɣ}	\ipa{a-pɯ-ŋu}	\ipa{tʂaŋ}	\ipa{ma}	\ipa{tɕe}	\ipa{aʑo}	\ipa{pɯ-nɯmto-t-a}	\ipa{ɕti}	\ipa{tɕe}	\\
%\textsc{dem}.\textsc{prox} girl \textsc{dem}.\textsc{prox} \textsc{lnk}  \textsc{1sg}:\textsc{gen} \textsc{irr}-\textsc{ipfv}-be be.fair:\textsc{fact} \textsc{lnk} \textsc{1sg} \textsc{pfv}-find-\textsc{tr}:\textsc{pst}-\textsc{1sg} be.\textsc{affirm}:\textsc{fact} \textsc{lnk} \\
%\glt `It is I who found the girl, it is fair that she should be mine.' (140517 buaishuohua, 101)
%\end{exe}

\bibliographystyle{unified}
\bibliography{bibliogj}
 \end{document}
 