\documentclass[oneside,a4paper,11pt]{article} 
\usepackage{fontspec}
\usepackage{natbib}
\usepackage{booktabs}
\usepackage{xltxtra} 
\usepackage{polyglossia} 
 \usepackage{geometry}
% \geometry{
% a4paper,
% total={210mm,297mm},
% left=25mm,
% right=25mm,
% top=20mm,
% bottom=20mm,
% }
\usepackage[table]{xcolor}
\usepackage{gb4e} 
\usepackage{multicol}
\usepackage{graphicx}
\usepackage{float}
\usepackage{hyperref} 
\hypersetup{bookmarks=false,bookmarksnumbered,bookmarksopenlevel=5,bookmarksdepth=5,xetex,colorlinks=true,linkcolor=blue,citecolor=blue}
\usepackage[all]{hypcap}
\usepackage{memhfixc}
\usepackage{lscape}
\usepackage{amssymb}
\usepackage{lineno}
 
\setmainfont[Mapping=tex-text,Numbers=OldStyle,Ligatures=Common]{Charis SIL} 
\newfontfamily\phon[Mapping=tex-text,Ligatures=Common,Scale=MatchLowercase]{Charis SIL} 
\newcommand{\ipa}[1]{{\phon{#1}}} 
\newcommand{\grise}[1]{\cellcolor{lightgray}\textbf{#1}}
\newfontfamily\cn[Mapping=tex-text,Ligatures=Common,Scale=MatchUppercase]{SimSun}%pour le chinois
\newcommand{\zh}[1]{{\cn #1}}
\newcommand{\Y}{\Checkmark} 
\newcommand{\N}{} 
\newcommand{\jpg}[2]{\ipa{#1} `#2'}  
\newcommand{\refb}[1]{(\ref{#1})}
\newcommand{\tld}{\textasciitilde{}}
\newcommand{\zhc}[2]{\zh{#1} \ipa{#2}} 
\newcommand{\fl}{$\rightarrow$} 
%\newcommand{\bleute}[1]{\cellcolor{green}\textbf{#1}}
%\newcommand{\rouge}[1]{\cellcolor{red}\textbf{#1}}
\newcommand{\bleute}{}
\newcommand{\rouge}{}
 \begin{document} 
\title{Pseudo-cleft constructions in Japhug}
\author{Guillaume Jacques\\ CNRS-CRLAO-INALCO}
\date{}
\maketitle
%\linenumbers
Clefts and related focus constructions, 15-16 February, 2018, Villejuif
\section*{Introduction}
Among the possible means of focalizing noun phrases, most if not all languages use an equative construction with a headless relative clause (or a relative clause with an overt head noun, if this head noun is distinct from the focalized noun phrase) in the topicalized position and the focalized noun phrase as the nominal predicate, as in English `[What matters] are \textbf{his ideas}' or `[The thing that matters] is \textbf{meaning}'.
Such constructions are generally referred to as \textbf{pseudo-clefts}, by contrast with \textbf{cleft} sentences, a type of construction (such as English `It is \textbf{his ideas} [that matter]') where the focalized noun is the head of the clause defining it, that clause being built like a relative, but sometimes presenting language-specific differences with relative clauses of the same type. 

\citet[123-4]{creissels06sgit2} points out for instance that in French clefting of dative phrases (\textit{C'est à Jean \textbf{que} tu as donné le livre}) differs from the corresponding relativization (\textit{La personne \textbf{à laquelle} tu as donné le livre}). In English, proper names (without articles) cannot be normally head of a restrictive relative in \textit{that} (?\textit{John that you saw} -- a non-restrictive relative is normally used instead) but clefting occurs with \textit{that} even with personal names (\textit{It is John that I saw}).  
 

\section{Relativization in Japhug} \label{sec:relatives}
Japhug has a wide range of relative clauses, which can be classified along the following parameters (\citealt{jacques16relatives}):

\begin{itemize}
\item Participial vs finite
\item Place of the head noun (prenominal, postnominal, head-internal, headless)
\item Relativized element (S, A, P, goal, dative, possessor, adjuncts)
\end{itemize}
 
 Examples (\ref{ex:wuma.Zo.kWpe}), (\ref{ex:mANidpon}), (\ref{ex:WkWnWmbrApW}), (\ref{ex:nWkAmbi}) illustrate head-internal participial relatives with  S, A and P relativization respectively (with subject \ipa{kɯ-} vs object \ipa{kɤ-} participles).
 
\begin{exe}
   \ex  \label{ex:wuma.Zo.kWpe}
\gll  ɯ-rʑaβ βdaʁmu nɯ [wuma ʑo tɕʰeme kɯ-pe] ci pjɤ-ŋu. \\
\textsc{3sg}.\textsc{poss}-wife lady \textsc{dem} really \textsc{emph} woman \textsc{nmlz}:S/A-be.good \textsc{indef} \textsc{ifr}.\textsc{ipfv}-be \\
\glt `His wife, the queen, was a very nice woman.' (28-smAnmi, 4)
    \end{exe}  
    
\begin{exe}
   \ex  \label{ex:mANidpon}
\gll [[\ipa{mɤŋi}  	\ipa{kɤ-kɯ-ɣe}]  	\ipa{χpɯn}  	\ipa{tʰɯ-kɯ-rgɤz}] 	\ipa{ci}  	\ipa{pjɤ-tu}  	\ipa{tɕe,}    	\\
Mangi \textsc{pfv:east-nmlz:S/A}-come[II] monk \textsc{pfv-nmlz:S/A}-old \textsc{indef} \textsc{ifr.ipfv}-exist   \textsc{lnk} \\
 \glt  `There was an old monk who had come from Mangi.' (08 kWqhi, 19)
   \end{exe}  
   
   \begin{exe}
   \ex  \label{ex:WkWnWmbrApW}
\gll [[\ipa{tɤ-pɤtso}  	\ipa{ci}  	\ipa{kɯ}]  	<yangma> 	\ipa{ɯ-kɯ-nɯmbrɤpɯ}] 	\ipa{ci}  	\ipa{jɤ-ɣe}  \\
\textsc{indef.poss}-child \textsc{indef} \textsc{erg} bicycle \textsc{3sg-nmlz:S/A}-ride \textsc{indef} \textsc{pfv}-come[II] \\
\glt `A boy who was riding a bicycle arrived.' (Pear story, Chenzhen, 5)
\end{exe}

\begin{exe}
\ex \label{ex:nWkAmbi}
\gll      \ipa{tɕe} 	[\ipa{ɬamu} 	\ipa{kɯ} 	\ipa{qajɣi} 	\ipa{nɯ-kɤ-mbi}] 	\ipa{nɯ} 	\ipa{tu-ndze} 	\ipa{pjɤ-ŋu.}   \\
\textsc{lnk} Lhamo \textsc{erg} bread \textsc{pfv-nmlz}:P-give \textsc{dem} \textsc{ipfv}-eat[III] \textsc{ipfv.ifr}-be  \\
 \glt `He was eating the bread that Lhamo had given him.' (The Raven, 111)
\end{exe}  

Example (\ref{ex:WsAmWmi}) shows a headless relative with relativization of comitative argument (using the oblique participle \ipa{sɤ-}, also occurring for time/place/instrument adjuncts).

 
\begin{exe}
   \ex \label{ex:WsAmWmi}
 \gll 
\ipa{tɕe}   	[\ipa{ɯʑo}   	\ipa{ɯ-sɤ-ɤmɯmi}]   	\ipa{nɯ}   	\ipa{dɤn}   	\ipa{ma}   	\ipa{ca}   	\ipa{kɯ-fse}   	\ipa{qaʑo}   	\ipa{kɯ-fse,}   	\ipa{tsʰɤt}   	\ipa{kɯ-fse,}   	 \ipa{ɯʑo}   	\ipa{cʰo}   	\ipa{kɯ-naχtɕɯɣ}   	\ipa{sɯjno,}   	\ipa{xɕaj}   	\ipa{ma}   	\ipa{mɤ-kɯ-ndza}   	\ipa{nɯ} \ipa{ra}   	\ipa{cʰo}   	\ipa{nɯ}   	\ipa{amɯmi-nɯ}   	\ipa{tɕe,}   \\
\textsc{lnk} it \textsc{3sg-nmlz:oblique}-be.in.good.terms \textsc{dem} be.many:\textsc{fact} because musk.deer \textsc{nmlz:S/A}-be.like sheep \textsc{nmlz:S/A}-be.like goat  \textsc{nmlz:S/A}-be.like it with  \textsc{nmlz:S/A}-be.identical herbs grass apart.from \textsc{neg-nmlz:A}-eat \textsc{dem} \textsc{pl} with \textsc{dem} be.in.good.term:\textsc{fact}-\textsc{pl} \textsc{lnk} \\
\glt `The (animals) that are in good terms with the rabbit are many, it is in good terms with those that only eat grass, like musk deer, sheep or goats.' (04 qala1, 33-4)
\end{exe} 
 
Examples (\ref{ex:pWpaGWt}), (\ref{ex:nWwGmbi}),  (\ref{ex:tutianW}) (\ref{ex:jowGtsWmnW}) illustrate head-internal finite clauses with finite verb forms for object, theme or goal relativization. Finite relative clauses are identical to the equivalent independent clauses, except for the possibility of reduplicating the first syllable of the verb to express the meaning `all', as in (\ref{ex:pWpaGWt}).
 
   \begin{exe}
\ex \label{ex:pWpaGWt}
\gll
\ipa{tɕe}  	[\ipa{nɯ} \ipa{ra}  	\textbf{\ipa{tɤrɤkusna}}  	\ipa{nɯ}  	\ipa{pɯ\textasciitilde{}pa-ɣɯt}]  	\ipa{nɯ}  	\ipa{lo-ji-ndʑi}  \\
\textsc{lnk} \textsc{dem} \textsc{pl} good.crops \textsc{dem} \textsc{total\textasciitilde{}pfv:3$\rightarrow$3:down}-bring \textsc{dem} \textsc{ifr}-plant-\textsc{du} \\
\glt `They^{du} planted all the crops that she had brought (from heaven).' (flood3.111)
\end{exe}
 
\begin{exe}
\ex \label{ex:nWwGmbi}
\gll
[\ipa{tɤ-wɯ} 	\ipa{kɯ} 	\textbf{\ipa{ʑmbrɯ}}  	\ipa{nɯ́-wɣ-mbi}] 	\ipa{nɯ} 	 	\ipa{cʰɤ-lɤt} \\
\textsc{indef.poss}-grandfather \textsc{erg} boat \textsc{pfv-inv}-give \textsc{dem} \textsc{ifr}-throw \\
\glt `He took the boat that the old man had given him.' (140430 jin e, 245)
\end{exe} 
  
\begin{exe}
   \ex \label{ex:tutianW}
 \gll \ipa{nɯ}  	[\textbf{\ipa{qajɯ}}   	\textbf{\ipa{kɯ-ɲaʁ}}  	\ipa{tu-ti-a}] 	\ipa{nɯ}  	\ipa{nɯ}  	\ipa{kɯ-fse}  	\ipa{ɲɯ-βze}  	\ipa{ɲɯ-ŋu}  \\
\textsc{dem} worm \textsc{nmlz:S/A}-black \textsc{ipfv}-say-\textsc{1sg} \textsc{dem} \textsc{dem} \textsc{nmlz:S/A}-be.like \textsc{ipfv}-grow \textsc{sens}-be \\
\glt `The black worm that I was talking about grows like that.' (28 kWpAz, 30)
\end{exe}
 
     \begin{exe}
   \ex \label{ex:jowGtsWmnW}
 \gll
[\ipa{\textbf{kʰa}}  	\ipa{jɤ́-wɣ-tsɯm-nɯ}]  	\ipa{nɯnɯ,}  	[\ipa{lonba} \ipa{ɕom} \ipa{kɯ} \ipa{nɯ-kɤ-sɯ-βzu}] 	\ipa{\textbf{kʰa}}  	\ipa{pjɤ-ŋu}  \\
house \textsc{pfv-inv}-take.away-\textsc{pl} \textsc{dem} all iron \textsc{erg} \textsc{pfv-nmlz:P-caus}-make house \textsc{ipfv.ifr}-be \\
\glt `The house to which he had taken them, it was a house made of iron.' (140505 liuhaohan zoubian tianxia, 148)
\end{exe} 
 
Table (\ref{tab:summary}) presents an overview of attested relative clauses in Japhug(HI=head-internal, PN=prenominal); note that finite relatives cannot be used for the relativization of subjects.
 
\begin{table}[H]
\caption{Summary of relative clauses in Japhug } \label{tab:summary}
\resizebox{\columnwidth}{!}{
\begin{tabular}{l|ccc|ccc}
\toprule
&\multicolumn{3}{c}{Participial Relative Clause} & \multicolumn{2}{c}{Finite Relative Clause} \\
Function & \ipa{kɯ-}  & \ipa{kɤ-}  & \ipa{sɤ-}  & Simple  & Relator noun \\
\midrule
S	& HI, PN\bleute{}& &&&& \\
A & HI, PN\bleute{}& &&&& \\
\hline
possessor & PN&&&&& \\
\hline
P & & HI, PN\rouge{}&& HI, PN\rouge{} &\\
semi-object & & HI, PN\rouge{}&& HI, PN\rouge{} &\\
T & & HI, PN\rouge{} && HI, PN\rouge{}&\\
R (secundative) & & HI, PN\rouge{} && HI, PN\rouge{}&\\
\hline 
goal & & &HI, PN  & HI, PN  &\\
\hline
R (indirective) & &&HI, PN\\
comitative & &&HI, PN\\
instrumental adjunct  & &&HI, PN \\
\hline
time adjunct  & &&HI, PN&&PN\\
place adjunct  &&&HI, PN&&PN\\ 
\bottomrule
\end{tabular}}
\end{table} 
 
\section{Pseudo-clefts in Japhug} \label{sec:pseudo-clefts}
Pseudo-cleft contruction are built by  combining a headless participial relative clause in S function (generally with the determiner \ipa{nɯ}) with a nominal predicate followed by a positive (\ipa{ŋu} `be') or negative (\ipa{maʁ} `not be') copula. The basic structure of this construction is summarized in (\ref{ex:pseudo.cleft}).

\begin{exe}
\ex \label{ex:pseudo.cleft}
\glt [\textsc{nmlz}-verb] \textsc{dem} noun \textsc{copula}
\end{exe}

Pseudo-clefts in Japhug do occur in S (\ref{ex:jAkWGe.nW}), O (\ref{ex:nWkArga})  and A (\ref{ex:WkWsWmphWl}) functions, but are on the whole very rare in the corpus.\footnote{The Japhug text corpus is available on the Pangloss archive (\citealt{michailovsky14pangloss}), at the address: \url{http://lacito.vjf.cnrs.fr/pangloss/corpus/list\textunderscore rsc.php?lg=Japhug} }

\begin{exe}
\ex \label{ex:jAkWGe.nW}
\gll [\ipa{stu}	\ipa{kɯ-mɤku}	\ipa{jɤ-kɯ-ɣe}]	\ipa{nɯ}	\ipa{rɟɤlpu}	\ipa{pjɤ-ŋu.}	  \\
most \textsc{nmlz}:S/A-be.before \textsc{pfv}-\textsc{nmlz}:S/A-come[II] \textsc{dem} king \textsc{ifr}.\textsc{ipfv}-be \\
\glt `The one who came first was the king.' (140514 xizajiang he lifashi, 66)
\end{exe}

\begin{exe}
   \ex   \label{ex:nWkArga}  
\gll [\ipa{pɣa}  	\ipa{ra}  	\ipa{nɯ-kɤ-rga}]  	\ipa{nɯ}  	\ipa{qaj}  	\ipa{ntsɯ}  	\ipa{ŋu}  \\
bird \textsc{pl} \textsc{3pl-nmlz:P}-like \textsc{dem} wheat always be:\textsc{fact} \\
\glt `(The food) that birds like is always wheat (not barley).' (23 pGAYaR, 29)
\end{exe} 

Pseudo-clefts can be used to mark contrastive focus, as in (\ref{ex:WkWsWmphWl}) (unlike for instance Mandarin, according to \citealt[125]{paris79nmlz}).

\begin{exe}
\ex   \label{ex:WkWsWmphWl}  
\gll \ipa{tɕeri}	[\ipa{nɯnɯ}	\ipa{ɯ-kɯ-sɯ-mpʰɯl}]	\ipa{nɯ}	\ipa{li}	\ipa{ɯ-zrɤm}	\ipa{ɲɯ-ɕti}	\ipa{ma}	\ipa{ɯ-rɣi}	\ipa{ɲɯ-maʁ}  \\
but \textsc{dem} \textsc{3sg}.\textsc{poss}-\textsc{nmlz}:S/A-\textsc{caus}-reproduce \textsc{dem} again  \textsc{3sg}.\textsc{poss}-root \textsc{sens}-be.\textsc{affirm} \textsc{lnk} \textsc{3sg}.\textsc{poss}-seeds \textsc{sens}-not.be \\
\glt `The thing that it reproduces with is its root, not its seeds.' (11-paRzwamWntoR, 113)
\end{exe} 
 

   %iɕqha <yazi> tɤ-mu kɯ ku-ɕɯm nɯnɯ, <tiane> ɣɯ ɯ-ŋgɯm pjɤ-ɕti rca wo matɕi.

%kɯ-mɤku lu-kɯ-ɣi nɯ zdɯm nɯ kɯ-wɣrum ŋu,
%tɕe nɯnɯ aʑo ŋu-a
%ɯ-qhu nɯ tɕu zdɯm kɯ-ɲaʁ ci lu-ɣi ŋu tɕe,
%nɯnɯ tɕe tɕethi tɤmujku nɯ lu-ɣi ŋu tɕe

%\begin{exe}
%\ex \label{ex:mAkWnaXtCWG}
%\gll \ipa{tɕeri}	[\ipa{mɤ-kɯ-naχtɕɯɣ}]	\ipa{tɕe,}	\ipa{ɯ-ndzrɯ}	\ipa{ɣɤʑu} \\
%but \textsc{neg}-\textsc{nmlz}:S/A-be.similar \textsc{lnk} \textsc{3sg}.\textsc{poss}-claw exist:\textsc{sens} \\
%\glt `What is different is that it has claws.' (21-pri, 37)
%\end{exe}


%\begin{exe}
%\ex \label{ex:kAnWmga.nW}
%\gll [\ipa{tɕe}	\ipa{paʁ}	\ipa{ndɤre}	\ipa{stu}	\ipa{kɤ-nɯmga}]	\ipa{nɯ}	\ipa{ɯ-ɕa}	\ipa{ŋu}	\ipa{qʰe} \\
%\textsc{lnk} pig \textsc{lnk} most \textsc{nmlz}:S/A-want.for \textsc{dem} \textsc{3sg}.\textsc{poss}-meat be:\textsc{fact} \textsc{lnk} \\
%\glt `Pigs are (raised) for their meat.' (05-paR, 100)
%\end{exe}


%\begin{exe}
%\ex \label{ex:aZo.pWnWmtota}
%\gll
%\ipa{kɯki}	\ipa{tɕʰeme}	\ipa{ki}	\ipa{ndɤre}	\ipa{aʑɯɣ}	\ipa{a-pɯ-ŋu}	\ipa{tʂaŋ}	\ipa{ma}	\ipa{tɕe}	\ipa{aʑo}	\ipa{pɯ-nɯmto-t-a}	\ipa{ɕti}	\ipa{tɕe}	\\
%\textsc{dem}.\textsc{prox} girl \textsc{dem}.\textsc{prox} \textsc{lnk}  \textsc{1sg}:\textsc{gen} \textsc{irr}-\textsc{ipfv}-be be.fair:\textsc{fact} \textsc{lnk} \textsc{1sg} \textsc{pfv}-find-\textsc{tr}:\textsc{pst}-\textsc{1sg} be.\textsc{affirm}:\textsc{fact} \textsc{lnk} \\
%\glt `It is I who found the girl, it is fair that she should be mine.' (140517 buaishuohua, 101)
%\end{exe}

%ɯ-si ɯ-mat jɤ-kɤ-ɣɯt (thamtɕɤt) nɯ ɯʑo kɯ pɯ-kɤ-ji nɯ pjɤ-ɕti ma kɯmaʁ pjɤ-maʁ. 

 

\begin{exe}
\ex \label{ex:stu.WkAnWmga}
 \gll tɕe paʁ ɣɯ stu ɯ-kɤ-nɯmga, iʑora ji-kɤ-nɯmga nɯ ɯ-ɕa ŋu tɕe \\
 \textsc{lnk} pig \textsc{gen} most \textsc{3sg}.\textsc{poss}-\textsc{nmlz}:P-want.from \textsc{1pl} \textsc{1pl}.\textsc{poss}-\textsc{nmlz}:P-want.from \textsc{dem} \textsc{3sg}.\textsc{poss}-meat be:\textsc{fact} \textsc{lnk} \\
 \glt  `What is most wanted from pigs, what we want from them is their meat.' (05-paR, 13)
\end{exe}

%tɕe nɤkinɯ, stu ʑo ɯ-kɤ-ndza nɯnɯ tɯ-ŋga, tɤ-rme kɯ-fse tɯ-ŋga nɯra ŋu.
%28-kWpAz, 79

%tɕe nɯnɯ tɯ-ci nɯ-kɤ-tɕɤt nɯ chɤci ŋu
% 31-cha, 133

%soz tɕe tɤʑri kɤ-kɤ-ta nɯnɯ, tɕe ɯ-tɯ-ci pjɤ-ŋu. 
%150818_muzhi_guniang, 248


%kɯ-mɤmu nɯnɯ tɯrme ra ɕti.

Pseudo-clefts can also be built with oblique nominalization, as in (\ref{ex:WsAdAn})

\begin{exe}
\ex \label{ex:WsAdAn}
\gll [\ipa{stu} 	\ipa{ɯ-sɤ-dɤn}] 	\ipa{nɯ} 	\ipa{stɤmku} 	\ipa{nɯra} 	\ipa{ŋu-nɯ}  \\
most \textsc{3sg}.\textsc{poss}-\textsc{nmlz}:\textsc{oblique}-be.many \textsc{dem} grassland \textsc{dem}:\textsc{pl} be:\textsc{fact}-\textsc{pl} \\
\glt `The (places) where they are the most numerous are the grasslands.' (19-qachGa mWntoR, 24)
\end{exe}


A rare type of pseudo-cleft involves a headless relative used in nominal predicate position, as in (\ref{ex:stu.Zo.WkAnWzdWG}).

\begin{exe}
\ex \label{ex:stu.Zo.WkAnWzdWG}
\gll ɲɤ-ɣɤwu matɕi tɕendɤre nɯ [stu ʑo ɯ-kɤ-nɯzdɯɣ] nɯ pjɤ-ɕti tɕe  \\
\textsc{ifr}-cry because \textsc{lnk} \textsc{dem} most \textsc{emph} \textsc{3sg}.\textsc{poss}-\textsc{nmlz}:P-worry \textsc{dem} \textsc{ifr}.\textsc{ipfv}-be.\textsc{affirm} \textsc{lnk} \\
\glt `He cried because it was what he was most worried about.' (140506 shizi he huichang de bailingniao, 65-66)
\end{exe}

No examples of pseudo-cleft with finite relative has been found in the corpus yet.

\section{Other focalization constructions} \label{sec:other}
A focalized NP can be formally undistinguishable from non-focalized ones even by intonation, as the demonstrative \ipa{nɯ} in the last clause of example (\ref{ex:nW.mAnWXtCWG}).

\begin{exe}
\ex \label{ex:nW.mAnWXtCWG}
\gll mtʰɯmɤr ɣɯ ɯ-zbroŋ nɯnɯ, tu-nɯ-ɬoʁ ɲɯ-ŋu.  tu-mɤmbɯr kɯ-fse ɲɯ-ŋu tɕe \textbf{nɯ} mɤ-naχtɕɯɣ. \\
seal \textsc{gen} \textsc{3sg}.\textsc{poss}-pattern \textsc{dem} \textsc{ipfv}:\textsc{up}-\textsc{auto}-come.out \textsc{sens}-be \textsc{ipfv}-be.protuberant \textsc{nmlz}:S/A-be.like \textsc{sens}-be \textsc{lnk} \textsc{dem} \textsc{neg}-be.the.same:\textsc{fact} \\
\glt `The patterns on the seals (called \ipa{mtʰɯmɤr}) are coming out, protuberant, that is how they differ (from the other types of seals).' (160706 thotsi, 67)
\end{exe}

When the focalized element is the first or second person, an overt pronoun is always found, as in (\ref{ex:aZo.asni.mWjYaR}).

\begin{exe}
\ex \label{ex:aZo.asni.mWjYaR}
\gll \textbf{nɤʑo} nɤ-sni ɲɯ-ɲaʁ ma \textbf{aʑo} a-sni mɯ́j-ɲaʁ tɕe, ɬɤndʐi wuma nɯ \textbf{nɤʑo} ɲɯ-tɯ-ŋu ma \textbf{aʑo} ɬɤndʐi ɲɯ-maʁ-a \\
\textsc{2sg} \textsc{2sg}.\textsc{poss}-heart \textsc{sens}-be.black \textsc{lnk} 1sg \textsc{1sg}.\textsc{poss}-heart  \textsc{neg}:\textsc{sens}-be.black \textsc{lnk} demon real \textsc{dem} \textsc{2sg} \textsc{sens}-2-be \textsc{lnk} \textsc{1sg} demon \textsc{sens}-not.be-\textsc{1sg} \\
\glt `You are evil, not me, you are the real demon, not me.' (2002lhandzi, 12)
\end{exe}

The most common construction for focalization is combining an overt NP (or pronoun) with a postverbal copula, as in (\ref{ex:CkAnWru})\footnote{These sentences are from a story where a child at school is bullied by another pupil, who forces him to bring water for him; the teacher (who guessed that the first child was being bullied) asks (\ref{ex:CtAtWnWrut}) to have him tell the one who forced him to do it, but the bullied child replies (\ref{ex:CtAtWnWrut}), as he fears reprisals.} and (\ref{ex:nW.YWnArea}), the equivalent of (\ref{ex:nW.mAnWXtCWG}) with an additional copula. Given the fact that many periphrastic tenses use copulas, there are many cases where this focalizing function of the copula is not completely transparent (for this reason, most of the examples presented below involve verbs in perfective and inferential form, which do not occur with the copulas in periphrastic tenses).

\begin{exe}
\ex  \label{ex:CkAnWru}
\begin{xlist}
\ex \label{ex:CtAtWnWrut}
\gll \textbf{nɤʑo} nɤ-tɯ-ci ɕ-tɤ-tɯ-nɯ-ru-t ɯ́-ŋu? \\
\textsc{2sg} \textsc{2sg.poss}-\textsc{indef.poss}-water \textsc{transloc}-\textsc{pfv}-\textsc{auto}-bring-\textsc{pst}:\textsc{tr}  \textsc{qu}-be:\textsc{fact} \\
\glt `Was it for yourself that you brought the water?'
\ex \label{ex:CtAnWruta}
\gll \textbf{aʑo} ɕ-tɤ-nɯ-ru-t-a ŋu \\
\textsc{1sg} \textsc{transloc}-\textsc{pfv}-\textsc{auto}-bring-\textsc{pst}:\textsc{tr}-\textsc{1sg} be:\textsc{fact} \\
\glt `I brought it for myself.'
\end{xlist}
\end{exe}
 
 \begin{exe}
\ex \label{ex:nW.YWnArea}
\gll  kʰɯɣɲɟɯ ri pɣɤtɕɯ ni ɲɯ-ɤnɯɣro-ndʑi tɕe, \textbf{nɯ} ɲɯ-nɤre-a ɕti wo  \\
window \textsc{loc} bird \textsc{du} \textsc{sens}-play-\textsc{du} \textsc{lnk} \textsc{dem} \textsc{ipfv}--\textsc{1sg}  be.\textsc{affirm}:\textsc{fact} \textsc{sfp} \\
\glt `On the window two birds were playing, this is what I was laughing about.' (2014-kWlAG, 396)
\end{exe}


The overt NP + postverbal copula construction is found to express focus on intransitive subject (\ref{ex:Wpi.mWpjArAzi}), transitive subject (without \ref{ex:aZo.pWnWmtota} or with \ref{ex:aZo.kW.tAndzata.Cti} optional ergative on the overt pronoun), object (\ref{ex:tutia.ɕti}) but also core argument-cum-beneficiary (as in \ref{ex:CtAtWnWrut}).

\begin{exe}
\ex \label{ex:Wpi.mWpjArAzi}
\gll tɕeri, ɯ-pi mɯ-pjɤ-rɤʑi qʰendɤre,  \textbf{ɯ-ɬaʁ} \textbf{nɯ} pjɤ-rɤʑi ɕti qʰe  \\
\textsc{lnk} \textsc{3sg}.\textsc{poss}-elder.sibling \textsc{neg}-\textsc{ifr}.\textsc{ipfv}-stay \textsc{lnk} \textsc{3sg}.\textsc{poss}-aunt \textsc{dem} \textsc{neg}-\textsc{ifr}.\textsc{ipfv}-stay be.\textsc{affirm}:\textsc{fact} \textsc{lnk} \\
\glt  `But his elder brother was not there, it was his brother's wife who was there.' (140512 alibaba, 62)
\end{exe}

\begin{exe}
\ex \label{ex:aZo.pWnWmtota}
\gll  kɯki tɕʰeme ki ndɤre aʑɯɣ a-pɯ-ŋu tʂaŋ ma tɕe \textbf{aʑo} pɯ-nɯ-mto-t-a ɕti tɕe \\
\textsc{dem}.\textsc{prox} girl  \textsc{dem}.\textsc{prox} \textsc{lnk} \textsc{1sg}.\textsc{gen} \textsc{irr}-\textsc{ipfv}-be be.fair:\textsc{fact} \textsc{lnk} \textsc{lnk} \textsc{1sg} \textsc{pfv}-\textsc{auto}-see-\textsc{pst}:\textsc{tr}-\textsc{1sg} be.\textsc{affirm}:\textsc{fact} \textsc{lnk} \\
\glt `This girl, it would be fair if she were mine, as it was I who found her.' (140517 buaishuohua)
\end{exe}

\begin{exe}
\ex \label{ex:aZo.kW.tAndzata.Cti}
\gll ɯʑo kɯ aʑo kɤ-ndza maʁ kɯ,  \textbf{aʑo} \textbf{kɯ} ɯʑo tɤ-ndza-t-a ɕti \\
\textsc{3sg} \textsc{erg} \textsc{1sg} \textsc{inf}-eat not.be:\textsc{fact} \textsc{erg} \textsc{1sg} \textsc{erg} \textsc{3sg} \textsc{pfv}-eat-\textsc{pst}:\textsc{tr}-\textsc{1sg}  be.\textsc{affirm}:\textsc{fact} \\
\glt   `Rather than me being eaten by him, it is I who ate him.' (2003twxtsa, 186)
\end{exe}

\begin{exe}
\ex \label{ex:tutia.ɕti}
\gll aʑo \textbf{pɤnmawombɤr} tu-ti-a ɕti ma, nɤj ɲɯ-ta-nɯ-ɤkʰɤzŋga maʁ  \\
\textsc{1sg} p.n. \textsc{ipfv}-say-\textsc{1sg} be.\textsc{affirm}:\textsc{fact} \textsc{lnk} \textsc{2sg} \textsc{ipfv}-1\fl{}2-\textsc{appl}-call not.be:\textsc{fact} \\
\glt `I am calling ``Padma 'Od 'bar", not you.' (`It is Padma 'Od 'bar that I am saying, I am not calling you'; Norbzang2012, 163)
\end{exe}

\section{Question}
Could the copula focalization construction discussed in section \ref{sec:other} be analyzed as a cleft construction? 
\begin{itemize}
\item In such an analysis, in example (\ref{ex:aZo.kW.tAndzata.Cti}) for instance  [\textit{aʑo kɯ ɯʑo tɤ-ndza-t-a}] would be a finite head-internal relative clause meaning `I who ate him', with ergatively marked \textit{aʑo kɯ} as its head. The absence of \textsc{1sg} indexation on the copula \textit{ɕti} here would be in this analysis no more surprising than the expletive pronoun and the third person copula in the English cleft construction `It is me who X'. 
\item More problematic here is that as shown in § \ref{sec:relatives}, finite relatives are not possible with subject (S and A) relativization, only for object and goal. The clause \textit{aʑo kɯ ɯʑo tɤ-ndza-t-a} thus cannot be interpreted as A-relativization (at best, it could mean `He whom I ate'). 
\item If in (\ref{ex:tutia.ɕti}) \textit{pɤnmawombɤr tu-ti-a} is analyzed as a head-internal relative (see \ref{ex:tutianW} for an equivalent example with the same verb form) we would have a non-restrictive head-internal relative headed by a personal name.
\item Another difference between finite relatives and the sequence preceding the copula in the focalization construction is the fact that finite relatives can be, and are often, followed by demonstratives or number markers -- all examples (\ref{ex:pWpaGWt}),  (\ref{ex:nWwGmbi}) and (\ref{ex:tutianW}) above take the demonstrative \ipa{nɯ}. In the case of the focalization construction, it is not possible to add a demonstrative.
\end{itemize}
However, as mentioned in the introduction, differences between cleft constructions and normal relative clauses are expected; would a cleft analysis be tenable?

 
%\begin{exe}
%\ex 
%\gll ɯ-pɯ ra ɲɤ-me-nɯ tɕe tɕe, [kɯki qaliaʁ kɯ ta-nɤma] ŋu nɯnɯ ko-tso ɲɯ-ŋu. \\
%\textsc{3sg}.\textsc{poss}-young \textsc{pl} \textsc{ifr}-not.exist-\textsc{pl} \textsc{lnk} \textsc{lnk} \textsc{dem}.\textsc{prox} eagle \textsc{erg} \textsc{pfv}:3\fl{}3'-do \textsc{dem} \textsc{ifr}-understood \textsc{sens}-be \\
%\glt `His youngs had disappeared, and he knew that this what something that the eagle had done' (huli yu shanying, 31)
%\end{exe}
%
%\zh{知道这事是山鹰所做}

%\section*{Conclusion}
\bibliographystyle{unified}
\bibliography{bibliogj}
 \end{document}
 