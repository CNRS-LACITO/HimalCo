\documentclass[oneside,a4paper,11pt]{article} 
\usepackage{fontspec}
\usepackage{natbib}
\usepackage{booktabs}
\usepackage{xltxtra} 
\usepackage{polyglossia} 
 \usepackage{geometry}
% \geometry{
% a4paper,
% total={210mm,297mm},
% left=25mm,
% right=25mm,
% top=20mm,
% bottom=20mm,
% }
\usepackage[table]{xcolor}
\usepackage{gb4e} 
\usepackage{multicol}
\usepackage{graphicx}
\usepackage{float}
\usepackage{hyperref} 
\hypersetup{bookmarks=false,bookmarksnumbered,bookmarksopenlevel=5,bookmarksdepth=5,xetex,colorlinks=true,linkcolor=blue,citecolor=blue}
\usepackage[all]{hypcap}
\usepackage{memhfixc}
\usepackage{lscape}
\usepackage{amssymb}
\usepackage{lineno}
 
\setmainfont[Mapping=tex-text,Numbers=OldStyle,Ligatures=Common]{Charis SIL} 
\newfontfamily\phon[Mapping=tex-text,Ligatures=Common,Scale=MatchLowercase]{Charis SIL} 
\newcommand{\ipa}[1]{{\phon\textit{#1}}} 
\newcommand{\grise}[1]{\cellcolor{lightgray}\textbf{#1}}
\newfontfamily\cn[Mapping=tex-text,Ligatures=Common,Scale=MatchUppercase]{SimSun}%pour le chinois
\newcommand{\zh}[1]{{\cn #1}}
\newcommand{\Y}{\Checkmark} 
\newcommand{\N}{} 
\newcommand{\jpg}[2]{\ipa{#1} `#2'}  
\newcommand{\refb}[1]{(\ref{#1})}
\newcommand{\tld}{\textasciitilde{}}
\newcommand{\zhc}[2]{\zh{#1} \ipa{#2}} 

 \begin{document} 
\title{Pseudo-cleft constructions in Japhug}
\author{Guillaume Jacques\\ CNRS-CRLAO-INALCO}
\maketitle
\linenumbers

\section*{Introduction}
Among the possible means of focalizing nouns phrases, most if not all languages use an equative construction with a headless relative clause (or a relative clause with a overt head noun, if this head noun is distinct from the focalized noun phrase) in the topicalized position and the focalized noun phrase as the nominal predicate, as in English `[What matters] are \textbf{his ideas}' or `[The thing that matters] is \textbf{meaning}'.
Such constructions are generally referred to as \textbf{pseudo-clefts}, by opposition with \textbf{cleft} sentences, a type of construction (such as English `It is \textbf{his ideas} [that matter]') where the focalized noun is the head of the clause defining it, that clause being built like a relative, but sometimes presenting language-specific differences with relative clauses of the same type
(see for instance \citealt[123-4]{creissels06sgit2} on cleft sentences in French with focalisation on dative phrases). 

 

\section{Relativization in Japhug}
 \citealt{jacques16relatives}
 
\section{Pseudo-clefts in Japhug}
Pseudo-cleft contruction are built by  combining a headless participial relative clause in S function (generally with the determiner \ipa{nɯ}) with a nominal predicate followed by a positive (\ipa{ŋu} `be') or negative (\ipa{maʁ} `not be') copula. The basic structure of this construction is summarized in (\ref{ex:pseudo.cleft}).

\begin{exe}
\ex \label{ex:pseudo.cleft}
\glt [\textsc{nmlz}-verb] \textsc{dem} noun \textsc{copula}
\end{exe}

Pseudo-clefts in Japhug do occur in S (\ref{ex:jAkWGe.nW}), O (\ref{ex:nWkArga})  and A (\ref{ex:WkWsWmphWl}) functions, but are on the whole very rare in the corpus.\footnote{The Japhug text corpus is available on the Pangloss archive (\citealt{michailovsky14pangloss}), at the address: \url{http://lacito.vjf.cnrs.fr/pangloss/corpus/list\textunderscore rsc.php?lg=Japhug} }

\begin{exe}
\ex \label{ex:jAkWGe.nW}
\gll [\ipa{stu}	\ipa{kɯ-mɤku}	\ipa{jɤ-kɯ-ɣe}]	\ipa{nɯ}	\ipa{rɟɤlpu}	\ipa{pjɤ-ŋu.}	  \\
most \textsc{nmlz}:S/A-be.before \textsc{pfv}-\textsc{nmlz}:S/A-come[II] \textsc{dem} king \textsc{ifr}.\textsc{ipfv}-be \\
\glt `The one who came first was the king.' (140514 xizajiang he lifashi, 66)
\end{exe}

\begin{exe}
   \ex   \label{ex:nWkArga}  
\gll [\ipa{pɣa}  	\ipa{ra}  	\ipa{nɯ-kɤ-rga}]  	\ipa{nɯ}  	\ipa{qaj}  	\ipa{ntsɯ}  	\ipa{ŋu}  \\
bird \textsc{pl} \textsc{3pl-nmlz:P}-like \textsc{dem} wheat always be:\textsc{fact} \\
\glt `(The food) that birds like is always wheat (not barley).' (23 pGAYaR, 29)
\end{exe} 

Pseudo-clefts can be used to mark contrastive focus, as in (\ref{ex:WkWsWmphWl}) (unlike for instance Mandarin, according to \citealt[125]{paris79nmlz}).

\begin{exe}
\ex   \label{ex:WkWsWmphWl}  
\gll \ipa{tɕeri}	[\ipa{nɯnɯ}	\ipa{ɯ-kɯ-sɯ-mpʰɯl}]	\ipa{nɯ}	\ipa{li}	\ipa{ɯ-zrɤm}	\ipa{ɲɯ-ɕti}	\ipa{ma}	\ipa{ɯ-rɣi}	\ipa{ɲɯ-maʁ}  \\
but \textsc{dem} \textsc{3sg}.\textsc{poss}-\textsc{nmlz}:S/A-\textsc{caus}-reproduce \textsc{dem} again  \textsc{3sg}.\textsc{poss}-root \textsc{sens}-be.\textsc{affirm} \textsc{lnk} \textsc{3sg}.\textsc{poss}-seeds \textsc{sens}-not.be \\
\glt `The thing that it reproduces with is its root, not its seeds.' (11-paRzwamWntoR, 113)
\end{exe} 
 

   %iɕqha <yazi> tɤ-mu kɯ ku-ɕɯm nɯnɯ, <tiane> ɣɯ ɯ-ŋgɯm pjɤ-ɕti rca wo matɕi.

%kɯ-mɤku lu-kɯ-ɣi nɯ zdɯm nɯ kɯ-wɣrum ŋu,
%tɕe nɯnɯ aʑo ŋu-a
%ɯ-qhu nɯ tɕu zdɯm kɯ-ɲaʁ ci lu-ɣi ŋu tɕe,
%nɯnɯ tɕe tɕethi tɤmujku nɯ lu-ɣi ŋu tɕe

%\begin{exe}
%\ex \label{ex:mAkWnaXtCWG}
%\gll \ipa{tɕeri}	[\ipa{mɤ-kɯ-naχtɕɯɣ}]	\ipa{tɕe,}	\ipa{ɯ-ndzrɯ}	\ipa{ɣɤʑu} \\
%but \textsc{neg}-\textsc{nmlz}:S/A-be.similar \textsc{lnk} \textsc{3sg}.\textsc{poss}-claw exist:\textsc{sens} \\
%\glt `What is different is that it has claws.' (21-pri, 37)
%\end{exe}


%\begin{exe}
%\ex \label{ex:kAnWmga.nW}
%\gll [\ipa{tɕe}	\ipa{paʁ}	\ipa{ndɤre}	\ipa{stu}	\ipa{kɤ-nɯmga}]	\ipa{nɯ}	\ipa{ɯ-ɕa}	\ipa{ŋu}	\ipa{qʰe} \\
%\textsc{lnk} pig \textsc{lnk} most \textsc{nmlz}:S/A-want.for \textsc{dem} \textsc{3sg}.\textsc{poss}-meat be:\textsc{fact} \textsc{lnk} \\
%\glt `Pigs are (raised) for their meat.' (05-paR, 100)
%\end{exe}


%\begin{exe}
%\ex \label{ex:aZo.pWnWmtota}
%\gll
%\ipa{kɯki}	\ipa{tɕʰeme}	\ipa{ki}	\ipa{ndɤre}	\ipa{aʑɯɣ}	\ipa{a-pɯ-ŋu}	\ipa{tʂaŋ}	\ipa{ma}	\ipa{tɕe}	\ipa{aʑo}	\ipa{pɯ-nɯmto-t-a}	\ipa{ɕti}	\ipa{tɕe}	\\
%\textsc{dem}.\textsc{prox} girl \textsc{dem}.\textsc{prox} \textsc{lnk}  \textsc{1sg}:\textsc{gen} \textsc{irr}-\textsc{ipfv}-be be.fair:\textsc{fact} \textsc{lnk} \textsc{1sg} \textsc{pfv}-find-\textsc{tr}:\textsc{pst}-\textsc{1sg} be.\textsc{affirm}:\textsc{fact} \textsc{lnk} \\
%\glt `It is I who found the girl, it is fair that she should be mine.' (140517 buaishuohua, 101)
%\end{exe}

%ɯ-si ɯ-mat jɤ-kɤ-ɣɯt (thamtɕɤt) nɯ ɯʑo kɯ pɯ-kɤ-ji nɯ pjɤ-ɕti ma kɯmaʁ pjɤ-maʁ. 

 

\begin{exe}
\ex \label{ex:stu.WkAnWmga}
 \gll tɕe paʁ ɣɯ stu ɯ-kɤ-nɯmga, iʑora ji-kɤ-nɯmga nɯ ɯ-ɕa ŋu tɕe \\
 \textsc{lnk} pig \textsc{gen} most \textsc{3sg}.\textsc{poss}-\textsc{nmlz}:P-want.from \textsc{1pl} \textsc{1pl}.\textsc{poss}-\textsc{nmlz}:P-want.from \textsc{dem} \textsc{3sg}.\textsc{poss}-meat be:\textsc{fact} \textsc{lnk} \\
 \glt  `What is most wanted from pigs, what we want from them is their meat.' (05-paR, 13)
\end{exe}

%tɕe nɤkinɯ, stu ʑo ɯ-kɤ-ndza nɯnɯ tɯ-ŋga, tɤ-rme kɯ-fse tɯ-ŋga nɯra ŋu.
%28-kWpAz, 79

%tɕe nɯnɯ tɯ-ci nɯ-kɤ-tɕɤt nɯ chɤci ŋu
% 31-cha, 133

%soz tɕe tɤʑri kɤ-kɤ-ta nɯnɯ, tɕe ɯ-tɯ-ci pjɤ-ŋu. 
%150818_muzhi_guniang, 248


%kɯ-mɤmu nɯnɯ tɯrme ra ɕti.

Oblique:

 \begin{exe}
\ex \label{ex:thongthar}
\gll [\ipa{qandʑi}   	\ipa{cʰɯ-sɤ-ɣnda}]   	\ipa{nɯ}   	\ipa{tʰoŋtʰɤr}   	  	\ipa{ɲɯ-rmi}    \\
bullet \textsc{ipf}-\textsc{nmlz:oblique}-ram   \textsc{dem} ramrod \textsc{testim}-call \\
 \glt `What is used to ram a bullet (into the muzzle of the gun) is called a ramrod.' (28-CAmWGdW, 55)
 \end{exe}
 
\begin{exe}
\ex \label{ex:WsAdAn}
\gll [\ipa{stu} 	\ipa{ɯ-sɤ-dɤn}] 	\ipa{nɯ} 	\ipa{stɤmku} 	\ipa{nɯra} 	\ipa{ŋu-nɯ}  \\
most \textsc{3sg}.\textsc{poss}-\textsc{nmlz}:\textsc{oblique}-be.many \textsc{dem} grassland \textsc{dem}:\textsc{pl} be:\textsc{fact}-\textsc{pl} \\
\glt `The (places) where they are the most numerous are the grasslands.' (19-qachGa mWntoR, 24)
\end{exe}

\begin{exe}
\ex \label{ex:stu.Zo.WkAnWzdWG}
\gll ɲɤ-ɣɤwu matɕi tɕendɤre nɯ [stu ʑo ɯ-kɤ-nɯzdɯɣ] nɯ pjɤ-ɕti tɕe  \\
\textsc{ifr}-cry because \textsc{lnk} \textsc{dem} most \textsc{emph} \textsc{3sg}.\textsc{poss}-\textsc{nmlz}:P-worry \textsc{dem} \textsc{ifr}.\textsc{ipfv}-be.\textsc{affirm} \textsc{lnk} \\
\glt `He cried because it was what he was most worried about.' (140506 shizi he huichang de bailingniao, 65-66)
\end{exe}

\section{Other focalization construction}

\begin{exe}
\ex \label{ex:nW.mAnWXtCWG}
\gll mtʰɯmɤr ɣɯ ɯ-zbroŋ nɯnɯ, tu-nɯ-ɬoʁ ɲɯ-ŋu.  tu-mɤmbɯr kɯ-fse ɲɯ-ŋu tɕe nɯ mɤ-naχtɕɯɣ. \\
seal \textsc{gen} \textsc{3sg}.\textsc{poss}-pattern \textsc{dem} \textsc{ipfv}:\textsc{up}-\textsc{auto}-come.out \textsc{sens}-be \textsc{ipfv}-be.protuberant \textsc{nmlz}:S/A-be.like \textsc{sens}-be \textsc{lnk} \textsc{dem} \textsc{neg}-be.the.same:\textsc{fact} \\
\glt `The patterns on the seals (called \ipa{mtʰɯmɤr}) are coming out, protuberant, that is how they differ from (the other types of seals).' (160706 thotsi, 67)
\end{exe}

Overt pronoun+copula at the end. NOT relative

\begin{exe}
\ex \label{ex:aZo.pWnWmtota}
\gll  kɯki tɕʰeme ki ndɤre aʑɯɣ a-pɯ-ŋu tʂaŋ ma tɕe aʑo pɯ-nɯ-mto-t-a ɕti tɕe \\
\textsc{dem}.\textsc{prox} girl  \textsc{dem}.\textsc{prox} \textsc{lnk} \textsc{1sg}.\textsc{gen} \textsc{irr}-\textsc{ipfv}-be be.fair:\textsc{fact} \textsc{lnk} \textsc{lnk} \textsc{1sg} \textsc{pfv}-\textsc{auto}-see-\textsc{pst}:\textsc{tr}-\textsc{1sg} be.\textsc{affirm}:\textsc{fact} \textsc{lnk} \\
\glt `This girl, it would be fair if she were mine, as it was I who found her.' (140517 buaishuohua)
\end{exe}
\begin{exe}
\ex \label{ex:Wpi.mWpjArAzi}
\gll 
tɕeri, ɯ-pi mɯ-pjɤ-rɤʑi qʰendɤre,  ɯ-ɬaʁ nɯ pjɤ-rɤʑi ɕti qʰe  \\
\\
\glt  `But his elder brother was not there, it was his brother's wife who was there.' (140512 alibaba, 62)
\end{exe}

%\begin{exe}
%\ex 
%\gll ɯ-pɯ ra ɲɤ-me-nɯ tɕe tɕe, [kɯki qaliaʁ kɯ ta-nɤma] ŋu nɯnɯ ko-tso ɲɯ-ŋu. \\
%\textsc{3sg}.\textsc{poss}-young \textsc{pl} \textsc{ifr}-not.exist-\textsc{pl} \textsc{lnk} \textsc{lnk} \textsc{dem}.\textsc{prox} eagle \textsc{erg} \textsc{pfv}:3\fl{}3'-do \textsc{dem} \textsc{ifr}-understood \textsc{sens}-be \\
%\glt `His youngs had disappeared, and he knew that this what something that the eagle had done' (huli yu shanying, 31)
%\end{exe}
%
%\zh{知道这事是山鹰所做}

\section*{Conclusion}
\bibliographystyle{unified}
\bibliography{bibliogj}
 \end{document}
 