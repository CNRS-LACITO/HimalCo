\documentclass[oldfontcommands,twoside,a4paper,12pt]{article} 
\usepackage{fontspec}
\usepackage{natbib}
\usepackage{booktabs}
\usepackage{xltxtra} 
\usepackage{polyglossia} 
 \usepackage{geometry}
 \geometry{
 a4paper,
 total={210mm,297mm},
 left=10mm,
 right=10mm,
 top=15mm,
 bottom=15mm,
 }
\usepackage[table]{xcolor}
\usepackage{color}
\usepackage{multirow}
\usepackage{gb4e} 
\usepackage{multicol}
\usepackage{graphicx}
\usepackage{float}
\usepackage{hyperref} 
\hypersetup{bookmarks=false,bookmarksnumbered,bookmarksopenlevel=5,bookmarksdepth=5,xetex,colorlinks=true,linkcolor=blue,citecolor=blue}
\usepackage{memhfixc}
\usepackage{lscape}
\usepackage[footnotesize,bf]{caption}
 

%%%%%%%%%%%%%%%%%%%%%%%%%%%%%%%
\setmainfont[Mapping=tex-text,Numbers=OldStyle,Ligatures=Common]{Charis SIL} 
\setsansfont[Mapping=tex-text,Ligatures=Common,Mapping=tex-text,Ligatures=Common,Scale=MatchLowercase]{Lucida Sans Unicode} 
 


\newfontfamily\phon[Mapping=tex-text,Ligatures=Common,Scale=MatchLowercase,FakeSlant=0.3]{Charis SIL} 
\newfontfamily\phondroit[Mapping=tex-text,Ligatures=Common,Scale=MatchLowercase]{Doulos SIL} 
\newcommand{\ipa}[1]{{\phon\textbf{#1}}} 
\newcommand{\ipab}[1]{{\phon #1}}
\newcommand{\ipapl}[1]{{\phondroit #1}} 
\newcommand{\captionft}[1]{{\captionfont #1}} 
\newfontfamily\cn[Mapping=tex-text,Ligatures=Common,Scale=MatchUppercase]{MingLiU}%pour le chinois
\newcommand{\zh}[1]{{\cn #1}}
\newcommand{\tgf}[1]{{\large\mo{#1}}}

\newcommand{\acc}{\textsc{acc}}
\newcommand{\antierg}{\textsc{antierg}}
\newcommand{\allat}{\textsc{all}}
\newcommand{\aor}{\textsc{aor}}
\newcommand{\assert}{\textsc{assert}}
\newcommand{\auto}{\textsc{auto}}
\newcommand{\caus}{\textsc{caus}}
\newcommand{\classif}{\textsc{class}}
\newcommand{\concessif}{\textsc{concsf}}
\newcommand{\comit}{\textsc{comit}}
\newcommand{\conj}{\textsc{conj}}
\newcommand{\const}{\textsc{const}}
\newcommand{\conv}{\textsc{conv}}
\newcommand{\cop}{\textsc{cop}}
\newcommand{\dat}{\textsc{dat}}
\newcommand{\dem}{\textsc{dem}}
\newcommand{\detm}{\textsc{det}}
\newcommand{\dir}{\textsc{dir1}}
\newcommand{\du}{\textsc{du}}
\newcommand{\duposs}{\textsc{du.poss}}
\newcommand{\dur}{\textsc{dur}}
\newcommand{\erg}{\textsc{erg}}
\newcommand{\evd}{\textsc{evd}}
\newcommand{\fut}{\textsc{fut}}
\newcommand{\gen}{\textsc{gen}}
\newcommand{\hypot}{\textsc{hyp}}
\newcommand{\ideo}{\textsc{ideo}}
\newcommand{\imp}{\textsc{imp}}
\newcommand{\impf}{\textsc{ipfv}}
\newcommand{\instr}{\textsc{instr}}
\newcommand{\intens}{\textsc{intens}}
\newcommand{\intrg}{\textsc{intrg}}
\newcommand{\inv}{\textsc{inv}}
\newcommand{\irreel}{\textsc{irr}}
\newcommand{\loc}{\textsc{loc}}
\newcommand{\med}{\textsc{med}}
\newcommand{\negat}{\textsc{neg}}
\newcommand{\neu}{\textsc{neu}}
\newcommand{\nmlz}{\textsc{nmlz}}
\newcommand{\nonps}{\textsc{n.pst}}
\newcommand{\opt}{\textsc{dir2}}
\newcommand{\perf}{\textsc{pfv}}
\newcommand{\pl}{\textsc{pl}}
\newcommand{\plposs}{\textsc{pl.poss}}
\newcommand{\poss}{\textsc{poss}}
\newcommand{\pot}{\textsc{pot}}
\newcommand{\prohib}{\textsc{prohib}}
\newcommand{\pst}{\textsc{pst}}
\newcommand{\recip}{\textsc{recip}}
\newcommand{\redp}{\textsc{redp}}
\newcommand{\refl}{\textsc{refl}}
\newcommand{\sg}{\textsc{sg}}
\newcommand{\sgposs}{\textsc{sg.poss}}
\newcommand{\stat}{\textsc{stat}}
\newcommand{\topic}{\textsc{top}}
\newcommand{\volit}{\textsc{vol}}

\newcommand{\racine}[1]{\begin{math}\sqrt{#1}\end{math}} 
\newcommand{\grise}[1]{\cellcolor{lightgray}\textbf{#1}} 
\newcommand{\tinynb}[1]{\tiny#1}
\begin{document}

\title{Exercices (marquage et alignement)}
\author{Guillaume Jacques\\Anton Antonov}
\maketitle

\section{Japhug}
 \begin{exe}
\ex
\gll
\ipa{rɟɤlpu}  	\ipa{nɯ}  	\ipa{mɯ-pjɤ-rɤʑit}  \\
roi \textsc{dem} \textsc{neg-evd.ipfv}-rester.là \\
 \glt Le roi n'était pas là.
\end{exe}

 \begin{exe}
\ex 
\gll \ipa{rɟɤlpu}  	\ipa{nɯ}  	\ipa{kɯ}  	\ipa{tɕhemɤpɯ}  	\ipa{nɯ}  	\ipa{pjɤ-mto}   \\
roi \textsc{dem} ??? fille \textsc{dem} \textsc{evd}-voir \\
\glt Le roi vit la fille.
\end{exe}

 \begin{exe}
\ex 
\gll 
\ipa{rɟɤlpu}  	\ipa{nɯ}  	\ipa{kɯ}  	\ipa{li}  	\ipa{ci}  	\ipa{ɯ-rʑaβ}  	\ipa{kɯ-ɕɤɣ}  	\ipa{ci}  	\ipa{ɲɤ-ɕar.}  	 \\
roi \textsc{dem} ??? encore \textsc{indef} \textsc{3sg.poss}-épouse \textsc{nmls}:S/A-être.nouveau \textsc{indef}  \textsc{evd}-chercher \\
\glt Le roi se remaria.
\end{exe}

 \begin{exe}
\ex 
\gll 
\ipa{tɕheme}  	\ipa{nɯ}  	\ipa{kɯ}  	\ipa{rɟɤlpu}  	\ipa{ɯ-ɕki}  	``\ipa{atu}  	\ipa{pɣɤtɕɯ}  	\ipa{nɯ}  	\ipa{pjɯ́-wɣ-sat}  	\ipa{ɲɯ-ra}"  	\ipa{to-ti}  	\\
fille \textsc{dem} ??? roi \textsc{3sg.poss}-??? là.haut oiseau \textsc{dem} \textsc{ipfv-inv}-tuer \textsc{testim}-falloir \textsc{evd}-dire \\
\glt La fille dit au roi: ``Il faut tuer l'oiseau là-haut."
\end{exe}


 \begin{exe}
\ex 
\gll \ipa{iɕqha}  	\ipa{tɯtsɣe}  	\ipa{ɯ-kɯ-βzu}  	\ipa{nɯ}  	\ipa{kɯ}  	\ipa{kɯki}  	\ipa{qala-pɯ}  	\ipa{nɯnɯ}  	\ipa{tɤɕime}  	\ipa{ɯ-ɕki}  	\ipa{ɲɤ-kho.}  	\\
à.l'instant commerce \textsc{3sg.poss-nmls}:S/A-faire \textsc{dem} ??? \textsc{dem.prox} lapin-petit \textsc{dem} princesse \textsc{3sg.poss}-??? \textsc{evd}-donner \\
\glt Le marchand donna le petit lapin à la princesse.
\end{exe}

 \begin{exe}
\ex 
\gll
\ipa{tɕheme}  	\ipa{nɯ}  	\ipa{kɯ}  	\ipa{qɤjɣi}  	\ipa{kɤ-kɯ-ɕke}  	\ipa{nɯ}  	\ipa{ra}  	\ipa{kɯ-lɤɣ}  	\ipa{nɯ}  	\ipa{na-mbi,}  \\
fille \textsc{dem} ??? pain \textsc{pfv-nmls}:S/A-être.brûlé \textsc{dem} \textsc{pl} \textsc{nmls}:S/A-faire.paître \textsc{dem} \textsc{pfv}:3$\rightarrow$3'-donner \\
\glt La fille donna les (morceaux de) pain brûlés au pâtre.
\end{exe}


\section{Sahaptien}

 \begin{exe}
\ex 
\gll
\ipa{i-wiyánawi-ya}  	\ipa{tílaaki}  \\
\textsc{3sg}:S/A-arriver-\textsc{passé} femme \\
\glt La femme est arrivée.
\end{exe}

 \begin{exe}
\ex 
\gll
\ipa{tílaaki}  	\ipa{i-q'ínun-a}  	\ipa{wínš-na}   \\
		femme \textsc{3sg}:S/A-voir-\textsc{passé} homme-??? \\
\glt La femme a vu l'homme.
\end{exe}

 \begin{exe}
\ex 
\gll
 \ipa{áswan}    	\ipa{i-ní-ya}  \ipa{tílaaki-na} 	\ipa{x̣ax̣áykʷ}   \\
		enfant \textsc{3sg}:S/A-voir-\textsc{passé} femme-??? argent \\
\glt L'enfant a donné de l'argent à la femme.
\end{exe}


 \begin{exe}
\ex 
\gll
\ipa{ku}  	\ipa{i-q'ínun-a}  \ipa{áswan}  	\ipa{níit(-na)}   \\
		et \textsc{3sg}:S/A-voir-\textsc{passé} enfant maison(-???) \\
\glt  et l'enfant a vu la maison.
\end{exe}

 \begin{exe}
\ex 
\gll
\ipa{ku}  	\ipa{i-yax̣n-a}  \ipa{tílaaki}  	\ipa{miyánaš}   \\
		et \textsc{3sg}:S/A-trouver-\textsc{passé} femme enfant \\
\glt  et la femme a trouvé son enfant.
\end{exe}

 
 \begin{exe}
\ex 
\gll
\ipa{ku=š}  	\ipa{ín}  	\ipa{á-q'ínun-a}  	\ipa{tílaaki-na}  \\
et=\textsc{1sg} \textsc{1sg}:??? \textsc{obv}-voir-\textsc{passé} femme-??? \\
\glt et j'ai vu la femme.
\end{exe}

 \begin{exe}
\ex 
\gll
\ipa{ku=š}  	\ipa{ína}  	\ipa{i-q'inun-a}  	\ipa{tílaaki-nɨm}  \\
et=\textsc{1sg} \textsc{1sg}:???  \textsc{3sg}:S/A-voir-\textsc{passé} femme-??? \\
\glt et la femme m'a vu.
\end{exe}

\end{document}

