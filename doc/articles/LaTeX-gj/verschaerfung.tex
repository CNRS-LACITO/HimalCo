\documentclass[oneside,a4paper,11pt]{article} 
\usepackage{fontspec}
\usepackage{natbib}
\usepackage{booktabs}
\usepackage{xltxtra} 
\usepackage{polyglossia} 
\setmainlanguage{french}
\usepackage[table]{xcolor}
\usepackage{tikz}
\usetikzlibrary{trees}
\usepackage{gb4e} 
\usepackage{multicol}
\usepackage{graphicx}
\usepackage{float}
\usepackage{hyperref} 
\hypersetup{bookmarksnumbered,bookmarksopenlevel=5,bookmarksdepth=5,colorlinks=true,linkcolor=blue,citecolor=blue}
\usepackage[all]{hypcap}
\usepackage{memhfixc}
\usepackage{lscape}
\usepackage{amssymb}
 
\bibpunct[: ]{(}{)}{,}{a}{}{,}

%\setmainfont[Mapping=tex-text,Numbers=OldStyle,Ligatures=Common]{Charis SIL} 
\newfontfamily\phon[Mapping=tex-text,Scale=MatchLowercase]{Charis SIL} 
\newcommand{\ipa}[1]{\textbf{{\phon\mbox{#1}}}} %API tjs en italique
%\newcommand{\ipab}[1]{{\scriptsize \phon#1}} 

\newcommand{\grise}[1]{\cellcolor{lightgray}\textbf{#1}}
\newfontfamily\cn[Mapping=tex-text,Ligatures=Common,Scale=MatchUppercase]{SimSun}%pour le chinois
\newcommand{\zh}[1]{{\cn #1}}
\newfontfamily\mleccha[Mapping=tex-text,Ligatures=Common,Scale=MatchLowercase]{Galatia SIL}%pour le grec
\newcommand{\grec}[1]{{\mleccha #1}}
\newfontfamily\tibetain{Microsoft Himalaya} 

\newcommand{\sg}{\textsc{sg}}
\newcommand{\pl}{\textsc{pl}}
\newcommand{\ro}{$\Sigma$}
\newcommand{\ra}{$\Sigma_1$} 
\newcommand{\rc}{$\Sigma_3$}  
\newcommand{\dhatu}[2]{|\ipa{#1}| `#2'}
\newcommand{\tibet}[3]{{\tibetain#1} \textit{\phon#2} `#3'}  
\newcommand{\tibetan}[1]{{\tibetain#1}}
\newcommand{\dhat}[1]{|\ipa{#1}|}
\newcommand{\change}[2]{*\ipa{#1} $\rightarrow$ \ipa{#2}}
 

\XeTeXlinebreakskip = 0pt plus 1pt %
 %CIRCG
 
\newcommand{\zhc}[2]{\zh{#1} \ipa{#2}} 


\begin{document}

\title{La Verschärfung du yod en tibétain à la lumière des emprunts à l'indo-aryen}
\author{Guillaume Jacques}
\maketitle

La Verschärfung du yod en tibétain après \ipa{r-}, une loi phonétique découverte par \citet{lifk59globa} sur la base de données comparatives, est attestée dans quelques emprunts au sanskrit (\citealt{hill11laws}), une observation d'interprétation difficile.

Dans ce travail, nous discuterons tout d'abord des preuves comparatives de la Verschärfung, en particulier à partir des données des langues rgyalronguiques et kuki-chin (\citealt{jacques13yod}), mais aussi dialectales (\citealt{wangsc12anduo, gong16amdo}), accompagnées de réflexions sur la typologie des changements phonétiques (\citealt{gong16ld}).

Dans une seconde partie, nous présenterons une typologie des emprunts indo-aryens en tibétain (\citealt{laufer16loanwords, bielmeier07wortschatz}), qui manifestent pour certain d'entre eux des caractéristiques inhabituelles (par exemple la troncation de syllabes pour les forcer dans un moule dissyllabique, comme dans le nom de la constellation \tibet{ནབས་སོ་}{nabs.so}{punarvasu-},  \citealt{jacques07naksatram}).

Enfin, nous discuterons des cas connus de mots sanskrit ou prakrit ayant subi la Verschärfung (comme \tibet{བེ་དུ་རྒྱ་}{be.du.rgya}{vaiḍūrya-}, cf \citealt[36-37]{snellgrove67gzibrjid}) et du problème plus général de l'utilisation des emprunts pour la datation des changements phonétiques.

\bibliographystyle{unified}
\bibliography{bibliogj}
\end{document}
