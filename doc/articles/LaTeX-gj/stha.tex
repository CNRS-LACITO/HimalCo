\documentclass[oneside,a4paper,11pt]{article} 
\usepackage{fontspec}
\usepackage{natbib}
\usepackage{booktabs}
\usepackage{xltxtra} 
\usepackage{polyglossia} 
\usepackage[table]{xcolor}
\usepackage{gb4e} 
\usepackage{multicol}
\usepackage{graphicx}
\usepackage{float}
\usepackage{hyperref} 
\hypersetup{bookmarksnumbered,bookmarksopenlevel=5,bookmarksdepth=5,colorlinks=true,linkcolor=blue,citecolor=blue}
\usepackage[all]{hypcap}
\usepackage{memhfixc}
\usepackage{lscape}

\setmainfont[Mapping=tex-text,Numbers=OldStyle,Ligatures=Common]{Charis SIL} 
\newfontfamily\phon[Mapping=tex-text,Ligatures=Common,Scale=MatchLowercase]{Charis SIL} 
\newcommand{\ipa}[1]{{\phon\mbox{\textbf{#1}}}}
\newcommand{\ipab}[1]{{\scriptsize \phon#1}} 

\newcommand{\grise}[1]{\cellcolor{lightgray}\textbf{#1}}
\newfontfamily\cn[Mapping=tex-text,Ligatures=Common,Scale=MatchUppercase]{SimSun}%pour le chinois
\newcommand{\zh}[1]{{\cn #1}}
\newcommand{\refb}[1]{(\ref{#1})}
\newcommand{\factual}[1]{\textsc{:fact}}
\newcommand{\rdp}{\textasciitilde{}}

\XeTeXlinebreaklocale 'zh' %使用中文换行
\XeTeXlinebreakskip = 0pt plus 1pt %
 %CIRCG
 \newcommand{\bleu}[1]{{\color{blue}#1}}
\newcommand{\rouge}[1]{{\color{red}#1}} 
\newcommand{\ch}[3]{\zh{#1} \ipa{#2} `#3'} 
\newcommand{\change}[2]{*\ipa{#1} $\rightarrow$ \ipa{#2}}
\newcommand{\deux}[1]{/#1/}
\newcommand{\trois}[1]{/#1/}

\newcommand{\tib}[1]{\cellcolor{lightgray}\textbf{#1}}
\newcommand{\idph}[1]{\cellcolor{gray}\textbf{#1}}



\begin{document} 
\title{Les désinences verbales sourdes aspirées de l'indo-iranien}
\author{Guillaume Jacques}
\maketitle
 
\sloppy
\section{Le problème}

L'indo-iranien se démarque de toutes les autres langues indo-européennes par la présence d'une série de sourdes aspirées (devenues fricatives en iranien), dont il est admis qu'elles sont d'origine secondaire, provenant en particulier des groupes occlusives sourde + laryngales de l'indo-européen (voir par exemple \citet[112-4]{mayrhofer05fortsetzung}).

Ces aspirées apparaissent dans de nombreuses désinences verbales. Dans le cas de la \textsc{2sg} du parfait, la reconstruction *\ipa{-th_2e} proposée par \citet{kurylowicz1927schwa} (en notation modernisée) est universellement admise. En revanche, certaines terminaisons du présent actif et médio-passif, présentent des aspirées inattendues au regard des autres langues indo-européennes.

Les tableaux \ref{tab:actif} et \ref{tab:medio} indiquent en grisé les formes qui nous intéressent: la \textsc{2pl} de l'actif \ipa{-tha}, et les \textsc{2du} et \textsc{3du} de l'actif (\ipa{-thas}) et du médio-passif  (\ipa{-(ā|e)the}), qui ont des aspirées à des formes où, contrairement au cas de la \textsc{2sg} du parfait, aucune trace de laryngales ne se laisse déceler ailleurs en indo-européen de façon certaine (\citealt[309-311]{burrow55skt})
Pour la seconde pluriel \ipa{-tha}, aucune trace, métrique ou segmentale, de laryngale n'a été rapportée dans les langues autres que l'indo-iranien. 

Pour celles du duel, la situation est plus compliquée en raison de la mauvaise préservation de ces formes dans les langues de la famille. Les formes du grec, quoiqu'indéniablement apparentées à celle de l'indo-iranien, sont peu informatives sur la présence ou non d'une laryngale. Certains auteurs ont proposé que le germanique aurait *\ipa{t} correspondant au sanskrit \ipa{th} au lieu de *\ipa{θ} comme on l'attendrait d'après la loi de Grimm (dossier cité dans \citealt[113]{mayrhofer05fortsetzung}), sur la base en particulier de la désinence de \textsc{2sg} du parfait \ipa{-t} (sanskrit \ipa{-tha} $\rightarrow$ *\ipa{th_2e}) et de celle de \textsc{2du} du présent  \ipa{-ts} en gothique (sanskrit \ipa{-thas} $\rightarrow$ *\ipa{th_2es}?). Cette solution n'est toutefois pas retenue de nos jours, et les formes germaniques peuvent s'expliquer de façon plus satisfaisante par extension analogique de l'allomorphe qui suivait les radicaux verbaux en dentale (\citealt[84]{hill03zusammenstoss}, \citealt[192]{ringe06PIE}). La présence ou non de laryngales ou d'une aspiration ancienne dans ces formes est donc incertaine.

Seuls le sanskrit et l'avestique seront mentionnés dans ce travail, car les formes en question ne sont pas attestées en vieux perse, et mal conservées ailleurs en indo-iranien.

On remarque immédiatement une différence importante entre sanskrit et avestique: alors que l'opposition d'aspiration distingue la deuxième de la troisième personne du duel en sanskrit,  on trouve en avestique des formes provenant d'aspirées (reflétées par la fricative \ipa{θ} $\leftarrow$ *\ipa{th}) et de sourde simples pour les troisièmes personnes du duel. \citet{martinez14avestan}, considérant la distribution du sanskrit comme originelle, interprètent les formes en \ipa{θ} de troisième duel en avestique comme une extension analogique de la deuxième duel. Toutefois, l'analogie se propage normalement de la troisième personne aux autres formes, sauf cas exceptionnels de verbes dont la première personne singulier est plus courante que la troisième personne (\citealt{jacques16ebde}), et l'hypothèse de Martínez et de Vaan est donc peu probable.

\begin{table}[H]
\caption{Paradigmes du présent actif en indo-iranien} \label{tab:actif}
\begin{tabular}{lllllll}
\toprule
 & 	Sanskrit  & 	Sanskrit & 	Avestique & 	Grec & 	\\
 &thématique&athématique&&\\
\midrule
1sg & 	\ipa{-āmi} & 	\ipa{-mi} & 	\ipa{-ā} & 	\ipa{-mi} & 	\\
2sg & 	\ipa{-asi} & 	\ipa{-si} & 	\ipa{-hī, šī} & 	\ipa{-si} & 	\\
3sg & 	\ipa{-ati} & 	\ipa{-ti} & 	\ipa{-tī} & 	\ipa{-ti} & 	\\
1du & 	\ipa{-āvas} & 	\ipa{-vas} & 	\ipa{-uuahī} & 	x & 	\\
2du & 	\ipa{-athas} \grise{}& 	\ipa{-thas} \grise{}& 	\ipa{x} \grise{}& 	\ipa{-ton} & 	\\
3du & 	\ipa{-atas} \grise{}& 	\ipa{-tas} \grise{}& 	\ipa{-tō, -θō} \grise{}& 	\ipa{-ton} & 	\\
1pl & 	\ipa{-āmas} & 	\ipa{-mas} & 	\ipa{-mahī} & 	\ipa{-me(n|s)} & 	\\
2pl & 	\ipa{-atha}\grise{} & 	\ipa{-tha} \grise{}& 	\ipa{-θā} \grise{}& 	\ipa{-tes} & 	\\
3pl & 	\ipa{-anti} & 	\ipa{-anti} & 	\ipa{-ṇtī, -aiṇti} & 	\ipa{-ousi} & 	\\
\bottomrule
\end{tabular}
\end{table}

\begin{table}[H]
\caption{Paradigmes du présent médio-passif  en indo-iranien}  \label{tab:medio}
\begin{tabular}{lllllll}
\toprule
 & 	Sanskrit  & 	Sanskrit & 	Avestique & 	Grec & 	\\
 &thématique&athématique&&\\
 \midrule
1sg & 	\ipa{-e} & 	\ipa{-̣e} & 	\ipa{-ē, -ōi} & 	\ipa{-mai} & 	\\
2sg & 	\ipa{-ase} & 	\ipa{-se} & 	\ipa{-hē, -ŋ́hē, -šē} & 	\ipa{-sai} & 	\\
3sg & 	\ipa{-ate} & 	\ipa{-te} & 	\ipa{-tē, -ē} & 	\ipa{-tai} & 	\\
1du & 	\ipa{-āvahe} & 	\ipa{-vahe} & 	\ipa{x} & 	x& 	\\
2du & 	\ipa{-ethe} \grise{}& 	\ipa{-āthe} \grise{}& 	\ipa{x} \grise{}& 		\ipa{-sthon} & 	\\
3du & 	\ipa{-ete}\grise{} & 	\ipa{-āte}\grise{} & 	\ipa{-aētē, -ōiθe, -āitē} \grise{}& 	\ipa{-sthon} & 	\\
1pl & 	\ipa{-āmahe} & 	\ipa{-mahe} & 	\ipa{-maidē} & 	\ipa{-metha}& 	\\
2pl & 	\ipa{-adhve} & 	\ipa{-dhve} & 	\ipa{-duiiē} & 	\ipa{-sthe} & 	\\
3pl & 	\ipa{-ante} & 	\ipa{-ate} & 	\ipa{-n̄tē, -aitē} & 	\ipa{-ntai} & 	\\
\bottomrule
\end{tabular}
\end{table}



 
\section{Une solution possible}
La solution proposée ici pour rendre compte de l'aspiration dans les formes indo-iraniennes citées ci-dessus part de plusieurs constations:

\begin{itemize}
\item L'aspiration dans la désinence de \textsc{2pl} actif n'a pas d'équivalent ailleurs en indo-européen, et il est vraisemblable qu'il s'agisse d'une innovation indo-iranienne.
\item La distribution au duel entre \ipa{th} à la deuxième personne et \ipa{t} à la troisième est propre au sanskrit; l'avestique suggère plutôt que la troisième personne du duel avait des formes aspirées et non-aspirées selon une distribution encore mal comprise, et que la restriction à des personnes spécifiques observée en sanskrit est secondaire.
\end{itemize}

Ces observations suggèrent que les formes aspirées (\ipa{-tha}, \ipa{-thas}, \ipa{-(e)the}) pourraient être à l'origine des variantes contextuelles de formes héritées sans aspiration, généralisées analogiquement à l'ensemble du lexique (imparfaitement en avestique).

xx

Extension \ipa{sedúr} (\citealt[342-3]{burrow55skt}
\section{Implications pour la phonologie historique de l'indo-iranien}



\bibliographystyle{unified}
\bibliography{bibliogj}

\end{document}