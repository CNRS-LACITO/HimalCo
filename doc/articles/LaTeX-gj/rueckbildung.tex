\documentclass[oldfontcommands,oneside,a4paper,11pt]{article} 
\usepackage{fontspec}
\usepackage{natbib}
\usepackage{booktabs}
\usepackage{xltxtra} 
\usepackage{polyglossia} 
\usepackage[table]{xcolor}
\usepackage{gb4e} 
\usepackage{multicol}
\usepackage{graphicx}
\usepackage{float}
\usepackage{hyperref} 
\hypersetup{bookmarks=false,bookmarksnumbered,bookmarksopenlevel=5,bookmarksdepth=5,xetex,colorlinks=true,linkcolor=blue,citecolor=blue}
\usepackage[all]{hypcap}
\usepackage{memhfixc}
\usepackage{lscape}
\bibpunct[: ]{(}{)}{,}{a}{}{,}
 
%\setmainfont[Mapping=tex-text,Numbers=OldStyle,Ligatures=Common]{Charis SIL} 
\newfontfamily\phon[Mapping=tex-text,Ligatures=Common,Scale=MatchLowercase,FakeSlant=0.3]{Charis SIL} 
\newcommand{\ipa}[1]{{\phon \mbox{#1}}} %API tjs en italique
 \newcommand{\ipab}[1]{{\phon \mbox{#1}}} %API tjs en italique
\newcommand{\grise}[1]{\cellcolor{lightgray}\textbf{#1}}
\newfontfamily\cn[Mapping=tex-text,Ligatures=Common,Scale=MatchUppercase]{MingLiU}%pour le chinois
\newcommand{\zh}[1]{{\cn #1}}

 

 \begin{document} 
 \title{Rückbildung in Japhug}
\author{Guillaume Jacques}
\maketitle
\sloppy


\section{Introduction}

\section{Denominal derivation}
\citet{jacques12incorp}
\citet{jacques14antipassive}

\section{Prefixal nominalization vs bare infinitive}
\citet{jacques14antipassive}
\begin{exe}
\ex \label{ex:bare.inf}
\gll \ipa{nɤʑo} 	\ipa{kɯ-fse} 	\ipa{a-ŋkhor} 	\ipa{nɯ} 	\ipa{ɯ-mto} 	\ipa{mɯ-pɯ-rɲo-t-a} \\
you \textsc{nmlz:S/A}-be.like \textsc{1sg.poss}-subject \textsc{dem} \textsc{3sg}-\textsc{bare.inf:}see \textsc{neg-pfv}-experience-\textsc{pst:tr-1sg} \\
\glt I never saw anyone like you among my subjects (Smanmi metog koshana1.157)
\end{exe}

\begin{exe}
\ex \label{ex:bare.inf.noun}
\gll \ipa{ndʑi-mi}   	\ipa{ɯ-tshoʁ}   	\ipa{ɯ-tshɯɣa}   	\ipa{nɯra}   	\ipa{wuma}   	\ipa{ʑo}   	\ipa{naχtɕɯɣ-ndʑi.}   \\
\textsc{3du.poss}-foot \textsc{3sg}-\textsc{bare.inf:}attach.to \textsc{3sg.poss}-form \textsc{dem:pl} very \textsc{emph}   be.similar:\textsc{fact}-\textsc{du}  \\
\glt The way their feet (of fleas and crickets) touch the ground is very similar (the cricket 17)
\end{exe}




\section{Inverse derivation}
\citet{garnier15inverse}

\begin{table}[H]
\caption{Examples of Inverse derivation} \centering \label{tab:rueckbildung}
\begin{tabular}{lllllll}
\toprule
Verb & Meaning &&Noun & Meaning\\
\midrule
\ipa{fkaβ} & `cover' & $\Rightarrow$ &\ipa{(tɤ)-fkaβ} &`cover, lid' \\
\ipa{ɕphɤt} & `patch'& $\Rightarrow$ &\ipa{(tɤ)-ɕphɤt} & `a patch'  \\
\ipa{sɯso} & think &  $\Rightarrow$ & \ipa{(tɯ)-sɯso} & thought \\
\midrule
\ipa{ndzom} & `freeze and become   &$\Leftarrow$ &\ipa{ndzom} &`bridge' \\
&crossable afoot (of a river)'&&&\\
\ipa{jpɣom} & `freeze' & $\Leftarrow$ &\ipa{tɤ-jpɣom}& `ice' \\
\ipa{mkɯm} & `have one's head turned& $\Leftarrow$ &\ipa{(tɤ)-mkɯm}& `pillow' \\
&   towards X while sleeping' &&&\\
\ipa{sɯjno} & `to weed'   &$\Leftarrow$ &\ipa{sɯjno} &`grass' \\
\ipa{znde} & `repair a wall'  &$\Leftarrow$ &\ipa{znde} &`wall' \\
\bottomrule
\end{tabular}
\end{table}

\ipa{nɤmkɯm} `use as a pillow'


\begin{exe}
\ex
\gll 
\ipa{nɯ}  	\ipa{maʁ}  	\ipa{nɤ}  \ipa{tɕe}   \ipa{soz}  	\ipa{tɤ-tɕɯ}  	\ipa{nɯ}  	\ipa{rɤru}  	\ipa{tɤkha}  	\ipa{tɕe}   	\ipa{tɕheme}  	\ipa{nɯ}  	\ipa{ɣɯ}  	\ipa{ɯ-jme}  	\ipa{pɕoʁ}  	\ipa{nɯ}  	\ipa{tɕu}  	\ipa{ntsɯ}  	\ipa{chɯ-mkɯm}  	\ipa{pjɤ-ŋu}   \\
\textsc{dem} not.be:\textsc{fact} \textsc{lnk} \textsc{lnk} morning \textsc{indef.poss}-boy \textsc{dem}  get.up:\textsc{fact} the.moment \textsc{lnk} girl \textsc{dem} \textsc{gen} \textsc{3sg.poss}-tail side \textsc{dem} \textsc{loc} \textsc{always} \textsc{ipfv:downstream}-have.one's.head.turned.towards \textsc{ipfv.ifr}-be \\
\glt In the morning, the husband would wake up with his head turned on the opposite side (towards the girls' feet). (2002rkongrjal2, 14)
\end{exe}

\begin{exe}
\ex
\gll \ipa{iʑora}  	\ipa{thamtham}  	\ipa{tɕe}  	\ipa{ku-nɯ-sɯ-jpɣom-i}  	\ipa{tɕe}  	\ipa{tɕendɤre,}  
<bingxiang>  	<binggui>  	\ipa{ɯ-ŋgɯ}  	\ipa{ri,}  	<baoxiandai>  	\ipa{ɯ-ŋgɯ}  	\ipa{chɯ́-wɣ-rku}  	\ipa{qhe} \ipa{tɕe}  	\ipa{nɯ}  	\ipa{ɲɯ́-wɣ-nɯ-sɯ-jpɣom}  	\ipa{qhe}  	\ipa{qartsɯ}  	\ipa{qhe}  	\ipa{tú-wɣ-ndza}  	\ipa{qhe,}  \ipa{nɯnɯ}  	\ipa{kɯ-ɕɤɣ}  	\ipa{ʑo}  	\ipa{ɲɯ-fse,}  	\ipa{ɲɯ-mɯm.}   \\
\textsc{1pl} now \textsc{lnk} \textsc{prs:ego-auto-caus}-freeze-\textsc{1pl} \textsc{lnk} \textsc{lnk} freezer freezer \textsc{3sg}-inside plastic.bag \textsc{3sg}-inside \textsc{ipfv-inv}-put.in \textsc{loc} \textsc{lnk} \textsc{dem} \textsc{ipfv-inv-auto-caus}-freeze \textsc{lnk} winter \textsc{lnk} \textsc{ipfv-inv}-eat \textsc{lnk} \textsc{dem} \textsc{nmlz}:S/A-be.new \textsc{emph} \textsc{sens}-be.like \textsc{sens}-be.tasty \\
\glt Now, we freeze it (mushrooms) and put it in the freezer, in a plastic bag. When we freeze it, even in winter when we eat it, it is like fresh, it is tasty. (hist-20-grWBgrWB, 63-67)
\end{exe} 


For the verbal use of \ipa{ndzom}, no examples are attestedin the corpus, but our main consultant tshendzin provided the following definition:

\begin{exe}
\ex
\gll \ipa{tɯ-ci}  	\ipa{ɯ-taʁ}  	\ipa{nɯ} \ipa{tɕu}  	\ipa{lonba}  	\ipa{ʑo}  	\ipa{a-kɤ-jpɣom}  	\ipa{ʑo}  	\ipa{tɕe,}  	\ipa{tɯ-ci}  	\ipa{ko-ndzom}  	\ipa{tu-kɯ-ti}  	\ipa{ŋgrɤl}  	\\
\textsc{indef.poss}-water \textsc{3sg}-on \textsc{dem} \textsc{loc} all \textsc{emph} \textsc{irr-pfv}-freeze \textsc{emph} \textsc{lnk}  \textsc{indef.poss}-water \textsc{ifr}-freeze.on.the.surface \textsc{ipfv-genr}-say be.usually.the.case:\textsc{fact} \\
\glt Whenever it freezes completely on the surface of a body of water, we say `the water froze on the surface'.
\end{exe} 

All three nouns `bridge',  `pillow' and `ice' are widely attested in the family. The first two are included in (\citealt[257, 272]{matisoff03}), and `pillow' has a cognate in Chinese (in \citealt{bs14oc}'s reconstruction, \zh{枕} \ipa{tɕim^x} < *\ipa{[t.k][ə]mʔ}). The noun `ice' does not appear to be widespread, but has at least a cognate in Chinese (\zh{冰} < \ipa{piŋ} < *\ipa{prəŋ} <  **\ipa{prəm} with labial dissimilation).

The cognates of these three words are primarily nouns, never as verbs. The only exception known to me is the verbal use of \zh{枕} \ipa{zhěn} `to use as a pillow' in Mandarin Chinese (example \ref{ex:zhenzhe}).


\begin{exe}
\ex \label{ex:zhenzhe}
\glt \zh{枕着胳膊睡觉}
\gll \ipa{zhěn-zhe} \ipa{gēbo} \ipa{shuìjiao}\\
use.as.pillow-\textsc{prog} arm sleep \\
\glt To sleep using one's arm as a pillow.
\end{exe} 

Given the widespread productive zero-derivation in Mandarin and in former stages of Chinese, there is however no need to suppose 


%\citet{jacques14esquisse}




\section{Conclusion}

\bibliographystyle{unified}
\bibliography{bibliogj}

 \end{document}
 