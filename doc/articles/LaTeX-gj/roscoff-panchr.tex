\documentclass[oldfontcommands,oneside,a4paper,11pt]{article} 
\usepackage{fontspec}
\usepackage{natbib}
\usepackage{booktabs}
\usepackage{xltxtra} 
\usepackage{longtable}
\usepackage{polyglossia} 
\usepackage[table]{xcolor}
\usepackage{gb4e} 
\usepackage{multicol}
\usepackage{graphicx}
\usepackage{float}
\usepackage{hyperref} 
\hypersetup{bookmarks=false,bookmarksnumbered,bookmarksopenlevel=5,bookmarksdepth=5,xetex,colorlinks=true,linkcolor=blue,citecolor=blue}
\usepackage[all]{hypcap}
\usepackage{memhfixc}
\usepackage{lscape}
 \usepackage{lineno}
\bibpunct[: ]{(}{)}{,}{a}{}{,}

\setmainfont[Mapping=tex-text,Numbers=OldStyle,Ligatures=Common]{Charis SIL} 
\newfontfamily\phon[Mapping=tex-text,Ligatures=Common,Scale=MatchLowercase,FakeSlant=0.3]{Charis SIL} 
\newcommand{\ipa}[1]{{\phon \mbox{#1}}} %API tjs en italique
\newcommand{\ipab}[1]{{\scriptsize \phon#1}} 

\newcommand{\grise}[1]{\cellcolor{lightgray}\textbf{#1}}
\newfontfamily\cn[Mapping=tex-text,Ligatures=Common,Scale=MatchUppercase]{MingLiU}%pour le chinois
\newcommand{\zh}[1]{{\cn #1}}

\newcommand{\sg}{\textsc{sg}}
\newcommand{\pl}{\textsc{pl}}
\newcommand{\ro}{$\Sigma$}
\newcommand{\ra}{$\Sigma_1$} 
\newcommand{\rc}{$\Sigma_3$}  


\XeTeXlinebreakskip = 0pt plus 1pt %
 %CIRCG
 


\begin{document} 
\linenumbers
\title{Panchronic phonology}
\author{Guillaume Jacques}
\maketitle
\section{Aspirated fricatives}
\citet{jacques11lingua}

\citet{carveth13aspirated}
\citet{carveth13aspirated}

\begin{table}[H] \centering
\begin{tabular}{llllllllllllllll}
\toprule
Aspirated fricatives & language \\
\midrule
sʰ ʃʰ & Chumashan \\
sʰ fʰ & Ofo (Siouan) \\
sʰ & Arawan languages  \\
sʰ ɕʰ çʰ & Phosul Horpa (ST) \\
sʰ ɕʰ ʂʰ & Shuiluo Pumi (ST) \\
sʰ ɕʰ ʂʰ xʰ & Cone Tibetan(ST)  \\
fʰ sʰ ɕʰ ʂʰ   & Heqing Bai (ST) \\
fʰ sʰ _ɬʰ ɬʲʰ ɕʰ   & Yanghao (Hmong-mien) \\
\bottomrule
\end{tabular}

Hierarchies:

\begin{exe}
\ex 
\glt sʰ > {ɕʰ, ʂʰ, ʃʰ} > {çʰ, xʰ, ɬʰ}
\glt sʰ > fʰ
\end{exe}
\end{table}

\citet{dixon04arawa}
\citet{dienst05kulina}

\subsection{Attested pathways}

\subsubsection{Tibetan}
\begin{exe}
\ex 
\glt *s $\rightarrow$ sʰ
\glt *ks/*ps $\rightarrow$ s
\glt *z $\rightarrow$ s
\glt *gz/*bz $\rightarrow$ z
\end{exe}


\bibliographystyle{linquiry2}
\bibliography{bibliogj}
\end{document}