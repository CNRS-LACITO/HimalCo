\documentclass[oldfontcommands,oneside,a4paper,11pt]{article} 
\usepackage{fontspec}
\usepackage{natbib}
\usepackage{booktabs}
\usepackage{xltxtra} 
\usepackage{longtable}
\usepackage{polyglossia} 
\usepackage[table]{xcolor}
\usepackage{gb4e} 
\usepackage{multicol}
\usepackage{graphicx}
\usepackage{float}
\usepackage{hyperref} 
\hypersetup{bookmarks=false,bookmarksnumbered,bookmarksopenlevel=5,bookmarksdepth=5,xetex,colorlinks=true,linkcolor=blue,citecolor=blue}
\usepackage[all]{hypcap}
\usepackage{memhfixc}
\usepackage{lscape}
 \usepackage{lineno}
\bibpunct[: ]{(}{)}{,}{a}{}{,}

\setmainfont[Mapping=tex-text,Numbers=OldStyle,Ligatures=Common]{Charis SIL} 
\newfontfamily\phon[Mapping=tex-text,Ligatures=Common,Scale=MatchLowercase,FakeSlant=0.3]{Charis SIL} 
\newcommand{\ipa}[1]{{\phon \mbox{#1}}} %API tjs en italique
\newcommand{\ipab}[1]{{\scriptsize \phon#1}} 

\newcommand{\grise}[1]{\cellcolor{lightgray}\textbf{#1}}
\newfontfamily\cn[Mapping=tex-text,Ligatures=Common,Scale=MatchUppercase]{MingLiU}%pour le chinois
\newcommand{\zh}[1]{{\cn #1}}

\newcommand{\sg}{\textsc{sg}}
\newcommand{\pl}{\textsc{pl}}
\newcommand{\ro}{$\Sigma$}
\newcommand{\ra}{$\Sigma_1$} 
\newcommand{\rc}{$\Sigma_3$}  


\XeTeXlinebreakskip = 0pt plus 1pt %
 %CIRCG
 


\begin{document} 
\linenumbers
\title{Panchronic phonology}
\author{Guillaume Jacques}
\maketitle
\section{Introduction}
Unidirectionality of sound changes? 

For instance, both s $\rightarrow$ t and t $\rightarrow$ s are attested, but not in the same contexts:

\begin{enumerate}

\item s $\rightarrow$ t (Vietnamese, syllable-initial, part of a chain shift, see \citealt{ferlus82spirantisation})
\item t $\rightarrow$ s (assibilation, followed by a high vowel)

\end{enumerate}



\section{Aspirated fricatives}
\citet{jacques11lingua}

\subsection{Attested inventories}
 
Excluding Korean, Oto-Manguean languages, !Xũ
\begin{table}[H] \centering
\begin{tabular}{llllllllllllllll}
\toprule
Aspirated fricatives & language & Reference\\
\midrule
sʰ ʃʰ & Chumashan &\citet{klar77chumashan}\\
fʰ sʰ& Ofo (Siouan)& \citet{reuse81ofo} \\
sʰ & Arawan languages  &\citet{dixon04arawa},
\citet{dienst05kulina}\\
sʰ ɕʰ çʰ & Phosul Horpa (ST)&\citet{jackson00puxi} \\
sʰ ɕʰ ʂʰ & Shuiluo Pumi (ST)&\citet{jacques11lingua} \\
sʰ ɕʰ ʂʰ xʰ & Cone Tibetan(ST) &\citet{jacques11lingua} \\
fʰ sʰ ɕʰ ʂʰ   & Heqing Bai (ST) &\citet{wang06songqi} \\
fʰ sʰ ɬʰ ɬʲʰ ɕʰ   & Yanghao (Hmong-mien) &\citet{ratliff10protohm}\\
\bottomrule
\end{tabular}

Hierarchies:

\begin{exe}
\ex 
\glt sʰ > {ɕʰ, ʂʰ, ʃʰ} > {çʰ, xʰ, ɬʰ}
\glt sʰ > fʰ
\end{exe}
\end{table}



\subsection{Attested pathways}
 

\subsubsection{Unvoiced  fricative $\rightarrow$ aspirated fricative 1}

\citet{sun86ndzorge} and \citet{jacques14cone} 

\begin{table}[H] \centering
\begin{tabular}{lllllll}
\toprule
Old Tibetan& Ndzorge& Chone & Meaning \\
\midrule

so & \ipa{sʰo}& \ipa{sʰɔ́} &tooth \\
zo & \ipa{so}& \ipa{sɔ̀} &eat \textsc{imp} \\
ksum & \ipa{sɤm}& \ipa{sṍ} &three \\
bzuŋ & \ipa{zoŋ}& \ipa{zʉ̀ː} &take \textsc{pfv}\\
\bottomrule
\end{tabular}
\end{table}
\begin{exe}
\ex 
\glt *s $\rightarrow$ sʰ
\glt *ks/*ps $\rightarrow$ s
\glt *z $\rightarrow$ s
\glt *gz/*bz $\rightarrow$ z
\end{exe}

\subsubsection{Unvoiced  fricative $\rightarrow$ aspirated fricative 2}
Shan?
\begin{exe}
\ex 
\glt *s $\rightarrow$ sʰ
\glt *tɕ $\rightarrow$ s
\end{exe}

\subsubsection{Aspirated affricate $\rightarrow$ aspirated fricative}
\begin{exe}
\ex 
\glt *s $\rightarrow$ tθ
\glt *ts $\rightarrow$ s
\glt *tsʰ $\rightarrow$ sʰ
\glt *kʰr/kʰʲ $\rightarrow$ tɕʰ
\end{exe}
\citet{wang79miaoyu}

Also found in Arawan languages (\citealt{dixon04arawa}, \citealt{dienst05kulina}) and in Hmong-Mien (\citealt{wang79miaoyu}, \citealt{ratliff10protohm}, \citealt{carveth13aspirated}):
\begin{exe}
\ex 
\glt *nts $\rightarrow$ s
\glt *ntsʰ, *tsʰ $\rightarrow$ sʰ
\glt *ntʂ $\rightarrow$ ɕ
\glt *ntʂʰ $\rightarrow$ ɕʰ
\end{exe}


\subsubsection{Aspirated stop + glide $\rightarrow$ aspirated fricative}
Chone Tibetan
\begin{exe}
\ex 
\glt *pʰʲ $\rightarrow$ *pɕʰ $\rightarrow$ ɕʰ
\glt *bʲ $\rightarrow$ ɕ  
\end{exe}

\subsubsection{s+aspirated stop/affricate $\rightarrow$ aspirated fricative}
Shuiluo Pumi

\begin{table}[H]
\centering
\begin{tabular}{lllll}
\toprule
gloss & 	Shuiluo & 	Mudiqing & 	Lanping \\ 	
\midrule
hide & 	\ipa{ʂû} & 	\ipa{tʂǔ} & 	\ipa{thə-̀stʃú} \\ 	
nest & 	\ipa{ʂuâ} & 	\ipa{tʂuɔ̂} & 	\ipa{stʃuá} \\ 	
jump & 	\ipa{ʂə̂} & 	\ipa{tɕə̂} & 	\ipa{tə-́stʃə́} \\ 	
stab & 	\ipa{ʂʰuɛi} & 	\ipa{tʂʰuɛ̌} & 	\ipa{xə̀-stʃʰà} \\ 	
saliva & 	\ipa{ʐǎ} & 	\ipa{dʐɐ̌} & 	\ipa{sdʒà} \\ 	
nail & 	\ipa{ʐɛ̃̌} & 	\ipa{dʐɛ̃̌} & 	\ipa{sdʒã̀} \\ 	
leak & 	\ipa{ʐə̌} & 	\ipa{dʑə̌} & 	\ipa{kʰə-̀sdʒə̀} \\ 	
chop & 	\ipa{ɕɛ̂} & 	\ipa{tɕɐ̂} & 	\ipa{tʰə̀-stʃɑ́} \\ 	
saddle & 	\ipa{ɕî} & 	\ipa{stʃɛ́tʂʰṍ} & 	\ipa{} \\ 	
twist & 	\ipa{ɕúwá} & 	\ipa{tɕú} & 	\ipa{nə̀-stʃɯú} \\ 	
feed & 	\ipa{ɕʰɛ̌} & 	\ipa{tɕʰǐ} & 	\ipa{thə̀-stʃhɛ́} \\ 	
scoop & 	\ipa{ɕʰã̌} & 	\ipa{tʂʰã̌} & 	\ipa{} \\ 	
pine & 	\ipa{ɕʰĩbṍ} & 	\ipa{tɕʰĩsẽ́} & 	\ipa{stʃʰɛ̃̀sbṍ} \\ 	
stand & 	\ipa{ɕʰĩ̌} & 	\ipa{tɕʰə̂} & 	\ipa{nə-stʃə́} \\ 	
swallow & 	\ipa{ʑĩ̂} & 	\ipa{diẽ̂} & 	\ipa{kʰə-̀sdʒɛ̃́} \\ 	
hail & 	\ipa{ʑĩ̌} & 	\ipa{dʑẽ̂} & 	\ipa{sdʒɛ̃́} \\ 	
trousers & 	\ipa{ʑə̌} & 	\ipa{dʑə̌} & 	\ipa{sdʒə́} \\ 	
light & 	\ipa{ʑĩ̌} & 	\ipa{dʑẽ̌} & 	\ipa{sdʒɛ̃́} \\ 	
beard & 	\ipa{asṍ} & 	\ipa{atió} & 	\ipa{àstiãú} \\ 	
pulse & 	\ipa{siɛ́} & 	\ipa{sèi} & 	\ipa{stié} \\ 	
pillar & 	\ipa{sɛ̃́} & 	\ipa{tɛ̃́} & 	\ipa{stã́} \\ 	
choose & 	\ipa{sʰɛ́} & 	\ipa{tʰí} & 	\ipa{tʰə̀-stʰié} \\ 	
deaf & 	\ipa{zabõ̌} & 	\ipa{dəbã̌} & 	\ipa{sdəbõ̀} \\ 	
\bottomrule
\end{tabular}
\end{table}
\begin{exe}
\ex 
\glt *stʂ $\rightarrow$ ʂ
\glt *stɕ  $\rightarrow$ ɕ
\glt *stɕʰ $\rightarrow$ ɕʰ
\glt *stʰ $\rightarrow$ sʰ
\glt *s $\rightarrow$ s
\glt *ɕ $\rightarrow$ ɕ
\end{exe}

But loss of aspiration contrast with dorsal fricatives:

\begin{exe}
\ex 
\glt *sk $\rightarrow$ x
\glt *skʰ $\rightarrow$ x
\end{exe}
\subsubsection{Aspirated sonorant $\rightarrow$ aspirated fricative}
Chone Tibetan
\begin{exe}
\ex 
\glt *rh $\rightarrow$ ʂʰ
\glt *sr/spr $\rightarrow$ ʂ
\end{exe}
\subsubsection{Oral fricative + h $\rightarrow$ aspirated fricative}

Chumashan (synchronic rule)

\begin{exe}
\ex 
\glt *s+h $\rightarrow$ sʰ
\glt *s+h $\rightarrow$ ʃʰ
\end{exe}
\subsubsection{Dissimilation}
Chumashan (synchronic rule)
\begin{exe}
\ex 
\glt *s+s $\rightarrow$ sʰ
\glt *s+ʃ/*ʃ+s $\rightarrow$ ʃʰ
\end{exe}

\subsubsection{Stress-related aspiration}
Ofo, Rgyalrongic languages (Horpa, Khroskyabs)

\subsection{Generalizations}
 

There are examples of ongoing loss of aspiration on fricatives (standard Burmese), or loss of aspiration at an unattested stage (dorsal fricatives in Shuiluo Pumi for instance), but no case where an aspiration contrast in fricatives is transphonologized into something else.

Non-attested but conceivable changes:
\begin{exe}
\ex 
\glt simple aspirated stop $\rightarrow$ aspirated fricative (1)
\glt *pʰ $\rightarrow$ fʰ
\glt *p $\rightarrow$ f
\glt *b $\rightarrow$ v
\end{exe}

\begin{exe}
\ex 
\glt simple aspirated stop $\rightarrow$ aspirated fricative (2)
\glt *p $\rightarrow$ fʰ
\glt *b $\rightarrow$ f
\end{exe}

\subsection{Perspectives for synchronic phonology}
\begin{exe}
\ex 
\glt The rarity of aspirated fricatives is not due to a dearth of potential origins; it is due to the difficulty of perceiving (and producing) the contrast and maintaining it.
\end{exe}
\begin{exe}
\ex 
\glt Rarity of aspirated fricatives in clusters (only attested in Horpa): several pathways leading to the creation of aspirated fricatives involve loss of clusters.
\end{exe}


\section{Obstruent debuccalization}
The clearest example of unidirectional sound change?
\subsection{Attested pathways}

  \subsubsection{Fricatives}
Two main pathways

Loss of oral articulation:
\begin{exe}
\ex
\glt voiceless fricative (s, x, f, θ) $\rightarrow$ h $\rightarrow \emptyset$ 
\end{exe}

Lenition to approximant:
\begin{exe}
\ex
\glt voiced fricative (v, z, ɣ) $\rightarrow$ ɰ (w, j)$\rightarrow \emptyset$ 
\end{exe}

Shibilants, a special case:

\begin{exe}
\ex
\glt ʂ ʃ ɕ $\rightarrow$  x $\rightarrow$ h $\rightarrow \emptyset$ 
\end{exe}


 \subsubsection{Stops}
 
 Three pathways:
 \begin{exe}
\ex
\glt stop $\rightarrow$  fricative $\rightarrow$ h $\rightarrow \emptyset$ 
\glt p $\rightarrow$  f/ɸ
\glt t $\rightarrow$  θ
\glt k $\rightarrow$  x
\end{exe}


Loss of oral articulation: (eg, final stops in ST languages, initial stops in Belhare and Yamphu \citealt{michailovsky10kiranti})
 \begin{exe}
\ex
\glt voiceless stop $\rightarrow$  ʔ $\rightarrow \emptyset$ 
\end{exe}

Lenition to approximant

\begin{exe}
\ex
\glt voiced stop  $\rightarrow$ voiced spirant (β, ð, ɣ) $\rightarrow$ ɰ (w, j)$\rightarrow \emptyset$ 
\end{exe} 

Not completely unidirectional (fortition):

\begin{exe}
\ex
\glt CjV $\rightarrow$ CgjV (North Germanic, Tibetan)
\glt β, ð, ɣ $\rightarrow$ b, d, g
\end{exe} 

  
 \subsubsection{Complete loss of a place of articulation}

 
  \begin{enumerate}
  \item  Word-initially, all consonants: Arandic languages (\citealt{koch04arandic})
\item In presyllables, all consonants:  Arem (Viet-Muong, \citealt{ferlus96monosyllabisme})
\item In codas, all consonants

\item In languages with only one series of stops, one of the stops
  \begin{enumerate}
\item   Arapaho (Algonquian):

 \begin{itemize}
\item \ipa{*k} $\rightarrow \emptyset$
\item \ipa{*p} $\rightarrow$ \ipa{*k} $\rightarrow$  \ipa{k/c}
\end{itemize}

 \begin{table}[H]
 \centering  \label{tab:k.zero}
\begin{tabular}{lllllll}
\toprule
Proto-algonquian & Arapaho & Ojibwe \\
\midrule
  \ipa{*kenw-aa} ``be long" & \textit{he-yóó}--    &  \textit{ginw-aa}-- \\
  \ipa{*nepyi} ``water" & \textit{néc}--    &  \textit{nibi}-- \\
 \ipa{*kespak-yaa--} ``be thick" & \textit{hooko-yóó}--    &  \textit{gipag-aa}-- \\
 \ipa{*waaposwa} ``rabbit" & \textit{nóóku}--    &  \textit{waabooz}-- \\
  \ipa{*čiipay-(a/i)} ``corpse" & \textit{θiik}--    &  \textit{jiibay}-- \\
  \ipa{*paaʔt-ee--} ``be smoky" & \textit{koot-éé}--    &  \textit{baat-e}-- ``be dry"\\
    \ipa{*pahkwee-n--} ``take out a piece" & \textit{koyei-n}--    &  \textit{bakwe-n}-- \\
        \ipa{*pa[n/r]ankw-} ``loose" & \textit{kono'-(óé)}--    &  \textit{banangw-(ad)}-- \\ 
   \ipa{*paʔs-askamikw-i} ``ravine" & \textit{koh'ówu'}--    &  \textit{basakamig(aa)}-- ``be a ravine"\\ 
\bottomrule
\end{tabular}
\end{table}

\item Oceanic languages (\citealt{blust04tk}):
 
 \begin{itemize}
\item \ipa{*k} $\rightarrow $\ipa{ʔ}
\item \ipa{*t} $\rightarrow$ \ipa{k}  
\end{itemize} 
 \end{enumerate}
 \end{enumerate}
 
 

 
  \subsection{Generalizations}
 \begin{enumerate}
\item Is there a hierarchy of debuccalizability depending on the place of articulation? %(V velar, L labial, C coronal)

\item  Rebuccalization, only attested as a contextual sound change:
 \begin{itemize}
\item hu > fu (Cantonese)
\item hi > ɕi (some Naxi dialects, see \citealt{michaud06neutralisation})
 \end{itemize}
`Verschärfung' of ʔ does not appear to be clearly attested
 \end{enumerate}
\bibliographystyle{linquiry2}
\bibliography{bibliogj}
\end{document}