\documentclass[oldfontcommands,oneside,a4paper,11pt]{article} 
\usepackage{fontspec}
\usepackage{natbib}
\usepackage{booktabs}
\usepackage{xltxtra} 
\usepackage{longtable}
\usepackage{polyglossia} 
%\usepackage[table]{xcolor}
\usepackage{gb4e} 
\usepackage{multicol}
\usepackage{graphicx}
\usepackage{float}
\usepackage{lineno}
\usepackage{textcomp}
\usepackage{hyperref} 
\hypersetup{bookmarks=false,bookmarksnumbered,bookmarksopenlevel=5,bookmarksdepth=5,xetex,colorlinks=true,linkcolor=blue,citecolor=blue}
\usepackage[all]{hypcap}
\usepackage{memhfixc}
\usepackage{lscape}
 

\setmainfont[Mapping=tex-text,Numbers=OldStyle,Ligatures=Common]{Charis SIL} \newfontfamily\phon[Mapping=tex-text,Ligatures=Common,Scale=MatchLowercase,FakeSlant=0.3]{Charis SIL} 
\newcommand{\ipa}[1]{{\phon #1}} %API tjs en italique
 
\newcommand{\grise}[1]{\cellcolor{lightgray}\textbf{#1}}
\newfontfamily\cn[Mapping=tex-text,Ligatures=Common,Scale=MatchUppercase]{MingLiU}%pour le chinois
\newcommand{\zh}[1]{{\cn #1}}


\XeTeXlinebreaklocale "zh" %使用中文换行
\XeTeXlinebreakskip = 0pt plus 1pt %
 %CIRCG
 


\begin{document} 

\title{Japhug Rgyalrong}
\author{Guillaume Jacques}
\maketitle
\linenumbers
 \section{Introduction}
 This paper focuses on the Japhug language (local name \ipa{kɯrɯ skɤt}) of Kamnyu village (\ipa{kɤmɲɯ}, Chinese \textit{Ganmuniao} \zh{干木鸟}) in Gdong-brgyad area (\ipa{ʁdɯrɟɤt}, Chinese  \textit{Longerjia} \zh{龙尔甲}), Mbarkhams county (Chinese \textit{Maerkang} \zh{马尔康}), Rngaba prefecture, Sichuan province, China.
 
 Japhug belongs to the Sino-Tibetan family, and is one of the four Rgyalrong languages, alongside Tshobdun, Zbu and Situ.\footnote{See  \citet{jackson00sidaba} for an overview of the Rgyalrong group. A grammar of Japhug is available (\citealt{jacques08}) as well as a series of articles on morphosyntax (see for instance  \citealt{jacques13harmonization} and
 \citealt{jacques14antipassive}) but little has been published specifically on phonology. } 
 
The description is based on the author’s fieldwork, and the word list and the short story in the appendix have been provided by Tshendzin (Chenzhen \zh{陈珍}, female, born 1950), a retired schoolteacher (bilingual in Japhug and Sichuan mandarin).

 Japhug has a very developed system of ideophones (\citealt{japhug14ideophones}), which present marked phonological features, in particular rare clusters. In the following discussion, phonemes or clusters limited to ideophones will be treated separately. In addition, about a quarter of the Japhug vocabulary is borrowed from Tibetan, and these loanwords, like the ideophones, fill some gaps in the phonotactic distribution of vowels and consonants, and these cases are clearly distinguished from the native vocabulary.

 
 \section{Consonants}
 In Japhug, syllables follow the template (C)(C)(C)V(C) with initial clusters containing at most three consonants, and at most one coda.
 
 \subsection{Simplet onsets}
 The consonant inventory of Japhug comprises 49 phonemes as listed in Table \ref{tab:consonants}. There is a general four-way contrast in stops and affricates between unvoiced unaspirated, unvoiced aspirated, voiced and prenasalized.
 
Since monosyllabic words are few,  most examples involve disyllabic words, whose first syllable illustrates the consonant in question followed by the vowel \ipa{ɯ}, the most common one. For some examples involving possessed nouns (on this topic see \citealt[6]{jacques14antipassive}), a possessive prefix must always be present, and we always give here the third person singular prefix  \ipa{ɯ--}.
 
 \begin{table}
 \caption{Consonantal phonemes in Japhug Rgyalrong} \label{tab:consonants}
 \resizebox{\columnwidth}{!}{
\begin{tabular}{llllllllll}
\toprule
& &Bilabial& Dental/Alveolar & Retroflex &Alveolo-palatal&Palatal & Velar&Uvular\\
\midrule
Stop&unvoiced & p & t & & & c & k & q\\
&aspirated & pʰ & tʰ & & & cʰ & kʰ & qʰ\\
&voiced & b & d & & & ɟ & g &  \\
&prenasalized& mb & nd & & & ɲɟ & ŋg & ɴɢ \\
Affricate&unvoiced &   & ts & tʂ& tɕ &  &   &  \\
&aspirated &  & tsʰ & tʂʰ& tɕʰ &  &   &  \\
&voiced &   & dz & dʐ& dʑ &  &   &  \\
&prenasalized&   & ndz & ndʐ & ndʑ &  &   &  \\
Nasal &&m& n   &&&ɲ &ŋ \\
Fricative &unvoiced&  & s & ʂ& ɕ &  &x   &χ  \\
&voiced&  & z & & ʑ &  &ɣ   &ʁ  \\
Approximant && w &&  &&j &\\
Rhotic &&&&r&&\\
Lateral &sonorant &&l\\
&  fricative &&ɬ\\
\bottomrule
\end{tabular}}
\end{table}

 \begin{table}
 \caption{Examples of the consonant phonemes} \label{tab:consonants2}
  \resizebox{\columnwidth}{!}{
 \begin{tabular}{lll|lll}
\toprule
p  & 	 \ipa{ɯ-\textbf{pɯ}} & 	 `its young' & 	tɕ & 	\ipa{ɯ-\textbf{tɕɯ}}   & 	 `his boy' \\ 
pʰ  & 	 \ipa{ɯ-\textbf{pʰɯ}} & 	 `its price' & 	tɕʰ & 	\ipa{\textbf{tɕhɯ}wur}   & 	 `blister' \\ 
b  & 	 \ipa{ba\textbf{bɯ}} & 	 `blackcurrant' & 	dʑ & 	\ipa{\textbf{dʑɯβ}dʑɯβ}   & 	 `rough' \\ 
mb  & 	 \ipa{\textbf{mbɯt}} & 	 `collapse'  & 	ndʑ & 	\ipa{\textbf{ndʑɯ}nɯ}   & 	 `Angelica sp.' \\ 
m  & 	  \ipa{tɯ\textbf{mɯ}}  & 	 `sky' & 	ɕ & 	\ipa{\textbf{ɕɯ}ŋgɯ}   & 	 `before' \\ 
w  & 	  \ipa{\textbf{wɯ}wɯ}  & 	 `Boletus sp.' & 	ʑ & 	\ipa{\textbf{ʑɯ}rɯʑɤri}   & 	 `progressively' \\ 
t  & 	  \ipa{\textbf{tɯ}boʁ}  & 	 `one group' & 	c & 	\ipa{\textbf{cɯ}} & 	 `stone' \\ 
tʰ  & 	  \ipa{\textbf{tʰɯ}ɣi}  & 	 `\textsc{imp:downstream}:come' & 	cʰ & 	\ipa{tɤ\textbf{cʰɯ}} & 	 `wedge' \\ 
d  & 	  \ipa{\textbf{dɯ}dɯt}  & 	 `turtledove' & 	ɟ & 	\ipa{wa\textbf{ɟɯ}} & 	 `earthquake' \\ 
nd  & 	  \ipa{\textbf{ndɯ}}  & 	 `appear (rainbow)' & 	ɲɟ & 	\ipa{\textbf{ɲɟɯ}} & 	 `open (it)' \\ 
ts  & 	  \ipa{\textbf{tsɯ}tho}   & 	 `goat kid' & 	ɲ & 	\ipa{\textbf{ɲɯɣ}ɲɯɣ} & 	 `soft and powdery' \\ 
tsʰ  & 	  \ipa{ɯ-\textbf{tsʰɯ}ɣa}   & 	 `appearance' & 	j & 	\ipa{ɯ-\textbf{jɯ}}   & 	 `its handle' \\ 
dz  & 	  \ipa{\textbf{dzɯr}dzɯr}   & 	 `straight' & 	k & 	\ipa{\textbf{kɯ}ki}   & 	 `this' \\ 
ndz   & 	  \ipa{\textbf{ndzɯ}pe}   & 	 `way of sitting' & 	kʰ & 	\ipa{\textbf{kʰɯ}na}   & 	 `dog' \\ 
n   & 	  \ipa{\textbf{nɯ}ŋa}   & 	 `cow' & 	g & 	\ipa{\textbf{gɯ}gɯɣ}   & 	 `very dark (sky)' \\ 
s   & 	  \ipa{\textbf{sɯ}mat}   & 	 `fruit' & 	ŋg & 	\ipa{ɯ-\textbf{ŋgɯ}}   & 	 `inside' \\ 
z   & 	  \ipa{\textbf{zɯ}mi}   & 	 `almost' & 	ŋ & 	\ipa{ɕa\textbf{ŋɯ}}   & 	 `heat (deer)' \\ 
l   & 	  \ipa{\textbf{lɯ}lu}   & 	 `cat' & 	x & 	\ipa{\textbf{xɯr}xɯr}   & 	 `round' \\ 
ɬ   & 	  \ipa{\textbf{ɬɯɣ}nɤɬɯɣ}   & 	 `breathing mouvement' & 	ɣ & 	\ipa{\textbf{ɣɯ}}   & 	 `genitive' \\ 
tʂ   & 	  \ipa{\textbf{tʂɯ}tʂu}   & 	 `on the road' & 	q & 	\ipa{\textbf{qɯ}qli}   & 	 `staring' \\ 
tʂʰ   & 	  \ipa{\textbf{tʂʰɯɣ}}   & 	 `maybe' & 	qʰ & 	\ipa{sɤ\textbf{qhɯ}qha}   & 	 `naughty' \\ 
dʐ   & 	\ipa{\textbf{dʐɯɣ}dʐɯɣ}   & 	 `strong (of tea)' & 	ɴɢ & 	\ipa{mɯ\textbf{ɴɢɯ}}  & 	 `Ligularia fischeria' \\ 
ndʐ & 	\ipa{\textbf{ndʐɯ}nbu}   & 	 `guest' & 	χ & 	\ipa{\textbf{χɯŋ}χɯŋ}   & 	 `a little orange' \\ 
ʂ & 	\ipa{\textbf{ʂɯŋ}ʂɯŋ}   & 	 `clear' & 	ʁ & 	\ipa{naŋ\textbf{ʁɯ}}   & 	 `shirt' \\ 
r & 	\ipa{\textbf{rɯ}}   & 	 `temporary place (nomads)' & 	  & 	 & 	 \\ 
\bottomrule
\end{tabular}}
\end{table}

Of these consonants, five are only attested in borrowings from Tibetan and/or ideophones: /ɬ/, /ʂ/, /dʐ/, /dʑ/ and /g/.

The analysis of prenasalized voiced stops and affricates, of palatal stops and of /ɬ/ as unitary phonemes rather than clusters /NC/, /C+j/ an /l+x/ respectively will be justified in section \ref{sec:non.clusters}. 

The /ɬ/ is slightly aspirated [ɬʰ], unlike other unvoiced fricatives.


As in many languages of the Tibetan area, the /r/ in a trilled retroflex voiced fricative [ɽ͡ʐ] in onset position, sometimes realized as a simpled voiced fricative [ʐ].


  \subsection{Consonants clusters} \label{sec:clusters}
  Japhug counts 396 clusters, including  307 clusters with two consonants and 89 with three consonants.
  
 A useful test to analyse and classify clusters is partial reduplication, a very productive process which can be applied to both verb and noun stems and has a variety of morphosyntactic functions (see \citealt{jacques07redupl}). In partial reduplication, the rhyme of the replicated syllable is changed to \ipa{ɯ} in the replicant.
 
 \begin{exe}
\ex
\glt \ipa{praʁ} $\rightarrow$ \ipa{pɯ-praʁ}
\ex
\glt \ipa{wraʁ} $\rightarrow$ \ipa{wrɯ-wraʁ}
\end{exe}
  
  
  \citet{japhug14ideophones}
 
 
 
 svarabhakti vowels \ipa{ʑŋgri} \ipa{ndʑrɯ} \ipa{zgri} not with unvoiced
 
 
 
 
 only word-internal in Tibetan loanwords tk tp 
 pt only in \ipa{sqaptɤɣ}
 
 
      \subsection{Clusters vs unitary phonemes} \label{sec:non.clusters}
      
      
palatal stops:      
      \ipa{ɲɟo} \ipa{ŋgjo}
      
      ɯ-mtsioʁ
	tso tɤtsoʁ 
	tɤtʂo (patʂo). ta-tʂoʁ
 tɕoŋtɣa mtɕoʁ
 co tɤcoʁcoʁ
 pa-kjo
     lɤ-qjoʁ
      \subsection{Codas}label{sec:coda}
    
    
     \section{Vowels} \label{sec:vowels}
     
          \subsection{Diphthongs}
     zɯu
     lɯa
     
     thɤlwa
     
     \section{Suprasegmentals}
     
         
     \section{Appended text}
\bibliographystyle{linquiry2}
\bibliography{bibliogj}
\end{document}