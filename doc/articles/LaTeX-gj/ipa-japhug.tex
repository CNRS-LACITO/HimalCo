\documentclass[oldfontcommands,oneside,a4paper,11pt]{article} 
\usepackage{fontspec}
\usepackage{natbib}
\usepackage{booktabs}
\usepackage{xltxtra} 
\usepackage{longtable}
\usepackage{polyglossia} 
\usepackage[table]{xcolor}
\usepackage{gb4e} 
\usepackage{multicol}
\usepackage{graphicx}
\usepackage{float}
\usepackage{lineno}
\usepackage{textcomp}
\usepackage{hyperref} 
\hypersetup{bookmarks=false,bookmarksnumbered,bookmarksopenlevel=5,bookmarksdepth=5,xetex,colorlinks=true,linkcolor=blue,citecolor=blue}
\usepackage[all]{hypcap}
\usepackage{memhfixc}
\usepackage{lscape}
 

\setmainfont[Mapping=tex-text,Numbers=OldStyle,Ligatures=Common]{Charis SIL} \newfontfamily\phon[Mapping=tex-text,Ligatures=Common,Scale=MatchLowercase,FakeSlant=0.3]{Charis SIL} 
\newcommand{\ipa}[1]{{\phon #1}} %API tjs en italique
 
\newcommand{\grise}[1]{\cellcolor{lightgray}\textbf{#1}}
\newfontfamily\cn[Mapping=tex-text,Ligatures=Common,Scale=MatchUppercase]{MingLiU}%pour le chinois
\newcommand{\zh}[1]{{\cn #1}}


\newcommand{\tib}[1]{\cellcolor{lightgray}\textbf{#1}}
\newcommand{\idph}[1]{\cellcolor{gray}\textbf{#1}}

\XeTeXlinebreaklocale "zh" %使用中文换行
\XeTeXlinebreakskip = 0pt plus 1pt %
 %CIRCG
 


\begin{document} 

\title{Japhug Rgyalrong}
\author{Guillaume Jacques}
\maketitle
\linenumbers
 \section{Introduction}
 This paper focuses on the Japhug language (local name \ipa{kɯrɯ skɤt}) of Kamnyu village (\ipa{kɤmɲɯ}, Chinese \textit{Ganmuniao} \zh{干木鸟}) in Gdong-brgyad area (\ipa{ʁdɯrɟɤt}, Chinese  \textit{Longerjia} \zh{龙尔甲}), Mbarkhams county (Chinese \textit{Maerkang} \zh{马尔康}), Rngaba prefecture, Sichuan province, China.
 
 Japhug belongs to the Sino-Tibetan family, and is one of the four Rgyalrong languages, alongside Tshobdun, Zbu and Situ.\footnote{See  \citet{jackson00sidaba} for an overview of the Rgyalrong group. A grammar of Japhug is available (\citealt{jacques08}) as well as a series of articles on morphosyntax (see for instance  \citealt{jacques13harmonization} and
 \citealt{jacques14antipassive}) but little has been published specifically on phonology. } 
 
The description is based on the author’s fieldwork, and the word list and the short story in the appendix have been provided by Tshendzin (Chenzhen \zh{陈珍}, female, born 1950), a retired schoolteacher (bilingual in Japhug and Sichuan mandarin).

 Japhug has a very developed system of ideophones (\citealt{japhug14ideophones}), which present marked phonological features, in particular rare clusters. In the following discussion, phonemes or clusters limited to ideophones will be treated separately. In addition, about a quarter of the Japhug vocabulary is borrowed from Tibetan, and these loanwords, like the ideophones, fill some gaps in the phonotactic distribution of vowels and consonants, and these cases are clearly distinguished from the native vocabulary.

 
 \section{Consonants}
 In Japhug, syllables follow the template (C)(C)(C)V(C) with initial clusters containing at most three consonants, and at most one coda.
 
 \subsection{Simplet onsets}
 The consonant inventory of Japhug comprises 49 phonemes as listed in Table \ref{tab:consonants}. There is a general four-way contrast in stops and affricates between unvoiced unaspirated, unvoiced aspirated, voiced and prenasalized.
 
Since monosyllabic words are few,  most examples involve disyllabic words, whose first syllable illustrates the consonant in question followed by the vowel \ipa{ɯ}, the most common one. For some examples involving possessed nouns (on this topic see \citealt[6]{jacques14antipassive}), a possessive prefix must always be present, and we always give here the third person singular prefix  \ipa{ɯ--}.
 
 \begin{table}
 \caption{Consonantal phonemes in Japhug Rgyalrong} \label{tab:consonants}
 \resizebox{\columnwidth}{!}{
\begin{tabular}{llllllllll}
\toprule
& &Bilabial& Dental/Alveolar & Retroflex &Alveolo-palatal&Palatal & Velar&Uvular\\
\midrule
Stop&unvoiced & p & t & & & c & k & q\\
&aspirated & pʰ & tʰ & & & cʰ & kʰ & qʰ\\
&voiced & b & d & & & ɟ & g &  \\
&prenasalized& mb & nd & & & ɲɟ & ŋg & ɴɢ \\
Affricate&unvoiced &   & ts & tʂ& tɕ &  &   &  \\
&aspirated &  & tsʰ & tʂʰ& tɕʰ &  &   &  \\
&voiced &   & dz & dʐ& dʑ &  &   &  \\
&prenasalized&   & ndz & ndʐ & ndʑ &  &   &  \\
Nasal &&m& n   &&&ɲ &ŋ \\
Fricative &unvoiced&  & s & ʂ& ɕ &  &x   &χ  \\
&voiced&  & z & & ʑ &  &ɣ   &ʁ  \\
Approximant && w &&  &&j &\\
Rhotic &&&&r&&\\
Lateral &sonorant &&l\\
&  fricative &&ɬ\\
\bottomrule
\end{tabular}}
\end{table}

 \begin{table}
 \caption{Examples of the consonant phonemes} \label{tab:consonants2}
  \resizebox{\columnwidth}{!}{
 \begin{tabular}{lll|lll}
\toprule
p  & 	 \ipa{ɯ-\textbf{pɯ}} & 	 `its young' & 	tɕ & 	\ipa{ɯ-\textbf{tɕɯ}}   & 	 `his boy' \\ 
pʰ  & 	 \ipa{ɯ-\textbf{pʰɯ}} & 	 `its price' & 	tɕʰ & 	\ipa{\textbf{tɕʰɯ}wur}   & 	 `blister' \\ 
b  & 	 \ipa{ba\textbf{bɯ}} & 	 `blackcurrant' & 	dʑ & 	\ipa{\textbf{dʑɯβ}dʑɯβ}   & 	 `rough' \\ 
mb  & 	 \ipa{\textbf{mbɯt}} & 	 `collapse'  & 	ndʑ & 	\ipa{\textbf{ndʑɯ}nɯ}   & 	 `Angelica sp.' \\ 
m  & 	  \ipa{tɯ\textbf{mɯ}}  & 	 `sky' & 	ɕ & 	\ipa{\textbf{ɕɯ}ŋgɯ}   & 	 `before' \\ 
w  & 	  \ipa{\textbf{wɯ}wɯ}  & 	 `Boletus sp.' & 	ʑ & 	\ipa{\textbf{ʑɯ}rɯʑɤri}   & 	 `progressively' \\ 
t  & 	  \ipa{\textbf{tɯ}boʁ}  & 	 `one group' & 	c & 	\ipa{\textbf{cɯ}} & 	 `stone' \\ 
tʰ  & 	  \ipa{\textbf{tʰɯ}ɣi}  & 	 `\textsc{imp:downstream}:come' & 	cʰ & 	\ipa{tɤ\textbf{cʰɯ}} & 	 `wedge' \\ 
d  & 	  \ipa{\textbf{dɯ}dɯt}  & 	 `turtledove' & 	ɟ & 	\ipa{wa\textbf{ɟɯ}} & 	 `earthquake' \\ 
nd  & 	  \ipa{\textbf{ndɯ}}  & 	 `appear (rainbow)' & 	ɲɟ & 	\ipa{\textbf{ɲɟɯ}} & 	 `open (it)' \\ 
ts  & 	  \ipa{\textbf{tsɯ}tʰo}   & 	 `goat kid' & 	ɲ & 	\ipa{\textbf{ɲɯɣ}ɲɯɣ} & 	 `soft and powdery' \\ 
tsʰ  & 	  \ipa{ɯ-\textbf{tsʰɯ}ɣa}   & 	 `appearance' & 	j & 	\ipa{ɯ-\textbf{jɯ}}   & 	 `its handle' \\ 
dz  & 	  \ipa{\textbf{dzɯr}dzɯr}   & 	 `straight' & 	k & 	\ipa{\textbf{kɯ}ki}   & 	 `this' \\ 
ndz   & 	  \ipa{\textbf{ndzɯ}pe}   & 	 `way of sitting' & 	kʰ & 	\ipa{\textbf{kʰɯ}na}   & 	 `dog' \\ 
n   & 	  \ipa{\textbf{nɯ}ŋa}   & 	 `cow' & 	g & 	\ipa{\textbf{gɯ}gɯɣ}   & 	 `very dark (sky)' \\ 
s   & 	  \ipa{\textbf{sɯ}mat}   & 	 `fruit' & 	ŋg & 	\ipa{ɯ-\textbf{ŋgɯ}}   & 	 `inside' \\ 
z   & 	  \ipa{\textbf{zɯ}mi}   & 	 `almost' & 	ŋ & 	\ipa{ɕa\textbf{ŋɯ}}   & 	 `heat (deer)' \\ 
l   & 	  \ipa{\textbf{lɯ}lu}   & 	 `cat' & 	x & 	\ipa{\textbf{xɯr}xɯr}   & 	 `round' \\ 
ɬ   & 	  \ipa{\textbf{ɬɯɣ}nɤɬɯɣ}   & 	 `breathing mouvement' & 	ɣ & 	\ipa{\textbf{ɣɯ}}   & 	 `genitive' \\ 
tʂ   & 	  \ipa{\textbf{tʂɯ}tʂu}   & 	 `on the road' & 	q & 	\ipa{\textbf{qɯ}qli}   & 	 `staring' \\ 
tʂʰ   & 	  \ipa{\textbf{tʂʰɯɣ}}   & 	 `maybe' & 	qʰ & 	\ipa{sɤ\textbf{qʰɯ}qʰa}   & 	 `naughty' \\ 
dʐ   & 	\ipa{\textbf{dʐɯɣ}dʐɯɣ}   & 	 `strong (of tea)' & 	ɴɢ & 	\ipa{mɯ\textbf{ɴɢɯ}}  & 	 `Ligularia fischeria' \\ 
ndʐ & 	\ipa{\textbf{ndʐɯ}nbu}   & 	 `guest' & 	χ & 	\ipa{\textbf{χɯŋ}χɯŋ}   & 	 `a little orange' \\ 
ʂ & 	\ipa{\textbf{ʂɯŋ}ʂɯŋ}   & 	 `clear' & 	ʁ & 	\ipa{naŋ\textbf{ʁɯ}}   & 	 `shirt' \\ 
r & 	\ipa{\textbf{rɯ}}   & 	 `temporary place (nomads)' & 	  & 	 & 	 \\ 
\bottomrule
\end{tabular}}
\end{table}

Of these consonants, five are only attested in borrowings from Tibetan and/or ideophones: /ɬ/, /ʂ/, /dʐ/, /dʑ/ and /g/.

The analysis of prenasalized voiced stops and affricates, of palatal stops and of /ɬ/ as unitary phonemes rather than clusters /NC/, /C+j/ an /l+x/ respectively will be justified in section \ref{sec:non.clusters}. 

The /ɬ/ is slightly aspirated [ɬʰ], unlike other unvoiced fricatives.


As in many languages of the Tibetan area, the /r/ in a trilled retroflex voiced fricative [ɽ͡ʐ] in onset position, sometimes realized as a simple voiced fricative [ʐ].


  \subsection{Consonants clusters} \label{sec:clusters}
  Japhug counts 398 clusters in syllable onset position, including  309 clusters with two consonants and 89 with three consonants. Clusters that are only possible at syllable boundaries are
  
  \subsubsection{Partial reduplication} \label{sec:redp}
 A useful test to analyse and classify clusters is partial reduplication, a very productive process which can be applied to both verb and noun stems and has a variety of morphosyntactic functions (see \citealt{jacques07redupl}). When partial reduplication is applied to a syllable, the rhyme of the replicated syllable is changed to \ipa{ɯ} in the replicant.
 
Some clusters are affected by the partial reduplication:   when the last consonant  of a cluster is one of the non-nasal sonorants (\ipa{r}, \ipa{l}, \ipa{j}, \ipa{w}, \ipa{ɣ} or \ipa{ʁ}), and the preceding consonant in neither a sonorant nor a sibilant fricative, the sonorant is deleted, as in example \ref{ex:medial.r}. Glides are always deleted in the replicant regardless of the preceding consonant.
 
 \begin{exe}
\ex \label{ex:medial.r}
\glt \ipa{praʁ} `cut, break'$\rightarrow$ \ipa{pɯ-praʁ}
\end{exe}

When the penultimate consonant of the cluster is a sonorant   and the last consonant is a non-nasal sonorant which is not a glide (\ipa{r}, \ipa{l}, \ipa{ɣ} or \ipa{ʁ}), this last consonant  is not deleted. 

 \begin{exe}
\ex  \label{ex:initial.r}
\glt \ipa{wraʁ} `attach' $\rightarrow$ \ipa{wrɯ-wraʁ}
\end{exe}
  
When the prenultimate consonant is a sibilant fricative   (\ipa{s}, \ipa{z}, \ipa{ɕ}, \ipa{ʑ}) and the last consonant is a non-nasal sonorant which is not a glide (\ipa{r}, \ipa{l}, \ipa{ɣ} or \ipa{ʁ}), there is   free variation between the two possibilities.

This morphophonological rule is thus crucial in analysis and classifying consonant clusters. Sonorants that undergo deletion when partial reduplication is applied are henceforth designated as \textit{medial} consonants, and it is postulated that they do not belong to the same constituent as the rest of the onset.

In the following, we present a complete list of consonant clusters in Japhug Rgyalrong. Groups only attested in Tibetan loanwords or ideophones (or deideophonic verbs), and not in the native vocabulary, are indicated in lightgray and gray respectively in the tables.
  
  
  \subsubsection{Consonant clusters whose last element is not a non-nasal sonorant}  
Clusters whose last consonant is not a non-nasal sonorant have a limited number of possible consonants in first position: /w/, /s/, /z/, /ɕ/, /ʑ/, /l/, /r/, /ʂ/, /j/, /ɣ/, /x/, /ʁ/, /χ/, /n/, /m/ and the homorganic nasal, except for a few clusters in stop+/ɕ/.

 Clusters beginning in /w/ or in a alveolar fricative /s/ /z/ are listed in Table \ref{prein.w.s}. 
 
 /w/ is realized as [f] or [ɸ] before unvoiced obstruents and as [w] or [β] before voiced ones. /w/ does not appear before nasal or prenasalized segments, and is not compatible with a labial or a uvular segment. Some clusters with /w/ + voiced obstruents (/wz/ and /wg/) are only attested in Tibetan loanwords. Clusters with three consonant whose first element is /w/ and the last one is not a sonorant are all restricted to Tibetan borrowings except /wxt/, which is realized as [xʷt] with a labiovelarized fricative. Not all speakers maintain the contrast between /wxt/ and /xt/, and the former is only attested in a single word /wxti/ `big'.
 
 /s/ and /z/ as first element of a cluster are only contrastive before a sonorant. With obstruents, the fricative has the same voicing value as the following consonant. All clusters of this type are attested in the native vocabulary.
 
 \begin{table}
 \caption{List of consonant clusters with \ipa{w}, \ipa{s} or \ipa{z} as a first element} \label{prein.w.s}
   \resizebox{\columnwidth}{!}{
\begin{tabular}{l|lll|lll|lll|lllllll}
\toprule
\ipa{p}  &	  &	  &	&	\ipa{sp}  &	\ipa{spoz}  &	incense&	  &	  &	&	\\
\ipa{pʰ}  &	  &	  &	&	  &	  &	&	  &	  &	&	\\
\ipa{b}  &	  &	  &	&	  &	  &	&	\ipa{zb}  &	\ipa{zbaʁ}  &	&	\\
\ipa{mb}  &	  &	  &	&	  &	  &	&	\ipa{zmb}  &	\ipa{tɤzmbɯr}  &	silt&	\\
\ipa{m}  &	  &	  &	&	\ipa{sm}  &	\ipa{smar}  &river	&	\ipa{zm}  &	\ipa{zmɤrɤβ}  &	eat sth with&	\\
\ipa{t}  &	\ipa{ft}  &	\ipa{ɯ-wtaʁ}  &	sign&	\ipa{st}  &	\ipa{staχpɯ}  &	pea&	  &	  &	&	\\
\ipa{tʰ}  &	  &	  &	&	\ipa{stʰ}  &	\ipa{stʰaβ}  &touch	&	  &	  &	&	\\
\ipa{d}  &	\ipa{wd}  &	\ipa{wdɯt}  &demon	&	  &	  &	&	\ipa{zd}  &	\ipa{zdɯm}  &cloud	&	\\
\ipa{nd}  &	  &	  &	&	  &	  &	&	\ipa{znd}  &	\ipa{znde}  &	wall&	\\
\ipa{n}  &	  &	  &	&	\ipa{sn}  &	\ipa{sna}  &	able, worthy&	\ipa{zn}  &	\ipa{znɤje}  &fell sorry, regret	&	\\
\ipa{ts}  &	\ipa{wts}  &	\ipa{wtsoʁ}  &female hybrid yak	&	  &	  &	&	  &	  &	&	\\
\ipa{tsʰ}  &	\ipa{wtsʰ}  &	\ipa{wtsʰi}  &	not serious (disease)&	  &	  &	&	  &	  &	&	\\
\ipa{dz}  &	  &	  &	&	  &	  &	&	  &	  &	&	\\
\ipa{ndz}  &	  &	  &	&	  &	  &	&	  &	  &	&	\\
\ipa{s}  &	\ipa{ws}  &	\ipa{wsaŋ}  &	fumigation&	  &	  &	&	  &	  &	&	\\
\ipa{z}  &	\ipa{wz} \tib{}  &	\ipa{wzaŋsa}  &	friend&	  &	  &	&	  &	  &	&	\\
\ipa{ɬ}  &	  &	  &	&	  &	  &	&	  &	  &	&	\\
\ipa{tɕ}  &	\ipa{wtɕ}  &	\ipa{wtɕar }  &	summer&	  &	  &	&	  &	  &	&	\\
\ipa{tɕʰ}  &	\ipa{wtɕʰ}  &	\ipa{wtɕʰur}  &pour down	&	  &	  &	&	  &	  &	&	\\
\ipa{dʑ}  &	  &	  &	&	  &	  &	&	  &	  &	&	\\
\ipa{ndʑ}  &	  &	  &	&	  &	  &	&	  &	  &	&	\\
\ipa{ɕ}  &	\ipa{wɕ}  &	\ipa{wɕaʁ}  &repent	&	  &	  &	&	  &	  &	&	\\
\ipa{ʑ}  &	\ipa{wʑ}  &	\ipa{wʑar }  &	buzzard&	  &	  &	&	  &	  &	&	\\
\ipa{tʂ}  &	\ipa{wtʂ}  &	\ipa{wtʂi}  & 	&	  &	  &	&	  &	  &	&	\\
\ipa{tʂʰ}  &	  &	  &	&	  &	  &	&	  &	  &	&	\\
\ipa{dʐ}  &	  &	  &	&	  &	  &	&	  &	  &	&	\\
\ipa{ndʐ}  &	  &	  &	&	  &	  &	&	  &	  &	&	\\
\ipa{ʂ}  &	  &	  &	&	  &	  &	&	  &	  &	&	\\
\ipa{c}  &	\ipa{wc}  &	\ipa{tɯ-wcaʁ }  &dorsal mat	&	\ipa{sc}  &	\ipa{scoʁ}  &scoop	&	  &	  &	&	\\
\ipa{cʰ}  &	  &	  &	&	\ipa{scʰ}  &	\ipa{scʰɤt}  &	come down (water level)&	  &	  &	&	\\
\ipa{ɟ}  &	\ipa{wɟ}  &	\ipa{wɟi}  &run after	&	  &	  &	&	\ipa{zɟ}  &	\ipa{nɯzɟɯ}  &	suffer losses&	\\
\ipa{ɲɟ}  &	  &	  &	&	  &	  &	&	\ipa{zɲɟ}  &	\ipa{zɲɟa}  &	plant sp.&	\\
\ipa{ɲ}  &	  &	  &	&	\ipa{sɲ}  &	\ipa{sɲaŋne}  &fasting	&	   &	 &	&	\\
\ipa{k}  &	\ipa{wk}  &	\ipa{wka}  &	order&	\ipa{sk}  &	\ipa{skɤm}  &	ox&	  &	  &	&	\\
\ipa{kʰ}  &	  &	  &	&	\ipa{skʰ}  &	\ipa{rɟɤskhi}  &pan	&	  &	  &	&	\\
\ipa{g}  &	\ipa{wg}  \tib{}&	\ipa{wgoz}  &	prepare&	  &	  &	& 	\ipa{zg}  &	\ipa{zga}  &	sauce&	\\
\ipa{ŋg}  &	  &	  &	&	  &	  &	&	\ipa{zŋg}  &	\ipa{akhɤzŋga}  &	call&	\\
\ipa{ŋ}  &	  &	  &	&	\ipa{sŋ}  &	\ipa{sŋaʁ}  &	curse (v)&	  &	  &	&	\\
\ipa{x}  &	  &	  &	&	  &	  &	&	  &	  &	&	\\
\ipa{q}  &	  &	  &	&	\ipa{sq}  &	\ipa{sqamnɯz}  &	twelve&	  &	  &	&	\\
\ipa{qʰ}  &	  &	  &	&	\ipa{sqʰ}  &	\ipa{sqʰi}  &	tripod&	  &	  &	&	\\
\ipa{ɴɢ}  &	  &	  &	&	  &	  &	&	  &	  &	&	\\
\ipa{χ}  &	  &	  &	&	  &	  &	&	  &	  &	&	\\
\midrule
&	\ipa{wxt}  &	\ipa{wxti}  &big	\\
&	\ipa{wst} \tib{} &	\ipa{wstɯn}  &serve	\\
&	\ipa{wrt}  \tib{} &	\ipa{wrtɤn}  &	trustworth\\
&	\ipa{wsk}  \tib{} &	\ipa{wskɤr}  & go around, 	\\
&	\ipa{wzg}  \tib{} &	\ipa{wzgɤr}  & delay	\\
&	\ipa{wzd}  \tib{} &	\ipa{wzdɯ}  & collect	\\
&	\ipa{wzɟ}  \tib{} &	\ipa{wzɟɯr}  & transform	\\
&	\ipa{wrɟ}  \tib{} &	\ipa{wrɟaŋ}  &stretch (skin)	\\
\bottomrule
\end{tabular}}
\end{table}
 
 Clusters with /l/, /r/ and /j/ as first element are listed in Table \ref{prein.l.r.j}. /r/ and /ʂ/ are in complementary distribution as first element of a cluster, the former appearing before voiced consonants and the latter after unvoiced ones (except before /ɣ/, see section \ref{sec:other.non.medial}). 
 
 There are some phonotactic constraints on the distribution of these consonants: /l/ is not compatible with coronal fricatives, /r/ and /ʂ/ never appear before retroflex fricatives and affricates, and /j/ is not attested before alveolo-palatal and palatal consonants.
 
The \ipa{j} glide, unlike other consonants, neither devoices nor fricativizes when occurring as first element of a cluster whose second element is an obstruent.
 
  \begin{table}
 \caption{List of consonant clusters with \ipa{l}, \ipa{r} or \ipa{j} as a first element} \label{prein.l.r.j}
   \resizebox{\columnwidth}{!}{
\begin{tabular}{l|lll|lll|lll|l}
\toprule
 \ipa{p}   &	\ipa{lp}  &	\ipa{tɯ-lpɤɣ}  &	one piece&	\ipa{ʂp}  &	\ipa{tɯ-ʂpa}  &	axe&	\ipa{jp}  &	\ipa{jpum}  &	thick&	\\
\ipa{pʰ}   &	  &	  &	&	\ipa{ʂpʰ} \idph{} &	\ipa{ɣɤʂpʰɤʂpʰɤβ}  &	flap wings&	  &	  &	&	\\
\ipa{b}   &	  &	  &	&	  &	  &	&	  &	  &	&	\\
\ipa{mb}   &	  &	  &	&	\ipa{rmb}  &	\ipa{armbat}  &near	&	  &	  &	&	\\
\ipa{m}   &	\ipa{lm}  &	\ipa{tɤlmɯz}  &	straw covering the balcony&	\ipa{rm}  &	\ipa{rmɤwja}  &peacock	&	\ipa{jm}  &	\ipa{jmɯt} &forget	&	\\
\ipa{t}   &	\ipa{lt}  &	\ipa{ltɤβ}  &	fold&	\ipa{ʂt}  &	\ipa{ʂtalu}  &horse year	&	\ipa{jt}  &	\ipa{ajtɯ}  &accumulate	&	\\
\ipa{tʰ}   &	\ipa{ltʰ} \idph{} &	\ipa{ltʰɯmɯmi}  &coming slowly (sleep)	&	\ipa{ʂtʰ}  &	\ipa{pɤʂtʰɤβ}  &midde	&	  &	  &	&	\\
\ipa{d}   &	\ipa{ld}  &	\ipa{ldɯɣi}  &	bharal&	\ipa{rd}  &	\ipa{rdɤstaʁ}  &	stone&	  &	   &	&	\\
\ipa{nd}   &	  &	  &	&	\ipa{rnd}  &	\ipa{rnde}  &	find&	  &	  &	&	\\
\ipa{n}   &	\ipa{ln}  &	\ipa{lni}  &wither	&	\ipa{rn}  &	\ipa{rnaʁ}  &deep	&	\ipa{jn}  &	\ipa{jnom}  &flexible	&	\\
\ipa{ts}   &	\ipa{lts}  &	\ipa{ɕɤltsaʁ}  &leather coat	&	\ipa{ʂts}  &	\ipa{ʂtsot}  &vengeance	&	\ipa{jts}  &	\ipa{tɤ-jtsi}  &pillar	&	\\
\ipa{tsʰ}   &	\ipa{ltsʰ} \idph{} &	\ipa{ltshɤltshɤt}  & small and weak	&	\ipa{ʂtsʰ}  &	\ipa{ʂtshom}  &	have a crack (bucket)&	\ipa{jtsʰ}  &	\ipa{jtsʰi}  &	give to drink&	\\
\ipa{dz}   &	  &	  &	&	\ipa{rdz} \idph{} &	\ipa{rdzardza}  &	insolent&	  &	  &	&	\\
\ipa{ndz}   &	  &	  &	&	\ipa{rndz}  &	\ipa{rndzɤkɤŋe}  &	 shade of the mountain&	  &	  &	&	\\
\ipa{s}   &	  &	  &	&	\ipa{ʂs}  \idph{} &	\ipa{ʂsɯβʂsɯβ}  &	hairy&	  &	  &	&	\\
\ipa{z}   &	  &	  &	&	\ipa{rz}  &	\ipa{tɯ-rzɯɣ}  &	one section&	  &	  &	&	\\
\ipa{ɬ}   &	  &	  &	&	  &	  &	&	  &	  &	&	\\
\ipa{tɕ}   &	\ipa{ltɕ} & \ipa{ʂtɤltɕaʁ}  &	horse whip  &	 	\ipa{ʂtɕ}  &	\ipa{nɯʂtɕe}  &tease	&	  &	  &	&	\\
\ipa{tɕʰ}   &	\ipa{ltɕʰ} \idph{} &	\ipa{ltɕʰɤltɕʰɤt}  &hanging (of fluffy objects)	&	\ipa{ʂtɕʰ}  &	\ipa{ʂtɕʰɯʁjɯ}  &caterpillar	&	  &	  &	&	\\
\ipa{dʑ}   &	\ipa{ldʑ}  \tib{} &	\ipa{ldʑaŋkɯ}  &	green&	  &	  &	&	  &	  &	&	\\
\ipa{ndʑ}   &	  &	  &	&	\ipa{rndʑ}  &	\ipa{cɯrndʑi}  &sand	&	  &	  &	&	\\
\ipa{ɕ}   &	  &	  &	&	\ipa{rɕ}  &	\ipa{arɕo}  &	completely finished&	  &	  &	&	\\
\ipa{ʑ}   &	  &	  &	&	\ipa{rʑ}  &	\ipa{tɤ-rʑaβ}  &wife	&	  &	  &	&	\\
\ipa{tʂ}   &	  &	  &	&	  &	  &	&	  &	  &	&	\\
\ipa{tʂʰ}   &	  &	  &	&	  &	  &	&	\ipa{jtʂʰ}  &	\ipa{qɤjtʂʰa}  &vulture	&	\\
\ipa{dʐ}   &	\ipa{ldʐ} \idph{}  &	\ipa{ldʐaŋldʐaŋ}  &	hanging (big object)&	  &	  &	&	  &	  &	&	\\
\ipa{ndʐ}   &	  &	  &	&	  &	  &	&	\ipa{jndʐ}  &	\ipa{jndʐɤz}  &thick (powder)	&	\\
\ipa{ʂ}   &	  &	  &	&	  &	  &	&	  &	  &	&	\\
\ipa{c}   &	\ipa{lc}  \idph{}&	\ipa{lcɯɣlcɯɣ}  &	drenching&	\ipa{ʂc}  &	\ipa{tɤ-ʂcoʁ}  &	mud&	  &	  &	&	\\
\ipa{cʰ}   &	\ipa{lcʰ}  &	\ipa{tɯ-lcʰɯɣ}  &	section (of a bag)&	\ipa{ʂcʰ}  &	\ipa{ɯ-ʂcʰɤβ}  &	interstice&	  &	  &	&	\\
\ipa{ɟ}   &	  &	  &	&	\ipa{rɟ}  &	\ipa{rɟaʁ}  &dance (v)	&	  &	  &	&	\\
\ipa{ɲɟ}   &	  &	  &	&	\ipa{rɲɟ}  &	\ipa{rɲɟaʁlo}  &	bolt&	  &	  &	&	\\
\ipa{ɲ}   &	  &	  &	&	\ipa{rɲ}  &	\ipa{rɲaŋ}  &	ancient&	  &	  &	&	\\
\ipa{j}   &	  &	  &	&	  &	  &	&	  &	  &	&	\\
\ipa{k}   &	  &	  &	&	\ipa{ʂk}  &	\ipa{ʂko}  &	hard&	\ipa{jk}  &	\ipa{tɤ-jkɯz}  &secret	&	\\
\ipa{kʰ}   &	  &	  &	&	\ipa{ʂkʰ}  &	\ipa{tɤ-ʂkʰom}  &	feather rachis&	   &	  &	&	\\
\ipa{g}   &	  &	  &	&	\ipa{rg}  &	\ipa{rga}  &	like&	  &	  &	&	\\
\ipa{ŋg}   &	  &	  &	&	\ipa{rŋg}  &	\ipa{rŋgɤm}  &hard piece	&	   &	   &	&	\\
\ipa{ŋ}   &	\ipa{lŋ}  \idph{} &	\ipa{lŋɤlŋɤt}  &hanging (fruit)	&	\ipa{rŋ}  &	\ipa{tɯ-rŋa}  &	face&	\ipa{jŋ}  &	\ipa{tɤ-jŋoʁ}  &	hook&	\\
\ipa{x}   &	\ipa{lx}  \idph{} &	\ipa{lxɤβlxɤβ}  &thick (clothes)	&	  &	  &	&	  &	  &	&	\\
\ipa{q}   &	  &	  &	&	\ipa{ʂq}  &	\ipa{ʂqoʁ}  &hug	&	\ipa{jq}  &	\ipa{jqe}  &	able to lift&	\\
\ipa{qʰ}   &	  &	  &	&	\ipa{ʂqʰ}  &	\ipa{tɤ-ʂqʰu}  &bark, skin&	   &	   &	&	\\
\ipa{ɴɢ}   &	  &	  &	&	\ipa{rɴɢ}  &	\ipa{ɕɯrɴɢo}  &Anisodus tanguticus 	&	  &	  &	&	\\
\ipa{χ}   &	  &	  &	&	\ipa{ʂχ}   &	\ipa{ʂχɯʂχi}  &with big nostrils	&	\ipa{jχ}  &	\ipa{ajχoʁ}  &	flap (belly)&	\\
\midrule
&&&&&&&\ipa{jmŋ} & \ipa{tɯ-jmŋo} &dream (n)\\
\bottomrule
\end{tabular}}
\end{table}
 
 \begin{table}
 \caption{List of consonant clusters with \ipa{ɕ/ʑ}, \ipa{x/ɣ} or \ipa{χ/ʁ} as a first element} \label{prein.Z.G.R}
   \resizebox{\columnwidth}{!}{
\begin{tabular}{l|lll|lll|lll|l}
\toprule
\ipa{p}  &	\ipa{ɕp}  &	\ipa{ɕpaʁ}  &	be thirsty &	\ipa{xp}  &	\ipa{tɯ-xpa}  &	&	\ipa{χp}  &	\ipa{χpi}  &	\\
\ipa{pʰ}  &	\ipa{ɕpʰ}  &	\ipa{ɕpʰɤt}  &	patch &	  &	  &	&	\ipa{χpʰ}  &	\ipa{taχpʰe}  &	\\
\ipa{b}  &	  &	  &	&	  &	  &	&	  &	  &	\\
\ipa{mb}  &	\ipa{ʑmb}  &	\ipa{ʑmbɤr}  &	&	\ipa{ɣmb}  &	\ipa{tɯ-ɣmba}  &	&	\ipa{ʁmb}  &	\ipa{aʁmbɯm}  &	\\
\ipa{m}  &	\ipa{ɕm}  &	\ipa{ɕmi}  &	&	\ipa{ɣm}  &	\ipa{tɯ-ɣmaz}  &	&	\ipa{ʁm}  &	\ipa{ʁmaʁ}  &	\\
\ipa{w}  &	  &	  &	&	  &	  &	&	  &	  &	\\
\ipa{t}  &	\ipa{ɕt}  &	\ipa{ɕte}  &	&	\ipa{xt}  &	\ipa{xtɯt}  &	&	\ipa{χt}  &	\ipa{χtorma}  &	\\
\ipa{tʰ}  &	\ipa{ɕtʰ}  &	\ipa{ɕthɯz}  &	&	\ipa{xtʰ}  &	\ipa{xtʰom}  &	&	\ipa{χtʰ}  &	\ipa{naχthɤβ}  &	\\
\ipa{d}  &	\ipa{ʑd}  &	\ipa{ʑdɯɣʑdɯɣ}  &	&	\ipa{ɣd}  &	\ipa{ɣdɤso}  &	&	\ipa{ʁd}  &	\ipa{ʁdɯɣ}  &	\\
\ipa{nd}  &	  &	  &	&	\ipa{ɣnd}  &	\ipa{ɣnde}  &	&	\ipa{ʁnd}  &	\ipa{ʁndɤr}  &	\\
\ipa{n}  &	\ipa{ɕn}  &	\ipa{ɕnat}  &	&	\ipa{ɣn}  &	\ipa{ɣnɤsqi}  &	&	\ipa{ʁn}  &	\ipa{ʁnaʁna}  &	\\
\ipa{ts}  &	  &	  &	&	\ipa{xts}  &	\ipa{xtsɤɕna}  &	&	\ipa{χts}  &	\ipa{χtso}  &	\\
\ipa{tsʰ}  &	  &	  &	&	\ipa{xtsʰ}  &	\ipa{xtsʰɯm}  &	&	\ipa{χtsʰ}  &	\ipa{χtsʰɤχtsʰɤt}  &	\\
\ipa{dz}  &	  &	  &	&	  &	  &	&	  &	  &	\\
\ipa{ndz}  &	  &	  &	&	\ipa{ɣndz}  &	  &	&	\ipa{ʁndz}  &	\ipa{ʁndzɤr}  &	\\
\ipa{s}  &	  &	  &	&	\ipa{xs}  &	\ipa{xsar}  &	&	\ipa{χs}  &	\ipa{χsɤr}  &	\\
\ipa{z}  &	  &	  &	&	\ipa{ɣz}  &	\ipa{ɣzɯ}  &	&	\ipa{ʁz}  &	\ipa{ʁzɤβ}  &	\\
\ipa{l}  &	  &	  &	&	  &	  &	&	  &	  &	\\
\ipa{ɬ}  &	  &	  &	&	  &	  &	&	  &	  &	\\
\ipa{tɕ}  &	  &	  &	&	\ipa{xtɕ}  &	\ipa{xtɕi}  &	&	\ipa{χtɕ}  &	\ipa{χtɕoŋ}  &	\\
\ipa{tɕʰ}  &	  &	  &	&	\ipa{xtɕʰ}  &	\ipa{xtɕʰɯt}  &	&	  &	  &	\\
\ipa{dʑ}  &	  &	  &	&	\ipa{ɣdʑ}  &	  &	&	  &	  &	\\
\ipa{ndʑ}  &	  &	  &	&	\ipa{ɣndʑ}  &	\ipa{ɣndʑɤβ}  &	&	  &	  &	\\
\ipa{ɕ}  &	  &	  &	&	\ipa{xɕ}  &	\ipa{xɕɤj}  &	&	\ipa{χɕ}  &	\ipa{χɕu}  &	\\
\ipa{ʑ}  &	  &	  &	&	\ipa{ɣʑ}  &	\ipa{ɣʑo}  &	&	\ipa{ʁʑ}  &	\ipa{ʁʑɯnɯ}  &	\\
\ipa{tʂ}  &	\ipa{ɕtʂ}  &	\ipa{ɕtʂaŋlaŋ}  &	&	\ipa{xtʂ}  &	  &	&	\ipa{χtʂ}  &	\ipa{χtʂɯɣdʑa}  &	\\
\ipa{tʂʰ}  &	  &	  &	&	  &	  &	&	  &	  &	\\
\ipa{dʐ}  &	  &	  &	&	  &	  &	&	  &	  &	\\
\ipa{ndʐ}  &	  &	  &	&	  &	  &	&	  &	  &	\\
\ipa{r}  &	  &	  &	&	  &	  &	&	  &	  &	\\
\ipa{ʂ}  &	  &	  &	&	\ipa{xʂ}  &	\ipa{xʂɤxʂɤt }  &	&	\ipa{χʂ}  &	\ipa{χʂɤχʂɤt}  &	\\
\ipa{c}  &	  &	  &	&	\ipa{xc}  &	\ipa{xcat}  &	&	\ipa{χc}  &	\ipa{χcoŋkroŋ}  &	\\
\ipa{cʰ}  &	  &	  &	&	\ipa{xcʰ}  &	\ipa{tɤlɤxchi}  &	&	\ipa{χcʰ}  &	\ipa{χcʰa}  &	\\
\ipa{ɟ}  &	  &	  &	&	\ipa{ɣɟ}  &	\ipa{ɣɟaβ}  &	&	\ipa{ʁɟ}  &	\ipa{ʁɟa}  &	\\
\ipa{ɲɟ}  &	  &	  &	&	  &	  &	&	  &	  &	\\
\ipa{ɲ}  &	  &	  &	&	\ipa{ɣɲ}  &	\ipa{ɯ-ɣɲaʁ}  &	&	\ipa{ʁɲ}  &	\ipa{ʁɲɤrpa}  &	\\
\ipa{j}  &	  &	  &	&	  &	  &	&	  &	  &	\\
\ipa{k}  &	\ipa{ɕk}  &	\ipa{ɕkom}  &	&	  &	  &	&	  &	  &	\\
\ipa{kʰ}  &	\ipa{ɕkʰ}  &	\ipa{ɕkhɤm}  &	&	  &	  &	&	  &	  &	\\
\ipa{g}  &	\ipa{ʑg}  &	\ipa{ʑgaʁ}  &	&	  &	  &	&	  &	  &	\\
\ipa{ŋg}  &	\ipa{ʑŋg}  &	\ipa{ʑŋgu}  &	&	  &	  &	&	  &	  &	\\
\ipa{ŋ}  &	\ipa{ɕŋ}  &	\ipa{ɕŋaʁɕŋaʁ}  &	&	  &	  &	&	  &	  &	\\
\ipa{x}  &	  &	  &	&	  &	  &	&	  &	  &	\\
\ipa{ɣ}  &	  &	  &	&	  &	  &	&	  &	  &	\\
\ipa{q}  &	\ipa{ɕq}  &	\ipa{ɕqɤjɤr}  &	&	  &	  &	&	  &	  &	\\
\ipa{qʰ}  &	\ipa{ɕqʰ}  &	\ipa{ɕqʰaloʁ}  &	&	  &	  &	&	  &	  &	\\
\ipa{ɴɢ}  &	\ipa{ʑɴɢ}  &	\ipa{ʑɴɢoloʁ}  &	&	  &	  &	&	  &	  &	\\

\end{tabular}}
\end{table}
 \subsubsection{Groups with a medial sonorant } \label{sec:medial}
  non-nasal sonorant (\ipa{r}, \ipa{l}, \ipa{j}, \ipa{w}, \ipa{ɣ} or \ipa{ʁ}) are included. The
  
 \begin{table}
 \caption{χχχχχ}
\begin{tabular}{llllllll}
&	\ipa{wɣr}  &	\ipa{wɣrum}  &white	\\
&	\ipa{wzj}  &	\ipa{wzjoz}  &	\\
  &	\ipa{wsr}  &	\ipa{wsroŋ}  &protect	\\
\end{tabular}


  \end{table} 
  
   \subsubsection{Other groups with non-nasal sonorants } \label{sec:other.non.medial}
  
 svarabhakti vowels \ipa{ʑŋgri} \ipa{ndʑrɯ} \ipa{zgri} not with unvoiced
 
 
 
 \subsubsection{Clusters at syllable boundaries}
 only word-internal in Tibetan loanwords tk tp 
 pt only in \ipa{sqaptɤɣ}
 
 
      \subsection{Clusters vs unitary phonemes} \label{sec:non.clusters}
      
      
palatal stops:      
      \ipa{ɲɟo} \ipa{ŋgjo}
      
      ɯ-mtsioʁ
	tso tɤtsoʁ 
	tɤtʂo (patʂo). ta-tʂoʁ
 tɕoŋtɣa mtɕoʁ
 co tɤcoʁcoʁ
 pa-kjo
     lɤ-qjoʁ
      \subsection{Codas} \label{sec:coda}
    
    
     \section{Vowels} \label{sec:vowels}
     
          \subsection{Diphthongs}
     zɯu
     lɯa
     
     thɤlwa
     
     \section{Suprasegmentals}
     
         
     \section{Appended text}
\bibliographystyle{linquiry2}
\bibliography{bibliogj}
\end{document}