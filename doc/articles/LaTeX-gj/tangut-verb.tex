\documentclass[oneside,a4paper,11pt]{article} 
\usepackage{fontspec}
\usepackage{natbib}
\usepackage{booktabs}
\usepackage{xltxtra} 
\usepackage{polyglossia} 
%\usepackage[table]{xcolor}
\usepackage{gb4e} 
\usepackage{multicol}
\usepackage{graphicx}
\usepackage{float}
\usepackage{lineno}
\usepackage{hyperref} 
\hypersetup{bookmarksnumbered,bookmarksopenlevel=5,bookmarksdepth=5,colorlinks=true,linkcolor=blue,citecolor=blue}
\usepackage[all]{hypcap}
\usepackage{memhfixc}
\usepackage{tangutex2}

%\setmainfont[Mapping=tex-text,Numbers=OldStyle,Ligatures=Common]{Charis SIL} 
\newfontfamily\phon[Mapping=tex-text,Ligatures=Common,Scale=MatchLowercase]{Charis SIL} 
\newcommand{\ipa}[1]{\textbf{\phon#1}} %API tjs en italique
 \newcommand{\jpg}[2]{\ipa{#1} `#2'} %API tjs en italique
\newcommand{\grise}[1]{\cellcolor{lightgray}\textbf{#1}}
\newfontfamily\cn[Mapping=tex-text,Ligatures=Common,Scale=MatchUppercase]{SimSun}%pour le chinois
\newcommand{\zh}[1]{{\cn #1}}
\newcommand{\tld}{\textasciitilde{}}
\newcommand{\tgz}[1]{\mo{#1} \tg{#1}}
\newcommand{\tgf}[1]{\begin{tabular}{l}\mo{#1}\\{\tiny #1}\end{tabular}}

\XeTeXlinebreaklocale "zh" %使用中文换行
\XeTeXlinebreakskip = 0pt plus 1pt %
 \newcommand{\bleu}[1]{{\color{blue}#1}}
\newcommand{\rouge}[1]{{\color{red}#1}} 
\newcommand{\refb}[1]{(\ref{#1})}
\newcommand{\factual}[1]{\textsc{:fact}}
\newcommand{\rdp}{\textasciitilde{}} 

\newcommand{\conv}{\textsc{conv}}
\newcommand{\dat}{\textsc{dat}}
\newcommand{\dem}{\textsc{dem}}
\newcommand{\evid}{\textsc{ifr}}
\newcommand{\erg}{\textsc{erg}}
\newcommand{\inv}{\textsc{inv}}
\newcommand{\pfv}{\textsc{pfv}} 
\newcommand{\prf}{\textsc{pfv}}
\newcommand{\refl}{\textsc{refl}}
\newcommand{\sg}{\textsc{sg}}
\newcommand{\pl}{\textsc{pl}}
\newcommand{\sens}{\textsc{sens}}

\sloppy
\begin{document} 

\section{Case marking}
Tangut has a relatively rich system of case markers, in comparison with those of 

\citet[143-161]{kepping85}

\begin{table}[H]
\caption{Case markers in Stau, Khroskyabs and Tangut}\label{tab:tangut} \centering
\begin{tabular}{ll|ll|llllll}
\toprule
Stau && Khroskyabs && Tangut & \\
\midrule
\ipa{-w} & \textsc{erg} &&& \mo{5880} \ipa{ŋwu²} & \textsc{instr} \\
\ipa{-j} & \textsc{gen} &\ipa{-ji} &\textsc{gen} &\mo{1139} \ipa{.jij¹} & \textsc{gen}, antiergative\\
\ipa{-ʁa} & \textsc{all} & \ipa{-ʁɑ} & \textsc{loc} & \mo{5856} \ipa{ɣa²} & \textsc{loc} \\
\ipa{-tɕʰa} & \textsc{loc} &&& \mo{0089} \ipa{tɕʰjaa¹}  &\textsc{loc} \\
\ipa{-kʰa} & \textsc{instr} &&& \mo{5993} \ipa{kʰa¹}  &in the middle of \\
\toprule
\end{tabular}
\end{table}

\section{Person Indexation }

\subsection{Suffixes}
\begin{table}[H]
\caption{Pronouns and person suffixes in Tangut (\citealt{kepping75agreement, kepping85})}\label{tab:pronoms.suffixes} \centering
\begin{tabular}{llllll} 
\toprule
\multicolumn{3}{c}{Pronoun} &\multicolumn{3}{c}{Suffix} \\
\midrule
\tgf{2098} & \ipa{ŋa²}  & 1\textsc{sg} & \tgf{2098} & \ipa{ŋa²}  &1\textsc{sg} \\
\tgf{3926} & \ipa{nja²} & 2\textsc{sg} & \tgf{4601} & \ipa{nja²} &2\textsc{sg} \\
\tgf{4028} &  \ipa{nji²} & 2\textsc{sg}  honorific or 2\textsc{pl} & \tgf{4884} & \ipa{nji²} & 1\textsc{pl} and 2\textsc{pl} \\
\bottomrule
\end{tabular}
\end{table}

\subsection{Stem alternation}

\begin{table}[H]
\caption{Attested forms of the ditransitive paradigm in Tangut}\centering  \label{tab:paradigm}
\begin{tabular}{lllll}
\toprule
	&	1\sg{}	&	2\sg{}	&	1/2\pl{}	&	3	\\
	\midrule
1\sg{}	&	?	&	A-\ipa{nja²}	&	?	&	 B-\ipa{ŋa²}	\\
2\sg{}	&	A-\ipa{ŋa²}	&	B-\ipa{nja²}	&	A-\ipa{nji²}	&	 B-\ipa{nja²}	\\
1/2\pl{}	&	 A-\ipa{ŋa²}	& ?	&	?	&	A-\ipa{nji²}	\\
3	&	A-\ipa{ŋa²}	&	A-\ipa{nja²}	&	?	&	A 	\\
\bottomrule
\end{tabular}
\end{table}

\bibliographystyle{unified}
\bibliography{bibliogj}
\end{document}