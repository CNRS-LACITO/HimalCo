\documentclass[oldfontcommands,oneside,a4paper,11pt]{article} 
\usepackage{fontspec}
\usepackage{natbib}
\usepackage{booktabs}
\usepackage{xltxtra} 
\usepackage{polyglossia} 
\usepackage[table]{xcolor}
\usepackage{gb4e} 
\usepackage{multicol}
\usepackage{graphicx}
\usepackage{float}
\usepackage{textcomp}
\usepackage{hyperref} 
\hypersetup{bookmarks=false,bookmarksnumbered,bookmarksopenlevel=5,bookmarksdepth=5,xetex,colorlinks=true,linkcolor=blue,citecolor=blue}
\usepackage[all]{hypcap}
\usepackage{memhfixc}
\usepackage{lscape}
 

%\setmainfont[Mapping=tex-text,Numbers=OldStyle,Ligatures=Common]{Charis SIL} 
\newfontfamily\phon[Mapping=tex-text,Ligatures=Common,Scale=MatchLowercase,FakeSlant=0.3]{Charis SIL} 
\newcommand{\ipa}[1]{{\phon #1}} %API tjs en italique
 
\newcommand{\grise}[1]{\cellcolor{lightgray}\textbf{#1}}
\newcommand{\bleute}[1]{\cellcolor{green}\textbf{#1}}
\newcommand{\rouge}[1]{\cellcolor{red}\textbf{#1}}
\newfontfamily\cn[Mapping=tex-text,Ligatures=Common,Scale=MatchUppercase]{SimSun}%pour le chinois
\newcommand{\zh}[1]{{\cn #1}}
\newcommand{\topic}{\textsc{dem}}
\newcommand{\tete}{\textsuperscript{\textsc{head}}}
\newcommand{\rc}{\textsubscript{\textsc{rc}}}
\newcommand{\refb}[1]{(\ref{#1})}
\XeTeXlinebreaklocale 'zh' %使用中文换行
\XeTeXlinebreakskip = 0pt plus 1pt %
 %CIRCG
\usepackage{endnotes}
\let\footnote=\endnote 


\begin{document} 
\title{Subjects, objects  and relativization in Japhug\endnote{
The glosses follow the Leipzig glossing rules; on S, A, P, T and R, see \citet{haspelmath11SAPTR}. Other abbreviations used here are: \textsc{appl} applicative, \textsc{antipass} antipassive,\textsc{dem} demonstrative, \textsc{dist} distal, \textsc{emph} emphatic, \textsc{fact} factual, \textsc{ifr} inferential, \textsc{indef} indefinite, \textsc{inv} inverse,  \textsc{lnk} linker, \textsc{pfv} perfective, \textsc{poss} possessor, SAP speech act participant (first or second person), \textsc{rc} relative clause, \textsc{sens} sensory. Words borrowed from Chinese are indicated between chevrons <> and are transcribed in pinyin. %\textsc{trop} tropative, 
I would like to thank Hilary Chappell, Aimée Lahaussois, Anton Antonov, Wu Tong and the anonymous reviewers for useful comments on previous versions of this article.  The examples are taken from a corpus that is progressively being made available on the Pangloss archive (\citealt{michailovsky14pangloss}). This research was funded by the HimalCo project (ANR-12-CORP-0006) and is related to the research strand LR-4.11 ‘‘Automatic Paradigm Generation and Language Description’’ of the Labex EFL (funded by the ANR/CGI). 
} }
\author{Guillaume Jacques}
\maketitle


\textbf{Abstract}: Japhug is a language with ergative alignment on NP arguments and direct-inverse verbal indexation. However, this article, through  a detailed description of relativizing constructions in Japhug, shows the existence of accusative pivots and proposes an unambiguous definition of `subjects' and `objects' in this language. 

\textsc{Keywords}:Japhug, Relativization, Subject, Object, Syntactic pivot


\zh{提要:茶堡话有作格格局的格标记,同时在动词上有正向/反向类型的人称范畴。虽然在动词和名词形态上没有主格/宾格格局,但通过对关系句的仔细考察可以证明茶堡话有非常清楚主格/宾格格局的句法枢纽,在这些枢纽的基础上可以提出对“主语”和“宾语”明确不含糊的定义。}


\zh{关键词:茶堡话,关系句,主语,宾语,句法枢纽}


\section{Introduction}
The present article deals with relative clauses in Japhug, and how these constructions provide evidence for the existence of syntactic pivots in this language. While previous publications have discussed relative clauses in Rgyalrong languages (in particular \citealt{jackson06guanxiju}, \citealt{jacksonlin07}, \citealt{jacques08zh}, \citealt{prins11kyomkyo}), this is the first systematic description of relative clauses in Japhug.\footnote{Correlatives, non-restrictive relative clauses and other non-canonical relative clauses are not described in this work.} This article is divided in five sections.

First, I provide background information on flagging and person indexation on the verb in Japhug.\footnote{Japhug is Sino-Tibetan language spoken by around 10000 speakers in Mbarkham county, Sichuan, China. Like the three other Rgyalrong languages (Situ, Zbu and Tshobdun), it is a polysynthetic language with complex morphology (see \citealt{jacques14antipassive}). }  Transitivity is morphologically marked in Japhug in an unambiguous way. In addition to plain intransitive  and transitive verbs, we find  semi-transitive verbs which share their morphological properties with intransitive verbs and some of their syntactic properties with transitive verbs. In addition,  both secundative and indirective ditransitive verbs are found.

Second, I present general information on relative clauses in Japhug, and in particular show the existence of both head-internal and prenominal relative clauses.

Third, I describe   non-finite relative clauses, whose main verb is in a participial form. Most relative clauses belong to this type, and three participles in \ipa{kɯ-}, \ipa{kɤ-} and \ipa{sɤ-} are used depending on the syntactic function of the relativized element.

Fourth, I study finite relative clauses,  whose main verb is not nominalized. The use of this type of relative is highly restricted.

Fifth, I summarize the data presented in the previous sections and show how it allows us to strictly define syntactic pivots that can be labeled as `subjects' and `objects' in Japhug. In addition, I discuss how this study is relevant for the typology of alignment in Sino-Tibetan languages and beyond.



\section{Flagging and indexation} \label{sec:flagging.indexation}
The present section presents background information on person marking and flagging in Japhug, and shows that neither `subjects' nor `objects' can be straightforwardly defined on the exclusive basis of morphogical marking on  verbs and on NPs.


\subsection{Flagging}
Japhug presents strict verb-final word order. The only elements that can occur post-verbally are sentence-final particles, some ideophones and adverbs (see \citealt[275-6]{japhug14ideophones}), and right-dislocated constituents.

Japhug has ergative alignment on all non-SAP arguments:  S and P are unmarked (examples \ref{ex:abs} and \ref{ex:erg}), while the A of transitive verbs receives the clitic \ipa{kɯ} (example \ref{ex:erg}). This clitic is obligatory with noun phrases and third person pronouns, but in the case of first and second person pronouns it is optional. The clitic \ipa{kɯ} can also be used to mark instruments.

\begin{exe}
\ex \label{ex:abs}
\gll \ipa{tɤ-tɕɯ}  	\ipa{nɯ}  	 	\ipa{jo-ɕe}   \\
\textsc{indef.poss}-boy \textsc{dem}   \textsc{ifr}-go \\
\glt The boy went (there).
\end{exe}

\begin{exe}
\ex \label{ex:erg}
\gll \ipa{tɤ-tɕɯ}  	\ipa{nɯ}  	\ipa{kɯ}  	\ipa{χsɤr}  	\ipa{qaɕpa}  	\ipa{nɯ}  	\ipa{cʰɤ-mqlaʁ}   \\
\textsc{indef.poss}-boy \textsc{dem} \textsc{erg} gold frog \textsc{dem} \textsc{ifr}-swallow \\
\glt The boy swallowed the golden frog. (Nyima Wodzer.1, 131)
\end{exe}


Japhug is a strictly postpositional language, and postpositional phrases can be headed by either postpositions (such as comitative \ipa{cʰo} or locative \ipa{zɯ}) or relators (which must take a possessive prefix, as dative \ipa{ɯ-ɕki}, temporal and locative  \ipa{ɯ-qʰu} `after' etc). Indirective ditransitive verbs such as \ipa{kʰo} `to pass over', \ipa{tʰu} `to ask', \ipa{rŋo} `to lend' or \ipa{ti} `say' mark their recipient with the dative \ipa{-ɕki} or \ipa{-pʰe},\footnote{The preference for one or the other dative marker depends on the speaker.} as in example \refb{ex:tathu}. 


 \begin{exe}
   \ex   \label{ex:tathu}
 \gll \ipa{tɤ-pɤtso}  	\ipa{ra}  	\ipa{kɯ}  	\ipa{nɯ-sloχpɯn}  	\ipa{ɯ-ɕki}  	\ipa{to-tʰu-nɯ}  \\
\textsc{indef.poss}-child \textsc{pl} \textsc{erg} \textsc{3pl.poss}-teacher \textsc{3sg-dat} \textsc{ifr}-ask-\textsc{pl} \\
\glt The children asked their teacher. (Looking at the snow, 11)
   \end{exe}  

Secundative  ditransitive verbs (such as \ipa{mbi} `give') mark both the theme and the recipient in the absolutive.
 

\subsection{Indexation of arguments}

Japhug verbs have two conjugations, transitive and intransitive. The intransitive conjugation indexes the person and number (singular, dual, plural) of the S, while the transitive conjugation indexes the person and number of both A and P. The indexation of arguments on transitive verb follows a quasi-canonical direct-inverse system (see \citealt{jacques10inverse}, \citealt{jacques14inverse}). The person marking prefixes and suffixes of the intransitive conjugation can be combined with either direct marking (via stem alternation), inverse marking (the \ipa{wɣ-} prefix) or portmanteau prefixes (the local scenario markers \ipa{kɯ-} $2\rightarrow1$ and \ipa{ta-} $1\rightarrow2$).


In the case of ditransitive verbs as well, only two arguments can be indexed: either the theme (in the case of indirective verbs, such as \ipa{tʰu} `ask') or the recipient (in the case of secundative verbs, such as \ipa{mbi} `give') is treated as the P, while the the third argument receives no indexation on the verb. If a  speech act participant (first or second person, henceforth SAP) occurs as the P of an indirective verb, it cannot be interpreted as the recipient. Thus, sentence \refb{ex:tAGwthua} cannot be translated as `he asked me'.

 \begin{exe}
   \ex   \label{ex:tAGwthua}
 \gll
\ipa{tɤ́-wɣ-tʰu-a} \\
\textsc{pfv-inv}-ask-\textsc{1sg}\\
\glt He asked about me / he asked for my hand in marriage (elicited)   
      \end{exe}  
      

\subsection{Semi-transitive verbs}
  Japhug has a special sub-category of intransitive verbs with two absolutive arguments. Only one of these arguments is indexed on the verb, regardless of any person or animacy hierarchy, as illustrated by example \refb{ex:aroa1}.
  

 \begin{exe}
   \ex   \label{ex:aroa1}
 \gll 
\ipa{aʑo}  	\ipa{tɤ-rɟit}  	\ipa{χsɯm}  	\ipa{aro-a}   \\
I \textsc{indef.poss}-child three have:\textsc{fact}-\textsc{1sg} \\
 \glt   I have three children. (elicited)
   \end{exe} 

 
In example \refb{ex:aroa1}, agreement with the plural possessed would lead one to expect a form *\ipa{aro-a-nɯ} if the verb were morphologically transitive.
  

  These  \textit{semi-transitive} verbs   include the possessive verb \ipa{aro} `to possess',  experiencer verbs \ipa{rga} `to like', \ipa{sɤŋo} `to listen' and \ipa{ru} `to look at' and verbs such as \ipa{rmi} `to be called ...', \ipa{rʑaʁ} `to spend ... nights'. The possessor or the experiencer is indexed on the verb, while the possessee/stimulus is not. The verb \ipa{ru} `to look' is a special case, as the stimulus can be optionally marked with the dative \ipa{-ɕki}. 

This semi-transitive construction, which involves two unmarked arguments, only one of which is indexed on the verb, is superficially similar to the `bi-absolutive' construction found in Nakh-Daghestanian (\citealt{forker12biabsolutive}). However, the Japhug semi-transitive verbs differ from bi-absolutive verbs in Daghestanian languages in that this construction is not restricted to a specific TAM category. A better typological parallel is provided by the  \textsc{vaio} (intransitive animate verbs with object) found in Algonquian languages (cf  \citealt[242]{valentine01grammar}): although Algonquian languages do not have ergative flagging, \textsc{vaio} verbs behave like Japhug semi-transitive verbs in that they are conjugated intransitively and the person/number of one of the arguments is indexed on the verb, while the other is not.

Interestingly, there is some overlap between Japhug semi-transitive verbs and Algonquian \textsc{vaio} verbs: both include verbs of perception and verbs of possession. Further typological comparisons on this specific issue is deferred to future research, as the present article strictly focuses on Japhug.

The absolutive argument that is not indexed on the verb will be referred to as the `semi-object' in the following sections. Its  syntactic status will be discussed in detail.

\subsection{Are there subjects and objects in Japhug?}
The data presented in this section show that neither flagging nor indexation on the verb provide any evidence for positing `objects' or `subjects' in Japhug. However, the following sections explore this question from the point of view of syntactic pivots (\citealt[275]{vanvalin97syntax}), and show that the study of relativization offers critical evidence for positing the existence of subject and objects in Japhug.

\section{Relative clauses in Japhug}
Relative clauses in Japhug can be classified in two ways, depending on the place of the head noun and on the form of the subordinate verb. In this section, I briefly present  the general types of relative clauses in Japhug depending on the first criterion.  

 \citet[314]{dixon10basic2}  describes of the `Canonical Relative Clause Construction' as follows:
 
\begin{itemize}
\item It involves two clauses (a main clause and a relative clause) making up one sentence. 
\item These two clauses share an argument (the Common Argument). 
\item The relative clause is a modifier of the Common Argument. 
\item The relative clause must have a predicate and its core arguments. 
\end{itemize}
 
Such a definition allows for various types of relative clauses, including head-internal ones, but it excludes correlatives, non-restrictive relative clauses (which are in apposition to the NP, and thus not modifiers in the proper sense) and headless (or free) relative clauses.

In linguistic theories that  allow for the existence of empty elements, constructions such as \refb{ex:pWkAsat2} with a stand-alone nominalized verb can be viewed as a special sub-type of relative clauses whose head noun has been deleted (\citealt[197-205]{dryer07noun.phrase}). 


   \begin{exe}
\ex \label{ex:pWkAsat2}
\gll [$\emptyset_i$ \ipa{pɯ-kɤ-sat}]_{RC}  $\emptyset_i$ 	\ipa{nɯ}  	\ipa{kɤ-mto}  	\ipa{nɯ}  	\ipa{pɯ-rɲo-t-a.}  \\
{ }  \textsc{pfv-nmlz:P}-kill { } \topic{} \textsc{inf}-see \topic{} \textsc{pfv}-experience-\textsc{pst:tr-1sg} \\
\glt I have already seen ones that had been killed (of owls). (Owls, 20)
  \end{exe}

Whatever the merits and demerits of such an approach from a theoretical point of view, there is a practical advantage of treating such constructions as relative clauses in the particular case of Japhug: all canonical relative clauses in Japhug (except  finite relative clauses with relativized time or place adjunct, see section \ref{sec:finite-adjunct}) can be turned into headless relative clauses by removing the head noun. Headless relative clauses are well-attested in many other languages of the Sino-Tibetan family (\citealt[128-9]{genetti08nmlz}), and appear to be relatively common in text corpora in these languages.

\citet[227]{coupe07mongsen} points out that head-internal relative clauses can be distinguished from other uses of nominalized clause by the fact that the deleted head can always be recovered. For instance, in example \refb{ex:pWkAsat2}, restoring the deleted head noun \ipa{pɣɤkʰɯ} `owl' is possible in either one of the two slots indicated by $\emptyset$.

The head deletion analysis of headless relative clauses however raises an issue in the case of Japhug:  when A or P arguments are relativized either head-internal or prenominal relative clauses are attested (see for instance examples \refb{ex:akanwrga1} and \refb{ex:akanwrga2}. If head-internal relative clauses are analyzed as having a gap corresponding the common argument (or adjunct), it is not obvious which, of the relative-internal or the post-relative gap, should be considered to be the head of the relative (see example \ref{ex:akanwrga3}).

     \begin{exe}
   \ex \label{ex:akanwrga1}
 \gll [\ipa{aʑo}  	\textbf{\ipa{tɯ-skɤt}}_i\tete{}	\ipa{stu}  	\ipa{a-kɤ-nɯ-rga}]\rc{}   $\emptyset_i$ 	\ipa{nɯ}  	\ipa{kɯrɯ-skɤt}  	\ipa{ŋu}  \\
I  \textsc{indef.poss}-language most \textsc{1sg-nmlz:P-appl}-like  { } \textsc{dem} Rgyalrong-language be:\textsc{fact}\\
\ex \label{ex:akanwrga2}
\gll [\ipa{aʑo}  	$\emptyset_i$ \ipa{stu}  	\ipa{a-kɤ-nɯ-rga}]\rc{}  	\textbf{\ipa{tɯ-skɤt}}_i\tete{}	 	\ipa{nɯ}  	\ipa{kɯrɯ-skɤt}  	\ipa{ŋu}  \\
I { } most \textsc{1sg-nmlz:P-appl}-like \textsc{indef.poss}-language \textsc{dem} Rgyalrong-language be:\textsc{fact} \\
\glt Rgyalrong is my favourite language. (elicited)
\ex \label{ex:akanwrga3}
\gll [\ipa{aʑo}  	$\emptyset_i$\textsuperscript{\textsc{head}?}  \ipa{stu}  	\ipa{a-kɤ-nɯ-rga}]\rc{}  	  	$\emptyset_i$\textsuperscript{\textsc{head}?}  \ipa{nɯ}  	\ipa{kɯrɯ-skɤt}  	\ipa{ŋu}  \\
I { } most \textsc{1sg-nmlz:P-appl}-like  { } \textsc{dem} Rgyalrong-language be:\textsc{fact} \\
\glt My favourite is Rgyalrong. (elicited)
\end{exe}

For this reason, I do not indicate the deleted head in the examples of head-internal relative clauses in this article. 

Post-nominal relative clauses have been described for some languages of the Sino-Tibetan family (see \citealt[130]{genetti08nmlz}). In the Japhug corpus however, all examples of relative clauses following the head noun examined so far can be interpreted as head-internal relative clauses.

\section{Non-finite relative clauses} \label{sec:nonfinite}
Non-finite relative clauses in Japhug have verbs in their participial form, and are formally distinct from independent main clause, which require a verb in finite form. All arguments and adjuncts that can be relativized by finite relative clauses (section \ref{sec:finite}) can also alternatively be relativized with non-finite relative clauses, but the reverse is not true.

In this section, I first present the morphology of the three participles attested in Japhug. Then, I discuss all types of non-finite relative clauses, classified by the syntactic function of the relativized element, including core arguments, possessor of arguments and oblique arguments or adjuncts.



\subsection{Participles}
Japhug verbs have a rich array of nominalized forms. Three nominalized forms are distinct from the rest in that they preserve some verbal characteristics: they can serve as predicates of subordinate clauses, take TAM or associated motion marking, and I thus refer to them as `participles'. 

Participles differ from finite verbs in three regards. First, they cannot serve as the predicate of a main clause. Second, they cannot take the personal prefixes and suffixes of intransitive and transitive conjugation (including direct/inverse marking); rather, in some cases they take a possessive prefix coreferent with one the arguments of the participle. It is not possible to index more than one argument on a participial form. Third, there are restrictions on the TAM marking on these verbs.

There are three participles in Japhug, the S/A participle in \ipa{kɯ-}, the P participle in \ipa{kɤ-} and the oblique participle in \ipa{sɤ-}. Examples \refb{ex:die} to \refb{ex:come} illustrate their basic functions. Their uses in building relative clauses are studied in the following sections. Their other functions (in clause linking or complementation) are not discussed in the present article (see \citealt{jacques14linking}).

The S/A participle refers to the S (in the case of an intransitive verb, example \ref{ex:die}) or the A (in transitive verbs, \ref{ex:kill}). In the case of transitive verbs, a possessive prefix coreferent with the P is obligatory when no overt NP corresponding to the P is present, and when no other prefix is added to the participle.

 \begin{exe}
\ex \label{ex:die}
\gll \ipa{kɯ-si}    \\
  \textsc{nmlz}:S/A-die \\
 \glt  `The dead one'
\end{exe}

 \begin{exe} 
\ex \label{ex:kill}
\gll \ipa{ɯ-kɯ-sat}    \\
  \textsc{3sg}-\textsc{nmlz}:S/A-kill \\
 \glt  `The one who kills him.'
\end{exe}

The P-participle corresponds to the P-argument. This form is homophonous with the infinitive.\footnote{The infinitive is not discussed in the present article, as it is irrelevant to the study of relativization.}

 \begin{exe} 
\ex \label{ex:kill2}
\gll \ipa{kɤ-sat}    \\
   \textsc{nmlz}:P-kill \\
 \glt  `The one that is killed.'
 \end{exe}
 
It can appear with an optional possessive prefix coreferent to the agent as in \refb{ex:kill3}.
  
  \begin{exe}
\ex \label{ex:kill3}
\gll \ipa{a-kɤ-sat}    \\
   \textsc{1sg-nmlz}:P-kill \\
 \glt  `The one that I kill.'
 \end{exe}

The \ipa{sɤ}-prefix (and its allomorphs \ipa{sɤz}- and \ipa{z}-) is used for non-core argument nominalization, in particular   recipient of indirective verbs, instruments, place and time. It takes a possessive prefix  which can be coreferent with S, A or P.

   \begin{exe}
\ex \label{ex:come}
\gll \ipa{ɯ-sɤ-ɣi}    \\
   \textsc{3sg-nmlz}:S-come \\
 \glt  `The place/moment where/when it comes.'
 \end{exe}
 
More complex participial forms, including negative, associated motion or TAM prefixes are also possible, as shown by example \refb{ex:WGWjAkWqru}.

 \begin{exe}
\ex \label{ex:WGWjAkWqru}
\gll
  	\ipa{ɯ-ɣɯ-jɤ-kɯ-qru}  	\ipa{tɤ-tɕɯ}  	   \\
  \textsc{3sg-cisloc-pfv-nmlz:}S/A-meet \textsc{indef.poss}-boy   \\
\glt The boy who had come to look for her. (The three sisters 231)
 \end{exe}

Table \ref{tab:template.nmlz} summarizes the template of participial verb forms.

\begin{table}[h]
\caption{The template of participial verb forms in Japhug} \centering \label{tab:template.nmlz}
\resizebox{\columnwidth}{!}{
\begin{tabular}{lllllll}
\toprule
-5 & -4&-3 &-2&-1&$\Sigma$\\
possessive & negative&associated   & TAM & participle prefix &enlarged  \\
prefix & prefix &motion prefix  &directional&&stem\\
\bottomrule
\end{tabular}}
\end{table}


\subsection{S and A Relativization}
The only way to relativize either S or A arguments in Japhug is by a non-finite relative with a verb in its \ipa{kɯ-} participle form. 

Relativization of the S is most often expressed by a head-internal relative. Since Japhug has strict verb final word order,  the participial verb follows the head noun as in \refb{ex:tchi}.

 \begin{exe}
   \ex   \label{ex:tchi}
 \gll  	\ipa{ɯ-ɣmbaj}  	\ipa{zɯ}  	[\ipa{tɕʰi}\tete{}  	\ipa{tu-kɯ-ndɯ}]\rc{}  	\ipa{ci}  	\ipa{pɯ-tu}  	\ipa{ɲɯ-ŋu}  		\\
\textsc{3sg.poss}-side \textsc{loc} ladder \textsc{ipfv-nmlz:S}-be.built  \textsc{indef} \textsc{pst.ipfv}-exist \textsc{sens}-be  \\
 \glt    There was a ladder which was leaning on the side (of the tower). (Slobdpon, 55)
   \end{exe} 

Adjectives are a subclass of stative verbs in Japhug,\footnote{They differ from other stative verbs in that they can take the tropative derivation, see \citet{jacques13tropative}. } and can only be used as noun modifiers in participial form, as in \refb{ex:mazw}. All attributive adjectives in Japhug thus form a head-internal relative.

 \begin{exe}
   \ex   \label{ex:mazw}
 \gll 
\ipa{nɯnɯ}  	\ipa{li}  	[\textbf{\ipa{smɤn}}\tete{}   	\ipa{mɤʑɯ}  	\ipa{kɯ-pe}]\rc{}  	\ipa{ɲɯ-ŋu.}  \\
\textsc{dem} again medicine not.only \textsc{nmlz:S}-good \textsc{sens}-be \\
 \glt    This is an even better medicine. (21, pri, 85)
   \end{exe} 

Prenominal relative clauses with relativized S are also possible, as in example  \refb{ex:mANidpon}, but they are more restricted and more commonly occur in the case of very long relative clauses or when several relative clauses share the same head noun.

\begin{exe}
   \ex  \label{ex:mANidpon}
\gll [[\ipa{mɤŋi}  	\ipa{kɤ-kɯ-ɣe}]\rc{}  	\ipa{χpɯn}\tete{}  	\ipa{tʰɯ-kɯ-rgɤz}]\rc{}  	\ipa{ci}  	\ipa{pjɤ-tu}  	\ipa{tɕe,}    	\\
Mangi \textsc{pfv:east-nmlz:S}-come[II] monk \textsc{pfv-nmlz:S}-old \textsc{indef} \textsc{ifr.ipfv}-exist   \textsc{lnk} \\
 \glt  There was an old monk who had come from Mangi. (08 kWqhi, 19)
   \end{exe} 


When the relativized element is the A, prenominal relative clauses such as \refb{ex:wkwtshi} are more common.

\begin{exe}
   \ex  \label{ex:wkwtshi}
\gll [\ipa{tɯ-nɯ}  	\ipa{ɯ-kɯ-tsʰi}]\rc{}  	\ipa{tɤ-pɤtso}\tete{}  	\ipa{ɣɯ}  	\ipa{ɯ-kɯ-mŋɤm}  	\ipa{ɲɯ-ŋu}  \\
\textsc{indef.poss}-breast \textsc{3sg-nmlz:A}-drink \textsc{indef.poss}-child \textsc{gen} \textsc{3sg.poss-nmlz:S}-be.painful \textsc{sens}-be \\
\glt It is a disease of children who drink milk from the breast. (25 kACAl, 61)
\end{exe}


Head-internal relative clauses in this case are relatively rare in the corpus. As shown by example  \refb{ex:WkWnWmbrApW}, the relativized A keeps ergative marking \ipa{kɯ-} in head-internal relative clauses.

\begin{exe}
   \ex  \label{ex:WkWnWmbrApW}
\gll [[\ipa{tɤ-pɤtso}  	\ipa{ci}  	\ipa{kɯ}]\tete{}  	<yangma> 	\ipa{ɯ-kɯ-nɯmbrɤpɯ}]\rc{}  	\ipa{ci}  	\ipa{jɤ-ɣe}  \\
\textsc{indef.poss}-child \textsc{indef} \textsc{erg} bicycle \textsc{3sg-nmlz:A}-ride \textsc{indef} \textsc{pfv}-come[II] \\
\glt A boy who was riding a bicycle arrived. (Pear story, Chenzhen, 5)
\end{exe}


As pointed out by \citet{jackson03caodeng} and \citet{jackson06guanxiju} concerning Tshobdun, a close relative of Japhug, the fact that both S and A arguments are relativized by the same constructions -- prenominal or head-internal participial relative clauses in \ipa{kɯ-} -- suggests  the existence of a `subject' pivot. This question is discussed in more detail in section \ref{sec:subject.object}.


\subsection{Possessor Relativization}

When possessors are relativized, the possessed noun remains \textit{in situ} and the verb are nominalized with the prefix \ipa{kɯ}- like S and A arguments. A resumptive possessive prefix on the possessed noun is obligatory whether the possessor is overt (as in \ref{ex:WRrWkWtu}) or not (\ref{ex:lrWba}).

      \begin{exe}
   \ex \label{ex:WRrWkWtu}
 \gll 
\ipa{akɯ}   	\ipa{zɯ}   	[\ipa{qapri}\tete{}   	\ipa{ci}   	\ipa{ɯ}\tete{}-\ipa{kɤχcɤl}  	\ipa{ɯ}\tete{}-\ipa{ʁrɯ}   	\ipa{kɯ-tu}]\rc{}   	\ipa{ci}   	\ipa{ɣɤʑu}   	\ipa{tɕe,}   \\
east \textsc{loc} snake \textsc{indef} \textsc{3sg.poss}-middle.of.the.head  \textsc{3sg.poss}-horn \textsc{nmlz:S}-exist \textsc{indef} exist:\textsc{sensory}  \textsc{lnk} \\
\glt In the east, there is a snake with a horn in the middle of his head.  (The divination, 43)
\end{exe}
 
       \begin{exe}
   \ex \label{ex:lrWba}
 \gll 
[\ipa{iɕqʰa}   	 \ipa{nɯ}\tete{}-\ipa{me}   	\ipa{lʁɯba}   	\ipa{kɯ-ŋu}]   	\ipa{ra}   	\ipa{ɣɯ}   	\ipa{nɯ-kʰɤru}   	\ipa{lɤ-nɯ-ɬoʁ,}   \\
the.aforementioned \textsc{3pl.poss}-daughter mute \textsc{nmlz:S}-be \textsc{pl} \textsc{gen} \textsc{3pl.poss}-kitchen.door \textsc{pfv:upstream-auto}-come.out \\
\glt  As he entered the  door of the kitchen of those whose daughter was mute.  (The divination2, 55)
\end{exe}
 
When the possessor is   first or second person,  the resumptive possessive prefixes  are not neutralized to third person (see example \ref{ex:kWtshoz}).

        \begin{exe}
   \ex \label{ex:kWtshoz}
 \gll 
[\ipa{nɤ}\tete{}-\ipa{mu}   	\ipa{nɤ}\tete{}-\ipa{wa}   	\ipa{kɯ-tsʰoz}]\rc{}   	\ipa{tɯ-ŋu,}   	\ipa{aʑo}   	[\ipa{a}\tete{}-\ipa{mu}   	\ipa{kɯ-me}]\rc{}   	\ipa{ŋu-a}   	\ipa{tɕe}    \\
\textsc{2sg.poss}-mother \textsc{2sg.poss}-father \textsc{nmlz:S}-complete 2-be:{fact} I \textsc{1sg.poss}-mother \textsc{nmlz:S}-not.exist be:{fact}-\textsc{1sg} \textsc{lnk} \\
\glt You are someone whose father and mother are all there, I am someone without a mother. (Nyima wodzer, 12)
\end{exe}

However, unlike S and A arguments, possessors cannot be relativized by prenominal or headless non-finite relative clauses; only head-internal relative clauses are possible.

\subsection{P Relativization}
Unlike S, A and possessors, P arguments can be relativized with either finite or non-finite relative clauses. The first type is discussed in section \ref{sec:finite.P}. The present section describes non-finite relativization of P arguments.

When the relativized participant is the P, both head-internal (\ref{ex:nandzwt}), prenominal (\ref{ex:tajmag}) or headless relative clauses are possible. In the case of non-finite relativization, the verb is in the P-participle form with a \ipa{kɤ-} prefix.

\begin{exe}
   \ex \label{ex:nandzwt}
\gll   [\ipa{tɯrme}\tete{}  	\ipa{mɤ-kɤ-nɯfse}]\rc{}  	\ipa{jɤ-ɣe}  	\ipa{tɕe}  	\ipa{tu-nɯ-ɤndzɯt}\\
person \textsc{neg-nmzl:P}-know \textsc{pfv}-come[II] \textsc{lnk} \textsc{ipfv-appl}-bark\\
  \glt  When an unknown person  comes, it barks at him.  (05 khWna, 9)
   \end{exe}  
   
\begin{exe}
   \ex \label{ex:tajmag}
   \gll
[\ipa{aʑo}  	\ipa{a-mɤ-kɤ-sɯz}]\rc{}  	\ipa{tɤjmɤɣ}\tete{}  	\ipa{nɯ}  	\ipa{kɤ-ndza}  	\ipa{mɤ-naz-a}  \\
\textsc{1sg} \textsc{1sg-neg-nmlz:P}-know mushroom \textsc{dem} \textsc{inf}-eat \textsc{neg}-dare:\textsc{fact}-\textsc{1sg} \\
\glt I do not dare to eat the mushrooms that I do not know. (23 mbrAZim,103)
\end{exe}
 

Unlike in Tshobdun (\citealt[10]{jacksonlin07}), in Japhug non-finite relative clauses with possessive prefixes are not restricted to a generic state of affairs, but can refer to particular situations as in the pseudo-cleft in \refb{ex:khu}.

     \begin{exe}
   \ex \label{ex:khu}
   \gll  \ipa{lɤ-fsoʁ}  	\ipa{ɯ-jɯja}  	\ipa{nɯ}  	\ipa{pjɯ-ru}  	\ipa{tɕe}  	[\ipa{ɯ-kɤ-nɯmbrɤpɯ}]\rc{}  	\ipa{nɯ}  	\ipa{kʰu}  	\ipa{pɯ-ɕti}  	\ipa{ɲɯ-ŋu,}  \\
\textsc{pfv}-be.clear    \textsc{3sg}-along  \textsc{dem} \textsc{ipfv:down}-look \textsc{lnk} \textsc{3sg-nmlz:P}-ride \topic{} tiger \textsc{pst.ipfv}-be.\textsc{assert}  \textsc{sens}-be \\
\glt As the day was breaking, looking down, he (progressively realized that) what he was riding was a tiger. (Tiger, 20)
\end{exe}


\subsection{T Relativization}
As mentioned in section \ref{sec:flagging.indexation}, both indirective and secundative verbs are found in Japhug. By definition, the theme of indirective verbs is treated as the P, and need not be discussed in this section.

The theme of secundative verbs like \ipa{mbi} `give' does not receive any flagging; it differs from the P-argument of monotransitive verbs in that there is no indexing on the verb of its person/number. However, the theme can be relativized by a non-finite relative (either prenominal or head-internal) with a P-participle in \ipa{kɤ}-  (example \ref{ex:nWkAmbi}) exactly like the P argument of a monotransitive verb. 

\begin{exe}
\ex \label{ex:nWkAmbi}
\gll      \ipa{tɕe} 	[\ipa{ɬamu} 	\ipa{kɯ} 	\ipa{qajɣi}\tete{} 	\ipa{nɯ-kɤ-mbi}]\rc{} 	\ipa{nɯ} 	\ipa{tu-ndze} 	\ipa{pjɤ-ŋu.}   \\
\textsc{lnk} Lhamo \textsc{erg} bread \textsc{pfv-nmlz}:P-give \topic{} \textsc{ipfv}-eat[III] \textsc{ipfv.ifr}-be  \\
 \glt    He was eating the bread that Lhamo had given him. (The Raven, 111)
\end{exe} 

For secundative verbs, non-finite relative clauses with a P-participle are thus ambiguous: from the form of the verb, it is impossible to determine whether the relativized participant is the theme or the recipient. The form \ipa{nɯ-kɤ-mbi} \textsc{pfv-nmlz}:P-give can thus mean either `(the thing) X has given him' or `(the person) whom X has given it to'.

\subsection{Semi-transitive verbs}
In the case of semi-transitive verbs, the absolutive argument that is indexed on the verb is relativized like a normal S with a non-finite relative with a \ipa{kɯ-} participle as in example \refb{ex:kWrga}, with the verb \ipa{rga} `like', whose experiencer is treated as the S, and the stimulus is the semi-object.

 \begin{exe}
   \ex   \label{ex:kWrga}
 \gll  [\textbf{\ipa{tɯrme}}\tete{}  	\ipa{kɯ-rga,}]\rc{}  	[\ipa{wuma}  	\ipa{ʑo}  	\ipa{kɤ-ndza}  	\ipa{kɯ-rga}]\rc{}  	\ipa{ɣɤʑu.} \\
person \textsc{nmlz:S}-like very \textsc{emph} \textsc{inf}-eat  \textsc{nmlz:S}-like exist:\textsc{sensory} \\
 \glt  There are persons who like it, who like to eat it. (22  BlamajmAG, 62)
   \end{exe} 
   
   On the other hand, the semi-object can be relativized by a non-finite (prenominal or head-internal) relative with a verb in \ipa{kɤ-} prefixed P-participle form, as in example \refb{ex:nWkArga} (a pseudo-cleft construction). The (optional) possessive prefix is coreferent here with the S.

\begin{exe}
   \ex   \label{ex:nWkArga}  
\gll [\ipa{pɣa}  	\ipa{ra}  	\ipa{nɯ-kɤ-rga}]  	\ipa{nɯ}  	\ipa{qaj}  	\ipa{ntsɯ}  	\ipa{ŋu}  \\
bird \textsc{pl} \textsc{3pl-nmlz:P}-like \topic{} wheat always be:\textsc{fact} \\
\glt (The food) that birds like is always wheat (not barley). (23 pGAYaR, 29)
   \end{exe} 

Thus, while semi-transitive verbs are  treated as intransitive verbs from the point of view of indexation and flagging, their semi-object can be relativized with the same construction as the P of a transitive verb.

\subsection{Oblique Relativization}
Other arguments and adjuncts, when they can be relativized in a non-finite relative clause,\footnote{Not all participants can be relativized in Japhug. In particular, the standard of comparison, which lies at the lower end of \citet{keenan77accessibility}'s accessibility hierarchy, cannot be relativized in this language.} require the use of a non-finite relative with a verb in the oblique participle in \ipa{sɤ-}.

This includes the recipient of indirective verbs (but not secundative verbs) as in \refb{ex:WsAfCAt}, comitative arguments in \ipa{cʰo} `with' (\ref{ex:WsAmWmi}), time adjuncts (\ref{ex:WsAji}), place adjuncts (\ref{ex:asAGi}) and instruments (\ref{ex:sAxtCAr}). Note that goals of motion / manipulation verbs can alternatively be relativized with the finite verb construction, see section \ref{sec:finite-adjunct}.

\begin{exe}
\ex \label{ex:WsAfCAt}
\gll
[\ipa{ɯ-sɤ-fɕɤt}]\rc{} 
\ipa{pjɤ-me} 	\ipa{qʰe} 	\ipa{tɕe} 	\ipa{tɤ-pɤtso} 	\ipa{ɯ-ɕki} 	\ipa{nɯ} 	\ipa{tɕu} 	\ipa{nɯra} 	\ipa{tɕʰi} 	\ipa{pɯ-kɯ-fse} 	\ipa{nɯra} 	\ipa{pjɤ-fɕɤt.} \\
\textsc{3sg-nmlz:oblique}-tell \textsc{ipfv.ifr}-not.exist \textsc{lnk} \textsc{lnk} \textsc{indef.poss}-child \textsc{3sg-dat} \textsc{dem} \textsc{loc} \textsc{dem:pl} what \textsc{pst-nmlz:S}-be.like  \textsc{dem:pl} \textsc{ifr}-tell \\
\glt She had no one (else) to tell it to, so she told the boy everything that had happened. (140515 congming de wusui xiaohai, 77)
\end{exe} 

\begin{exe}
   \ex \label{ex:WsAmWmi}
 \gll 
\ipa{tɕe}   	[\ipa{ɯʑo}   	\ipa{ɯ-sɤ-ɤmɯmi}]\rc{}   	\ipa{nɯ}   	\ipa{dɤn}   	\ipa{ma}   	\ipa{ca}   	\ipa{kɯ-fse}   	\ipa{qaʑo}   	\ipa{kɯ-fse,}   	\ipa{tsʰɤt}   	\ipa{kɯ-fse,}   	 \ipa{ɯʑo}   	\ipa{cʰo}   	\ipa{kɯ-naχtɕɯɣ}   	\ipa{sɯjno,}   	\ipa{xɕaj}   	\ipa{ma}   	\ipa{mɤ-kɯ-ndza}   	\ipa{nɯ} \ipa{ra}   	\ipa{cʰo}   	\ipa{nɯ}   	\ipa{amɯmi-nɯ}   	\ipa{tɕe,}   \\
\textsc{lnk} it \textsc{3sg-nmlz:oblique}-be.in.good.terms \topic{} be.many:\textsc{fact} because water.deer \textsc{nmlz:S}-be.like sheep \textsc{nmlz:S}-be.like goat  \textsc{nmlz:S}-be.like it with  \textsc{nmlz:S}-be.identical herbs grass apart.from \textsc{neg-nmlz:A}-eat \textsc{dem} \textsc{pl} with \textsc{dem} be.in.good.term:\textsc{fact}-\textsc{pl} \textsc{lnk} \\
\glt The (animals) that are in good terms with the rabbit are many, it is in good terms with those that only eat grass, like water deer, sheep or goats. (04 qala1, 33-4)
\end{exe}

\begin{exe}
   \ex \label{ex:WsAji}
   \gll
   \ipa{tɕe} 	\ipa{nɯnɯ} 	\ipa{ʑaka} 	[\ipa{ɯ-sɤ-ji}]\rc{} 	\ipa{ɲɯ-ŋu} 	\ipa{tɕe}\\
   \textsc{lnk} \textsc{dem} each \textsc{3sg-nmlz:oblique}-plant \textsc{sens}-be \textsc{lnk}\\
\glt These are the (periods) when people plant each of these (crops). (15 tChWma, 19)
\end{exe}

\begin{exe}
   \ex \label{ex:asAGi}
 \gll
\ipa{kɯki}   	\ipa{tɯ-ci}   	\ipa{ki}   	\ipa{ɯ-tɯ-rnaʁ}   	\ipa{mɯ́j-rtaʁ}   	\ipa{tɕe,}   	\ipa{aʑo}   	[\ipa{a-sɤ-ɣi}]\rc{}   	\ipa{mɯ́j-kʰɯ}   \\
this \textsc{indef.poss}-water this \textsc{3sg-nmlz:degree}-deep \textsc{neg:sens}-deep \textsc{lnk} I \textsc{1sg-nmlz:oblique}-come \textsc{neg:sens}-be.able \\
\glt The water is not deep enough, there is not (enough) place for me to come. (Go by yourself,4)
\end{exe}


 \begin{exe}
  \ex  \label{ex:sAxtCAr}  
  \gll [\ipa{nɯ-mtʰɤɣ}  	\ipa{sɤ-xtɕɤr}]  	\ipa{xɕɤfsa}  	\ipa{ma}  	\ipa{pjɤ-me}  \\
\textsc{3pl.poss}-waist \textsc{nmlz:oblique}-tie thread apart.from \textsc{ifr.ipfv}-not.exist \\
\glt They only had threads to tie their waists (the only things that they could use to tie their waists were threads). (Milaraspa translation)
   \end{exe} 

The fact that instruments are relativized with the oblique participle, rather than with the S/A participle is significant, as the instrument receives the same ergative marker \ipa{kɯ} as A arguments.\footnote{It differs from A arguments in that it is not indexed on the verb and by the fact that the verb optionally takes a causative prefix as in \refb{ex:kuwGsWxtCAr}  when an instrumental adjunct is added.}

 \begin{exe}
  \ex   \label{ex:kuwGsWxtCAr}  
\gll \ipa{ɯnɯnɯ}  	\ipa{ri}  	\ipa{qase}  	\ipa{kɯ}  	\ipa{kú-wɣ-sɯ-xtɕɤr}  \\
\textsc{dem} \textsc{loc} leather.rope \textsc{erg} \textsc{ipfv-inv-caus}-tie \\
\glt There, one ties it with a leather rope. (24 mbGo, 97)
   \end{exe} 
   

\section{Finite relative clauses} \label{sec:finite}
Finite relative clauses differ from non-finite ones in that the main verb of the relative is not in participial form, but takes full person and TAM marking, and no nominalization prefix.

Like non-finite relative clauses, finite relative clauses can be either prenominal, head-internal or headless, but they are available for a much more limited range of participants: P arguments, semi-objects, T of ditransitive verbs and some adjuncts. 

Although non-finite relative clauses are generally identical to the corresponding independent clause, they do present some subtle differences that are described in this section.

\subsection{Simple finite relative} \label{sec:finite.P}
Simple finite relative clauses are those head-internal, headless or prenominal finite relative clauses that cannot take a  relator noun.\footnote{Relator nouns differ from normal head nouns in that they have an obligatory third person singular possessive prefix \ipa{ɯ-}. They only occur in prenominal relative clauses.}  

P arguments can be relativized with simple finite relative clauses as in \refb{ex:tutianw}.\footnote{The P of \ipa{ti} `to say' is never the addressee, it always refers to the words that are said.} 

     \begin{exe}
   \ex \label{ex:tutianw}
 \gll [\ipa{nɯ}  	\ipa{qajɯ}\tete{}  	\ipa{kɯ-ɲaʁ}  	\ipa{tu-ti-a}]\rc{}  	\ipa{nɯ}  	\ipa{nɯ}  	\ipa{kɯ-fse}  	\ipa{ɲɯ-βze}  	\ipa{ɲɯ-ŋu}  \\
\textsc{dem} worm \textsc{nmlz:S}-black \textsc{ipfv}-say-\textsc{1sg} \topic{} \textsc{dem} \textsc{nmlz:S}-be.like \textsc{ipfv}-grow \textsc{sens}-be \\
\glt The black worm that I was talking about grows like that. (28 kWpAz, 30)
\end{exe}

Since there is a restriction against combining a TAM prefix with a possessive prefix coreferent with the A in \ipa{kɤ-} P-participles (a form such as **\ipa{a-tu-kɤ-ti} \textsc{1sg-ipfv-nmlz:P}-say is not possible), the only way to specify both the A and the TAM on the verb when relativizing the P is to use a finite relative instead of a non-finite one.


In addition, both the R and the T of secundative ditransitive verbs are treated as the P of monotransitive verbs, and can be relativized with a finite relative as in \refb{ex:nWGmbia}.

       \begin{exe}
   \ex \label{ex:nWGmbia}
   \gll [\ipa{aʑo} 	\ipa{nɯ́-wɣ-mbi-a}]\rc{} 	\ipa{maka} 	\ipa{ʑo} 	\ipa{me} \\
 \textsc{1sg}  \textsc{pfv-inv}-give-\textsc{1sg} at.all \textsc{emph} not.exist:\textsc{fact} \\
\glt  He did not give me anything. (140430 yufu he tade qizi, 48)
\end{exe}

Semi-objects can also be relativized with the same construction, as in  \refb{ex:aroa}. Such examples, however, are not found in the corpus and can only be elicited.
 \begin{exe}
   \ex   \label{ex:aroa}  
\gll
[\ipa{aʑo}  	\ipa{qaʑo}\tete{}  	\ipa{aro-a}]\rc{}  	\ipa{nɯ} \ipa{ra}  	\ipa{kɯki}  	\ipa{ŋu}  \\
I sheep possess:\textsc{fact}-\textsc{1sg} \topic{} \textsc{pl} \textsc{dem.prox} be:\textsc{fact} \\
\glt The sheep which I own are these ones. (elicitation)
   \end{exe} 


\subsection{Inverse marking}
Finite relative clauses are attested with inverse marking in Japhug as in example \refb{ex:nWwGmbi}, a fact that distinguishes Japhug from Tshobdun, where such sentences are reportedly ungrammatical  (\citealt{jacksonlin07}).\footnote{Direct/inverse marking is not possible in participial verb forms, so that there is no equivalent to such sentences in non-finite relative clauses.}

\begin{exe}
\ex \label{ex:nWwGmbi}
\gll
[\ipa{tɤ-wɯ} 	\ipa{kɯ} 	\ipa{ʑmbrɯ}\tete{} 	\ipa{nɯ́-wɣ-mbi}]\rc{} 	\ipa{nɯ} 	 	\ipa{cʰɤ-lɤt} \\
\textsc{indef.poss}-grandfather \textsc{erg} boat \textsc{pfv-inv}-give \textsc{dem} \textsc{ifr}-throw \\
\glt He took the boat that the old man had given him. (140430 jin e, 245)
\end{exe}

In \refb{ex:nWwGmbi}, the relativized element is the theme of the verb `give' \ipa{mbi}. We saw in the previous sections that although \ipa{mbi} is a secundative verb, whose R  is treated as the P of a monotransitive verb in the verbal morphology, both  T  and R can be relativized using the same constructions.

The presence of direct vs inverse morphology on the verb of the relative has no influence on the syntactic pivot of the construction: whether the verb takes the prefix \ipa{-wɣ-} or not, only the theme or the recipient can be relativized in such construction, never the A. In this regard, Japhug radically differs from languages such as Movima (\citealt[526-7]{haude09hierarchical}) which display a strict syntactic pivot: direct is used when the relativized participant is the P, while inverse appears when it is the A.


\subsection{Other finite relative clauses} \label{sec:finite-adjunct}
The goal of motion verbs and manipulation verbs like \ipa{ɕe} `go' and \ipa{tsɯm} `take away' can be relativized with finite relative clauses, as in examples \refb{ex:jowGtsWmnW} and \refb{ex:jaria.nW}. Although the goal is not indexed in verb morphology, it is nevertheless included in the verb argument structure.

     \begin{exe}
   \ex \label{ex:jowGtsWmnW}
 \gll
[\ipa{\textbf{kʰa}}\tete{}  	\ipa{jɤ́-wɣ-tsɯm-nɯ}]_{\textsc{rc}}  	\ipa{nɯnɯ,}  	\ipa{lonba}  	[\ipa{ɕom}  	\ipa{kɯ}  	\ipa{nɯ-kɤ-sɯ-βzu}]_{\textsc{rc}}  	\ipa{\textbf{kʰa}}  	\ipa{pjɤ-ŋu}  \\
house \textsc{pfv-inv}-take.away-\textsc{pl} \textsc{dem} all iron \textsc{erg} \textsc{pfv-nmlz:P-caus}-make house \textsc{ipfv.ifr}-be \\
\glt The house to which he had taken them, it was a house made of iron.(140505 liuhaohan zoubian tianxia, 148)
\end{exe}


     \begin{exe}
   \ex \label{ex:jaria.nW}
 \gll
[\ipa{aʑo}  	\ipa{sɤtɕha}\tete{}  	\ipa{jɤ-ari-a}]_{\textsc{rc}}  	\ipa{nɯ}  	\ipa{nɯnɯ}  	\ipa{kɯ-ɤrqhi}  	\ipa{ci}  	\ipa{ŋu.}  \\
\textsc{1sg} place \textsc{pfv}-go[II]-\textsc{1sg} \textsc{dem} \textsc{dem} \textsc{nmlz}:S/A-be.far \textsc{indef} be:\textsc{fact} \\
\glt The place where I have gone is far away. (elicited)
\end{exe}


In addition, time and place adjuncts can be relativized with prenominal finite relative clauses with  relator head nouns such as \ipa{ɯ-raŋ} `time', \ipa{ɯ-sŋi} `the day' or \ipa{ɯ-sta} `the place', as in examples  \refb{ex:WraN2} and \refb{ex:Wsta2}.

\begin{exe}
   \ex \label{ex:WraN2}
 \gll [\ipa{nɤ-ɕɣa}   	\ipa{xtɕi}]\rc{}   	\ipa{ɯ-raŋ}\tete{}   	\ipa{ri}   	\ipa{nɯ}   	\ipa{tɯ́-wɣ-nɤzda}   	\ipa{ŋu}   	\ipa{ri}   \\
 \textsc{2sg.poss}-tooth be.small:\textsc{fact} \textsc{3sg.poss}-time \textsc{loc} \topic{} 2-\textsc{inv}-be.with:\textsc{fact} be:\textsc{fact} but \\
\glt While you are young, she will be with you. (Slobdpon2, 60)
\end{exe}

\begin{exe}
   \ex \label{ex:Wsta2}
 \gll
\ipa{iɕqʰa,}  	[\ipa{pɯ-nɤŋkɯŋke}]  	\ipa{ɯ-sta}  	\ipa{nɯ} \ipa{ra} \ipa{rcanɯ} 	\ipa{tɯ-ɕnaβ}  	\ipa{pɯ-kɤ-βde}  	\ipa{ʑo}  	\ipa{fse,}  \\
the.aforementioned \textsc{pst.ipfv}-walk.around \textsc{3sg.poss}-place \topic{} \textsc{pl} \textsc{unexpected} \textsc{indef.poss}-snot \textsc{pfv:down-nmlz:P}-throw.away \textsc{emph} be.like:\textsc{fact} \\
\glt The places where it has been look like spilled snot. (26 qro, 138)
\end{exe}

This construction differs from simple finite relative clauses in two ways. First, only prenominal relativization is possible. Second, the  head noun has an obligatory third person possessive prefix \ipa{ɯ-}, which is impossible in the case of a prenominal simple finite relative.

\subsection{Nominalized status of finite relative clauses}
All relative clauses in Japhug, apart from correlative clauses, are instances of clausal nominalization, a situation extremely common across Sino-Tibetan languages (see in particular \citealt{genetti08nmlz} and \citealt{bickel99nmlz}).

As we have seen in the previous sections, relative clauses in Japhug can either have a finite or a non-finite verb, the first case being limited to the relativization of P, semi-object or theme of ditransitive verbs (with head-internal of prenominal relative clauses) or that of time / place adjuncts (prenominal relative clauses). 

While non-finite relative clauses superficially appear to be entirely similar to independent clauses, there are three pieces of evidence showing crucial differences between them.

First, the use of possessive prefixes on nouns in relative clauses is distinct from main clauses.

In Japhug, the possessive prefixes of inalienably possessed nouns have coreference constraints: for instance, with the noun \ipa{-pɤro} `present', the possessive prefix always refers to the person giving the present, never to the recipient as in \refb{ex:YWtambi}. 

	
\begin{exe}
\ex \label{ex:YWtambi}
\gll
	\ipa{a-pɤro}  	\ipa{ɲɯ-ta-mbi}  	\ipa{ŋu}  \\
	\textsc{1sg.poss}-present \textsc{ipfv}-1$\rightarrow$2-give be:\textsc{fact} \\
	\glt I give it to you as a present. (Elicited)
 	  \end{exe} 
	 
In relative clauses, including nominalized and non-nominalized ones, it is possible with nouns of this type to use either the possessive prefix corresponding to the giver (as in example \ref{ex:apAro}) or to neutralize the giver and use the indefinite possessor prefix \ipa{tɯ}-/\ipa{tɤ}- (\ref{ex:tApAro}).
	 

 		\begin{exe}
\ex \label{ex:apAro}
\gll
[\ipa{a-pɤro}  	\ipa{nɯ-mbi-t-a}]  	\ipa{nɯ}  	\ipa{a-rɟit}  	\ipa{ŋu}  \\
	\textsc{1sg.poss}-present \textsc{pfv}-give-\textsc{pst:tr-1sg} 	  \topic{} \textsc{1sg.poss}-child be:\textsc{fact} \\
\glt The one to whom I gave a present is my child. (Elicited)
 	  \end{exe} 

		\begin{exe}
\ex \label{ex:tApAro}
\gll
	[\ipa{tɤ-pɤro}  	\ipa{nɯ-mbi-t-a}]  	\ipa{tɤ-rɟit}  	\ipa{nɯ}  	\ipa{a-tɕɯ}  	\ipa{ŋu}   \\
	\textsc{indef.poss}-present \textsc{pfv}-give-\textsc{pst:tr-1sg} 	\textsc{indef.poss}-child \topic{} \textsc{1sg.poss}-son be:\textsc{fact} \\
\glt The child to whom I gave a present is my son. (Elicited)
\end{exe} 


Second, the main verbs of finite relative clauses can undergo totalitative reduplication. Totalitative reduplication is the  reduplication of the first syllable of a verb form, expressing the meaning `all', and is normally only possible on nominalized verbs as in \refb{ex:kakwnandza}.

\begin{exe}
   \ex  \label{ex:kakwnandza}
\gll [<quanxian>  	\ipa{tɕe}  	\ipa{kɯ\textasciitilde{}kɤ-kɯ-nɤndza}]\rc{}  	\ipa{nɯ}  	\ipa{ɲɤ-ɣɤme.}      	\\
the.whole.county \textsc{lnk}  \textsc{total\textasciitilde{}pfv-nmlz:S}-have.leprosy \topic{} \textsc{ifr}-suppress \\
 \glt  (This doctor) cured (suppressed) all lepers in the whole county. (25 khArWm, 72)
   \end{exe} 


However, in finite relative clauses, totalitative reduplication is also possible on a finite verb form as in \refb{ex:pWpaGWt} and \refb{ex:pwpwfcata}.

  \begin{exe}
\ex \label{ex:pWpaGWt}
\gll
\ipa{tɕe}  	[\ipa{nɯ} \ipa{ra}  	\textbf{\ipa{tɤrɤkusna}}\tete{}  	\ipa{nɯ}  	\ipa{pɯ\textasciitilde{}pa-ɣɯt}]\rc{}  	\ipa{nɯ}  	\ipa{lo-ji-ndʑi}  \\
\textsc{lnk} \textsc{dem} \textsc{pl} good.crops \topic{} \textsc{total\textasciitilde{}pfv:3$\rightarrow$3:down}-bring \topic{} \textsc{ifr}-plant-\textsc{du} \\
\glt They^{du} planted all the crops that she had brought (from heaven). (flood3.111)
\end{exe}

  \begin{exe}
   \ex \label{ex:pwpwfcata}
 \gll \ipa{ɯ-ro}   	\ipa{nɯ} \ipa{ra,}   	[\ipa{iɕqʰa}   	\ipa{pɯ\textasciitilde{}pɯ-fɕat-a}]\rc{}   	\ipa{nɯ} \ipa{ra}  	\ipa{kɯ}   	\ipa{tɕe}   	\ipa{tɕe}   	\ipa{sɯjno}   	\ipa{tu-ndza-nɯ}    \\
 \textsc{3sg.poss}-rest \topic{} \textsc{pl} the.aforementioned \textsc{total\textasciitilde{}pfv}-tell-\textsc{1sg} \topic{} \textsc{pl} \textsc{erg} \textsc{lnk} \textsc{lnk} grass \textsc{ipfv}-eat-\textsc{pl} \\
\glt The rest, all the (animals) that I have talked about before eat grass. (05 khWna, 42)
\end{exe}


Totalitative reduplication, on the other hand, is impossible in the case of main clauses or subordinate clauses other than relative clauses. Apart from totalitative reduplication, reduplication of the first syllable of the verb form is only found in the protasis of some conditionals (\citealt{jacques14linking}), but this is an entirely different phenomenon.

Third, evidential marking is neutralized in relative clauses (not an uncommon phenomenon, see \citealt[253-6]{aikhenvald06}): it is not possible to use the inferential in a finite relative, only the perfective is possible. For instance, in example \refb{ex:paBde}, while the verb of the main clause is in the inferential, that of the relative \ipa{pa-βde} \textsc{pfv:down}:3$\rightarrow$3'-throw `he threw it down' is in the perfective form. Replacing this form  by the equivalent inferential \ipa{pjɤ-βde}  \textsc{ifr:down}-throw would result in an agrammatical sentence.

\begin{exe}
\ex  \label{ex:paBde}
\gll  [\ipa{tɤ-tɕɯ-pɯ} 	\ipa{kɯ} 	\ipa{rdɤstaʁ} 	\ipa{pa-βde}] 	\ipa{nɯ} 	\ipa{jo-nɯɴqhu-ndʑi} \ipa{tɕe} \\
\textsc{indef.poss}-boy-small \textsc{erg} stone \textsc{pfv:down}:3$\rightarrow$3'-throw \textsc{dem} \textsc{ifr}-follow-\textsc{du} \textsc{lnk} \\
\glt `They followed the stones that the little boy had thrown down (along the way).' (Tangguowu, 42)
\end{exe}

These three independent pieces of evidence show that finite relative clauses in Japhug present properties showing their nominalized status, despite the fact that the verb is in a non-nominalized form.

\section{Defining subject and object in Japhug} \label{sec:subject.object}
We saw in section \ref{sec:flagging.indexation} that neither flagging nor indexation of arguments of the verb offer clear evidence for  the existence of either subjects or objects in Japhug.



However, data from relativization provide evidence of \textit{restrictive neutralization} (\citealt[275]{vanvalin97syntax}) of several types of arguments in specific constructions. Table \ref{tab:summary} summarizes all relativisation constructions described in this article (in this table, HI stands for \textit{head-internal} relative and PN for \textit{prenominal} relative).



\begin{table}[H]
\caption{Summary of relative clauses in Japhug } \label{tab:summary}
\resizebox{\columnwidth}{!}{
\begin{tabular}{l|ccc|ccc}
\toprule
&\multicolumn{3}{c}{Participial Relative Clause} & \multicolumn{2}{c}{Finite Relative Clause} \\
Function & \ipa{kɯ-}  & \ipa{kɤ-}  & \ipa{sɤ-}  & Simple  & Relator noun \\
\midrule
S	& HI, PN \bleute{}& &&&& \\
A & HI, PN \bleute{}& &&&& \\
\hline
possessor & PN&&&&& \\
\hline
P & & HI, PN \rouge{}&& HI, PN \rouge{} &\\
semi-object & & HI, PN \rouge{}&& HI, PN \rouge{} &\\
T & & HI, PN \rouge{} && HI, PN \rouge{}&\\
R (secundative) & & HI, PN \rouge{} && HI, PN \rouge{}&\\
\hline 
goal & & &HI, PN  & HI, PN  &\\
\hline
R (indirective) & &&HI, PN\\
comitative & &&HI, PN\\
instrumental adjunct  & &&HI, PN \\
\hline
time adjunct  & &&HI, PN&&PN\\
place adjunct  &&&HI, PN&&PN\\ 
\bottomrule
\end{tabular}}
\end{table}


The first neutralization is that of S and A (\textbf{subjects}), first noticed by \citet{jackson03caodeng}  in the case of the related language Tshobdun. Both S and A, and these two types of participants exclusively, can be relativized by headless, prenominal and head-internal relative clauses with a verb in the \ipa{kɯ-} participial form. Possessors can also be relativized with non-finite relative clauses and \ipa{kɯ-} participles, but only with head-internal relative clauses, not prenominal or head-internal ones. 

%There are two other areas of Japhug grammar where this neutralization is observed: the prefix on \ipa{kɤ-} participles and some types of control verbs.

There is another area  of Japhug grammar where this neutralization is observed: the optional possessive prefix on \ipa{kɤ-} participles also follows an accusative alignment.   For transitive and ditransitive verbs  the prefix always refers to the A (example \ref{ex:kill3} above), while in the case of semi-transitive verbs, it refers to the S, as in \refb{ex:nWkArga2}, thus displaying S/A restrictive neutralization.\footnote{The \ipa{kɤ-} participle does not exist for plain intransitive verbs, so that semi-transitive verbs are the only ones where one can test the behaviour of S with an object participle.}

 \begin{exe}
   \ex   \label{ex:nWkArga2} 
\gll [\ipa{pɣa}  	\ipa{ra}  	\ipa{nɯ-kɤ-rga}]\rc{}  	\ipa{nɯ}  	\ipa{qaj}  	\ipa{ntsɯ}  	\ipa{ŋu}  \\
bird \textsc{pl} \textsc{3pl-nmlz:P}-like \topic{} wheat always be:\textsc{fact} \\
\glt (The food) that birds like is always wheat. (23 pGAYaR, 24)
   \end{exe} 


The second neutralization is that of P, T and R of secundative ditransitive verbs, T of indirective verbs and the semi-object of semi-transitive verbs (\textbf{objects}), the four of which can be relativized in both non-finite relative clauses with a \ipa{kɤ-} participle and simple finite relative clauses (see Table \ref{tab:summary}). The goal of motion / manipulation verbs can be relativized in a finite relative, but require the oblique participle \ipa{sɤ-} when grammaticalized with a participial relative, showing that it should be kept distinct.

The accusative alignment shown by the existence of subject and objects in Japhug contrasts with the presence of ergative and neutral alignment  in other areas of the grammar. Ergative alignment is found in generic person marking (see \citealt{jacques12demotion}),\footnote{Generic S/P is marked by \ipa{kɯ-}, while generic A is marked with the inverse prefix \ipa{-wɣ-}.} while neutral alignment is extremely common especially in clause linking constructions (\citealt{jacques14linking}), and also in control constructions. For instance, the S of the verb \ipa{rga}  `like' can be coreferent with either the S  (\ref{ex:kAnWrAGo.rganW}), the A (\ref{ex:kAnArtoXpjAt.pWrgaa}) and even the P (\ref{ex:YWrganW}) of its infinitival or finite  complement verbs (the complement verb in \ref{ex:kAnWrAGo.rganW} is intransitive, while it is transitive in \ref{ex:kAnArtoXpjAt.pWrgaa} and \ref{ex:YWrganW}). This is still a syntactic pivot, since  coreference is only possible with core arguments, not with oblique arguments (including recipients of indirective verbs) or adjuncts.
    \begin{exe}
   \ex   \label{ex:kAnWrAGo.rganW} 
\gll
\ipa{tsuku}  	\ipa{tɕe}  	\ipa{kɤ-nɯrɤɣo}  	\ipa{wuma}  	\ipa{ʑo}  	\ipa{rga-nɯ}  	\ipa{tɕe}  \\
some \textsc{lnk} \textsc{inf}-sing really \textsc{emph} like:\textsc{fact-pl}  \textsc{lnk} \\
  \glt Some people like to sing. (26 kWrNukWGndZWr, 104)
     \end{exe}  
 
   \begin{exe}
   \ex   \label{ex:kAnArtoXpjAt.pWrgaa} 
\gll
  	\ipa{aʑo}  	\ipa{qajɯ}  	\ipa{nɯ} \ipa{ra}  	\ipa{kɤ-nɤrtoχpjɤt}  	\ipa{pɯ-rga-a}  	\ipa{tɕe}  	\\
  	\textsc{1sg} bugs \textsc{dem} \textsc{pl} \textsc{inf}-observe \textsc{pst.ipfv}-like-\textsc{1sg} \textsc{lnk}  \\
 \glt I liked to observe bugs. (26 quspunmbro, 15)
     \end{exe}  
 
  \begin{exe}
   \ex   \label{ex:YWrganW} 
\gll
\ipa{maka}  	\ipa{tu-kɤ-nɤjoʁjoʁ,}  	\ipa{tu-kɤ-fstɤt}  	\ipa{nɯ}  	\ipa{ɲɯ-rga-nɯ}  \\
at.all \textsc{ipfv-inf}-flatter \textsc{ipfv-inf}-praise \textsc{dem} \textsc{ipfv}-like-\textsc{pl} \\
\glt They like to be flattered or praised. (140427 yuanhou, 53)
    \end{exe}  
 

It is possible that other control verbs or that some clause linking constructions have  stricter syntactic pivots. Although there is some evidence for this, I defer a fuller investigation of pivots in complementation and clause linking construction to future research. Since the study of syntactic pivots critically depends on negative data (the impossibility of saying a particular sentence, or the impossibility to interpret an attested sentence in a particular way), even a relatively large corpus is not sufficient to ascertain the existence of pivots. While elicitation is necessary, it is often difficult to be sure that a particular judgement on the agrammaticality of a sentence is due to syntactic rather than pragmatic reasons. 
 
Table \ref{tab:japhug.pivot} summarizes the syntactic pivots  attested in Japhug. The symbol P' is used for the semi-object of semi-transitive verbs, and T_1 / R_1 vs T_2 / R_2 for the arguments of secundative vs indirective transitive verbs.

\begin{table}[H]
\caption{Syntactic pivots in Japhug} \label{tab:japhug.pivot} \centering
\begin{tabular}{llllll}
\toprule
Pivot & Construction \\
\midrule
\{S, A\}  & prenominal relativization with \ipa{kɯ-} participle  \\
 (subject)&(possessive prefix) on \ipa{kɤ-} participles in relative clauses \\
 \midrule
\{P, P', R_1, T\}  &relativization with \ipa{kɤ-} participle \\
(object)  & \\
\midrule
\{P, P', R_1, T, goal\} &
relative clauses with a finite main verb \\
(extended object) &(without relator noun)\\
\midrule
 \{S, P, R_1, T_2\} & generic person marking\\
 (absolutive argument)\\
 \midrule
  \{S, A, P, P', R_1, T\} & control constructions (\ipa{rga}  `like') \\
 (core argument)\\
\bottomrule
\end{tabular}
\end{table}

In the Sino-Tibetan family, while many languages such as Standard Mandarin (\citealt{lapolla93subject}) or Lhasa Tibetan  (\citealt{tournadre96erg}) lack strict syntactic pivots, such pivots are well-attested in Kiranti languages (\citealt{bickel01light.verbs}, \citealt{bickel04hidden}). Table \ref{tab:belhare} recapitulates some of the main syntactic pivots in Belhare (the notation of the arguments is slightly modified from \citealt{bickel04hidden}).

\begin{table}[H]
\caption{Syntactic pivots in Belhare (after \citealt{bickel04hidden})} \label{tab:belhare} \centering
\begin{tabular}{llllll}
\toprule
Pivot & Construction \\
\midrule
\{S, A\} & embedded non-finite \ipa{-si} and \ipa{-sa} clauses\\
& root nominalization\\
\{S, P, R, T\} & internally headed relativisation \\
\{S, P, R\} & control by \ipa{khes} `must', \ipa{nus-} `say'\\
\bottomrule
\end{tabular}
\end{table}

 
We see that Japhug resembles Belhare in two respects: it has both ergative and accusative pivots, and the accusative pivot is involved in relativization (`root nominalization' in Belhare). However, accusative pivots  are  more widespread in Japhug, while Belhare tends to favour ergative pivots.  


Sino-Tibetan is perhaps one of the most diverse language family from the point of view of morphosyntactic typology, and especially alignment. While isolating Sino-Tibetan languages like Mandarin tend to lack syntactic pivots, languages with richer morphology display a bewildering diversity of alignment types: some languages favour accusative pivots, others ergative pivots, and present at least three distinct   pivots depending on the  construction. Only purely accusative, or purely ergative languages appear to be unattested in Sino-Tibetan.


\section{Conclusion}

The present article has shown that, although Japhug has ergative alignment on NP arguments, and a direct-inverse system in person indexation on the verb, it is nevertheless possible to rigorously define `subjects' and `objects' in this language, by taking into account restrictive neutralization phenomena observed in relative constructions.

This work shows that ergative and accusative pivots co-exist in Japhug, and that this language cannot be meaningfully classified as `ergative', `accusative' or `hierarchical'. In addition, it shows that unlike languages such as Movima, inverse marking in relative clauses does not necessarily influence the syntactic pivot of the relativized participant.

\theendnotes

\bibliographystyle{unified}
\bibliography{bibliogj}
\end{document}