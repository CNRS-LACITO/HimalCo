
\documentclass[oldfontcommands,oneside,a4paper,11pt]{article} 
\usepackage{fontspec}
\usepackage{natbib}
\usepackage{booktabs}
\usepackage{xltxtra} 
\usepackage{polyglossia} 
\usepackage[table]{xcolor}
\usepackage{gb4e} 
\usepackage{multicol}
\usepackage{graphicx}
\usepackage{float}
\usepackage{hyperref} 
\hypersetup{bookmarks=false,bookmarksnumbered,bookmarksopenlevel=5,bookmarksdepth=5,xetex,colorlinks=true,linkcolor=blue,citecolor=blue}
\usepackage[all]{hypcap}
\usepackage{memhfixc}
\usepackage{lscape}
 \usepackage{lineno}
\bibpunct[: ]{(}{)}{,}{a}{}{,}
 
%\setmainfont[Mapping=tex-text,Numbers=OldStyle,Ligatures=Common]{Charis SIL} 
\newfontfamily\phon[Mapping=tex-text,Ligatures=Common,Scale=MatchLowercase,FakeSlant=0.3]{Charis SIL} 
\newcommand{\ipa}[1]{{\phon \mbox{#1}}} %API tjs en italique
 \newcommand{\ipab}[1]{{\phon \mbox{#1}}} %API tjs en italique
\newcommand{\grise}[1]{\cellcolor{lightgray}\textbf{#1}}
\newfontfamily\cn[Mapping=tex-text,Ligatures=Common,Scale=MatchUppercase]{MingLiU}%pour le chinois
\newcommand{\zh}[1]{{\cn #1}}

 

 \begin{document} 
 
\title{Grammaire japhug: plan de travail}
 
\author{Guillaume Jacques}
\maketitle
 
 
 \section{Publications}
Articles déjà publiés dans le tableau \ref{tab:ecrit}. \citet{jacques10inverse}, \citet{jacques10refl} et \citet{jacques12demotion} à récrire en \LaTeX.
 
\begin{table}
\caption{Déjà publié} \label{tab:ecrit} \centering
\begin{tabular}{llll}
\toprule
Chapitre &section& nb de pages & référence \\
\midrule
Indexation & Direct /inverse&30 & \citet{jacques10inverse}\\
Voix &réfléchi &7 & \citet{jacques10refl} \\
&passif, anticausatif etc &27 & \citet{jacques12demotion} \\
&incorporation & 24& \citet{jacques12incorp} \\
&tropatif/ applicatif &13 & \citet{jacques13tropative} \\
&antipassif &22 & \citet{jacques14antipassive} \\
Autres catégories & Mouvement associé  & 30&\citet{jacques13harmonization} \\
Idéophones & &31 &\citet{japhug14ideophones} \\
Clause linking & &64 &\citet{jacques14linking} \\
\bottomrule
\end{tabular}
\end{table} 


 
\section{Articles en cours d'évaluation} \label{sec:eval}
Sans doute un gros boulot de récriture durant l'année 2015.
\begin{enumerate}
\item Relatives (Linguistic Anthropology)
\item Causatif (FLH)
\item Illustration of the IPA (JIPA)
\item Spontané / autobénéfactif (LTBA)
\item Comparatives (Diachronica)
\item Générique (ouvrage collectif)
\item Numerals (festschrift)
\item Sketch (Routledge)
\end{enumerate} 
 
\section{Articles à écrire} \label{sec:aecrire}
A finir en 2015, ou début 2016

\begin{enumerate}
\item Emprunts (pour Eitan)
\item Complémentation
\item Evidentialité 
\item Indirect speech
\item Grammaticalisation (pour Malchukov). Inclura l'idée sur le comitatif (\ipa{kɤɣɯjɯjaʁ} < \ipa{kɯ}+\ipa{aɣɯ}-\ipa{jaʁ}) et le progressif
\end{enumerate}


\section{Projet} 
 
Lorsque les articles dans \ref{sec:eval} et \ref{sec:aecrire} seront tous publiés ou acceptés (2016), début de la rédaction de la grammaire, environ trois ans ($\rightarrow$ 2018/2019).
   
  
\bibliographystyle{unified}
\bibliography{bibliogj}

 \end{document}
 