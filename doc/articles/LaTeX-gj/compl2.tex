\documentclass[oldfontcommands,oneside,a4paper,11pt]{article} 
\usepackage{fontspec}
\usepackage{natbib}
\usepackage{booktabs}
\usepackage{xltxtra} 
\usepackage{longtable}
\usepackage{polyglossia} 
%\usepackage[table]{xcolor}
\usepackage{gb4e} 
\usepackage{multicol}
\usepackage{graphicx}
\usepackage{float}
\usepackage{lineno}
\usepackage{textcomp}
\usepackage{hyperref} 
\hypersetup{bookmarks=false,bookmarksnumbered,bookmarksopenlevel=5,bookmarksdepth=5,xetex,colorlinks=true,linkcolor=blue,citecolor=blue}
\usepackage[all]{hypcap}
\usepackage{memhfixc}
\usepackage{lscape}
 

\setmainfont[Mapping=tex-text,Numbers=OldStyle,Ligatures=Common]{Charis SIL} \newfontfamily\phon[Mapping=tex-text,Ligatures=Common,Scale=MatchLowercase,FakeSlant=0.3]{Charis SIL} 
\newcommand{\ipa}[1]{{\phon#1}} %API tjs en italique
 \newcommand{\jpg}[2]{\ipa{#1} `#2'} %API tjs en italique
\newcommand{\grise}[1]{\cellcolor{lightgray}\textbf{#1}}
\newfontfamily\cn[Mapping=tex-text,Ligatures=Common,Scale=MatchUppercase]{MingLiU}%pour le chinois
\newcommand{\zh}[1]{{\cn #1}}
\newcommand{\tld}{\textasciitilde{}}

\XeTeXlinebreaklocale "zh" %使用中文换行
\XeTeXlinebreakskip = 0pt plus 1pt %
 %CIRCG
 


\begin{document} 

\title{Complementation in Japhug Rgyalrong}%\footnote{} 
\author{Guillaume Jacques}
\maketitle
\linenumbers
 
 
\section{Introduction}
 \citet[9]{dixon06complementation}
 \citet{sun12complementation}
 \citet{jacques08}

\section{Background information}

\subsection{Complement clauses vs complementation strategies} \label{sec:strategies}

\section{Complement types}
\subsection{Infinitive} 

\ipa{kɯ-} vs \ipa{kɤ-}

\subsubsection{Coreference between matrix and complement clause}
The coreference restrictions of the arguments in the complement and in the matrix clause differ from verb to verb. Some infinitival complements present an ergative pivot: the S or A of the matrix clause is necessarily coreferent with the A or S of the complement clause (this is the case with the verb \ipa{spa} `be able', see section \ref{sec:spa}).


Other verbs also allow coreference with the P of the complement clause. This is for instance the case of the semi-transitive verb \jpg{rga}{like}, whose S can be coreferent with either the S  (\ref{ex:kAnWrAGo.rganW}), the A (\ref{ex:kAnArtoXpjAt.pWrgaa}) and even the P (\ref{ex:YWrganW}) of its infinitival complement clause (\citealt{jacques16relatives}).


 \begin{exe}
   \ex   \label{ex:kAnWrAGo.rganW} 
\gll
\ipa{tsuku}  	\ipa{tɕe}  	\ipa{kɤ-nɯrɤɣo}  	\ipa{wuma}  	\ipa{ʑo}  	\ipa{rga-nɯ}  	\ipa{tɕe}  \\
some \textsc{lnk} \textsc{inf}-sing really \textsc{emph} like:\textsc{fact-pl}  \textsc{lnk} \\
  \glt Some people like to sing. (26 kWrNukWGndZWr, 104)
     \end{exe}  
 
   \begin{exe}
   \ex   \label{ex:kAnArtoXpjAt.pWrgaa} 
\gll
  	\ipa{aʑo}  	\ipa{qajɯ}  	\ipa{nɯ} \ipa{ra}  	\ipa{kɤ-nɤrtoχpjɤt}  	\ipa{pɯ-rga-a}  	\ipa{tɕe}  	\\
  	\textsc{1sg} bugs \textsc{dem} \textsc{pl} \textsc{inf}-observe \textsc{pst.ipfv}-like-\textsc{1sg} \textsc{lnk}  \\
 \glt I liked to observe bugs. (26 quspunmbro, 15)
     \end{exe}  
 
  \begin{exe}
   \ex   \label{ex:YWrganW} 
\gll
\ipa{maka}  	\ipa{tu-kɤ-nɤjoʁjoʁ,}  	\ipa{tu-kɤ-fstɤt}  	\ipa{nɯ}  	\ipa{ɲɯ-rga-nɯ}  \\
at.all \textsc{ipfv-inf}-flatter \textsc{ipfv-inf}-praise \textsc{dem} \textsc{ipfv}-like-\textsc{pl} \\
\glt They like to be flattered or praised. (140427 yuanhou, 53)
    \end{exe}  

%S, A, P of ɣɤkhɯ

Absence of coreference between matrix clause and infinitival complement clause occur in two cases. First, with intransitive impersonal modal verbs (such as \jpg{ra}{have to}, section \ref{sec:ra}), the whole complement clause is treated as the S of the matrix verb. Second, the verb \jpg{rɲo}{experience} presents a special case treated in section (\ref{sec:rɲo}). 
 

    
 \subsection{S/A participles} 
This type of complements\footnote{Following Dixon, \citet{sun12complementation} treats complements of this type with motion verbs  in Tshobdun as a relativization strategy rather than as proper complements, since their are not core arguments of the verb. However, for the reasons explained in section (\ref{sec:strategies}), I do not consider them to be complementation strategies, as they are syntactically non-core argument of the verb.} are rarer than infinitival complements, and only attested with some motion verbs (\jpg{ɕe}{go}, \jpg{ɣi}{come}), verbs of pretence (\jpg{nɯɕpɯz}{pretend}, \jpg{ʑɣɤpa}{pretend}) and one aspectual verb  (\jpg{rɤŋgat}{be about to}).

XXXX
\subsubsection{Coreference between matrix and complement clause}
Unlike motion verbs, verb of pretence such \jpg{nɯɕpɯz}{pretend} cannot be used with infinitival complement to express coreference of the S/A of the matrix clause with the P of the verb of the complement clause. The only way to express this meaning is by reflexivizing the verb of the complement clause with the prefix \ipa{ʑɣɤ-}, as in example (\ref{ex:pWkWZGAnWBlu} ).

  \begin{exe}
   \ex   \label{ex:pWkWZGAnWBlu} 
\gll \ipa{pɯ-kɯ-ʑɣɤ-nɯβlu} \ipa{to-nɯɕpɯz} \\
\textsc{pfv-nmlz:S/A-refl}-cheat \textsc{ifr}-pretend \\
\glt  `He pretended having been duped (=let himself be cheated).' (Elicitation)
    \end{exe}  
    
    
 \subsection{Bare infinitive and \ipa{tɯ-} infinitives} 
A handful of verbs, namely \jpg{ʑa}{begin}, \jpg{sɤʑa}{begin} and \jpg{rɲo}{experience, have already} are compatible with bare infinitive complements. These verbs also allow \ipa{kɤ-} infinitives and/or finite complement clauses, though in the case of \jpg{ʑa}{begin} these are extremely rare.

Bare infinitives occur only with transitive verbs, and are formed by combining the stem 1 of the verb with a possessive prefix coreferent with the P of the complement clause, as in examples (\ref{ex:Wmto}).

\begin{exe}
\ex \label{ex:Wmto}
\gll \ipa{nɤʑo} 	\ipa{kɯ-fse} 	\ipa{a-ŋkʰor} 	\ipa{nɯ} 	\ipa{ɯ-mto} 	\ipa{mɯ-pɯ-rɲo-t-a} \\
you \textsc{nmlz:stative}-be.like \textsc{1sg.poss}-subject \textsc{top} \textsc{3sg}-\textsc{bare.inf:}see \textsc{neg-pfv}-experience-\textsc{pst:tr-1sg} \\
\glt  `I never saw anyone like you among my subjects.' (Smanmi metog koshana1.157)
\end{exe} 

Bare infinitives are in complementary distribution with \ipa{tɯ-} infinitives,  which occur when the verb of the complement is morphologically intransitive, and thus lacks a P. They are only compatible with polarity prefixes (as in example \ref{ex:mAtWrga} below), and cannot take TAM or possessive prefixes.

XXX


Semi-transitive verbs are treated like intransitive verbs: they cannot form a bare infinitive, and use \ipa{tɯ-} infinitives instead (example \ref{ex:mAtWrga}), although their semi-object does present some object-like syntactic properties (see \citealt{jacques16relatives}).  

\begin{exe}
\ex  \label{ex:mAtWrga}
\gll \ipa{qaɟy} 	\ipa{ɯ-me} 	\ipa{nɯnɯ,} 	\ipa{tɕendɤre} 	\ipa{kʰro} 	\ipa{mɤ-tɯ-rga} 	\ipa{to-ʑa} \\
fish \textsc{3sg.poss}-daughter \textsc{dem} \textsc{lnk} a.lot \textsc{neg-inf}-like \textsc{ifr}-start \\
\glt `He started not liking the mermaid that much.' (hist150819 haidenver, 154)
\end{exe}
 
The only exception to this distribution are some transitive verbs used in complex predicates referring to weather phenomena, in particular \ipa{lɤt} `throw' and \ipa{βzu} `make, do', as in (\ref{ex:tWmlAt}). Note that in this construction, although the verbs remain morphologically transitive, they cannot take an overt A marked with the ergative.
 
\begin{exe}
\ex  \label{ex:tWmlAt}
\gll
\ipa{tɯ-mɯ} 	\ipa{kɯ-wxtɯ\tld{}wxti} 	\ipa{ʑo} 	\ipa{tɯ-lɤt} 	\ipa{pjɤ-ʑa} \\
\textsc{indef.poss}-sky \textsc{nmlz:S/A-emph}\tld{}be.big \textsc{emph} \textsc{inf}-throw \textsc{ifr}-start \\
\glt `A big rain started.' (hist150819 haidenver, 104)
\end{exe}


%\ipa{hanɯni} 	\ipa{ɯ-rŋa} 	\ipa{ra} 	\ipa{tɯ-ɣɯrni} 	\ipa{tɯ-βzu} 	\ipa{ɲɤ-ʑa} 
%hist150820 meili de meiguihua

Etymology: action nominalization or indefinite possessor?

Bare infinitive as an action nominal (\citealt{jacques14antipassive}):

\begin{exe}
\ex \label{ex:bare.inf.noun}
\gll \ipa{ndʑi-mi}   	\ipa{ɯ-tsʰoʁ}   	\ipa{ɯ-tsʰɯɣa}   	\ipa{nɯra}   	\ipa{wuma}   	\ipa{ʑo}   	\ipa{naχtɕɯɣ-ndʑi.}   \\
\textsc{3du.poss}-foot \textsc{3sg}-\textsc{bare.inf:}attach.to \textsc{3sg.poss}-form \textsc{dem:pl} very \textsc{emph}  \textsc{npst}:similar-\textsc{du}  \\
\glt `The way their feet (of fleas and crickets) touch the ground is very similar.' (26-mYaRmtsaR, 17)
\end{exe}
 
 \subsection{Complementation strategies}  
 
 \subsection{Finite} 

  \section{Morphosyntactic properties of complement clauses} 

 \subsection{Alignment} 

  \subsection{Raising of TAM and person/number marking} 
  
  \subsection{Case marking mismatch} 
  
  \subsection{Semi-finiteness} 
like relative clauses, \citealt{jacques16relatives}

  \subsection{Multiple complements}
 \ipa{ɯ-zda} 	\ipa{nɯ} 	\ipa{ra} 	\ipa{pɣɤtɕɯ} 	\ipa{kɯ-ɣɤwu,} 	\ipa{kʰɯna} 	\ipa{kɯ-ɤndzɯt,} 	\ipa{lɯlu} 	\ipa{kɯ-ɣɤwu} 	\ipa{qachɣa} 	\ipa{kɯ-mbri} 	\ipa{kɯ-fse,} 	\ipa{nɯra} 	\ipa{tu-nɯɕpɯz} 	\ipa{ɲɯ-spe.} 	
 
  \section{Complement-taking verbs} 
  \subsection{Modal verb}
  
    \subsubsection{\jpg{spa}{be able to}} \label{sec:spa}
The verb \jpg{spa}{be able to} is the only transitive modal verb.\footnote{The causative verb \jpg{sɯxcʰa}{(cause to) have the ability to} is also formally transitive, but is only used in inverse forms, see section XXX)} It originates from the abilitative form of the verb \ipa{pa} `do' (\citealt{jacques15causative}) and has a cognate in Tangut  (\citealt{jacques14esquisse}), showing that its lexicalization occurred even earlier than proto-Gyalrongic.

The verb \jpg{spa}{be able to} takes both infinitival (examples \ref{ex:rYo:inf:A} and \ref{ex:rYo:inf:S}) or finite complements, and its A is coreferent with the S or the A of the complement clause.

\begin{exe}
\ex  \label{ex:rYo:inf:A}
\gll
\ipa{nɯ} 	\ipa{ɯ-mdoʁ} 	\ipa{nɯ} 	\ipa{aj} 	\ipa{kɤ-ti} 	\ipa{mɯ́j-spe-a} \\
\textsc{dem} 3sg.poss-colour \textsc{dem} \textsc{1sg} \textsc{inf}-say \textsc{neg:sens}-be.able[III]-1sg \\
\glt I am not able to name its colour. (06-qaZmbri, 57)
\end{exe}

\begin{exe}
\ex  \label{ex:rYo:inf:S}
 \ipa{kɤ-nɤre} 	\ipa{ɯ-tá-spa?}\\
 \textsc{inf}-laugh \textsc{q-pfv}:3$\rightarrow$3'-be.able.to\\
 \glt `Is he now able to laugh?' (conversation, 2014, of a three month old infant)
\end{exe}
    
  \subsubsection{\jpg{ra}{have to}} \label{sec:ra}
  \ipa{tɤ-pɤtso} 	\ipa{nɯ,} 	\ipa{tɯ-pɤrme} 	\ipa{roro} 	\ipa{jamar} 	\ipa{tɕe} 	\ipa{tɕe} 	\ipa{tɯ-nɯ} 	\ipa{kɤ-sɯβde} 	\ipa{pjɤ-ra} 
  
  \subsection{Phasal verbs and other aspectual auxiliaries}
\subsubsection{\jpg{rɲo}{experience}}   \label{sec:rɲo}

Infinitival complement without argument coreferent with the matrix clause:
\ipa{aʑo} 	\ipa{pɯ-xtɕɯ-xtɕi-a} 	\ipa{ʑo} 	\ipa{ri} 	\ipa{tɯxtɤŋɤm} 	\ipa{nɯ-atɯɣ-a} 	\ipa{tɕe,} 	\ipa{nɯ} 	\ipa{kɤ-mŋɤm} 	\ipa{pɯ-rɲo-t-a} 
  (24-pGArtsAG, 121)
  
  \subsection{Motion verbs}
  
motion verbs and associated motion   \citet{jacques13harmonization}
  
\subsection{Causative verbs}


 %mɯ-tu-kɤ-nɤtɯti kɯ-ra tɤ́-wɣ-sɯ-βzu-a-nɯ ndʐa ɕti ma

\subsection{Complements of stative verbs}

enough to
\ipa{koŋla} 	\ipa{pjɯ-nɤrte} 	\ipa{ʑo} 	\ipa{kɯ-rtaʁ} 	\ipa{ʑo} 	\ipa{kɯ-wxti} 	\ipa{ɲɯ-βze} 	\ipa{ŋgrɤl} 

 \section{Conclusion}
 
\bibliographystyle{unified}
\bibliography{bibliogj}
\end{document}