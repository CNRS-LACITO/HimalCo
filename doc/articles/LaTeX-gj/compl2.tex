\documentclass[oldfontcommands,oneside,a4paper,11pt]{article} 
\usepackage{fontspec}
\usepackage{natbib}
\usepackage{booktabs}
\usepackage{xltxtra} 
\usepackage{longtable}
\usepackage{polyglossia} 
%\usepackage[table]{xcolor}
\usepackage{gb4e} 
\usepackage{multicol}
\usepackage{graphicx}
\usepackage{float}
\usepackage{lineno}
\usepackage{textcomp}
\usepackage{hyperref} 
\hypersetup{bookmarks=false,bookmarksnumbered,bookmarksopenlevel=5,bookmarksdepth=5,xetex,colorlinks=true,linkcolor=blue,citecolor=blue}
\usepackage[all]{hypcap}
\usepackage{memhfixc}
\usepackage{lscape}
 

\setmainfont[Mapping=tex-text,Numbers=OldStyle,Ligatures=Common]{Charis SIL} \newfontfamily\phon[Mapping=tex-text,Ligatures=Common,Scale=MatchLowercase,FakeSlant=0.3]{Charis SIL} 
\newcommand{\ipa}[1]{{\phon#1}} %API tjs en italique
 \newcommand{\jpg}[2]{\ipa{#1} `#2'} %API tjs en italique
\newcommand{\grise}[1]{\cellcolor{lightgray}\textbf{#1}}
\newfontfamily\cn[Mapping=tex-text,Ligatures=Common,Scale=MatchUppercase]{MingLiU}%pour le chinois
\newcommand{\zh}[1]{{\cn #1}}
\newcommand{\tld}{\textasciitilde{}}

\XeTeXlinebreaklocale "zh" %使用中文换行
\XeTeXlinebreakskip = 0pt plus 1pt %
 %CIRCG
 


\begin{document} 

\title{Complementation in Japhug Rgyalrong}%\footnote{} 
\author{Guillaume Jacques}
\maketitle
\linenumbers
 
 
\section{Introduction}
 \citet[9]{dixon06complementation}
 \citet{sun12complementation}
 \citet{jacques08}

\section{Background information}

\subsection{Participles vs infinitives} 
All Gyalrong languages, including Situ (\citealt{youjing03zhuokeji}), Tshobdun (\citealt{sun12complementation}) and Japhug, have a distinction between infinitives and participles. The distinction is quite subtle, as there are both infinitives and participles in \ipa{kɯ-} and \ipa{kɤ-} (or \ipa{kə-} and \ipa{kɐ-} depending on the transcription system). Since both categories are formally similar and can occur in similar contexts, it is crucial to clearly explain the distinction between the two, especially since Japhug slightly differs from the other Gyalrong languages in this regard.

The system of core argument participles in Japhug is relatively straightforward (\citealt{jacques16relatives}). The prefix \ipa{kɯ-} is used to build the S-participle of intransitive verbs, and the A-participle of transitive verbs. A-participles differ from S-participles in having in addition a possessive prefix coreferent with the P, as in (\ref{ex:akWfstWn}).

\begin{exe}
\ex \label{ex:akWfstWn}
\gll \ipa{a-me} 	\ipa{a-kɯ-fstɯn} 	\ipa{ŋu} \\
\textsc{1sg.poss}-daughter \textsc{1sg.poss}-\textsc{nmlz}:S/A-serve be:\textsc{fact} \\
\glt `My daughter is the one who takes care of me.' (The prince, 74)
\end{exe}

The prefix \ipa{kɤ-} on the other hand serves to build the P-participle, and can optionally take a possessive prefix coreferent with the A, as in (\ref{ex:tajmag}). Participles are compatible with polarity and associated motion prefixes (\citealt{jacques16relatives}).

\begin{exe}
   \ex \label{ex:tajmag}
   \gll
\ipa{aʑo}  	\ipa{a-mɤ-kɤ-sɯz}   	\ipa{tɤjmɤɣ}  	\ipa{nɯ}  	\ipa{kɤ-ndza}  	\ipa{mɤ-naz-a}  \\
\textsc{1sg} \textsc{1sg-neg-nmlz:P}-know mushroom \textsc{dem} \textsc{inf}-eat \textsc{neg}-dare:\textsc{fact}-\textsc{1sg} \\
\glt `I do not dare to eat the mushrooms that I do not know.' (23 mbrAZim,103)
\end{exe}

There are four types of infinitives in Japhug: \ipa{kɯ-}, \ipa{kɤ-}, \ipa{tɯ-} and bare infinitives. The latter two are restricted to very specific constructions (see \ref{sec:bareinf}), and only the former two types are discussed in this section. 

Infinitives in \ipa{kɯ-} and \ipa{kɤ-} are used in three types of constructions: citation form, manner converb (\citealt{jacques14linking}) and infinitival complement. Infinitives in \ipa{kɤ-} are by far the most common form in Japhug. The \ipa{kɯ-} form is restricted to stative verbs (including adjectives, copulas and existential verbs), impersonal auxiliaries and some dynamic intransitive verbs when they do not have a human S, and even with these verbs, \ipa{kɤ-} infinitives are used in several contexts.

The infinitives in \ipa{kɯ-} and \ipa{kɤ-} are used as the citation form; 

They formally differ from participles in several ways. First, they cannot take possessive prefixes.

\subsection{Complement clauses vs complementation strategies} \label{sec:strategies}

\section{Complement types}
\subsection{Infinitive} 

 
 

\subsubsection{Coreference between matrix and complement clause}
The coreference restrictions of the arguments in the complement and in the matrix clause differ from verb to verb. Some infinitival complements present an ergative pivot: the S or A of the matrix clause is necessarily coreferent with the A or S of the complement clause (this is the case with the verb \ipa{spa} `be able', see section \ref{sec:spa}).


Other verbs also allow coreference with the P of the complement clause. This is for instance the case of the semi-transitive verb \jpg{rga}{like}, whose S can be coreferent with either the S  (\ref{ex:kAnWrAGo.rganW}), the A (\ref{ex:kAnArtoXpjAt.pWrgaa}) and even the P (\ref{ex:YWrganW}) of its infinitival complement clause (\citealt{jacques16relatives}).


 \begin{exe}
   \ex   \label{ex:kAnWrAGo.rganW} 
\gll
\ipa{tsuku}  	\ipa{tɕe}  	\ipa{kɤ-nɯrɤɣo}  	\ipa{wuma}  	\ipa{ʑo}  	\ipa{rga-nɯ}  	\ipa{tɕe}  \\
some \textsc{lnk} \textsc{inf}-sing really \textsc{emph} like:\textsc{fact-pl}  \textsc{lnk} \\
  \glt Some people like to sing. (26 kWrNukWGndZWr, 104)
     \end{exe}  
 
   \begin{exe}
   \ex   \label{ex:kAnArtoXpjAt.pWrgaa} 
\gll
  	\ipa{aʑo}  	\ipa{qajɯ}  	\ipa{nɯ} \ipa{ra}  	\ipa{kɤ-nɤrtoχpjɤt}  	\ipa{pɯ-rga-a}  	\ipa{tɕe}  	\\
  	\textsc{1sg} bugs \textsc{dem} \textsc{pl} \textsc{inf}-observe \textsc{pst.ipfv}-like-\textsc{1sg} \textsc{lnk}  \\
 \glt I liked to observe bugs. (26 quspunmbro, 15)
     \end{exe}  
 
  \begin{exe}
   \ex   \label{ex:YWrganW} 
\gll
\ipa{maka}  	\ipa{tu-kɤ-nɤjoʁjoʁ,}  	\ipa{tu-kɤ-fstɤt}  	\ipa{nɯ}  	\ipa{ɲɯ-rga-nɯ}  \\
at.all \textsc{ipfv-inf}-flatter \textsc{ipfv-inf}-praise \textsc{dem} \textsc{ipfv}-like-\textsc{pl} \\
\glt They like to be flattered or praised. (140427 yuanhou, 53)
    \end{exe}  

%S, A, P of ɣɤkhɯ
Some verbs selecting S/A participle complement clauses use infinitive complement when there is coreference between the S of the matrix clause and the P of the complement clause (see section \ref{sec:SAparticiple.coref}).

Absence of core argument coreference between matrix clause and infinitival complement clause occurs in two cases. 

First, with intransitive impersonal modal verbs (such as \jpg{ra}{have to}, section \ref{sec:ra}), the whole complement clause is treated as the S of the matrix verb. 

Second, there are examples of coreference between the S/A of the matrix clause and the possessor of the S rather than the S in the case of infinitival clause with experiencer verbs. This is for instance the case of the verb \ipa{mŋɤm} `hurt', which can only take a body part as its S -- the experiencer is indicated by a possessive prefix on the body part, as in (\ref{ex:YWmNAm}).
 
 \begin{exe}
\ex \label{ex:YWmNAm}
\gll \ipa{a-xtu} 	\ipa{ɲɯ-mŋɤm} \\
\textsc{1sg.poss}-belly \textsc{sens}-hurt \\
\glt `My belly hurts.' 
\end{exe}
 
This verb can nevertheless be  used as complement of verbs such as \jpg{rɲo}{experience} when the experiencer is coreferent with the A, as in example (\ref{ex:kAmNAm}).
 
 \begin{exe}
\ex \label{ex:kAmNAm}
\gll \ipa{aʑo} 	\ipa{pɯ-xtɕɯ\tld{}xtɕi-a} 	\ipa{ʑo} 	\ipa{ri} 	\ipa{tɯxtɤŋɤm} 	\ipa{nɯ-atɯɣ-a} 	\ipa{tɕe,} 	\ipa{nɯ} 	\ipa{kɤ-mŋɤm} 	\ipa{pɯ-rɲo-t-a} \\
\textsc{1sg} \textsc{pst:ipfv-emph}\tld{}be.small-\textsc{1sg} \textsc{emph} \textsc{loc} dysentery \textsc{pfv}-meet-\textsc{1sg} \textsc{lnk} \textsc{inf}-hurt \textsc{pfv}-experience-\textsc{1sg} \\
\glt `When I was very small, I had dysentery, (my belly) ached.'  (24-pGArtsAG, 121)
\end{exe}

 \subsection{S/A participles} 
This type of complements\footnote{Following Dixon, \citet{sun12complementation} treats complements of this type with motion verbs  in Tshobdun as a relativization strategy rather than as proper complements, since their are not core arguments of the verb. However, for the reasons explained in section (\ref{sec:strategies}), I do not consider them to be complementation strategies, as they are syntactically non-core argument of the verb.} are rarer than infinitival complements, and only attested with some motion verbs (\jpg{ɕe}{go}, \jpg{ɣi}{come}), verbs of pretence (\jpg{nɯɕpɯz}{pretend}, \jpg{ʑɣɤpa}{pretend}) and one aspectual verb  (\jpg{rɤŋgat}{be about to}).

XXXX
\subsubsection{Coreference between matrix and complement clause} \label{sec:SAparticiple.coref}
Unlike motion verbs, verb of pretence such \jpg{nɯɕpɯz}{pretend} cannot be used with infinitival complement to express coreference of the S/A of the matrix clause with the P of the verb of the complement clause. The only way to express this meaning is by reflexivizing the verb of the complement clause with the prefix \ipa{ʑɣɤ-}, as in example (\ref{ex:pWkWZGAnWBlu} ).

  \begin{exe}
   \ex   \label{ex:pWkWZGAnWBlu} 
\gll \ipa{pɯ-kɯ-ʑɣɤ-nɯβlu} \ipa{to-nɯɕpɯz} \\
\textsc{pfv-nmlz:S/A-refl}-cheat \textsc{ifr}-pretend \\
\glt  `He pretended having been duped (=let himself be cheated).' (Elicitation)
    \end{exe}  
    
    
 \subsection{Bare infinitive and \ipa{tɯ-} infinitives} \label{sec:bareinf}
A handful of verbs, namely \jpg{ʑa}{begin}, \jpg{sɤʑa}{begin} and \jpg{rɲo}{experience, have already} are compatible with bare infinitive complements. These verbs also allow \ipa{kɤ-} infinitives and/or finite complement clauses, though in the case of \jpg{ʑa}{begin} these are extremely rare.

Bare infinitives occur only with transitive verbs, and are formed by combining the stem 1 of the verb with a possessive prefix coreferent with the P of the complement clause, as in examples (\ref{ex:Wmto}).

\begin{exe}
\ex \label{ex:Wmto}
\gll \ipa{nɤʑo} 	\ipa{kɯ-fse} 	\ipa{a-ŋkʰor} 	\ipa{nɯ} 	\ipa{ɯ-mto} 	\ipa{mɯ-pɯ-rɲo-t-a} \\
you \textsc{nmlz:stative}-be.like \textsc{1sg.poss}-subject \textsc{top} \textsc{3sg}-\textsc{bare.inf:}see \textsc{neg-pfv}-experience-\textsc{pst:tr-1sg} \\
\glt  `I never saw anyone like you among my subjects.' (Smanmi metog koshana1.157)
\end{exe} 

Bare infinitives are in complementary distribution with \ipa{tɯ-} infinitives,  which occur when the verb of the complement is morphologically intransitive, and thus lacks a P. They are only compatible with polarity prefixes (as in example \ref{ex:mAtWrga} below), and cannot take TAM or possessive prefixes.

XXX


Semi-transitive verbs are treated like intransitive verbs: they cannot form a bare infinitive, and use \ipa{tɯ-} infinitives instead (example \ref{ex:mAtWrga}), although their semi-object does present some object-like syntactic properties (see \citealt{jacques16relatives}).  

\begin{exe}
\ex  \label{ex:mAtWrga}
\gll \ipa{qaɟy} 	\ipa{ɯ-me} 	\ipa{nɯnɯ,} 	\ipa{tɕendɤre} 	\ipa{kʰro} 	\ipa{mɤ-tɯ-rga} 	\ipa{to-ʑa} \\
fish \textsc{3sg.poss}-daughter \textsc{dem} \textsc{lnk} a.lot \textsc{neg-inf}-like \textsc{ifr}-start \\
\glt `He started not liking the mermaid that much.' (hist150819 haidenver, 154)
\end{exe}
 
The only exception to this distribution are some transitive verbs used in complex predicates referring to weather phenomena, in particular \ipa{lɤt} `throw' and \ipa{βzu} `make, do', as in (\ref{ex:tWmlAt}). Note that in this construction, although the verbs remain morphologically transitive, they cannot take an overt A marked with the ergative.
 
\begin{exe}
\ex  \label{ex:tWmlAt}
\gll
\ipa{tɯ-mɯ} 	\ipa{kɯ-wxtɯ\tld{}wxti} 	\ipa{ʑo} 	\ipa{tɯ-lɤt} 	\ipa{pjɤ-ʑa} \\
\textsc{indef.poss}-sky \textsc{nmlz:S/A-emph}\tld{}be.big \textsc{emph} \textsc{inf}-throw \textsc{ifr}-start \\
\glt `A big rain started.' (hist150819 haidenver, 104)
\end{exe}


%\ipa{hanɯni} 	\ipa{ɯ-rŋa} 	\ipa{ra} 	\ipa{tɯ-ɣɯrni} 	\ipa{tɯ-βzu} 	\ipa{ɲɤ-ʑa} 
%hist150820 meili de meiguihua

Etymology: action nominalization or indefinite possessor?

Bare infinitive as an action nominal (\citealt{jacques14antipassive}):

\begin{exe}
\ex \label{ex:bare.inf.noun}
\gll \ipa{ndʑi-mi}   	\ipa{ɯ-tsʰoʁ}   	\ipa{ɯ-tsʰɯɣa}   	\ipa{nɯra}   	\ipa{wuma}   	\ipa{ʑo}   	\ipa{naχtɕɯɣ-ndʑi.}   \\
\textsc{3du.poss}-foot \textsc{3sg}-\textsc{bare.inf:}attach.to \textsc{3sg.poss}-form \textsc{dem:pl} very \textsc{emph}  \textsc{npst}:similar-\textsc{du}  \\
\glt `The way their feet (of fleas and crickets) touch the ground is very similar.' (26-mYaRmtsaR, 17)
\end{exe}
 
 \subsection{Complementation strategies}  
 
 \subsection{Finite} 

  \section{Morphosyntactic properties of complement clauses} 

 \subsection{Alignment} 

  \subsection{Raising of TAM and person/number marking} 
  
  \subsection{Case marking mismatch} 
  
  \subsection{Semi-finiteness} 
like relative clauses, \citealt{jacques16relatives}

 \subsection{Dislocated complement} 

Discontinuous clauses are rare in Japhug, but examples of elements from the matrix clause inserted in the complement clause can be found in the corpus. In example (\ref{ex:lWlu.kW.aZo}), the \textsc{1sg} pronoun \ipa{aʑo} (A of the matrix clause) appears between the A \ipa{lɯlu} 	\ipa{kɯ} `the cat' and the P \ipa{ʁnɯz} `two' of the complement clause. This sentence was judged to be correct by our consultant.
 
 \begin{exe}
\ex \label{ex:lWlu.kW.aZo}
\gll \ipa{tɕe} 	[\ipa{lɯlu} 	\ipa{kɯ} 	\ipa{aʑo} 	\ipa{ʁnɯz} 	\ipa{ʑo} 	\ipa{ka-ndo}] 	\ipa{pɯ-mto-t-a} \\
\textsc{lnk} cat \textsc{erg} \textsc{1sg} two \textsc{emph} \textsc{pfv}:3$\rightarrow$3'-take \textsc{pfv}-see-\textsc{pst:tr-1sg} \\
\glt `I saw a cat catching two of them.' (22-kumpGatCW, 61)
\end{exe}

  \subsection{Multiple complements}
 \ipa{ɯ-zda} 	\ipa{nɯ} 	\ipa{ra} 	\ipa{pɣɤtɕɯ} 	\ipa{kɯ-ɣɤwu,} 	\ipa{kʰɯna} 	\ipa{kɯ-ɤndzɯt,} 	\ipa{lɯlu} 	\ipa{kɯ-ɣɤwu} 	\ipa{qachɣa} 	\ipa{kɯ-mbri} 	\ipa{kɯ-fse,} 	\ipa{nɯra} 	\ipa{tu-nɯɕpɯz} 	\ipa{ɲɯ-spe.} 	
 
\subsection{Complements followed by a postposition} 
It is possible for a complement to be followed by the postposition \ipa{ma} `apart from' to express a restriction, as in example (\ref{ex:manWma.compl}).

\begin{exe}
\ex \label{ex:manWma.compl}
\gll \ipa{kɤ-mtsʰɤm} 	\ipa{ma} 	\ipa{nɯ} 	\ipa{ma} 	\ipa{mɯ-pɯ-rɲo-t-a} \\
\textsc{inf}-hear apart.from \textsc{dem} apart.from \textsc{neg-pfv}-experience-\textsc{pst:tr-1sg} \\
\glt `I only heard about it.' (I did not see it and even do not claim that it exists, of a mythological animal) (20-RmbroN, 118)
\end{exe}
 
  \section{A classification of complement-taking verbs} 
  \subsection{Modal verb}
  
    \subsubsection{\jpg{spa}{be able to}} \label{sec:spa}
The verb \jpg{spa}{be able to} is the only transitive modal verb.\footnote{The causative verb \jpg{sɯxcʰa}{(cause to) have the ability to} is also formally transitive, but is only used in inverse forms, see section XXX)} It originates from the abilitative form of the verb \ipa{pa} `do' (\citealt{jacques15causative}) and has a cognate in Tangut  (\citealt{jacques14esquisse}), showing that its lexicalization occurred even earlier than proto-Gyalrongic.

The verb \jpg{spa}{be able to} takes both infinitival (examples \ref{ex:rYo:inf:A} and \ref{ex:rYo:inf:S}) or finite complements, and its A is coreferent with the S or the A of the complement clause.

\begin{exe}
\ex  \label{ex:rYo:inf:A}
\gll
\ipa{nɯ} 	\ipa{ɯ-mdoʁ} 	\ipa{nɯ} 	\ipa{aj} 	\ipa{kɤ-ti} 	\ipa{mɯ́j-spe-a} \\
\textsc{dem} 3sg.poss-colour \textsc{dem} \textsc{1sg} \textsc{inf}-say \textsc{neg:sens}-be.able[III]-1sg \\
\glt I am not able to name its colour. (06-qaZmbri, 57)
\end{exe}

\begin{exe}
\ex  \label{ex:rYo:inf:S}
\gll 
 \ipa{kɤ-nɤre} 	\ipa{ɯ-tá-spa?}\\
 \textsc{inf}-laugh \textsc{q-pfv}:3$\rightarrow$3'-be.able.to\\
 \glt `Is he now able to laugh?' (conversation, 2014, of a three month old infant)
\end{exe}
    
  \subsubsection{\jpg{ra}{have to}} \label{sec:ra}
  \ipa{tɤ-pɤtso} 	\ipa{nɯ,} 	\ipa{tɯ-pɤrme} 	\ipa{roro} 	\ipa{jamar} 	\ipa{tɕe} 	\ipa{tɕe} 	\ipa{tɯ-nɯ} 	\ipa{kɤ-sɯβde} 	\ipa{pjɤ-ra} 
  
  \subsection{Phasal verbs and other aspectual auxiliaries}
\subsubsection{\jpg{rɲo}{experience}}   \label{sec:rɲo}

 
 
  
  \subsection{Motion verbs}
  
motion verbs and associated motion   \citet{jacques13harmonization}
  
\subsection{Causative verbs}


 %mɯ-tu-kɤ-nɤtɯti kɯ-ra tɤ́-wɣ-sɯ-βzu-a-nɯ ndʐa ɕti ma

\subsection{Complements of stative verbs}

enough to
\ipa{koŋla} 	\ipa{pjɯ-nɤrte} 	\ipa{ʑo} 	\ipa{kɯ-rtaʁ} 	\ipa{ʑo} 	\ipa{kɯ-wxti} 	\ipa{ɲɯ-βze} 	\ipa{ŋgrɤl} 

\subsection{Complements of nouns and noun-verb collocations}
\ipa{mɯ-tu-kɤ-mbro} 	\ipa{ftɕaka} 	\ipa{tu-βze-a} 	\ipa{ŋu} 

 \section{Conclusion}
 
\bibliographystyle{unified}
\bibliography{bibliogj}
\end{document}