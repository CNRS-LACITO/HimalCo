\documentclass[oneside,a4paper,11pt]{article} 
\usepackage{fontspec}
\usepackage{natbib}
\usepackage{booktabs}
\usepackage{xltxtra} 
\usepackage{polyglossia} 
%\usepackage[table]{xcolor}
\usepackage{gb4e} 
\usepackage{multicol}
\usepackage{graphicx}
\usepackage{float}
\usepackage{lineno}
\usepackage{textcomp}
\usepackage{hyperref} 
\hypersetup{bookmarksnumbered,bookmarksopenlevel=5,bookmarksdepth=5,colorlinks=true,linkcolor=blue,citecolor=blue}
\usepackage[all]{hypcap}
\usepackage{memhfixc}
 

%\setmainfont[Mapping=tex-text,Numbers=OldStyle,Ligatures=Common]{Charis SIL} 
\newfontfamily\phon[Mapping=tex-text,Ligatures=Common,Scale=MatchLowercase]{Charis SIL} 
\newcommand{\ipa}[1]{\textbf{\phon#1}} %API tjs en italique
 \newcommand{\jpg}[2]{\ipa{#1} `#2'} %API tjs en italique
\newcommand{\grise}[1]{\cellcolor{lightgray}\textbf{#1}}
\newfontfamily\cn[Mapping=tex-text,Ligatures=Common,Scale=MatchUppercase]{SimSun}%pour le chinois
\newcommand{\zh}[1]{{\cn #1}}
\newcommand{\tld}{\textasciitilde{}}

\XeTeXlinebreaklocale "zh" %使用中文换行
\XeTeXlinebreakskip = 0pt plus 1pt %
 


\begin{document} 

\title{Complementation in Japhug Rgyalrong\footnote{The glosses follow the Leipzig glossing rules. Other abbreviations used here are: \textsc{auto}  autobenefactive-spontaneous, \textsc{anticaus} anticausative, \textsc{antipass} antipassive, \textsc{appl} applicative, \textsc{dem} demonstrative,  \textsc{emph} emphatic, \textsc{fact} factual, \textsc{genr} generic, \textsc{ifr} inferential, \textsc{indef} indefinite, \textsc{inv} inverse,  \textsc{lnk} linker, \textsc{pfv} perfective, \textsc{poss} possessor, \textsc{pres} egophoric present, \textsc{prog} progressive, \textsc{sens} sensory. The examples are taken from a corpus that is progressively being made available on the Pangloss archive (\citealt{michailovsky14pangloss}). This research was funded by the HimalCo project (ANR-12-CORP-0006) and is related to the research strand LR-4.11 ‘‘Automatic Paradigm Generation and Language Description’’ of the Labex EFL (funded by the ANR/CGI). Acknowledgements   will be added after editorial decision.}} 
\author{Guillaume Jacques}
\maketitle
\linenumbers
 
 
\section{Introduction}
 \citet[9]{dixon06complementation}
 \citet{sun12complementation}
 \citet{jacques08}

\section{Background information}

\subsection{Case marking and transitivity in Japhug} \label{sec:transitivity}


\subsubsection{Semi-transitive verbs}

\subsubsection{Unmarked adjuncts} \label{sec:adjuncts}

goal


Unlike languages such as English, Hungarian or Mandinka (\citealt{creissels14functive}), Japhug lacks a special marker for essive / functive meaning.\footnote{Essive noun phrases are not arguments of the sentence, but are used to indicate `the property of fulfilling the role of an N' (\citealt[606]{creissels14functive}); in English for instance, they are marked by \textit{as} in a sentence such as `I am saying this \textit{as your friend}'.} Bare noun phrases, without any case marker, can be interpreted as essive adjuncts, as \ipa{nɤ-rʑaβ} `your wife' in (\ref{ex:YWtambi}), which is neither the object (recipient) or the theme of the verb \jpg{mbi}{give}

\begin{exe}
\ex \label{ex:YWtambi}
\gll \ipa{a-me} 	\ipa{nɯ} 	\ipa{nɤ-rʑaβ} 	\ipa{ɲɯ-ta-mbi} 	\ipa{ŋu} \\
\textsc{1sg.poss}-daughter \textsc{dem} \textsc{2sg.poss}-wife \textsc{ipfv}-1$\rightarrow$2-give be:\textsc{fact} \\
\glt `I will give you my daughter in marriage. (=I will give her to you as your wife)'
\end{exe}

Noun phrases headed by the possessed noun \jpg{ɯ-spa}{material} are often used as an essive adjuncts as in (\ref{ex:zGAmbu.Wspa}), a construction that is in the process of grammaticalizing into a purposive phrase, and serves as a grammaticalization strategy with verbs of manipulation (see section \ref{sec:essive}).

\begin{exe}
\ex \label{ex:zGAmbu.Wspa}
\gll \ipa{zɣɤmbu} 	\ipa{ɯ-spa} 	\ipa{ɲɯ-nɯ-pʰɯt-nɯ} 	\ipa{ŋgrɤl} \\
broom \textsc{3sg.poss}-material \textsc{ipfv-auto}-cut-\textsc{pl} be.usually.the.case:\textsc{fact} \\
\glt `They cut it to make brooms. (=as a material for brooms.)' (140505 sWjno, 22)
\end{exe}

\subsection{Participles vs infinitives} \label{sec:part.inf}
All Gyalrong languages, including Situ (\citealt{youjing03zhuokeji}), Tshobdun (\citealt{sun12complementation}) and Japhug, have a distinction between infinitives and participles. The distinction is quite subtle, as there are both infinitives and participles in \ipa{kɯ-} and \ipa{kɤ-} (or \ipa{kə-} and \ipa{kɐ-} depending on the transcription system). Since both categories are formally similar and can occur in similar contexts, it is crucial to clearly explain the distinction between the two, especially since Japhug slightly differs from the other Gyalrong languages in this regard.

\subsubsection{Participles}
The system of core argument participles in Japhug is relatively straightforward (\citealt{jacques16relatives}). The prefix \ipa{kɯ-} is used to build the S-participle of intransitive verbs, and the A-participle of transitive verbs. A-participles differ from S-participles in having in addition a possessive prefix coreferent with the P, as in (\ref{ex:akWfstWn}).

\begin{exe}
\ex \label{ex:akWfstWn}
\gll \ipa{a-me} 	\ipa{a-kɯ-fstɯn} 	\ipa{ŋu} \\
\textsc{1sg.poss}-daughter \textsc{1sg.poss}-\textsc{nmlz}:S/A-serve be:\textsc{fact} \\
\glt `My daughter is the one who takes care of me.' (The prince, 74)
\end{exe}

The prefix \ipa{kɤ-} on the other hand serves to build the P-participle, and can optionally take a possessive prefix coreferent with the A, as in (\ref{ex:tajmag}). Participles are compatible with polarity and associated motion prefixes (\citealt{jacques16relatives}).

\begin{exe}
   \ex \label{ex:tajmag}
   \gll
\ipa{aʑo}  	\ipa{a-mɤ-kɤ-sɯz}   	\ipa{tɤjmɤɣ}  	\ipa{nɯ}  	\ipa{kɤ-ndza}  	\ipa{mɤ-naz-a}  \\
\textsc{1sg} \textsc{1sg-neg-nmlz:P}-know mushroom \textsc{dem} \textsc{inf}-eat \textsc{neg}-dare:\textsc{fact}-\textsc{1sg} \\
\glt `I do not dare to eat the mushrooms that I do not know.' (23 mbrAZim,103)
\end{exe}

\subsubsection{Infinitives} \label{sec:infinitives}
There are four types of infinitives in Japhug: \ipa{kɯ-}, \ipa{kɤ-}, \ipa{tɯ-} and bare infinitives. The latter two are restricted to very specific constructions (see \ref{sec:bareinf}), and only the former two types are discussed in this section. 

Infinitives in \ipa{kɤ-} are by far the most common form in Japhug. The \ipa{kɯ-} form is restricted to stative verbs (including adjectives, copulas and existential verbs) and impersonal auxiliaries, but even with these verbs, \ipa{kɤ-} infinitives are used in several contexts. 

Aside from complement clauses (for which see section \ref{sec:infinitives.compl}), infinitives are used in two types of constructions, a brief overview of which is provided below.

First, infinitives occur as the citation forms of verbs and in metalinguistic discussion in Japhug, as in examples (\ref{ex:mAkWBdi}) and (\ref{ex:kAnARarphAB}) for stative vs non-stative infinitives.\footnote{This not the only available construction to express this -- the imperfective with generic person marking is also used (\citealt{jacques15generic}).}

\begin{exe}
\ex  \label{ex:mAkWBdi}
 \gll \ipa{ɯnɯnɯ} 	\ipa{tɕe} 	\ipa{tɕe} 	\ipa{ɯ-tɯ-tʂɯβ} 	\ipa{mɤ-kɯ-βdi} 	\ipa{tu-kɯ-ti} 	\ipa{ŋu} \\ 
 \textsc{dem} \textsc{lnk} \textsc{lnk} \textsc{3sg.poss-nmlz:action}-sew \textsc{neg-inf:stat}-be.good  \textsc{ipfv-genr}:A-say be:\textsc{fact}  \\
\glt  `People call this `badly sewn'.'  (12-kAtsxWb, 12)
\end{exe}

\begin{exe}
\ex \label{ex:kAnARarphAB}
 \gll \ipa{pjɯ-sɯ-ʁndi} 	\ipa{tɕe} 	\ipa{pjɯ-sɯ-sat.} \ipa{tɕe} 	\ipa{nɯ} 	\ipa{koʁmɯz} 	\ipa{nɤ} 	\ipa{cʰɯ-nɯtsɯm} 	\ipa{ɲɯ-ra.} \ipa{tɕe} 	\ipa{nɯnɯ} 	\ipa{kɤ-nɤʁarphɤβ} 	\ipa{tu-kɯ-ti} 	\ipa{ŋu} \\
 \textsc{ipfv-caus}-hit[III]  \textsc{lnk} \textsc{ipfv-caus}-kill \textsc{lnk} \textsc{dem} only.then \textsc{lnk} \textsc{ipfv:downstream}-take.away \textsc{sens}-have.to \textsc{lnk} \textsc{dem} \textsc{inf}-strike.with.wings \textsc{ipfv-genr}:A-say be:\textsc{fact}  \\
 \glt `It strikes it and kills it (with its wings) and only then takes it away. This is called \ipa{kɤ-nɤʁarphɤβ} `strike with one's wings'.' (hist150819 RarphAB, 11)
\end{exe}

In the topical position, the infinitive is neutralized to the \ipa{kɤ-} form even for stative verbs, as in (\ref{ex:kArZi}).

\begin{exe}
\ex \label{ex:kArZi}
 \gll
 \ipa{kɤ-rʑi} 	\ipa{ri} 	\ipa{pjɤ-rʑi,} 	  \\
 \textsc{inf}-be.heavy also \textsc{ifr.ipfv}-be.heavy \\
 \glt `As for being heavy, (the old man) was heavy.'  (140511 xinbada, 138)
\end{exe}

Second, infinitives are used as converbs to indicate the manner which the action of the main clause occurs (example \ref{ex:kANke.jari}), or a background event (\citealt{jacques14linking}). The \ipa{kɯ-} infinitive form occurs with stative verbs (as \jpg{sɤscit}{nice (of an environment)} in example \ref{ex:kWsAscWscit}) but it is also attested with a handful of dynamic verbs in lexicalized form such as \ipa{mɤ-kɯ-mbrɤt} `without stop' in (\ref{ex:mAkWmbrAt}).\footnote{The implied S of the verb \jpg{mbrɤt}{break, stop suddenly} (the anticausative of  \jpg{prɤt}{break}) in this sentence is the work of the subject.  } This latter use is the last trace of the contrast between human \ipa{kɐ-} and non-human \ipa{kə-} action nominals reported by \citet[476]{sun12complementation} and \citealt{jackson14morpho}, which otherwise appears to have been lost in the variety of Japhug under study.

\begin{exe}
\ex \label{ex:kANke.jari}
\gll
\ipa{kɤ-ŋke} 	\ipa{jɤ-ari} 	\ipa{pɯ-ra} \\
\textsc{inf}-walk \textsc{pfv}-go[II] \textsc{pst.ipfv}-have.to \\
\glt He had to go on foot. (elicited)
\end{exe}

\begin{exe}
\ex \label{ex:kWsAscWscit}
\gll
\ipa{ɕɤr} 	\ipa{tɕe} 	\ipa{nɯtɕu} 	\ipa{kɯ-sɤ-scɯ\tld{}scit} 	\ipa{ʑo} 	\ipa{ɕ-ku-nɯ-rŋgɯ} 	\ipa{ŋu} \\
night \textsc{lnk} \textsc{dem:loc} \textsc{inf:stat-deexp-emph}\tld{}be.happy \textsc{emph} \textsc{transloc-ipfv-auto}-lie.down be:\textsc{fact} \\
\glt `In the night, he goes in there to sleep cosily.' (26-NalitCaRmbWm, 35)
\end{exe}

\begin{exe}
\ex \label{ex:mAkWmbrAt}
\gll
 \ipa{nɯ} 	\ipa{maka} 	\ipa{mɤ-kɯ-mbrɤt} 	\ipa{ʑo} 	\ipa{ɲɯ-rɤma} 	\ipa{ɲɯ-ɕti} 	\ipa{tɕe} \\
 \textsc{dem} at.all \textsc{neg-inf-anticaus:}break \textsc{emph} \textsc{ipfv}-work \textsc{sens}-be:\textsc{affirm} \textsc{lnk} \\
\glt `It works without stop.' (hist-26-GZo.txt 67)
\end{exe}
 
\subsubsection{Japhug vs Tshobdun}
The inventory of participial and infinitive forms presented above for Japhug differs from other previosu descriptions of Gyalrong languages. In Tsobdun, \citet[476]{sun12complementation} describes five types of verbs forms with \ipa{kə-} or \ipa{kɐ-} prefixes (Table \ref{tab:tshobdun.nmlz}).

\begin{table}[H]
\caption{Nominalization types in Tshobdun (\citealt[476]{sun12complementation}) } \label{tab:tshobdun.nmlz} \centering
\begin{tabular}{lllllll}
\toprule
Type & Scope & Finiteness & Argument & Prefix \\
&&&coding \\
\midrule
purposive & clausal & non-finite& possessor & \ipa{kə-} \\
participant & clausal & non-finite& possessor & \ipa{kə-} (subject) \\
&&&& \ipa{kɐ-} (object)\\
infinitive & clausal & non-finite& normal &  \ipa{kɐ-}  \\
action / state & clausal & non-finite& normal &  \ipa{kɐ-} [+human] \\
&&&&\ipa{kə-}  [-human] \\
finite  & clausal & finite& normal &  \ipa{kə-} \\
\bottomrule
\end{tabular}
\end{table}

XXX conflating categories

\subsection{Complement clauses vs complementation strategies} \label{sec:strategies}

\citet[15]{dixon06complementation} defines \textit{complement clauses} as subordinate clauses which function as one of the the core argument of a main clause. 

He introduces the term `complementation strategy' to refer to constructions corresponding with a meaning expressed by complement clauses in some languages, which either are not core arguments or the verb of the main clause or are not clauses with a complete argument structure \citealt[34-40]{dixon06complementation}). Complement strategies include nominalizations (when the verb sheds its argument structure as it becomes a noun), relative clauses (which are formally modifier of a core argument, not core arguments themselves), serial verb constructions and clause linking.

In Japhug, due to the existence of semi-objects, which are demonstrably core arguments despite not being indexed on the verb (see \citet{jacques16relatives}), and due to the presence of unmarked adjuncts with various syntactic properties (see \ref{sec:transitivity}), the status of a particular clause as a core argument or non-core-argument is not always trivial to determine. In this paper, I keep Dixon's sensible distinction between complement clauses and complementation strategies, but given the gradient nature of the opposition between core and oblique arguments in Japhug, I use the term `complement clauses' to refer to all clauses which are part of the verb's argument structure. In particular, in consider the purposive clauses of motion verb to be complement clauses, for reasons that are exposed in more detail in section (\ref{sec:SApart}).
 

\section{Complement types} \label{sec:complement.types}


\subsection{Infinitive} \label{sec:infinitives.compl}
The most common type of complement clauses in Japhug are \ipa{kɤ-} and \ipa{kɯ-} infinitival complements. As seen in section \ref{sec:infinitives}, there are \ipa{kɤ-} and \ipa{kɯ-} infinitives in Japhug, the latter found in the citation form of stative verbs and modal impersonal auxiliary verbs. In complement clauses, stative verbs take the \ipa{kɤ-} infinitive like dynamic verbs in many cases.

\subsubsection{Case marking} \label{sec:case.infinitive}
While infinitives bear no person indexation markers, noun phrases receive the same case markers in infinitive clauses as in main clauses, showing that infinitives have the same argument structures as finite verb forms.

When an argument is shared between the complement and the matrix clause, it often has a different syntactic function in the two clauses, as in \ref{ex:kAstu}, where \ipa{tɤɕime} `princess' is A in the complement clause (\jpg{stu}{do like this} is transitive) and S in the matrix clause (\jpg{cʰa}{can} is intransitive). 

\begin{exe}
\ex \label{ex:kAstu}
\gll [\ipa{tɤɕime} 	\ipa{nɯ} 	\ipa{kɯ} 	\ipa{nɯra} 	\ipa{kɤ-stu}] 	\ipa{pjɤ-cʰa} \\
princess \textsc{dem} \textsc{erg} \textsc{dem:pl} \textsc{inf}-do.like.this \textsc{ifr}-can \\
\glt `The princess succeeded in doing it.' (140511 alading, 252)
\end{exe}

In this sentence, the noun takes the ergative marker \ipa{kɯ} following the verb of the complement clause, showing that it belongs to the complement clause rather than to the matrix clause directly. This is the most commonly observed pattern in Japhug texts: in infinitival clauses, the shared arguments more often take the case marking selected by the verb of the complement clause than that of the matrix clause.

\subsubsection{Coreference between matrix and complement clause} \label{sec:inf.coref}
Coreference restrictions between complements in \ipa{kɤ-} finitives and the matrix clauses vary from verb to verbs, and three cases can be distinguished.


First, in the case of impersonal verbs such as \jpg{ra}{have to, need} (see section \ref{sec:ra}), the complement clause is the S and there is no argument coereference between the matrix clause and the complement clause.

Second, with a few transitive complement-taking verbs such as \ipa{spa} `be able' (see section \ref{sec:spa}), the subjects of both clauses must be coreferent.

Third, for most verbs taking infinitives (like the semi-transitive \jpg{rga}{like} or the transitive \jpg{rɲo}{experience}), the subject of the matrix clauses can be coreferent to either the S  (\ref{ex:kAnWrAGo.rganW}), the A (\ref{ex:kAnArtoXpjAt.pWrgaa}) and even the P (\ref{ex:YWrganW} and \ref{ex:kAmtsWG.P})  of its infinitival complement clause (\citealt{jacques16relatives}).

 \begin{exe}
   \ex   \label{ex:kAnWrAGo.rganW} 
\gll
\ipa{tsuku}  	\ipa{tɕe}  	\ipa{kɤ-nɯrɤɣo}  	\ipa{wuma}  	\ipa{ʑo}  	\ipa{rga-nɯ}  	\ipa{tɕe}  \\
some \textsc{lnk} \textsc{inf}-sing really \textsc{emph} like:\textsc{fact-pl}  \textsc{lnk} \\
  \glt Some people like to sing. (26 kWrNukWGndZWr, 104) 
 \glt I liked to observe bugs. (26 quspunmbro, 15) (S=S)
     \end{exe}  
 
   \begin{exe}
   \ex   \label{ex:kAnArtoXpjAt.pWrgaa} 
\gll
  	\ipa{aʑo}  	\ipa{qajɯ}  	\ipa{nɯ} \ipa{ra}  	\ipa{kɤ-nɤrtoχpjɤt}  	\ipa{pɯ-rga-a}  	\ipa{tɕe}  	\\
  	\textsc{1sg} bugs \textsc{dem} \textsc{pl} \textsc{inf}-observe \textsc{pst.ipfv}-like-\textsc{1sg} \textsc{lnk}  \\
 \glt I liked to observe bugs. (26 quspunmbro, 15) (A=S)
     \end{exe}  
 
  \begin{exe}
   \ex   \label{ex:YWrganW} 
\gll
\ipa{maka}  	\ipa{tu-kɤ-nɤjoʁjoʁ,}  	\ipa{tu-kɤ-fstɤt}  	\ipa{nɯ}  	\ipa{ɲɯ-rga-nɯ}  \\
at.all \textsc{ipfv-inf}-flatter \textsc{ipfv-inf}-praise \textsc{dem} \textsc{ipfv}-like-\textsc{pl} \\
\glt They like to be flattered or praised. (140427 yuanhou, 53) (P=S)
    \end{exe}  
      \begin{exe}
   \ex   \label{ex:kAmtsWG.P} 
\gll 
\ipa{aʑo} 	\ipa{kɤ-mtsɯɣ} 	\ipa{mɯ-pɯ-rɲo-t-a} 	\ipa{ri,} 	\ipa{χpɤltɕɯn} 	\ipa{kɯ} 	\ipa{pjɤ-rɲo} 	
 \\
\textsc{1sg} \textsc{inf}-bite \textsc{neg-pfv}-experience-\textsc{pst:tr-1sg} but Dpalcan \textsc{erg} \textsc{ifr}-experience \\
\glt I have never been stung (by a wasp), but Dpalcan has. (26-ndzWrnaR, 19) (P=A)
    \end{exe}  

The subject of the matrix clause can even be coreferent with the possessor of the S, as in example (\ref{ex:kAmNAm}). The that the the verb \ipa{mŋɤm} `hurt' in this infinitive clause can only take a body part as its S -- the experiencer is indicated by a possessive prefix on the body part, as in (\ref{ex:YWmNAm}).
 
 \begin{exe}
\ex \label{ex:kAmNAm}
\gll \ipa{aʑo} 	\ipa{pɯ-xtɕɯ\tld{}xtɕi-a} 	\ipa{ʑo} 	\ipa{ri} 	\ipa{tɯxtɤŋɤm} 	\ipa{nɯ-atɯɣ-a} 	\ipa{tɕe,} 	\ipa{nɯ} 	\ipa{kɤ-mŋɤm} 	\ipa{pɯ-rɲo-t-a} \\
\textsc{1sg} \textsc{pst:ipfv-emph}\tld{}be.small-\textsc{1sg} \textsc{emph} \textsc{loc} dysentery \textsc{pfv}-meet-\textsc{1sg} \textsc{lnk} \textsc{dem} \textsc{inf}-hurt \textsc{pfv}-experience-\textsc{1sg} \\
\glt `When I was very small, I had dysentery, (my belly) ached.'  (24-pGArtsAG, 121)
\end{exe}

 \begin{exe}
\ex \label{ex:YWmNAm}
\gll \ipa{a-xtu} 	\ipa{ɲɯ-mŋɤm} \\
\textsc{1sg.poss}-belly \textsc{sens}-hurt \\
\glt `My belly hurts.' 
\end{exe}
 
 \subsubsection{Stative infinitive} \label{sec:stative.inf}
Stative verbs, when occurring in a complement clause, generally take the \ipa{kɤ-} infinitive, as in example  (\ref{ex:rYo}) and (\ref{ex:kAscit}). The main verb of the complement clauses in these examples have the \ipa{kɤ-} infinitive, even though both \jpg{tu}{exist} and \jpg{scit}{be happy} are stative verbs and have a citation form with the \ipa{kɯ-} prefix.

\begin{exe}
\ex \label{ex:rYo}
\gll \ipa{a-rŋɯl} 	\ipa{kɤ-tu} 	\ipa{pɯ-rɲo-t-a} \\
\textsc{1sg.poss}-money \textsc{inf}-exist \textsc{pst:ipfv}-experience-\textsc{pst:tr-1sg} \\
\glt `I used to have money'. (elicited)
\end{exe}

\begin{exe}
\ex \label{ex:kAscit}
 \gll \ipa{kɤ-scit} 	\ipa{pjɤ-ŋgrɯ} 	\ipa{ɲɯ-ŋu}  \\
 inf-be.happy ifr-succeed sens-be \\
 \glt `She succeeded in being happy.' (150818 muzhi guniang, 6)
 \end{exe} 
 
In complement clauses, \ipa{kɯ-} infinitives are uncommon, and only occur in three cases.
 
 First, the conversion to \ipa{kɤ-} infinitive only applies to stative verbs, not to  impersonal modal verbs such as \jpg{ra}{have to, need}. When the latter occur in a complement clause, as in example (\ref{ex:kAndza.kWra}), they always have the \ipa{kɯ-} prefix.

\begin{exe}
\ex \label{ex:kAndza.kWra}
\gll 
\ipa{smɤn} 	\ipa{kɤ-ndza} 	\ipa{kɯ-ra} 	\ipa{pɯ-rɲo-t-a} \\ 
medecine \textsc{inf}-eat \textsc{inf}-have.to  \textsc{pst:ipfv}-experience-\textsc{pst:tr-1sg} \\
\glt  `I used to have to take medecine.' 
\end{exe}
 
 Second, complements of verb of perception or thought like \jpg{sɯpa}{consider as} or \jpg{sɯχsɤl}{recognize, realize} can take complement clauses with stative infinitives as objects, as in (\ref{ex:kWNu.nW}).
\begin{exe}
\ex \label{ex:kWNu.nW}
\gll \ipa{ɯʑo} 	\ipa{srɯnmɯ} 	\ipa{kɯ-ŋu} 	\ipa{nɯ} 	\ipa{tɤ-wa} 	\ipa{nɯ} 	\ipa{kɯ} 	\ipa{mɯ-pjɤ-sɯχsɤl,} 	\ipa{ɯ-nmaʁ} 	\ipa{nɯ} 	\ipa{kɯ} \\
\textsc{3sg} râkshasî \textsc{stat.inf}-be \textsc{dem} \textsc{indef.poss}-father \textsc{dem} \textsc{erg} \textsc{neg-ifr}-recognize \textsc{3sg.poss}-husband \textsc{dem} \textsc{erg} \\
\glt  `That she was a râkshasî, the father did not realize it, her husband.' (28-smAnmi, 62)
\end{exe}

Third, some adjectives like \jpg{pʰɤn}{be efficient} can take stative infinitives (section \ref{sec:adj.infinitive})

 \subsection{S/A participles} \label{sec:SApart}
In Japhug, a handful of verbs select clauses with S/A-participles rather than infinitival clauses, including  some motion verbs (\jpg{ɕe}{go}, \jpg{ɣi}{come}, but not \jpg{rɟɯɣ}{run}) and one aspectual verb  (\jpg{rɤŋgat}{be about to}). These verbs are all morphologically intransitive.

Example (\ref{ex:WkWnAjo}) illustrates this construction, with the A-participle \ipa{ɯ-kɯ-n-nɤjo} `waiting for him'. Note that the common argument shared between the participial clause (whose verb is transitive) and the matrix clause (whose verb is intransitive) takes the ergative, showing that it belongs to the participial clause, a pattern already observed with infinitival complement (see section \ref{sec:case.infinitive}).

\begin{exe}
\ex \label{ex:WkWnAjo}
\gll [\ipa{ɯ-wa} 	\ipa{nɯ} 	\ipa{kɯ} 	\ipa{kʰapa} 	\ipa{tɕe} 	\ipa{ɯ-kɯ-n-nɤjo}] 	\ipa{pjɤ-ɣi} \\
\textsc{3sg.poss}-father \textsc{dem} \textsc{erg} downstairs \textsc{lnk} \textsc{3sg-nmlz:S/A-auto}-wait \textsc{ifr:down}-come \\
\glt `His father had come downstairs to wait for him.' (140506 loBzi, 5)
\end{exe}


A similar construction is also found in Tshobdun, where \citet{sun12complementation} (Following Dixon) treats these clauses as a relativization strategy rather than as proper complements, since these clauses are not core arguments of the verb. However, this construction is highly grammaticalized and specific to a class of verbs which do not form a natural semantic class, which suggests that the participial clauses are selected by these verbs' argument structure. Though not core arguments, the participial clauses can be treated as an unmarked oblique arguments of the verb, and therefore be analyzed as genuine complement clauses.

\subsubsection{S/A- vs P-participles} \label{sec:SAparticiple.coref}
In examples such as (\ref{ex:WkWnAjo}), the S or A of the participial clause is obligatorily coreferent with the S of the matrix clause.

Coreference between the P of the participial clause and the S of the matrix clause is possible but requires using the P-participle. Several examples of this construction are found in the corpus with the verb \jpg{nɤkʰu}{invite to one's home as a guest}, as in (\ref{ex:kAnAkhu}).

\begin{exe}
\ex \label{ex:kAnAkhu}
\gll <xingqi> 	\ipa{raŋri} 	\ipa{ʑo} 	\ipa{tɕe} 	\ipa{nɯnɯ} \ipa{sɤβʑɯ} 	\ipa{ɣɯ} 	\ipa{ɯ-kʰa} 	\ipa{nɯtɕu} 	\ipa{kɤ-nɤkʰu}] 	\ipa{ju-ɣi} 	\ipa{pjɤ-ŋu} \\
week each \textsc{emph} \textsc{lnk} \textsc{dem} mouse \textsc{gen} \textsc{3sg.poss}-house \textsc{dem:loc} \textsc{nmlz:P}-invite \textsc{ipfv}-come \textsc{ipfv.ifr}-be \\
\glt `He would come as a guest to the mouse's house as a guest.' (150818 muzhi guniang, 299).
\end{exe}

We know that this form is the P-participle rather than the infinitive because it is possible to optionally add a possessive prefix coreferent with the A, as in (\ref{ex:akAnAkhu}).

\begin{exe}
\ex \label{ex:akAnAkhu}
\gll 
\ipa{a-kɤ-nɤkʰu} 	\ipa{jo-ɣi}  \\
 \textsc{1sg.poss-nmlz:P}-invite \textsc{ifr}-come \\
\glt `He came to my house as a guest.' (elicited)
\end{exe}

Coreference between the P of the complement clause and the S of the matrix clause is possible only if the P has control over the action, something that is possible for only verb few transitive verbs and explains the rarity of this construction. 
    
\subsubsection{Complements of participles}
When a complement-bearing verb is itself in the S/A-participle form, it is possible for the complement either to be in the expected form (infinitive or finite), or to be in S/A-participle form itself by contagion, as in example (\ref{ndZikWsAndu}).

\begin{exe}
\ex \label{ndZikWsAndu}
\gll [\ipa{rŋɯl} 	\ipa{kɯ} 	\ipa{ndʑi-kɯ-sɤndu}] 	\ipa{kɯ-cha} 	\ipa{kɯ-fse} 	\ipa{pɯ\tld{}pɯ-tu} 	\ipa{nɤ} \\
silver \textsc{erg} \textsc{3du-nmlz}:S/A-exchange \textsc{nmlz}:S/A-can \textsc{nmlz}:S/A-be.like 
\textsc{cond}\tld{}\textsc{pst.ipfv}-exist if \\
\glt `If there was someone who could exchange (the life of two brothers) with money, ...' (140507 jinniao, 339)
\end{exe}

Such examples are rare, but not considered to be mistakes by consultants when listening again to the recordings.

\subsubsection{Complement or relative clause?} \label{sec:relative.q}
The transitive verb \jpg{nɯɕpɯz}{pretend, imitate} and the semi-transitive \jpg{ʑɣɤpa}{pretend} superficially appear to allow participial complements like motion verbs, as could be deduced from examples such as (\ref{ex:YWkWrABraR}). 

\begin{exe}
\ex \label{ex:YWkWrABraR}
\gll \ipa{tɤ-mu} 	\ipa{nɯ} 	\ipa{kɯ} 	\ipa{ɯ-ku} 	\ipa{ci} 	\ipa{ɲɯ-kɯ-rɤβraʁ} 	\ipa{to-nɯɕpɯz} \\
\textsc{indef.poss}-mother \textsc{dem} \textsc{erg} \textsc{3sg.poss}-head \textsc{indef} \textsc{ipfv-nmlz}:S/A-scratch \textsc{ifr}-pretend \\
\glt `The (rakshasi)-mother pretended to scratch her head.' (Slob.dpon1, 
\end{exe}

However, unlike verbs such as \jpg{ɕe}{go} or \jpg{rɤŋgat}{be about to},  coreference between the subject of pretence verbs and the subject of the verb in participial form is not required, as in example (\ref{ex:pGatCW.kWGAwu}).

\begin{exe}
\ex \label{ex:pGatCW.kWGAwu}
\gll  \ipa{ɯ-zda} 	\ipa{nɯra,} 	\ipa{pɣɤtɕɯ} 	\ipa{kɯ-ɣɤwu,} 	\ipa{kʰɯna} 	\ipa{kɯ-ɤndzɯt,} 	\ipa{lɯlu} 	\ipa{kɯ-ɣɤwu} 	\ipa{qacʰɣa} 	\ipa{kɯ-mbri} 	\ipa{kɯ-fse,} 	\ipa{nɯra} 	\ipa{tu-nɯɕpɯz} 	\ipa{ɲɯ-spe.} \\
\textsc{3sg.poss}-companion \textsc{dem:pl} bird \textsc{nmlz}:S/A-cry dog \textsc{nmlz}:S/A-bark cat \textsc{nmlz}:S/A-cry fox \textsc{nmlz}:S/A-cry \textsc{inf:stat}-be.like \textsc{dem:pl} \textsc{ipfv}-imitate \textsc{sens}-be.able[III] \\
\glt `It is able to imitate other animals, cry like a bird, bark like a dog, mew like a cat or call like a fox.' (27-kikakCi, 141)
\end{exe}

Since the verb \jpg{nɯɕpɯz}{pretend, imitate} is transitive and can take as its P the person imitated by the A, as in (\ref{ex:tuWGnWCpWzndZi}), a different analysis of (\ref{ex:YWkWrABraR}) and (\ref{ex:pGatCW.kWGAwu}) offers itself: the phrases containing the S/A-participles there are not complement clauses, but in fact head-internal relatives.\footnote{In Tshobdun, \citet[481]{sun12complementation} also argues that these clauses are distinct from the complements of motion verbs, but in this language it is not possible to analyze them as participles. } 

\begin{exe}
\ex \label{ex:tuWGnWCpWzndZi}
\gll \ipa{ɣzɯ} 	\ipa{ra} 	\ipa{kɯ} 	\ipa{li} 	\ipa{ʑɤni} 	\ipa{tú-wɣ-nɯɕpɯz-ndʑi} \\
monkey \textsc{pl} \textsc{erg} again \textsc{3du} \textsc{ipfv-inv}-imitate-\textsc{du} \\
\glt `The monkeys imitated the two of them (and repeatedly threw back the coconuts at them).' (140511 xinbada, 262)
\end{exe} 

Examples (\ref{ex:YWkWrABraR}) and (\ref{ex:pGatCW.kWGAwu}) could be literally translated as `The mother pretended to be someone who is scratching her head' and `It is able to imitate a crying bird' respectively.

Yet, it is possible that the use of participles as complements with motion verbs and with \jpg{rɤŋgat}{be about to} was grammaticalized from a construction with such a head-internal relative clause in essive function instead of P function, as in (\ref{ex:WkWnAjo2}).

\begin{exe}
\ex \label{ex:WkWnAjo2}
\gll 	\ipa{ɯ-kɯ-nɤjo} 	\ipa{pjɤ-ɣi} \\
 \textsc{3sg-nmlz:S/A}-wait \textsc{ifr:down}-come \\
\glt *`He came as someone waiting for him' $\Rightarrow$ `He came to wait for him'. (from example \ref{ex:WkWnAjo})
\end{exe}

Reanalysis was complete when coreference between the subjects of the matrix and of the participle became obligatory.

%\begin{exe}
%\ex   \label{ex:pWkWZGAnWBlu} 
%\gll \ipa{pɯ-kɯ-ʑɣɤ-nɯβlu} \ipa{to-nɯɕpɯz} \\
%\textsc{pfv-nmlz:S/A-refl}-cheat \textsc{ifr}-pretend \\
%\glt  `He pretended having been duped (=He pretended to be someone who has let himself be cheated).' (Elicitation)
%\end{exe}  

 \subsection{Bare infinitive and \ipa{tɯ-} infinitives} \label{sec:bareinf}
Several phasal verbs, such as \jpg{ʑa}{begin}, \jpg{sɤʑa}{begin}, \jpg{stʰɯt}{finish}, \jpg{jɤɣ}{finish}, \ipa{ɣɤ-} causative verbs such as \jpg{ɣɤtɕʰom}{overdo, do too much} or \jpg{ɣɤβdi}{do well} and the aspectual verb \jpg{rɲo}{experience, have already} are compatible with bare infinitival and \ipa{tɯ-} infinitival complements. The verbs \jpg{sɤʑa}{begin} and \jpg{rɲo}{experience} are more commonly used with \ipa{kɤ-} infinitives. With the exception of \jpg{jɤɣ}{finish}, these verbs are all transitive.


\subsubsection{Bare infinitives and transitivity}
Bare infinitives, and are formed by combining the stem 1 of the verb with a possessive prefix coreferent with the object of the complement clause, as in examples (\ref{ex:Wmto}). Intransitive verbs have no bare infinitives.

\begin{exe}
\ex \label{ex:Wmto}
\gll \ipa{nɤʑo} 	\ipa{kɯ-fse} 	\ipa{a-ŋkʰor} 	\ipa{nɯ} 	\ipa{ɯ-mto} 	\ipa{mɯ-pɯ-rɲo-t-a} \\
you \textsc{nmlz:stative}-be.like \textsc{1sg.poss}-subject \textsc{top} \textsc{3sg}-\textsc{bare.inf:}see \textsc{neg-pfv}-experience-\textsc{pst:tr-1sg} \\
\glt  `I never saw anyone like you among my subjects.' (Smanmi metog koshana1.157)
\end{exe} 

Bare infinitives are in complementary distribution with \ipa{tɯ-} infinitives,  which occur when the verb of the complement is morphologically intransitive, and thus lacks an object. In this case, there is obligatory coreference between the A of the matrix verb and the S of the \ipa{tɯ-} infinitival complement. If overt, the noun phrase corresponding to the shared argument is generally in the absolutive form as in example \ref{ex:tWNke}, following the verb of the complement clause, though a few examples such as (\ref{ex:tWnWrAGo}) in the ergative are also attested.

\begin{exe}
\ex \label{ex:tWNke}
\gll
<xinbada> 	\ipa{nɯ} 	\ipa{tɕe} 	\ipa{li} 	\ipa{tɯ-ŋke} 	\ipa{to-ʑa} \\
Sinbad \textsc{dem} \textsc{lnk} again  \textsc{inf}-walk \textsc{ifr}-begin \\
\glt `Sinbad started to walk again.' (140511 xinbada, 217)
\end{exe}


\begin{exe}
\ex \label{ex:tWnWrAGo}
\gll \ipa{pɣɤtɕɯ} 	\ipa{nɯ} 	\ipa{kɯ} 	\ipa{nɯɕɯmɯma} 	\ipa{ʑo} 	\ipa{tɯ-nɯrɤɣo} 	\ipa{cʰɤ-ʑa} \\
bird \textsc{dem} \textsc{erg} immediately \textsc{emph} \textsc{inf}-sing \textsc{ifr}-begin \\
\glt `The bird immediately started to sing.' (140514 huishuohua de niao, 221)
\end{exe}

These infinitives are only compatible with polarity prefixes (as in example \ref{ex:mAtWrga} below), and cannot take TAM or possessive prefixes.

Crucially, semi-transitive verbs are treated like intransitive verbs: they cannot form a bare infinitive, and use \ipa{tɯ-} infinitives instead (example \ref{ex:mAtWrga}), although their semi-object does present some object-like syntactic properties (see \citealt{jacques16relatives}).  

\begin{exe}
\ex  \label{ex:mAtWrga}
\gll \ipa{qaɟy} 	\ipa{ɯ-me} 	\ipa{nɯnɯ,} 	\ipa{tɕendɤre} 	\ipa{kʰro} 	\ipa{mɤ-tɯ-rga} 	\ipa{to-ʑa} \\
fish \textsc{3sg.poss}-daughter \textsc{dem} \textsc{lnk} a.lot \textsc{neg-inf}-like \textsc{ifr}-start \\
\glt `He started not liking the mermaid that much.' (hist150819 haidenver, 154)
\end{exe}
 
The only exception to this distribution are some transitive verbs used in complex predicates referring to weather phenomena, in particular \ipa{lɤt} `throw' and \ipa{βzu} `make, do', as in (\ref{ex:tWmlAt}). Note that in this construction, although the verbs remain morphologically transitive, they cannot take an overt A marked with the ergative.
 
\begin{exe}
\ex  \label{ex:tWmlAt}
\gll
\ipa{tɯ-mɯ} 	\ipa{kɯ-wxtɯ\tld{}wxti} 	\ipa{ʑo} 	\ipa{tɯ-lɤt} 	\ipa{pjɤ-ʑa} \\
\textsc{indef.poss}-sky \textsc{nmlz:S/A-emph}\tld{}be.big \textsc{emph} \textsc{inf}-throw \textsc{ifr}-start \\
\glt `A big rain started.' (hist150819 haidenver, 104)
\end{exe}


%\ipa{hanɯni} 	\ipa{ɯ-rŋa} 	\ipa{ra} 	\ipa{tɯ-ɣɯrni} 	\ipa{tɯ-βzu} 	\ipa{ɲɤ-ʑa} 
%hist150820 meili de meiguihua

\subsubsection{Coreference restrictions}
Bare infinitives and \ipa{kɤ-} infinitives strongly differ as to their coreference restrictions. With \ipa{kɤ-} infinitives, the subject of the matrix clause can be coreferent with either the subject, the object or even the possessor of the intransitive subject of the complement clause (see example \ref{ex:kAmNAm} above). This ambiguity is particularly clear with the verb \ipa{nɤkʰu} `invite to one's home as a guest' (see examples \ref{ex:kAnAkhu1} and \ref{ex:kAnAkhu2}), as with this verb both arguments are equal in term of volition and control.

\begin{exe}
\ex  \label{ex:kAnAkhu1}
\gll
\ipa{ɯʑo} 	\ipa{kɯ} 	\ipa{kɤ-nɤkʰu} 	\ipa{pɯ-rɲo-t-a} \\
\textsc{3sg} \textsc{erg} \textsc{inf}-invite \textsc{pfv}-experience-\textsc{pst:tr-1sg} \\
\glt `I have been to his house as guest.'  (= `He has invited me to come to his house as a guest') (P=A)
\ex  \label{ex:kAnAkhu2}
\gll
\ipa{ɯʑo} 	\ipa{kɤ-nɤkʰu} 	\ipa{pɯ-rɲo-t-a} \\
\textsc{3sg}  \textsc{inf}-invite \textsc{pfv}-experience-\textsc{pst:tr-1sg} \\
\glt `He has been to my house as guest.' (= `I have invited him to come to my house as a guest.') (A=A)
\end{exe}

In the case of bare infinitives, on the other hand, the subjects of the matrix and complement clause must be identical, as shown by examples (\ref{ex:nAkhu1}) and (\ref{ex:nAkhu2}). The object of the matrix clause can however be neutralized to third person, as in (\ref{ex:nAkhu1}).

\begin{exe}
\ex  \label{ex:nAkhu1}
\gll \ipa{a-nɤkʰu} 	\ipa{pa-rɲo} \\
\textsc{1sg.poss-bare.inf:}invite \textsc{pfv:3$\rightarrow$3'}-experience \\
\glt `I have been to his house as guest.' 
\ex  \label{ex:nAkhu2}
\gll \ipa{ɯʑo} 	\ipa{ɯ-nɤkʰu} 	\ipa{pɯ-rɲo-t-a} \\
\textsc{3sg}  \textsc{3sg.poss-bare.inf:}invite \textsc{pfv}-experience-\textsc{pst:tr-1sg} \\
\glt `He has been to my house as guest.'
\end{exe}

This generalization is observed for all transitive verbs taking bare infinitive complement clauses. The intransitive impersonal verb \jpg{jɤɣ}{finish} takes the bare infinitive clause in S function, and remains in third person singular regardless of the subject and object of the complement clause, as in (\ref{ex:Wti.tojAG}), where although the subject of the complement clause is third person plural, no plural marker can appear on \ipa{jɤɣ}.

\begin{exe}
\ex \label{ex:Wti.tojAG}
\gll \ipa{nɯra} 	\ipa{ɯ-ti} 	\ipa{to-jɤɣ} \ipa{tɕe} \\
\textsc{dem:pl} \textsc{3sg.poss-bare.inf}:say \textsc{ifr}-finish \textsc{lnk}\\
\glt `After having finished saying that, (they went to the park)' (140515 congming de wusui xiaohai, 15)
\end{exe}

\subsubsection{Historical origin}
Bare infinitives probably derive from action nominals. There are marginal examples in Japhug of bare infinitives used in this way, as in example \ref{ex:bare.inf.noun})  (\citealt{jacques14antipassive}.

\begin{exe}
\ex \label{ex:bare.inf.noun}
\gll \ipa{ndʑi-mi}   	\ipa{ɯ-tsʰoʁ}   	\ipa{ɯ-tsʰɯɣa}   	\ipa{nɯra}   	\ipa{wuma}   	\ipa{ʑo}   	\ipa{naχtɕɯɣ-ndʑi.}   \\
\textsc{3du.poss}-foot \textsc{3sg}-\textsc{bare.inf:}attach.to \textsc{3sg.poss}-form \textsc{dem:pl} very \textsc{emph}  \textsc{npst}:similar-\textsc{du}  \\
\glt `The way their feet (of fleas and crickets) touch the ground is very similar.' (26-mYaRmtsaR, 17)
\end{exe}

The complementary distribution between the bare infinitive and the \ipa{tɯ-} infinitive is puzzling. There are at least two possible ways of analyzing the origin of  \ipa{tɯ-} infinitives. 


First, they could be related to the \ipa{tɯ-} action nominals (\citealt{jacques14antipassive}), found in light verb constructions such as (\ref{ex:tWrJaR}) or (\ref{ex:tWtWtsxaB})\footnote{When the \ipa{tɯ-} nominalization prefix is reduplicated as in (\ref{ex:tWtWtsxaB}), it conveys the meaning of several persons/objects being subjected to the same action together. } and used to build abstract nouns (\ipa{si} `die' $\rightarrow$ \ipa{tɯ-si} `death'). This solution is attractive due to the fact that \ipa{tɯ-} action nominals are relatively common, but it does not account well for the complementary distribution of bare infinitives and \ipa{tɯ-} infinitives, since both intransitive and transitive verbs can build \ipa{tɯ-} action nominals.

\begin{exe}
\ex \label{ex:tWrJaR}
\gll \ipa{tɯ-rɟaʁ} \ipa{pɯ-βzu-t-a} \\
\textsc{nmlz:action}-dance \textsc{pfv}-do-\textsc{pst:tr-1sg} \\
\glt `I danced.'
\end{exe}

\begin{exe}
\ex \label{ex:tWtWtsxaB}
\gll
\ipa{kuxtɕo} 	\ipa{cʰondɤre}  	\ipa{kɯrtsɤɣ} 	\ipa{nɯra} 	\ipa{ɯ-pa} 	\ipa{nɯtɕu} 	\ipa{tɯ\tld{}tɯ-tʂaβ} 	\ipa{ʑo} 	\ipa{pjɤ-βzu} \\
basket \textsc{comit} leopard \textsc{dem:pl} \textsc{3sg}-down \textsc{dem:loc} \textsc{together}\tld{}\textsc{nmlz:action}-cause.to.roll \textsc{emph} \textsc{ifr:down}-make \\
\glt `(The rabbit) caused the leopard and the basket to roll down together.' (The rabbit 2002, 72)
\end{exe}

Second, one could interpret the \ipa{tɯ-} here as the indefinite possessor prefix \ipa{tɯ-}, which is added to inalienably possessed nouns when no definite possessor is present (\citealt{jacques15generic}). In this hypothesis, the bare infinitive takes a possessive prefix coreferent with its P with transitive verbs, but in the case of intransitive verbs, given the absence of P argument, the indefinite possessor is used instead.

 \subsection{Finite complements} 
Finite subordinate clauses are common in Japhug and other Gyalrong languages. For instance, some specific types of relative clauses (\citealt{jackson06guanxiju, jacques16relatives} or temporal subordinate clauses (\citealt{jacques14linking}) take a verb in finite form, in some case without subordinator.\footnote{Finite relatives, however, have some properties that distinguish them from corresponding independent clauses (\citealt[18-21]{jacques16relatives}).} Likewise, finite complement clauses are common in Japhug.


\subsubsection{TAM forms} \label{sec:TAM.finite}
Like finite relatives, there are constraints on the TAM forms that the main verb of a finite complement can take. three cases can be distinguished.

The most common type of finite complement have their verb in the imperfective, regardless of the TAM category of the complement-taking verb, whether in sensory or imperfective form as in (\ref{ex:tundzxi}), in irrealis form as in (\ref{ex:tukWtɕata}) or even in inferential or perfective forms as in (\ref{ex:chWtinW}) adn (\ref{ex:tAtWsWsot}).

\begin{exe}
\ex \label{ex:tundzxi}
\gll 
\ipa{tɤjpa} 	\ipa{kɯ-xtɕɯ\tld{}xtɕi} 	\ipa{ka-lɤt} 	\ipa{ri,} 	\ipa{mɯ́j-ʁdɯɣ,} 	\ipa{pɤjkʰu} 	\ipa{tu-ndʐi} 	\ipa{ɲɯ-cʰa} \\
snow \textsc{inf:stat-emph}\tld{}be.small \textsc{pfv}:3$\rightarrow$3'-throw but \textsc{neg:sens}-be.serious still \textsc{ipfv}-melt \textsc{sens}-can \\
\glt `There was a little snow, but it doesn't matter, it can still melt.' (conversation, 2015/12/17)
\end{exe}

\begin{exe}
\ex \label{ex:tukWtɕata}
\gll 
\ipa{nɤʑo} 	\ipa{koŋla} 	\ipa{ʑo} 	\ipa{tu-kɯ-tɕat-a} 	\ipa{a-pɯ-tɯ-cʰa} 	\ipa{a-pɯ-ŋu} 	\\
\textsc{2sg} really \textsc{emph} \textsc{ipfv}-2$\rightarrow$1-take.out-\textsc{1sg} \textsc{irr-ipfv}-2-can \textsc{irr-ipfv}-be \\
\glt `If you can take me out of here...' (140516 huli de baofu, 82)
\end{exe}

\begin{exe}
\ex \label{ex:chWtinW}
\gll 
\ipa{ko-spa-nɯ} 	\ipa{qʰendɤre,} 	\ipa{nɯnɯ} 	\ipa{rɤɣo} 	\ipa{cʰɯ-tɯ-ʑa} 	\ipa{qʰe,} 	\ipa{tɤrcɯrca} 	\ipa{ʑo} 	\ipa{cʰɯ-ti-nɯ} 	\ipa{to-cʰa-nɯ.} \\
\textsc{ifr}-be.able-\textsc{pl} \textsc{lnk} \textsc{dem} song \textsc{ipfv-imm}-start \textsc{lnk} together \textsc{emph} \textsc{ipfv}-say-\textsc{pl} \textsc{ifr}-can-\textsc{pl} \\
\glt `They had learned (its songs) and as soon as it would start his song, they had become able to sing together with it.' (140519 yeying, 156)
\end{exe}

\begin{exe}
\ex \label{ex:tAtWsWsot}
\gll \ipa{a-rɤɣo} 	\ipa{ɲɯ-tɯ-sɤŋo} 	\ipa{tɤ-tɯ-sɯso-t} 	\ipa{tɕe,} 	\ipa{a-ɣɯ-jɤ-kɯ-sɯ-ɣe-a} 	\ipa{qʰe} 	\ipa{nɯ} 	\ipa{ɕti} \\
\textsc{1sg.poss}-song \textsc{ipfv}-2-listen \textsc{pfv}-2-think-\textsc{tr:pst} \textsc{lnk} \textsc{irr-cisloc-pfv-2$\rightarrow$1-caus-come}-1sg \textsc{lnk} \textsc{dem} be.\textsc{affirm:fact} \\
\glt `When you want to listen to my song, just come and ask me.' (140519 yeying, 238)
\end{exe}

Second, some complement-taking verbs, including impersonal modal verbs such as \jpg{ntsʰi}{have better}, \jpg{ɬoʁ}{have to} or \jpg{ra}{have to, need} or transitive verbs such as \jpg{sɯso}{think} allow complements in the irrealis form, as (\ref{ex:amApWwGnWClWG}).\footnote{The same is also found in Tshobdun, see \citet[483]{sun12complementation}.}

\begin{exe}
\ex \label{ex:amApWwGnWClWG}
\gll 
\ipa{ndɤre} 	\ipa{kɯ-xtɕɯ\tld{}xtɕi} 	\ipa{a-mɤ-pɯ́-wɣ-nɯ-ɕlɯɣ} 	\ipa{ɲɯ-ra} 	\ipa{ma} 	\ipa{rca} 	\ipa{nɯ} 	\ipa{ɲɯ-ndoʁ} 	\ipa{qʰe} 	\ipa{ɕlaʁ} 	\ipa{ʑo} 	\ipa{pjɯ-ɴɢrɯ} 	\ipa{ɲɯ-ɕti} \\
\textsc{lnk} \textsc{inf:stat-emph}\tld{}be.small \textsc{irr-neg-pfv:down-inv-auto}-drop \textsc{sens}-need because \textsc{unexpectedly} \textsc{dem} \textsc{sens}-be.brittle \textsc{lnk} at.once \textsc{emph} \textsc{ipfv-acaus}:break \textsc{sens}-be:\textsc{affirm} \\
\glt `However, one should not let it drop even a little, otherwise, as it is very brittle, it would break.' (30-Com, 27)
\end{exe}


Third, some transitive or semi-transitive verb-taking complements, such as \jpg{cʰa}{can}, may occur in a construction in which both the verb of the matrix clause and that of the complement clause share the same subject and the same TAM category, in particular in the case of inferential or perfective forms (as in \ref{ex:lomWrkW} and \ref{ex:nanACqa}), which are otherwise not found in complement clauses of matrix verbs other than verbs of speech or thought.\footnote{Note the difference of meaning of \jpg{cʰa}{can} with an imperfective finite complement (as in example \ref{ex:chWtinW}) and when the complement clause takes the inferential -- `become able to X' in the former case, and `succeed in Xing' in the second.}
\begin{exe}
\ex \label{ex:lomWrkW}
\gll 
\ipa{tarmgɯnku} 	\ipa{ɯ-ʑɯβdaʁ} 	\ipa{nɯnɯ} 	\ipa{kɯ-mɯrkɯ} 	\ipa{cʰɤ-ɕe.} 	\ipa{tɕe} 	\ipa{tɯ-ci,} 	\ipa{nɤki} 	\ipa{tɯ-tsʰɤʁrɯ} 	\ipa{nɯ,} 	\ipa{ɯnɯnɯ} 	\ipa{lo-mɯrkɯ} 	\ipa{pjɤ-cʰa.} \\
Dar.mgon \textsc{3sg.poss}-tutelary.spirit \textsc{dem} \textsc{nmlz}:S/A-steal \textsc{ifr:downstream}-go \textsc{lnk} \textsc{indef.poss}-water \textsc{dem} one-goat.horn \textsc{dem} \textsc{dem} \textsc{ifr}-steal \textsc{ifr}-can \\
\glt The tutelary spirit of Darmgon went to steal it, and the goat horn full of water, he succeeded in stealing it. (02-montagnes-kamnyu-cz)
\end{exe}


\begin{exe}
\ex \label{ex:nanACqa}
\gll \ipa{tɤ-ari-ndʑi} 	\ipa{ndɤre,} 	\ipa{ʑŋgri} 	\ipa{cʰo} 	\ipa{ra} 	\ipa{nɯ-pʰe} 	\ipa{lonba,} 	\ipa{tɤŋe} 	\ipa{cʰo} 	\ipa{slɤŋe} 	\ipa{ra} 	\ipa{nɯ-pʰe} 	\ipa{ra} 	\ipa{lonba} 	\ipa{na-nɤɕqa} 	\ipa{pɯ-cʰa} 	\ipa{ɲɯ-ŋu.} \\ 
\textsc{pfv}-go[II]-\textsc{du} \textsc{lnk} star \textsc{comit} \textsc{pl} \textsc{3pl-dat} all sun \textsc{comit} moon \textsc{pl} \textsc{3pl-dat} \textsc{pl} all \textsc{pfv}:3$\rightarrow$3'-resist \textsc{pfv}-can \textsc{sens}-be \\
\glt `When they went up(the sky), he succeeded in resisting (the cold) near the stars, the sun and the moon. (the demon 2003, 108-109)
\end{exe}

Constructions like (\ref{ex:lomWrkW})or (\ref{ex:nanACqa}), where the verbs of both the matrix and the complement clause share the same TAM form and the same subject superficially resemble serial verb constructions (section \ref{sec:serial}). They can be distinguished however by the fact in serial verb constructions, both verbs must share not only the same TAM marking and subject, but also the same transitivity and (if applicable) the same object.

Fourth, complements in direct or hybrid indirect speech (in the case of verbs of speech and thought) do not have restrictions on TAM or person marking as other types of finite complements (see section XXX).

\subsubsection{Coreference restrictions} \label{sec:finitie.coref}
Verbs taking finite complements can be divided into three groups depending on the coreference restrictions between complement and matrix clause.

First, verbs such as \jpg{cʰa}{can} require subject coreference between the clauses. For instance, in (\ref{ex:matWcha}) and (\ref{ex:machaa}), the semi-transitive main verb \jpg{cʰa}{can} respectively takes second and first singular person indexation, like the subject of the transitive complement verbs. Absence of person indexation on the verb of the matrix clause or indexing the object of the complement clause is impossible.

\begin{exe}
\ex \label{ex:matWcha}
\gll \ipa{aʑo} 	\ipa{ɲɯ-kɯ-ɕɯɣ-mu-a} 	\ipa{mɤ-tɯ-cʰa} \\
\textsc{1sg} \textsc{ipfv-2$\rightarrow$1-caus}-be.afraid \textsc{neg}-2-can:\textsc{fact} \\
\glt `You cannot scare me.' (140516 guowang halifa, 54)
\end{exe}

\begin{exe}
\ex \label{ex:machaa}
\gll
\ipa{cʰɯ-ta-ɕɯ-fka} 	\ipa{mɤ-cʰa-a} \\
\textsc{ipfv-1$\rightarrow$2-caus}-be.satiated \textsc{neg}-can:\textsc{fact-1sg} \\
\glt `(If you eat me, as I am lean,) I will not be able to satiate your hunger.' (140516 guowang halifa, 92)
\end{exe}

Second, other verbs such as \jpg{rga}{like} allow coreference of the subject of the main clause with either the subject or the object of the complement clause (as in \ref{ex:tukWnAjoRjoRa}).

\begin{exe}
\ex \label{ex:tukWnAjoRjoRa}
\gll \ipa{nɤʑo} \ipa{kɯ} \ipa{tu-kɯ-nɤjoʁjoʁ-a} \ipa{nɯra}	\ipa{rga-a} \\
\textsc{2sg} \textsc{erg} ipfv-2$\rightarrow$1-flatter-\textsc{1sg} \textsc{dem:pl} like:\textsc{fact-1sg} \\
\glt `I like when you flatter me.' (elicited)
\end{exe}

Unlike infinitival complements (as in example (\ref{ex:kAmNAm}), coreference with a possessor is not possible. While it is possible to say (\ref{kAmNAm2}), it is not grammatical to use a finite complement here, and one cannot replace the infinitive \ipa{kɤ-mŋɤm} (\textsc{inf}-hurt) with a finite form such as \ipa{ɲɯ-mŋɤm} (\textsc{sens}-hurt) for instance.

\begin{exe}
\ex \label{kAmNAm2}
\gll \ipa{a-xtu} \ipa{kɤ-mŋɤm} \ipa{mɤ-rga-a} \\
\textsc{1sg.poss}-belly \textsc{inf}-hurt \textsc{neg}-like:\textsc{fact-1sg} \\
\glt `I don't like to have belly ache.' (elicited)
\end{exe}

Third, in the case of complement clauses in S function, the verb of the matrix remains in third person singular form regardless of the person of the subject and object in the complement clause, as in (\ref{ex:zYWCara}).

\begin{exe}
\ex \label{ex:zYWCara}
\gll \ipa{ki} 	\ipa{maka} 	\ipa{qala} 	\ipa{kɯ} 	\ipa{pjɤ́-wɣ-nɯβlu-a} 	\ipa{tɕe} 	\ipa{z-ɲɯ-ɕar-a} 	\ipa{ɲɯ-ntsʰi}\\ 
\textsc{dem} at.all rabbit \textsc{erg} \textsc{ifr-inv}-cheat-\textsc{1sg} \textsc{lnk} \textsc{transloc-ipfv}-search-\textsc{1sg} \textsc{sens}-have.better \\
\glt `That rabbit has cheated me, I have to go to look for him.' (31-qala, 39)
\end{exe}

 \subsection{Complementation strategies}  
 
  
 

\subsubsection{Relatives}   \label{sec:relative}
Not all clauses occurring in subject, object or adjunct function are complement clauses. 

Most relative clauses in Japhug are headless  (\citealt{jacques16relatives}). When headless relative clauses occur in subject or object function, they may be superficially similar to complement clauses.

In example (\ref{ex:pjACWnNo}), the main verb \jpg{cʰa}{can} takes a finite complement clause (with TAM form agreement between the matrix and the complement clause, see section \ref{sec:TAM.finite}). Note that the subject of the matrix clause and that of the complement clause are coreferent.

\begin{exe}
\ex \label{ex:pjACWnNo}
\gll [\ipa{sɯŋgi} 	\ipa{nɯ} 	\ipa{pjɤ-ɕɯ-nŋo}] 	\ipa{pjɤ-cʰa} \\
lion \textsc{dem} \textsc{ifr-caus}-lose \textsc{ifr}-can \\
\glt `(The mosquito) succeeded in defeating the lion.' (140426 wenzi he shizi, 27)
\end{exe}

In example (\ref{ex:tWtatWt}), despite the fact that we have the same form \ipa{pjɤ-cʰa} in the matrix clause as in (\ref{ex:pjACWnNo}), the subordinate clause \ipa{tɯ\tld{}ta-tɯt} `all that he had said' is here a headless relative clause. Evidence for this analysis is as follows.

 First, the subject of both clauses are not coreferent (this example cannot be interpreted as meaning `the boy succeeded in saying all these things'). If the subordinate clause in (\ref{ex:tWtatWt}) were a complement clause, subject coreference would be expected with the verb 
\jpg{cʰa}{can} (see section \ref{sec:finitie.coref}). Second, the verb of the relative has totalitative reduplication, which is attested in relatives and some temporal subordinate clauses (\citealt[295]{jacques14linking}), but not in complement clauses. Third, the verb of the relative clause is in the perfective form while that of the the main clause is in the inferential; in finite complement clauses, the verb should be either in imperfective form or in the same form as the verb \jpg{cʰa}{can} in the main clause (see section \ref{sec:TAM.finite}). Fourth, although demonstrative and plural markers can follow a complement clause (see section \ref{sec:demonstratives}), only relative clauses can take both preclausal and postclausal demonstratives as in (\ref{ex:tWtatWt}), like any noun phrase. Fifth, it is possible to make a sentence with an overt head noun in (\ref{ex:tWtatWt}), while this option does not exist with (\ref{ex:pjACWnNo}). 

\begin{exe}
\ex \label{ex:tWtatWt}
\gll 
\ipa{tɤ-pɤtso} 	\ipa{nɯ} \ipa{kɯ}	\ipa{nɯra} 	[\ipa{tɯ\tld{}ta-tɯt}] 	\ipa{nɯra} 	\ipa{pjɤ-cʰa} \\
\textsc{indef.poss}-child \textsc{dem} \textsc{erg} \textsc{dem:pl} \textsc{total\tld{}pfv}:3$\rightarrow$3'-say[II] \textsc{dem:pl} \textsc{ifr}-can \\
\glt `The child had succeeded in doing everything that (the old king) had said.' (140428 yonggan de xiaocaifeng, 256)
\end{exe}

The contrast between finite headless relative and complement clauses in object function is not always as clear as that between (\ref{ex:tWtatWt}) and (\ref{ex:pjACWnNo}), but one of the criteria presented above, or a combination thereof, can be used to discriminate between the two.

Formal near-ambiguity between headless relative and complement clause can also exist in the case of non-finite subordinate clauses. In (\ref{ex:amApWNgrW}), the intransitive verb \jpg{ŋgrɯ}{succeed} takes an infinitive complement in \ipa{kɤ-} as its subject, while in (\ref{ex:nAkAnWsmWlAm}), its subject is a headless relative clause with an object participle, as is shown by the fact that it can take a possessive prefix coreferent with its A. Note also the semantic difference between a relative (`may what you wish for come true') and a complement clause (??`may you succeed to wish it.').

\begin{exe}
\ex \label{ex:amApWNgrW}
\gll 
\ipa{kɤ-nɯsmɤn} 	\ipa{a-mɤ-pɯ-ŋgrɯ} 	\ipa{cʰo} 	\ipa{ɯ-nɯsmɤn} 	\ipa{a-mɤ-tɤ-βdi} 	\ipa{tɕe} 	\ipa{pjɯ-kɯ-sat} 	\ipa{ɲɯ-ŋgrɤl} \\
\textsc{inf}-treat \textsc{irr-neg-ipfv}-succeed \textsc{comit} \textsc{3sg.poss-bare.inf}:treat  \textsc{irr-neg-ipfv}-be.good \textsc{lnk} \textsc{ipfv-genr:S/P}-kill \textsc{sens}-be.usually.the.case \\
\glt `If one cannot treat it (rabbies), if one does treat it well, it is fatal.' (29-chWsYu, 29)
\end{exe}

\begin{exe}
\ex \label{ex:nAkAnWsmWlAm}
\gll   \ipa{nɤ-kɤ-nɯsmɯlɤm} 	\ipa{nɯ} 	\ipa{a-pɯ-ngrɯβ} \\
\textsc{2sg.poss-nmlz:P}-wish \textsc{dem} \textsc{irr-pfv}-succeed \\
\glt `May your wishes (=the things that you wish for) succeed.' (elicited)
\end{exe}  

A more delicate ambiguous case of relative vs complement clause is discussed in section (\ref{sec:relative.q}) in the case of pretence verbs.

Besides plain finite relatives and participial relatives, correlatives (with an interrogative pronoun) can also appear in subject or object function, as in (\ref{ex:tAstuta}). 

\begin{exe}
\ex \label{ex:tAstuta}
\gll 
[\ipa{ɯ-wa} 	\ipa{tɕʰi} 	\ipa{tɤ-stu-t-a}] 	\ipa{nɯ} 	\ipa{tu-ste-a} 	\ipa{ɲɯ-ɬoʁ} \\
\textsc{3sg.poss-}father what \textsc{pfv}-do.like-\textsc{pst:tr}-\textsc{1sg} \textsc{ipfv}-do.like[III]-\textsc{1sg} \textsc{sens}-have.to \\
\glt `I have to deal with him in the same way as I dealt with his father. (=How I treated his father, I have to treat him like that)' (Slob.dpon2, 159)
\end{exe}

There are however cases where a surface form can be analyzed either as a complement clause or as a relative, in particular with verbs of perception. In example (\ref{ex:WftaR.tAkAta}), the verb form \ipa{tɤ-kɤ-ta} can either be analyzed as the object participle, or the infinitive of \jpg{ta}{put}, resulting in two slightly different but semantically nearly identical translations.

\begin{exe}
\ex  \label{ex:WftaR.tAkAta}
\gll 
\ipa{nɯɕɯmɯma} 	\ipa{ʑo} 	\ipa{iɕqha} 	\ipa{kɯm} 	\ipa{nɯtɕu} 	\ipa{ɯ-ftaʁ} 	\ipa{tɤ-kɤ-ta} 	\ipa{nɯ} 	\ipa{pjɤ-mto} \\
immediately \textsc{emph} the.aforementioned door \textsc{dem:loc} \textsc{3sg.poss}-mark \textsc{pfv-inf/nmlz:P}-put \textsc{dem} \textsc{ifr}-see \\
\glt `She immediately saw the mark that had been put on the door / that someone had put a mark on the door.' (140512 alibaba, 183)
\end{exe}

Similar ambiguities can also occur with finite relatives. Example (\ref{ex:tatWt.nWra}) can be either parsed as having a finite complement clause \ipa{nɯra} 	\ipa{ta-tɯt} `He said these words',\footnote{With verbs of cognition, speech or perception, there are no coreference or TAM restrictions between the complement and the matrix clause, see (\ref{sec:cognition}).} or a relative \ipa{ta-tɯt} `what he said' (with preclausal and postclausal demonstratives \ipa{nɯra} `these'), with little semantic difference.

\begin{exe}
\ex \label{ex:tatWt.nWra}
\gll \ipa{ɯ-ʁjoʁ} 	\ipa{nɯ} 	\ipa{kɯ} 	\ipa{nɯra} 	\ipa{ta-tɯt} 	\ipa{nɯra} 	\ipa{pjɤ-mtsʰɤm} \\
\textsc{3sg.poss}-servant \textsc{dem} \textsc{erg} \textsc{dem:pl} \textsc{pfv}:3$\rightarrow$3'-say[II] \textsc{dem:pl}  \textsc{ifr}-hear \\
\glt `His servant heard what (the king) had said / that he had said these (words).' (140428 yonggan de xiaocaifeng, 265)
\end{exe}

In the case of example (\ref{ex:pWwGnWmtChu}), the speaker started by saying the relative clause [\ipa{spjaŋkɯ} \ipa{kɯ} \ipa{sɯŋgi} \ipa{ɯ-ɕki} \ipa{tɤ-kɤ-tɯt}] `what the wolf had said to the lion' and then corrected herself and said the verb form \ipa{pɯ́-wɣ-nɯmtɕʰu}, which can either be interpreted as a relative (`The slandering words that the wolf said')\footnote{The verb \jpg{nɯmtɕʰu}{slander} takes the person slandered as its object, but it can be construed as a secundative ditransitive verb whose third argument (the slandering words) can also be relativized with finite relative clause, like the theme of the verb \jpg{mbi}{give} (see (\citealt[16-17]{jacques16relatives}).} or alternatively as a complement clause (`that the wolf had slandered him'), showing the close proximity of these two possible analyses.

\begin{exe}
\ex \label{ex:pWwGnWmtChu}
\gll 
\ipa{spjaŋkɯ} 	\ipa{kɯ} 	\ipa{sɯŋgi} 	\ipa{ɯ-ɕki} 	\ipa{tɤ-kɤ-tɯt,...} 	\ipa{pɯ́-wɣ-nɯmtɕʰu} 	\ipa{nɯnɯra} 	\ipa{pjɤ-mtsʰɤm.} \\
wolf \textsc{erg} lion \textsc{3sg-dat} \textsc{pfv-nmlz:P}-say[II] \textsc{pfv-inv}-slander \textsc{dem:pl} ifr-hear \\
\glt `(The fox) heard what the wolf had said to the lion... (that the wolf) had slandered him.' (140425 shizi lang huli, 16)
\end{exe}

  \subsubsection{Degree nominal}  \label{sec:degree}
Adjectives of degree like \jpg{rtaʁ}{be enough}, \jpg{tɕʰom}{be too much}, \jpg{naχtɕɯɣ}{be identical} or \jpg{saχaʁ}{be extremely} can be used with infinitival and finite complements (section \ref{sec:degree.complement}), but the most common construction involves degree nominals, build by prefixing \ipa{tɯ-} and a possessive prefix to the stem of the verb, as in (\ref{ex:nWrtaR}) or (\ref{ex:YWtChom}). Although most degree nominals in the corpus derive from adjectives, there are also a few examples of dynamic verbs, as (\ref{ex:nWrtaR}).


\begin{exe}
\ex \label{ex:nWrtaR}
\gll \ipa{ɯ-tɯ-ɤla} 	\ipa{nɯ-rtaʁ} 	\ipa{ʑo} 	\ipa{tɕe} 	\ipa{tɕe} 	\ipa{chɯ́-wɣ-tɕɤt} 	\ipa{tɕe} 	\ipa{ɲɯ́-wɣ-χtɕi} \\
\textsc{3sg.poss-nmlz:degree}-boil \textsc{pfv}-be.enough \textsc{emph} \textsc{lnk} \textsc{lnk} \textsc{ipfv-inv}-take.out  \textsc{lnk} \textsc{ipfv-inv}-wash \\
\glt `When it has boiled enough, one takes it out and washes it.' (30-tasa, 4)
\end{exe}

\begin{exe}
\ex \label{ex:YWtChom}
\gll
\ipa{kɯki} 	\ipa{kʰa} 	\ipa{ki} 	\ipa{ɯ-tɯ-xtɕi} 	\ipa{ɲɯ-tɕʰom} \\
\textsc{dem} house \textsc{dem} \textsc{3sg.poss-nmlz:degree}-be.small \textsc{sens}-be.too.much \\
\glt `This house is too small.' (140430 yufu he tade qizi, 83)
\end{exe}

In this construction, the degree nominal is the S of the adjective of degree. The possessive prefix refers to the S of the nominalized verb. Thus, example (\ref{ex:YWtChom}) for instance can be literally translated as `The smallness of this house is excessive'. 

This construction does not fulfil all conditions for being a proper complement clause in Dixon's definition. Degree nominals, unlike participles, can only take one argument and is not compatible with TAM marking.

\subsubsection{Serial verb constructions} \label{sec:serial}
In Japhug, as in Tshobdun (\citealt[490-1]{sun12complementation}), we find a serial verb constructions comprising two verbs sharing TAM category, core argument(s) (both subject and object in the case of transitive verbs) and transitivity. One of the verbs expresses the main action, and the other describes the manner in which the action is performed. Unlike Tshobdun, there is no constraint in Japhug against inserting a linker such as \ipa{tɕe} between the two verbs in the serial construction.

This construction is most common with deideophonic verbs, \footnote{On deideophonic verbs and their morphosyntactic properties, see \citet{jackson04zhuangmaoci} and \citet{japhug14ideophones}.} as exemplified by (\ref{ex:totChW}), where \jpg{nɯdrɯβ}{gore again and again} can only be used in this construction together with \jpg{tɕʰɯ}{gore}. The ideophonic verb can either follow (\ref{ex:totChW}) or precede  the main verb (\ref{ex:totChW2}), the latter construction being by far more common.

\begin{exe}
\ex \label{ex:totChW}
\gll \ipa{iɕqʰa} 	\ipa{srɯnmɯ} 	\ipa{nɯ} 	\ipa{to-tɕʰɯ} 	\ipa{to-nɯdrɯβ}  \ipa{tɕe} 	\ipa{pjɤ-sat} \\
the.aforementioned râkshasî \textsc{dem} \textsc{ifr}-gore \textsc{ifr}-repeatedly.gore \textsc{emph} \textsc{lnk} \textsc{ifr}-kill \\
\glt `(The rhinoceros) gored the râkhsasî repeatedly and killed her.' (28-smAnmi, 403)
\end{exe}

\begin{exe}
\ex \label{ex:totChW2}
\gll 	\ipa{srɯnmɯ} 	\ipa{nɯ} 	\ipa{to-nɯdrɯβ} 	\ipa{ʑo} 	 	\ipa{to-tɕʰɯ} \\
 râkshasî \textsc{dem}  \textsc{ifr}-repeatedly.gore  \textsc{emph}  \textsc{ifr}-gore \\
 \glt `(The rhinoceros) gored the râkhsasî repeatedly and killed her.' (elicited on the basis of \ref{ex:totChW})
\end{exe}	

The second most common type of serial verb construction in Japhug involves the manner deixis verbs \jpg{stu}{do like this} (transitive) and \jpg{fse}{be like this} (intransitive). 

Examples like (\ref{ex:tuWGstu}) could seem to indicate that \jpg{stu}{do like this} and the lexical verb do not share the same object, as  \jpg{ki}{this}, which obligatorily occurs before the manner deixis verbs, appears to be its object. 

\begin{exe}
\ex \label{ex:tuWGstu}
\gll 	
\ipa{ɯ-ru} 	\ipa{nɯ} 	\ipa{ki} 	\ipa{tú-wɣ-stu} 	\ipa{pjɯ́-wɣ-qlɯt} \\
\textsc{3sg.poss}-stalk \textsc{dem} \textsc{dem:prox} \textsc{ipfv-inv}-do.like \textsc{ipfv-inv}-break \\
\glt `One breaks its stalk like this.' (14-tasa, 81)
\end{exe}	

However, when the lexical verbs takes a non-third person object, the manner deixis verb indexes it as its object too, as in (\ref{ex:kuWGstuanW}): \jpg{stu}{do like this} is in fact a secondative ditransitive verb, and the demonstrative is an unmarked T argument.

\begin{exe}
\ex \label{ex:kuWGstuanW}
\gll 	
 \ipa{aʑo} 	\ipa{kɯki} 	\ipa{ntsɯ} 	\ipa{kú-wɣ-stu-a-nɯ} 	\ipa{tɕe,} 	\ipa{kú-wɣ-znɯkʰrɯm-a-nɯ} \\
 \textsc{1sg} \textsc{dem:prox} always \textsc{ipfv-inv}-do.like-\textsc{1sg-pl} \textsc{lnk} \textsc{ipfv-inv}-punish-\textsc{1sg-pl} \\
 \glt `They punished me like this.' (Gesar, 278)
\end{exe}	

The verb \jpg{stu}{do like this} cannot be used with intransitive verbs in a serial construction. Instead, its intransitive counterpart \jpg{fse}{be like this} occurs with a demonstrative such as \ipa{ki} as in (\ref{ex:ki.fsea}).

\begin{exe}
\ex \label{ex:ki.fsea}
\gll \ipa{aʑo} 	\ipa{nɯ} 	\ipa{sŋiɕɤr} 	\ipa{ʑo} 	\ipa{kutɕu} 	\ipa{ki} 	\ipa{fse-a} 	\ipa{ndzur-a} 	\ipa{ntsɯ} 	\ipa{ɲɯ-ra} 	\ipa{tɕe,} \\
\textsc{1sg} \textsc{dem} night.and.day \textsc{emph} here \textsc{dem:prox} be.like:\textsc{fact-1sg} stand:\textsc{fact-1sg} always \textsc{sens}-have.to like \\
\glt `I have to stand like this night and day.' (The divination, 2002, 44)
\end{exe}


%\begin{exe}
%\ex 
%\gll \ipa{tɤɕime} 	\ipa{kɯ} 	[\ipa{tɕʰi} 	\ipa{a-tɤ-fse-a} 	\ipa{tɕe}] 	\ipa{mɤ́-wɣ-mto-a}  \\
%princess \textsc{erg} what \textsc{irr-pfv}-be.like-\textsc{1sg} \textsc{lnk} \textsc{neg-inv}-see:\textsc{fact-1sg} \\
%\glt `What should I do not to be seen by the princess?' (140505 xiaohaitu)
%\end{exe}


Both Tshobdun and Japhug have a few cognate manner verbs (other than deideophonic and manner deixis verbs) which can appear in serial verb constructions, such as Tshobdun \jpg{nɐʃeʃet}{exert oneself} and Japhug  \jpg{nɤxɕɤt}{do with force}  (denominal verbs derived from the Tibetan loanword \jpg{ɯ-xɕɤt}{force, strength}).

However, the inventory of verbs compatible with serial constructions is not the same in Japhug and Tshobdun. For instance, while Tshobdun \jpg{wɐtʃʰɐm}{overdo} requires a serial construciton, while its Japhug cognate \jpg{ɣɤtɕʰom}{overdo, do too much}  only occurs with infinival complements (\ipa{kɤ-} or bare infinitives).

Finally, the phasal verb \jpg{ʑa}{begin}, though usually taking a bare infinitive or \ipa{tɯ-} infinitive complement, can also appear in a serial verb construction expressing the specific meaning `start doing X from ... until...', as in (\ref{ex:chWwGmphWr}).

   \begin{exe}
\ex \label{ex:chWwGmphWr}
 \gll  \ipa{tɕe} 	\ipa{βzɯr} 	\ipa{ri} 	\ipa{tɕe} 	\ipa{cʰɯ́-wɣ-mpʰɯr} 	\ipa{cʰɯ́-wɣ-ʑa} 	\ipa{tɕe} 	\ipa{mɤpɕoʁ} 	\ipa{cʰu} 	\ipa{βzɯr} 	\ipa{nɯ-ɕki} 	\ipa{mɤɕtʂa} 	\ipa{cʰɯ́-wɣ-mpʰɯr}  \\
 \textsc{lnk} corner \textsc{loc} \textsc{lnk} \textsc{ipfv-inv-}wrap \textsc{ipfv-inv-}begin \textsc{lnk}  opposite.side \textsc{loc} corner \textsc{3pl-dat} until \textsc{ipfv-inv-}wrap  \\
 \glt  One starts to wrap it up from one corner until the opposite corner.' (30-mboR, 20)
\end{exe}
 
 
Serial verb constructions may superficially resemble finite complements when the TAM category of the complement verb agrees with that of the matrix verb (as in \ref{ex:nanACqa} in section \ref{sec:TAM.finite}). They can be distinguished by the fact that serial verb constructions require both verbs to share the same transitivity, subject and object, while in the case of finite complements, only subject coreference is required.

\subsubsection{Coordination}
In Japhug, some attitudinal verbs such as \jpg{ʁnɯ}{suspect}, \jpg{nɯsɯmɲiz}{hesitate}, \jpg{nɯʁlɯmbɯɣ}{guess, estimate} or \jpg{nɯʁjɯβtsʰɤt}{guess, estimate} do not take complement clauses. Rather, they occur in a coordinating construction strikingly similar to that described in Tshobdun  by \citet[487-8]{sun12complementation}: the attitudinal verb is followed by the affirmative copula \jpg{ɕti}{be} and an adversative linker such as \jpg{ri}{but} , as in (\ref{ex:tunWRlWmbWGa}).

\begin{exe}
\ex \label{ex:tunWRlWmbWGa}
\gll \ipa{nɯ} 	\ipa{tu-nɯʁlɯmbɯɣ-a} 	\ipa{ɕti} 	\ipa{ri,} 	\ipa{ɯʑo} 	\ipa{kɯ} 	\ipa{kɤ-nɤma} 	\ipa{nɯ} 	\ipa{sɤpe} \\
\textsc{dem} \textsc{ipfv}-guess-\textsc{1sg} be.\textsc{affirm:fact} \textsc{lnk} \textsc{3sg} \textsc{erg} \textsc{nmlz:P}-work \textsc{dem} do.well:\textsc{fact} \\
\glt `I guess that he will perform this task well.'
\end{exe}

There are apparent cases of reported speech complements with these verbs, as example (\ref{ex:kunWsWmRYiza}), but such sentences result from the ellipsis of a cognition verb such as \jpg{sɯso}{think} -- it is possible (and slightly better) to insert here \ipa{ɲɯ-sɯsam-a} 	\ipa{tɕe} (\textsc{sens}-think[III]-\textsc{1sg} \textsc{lnk}) `I think' before the verb \ipa{ku-nɯsɯmʁɲiz-a}  `I hesitate'.

\begin{exe}
\ex \label{ex:kunWsWmRYiza}
\gll 
\ipa{ku-ɕe-a} 	\ipa{ɕi} 	\ipa{ma-kɤ-ɕe-a} 	\ipa{kɯ} (\ipa{ɲɯ-sɯsam-a} 	\ipa{tɕe}) \ipa{ku-nɯsɯmʁɲiz-a} \\
\textsc{ipfv:east}-go-\textsc{1sg} \textsc{qu} \textsc{neg-imp}-go-\textsc{1sg} \textsc{qu} \textsc{sens}-think[III]-\textsc{1sg} \textsc{lnk} \textsc{ego.prs}-hesitate-\textsc{1sg} \\
\glt `I hesitate whether to go or not.' (elicited)
\end{exe}

\subsubsection{Essive noun phrase} \label{sec:essive}
While motion verbs use participial complements (section \ref{sec:SAparticiple.coref}), manipulation verbs such \jpg{tsɯm}{take away} or \jpg{ɣɯt}{bring} cannot. Examples such as (\ref{ex:kAntsGe.jotsWm}), with \ipa{kɤ-} prefixed verb forms appearing before a manipulation verb, could appear to be an example of an infinitival complement.

\begin{exe}
\ex \label{ex:kAntsGe.jotsWm}
\gll \ipa{ɯ-mbro} 	\ipa{ɯ-ndʐi} 	\ipa{nɯra} 	\ipa{kɤ-ntsɣe} 	\ipa{jo-tsɯm} \\
\textsc{3sg.poss}-horse \textsc{3sg.poss}-skin \textsc{dem:pl} ???-sell \textsc{ifr}-take.away \\
\glt `He took the horses' skin to (the market) to sell them.' (150814 kelaosi, 85)
\end{exe}

However, in all examples such as (\ref{ex:kAntsGe.jotsWm}), the noun \jpg{ɯ-spa}{material} can be added after the verb prefixed in \ipa{kɤ-} without changing the meaning (\ipa{kɤ-ntsɣe} \ipa{ɯ-spa} \ipa{jo-tsɯm}). This indicates that the syntactic function of \ipa{kɤ-ntsɣe} here is in fact that of an essive adjunct (see section \ref{sec:adjuncts}), meaning literally `He took the horses' skin (there) as something to sell', and that it should be analyzed not as an infinitive form, but as an object participle meaning `which is to be sold'.

While such construction could eventually be reanalyzed as a complement clause, it does not belong to the argument structure of the main verb and should thus be classified as a complementation strategy.
 
   \section{Morphosyntactic properties of complement clauses} 

 \subsection{Syntactic pivots} 
Table (\ref{tab:coref}) presents a summary of coereference restrictions between matrix and complement clause in Japhug, based on the data in section (\ref{sec:complement.types}).

Some verb classes are named by a representative example (for instance \jpg{rga}{like}) because at this stage of research, it is not yet clear to what extent the class of all verbs with the same behavior can be given a simple functional label.

\begin{table}[H]
\caption{Coreference restrictions in complement clauses in Japhug} \label{tab:coref} \centering
\begin{tabular}{llcc}
\toprule
Verb class & 	Complement type & 	Coference & 	\\
&& complement = main clause \\
\midrule
motion verb & 	\ipa{kɤ-} participle & 	\{P\}=\{S\} & 	\\
\midrule
motion verb & 	\ipa{kɯ-} participle & 	\{S,A\}=\{S\} & 	\\
transitive verb & 	bare infinitive & 	\{S,A\}=\{S,A\} & 	\\
\ipa{spa} & 	infinitive & 	\{S,A\}=\{S,A\} & 	\\
\ipa{cʰa} & 	finite & 	\{S,A\}=\{S,A\} & 	\\
\midrule
\ipa{rga} & 	infinitive & 	\{S,A,P,P'\}=\{S,A\} & 	\\
\ipa{rga} & 	finite & 	\{S,A,P\}=\{S,A\} & 	\\
\midrule
impersonal & 	bare infinitive & 	zero & 	\\
impersonal & 	finite, infinitive & 	zero & 	\\
\bottomrule
\end{tabular}
\end{table}

This table confirms the observation that although Gyalrong languages have ergative case marking, syntactic pivots mainly follow an accusative alignment, with restrictive neutralization of S and A (\citealt{jackson03caodeng, jacques16relatives}). Note that the construction with motion verbs and \ipa{kɤ-} participial complements, despite showing obligatory coreference between the intransitive subject of the main clause and the object of the complement, cannot be considered to be ergative, since the same construction cannot express coreference between the intransitive subject of the main clause and that of the complement clause.

Some verbs such as \jpg{rga}{like} present looser coreference restrictions, with differing rules depending on the construction.

 \subsection{Plural and demonstrative markers} \label{sec:demonstratives}
Finite, infinitival or participial complements in Japhug can be optionally followed by the distal demonstrative \jpg{nɯ}{that} (\ref{ex:tWmbro.nW}), the plural \ipa{ra} (\ref{ex:kANke.ra}) or a combination of the two \jpg{nɯra}{those}. In example (\ref{ex:tWmbro.nW}), \jpg{nɯ}{that}  has a topicalizing function (`as for growing high, it can grow high, but on the other hand it cannot grow thick').

 \begin{exe}
\ex \label{ex:tWmbro.nW}
\gll 
\ipa{tu-mbro} 	\ipa{nɯ} 	\ipa{ɲɯ-cʰa} 	\ipa{ri} 	\ipa{tu-mbro} 	\ipa{tɕe} 	\ipa{ʁnɯ-rtsɤɣ,} 	\ipa{χsɤ-rtsɤɣ} 	\ipa{jamar} 	\ipa{tu-mbro} 	\ipa{ɲɯ-cʰa} 	\ipa{ri}  \\
\textsc{ipfv}-be.high \textsc{dem} \textsc{sens}-can \textsc{lnk} \textsc{ipfv}-be.high \textsc{lnk} two-stair three-stair about \textsc{ipfv}-be.high  \textsc{sens}-can \textsc{lnk} \\
\glt `Although it can grow high, although it can grow two or three stairs high,...' (16-CWrNgo, 151)
\end{exe}
 
The marker \ipa{ra} is an associative plural; in example such as (\ref{ex:kANke.ra}), the use of \ipa{ra} implies an open list of activities (`crawl, walk etc'). 
 
 \begin{exe}
\ex \label{ex:kANke.ra}
\gll   \ipa{kɤ-nɯrtsɯ} 	\ipa{kɤ-ŋke} 	\ipa{ra} 	\ipa{tɤ-cʰa} 	\ipa{tɕe}   \\
  \textsc{inf}-crawl   \textsc{inf}-walk \textsc{pl} \textsc{pfv}-can \textsc{lnk} \\
\glt `When (the baby) becomes able to crawl or to walk, ...'  (140426 tApAtso kAnWBdaR, 65)
\end{exe}

In Tshobdun, \citet[481]{sun12complementation} analyzes these demonstratives as complementizers, an analysis which would imply a grammaticalization pathway identical to that of English `that'. It will not attempt  at solving this complex question in the present paper. An argument for a special status of demonstratives with complement clauses is that they can only be post-clausal, whereas in the case of relatives (or any noun phrase) demonstratives can be pre-clausal or circum-clausal (see section \ref{sec:relative}).

 \subsection{Discontinuous complement} 

Discontinuous clauses are rare in Japhug. The only clear example in our corpus is (\ref{ex:lWlu.kW.aZo}). In this example, the \textsc{1sg} pronoun \ipa{aʑo} (the subject of the matrix clause, which has no syntactic role in the complement clause) appears between the A \ipa{lɯlu} 	\ipa{kɯ} `the cat' and the P \ipa{ʁnɯz} `two' of the complement clause. Despite the rarity of this construction, this sentence was not considered to be unusual by our consultant when listening again to the recording.
 
 \begin{exe}
\ex \label{ex:lWlu.kW.aZo}
\gll \ipa{tɕe} 	[\ipa{lɯlu} 	\ipa{kɯ} 	\ipa{aʑo} 	\ipa{ʁnɯz} 	\ipa{ʑo} 	\ipa{ka-ndo}] 	\ipa{pɯ-mto-t-a} \\
\textsc{lnk} cat \textsc{erg} \textsc{1sg} two \textsc{emph} \textsc{pfv}:3$\rightarrow$3'-take \textsc{pfv}-see-\textsc{pst:tr-1sg} \\
\glt `I saw a cat catching two of them.' (22-kumpGatCW, 61)
\end{exe}
 
\subsection{Restrictive} 
To express a restriction (`only') having scope over a complement clause, the postposition \ipa{ma} `apart from' is used after the complement, sometimes with the postposition repeated two times [X \ipa{ma} \ipa{nɯ} \ipa{ma}] `apart from X, apart from it' as in example (\ref{ex:manWma.compl})

\begin{exe}
\ex \label{ex:manWma.compl}
\gll \ipa{kɤ-mtsʰɤm} 	\ipa{ma} 	\ipa{nɯ} 	\ipa{ma} 	\ipa{mɯ-pɯ-rɲo-t-a} \\
\textsc{inf}-hear apart.from \textsc{dem} apart.from \textsc{neg-pfv}-experience-\textsc{pst:tr-1sg} \\
\glt `I only heard about it.' (I did not see it and even do not claim that it exists, of a mythological animal) (20-RmbroN, 118)
\end{exe}
 
  \section{A classification of complement-taking verbs} 
  \subsection{Modal verb}
  
    \subsubsection{\jpg{spa}{be able to}} \label{sec:spa}
The verb \jpg{spa}{be able to} is the only transitive modal verb.\footnote{The causative verb \jpg{sɯxcʰa}{(cause to) have the ability to} is also formally transitive, but is only used in inverse forms, see section XXX)} It originates from the abilitative form of the verb \ipa{pa} `do' (\citealt{jacques15causative}) and has a cognate in Tangut  (\citealt{jacques14esquisse}), showing that its lexicalization occurred even earlier than proto-Gyalrongic.

The verb \jpg{spa}{be able to} takes both infinitival (examples \ref{ex:rYo:inf:A} and \ref{ex:rYo:inf:S}) or finite complements, and its A is coreferent with the S or the A of the complement clause.

\begin{exe}
\ex  \label{ex:rYo:inf:A}
\gll
\ipa{nɯ} 	\ipa{ɯ-mdoʁ} 	\ipa{nɯ} 	\ipa{aj} 	\ipa{kɤ-ti} 	\ipa{mɯ́j-spe-a} \\
\textsc{dem} 3sg.poss-colour \textsc{dem} \textsc{1sg} \textsc{inf}-say \textsc{neg:sens}-be.able[III]-1sg \\
\glt I am not able to name its colour. (06-qaZmbri, 57)
\end{exe}

\begin{exe}
\ex  \label{ex:rYo:inf:S}
\gll 
 \ipa{kɤ-nɤre} 	\ipa{ɯ-tá-spa?}\\
 \textsc{inf}-laugh \textsc{q-pfv}:3$\rightarrow$3'-be.able.to\\
 \glt `Is he now able to laugh?' (conversation, 2014, of a three month old infant)
\end{exe}

\ipa{nɯra} 	\ipa{ɲɯ-nɯrle} 	\ipa{ra} 	\ipa{mɯ́j-spe.} 

    
  \subsubsection{\jpg{ra}{have to}} \label{sec:ra}
  \ipa{tɤ-pɤtso} 	\ipa{nɯ,} 	\ipa{tɯ-pɤrme} 	\ipa{roro} 	\ipa{jamar} 	\ipa{tɕe} 	\ipa{tɕe} 	\ipa{tɯ-nɯ} 	\ipa{kɤ-sɯβde} 	\ipa{pjɤ-ra} 
  
  \subsection{Phasal verbs and other aspectual auxiliaries}
\subsubsection{\jpg{rɲo}{experience}}   \label{sec:rɲo}
\subsubsection{\jpg{ʑa}{begin}}   \label{sec:Za}
   \subsection{Verbs of cognition} \label{sec:cognition}
% ɯʑo srɯnmɯ kɯ-ŋu nɯ tɤ-wa nɯ kɯ mɯ-pjɤ-sɯχsɤl, ɯ-nmaʁ nɯ kɯ.
  
  \subsection{Motion verbs}
  
motion verbs and associated motion   \citet{jacques13harmonization}
  
\subsection{Causative verbs}


 %mɯ-tu-kɤ-nɤtɯti kɯ-ra tɤ́-wɣ-sɯ-βzu-a-nɯ ndʐa ɕti ma

%tɕetha ndɤre nɤʑo kɤ-sɯso mɤ-kɯ-ŋgrɯ ɲɯ-tɕat-a ŋu
  \begin{exe}
\ex 
\gll \ipa{kʰa} 	\ipa{ɯ-ʁɤri} 	\ipa{nɯtɕu} 	\ipa{ɯ-fkrɤm} 	\ipa{a-kɤ-tɯ-ɣɤ-βdi} 	\ipa{tɕe,} 	\ipa{ɕ-pɯ-sɤtse} \\
house \textsc{3sg.poss}-front.of \textsc{dem:loc} \textsc{3sg.poss-bare.inf}:place \textsc{irr-pfv-2-caus}-be.good \textsc{lnk} \textsc{transloc-imp}-stick.into[III] \\
\glt `Place these in front of your house in orderly fashion and stick them (into the ground).' (Smanmi 2003, 129)
  \end{exe}
  
\subsection{Complements of adjectives} \label{sec:adj}
Adjectives in Japhug can be formally defined as the subclass of stative verbs allowing the tropative derivation (\citealt{jacques13tropative}).\footnote{This definition excludes some noun-like property words.} We can distinguish two classes of adjectives depending on the complements they can take.


\subsubsection{Infinitival and finite complements} \label{sec:adj.infinitive}
A few adjectives are semi-transitive (see section \ref{sec:transitivity}), like \jpg{mkʰɤz}{be expert, be knowledgeable} and optionally take either a noun (\ref{ex:CoNBzu.mkhAz}) or an complement clause (\ref{ex:mkhAztCi}) addition to their S. The complement clause can be either infinitival or finite, with a verb in the imperfective.

\begin{exe}
\ex \label{ex:CoNBzu.mkhAz}
\gll 
\ipa{ɯ-nmaʁ} 	\ipa{jɤ-kɯ-ɣe} 	\ipa{nɯ} 	\ipa{ɕoŋβzu} 	\ipa{mkʰɤz} 	\ipa{tɕe} \\
\textsc{3sg.poss}-husband \textsc{pfv-nmlz}:S/A-come[II] \textsc{dem} carpentry be.expert:\textsc{fact} \textsc{lnk} \\
\glt `Her husband (who came to live in her family) is very good at carpentry.' (14-tApitaRi, 273)
\end{exe}

\begin{exe}
\ex \label{ex:mkhAztCi}
\gll \ipa{tɕiʑo} 	\ipa{rcanɯ,} 	\ipa{kɤ-taʁ} 	\ipa{wuma} 	\ipa{ʑo} 	\ipa{mkʰɤz-tɕi} 	 \\
\textsc{1du}  \textsc{unexpected} \textsc{inf}-weave really \textsc{emph} be.expert:\textsc{fact}-\textsc{1du} \\
\glt `We are very good at weaving.' (140521, huangdi de xinzhuang, 20)
\end{exe}

Adjectives such as \jpg{ɴqa}{be difficult}, \jpg{mbat}{be easy}, which unlike \jpg{mkʰɤz}{be expert} do not have a semi-object, can also take infinitival or finite complement clauses as their S (as in \ref{ex:YWmbat}).

\begin{exe}
\ex \label{ex:YWmbat}
\gll
<gang> 	\ipa{stʰɯci} 	\ipa{mɯ́j-rko} 	\ipa{qʰe,} 	\ipa{ɲɯ-mpɯ} 	\ipa{qʰe} 	[\ipa{tu-ŋgɤɣ,} 	\ipa{ɲɯ-ɤjʁu} 	\ipa{nɯra} 	\ipa{ɲɯ-mbat} \\
steel as.much \textsc{neg:sens}-be.hard \textsc{lnk} \textsc{sens}-be.soft \textsc{lnk} \textsc{ipfv-anticaus}:bend \textsc{ipfv}-be.curved  \textsc{dem:pl}] \textsc{sens}-be.easy \\
\glt `(Iron) is not as hard as steel, it is soft and bends easily.' (30-Com, 42)
\end{exe}

In nearly all cases, the infinitival complements of stative verbs is in the \ipa{kɤ-} infinitive form. The only exception found in the corpus is the verb  \jpg{pʰɤn}{be efficient}, as in example (\ref{ex:kWGAmna}) where \ipa{kɯ-ɣɤmna} is the stative infinitive of the verb \ipa{ɣɤmna} `easy to heal, heal fast' (with the abilitative \ipa{ɣɤ-} prefix). The form \ipa{kɤ-ɣɤmna} with the \ipa{kɤ-} infinitive would also be possible, but this would be the infinitive of the homophonous transitive verb \ipa{ɣɤmna} `heal' (with the causative \ipa{ɣɤ-} prefix) and the meaning would be `it is efficient to heal (this disease)'.
 
 
 \begin{exe}
\ex \label{ex:kWGAmna}
\gll \ipa{smɤn} 	\ipa{tu-βzu-nɯ} 	\ipa{tɕe} 	\ipa{tɕe} 	\ipa{ʁo} 	\ipa{kɯ-ɣɤmna} 	\ipa{ɲɯ-pʰɤn} \\
medicine \textsc{ipfv}-make-\textsc{pl} \textsc{lnk} \textsc{lnk} \textsc{adversative}  \textsc{nmlz:S/A}-\textsc{abil}-heal \textsc{sens}-be.efficient \\
\glt `When they use medicine, on the other hand, it is efficient to (make this disease) heal faster.' (27-kharwut, 103)
\end{exe}

Not all infinitival clauses occurring with adjectives are complement clauses. In (\ref{ex:turACi}), the clause whose main verb is the negative infinitive \ipa{mɤ-kɤ-cʰa} is neither a core argument, an adjunct or a purposive clause selected by the predicate of its matrix clause \jpg{rʑi}{be heavy}. Instead, it is an infinitival manner clause (see section \ref{sec:infinitives} and \citealt{jacques14linking}).

\begin{exe}
\ex \label{ex:turACi}
\gll [\ipa{maka} 	\ipa{tu-rɤɕi} 	\ipa{mɤ-kɤ-cʰa}] 	\ipa{ʑo} 	\ipa{kɯ-rʑi} 	\ipa{pjɤ-ɕti} \\
at.all \textsc{ipfv}-pull \textsc{neg-inf}-can \textsc{emph} \textsc{nmlz}:S/A-be.heavy \textsc{ipfv.ifr}-be:\textsc{affirm} \\
\glt `It was so heavy that (the fisherman) could not pull it out (of the water).' (140512 yufu yu mogui, 35)
\end{exe}

\subsubsection{Adjectives of degree} \label{sec:degree.complement}
Adjectives of degree, like \jpg{rtaʁ}{be enough} or \jpg{tɕʰom}{be too much} are compatible with three distinct constructions: finite complement clauses (with a verb in the imperfective, as in example \ref{ex:pjWnArte}), infinitival complement clauses (as in \ref{ex:kWfsoR.kWtChom}) or, most commonly, the degree nominal complementation strategy (example \ref{ex:kotChom}, see also section \ref{sec:degree}).

\begin{exe}
\ex \label{ex:pjWnArte}
\gll \ipa{koŋla} 	\ipa{pjɯ-nɤrte} 	\ipa{ʑo} 	\ipa{kɯ-rtaʁ} 	\ipa{ʑo} 	\ipa{kɯ-wxti} 	\ipa{ɲɯ-βze} 	\ipa{ŋgrɤl}  \\
completely \textsc{ipfv}-wear.as.a.hat \textsc{emph} \textsc{nmlz}:S/A-enough \textsc{emph} \textsc{nmlz}:S/A-be.big \textsc{ipfv}-grow be.usually.the.case:\textsc{fact} \\
\glt `(Leaves of the burdock) can grow big enough to be worn as hats.' (13-tCamu, 38)
\end{exe}
 
\begin{exe}
\ex \label{ex:kWfsoR.kWtChom}
\gll [\ipa{tɤŋe} 	\ipa{kɯ-fse} 	\ipa{kɯ-fsoʁ}] 	\ipa{kɯ-tɕhom} 	\ipa{kɯ-fse} 	\ipa{nɯra} 	\ipa{ju-kɯ-ru} 	\ipa{rcanɯ} \\
sun \textsc{inf:stat}-be.like \textsc{inf:stat}-be.bright \textsc{inf:stat}-be.too.much \textsc{inf:stat}-be.like \textsc{dem:pl} \textsc{ipfv-genr}:S/P-look \textsc{unexpected} \\
\glt `(When one gets this eye disease), one looks (with the eyes half-closed) as (ones does when one's eyes are dazzled) when the sun is too bright.' (27-tApGi, 7)
\end{exe}

\begin{exe}
\ex \label{ex:kotChom}
\gll
\ipa{ma} 	\ipa{ɯ-tɯ-smi} 	\ipa{ko-tɕʰom} 	\ipa{qʰe} \\
because \textsc{3sg.poss-nmlz:degree}-be.cooked \textsc{ifr}-be.too.much \textsc{lnk} \\
\glt `Because if it cooks too much (it is not as tasty).' (Conversation 14.05.10)
\end{exe}

 
%  \begin{exe}
%\ex \label{ex:XtCoN}
%\gll
% \ipa{χtɕoŋ} 	\ipa{nɯ} 	\ipa{ɲɯ-pʰɤn} 	\ipa{ɲɯ-ti-nɯ} 	\ipa{ri}  \\
% rheumatism \textsc{dem} \textsc{sens}-be.efficient \textsc{sens}-say-\textsc{pl} but \\
% \glt `People say it is good against rheumatism, but...' (20-sWrna, 146)
%\end{exe}
% 
%Example (\ref{ex:tWmtshi}) is not an example of complement clause, as the S-participle of \jpg{mŋɤm}{hurt} is lexicalized in the sense of `disease'.
%
%  \begin{exe}
%\ex \label{ex:tWmtshi}
%\gll 
%[\ipa{tɯ-mtsʰi} 	\ipa{kɯ-mŋɤm}] 	\ipa{wuma} 	\ipa{ʑo} 	\ipa{pʰɤn} 	\ipa{tu-ti-nɯ} 	\ipa{ŋgrɤl}  \\
% \textsc{indef.poss-liver}  \textsc{nmlz}:S/A-hurt really \textsc{emph} be.efficient:\textsc{fact} \textsc{ipfv}-say-\textsc{pl} be.usually.the.case:\textsc{fact} \\
% \glt (05-qaZo 38)
%\end{exe}
 

 

\subsection{Complements of nouns and noun-verb collocations}
\ipa{mɯ-tu-kɤ-mbro} 	\ipa{ftɕaka} 	\ipa{tu-βze-a} 	\ipa{ŋu} 

\ipa{rgɤtpu} 	\ipa{nɯnɯ} 	\ipa{jɤ-nɯɣe} 	\ipa{ɣɯ} 	\ipa{ɯ-zgra} 	\ipa{nɯ} 	\ipa{pa-mtshɤm} 	\ipa{tɕe,} 


 \section{Conclusion}
 
\bibliographystyle{unified}
\bibliography{bibliogj}
\end{document}