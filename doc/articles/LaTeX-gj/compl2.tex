\documentclass[oldfontcommands,oneside,a4paper,11pt]{article} 
\usepackage{fontspec}
\usepackage{natbib}
\usepackage{booktabs}
\usepackage{xltxtra} 
\usepackage{polyglossia} 
%\usepackage[table]{xcolor}
\usepackage{gb4e} 
\usepackage{multicol}
\usepackage{graphicx}
\usepackage{float}
\usepackage{lineno}
\usepackage{textcomp}
\usepackage{hyperref} 
\hypersetup{bookmarks=false,bookmarksnumbered,bookmarksopenlevel=5,bookmarksdepth=5,xetex,colorlinks=true,linkcolor=blue,citecolor=blue}
\usepackage[all]{hypcap}
\usepackage{memhfixc}
 

%\setmainfont[Mapping=tex-text,Numbers=OldStyle,Ligatures=Common]{Charis SIL} 
\newfontfamily\phon[Mapping=tex-text,Ligatures=Common,Scale=MatchLowercase]{Charis SIL} 
\newcommand{\ipa}[1]{\textbf{\phon#1}} %API tjs en italique
 \newcommand{\jpg}[2]{\ipa{#1} `#2'} %API tjs en italique
\newcommand{\grise}[1]{\cellcolor{lightgray}\textbf{#1}}
\newfontfamily\cn[Mapping=tex-text,Ligatures=Common,Scale=MatchUppercase]{SimSun}%pour le chinois
\newcommand{\zh}[1]{{\cn #1}}
\newcommand{\tld}{\textasciitilde{}}

\XeTeXlinebreaklocale "zh" %使用中文换行
\XeTeXlinebreakskip = 0pt plus 1pt %
 %CIRCG
 


\begin{document} 

\title{Complementation in Japhug Rgyalrong}%\footnote{} 
\author{Guillaume Jacques}
\maketitle
\linenumbers
 
 
\section{Introduction}
 \citet[9]{dixon06complementation}
 \citet{sun12complementation}
 \citet{jacques08}

\section{Background information}

\subsection{Transitivity in Japhug} \label{sec:transitivity}


\subsubsection{Semi-transitive verbs}

\subsubsection{Unmarked adjuncts}

goal


Essive / functive (\citealt{creissels14functive}))
\begin{exe}
\ex \label{ex:YWtambi}
\gll \ipa{a-me} 	\ipa{nɯ} 	\ipa{nɤ-rʑaβ} 	\ipa{ɲɯ-ta-mbi} 	\ipa{ŋu} \\
\textsc{1sg.poss}-daughter \textsc{dem} \textsc{2sg.poss}-wife \textsc{ipfv}-1$\rightarrow$2-give be:\textsc{fact} \\
\glt `I will give you my daughter in marriage.'
\end{exe}

\subsection{Participles vs infinitives} \label{sec:part.inf}
All Gyalrong languages, including Situ (\citealt{youjing03zhuokeji}), Tshobdun (\citealt{sun12complementation}) and Japhug, have a distinction between infinitives and participles. The distinction is quite subtle, as there are both infinitives and participles in \ipa{kɯ-} and \ipa{kɤ-} (or \ipa{kə-} and \ipa{kɐ-} depending on the transcription system). Since both categories are formally similar and can occur in similar contexts, it is crucial to clearly explain the distinction between the two, especially since Japhug slightly differs from the other Gyalrong languages in this regard.

\subsubsection{Participles}
The system of core argument participles in Japhug is relatively straightforward (\citealt{jacques16relatives}). The prefix \ipa{kɯ-} is used to build the S-participle of intransitive verbs, and the A-participle of transitive verbs. A-participles differ from S-participles in having in addition a possessive prefix coreferent with the P, as in (\ref{ex:akWfstWn}).

\begin{exe}
\ex \label{ex:akWfstWn}
\gll \ipa{a-me} 	\ipa{a-kɯ-fstɯn} 	\ipa{ŋu} \\
\textsc{1sg.poss}-daughter \textsc{1sg.poss}-\textsc{nmlz}:S/A-serve be:\textsc{fact} \\
\glt `My daughter is the one who takes care of me.' (The prince, 74)
\end{exe}

The prefix \ipa{kɤ-} on the other hand serves to build the P-participle, and can optionally take a possessive prefix coreferent with the A, as in (\ref{ex:tajmag}). Participles are compatible with polarity and associated motion prefixes (\citealt{jacques16relatives}).

\begin{exe}
   \ex \label{ex:tajmag}
   \gll
\ipa{aʑo}  	\ipa{a-mɤ-kɤ-sɯz}   	\ipa{tɤjmɤɣ}  	\ipa{nɯ}  	\ipa{kɤ-ndza}  	\ipa{mɤ-naz-a}  \\
\textsc{1sg} \textsc{1sg-neg-nmlz:P}-know mushroom \textsc{dem} \textsc{inf}-eat \textsc{neg}-dare:\textsc{fact}-\textsc{1sg} \\
\glt `I do not dare to eat the mushrooms that I do not know.' (23 mbrAZim,103)
\end{exe}

\subsubsection{Infinitives} \label{sec:infinitives}
There are four types of infinitives in Japhug: \ipa{kɯ-}, \ipa{kɤ-}, \ipa{tɯ-} and bare infinitives. The latter two are restricted to very specific constructions (see \ref{sec:bareinf}), and only the former two types are discussed in this section. 

Infinitives in \ipa{kɤ-} are by far the most common form in Japhug. The \ipa{kɯ-} form is restricted to stative verbs (including adjectives, copulas and existential verbs) and impersonal auxiliaries, but even with these verbs, \ipa{kɤ-} infinitives are used in several contexts. 

Infinitives are used in three types of constructions, a brief overview of which is provided below.

First, infinitives occur as the citation forms of verbs and in metalinguistic discussion in Japhug, as in examples (\ref{ex:mAkWBdi}) and (\ref{ex:kAnARarphAB}) for stative vs non-stative infinitives.\footnote{This not the only available construction to express this -- the imperfective with generic person marking is also used (\citealt{jacques15generic}).}

\begin{exe}
\ex  \label{ex:mAkWBdi}
 \gll \ipa{ɯnɯnɯ} 	\ipa{tɕe} 	\ipa{tɕe} 	\ipa{ɯ-tɯ-tʂɯβ} 	\ipa{mɤ-kɯ-βdi} 	\ipa{tu-kɯ-ti} 	\ipa{ŋu} \\ 
 \textsc{dem} \textsc{lnk} \textsc{lnk} \textsc{3sg.poss-nmlz:action}-sew \textsc{neg-inf:stat}-be.good  \textsc{ipfv-genr}:A-say be:\textsc{fact}  \\
\glt  `People call this `badly sewn'.'  (12-kAtsxWb, 12)
\end{exe}

\begin{exe}
\ex \label{ex:kAnARarphAB}
 \gll \ipa{pjɯ-sɯ-ʁndi} 	\ipa{tɕe} 	\ipa{pjɯ-sɯ-sat.} \ipa{tɕe} 	\ipa{nɯ} 	\ipa{koʁmɯz} 	\ipa{nɤ} 	\ipa{cʰɯ-nɯtsɯm} 	\ipa{ɲɯ-ra.} \ipa{tɕe} 	\ipa{nɯnɯ} 	\ipa{kɤ-nɤʁarphɤβ} 	\ipa{tu-kɯ-ti} 	\ipa{ŋu} \\
 \textsc{ipfv-caus}-hit[III]  \textsc{lnk} \textsc{ipfv-caus}-kill \textsc{lnk} \textsc{dem} only.then \textsc{lnk} \textsc{ipfv:downstream}-take.away \textsc{sens}-have.to \textsc{lnk} \textsc{dem} \textsc{inf}-strike.with.wings \textsc{ipfv-genr}:A-say be:\textsc{fact}  \\
 \glt `It strikes it and kills it (with its wings) and only then takes it away. This is called \ipa{kɤ-nɤʁarphɤβ} `strike with one's wings'.' (hist150819 RarphAB, 11)
\end{exe}

In the topical position, the infinitive is neutralized to the \ipa{kɤ-} form even for stative verbs, as in (\ref{ex:kArZi}).

\begin{exe}
\ex \label{ex:kArZi}
 \gll
 \ipa{kɤ-rʑi} 	\ipa{ri} 	\ipa{pjɤ-rʑi,} 	  \\
 \textsc{inf}-be.heavy also \textsc{ifr.ipfv}-be.heavy \\
 \glt `As for being heavy, (the old man) was heavy.'  (140511 xinbada, 138)
\end{exe}

Second, infinitives are used as converbs to indicate the manner which the action of the main clause occurs (example \ref{ex:kANke.jari}), or a background event (\citealt{jacques14linking}). The \ipa{kɯ-} infinitive form occurs with stative verbs (as \jpg{sɤscit}{nice (of an environment)} in example \ref{ex:kWsAscWscit}) but it is also attested with a handful of dynamic verbs in lexicalized form such as \ipa{mɤ-kɯ-mbrɤt} `without stop' in (\ref{ex:mAkWmbrAt}).\footnote{The implied S of the verb \jpg{mbrɤt}{break, stop suddenly} (the anticausative of  \jpg{prɤt}{break}) in this sentence is the work of the subject.  } This latter use is the last trace of the contrast between human \ipa{kɐ-} and non-human \ipa{kə-} action nominals reported by \citet[476]{sun12complementation} and \citealt{jackson14morpho}, which otherwise appears to have been lost in the variety of Japhug under study.

\begin{exe}
\ex \label{ex:kANke.jari}
\gll
\ipa{kɤ-ŋke} 	\ipa{jɤ-ari} 	\ipa{pɯ-ra} \\
\textsc{inf}-walk \textsc{pfv}-go[II] \textsc{pst.ipfv}-have.to \\
\glt He had to go on foot. (elicited)
\end{exe}

\begin{exe}
\ex \label{ex:kWsAscWscit}
\gll
\ipa{ɕɤr} 	\ipa{tɕe} 	\ipa{nɯtɕu} 	\ipa{kɯ-sɤ-scɯ\tld{}scit} 	\ipa{ʑo} 	\ipa{ɕ-ku-nɯ-rŋgɯ} 	\ipa{ŋu} \\
night \textsc{lnk} \textsc{dem:loc} \textsc{inf:stat-deexp-emph}\tld{}be.happy \textsc{emph} \textsc{transloc-ipfv-auto}-lie.down be:\textsc{fact} \\
\glt `In the night, he goes in there to sleep cosily.' (26-NalitCaRmbWm, 35)
\end{exe}

\begin{exe}
\ex \label{ex:mAkWmbrAt}
\gll
 \ipa{nɯ} 	\ipa{maka} 	\ipa{mɤ-kɯ-mbrɤt} 	\ipa{ʑo} 	\ipa{ɲɯ-rɤma} 	\ipa{ɲɯ-ɕti} 	\ipa{tɕe} \\
 \textsc{dem} at.all \textsc{neg-inf-anticaus:}break \textsc{emph} \textsc{ipfv}-work \textsc{sens}-be:\textsc{affirm} \textsc{lnk} \\
\glt `It works without stop.' (hist-26-GZo.txt 67)
\end{exe}

Third, infinitives are one way to build complement clauses in Japhug. Complement clauses nearly always have the \ipa{kɤ-} infinitive, even stative verbs, as in (\ref{ex:rYo}), where the verb of the complement clause \jpg{tu}{exist}, whose citation form is \ipa{kɯ-tu} (with a \ipa{kɯ-} infinitive), is used with the \ipa{kɤ-} infinitive.

\begin{exe}
\ex \label{ex:rYo}
\gll \ipa{a-rŋɯl} 	\ipa{kɤ-tu} 	\ipa{pɯ-rɲo-t-a} \\
\textsc{1sg.poss}-money \textsc{inf}-exist \textsc{pst:ipfv}-experience-\textsc{pst:tr-1sg} \\
\glt `I used to have money'. (elicited)
\end{exe}

Complement clauses with verb prefixed with \ipa{kɯ-} can be analyzed as S/A participles (\ref{sec:SApart}) in nearly all cases.\footnote{In particular, when transitive verbs appear as complement, they take a possessive prefix coreference with the P.}  Complement clauses with \ipa{kɯ-} infinitives are found when  impersonal modal verbs such as \jpg{ra}{have to, need} occurs in a complement clause, as in example (\ref{ex:kAndza.kWra}), or with complement clauses of stative verbs (see example \ref{ex:kWfsoR.kWtChom} in section \ref{sec:adj}).

\begin{exe}
\ex \label{ex:kAndza.kWra}
\gll 
\ipa{smɤn} 	\ipa{kɤ-ndza} 	\ipa{kɯ-ra} 	\ipa{pɯ-rɲo-t-a} \\ 
medecine \textsc{inf}-eat \textsc{inf}-have.to  \textsc{pst:ipfv}-experience-\textsc{pst:tr-1sg} \\
\glt  `I used to have to take medecine.' 
\end{exe}

\subsection{Complement clauses vs complementation strategies} \label{sec:strategies}

\citet[15]{dixon06complementation} defines \textit{complement clauses} as subordinate clauses which function as one of the the core argument of a main clause. 

He introduces the term `complementation strategy' to refer to constructions corresponding with a meaning expressed by complement clauses in some languages, which either are not core arguments or the verb of the main clause or are not clauses with a complete argument structure \citealt[34-40]{dixon06complementation}). Complement strategies include nominalizations (when the verb sheds its argument structure as it becomes a noun), relative clauses (which are formally modifier of a core argument, not core arguments themselves), serial verb constructions and clause linking.

In Japhug, due to the existence of semi-objects, which are demonstrably core arguments despite nor being indexed on the verb (see \citet{jacques16relatives}), and due to the presence of unmarked adjuncts with various syntactic properties (see \ref{sec:transitivity}), the status of a particular clause as a core argument or non-core-argument is not always trivial to determine. In this paper, I keep Dixon's sensible distinction between complement clauses and complementation strategies, but given the gradient nature of the opposition between core and oblique arguments in Japhug, I use the term `complement clauses' to refer to all clauses which are part of the verb's argument structure. In particular, in consider the purposive clauses of motion verb to be complement clauses, for reasons that are exposed in more detail in section (\ref{sec:SApart}).
 

\section{Complement types}


\subsection{Infinitive} \label{sec:infinitives.compl}
The most common type of complement clauses in Japhug are \ipa{kɤ-} and \ipa{kɯ-} infinitival complements. As seen in section \ref{sec:infinitives}, there are \ipa{kɤ-} and \ipa{kɯ-} infinitives in Japhug, the latter found in the citation form of stative verbs and modal impersonal auxiliary verbs. In complement clauses, stative verbs take the \ipa{kɤ-} infinitive like dynamic verbs except if their matrix verb is itself a stative verb. The only verb that consistently take \ipa{kɯ-} infinitives in complement clauses are auxiliaries.

\subsubsection{Case marking} \label{sec:case.infinitive}
While infinitives bear no person indexation markers, noun phrases receive the same case markers in infinitive clauses as in main clauses, showing that infinitives have the same argument structures as finite verb forms.

When an argument is shared between the complement and the matrix clause, it often has a different syntactic function in the two clauses, as in \ref{ex:kAstu}, where \ipa{tɤɕime} `girl' is A in the complement clause (\jpg{stu}{do like this} is transitive) and S in the matrix clause (\jpg{cʰa}{can} is intransitive). 

\begin{exe}
\ex \label{ex:kAstu}
\gll [\ipa{tɤɕime} 	\ipa{nɯ} 	\ipa{kɯ} 	\ipa{nɯra} 	\ipa{kɤ-stu}] 	\ipa{pjɤ-cʰa} \\
girl \textsc{dem} \textsc{erg} \textsc{dem:pl} \textsc{inf}-do.like.this \textsc{ifr}-can \\
\glt `The girl succeeded in doing it.' (140511 alading, 252)
\end{exe}

In this sentence, the noun takes the ergative marker \ipa{kɯ} following the verb of the complement clause, showing that it belongs to the complement clause rather than to the matrix clause directly. This is the most commonly observed pattern in Japhug texts: in infinitival clauses, the shared arguments more often take the case marking selected by the verb of the complement clause than that of the matrix clause.

\subsubsection{Coreference between matrix and complement clause}
The coreference restrictions of the arguments in the complement and in the matrix clause differ from verb to verb. Some infinitival complements present an accusative pivot: the S or A of the matrix clause is necessarily coreferent with the A or S of the complement clause (this is the case with the verb \ipa{spa} `be able', see section \ref{sec:spa}).

Most verbs also allow coreference with the P of the complement clause. This is for instance the case of the semi-transitive verb \jpg{rga}{like}, whose S can be coreferent with either the S  (\ref{ex:kAnWrAGo.rganW}), the A (\ref{ex:kAnArtoXpjAt.pWrgaa}) and even the P (\ref{ex:YWrganW}) of its infinitival complement clause (\citealt{jacques16relatives}).

 \begin{exe}
   \ex   \label{ex:kAnWrAGo.rganW} 
\gll
\ipa{tsuku}  	\ipa{tɕe}  	\ipa{kɤ-nɯrɤɣo}  	\ipa{wuma}  	\ipa{ʑo}  	\ipa{rga-nɯ}  	\ipa{tɕe}  \\
some \textsc{lnk} \textsc{inf}-sing really \textsc{emph} like:\textsc{fact-pl}  \textsc{lnk} \\
  \glt Some people like to sing. (26 kWrNukWGndZWr, 104)
     \end{exe}  
 
   \begin{exe}
   \ex   \label{ex:kAnArtoXpjAt.pWrgaa} 
\gll
  	\ipa{aʑo}  	\ipa{qajɯ}  	\ipa{nɯ} \ipa{ra}  	\ipa{kɤ-nɤrtoχpjɤt}  	\ipa{pɯ-rga-a}  	\ipa{tɕe}  	\\
  	\textsc{1sg} bugs \textsc{dem} \textsc{pl} \textsc{inf}-observe \textsc{pst.ipfv}-like-\textsc{1sg} \textsc{lnk}  \\
 \glt I liked to observe bugs. (26 quspunmbro, 15)
     \end{exe}  
 
  \begin{exe}
   \ex   \label{ex:YWrganW} 
\gll
\ipa{maka}  	\ipa{tu-kɤ-nɤjoʁjoʁ,}  	\ipa{tu-kɤ-fstɤt}  	\ipa{nɯ}  	\ipa{ɲɯ-rga-nɯ}  \\
at.all \textsc{ipfv-inf}-flatter \textsc{ipfv-inf}-praise \textsc{dem} \textsc{ipfv}-like-\textsc{pl} \\
\glt They like to be flattered or praised. (140427 yuanhou, 53)
    \end{exe}  

%S, A, P of ɣɤkhɯ
Some verbs selecting S/A participle complement clauses use infinitive complement when there is coreference between the S of the matrix clause and the P of the complement clause (see section \ref{sec:SAparticiple.coref}).

Absence of core argument coreference between matrix clause and infinitival complement clause occurs in two cases. 

First, with intransitive impersonal modal verbs (such as \jpg{ra}{have to}, section \ref{sec:ra}), the whole complement clause is treated as the S of the matrix verb. 

Second, there are examples of coreference between the S/A of the matrix clause and the possessor of the S rather than the S in the case of infinitival clause with experiencer verbs. This is for instance the case of the verb \ipa{mŋɤm} `hurt', which can only take a body part as its S -- the experiencer is indicated by a possessive prefix on the body part, as in (\ref{ex:YWmNAm}).
 
 \begin{exe}
\ex \label{ex:YWmNAm}
\gll \ipa{a-xtu} 	\ipa{ɲɯ-mŋɤm} \\
\textsc{1sg.poss}-belly \textsc{sens}-hurt \\
\glt `My belly hurts.' 
\end{exe}
 
This verb can nevertheless be  used as complement of verbs such as \jpg{rɲo}{experience} when the experiencer is coreferent with the A, as in example (\ref{ex:kAmNAm}).
 
 \begin{exe}
\ex \label{ex:kAmNAm}
\gll \ipa{aʑo} 	\ipa{pɯ-xtɕɯ\tld{}xtɕi-a} 	\ipa{ʑo} 	\ipa{ri} 	\ipa{tɯxtɤŋɤm} 	\ipa{nɯ-atɯɣ-a} 	\ipa{tɕe,} 	\ipa{nɯ} 	\ipa{kɤ-mŋɤm} 	\ipa{pɯ-rɲo-t-a} \\
\textsc{1sg} \textsc{pst:ipfv-emph}\tld{}be.small-\textsc{1sg} \textsc{emph} \textsc{loc} dysentery \textsc{pfv}-meet-\textsc{1sg} \textsc{lnk} \textsc{dem} \textsc{inf}-hurt \textsc{pfv}-experience-\textsc{1sg} \\
\glt `When I was very small, I had dysentery, (my belly) ached.'  (24-pGArtsAG, 121)
\end{exe}

 \subsection{S/A participles} \label{sec:SApart}
In Japhug, a handful of verbs select clauses with S/A-participles rather than infinitival clauses, including  some motion verbs (\jpg{ɕe}{go}, \jpg{ɣi}{come}, but not \jpg{rɟɯɣ}{run}) and one aspectual verb  (\jpg{rɤŋgat}{be about to}). These verbs are all morphologically intransitive.

Example (\ref{ex:WkWnAjo}) illustrates this construction, with the A-participle \ipa{ɯ-kɯ-n-nɤjo} `waiting for him'. Note that the common argument shared between the participial clause (whose verb is transitive) and the matrix clause (whose verb is intransitive) takes the ergative, showing that it belongs to the participial clause, a pattern already observed with infinitival complement (see section \ref{sec:case.infinitive}).

\begin{exe}
\ex \label{ex:WkWnAjo}
\gll [\ipa{ɯ-wa} 	\ipa{nɯ} 	\ipa{kɯ} 	\ipa{kʰapa} 	\ipa{tɕe} 	\ipa{ɯ-kɯ-n-nɤjo}] 	\ipa{pjɤ-ɣi} \\
\textsc{3sg.poss}-father \textsc{dem} \textsc{erg} downstairs \textsc{lnk} \textsc{3sg-nmlz:S/A-auto}-wait \textsc{ifr:down}-come \\
\glt `His father had come downstairs to wait for him.' (140506 loBzi, 5)
\end{exe}


A similar construction is also found in Tshobdun, where \citet{sun12complementation} (Following Dixon) treats these clauses as a relativization strategy rather than as proper complements, since these clauses are not core arguments of the verb. However, this construction is highly grammaticalized and specific to a class of verbs which do not form a natural semantic class, which suggests that the participial clauses are selected by these verbs' argument structure. Though not core arguments, the participial clauses can be treated as an unmarked oblique arguments of the verb, and therefore be analyzed as genuine complement clauses.

\subsubsection{S/A- vs P-participles} \label{sec:SAparticiple.coref}
In examples such as (\ref{ex:WkWnAjo}), the S or A of the participial clause is obligatorily coreferent with the S of the matrix clause.

Coreference between the P of the participial clause and the S of the matrix clause is possible but requires using the P-participle. Several examples of this construction are found in the corpus with the verb \jpg{nɤkʰu}{invite to one's home as a guest}, as in (\ref{ex:kAnAkhu}).

\begin{exe}
\ex \label{ex:kAnAkhu}
\gll <xingqi> 	\ipa{raŋri} 	\ipa{ʑo} 	\ipa{tɕe} 	\ipa{nɯnɯ} \ipa{sɤβʑɯ} 	\ipa{ɣɯ} 	\ipa{ɯ-kʰa} 	\ipa{nɯtɕu} 	\ipa{kɤ-nɤkʰu}] 	\ipa{ju-ɣi} 	\ipa{pjɤ-ŋu} \\
week each \textsc{emph} \textsc{lnk} \textsc{dem} mouse \textsc{gen} \textsc{3sg.poss}-house \textsc{dem:loc} \textsc{nmlz:P}-invite \textsc{ipfv}-come \textsc{ipfv.ifr}-be \\
\glt `He would come as a guest to the mouse's house as a guest.' (150818 muzhi guniang, 299).
\end{exe}

We know that this form is the P-participle rather than the infinitive because it is possible to optionally add a possessive prefix coreferent with the A, as in (\ref{ex:akAnAkhu}).

\begin{exe}
\ex \label{ex:akAnAkhu}
\ipa{a-kɤ-nɤkʰu}] 	\ipa{jo-ɣi}  \\
 \textsc{1sg.poss-nmlz:P}-invite \textsc{ifr}-come \\
\glt `He came to my house as a guest.' (elicited)
\end{exe}

Coreference between the P of the complement clause and the S of the matrix clause is possible only if the P has control over the action, something that is possible for only verb few transitive verbs and explains the rarity of this construction. 
    
\subsubsection{Complements of participles}
When a complement-bearing verb is itself in the S/A-participle form, it is possible for the complement either to be in the expected form (infinitive or finite), or to be in S/A-participle form itself by contagion, as in example (\ref{ndZikWsAndu}).

\begin{exe}
\ex \label{ndZikWsAndu}
\gll [\ipa{rŋɯl} 	\ipa{kɯ} 	\ipa{ndʑi-kɯ-sɤndu}] 	\ipa{kɯ-cha} 	\ipa{kɯ-fse} 	\ipa{pɯ\tld{}pɯ-tu} 	\ipa{nɤ} \\
silver \textsc{erg} \textsc{3du-nmlz}:S/A-exchange \textsc{nmlz}:S/A-can \textsc{nmlz}:S/A-be.like 
\textsc{cond}\tld{}\textsc{pst.ipfv}-exist if \\
\glt `If there was someone who could exchange (the life of two brothers) with money, ...' (140507 jinniao, 339)
\end{exe}

Such examples are rare, but not considered to be mistakes by consultants when listening again to the recordings.

\subsubsection{Complement or relative clause?}
The transitive verb \jpg{nɯɕpɯz}{pretend, imitate} and the semi-transitive \jpg{ʑɣɤpa}{pretend} superficially appear to allow participial complements like motion verbs, as could be deduced from examples such as (\ref{ex:YWkWrABraR}). 

\begin{exe}
\ex \label{ex:YWkWrABraR}
\gll \ipa{tɤ-mu} 	\ipa{nɯ} 	\ipa{kɯ} 	\ipa{ɯ-ku} 	\ipa{ci} 	\ipa{ɲɯ-kɯ-rɤβraʁ} 	\ipa{to-nɯɕpɯz} \\
\textsc{indef.poss}-mother \textsc{dem} \textsc{erg} \textsc{3sg.poss}-head \textsc{indef} \textsc{ipfv-nmlz}:S/A-scratch \textsc{ifr}-pretend \\
\glt `The (rakshasi)-mother pretended to scratch her head.' (Slob.dpon1, 
\end{exe}

However, unlike verbs such as \jpg{ɕe}{go} or \jpg{rɤŋgat}{be about to},  coreference between the subject of pretence verbs and the subject of the verb in participial form is not required, as in example (\ref{ex:pGatCW.kWGAwu}).

\begin{exe}
\ex \label{ex:pGatCW.kWGAwu}
\gll  \ipa{ɯ-zda} 	\ipa{nɯra,} 	\ipa{pɣɤtɕɯ} 	\ipa{kɯ-ɣɤwu,} 	\ipa{kʰɯna} 	\ipa{kɯ-ɤndzɯt,} 	\ipa{lɯlu} 	\ipa{kɯ-ɣɤwu} 	\ipa{qacʰɣa} 	\ipa{kɯ-mbri} 	\ipa{kɯ-fse,} 	\ipa{nɯra} 	\ipa{tu-nɯɕpɯz} 	\ipa{ɲɯ-spe.} \\
\textsc{3sg.poss}-companion \textsc{dem:pl} bird \textsc{nmlz}:S/A-cry dog \textsc{nmlz}:S/A-bark cat \textsc{nmlz}:S/A-cry fox \textsc{nmlz}:S/A-cry \textsc{inf:stat}-be.like \textsc{dem:pl} \textsc{ipfv}-imitate \textsc{sens}-be.able[III] \\
\glt `It is able to imitate other animals, cry like a bird, bark like a dog, mew like a cat or call like a fox.' (27-kikakCi, 141)
\end{exe}

Since the verb \jpg{nɯɕpɯz}{pretend, imitate} is transitive and can take as its P the person imitated by the A, as in (\ref{ex:tuWGnWCpWzndZi}), a different analysis of (\ref{ex:YWkWrABraR}) and (\ref{ex:pGatCW.kWGAwu}) offers itself: the phrases containing the S/A-participles there are not complement clauses, but in fact head-internal relatives.

\begin{exe}
\ex \label{ex:tuWGnWCpWzndZi}
\gll \ipa{ɣzɯ} 	\ipa{ra} 	\ipa{kɯ} 	\ipa{li} 	\ipa{ʑɤni} 	\ipa{tú-wɣ-nɯɕpɯz-ndʑi} \\
monkey \textsc{pl} \textsc{erg} again \textsc{3du} \textsc{ipfv-inv}-imitate-\textsc{du} \\
\glt `The monkeys imitated the two of them (and repeatedly threw back the coconuts at them).' (140511 xinbada, 262)
\end{exe} 

Examples (\ref{ex:YWkWrABraR}) and (\ref{ex:pGatCW.kWGAwu}) could be literally translated as `The mother pretended to be someone who is scratching her head' and `It is able to imitate a crying bird' respectively.

Yet, it is possible that the use of participles as complements with motion verbs and with \jpg{rɤŋgat}{be about to} was grammaticalized from a construction with such a head-internal relative clause in essive function instead of P function, as in (\ref{ex:WkWnAjo2}).

\begin{exe}
\ex \label{ex:WkWnAjo2}
\gll 	\ipa{ɯ-kɯ-nɤjo} 	\ipa{pjɤ-ɣi} \\
 \textsc{3sg-nmlz:S/A}-wait \textsc{ifr:down}-come \\
\glt *`He came as someone waiting for him' $\Rightarrow$ `He came to wait for him'. (from example \ref{ex:WkWnAjo})
\end{exe}

Reanalysis was complete when coreference between the subjects of the matrix and of the participle became obligatory.

%\begin{exe}
%\ex   \label{ex:pWkWZGAnWBlu} 
%\gll \ipa{pɯ-kɯ-ʑɣɤ-nɯβlu} \ipa{to-nɯɕpɯz} \\
%\textsc{pfv-nmlz:S/A-refl}-cheat \textsc{ifr}-pretend \\
%\glt  `He pretended having been duped (=He pretended to be someone who has let himself be cheated).' (Elicitation)
%\end{exe}  

 \subsection{Bare infinitive and \ipa{tɯ-} infinitives} \label{sec:bareinf}
A handful of transitive verbs, namely \jpg{ʑa}{begin}, \jpg{sɤʑa}{begin} and \jpg{rɲo}{experience, have already} are compatible with bare infinitival and \ipa{tɯ-} infinitival complements. The verbs \jpg{sɤʑa}{begin} and \jpg{rɲo}{experience} are more commonly used with \ipa{kɤ-} infinitives.


\subsubsection{Bare infinitives and transitivity}
Bare infinitives occur only with transitive verbs, and are formed by combining the stem 1 of the verb with a possessive prefix coreferent with the P of the complement clause, as in examples (\ref{ex:Wmto}).

\begin{exe}
\ex \label{ex:Wmto}
\gll \ipa{nɤʑo} 	\ipa{kɯ-fse} 	\ipa{a-ŋkʰor} 	\ipa{nɯ} 	\ipa{ɯ-mto} 	\ipa{mɯ-pɯ-rɲo-t-a} \\
you \textsc{nmlz:stative}-be.like \textsc{1sg.poss}-subject \textsc{top} \textsc{3sg}-\textsc{bare.inf:}see \textsc{neg-pfv}-experience-\textsc{pst:tr-1sg} \\
\glt  `I never saw anyone like you among my subjects.' (Smanmi metog koshana1.157)
\end{exe} 

Bare infinitives are in complementary distribution with \ipa{tɯ-} infinitives,  which occur when the verb of the complement is morphologically intransitive, and thus lacks a P. In this case, there is obligatory coreference between the A of the matrix verb and the S of the \ipa{tɯ-} infinitival complement. If overt, the noun phrase corresponding to the shared argument is generally in the absolutive form as in example \ref{ex:tWNke}, following the verb of the complement clause, though a few examples such as (\ref{ex:tWnWrAGo}) in the ergative are also attested.

\begin{exe}
\ex \label{ex:tWNke}
\gll
<xinbada> 	\ipa{nɯ} 	\ipa{tɕe} 	\ipa{li} 	\ipa{tɯ-ŋke} 	\ipa{to-ʑa} \\
Sinbad \textsc{dem} \textsc{lnk} again  \textsc{inf}-walk \textsc{ifr}-begin \\
\glt `Sinbad started to walk again.' (140511 xinbada, 217)
\end{exe}


\begin{exe}
\ex \label{ex:tWnWrAGo}
\gll \ipa{pɣɤtɕɯ} 	\ipa{nɯ} 	\ipa{kɯ} 	\ipa{nɯɕɯmɯma} 	\ipa{ʑo} 	\ipa{tɯ-nɯrɤɣo} 	\ipa{cʰɤ-ʑa} \\
bird \textsc{dem} \textsc{erg} immediately \textsc{emph} \textsc{inf}-sing \textsc{ifr}-begin \\
\glt `The bird immediately started to sing.' (140514 huishuohua de niao, 221)
\end{exe}

These infinitives are only compatible with polarity prefixes (as in example \ref{ex:mAtWrga} below), and cannot take TAM or possessive prefixes.

Semi-transitive verbs are treated like intransitive verbs: they cannot form a bare infinitive, and use \ipa{tɯ-} infinitives instead (example \ref{ex:mAtWrga}), although their semi-object does present some object-like syntactic properties (see \citealt{jacques16relatives}).  

\begin{exe}
\ex  \label{ex:mAtWrga}
\gll \ipa{qaɟy} 	\ipa{ɯ-me} 	\ipa{nɯnɯ,} 	\ipa{tɕendɤre} 	\ipa{kʰro} 	\ipa{mɤ-tɯ-rga} 	\ipa{to-ʑa} \\
fish \textsc{3sg.poss}-daughter \textsc{dem} \textsc{lnk} a.lot \textsc{neg-inf}-like \textsc{ifr}-start \\
\glt `He started not liking the mermaid that much.' (hist150819 haidenver, 154)
\end{exe}
 
The only exception to this distribution are some transitive verbs used in complex predicates referring to weather phenomena, in particular \ipa{lɤt} `throw' and \ipa{βzu} `make, do', as in (\ref{ex:tWmlAt}). Note that in this construction, although the verbs remain morphologically transitive, they cannot take an overt A marked with the ergative.
 
\begin{exe}
\ex  \label{ex:tWmlAt}
\gll
\ipa{tɯ-mɯ} 	\ipa{kɯ-wxtɯ\tld{}wxti} 	\ipa{ʑo} 	\ipa{tɯ-lɤt} 	\ipa{pjɤ-ʑa} \\
\textsc{indef.poss}-sky \textsc{nmlz:S/A-emph}\tld{}be.big \textsc{emph} \textsc{inf}-throw \textsc{ifr}-start \\
\glt `A big rain started.' (hist150819 haidenver, 104)
\end{exe}


%\ipa{hanɯni} 	\ipa{ɯ-rŋa} 	\ipa{ra} 	\ipa{tɯ-ɣɯrni} 	\ipa{tɯ-βzu} 	\ipa{ɲɤ-ʑa} 
%hist150820 meili de meiguihua

\subsubsection{Coreference restrictions}
Bare infinitives and \ipa{kɤ-} infinitives strongly differ as to their coreference restrictions. With \ipa{kɤ-} infinitives, the A of the matrix clause can be coreferent with either the A, the P  or even the possessor of the S of the complement clause (see example \ref{ex:kAmNAm} above). This ambiguity is particularly clear with the verb \ipa{nɤkʰu}`invite to one's home as a guest' (see examples \ref{ex:kAnAkhu1} and \ref{ex:kAnAkhu2}), as with this verb both arguments are equal in term of volition and control.

\begin{exe}
\ex  \label{ex:kAnAkhu1}
\gll
\ipa{ɯʑo} 	\ipa{kɯ} 	\ipa{kɤ-nɤkʰu} 	\ipa{pɯ-rɲo-t-a} \\
\textsc{3sg} \textsc{erg} \textsc{inf}-invite \textsc{pfv}-experience-\textsc{pst:tr-1sg} \\
\glt `I have been to his house as guest.'  (= `He has invited me to come to his house as a guest') (P=A)
\ex  \label{ex:kAnAkhu2}
\gll
\ipa{ɯʑo} 	\ipa{kɤ-nɤkʰu} 	\ipa{pɯ-rɲo-t-a} \\
\textsc{3sg}  \textsc{inf}-invite \textsc{pfv}-experience-\textsc{pst:tr-1sg} \\
\glt `He has been to my house as guest.' (= `I have invited him to come to my house as a guest.') (A=A)
\end{exe}

In the case of bare infinitive, on the other hand, the A and P of the matrix and complement clause must be identical, as shown by examples (\ref{ex:nAkhu1}) and (\ref{ex:nAkhu2}).
\begin{exe}
\ex  \label{ex:nAkhu1}
\gll \ipa{a-nɤkʰu} 	\ipa{pa-rɲo} \\
\textsc{1sg.poss-bare.inf:}invite \textsc{pfv:3$\rightarrow$3'}-experience \\
\glt `I have been to his house as guest.' 
\ex  \label{ex:nAkhu2}
\gll \ipa{ɯʑo} 	\ipa{ɯ-nɤkʰu} 	\ipa{pɯ-rɲo-t-a} \\
\textsc{3sg}  \textsc{3sg.poss-bare.inf:}invite \textsc{pfv}-experience-\textsc{pst:tr-1sg} \\
\glt `He has been to my house as guest.'
\end{exe}

\subsubsection{Historical origin}
Bare infinitives probably derive from action nominals. There are marginal examples in Japhug of bare infinitives used in this way, as in example \ref{ex:bare.inf.noun})  (\citealt{jacques14antipassive}.

\begin{exe}
\ex \label{ex:bare.inf.noun}
\gll \ipa{ndʑi-mi}   	\ipa{ɯ-tsʰoʁ}   	\ipa{ɯ-tsʰɯɣa}   	\ipa{nɯra}   	\ipa{wuma}   	\ipa{ʑo}   	\ipa{naχtɕɯɣ-ndʑi.}   \\
\textsc{3du.poss}-foot \textsc{3sg}-\textsc{bare.inf:}attach.to \textsc{3sg.poss}-form \textsc{dem:pl} very \textsc{emph}  \textsc{npst}:similar-\textsc{du}  \\
\glt `The way their feet (of fleas and crickets) touch the ground is very similar.' (26-mYaRmtsaR, 17)
\end{exe}

The complementary distribution between the bare infinitive and the \ipa{tɯ-} infinitive is puzzling. There are at least two possible ways of analyzing the origin of  \ipa{tɯ-} infinitives. 


First, they could be related to the \ipa{tɯ-} action nominals (\citealt{jacques14antipassive}), found in light verb constructions such as (\ref{ex:tWrJaR}) or (\ref{ex:tWtWtsxaB})\footnote{When the \ipa{tɯ-} nominalization prefix is reduplicated as in (\ref{ex:tWtWtsxaB}), it conveys the meaning of several persons/objects being subjected to the same action together. } and used to build abstract nouns (\ipa{si} `die' $\rightarrow$ \ipa{tɯ-si} `death'). This solution is attractive due to the fact that \ipa{tɯ-} action nominals are relatively common, but it does not account well for the complementary distribution of bare infinitives and \ipa{tɯ-} infinitives, since both intransitive and transitive verbs can build \ipa{tɯ-} action nominals.

\begin{exe}
\ex \label{ex:tWrJaR}
\gll \ipa{tɯ-rɟaʁ} \ipa{pɯ-βzu-t-a} \\
\textsc{nmlz:action}-dance \textsc{pfv}-do-\textsc{pst:tr-1sg} \\
\glt `I danced.'
\end{exe}

\begin{exe}
\ex \label{ex:tWtWtsxaB}
\gll
\ipa{kuxtɕo} 	\ipa{cʰondɤre}  	\ipa{kɯrtsɤɣ} 	\ipa{nɯra} 	\ipa{ɯ-pa} 	\ipa{nɯtɕu} 	\ipa{tɯ\tld{}tɯ-tʂaβ} 	\ipa{ʑo} 	\ipa{pjɤ-βzu} \\
basket \textsc{comit} leopard \textsc{dem:pl} \textsc{3sg}-down \textsc{dem:loc} \textsc{together}\tld{}\textsc{nmlz:action}-cause.to.roll \textsc{emph} \textsc{ifr:down}-make \\
\glt `(The rabbit) caused the leopard and the basket to roll down together.' (The rabbit 2002, 72)
\end{exe}

Second, one could interpret the \ipa{tɯ-} here as the indefinite possessor prefix \ipa{tɯ-}, which is added to inalienably possessed nouns when no definite possessor is present (\citealt{jacques15generic}). In this hypothesis, the bare infinitive takes a possessive prefix coreferent with its P with transitive verbs, but in the case of intransitive verbs, given the absence of P argument, the indefinite possessor is used instead.



 
 \subsection{Finite complements} 

 imperfective, irrealis, same TAM category as main clause

 \subsection{Complementation strategies}  
 
  \subsubsection{Relatives}  
  
  \subsubsection{Degree nominal}  \label{sec:degree}
Adjectives of degree like \jpg{rtaʁ}{be enough}, \jpg{tɕʰom}{be too much}, \jpg{naχtɕɯɣ}{be identical} or \jpg{saχaʁ}{be extremely} can be used with infinitival and finite complements (section \ref{sec:degree.complement}), but the most common construction involves degree nominals, build by prefixing \ipa{tɯ-} and a possessive prefix to the stem of the verb, as in (\ref{ex:nWrtaR}) or (\ref{ex:YWtChom}). Although most degree nominals in the corpus derive from adjectives, there are also a few examples of dynamic verbs, as (\ref{ex:nWrtaR}).


\begin{exe}
\ex \label{ex:nWrtaR}
\gll \ipa{ɯ-tɯ-ɤla} 	\ipa{nɯ-rtaʁ} 	\ipa{ʑo} 	\ipa{tɕe} 	\ipa{tɕe} 	\ipa{chɯ́-wɣ-tɕɤt} 	\ipa{tɕe} 	\ipa{ɲɯ́-wɣ-χtɕi} \\
\textsc{3sg.poss-nmlz:degree}-boil \textsc{pfv}-be.enough \textsc{emph} \textsc{lnk} \textsc{lnk} \textsc{ipfv-inv}-take.out  \textsc{lnk} \textsc{ipfv-inv}-wash \\
\glt `When it has boiled enough, one takes it out and washes it.' (30-tasa, 4)
\end{exe}

\begin{exe}
\ex \label{ex:YWtChom}
\gll
\ipa{kɯki} 	\ipa{kʰa} 	\ipa{ki} 	\ipa{ɯ-tɯ-xtɕi} 	\ipa{ɲɯ-tɕʰom} \\
\textsc{dem} house \textsc{dem} \textsc{3sg.poss-nmlz:degree}-be.small \textsc{sens}-be.too.much \\
\glt `This house is too small.' (140430 yufu he tade qizi, 83)
\end{exe}

In this construction, the degree nominal is the S of the adjective of degree. The possessive prefix refers to the S of the nominalized verb. Thus, example (\ref{ex:YWtChom}) for instance can be literally translated as `The smallness of this house is excessive'. 

This construction does not fulfill all conditions for being a proper complement clause in Dixon's definition. The degree nominal, unlike participles, can only take one argument and is not compatible with TAM marking.
 
  \section{Morphosyntactic properties of complement clauses} 

 \subsection{Syntactic pivots} 

  \subsection{Raising of TAM and person/number marking} 

 \subsection{Plural and demonstrative markers} 
  \ipa{kɤ-nɯrtsɯ} 	\ipa{kɤ-ŋke} 	\ipa{ra} 	\ipa{tɤ-cha} 	\ipa{tɕe}   
  
  (140426 tApAtso kAnWBdaR, 65)
  \subsection{Semi-finiteness} 
like relative clauses, \citealt{jacques16relatives}

 \subsection{Discontinuous complement} 

Discontinuous clauses are rare in Japhug. The only clear example in our corpus is (\ref{ex:lWlu.kW.aZo}). In this example, the \textsc{1sg} pronoun \ipa{aʑo} (the subject of the matrix clause, which has no syntactic role in the complement clause) appears between the A \ipa{lɯlu} 	\ipa{kɯ} `the cat' and the P \ipa{ʁnɯz} `two' of the complement clause. Despite the rarity of this construction, this sentence was not considered to be unusual by our consultant when listening again to the recording.
 
 \begin{exe}
\ex \label{ex:lWlu.kW.aZo}
\gll \ipa{tɕe} 	[\ipa{lɯlu} 	\ipa{kɯ} 	\ipa{aʑo} 	\ipa{ʁnɯz} 	\ipa{ʑo} 	\ipa{ka-ndo}] 	\ipa{pɯ-mto-t-a} \\
\textsc{lnk} cat \textsc{erg} \textsc{1sg} two \textsc{emph} \textsc{pfv}:3$\rightarrow$3'-take \textsc{pfv}-see-\textsc{pst:tr-1sg} \\
\glt `I saw a cat catching two of them.' (22-kumpGatCW, 61)
\end{exe}


  



 
\subsection{Restrictive} 
To express a restriction (`only') having scope over a complement clause, the postposition \ipa{ma} `apart from' is used after the complement, sometimes with the postposition repeated two times [X \ipa{ma} \ipa{nɯ} \ipa{ma}] `apart from X, apart from it' as in example (\ref{ex:manWma.compl})

\begin{exe}
\ex \label{ex:manWma.compl}
\gll \ipa{kɤ-mtsʰɤm} 	\ipa{ma} 	\ipa{nɯ} 	\ipa{ma} 	\ipa{mɯ-pɯ-rɲo-t-a} \\
\textsc{inf}-hear apart.from \textsc{dem} apart.from \textsc{neg-pfv}-experience-\textsc{pst:tr-1sg} \\
\glt `I only heard about it.' (I did not see it and even do not claim that it exists, of a mythological animal) (20-RmbroN, 118)
\end{exe}
 
  \section{A classification of complement-taking verbs} 
  \subsection{Modal verb}
  
    \subsubsection{\jpg{spa}{be able to}} \label{sec:spa}
The verb \jpg{spa}{be able to} is the only transitive modal verb.\footnote{The causative verb \jpg{sɯxcʰa}{(cause to) have the ability to} is also formally transitive, but is only used in inverse forms, see section XXX)} It originates from the abilitative form of the verb \ipa{pa} `do' (\citealt{jacques15causative}) and has a cognate in Tangut  (\citealt{jacques14esquisse}), showing that its lexicalization occurred even earlier than proto-Gyalrongic.

The verb \jpg{spa}{be able to} takes both infinitival (examples \ref{ex:rYo:inf:A} and \ref{ex:rYo:inf:S}) or finite complements, and its A is coreferent with the S or the A of the complement clause.

\begin{exe}
\ex  \label{ex:rYo:inf:A}
\gll
\ipa{nɯ} 	\ipa{ɯ-mdoʁ} 	\ipa{nɯ} 	\ipa{aj} 	\ipa{kɤ-ti} 	\ipa{mɯ́j-spe-a} \\
\textsc{dem} 3sg.poss-colour \textsc{dem} \textsc{1sg} \textsc{inf}-say \textsc{neg:sens}-be.able[III]-1sg \\
\glt I am not able to name its colour. (06-qaZmbri, 57)
\end{exe}

\begin{exe}
\ex  \label{ex:rYo:inf:S}
\gll 
 \ipa{kɤ-nɤre} 	\ipa{ɯ-tá-spa?}\\
 \textsc{inf}-laugh \textsc{q-pfv}:3$\rightarrow$3'-be.able.to\\
 \glt `Is he now able to laugh?' (conversation, 2014, of a three month old infant)
\end{exe}
    
  \subsubsection{\jpg{ra}{have to}} \label{sec:ra}
  \ipa{tɤ-pɤtso} 	\ipa{nɯ,} 	\ipa{tɯ-pɤrme} 	\ipa{roro} 	\ipa{jamar} 	\ipa{tɕe} 	\ipa{tɕe} 	\ipa{tɯ-nɯ} 	\ipa{kɤ-sɯβde} 	\ipa{pjɤ-ra} 
  
  \subsection{Phasal verbs and other aspectual auxiliaries}
\subsubsection{\jpg{rɲo}{experience}}   \label{sec:rɲo}
\subsubsection{\jpg{ʑa}{begin}}   \label{sec:Za}

tɕe βzɯr ri tɕe chɯ́-wɣ-mphɯr chɯ́-wɣ-ʑa tɕe


   \subsection{Verbs of perception}
% ɯʑo srɯnmɯ kɯ-ŋu nɯ tɤ-wa nɯ kɯ mɯ-pjɤ-sɯχsɤl, ɯ-nmaʁ nɯ kɯ.
  
  \subsection{Motion verbs}
  
motion verbs and associated motion   \citet{jacques13harmonization}
  
\subsection{Causative verbs}


 %mɯ-tu-kɤ-nɤtɯti kɯ-ra tɤ́-wɣ-sɯ-βzu-a-nɯ ndʐa ɕti ma

\subsection{Complements of adjectives} \label{sec:adj}
Adjectives in Japhug can be formally defined as the subclass of stative verbs allowing the tropative derivation (\citealt{jacques13tropative}).\footnote{This definition excludes some noun-like property words.} We can distinguish two classes of adjectives depending on the complements they can take.


\subsubsection{Infinitival and finite complements} \label{sec:adj.infinitive}
A few adjectives are semi-transitive (see section \ref{sec:transitivity}), like \jpg{mkʰɤz}{be expert, be knowledgeable} and optionally take either a noun (\ref{ex:CoNBzu.mkhAz}) or an complement clause (\ref{ex:mkhAztCi}) addition to their S. The complement clause can be either infinitival or finite, with a verb in the imperfective.

\begin{exe}
\ex \label{ex:CoNBzu.mkhAz}
\gll 
\ipa{ɯ-nmaʁ} 	\ipa{jɤ-kɯ-ɣe} 	\ipa{nɯ} 	\ipa{ɕoŋβzu} 	\ipa{mkʰɤz} 	\ipa{tɕe} \\
\textsc{3sg.poss}-husband \textsc{pfv-nmlz}:S/A-come[II] \textsc{dem} carpentry be.expert:\textsc{fact} \textsc{lnk} \\
\glt `Her husband (who came to live in her family) is very good at carpentry.' (14-tApitaRi, 273)
\end{exe}

\begin{exe}
\ex \label{ex:mkhAztCi}
\gll \ipa{tɕiʑo} 	\ipa{rcanɯ,} 	\ipa{kɤ-taʁ} 	\ipa{wuma} 	\ipa{ʑo} 	\ipa{mkʰɤz-tɕi} 	 \\
\textsc{1du}  \textsc{unexpected} \textsc{inf}-weave really \textsc{emph} be.expert:\textsc{fact}-\textsc{1du} \\
\glt `We are very good at weaving.' (140521, huangdi de xinzhuang, 20)
\end{exe}

Adjectives such as \jpg{ɴqa}{be difficult}, \jpg{mbat}{be easy}, which unlike \jpg{mkʰɤz}{be expert} do not have a semi-object, can also take infinitival or finite complement clauses as their S (as in \ref{ex:YWmbat}).

\begin{exe}
\ex \label{ex:YWmbat}
\gll
<gang> 	\ipa{stʰɯci} 	\ipa{mɯ́j-rko} 	\ipa{qʰe,} 	\ipa{ɲɯ-mpɯ} 	\ipa{qʰe} 	[\ipa{tu-ŋgɤɣ,} 	\ipa{ɲɯ-ɤjʁu} 	\ipa{nɯra} 	\ipa{ɲɯ-mbat} \\
steel as.much \textsc{neg:sens}-be.hard \textsc{lnk} \textsc{sens}-be.soft \textsc{lnk} \textsc{ipfv-anticaus}:bend \textsc{ipfv}-be.curved  \textsc{dem:pl}] \textsc{sens}-be.easy \\
\glt `(Iron) is not as hard as steel, it is soft and bends easily.' (30-Com, 42)
\end{exe}

In nearly all cases, the infinitival complements of stative verbs is in the \ipa{kɤ-} infinitive form. The only exception found in the corpus is the verb  \jpg{pʰɤn}{be efficient}, as in example (\ref{ex:kWGAmna}) where \ipa{kɯ-ɣɤmna} is the stative infinitive of the verb \ipa{ɣɤmna} `easy to heal, heal fast' (with the abilitative \ipa{ɣɤ-} prefix). The form \ipa{kɤ-ɣɤmna} with the \ipa{kɤ-} infinitive would also be possible, but this would be the infinitive of the homophonous transitive verb \ipa{ɣɤmna} `heal' (with the causative \ipa{ɣɤ-} prefix) and the meaning would be `it is efficient to heal (this disease)'.
 
 
 \begin{exe}
\ex \label{ex:kWGAmna}
\gll \ipa{smɤn} 	\ipa{tu-βzu-nɯ} 	\ipa{tɕe} 	\ipa{tɕe} 	\ipa{ʁo} 	\ipa{kɯ-ɣɤmna} 	\ipa{ɲɯ-pʰɤn} \\
medicine \textsc{ipfv}-make-\textsc{pl} \textsc{lnk} \textsc{lnk} \textsc{adversative}  \textsc{nmlz:S/A}-\textsc{abil}-heal \textsc{sens}-be.efficient \\
\glt `When they use medicine, on the other hand, it is efficient to (make this disease) heal faster.' (27-kharwut, 103)
\end{exe}

Not all infinitival clauses occurring with adjectives are complement clauses. In (\ref{ex:turACi}), the clause whose main verb is the negative infinitive \ipa{mɤ-kɤ-cʰa} is neither a core argument, an adjunct or a purposive clause selected by the predicate of its matrix clause \jpg{rʑi}{be heavy}. Instead, it is an infinitival manner clause (see section \ref{sec:infinitives} and \citealt{jacques14linking}).

\begin{exe}
\ex \label{ex:turACi}
\gll [\ipa{maka} 	\ipa{tu-rɤɕi} 	\ipa{mɤ-kɤ-cʰa}] 	\ipa{ʑo} 	\ipa{kɯ-rʑi} 	\ipa{pjɤ-ɕti} \\
at.all \textsc{ipfv}-pull \textsc{neg-inf}-can \textsc{emph} \textsc{nmlz}:S/A-be.heavy \textsc{ipfv.ifr}-be:\textsc{affirm} \\
\glt `It was so heavy that (the fisherman) could not pull it out (of the water).' (140512 yufu yu mogui, 35)
\end{exe}

\subsubsection{Adjectives of degree} \label{sec:degree.complement}
Adjectives of degree, like \jpg{rtaʁ}{be enough} or \jpg{tɕʰom}{be too much} are compatible with three distinct constructions: finite complement clauses (with a verb in the imperfective, as in example \ref{ex:pjWnArte}), infinitival complement clauses (as in \ref{ex:kWfsoR.kWtChom}) or, most commonly, the degree nominal complementation strategy (example \ref{ex:kotChom}, see also section \ref{sec:degree}).

\begin{exe}
\ex \label{ex:pjWnArte}
\gll \ipa{koŋla} 	\ipa{pjɯ-nɤrte} 	\ipa{ʑo} 	\ipa{kɯ-rtaʁ} 	\ipa{ʑo} 	\ipa{kɯ-wxti} 	\ipa{ɲɯ-βze} 	\ipa{ŋgrɤl}  \\
completely \textsc{ipfv}-wear.as.a.hat \textsc{emph} \textsc{nmlz}:S/A-enough \textsc{emph} \textsc{nmlz}:S/A-be.big \textsc{ipfv}-grow be.usually.the.case:\textsc{fact} \\
\glt `(Leaves of the burdock) can grow big enough to be worn as hats.' (13-tCamu, 38)
\end{exe}
 
\begin{exe}
\ex \label{ex:kWfsoR.kWtChom}
\gll [\ipa{tɤŋe} 	\ipa{kɯ-fse} 	\ipa{kɯ-fsoʁ}] 	\ipa{kɯ-tɕhom} 	\ipa{kɯ-fse} 	\ipa{nɯra} 	\ipa{ju-kɯ-ru} 	\ipa{rcanɯ} \\
sun \textsc{inf:stat}-be.like \textsc{inf:stat}-be.bright \textsc{inf:stat}-be.too.much \textsc{inf:stat}-be.like \textsc{dem:pl} \textsc{ipfv-genr}:S/P-look \textsc{unexpected} \\
\glt `(When one gets this eye disease), one looks (with the eyes half-closed) as (ones does when one's eyes are dazzled) when the sun is too bright.' (27-tApGi, 7)
\end{exe}

\begin{exe}
\ex \label{ex:kotChom}
\gll
\ipa{ma} 	\ipa{ɯ-tɯ-smi} 	\ipa{ko-tɕʰom} 	\ipa{qʰe} \\
because \textsc{3sg.poss-nmlz:degree}-be.cooked \textsc{ifr}-be.too.much \textsc{lnk} \\
\glt `Because if it cooks too much (it is not as tasty).' (Conversation 14.05.10)
\end{exe}

 
%  \begin{exe}
%\ex \label{ex:XtCoN}
%\gll
% \ipa{χtɕoŋ} 	\ipa{nɯ} 	\ipa{ɲɯ-pʰɤn} 	\ipa{ɲɯ-ti-nɯ} 	\ipa{ri}  \\
% rheumatism \textsc{dem} \textsc{sens}-be.efficient \textsc{sens}-say-\textsc{pl} but \\
% \glt `People say it is good against rheumatism, but...' (20-sWrna, 146)
%\end{exe}
% 
%Example (\ref{ex:tWmtshi}) is not an example of complement clause, as the S-participle of \jpg{mŋɤm}{hurt} is lexicalized in the sense of `disease'.
%
%  \begin{exe}
%\ex \label{ex:tWmtshi}
%\gll 
%[\ipa{tɯ-mtsʰi} 	\ipa{kɯ-mŋɤm}] 	\ipa{wuma} 	\ipa{ʑo} 	\ipa{pʰɤn} 	\ipa{tu-ti-nɯ} 	\ipa{ŋgrɤl}  \\
% \textsc{indef.poss-liver}  \textsc{nmlz}:S/A-hurt really \textsc{emph} be.efficient:\textsc{fact} \textsc{ipfv}-say-\textsc{pl} be.usually.the.case:\textsc{fact} \\
% \glt (05-qaZo 38)
%\end{exe}
 

 

\subsection{Complements of nouns and noun-verb collocations}
\ipa{mɯ-tu-kɤ-mbro} 	\ipa{ftɕaka} 	\ipa{tu-βze-a} 	\ipa{ŋu} 

 \section{Conclusion}
 
\bibliographystyle{unified}
\bibliography{bibliogj}
\end{document}