%aʑo me, nɤʑo me.

\documentclass[oldfontcommands,oneside,a4paper,11pt]{article} 
\usepackage{fontspec}
\usepackage{natbib}
\usepackage{booktabs}
\usepackage{xltxtra} 
\usepackage{longtable}
\usepackage{polyglossia} 
%\usepackage[table]{xcolor}
\usepackage{gb4e} 
\usepackage{multicol}
\usepackage{graphicx}
\usepackage{float}
\usepackage{lineno}
\usepackage{textcomp}
\usepackage{hyperref} 
\hypersetup{bookmarks=false,bookmarksnumbered,bookmarksopenlevel=5,bookmarksdepth=5,xetex,colorlinks=true,linkcolor=blue,citecolor=blue}
\usepackage[all]{hypcap}
\usepackage{memhfixc}
\usepackage{lscape}
 

\setmainfont[Mapping=tex-text,Numbers=OldStyle,Ligatures=Common]{Charis SIL} \newfontfamily\phon[Mapping=tex-text,Ligatures=Common,Scale=MatchLowercase,FakeSlant=0.3]{Charis SIL} 
\newcommand{\ipa}[1]{{\phon #1}} %API tjs en italique
 
\newcommand{\grise}[1]{\cellcolor{lightgray}\textbf{#1}}
\newfontfamily\cn[Mapping=tex-text,Ligatures=Common,Scale=MatchUppercase]{MingLiU}%pour le chinois
\newcommand{\zh}[1]{{\cn #1}}


\XeTeXlinebreaklocale "zh" %使用中文换行
\XeTeXlinebreakskip = 0pt plus 1pt %
 %CIRCG
 


\begin{document} 

\title{Complementation in Japhug Rgyalrong\footnote{
 
} }
\author{Guillaume Jacques}
\maketitle
\linenumbers
 
 \citet[9]{dixon06complementation}
 \citet{sun12complementation}
 \citet{jacques08}

  \section{Case marking mismatch} 
 
  \section{Multiple complements}
 \ipa{ɯ-zda} 	\ipa{nɯ} 	\ipa{ra} 	\ipa{pɣɤtɕɯ} 	\ipa{kɯ-ɣɤwu,} 	\ipa{khɯna} 	\ipa{kɯ-ɤndzɯt,} 	\ipa{lɯlu} 	\ipa{kɯ-ɣɤwu} 	\ipa{qachɣa} 	\ipa{kɯ-mbri} 	\ipa{kɯ-fse,} 	\ipa{nɯra} 	\ipa{tu-nɯɕpɯz} 	\ipa{ɲɯ-spe.} 	

\section{Infinitive} 

\citet{jacques16relatives}

The S of the verb \ipa{rga}  `like' can be coreferent with either the S  (\ref{ex:kAnWrAGo.rganW}), the A (\ref{ex:kAnArtoXpjAt.pWrgaa}) and even the P (\ref{ex:YWrganW}) of its infinitival or finite  complement verbs (the complement verb in \ref{ex:kAnWrAGo.rganW} is intransitive, while it is transitive in \ref{ex:kAnArtoXpjAt.pWrgaa} and \ref{ex:YWrganW}). 


 \begin{exe}
   \ex   \label{ex:kAnWrAGo.rganW} 
\gll
\ipa{tsuku}  	\ipa{tɕe}  	\ipa{kɤ-nɯrɤɣo}  	\ipa{wuma}  	\ipa{ʑo}  	\ipa{rga-nɯ}  	\ipa{tɕe}  \\
some \textsc{lnk} \textsc{inf}-sing really \textsc{emph} like:\textsc{fact-pl}  \textsc{lnk} \\
  \glt Some people like to sing. (26 kWrNukWGndZWr, 104)
     \end{exe}  
 
   \begin{exe}
   \ex   \label{ex:kAnArtoXpjAt.pWrgaa} 
\gll
  	\ipa{aʑo}  	\ipa{qajɯ}  	\ipa{nɯ} \ipa{ra}  	\ipa{kɤ-nɤrtoχpjɤt}  	\ipa{pɯ-rga-a}  	\ipa{tɕe}  	\\
  	\textsc{1sg} bugs \textsc{dem} \textsc{pl} \textsc{inf}-observe \textsc{pst.ipfv}-like-\textsc{1sg} \textsc{lnk}  \\
 \glt I liked to observe bugs. (26 quspunmbro, 15)
     \end{exe}  
 
  \begin{exe}
   \ex   \label{ex:YWrganW} 
\gll
\ipa{maka}  	\ipa{tu-kɤ-nɤjoʁjoʁ,}  	\ipa{tu-kɤ-fstɤt}  	\ipa{nɯ}  	\ipa{ɲɯ-rga-nɯ}  \\
at.all \textsc{ipfv-inf}-flatter \textsc{ipfv-inf}-praise \textsc{dem} \textsc{ipfv}-like-\textsc{pl} \\
\glt They like to be flattered or praised. (140427 yuanhou, 53)
    \end{exe}  
    
\section{Complements of stative verbs}

enough to
\ipa{koŋla} 	\ipa{pjɯ-nɤrte} 	\ipa{ʑo} 	\ipa{kɯ-rtaʁ} 	\ipa{ʑo} 	\ipa{kɯ-wxti} 	\ipa{ɲɯ-βze} 	\ipa{ŋgrɤl} 

 
 
\bibliographystyle{unified}
\bibliography{bibliogj}
\end{document}