\documentclass[oneside,a4paper,11pt]{article} 
\usepackage{fontspec}
\usepackage{natbib}
\usepackage{booktabs}
\usepackage{xltxtra} 
\usepackage{polyglossia} 
\usepackage[table]{xcolor}
\usepackage{gb4e} 
\usepackage{multicol}
\usepackage{graphicx}
\usepackage{float}
\usepackage{hyperref} 
\hypersetup{bookmarksnumbered,bookmarksopenlevel=5,bookmarksdepth=5,colorlinks=true,linkcolor=blue,citecolor=blue}
\usepackage[all]{hypcap}
\usepackage{memhfixc}
\usepackage{lscape}
\usepackage{amssymb}
 
\bibpunct[: ]{(}{)}{,}{a}{}{,}

%\setmainfont[Mapping=tex-text,Numbers=OldStyle,Ligatures=Common]{Charis SIL} 
\newfontfamily\phon[Mapping=tex-text,Scale=MatchLowercase]{Charis SIL} 
\newcommand{\ipa}[1]{\textbf{{\phon\mbox{#1}}}} %API tjs en italique
%\newcommand{\ipab}[1]{{\scriptsize \phon#1}} 

\newcommand{\grise}[1]{\cellcolor{lightgray}\textbf{#1}}
\newfontfamily\cn[Mapping=tex-text,Ligatures=Common,Scale=MatchUppercase]{SimSun}%pour le chinois
\newcommand{\zh}[1]{{\cn #1}}
\newfontfamily\mleccha[Mapping=tex-text,Ligatures=Common,Scale=MatchLowercase]{Galatia SIL}%pour le grec
\newcommand{\grec}[1]{{\mleccha #1}}


\XeTeXlinebreakskip = 0pt plus 1pt %
 %CIRCG
 
\begin{document}

\title{ST phylogeny: first overview of the results from the point of view of linguistics}
\author{Guillaume Jacques}
\maketitle

\section{Preliminary comments}
Based on the last trees by Valentin and Simon, we have a root at around 7000 BP, and if we focus exclusively on the clades with probability above 0.8, the following groups can be accepted:

\begin{enumerate}
\item Sal (Garo, Rabha, Jinghpo), \citet{burling83sal}, 0.87/ 0.96
\item Tibetan+Burmo-Qiangic (never explicitely proposed), 0.91 / 0.98
\item Burmo-Qiangic (\citealt{jacques.michaud11naish, jacques14esquisse}), 0.94 / 0.98
\item Kuki-Chin-Tangkhul (never explicitely proposed), 1/1
\item Kiranti 
\item West Himalayish
\end{enumerate}

While the results do not solve the higher phylogeny of the ST family, in particular the vexing question of the status of Chinese, they nevertheless disprove a certain number of hypotheses, as described in detail in the following (which can be re-used for the paper).
 
\section{Discussion}
While the results confirm clades whose existence are already universally accepted (Kiranti, Tibetic, Sinitic, Lolo-Burmese), they also provide clear evidence for non-trivial groupings that disprove a considerable number of proposed hypotheses on the phylogeny of the ST family.

\subsection{Sal}
The Sal subgroup, comprising Bodo-Garo, Jinghpo, Northern Naga and Sak, was proposed by \citet{burling83sal} on the basis of a certain number of lexical innovations, and is confirmed by the present results. The main opponent of Burling's view, \citet{matisoff03}, argued instead to group Bodo-Garo with most languages of North-Eastern India into a `Kamarupan' grouping (not based on actual innovations), a view that is incompatible with our results.

There is some evidence that Sal might be one of the earliest branching clades of ST alongside Sinitic, as shown by the fact that it occurs with above 0.5 probability in a group with Sinitic as opposed to all other languages of the family. 

\subsection{Burmo-Qiangic and Tibeto-Qiangic}
One of the most important finding is the confirmation of the Burmo-Qiangic branch on the one hand, and the unforeseen grouping of Tibetic with Burmo-Qiangic at a slightly higher level on the other. These two clades are highly relevant to a major debate in ST comparative linguistics concerning the antiquity of person indexation: some branches (in the sample, Kiranti, Gyalrongic, Dulong, and to a lesser extent Kuki-Chin, West Himalayish and Jinghpo) have person indexation and relatively elaborated voice morphology, while other branches (in particular Lolo-Burmese and Chinese) are close to the `isolating' type, with no person indexation and only frozen evidence for verbal derivations.

Some scholars, in particular \citet{driem93agreement, delancey10agreement, jacques12agreement, delancey14second}, argue that person indexation, in particular as attested in Southern Kiranti and Situ Gyalrong, is an archaism and goes back if not to proto-ST, at least to a higher subgroup comprising Gyalrongic, Kiranti and all more closely related languages. In this view, many languages without person indexation have simply lost it, a process that is historically attested in some branches (\citealt{delancey15complexity}).

Another line of argumentation, represented by \citet{lapolla01migration, lapolla13subgrouping}, while accepting the striking similarity shared by the verbal morphology of Kiranti and Rgyalrongic, argues that this common morphology is not a common archaism, but rather a common innovation defining a `Rung' group containing Kiranti, Dulong, Gyalrongic and West-Himalayish. In this hypothesis, languages like Lolo-Burmese without person indexation simply never had it, and are in this regard more archaic than the `Rungic' languages.

The present results show that the `Rungic' hypothesis, in all its past and present formulations, is untenable. Gyalrongic languages, with a considerable certainty (this is a result that occured in absolutely all tests performed during this research project), belong to a clade exclusively shared with Lolo-Burmese excluding Kiranti, West-Himalayish and Dulong. Since the morphologically common element shared by Gyalrongic and Kiranti are not controversial (accepted even by LaPolla), the natural consequence is that person indexation goes back at least to the common ancestor of Gyalrongic and Kiranti, which according to our results include at least Lolo-Burmese and Tibetic: these two branches, therefore, have lost the person indexation system attested by Gyalrongic and Kiranti (on possible traces of person indexation in Tibetic, see \citealt{jacques10zos, driem11TB}).

While the present results do not settle the question of the antiquity of person indexation at the higher level (in particular whether groups like Sinitic or Tani never developed it or lost it), note that simpler, but partially cognate system exist in Jinghpo and Northern Naga in the Sal group (\citealt{delancey11nocte}), Kuki-Chin and Dulong (\citealt{delancey10agreement}). By the same reason as above, it follows that at least Bodo-Garo (in the Sal group) and Ukhrul have lost person indexation.

\section{Sinitic}
The status of Sinitic is the topic of a major controversy in ST. Many scholars assume that Sinitic is the earliest branch of ST, and some specialists have proposed possible innovations common to all non-Sinitic languages (\citealt{handel08st, sagart17candidate}). Our results neither confirm nor contradict this hypothesis. However, they disprove the `Sino-Bodic' hypothesis (\citealt{driem97sinobodic}) for which potential supporting evidence had also been proposed (\citealt{hill14jrn}), and also \citet{post14th}'s vision of Sinitic as a sub-branch of a clade comprising Lolo-Burmese and Tibetan.

%post17neast 
%delancey15central


\bibliographystyle{unified}
\bibliography{bibliogj}
\end{document}
