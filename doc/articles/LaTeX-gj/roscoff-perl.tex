\documentclass[oldfontcommands,oneside,a4paper,11pt]{article} 
\usepackage{fontspec}
\usepackage{natbib}
\usepackage{booktabs}
\usepackage{xltxtra} 
\usepackage{longtable}
\usepackage{polyglossia} 
\usepackage[table]{xcolor}
\usepackage{gb4e} 
\usepackage{multicol}
\usepackage{graphicx}
\usepackage{float}
\usepackage{hyperref} 
\hypersetup{bookmarks=false,bookmarksnumbered,bookmarksopenlevel=5,bookmarksdepth=5,xetex,colorlinks=true,linkcolor=blue,citecolor=blue}
\usepackage[all]{hypcap}
\usepackage{memhfixc}
\usepackage{lscape}
 \usepackage{lineno}
\bibpunct[: ]{(}{)}{,}{a}{}{,}

\setmainfont[Mapping=tex-text,Numbers=OldStyle,Ligatures=Common]{Charis SIL} 
\newfontfamily\phon[Mapping=tex-text,Ligatures=Common,Scale=MatchLowercase,FakeSlant=0.3]{Charis SIL} 
\newcommand{\ipa}[1]{{\phon \mbox{#1}}} %API tjs en italique
\newcommand{\ipab}[1]{{\scriptsize \phon#1}} 

\newcommand{\grise}[1]{\cellcolor{lightgray}\textbf{#1}}
\newfontfamily\cn[Mapping=tex-text,Ligatures=Common,Scale=MatchUppercase]{MingLiU}%pour le chinois
\newcommand{\zh}[1]{{\cn #1}}

\newcommand{\sg}{\textsc{sg}}
\newcommand{\pl}{\textsc{pl}}
\newcommand{\ro}{$\Sigma$}
\newcommand{\ra}{$\Sigma_1$} 
\newcommand{\rc}{$\Sigma_3$}  


\XeTeXlinebreakskip = 0pt plus 1pt %
 %CIRCG
 


\begin{document} 

\title{Paradigm generation}
\author{Guillaume Jacques}
\maketitle
\section{Introduction}
\begin{enumerate}
\item PERL
\item Python
\item XFST (\citealt{bessley03fsm})
\end{enumerate}

Implementations: \citet{jacques12khaling}, \citet{walther14compactness}



\section{Perl regular expressions}


\subsection{Header}
\begin{verbatim}
   use strict;
   use warnings;
   use utf8;
\end{verbatim}

\subsection{Variables}

\begin{verbatim}

   my $variable1 = "bonjour";
   my $variable2 = substr($variable1,1,4);
   print $variable2."\n";

\end{verbatim}

\subsection{Special characters}
\begin{verbatim}
   \n Newline
   \r Carriage return
   \t	tab 
\end{verbatim}

\subsection{Regular expressions}
\subsubsection{Bind $=\sim$ and  $!\sim$}
\begin{verbatim}
   if ($variable =~  /on/)  { 
      ....
      }
 \end{verbatim} %$
 
Enhanced regular expressions:
 
 \begin{verbatim}
    if ($variable1 =~  /(.{3,10})\1/) {
        print $1."\n";
        }
\end{verbatim} %$



\subsubsection{Wildcards}
\begin{verbatim}
    	.   Match any character
    	\w  Match "word" character (alphanumeric plus "_")
    	\W  Match non-word character
    	\s  Match whitespace character
    	\S  Match non-whitespace character
    	\d  Match digit character
    	\D  Match non-digit character
    	*  Match 0 or more times
    	+  Match 1 or more times
    	?  Match 1 or 0 times
    	*?  Greedy matching
    	{n}    Match exactly n times
    	{n,}   Match at least n times
    	{n,m}  Match at least n but not more than m times
    	/^xxxx/	String beginning with xxxx
    	/xxxx$/  String ending in xxxx
    	[^abc] Any character except a, b or c
\end{verbatim} %$

\subsubsection{Pattern Matching}
\begin{verbatim}
    my  $variable = "bonjour tout le monde!!";
    $variable =~  m/(o.*?u)/;
    print $1."\n";
\end{verbatim} %$


\subsubsection{Substitution}
\begin{verbatim}
    $variable =~  s/xx/yy/;
\end{verbatim} %$

\subsubsection{Translation}
\begin{verbatim}
    $variable =~  tr/[a-z]/[A-z]/;
\end{verbatim} %$
\section{Paradigm generation}
\begin{enumerate}
\item Easier to directly generate \LaTeX{} code.
\item Generate stems from the underlying forms stored in the dictionary by applying a series of regular expressions.
\item Concatenate the stems with prefixes and suffixes.
\end{enumerate}

\bibliographystyle{linquiry2}
\bibliography{bibliogj}
\end{document}