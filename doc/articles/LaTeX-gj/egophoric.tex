\documentclass[oldfontcommands,oneside,a4paper,11pt]{article} 
\usepackage{fontspec}
\usepackage{natbib}
\usepackage{booktabs}
\usepackage{xltxtra} 
\usepackage{polyglossia} 
\usepackage[table]{xcolor}
\usepackage{gb4e} 
\usepackage{multicol}
\usepackage{graphicx}
\usepackage{float}
\usepackage{hyperref} 
\usepackage{lineno}
\hypersetup{bookmarks=false,bookmarksnumbered,bookmarksopenlevel=5,bookmarksdepth=5,xetex,colorlinks=true,linkcolor=blue,citecolor=blue}
\usepackage[all]{hypcap}
\usepackage{memhfixc}
\usepackage{lscape}

\bibpunct[: ]{(}{)}{,}{a}{}{,}

%\setmainfont[Mapping=tex-text,Numbers=OldStyle,Ligatures=Common]{Charis SIL} 
\newfontfamily\phon[Mapping=tex-text,Ligatures=Common,Scale=MatchLowercase]{Charis SIL} 
\newcommand{\ipa}[1]{{\phon\textit{#1}}} %API tjs en italique
\newcommand{\ipab}[1]{{\scriptsize \phon#1}} 

\newcommand{\grise}[1]{\cellcolor{lightgray}\textbf{#1}}
\newfontfamily\cn[Mapping=tex-text,Ligatures=Common,Scale=MatchUppercase]{SimSun}%pour le chinois
\newcommand{\zh}[1]{{\cn #1}}
\newcommand{\refb}[1]{(\ref{#1})}
\newcommand{\factual}[1]{\textsc{:fact}}
\newcommand{\rdp}{\textasciitilde{}}

\XeTeXlinebreaklocale 'zh' %使用中文换行
\XeTeXlinebreakskip = 0pt plus 1pt %
 %CIRCG
 \newcommand{\bleu}[1]{{\color{blue}#1}}
\newcommand{\rouge}[1]{{\color{red}#1}} 
\newcommand{\ro}{$\Sigma$}

\begin{document} 
\title{Egophoricity and Person Indexation in Japhug\footnote{ The glosses follow the Leipzig glossing rules. Other abbreviations used here are: \textsc{auto}  autobenefactive-spontaneous, \textsc{anticaus} anticausative, \textsc{antipass} antipassive, \textsc{appl} applicative, \textsc{dem} demonstrative,  \textsc{dist} distal, \textsc{emph} emphatic, \textsc{fact} factual, \textsc{genr} generic, \textsc{ifr} inferential, \textsc{indef} indefinite, \textsc{inv} inverse,  \textsc{lnk} linker, \textsc{pfv} perfective, \textsc{poss} possessor, \textsc{pres} egophoric present, \textsc{prog} progressive, \textsc{sens} sensory. The examples are taken from a corpus that is progressively being made available on the Pangloss archive  (\citealt{michailovsky14pangloss}, 
 \url{http://lacito.vjf.cnrs.fr/pangloss/corpus/list\textunderscore rsc.php?lg=Japhug}). This research was funded by the HimalCo project (ANR-12-CORP-0006) and is related to the research strand LR-4.11 ‘‘Automatic Paradigm Generation and Language Description’’ of the Labex EFL (funded by the ANR/CGI). Acknowledgements   will be added after editorial decision. %Marc Bavant, Lauren Gawne, Simeon Floyd, Nathan W. Hill, Theo Lap, Holger Markgraf, Alexis Michaud, Nicolas Tournadre
} }
\author{Guillaume Jacques}
\maketitle
%\linenumbers

\section*{Introduction}
Evidential systems with an egophoric / heterophoric contrast have been first described in several languages of the Himalayan area, in particular Newar and Tibetan under the label `conjunct/disjunct' (see \citealt{hale80conjunct}, \citealt{delancey92conjunct}). This phenomenon has then been documented in various areas including South America, the Caucasus and Highland New Guinea (in particular \citealt{creissels08akhvakh}, \citealt{curnow02conjunct}, \citealt{sanroque12evidentiality}, \citealt{sanroque15interrogativity}). 

Early research has tended to describe `conjunct/disjunct' systems as a form of person agreement (see the survey of the literature in \citealt{hill17evidential}), in which the `conjunct' form is used to mark first person in declarative clauses, the second person in interrogative ones and `same subject' in complement clauses, while the `disjunct' form is used to mark second person in declarative clauses, first/third person in interrogative ones and non-same subject in complement clauses (see Table \ref{tab:conjunct}).

\begin{table}[H]
\caption{The conjunct / disjunct model} \label{tab:conjunct} \centering
\begin{tabular}{lllllll}
\toprule
& Conjunct & Disjunct \\
\midrule
Declarative & 1st person & 2/3nd person \\
Interrogative & 2st person & 1/3nd person \\
Complement clause & same subject & distinct subject \\
\bottomrule
\end{tabular}
\end{table}

More recent research, in particular \citet{sun93evidentiality} and \citet{tournadre08conjunct} on Tibetan, has shown that `conjunct/disjunct' systems are better analyzed as an evidential system, only indirectly linked to person marking. Tournadre proposed the term `egophoric' to replace `conjunct', a misleading label based on the use of this form to mark `same subject' in complement clauses, a phenomenon better analyzed in terms of Hybrid Indirect Speech (see section \ref{sec:hybrid}). Moreover, Tournadre showed that the evidential contrast is ternary rather than binary in Tibetan.

Yet, very few languages have both person indexation and egophoric marking; none of the languages included in the forthcoming volume on egophoricity (\citealt{norcliffe17egophoricity}) have person indexation, and in the Sino-Tibetan family, while languages with egophoric marking such as Newar, Pumi (\citealt{daudey14volition}) and Bunan (\citealt{widmer17epistemization}) have remnants of person indexation completely or partially reanalyzed as evidential categories, the only language group where both a fully fledged person indexation system and an evidential system containing an egophoric category are both present is the Gyalrong branch of Sino-Tibetan, comprising Situ, Japhug, Tshobdun and Zbu (\citealt{sun18evidentials}). While previous work has partially described the use of evidential categories in Gyalrong languages (see in particular \citealt{youjing03zhuokeji}, \citealt{jackson03caodeng}, \citealt[617-620]{jacques17sketch}), much descriptive work is still needed before these languages can be profitably used by typologists working on evidentiality.

TAME systems in Gyalrong languages are highly complex, comprising more than ten basic TAME forms, augmented by periphrastic TAME categories and secondary affixes. A satisfying description of the TAME of any such language therefore requires a book-length monograph. The present study is of more limited scope: studying the ternary evidential contrast in the present imperfective of stative verbs. This choice is motivated by three reasons.

First, the ternary evidential contrast between factual, sensory and egophoric only exists in the present, since the egophoric marker is incompatible with past and future tenses (\ref{sec:morph}). Second, stative verbs have less TAME distinctions than dynamic verbs. Third, stative verbs, having only one argument, present less complex interaction between person and evidentiality than transitive verbs.

This paper studies the ternary evidential contrasts in the three main constructions relevant to the topic at hand: declarative clauses, interrogative clauses and reported speech. It systematically discusses the semantic differences between the three categories and their relationship with person marking. It also discusses the historical origin of sensory and egophoric prefixes in Japhug and other Gyalrong languages.


\section{Morphological categories} \label{sec:morph}

\subsection{The ternary system}
Stative verbs have only three distinct forms in the present: the Factual Non-Past, the Sensory Imperfective and the Egophoric Imperfective Present, henceforth referred to as Factual, Sensory and Egophoric. While the three forms require stem alternation in the case of transitive verbs with singular subject and third person object (see \citealt{jackson00sidaba}, \citealt[267]{jacques14linking}), no stem alternation occurs with stative verbs. Therefore, these forms are only marked by affixation for this category of verbs.

The Sensory form is built by combining the stem with the  prefix \ipa{ɲɯ-} (in the negative \ipa{mɯ́j-}), the Egophoric with the prefix \ipa{ku-} (its negative form is \ipa{mɯ-ku-}) and the Factual has no prefix, and consists of the bare stem (its negative form is marked by the prefix \ipa{mɤ-}), as indicated in Table \ref{tab:three}.

The existential verbs \ipa{tu} `exist' and \ipa{me} `not exist' have suppletive Sensory forms \ipa{ɣɤʑu} `exist.\textsc{sens}' and \ipa{maŋe} `not exist.\textsc{sens}'. The suppletive verbs have irregular second person forms (\ipa{ɣɤtɤʑu} and \ipa{mataŋe} respectively, see \citealt{jacques12agreement}).

\begin{table}[H]
\caption{The three present evidential forms of stative verbs in Japhug} \label{tab:three} \centering
\begin{tabular}{llllll}
\toprule
Form & Regular stative verb & Existential verbs \\
&(\ipa{pe} `be good') & (\ipa{tu} `exist') \\
\midrule
Factual & \ipa{pe} & \ipa{tu} \\
Factual, negative & \ipa{mɤ-pe} & \ipa{me} \\
Sensory & \ipa{ɲɯ-pe} & \ipa{ɣɤʑu} \\
Sensory, negative & \ipa{mɯ́j-pe} & \ipa{maŋe} \\
Egophoric &  \ipa{ku-pe} & \ipa{ku-tu} \\
Egophoric, negative & \ipa{mɯ-ku-pe} & \ipa{ku-me} \\
\midrule
\end{tabular}
\end{table}

A fourth category is also possible in the present tense, the Imperfective, but it has an inchoative meaning, and turns a stative verb into a dynamic one; it will therefore not be considered in this paper.

The Sensory and the Factual are not restricted to present tense. Sensory is also used past tense imperfective, and the Factual in future tense, with various aspectual meanings. These uses will not concern us in the present paper.


\subsection{Person Indexation}
Stative verbs are a subclass of intransitive verbs, and can only index one argument, the intransitive subject (S), following the paradigm in Table \ref{tab:indexation}.


\begin{table}[H]
\caption{Intransitive person indexation paradigm} \label{tab:indexation} \centering
\begin{tabular}{llllll}
\toprule 
\textsc{1sg} & \ro{}\ipa{-a} \\
\textsc{1du} & \ro{}\ipa{-tɕi} \\
\textsc{1pl} & \ro{}\ipa{-j} \\
\midrule
\textsc{2sg} & \ipa{tɯ-}\ro{} \\
\textsc{2du} & \ipa{tɯ-}\ro{}\ipa{-ndʑi} \\
\textsc{2pl} & \ipa{tɯ-}\ro{}\ipa{-nɯ} \\
\midrule
\textsc{3sg} & \ro{} \\
\textsc{3du} & \ro{}\ipa{-ndʑi} \\
\textsc{3pl} & \ro{}\ipa{-nɯ} \\
\bottomrule
\end{tabular}
\end{table}

\citet[275]{jacques16complementation} \ipa{mkhɤz}

\begin{exe}
\ex \label{ex:CoNBzu.mkhAz}
\gll 
\ipa{ɯ-nmaʁ} 	\ipa{jɤ-kɯ-ɣe} 	\ipa{nɯ} 	\ipa{ɕoŋβzu} 	\ipa{mkʰɤz} 	\ipa{tɕe} \\
\textsc{3sg.poss}-husband \textsc{pfv-nmlz}:S/A-come[II] \textsc{dem} carpentry be.expert:\textsc{fact} \textsc{lnk} \\
\glt `Her husband (who came to live in her family) is very good at carpentry.' (14-tApitaRi, 273)
\end{exe}
\section{Declarative clauses}

\citealt{hill17evidential} ‘personal’, ‘experiential’, and ‘factual’

nɤ-tɕɯ tɤ-sci, aʑo ɲɯ-rga-a
phone conversation, 2013

\section{Interrogative clauses}

"wo, ɯ-ku-tɯ-pe, ɯ-ɲɯ-tɯ-cha" nɯra ntsɯ to-ti.
140425 shizi huli he lu, 16)

\section{Reported Speech} \label{sec:hybrid}

nɤ-rʑaβ nɯ kɯ-xtɕi nɯnɯ tɤ-ngo tɕe nɯ-si nɤ-tɕɯ nɯnɯ, kɯ nɯnɯ ɯ-mu kɤ-nɯzdɯɣ kɯ, tɕe ɲɯ-nɯzdɯɣ tɕe kha na-βde tɕe /ja/ jɤ-ari ɕti tɕe, ŋotɕu nɯ-ari mɤxsi ma a-tɕha maka ku-me" to-ti.
101-103
fushang he yaomo

"nɤ-rʑaβ kɯ-xtɕi nɯnɯ ɯ-kɯ-mŋɤm kɤ-ndzoʁ tɕe nɯ-si,
nɤ-tɕɯ nɯnɯ kɯ nɯnɯ ɯ-mu kɤ-nɯzdɯɣ kɯ tɕe kha na-βde tɕe, jɤ-a<nɯ>ri ɕti tɕe, maka ɯ-tɕha ku-me ɲɯ-ti ɕti.
114-116
fushang he yaomo1
妻子说小妾染病身亡,儿子悲痛余竟离家出走,一直没有消息

"aʑo ɲɯ-khe" a-mɤ-nɯ-sɯso-nɯ 
huangdi de xinzhuang

\section{Historical origin}

\section*{Conclusion}

\bibliographystyle{unified}
\bibliography{bibliogj}

\end{document}