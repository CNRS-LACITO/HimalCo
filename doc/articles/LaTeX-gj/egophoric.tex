\documentclass[11pt]{article} 
\usepackage{fontspec}
\usepackage{natbib}
\usepackage{booktabs}
\usepackage{xltxtra} 
\usepackage{polyglossia} 
\usepackage[table]{xcolor}
\usepackage{gb4e} 
\usepackage{multicol}
\usepackage{graphicx}
\usepackage{float}
\usepackage{hyperref} 
\usepackage{lineno}
\hypersetup{bookmarks=false,bookmarksnumbered,bookmarksopenlevel=5,bookmarksdepth=5,xetex,colorlinks=true,linkcolor=blue,citecolor=blue}
\usepackage[all]{hypcap}
\usepackage{memhfixc}
%\usepackage{lscape}

%\bibpunct[: ]{(}{)}{,}{a}{}{,}

%\setmainfont[Mapping=tex-text,Numbers=OldStyle,Ligatures=Common]{Charis SIL} 
\newfontfamily\phon[Mapping=tex-text,Ligatures=Common,Scale=MatchLowercase]{Charis SIL} 
\newcommand{\ipa}[1]{{\phon\textit{#1}}} %API tjs en italique
\newcommand{\ipab}[1]{{\scriptsize \phon#1}} 

\newcommand{\grise}[1]{\cellcolor{lightgray}\textbf{#1}}
\newfontfamily\cn[Mapping=tex-text,Ligatures=Common,Scale=MatchUppercase]{SimSun}%pour le chinois
\newcommand{\zh}[1]{{\cn #1}}
\newcommand{\refb}[1]{(\ref{#1})}
\newcommand{\factual}[1]{\textsc{:fact}}
\newcommand{\rdp}{\textasciitilde{}}

\XeTeXlinebreaklocale 'zh' %使用中文换行
\XeTeXlinebreakskip = 0pt plus 1pt %
 %CIRCG
 \newcommand{\bleu}[1]{{\color{blue}#1}}
\newcommand{\rouge}[1]{{\color{red}#1}} 
\newcommand{\ro}{$\Sigma$}

\begin{document} 
\title{Egophoric marking and Person Indexation in Japhug\footnote{ The glosses follow the Leipzig glossing rules. Other abbreviations used here are: \textsc{auto}  autobenefactive-spontaneous, \textsc{anticaus} anticausative, \textsc{antipass} antipassive, \textsc{appl} applicative, \textsc{dem} demonstrative,  \textsc{dist} distal, \textsc{emph} emphatic, \textsc{fact} factual, \textsc{genr} generic, \textsc{ifr} inferential, \textsc{indef} indefinite, \textsc{inv} inverse,  \textsc{lnk} linker, \textsc{pfv} perfective, \textsc{poss} possessor, \textsc{pres} egophoric present, \textsc{prog} progressive, \textsc{sens} sensory. The text examples are taken from a corpus that is progressively being made available on the Pangloss archive (\citealt{michailovsky14pangloss}, 
 \url{http://lacito.vjf.cnrs.fr/pangloss/corpus/list\textunderscore rsc.php?lg=Japhug}). This research was funded by the HimalCo project (ANR-12-CORP-0006) and is related to the research strand LR-4.11 ‘‘Automatic Paradigm Generation and Language Description’’ of the Labex EFL (funded by the ANR/CGI). I would like to thank Marc Bavant, Lauren Gawne, Simeon Floyd, Nathan W. Hill, Theo Lap, Holger Markgraf, Alexis Michaud, Nicolas Tournadre and two anonymous reviewers for useful comments on previous versions of this work.
} }
%\author{Guillaume Jacques}  
\maketitle
%\linenumbers

\textbf{Abstract}: Japhug, like other Gyalrong languages, is one of the very few languages with both a full-fledged person indexation system and an egophoric evidential category. A detailed account of the uses and meanings of the Egophoric and its interaction with person is thus of interest to the typology of evidential systems.

This paper describes the uses of Egophoric marking in Japhug and of the two other evidential categories with which it contrasts (Factual and Sensory), as well as their interaction with person indexation. Due to the limited distribution of the Egophoric in Japhug (it only occurs in present contexts), the present paper exclusively focuses on the uses of evidentials with stative verbs in present (imperfective) contexts, where minimal pairs are available in the corpus.

\textbf{Keywords}: Egophoric, Factual, Sensory, Mirative, Hybrid Indirect Speech, Conjunct/Disjunct, Person indexation

\sloppy
\section*{Introduction}
The interaction between Evidentiality and Person is a well-establish phenomenon (\citealt[217-238]{aikhenvald06}, \citealt{aikhenvald15evd}, \citealt{sun18evidentials}), and Egophoric marking, a phenomenon documented in the Himalayas, South America, the Caucasus and Highland New Guinea\footnote{Recent references include \citet{tournadre08conjunct}, \citet{hill17evidential}, \citet{delancey18tibetic}, \citet{creissels08akhvakh}, \citet{curnow02conjunct}, \citet{sanroque12evidentiality} and \citet{sanroque17interrogativity}; earlier work such as \citet{yukawa71rinkaku} and \citet{bendix74newari} had correctly described the phenomena before the term `Egophoric' was coined by \citealt{tournadre96erg}.  A distinct research tradition refers to the contrast between Egophoric and other evidential categories as `conjunct / disjunct' (\citealt{hale80conjunct}, \citealt{delancey90erg}).}  is one of the most person-sensitive evidential categories. 

%Despite early research correctly describing the evidential systems of Newar and Lhasa Tibetan (\citealt{yukawa71rinkaku}, \citealt{bendix74newari}), the mainstream `conjunct/disjunct' approach (see \citealt{hale80conjunct}, \citealt{delancey92conjunct}) has analyzed the Newar and Tibetan systems as a form of person agreement (see the survey of the literature in \citealt{hill17evidential}), in which the `conjunct' form is used to mark first person in declarative clauses, the second person in interrogative ones and `same subject' in complement clauses, while the `disjunct' form is used to mark second person in declarative clauses, first/third person in interrogative ones and non-same subject in complement clauses (see Table \ref{tab:conjunct}).
%
%\begin{table}[H]
%\caption{The conjunct / disjunct model} \label{tab:conjunct} \centering
%\begin{tabular}{lllllll}
%\toprule
%& Conjunct & Disjunct \\
%\midrule
%Declarative & 1st person & 2/3nd person \\
%Interrogative & 2st person & 1/3nd person \\
%Complement clause & same subject & distinct subject \\
%\bottomrule
%\end{tabular}
%\end{table}
%
%Other researchers, including \citet{yukawa71rinkaku},  and more recently \citet{tournadre08conjunct} have shown that the Tibetan evidential system is only indirectly linked to person marking. Tournadre proposed the term `egophoric' to replace `conjunct', a misleading label based on the use of this form to mark `same subject' in complement clauses, a phenomenon better analyzed in terms of Hybrid Indirect Speech (see section \ref{sec:hybrid}). Moreover, Tournadre argued, like \citet{yukawa71rinkaku}, that the evidential contrast is ternary rather than binary in Tibetan. The Yukawa-Tournadre model is becoming the mainstream view among Tibetologists (\citealt{hill17evidential, delancey18tibetic}).

Very few languages have both person indexation and egophoric marking; none of those included in the forthcoming volume on egophoricity (\citealt{norcliffe17egophoricity}) have person indexation, and in the Sino-Tibetan family, while languages with egophoric marking such as Newar, Pumi (\citealt{daudey14volition}) and Bunan (\citealt{widmer17epistemization}) have remnants of person indexation completely or partially reanalyzed as evidential categories, the only language group where both a fully fledged person indexation system and an evidential system containing an egophoric category are both present is the Gyalrong branch of Sino-Tibetan, comprising Situ, Japhug, Tshobdun and Zbu (\citealt{sun18evidentials}). While previous work has partially described the use of evidential categories in Gyalrong languages (see in particular \citealt{youjing03zhuokeji}, \citealt{jackson03caodeng}, \citealt[617-620]{jacques17sketch}), much descriptive work is still needed before these languages can be profitably used by typologists working on evidentiality.

TAME systems in Gyalrong languages are highly complex, comprising more than ten basic TAME forms, augmented by periphrastic TAME categories and secondary affixes. A satisfying description of the TAME of any such language therefore requires a book-length monograph. The present study is of more limited scope: studying the tripartite evidential contrast in the present imperfective of stative verbs. This choice is motivated by three reasons.

First, the tripartite evidential contrast between Factual, Sensory and Egophoric only exists in the present, since the Egophoric marker is incompatible with past and future tenses (see section \ref{sec:morph}). Second, stative verbs have fewer TAME distinctions than dynamic verbs. Third, stative verbs, having only one core argument, present fewer interactions between person and evidentiality than transitive verbs.

By focusing on such a restricted topic, we isolate the evidential contrast, and study minimal pairs which have exactly the same tense, aspect and modality parameters, to avoid any possible interference which could make the analysis of the semantic contrasts more difficult.

This paper studies the tripartite evidential contrasts in the three main constructions relevant to the topic at hand: declarative clauses, interrogative clauses and reported speech. It systematically discusses the semantic differences between the three categories and their relationship with person marking. %It also discusses the historical origin of sensory and egophoric prefixes in Japhug and other Gyalrong languages.

As evidential markers require a very clear context, elicited examples have been avoided in the present paper, which contains data coming either from narratives or from conversations.\footnote{ The examples are taken from a corpus that is progressively being made available on the Pangloss archive  (\citealt{michailovsky14pangloss}, 
 \url{http://lacito.vjf.cnrs.fr/pangloss/corpus/list\textunderscore rsc.php?lg=Japhug}). Some examples are taken from stories translated from Chinese (systematically identified by the label ``translation" before the reference of the story), but have been rechecked thoroughly and no example suspect of containing a calque from Chinese has been included. Note that since Mandarin has neither evidential marking nor person indexation, calquing would have little direct interference with the topic at hand in any case.}

\section{Morphological categories} \label{sec:morph}
This section describes the morphology of evidential categories in the present tense in Japhug, as well as person indexation. The meaning of these categories is discussed in sections \ref{sec:declarative} and \ref{sec:interrogative}.

\subsection{The tripartite system}
Stative verbs have only three distinct forms in the present imperfective: the Factual Non-Past, the Sensory Imperfective and the Egophoric Imperfective Present, henceforth referred to as \textsc{Factual}, \textsc{Sensory} and \textsc{Egophoric}. Although the three forms require stem alternation in the case of transitive verbs with singular subject and third person object (see \citealt{jackson00sidaba}, \citealt[267]{jacques14linking}), no stem alternation occurs with stative verbs. Therefore, these forms are only marked by affixation for this category of verbs.

The Sensory form is built by combining the stem with the  prefix \ipa{ɲɯ-} (in the negative \ipa{mɯ́j-}), the Egophoric with the prefix \ipa{ku-} (its negative form is \ipa{mɯ-ku-}) and the Factual has no prefix, and consists of the bare stem (its negative form is marked by the prefix \ipa{mɤ-}), as indicated in Table \ref{tab:three}. Some verbs also form their Imperfective with the prefixes \ipa{ɲɯ-} or \ipa{ku-}, and have thus syncretisms (for instance, the Imperfective \textsc{3sg} of \ipa{rga} `be happy' and \ipa{ŋgɤr} `be narrow' are \ipa{ɲɯ-rga} and \ipa{ku-ŋgɤr} respectively, and identical with the corresponding Sensory and Egophoric forms respectively).

The existential verbs \ipa{tu} `exist' and \ipa{me} `not exist' have suppletive Sensory forms \ipa{ɣɤʑu} `exist.\textsc{sens}' and \ipa{maŋe} `not exist.\textsc{sens}'.\footnote{Thus, despites the fact that \ipa{me} forms its Imperfective with the prefix \ipa{ɲɯ-}, there is no ambiguity with the Sensory.} The suppletive verbs have irregular second person forms (\ipa{ɣɤtɤʑu} and \ipa{mataŋe} respectively, see \citealt{jacques12agreement}).

\begin{table}[H]
\caption{The three present evidential forms of stative verbs in Japhug} \label{tab:three} \centering
\begin{tabular}{llllll}
\toprule
Form & Regular stative verb & Existential verbs \\
&(\ipa{pe} `be good') & (\ipa{tu} `exist') \\
\midrule
Factual & \ipa{pe} & \ipa{tu} \\
Factual, negative & \ipa{mɤ-pe} & \ipa{me} \\
Sensory & \ipa{ɲɯ-pe} & \ipa{ɣɤʑu} \\
Sensory, negative & \ipa{mɯ́j-pe} & \ipa{maŋe} \\
Egophoric &  \ipa{ku-pe} & \ipa{ku-tu} \\
Egophoric, negative & \ipa{mɯ-ku-pe} & \ipa{ku-me} \\
\midrule
\end{tabular}
\end{table}

A fourth category is also possible in the present tense, the Imperfective, but it has an inchoative meaning, and turns a stative verb into a dynamic one (for instance the imperfective \ipa{tu-pe} \textsc{ipfv}-be.good means `it becomes better/good); it will therefore not be considered in this paper.\footnote{This is actually, together with the fact of having an infinitive in \ipa{kɯ-} instead of \ipa{kɤ-} (\citealt[227]{jacques16complementation}), one of the defining properties of stative verbs in Japhug.}

The Sensory and the Factual are not restricted to present tense. Sensory is also used in past tense imperfective, and the Factual in future tense, with various aspectual meanings. These uses will not concern us in the present paper.

In addition to verbal morphology, evidentiality and epistemic modality are partially marked by sentence final particles such as \ipa{kʰi} `hearsay' and \ipa{tʰaŋ} `probably'. The interaction between these particles and the three-way evidential marking is deferred to further research.

\subsection{Person Indexation}
Stative verbs are a subclass of intransitive verbs, and can only index one argument, the intransitive subject (S), following the paradigm in Table \ref{tab:indexation} (the symbol \ro{} represents the verb stem). Third person singular is zero-marked. The possessive prefixes, found on nouns and on some non-finite verb forms, are also indicated for comparison.

\begin{table}[H]
\caption{Intransitive person indexation and possessive paradigms} \label{tab:indexation} \centering
\begin{tabular}{llllll}
\toprule 
Person & Indexation affixes & Possessive prefixes \\
\midrule
\textsc{1sg} & \ro{}\ipa{-a} & \ipa{a-} \\
\textsc{1du} & \ro{}\ipa{-tɕi} & \ipa{tɕi-} \\
\textsc{1pl} & \ro{}\ipa{-j} & \ipa{ji-} \\
\midrule
\textsc{2sg} & \ipa{tɯ-}\ro{} & \ipa{nɤ-} \\
\textsc{2du} & \ipa{tɯ-}\ro{}\ipa{-ndʑi} &\ipa{ndʑi-}\\
\textsc{2pl} & \ipa{tɯ-}\ro{}\ipa{-nɯ} & \ipa{nɯ-}\\
\midrule
\textsc{3sg} & \ro{} & \ipa{ɯ-} \\
\textsc{3du} & \ro{}\ipa{-ndʑi} & \ipa{ndʑi-}\\
\textsc{3pl} & \ro{}\ipa{-nɯ} &\ipa{nɯ-}  \\
\bottomrule
\end{tabular}
\end{table}

Although stative verbs can only index one argument, some of them are semi-transitive and can take a second absolutive argument (noun phrase or complement clause), for instance \ipa{mkhɤz} `be expert' (\citealt[275]{jacques16complementation}):

\begin{exe}
\ex \label{ex:CoNBzu.mkhAz}
\gll 
\ipa{ɯ-nmaʁ} 	\ipa{jɤ-kɯ-ɣe} 	\ipa{nɯ} 	\ipa{ɕoŋβzu} 	\ipa{mkʰɤz} 	\ipa{tɕe} \\
\textsc{3sg.poss}-husband \textsc{pfv-nmlz}:S/A-come[II] \textsc{dem} carpentry be.expert:\textsc{fact} \textsc{lnk} \\
\glt `Her husband (who came to live in her family) is very good at carpentry.' (14-tApitaRi, 273)
\end{exe}

Semi-transitive stative verbs are however very few, and the second absolutive argument is nearly always third person. Exceptions like \refb{ex:WYWfsea} with a second person additional argument are rare and are not considered in this paper. For the purpose of this study, only the person of the argument indexed on the verb will be taken into account.


\begin{exe}
\ex \label{ex:WYWfsea}
\gll \ipa{a-ʁi,} 	\ipa{nɤʑo} 	\ipa{ɯ-ɲɯ́-fse-a} \\
\textsc{1sg.poss}-younger.sibling \textsc{2sg} \textsc{qu-sens}-be.like-\textsc{1sg} \\
\glt `Sister, do I look like you?' (2003kongzong, 293)
\end{exe}

\section{Declarative clauses} \label{sec:declarative}


\subsection{Factual}

Used in the present with stative verbs, the Factual expresses a fact regarded as true by the speaker or belonging to generally accepted knowledge. 

It is compatible with all persons, including \textsc{1sg} (examples \ref{ex:NgWa} and \ref{ex:CqraRa}, with the suffix \ipa{-a}), \textsc{2sg} (example \ref{ex:tWmkhAz}, with the prefix \ipa{tɯ-}) and \textsc{3sg} (example \ref{ex:sAjndAt}, no affix).

With the first person, the Factual can be used to tell something about oneself that the addressee may not know, but which all persons familiar with the speaker are aware of, as in \refb{ex:NgWa}. In example \refb{ex:CqraRa}, the Factual is appropriate to express the overconfidence of the speaker in his abilities, which he believes to be obvious and well-known.

\begin{exe}
\ex \label{ex:NgWa}
\gll \ipa{aʑo} 	\ipa{nɯra} 	\ipa{fse-a} 	\ipa{tɕe} 	\ipa{ŋgɯ-a} 	\ipa{tɕe,} \\
\textsc{1sg} \textsc{dem:pl} be.like:\textsc{fact-1sg} \textsc{lnk} be.poor:\textsc{fact-1sg} \textsc{lnk} \\
\glt `I am poor like that.' (translation, 150824 kelaosi, 55)
\end{exe}

\begin{exe}
\ex \label{ex:CqraRa}
\gll
\ipa{aʑo} 	\ipa{kɯnɤ} 	\ipa{wuma} 	\ipa{ʑo} 	\ipa{ɕqraʁ-a} 	\ipa{tɕe,} 	\ipa{a-kɯ-nɯβlu} 	\ipa{kɯ-cʰa} \ipa{me,}  \\
\textsc{1sg} too really \textsc{emph} be.smart:\textsc{fact-1sg} \textsc{lnk} \textsc{1sg}-\textsc{nmlz}:S/A-cheat \textsc{nmlz}:S/A-can not.exist:\textsc{fact} \\
\glt `I am very smart too, nobody can cheat me.' (translation, 150830 afanti, 120)
\end{exe}

In \refb{ex:tWmkhAz}, the Factual occurs with a verb in the second person to state a fact about the addressee considered to be obviously true by the speaker:

\begin{exe}
\ex \label{ex:tWmkhAz}
\gll \ipa{nɤʑo} 	\ipa{stu} 	\ipa{ʑo} 	\ipa{tɯ-mkʰɤz} 	\ipa{tɕe,} 	\ipa{tɕe} 	\ipa{nɤʑo} 	\ipa{ɕ-tɤ-nɤme} \\
\textsc{2sg} most \textsc{emph} 2-be.expert:\textsc{fact}   \textsc{lnk} \textsc{lnk} \textsc{2sg} \textsc{transloc-imp}-do[III] \\
\glt `You are the best at it, do it!' (translation, 150822 laoye zuoshi zongshi duide, 37)
\end{exe}

In \refb{ex:sAjndAt}, the Factual is used with two adjectival verbs to describe facts about the swallow that the speaker is relatively confident in and consider to be generally well-known. This use is in contrast with that of the Sensory to report facts which the speaker has less confidence in, for instance concerning animals he/she has never seen (see section \ref{sec:sensory}).

\begin{exe}
\ex \label{ex:sAjndAt}
\gll	\ipa{wuma} 	\ipa{ʑo} 	\ipa{pe} 	\ipa{tɕe,} 	\ipa{sɤjndɤt} 	\ipa{tɕe,} \\
really  \textsc{emph} be.good:\textsc{fact} \textsc{lnk}  be.cute:\textsc{fact} \textsc{lnk} \\
\glt `(The swallow) is very nice, it is cute.' (03-mWrmWmbjW, 6)
\end{exe}


\subsection{Sensory} \label{sec:sensory}
The Sensory is used to express access to information through any of the senses, most commonly vision, but also hearing (\ref{ex:tumbri}), touch (\ref{ex:YWmpW}), smell (\ref{ex:tunAmnAmnW}) and taste (\ref{ex:YWmWm}). It implies the discovery of a previously unknown fact or confirmation of an uncertain fact.

\begin{exe}
\ex \label{ex:tumbri}
\gll
\ipa{tu-mbri} 	\ipa{tɕe} 	\ipa{ɯ-skɤt} 	\ipa{wuma} 	\ipa{ʑo} 	\ipa{ɲɯ-mpɕɤr} \\
\textsc{ipfv}-cry \textsc{lnk} \textsc{3sg.poss}-voice really \textsc{emph} \textsc{sens}-be.beautiful \\
\glt `When it cries, its voice is very beautiful.' (translation, 04-cuiniao, 26)
\end{exe}
\begin{exe}
\ex \label{ex:YWmpW}
\gll \ipa{ɲɯ́-wɣ-nɤmɤle} 	\ipa{tɕe} 	\ipa{ɲɯ-mpɯ.} \\
\textsc{ipfv-inv}-touch \textsc{lnk} \textsc{sens}-be.soft \\
\glt `It is soft to the touch.' (19 khWlu, 25)
\end{exe}

\begin{exe}
\ex \label{ex:tunAmnAmnW}
\gll \ipa{tɕe} 	\ipa{nɯ} 	\ipa{tu-nɤmnɤm-nɯ} 	\ipa{tɕe,} 	\ipa{cɤmtsho} 	\ipa{ɯ-di,} 	\ipa{pɯ\rdp{}pɯ-ŋu} 	\ipa{nɤ,} 	\ipa{ɯ-di} 	\ipa{ɲɯ-mnɤm,} \ipa{tɕe} 	\ipa{nɯnɯ} 	\ipa{tɕu} 	\ipa{ɯ-fsa} 	\ipa{tu-ta-nɯ} 	\ipa{ɲɯ-ŋgrɤl.} \\
\textsc{lnk} \textsc{dem} \textsc{ipfv}-smell-\textsc{pl} \textsc{lnk} musk \textsc{3sg.poss}-smell \textsc{cond\rdp{}pst.ipfv}-be \textsc{lnk} \textsc{3sg.poss}-smell \textsc{sens}-be.smell \textsc{lnk} \textsc{dem} \textsc{loc} \textsc{3sg.poss}-snare \textsc{ipfv}-put-\textsc{pl} \textsc{sens}-be.usually.the.case \\
\glt (The hunters) smell (the places where they find deer hair); if it is smell of musk, it is very strong. And they put the snare there. (27 kikakCi, 68)
\end{exe}

\begin{exe}
\ex \label{ex:YWmWm}
\gll
\ipa{tú-wɣ-ndza} 	\ipa{tɕe} 	\ipa{wuma} 	\ipa{ʑo} 	\ipa{ɲɯ-mɯm} 	\ipa{ɲɯ-ti} \\
\textsc{ipfv-inv}-eat \textsc{lnk} really \textsc{emph} \textsc{sens}-be.tasty \textsc{sens}-say \\
\glt `She said: `(These ferns, prepared this way) are very nice to eat'.' (said just after eating them; conversation 14.05.10)
\end{exe}

Although in the above examples there is no implication that the person producing the sound or the objects mentioned in the sentences are not visible to the speaker, in these contexts vision is largely irrelevant to determine the property in question, and there is not ambiguity as to which sense was responsible for obtaining the information.  

Like the Factual, the Sensory can occur with all persons, including the \textsc{1sg} (examples \ref{ex:YWrgaa} and \ref{ex:YWsAjloRa}) and the second person (example \ref{ex:YWtWmkhAz}).

With second person subjects, the Sensory is very commonly used to state a fact about the addressee that the speaker noticed (not something he knew previously). For instance, in contrast to \refb{ex:tWmkhAz} above in which the addressee's (recent) actions are irrelevant, a sentence such as \refb{ex:YWtWmkhAz} can be used if the speaker witnessed something revealing the proficiency of the addressee.

\begin{exe}
\ex \label{ex:YWtWmkhAz}
\gll 
 \ipa{ɲɯ-tɯ-mkʰɤz} \\
\textsc{sens}-2-be.expert \\
\glt `You are good at it.' (heard in several conversations)
\end{exe}

With first person subjects, the Sensory is not rare. It is common with verbs such as \ipa{rga} `be happy'\footnote{This verb should not be confused with the homophonous semi-transitive verb \ipa{rga} `like', from which it is demonstrably distinct, even though both are borrowed from Tibetan \ipa{dga}.} whose intransitive subject is the experiencer, as in \refb{ex:YWrgaa}.\footnote{As pointed out by an anonymous reviewer, this example is related to use of the Sensory prefix with endopathic predicates, as in (\ref{ex:YWmNAm}). }

\begin{exe}
\ex \label{ex:YWrgaa}
\gll 
 \ipa{nɤ-tɕɯ}  \ipa{tɤ-sci}  \ipa{tɕe}  \ipa{ɲɯ-pe}  \ipa{tɕe}  \ipa{papa,}  \ipa{aʑo} \ipa{ɲɯ-rga-a} \\
 \textsc{2sg.poss}-child \textsc{pfv}-born \textsc{lnk} \textsc{sens}-good \textsc{lnk} good  \textsc{1sg} \textsc{sens}-be.happy-\textsc{1sg} \\
\glt `It is nice that your son is born, I am happy.' (Tshendzin, conversation, 2013)
\end{exe}

With non-experiencer adjectival stative verbs, it can occur if the speaker discovers something about oneself, for instance from the behaviour of others as in \refb{ex:YWsAjloRa}.\footnote{This example is taken from the translation of Andersen's story `The Ugly Duckling', when a hunting dog appears before the eponymous character but does not bite him. }

\begin{exe}
\ex \label{ex:YWsAjloRa}
\gll 
\ipa{aʑo} 	\ipa{ndɤre} 	\ipa{ɲɯ-sɤjloʁ-a} 	\ipa{tɕe,} \ipa{tɤrʁaʁkɕi} 	\ipa{kɯnɤ} 	\ipa{ʑo} 	\ipa{kú-wɣ-mtsɯɣ-a} 	\ipa{mɯ́j-sɯsɤm} \\
\textsc{1sg} on.the.other.hand \textsc{sens}-be.ugly-\textsc{1sg} \textsc{lnk} hunting.dog also \textsc{emph} \textsc{ipfv-inv}-bite-\textsc{1sg} \textsc{neg:sens}-think[III] \\
\glt `I am (so) ugly that even a hunting dog does not want to bite me.'  (translation, 140519 chou xiaoya, 86)
\end{exe}

The Sensory is also used concerning information that is somehow part of common knowledge, but that the speaker has not had the opportunity to personally confirm. For instance, it is commonly used instead of the Factual for describing facts about animals that do not live in Tibetan areas and that the speaker only knows through indirect channels. Compare for instance the forms of the stative verbs \ipa{sɤɣ-mu} `be terrifying' and \ipa{mpɕɤr} `be beautiful': they appear in the Factual when referring to  spiders or flowers found in the area (\ref{ex:sAGmu} and \ref{ex:mpCAr}) and in the Sensory when referring to lions and gnus, which the speaker has only seen in zoos or in the television  (\ref{ex:YWsAGmu} and \ref{ex:YWmpCAr}).
 
 \begin{exe}
\ex \label{ex:sAGmu}
\gll 
\ipa{ŋgoŋpu}  	\ipa{ɴɢoɕna}  	\ipa{kɤ-ti}  	\ipa{ci}  	\ipa{tu}  	\ipa{tɕe,}  	\ipa{nɯnɯ}  	\ipa{wxti}  	\ipa{nɯ}  	\ipa{stoʁ}  	\ipa{jamar}  	\ipa{tu.}  	\ipa{kú-wɣ-rtoʁ}  	\ipa{tɕe}  	\ipa{sɤɣ-mu.}  \\
disaster spider \textsc{nmlz}:P-say \textsc{indef} exist\factual{} \textsc{lnk} \textsc{dem} be.big\factual{}  \textsc{dem} bean about exist\factual{} \textsc{ipfv-inv}-look.at \textsc{lnk} \textsc{deexp}-be.afraid\factual{}  \\
\glt `There is one that is  called `disaster spider', it is big, like the size of a bean. It is terrifying to look at it.' (26 mYaRmtsaR, 151)
\end{exe}

\begin{exe}
\ex \label{ex:YWsAGmu}
\gll 
\ipa{sɯŋgi}  	\ipa{nɯ}  	\ipa{ɲɯ-sɤɣ-mu.}  \\
lion \textsc{dem} \textsc{sens-deexp}-be.afraid \\
\glt `The lion is terrifying.' (20 sWNgi, 64)
\end{exe}


\begin{exe}
\ex \label{ex:mpCAr}
\gll
\ipa{nɯnɯ}  	\ipa{ɯ-mɯntoʁ}  	\ipa{nɯ}  	\ipa{mpɕɤr.}  \\
\textsc{dem} \textsc{3sg.poss}-flower \textsc{dem} be.beautiful\factual{} \\
\glt `Its flower is beautiful.' (15 babW, 105)
\end{exe}


\begin{exe}
\ex \label{ex:YWmpCAr}
\gll 
<jiaoma> 	\ipa{nɯ}  	\ipa{ɲɯ-mpɕɤr}  \\
gnu \textsc{dem} \textsc{sens}-be.beautiful \\
\glt `The Gnu is beautiful.' (20 RmbroN, 128)
\end{exe}

As in other languages of the area (\citealt{tournadre14evidentiality}), the Sensory form is used for endopathic sensations (pain, itch, cold etc)  relating to the speaker, as in example \refb{ex:YWmNAm}.

\begin{exe}
\ex \label{ex:YWmNAm}
\gll
\ipa{tʰam} 	\ipa{tɕe} 	\ipa{mɯ́j-cʰa-a,} 	\ipa{a-mi} 	\ipa{ɲɯ-mŋɤm.} \\
now \textsc{lnk} \textsc{neg:sens}-can-\textsc{1sg} \textsc{1sg.poss}-foot \textsc{sens}-hurt \\
\glt `Now I can't, my foot hurts.' (21 kuGrummAG, 24)
\end{exe}


Unlike in Lhasa Tibetan where the Sensory \ipa{'dug} cannot be used for non-personal endopathic feelings (\citealt{tournadre14evidentiality}), this possibility is available to the Japhug Sensory. 

In \refb{ex:tWCGa.YWmNAm}, the Sensory is used in a generic sentence, when the speaker has experienced himself the feeling and recounts his experience while presenting it as a generic fact, and thus do not count as a real example of Sensory with non-first person.

\begin{exe}
\ex \label{ex:tWCGa.YWmNAm}
\gll
\ipa{kɯ-maqʰu}  	\ipa{qʰe}  	\ipa{tɯ-ɕɣa}  	\ipa{ɲɯ-mŋɤm}  \\
\textsc{nmlz}:S/A-be.after \textsc{lnk} \textsc{genr.poss}-tooth \textsc{sens}-hurt \\
\glt `Afterwards tooths hurt.' (27 tApGi, 66)
\end{exe}

In example \refb{ex:nWrqo.YWmNAm}, which describes the effects of foot and mouth disease on cattle, the speaker  infers that the cattle suffering from the disease are in pain (because of their whining), yet uses the Sensory due to the fact that she describes an event she has directly witnessed by vision and hearing. 

\begin{exe}
\ex \label{ex:nWrqo.YWmNAm}
\gll \ipa{nɯ-mci} 	\ipa{kɤ-rɤwum} 	\ipa{maka} 	\ipa{mɯ́j-cʰa-nɯ} 	\ipa{tɕe} 	\ipa{nɯ-mci} 	\ipa{tu-ɣɤrɯβrɯβ} 	\ipa{ʑo} 	\ipa{ɲɯ-ŋu.}  
\ipa{tɕe} 	\ipa{nɯ-rqo} 	\ipa{ɲɯ-mŋɤm} 	\ipa{rca,} \\
\textsc{3pl.poss}-saliva \textsc{inf}-collect at.all \textsc{neg:sens}-can-\textsc{pl} \textsc{lnk} \textsc{3pl.poss}-saliva \textsc{ipfv}-flow.continuously \textsc{emph} \textsc{sens}-be \textsc{lnk} \textsc{3pl.poss}-throat \textsc{sens}-hurt \textsc{sfp}  \\
\glt `They cannot keep the saliva in their mouth, and it flows continuously. Their throat hurt.' (27-kharwut, 6)
\end{exe}

In \refb{ex:Wmi.YWmNAm} likewise we have the Sensory used with \ipa{mŋɤm} `hurt' to describe an event visually witnessed by the speaker.

\begin{exe}
\ex \label{ex:Wmi.YWmNAm}
\gll
\ipa{kɯɕnɤsqi} 	\ipa{tʰɯ-azɣɯt} 	\ipa{ri,} \ipa{tɕe} 	\ipa{pɤjkʰu} 	\ipa{ɯ-mi} 	\ipa{ɲɯ-mŋɤm} 	\ipa{tɕe} 	\ipa{ri,} 	\ipa{nɯ} 	\ipa{kɯnɤ} 	\ipa{kʰa} 	\ipa{tsʰitsuku} 	\ipa{ɲɯ-nɤme} 	\ipa{ɕti.} \\
seventy \textsc{pfv}-reach but \textsc{lnk} already \textsc{3sg.poss}-foot \textsc{sens}-hurt \textsc{lnk} but \textsc{dem} also house some.things \textsc{sens}-work[III] be:\textsc{affirm}\factual{} \\
\glt `He is seventy, his foot hurts already, but even like that he does all sorts of work at home.' (14 : tApitaRi, 49-50)
\end{exe}

 

\subsection{Egophoric} \label{sec:egophoric}
The Egophoric, while common in conversations, is nearly non-existent in narrative and procedural texts (outside of quotations) unlike the Factual and the Sensory. It is used to describe information that is not directly shareable, which the speaker obtains through his own `personal involvement in a state of affairs' (\citealt{floyd16egophoricity}). 

In declarative sentences, Egophoric can occur with first person, in particular with stative verbs whose subject is an experiencer (as \ipa{ku-scit-i} `we are happy' in \ref{ex:kusciti}; this is a very common form, independently heard in conversation, for instance in new year's greetings). It is however also compatible with third person, in the case of people from the same household, with whom the speaker shares his/her life (the king -- the speaker's husband -- and the servants in example \ref{ex:kusciti}), or with concrete or abstract nouns with a first person possessive prefix (as \ipa{a-ʁa} `my free time' in \ref{ex:kume}). This usage is similar to the so-called `broad scope' egophoric observed in Lhasa Tibetan and other Tibetic languages (\citealt[89]{gawne17bodish})

\begin{exe}
\ex \label{ex:kusciti}
\gll
\ipa{tɕʰeme} 	\ipa{nɯ} 	\ipa{kɯ} 	\ipa{`wuma} 	\ipa{ʑo} 	\ipa{ku-scit-i,} \ipa{rɟɤlpu} 	\ipa{ri} 	\ipa{a-taʁ} 	\ipa{wuma} 	\ipa{ku-sna} \ipa{ʁjoʁ} 	\ipa{ra} 	\ipa{ri} 	\ipa{wuma} 	\ipa{ʑo} 	\ipa{ku-pe-nɯ'} \ipa{to-ti,} \\
girl \textsc{dem} \textsc{erg} really \textsc{emph} \textsc{prs:egoph}-be.happy-\textsc{1pl}  roi also \textsc{1sg}-on really \textsc{prs:egoph}-be.kind servant \textsc{pl} also really \textsc{emph}   \textsc{prs:egoph}-be.good \textsc{ifr}-say \\
\glt `The girl said: `We are very happy, the king is very kind to me, the servants are very nice.''
(The frog 2002, 122-4)
\end{exe}


\begin{exe}
\ex \label{ex:kume}
\gll 
\ipa{aʑo}  	\ipa{kɯre}  	\ipa{a-ʁa}  	\ipa{ku-me}  \ipa{tɯ-mgo} 	\ipa{ku-osɯ-βzu-a} 	\ipa{ɕti} 	  	\\
\textsc{1sg} here \textsc{1sg.poss}-free.time \textsc{prs:egoph}-not.exist \textsc{indef.poss}-food \textsc{prs:egoph}-\textsc{prog}-make-\textsc{1sg} be.\textsc{affirm}\factual{}  \\
\glt `I don't have time, I am making food.'  (Rkangrgyal, 47)
\end{exe}

There is no syntactic rule requiring co-occurrence of egophoric marking with a first person subject or a subject with first person possessor. As shown by example (\ref{ex:WgrAl.kume}) with the collocation \ipa{ɯ-grɤl}+\ipa{me} (one of whose meanings is `be innumerable'), the Egophoric can occur even if no first person marker is present in the clause.\footnote{Example (\ref{ex:WgrAl.kume}) is a sentence pronounced in the story by a raven, who tells another raven that due to an infectious disease, a lot of cattle from nomad areas died, so that it has a lot of meat to eat. The use of egophoric here implies that it has partaken in the feast, rather than being simply witness of the fact. }

\begin{exe}
\ex \label{ex:WgrAl.kume}
\gll 
\ipa{ɕa} 	\ipa{ɯ-ndza} 	\ipa{ɯ-grɤl} 	\ipa{ku-me} 	 \\
meat \textsc{3sg-bare.inf}:eat \textsc{3sg.poss}-order \textsc{prs:egoph}-not.exist   \\
\glt `There is an immense amount of meat to eat.' (2003kandZislama, 129)
 \end{exe}
%a-wi cho a-wa ni nɤʑo nɤ-ndʐa kɯ wuma ʑo ku-nɯsɯmɯzdɯɣ-ndʑi tɕe,
%150819 haidenver, 494

No example of Egophoric with second person subject in declarative sentences has been found in the corpus, nor could such example be elicited.

In the case of experiencer stative verbs such as \ipa{scit} `be happy' with a first person subject, there is a tripartite contrast between Egophoric, Sensory and Factual. The following near-minimal pairs can help understand how each form is used in such a context.

\begin{exe}
\ex \label{ex:YWscita}
\gll  \ipa{nɯtɕu} 	\ipa{ɲɯ-scit-a} 	\ipa{ɕti} 	\ipa{li} 	\ipa{tɕe} 	\ipa{tɕe} 	\ipa{a-zda} 	\ipa{ri} 	\ipa{ɲɯ-pe-nɯ,} \\
\textsc{dem:loc} \textsc{sens}-be.happy-\textsc{1sg} be.\textsc{affirm}:\textsc{fact} again \textsc{lnk} \textsc{lnk} \textsc{1sg}-companion also \textsc{sens}-be.good-\textsc{pl} \\
\glt `I am very happy there, the people with me are very nice.' (140501 jingli, 149)
\end{exe}


\begin{exe}
\ex \label{ex:YWscita2}
\gll 
\ipa{nɯ} 	\ipa{tɤ-ŋu} 	\ipa{tɕe,} 	\ipa{aʑo} 	\ipa{ndɤre,} 	\ipa{ʁloŋbutɕhi} 	\ipa{sɤz} 	\ipa{ndɤre} 	\ipa{ɲɯ-scit-a} 	\ipa{tɕe} 	\ipa{a-kʰi} 	\ipa{ɲɯ-ŋgɯ} \\
\textsc{dem} \textsc{pfv}-be \textsc{lnk} \textsc{1sg} on.the.other.hand elephant \textsc{comp} on.the.other.hand \textsc{sens}-be.happy-\textsc{1sg} \textsc{lnk} \textsc{1sg.poss}-luck \textsc{sens}-be.lucky \\
\glt `Since it is like that, I am happier than the elephant, I am luckier than him.' (translation, 140425 shizi puluomixiusi he daxiang, 41)
\end{exe}


\begin{exe}
\ex \label{ex:sciti}
\gll \ipa{χsɯ-xpa} 	\ipa{jɤ-tsu-j,} 	\ipa{nɯsthɯci} 	\ipa{ʑo} \ipa{scit-i,} 	\ipa{amɯmi-j}  \\
three-year \textsc{pfv}-reach-\textsc{1sg} so.much \textsc{emph} be.happy:\textsc{fact-1pl} be.in.good.terms:\textsc{fact-1pl} \\
\glt `We have been together for three years now, we are so happy together.' (2005slobdpon2, 95)
\end{exe}

In \refb{ex:sciti}, the speakers (humans stranded on an island) include the addressees (râkshasîs in human shape) in the first plural, and state their happiness together as an commonly agreed fact (the first step in a plan to cheat the râkshasîs), hence the use of the Factual. 

In \refb{ex:YWscita2}, the speaker, after a discussion, realizes that he is happier than the lion, and therefore chooses the Sensory. In \refb{ex:YWscita} the use of the Sensory rather than the Egophoric is more subtle; the speaker, talking about her life at work, does not suddenly realizes that she is happy at work. Rather, she expressed that when thinking about it, she feels that she is happy, as opposed to the continuous conscience of being in a state of happiness implied by the use of the Egophoric in \refb{ex:kusciti}.

% suggests that one has come to know of one’s illness in a similar way to how one knows of one’s own actions or long term acquaintances. hill12mirativity

\subsection{Japhug and Tibetan}
The tripartite Japhug evidential system in present tense is very similar to the one observed in Tibetic languages such as Lhasa Tibetan between the Imperfective Factual \ipa{gi.yod.pa.red}, Sensory \textit{gis} and Egophoric \ipa{gi.yod} (\citealt[295+]{tournadre08conjunct}),\footnote{\citet{hill17evidential} propose to replace `egophoric' by ‘personal’ and `sensory' or `testimonial' by ‘experiential’, but in the interest of continuity with previous publications on Rgyalrong languages, I will keep the terminology used in \citet{jacques17sketch}. } though with minor differences in the use of Sensory for endopathic sensations (see example \ref{ex:YWmNAm} above). Surprisingly, the Japhug evidential categories are actually more similar to those of Lhasa Tibetan than some Tibetic varieties like Yolmo are (on which see \citealt{gawne13copulas}). 
 
\subsection{The expression of surprise}
Given the debated status of the expression of surprise and its relationship to evidentiality (\citealt{hill12mirativity}, \citealt{delancey12still}, \citealt{aikhenvald12mirativity}), it is useful to provide data on the mirative use of the Sensory evidential in Japhug. 

Japhug has two interjections specifically used to express surprise (\ipa{amaŋ} and \ipa{mtsʰɤri}, the second being of Tibetan origin). When the predicate of the sentence is a stative verb, it is possible to select the Sensory evidential as in \refb{ex:amang.YWmbro}, as expected for a visual perception. This use of the Sensory evidential is what motivated its analysis as a mirative marker in some languages (\citealt{hill12mirativity}).

\begin{exe}
\ex \label{ex:amang.YWmbro}
 \gll \ipa{amaŋ,}	\ipa{nɯstʰɯci}	\ipa{ɲɯ-mbro} \\
 \textsc{interjection:surprise} so.much \textsc{sens}-be.high \\
 \glt `It is so high!' (translation, 150826 liyu tiao longmen, 75)
\end{exe}

Another possibility is to use a non-finite verb form, the degree noun (built by combining a possessive prefix coreferent with the subject with the \ipa{-tɯ-} nominalization prefix, see \citealt[10-11]{jacques16comparative}), focusing on the surprisingly high degree of the observed property.

\begin{exe}
\ex \label{ex:amang.WtWsAre}
 \gll 
\ipa{amaŋ,}	\ipa{nɯ}	\ipa{ɯ-tɯ-sɤre,}	\ipa{mtshɤri,}	\ipa{ɯ-tɯ-sɤmtshɤr}	\ipa{nɯ}	\\
 \textsc{interjection:surprise} \textsc{dem} \textsc{3sg-nmlz:degree}-be.funny  \textsc{interjection:surprise} \textsc{3sg-nmlz:degree}-be.surprising \textsc{sfp} \\
 \glt `It is so funny, so surprising!' (translation, 150830 baihe jiemei, 112)
\end{exe}

\section{Interrogative clauses} \label{sec:interrogative}
Like most languages of the Tibetosphere, interrogatives sentences generally adopt the perspective of the addressee rather than that of the speaker, causing a phenomenon referred to as `anticipation rule' (\citealt[244]{tournadre14evidentiality}) or `flipping' (\citealt{sanroque17interrogativity}):   the speaker anticipates the answer of the addressee and uses the form that he expect the addressee will choose to respond to the question. For instance, in example \refb{ex:WtWsWz}, the speaker uses the Factual because she expects and answer with the Factual such as \ipa{sɯz-a} know:\textsc{fact}-\textsc{1sg} `I know'.

\begin{exe}
\ex \label{ex:WtWsWz}
\gll 
\ipa{nɤj}	\ipa{ɯ-tɯ́-sɯz?} \\
\textsc{2sg} \textsc{qu}-2-know:\textsc{fact} \\
\glt `Do you know that?' (19 GzW, 8)
\end{exe}

As a result of this change of perspective, compatibilities between evidential markers and first vs second person are always reversed between declarative and interrogative sentences. In particular, as discussed in section \ref{sec:egophoric}, the Egophoric is used with a \textit{first person} subject or a third person subject with first person possessor (or belonging to the same household as the speaker) in declarative sentences, never with a second person subject. In interrogative sentence, this person constraint is reversed: the Egophoric appears with \textit{second person} subject (as in example \ref{ex:WkutWscitnW}) or third person with second person possessive prefix. 


\begin{exe}
\ex \label{ex:WkutWscitnW}
\gll	`\ipa{ɯ-kú-tɯ-scit-nɯ?}' 	\ipa{ra} 	\ipa{to-ti,} \\
  \textsc{qu-prs:egoph}-2-be.happy-\textsc{pl} \textsc{pl} \textsc{evd}-say \\
\glt She said: `Are you (and your husband) happy?' (The frog 2002, 121)
\end{exe}

The addressee perspective however is not a syntactic rule. The addressee is free to adopt the evidential form suggested by the speaker who asked the question, or to choose another form if he sees fit: see \citet{garrett07symbiosis} for an account of this phenomenon in Tibetan. It is also possible to have in the same question two verbs referring to the addressee with the Egophoric in one case and the Sensory in the other, as in \refb{ex:WkutWpe}. 

\begin{exe}
\ex \label{ex:WkutWpe}
\gll 
\ipa{wo,} 	\ipa{ɯ-kú-tɯ-pe,} 	\ipa{ɯ-ɲɯ́-tɯ-cʰa} 	\ipa{nɯra} 	\ipa{ntsɯ} 	\ipa{to-ti.} \\
\textsc{interj} \textsc{qu-egoph:prs}-2-be.good \textsc{qu-sens}-2-be.fine \textsc{dem:pl} always \textsc{ifr}-say  \\
\glt `(The fox) said (to the deer) `Are you feeling well, are you fine?' (translation, 140425 shizi huli he lu, 16)
\end{exe}

Sentences \refb{ex:WYWpendZi} and \refb{ex:Wkupe} illustrate the difference of use of the Sensory and Egophoric forms in third person subject interrogative contexts. These questions expect answers in the Sensory and Egophoric forms respectively. Question \refb{ex:WYWpendZi} was asked when I phoned from my parents' home (when I came for the holidays). The Sensory is used because my parents and I do not live in the same household, and the expectation was that I had just realized whether or not they were well after having arrived at their place. 

Question \refb{ex:Wkupe} on the other hand, asked about my son, expects an answer in the Egophoric because since I live with him in the same house, I always know whether he is fine or not (I did not `discover' whether he was fine at a certain point).

\begin{exe}
\ex \label{ex:WYWpendZi}
\gll 
\ipa{nɤ-mu}  	\ipa{nɤ-wa}  	\ipa{ni}  	\ipa{ɯ-ɲɯ́-pe-ndʑi?}  \\
\textsc{2sg.poss}-mother \textsc{2sg.poss}-father \textsc{du} \textsc{qu-sens}-be.good-\textsc{du} \\
\glt `Are your parents well?' (conversation 2014.12)
\end{exe}


\begin{exe}
\ex \label{ex:Wkupe}
\gll \ipa{nɤ-tɕɯ} \ipa{ɯ-kú-pe?}\\
\textsc{2sg.poss}-son \textsc{qu-egoph}-be.good\\
\glt `Is your son well?' (conversation 2014.02)
\end{exe}

The Factual would not be appropriate in these contexts because neither involve a permanent state part of common knowledge.


\section{Reported Speech} \label{sec:hybrid}

%subsection{Conjunct/Disjunct}
% The term `conjunct / disjunct' was chosen by  \citet{hale80conjunct} specifically to refer to the use of these forms in complement clauses with a verb of speech: the `conjunct' appears when  subject of the complement clause is coreferent with that of the main clause, as in the Tibetan example \refb{ex:yin}, taken from (\citealt[295]{delancey90erg}).
%
%\begin{exe}
%\ex \label{ex:yin}
%\gll   \ipa{kho-s} 	\ipa{kho} 	\ipa{bod=pa} 	\ipa{yin} 	\ipa{zer}-\ipa{gyis} \\
%He-\textsc{erg} he Tibetan be:\textsc{conjunct}  say-\textsc{impf/disjunct} \\
%\glt `He_i says that he_i is a Tibetan.'  
%   \end{exe}
%
%The \textit{disjunct} form (in Tibetan, the copula \ipa{red}) appears in all other situations, in particular when the subject of the complement clauses is not coreferent with that of the main clause, as in \refb{ex:red}.
%
%\begin{exe}
%\ex \label{ex:red}
%\gll \ipa{kho-s} 	\ipa{kho} 	\ipa{bod=pa} 	\ipa{red} 	\ipa{zer}-\ipa{gyis} \\
%He-\textsc{erg} he Tibetan be:\textsc{disjunct}  say-\textsc{impf/disjunct}\\
%\glt `He_i says that he_j is a Tibetan.'
%   \end{exe}
%   


% The terms `conjunct' vs `disjunct' describe the use of these markers in syntactic terms (person indexation and coreference). \citet{tournadre08conjunct} however argues that the contrast observed between \refb{ex:yin} and \refb{ex:red} is better described in terms of \textit{Hybrid Indirect Speech}, a type of Indirect Speech in which the verb form in the quoted sentence reproduces exactly the form used by the original speaker (Egophoric in \ref{ex:yin} since the speaker is talking about himself, non-egophoric in \ref{ex:red} since he is talking about someone else), while the pronoun represents the point of view of the current speaker. Evidence from Japhug, and Gyalrongic languages in general, support  Tournadre's analysis.

%subsection{Hybrid Indirect Speech and person marking in Japhug}
Hybrid Indirect Speech is a well-established feature of Tibetic languages (\citealt{tournadre08conjunct}), and has also been documented in Gyalrongic languages, in particular Japhug (\citealt[241-244]{jacques16complementation}) and Stau (\citealt{jacques17stau}). Examples \refb{ex:nWGi.kAsWso} and \refb{ex:tunAmea}, taken from \citet[242-3]{jacques16complementation},  illustrate the effects of Hybrid Indirect Speech on person indexation and pronouns. As in Tibetan, the verb form represents the point of view of the original speaker (highlighted in blue the following examples), while the pronouns and possessive prefixes represent that of the current speaker (highlighted in red). 

\begin{exe}
\ex \label{ex:nWGi.kAsWso}
\gll 
\ipa{ma} \ipa{nɤ-wa}  	\ipa{kɯ}  	[\rouge{\ipa{nɤʑo}} 	\bleu{\ipa{nɯɣi}}]  	\ipa{kɤ-sɯso}  	\ipa{kɯ}  	\ipa{kʰa}  	\ipa{ɯ-rkɯ} \ipa{tɕe} 	\ipa{ʁmaʁ}  	\ipa{χsɯ-tɤxɯr}  	\ipa{pa-sɯ-lɤt}  	\ipa{ɕti}  	\ipa{tɕe}  \\
\textsc{lnk} \textsc{2sg.poss}-father \textsc{erg} \textsc{2sg} {come.back:\textsc{fact}}  \textsc{inf}-think \textsc{erg} house \textsc{3sg.poss}-side \textsc{lnk} soldier three-circle \textsc{pfv:3$\rightarrow$3'-caus}-throw be.\textsc{affirm}:\textsc{fact} \textsc{lnk}\\
\glt \textbf{Direct}: `Your father, thinking `\bleu{He is coming back}',   put three circles of soldiers around the house.' 
\glt  \textbf{Indirect}: `Your father, thinking that \rouge{you are coming back},'
\glt  \textbf{Hybrid indirect}: `Your father, thinking that `\rouge{you}' \bleu{is coming back},' (qachGa 2003, 154)
\end{exe}

In \refb{ex:nWGi.kAsWso}, the \textsc{2sg} pronoun \ipa{nɤʑo} normally requires a verb in second person form \ipa{tɯ-nɯɣi} 2-come.back:\textsc{fact} `you are coming back'; the use of the third person \ipa{nɯɣi} `he is coming back' reflects the point of view of the original speaker (the father), while the pronoun  \ipa{nɤʑo} represents that of the current speaker. Example \refb{ex:tunAmea} shows that the shift in speaker perspective applies not only to pronouns, but also to possessive prefixes.

\begin{exe}
\ex \label{ex:tunAmea}
\gll  \ipa{tɕendɤre}  	\ipa{ta-ʁi}  	\ipa{nɯ}  	\ipa{kɯ}  	[\rouge{\ipa{ɯ-pi}}  	\ipa{ɣɯ}  	\ipa{ɯ-sci}  	\bleu{\ipa{tu-nɤme-a}}  	\ipa{ra}] 	\ipa{ɲɤ-sɯso}  	\ipa{tɕe,}  	\\
\textsc{lnk}  \textsc{indef.poss}-younger.sibling \textsc{dem} \textsc{erg}  {\textsc{3sg.poss}-elder.sibling}  \textsc{gen} \textsc{3sg.poss}-revenge {\textsc{ipfv}-make[III]-\textsc{1sg}} have.to:\textsc{fact} \textsc{ifr}-think \textsc{lnk} \\
\glt  \textbf{Direct}: `The (younger) sister thought ``\bleu{I have to get revenge} on \bleu{my brother}".'
\glt  \textbf{Indirect}:  `The (younger) sister$_i$ \rouge{wanted to get revenge on her$_i$ brother}.'
\glt  \textbf{Hybrid indirect}:  `The (younger) sister$_i$ thought \bleu{I_i have to get revenge} on \rouge{her$_i$ brother}".' (translation, xiong he mei, 17)
\end{exe}
 
 The presence of Hybrid Indirect Speech is never obligatory; Direct Speech is always a possibility, and moreover since pronouns are not overt in most quotations in our corpus, it is rarely possible to distinguish between the two, since the verb form, the only obligatory element, will be the same regardless.\footnote{Indirect Speech is also marginally attested in translations from written Chinese.}
 
%subsection{Hybrid Indirect Speech and egophoric marking}
The shift of viewpoint caused by Hybrid Indirect Speech has effect not only on person marking, but also on person-sensitive evidential marking. %like the Egophoric.

Examples  \refb{ex:atCha.kume} and \refb{ex:WtCha.kume} are from two versions of the same story, translated from Chinese. The first translation \refb{ex:atCha.kume}, uses Direct Speech: \ipa{a-tɕʰa}  \ipa{maka} 	\ipa{ku-me} `I do not have any news': the possessor of the inalienably possessed abstract noun \ipa{--tɕʰa} corresponds (in this construction with an existential verb) to the person who receives the information, not the person whom the information is about.\footnote{In the original text, the corresponding clause \zh{一直没有消息} `there have not been any news (since then)' has no explicit person marker; person marking and evidential markers here cannot be due to calquing from Chinese. }
 

\begin{exe}
\ex \label{ex:atCha.kume}
 \gll %\ipa{``nɤ-rʑaβ} 	\ipa{nɯ} 	\ipa{kɯ-xtɕi} 	\ipa{nɯnɯ} 	\ipa{tɤ-ngo} 	\ipa{tɕe} 	\ipa{nɯ-si} 	
 \ipa{``nɤ-tɕɯ} 	\ipa{nɯnɯ,} 	\ipa{kɯ} 	\ipa{nɯnɯ} 	\ipa{ɯ-mu} 	\ipa{kɤ-nɯzdɯɣ} 	\ipa{kɯ,} 	... 	\ipa{kʰa} 	\ipa{na-βde} 	\ipa{tɕe} 	 	\ipa{jɤ-ari} 	\ipa{ɕti} 	\ipa{tɕe,} 	\ipa{ŋotɕu} 	\ipa{nɯ-ari} 	\ipa{mɤxsi} 	\ipa{ma} 	\bleu{\ipa{a-tɕʰa}} 	\ipa{maka} 	\bleu{\ipa{ku-me"}} 	\ipa{to-ti}  \\
%\textsc{2sg.poss}-wife \textsc{dem} \textsc{nmlz}:S/A-be.small \textsc{dem} \textsc{pfv}-be.sick \textsc{lnk} \textsc{pfv}-die 
\textsc{2sg.poss}-son \textsc{dem} \textsc{erg} \textsc{dem} \textsc{3sg.poss}-mother \textsc{inf}-\textsc{worry.about}  \textsc{erg} ... house \textsc{pfv}:3$\rightarrow$3'-leave \textsc{lnk} \textsc{pfv}-go[II] be:\textsc{affirm:fact} \textsc{lnk} where \textsc{pfv}-go[II] \textsc{neg:genr}:know \textsc{lnk} \bleu{\textsc{1sg.poss}-news} at.all \bleu{\textsc{egoph:prs}-not.exist} \textsc{ifr}-say \\
\glt `Your son left out of grief, I don't know where he went, I don't have any news (from him).' (translation, fushang he yaomo, 101-103)
\end{exe}

The second version \refb{ex:WtCha.kume} uses Hybrid Indirect Speech, and we observe a mismatch between the possessive prefix on the noun \ipa{ɯ-tɕʰa} (third person) and the Egophoric form of the predicate.

 \begin{exe}
\ex \label{ex:WtCha.kume}
 \gll %\ipa{"nɤ-rʑaβ} 	\ipa{kɯ-xtɕi} 	\ipa{nɯnɯ} 	\ipa{ɯ-kɯ-mŋɤm} 	\ipa{kɤ-ndzoʁ} 	\ipa{tɕe} 	\ipa{nɯ-si,} 	
 \ipa{``nɤ-tɕɯ} 	\ipa{nɯnɯ} 	\ipa{kɯ} 	\ipa{nɯnɯ} 	\ipa{ɯ-mu} 	\ipa{kɤ-nɯzdɯɣ} 	\ipa{kɯ} 	\ipa{tɕe} 	\ipa{kha} 	\ipa{na-βde} 	\ipa{tɕe,} 	\ipa{jɤ-a<nɯ>ri} 	\ipa{ɕti} 	\ipa{tɕe,} 	\ipa{maka} 	\rouge{\ipa{ɯ-tɕʰa}} 	\bleu{\ipa{ku-me}"} 	\ipa{ɲɯ-ti} 	\ipa{ɕti.} \\
% \textsc{2sg.poss}-wife  \textsc{nmlz}:S/A-be.small \textsc{dem}   \textsc{3sg.poss}-\textsc{nmlz}:S/A-hurt \textsc{pfv}-\textsc{auticaus}:attach \textsc{lnk} \textsc{pfv}-die 
 \textsc{2sg.poss}-son \textsc{dem} \textsc{erg} \textsc{dem} \textsc{3sg.poss}-mother \textsc{inf}-\textsc{worry.about}  \textsc{erg} \textsc{lnk} house \textsc{pfv}:3$\rightarrow$3'-leave \textsc{lnk} \textsc{pfv-<auto>}go[II] \textsc{affirm:fact} \textsc{lnk} at.all \rouge{\textsc{3sg.poss}-news} \bleu{\textsc{egoph:prs}-not.exist} \textsc{sens}-say \textsc{affirm:fact}  \\
\glt  `Your son left out of grief, he went (away) on his own, 
\glt \textbf{Direct}: `\bleu{I don't have} any news (from him).' 
\glt \textbf{Indirect}: \rouge{She (his wife) does not have} any news (from him).
\glt \textbf{Hybrid indirect}: \rouge{She (his wife)} \bleu{don't have} any news (from him).
\glt (translation, fushang he yaomo1, 114-116)
\end{exe}

In this construction, the possessive prefix on \ipa{-tɕʰa} can only refer to the person who receives the information, not the person the information is about. Thus, a sentence such as (\ref{ex:WtCha.pWme}) cannot be translated as `X did not receive any news from him'.

\begin{exe}
\ex \label{ex:WtCha.pWme}
\gll \ipa{ɯ-tɕʰa} \ipa{pɯ-me} \\
\textsc{3sg.poss}-news \textsc{pst.ipfv}-not.exist \\
\glt `He has not received any news/answers.'(elicited)
\end{exe}


The verb form \ipa{ku-me} with the Egophoric represents the point of view of the original speaker (the first wife), while the possessive prefix on the noun represents the point of view of the current speaker, for whom the original speaker is third person.\footnote{Note however that the first sentence in example (\ref{ex:WtCha.kume}) is in Direct Speech, as shown by the second person prefix \ipa{nɤ-} on the noun \ipa{-tɕɯ} `son'. If this sentence were in Hybrid Indirect Speech, we would instead have the perspective of the current speaker, and use a first person possessive prefix \ipa{a-tɕɯ} `my son'.} 

%This example is comparable to the Tibetan example \refb{ex:yin} above (repeated as \refb{ex:yin2} with colour code), where the verb was in Egophoric form with a third person pronoun.
%\begin{exe}
%\ex \label{ex:yin2}
%\gll   \ipa{kho-s} 	\rouge{\ipa{kho}} 	\ipa{bod=pa} 	\bleu{\ipa{yin}} 	\ipa{zer}-\ipa{gyis} \\
%He-\textsc{erg} \rouge{he} Tibetan \bleu{be:\textsc{egophoric}}  say-\textsc{impf/disjunct} \\
%\glt `He_i says that he_i is a Tibetan.'  
%   \end{exe}

%While clear examples of Hybrid Indirect Speech with the Egophoric have not yet been discovered in the corpus at hand, we do find examples of reported speech with an evidential mismatch between the current speaker and the original speaker involving the Sensory evidential. %However, the converse situation, with a non-Egophoric form combined with a first person pronoun, is less conspicuous since both Factual and Sensory are compatible in certain contexts with first person, but is equally instructive. 

Example \refb{ex:YWkhe} illustrate the converse situation, with the Sensory used with the first person. The verb \ipa{ɲɯ-kʰe} `he is stupid' is a third person form in the Sensory, while the pronoun \ipa{aʑo} is first person (in direct speech, the third person pronoun \ipa{ɯʑo} `he' instead would be been expected).

 \begin{exe}
\ex \label{ex:YWkhe}
 \gll
\rouge{\ipa{aʑo}} 	\bleu{\ipa{ɲɯ-kʰe}} 	\ipa{a-mɤ-nɯ-sɯso-nɯ} \\
\rouge{\textsc{1sg}} \bleu{\textsc{sens}-be.stupid} \textsc{irr-neg-pfv}-think-\textsc{pl}  \\
\glt  \textbf{Direct}: `Let us hope that they will not think `\bleu{He is stupid}.'
\glt  \textbf{Indirect}:  `Let us hope that they will not think that \rouge{I am stupid}.'
\glt  \textbf{Hybrid indirect}: `Let us hope that they will not think that \rouge{I} \bleu{is stupid}.' (translation, huangdi de xinzhuang)
\end{exe}

In addition to mismatch in person indexation, the evidential form of the verb -- the Sensory \ipa{ɲɯ-} -- also expresses the point of view of the original speaker, namely the words that the current speaker attributes to other people. It does not imply that the speaker discovers a fact about himself (unlike examples like \ref{ex:YWsAjloRa} above), since he does not consider himself to be stupid.

Future research on Japhug narratives will hopefully reveal converse examples with a verb in Egophoric form in reported speech combined with a third or second person pronoun or possessive prefix.

\section*{Conclusion}
This paper is the first step towards a comprehensive description of the evidential system in Japhug, and of that of Rgyalrong languages in general. Focusing on a narrow topic, stative verbs in present tense, it presents the use of the tripartite evidential system in this context (Factual, Sensory, Egophoric) and their interaction with person indexation. It shows that the use of the Egophoric marker in declarative clauses, interrogative clauses and reported speech cannot be described by a syntactic rule, and is only indirectly related to person indexation. Future research will have to study the evidential contrasts of Stative verbs in past tense, and then extend this study to dynamic intransitive and transitive verbs.

%This work supports the findings of \citet{tournadre08conjunct} and more recent publications such as \citet{hill17evidential} on Tibetic languages, which argue against the applicability of the `conjunct/disjunct' model. 
Japhug and other Rgyalrong languages, which combine a Tibetan-like evidential system with polypersonal indexation on the verb, provide a testing ground for studying the interaction between evidentiality and person in cross-linguistic perspective.

\bibliographystyle{unified}
\bibliography{bibliogj}

\end{document}