\documentclass[xcolor=table]{beamer}
\usepackage{fontspec}
\usepackage{natbib}
\usepackage{gb4e} 
\usepackage[table]{xcolor}
%\usepackage{color}
\usepackage{graphicx}

 \setmainfont[Mapping=tex-text]{Charis SIL}
\let\sfdefault\rmdefault
%\newcommand{\racine}[1]{\begin{math}\sqrt{#1}\end{math}} 
\newfontfamily\phon[Mapping=tex-text,Ligatures=Common,Scale=MatchLowercase,FakeSlant=0.3]{Charis SIL} 
\newcommand{\ipa}[1]{{\phon \mbox{#1}}} %API tjs en italique
\newcommand{\grise}[1]{\cellcolor{lightgray}\textbf{#1}} 
\newcommand{\ra}{$\Sigma_1$} 
\newcommand{\rc}{$\Sigma_3$} 
\newcommand{\ro}{$\Sigma$} 
 \begin{document}

 \title{Analogical leveling in the Khaling verbal paradigms}
 \author{Guillaume Jacques}
 \maketitle
 
  \begin{frame} 
 \frametitle{Stem alternation in Khaling} 
 
  
  \begin{table}
\centering

\begin{tabular}{lll}


\textsc{1sg>3.npst} &  \ipa{ɦod-u} \\ 
\textsc{du} &  \ipa{ɦɵʦ-i} \\ 
\textsc{1pl} &  \ipa{ɦoɔç-ki} \\ 
\textsc{2pl} &  \ipa{ʔi-ɦoɔ̂n-ni} \\ 
\textsc{3>3sg.npst} &  \ipa{ɦɵ̄ːd-ʉ} \\ 
\textsc{3>3n.sg.npst} &  \ipa{ɦɵ̂ːt-nu} \\ 
\textsc{3>2.npst} &  \ipa{ʔi-ɦoɔ̂j} \\ 
\textsc{1sg>3.pst} &  \ipa{ɦôːt-ʌ} \\ 
\textsc{pst} &  \ipa{ɦɵs-ti} \\ 
\textsc{3>3sg.pst} &  \ipa{ɦɵ̂ːt-ɛ} \\ 
\end{tabular}
\end{table}
  \end{frame}   


 \begin{frame} 
 \frametitle{Khaling verbal paradigms} 
\framesubtitle{Stem alternations}

Données tirées de \citet{jacques12khaling}


\begin{table}[H]
\caption{Khaling transitive and intransitive paradigms (Non-Past)} \label{tab:khaling2}
\resizebox{\columnwidth}{!}{
\begin{tabular}{l|l|l|l|l|l|l|l|l|l|l|l|}
\textsc{} & 	\textsc{1s} & 	\textsc{1di} & 	\textsc{1de} & 	\textsc{1pi} & 	\textsc{1pe} & 	\textsc{2s} & 	\textsc{2d} & 	\textsc{2p} & 	\textsc{3s} & 	\textsc{3d} & 	\textsc{3p} \\ 
\hline	
\textsc{1s} & 	\multicolumn{5}{c|}{\grise{}} &	\ipa{loɔ̂m-tɛni}   &	\ipa{loɔ̂m-tɛnsu}   &	\ipa{loɔ̂m-tɛnnu}   &	\ipa{lob-utʌ}   &	\ipa{lob-utʌ-su}   &	\ipa{lob-utʌ-nu}   \\	
\cline{7-12}
\textsc{1di} & \multicolumn{8}{c|}{\grise{}} &	\multicolumn{3}{c|}{	\ipa{lɵp-iti} }	   \\	
\cline{7-12}
\textsc{1de} & 	\multicolumn{5}{c|}{\grise{}} &		\ipa{ʔi-lɵp-tɛ}   &		\ipa{ʔi-lɵp-iti}   &	\ipa{ʔi-lɵp-tɛnu}  	 &	 \multicolumn{3}{c|}{	\ipa{lɵp-utu} }	   \\	
\cline{7-12}
\textsc{1pi} & 	\multicolumn{8}{c|}{\grise{}} &	 \multicolumn{3}{c|}{\ipa{loɔp-tiki}}   \\	
\cline{7-12}
\textsc{1pe} & 	\multicolumn{5}{c|}{\grise{}} &		\ipa{ʔi-lɵp-tɛ}   &		\ipa{ʔi-lɵp-iti}   &	\ipa{ʔi-lɵp-tɛnu}  	  &\multicolumn{3}{c|}{\ipa{loɔp-tʌkʌ} }	   \\	
\cline{2-2}
\cline{4-4}
\cline{6-12}
\textsc{2s} & 	\ipa{ʔi-lɵp-ʌtʌ}  	 &	\grise{} &	 	  &	\grise{} &	  	 &	\grise{} &	\grise{} &	\grise{} &	\ipa{ʔi-lɵ̂ːp-tɛ} 	  &	\ipa{ʔi-lɵ̂ːp-tɛsu}  	 &		\ipa{ʔi-lɵ̂ːp-tɛnu}   \\	
\cline{2-2}	
\cline{10-12}
\textsc{2d} &	 	\ipa{ʔi-lɵp-ʌtʌ-su}   &	\grise{} &	 	  &	\grise{} &		   &	\grise{} &	\grise{} &	\grise{} &	 \multicolumn{3}{c|}{	\ipa{ʔi-lɵp-iti} }	   \\	
\cline{2-2}	
\cline{10-12}
\textsc{2p} & 		\ipa{ʔi-lɵp-ʌtʌ-nu}   &	\grise{} &	 	  &	\grise{} &	  	 &	\grise{} &	\grise{} &	\grise{} &	\multicolumn{3}{c|}{	\ipa{ʔi-lɵp-tɛnu}}	   \\	
\cline{2-3}	
\cline{5-5}	
\cline{7-12}
\textsc{3s} & 		\ipa{ʔi-lɵp-ʌtʌ}   &	   	&	  	 &	  	 &		   &		   &	  	 &	 	  &	\ipa{lɵ̂ːp-tɛ}   &	   &	   \\	
\cline{2-2}	
\cline{10-10}
\textsc{3d} &	 	\ipa{ʔi-lɵp-ʌtʌ-su}   &		\ipa{ʔi-lɵp-iti}   &		\ipa{ʔi-lɵp-utu}   &		\ipa{ʔi-loɔp-tiki}   &		\ipa{ʔi-loɔp-tʌkʌ}   &	\ipa{ʔi-lɵp-tɛ} 	  &		\ipa{ʔi-lɵp-iti}   &		\ipa{ʔi-lɵp-tɛnu}   &	  \multicolumn{2}{r|}{\ipa{lɵ̂ːp-tɛsu}}   &	   \\	
\cline{2-2}	
\cline{10-11}
\textsc{3p} &	 	\ipa{ʔi-lɵp-ʌtʌ-nu}  	 &	  	 &	   	&	 	  &	 	  &	 	  &		   &	   	&	   \multicolumn{3}{r|}{\ipa{lɵ̂ːp-tɛnu}   } \\	
\hline
\textsc{intr}&\ipa{sɵp-ʌtʌ}   &	\ipa{sɵp-iti}   &	\ipa{sɵp-utu}   &	\ipa{soɔp-tiki}   &	\ipa{soɔp-tʌkʌ}   &		\ipa{ʔi-sɵp-tɛ}   &		\ipa{ʔi-sɵp-iti}   &	\ipa{ʔi-sɵp-tɛnu}   &	  	   \ipa{sɵp-tɛ}   &	\ipa{sɵp-iti}   &	\ipa{sɵp-tɛnu}   	  	   \\	
\hline
\end{tabular}}
\end{table} 
  \end{frame} 
  
  \begin{frame} 
 \frametitle{Free variation (--k)} 
 
   \begin{table}[H]
\label{uk.vi}
\caption{Intransitive verb \ipa{ʣhuk} ``escape"  }
\begin{tabular}{l|l|l|l|l|l|l|l|l|l|l}  \hline
&non-past & past & imperative \\ 
1s &ʣhûːŋʌ / ʣhûŋŋʌ \grise{} &ʣhʉkʌtʌ \\ 
1di &ʣhʉki &ʣhʉkiti   \\
1de &ʣhʉku &ʣhʉkutu   \\ 
1pi &ʣhukki &ʣhuktiki   \\ 
1pe &ʣhukkʌ &ʣhuktʌkʌ   \\ 
2s & ʔiʣhûː & ʔiʣhʉktɛ &ʣhʉkje  \\ 
2d & ʔiʣhʉki & ʔiʣhʉkiti &ʣhʉkije    \\
2n & ʔiʣhûːni  & ʔiʣhʉktɛnu &ʣhʉknuje  \\ 
3s & ʣhûː & ʣhʉktɛ   \\ 
3d & ʣhʉki & ʣhʉkiti   \\ 
3n & ʣhûːnu  & ʣhʉktɛnu \\ 
\hline
\end{tabular}
\end{table}
  \end{frame}   
  
    
  \begin{frame} 
 \frametitle{Free variation (reflexive paradigm)} 
 

\begin{table}[h]
\caption{The conjugation of |\ipa{nɛm-si}|  ``to immerse oneself"  } \centering \label{tab:nyamsi}
\begin{tabular}{l|l|l|l}  
\hline
& non-past & past & imperative\\
\hline
\textsc{1s}  &  \ipa{nɛ̄m-si-ŋʌ}   &  \ipa{nɛ̄m-\textbf{tʌsu}} / \ipa{nɛ̄m-siŋʌtʌ}  \grise{}\\ 
\textsc{1di}  &  \ipa{nɛ̄m-si-ji}   &  \ipa{nɛ̄m-sî-jti} \\
\textsc{1de}  &  \ipa{nɛ̄m-si-ju}   &  \ipa{nɛ̄m-sî-jtu} \\ 
\textsc{1pi}  &  \ipa{nɛ̄m-si-ki}   &  \ipa{nɛ̄m-si-ktiki} \\ 
\textsc{1pi}  &  \ipa{nɛ̄m-si-kʌ}   &  \ipa{nɛ̄m-si-ktʌkʌ} \\ 
\hline
\textsc{2s}  &  \ipa{ʔi-nɛ̄m-si}   &  \ipa{ʔi-nɛ̄m-tɛ-si}   &  \ipa{nɛ̄m-si-je} \\ 
\textsc{2d}  &  \ipa{ʔi-nɛ̄m-si-ji}   &  \ipa{ʔi-nɛ̄m-sî-jti}   &  \ipa{nɛ̄m-sî-jje} \\
\textsc{2p}  &  \ipa{ʔi-nɛ̄m-si-ni}   &  \ipa{ʔi-nɛ̄m-\textbf{tɛnnu}}   &  \ipa{nɛ̄m-\textbf{nuje}} \\ 
\hline
\textsc{3s}  &  \ipa{nɛ̄m-si}   &  \ipa{nɛ̄m-tɛ-si} \\ 
\textsc{3d}  &  \ipa{nɛ̄m-si-ji}   &  \ipa{nɛ̄m-sî-jti} \\ 
\textsc{3p}  &  \ipa{nɛ̄m-si-nu}   &  \ipa{nɛ̄m-\textbf{tɛnnu}}   \\ 
\hline
\end{tabular}
\end{table}

\textsc{Watkins' laws}
  \end{frame}   
  
  \begin{frame} 
 \frametitle{Analogy} 
 \framesubtitle{The open-syllable paradigms}  
  
 \begin{table}[H]
\label{o.vi}
\caption{Intransitive verb \ipa{ɦo} ``come" }
\begin{tabular}{l|l|l|l|l|l|l|l|l|l|l}  \hline
&non-past & past & imperative \\ 
1s &ɦɵŋʌ &ɦɵŋʌtʌ \\ 
1di &ɦɵji &ɦɵ̂jti   \\
1de &ɦɵju &ɦɵ̂jtu   \\ 
1pi &ɦɵki &ɦɵktiki   \\ 
1pe &ɦɵkʌ &ɦɵktʌkʌ   \\ 
2s & ʔiɦɵ & ʔiɦōːtɛ &ɦōːje  \\ 
2d & ʔiɦɵji & ʔiɦɵ̂iti &ɦɵ̂ije    \\
2n & ʔiɦɵni  & ʔiɦōːtnu / ʔiɦōːtɛnu \grise{}&ɦônje  \\ 
3s & ɦɵ & ɦōːtɛ   \\ 
3d & ɦɵji & ɦɵ̂iti     \\ 
3n & ɦɵnu  & ɦōːtnu / ɦōːtɛnu \grise{} \\ 
\hline
\end{tabular}
\end{table}
  \end{frame} 
  
  \begin{frame} 
 \frametitle{Analogy} 
 \framesubtitle{The reflexive paradigm}
 
\begin{table}[H]
 \centering 
\caption{ \ipa{nɵ̂nsinɛ}  ``to rest"  } \label{tab:no}
\begin{tabular}{l|l|l|l|l|l|l|l|l|l|l|l|l} 
 \hline
&original paradigm& analogical paradigm \\
\hline
\textsc{1s} & \ipa{nɵ̂nsiŋʌ} &\ipa{nɵ̂nsiŋʌ} \\ 
\textsc{1di} & \ipa{nɵssiji} &\ipa{nɵ̂nsiji} \grise{}  \\
\textsc{1de} & \ipa{nɵssiju} & \ipa{nɵ̂nsiju} \grise{}  \\ 
\textsc{1pi} & \ipa{nɵssiki} & \ipa{nɵ̂nsiki}  \grise{} \\ 
\textsc{1pe} & \ipa{nɵssikʌ} & \ipa{nɵ̂nsikʌ} \grise{}  \\ 
\textsc{2s} & \ipa{ʔinɵ̂nsi} & \ipa{ʔinɵ̂nsi}   \\ 
\textsc{2d} & \ipa{ʔinɵssiji} & \ipa{ʔinɵ̂nsiji}  \grise{}  \\
\textsc{2p} & \ipa{ʔinɵ̂nsini}  & \ipa{ʔinɵ̂nsini}    \\ 
\textsc{3s} & \ipa{nɵ̂nsi} & \ipa{nɵ̂nsi}   \\ 
\textsc{3d} & \ipa{nɵssiji} & \ipa{nɵ̂nsiji}  \grise{} \\ 
\textsc{3p} & \ipa{nɵ̂nsinu}  & \ipa{nɵ̂nsinu} \\ 
\hline
\end{tabular}
\end{table}
 
  \end{frame}     
  
 \begin{frame} 
 \frametitle{References}
 
 \bibliographystyle{Linquiry2}
\bibliography{bibliogj}
 \end{frame}
\end{document}