\documentclass[oldfontcommands,oneside,a4paper,11pt]{article} 
\usepackage{fontspec}
\usepackage{natbib}
\usepackage{booktabs}
\usepackage{xltxtra} 
\usepackage{polyglossia} 
%\setmainlanguage{french}
 \usepackage{geometry}
 \geometry{
 a4paper,
 total={210mm,297mm},
 left=10mm,
 right=10mm,
 top=15mm,
 bottom=15mm,
 }
\usepackage{bibentry}
\usepackage[table]{xcolor}
\usepackage{gb4e} 
\usepackage{graphicx}
\usepackage{float}
\usepackage{hyperref} 
\hypersetup{bookmarks=false,bookmarksnumbered,bookmarksopenlevel=5,bookmarksdepth=5,xetex,colorlinks=true,linkcolor=blue,citecolor=blue}
\usepackage[all]{hypcap}
\usepackage{memhfixc}

\bibpunct[: ]{(}{)}{,}{a}{}{,}
%%%%%%%%%quelques options de style%%%%%%%%
%\setsecheadstyle{\SingleSpacing\LARGE\scshape\raggedright\MakeLowercase}
%\setsubsecheadstyle{\SingleSpacing\Large\itshape\raggedright}
%\setsubsubsecheadstyle{\SingleSpacing\itshape\raggedright}
%\chapterstyle{veelo}
%\setsecnumdepth{subsubsection}
%%%%%%%%%%%%%%%%%%%%%%%%%%%%%%%
\setmainfont[Mapping=tex-text,Numbers=OldStyle,Ligatures=Common]{Charis SIL} %ici on définit la police par défaut du texte

\newfontfamily\phon[Mapping=tex-text,Ligatures=Common,Scale=MatchLowercase,FakeSlant=0.3]{Charis SIL} 
\newcommand{\ipa}[1]{{\phon #1}} %API tjs en italique
 \newcommand{\ipapl}[1]{{\phon #1}} %API tjs en italique
\newcommand{\grise}[1]{\cellcolor{lightgray}\textbf{#1}}
\newfontfamily\cn[Mapping=tex-text,Ligatures=Common,Scale=MatchUppercase]{SimSun}%pour le chinois
\newcommand{\zh}[1]{{\cn #1}}

\newcommand{\jg}[1]{\ipa{#1}\index{Japhug #1}}
\newcommand{\wav}[1]{#1.wav}
\newcommand{\tgz}[1]{\mo{#1} \tg{#1}}


 
\begin{document}
 \sloppy
\section*{{\LARGE Curriculum Vitae}}
Guillaume Jacques, Directeur de recherche 2, CNRS (CRLAO, UMR 8563).  Né le 16 septembre 1979 à Paris, marié, un fils.
\sloppy
\section*{Compétences et centres d’intérêt}
Documentation de langues en danger (japhug, khaling, stau), linguistique historique (sino-tibétain, indo-européen, sémitique, algonquien, sioux), typologie morphosyntaxique (incorporation, voix et valence, systèmes d'indexation complexes, pivots syntaxiques), linguistique panchronique (principes généraux des changements phonétiques, de la grammaticalisation et des changements analogiques), morphologie computationnelle.


\section*{Parcours professionnel}
\begin{itemize}
\item 2009-2015 Chargé de recherche au CNRS (CR1), Centre de recherches linguistiques sur l'Asie Orientale (UMR 8563)
\item 2005-2009 Maître de conférences, Université Paris Descartes, département de sciences du langage.
\end{itemize}
 
\section*{Activités éditoriales}
\begin{itemize}
\item Rédacteur en chef de la revue \textit{Cahiers de Linguistique -- Asie orientale} (Brill, depuis 2013) et futur responsable de la linguistique historique à partir de 2015 pour la revue \textit{Linguistic Vanguard} (Mouton de Gruyter)
\item membre du comité éditorial de \textit{Diachronica} (Benjamins, depuis 2008) et de \textit{Linguistics of the Tibeto-Burman Area}  (Benjamins, depuis 2014).
\item Relecteur pour les revues suivantes (en plus des précédentes): \textit{Lingua}, \textit{Linguistic Typology}, \textit{Studies in Language}, \textit{Folia Linguistica}, \textit{Journal of the International Phonetic Alphabet},  \textit{Language and Linguistics}, \textit{Transactions of the Philological Society}, \textit{Journal of Chinese Linguistics}, \textit{SKY journal of linguistics}, \textit{Langages}, \textit{Yuyanxue luncong} \zh{语言学论丛}
\end{itemize}

\section*{Thèses soutenues}
\begin{itemize}
\item 2015. Gao Yang, EHESS, \textit{Description de la langue muya} (Sino-tibétain, birmo-qianguique, en co-direction avec Laurent Sagart). Mention \textit{très honorable avec félicitation du jury}.
\end{itemize}

\section*{Thèses en cours}
\begin{itemize}
\item 2013-présent. Gong Xun, Normale supérieure-INALCO, \textit{Etude descriptive et historique de la langue zbu} (Sino-tibétain, birmo-qianguique)
\item 2013-présent. Lai Yunfan, Paris 3, \textit{Grammaire du Kroskyabs} (Sino-tibétain, birmo-qianguique,en co-direction avec Pollet Samvelian)
\end{itemize}
\section*{Projets de recherche}
\begin{itemize}
\item 2008-2012: Projet ANR \textbf{PASQi} (avec Katia Chirkova (porteur) et Alexis Michaud) 
\item  2013-2015: Projet ANR Corpus \textbf{HimalCo} (porteur du projet; avec Alexis Michaud, Aimée Lahaussois et Séverine Guillaume) (\url{http://himalco.hypotheses.org/})
\item 2010-présent: Responsable de l'opération PPC2-\textit{Evolutionary approaches to phonology} et de l'opération LR4.11-\textit{Automatic paradigm generation and language description} dans le cadre du Labex \textit{Empirical Foundations of Linguistics}.(\url{http://www.labex-efl.org/?q=fr/recherche/axe6})
\end{itemize}
 
     
\section*{Invitations à l'étranger}
\begin{itemize}
\item   2010 (janvier-juin) Visiting scholar, Research Centre for Linguistic Typology, LaTrobe University, Melbourne, Australie.
 \item   \textbf{séminaires invités}:  Université de Genève (2003),  Université d'Oxford (2009), School of Oriental and African Studies, Londres (2011), Université de Berne (2012),  Université de Zürich (2013).
 \item \textbf{conférences invitées}: Université Fudan, Shanghai (2005), Academia Sinica, Taiwan (2008, 2009 et 2010), Musée d'Ethnologie, Osaka (2009), School of Oriental and African Studies, Londres (2013), Université de Washington à Seattle (2013),  Université de Cambridge (2014).
  \end{itemize}
    \bibliographystyle{unified}
  \nobibliography{bibliogj.bib}
\section*{Publications}

    \begin{enumerate}
    \item Monographies: Brill (1), Minzu chubanshe (1) ; Participations à des volumes collectifs: Mouton de Gruyter (1), Brill (3)
\item Revues A : \textit{Linguistic Inquiry} (1), \textit{Lingua} (3), \textit{Studies in Language} (1), \textit{Linguistic Typology} (1), \textit{Diachronica} (2), \textit{Anthropological Linguistics} (1), \textit{Bulletin of the School of Oriental and African Studies} (3),  \textit{Cahiers de linguistique d’Asie Orientale} (3), \textit{Etudes mongoles et sibériennes, centrasiatiques et tibétaines} (1), \textit{Journal of Chinese Linguistics} (3)
\item Revues B : \textit{Folia Linguistica Historica} (2), \textit{Transactions of the Philological Society} (1), \textit{Journal of the American Oriental Society} (1), \textit{Linguistics of the Tibeto-Burman Area} (4), \textit{Minzu yuwen} (3), \textit{Amerindia} (1), \textit{Faits de langues} (1), \textit{La linguistique} (1), \textit{Central Asiatic Journal} (1)
\item Revues C : \textit{Language and Linguistics Compass} (1) 
\item Autre revues :   \textit{Language and Linguistics} (5; Q2 dans Scimago),   \textit{Studia Etymologica Cracoviensia} (1), \textit{Revue d'études tibétaines} (2)
\end{enumerate}

    \subsection*{5 publications principales}

      \begin{enumerate}
\item {\bibentry{jacques14antipassive}}
\item  \bibentry{jacques14esquisse}
\item {\bibentry{jacques14auditory}}
\item {\bibentry{antonov14need}}
\item {\bibentry{jacques13harmonization}}
  \end{enumerate}
  
     \subsection*{5 publications les plus récentes}
           \begin{enumerate}
 \item {\bibentry{jacques15khaling}}
\item  \bibentry{jacques15japhug}
\item {\bibentry{jacques15causative}}
\item {\bibentry{jacques15spontaneous}}
\item {\bibentry{jacques15sr}}
  \end{enumerate}
  
\end{document}