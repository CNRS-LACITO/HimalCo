\documentclass[oneside,a4paper,11pt]{article} 
\usepackage{fontspec}
\usepackage{natbib}
\usepackage{booktabs}
\usepackage{xltxtra} 
\usepackage{polyglossia} 
\usepackage[table]{xcolor}
\usepackage{gb4e} 
\usepackage{multicol}
\usepackage{graphicx}
\usepackage{float}
\usepackage{hyperref} 
\hypersetup{bookmarks=false,bookmarksnumbered,bookmarksopenlevel=5,bookmarksdepth=5,xetex,colorlinks=true,linkcolor=blue,citecolor=blue}
\usepackage[all]{hypcap}
\usepackage{memhfixc}
%\usepackage{lscape}
\usepackage{amssymb}
 
\usepackage{tangutex2}
%\setmainfont[Mapping=tex-text,Numbers=OldStyle,Ligatures=Common]{Charis SIL} 
\newfontfamily\phon[Mapping=tex-text,Ligatures=Common,Scale=MatchLowercase]{Charis SIL} 
\newcommand{\ipa}[1]{{\phon\textbf{#1}}} 
\newcommand{\grise}[1]{\cellcolor{lightgray}\textbf{#1}}
\newfontfamily\cn[Mapping=tex-text,Ligatures=Common,Scale=MatchUppercase]{SimSun}%pour le chinois
\newcommand{\zh}[1]{{\cn #1}}
\newcommand{\Y}{\Checkmark} 
\newcommand{\N}{} 
\newcommand{\dhatu}[2]{|\ipa{#1}| `#2'}
\newcommand{\jpg}[2]{\ipa{#1} `#2'}  
\newcommand{\refb}[1]{(\ref{#1})}
\newcommand{\tld}{\textasciitilde{}}

\newcommand{\sg}{\textsc{sg}}
\newcommand{\du}{\textsc{du}}
\newcommand{\pl}{\textsc{pl}}

 \begin{document} 
 
 \title{Tangut as a West Rgyalrongic language}
  \author{Guillaume Jacques, Lai Yunfan and Gong Xun}
 \maketitle 
 
\section*{Introduction}

\citet{jackson00sidaba, jackson00puxi}

\section{Lexicon} 
\citet{jacques14esquisse}
\subsection{Rgyalrongic innovations}

\subsection{West-Rgyalrongic innovations}

The verb \mo{5449} \ipa{tjị¹} `to put' (pre-Tangut *S-tja, stem B \mo{5633} \ipa{tjọ¹}) finds an exact cognate in Khroskyabs \ipa{stî} (\citealt{lai17khroskyabs}). This verb is related to a root widespread in the TH family, but in other languages of the family, it has a simple stop initial, as in Japhug \ipa{ta} `put'. The presence of a \ipa{st-} cluster in Khroskyabs, and of tense voice in Tangut, indicative of a *S-t- cluster (\citealt{gong99jinyuanyin}). This unexpected \ipa{s-} must be a lexicalized prefix. A potential explanation for it would be to analyze it as a trace of the cognate of the translocative associated motion prefix \ipa{ɕɯ-} prefix found in Japhug and other modern Core Rgyalrong languages (see \citealt{jacques13harmonization}). In Japhug texts, this prefix does occur in direct contact with the verb root in the Non-Past Factual form, as in (\ref{ex:CWtanW}).

\begin{exe}
\ex \label{ex:CWtanW}
\gll \ipa{nɯtɕu} 	\ipa{fsaŋ} 	\ipa{ɕɯ-ta-nɯ} 	\ipa{ra.}  \\
 there fumigation \textsc{transloc}-put:\textsc{fact-pl} have.to:\textsc{fact} \\
 \glt `They have to go and put juniper fumigation offerings there.' (hist140522 Kamnyu zgo, 298)
\end{exe}

This form being relatively common, it is possible that the translocative prefix was reanalyzed as part of the root. Other possibilities could be taken into consideration (an incorporated noun for instance, as this phenomenon is well-attested in Khroskyabs and Tangut, see \citealt{jacques11tangut.verb}), but in any case, this verb is a good example of innovative feature that is unlikely to have independently arisen twice.

\section{Case marking} 
\citet{jacques17stau}

\begin{table}[H]
\caption{Case markers in Stau, Khroskyabs and Tangut}\label{tab:tangut} \centering
\begin{tabular}{ll|ll|llllll}
\toprule
Stau && Khroskyabs && Tangut & \\
\midrule
\ipa{-w} & \textsc{erg} &&& \mo{5880} \ipa{ŋwu²} & \textsc{instr} \\
\ipa{-j} & \textsc{gen} &\ipa{-ji} &\textsc{gen} &\mo{1139} \ipa{.jij¹} & \textsc{gen}, antiergative\\
\ipa{-ʁa} & \textsc{all} & \ipa{-ʁɑ} & \textsc{loc} & \mo{5856} \ipa{ɣa²} & \textsc{loc} \\
\ipa{-tɕʰa} & \textsc{loc} &&& \mo{0089} \ipa{tɕʰjaa¹}  &\textsc{loc} \\
\ipa{-kʰa} & \textsc{instr} &&& \mo{5993} \ipa{kʰa¹}  &in the middle of \\
\toprule
\end{tabular}
\end{table}

\section{Verbal morphology} 

\subsection{Inverse marking and stem alternations} \label{}

\subsubsection{Western Rgyalrongic vs Core Rgyalrong}
\citet{jackson02rentongdengdi}, \citet{jacques10inverse}, \citet{gongxun14agreement}, \citet{lai15person}, 
\citet{jackson00sidaba, jackson00puxi}


\subsubsection{Tangut stem B and stem III}
\citet{gong16stems}, \citet{kepping85}
\citet{gong01huying}

\begin{table}[H]
\caption{Attested forms of the ditransitive paradigm in Tangut}\centering  \label{tab:paradigm}
\begin{tabular}{lllll}
\toprule
	&	1\sg{}	&	2\sg{}	&	1/2\pl{}	&	3	\\
	\midrule
1\sg{}	&	?	&	A-\ipa{nja²}	&	?	&	 B-\ipa{ŋa²}	\\
2\sg{}	&	A-\ipa{ŋa²}	&	B-\ipa{nja²}	&	A-\ipa{nji²}	&	 B-\ipa{nja²}	\\
1/2\pl{}	&	 A-\ipa{ŋa²}	& ?	&	?	&	A-\ipa{nji²}	\\
3	&	A-\ipa{ŋa²}	&	A-\ipa{nja²}	&	?	&	A 	\\
\bottomrule
\end{tabular}
\end{table}

\subsection{The status of person indexation suffixes}
\citet{lapolla92}
\citet{kepping85}
\citet{jacques10inverse}
\citet{jacques16th}

\citet{jouon06}
\begin{table}[H]
\caption{Pronouns and person suffixes in Tangut (\citealt{kepping75agreement, kepping85})}\label{tab:pronoms.suffixes} \centering
\begin{tabular}{llllll} 
\toprule
\multicolumn{3}{c}{Pronoun} &\multicolumn{3}{c}{Suffix} \\
\midrule
\mo{2098} & \ipa{ŋa²}  & 1\textsc{sg} & \mo{2098} & \ipa{ŋa²}  &1\textsc{sg} \\
\mo{3926} & \ipa{nja²} & 2\textsc{sg} & \mo{4601} & \ipa{nja²} &2\textsc{sg} \\
\mo{4028} &  \ipa{nji²} & 2\textsc{sg}  honorific or 2\textsc{pl} & \mo{4884} & \ipa{nji²} & 1\textsc{pl} and 2\textsc{pl} \\
\bottomrule
\end{tabular}
\end{table}
 

\subsection{Aspiration alternation}


\subsection{Orientation prefixes}

\subsection{TAM suffixes}
\mo{0734} \ipa{mo²} 

\mo{3916} \ipa{sji²}  nmlz, evd

\section*{Conclusion}
 

\bibliographystyle{unified}
\bibliography{bibliogj}

 \end{document}
 