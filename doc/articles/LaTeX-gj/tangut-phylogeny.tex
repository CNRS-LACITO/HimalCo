\documentclass[oneside,a4paper,11pt]{article} 
\usepackage{fontspec}
\usepackage{natbib}
\usepackage{booktabs}
\usepackage{xltxtra} 
\usepackage{polyglossia} 
\usepackage[table]{xcolor}
\usepackage{gb4e} 
\usepackage{multicol}
\usepackage{graphicx}
\usepackage{float}
\usepackage{hyperref} 
\hypersetup{bookmarks=false,bookmarksnumbered,bookmarksopenlevel=5,bookmarksdepth=5,xetex,colorlinks=true,linkcolor=blue,citecolor=blue}
\usepackage[all]{hypcap}
\usepackage{memhfixc}
%\usepackage{lscape}
\usepackage{amssymb}
 
\usepackage{tangutex2}
%\setmainfont[Mapping=tex-text,Numbers=OldStyle,Ligatures=Common]{Charis SIL} 
\newfontfamily\phon[Mapping=tex-text,Ligatures=Common,Scale=MatchLowercase]{Charis SIL} 
\newcommand{\ipa}[1]{{\phon\textbf{#1}}} 
\newcommand{\grise}[1]{\cellcolor{lightgray}\textbf{#1}}
\newfontfamily\cn[Mapping=tex-text,Ligatures=Common,Scale=MatchUppercase]{SimSun}%pour le chinois
\newcommand{\zh}[1]{{\cn #1}}
\newcommand{\Y}{\Checkmark} 
\newcommand{\N}{} 
\newcommand{\dhatu}[2]{|\ipa{#1}| `#2'}
\newcommand{\jpg}[2]{J. \ipa{#1} `#2'}  
\newcommand{\wobzi}[2]{K. \ipa{#1} `#2'}  
\newcommand{\stau}[2]{S. \ipa{#1} `#2'}  
\newcommand{\tangut}[3]{\mo{#1}^{#1} \ipa{#2} `#3'}  
\newcommand{\refb}[1]{(\ref{#1})}
\newcommand{\tld}{\textasciitilde{}}

\newcommand{\sg}{\textsc{sg}}
\newcommand{\du}{\textsc{du}}
\newcommand{\pl}{\textsc{pl}}

 \begin{document} 
 
 \title{Tangut as a West Rgyalrongic language}
%  \author{Guillaume Jacques, Lai Yunfan and Gong Xun}
 \maketitle 
 
\section*{Introduction}

\citet{jackson00sidaba, jackson00puxi}

\section{Lexicon} 
\citet{jacques14esquisse}
\subsection{Rgyalrongic lexical commonalities}

\begin{enumerate}
\item \tangut{4425}{ɕju¹}{cool}, \jpg{ɣɤɕu}{cool} 
\item \tangut{0328}{ku¹}{blind}, \jpg{ɕquwa}{blind} 
\item \tangut{4600}{ŋwụ¹}{oath}, \jpg{kɯjŋu}{oath} 
\item \tangut{2795}{rur¹}{pasture}, \jpg{rɯŋgu}{pasture} 
\item \tangut{5165}{twụ¹}{place}, \jpg{ɯ-stu}{place, direction}. In Japhug, this noun is a lexicalized oblique participle from the verb \jpg{tu}{exist}. Note that in Tangut \mo{5165} \ipa{twụ¹} is completely opaque, as the corresponding existential verb \tangut{0930}{dju¹}{exist} has a voiced initial (like Situ XXX). While the alternation between the two alternating forms (unvoiced unaspirated vs prenasalized) is not explained, Tangut shows that both forms existed in proto-Rgyalrongic.
\item \tangut{4557}{zjur²}{torch}, \jpg{tɤtʂu}{lamp} 
\item \tangut{5149}{duu¹}{accumulate}, \jpg{ajtɯ}{accumulate} 
\item \tangut{1464}{tsur¹}{kick}, \jpg{tɯ-qartsɯ}{kick}
\item \tangut{1638}{gji¹}{clear (water)}, \jpg{amgri}{clear (water)} 
\item \tangut{1321}{zjị¹}{shoe}, \jpg{tɯ-xtsa}{shoe} \wobzi{jzî}{shoe}
\item \tangut{0385}{.wjị²}{be able}, \jpg{spa}{be able} 
\item \tangut{4966}{.wẹ¹}{rust}, \jpg{sɣa}{rust} 
\item \tangut{0439}{ɣiẹ¹}{cook}, \jpg{sqa}{cook}
\item \tangut{3596}{ɣiwe¹}{power}, \jpg{βʁa}{win}  
\item \tangut{4092}{khie¹}{hate}, \jpg{qʰa}{hate}
\item \tangut{1195}{khie²}{yak}, \jpg{qra}{female yak}
\item \tangut{5957}{tser¹}{sell}, \jpg{ntsɣe}{sell}
\item \tangut{5356}{tjịj¹}{alone}, \jpg{-sti}{alone} 
\item \tangut{1670}{sjwij¹}{whet}, \jpg{fse}{whet, sharpen} 
\item \tangut{3574}{tsjij²}{understand}, \jpg{tso}{understand} 
\item \tangut{0045}{zar¹}{spicy}, \jpg{mɤrtsaβ}{spicy} 
\item \tangut{0527}{.wjạ¹}{goiter}, \jpg{ɯ-zbɤβ}{goiter}. Lexicalization
\item \tangut{1894}{.jar¹}{wife}, \jpg{tɤ-rʑaβ}{wife} 
\item \tangut{0596}{ndzja¹}{grow}, \jpg{ndzɤt}{grow bigger} 
\item \tangut{4052}{dạ²}{cold}, \jpg{mɯɕtaʁ}{cold} 
\item \tangut{3547}{lia²}{drunk}, cf XXX zbu
\item \tangut{4680}{khia²}{ploughshare}, \jpg{qraʁ}{ploughshare} 
\item \tangut{5528}{bar¹}{drum}, \jpg{tɤ-rmbɣo}{drum} 
\item \tangut{1254}{dźjwɨr¹}{mill, grindstone}, \jpg{ɣndʑɯr}{grind} \stau{ɣdʑʚ}{grind}
\item \tangut{5570}{ŋwə¹}{be drowsy}, \jpg{nɯndzɯlŋɯz}{dose off} 
\item \tangut{0099}{thjwɨ¹}{finish}, \jpg{sthɯt}{finish}
\item \tangut{0074}{khwə¹}{half}, \jpg{ɯ-qiɯ}{half}
\item \tangut{4796}{zjɨr¹}{south}, \jpg{zrɯ}{sunny side of the mountain}
\item \tangut{5120}{swew¹}{bright}, \jpg{fsoʁ}{bright} 
\item \tangut{3299}{lwew¹}{steam}, \jpg{tɤ-jlɤβ}{steam}
\item \tangut{0148}{tọ¹}{congeal}, \jpg{stɤm}{congeal}
\item \tangut{2005}{tśior¹}{mud}, \jpg{tɤrcoʁ}{mud}
\item \tangut{2857}{ŋo²}{illness}, \jpg{tɤŋɤm}{disease} 
\end{enumerate}

\subsection{West-Rgyalrongic lexical commonalities}
\begin{enumerate}
\item \tangut{1278}{.jɨ²}{say}, \stau{jə}{say}
\item \tangut{0756}{dźju²}{meet}, \stau{dʑə}{meet}
\item \tangut{0020}{tśja¹}{road}, \stau{tɕe}{road}, \wobzi{xxx}{} construct state
\item  The verb \tangut{5449}{tjị¹}{to put} (pre-Tangut *S-tja, stem B \mo{5633} \ipa{tjọ¹}) finds an exact cognate in Khroskyabs \ipa{stî} (\citealt{lai17khroskyabs}). This verb is related to a root widespread in the TH family, but in other languages of the family, it has a simple stop initial, as in Japhug \ipa{ta} `put'. The presence of a \ipa{st-} cluster in Khroskyabs, and of tense voice in Tangut, indicative of a *S-t- cluster (\citealt{gong99jinyuanyin}). This unexpected \ipa{s-} must be a lexicalized prefix. A potential explanation for it would be to analyze it as a trace of the cognate of the translocative associated motion prefix \ipa{ɕɯ-} prefix found in Japhug and other modern Core Rgyalrong languages (see \citealt{jacques13harmonization}). In Japhug texts, this prefix does occur in direct contact with the verb root in the Non-Past Factual form, as in (\ref{ex:CWtanW}).

\begin{exe}
\ex \label{ex:CWtanW}
\gll \ipa{nɯtɕu} 	\ipa{fsaŋ} 	\ipa{ɕɯ-ta-nɯ} 	\ipa{ra.}  \\
 there fumigation \textsc{transloc}-put:\textsc{fact-pl} have.to:\textsc{fact} \\
 \glt `They have to go and put juniper fumigation offerings there.' (hist140522 Kamnyu zgo, 298)
\end{exe}

This form being relatively common, it is possible that the translocative prefix was reanalyzed as part of the root. Other possibilities could be taken into consideration (an incorporated noun for instance, as this phenomenon is well-attested in Khroskyabs and Tangut, see \citealt{jacques11tangut.verb}), but in any case, this verb is a good example of innovative feature that is unlikely to have independently arisen twice.
\end{enumerate}
\section{Case marking} 
\citet{jacques17stau}

\begin{table}[H]
\caption{Case markers in Stau, Khroskyabs and Tangut}\label{tab:tangut} \centering
\begin{tabular}{ll|ll|llllll}
\toprule
Stau && Khroskyabs && Tangut & \\
\midrule
\ipa{-w} & \textsc{erg} &&& \mo{5880} \ipa{ŋwu²} & \textsc{instr} \\
\ipa{-j} & \textsc{gen} &\ipa{-ji} &\textsc{gen} &\mo{1139} \ipa{.jij¹} & \textsc{gen}, antiergative\\
\ipa{-ʁa} & \textsc{all} & \ipa{-ʁɑ} & \textsc{loc} & \mo{5856} \ipa{ɣa²} & \textsc{loc} \\
\ipa{-tɕʰa} & \textsc{loc} &&& \mo{0089} \ipa{tśhjaa¹}  &\textsc{loc} \\
\ipa{-kʰa} & \textsc{instr} &&& \mo{5993} \ipa{kʰa¹}  &in the middle of \\
\toprule
\end{tabular}
\end{table}

\section{Verbal morphology} 

\subsection{Inverse marking and stem alternations} \label{}

\subsubsection{Western Rgyalrongic vs Core Rgyalrong}
\citet{jackson02rentongdengdi}, \citet{jacques10inverse}, \citet{gongxun14agreement}, \citet{lai15person}, 
\citet{jackson00sidaba, jackson00puxi}


\subsubsection{Tangut stem B and stem III}
\citet{gong16stems}, \citet{kepping85}
\citet{gong01huying}

\begin{table}[H]
\caption{Attested forms of the ditransitive paradigm in Tangut}\centering  \label{tab:paradigm}
\begin{tabular}{lllll}
\toprule
	&	1\sg{}	&	2\sg{}	&	1/2\pl{}	&	3	\\
	\midrule
1\sg{}	&	?	&	A-\ipa{nja²}	&	?	&	 B-\ipa{ŋa²}	\\
2\sg{}	&	A-\ipa{ŋa²}	&	B-\ipa{nja²}	&	A-\ipa{nji²}	&	 B-\ipa{nja²}	\\
1/2\pl{}	&	 A-\ipa{ŋa²}	& ?	&	?	&	A-\ipa{nji²}	\\
3	&	A-\ipa{ŋa²}	&	A-\ipa{nja²}	&	?	&	A 	\\
\bottomrule
\end{tabular}
\end{table}

\subsection{The status of person indexation suffixes}
\citet{lapolla92}
\citet{kepping85}
\citet{jacques10inverse}
\citet{jacques16th}

\citet{jouon06}
\begin{table}[H]
\caption{Pronouns and person suffixes in Tangut (\citealt{kepping75agreement, kepping85})}\label{tab:pronoms.suffixes} \centering
\begin{tabular}{llllll} 
\toprule
\multicolumn{3}{c}{Pronoun} &\multicolumn{3}{c}{Suffix} \\
\midrule
\mo{2098} & \ipa{ŋa²}  & 1\textsc{sg} & \mo{2098} & \ipa{ŋa²}  &1\textsc{sg} \\
\mo{3926} & \ipa{nja²} & 2\textsc{sg} & \mo{4601} & \ipa{nja²} &2\textsc{sg} \\
\mo{4028} &  \ipa{nji²} & 2\textsc{sg}  honorific or 2\textsc{pl} & \mo{4884} & \ipa{nji²} & 1\textsc{pl} and 2\textsc{pl} \\
\bottomrule
\end{tabular}
\end{table}
 

\subsection{Aspiration alternation}


\subsection{Orientation prefixes}

\subsection{TAM suffixes}
\mo{0734} \ipa{mo²} 

\mo{3916} \ipa{sji²}  nmlz, evd

\section*{Conclusion}
 

\bibliographystyle{unified}
\bibliography{bibliogj}

 \end{document}
 