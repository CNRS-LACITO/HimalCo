\documentclass[oldfontcommands,oneside,a4paper,11pt]{article} 
\usepackage{fontspec}
\usepackage{natbib}
\usepackage{booktabs}
\usepackage{xltxtra} 
\usepackage{polyglossia} 
\usepackage[table]{xcolor}

\usepackage{multicol}
\usepackage{graphicx}
\usepackage{lineno}
\usepackage{float}
\usepackage{hyperref} 
\hypersetup{bookmarks=false,bookmarksnumbered,bookmarksopenlevel=5,bookmarksdepth=5,xetex,colorlinks=true,linkcolor=blue,citecolor=blue}
\usepackage[all]{hypcap}
\usepackage{memhfixc}
\usepackage{lscape}
\usepackage{tikz}
%
\usetikzlibrary{trees}
\usepackage{gb4e} 
\bibpunct[: ]{(}{)}{,}{a}{}{,}
 
%\setmainfont[Mapping=tex-text,Numbers=OldStyle,Ligatures=Common]{Charis SIL}  
\newfontfamily\phon[Mapping=tex-text,Ligatures=Common,Scale=MatchLowercase,FakeSlant=0.3]{Charis SIL} 
\newcommand{\ipa}[1]{{\phon #1}} %API tjs en italique
 
\newcommand{\grise}[1]{\cellcolor{lightgray}\textbf{#1}}
\newfontfamily\cn[Mapping=tex-text,Ligatures=Common,Scale=MatchUppercase]{MingLiU}%pour le chinois
\newcommand{\pform}[2]{\ipa{#1} $\leftarrow$ \ipa{*#2}}
\newcommand{\zh}[1]{{\cn #1}}
\newcommand{\zhc}[3]{{\cn #1} \pform{#2}{#3}}
    


\begin{document} 
\title{ST linguistic paleontology}
\author{Guillaume Jacques}
\maketitle

\section{Introduction}
\citet{sagart11rice}

\citet{bradley11crops}


data from \citet{xu83jingpo}, \citet{bodrgya}, \citet{jackson93}
  \section{Domesticated animals} 
\begin{table}[H]
\begin{tabular}{llllll}
\toprule
&cow &  sheep & pig \\
\midrule
Chinese & \zhc{牛}{ŋjuw}{ŋʷə}&  \zhc{羊}{jaŋ}{ɢaŋ} & ?\\
Japhug& \pform{nɯŋa}{*--ŋwa} & \pform{qaʑo}{--jaŋ} & \pform{paʁ}{paq}\\
Tibetan& X & \pform{gjaŋ}{kə-jaŋ} & \pform{pʰag}{pak}\\
\bottomrule
Jingpo &\ipa{ŋā}&& \\
Deng & & \ipa{kɯ̀ jǒŋ} \\
\end{tabular}
\end{table}
  
  
  \zhc{癢}{jaŋ^b}{Cə.ɢaŋʔ} `itch', \pform{gja}{kə-ja}, \pform{rɤʑa}{--ja}
  
  \zhc{祥}{zjaŋ}{s.ɢaŋʔ} `auspicious', \pform{gjaŋ}{kə-jaŋ}
  
  
\ipa{nɯ--} in \ipa{nɯŋa}, perhaps from the possessed noun \ipa{--nɯ} `breast, udder'.  
  
For pig,  \zhc{富}{pjuw^c}{pək-s} (\citealt[]{sagart11homeland})\footnote{Unrelated to \pform{pʰʲug-po}{pluk} `rich', an adjective derived from \ipa{lug} `sheep', see also \ipa{lhug-po} `rich' from the same word family.}


Yak
\zhc{犛}{maew}{mrˁə} or \pform{li}{rə} 

\pform{qambrɯ}{--mrə} `male yak'  \pform{ɴbri}{mri} `female yak'
  
  \section{Weaving technology}
  
\begin{table}[H]
\caption{Verb roots related to weaving in Sino-Tibetan languages}
\begin{tabular}{llllll}
\toprule
&to weave &  to spin  \\
\midrule
Chinese & \zhc{織}{tɕik}{tək}&  \zhc{紡}{pjaŋ^b}{paŋʔ} & \\
Japhug& \pform{taʁ}{taq} & \pform{pɣo}{paˠŋ} & \\
Tibetan& \pform{ɴtʰag, btags}{*tak} & \pform{ɴpʰaŋ}{N-paŋ} `spinning wheel'& \\
Jinghpo & \pform{dàʔ}{dàk}&\ipa{kabāŋ} `spinning wheel' \\
proto-Tani & &\ipa{*poŋ} `spindle', \citet[227]{jackson93}\\
\bottomrule
\end{tabular}
\end{table}

\zhc{織}{tɕi^c}{tək-s} 
\bibliographystyle{unified}
\bibliography{bibliogj}
\end{document}
