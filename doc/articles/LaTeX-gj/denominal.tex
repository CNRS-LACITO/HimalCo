\documentclass[oldfontcommands,oneside,a4paper,11pt]{article} 
\usepackage{fontspec}
\usepackage{natbib}
\usepackage{booktabs}
\usepackage{xltxtra} 
\usepackage{polyglossia} 
\usepackage[table]{xcolor}
\usepackage{gb4e} 
\usepackage{multicol}
\usepackage{graphicx}
\usepackage{float}
\usepackage{hyperref} 
\hypersetup{bookmarks=false,bookmarksnumbered,bookmarksopenlevel=5,bookmarksdepth=5,xetex,colorlinks=true,linkcolor=blue,citecolor=blue}
\usepackage[all]{hypcap}
\usepackage{memhfixc}
\usepackage{lscape}
\bibpunct[: ]{(}{)}{,}{a}{}{,}
 
%\setmainfont[Mapping=tex-text,Numbers=OldStyle,Ligatures=Common]{Charis SIL} 
\newfontfamily\phon[Mapping=tex-text,Ligatures=Common,Scale=MatchLowercase,FakeSlant=0.3]{Charis SIL} 
\newcommand{\ipa}[1]{{\phon \mbox{#1}}} %API tjs en italique
 \newcommand{\ipab}[1]{{\phon \mbox{#1}}} %API tjs en italique
\newcommand{\grise}[1]{\cellcolor{lightgray}\textbf{#1}}
\newfontfamily\cn[Mapping=tex-text,Ligatures=Common,Scale=MatchUppercase]{MingLiU}%pour le chinois
\newcommand{\zh}[1]{{\cn #1}}

 

 \begin{document} 
 \title{Denominal verbs : a panchronic account}
\author{Guillaume Jacques}
\maketitle
\sloppy

\section{Intro}


\section{Grammaticalization pathways involving denominal markers}

\citet{jacques14antipassive}
\citet{jacques15causative}

\citet{jacques13tropative}
\citet{jacques12incorp}




\section{The origin of denominal morphology}

\begin{exe}
\item Light verbs (do, send, give, become etc)
\item Morphemes often found on verbs + zero-derivation, reanalysis as denominal markers when zero-derivation disappears, by over-generalization  (\citealt{heath98hermit}) Siouan?
\item Generalization of the most common conjugation pattern to borrowings (verbs being borrowed first as nouns, \citealt{wohlgemuth09verbal}), Japanese demoru etc (Pellard)
\end{exe}
\section{Conclusion}
\bibliographystyle{unified}
\bibliography{bibliogj}

 \end{document}
 