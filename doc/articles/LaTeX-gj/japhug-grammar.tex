\documentclass[oldfontcommands,oneside,a4paper,11pt]{memoir} 
\RequirePackage{lineno} 
\usepackage{xunicode}%packages de base pour utiliser xetex
\usepackage{fontspec}
\usepackage{natbib}
\usepackage{booktabs}
\usepackage{xltxtra} 
\usepackage{longtable}
\usepackage{polyglossia} 
\usepackage[table]{xcolor}
\usepackage{gb4e} 
\usepackage{graphicx}
\usepackage{float}
\usepackage{hyperref} 
\hypersetup{bookmarks=false,bookmarksnumbered,bookmarksopenlevel=5,bookmarksdepth=5,xetex,colorlinks=true,linkcolor=blue,citecolor=blue}
\usepackage[all]{hypcap}
\usepackage{memhfixc}

\bibpunct[~: ]{(}{)}{,}{a}{}{,}
%%%%%%%%%quelques options de style%%%%%%%%
%\setsecheadstyle{\SingleSpacing\LARGE\scshape\raggedright\MakeLowercase}
%\setsubsecheadstyle{\SingleSpacing\Large\itshape\raggedright}
%\setsubsubsecheadstyle{\SingleSpacing\itshape\raggedright}
%\chapterstyle{veelo}
\setsecnumdepth{subsubsection}
%%%%%%%%%%%%%%%%%%%%%%%%%%%%%%%
\setmainfont[Mapping=tex-text,Numbers=OldStyle,Ligatures=Common]{Charis SIL} %ici on définit la police par défaut du texte

\newfontfamily\phon[Mapping=tex-text,Ligatures=Common,Scale=MatchLowercase,FakeSlant=0.3]{Charis SIL} 
\newcommand{\ipa}[1]{{\phon #1}} %API tjs en italique
\newcommand{\ipab}[1]{{\scriptsize \phon#1}} 
\newcommand{\grise}[1]{\cellcolor{lightgray}\textbf{#1}}
\newfontfamily\cn[Mapping=tex-text,Ligatures=Common,Scale=MatchUppercase]{MingLiU}%pour le chinois
\newcommand{\zh}[1]{{\cn #1}}

\newcommand{\jg}[1]{\ipa{#1}\index{Japhug #1}}
\newcommand{\wav}[1]{}%#1.wav}

\newcommand{\acc}{\textsc{acc}}
 \newcommand{\acaus}{\textsc{acaus}}
 \newcommand{\advers}{\textsc{advers}}
\newcommand{\apass}{\textsc{apass}}
\newcommand{\appl}{\textsc{appl}}
\newcommand{\allat}{\textsc{all}}
\newcommand{\aor}{\textsc{aor}}
\newcommand{\assert}{\textsc{assert}}
\newcommand{\auto}{\textsc{autoben}}
\newcommand{\caus}{\textsc{caus}}
\newcommand{\cl}{\textsc{cl}}
\newcommand{\cisl}{\textsc{cisl}}
\newcommand{\classif}{\textsc{class}}
\newcommand{\concsv}{\textsc{concsv}}
\newcommand{\comit}{\textsc{comit}}
\newcommand{\compl}{\textsc{compl}} %complementizer
\newcommand{\comptv}{\textsc{comptv}} %comparative
\newcommand{\cond}{\textsc{cond}}
\newcommand{\conj}{\textsc{conj}}
\newcommand{\coord}{\textsc{coord}}
\newcommand{\const}{\textsc{const}}
\newcommand{\conv}{\textsc{conv}}
\newcommand{\cop}{\textsc{cop}}
\newcommand{\dat}{\textsc{dat}}
\newcommand{\dem}{\textsc{dem}}
\newcommand{\degr}{\textsc{degr}}
\newcommand{\deexp}{\textsc{deexp}}
\newcommand{\dist}{\textsc{dist}}
\newcommand{\du}{\textsc{du}}
\newcommand{\duposs}{\textsc{du.poss}}
\newcommand{\dur}{\textsc{dur}}
\newcommand{\erg}{\textsc{erg}}
\newcommand{\emphat}{\textsc{emph}}
\newcommand{\evd}{\textsc{evd}}
\newcommand{\fut}{\textsc{fut}}
\newcommand{\gen}{\textsc{gen}}
\newcommand{\genr}{\textsc{genr}}
\newcommand{\hort}{\textsc{hort}}
\newcommand{\hypot}{\textsc{hyp}}
\newcommand{\ideo}{\textsc{ideo}}
\newcommand{\imp}{\textsc{imp}}
\newcommand{\indef}{\textsc{indef}}
\newcommand{\inftv}{\textsc{inf}}
\newcommand{\instr}{\textsc{instr}}
\newcommand{\intens}{\textsc{intens}}
\newcommand{\intrg}{\textsc{intrg}}
\newcommand{\inv}{\textsc{inv}}
\newcommand{\ipf}{\textsc{ipf}}
\newcommand{\irr}{\textsc{irr}}
\newcommand{\loc}{\textsc{loc}}
\newcommand{\med}{\textsc{med}}
\newcommand{\mir}{\textsc{mir}}
\newcommand{\negat}{\textsc{neg}}
\newcommand{\neu}{\textsc{neu}}
\newcommand{\nmlz}{\textsc{nmlz}}
\newcommand{\npst}{\textsc{n.pst}}
\newcommand{\pfv}{\textsc{pfv}}
\newcommand{\pl}{\textsc{pl}}
\newcommand{\plposs}{\textsc{pl.poss}}
\newcommand{\pass}{\textsc{pass}}
\newcommand{\poss}{\textsc{poss}}
\newcommand{\pot}{\textsc{pot}}
\newcommand{\pres}{\textsc{pres}}
\newcommand{\prohib}{\textsc{prohib}}
\newcommand{\prox}{\textsc{prox}}
\newcommand{\pst}{\textsc{pst}}
\newcommand{\qu}{\textsc{qu}}
\newcommand{\recip}{\textsc{recip}}
\newcommand{\redp}{\textsc{redp}}
\newcommand{\refl}{\textsc{refl}}
\newcommand{\sg}{\textsc{sg}}
\newcommand{\sgposs}{\textsc{sg.poss}}
\newcommand{\stat}{\textsc{stat}}
\newcommand{\topic}{\textsc{top}}
\newcommand{\volit}{\textsc{vol}}
\newcommand{\transloc}{\textsc{transl}}
\newcommand{\cisloc}{\textsc{cisl}}
\newcommand{\quind}{\textsc{qu.ind}} %revoir glose
\newcommand{\trop}{\textsc{trop}} 
 \newcommand{\abil}{\textsc{abil}}  
 \newcommand{\facil}{\textsc{facil}}  
 
\XeTeXlinebreaklocale "zh" %使用中文换行 
\XeTeXlinebreakskip = 0pt plus 1pt % 
\makeindex 
\begin{document}
\linenumbers

\chapter{Introduction}


nɯ tukɯti ɲɯkhɯ

nɯnɯ kɤti rkɯn
\chapter{Parts of speech}
\chapter{Nominal morphology}
\section{Status constructus}

\section{Fossilized participles} \label{sec:fossil.participle}
Half-lexicalized: sɤcɯ


 \ipa{ndʑɤβ} ``to burn'' 
> \ipa{ɣndʑɤβ} ``devastating fire'', a noun with a Tibetan cognate \ipa{gzhob} ``burning smell'', 
ɣɲɟɯ from \ipa{ɲɟɯ} ``to be opened''

tɯrnɤɣɲɟɯ
tɯɕnɤɣɲɟɯ
\ipa{kʰɯɣɲɟɯ} ``chimney''   ``the smoke opening''.

cf \ref{subsub:anticaus.morph}

\section{Noun-noun}
khɯzɤpɯ

tɕhemɤpɯ
\section{Noun-verb}
tʂɤɕphɤt
phoŋsti
tɤkɤɣrum tɯku + ɣrum (*tɤŋgɤɣrum
zɲɟɤʑru
kɯngɯt tɤrqhɤɴɢaʁ 



\section{Verb-Noun}


ndzɤpri


\section{Measure}
xtɯrɲɟi

jpomxtshɯm

\section{Reduplication}
fsapaq	kɤ-χsu	mɯ́-j-mbat	ma	maka	a-ki	prɤku	rgali	stɯ-sti	kɯ	ʑo ji-tɯ-βɣi	lonba	ʑo,	nɯ	kɯ-kɯ-jndʐɤs	ʑo	tha-ɕkɯt	qhe,
conversation tarrdo 23-24


\chapter{Numerals and classifiers}


\begin{table}[H]
\caption{Numerals and numeral prefixes in Japhug} \centering
\begin{tabular}{lllllll}
\toprule
&Numeral & prefix \\
\midrule
1 &\ipa{ci} &\ipa{tɯ--} \\
2 &\ipa{ʁnɯz} &\ipa{ʁnɯ--} \\
3 &\ipa{χsɯm} &\ipa{χsɯ--} \\
4 &\ipa{kɯβde} &\ipa{kɯβde-} \\
5 &\ipa{kɯmŋu} &\ipa{kɯmŋu--} \\
6 &\ipa{kɯtʂɤɣ-} &\ipa{kɯtʂɤ--} \\
7 &\ipa{kɯɕnɯz} &\ipa{kɯɕnɯ--} \\
8 &\ipa{kɯrcat} &\ipa{kɯrcɤ-} \\
9 &\ipa{kɯngɯt} &\ipa{kɯngɯ--} \\
10 &\ipa{sqi} &\ipa{sqɯ--} \\
\bottomrule
\end{tabular}
\end{table}
tɕe ndʑi-rmi mɯ́j-naχtɕɯɣ ri, ci nɤ qaʑmbri, ci nɤ qafsa ɲɯ-rmi
qaʑmbri 78

tɯsla > kɤrɤsla
tɯxpa > kɤrɤxpa


time ordinals
\chapter{Postpositions} \label{chapt:postpositions}


\section{Ergative} \label{sec:erg}
with inanimate:
smi kɯ ɯrkɯ ɯsi nɯra kɤtɕɤβ pjɤsthɯt tɕɤn,
xuehaizi 19,

\subsection{Instrumental}






\subsection{Verbal complement}
tɯ-qioʁ	kɯ	tó-wɣ-sɯ-khɤt	ʑo
Gesar 266

> kɯ, only \jg{khɤt}
 aɕki tɯjʁo kɯ takhɤt ʑo 
 tama kɯ takhɤt ʑo
 
\subsection{Comitative}
ɯrte ra kɯ ʑlasɯβɟi tɕe
pearstory.12

ɯrte tondo tɕe josɯβɟinɯ 

\subsection{Expression of degree}
ɯmi kɯzri ɯthɯrʑi kɯ khɯna kɯ mɯpjɤ́wɣɕaβ (thème 44)
sɯmɯzdɯɣ kɯ ɕɤr ɯʑɯβ mucin mɯpɯɣe ɲɯŋu. 
Slobdpon 174
\subsection{Intransitive verbs}

\subsection{Across a verb}
li laʁnɤrʑaʁ totsu tɕe, nɯnɯ qaliaʁ ɯpɯ nɯ kɯ juɕe tɕe, tɕe li kɤnɯqambɯmbjom kospa
ziji qu ba 9


\subsection{With abstract nouns}
ɯxɕɤt kɯ
kɤti kɯ
ɯrʑaβ nɯzdɯɣ kɤti kɯ pɯsaχaʁ ʑo ɲɯŋu
Slobdpon 110

\section{Genitive} \label{sec:genitive}

Special forms \ref{sec:pronouns.gen}

Possession (see \ref{sec:possession})
\begin{exe}
\ex
\gll
\ipa{aʑɯɣ} 	\ipa{kɯ́nɤ} 	\ipa{a-mphrɯmɯ} 	\ipa{a-pɯ-tɯ-sɯ-re} 	\ipa{ɯ-tɯ́-cha}  \ipa{to-ti}  \\
me:\gen{} too 1\sg{}-divination  \irr{}-\pfv{}-\caus{}-do.divination[III] \qu{}-2-can \evd{}-say\\
  \glt He said, ``Can you also ask him (the monk) to do a divination for me? (The divination,  32)
\end{exe} 


Dative:

\begin{exe}
\ex
\gll
 \ipa{tɕeri}	\ipa{ɯʑɤɣ}	\ipa{tɤ-mda}	\ipa{tɕe,}  \\
  but she:\gen{} \aor{}-arrive \coord{} \\
  \glt But when her turn came, (The three sisters,  141)
\end{exe} 

ɴɢoɕna ɣɯ nɯ stoŋtsu,
qaɟɯ ɣɯ nɯ khrɯtsu tu tu-ti-nɯ ɲɯ-ŋu.
 Les araignées en ont mille, et les poissons dix mille
(117-118 26-mYaRmtsaR )
 
\section{Locative} \label{sec:loc}

\section{Relational nouns}
\begin{exe} 
 \ex 
\gll  \jg{rgɯnba} \jg{ɲɯ-rɤntɕhom-nɯ}  \\ 
 monastery \const{}-dance-\pl{} \\ 
 \glt  People dance in the monastery. (Dpalcan 2010, \wav{vi-rAntChom})
\end{exe} 


ɯ-tsa aussi???

tɯ-mi kɯ-kɯ-fse nɯ ɯ-tsa nɯnɯ tu-qrɯ-nɯ tɕe (314)
30-komAr 

"tú-ɣ-ndza	nɤ	mɤ-jɤɣ	nɤ	ma	nɤʑo	nɤ-ndʐa	pɯ-ɣi-a	ɕti	tɕe
benefactive

\section{time}
qachɣa nɯ pɯ-si, ɯʑo nɯ ɯ-qhu ɕanpɕi nɯ tɕu sɤtɕha kɤ-ɣi mɯ-nɯ-nɤs
Tiger 42

tɕendɤre	li	iʑora	jɯ-raŋ,	ŋgɯsqɯ-xpa	ɕɯŋgɯ	nɯ	ra	tɕheme	ra	uɣma	ʑo	kɯ-taq	pɯ-dɤn,
101 taXtsa



nɯ mɤɕtʂa nɯ kɤti mɯpɯmtshama
\wav{8_nWmACtsxa}
phone conversation, nov 2011


ɯ-mphru 一个接着一个
kɯrtsɤɣ kɯ-sɤɣmɯɣmu ʑo ci jo-ɣi tɕe, ɯqaʑo ci ɯ-mphru ci ʑo pjɤ-sat.
histoire 54

\section{also}

paʁ kɯ tɕi ndze, nɯŋa kɯ tɕi ndze, jla kɯ tɕi ndze.


\chapter{Pronouns and indefinites}
\section{Pronouns and possessive prefixes} \label{sec:pronouns}

The pronominal system of Japhug distinguishes singular, dual and plural, but does not present any inclusive / exclusive distinction as other languages such as Tshobdun (see \citealt{jackson98morphology}).


Alongside the free pronouns, a system of pronominal prefixes is used not only to express  possession on noun (see section \ref{sec:possession}  for an account of the possessive constructions), but also appears in various constructions in the verbal system. These prefixes do not distinguish the second and the third person in the dual and plural forms:


\begin{table}[H] \centering
\caption{Pronouns and possessive prefixes }\label{tab:pronoun}
\begin{tabular}{lllllllll} \toprule
 Free pronoun & Prefix & \\
\midrule
 \ipa{aʑo},    \ipa{ɤj} &	\ipa{a--}  &		1\sg{} \\
\ipa{nɤʑo},  \ipa{nɤj} &	\ipa{nɤ--}  &			2\sg{} \\
\ipa{ɯʑo}  &	\ipa{ɯ--}  &			3\sg{} \\
\ipa{tɕiʑo}  &	\ipa{tɕi--}  &			1\du{} \\
\ipa{ndʑiʑo}  &	\ipa{ndʑi--}  &		2\du{} \\	
\ipa{ʑɤni}  &	\ipa{ndʑ-i-}  &		3\du{} \\	
\ipa{iʑo}, \ipa{iʑora},   \ipa{iʑɤra}   &	\ipa{i--}  &			1\pl{} \\
\ipa{nɯʑo}, \ipa{nɯʑora},   \ipa{nɯʑɤra}  &	\ipa{nɯ--}  &			2\pl{} \\
\ipa{ʑara}  &	\ipa{nɯ--}  &			3\pl{} \\
\bottomrule
\end{tabular}
\end{table}Free pronouns and possessive prefixes are remarkably similar in Kamnyu Japhug. In the eastern Japhug dialects, a different 1\sg{} pronoun distinct from the possessive prefix  is used: \ipa{ŋa}. In the table above, we observe that all the pronouns except the third person dual and plural are formed by adding the root --\ipa{ʑo} to the corresponding possessive prefix. This root is that of the generic pronoun \ipa{tɯ-ʑo}, which appears mainly in gnomic statements:

tɯʑo tɯkɤsɯso nɯ tukɯnɯti ɕti ma
One says what he thinks
\ipa{8_tWZo}

\begin{exe}
\ex
\gll 
\ipa{tɯ-zda} \ipa{pjɯ́-wɣ-z-ɣɤtɕa,} \textbf{\ipa{tɯʑo}}  \ipa{ntsɯ}  \ipa{pjɯ-kɯ-ʑɣɤ-ɣɤŋgi}   	\ipa{tɕe,}  \ipa{pɯ-kɯ-nɯ-ɣɤtɕa} 	\ipa{kɯ́nɤ}   	\ipa{pjɯ-kɯ-ʑɣɤ-ɣɤŋgi}   	\ipa{tɕe,}    \ipa{ɯ-mbrɤzɯ}   	\ipa{kɯ-tu}   	\ipa{me}  	\ipa{tu-kɯ-ti}   	\ipa{ɲɯ-ŋu.}   \\
\neu{}-companion \ipf{}-\inv{}-\caus{}-be.wrong oneself always \ipf{}-\genr{}:S/O-\refl{}-be.right \coord{} \aor{}-\genr{}:S/O-\auto{}-be.wrong also \ipf{}-\genr{}:S/O-\refl{}-be.right \coord{} \3sg{};\poss{}-result \nmlz{}:\stat{}-have \npst{}:not.have \ipf{}-\genr-say \const{}-be \\
\glt  If one considers that one's companion is wrong, and always considers himself to be right even if one is wrong, there is can be no good result. (Mouse and sparrow, 80-82)
\end{exe} 


The first and second person singular pronouns  \ipa{aʑo} and \ipa{nɤʑo} also have the shorter monosyllabic forms \ipa{ɤj} and \ipa{nɤj} respectively. These short forms are considerably less common in stories (in the reported speech of the characters), but appear frequently in free conversations.


Third person pronouns can be used with inanimate referents, as the third person dual \ipa{ʑɤni} in example \ref{ex:rNgW}.
\begin{exe}
\ex \label{ex:rNgW}
\gll \ipa{tɕe}   	\ipa{rŋgɯ}   	\ipa{nɯ}   	\ipa{to-kɤmɯrpu-ndʑi-chɯ}   	\ipa{tɕe,}   	\ipa{tɕendɤre}   	\ipa{ʑɤni}   	\ipa{pjɤ-nɯ-ɴɢrɯ-ndʑi}   \\
\textsc{coord} boulder \textsc{top} \textsc{evd}-knock.together-\textsc{du-med} \textsc{coord} \textsc{coord} they^{d} \textsc{evd-auto}-crush-\textsc{du} \\
\glt The boulders collided and they were pulverized. (smanmi4.82-83)
\end{exe}




\section{Genitive forms} \label{sec:pronouns.gen}
The form of pronouns and personal prefixes undergoes very morphophonological changes in combination with postpositions and relational nouns. However, in combination with the genitive suffix \ipaipa{ɣɯ} (cf \ref{sec:genitive})  personal pronouns have the following special forms:


% (cf. \ref{sec:possession})


\begin{table}[H] \centering
\caption{Pronouns and possessive prefixes }\label{tab:pronoun}
\begin{tabular}{lllllllll} \toprule
 Free pronoun & Genitive & \\
\midrule
 \ipa{aʑo}  &	\ipa{aʑɯɣ}  &		1\sg{} \\ 
\ipa{nɤʑo}  &	\ipa{nɤʑɯɣ}  &			2\sg{} \\ 
\ipa{ɯʑo}  &	\ipa{ɯʑɤɣ}  &			3\sg{} \\ 
\ipa{tɕiʑo}  &	\ipa{tɕiʑɤɣ}  &			1\du{} \\ 
\ipa{ndʑiʑo}  &	\ipa{ndʑiʑɤɣ}  &		2\du{} \\	 
\ipa{ʑɤni}  &	\ipa{ʑɤniɣɯ}  &		3\du{} \\	 
\ipa{iʑo}  &	\ipa{iʑɤɣ}, 	\ipa{iʑɤra ɣɯ}   &			1\pl{} \\ 
\ipa{nɯʑo}  &	\ipa{nɯʑɤɣ}, 	\ipa{nɯʑɤra ɣɯ}  &			2\pl{} \\ 
\ipa{ʑara}  &	\ipa{ʑaraɣ},   \ipa{ʑara ɣɯ}&			3\pl{}  \\  
\bottomrule
\end{tabular}
\end{table}

While some degree of variation exists with dual and plural pronouns (for instance the regular \ipa{iʑo ɣɯ} is found alongside \ipa{iʑɤɣ} and \ipa{iʑɤra ɣɯ}), for the singular pronouns only one form is attested:

\begin{exe}
\ex
\gll \ipa{aʑɯɣ} 	\ipa{ndʐa} 	\ipa{ŋu} 	\ipa{ɕi,} 	\ipa{nɤʑɯɣ} 	\ipa{ndʐa} 	\ipa{ŋu,} 	\ipa{ɤj} 	\ipa{mɯ́j-tso-a}   \\
1\sg{}:\gen{} reason \npst{}:be \textsc{qu} 2\sg{}:\gen{} reason \npst{}:be 1\sg{} \negat{}:\const{}-understand-1\sg{}\\
\glt  I don't know if it is because of me, or because of you. (that the phone line is not working well) (phone conversation, 2011, \wav{8_ndzxa})
\end{exe} 

In the genitive forms of the pronouns, the vowel of the genitive marker is generally dropped, and the pronominal root--\ipa{ʑo} undergoes vowel change to --\ipa{ʑɯɣ} (in the case of first and second person) and --\ipa{ʑɤɣ} (in other forms). Note that \ipa{ʑaraɣ} is the only case of the rhyme /aɣ/ in Japhug.


\section{Interrogative pronouns}
The interrogative pronouns in Japhug include the following:
\begin{table}[H] \centering
\caption{Interrogative pronouns }\label{tab:interrog.pronoun}
\begin{tabular}{lllllllll} \toprule
Form & Meaning   \\
\midrule
\ipa{ɕu} & who \\
\ipa{tʰɤstɯɣ} & how many \\
\ipa{tʰɤjtɕu} & when \\
\ipa{ŋotɕu}, \ipa{ŋoj} & where \\
\ipa{tɕʰi} & what \\
\ipa{tɕʰindʐa} & why \\
\bottomrule
\end{tabular}
\end{table}
The interrogative  ``what'' varies considerably across Japhug dialects. In Kamnyu we find \ipa{tɕʰi}, visibly borrowed from Tibetan \ipa{chi}. Neighbouring dialects of Gdongbrgyad xiang have either \ipa{tsʰi} (in Mangi) or \ipa{tʰi} (in Rqaco), which represents the original Rgyalrongic root for this interrogative pronoun. The Eastern dialects of Gsardzong and Datshang have \ipa{xto} instead.

The interrogative \ipa{ɕu} ``who'' is used with a human referent. It can occur in all syntactic roles, including S,A and O:
\begin{exe}
\ex
\gll  \ipa{mɤ-ta-mbi} 	\ipa{nɤʑo} 	\ipa{qaɕpa} 	\ipa{ɕu} 	\ipa{kɯ} 	\ipa{tɯ́-wɣ-mbi}    \\
 \negat{}-1>2-\npst{}:give you frog who \erg{} 2-\inv{}-\npst{}:give \\
\glt We won't give her to you, who would give her to you, a frog?   (The frog, 09)
\end{exe} 

It can also be used with the genitive suffix, though without a special form like the personal pronouns:
\begin{exe}
\ex
\gll  \ipa{χawo!} 	\ipa{ki} 	\ipa{ɕu} 	\ipa{ɣɯ} 	\ipa{ku-nɯ-ŋu} 	\ipa{kɯɣe?}    \\
 \textsc{interj} this who \gen{} \pres{}-\auto{}-be  \textsc{part}:hoping \\
\glt Oh, who does this belong to? (The frog, 09)
\end{exe}  



To express the meaning ``how many'' or ``how much'', the most common pronoun is \ipa{thɤstɯɣ}, though the combination \ipa{tɕʰi jarmar} (see \ref{adv:degree} on the forms of this adverb) is also used with this meaning:


\begin{exe}
\ex
\gll  \ipa{zgo} 	\ipa{thɤstɯɣ} 	\ipa{ja-pɣaʁ-ndʑi,} 	\ipa{tɯ-ci} 	\ipa{tɕʰi} 	\ipa{jarma} 	\ipa{ja-pjɤl-ndʑi} 	\ipa{mɤ-xsi} 	\ipa{ma,}       \\
 mountain how.many \aor{}:3>3-turn.over-\du{} \neu{}-water what about \aor{}:3>3-go.around-\du{} \negat{}-\genr{}:know apart.from \\
\glt No one knows how many mountains they crossed, around how many rivers they went.  (The raven, 50)
\end{exe}  
  
To ask about money, only 	\ipa{thɤstɯɣ}  is found:

\begin{exe}
\ex
 \gll    \ipa{nɤʑo} 	\ipa{thɤstɯɣ} 	\ipa{tɯ-khɤm}    \\
 you how.much 2-\npst{}:give[III]  \\
\glt  How much do you give (Bargaining, 13)
\end{exe} 

	\ipa{tʰɤstɯɣ} can be used with classifiers: it assumes the conjunct form \ipa{tʰɤstɯ-}, and takes the place of the numeral prefix:
 \begin{exe}
\ex
 \gll   \ipa{phɤri} 	\ipa{ku-ru} 	\ipa{tɕe} 	\ipa{qaʑo} 	\ipa{kɯ} 	\ipa{ɯ-qe} 	\ipa{thɤstɯ-rdoʁ} 	\ipa{pa-lɤt} 	\ipa{mtɤm} 	\ipa{tu-kɯ-ti} 	\ipa{ɲɯ-ŋgrɤl}   \\
  on.the.other.side \ipf{}:east-look \coord{} sheep \erg{} 3\sg{}.\poss{}-dung how.many-\textsc{cl} \aor{}:3>3-throw \npst{}:see[III] \ipf{}-\genr{}-say \ipf{}-be.usually.the.case \\
\glt When it looks across the river, it can see how many dungs have been left by the sheep, it is said.  (Dictionary entry, eagle)
\end{exe} 


	\ipa{tʰɤstɯɣ} is also used to as about a length of time:

\begin{exe}
\ex
 \gll   \ipa{nɤʑo} 	\ipa{tɤrʑaʁ} 	\ipa{thɤstɯɣ} 	\ipa{jamar} 	\ipa{kɤ-βzjoz} 	\ipa{kɤ-tɯ-spa-t?}  \\
 you time how.much about \inftv{}-learn \aor{}-2-be.able-\pst{} \\
\glt   How long did you need to learn it? (el., Dpalcan)
\end{exe} 
 

To ask about the precise moment when an event took place, the pronoun \ipa{tʰɤjtɕu} ``when'' should be used instead:

\begin{exe}
\ex
\gll  \ipa{tʰɤjtɕu} 	\ipa{lɤ-tɯ-nɯɣe} 	\ipa{pɯ-ŋu} 	\ipa{ra} 	\ipa{nɤ?}    \\
 when \aor{}-2-come.back[II] \pst{}.\ipf{}-be \pl{} \textsc{part} \\
\glt  When did you come back home? (Tarrdo conversation, 01)
\end{exe} 

The interrogative \ipa{ŋotɕu} and its reduced form \ipa{ŋoj} can be used to ask either about a location or a direction:

\begin{exe}
\ex
\gll     \ipa{ŋotɕu} \ipa{ku-tɯ-rɤʑi}?   \\
  where \pres{}-2-stay \\
\glt Where are you? (Conversation, 2005)
\end{exe} 
 

 
\begin{exe}
\ex
\gll     \ipa{qala} \ipa{ŋoj} \ipa{nɯ-ari}  \\
  rabbit where \aor{}:west-go[II]\\
\glt Where did the rabbit go?  (The rabbit and the bear, 21)
\end{exe} 
 
 \ipa{ŋotɕu} also appears in the following idiomatic expression:
 
 \begin{exe}
\ex
\gll     \ipa{kɯki} 	\ipa{ŋotɕu} 	\ipa{ɲɯ-ŋgrɤl?}   \\
 this where \ipf{}-be.usually.the.case \\
\glt How could this be possible?  (The raven, 32)
\end{exe} 

This sentence is used to express indignation (as in Chinese \zh{哪有这样的道理?}); in the story from which it is quoted, the husband says this sentence after his wife, quoting the words of a raven, says that she will have luck, not her husband. 

 \ipa{tɕʰi} is by far the most common of all interrogative pronouns, and can appear in a broad variety of contexts. It can be used to ask about a thing:


 \begin{exe}
\ex
\gll    \ipa{nɯnɯ} 	\ipa{tɕʰi} 	\ipa{pɯ-rmi} 	\ipa{kɯ?}       \\
that what \pst{}.\ipf{}-be.called \textsc{qu} \\
\glt  What was he called, again? (Gesar, 249)
\end{exe} 
 
 \ipa{tɕʰi} also occurs rhetorical questions with the irrealis (\ref{sub:irrealis}), in the meaning ``how could it be that...'':
 
 \begin{exe}
\ex
\gll   \ipa{tʰaʁɕa} 	\ipa{ʁo} 	\ipa{tɕʰi} 	\ipa{a-pɯ-rtaβ-a} 	\ipa{kɯ} \\
 weaving.comb \textsc{advers} what \irr{}-\pfv{}-hit-1\sg{} \textsc{qu} \\
\glt  How (could you imagine that) I would hit her with the weaving comb? (The bird, 61)
\end{exe} 

  \begin{exe}
\ex
\gll  \ipa{aʑo} 	\ipa{tɕhi} 	\ipa{a-pɯ-ŋu-a?} \\
1\sg{} what \irr{}-\ipf{}-be-1\sg{}  \\
\glt  How could it be me? (not to be interpreted as: ``What would I be?", The prince, 60)
\end{exe} 


To ask about manner, \ipa{tɕhi} is either combined with the stative verb \ipa{fse} ``to be in such a way, to be like'' or the transitive \ipa{stu} ``to do in such a way'':

   \begin{exe}
\ex
\gll  \ipa{iʑɤra} 	\ipa{ɣɯ} 	\ipa{tɕhi} 	\ipa{tu-fse} 	\ipa{tɕe} 	\ipa{ji-tɯ-ci} 	\ipa{ɣɤʑu} 	\ipa{tu-tɯ-the} 	\ipa{ɯ-tɯ́-cha}  \\
we \gen{} what  \ipf{}-be.in.such.a.way \coord{} 1\pl{}.\poss{}-\neu{}-water be.there.\textsc{sensory} \ipf{}-2-ask[III] \textsc{qu}-2-\npst{}:be  \\
\glt  Can you ask how can we have water? (The divination2, 14)
\end{exe} 
 
   \begin{exe}
\ex
\gll  \ipa{nɯnɯ} 	\ipa{ra} 	\ipa{kɯ} 	\ipa{tɕʰi} 	\ipa{a-tɤ-stu-nɯ} 	\ipa{tɕe} 	\ipa{nɯ-tɯ-ci} 	\ipa{ɣɤʑu}   \\
\dem{}.\textsc{distal} \pl{} \erg{} what \irr{}-\pfv{}-do.this.way-\pl{} \coord{} 3\pl{}.\poss{}-\neu{}-water be.there.\textsc{sensory}  \\
\glt  What should these (people) do so that they can have water? (The divination2, 38)
\end{exe} 

Alternatively to the construction above (with both verbs in non-finite forms) the verb \ipa{fse} can be used as the main predicate with a verbal complement:
   \begin{exe}
\ex
\gll  \ipa{kɤ-pʰɣo} \ipa{tɕʰi} \ipa{a-tɤ-fse-j}    \\
\inftv{}-flee what \irr{}-\pfv{}-be.in.such.a.way-1\pl{}   \\
\glt  How do we flee? (Slobdpon 69)
\end{exe} 

To ask about the reason of an action,  \ipa{tɕʰi} is combined with the noun \ipa{ɯ-ndʐa} ``reason'' as \ipa{tɕʰindʐa} ``why'':
  \begin{exe}
\ex
\gll    \ipa{ama} 	\ipa{a-pi} 	\ipa{khu} 	\ipa{tɕʰindʐa} 	\ipa{ku-tɯ-nɤpʰɯpʰɣo} 	\ipa{tɕe} 	\ipa{nɤʑo} 	\ipa{kɯ-fse} 	\ipa{kɯ-sɤɣ-mu} 	\ipa{me}   \\
\textsc{part}:surprise 1\sg{}.\poss{}-elder.sibling   tiger why \pres{}-2-\textsc{atelic}:flee \coord{} 2\sg{} \nmlz{}:\stat{}-be.in.such.a.way \nmlz{}:\stat{}-\textsc{deexp}-fear \npst{}:not.be.there \\
\glt Oh, elder brother Tiger, why are you running around, there is nothing as fearsome as you?  (The tiger, 26)
\end{exe} 

 

\section{Indefinite and definite marking}
Japhug does not have have genuine definite or indefinite articles, though and presents several strategies to mark definiteness. First, it is possible to use a noun with an indefinite or a definite marker. Second, Japhug has a series of indefinite pronouns, and interrogative pronouns can also be used as indefinites in some contexts. Third, non-referentiality or non-identifiability of an entity can be expressed by various syntactic constructions, including relative clauses and incorporation.

%identifiability+referentiality

\subsection{Indefinite and definite markers}
Like many languages (\citealt[130]{creissels06sgit1}), Japhug  uses bare nouns without any definiteness marking. Bare nouns are most often non-referential:

  \begin{exe}
\ex
\gll \ipa{ʁnaʁna} 		\ipa{tɕheme} 	\ipa{tɯ-tɤ-tu} 	\ipa{nɤ,} 	\ipa{kɤndʑɯsqhɤj} 	\ipa{tu-kɤ-sɯ-βzu} \\
both \emphat{} \gen{} 3\du{}.\poss{}-child girl \textsc{if}-\aor{}-have \textsc{if} group.of.sisters \ipf{}-\inftv{}-\caus{}-make \\
\glt If both of them have girls, let them be sisters. (zrɤntɕɯ tɯrme 4)
\end{exe}

They are  used in nominal predicates with a copula:

  \begin{exe}
\ex
\gll \ipa{aʑo} 	\ipa{tɕʰeme} 	\ipa{ɲɯ-ŋu-a}  \ipa{tɕe} \\
I girl \const{}-be-1\sg{} \coord{} \\
\glt I am a girl. (Nyima Vodzer 144)
\end{exe}

Bare nouns are rare with referential nouns (except in answers to questions), but examples can be found:

  \begin{exe}
\ex
\gll \ipa{qacʰɣa} 	\ipa{kɯ} 	\ipa{maχtɕɯ} \ipa{tɤ-tɯt-a} \ipa{nɯ} \ipa{mɤ-tɯ-ste} 	\ipa{ti} 	\ipa{ɲɯ-ŋu}  \\
fox \erg{} in.principle \aor{}-say[II]-1\sg{} \compl{} \negat{}-2-\npst{}:do.this.way[III] \npst{}:say \ipf{}-be \\
\glt The fox says: ``You do not do as I told you to." (The fox, 44)
\end{exe}

Personal names generally occur as bare nouns, without any definiteness marker (though, as we will see, these markers are not agrammatical with personal names):

  \begin{exe}
\ex
\gll  \ipa{ɯrɟɤnpanma} 	\ipa{kɯ} 	\ipa{ʁlaŋsaŋtɕhin} 	\ipa{ɯ-ɕki}  \\
 Padmasambhava \erg{} Gesar 3\sg{}-\textsc{dat} \\
\glt Padmasambhava (told) Gesar:
\end{exe}

Bare nouns are however relatively rare. Most nominal phrases contain either a demonstrative, an indefinite or a topic marker (cf chapter \ref{chapt:noun.phrase}). 


The numeral \ipa{ci} is used as an indefinite marker, placed at the end of the noun phrase:
  \begin{exe}
\ex
\gll \ipa{tɕʰeme} 	\ipa{kɯ-mpɕɯ-mpɕɤr} 	\ipa{ci} 	\ipa{ɲɤ-nɯɬoʁ} \\
girl \nmlz{}:\stat{}-\textsc{intens}-beautiful \textsc{indef}  \evd{}-appear \\
\glt A very beautiful girl appeared (out of it). (The flood, 39)
\end{exe}
\ipa{ci} must be used to introduce a new referent in a story.

On its own, \ipa{ci} can also serve as an indefinite pronoun, meaning ``one of (a group)'':
  \begin{exe}
\ex
\gll \ipa{ci} 	\ipa{ɣɯ} 	\ipa{tɤ-tɕɯ,} 	\ipa{ci} 	\ipa{ɣɯ} 	\ipa{tɕheme} 	\ipa{tɯ-tɤ-tu} 	\ipa{nɤ,} 	\ipa{ʁzɤmi} 	\ipa{ku-kɤ-sɯ-βzu} \\
\textsc{indef} \gen{} \neu{}-boy \textsc{indef} \gen{} girl \textsc{if}-\aor{}-have \textsc{if} husband.and.wife \ipf{}-\inftv{}-\caus{}-make \\
\glt If one of them has a boy, and the other one has a girl, let us make them husband and wife. (zrɤntɕɯ tɯrme 5)
\end{exe}


ci thɯ-kɯ-rgɯ-rgɤz ɲɯ-ɕti tɕe, ci kɯ-xtɕɯ-xtɕi ɲɯ-ɕti tɕe,
Nyima wodzer4.127

Japhug has no definite article. The distribution of the ubiquitous marker \textit{nɯ} is close to that of a definite marker, but since it does not normally appear in focalized noun phrases such as answers to questions,  it is more statisfying to treat it as a topic marker (see section \ref{sec:topic}).

The marker  \ipa{iɕqʰa} ``the aforementioned'' indicates both definiteness and topicality.\footnote{Used alone, \ipa{iɕqʰa} is a temporal adverb meaning ``just before'', see \ref{chapt:adv}.} It is used on referents that have been previously mentioned in the same story, usually only a few sentences back:
  \begin{exe}
\ex \label{ex:indef}
\gll \textbf{\ipa{``razri}} 	\textbf{\ipa{kɤtɯm}} 	\textbf{\ipa{ci}} 	\ipa{ɲɯ-ra,} 	\ipa{taqaβ} 	\ipa{ci} 	\ipa{ɲɯ-ra"} \ipa{to-ti} \ipa{qhe}   \\
 thread ball \textsc{indef} \const{}-need needle \textsc{indef} \const{}-need \evd{}-say \coord{}  \\
\glt He told (Rgyabza) ``I need a ball of thread and a needle''.
\ex \label{ex:icqha}
\gll \ipa{tɕendɤre} 	\ipa{ɲo-kho} 	\ipa{qhe,} 	\ipa{tɕe} 	\ipa{ɯ-ndzɤtshi} 	\ipa{kɤ-tsɯm} 	\ipa{nɯ} 	\ipa{tɕu} 	\ipa{qhe} 	\ipa{tɕe,} \textbf{\ipa{iɕqʰa}} 	\textbf{\ipa{kɤtɯm}} 	\textbf{\ipa{nɯ} }	\ipa{ɯʑo} 	\ipa{kɯ} 	\ipa{ko-ndo,} 	\ipa{taqaβ-rna} 	\ipa{nɯ} 	\ipa{ɲɤ-rku} \ipa{qhe,}  \\
 \coord{} \evd{}-give \coord{} \coord{} 3\sg{}.\poss{}-meal \inftv{}-bring \textsc{compl}  \loc{} \coord{} \coord{} the.aforementioned ball \topic{} he \erg{} \evd{}-take needle-ear \topic{} \evd{}put.in \coord{} \\
\glt She gave it to him. While (people) brought his meal, he took the ball of thread and put it into the ear of the needle. (Gesar 270-2)
\end{exe}
The referent ``ball of thread'', first introduced in sentence \ref{ex:indef}, appears again two sentences later with both the topic markers \ipa{nɯ} and \textit{iɕqʰa}. 

Unlike \ipa{nɯ}, \ipa{iɕqʰa} is relatively rare depending one the speaker and the type of discourse. It is possible to find stories long more than 80 sentences without any occurrence of \ipa{iɕqʰa}. 

There seems to be a limit to the number of sentences that can separate a noun phrase in \ipa{iɕqʰa} from its preceding occurrence (probably no more than five-six), but this topic deserves of systematic study based on all available stories.

 \ipa{iɕqʰa} as a definite marker not only occurs with nouns, but also with bare demonstratives such as \ipa{nɯnɯ} and with personal names:

  \begin{exe}
\ex
\gll  \ipa{tɕendɤre} 	\ipa{iɕqʰa} 	\ipa{ʁlaŋsaŋtɕʰin} 	\ipa{χsɯm} 	\ipa{ma} 	\ipa{mɯ-tɤ-kɯ-rʑaʁ} 	\ipa{nɯ,} \\
 \coord{} the.aforementioned Gesar three apart.from \negat{}-\aor{}-\nmlz{}:S-pass.day \topic{} \\
\glt Gesar, who was only three days old,  (Gesar 81)
\end{exe}

\subsection{Indefinitive pronouns}
 Japhug has some indefinite pronouns that are etymologically linked with corresponding interrogatives.


\begin{table}[H] \centering
\caption{Indefinite pronouns }\label{tab:indef.pronoun}
\begin{tabular}{lllllllll} \toprule
Form & Meaning   \\
\midrule
\ipa{tʰɯci}, \ipa{tʰɯtʰɤci} & something \\
\ipa{cɯscʰɯz} & somewhere \\
\ipa{tsʰitsuku} & whatever \\
\bottomrule
\end{tabular}
\end{table}


The most common 

\ipa{tʰɯci}, \ipa{tʰɯtʰɤci}



\ipa{cɯscʰɯz}

tshi tsu ku

\begin{exe} 
 \ex 
\gll   \jg{nɯnɯ} \jg{ŋotɕu} \jg{tɤ-tɯ-nɯ-tɕhɯ-ntɕhoz} \jg{khɯ} \\
this where \aor{}-2-\auto{}-\redp{}-use \npst{}:be.possible \\
 \glt You can use (this word) wherever you want. (Dpalcan 2010)
\end{exe} 

pɣɤtɕɯ	nɯ	sɯ-ku	nɯ	ra	tɕhi	kɤ-cha	ʑo	pjɤ-kra	tɕe,
29

ŋotɕu	tɯ-ʁjis	tɕhi	kɯ-ɣi	nɯ	tú-ɣ-nɯχtɯ	tɕe	ɲɯ-pe,
83 thaXtsa

135	tɯ-mgo	zmɤrɤβ	nɯ	tɕhi	pɯ-nɯ-ŋu-ŋu	ʑo	tú-ɣ-nɯ-βzu	jɤk	ma	khɯ,
tsampa


"tɕe mɤʑɯ thɯthɤci ʑo mɯ́jnaχtɕɯɣ kɯ akɤtɯχpjɤt ra" toti
yici bi yici you jinbu 17


A	70	tɕe	ɕnɤloq	nɯnɯ	reri	nɯ	tɯ-kɤ-taq	nɯ	lú-ɣ-βraq	qhe,	tɕe	ɕnɤloq	nɯ	cɯschɯs	kú-ɣ-βraq	qhe,
C	70	就把要织的带子套住在鼻圈上, 鼻圈随便拴在什么地方
A	83	tɕe	nɯnɯ	ra,	tɯ-...	ŋotɕu	tɯ-ʁjis	tɕhi	kɯ-ɣi	nɯ	tú-ɣ-nɯχtɯ	tɕe	ɲɯ-pe,
B	83
C	83
D	83	那些(线),你要想要什么线就可以买什么就好了,

ɕu pɯnɯŋɯŋu kɯ, aʑo amu mpɕɤr ɲɯsɯsɤm ɕti

ɕu pɯnɯŋɯŋu nɤ, tɕhi pɯnɯfsɯfse, ɯkɯnɤmpɕɤr tu ɕti
There is always someone who finds you beautiful
\ipa{8_WkWnAmpCAr}

tɯʑo mɯpɯ́wɣnɤmpɕɤr kɯ́nɤ, kɯmaʁ tɯrme ɯkɯnɤmpɕɤr tu ɕti
\ipa{8_WkWnAMpCAr2}



negative with maka
ɯʑo kɯ ta-tɯt maka kɯ-tu me
He said nothing 什么也没有说
\subsection{Competing constructions}

\subsubsection{Relatives}
nɤʑo	nɯ-nɯ-ɣɤwu	ma,	nɤ-kɯ-nɯɣmu	me	ma	ma-ta-mbi
Frog 38


iɕqha tɯrme kɯngo ɣɤʑu tɕe, ɯkɯrtoʁ jaria wo!
Someone was ill \wav{8_kWngoGAZu}


	kɤ-mɯnmu	kɯ́nɤ	mɯ-pjɤ-mɯnmu				
	Raven 58


A	78	nɤj	thɤstɯɣ	tɯ-chɯ-cha	ʑo	nɤ,	a-tɤ-tɯ-nɯ-rke	qhe	nɯnɯ	nɤʑo	ɣɯ	nɤ-rkus	ŋu"	to-ti,


tɕhi kɯfse pɯtɯnɯmbɣom kɯ́nɤ, zgo nɯ kutɯpɣaʁ ndɤre ra
\wav{8_kutWnWmbGom}

nɯ apɯŋu ma dianhua ɯkɯlɤt ɣɤʑu!
\wav{WkWlAt}

\section{Demonstratives} \label{sec:demonstratives}
\chapter{The structure of the noun phrase} \label{chapt:noun.phrase}

\section{Possession} \label{sec:possession}
1. Possessive prefix only

2. Possessive and genitive

3. Elision of the head, only preserves the genitive form

divination
47	tɕendɤre chɯ-rɯsɯso tɕe ɯʑɤɣ nɯ kɤ-nɯ-thu na-nɯ-jmɯt ɲɯ-ŋu,


Dative raised as the possessor

mihi est

mine
180	tɕe	tɕheme	nɯ	kɯ	"χawo	kɯ	ra	tɕiʑɤɣ	a-pɯ-ŋú-nɯ	kɯ-ɣe


double possession


  tɕe tɯsqar nɯ nɯ rmi
  tWsqar2.144
\section{Nominal complements}  \label{sec:nom.comp}
ftsoʁ kɯngɯt ɣɯ ɯphɯ srɯnloʁ pjɤkɤrkuchɯ
Gesar 238

Non-verbal complements left of noun, that explains the difference between kɯmaʁ "other" and kɯ-maʁ "the one who is not"


wuma ʑo mbro stu kɯ-sna ʑo nɯ to-z-rɤŋgat-nɯ tɕe ɲɤ-ɕe ɲɯ-ŋu
smanmi4.31
\section{Apposition}
nɤʑo qaɕpa
iʑo tɕheme mɤŋgrɤl

\section{Reduplication}


tɕe ɯ-mɤlɤjaʁ nɯ kɯtʂɤ-ldʑa ɣɤʑu low.
ɯ-phaʁ ntsi χsɯ-ldʑa, ɯ-phaʁ ntsi χsɯ-ldʑa ɣɤʑu.
x-25-akWzgumba
(βɣaza)
> distributive "each"
\chapter{Ideophones}


mtshalu kɯ kɤ́wɣmtsɯɣa tɕe, aβri xɯβnɤxɯβ ɲɯti
\section{Deideophonic verbs}
khɯtsa ra ɲɤɣɤrqhɯβrqhɯβ ʑo ɲɤmɯrpunɯ
khɯtsa ra ɲɤmɯrpunɯ tɕe ɲɤɣɤrqhɯβrqhɯβ ʑo
ɲɯsɤrqhɯβrqhɯβ ʑo paβde

\section{Interjections}

kɤ-rɯɕmi tɤ-stʰɯt-i tɕe, aʑo kɯ "woe" tɤ-tɯt-a tɕe, nɤʑo kɯ "woe" tu-tɯ-ti mɤ-ra"
\wav{reponse-oe}
\chapter{Adverbs} \label{chapt:adv}
\section{Adverbs of degree} \label{adv:degree}
wuma ʑo tɯʑo tɯ-fsu ʑo tu-kɯ-zɣɯt nɯ rkɯn.

tɕe pɤjka wuma nɯnɯ tɕe, pjɯ-kɤ-nɯji ŋu tɕe,
wuma = really, not very

ntsɯ


about

"a little"
ɯʑo kɯ-xtɕɯ-xtɕi ʑo qiaβ
21  jmɤɣni ɕɯrʁaʁ

verbs:
pandɤɕku nɯnɯ iʑora i-sɤtɕha nɤkɤro ʑo dɤn 

\section{Quantifiers}
ci

ɯ-si nɯnɯ kɯ-nɤrko ci kɯ-ngɯt ci ŋu,

ʑɯrɯʑɤri qhe ci rɯdaʁ nɯ dɯxpa ma (the other one)
nɯ-kɤ-ndza ɯ-spa ɲɯ-ɕti qhe,

some
ma tsuku rcanɯ, ɯ-mɯntoʁ kɯ-wxti tɕe, khɯtsa jamar ɲɯ-kɤ-βzu tu.
ri nɯnɯ rkɯn.


each, all
ɯ-mat roŋri ʑo nɯ ɯ-ru tɯka ntsɯ tu.

zgo roŋri nɯnɯ zgo kɯmbro ra nɯ-rmi ʑakastaka tu
kamnyu2.3

rɯri

ɯzda ra kɯnɤ nɯrʑaβ tɯka nanɯɕarnɯ ɲɯŋu.
Slobdpon 39

χsɯxpa jotsuj, nɯtshɯci ʑo asciti, amɯmij, jirɟit tɯka totu ɲɯŋu.
Slobdpon 81

rɯŋgu kɯnɤ rɯŋgu roŋri ʑo tu maʁ
不是每个都有的

revize ascit


\ipa{jamar}, \ipa{jarma}, \ipa{jarmar}





kɯrɯ tɕheme rɯri ɣɯ nɯthaχtsa tu
kɯrɯ tɕheme ra ɣɯ nɯthaχtsa tɯka (tɯkaka) tu
\ipa{8_rWri}

\xv kɤndza thɯ́-wɣ-kro tɕe, tɯrme roŋri ɣɯ ɲɯ́-wɣ-kho ra
\xn 分食物的时候,要分给每一个人

\xv tɯrme roŋri ɣɯ nɯ-mɲaʁ tu ɕti
\xn 每个人有眼睛

\xv ʑaka tunɯsaχsɯtɕi ɕti
\ipa{8_ʑaka}

nɯra ʑakastaka pjɤ-nɯ-fɕɤt-nɯ tɕe, 
smanmi4.239




\ipa{raŋ}

\xv nɤʑo tu-tɯ-ti mɤ-ra ma aʑo raŋ tu-nɯti-a jɤɣ
\xn 不用你说,我自己说
\ipa{8_raN1}

\xv kɯki laχtɕha ki aʑo raŋ ɣɯ pɯ-kɯ-nɯ-tɯ-tu ɕti
\xn 这个东西是我自己曾经有过的
\ipa{8_raN2}


ɕɤχcɤl kɯnɤ ʑo kɤ-nɯ-rŋgɯ mɯ-pjɤ-khɯ \wav{8_CAXcAl}
She could not sleep even in the middle of the night (theème 56
\section{Deverbal adverbs}

tɯrme fsɯfse ʑo fse

nɤkɤro wxti
kɤntɕhɯ

\section{Direction}
\begin{table}[H]
\caption{Locative nouns and adverbs}\label{tab:adv.loc}
\begin{tabular}{llllllllllllllllll} \toprule
ipa{ɯ-taʁ}  &  	ipa{atu}  &  	ipa{tɕɤtu}  &  	\\
ipa{ɯ-pa}  &  	ipa{aki}  &  	ipa{tɕɤki}  &  	\\
&	ipa{alo}  &  	ipa{tɕɤlo}  &  	ipa{lochu}  \\  
&	ipa{athi}  &  	ipa{tɕɤthi}  &  	ipa{thɯchu}  \\  
&	ipa{akɯ}  &  	ipa{tɕɤkɯ}  &  	ipa{kɯchu}  \\  
&	ipa{andi}  &  	ipa{tɕɤndi}  &  	ipa{ndɯchu}  \\  
\bottomrule
\end{tabular}
\end{table}

kɯchu phɤri, ndɯchu phɤri



\chapter{The verbal template: a general overview} \label{chapt:template}
 \cite[218]{bickel07inflectional} cites the following criteria distinguishing templatic morphology from layered one: 

\begin{enumerate}
\item ``There can be more than one root or head.''
\item ``Dependencies can obtain between non-adjacent formatives.''
\item ``Allomorphy of more inward formatives and the position of formatives in the string can be determined by their formal categories, or by phonological principles, rather than their syntactic or semantic functions.''
\end{enumerate}
\begin{table}[H]
\caption{ The Rgyalrong verbal template (Japhug)}\label{tab:template}
\begin{tabular}{llllllllllllllllll} \toprule
 
\ipab{a-}  &  	\ipab{mɯ- }   &  	\ipab{ɕɯ-}   &\ipab{tɤ-} &  	\ipab{tɯ-}  &  	\ipab{wɣ-}   &  	\ipab{ʑɣɤ-}  &  	\ipab{sɯ-}   & \ipab{rɤ-} & \ipab{nɤ-} &   	\ipab{a-}  &  	\ipab{nɯ-}  &  	\ipab{ɣɤ-}  &  	\ipab{noun}    &  	 \begin{math}\Sigma\end{math}    &  	\ipab{-a}  &  	\ipab{-t}  &  	\ipab{-nɯ}   &  \\
   &  	\ipab{mɤ-}   &  	\ipab{ɣɯ-}   &\ipab{pɯ-}&  	  & & 	  &  	  &  	  &  	  &  	 \ipab{sɤ-}&&  	\ipab{rɯ-}  &  	  &  	  &  	  &  	  &  	\ipab{-ndʑi} &  \\
  &  	   &     &  etc.	  & & 	  &  	  &&  	  &  	  &  	 & &  etc.	  &  	  &  	  &  	  &  	  &  	  &  \\
1  &  	2  &  	3  &  	4  &  	5  &  	6  &  	7  &  	8  &  	9  &  	10  &  	11  &  	12  &  	13  &  	14  &  	15  & 16 &17 & 18\\
\bottomrule
\end{tabular}
\end{table}
 

\begin{enumerate}


\item Irrealis  a-, Interrogative ɯ́-, conative jɯ-
\item negation ma- / mɤ- / mɯ- / mɯ́j-
\item \textbf{Translocative / Cislocative ɕɯ-/ɕ-/s-/z- and ɣɯ-}
\item Directional prefixes (tɤ- pɯ- lɤ- thɯ- kɤ- nɯ- jɤ-, tu- pjɯ- lu- chɯ- ku- ɲɯ- ju-) permansive nɯ-, apprehensive ɕɯ-
\item Second person (tɯ-, kɯ- 2>1 and ta- 1>2)
\item Inverse -wɣ- / Generic S/O prefix kɯ-. In some dialects of Japhug (Gsar-rdzong, Da-tshang), the aorist direct prefix -a- occurs in this this slot; in Kamnyu Japhug, the aorist direct prefix fused with the directional prefixes and the apprehensive. Progressive \ipa{asɯ}-. Note that the inverse is actually \textit{infixed} within the progressive as in \ipa{ɲɯ-tɯ-ɤ́<wɣ>sɯ-zgroʁ} \const{}-2-\prog{}<\inv{}>-attach ``he is attaching you". For this reason we do not posit a distinct slot.
\item Reflexive ʑɣɤ-  
\item Causative sɯ-/z-/sɯɣ-/ɕɯ-/ɕ-/ɕɯɣ-/ʑ-/ɣɤ-, Habilitative sɯ-
\item  Antipassive  sɤ-/rɤ-
\item tropative nɤ-, applicative nɯ-
\item Passive or Intransitive thematic marker a- / Deexperiencer sɤ-
\item Autobenefactive-spontaneous (appears in this position only when the passive/intransitive determiner is present, otherwise appears between positions 6 and 7) nɯ-
\item Other derivation prefixes nɯ- ɣɯ- rɯ- nɤ- ɣɤ- rɤ-
\item Denominal / incorporated noun.  
\item Verb root 
\item Past 1sg/2sg transitive -t (aorist and evidential)
\item 1sg\footnote{\citet{gongxun} proposed that  in the Rgyalrong languages, the 1sg suffix may not constitute a templatic slot distinct from the verb stem: especially in Zbu, the first person form is not always predictable and there is not clear boundary between the verb stem and the first person suffix.}

\item Personal agreement suffixes (-tɕi, -ji, -nɯ, -ndʑi)
\end{enumerate}
a-mɤ-ɕ-tɤ-tɯ́-wɣ-z-nɯmbrɤpɯ

a-mɤ-ɕ-tɤ-tɯ-znɤpɤmbat
\wav{8_template01}



\section{Contracting prefixes} \label{sec:contracting}
\ref{sub:passive}

\chapter{Verbal flexional morphology: personal agreement} \label{chapt:flexional.agr}


\section{Person marking}


person mismatch
iʑora tɕe ɕkɤphɤr tu-ti-nɯ tsuku kɯ ɕkɤjwaʁ tu-ti-nɯ ŋu ma
83
hist-07_Cku.wav
1pl used with 3pl verb form

iʑora kɯ kuxtɕo tu-βzu-nɯ ŋgrɤl
\section{Intransitive paradigm} \label{sec:intr.paradigm}
 
\section{Existential verbs} \label{sec:existential}


sensory
nɤʑo nɯtɕu ɣɤtɤʑu

> So you were here ! (mirative)


Special usage of existential verbs:
ŋotɕu ɕ-pɯ-tɯ-tú-nɯ?
sras.56

\section{Transitive paradigm} \label{sec:trans.paradigm}
\section{Generic} \label{sec:generic}


nɯ ma tɯ-kɯ-sɯ-rtsa me
gram-tWkWsWrtsa_me
> revérifier


nɯnɯ phe kɯ-rɤthu ju-kɯ-ɕe tɕe tu-kɯ-sɯɣɕɤt ŋu kɤ-ti ɲɯ-ŋu tɕe,
mphrWmW 3


"a-pa kɯnɤ, nɤkinɯ, srɯnmɯ ɯ-qhu chɯ-βze ɲɯ-ɕti ma
tɕiʑo mɯ́j-kɯ-nɯkon tɕe, mɤ-nɯɕe-tɕi.

Nyima4.95

>still possible to use pronouns with generic in some cases. = mɯ-ɲɯ́-wɣ-nɯkon-tɕi

tɕe nɯnɯ tɤftsa nɯ tɤrpɯ sɤz lu-kɯ-moŋlo mɯ-pjɤ-ŋgrɤl
with overt nouns
\subsection{Embedded generic}
qartshaz ɯ-se ɣɯ-tshi mɤ-βdi ma (456)
qartshaz nɯnɯ tɯxpa mɤɕtʂa chɯ-rɤrɟit mɤ-ŋgrɤl tɕe
nɯ kɯ-fse tɕe tɯʑo tɯ-rɟit kɯnɤ ʑa mɤ-sci tu-ti-nɯ

  \wav{x-27-qartshaz}  .
  ɣɯ-tshi mɤ-βdi /kɤ-tshi mɤβdi
  spécifique à βdi ?
  kɯki tɯŋga ɣɯ-ŋga mɤ-βdi
  kɯ-ɕe mɤ-βdi
  \wav{gram_embedded_generic}
  
\section{Transitivity} \label{sec:transitivity}
\subsection{Semi-transitive}

Attention à l'article ɕ-kɤ-re
\subsection{Ambitransitive} \label{sub:ambitransitive}

\subsection{Deponent} \label{sub:deponent}
Conjugates like a transitive verb, but agrees with an argument without ergative marker.
nɯ ɯ-ŋgɯ nɯ ɯ-pɯ nɯnɯ ra ɲɯ-βzu-nɯ.
x-akWzgumba
\chapter{Verbal flexional morphology: TAM}
\section{Stem alternation}
\section{Directional prefixes}

\section{Past} \label{sec:past}


relic of the past suffix:
rɤʑi / rɤʑit
rɤɕi / rɤɕit
amdzɯ / amdzɯt
\subsection{Aorist} \label{sub:aor}

> cf. converb -tɯ-
\subsection{Evidential} \label{sub:evd}
ɯsmɤn ra latsɯm tɕe ɕtoznɯsmɤn tɕe tope
\wav{8_znWsmAn}
Chen Zhen Oct 2011: witnessed the first event, not the second one

aʑo kɯspoʁ mɯpjɤmtota tɕe pjɤɕqhlata

\subsection{Past Imperfective} \label{sub:pst.ipf}

mɯpɯcha tɕe, tɕe tham tɕe ɯkɯmŋɤm tomna tɕe, kɤrɯndzɤtshi tocha
\wav{8_tocha}
Chen Zhen Oct 2011

nɤʑo nɤmu pɯxtɕi tɕe, ɯpɯ́mpɕɤr?

transitive verbs derived from experiencer intransitives, tropative/applicative
\section{Non-past}
\subsection{Unprefixed non-past}
\subsection{Present}
\subsection{Constative}



dianhua pɯ-mbrɤt > ɲɯ-ŋu tɕe
\subsection{Imperfective}

\subsection{Imperative}
phɤnba kɤβzu mɤkɯcha ci pɯŋu kɯ́nɤ, matɤkɯrɯʁdɯxpa ra

kɯki matɤ́wɣndza ra
\subsection{Irrealis} \label{sub:irrealis}
a-kɤ-tɯ-tso tɕe, tɕetha a-ʁa ɯ-ɣɤʑu nɤ tɕe a-pɯ-ŋgrɯ, a-ʁa maŋe nɤ tɕe a-ma-tɤ-kɯ-mpɕa-a
\wav{8_apWNgrW}
\section{Progressive}

\jg{asɯ-}

ɲɯ-tɯ-ɤ́<wɣ>sɯ-zgroʁ
ɲɯ-kɯ-ɤsɯ-zgroʁ-a
\wav{8_YWtAwGsWzgroR}
\section{Reduplication}
\subsection{Condition}
+action occuring systematically
\subsection{Increase}


tɕe tɤjpa tɤpɤtso cho cici chɤnɯrɤɣɯɣondʑi, cici tonɤmtsɯmtsaʁndʑi tɕe,
xuehaizi 15 他和雪孩子又唱又跳,玩得很开心。


ɯ-kɯ-spa ɲɯ-ɲɯ-me ɲɯ-ŋu, 
8_XpiWkWspa


tɕendɤre ʑɯrɯʑɤri nɤ chɯchɯmɤɕindʑi nɤ chɯchɯmɤɕindʑi tɕe tɕendɤre
125, déluge2
\section{Cislocative / Translocative} \label{sec:cisloc}
The translocative  \ipa{ɕ-} and cislocative  \ipa{ɣɯ-} prefixes  are the normal way to express ``to go do'', ``come to do'' in Japhug:
\begin{exe}
\ex 
\gll mphrɯmɯ	ɕ-pɯ-sɯ-re	tɕe,	ɕ-tɤ-the	ra \\
 divination \textsc{transloc}-\textsc{imp}-\textsc{caus}-look[III] \textsc{cnj} \textsc{transloc}-\textsc{imp}-ask[III] \textsc{n.pst}:have.to \\
 \glt You have to go to make him do a divination, go to ask him.
 
 \ex
\gll 	ɯ-ɕki	zɯ	ɣɯ-tɤ-nɯ-thu-nɯ	\\
3\textsc{sg}-\textsc{dat}  \textsc{loc}  \textsc{cisloc}-\textsc{imp}-\textsc{auto}-ask-\textsc{pl} \\
\glt 	Please come  to ask her. (the prince, 66)
\end{exe}
They exhibit S/A accusative alignment:


\begin{exe}
\ex 
\gll nɯ-wa	nɯ	to-ɣi	tɕe,	ɣɯ-pjɤ́-wɣ-sɯ-ɣe-nɯ \\
3\textsc{pl}-father \textsc{top} \textsc{evd:up}-come \textsc{cnj} \textsc{cisloc}-\textsc{evd:down}-\textsc{inv}-\textsc{caus}-come-\textsc{pl} \\
\glt 	Their father came (up) and invited them down (to the Naga realm) (Not: they came to be invited down)
\end{exe}



tham kɯ́nɤ nɯʑɤra ɣɯ-tu-tɯ-thú-nɯ ɲɯ-ŋu,
sras.76

Accusative alignement
ɣɯ-tɤ́-wɣ-qur-a 
He came to help me > S = A \wav{8_GWtAwGqura}
\ref{subsub:antipass.syntax} use of antipassive with trans/cis


\section{Persisting situation}


\section{Conative}
The conative \ipa{jɯ}-- prefix

\citet{nigel13conative}

\begin{exe}
\ex
\gll 
\ipa{χsɯ-tɤxɯr}   	\ipa{zɯmi,}   	\ipa{χsɯ-tɤxɯr}   	\ipa{jɯ-ko-ɕe}   	\ipa{ʑo}   	   	\ipa{tɕe,}   	\ipa{nɯ}   	\ipa{ma}   	\ipa{mɯ-ɲɤ-cha}   	\ipa{tɕe,}   \\
three-turn almost three-turn \textsc{conative-evd}-go \textsc{emph} \textsc{coord} \textsc{dem} a.part.from \textsc{neg-evd:perm}-can \textsc{coord} \\
\glt As he was about to finish the third turn, he could not (run) anymore. (The prince, 109-110)
\end{exe}

\begin{exe}
\ex
\gll
\ipa{ɯ-nmaʁ}   	\ipa{nɯ}   	\ipa{japandʐi}   	\ipa{ri,}   	\ipa{wuma}   	\ipa{ʑo}   	\ipa{tɤ-ngo}   	\ipa{tɕe}   	\ipa{zɯmi}   	\ipa{ʑo}   	\ipa{jɯ-nɯ-si,}   	<zhouyiyuan>  	\ipa{ri}   	\ipa{pɯ-rɤʑi}   	\ipa{pɯ-ra}   \\
\textsc{3sg.poss}-husband \textsc{top} last.year \textsc{loc} very \textsc{emph} \textsc{aor}-be.sick \textsc{coord} almost \textsc{emph} \textsc{conative-aor}-die district.hospital \textsc{loc} \textsc{pst.ipfv}-remain \textsc{pst.ipfv}-have.to \\
\glt Last year her husband became very sick and almost died, he had to stay at the hospital. (Relatives, 351)
\end{exe}

Alternative conative construction:

	tɕe ɣɯ-tɕhɯ pɯ-ŋu ri, ci nɯ mɤ́-wɣ-sɯɣcha pɯ-ŋu jamar ʑo qartshaz nɯ jɤ-nɯɬoʁ ndɤre,
(Lobzang1.70)

%\begin{exe}
%\ex
%\gll 
% \ipa{jisŋi}   	\ipa{pɯ-nɯkheixwi-j}   	\ipa{ɯ-raŋ}   	\ipa{zɯ,}   	\ipa{a-zda}   	\ipa{ci}   	\ipa{pɯ-tu}   	\ipa{tɕe,}   	\ipa{jɯ-pɯ-nɯʑɯβ}   	\ipa{ʑo}   	\ipa{ri,}   	\ipa{kɯ-maqhu}   	\ipa{ʁo}   	\ipa{nɯ-aχsom}   	\ipa{pɯ-ra}   \\
% today \textsc{pst}-have.a.meeting-\textsc{1pl} \textsc{3sg.poss}-time \textsc{loc} \textsc{1sg.poss-}companion \textsc{indef} \textsc{pst.ipfv}-exist \textsc{coord} conative-\textsc{aor}-fall.asleep \textsc{emph}  \textsc{loc}, \textsc{nmlz:}S/A-be.after \textsc{top:contr} \textsc{aor}-be.awake   \textsc{pst.ipfv}-have.to \\
%\glt There was a meeting yesterday, someone there was about to fall asleep but he had to XXXX (elicited)
%\end{exe}
%\wav{8_conative}



\section{Interrogative}

nɯʑo tɕi ɯβrɤ-tɯ-ŋu-nɯ?" sras.60

nɤ-tɤɲi taʁ kɤ-rɤt nɯ ɯβrɤ-kɯ-z-nɤmɲo-a-nɯ
sras.62

ɯβrɤ-ja-ɣɯt-ma 他没有拿来吧(希望他不拿来)
ɯβrɤ-jɤ-kɯ-ɣɯt-chɯ 他没有拿来吗? (希望他拿来,但是没有拿来)
ɯβrɤ-ju-ɣɯt 最好是不会拿来

ɯmɤ-kɤ-tɯ-tshi-t-chɯ 你是不是喝了?
ɯmɤ-tɤ-tɯ-rɯndzɤtshi-chɯ 是不是吃了
ɯβrɤ-kɤ-tɯ-tshi-t-ma 你没有喝吧

nɤʑo ɯmɤ-lɤ-tɯ-βzi-chɯ 你喝醉了吧

\section{Apprehensive}
92 "e, a-tɕɯ ki stɤβtshɤt ki mɯ-ɕɯ-cha kɯ" ɲɤ-sɯso tɕe, nɤki,

ɯ-mgo ʑo mɯ-chɤ-nɯ-ɣɤkhɯ tɕe, "a! a-χtɕɤs nɤ-tɯ-ndɯn nɯ ɲɯ-ŋu ri,
94 nɤʑo stɤβtshɤt ci mɯ-ɕɯ-tɯ-cha ɲɯ-sɯsám-a tɕe, nɯ ɲɯ-nɯzdɯ́ɣ-a wo" to-ti,


ɣɯ-nɯ-jmɯt nɯ-sɯso-t-a

I was afraid that I would forget

ɕa-nɯjmɯt nɯ-sɯso-t-a
I was afraid that he had forgotten/
ɕɯ-ngo nɯ-sɯso-t-a

\section{Negation}

tɕe qhɤrɟɤl-rtsɯlma-rɯntɕin chonɤ rɯntɕin-thaŋɬa ɣɯ ɯ-tɕɯ ni mɯ-ɲɯ-ta-mbi-ndʑi maʁ,
115

Double negation
nɯ kɤti mɤkɯkhɯ me


nɤ-kɤ-ndza mɤ-rtaʁ-a tɕe ndza-mɤ-ndza me-a."
I does not matter whether you eat me or not.
The wolf and the dog

mɤmbrɯmɤmbrɤt?
\chapter{Compound tenses}

\section{曾经}
rɲo
\section{About to}

qala ɯmu kɯ ɯpɕi tɕe kɤndza ɯkɯɕar ɕe pjɤŋu
xuehaizi 4


dian ʑatsa arɕo ɲɯŋu

\section{Perfect}
ndʑi-sɤtɕha nɯ nɯ ɣɯ jil nɯnɯ rcánɯ khu kɯ lonba ʑo tha-ɕkɯt ɲɯ-ŋu
khu.01 ?

\section{Conative}
aʑo a-tɕɯ sat pjɤ-ŋu ri, nɤʑo ɯ-sroʁ ko-tɯ-ri tɕɤn, 
smanmi4.53


tɕendɤre, nɤʑo tɯ́-wɣ-ɕaβ tɤ-ŋu tɕe, aʑo kɯ ʁe nɯ ɲɯ-ɕthɯz-a
aʑo ɣɯ-ɕaβ-a tɤ-ŋu tɕe, nɤʑo χcha nɯ a-nɯ-tɯ-ɕthɯz tɕe ku-ɕe-tɕi" to-ti
smanmi4.140
\section{Modal verbs}

\subsection{\ipa{ŋgrɤl} ``to be usually the case"}
\wav{8_NgrAl}
iʑora shendanjie mɤŋgrɤl
We don't have Christmas



\chapter{Non-finite verbal morphology} \label{chapt:non.finite}
\section{Nominalization and participles} \label{sec:nmlz}
\subsection{S/A participle} \label{sub:S/A.part}


kɯki ɯ-ku-kɯ-ndɯn kɯ-cha ci ŋu
sras 48

\subsection{O participle} \label{sub:O.part}
tɕe aʑo a-mɤ-kɤ-sɯz tɤjmɤɣ nɯ kɤ-ndza mɤ-naz-a
107mbrAZWm

\begin{exe}
\ex
\gll      \ipa{tɕe} 	\ipa{ɬamu} 	\ipa{kɯ} 	\ipa{qɤjɣi} 	\ipa{nɯ-kɤ-mbi} 	\ipa{nɯ} 	\ipa{tu-ndze} 	\ipa{pjɤ-ŋu.}   \\
\coord{} Lhamo \erg{} bread \aor{}-\nmlz{}:O-give \topic{} \ipf{}-eat[III] \ipf.\evd{}-be  \\
 \glt    He was eating the bread that Lhamo had given him. (The Raven 111)
\end{exe} 

nɯ ma ɯ-kɤ-mbi maŋe tɕe, "a-me ta-mbi ra" to-ti tɕe, 再也拿不出东西给他
Smanmi 4.166


pɣa ra nɯ-kɤ-rga nɯ qɤj ntsɯ ŋu

23-pGaYaR.24
\subsection{Infinitive} \label{sub:infinitive}
\subsection{Non-core argument participle} \label{sub:NCA.part}

tɤjmɤɣ ɯ-sɤɣɬoʁ thɤjtɕu ŋu?
\subsubsection{place} 
kɯki tɯci ki ɯ-tɯ-rnaʁ mɯ́j-rtaʁ tɕe, aʑo a-sɤ-ɣi mɯ́j-khɯ tɕe nɤʑo jɤ-nɯɕe
ziji  qu ba 4
\subsubsection{time} 

tɤjmɤɣ ɯ-sɤɣ-ɬoʁ nɯ thɤjtɕu ŋu

\subsubsection{instrument} 
with sɤ-



with kɯ-


ɯʑo nɯ  ɯ-kɯ-sɯ-mphɯl nɯ, ɯ-zrɤm ɲɯ-ɕti ma ɯ-rɣi maŋe
qaʑmbri, l.38


\subsection{Action noun} \label{sub:action.noun}
\subsection{Degree participle} \label{sub:degree.part}

\subsection{Simultaneous action} \label{sub:simult.part}

  with the verb \ipa{βzu}  and an additional prefix \ipa{tɯ}-- meaning ``one", one can build a special construction meaning ``to \textsc{verb} to \textsc{object1} and \textsc{object2} simultaneously", as in:
 

   \begin{exe} %\label{ex:tWtWtsxaB}
\ex
\gll  \ipa{pri} \ipa{cho} \ipa{jlɤkrɯ} \ipa{nɯra} \ipa{tɯ-tɯ-tʂaβ} \ipa{ʑo} \ipa{pjɤ-βzu} \\
  bear and basket \textsc{dem.pl}  \textsc{one-nmlz:action}-cause.to.roll.down \textsc{emph} \textsc{evd}-do \\
 \glt  He causes the bear to roll down together with his basket. (The rabbit, 48)
\end{exe} 

tɯtɯɕe ʑo joβzundʑi
他们俩同时去了

tɕiʑo tɯtɯndza βzutɕi jɤɣ wo = tɯtɯrca ndzatɕi
\wav{8_simult}

nɯra tɯ-tɯ-tʂɯβ chɯ-βzu-nɯ tɕe
30-kWrWxtsa, 21
\subsection{Neologisms}

tɯkɤrme mɤkɯsɤci 浴帽
\section{Other non-finite forms} \label{sec:non.finite}

\subsection{tɯ- Converb}
 pjɯ-tɯ-mto tɕe tɤ́wɣnɯɣɤŋoʁa
 
pri pjɯ-tɯ-mto tɕe jo-phɣo (ambiguous)

ɯʑo pri pjɯ-tɯ-mto tɕe jo-phɣo
\wav{8_converb_tW}

tutɯɕɯɴqoʁ ʑo tɕe, tɯrme kɯdɯdɤn ʑo kɯnɤmɲo juɣinɯ pjɤŋgrɤl
histoire 07.6

\subsection{kɤ- Converb}

nɯ-mɤ-kɤ-sɯz > with them knowing



\subsection{Purposive}
\jg{a-mɤ-tu-sɤ-ɕqhɯ-ɕqhe} smɤn ci tɤndzata
amɤɲɯsɤɕpɯɕpaʁ tʂha kutshia
fso tɕe atusɤnɯmtɕɯmtɕi, ʑa kunɯrŋgɯa ra

amu ɯɲɯsɤrgɯrga, 
\wav{8_sArgWrga}

amɤtusɤʁndɯʁndɯ, mɯtɤnɯkhajata
\wav{8_sARndWRndW}


\wav{8_sAndzWndza}
βʑɯ nɯni ko-kɤnbaʁ-ndʑi-chɯ tɕe, lɯlu kɯ ndʑi-mɤ-tu-sɤ-ndzɯ-ndza ɲɯŋu

kɯm ɲɯ-mbɤr tɕe, a-ku ɯ-mɤ-tu-sɤ-rpɯ-rpu pɯ-phaβ-a
\wav{8_sArpWrpu}



tɕe nɯ ɯ-pa nɯnɯ li khɤxtu nɯnɯ, tɯci, tɯftsaʁ kɯ pjɯ-sɯspoʁ ŋgrɤl tɕe,
tɕe ɯ-mɤ-pjɯ-sɤ-sɯspoʁ, nɯnɯ tɕu (exemple de PURPOSIVE)
nɤki, tɤrɤm kɯ-fse ɲɯ́-wɣ-ta nɯ maʁ nɤ cɯpa kɯ-fse ɲɯ́-wɣ-ta tɕe,
x-26-tChWra


\subsection{Gerund}
sɤ-ɕqhɯ-ɕqhe

\subsection{Naked infinitive}
\subsubsection{a}
\jg{pɯrɲota}

ɯβraʁ kɤʁzaβa  我拴好了

kɯm ɯɣɤβdi kɤʁzaβa 用木板把门顶住

smɤɣ ɯsaʁ nɯʁzaβa 

ɯphɯ ɯkho nɯkɯro nɯnɯ nɯ́wɣfsɯɣa
他把零钱还给我了

ɯxtɕɤr ɲɤsɯɣ

sɯlɤm-grɯβdɤn kɯ ɯ-thu tɤ-tɯ-z-mɤku-t ŋu tɕe, stɤβtshɤt ɯ-βrɤ-tɯ-nɯ-βze
sras

cha (kɤtshi, ɯtshi) koɣɤtɕhom

ɯ-tʂɯβ thɯ-jɤɣ tɕe tɕe chɯ-pɣaʁ-nɯ
23, kɯrɯxtsa30
\subsubsection{b}
\begin{exe} 
 \ex 
\gll   \jg{nɯɕimɯma} \jg{tu-kɯ-ndza-a} \jg{ri} \jg{nɤ-kɤ-ndza} \jg{mɤ-rtaʁ-a} \jg{tɕe} \jg{ndza-mɤ-ndza} \jg{me-a}  \\ 
 immediately \ipf{}-2>1-eat-1\pl{} but 2\sgposs{}-\nmlz{}:O-eat \negat{}-\npst{}:enough-1\sg{} \conj{} eat-\negat{}-eat \npst{}:non.exist-1\sg{} \\ 
 \glt  If you eat me immediately, I will not be enough for you to eat, eating me is not worth it. (Aesop adaptation)
\end{exe} 

tɤ́wɣndzaa nɯ kɯkhɯ, mɯtɤ́wɣndzaa nɯ kɯkhɯ kɯfse ɯskɤt ɲɯŋu.
\wav{8_ndzamandza-explication}

ʁndɯ mɤ-ʁndɯ tɯ-me

ɕar mɤɕar me
βɟi mɤβɟi me

ɣi mɤɣi maŋea/maŋe

ɣi mɤɣi mataŋe
\wav{8_GimaGi}

  
  
\chapter{Verbal derivational morphology} \label{chapt:derivational}
This chapter deals with the verbal derivational morphology of the Japhug verb, excluding nominalization, which is treated in section \ref{sec:nmlz}. Three main characteristics distinguish derivational  from flexional morphology. 

First, the derivational morphology is not productive to the same extend as flexion: some expected derived verb forms are non-attested, and sometimes only the derived verb exists, while the original from which it is derived has been lost. Second, the meaning of the derived verb forms is not always predictable from the basic verb (a typical example in  Japhug is \jg{a-βzu} ``to become'', originally a passive of \jg{βzu} ``to make''). Third, derivational morphology cannot be controlled by the syntax; this is one of the reason why nominalization prefixes are not included in this chapter, as their presence is required in many constructions such as complement or purposive clauses.  


The flexional affixes presented in the previous chapters have functions related to person marking and TAM, while those of the derivational prefixes mainly concern valency and modality.

As we have seen in chapter \ref{chapt:template}, the Japhug verb can be described using the following template:


\begin{table}[H]
\caption{The Japhug verbal template: derivational prefixes}\label{tab:template:derivational}
\begin{tabular}{llllll|llllllll|lllll} \toprule
 
\ipab{a-}  &  	\ipab{mɯ- }   &  	\ipab{ɕɯ-}   &\ipab{tɤ-} &  	\ipab{tɯ-}  &  	\ipab{wɣ-}   &

  	 \grise{\ipab{ʑɣɤ-}}  &  	\grise{\ipab{sɯ-}}  & \grise{\ipab{rɤ-}}& \grise{\ipab{nɤ-}} &   	 \grise{\ipab{a-}}   &  	\grise{\ipab{nɯ-}}  &  	\grise{\ipab{ɣɤ-}}  &  	\grise{\ipab{noun}}    &  	 \begin{math}\Sigma\end{math}    &  	\ipab{-a}  &  	\ipab{-t}  &  	\ipab{-nɯ}   &  \\
  	 
  	 
   &  	\ipab{mɤ-}   &  	\ipab{ɣɯ-}   &\ipab{pɯ-}&  	  &  	 
    & \grise{ }	  &  	 \grise{ }	  &  	  \grise{ }	  &  	   \grise{ }	&  	\grise{\ipab{sɤ-}}&  \grise{ }	 &  	\grise{\ipab{rɯ-}}  &  	 \grise{ }	  &  	  &  	  &  	  &  	\ipab{-ndʑi} &  \\
  &  	   &     &  etc.	  & & 	  &  	  &  	 & &  	  &  	 & &  etc.	  &  	  &  	  &  	  &  	  &  	  &  \\
1  &  	2  &  	3  &  	4  &  	5  &  	6  &  	7  &  	8  &  	9  &  	10  &  	11  &  	12  &  	13  &  	14  &  	15  & 16 &17&18\\
\bottomrule
\end{tabular}
\end{table}
 
The derivational prefixes occur in the shaded areas, slots 7-14. Slot 14 represents the incorporated noun (on nominal incorporation see \ref{sub:incorporation}). No verb form with all seven slots filled has been brought to light. Verbs with two or three derivational prefixes, such as \jg{nɤ-sɤ-scit} ``to consider to be nice'' (slots 10-11) or \jg{ʑɣɤ-ɣɤ-ʑo} ``cause oneself to become lighter'' (slots 7-11) commonly appear spontaneously in traditional stories. More complex verbs with three or four derivational prefixes are rare but can sometimes be accepted by native speakers. This study, however, will mainly focus on data from natural, rather than elicited speech, and artificial examples will be avoided.

\end{itemize}	

Each slot contains the following prefixes:
\begin{itemize}
\item Slot 7: the reflexive \ipa{ʑɣɤ-} (cf. \ref{sub:reflexive})
\item Slot 8: the causative \ipa{sɯ-} (cf. \ref{sub:caus1}) and abilitative \ipa{sɯ-} (cf. \ref{sub:abilitative}). 
\item Slot 9: the antipassive \ipa{sɤ-} and \ipa{rɤ}--
\item Slot 10: the tropative \ipa{nɤ-} (cf. \ref{sub:tropative}, the applicative \ipa{nɯ-} (cf. \ref{sub:applicative}, the causative of stative verbs \ipa{ɣɤ-}, the facilitative \ipa{nɯɣɯ-} (cf. \ref{sub:facilitative}).
\item Slot 11: the passive or intransitive determiner \ipa{a-} (cf. \ref{sub:passive} and \ref{sec:contracting}), the deexperiencer \ipa{sɤ-}
\item Slot 12: the autobenefactive-spontaneous \ipa{nɯ-} (cf. \ref{sub:autoben})
\item Slot 13: the denominal / deideophonic / incorporation prefixes \ipa{nɯ-}, \ipa{nɤ-}, \ipa{sɯ-}, \ipa{ɣɯ-} (cf. \ref{sec:denominal}), the vertitive \ipa{nɯ-} (cf. \ref{sub:vertitive}), 
\end{itemize}	
These slots are closer to the verb stem than the flexional prefixes (slots 1-6), following a general tendency observed in the world's languages (cite XXXXX).  The absence of any productive derivational \ipa{suffix} in Japhug is surprising in view of \citet[93]{greenberg66}'s universal 27, according to which OV languages tend to favor suffixes rather than prefixes. The only trace of a derivational suffix in Japhug will be presented in \ref{sub:applicative.t}.

Japhug verb morphology, as mentioned in chapter \ref{chapt:template} is templatic rather than layered (hierarchical).  Four  properties of the derivational prefixal chain in Japhug shows that it cannot be viewed as purely layered morphology, based on \citet[218]{bickel07inflectional} and \citet[12-5]{rice2000scope}'s criteria.

	First, the relative position of the prefixes is rigid and independent of their semantic scope. For instance, the negation on a verb with causative prefix can either refer to the action of the verb (cause not to do) or to the causation itself (not cause to do) without change in the relative order of the prefixes (cf. \ref{subsub:causation}). 
	
   
Second, affix recursion is impossible. For instance, it is impossible to add two times the same causative prefix \ipa{sɯ-} to a verb; this may not be unexpected however, as according to \citet[61]{dixon00causative} no such example exists in the worlds' languages.

Third, the relative order of the prefixes can change for purely arbitrary morphological reasons (presence/absence of another affix). For instance, the autobenefactive-spontaneous (\ref{sub:autoben}) appears in slot 12 only when the passive / intransitive determiner is present, and otherwise appears between positions 6 and 7. 

Fourth, we observe some cases of discontinuous verbal stems (cf. the intransitive determiner \ipa{-a-}, \ref{subsub:intransitive.det}). 

 
 However, discontinuous dependencies, which are plentiful in the domain of inflectional morphology, are  less common in derivational morphology and some phenomena show that part of the prefixes are subject to a layered ordering principle. This is the case in particular of the reciprocal derivation, which can occur both before and after the causative (\ref{subsub:recip.compat}):
 
    \begin{exe}
\ex
\gll   \ipa{sɯ-ɤ-sɯ-sat} \\
		\caus{}-\recip{}-kill-\textsc{reduplicated.syllable} \\
 \glt to  cause to kill each other.
\gll   \ipa{a-sɯ-ɴqʰɯ-ɴqʰi} \\
		\recip{}-\caus{}-dirty-\textsc{reduplicated.syllable} \\
 \glt to  cause each other to become dirty.
\end{exe} 
The causative and the reciprocal is the only case in which semantic scope can influence morpheme ordering in Japhug. Other valency-increasing prefixes, such as the applicative, only appear after the reciprocal (\ref{subsub:appl.compat}) .The reciprocal is not included in the template above; it can occur either before or after slot 8.
Only few other violations of the template are attested (see \ref{subsub:counterexample.caus}).


Apart from the derivational prefixes presented above, Japhug derivational morphology also present non-concatenative processes, in particular the prenasalisation of the initial (anticausative, cf. \ref{sub:anticausative}) and reduplication (cf. \ref{sub:reciprocal}, \ref{sub:intensive.redp}).

In this chapter, morphological processes are classified on the basis of their function rather than of their form or position in the prefixal chain. We distinguish five categories of derivational processes: valency-increasing (\ref{sec:valency.increasing}), valency-decreasing (\ref{sec:valency.decreasing}), modal (\ref{sec:derivation.modal}), denominal and deideophonic (\ref{sec:denominal}) and composition of two verb roots (\ref{sec:verbal.composition}).


\section{Valency-increasing} \label{sec:valency.increasing}
As we have seen in section \ref{sec:transitivity}, transitivity marking in the Japhug verbal system is quite strict except for a handful of ambitransitive verbs. Any change from intransitive to transitive or vice-versa requires overt morphological marking for the great majority of the verbs.

Four derivational prefixes increasing the valency of the verb are know in Japhug. The most common one is the causative of dynamic verbs \ipa{sɯ-} and its many allomorphs. We also find a causative of stative verbs \ipa{ɣɤ-}, an applicative \ipa{nɯ-} and a tropative \ipa{nɤ-}. Additionally, we will briefly treat discuss the relic \ipa{-t} applicative (\ref{sub:applicative.t}).

\subsection{Causative (dynamic verbs)} \label{sub:caus1}
The causative prefix \ipa{sɯ-} is the most productive derivational prefix in Japhug; it can be applied to virtually any dynamic verbs. Cognates of this prefix are not only found in other Rgyalrongic languages (Situ, \citet{linxr93jiarong}, Tshobdun, \citet{jackson06paisheng}), but relics of a similar prefix have been found in most branches of Sino-Tibetan.

It normally occurs in slot 8, after the reflexive \ipa{ʑɣɤ-} and before the tropative \ipa{nɤ-}, though some exceptions have been detected (cf. \ref{subsub:counterexample.caus}).

\subsubsection{Morphophonology} \label{subsub:caus:morphophon}
The causative prefix presents  considerable allomorphy, and numerous irregular forms. It has three regular allomorphs \ipa{sɯ-}, \ipa{sɯɣ-} and \ipa{z-} depending on the following element.

The \ipa{z-} allomorph appears before all derivational prefixes (or unanalysable prefixal element synchronically belonging to the verb root) in sonorant initial (beginning in \ipa{r-}, \ipa{n-}, \ipa{ɣ-} or \ipa{m-}). The following table illustrates some examples of this allomorph:

\begin{table}[H]
\caption{Examples of the \ipa{z}- allomorph of the causative prefix}\label{tab:causative.z}
\begin{tabular}{lllllllll} \toprule
nature of the prefix & base  & &derived  \\
\midrule
non-analysable &  \ipa{mɯnmu} &move& \ipa{z-mɯnmu} &cause to move\\
relic prefix &  \ipa{mɯrmbɯ} &be filled& \ipa{z-mɯrmbɯ} &fill\\
denominal &  \ipa{nɤma} &work& \ipa{z-nɤma} &cause to work\\
non-analysable &  \ipa{nɯna} &rest& \ipa{z-nɯna} &stop\\
antipassive \ipa{rɤ-} &  \ipa{rɤrɤt} &write& \ipa{z-rɤrɤt} &cause to write\\

\bottomrule
\end{tabular}
\end{table}

%causative \ipa{ɣɤ-} &  \ipa{ɣɤme} &destroy, kill& \ipa{z-ɣɤme} &cause s.o. to kill\\
The distribution of the \ipa{sɯ-} and \ipa{sɯɣ-} allomorphs depends on both phonology and morphology. The former latter allomorph occurs when the base verb is intransitive, has not prefixal element, has no initial cluster and no velar or uvular initial consonant, and the former appears in all other cases.
\begin{table}[H]
\caption{The \ipa{sɯ}- and  \ipa{sɯɣ-} allomorphs of the causative prefix}\label{tab:causative.sW}
\begin{tabular}{lllllllll} \toprule
 transitivity & base & & derived & \\
 \midrule
 intr. & \ipa{ɕe} & go & \ipa{sɯɣ-ɕe} & send \\
  tr. & \ipa{ɕɯm} & brood & \ipa{sɯ-ɕɯm} & cause to brood \\
  intr. & \ipa{ndzur} & stand & \ipa{sɯɣ-ndzur} & cause to stand up \\
  tr. & \ipa{ndza} & eat & \ipa{sɯ-ndza} & cause to eat \\ 
    intr. & \ipa{tso} & understand & \ipa{sɯɣ-tso} & cause to understand \\
  tr. & \ipa{tsɯm} & take away & \ipa{sɯ-ndza} & cause to take away \\ 
 \bottomrule
\end{tabular}
\end{table}
 
A four predictable allomorph of \ipa{sɯ-} appears with contracting verbs (cf. \ref{sub:passive}), where \ipa{sɯ-}  and the \ipa{a-} passive prefix or intransitive thematic element merge as \ipa{sɤ-}, as in the following examples:
\begin{table}[H]
\caption{The \ipa{sɤ}-   allomorph  of the causative prefix}\label{tab:causative.sA}
\begin{tabular}{lllllllll} \toprule
  base & & derived & \\
 \midrule
 \ipa{ambrɤqɤt} & be different & \ipa{sɤmbrɤqɤt} & distinguish \\
\ipa{andɯja} & gather & \ipa{sɤndɯja} & cause to gather \\
 \ipa{amɲɤm} & be homogeneous & \ipa{sɤmɲɤm} & homogenize \\
 \bottomrule
\end{tabular}
\end{table}

The causative has four additional irregular allomorphs: \ipa{ɕɯ-}, \ipa{ɕɯɣ-}, \ipa{ɕ-}, \ipa{ʑ-}, \ipa{s-} and \ipa{j-}. All known examples are presented in the following table:

\begin{table}[H]
\caption{The irregular allomorphs of the causative prefix}\label{tab:causative.irregular}
\begin{tabular}{lllllllll} \toprule
  base & & derived & \\
 \midrule
\jg{fka}  &  be full (after eating) &  \jg{ɕɯ-fka}  &  cause to be full  \\ 
\jg{fkaβ}  &  cover &  \jg{ɕɯ-fkaβ}  &  cover with  something \\ 
\jg{mbɣom}  &  be in a hurry &  \jg{ɕɯ-mbɣom}  & cause to be in a hurry \\ 
\jg{mnɤm}  &  have an odour &  \jg{ɕɯ-mnɤm}  & cause to have an odour \\ 
\jg{mŋɤm}  &   hurt  (of a body part) &  \jg{ɕɯ-mŋɤm}  & hurt (somebody)   \\ 
\jg{ntaβ}  & be stable  &  \jg{ɕɯ-ntaβ}  & put      \\ 
\jg{ngo}  & sick   &  \jg{ɕɯ-ngo}  & cause to become sick  \\ 
\jg{nŋo}  & lose   &  \jg{ɕɯ-nŋo}  & win   \\ 
\jg{ɴqoʁ}  & be hung  &  \jg{ɕɯ-ɴqoʁ}  & hang   \\ 
\jg{rŋo}  & borrow  &  \jg{ɕɯ-rŋo}  & lend     \\ 
\jg{(tɯ-mbrɯ) ŋgɯ}  & be / become angry  &  \jg{(tɯ-mbrɯ) ɕɯ-ŋgɯ}  & anger someone     \\ 
\jg{rŋgɯ}  & lie down  &  \jg{ɕɯ-rŋgɯ}  & cause to lie down, \\
&&&ferment (alcohol)   \\ 
\jg{rga}  & be glad  &  \jg{ɕɯ-rga}  &  please somebody  \\ 
\midrule
\jg{mu}  & be afraid  &  \jg{ɕɯɣ-mu}  & frighten   \\ 
\midrule
\jg{pʰɣo}  &  flee  &  \jg{ɕpʰɣo}  &  take away     \\ 
\jg{lɯɣ}  &  get loose, get free  &  \jg{ɕlɯɣ}  &  drop   \\ 
\midrule
\jg{ɴqoʁ}  & be hung&  \jg{ʑɴɢoʁ}  & hang on a hook  \\ 
\jg{ŋga}  & wear  &  \jg{ʑŋga}  & help someone to wear  \\ 
\jg{mbri}  & cry    &  \jg{ʑmbri}  & play (an instrument) \\ 
\midrule
\jg{tsʰi}  & drink   &  \jg{jtsʰi}  & give to drink  \\ 
\midrule
\jg{qanɯ}  & dark   &  \jg{sqanɯ}  & put in darkness  \\ 
 \bottomrule
\end{tabular}
\end{table}
Some of the verbs above, such as \jg{tsʰi} ``to drink'', \jg{pʰɣo} ``to flee'',  \jg{lɯɣ} ``to get free'' and \jg{rga}  ``to be glad, to like'' , can appear with the regular causative \jg{sɯ-}. In the first cases the meaning of the derived verbs are slightly different:
\begin{enumerate}
\item \jg{sɯ-tsʰi} means ``make s.o. drink'' rather than ``give s.t. to drink to s.o.''
\item \jg{sɯ-pʰɣo} ``make someone escape'' instead of ``take away''
\item \jg{sɯɣ-lɯɣ} ``cause to get free'' instead of ``drop''
\end{enumerate}
We observe that the regular causatives also have a regular semantic derivation from the basic verbs. The irregular causatives of these verbs can be used with the additional regular causative (\jg{sɯ-ɕlɯɣ} ``cause to drop'').   We can infer from there two facts that some of the irregular causatives (not including those in \ipa{ɕɯ-}) are not synchronically causatives of the original verb anymore, as their meaning has started to evolve independently.

The presence of a Tibetan loanword \jg{rga} ``to be glad, to like'' in this list shows that these allomorphs were still productive relatively recently. In the case of the \ipa{ɕɯ-}, \ipa{ɕɯɣ-} and \ipa{s-}, it seems possible to recover their original distribution. 

The original distribution of the allomorphs in alveolo-palatals \ipa{ɕɯ-}, \ipa{ɕɯɣ-}, \ipa{ɕ-} and \ipa{ʑ-} is unclear. They are only marginally restricted by the place of articulation of the initial consonant (they occur with labial, dental, velar and uvular - all except alveolo-palatal and palatal consonants), and occur with both simple initials and complex clusters. 

\ipa{ɕɯɣ-}, like \ipa{sɯɣ-},  probably occurred with intransitive verbs without initial cluster, possibly with labial initials. We will see that a similar \ipa{-ɣ-} instrusive element appears with the applicative and the tropative prefixes.

\ipa{s-} seems to occur with polysyllabic verb stems whose first element begins with a voiceless stop.

It is interesting to note that \jg{ɴqoʁ} has two distinct irregular causatives with different meanings; \jg{ʑɴɢoʁ} is not  anymore a causative from a synchronic point of view, since it can appear with a causative prefix \jg{sɯ-} (meaning ``hang with something'', see \ref{subsub:causation}). The causative \jg{jtsʰi} of \jg{tsʰi} is only found in the Kamnyu dialect of Japhug. In most Japhug dialects, where the verb ``to drink'' is \jg{tʰi}, its irregular causative is \jg{ɕtʰi}. The Kamnyu form results from a dissimilation *\ipapl{ɕtʰi} > *\ipapl{ɕtsʰi} > \ipa{jtsʰi}.


A few causative verbs not derived from intransitives in \ipa{a}-- have the allomorph \ipa{sɤ}--, in particular \ipa{sɤpe} ``do well'' (from \ipa{pe} ``be good'') and \ipa{sɤrmi} ``give a name'' (from \ipa{rmi} ``be named''). The latter could however perhaps be analysed as a denominal verb from \ipa{tɤ-rmi} ``name''.


\subsubsection{Syntactic constructions} \label{subsub:causation}
The causative prefix \ipa{sɯ-} is the most common morphosyntactic device to express causation in Japhug, though not the only one (see \ref{sec:causation.complement}).

When the causative is applied to an intransitive verb, the S becomes the O of the derived verb (\citet[45]{dixon00causative}) . The A of the derived verb corresponds to the causer  or the stimulus of the causation. This is The following example of the verb \jg{sɯ-ɤʁdɤt} [saʁdɤt] ``to cause to slip'', causative of \jg{aʁdɤt} ``to slip'', illustrates this principle:
\begin{exe}
\ex
\gll \ipa{tɯqe} 	\ipa{kɯ} 	\ipa{pɯ́-wɣ-sɯ-ɤʁdɤt} 	\ipa{nɯ} 	\ipa{pɯ-atɤr} 	\ipa{ɲɯ-ŋu} \\
dung \erg{} \pfv{}-\inv{}-\caus{}-slip \dem{} \pfv{}-fall \ipf{}-be \\
 \glt The dung caused him to slip and he fell down (The Demon, 51)
\end{exe} 



With transitive verbs, a different situation is observed. The causative derivation adds an argument, the causer, which becomes the A of the causative verb. Since Japhug verbs cannot encode more than two arguments in their morphology, one of the argument of the base verb must be demoted to leave place for the causator.  Three types of derivations are observed.

First, the causee (the A of the original verb) becomes the O of the derived verb, while the original O is demoted. Most causatives are formed this way:\footnote{3>1 and 2>1 forms are chosen because these are the only ones where the number and person of both arguments are indexed on the verb.}
 \begin{exe}
\ex
\gll  \ipa{aʑo} 	\ipa{ɯ-ʁɟo} 	\ipa{ɲɯ́-wɣ-jtshi-a-ndʑi} 	\ipa{pɯ-ɕti}  \\
  I 3\sg{}:\poss{}-diluted.wine \ipf{}-\inv{}-\caus{}:drink-1\sg{}-\du{} \pst{}.\ipf{}-be.\emphat{} \\
  \glt {They_d} gave me diluted wine to drink. (Kunbzang, 71)
\end{exe} 
 
 \begin{exe}
\ex
\gll  \ipa{nɤ-tɤɲi} 	\ipa{taʁ} 	\ipa{kɤ-rɤt} 	\ipa{nɯ} 	\ipa{ɯβrɤ-kɯ-z-nɤmɲo-a-nɯ} \\
  2\sg{}-staff on \nmlz{}:O-write \topic{} \quind{}-2>1-observe-1\sg{}-\pl{} \\
  \glt Could you_p show me what is written on your staff. (The prince, 61)
\end{exe} 

 \begin{exe}
\ex
\gll \ipa{nɤ-pi} 	\ipa{ni} 	\ipa{kɯ} 	\ipa{nɯ} 	\ipa{ló-wɣ-sɯ-tɕat-a-ndʑi} 	\ipa{ɕti}  \\
2\sg{}.\poss{}-elder.sibling \du{} \erg{} \dem{} \evd{}-\inv{}-\caus{}-take.out-1\sg{}-\du{} \npst{}:be.\emphat{} \\
  \glt Your two elder sisters forced me to spit it out. (The three sisters, 254)
\end{exe} 

Second, we find ambiguous causative forms for some verbs, where either the agent or the patient of the original verb is preserved: in other words, the O of the causative verb can either correspond to the original A of the original O. These to types of derivation would respectively belong to types iv. and v. in \citet[48]{dixon00causative}'s typology): passage of either the A or the O of the original verb to non-core status. Note that in Japhug the non-core status of these arguments is indicated by the absence of agreement on the verb, not by any overt marking on the nominal phrases. 
 
As an example, the causative of \ipa{qur} ``to help''  \ipa{sɯ-qur} has two meanings:

\begin{exe} 
\ex \label{ex:caus:show.2>3>1}
\gll   \ipa{tɤ-kɯ-sɯ-qur-a-ndʑi}  \\
 \pfv{}-2>1-\caus{}-see-1\sg{}-\du{}  \\
 \glt  You_d caused me to help him. OR You_d caused him to help me. \wav{8_tAkWsWqura}
\end{exe} 

Similarly the causative of \ipa{mto} ```to see'' \ipa{sɯ-mto} means either ``to cause X to be seen'' or ``to show to X (to cause X to see)'':

\begin{exe} 
\ex \label{ex:caus:show.2>3>1}
\gll  \ipa{kɯm} 	\ipa{pɯ-a-pa} 	\ipa{ɕti} 	\ipa{ri,} 	\ipa{kɯm} 	\ipa{lɤ-tɯ-cɯ-t} 	\ipa{tɕe,} 	\ipa{tɯrme} 	\ipa{ra} 	\ipa{kɯ} 	\ipa{pɯ-kɯ-sɯ-mto-a}  \\
door \pst{}.\ipf{}-\pass{}-close \npst{}:be.\emphat{} but door \pfv{}-2-open-\pst{} \coord{} people \pl{} \erg{}  \pfv{}-2>1-\caus{}-see-1\sg{}  \\
 \glt The door was closed, but you opened it, you caused me to be seen by the people. (elicited, Chen Zhen 2011) \wav{8_pWkWsWmtoa} 
\end{exe} 


\begin{exe}
\ex
\gll \ipa{kɯki} 	\ipa{laχtɕha} 	\ipa{ki} 	\ipa{wuma} 	\ipa{ʑo} 	\ipa{nɯ-ɕar-a} 	\ipa{ri,} 	\ipa{aʑo} 	\ipa{mɯ-pɯ-mto-t-a} 	\ipa{ri,} 	\ipa{nɤʑo} 	\ipa{kɯ} 	\ipa{pɯ-kɯ-sɯ-mto-a}   \\
this thing this very \emphat{} \pfv{}-search-1\sg{} but I \negat{}-\pfv{}-\caus{}-see-\pst{}-1\sg{} but you \erg{}  \pfv{}-2>1-\caus{}-see-1\sg{} \\
 \glt I looked for this thing for a long time, but could not find it, but you showed it to me. (elicited, Chen Zhen 2011)
\end{exe} 

\begin{exe}
\ex
\gll   \ipa{koŋla} 	\ipa{tɤjpa} 	\ipa{pjɯ-kɯ-sɯ-mto-j} 	\ipa{ɯ-jɤɣ?}   \\
real snow \ipf{}-2>1-\caus{}-see-1\pl{} \intrg{}-\npst{}:be.possible  \\
 \glt Can you show us real snow? (The snow08.25)
\end{exe} 


Here is an example of O-preservation from a narrative:
\begin{exe} 
\ex \label{ex:caus:beaten}
\gll  \ipa{nɯnɯ} 	\ipa{ni} 	\ipa{pjɤ́-wɣ-sɯ-ʁndɯ-ndʑi} 	\ipa{tɕe}  \\
\dem{} \du{} \evd{}-\inv{}-\caus{}-beat-\du{} \coord{} \\
\glt They_p had the two of them beaten (by people). (not to be understood as: ``They made two people beat them'', Fox 126)
\end{exe} 
Preservation of the original O   instead of the A occurs in verbs with  human patients, when the patient is higher than the agent on the empathy hierarchy (in example \ref{ex:caus:show.2>3>1}, first person > third person indefinite), or, when all arguments are third person, when the O of the original verb is more topical than the A (example \ref{ex:caus:beaten}).


Third, the causative is obligatory in sentences with an overt instrument in the ergative case, as in the following sentence:

\begin{exe}
\ex
\gll  \ipa{ɯ-χto} 	\ipa{nɯ} 	\ipa{mbrɯtɕɯ} 	\ipa{kɯ} 	\ipa{kú-wɣ-sɯ-rkhe}  \\
3\sg{}.\poss{}-slit \topic{} knife \erg{}  \ipf{}-\inv{}-\caus{}-carve \\
 \glt  The slit is carved with a knife. (Colored belts 13)
\end{exe} 

The irregular causatives are also used in this way:
\begin{exe}
\ex
\gll \ipa{khɤlɤβ} 	\ipa{kɯ} 	\ipa{tɯthɯ} 	\ipa{pɯ-ɕɯ-fkaβ-a}  \\
cover \erg{} pan \aor{}-\caus{}-cover-1\sg{} \\
 \glt  I covered the pan with a cover.
\end{exe} 
 
 In the sentences above, using the verbs \jg{kú-wɣ-rkhe} and \jg{pɯ-fkaβ-a} without the causative with the overt instrument would be  ungrammatical. 
 %In this construction the O of the causative verb is the instrument, and the original O cannot be encoded > vérifier ɯʑo kɯ scoʁ kɯ tɤ́wɣsɯʁndɯa, ɯʑo
 
 
 Animates can occur as instruments in some rare cases:
 
\begin{exe}
\ex
\gll  \ipa{βʑar} 	\ipa{nɯnɯ} 	\ipa{kɯ,} 	\ipa{nɯnɯ,} 	\ipa{pɣa} 	\ipa{kɯ-xtɕi} 	\ipa{nɯ} 	\ipa{ra} 	\ipa{ʁɟa} 	\ipa{ʑo} 	\ipa{tu-ndze} 	\ipa{ʁɟa} 	\ipa{ʑo} 	\ipa{nɯnɯ} 	\ipa{kɯ} 	\ipa{ɯ-xtu} 	\ipa{chɯ-nɯ-sɯ-χse} 	\ipa{ɲɯ-ŋu} 	\ipa{khi}  \\
 buzzard \dem{} \erg{} \dem{} bird \nmlz{}:S-small \topic{} \pl{} entire \emphat{} \ipf{}-eat[III] entire \emphat{} \dem{} \erg{} 3\sg{}.\poss{}-belly \ipf{}-\auto{}-\caus{}-feed[III] \ipf{}-be \textsc{hearsay} \\
\glt  The buzzard always eat small bird, and always nourishes himself with them, it is said. (The buzzard, 3)
\end{exe} 
 

The causative also occurs in very special syntactic constructions involving stative verbs. First, the causative form of a stative verb can occur with the infinitive of an action verb as its complement, expressing the manner of the action:
\begin{exe}
\ex
\gll 
 \ipa{kɤ-ɣndʑɯr} 	\ipa{chɤ-sɯ-ɤmɲɤm}  \\
 \inftv{}-grind \evd{}-\caus{}-homogeneous \\
 \glt  He ground it homogeneously.
\end{exe} 
In this construction, we observe raising of the directional prefix of the complement verb (in the example above, for instance, the intrinsic directional prefix of \jg{ɣndʑɯr} ``to grind'' is \ipa{thɯ- / chɤ-} ``downstream''). Both the causative verb and the complement transitive verb share the same A and O.

Second, the causativized stative verb occurs as the first element of a serial verb construction, expressing again the manner or circumstances of the second verb:

\begin{exe}
\ex
\gll \ipa{a-tʂha} 	\ipa{ci} 	\ipa{pɯ-z-mɤke} 	\ipa{pɯ-rke} \\
1\sg{}.\poss{}-tea a.little \imp{}-\caus{}-be.before[III] \imp{}-put.in[III] \\
\glt  Serve me some tea first.
\end{exe} 
As in all such constructions (see \ref{sec:serial.verb}), both verbs share the verb TAM and person features.

 
The causee (the original A) can be marked with the ergative, as seen in the examples above. When the causee is an instrument, ergative marking is obligatory, and one can find sentences with two ergatives, though those are rarely attested in stories:
\begin{exe}
\ex
\gll \ipa{nɤ-pi} 	\ipa{ni} 	\ipa{kɯ} 	\ipa{scoʁ} 	\ipa{kɯ} 	\ipa{tú-wɣ-sɯ-ʁndɯ-a-ndʑi} 	\ipa{pɯ-ɕti} 	\ipa{tɕe,} 	\ipa{nɤʑo} 	\ipa{kɯ́nɤ} 	\ipa{nɯ} 	\ipa{tɤ-ste} 	\ipa{jɤɣ}  \\
2\du{}.\poss{}-elder.sibling \du{} \erg{} ladle \erg{} \ipf{}-\inv{}-\caus{}-hit-1\sg{}-\du{} \pst{}.\ipf{}-be.\emphat{} \coord{} \dem{} \imp{}-do.this.way[III] \npst{}:be.possible  \\
\glt   Your two sisters hit me with a ladle, you can do the same. (Sentence retold by Chen Zhen from the story ``The three sisters'') \wav{8_scoRkW}
\end{exe} 


However, we do find  causees without ergative. First, topicalized ones (with a pause after the topicalizer):
\begin{exe}
\ex
\gll  \ipa{phuɲi} 	\ipa{nɯ}, 	\ipa{tʂaqhu} 	\ipa{rŋgɯ} 	\ipa{taʁ} 	\ipa{a-ʑ-lɤ-sɯ-rpe} 	\ipa{ra}  \\
 broom.shrub \topic{} side.of.the.road rock on \irr{}-\transloc{}-\pfv{}:upstream-\caus{}-bump.into[III] \npst{}:have.to   \\
 \glt   With the broom shrub, you will have to touch the rock on the road. (Smanmi2.62)
\end{exe} 
 

 
Second, when the causee is human, the ergative rarely appears (though it is not agrammatical): 
 
 \begin{exe}
\ex
\gll \ipa{tɕheme} 	\ipa{nɯ} 	\ipa{kɯjŋu} 	\ipa{kɯ-wxtɯ-wxti} 	\ipa{ʑo} 	\ipa{pa-sɯ-ta-ndʑi} \\
girl \topic{} oath \nmlz{}:\stat{}-\intens{}-big \emphat{} \aor{}:3>3-\caus{}-put-\du{} \\
 \glt   They forced the girl to make a great oath. (Fox, 141)
\end{exe} 
  
  
When the original verb is intransitive, the causee is not marked with ergative:
 
 \begin{exe}
\ex
\gll \ipa{tɤse} 	\ipa{mtshu} 	\ipa{tú-wɣ-sɯ-mtshɤt} \\
blood lake \ipf{}-\inv{}-\caus{}-full \\
 \glt   We have to fill the lake with blood. (Smanmi2.95)
\end{exe} 
The stative verb\jg{mtsʰɤt} ``be full'' can appear with both the container and the containee without case marking (the container is the real S, while the containee is an adjunct). Adding causative marking on the verb does not promote the containee to core argument status.

\subsubsection{Compatibilities} \label{subsub:caus1.compatibility}
The  causative \jg{-sɯ-} is highly productive and can appear with various other derivational prefixes, including the  reflexive \jg{-ʑɣɤ-}, the causative \jg{ɣɤ-}, the passive \jg{a-}, the autobenefactive-spontaneous \jg{-nɯ-} and the derivational prefixes in slot 13 (cf. table \ref{tab:template:derivational}).

It is incompatible with the anticausative (\ref{sub:anticausative}) and the deexperiencer \jg{sɤ-} (\ref{sub:deexperiencer}).

%tropative, facilitative

The reflexive  \jg{-ʑɣɤ-} occurs in slots 7, just before the causative. 	Only the order  \jg{-ʑɣɤ-sɯ-} is attested, and the reverse order is unintelligible to Japhug speakers.

 The combination of these two prefixes could potentially have to interpretations: either X cause Y to do to X (scope of the reflexive over the causative), or X cause to Y to do to Y (scope of the causative over the reflexive). However, only the first interpretation is possible, as shown by the examples:
%\citet[214]{bickel07inflectional}
\begin{exe}
\ex 
\gll \ipa{pɯ-ʑɣɤ-sɯ-sat}  \\
  \aor{}-\refl{}-\caus{}-kill \\
\glt   He_i caused (him, them) to kill himself_i
\end{exe}
	\begin{exe}
\ex 
\gll \ipa{ɯʑo} 	\ipa{mɯ-to-tɯndzaŋspa} 	\ipa{tɕe} 	\ipa{pjɤ-ʑɣɤ-sɯ-mto} \\
 he \negat{}-\evd{}-careful \coord{} \evd{}-\refl{}-\caus{}-see \\
\glt   He wasn't careful enough and got himself seen. \wav{8_ZGAsWmto}
\end{exe}

The two sentences above cannot be understood as  ``He_i caused (him, them)_j to kill (him,them)selve(s)_j'' or ``he_i caused him_j to see himself_j''.


The causative commonly appears with the autobenefactive-spontaneous \ipa{nɯ-}, which is located between positions 6 and 7 if no passive prefix in present in the verb form. The following example illustrates the combination of these prefixes:


 \begin{exe}
\ex 
\gll \ipa{ɯ-sci} 	\ipa{iɕqha} 	\ipa{ɯ-sroʁ} 	\ipa{nɯ} 	\ipa{ɯ-kɤ-kɯ-ri} 	\ipa{nɯnɯ} \ipa{ʑ-la-nɯ-sɯ-ɣe-nɯ} \\
3\sg{}.\poss{}-replacement the.aforementioned 3\sg{}.\poss{}-life \topic{} 3\sg{}-\aor{}-\nmlz{}:A-save \dem{} \transloc{}-\aor{}:3>3-\auto{}-\caus{}-come-\pl{} \\
\glt In his_j place, they (send people) to invite him_i to come, he_i who saved her life.    (the demon, 162)
\end{exe}

 The causative also appears in combination with the passive as \ipa{sɤ}-- < \ipa{sɯ-ɤ}--, but only in a limited number of verbs:
 
 
  \textbf{\ipa{sɤmbi}} ``to require something from someone'' is a causative form derived from the passive \ipa{a-mbi} ``to be given'' of the verb \ipa{mbi} ``to give''. Etymologically, the verb means ``to cause someone to give to oneself''. The exact pathway of derivation is however not entirely clear. While \ipa{mbi} ``to give'' has the recipient encoded as the O, not the object given, its passive form \ipa{a-mbi} has the object given as the S (see \Ref
  
  \ipa{sɤmbi} is labile, as it can be used with both transitive and intransitive morphology. One can say both:
   \begin{exe}
\ex
\gll   \ipa{nɯ-sɯ-ɤ-mbi-a}  / \ipa{nɯ-sɯ-ɤ-mbi-t-a}\\
\aor{}-\caus{}-\pass{}-give-1\sg{} /  \aor{}-\caus{}-\pass{}-give-\pst{}-1\sg{} \\
 \glt I asked for it. (el, Chen Zhen \wav{8_sAmbi})
\end{exe}   
  The form \ipa{nɯ-sɯ-ɤ-mbi-t-a} with the past transitive 1/2\sg{}>3 --\ipa{t} suffix (see \ref{sec:transitivity}) is morphologically transitive, while the other one is intransitive. However, even the morphologically intransitive \ipa{nɯ-sɯ-ɤ-mbi-a} can appear with direct objects; the person asked has dative marking:
    \begin{exe}
\ex
\gll \ipa{a-pi} 	\ipa{ɯ-ɕki} 	\ipa{kɯmtɕhɯ} 	\ipa{ci}   \ipa{nɯ-sɯ-ɤ-mbi-a}    \\
1\sg{}.\poss{}-elder.sibling 3\sg{}-\dat{} toy \indef{} \aor{}-\caus{}-\pass{}-give-1\sg{}   \\
 \glt I asked my brother for a toy. (el, Chen Zhen \wav{8_sAmbi})
\end{exe}   
  
  
  
    \textbf{\ipa{sɤjtsʰi}} ``to ask for something to drink'' derives from the irregular lexicalized causative \ipa{jtsʰi} ``to give to drink''. The etymological causative prefixes \ipa{j}-- being fossilized and not analysed synchronically as such in modern Japhug, this form is not a counterexample to the verbal template. As \ipa{mbi} ``to give'', \ipa{jtsʰi} has the recipient coded as the O:
      \begin{exe}
\ex
\gll   \ipa{a-wɯ} 	\ipa{tɯ-ci} 	\ipa{ɲɯ-kɯ-jtshi-tɕi} 	\ipa{ɯ́-jɤɣ}  \\
1\sg{}.\poss{}-grandfather \neu{}-water \ipf{}-2>1-give.to.drink-1\du{} \qu{}-\npst{}:could \\
 \glt Grandfather, could you give us water to drink? (Nima Vodzer 72)
\end{exe}   
 The resulting derived verb \ipa{sɯ-ɤ-jtshi} ``to ask for something to drink'', like \ipa{sɯ-ɤ-mbi}, is labile and can appear with intransitive morphology:
    
    \begin{exe}
\ex
\gll   \ipa{tɯ-ci} 	\ipa{z-ɲɯ-kɯ-sɯ-ɤ-jtshi} 	\ipa{ɲɯ-ntshi}  \\
\neu{}-water \transl{}-\ipf{}-\genr{}:S/O-\caus{}:\pass{}-give.to.drink \ipf{}-have.to  \\
 \glt  (We) have better go to ask for water to drink.   (Nima Vodzer 70)
\end{exe} 
  
  \textbf{\ipa{sɤβzu}} ``to prepare, to make ready to use'' derives from \ipa{a-βzu}, a verb whose meaning in modern Japhug is ``to grow'', but which originally was the passive of \ipa{βzu} ``to make''. \ipa{sɤβzu}  is therefore etymologically ``to cause to be made''. Unlike   \ipa{sɤmbi} and \ipa{sɤjtshi}, \ipa{sɤβzu} is always labile. The reflexive prefix \ipa{ʑɣɤ}-- can further be added to form the verb \ipa{ʑɣɤ-sɯ-ɤβzu} ``to transform oneself into".

  \textbf{\ipa{sɤpa}} ``transform (tr.)'' is the causative of \ipa{apa} ``become'', itself the passive of the verb \ipa{pa}, which means ``close (the door)'' in modern Japhug (among other meaning) but used to be the  regularly verb ``to do'' in Rgyalrongic languages. \ipa{sɤpa} is always transitive.
  
  This verb can in turn be combined with the reflexive \ipa{ʑɣɤ}-- to form \ipa{ʑɣɤ-sɯ-ɤ-pa} ``to transform oneself into'':
  
  \begin{exe}
\ex
\gll \ipa{ɯ-tɕɯ} 	\ipa{nɯ} 	\ipa{ɕkɤrɯ} 	\ipa{na-sɯ-ɤpa,} 	\ipa{ɯʑo} 	\ipa{xtɯt} 	\ipa{nɯ-ʑɣɤ-sɯ-ɤpa} 	\ipa{ɲɯ-ŋu} \\
3\sg{}.\poss{}-son \topic{} serow \aor{}.3>3-\caus{}-become she wild.cat \aor{}-\refl{}-\caus{}-become \ipf{}-be\\
 \glt She changed her son into a serow, and herself into a wild cat.  (Lobzang 54)
\end{exe} 


It is quite clear that the combination of the causative with the passive is not productive in Japhug, and that the analysis proposed above is only true in a diachronic perspective. The four examples above have to be divided into two groups:   \ipa{sɤmbi} and \ipa{sɤjtshi}, which are labile verbs, and \ipa{sɤβzu} and \ipa{sɤpa}, which are exclusively transitive. In the second group,  the element \textit{a/ɤ}-- is not synchronically a passive prefix per se, as the meaning of the derived verb and that of the based verb are too divergent, and the two verbs have evolved separately.


The causative \ipa{sɯ-} is also compatible with the causative \ipa{ɣɤ-}, though such examples are unusual. It is usually used with a noun of instrument:
 \begin{exe}
\ex  \label{ex:double.caus}
\gll \ipa{smɤnba} 	\ipa{kɯ} 	\ipa{smɤn} 	\ipa{ɲo-kho} 	\ipa{tɕe,} 	\ipa{ɯ-kɯ-mŋɤm} 	\ipa{to-z-ɣɤ-mna}    \\
  doctor \erg{} medicine \evd{}-give \coord{} 3\sg{}-\nmlz{}:\stat{}-hurt \evd{}-\caus{}-\caus{}-be.cured \\
\glt The doctor gave him a medicine and cured him with it. \wav{8_zGAmna}
\end{exe}
 


%The existence of many \ipa{sɤ-} derivation prefixes homophonous with the combination \ipa{sɯ-ɤ-} makes


 
 



\subsubsection{The semantics of the causative } \label{subsub:caus.semantics}
In his cross-linguistic overview of causatives, \citet[62-68]{dixon00causative} proposes nine parameters to study the semantic specificities of causative constructions. The first two, state vs. action and transitivity, are treated in the morphology, and will not concern us here. 

Of the seven remaining parameters, three ones (Control, Volition, Affectedness) relate to the causee, and four (Directness, Intention, Naturalness, Involvement) to the causer; in this section, we will regroup them into four groups by combining directness, naturalness and involvement, as these three parameters are most often intertwined in our examples.

 
 We will show that the \ipa{sɯ-} causative in Japhug has a wide range of uses, and can appear independently of these parameters. Is it compatible with either obligation, authorisation, accompaniment and various modes of causation. 
 
   \textbf{Control}. The prefix \ipa{sɯ-} occurs both with actions on which the causee has control, but also with action on which he/it has no control, either because it is inanimate, or because the action itself is not controllable:
 \begin{exe}
\ex 
\gll  \ipa{ɯrɟɤnpanma} 	\ipa{kɯ} 	\ipa{tɯɣ} 	\ipa{pjɤ-lɤt,} 	\ipa{tɕendɤre} 	\ipa{nɯ-wa} 	\ipa{ko-z-nɤndza,}  \\
Padmasambhava \erg{} poison \evd{}-use \coord{} 3\pl{}.\poss{}-father \evd{}-\caus{}-have.leprosy \\
 \glt Padmasambhava used a poison, and caused their father to contract leprosy.    (Gesar 15)
\end{exe}

 \textbf{Volition}. The causative \ipa{sɯ-} appears regardless whether causee acts willingly (`let, ask') or unwillingly (`make, force').
 
 This first example shows that the causative can be used when doing a favour to someone:
  \begin{exe}
\ex 
\gll   \ipa{a-mu} 	\ipa{ndʑu} 	\ipa{ci} \ipa{nɤ}	\ipa{ʑo} 	\ipa{a-mɤ-nɯ-tɯ-sɯ-qlɯt-nɯ} \\
  1\sg{}.\poss{}-mother chopsticks a.little even \emphat{} \irr{}-\negat{}-\pfv{}-2-\caus{}-break-\pl{} \\
 \glt    Please make sure that my mother does not even need to break chopsticks (go out to break twigs from the trees to make chopsticks; this idiomatic expression means ``take care of her every need''). (Slobdpon, 220)
\end{exe}

 It can also be used when one asks someone to do something:
 \begin{exe}
\ex 
\gll  \ipa{βlama} 	\ipa{kɯ-wxti} 	\ipa{ʑo} 	\ipa{ɲɤ-sqar-nɯ} 	\ipa{tɕe,} 	\ipa{tɤ-rpi} 	\ipa{kɯ-wxtɯ-wxti} 	\ipa{ʑo} 	\ipa{ɲɤ-sɯ-βzu-nɯ}  \\
lama \nmlz{}:\stat{} \emphat{} \evd{}-ask.to.do-\pl{} \coord{} \neu{}-sutra \nmlz{}:\stat{}-\intens{}-big \emphat{} \evd{}-\caus{}-do-\pl{} \\
 \glt   They employed a great lama and asked him to recite a major sutra. (Rkang-rgyal, 19-20)
\end{exe}

Finally, it can also express coercion, with adverbs such as \ipa{tɤrkoz} 	or \ipa{mɤkɯftshi} ``forcefully'':
 \begin{exe}
\ex 
\gll \ipa{kɤ-ndza} 	\ipa{a-ʁjiz} 	\ipa{mɯ́j-ɣi} 	\ipa{ri} 	\ipa{ɯʑo} 	\ipa{kɯ} 	\ipa{tɤrkoz} 	\ipa{thɯ́-wɣ-sɯ-ndza-a} \\
\nmlz{}:O-eat 1\sg{}-will \negat{}:\const{}-come but he \erg{} forcefully \aor{}-\inv{}-\caus{}-eat-1\sg{} \\
 \glt   I did not want to eat it, but he forced me to. (Chen Zhen, 2005)
\end{exe}

 
   \textbf{Intention}. The causative prefix can appear with unintentional actions:
 \begin{exe}
\ex 
\gll 
\ipa{tɯ-ŋga} 	\ipa{ɲɤ-sɯ-ɤrŋi-t-a} \\
	\neu{}-clothes \evd{}-\caus{}-blue-\pst{}-1\sg{} \\
 \glt   I caused the clothes to become blue (unintentionally, by washing them the wrong way; el., Chen Zhen)
\end{exe}
Notice the use of the evidential with first person for unintentional actions (cf. \ref{sub:evd}).
% \begin{exe}
%\ex 
%\gll 
% 	\ipa{tɯthɯ} 	\ipa{kɯ} 	\ipa{a-jaʁ} 	\ipa{ʂaʁ} 	\ipa{kɤ-nɯ-sɯ-ta-t-a} \\
% 	pan \erg{} 1\sg{}.\poss{}-hand burn \aor{}-\auto{}-\caus{}-put-\pst{}-1\sg{} \\
% 	\glt   I burned my hands with the pan.
%\end{exe} > ko-nɯ-sɯ
 	

   \textbf{Directness, naturalness and involvement}.  The causative prefix \ipa{sɯ-} can express various degrees of involvement on the part of the causer, as exemplified by the following example:
 
  \begin{exe}
\ex 
\gll \ipa{ɯ-mbro} 	\ipa{kɯ} 	\ipa{qapri} 	\ipa{tɯ-rdoʁ} 	\ipa{nɯ} 	\ipa{pjɤ-z-rɤtɕaʁ} 	\ipa{tɕe,} 	\ipa{tɤte} 	\ipa{kɯ-wɣrum} 	\ipa{nɯ} 	\ipa{lo-sɯ-qioʁ} \\
3\sg{}.\poss{}-horse \erg{} snake one-piece \pl{} \evd{}-\caus{}-trample \coord{} that.is \nmlz{}:\stat{}-white \topic{} \evd{}\caus{}-vomit \\
 \glt  (Nyima Wodzer) had his horse trample one of the snakes, and caused it to throw up the white one (snake). (Nyima Wodzer,30)
\end{exe}
In the first  clause, the causee of the verb verb \ipa{pjɤ-z-rɤtɕaʁ} 	``he caused him to trample'' is the horse, while in the second one, the causee is the snake that was trampled; the causer (the character Nyima Wodzer) in the second case only acts indirectly (through the action of his horse).
 
 
 
 
 The causative is also used to express authorisation, where the causer's involvement is even more indirect, and only amount to an absence of action:

 \begin{exe}
\ex 
\gll \ipa{ku-kɯ-z-rɤʑi-a-nɯ} 	\ipa{ɲɯ-ntshi} \\
\ipf{}-2>1-\caus{}-stay-1\sg{}-\pl{} \ipf{}-have.better \\
 \glt  Could you let me stay? (The raven, 68)
\end{exe}

Finally, it can even  appear in situations where the ``causer'' merely omits to act upon a naturally occurring event:

  \begin{exe}
\ex 
\gll \ipa{tɤmthɯm} 	\ipa{ɲɤ-z-ɣɤdi-t-a}  \\
 meat \evd{}-\caus{}-be.smelly-\pst{}-1\sg{} \\
 \glt  I let the meat get spoiled.
\end{exe}

	 



The  scope of the negation in relation to the causative presents ambiguities. The negative prefix can either have scope over the base verb (cause not to do = hinder) or over the causative (not cause to do). This may be an effect of the rigid  verbal template, as the relative order of the negation and the causative are strictly fixed.

Examples with negation with negative in the sense of ``hinder'', ``cause not to do'') are quite common:


  \begin{exe}
\ex
\gll \ipa{a-ʑɯβ} \ipa{mɯ́j-sɯ-ɣe-nɯ} \\
	1\sg{}-sleep \negat{}:\const{}-\caus{}-come-\pl{} \\
    \glt They don't let me sleep. NOT ``They do not cause me to sleep'' (Dpalcan, 2010, elicitation)
  \end{exe} 
  
  \begin{exe}
\ex
\gll  \ipa{ɯ-tɯ-ɣɤcraŋlaŋ} 	\ipa{kɯ} 	\ipa{koŋla} 	\ipa{mɯ́j-kɯ-z-rɤ-βzjoz} 	\ipa{ʑo} \\
3\sg{}-\nmlz{}:\degr{}-make.noise \erg{} really \negat{}:\const{}-\genr{}:S/O-\caus{}-\apass{}-learn \emphat{} \\
  \glt  They make so much noise that they do not let people study at all.
  \end{exe}   
  
   \begin{exe}
\ex
\gll   \ipa{nɯtɕu} 	\ipa{ku-je} 	\ipa{tɕe} 	\ipa{tɯ-ci} 	\ipa{tɯ-mɯm} 	\ipa{mɯ-pjɤ-sɯ-tshi} 	\ipa{ɲɯ-ŋgrɤl} \\
there \ipf{}-keep.in.enclosure \coord{} \neu{}-water one-swallow \negat{}-\evd{}-\caus{}-drink \ipf{}-be.usually.the.case \\
\glt (The male deer) keep (the female) in a place, and do not let them drink even a swallow of water. (dictionary entry on ``deer'', 2005)
  \end{exe} 
  
\begin{exe}
\ex
\gll  \ipa{aʑo} 	\ipa{ɲo-nɯ-jmɯt-a} 	\ipa{tɕe,} 	\ipa{rɟɤlpu} 	\ipa{ɯ-ɕki} 	\ipa{pɣɤtɕɯ} 	\ipa{kɯ} 	\ipa{mɯ-tɤ-sɯ-tɯt-a}   \\
 I \evd{}-\autoben{}-forget-1\sg{} \coord{} king 3\sg{}-\dat{} bird \erg{} \negat{}-\aor{}-\caus{}-say[II]-1\sg{} \\
  
  \glt  I forgot it, so that because of me the bird did not convey its message to the king (litterally: I did not let the bird tell the king)\footnote{This example is adapted from a traditional story; the speaker here is the shepherd Askyabs \ipa{kɯlɤɣ acɤβ}, whom a bird (in fact a reincarnated queen) asks to deliver a message to the king. The bird itself does not go to see the king directly.  } \wav{8_mWtAsWtWta2}
  \end{exe} 

  However, the other interpretation, with the scope of the negation on the causation is also possible:

  \begin{exe}
\ex
\gll \ipa{aʑo} 	\ipa{ɕ-tɤ-nɯ-tɯt-a} 	\ipa{ma} 	\ipa{tɯrme} 	\ipa{mɯ-tɤ-sɯ-tɯt-a} \\
I \transloc{}-\pfv{}-\auto{}-say[II]-1\sg{} apart.from people \negat{}-\aor{}-\caus{}-1\sg{} \\

  \glt    I went to convey (the message) myself,  and I did not make anybody else convey it. \wav{8_mWtAsWtWta}
  \end{exe} 
 
As mentioned in \ref{subsub:caus:morphophon},  in the case of the verb ``to drink'' \jg{tsʰi}, two distinct causatives are found, a regular one \jg{sɯ-tsʰi} and an irregular one \jg{jtsʰi}. The second one is used when the causee (which is encoded as O) has control and is willing to perform the action. The second one appears when the causee has no control or is unwilling.
%  \begin{exe}
%\ex
%\gll
%
%
%  \glt   
%  \end{exe} 


Other homophonous \ipa{sɯ-} prefixes exist in Japhug apart from  the causative: the abilitative (\ref{sub:abilitative}) and one of the denominal prefixes (\ref{sec:denominal}) have the same form, and ambiguity appears in some cases: \ipa{sɯ-ndza} for instance can mean either ``cause to eat'', ``eat with'' or ``be able to eat''.

We find several examples of causative prefixes, but whose semantics would rather remind of the tropative (cf. \ref{sub:tropative}): \jg{znɤja} ``consider to be a shame'', \jg{sɯpa} ``regard as'' and \jg{znɤkɤro} ``consider to be acceptable''.

The intransitive verb \jg{nɤja} means ``to be a shame, to be a pity''.
  \begin{exe}
\ex
\gll \ipa{iɕqha} 	\ipa{laχtɕha} 	\ipa{pjɤ-ɴɢrɯ,} 	\ipa{pɯ-nɤja} \\
the.aforementioned thing \evd{}-\acaus{}:break \aor{}-be.a.shame \\
  \glt  That thing broke, what a shame!
   \end{exe}

   The transitive \jg{z-nɤja}, rather than meaning ``to cause to be a shame'' as expected regularly, rather means ``to regret, be reluctant'' (Chinese \zh{不舍得}), in other words ``to consider something to be a pity'':
   
     \begin{exe}
\ex 
\gll \ipa{wuma} 	\ipa{ʑo} 	\ipa{pɯ-znɤja-t-a}  \\
very \emphat{} \aor{}-regret-\pst{}-1\sg{} \\
  \glt  I regretted it very much (a lost cellphone cover, Dpalcan, conversation, 2010)
   \end{exe}
Another verb having unpredictable semantics with the prefix \ipa{sɯ-} is the transitive verb \ipa{sɯ-pa} ``to consider, to regard as''. The original verb here is \ipa{pa} ``to close'', etymologically ``to do'':

     \begin{exe}
\ex 
\gll \ipa{tɤkhe-pɣɤtɕɯ} 	\ipa{nɯ} 	\ipa{ɯʑo} 	\ipa{pɣɤtɕɯ} 	\ipa{nɯ} 	\ipa{kɯ-khe} 	\ipa{tu-sɯpa-nɯ} \\
stupid-bird \topic{} he bird \topic{} \nmlz{}:\stat{}-stupid \ipf{}-consider-\pl{} \\
 \glt The \ipa{tɤkhe-pɣɤtɕɯ} is considered to be a stupid bird (the buzzard, 13)
   \end{exe}

%   XXXXX ajouter sɯχsɤl
 \subsubsection{The causative \ipa{sɯ-} with stative verbs} \label{subsub:caus.sW.stative}
 Although the prefix \ipa{ɣɤ-}, rather than \ipa{sɯ-}, is used with most stative verbs, some stative verbs only appear with \ipa{sɯ-}. The following non-exhaustive list illustrates some examples:
 \begin{table}[H]
\caption{Examples of the \ipa{sɯ}- causative with stative verbs }\label{tab:causative.sW.stative}
\begin{tabular}{lllllll} \toprule
  base  & &causative  \\
\midrule
   \ipa{arŋi} & blue & \ipa{sɯ-ɤrŋi} \\
 \ipa{wɣrum} & white & \ipa{sɯ-wɣrum} \\
 \ipa{ɲaʁ} & black & \ipa{sɯɣ-ɲaʁ} \\
  \ipa{ɣɯrni} & red & \ipa{z-ɣɯrni} \\
    \ipa{mɤrtsaβ} & spicy & \ipa{z-mɤrtsaβ} \\
       \ipa{mŋɤm} & be painful & \ipa{ɕɯ-mŋɤm} \\
\bottomrule
\end{tabular}
\end{table}
   The distinction between the two prefixes is partly arbitrary, but \ipa{ɣɤ-}, as will be seen in  \ref{subsub:caus-g:semantics}, implies  direct involvement of the actor in the causation process  and absence of control and volition on the part of the causee. 
   
   However, the distribution of  \ipa{ɣɤ-} and \ipa{sɯ-} with stative verbs is not purely determined by semantics, but also by the morphological structure of the verbs in question. 
   
    Stative verbs with a prefixal element (\ipa{mɤ-}, \ipa{rɤ-}, \ipa{ɣɯ-} etc), never appear with \ipa{sɯ-}, with the exception of a few verbs with the intransitive determiner \ipa{a-} such as \ipa{ala}  ``soaked'' > \ipa{ɣɤ-la}  ``soak''. This explains for instance why the causative of \ipa{mɤrtsaβ}  ``spicy'' is  in \ipa{z-} rather than \ipa{ɣɤ-}, while almost all other stative verbs denoting feelings or taste have a causative in \ipa{ɣɤ-}, for instance \ipa{tɕur}  ``sour'' > \ipa{ɣɤ-tɕur}  ``make too sour'',   \ipa{tsri}  ``salty'' > \ipa{ɣɤ-tsri}  ``make too salty'' etc.  
  
Color stative verbs and stative verbs related to disease and pain (\ipa{ngo} ``sick'', \ipa{mŋɤm} etc) also do  form their causative with \ipa{sɯ-} and its variants rather than with \ipa{ɣɤ-}, as seen in the table above.
  
  Very few stative verbs have been found to be compatible with both \ipa{ɣɤ-} and \ipa{sɯ-}; these cases will be seen in \ref{subsub:caus-g:semantics}.
 
 
 
 
 \subsubsection{Counterexample to the template} \label{subsub:counterexample.caus}
 The causative \ipa{sɯ-} appears in position 7 of the template (cf. table \ref{tab:template:derivational} p.\pageref{tab:template:derivational}), after the reflexive \ipa{ʑɣɤ-} and the autobenefactive-spontaneous \ipa{nɯ-} but before all other derivational prefixes. Only one counterexample to this position has been discovered: the causative \ipa{sɤzmbrɯ} ``to cause someone to become angry'', whose corresponding intransitive verb is \ipa{sɤmbrɯ} ``to get angry'', derived from the noun \ipa{tɯ-mbrɯ} ``anger''.
  
 
 This causative is anomalous in two ways. First, the prefix is realized \ipa{-z-} instead of \ipa{sɯ-}, though no derivation prefix in sonorant is found. Second, instead of appearing in position 8, it is inserted between the ``denominal'' \ipa{sɤ-} and the verb stem, a position where no prefix normally occurs.

\subsection{Causative (stative verbs)} \label{sub:caus2}

The causative \ipa{ɣɤ-} has a much more restricted usage that \ipa{sɯ-} studied in the previous section. It appears with most stative verbs, though as we have seen in \ref{subsub:caus.sW.stative},  some stative verbs also appear with \ipa{sɯ-}; the semantic differences between the two prefixes for stative verbs will be studied in \ref{subsub:caus-g:semantics}.


Unlike \ipa{sɯ-}, \ipa{ɣɤ-} presents no allomorphy. With verbs having the intransitive determiner \ipa{a-}, this syllable is absorbed by the prefix. For instance, the causative of \jg{artɯm} ``round'' is \jg{ɣɤ-rtɯm} ``to coil (threads)''.
\subsubsection{Syntactic constructions} \label{subsub:caus-g:syntax}
Unlike \ipa{sɯ-}, \ipa{ɣɤ-} only appears with stative intransitive verbs. The added argument, the causer, is always the A, while the original S becomes the O.

\begin{exe}
\ex 
\gll \ipa{ɯ-mke}  	\ipa{cho-ɣɤ-rɲɟi} \\
3\sg{}-neck \evd{}-\caus{}-long \\
 \glt  He stretched his neck (el. Dpalcan 2010)
   \end{exe}
 
 \begin{exe}
\ex 
\gll   \ipa{ɯ-phɯ} \ipa{ɲɯ-wxti} \ipa{tɕe}, \ipa{nɯ} \ipa{ra} \ipa{thamtɕɤt} \ipa{ma-tɤ-tɯ-ɣɤ-wxti},\\
 3\sg{}.\poss{}-price \const{}-big \coord{} \dem{} \pl{} all \negat{}-\imp{}-2-\caus{}-big \\
 \glt  It is expensive, don't make it that expensive. (Bargaining, 11) 
   \end{exe}
 
 
As the prefix \ipa{sɯ-} (see \ref{subsub:causation}), causative verbs with \ipa{ɣɤ-} are used with a complement in \ipa{kɤ-} infinitive to express the manner in which the action takes place:

\begin{exe}
\ex 
\gll \ipa{paʁndza}  	\ipa{kɤ-rɤkrɯ}  	\ipa{pa-ɣɤ-ndɯβ}  \\
hogwash \inftv{}-cut \aor{}:3>3-\caus{}-fine \\
 \glt  He chopped the hogwash very fine (el. Dpalcan 2010)
   \end{exe}
   
 \begin{exe}
\ex 
\gll \ipa{kɯm}  	\ipa{tɤ-ɣɤ-βdi-t-a}  	\ipa{tɕe,}  	\ipa{kɤ-cɯ}  	\ipa{tɤ-ɣɤ-khɯ-t-a}  	  \\
door \evd{}-\caus{}-good-\pst{}-1\sg{} \coord{} \inftv{}-open \evd{}-\caus{}-be.possible-\pst{}-1\sg{} \\
 \glt   I repaired the door, so that it can be opened (literally: I made the door openable, el. Chen Zhen, 2011)
   \end{exe}

    
As with the construction involving the prefix \ipa{sɯ-}, we observe that the directional prefix of the complement verb (in the infinitive) is raised to the causativized stative verb: \ipa{pɯ-} ``down'' and  \ipa{tɤ-} ``down'' are respectively the intrinsic directional prefixes of \ipa{rɤkrɯ} ``cut'' and  \ipa{cɯ} ``open'' (this verb also occurs with \ipa{kɤ-} ``towards east''). 
 
In this construction, the scope of the negation is normally on the causativized stative verb, not on the whole action:
  \begin{exe}
\ex 
 \gll \ipa{kɤ-rɤt} \ipa{mɯ-pjɤ-tɯ-ɣɤ-βdi-t} \\
\inftv{}-write \negat{}-\evd{}-2-\caus{}-good-\pst{} \\
 \glt    You did not write it well (``you wrote it badly'', not in the sense ``you did not write it at all''), (el. Chen Zhen 2011)
   \end{exe}  
   \begin{exe}
\ex 
 \gll  	 \ipa{khɤdaʁ}  	\ipa{tɤ-sɯfsaŋ}  	\ipa{tɕe}  	\ipa{thɯ-mqlaʁ}  	\ipa{ma}  	\ipa{kɤ-sci}  	\ipa{mɯ-nɯ-tɯ-ɣɤ-khɯ-t}  \\
Khatag \imp{}-fumigate \coord{} \imp{}-swallow otherwise \inftv{}-be.born \negat{}-\aor{}-\caus{}-be.possible-\pst{} \\
 \glt   Fumigate a \ipa{khatag} and swallow it, otherwise you would make my birth impossible (\textsc{not}: ``you did not make my birth possible''). (Gesar, 61)
   \end{exe}  
   
   
   The raising of the directional prefix from the complement verb to the causativizer stative verb can remain even when the complement verb is elided. For instance, with a verb such as \ipa{ɣɤ-xtɯt} ``shorten'', one can distinguish between:
   
   \begin{exe}
\ex 
 \gll  	 \ipa{nɯ-ɣɤ-xtɯt-a} \ipa{tɤ-ɣɤ-xtɯt-a}   	   \\
 \aor{}-\caus{}-short-1\sg{} \aor{}-\caus{}-short-1\sg{} \\
 \glt    I made it shorter. (el., Chen Zhen)
   \end{exe}     
   
 The first form means ``shorten by cutting (clothes)'', as the implicit complement verb is \ipa{qrɯ} ``cut'', whose intrinsic directional prefix (in the meaning ``to cut clothes'') is \ipa{nɯ}--.  \ipa{nɯ-ɣɤ-xtɯt-a} is actually a short form for:
 
   \begin{exe}
\ex 
 \gll  	\ipa{tɯ-ŋga} \ipa{kɤ-qrɯ} \ipa{nɯ-ɣɤ-xtɯt-a}  	   \\
 \neu{}-clothes \inftv{}-cut \aor{}-\caus{}-short-1\sg      \\
 \glt    I made the clothes shorter. (el., Chen Zhen)
   \end{exe}
 \ipa{tɤ-ɣɤ-xtɯt-a} , with the prefix \ipa{tɤ}-- ``up'' instead means that the clothes were made shorter by rolling sleeves up, without cutting the cloth.
 
 
\subsubsection{Compatibilities} \label{subsub:caus2:compat}
The prefix \ipa{ɣɤ-}, located in slot 10 of the template, is incompatible with other valency-increasing prefixes such as the tropative  \ipa{nɤ-} and the applicative  \ipa{nɯ-}. However, both the reflexive  \ipa{ʑɣɤ-} (position 7) and causative  \ipa{sɯ-} (position 8) can appear before it:
 \begin{exe}
\ex 
\gll \ipa{nɤʑo}  	\ipa{tɤ-muj}  	\ipa{sthɯci}  	\ipa{a-tɤ-tɯ-ʑɣɤ-ɣɤ-ʑo,}  	\ipa{nɤ-mbro}  	\ipa{qale}  	\ipa{sthɯci}  	\ipa{a-nɯ-ʑɣɤ-ɣɤ-mbjom}  	  \\
you \neu{}-feather  so.much \irr{}-\pfv{}-2-\refl-\caus{}-light, 2\sg{}.\poss{}-horse wind so.much \irr{}-\pfv{}-\refl-\caus{}-fast \\
 \glt   May you be as light as a feather, and your horse as swift as the wind. (Smanmi Metog Koshana, 62)
   \end{exe}
Example  \ref{ex:double.caus} above (\ref{subsub:caus1.compatibility}) illustrates a verb with both causative prefixes.

The causative \ipa{ɣɤ-}, as mentioned above, cannot appear when any of the following positions 11-14 is filled. It never occurs with the deexperiencer \ipa{sɤ-}, any denominal prefix, an incorporated noun, and an anticausative verb. It is also incompatible with bisyllabic stative verbs whose first element is probably prefixal etymologically (and thus in position 13), such as \ipa{mɤ-} or \ipa{ɣɯ-}. 
%\xv nɤsɯm nɯnɯɣɤβdi \xn 你放心吧/你死了这条心吧


It is however compatible with the reflexive \ipa{a}--+\textsc{reduplication} (see section \ref{subsub:recip.compat} ) in forms such as \ipa{rlaʁ}    ``to disappear'' > \ipa{ɣɤ-rlaʁ} ``to lose, to cause to be destroyed'' > \ipa{a-ɣɤ-rlɯ-rlaʁ} ``to destroy each other''. The reverse order is however not possible.
 

\subsubsection{Semantic} \label{subsub:caus-g:semantics}
The semantic opposition between the two causative \ipa{sɯ}-- and \ipa{ɣɤ}-- is not entirely clear. \citet{jackson06paisheng}, on the related Tshobdun and Zbu languages, proposed that the cognate prefixes \ipa{sə}-- and \ipa{wɐ}-- were distinguished in that the former expresses the \ipa{augmentation of a degree} (\zh{增加某状态之程度}), while the second expresses the \ipa{direct realization of a state}  (\zh{直接造成某种状态}), quoting the following examples from Tshobdun:
 \begin{exe}
\ex 
\gll  \ipa{ɐ́-ⁿge}  \ipa{ɐ-tə-tə-sə-rzɐʔ} \\
   1\sg{}.\poss{}-clothes \irr{}-\pfv{}-2-\caus{}-long[III] \\
 \glt    Please make my clothes longer. (\zh{把我的衣服弄长一些})
  \begin{exe}
   \end{exe}
   \ex 
\gll  \ipa{ɐ́-ⁿge}  \ipa{ɐ-tə-tə-wɐ-rzɐʔ} \\
     1\sg{}.\poss{}-clothes \irr{}-\pfv{}-2-\caus{}-long[III] \\
 \glt    Please make my clothes long. (\zh{把我的衣服做长})
   \end{exe}

In Japhug, this type of minimal pair is very difficult to discover. Most stative verbs only have either \ipa{ɣɤ}-- or \ipa{sɯ}--, and their distribution is partially determined by the phonological structure of the word as mentioned above: only \ipa{sɯ}-- is possible when the verb stem contains a prefixal syllable before the root (as in \ipa{mɤrtsaβ} ``spicy'' >  \ipa{z-mɤrtsaβ}). 

A few stative verbs, such as the following, can however appear with both prefixes:
\begin{itemize}
\item \ipa{ntaβ} ``be stable'' > \ipa{ɕɯ-ntaβ}, \ipa{ɣɤ-ntaβ} ``put''
\item \ipa{zbaʁ} ``dry'' > \ipa{sɯ-zbaʁ}, \ipa{ɣɤ-zbaʁ} ``dry s.t.''
\item \ipa{xtɯt} ``short'' > \ipa{sɯ-xtɯt}, \ipa{ɣɤ-xtɯt} ``shorten''
\end{itemize}
The \ipa{ɣɤ}-- variant is however  the most common one and the only one found in traditional texts up to now. The speakers consulted could not tell  clear differences between minimal pairs; the forms with \ipa{sɯ}-- have a more restricted usage. For instance, one can say:

    \begin{exe}
   \ex 
  \gll    \ipa{ɯ-sɯm} \ipa{ɲɯ-ntaβ}   \\
  3\sg{}-spirit \const{}-stable \\
 \glt   He is relieved (his mind is at rest).
\gll    \ipa{ɯ-sɯm} \ipa{nɯ-ɣɤ-ntaβ-a}   \\
  3\sg{}-spirit \aor{}-\caus{}-stable-1\sg{} \\
 \glt   I relieved him. (el., Chen Zhen)
   \end{exe} 
Using the causative \ipa{ɕɯ-ntaβ} instead of \ipa{ɣɤ-ntaβ} here would be impossible. However, this kind of isolated fact notwithstanding, no systematically re-verifiable semantic difference could be detected between these pairs of verbs.




To see whether some differences can still be found between the two prefixes, let us now apply Dixon's criteria to the causative forms in \ipa{ɣɤ}--.

 \textbf{control, volition (causee)}. The causee can be either inanimate, animate or even human:
   \begin{exe}
   \ex 
\gll  \ipa{paʁ}  	\ipa{nɯ}  	\ipa{stoʁ}  	\ipa{khro}  	\ipa{pɯ-mbi-j}  	\ipa{tɕe}  	\ipa{thɯ-ɣɤ-tshu-j}  \\
 pig \topic{} bean much \aor{}-give-1\pl{} \coord{} \aor{}-\caus{}-fat-1\pl{}\\
 \glt We gave the pig a lot of beans, and made it fat. (el. Dpalcan 2010)
   \end{exe} 
    \begin{exe}
   \ex 
\gll \ipa{ʁlaŋ}  	\ipa{ɣɯ}  	\ipa{ɯ-rɟaβlun}  	\ipa{nɯ}  	\ipa{ra}  	\ipa{to-ɣɤ-sna}      \\
Gling \gen{} 3\sg{}.\poss{}-minister \topic{} \pl{} \evd{}-\caus{}-be.well  \\
 \glt  He healed the ministers of Gling. (Gesar 325)
   \end{exe} 

   However, with causative verbs with \ipa{ɣɤ}--   the causee normally has no control over the action; the only exception is \ipa{ɣɤ-khɯ}, the causative form of \ipa{khɯ} ``agree, be possible''. It can be used in sentences such as:
   \begin{exe}
   \ex 
\gll    \ipa{tɤ-ndzɯmbra-t-a} \ipa{tɕe} \ipa{tɤ-ɣɤ-khɯ-t-a}\\
 \aor{}-educate-\pst{}-1\sg{} \coord{} \aor{}-\caus{}-agree/be.possible-\pst{}-1\sg{}\\
 \glt   Following my advice, he agreed (I made him agree). (el. Chen Zhen 2011)
   \end{exe} 
  In this sentence, the causee has some degree of control (he can decide whether to agree or not), but he is certainly unwilling to do so.
  
  We can conclude that unlike \ipa{sɯ}-- causatives, \ipa{ɣɤ}-- causatives preclude volition of the causee.
  
 
 \textbf{intention (causer)}. Although most examples of causatives in \ipa{ɣɤ}-- involve intentional animate agents, these forms  can also be used to express non-intentional events:
 
 \begin{exe}
   \ex 
\gll   \ipa{aʑo}  	\ipa{mɯ-to-rɯndzaŋspa-a}  	\ipa{tɕe,}  	\ipa{tɯ-ŋga}  	\ipa{kɤ-qrɯ}  	\ipa{ɲɤ-ɣɤ-xtɯt-a}  \\
I \negat{}-\evd{}-pay.attention-1\sg{} \coord{} \neu{}-clothes \inftv{}-cut \evd{}-\caus{}-short-1\sg{} \\
 \glt   I mistakenly cut the clothes too short. (el., Chen Zhen\wav{8_GAxtWt})
   \end{exe} 
The use of the evidential \ipa{ɲɤ}--   with a verb in the first person instead of the aorist \ipa{nɯ}-- clearly indicates that the act in question was not done on purpose (see \ref{sub:evd} on this use of the evidential).
   
 To insist on the intentionality of the action, the alternative synthetic construction with the verb \ipa{βzu} can be used instead, see section \ref{sec:causation.complement}.
  
 \ipa{ɣɤ}-- causatives are compatible with inanimate (and therefore non-intentional) causers. For instance, examine the following sentence:
     \begin{exe}
   \ex 
\gll  \ipa{qale} \ipa{kɯ} \ipa{to-sɯ-zbaʁ}\\
     wind \erg{} \evd{}-\caus{}-dry \\
 \glt The wind dried it.     (el. Dpalcan and Chen Zhen)
   \end{exe}  
  \ipa{zbaʁ} ``be dry'' is one of the the rare stative verbs that can be used with both causative prefixes. It is possible here to replace \ipa{to-sɯ-zbaʁ} with its equivalent \ipa{to-ɣɤ-zbaʁ} without noticeable change of meaning (however, the form with \ipa{sɯ}-- was proposed more spontaneously by both speakers).
  
  
  \textbf{Directness, naturalness and involvement (causer)}. Causatives in \ipa{ɣɤ}-- are most often used in direct causation. However, we do find cases where the nature of the causation is not one of \ipa{changing} the property of the causee, but rather to \ipa{select} a causee which has the property in question:
     \begin{exe}
   \ex 
\gll \ipa{tɯ-ŋga}  	\ipa{tú-wɣ-ɣɤ-jaʁ}  	\ipa{ma}  	\ipa{ɲɯ-ɣɤndʐo}   \\
   \neu{}-clothes \ipf{}-\inv{}-\caus{}-thick because \const{}-cold   \\
 \glt    One has to wear thick clothes, as it is cold.  (el. Chen Zhen) \wav{8_GAjaR}
   \end{exe}  
    
     
In this case  the clothes themselves are not affected by the action: none of their intrinsic properties is modified.  

%   \begin{exe}
%   \ex 
%\gll   	\ipa{nɤʑo}  	\ipa{tɕheme}  	\ipa{tɯ-ɕti}  	\ipa{tɕe,}  	\ipa{nɤ-ŋga}  	\ipa{kɤ-ɣɤ-xtɯt}  	\ipa{mɤ-ra}   \\
%you woman 2-\npst{}:be.\emphat{} \coord{} 2\sg{}.\poss{}-clothes \inftv{}-\caus{}-short \negat{}-\npst{}:have.to   \\
% \glt Since you are a woman, you should not wear you clothes short (by rolling them up).      (el. Chen Zhen)
%   \end{exe}   


However, the \ipa{ɣɤ}-- prefix cannot be used when the causer does not have a direct role in the action, unlike \ipa{sɯ}--. This important semantic difference explains  the incompatibility of \ipa{ɣɤ}-- with some stative verbs. For instance, one can only say  \ipa{sɯ-tsɣi} ``cause to rot'':
     \begin{exe}
   \ex 
\gll \ipa{kɯki}  	\ipa{kɤndza}  	\ipa{ki}  	\ipa{ʑa}  	\ipa{mɯ-tɤ-nɯβdaʁ-a}  	\ipa{tɕe,}  	\ipa{pɯ-sɯ-tsɣi-t-a}    \\
   \dem{}.\prox{} food \dem{}.\prox{} long.time \negat{}-\aor{}-take.care-1\sg{} \coord{} \aor{}-\caus{}-be.rotten-\pst{}-1\sg{}  \\
 \glt  I did not take care of this food and let it rot.    (el. Chen Zhen) \wav{8_sWtsGi}
   \end{exe} 
My consultant Chen Zhen pointed out that saying *pɯ-ɣɤ-tsɣi-t-a would imply that the speaker did it on purpose and had a direct role in the process of rotting.
 

In conclusion, causatives of stative verbs in \ipa{ɣɤ}-- differs from those in \ipa{sɯ}--  in that:
\begin{itemize}
\item  The causee has little or no control over the change of state.
\item The causer always has a direct involvement in the change of state.
\end{itemize}
These two properties of \ipa{ɣɤ}-- causatives  do not explain all the peculiarities of the distribution between the two prefixes, for instance why only \ipa{sɯ}-- is compatible with  stative verbs expressing colours (\ipa{sɯɣ-ɲaʁ} ``make black'') of pain/diseases (\ipa{ɕɯ-mŋɤm} ``cause pain''). However, this study shows that Sun's analysis of the Tshobdun causatives prefixes cannot be directly applied to Kamnyu Japhug.
 


 



\subsection{Applicative} \label{sub:applicative}
The applicative is a derivation by means of which an oblique argument is promoted to the O role. In Japhug, only intransitive verbs can be subject to the applicative derivation. 

The applicative in Japhug is marked by the prefix \ipa{nɯ}-- or \ipa{nɯɣ}--. It is only moderately productive. Only the following  examples have been discovered:

\begin{table}[H]
\caption{Examples of the \ipa{nɯ}- applicative prefix}\label{tab:applicative}
\begin{tabular}{lllllllll} \toprule
basic verb  & &derived  verb &\\
\midrule
 \ipa{aʑɯʑu}  & wrestle	& \ipa{nɤʑɯʑu}  & wrestle with\\
\ipa{akhu}  &	shout&\ipa{nɤkhu}  & shout at \\
\ipa{andzɯt}  &	bark&\ipa{nɤndzɯt}  & bark at \\
\ipa{amdzɯ}  &sit & \ipa{nɤmdzɯ}  &look after\\
    \ipa{stu}  &believe (vi)	& \ipa{nɤstu}  & believe (vt)\\
\midrule
\ipa{mbɣom}  &	be hurried & \ipa{nɯmbɣom}  & look  forward to, miss s.o.\\
\ipa{ŋke}  &go on foot	& \ipa{nɯŋke}  & look for \\
\ipa{rga}  &	like (vi) & \ipa{nɯrga}  &like (vt) \\
\ipa{sŋom}  &	envy (vi) & \ipa{nɯsŋom}  &envy (vt) \\
\midrule
  \ipa{bɯɣ}  &miss (vi)	& \ipa{nɯɣbɯɣ}  & miss (vt)\\
  \ipa{mu} & be afraid & \ipa{nɯɣmu} & be afraid of \\
\bottomrule
\end{tabular}
\end{table}

XXXXXXXXX \citet{peterson07appl}
> comitative, allative ( \citet{peterson07appl}) but mainly  stimulus applicative
\subsubsection{Morphophonology}
The applicative prefixes has three distinct allomorphs: \ipa{nɯ}--, \ipa{nɤ}-- and \ipa{nɯɣ}--. Of these, \ipa{nɯ}-- is homonymous with many other derivational prefixes(denominal and autobenefactive-spontaneous) and even flexional prefixes (aorist/imperative directional ``towards east'', 2/3 plural possessive). \ipa{nɤ}-- is identical to the tropative prefix, or to one allomorph of the \ipa{nɯ}-- denominal prefix. Even for the last allomorph \ipa{nɯɣ}--, the facilitative of transitive verbs \ipa{nɯɣɯ}-- has an irregular homophonous allomorph \ipa{nɯɣ}-- with at least one verb. 

\ipa{nɯ}-- is obviously the most basic allomorph, and the only one to appear with a Tibetan loanword (\ipa{rga} ``like''). 

\ipa{nɤ}-- results from the fusion of the applicative \ipa{nɯ}-- with the \ipa{a}-- determiner of many intransitive verbs. This rule is the same as that according to which the causative \ipa{sɯ}-- is realised as \ipa{sɤ}-- with verbs of this type (cf. \ref{subsub:caus:morphophon}). \ipa{nɤkhu} ``shout at'', \ipa{nɤmdzɯ} ``look after'' and \ipa{nɤʑɯʑu} ``wrestle with'' could therefore be rewritten as \ipa{nɯ-ɤkhu}, \ipa{nɯ-ɤmdzɯ} and \ipa{nɯ-ɤʑɯʑu}.

\ipa{nɯɣ}-- only appears with two examples that have a labial onset without cluster. The distribution of \ipa{nɯ}-- and \ipa{nɯɣ}-- might have been at an earlier stage like that of \ipa{sɯ}-- and \ipa{sɯɣ}--, the latter occurring with intransitive verbs whose initial does not contain a cluster or a velar/uvular. 

For most verbs, the applicative is formally difficult to distinguish from the autobenefactive-spontaneous \ipa{nɯ}--; only the meaning and the fact that the former adds an argument, while the latter does not change the valency of the verb, can allow to distinguish between the two. The two prefixes however are distinct in the case of verbs prefixed with the determiner \ipa{a}--: while the applicative (being located in slot 10) appears \ipa{before} the \ipa{a}--, the spontaneous-autobenefactive appears after it. The difference between \ipa{nɤkhu} ``shout at'' and \ipa{anɯkhu} ``shout'' can be analysed as follows:


\begin{tabular}{IIIII}
Position & 10 & 11 & 12 & 15 \\

& & \ipa{a}-- & &\ipa{khu} \\
applicative& \ipa{nɯ}--& \ipa{a}-- & &\ipa{khu} \\
autobenefactive& & \ipa{a}-- & \ipa{nɯ}--&\ipa{khu} \\
\end{tabular}

A counterexample however is found: the autobenefactive-spontanenous of \ipa{atɯɣ} ``to meet'' is  \ipa{nɯ-ɤtɯɣ}, with this prefix in position 10.

\subsubsection{Compatibilities} \label{subsub:appl.compat}

 The applicative \ipa{nɯ}-- is compatible with the causative \ipa{sɯ}--, with which it regularly combines as \ipa{z-nɯ}--:
    \begin{exe}
   \ex 
\gll \ipa{nɯ} 	\ipa{kɯ-fse} 	\ipa{ci} 	\ipa{jɯm} 	\ipa{ko-z-nɯ-ŋke-j} 	\ipa{tɕe,} 	\ipa{nɯ-me} 	\ipa{nɯ} 	\ipa{ɯ-rca} 	\ipa{tɤ-ɣe-j}    \\
   \dem{} \nmlz{}:\stat{}-be.like.that \indef{} wife \evd{}:east-\caus{}-\appl{}-walk-1\pl{} \coord{} 2\pl{}.\poss{}-girl \topic{} 3\sg{}.\poss{}-following \aor{}:up-come[II]-1\pl{}    \\
 \glt So (the king) sent us to look for a wife (for his son), and we followed your daughter here. (The prince, 70)
   \end{exe} 

The applicative can also be combine with both the deexperiencer \ipa{sɤ}-- and the antipassive \ipa{sɤ}--, producing homophonous forms, as \ipa{sɤ-nɯ-rga}, which can either be interpreted as ``to be likeable" (deexperiencer) or ``to like people'' (antipassive).
 
The  applicative is also compatible with the reciprocal in examples such as \ipa{anɯɣbɯɣbɯɣ} ``to miss each other'' or \ipa{anɯrgɯrga} ``to like each other'' (see \ref{subsub:recip.compat}), and with the reflexive as in \ipa{ʑɣɤ-nɤstu} ``to believe in oneself'' (see \ref{subsub:reflexive.compat}).
 
We failed to find other combinations of prefixes with the applicative, but this might reflect the rarity of the applicative rather than a structural principle of the language. For another possible case of applicative, see \ref{subsub:deexp.pairs}.
 

 
\subsubsection{Syntactic constructions} \label{subsub:appl.syntax}
The applicative makes an intransitive (in some case even stative) verb transitive. The original S becomes the A, and a new argument is promoted to O status.

\textbf{Promotion of a dative argument
    \begin{exe}
   \ex 
\gll   jilco ra nɯ-ɕki li ɲɤ-k-ɤkhɤzŋga-chɯ \\
     \\
 \glt jilco ra ɲɤ-nɯ-ɤkhɤzŋga 54
   \end{exe} 
\wav{8_nWCki}}

Note  that the intransitive verb \ipa{rga} ``like, be glad'' appears on its own with an infinitive complement without the applicative prefix:
    \begin{exe}
   \ex 
\gll \ipa{kɤ-rɯɕmi} 	\ipa{mɯ́j-rga}   \\
\inftv{}-speak \negat{}:\const{}-like     \\
 \glt  He does not like to speak. (el., Dpalcan)
   \end{exe} 

 However with a definite patient the applicative is necessary:
 
     \begin{exe}
   \ex 
\gll \ipa{iɕqha} 	\ipa{tɕheme} 	\ipa{nɯ}  	\ipa{ɲɯ-nɯ-rge-a}   \\
the.aforementioned woman \topic{}  \const{}-\appl{}-like[III]-1\sg{}     \\
 \glt  I like this woman. (el., Dpalcan)
   \end{exe} 

The added argument can be relativized:
     \begin{exe}
   \ex 
\gll  \ipa{thaχtsa} 	\ipa{nɯ} 	\ipa{iʑo} 	\ipa{kɯrɯ} 	\ipa{tɕheme} 	\ipa{ra} 	\ipa{nɯnɯ} 	\ipa{mɤlɤn} 	\ipa{ʑo} 	\ipa{pjɯ-tu} 	\ipa{kɯ-ra} 	\ipa{tɕe}, \ipa{stu} 	\ipa{ji-kɤ-nɯ-rga} 	\ipa{ɕti,}   \\
     coloured.belt \topic{} we Tibetan woman \pl{} \dem{} absolutely \emphat{} \ipf{}-be.there \nmlz{}:\stat{}-have.to \coord{} most 1\pl{}.\poss{}-\nmlz{}:O-\appl{}-like \npst{}:be.\emphat{}\\
 \glt  Coloured belts are something we Tibetan women need to have absolutely, it is what we like most. (Coloured belts, 93)
   \end{exe} 
Applicative verbs can also be used with infinitive complements:
     \begin{exe}
   \ex 
\gll  \ipa{tɯ-ŋga} 	\ipa{kɤ-χtɯ} 	\ipa{ɕ-pɯ-nɯ-ŋke-t-a}  \\
  \neu{}-clothes \inftv{}-buy \transloc{}-\aor{}:down-\appl{}-walk-\pst{}-1\sg{}\\
 \glt   I walked around to buy clothes. (el., Dpalcan)
   \end{exe} 
Note that movement verbs normally appear with complements using the S/A participle form instead (cf. \ref{sub:S/A.part}), but here the applicative of ``to walk'' \ipa{nɯ-ŋke} ``to look for'', a transitive verb (unlike other movement verbs such as \ipa{ɕe} ``to go'', \ipa{ɣi} ``to come'', \ipa{rɟɯɣ} ``to run'' etc which are intransitive), appears with a \ipa{kɤ}-- infinitive complement. In this construction, The S/A of the complement verb must be coreferent with that of the A of the applicative verb.
 
 When the S/A of the complement clause is not coreferent with the A of the applicative verb, a finite form is necessary:
      \begin{exe}
   \ex 
\gll   \ipa{ɯʑo} \ipa{ju-nɯɣi} \ipa{ɲɯ-nɯ-mbɣom-a}  \\
 he \ipf{}-come.back \const{}-\appl{}-be.ina.a.hurry-1\sg{} \\
 \glt   I am looking forward to  his coming back. (el., Chen Zhen)
   \end{exe} 
With the applicative verb \ipa{nɯ-mbɣom} ``to look forward to'', finite complements are always in the imperfective form, even when the verb is in the aorist:

      \begin{exe}
   \ex 
\gll   \ipa{jɯfɕɯr} \ipa{a-ʑɯβ} \ipa{mɯ-pɯ-ɣe} \ipa{tɕe,} \ipa{lu-fsoʁ} \ipa{tɤnɯmbɣoma}  \\
yesterday 1\sg{}.\poss{}-sleep \negat{}-\aor{}-come[II] \coord{} \ipf{}-be.clear
 \aor{}-\appl{}-be.ina.a.hurry-1\sg{} \\
 \glt   Yesterday I could not sleep, I looked forward for the daybreak. (el., Chen Zhen)
   \end{exe} 
 
\subsubsection{Semantics} \label{subsub:appl.sem}
The applicative verbs in Japhug can be divided into three groups depending on the semantics of the original verb: experiencer verbs, action verbs and reciprocal action verb.

For the experiencer verbs (\ipa{rga} ``to like'', \ipa{bɯɣ} ``to miss'', \ipa{mu} ``to be afraid'' etc), the applicative adds the stimulus of the feeling:
   \begin{exe}
   \ex 
\gll \ipa{ɲɯ-ta-nɯɣ-bɯɣ-nɯ}   \\
\ipf{}-1>2-\appl{}-miss-\pl{}   \\
 \glt  I miss {you_p}. (el., Chen Zhen)
   \end{exe} 
   
      \begin{exe}
   \ex 
\gll  \ipa{nɤʑo} 	\ipa{tɕhi} 	\ipa{tɯ-nɯɣ-me?}    \\
   you what 2-\npst{}:\appl{}-be.afraid[III]\\
 \glt   What are you afraid of? (Gesar, 379)
   \end{exe} 
 
 


 Unlike most transitive verbs, applicative verbs derived from experiencer verbs can be used with the past imperfective or the evidential imperfective, and the periphrastic past imperfective is not necessary (cf. \ref{sub:pst.ipf}):
   \begin{exe}
   \ex 
\gll \ipa{lɯlu} 	\ipa{nɯ} 	\ipa{wuma} 	\ipa{pjɤ-nɯɣ-mu-ndʑi} 	\ipa{ɕti}   \\
cat \topic{} very \evd{}.\ipf{}-\appl{}-be.afraid-\du{} \npst{}:be.\emphat{} \\
 \glt   {They_d} were very afraid of the cat. (The mouse and the sparrow, 15)
   \end{exe} 
 
  
 With the action verbs \ipa{ŋke} ``walk'' and \ipa{akhu} ``shout, call'', the added argument is the goal towards which the action is directed. The derived applicative verbs, unlike their corresponding intransitive counterpart, are intrinsically telic verbs:
 
   \begin{exe}
   \ex 
\gll \ipa{ɯ-rkɤrkɯ} \ipa{jilco} 	\ipa{nɯ} 	\ipa{ra} 	\ipa{tɯ-sŋi} 	\ipa{ɲo-z-nɤʁaʁ,} 	\ipa{ɲo-nɯ-ɤkhu}  \\
 3\sg{}.\poss{}-around neighbour \topic{} \pl{} one-day \evd{}-\caus{}-have.a.good.time \evd{}-\appl{}-call \\
 \glt    One day, she invited  neighbours from all places around, she invited them. (The raven 98)
   \end{exe} 
  
   \begin{exe}
   \ex 
\gll \ipa{a-ɣe} 	\ipa{rɟɤlpu} 	\ipa{ɕɯ-nɯ-ɤkhu-tɕi} \\
  1\sg{}.\poss{}-grandson king \transloc{}-\npst{}:\appl{}-call-1\du{} \\
 \glt    Grandson, let us go to invite the king. (Kunbzang 342)
   \end{exe} 
  
  In the case of  \ipa{amdzɯ} ``sit'', whose applicative \ipa{nɤmdzɯ} means ``take care for, look after'', the semantic derivation is less transparent, though it reminds of idiomatic expression such as ``baby-sitting'':
   \begin{exe}
   \ex 
\gll \ipa{ki} 	\ipa{tɤpɤtso} 	\ipa{kɤ-nɯ-ɤmdzi} 	\ipa{a-mɤ-pɯ-ndʐaβ}  \\
\dem{}.\prox{} child \imp{}-\appl{}-sit[III] \irr{}-\negat-\pfv{}-\acaus{}:make.fall \\
 \glt    Look after this child, do not let him fall. (el., Dpalcan)
   \end{exe} 


Finally, with the intrinsically reciprocal \ipa{aʑɯʑu} ``wrestle'', the applicative is used as an ``anti-reciprocal'':

   \begin{exe}
   \ex 
\gll \ipa{tɤ́-wɣ-nɯ-ɤʑɯʑu-a} 	   \\
\aor{}-\inv{}-\appl{}-wrestle-1\sg{}  \\
 \glt    He wrestled with me. (el., Dpalcan)
   \end{exe} 
The applicative however cannot be combined with reciprocal derivation in a regular way. \ipa{aʑɯʑu} itself is historically the reciprocal of a non-attested transitive verb *ʑu ``wrestle'', but is not analysable as such synchronically since the base verb was lost. One reason why the combination of applicative with reciprocal is not more common is that   it would be homophonous with the atelic derivation (\ref{sub:atelic}). The reciprocal is formed by combining the reduplicated verb stem with the \ipa{a}-- prefix. Adding the applicative \ipa{nɯ}-- yields \ipa{nɯ-ɤ}-- > \ipa{nɤ}-- with reduplication, which is exactly similar to the atelic form of the verb. This combination would also derive a transitive verb out of a transitive one (through a stage as an intransitive reflexive), exactly as the atelic derivation.

We notice that neither place nor instrument can be promoted to O using the applicative in Japhug; its range of uses is quite limited.

\subsection{Tropative} \label{sub:tropative}
The tropative \ipa{nɤ-}, located in slot 10 of the verbal template, is a very productive derivation that can be applied to most stative verbs, having the meaning ``to consider to be X''. 

The following table shows a few examples of tropative verbs in Japhug; this list is by no means exhaustive, but constitutes a representative sample:
\begin{table}[H]
\caption{Examples of the \ipa{nɤ}- tropative prefix in Japhug}\label{tab:tropative}
\begin{tabular}{lllllllll} \toprule
basic verb  & &derived  verb &\\
\midrule
 \ipa{wxti} & be big & \ipa{nɤ-wxti} & consider to be too big \\
 \ipa{zri} & be long & \ipa{nɤ-zri} & consider to be too long \\
  \ipa{chi} &be sweet & \ipa{nɤx-chi} &consider to be too sweet \\
    \ipa{mnɤm} & have an odour & \ipa{nɤ-mnɤm} & smell (tr.) \\
     \ipa{mɯm} & be tasty & \ipa{nɤ-mɯm} & consider to be tasty, like (food) \\
      \midrule
  \ipa{maʁ} & not be & \ipa{nɤɣ-maʁ} & consider to not to be right \\
  \ipa{mbat} & be easy & \ipa{nɤɣ-mbat} & finish easily \\
\bottomrule
\end{tabular}
\end{table}
The examples show that the morphophonology and semantic derivation of this prefix is not entirely straightforward. Each of these aspects will be discussed in the following sections.

\subsubsection{Morphophonology}


Aside from the regular \ipa{nɤ-} allomorph, one also find a \ipa{nɤɣ-} / \ipa{nɤx-} allomorph on a few verbs. A similar allomorphy is found with the causative prefix (which has \ipa{sɯɣ-} and \ipa{ɕɯɣ-} allomorphs alongside \ipa{sɯ-}) and the applicative \ipa{nɯ-} (which has the allomorph \ipa{nɯɣ-} in a few examples). The  \ipa{sɯ-} vs. \ipa{sɯɣ-} allomorphy is still productive: the latter allomorph is found when the original verb was intransitive, without an initial consonant cluster and without initial velar or uvular. It is possible that a similar distribution existed at a former stage for the \ipa{nɤ-} / \ipa{nɤɣ-} allomorphs, but the data at hand do not permit a firm conclusion.


\subsubsection{Syntax}

As in a causative derivation, the S of the original verb becomes the O of the derived transitive verb, while the added argument  becomes the A of the derived verb. For instance, the stative verb  \ipa{mpɕɤr} ``be beautiful'' have the derived transitive verb \ipa{nɤ-mpɕɤr} ``consider to be beautiful'':


 \begin{exe}
\ex
\gll  \ipa{ɯ-mdoʁ} 	\ipa{maka} 	\ipa{mɯ́j-nɤsci} 	\ipa{tɕe,} 	\ipa{nɯ} 	\ipa{ni} 	\ipa{stu} 	\ipa{nɯ-kɤ-nɤ-mpɕɤr} 	\ipa{ɲɯ-ŋu} 
\\
3\sg{}.\poss{}-colour at.all \negat{}:\const{}-change \coord{} \distal{}.\dem{} \du{} most 3\pl{}-\nmlz{}:O-\trop{}-beautiful \ipf{}-be \\
 \glt  Its colour does not change, and these two are the ones that they consider the most beautiful. (Coloured belts, 85)
\end{exe} 

The difference between tropative and causative derivation lies therefore in the semantics of the derivation: the added argument is not a causer, but an \ipa{experiencer}. However, some examples of causative verbs in \ipa{sɯ}-- have a semantics closer to that of a tropative verb (see \ref{subsub:caus.semantics}).

\subsubsection{Semantics}
The semantics of the derived verb is not always simply ``to consider to be X''. In the case of stative verbs whose meaning is neutral (not explicitely positive like ``beautiful''), the tropative generally has the meaning ``to consider to be too X''. Second, in the case of \ipa{nɤɣ-maʁ} ``consider to not to be right'', it seems that of of the original meanings of   \ipa{maʁ}  was ``not to be right'', as the nominalized form \ipa{kɯ-maʁ} can mean ``something which is not right''. Here the tropative preserved the original meaning of the verb, and the base verb underwent an independent semantic change.

Unlike most transitive verbs, tropative verbs can be used with the past imperfective or the evidential imperfective, and the periphrastic past imperfective is not necessary (\ref{sub:pst.ipf}). For instance, with the verb \ipa{nɤ-mɯm} ``consider tasty, like'', one can say both \ipa{to-nɤ-mɯm} in the evidential  ``he liked it'' (he used to consider it not tasty, but now he likes it) and \ipa{pjɤ-nɤ-mɯm} in the imperfective evidential  ``he used to like it''.

Interestingly, the agent (experiencer) of the tropative verb need not be an animate, as illustrated by the following example:
 \begin{exe}
\ex
\gll   \ipa{ɕomskrɯt} \ipa{nɯ} \ipa{ŋgɤm} \ipa{ɯ-taʁ} \ipa{kɤ-sɯ-ɤtsa-t-a} \ipa{tɕe}, \ipa{ɲɯ-nɤ-mpi} \ipa{tɕe}, \ipa{ɲɯ-ɣɤrɲɯɣrɲɯɣ} \ipa{ʑo} \ipa{lɤ-ari} \\
wire \topic{} earth.mount 3\sg{}.\poss{}-on \aor{}-\caus{}-inserted-\pst{}-1\sg{} \coord{} \const{}-\trop{}-soft[III] \coord{} \const{}-pierce.quickly \emphat{} \aor{}:upstream-go[II] \\
 \glt   I planted the wire into the earth mount,  the earth was soft, so that it went into it very quickly. (Chen Zhen, 2011)
\end{exe} 
 The verb \ipa{ɲɯ-nɤ-mpi} here actually means literally ``(the wire) considered (the mount) to be very soft''.

The semantic link between of tropative verbs and their base verbs is generally transparent, but the verb  verb \ipa{nɤrko} ``to endure'' is an exception. This verb is historically the tropative of \ipa{rko} ``to be hard'', but underwent a semantic change ``to consider to be hard" > ``to endure''. This verb can in turn undergo reciprocal derivation \ipa{a-nɤrkɯ-rko} ``to force each other''.
 
\subsubsection{Compatibility} \label{subsub:trop.compat}
The tropative is compatible with other derivation prefixes, in particular the deexperiencer \ipa{sɤ-} (see \citealt{jacques11demotion}). For instance, we observe the derivation chain:

 \begin{exe}
\ex
 \glt  \ipa{scit} ``happy'' (of a person)
 \glt Deexperiencer: \ipa{sɤ-scit} ``funny, nice''. The literal meaning of this stative verb is in fact ``to be such that people are happy''. 
 \glt Tropative: \ipa{nɤ-sɤ-scit} ``to consider to be nice''
\end{exe} 
The doubly derived verb can be illustrated by the following example:
 \begin{exe}
\ex
\gll   \ipa{nɯ-sɤ-rma} 	\ipa{pjɯ-ɣɤrcoʁ} 	\ipa{pjɤ-ra} 	\ipa{ma} 	\ipa{kɯ-zbaʁ} 	\ipa{nɯ} 	\ipa{mɯ́j-nɤ-sɤ-scit-nɯ} 	\ipa{ɲɯ-ŋgrɤl.} \\
 3\pl{}.\poss{}-\nmlz{}:\obl{}-live \ipf{}-be.muddy \evd{}.\ipf{}-have.to because \nmlz{}:\stat{}-dry \topic{} \negat{}:\const{}-\trop{}-\textsc{deexp}-happy-\pl{} \ipf{}-be.usually.the.case \\
 \glt   (The frogs)  had to live in a muddy place, because they did not like dry (places).  (Aesop adaptation, the frogs)
\end{exe} 
  
 It is possible to combine the tropative with the reflexive \ipa{ʑɣɤ-} in examples such as \ipa{ʑɣɤ-nɤ-mpɕɤr} ``to consider oneself to be beautiful'':
  \begin{exe}
\ex
\gll  \ipa{tɕhemɤpɯ} 	\ipa{ra} 	\ipa{kɯ} 	\ipa{aʑo} \ipa{mpɕar-a} 	\ipa{ntsɯ} 	\ipa{sɯso-nɯ} 	\ipa{ɕti} 	\ipa{nɤ,} 	\ipa{ɲɯ-ʑɣɤ-nɤ-mpɕɤr-nɯ} 	\ipa{ɕti} \\
gilr \pl{} \erg{} 1\sg{} \npst{}:beautiful-1\sg{} always \npst{}:think-\pl{} \npst{}:be.\emphat{} \coord{} \const{}-\refl{}-\trop{}-beautiful-\pl{}  \npst{}:be.\emphat{}\\
 \glt    Girls always think they are beautiful, they consider themselves to be beautiful. (conversation)
\end{exe} 
 
However, to express the meaning ``to consider oneself X'', one usually uses a different construction, with the complex prefix \ipa{znɤ-} and reduplication of the verb stem, described in \ref{sub:refl.trop}.

Finally, the tropative also appears with the reciprocal, as in \ipa{a-nɤ-mpɕɯ-mpɕɤr} ``to consider each other to be beautiful'' (see section \ref{subsub:recip.compat}).
  
\subsection{Relic suffix} \label{sub:applicative.t}
In addition to the four prefixes studied in the previous chapters, another valency-increasing derivational affix is found in Japhug: the applicative --\ipa{t} suffix. This suffix is only attested by one single pair of verbs: \ipa{ɣi} ``come'' and \ipa{ɣɯt} ``bring''. No Rgyalrongic languages has a productive --\ipa{t} suffix, but all preserve this pair of verbs. It constitutes the only derivational suffix in the Japhug language; it will not be analysed in the glosses, as it is entirely lexicalized.

The applicative --\ipa{t} is more common in other branches of Sino-Tibetan, especially  Kiranti, where it is exceptionally well attested (see for instance \citealt{michailovsky85dental}). Notice that in Khaling for instance, the pair \textsc{pi} (1\sg{} \ipa{pi-ŋa}, 1\textsc{di} \ipa{pi-ji}) ``come (along a horizontal plane)'' and \textsc{pid} (1\sg{}>3\sg{} \ipa{pid-u}, 1\textsc{di}>3 \ipa{pits-i}) ``bring (along a horizontal plane)'' , the cognates of \ipa{ɣi} and \ipa{ɣɯt}, is also found.
 
The intransitivized form of \ipa{ɣɯt}, \ipa{zɣɯt}, presents a unique valency decreasing derivation which will be discussed in \ref{sub:relic.val.decrease}.





\section{Valency-decreasing} \label{sec:valency.decreasing}

Japhug counts six distinct valency-decreasing derivations, to which incorporation must be added (see section \ref{sub:incorporation}).


\subsection{Antipassive} \label{sub:antipassive}
Like many languages with ergative alignment, Japhug has antipassive marking, a derivation which preserves the agent of a transitive and suppresses its patient. As   in the closely related Tshobsun language  (\citealt[8]{jackson06paisheng}), two antipassive prefixes are found: \ipa{rɤ}-- and \ipa{sɤ}--. The difference between the two will be explored in detail in \ref{subsub:antipass.morpho} and \ref{subsub:antipass.syntax}.

\subsubsection{Morphophonology} \label{subsub:antipass.morpho}
The antipassive prefixes are relatively productive, but cannot be added to any transitive verb. These prefixes present few morphophonological peculiarities.

The following table lists the best examples of the antipassive \ipa{rɤ}--. As in Tshobdun (\citealt[8]{jackson06paisheng}), this prefix is typically used when the suppressed argument is non-human, though we will see in \ref{subsub:antipass.syntax} that a few exceptions exist. The meaning of the derived verb is only indicated when it cannot be predicted from that of the transitive one:
\begin{table}[H]
\caption{Examples of the antipassive prefix  \ipa{rɤ}--}\label{tab:antipassive1}
\begin{tabular}{lllllllll} \toprule
basic verb  & &derived  verb &\\
\midrule
\ipa{roʁ}   &	to carve &  	\ipa{rɤ-roʁ}   &	 \\  
\ipa{tɕɤβ}   &	to burn &  	\ipa{rɤ-tɕɤβ}   &	to burn land \\  
\ipa{ɕphɤt}   &	to patch &  	\ipa{rɤ-ɕphɤt}   &	 \\  
\ipa{ɕtʂat}   &	to spare &  	\ipa{rɤ-ɕtʂat}   &	 \\  
\ipa{ɕtʂɯ}   &	to entrust to &  	\ipa{rɤ-ɕtʂɯ}   &	 \\  
\ipa{fɕɤt}   &	to tell &  	\ipa{rɤ-fɕɤt}   &	 \\  
\ipa{fse}   &	to whet &  	\ipa{rɤ-fse}   &	 \\  
\ipa{ftɕɤz}   &	to castrate &  	\ipa{rɤ-ftɕɤz}   &	 \\  
\ipa{mɲo}   &	to prepare &  	\ipa{rɤ-mɲo}   &	 \\  
\ipa{ndɯn}   &	to read aloud &  	\ipa{rɤ-ndɯn}   &	 \\  
\ipa{pɣaʁ}   &	to turn over &  	\ipa{rɤ-pɣaʁ}   &	to reclaim land \\  
\ipa{rkɤz}   &	to carve &  	\ipa{rɤ-rkɤz}   &	 \\  
\ipa{βzjoz}   &	to learn &  	\ipa{rɤ-βzjoz}   &	 \\  
\ipa{mbi}   &	to give &  	\ipa{rɤ-mbi}   &	 \\  
\ipa{ŋa}   &	to owe money &  	\ipa{rɤ-nŋa}   &	 \\  
\ipa{skɤr}   &	to weigh &  	\ipa{rɤ-skɤr}   &	 \\  
\ipa{thu}   &	to ask &  	\ipa{rɤ-tʰu}   &	 \\  
\ipa{tʂɯβ}   &	to sew &  	\ipa{rɤ-tʂɯβ}   &	 \\   
\ipa{scɤt}   &	to move &  	\ipa{rɤ-scɤt}   &	 \\  
\ipa{fsoʁ}   &	to earn &  	\ipa{rɤ-fsoʁ}   &	 \\  
\ipa{ɕar}   &	to search &  	\ipa{rɤ-ɕar}   &	 \\ 
\bottomrule
\end{tabular}
\end{table}
The formation of the \ipa{rɤ}-- antipassive is perfectly regular, and only one allomorph is found. This prefix is formally similar to the denominative \ipa{rɤ}--, and there is one case where ambiguity exists between the two prefixes. The verb \ipa{rɤ-znde} ``make a wall (vi)'' can indeed be analysed either as a denominal form of \ipa{znde} ``stone wall'' or of the transitive verb \ipa{znde}  ``make a wall''.  Note that both options are equally likely, as the prefix \ipa{rɤ}-- is attested with other denominal expressing the building of the base noun, for instance \ipa{tɤ-loʁ} ``nest'' > \ipa{rɤ-loʁ} ``build a nest (vi)''.


With the  \ipa{sɤ}-- antipassive examples are slightly less numerous. Verbs formed with this antipassive differ from the former group in that the suppressed argument is human, though one special case  will be discussed in \ref{subsub:antipass.semantics}.

\begin{table}[H]
\caption{Examples of the antipassive prefix \ipa{sɤ}--}\label{tab:antipassive2}
\begin{tabular}{lllllllll} \toprule
basic verb  & &derived  verb &\\
\midrule
\ipa{ɣɤmɯ}   &	to praise &  	\ipa{sɤ-ɣɤmɯ, sɤz-ɣɤmɯ}   &	 \\  
\ipa{ɣɤxpra}   &	to send &  	\ipa{sɤ-ɣɤxpra}   &	 \\  
\ipa{sɤŋo}   &	to listen &  	\ipa{sɤ-sɤŋo}   &to listen to people's advice	 \\  
\ipa{tɕʰɯ}   &	to gore &  	\ipa{sɤ-tɕʰɯ}   &	 \\  
\ipa{mtsɯɣ}   &	to bite &  	\ipa{sɤ-mtsɯɣ}   &	 \\  
\ipa{nɤmtsioʁ}   &	to peck &  	\ipa{sɤ-nɤmtsioʁ}   &	 \\  
\ipa{ɕaβ}   &	to catch up &  	\ipa{sɤ-ɕaβ}   &	 \\  
\ipa{nɯrtɕa}   &	to tease &  	\ipa{sɤ-nɯrtɕa}   &	 \\ 
\ipa{ʁndɯ}   &	to hit &  	\ipa{sa-ʁndɯ}   &	 \\  
\ipa{nɤre}   &	to laugh &  	\ipa{sɤ-nɤre}   &	to laugh at people \\  
\ipa{nɤsɤɣ}   &	to be jealous of &  	\ipa{sɤ-nɤsɤɣ}   &	 \\  
\ipa{nɯrɯtʂa}   &	to envy &  	\ipa{sɤ-nɯrɯtʂa}   &	 \\  
\ipa{sɯxɕɤt}   &	to teach &  	\ipa{sɤ-sɯxɕɤt}   &	 \\  
\ipa{tʰu}   &	to ask &  	\ipa{sɤ-tʰu}   &	to ask in marriage \\  
\ipa{ɕar}   &	to search &  	\ipa{sɤ-ɕar}   &	to search people \\ 
\bottomrule
\end{tabular}
\end{table}

The only cases of allomorphy with the prefix \ipa{sɤ}-- are the allomorph \ipa{sa}-- occuring when the prefix is folloded by a complex onset whose element is a uvular fricative, and the allomorph \ipa{sɤz}-- which appears in the verb \ipa{sɤz-ɣɤmɯ} ``to praise people''.



\subsubsection{Syntax} \label{subsub:antipass.syntax}
The antipassive prefixes make intransitive verbs out of transitive ones, and the S of the derived verb corresponds to the A of the base verb: this is an agent-preserving derivation, unlike the passive and the antipassive. The verb assumes intransitive morphology and the S   appears without ergative marking:

  \begin{exe}
\ex
\gll \ipa{tɤ-rʑaβ} 	\ipa{nɯ} 	\ipa{pjɤ-rɤ-ɕphɤt}   \\
    \neu{}-wife \topic{} \evd{}-\apass{}-mend\\
 \glt    The wife mended (clothes). (The raven 19)
\end{exe} 
 

 
 

As mentioned in the previous section,  the prefix \ipa{rɤ}-- is used when the suppressed O was non-human, and \ipa{sɤ}-- when it is human (see \ref{subsub:anticaus.semantics} for more details). However, bitransitive verbs constitute an a special case.

The three verbs  \jg{rɤ-mbi} ``to give to someone'', \jg{rɤ-nŋa} ``to owe someone'', \jg{rɤ-thu} ``to ask someone'' have the prefix \ipa{rɤ}-- though the deleted argument is human. Their base verbs \jg{mbi} ``to give'', \jg{ŋa} ``to owe'' and \jg{tʰu} ``to ask''  are bitransitive verbs, and their recipient is encoded as the O (cf \ref{sec:bitransitive}). The allomorphy between \ipa{ŋa} and \ipa{rɤ-nŋa} is explained in \ref{subsub:antipass.dia}.

The following examples show that the external argument can still appear in the sentence:

 \begin{exe}
\ex
\gll    \ipa{stoʁ} 	\ipa{nɯ-rɤ-mbi-a}   \\
   bean \aor{}-\apass{}-give-1\sg{} \\
 \glt    I gave beans. (el. Dpalcan)
\end{exe} 

The recipient (the original O) can be reintroduced in the sentence with oblique marking:

 
 \begin{exe}
\ex
\gll   \ipa{nɯnɯ} 	\ipa{phe} 	\ipa{kɯ-rɤ-thu} 	\ipa{ju-kɯ-ɕe} 	\ipa{tɕe} 	\ipa{tu-kɯ-sɯɣɕɤt} 	\ipa{ŋu} 	\ipa{kɤ-ti} 	\ipa{ɲɯ-ŋu}    \\
\dem{}.\textsc{distal} \dat{} \nmlz{}:S/A-\apass{}-ask \ipf{}-\genr{}:S/O-go \coord{} \ipf{}\genr{}:S/O-teach \npst{}:be \nmlz{}:O-say \ipf{}-be\\
 \glt   It is said that if you go  to ask him (instructions), he teaches you. (Divination2, 02)
\end{exe} 


Of these three verbs, two (\ipa{mbi} and \ipa{ŋa}) lack a human antipassive in \ipa{sɤ}--; the verb \ipa{sɤmbi} ``to ask for'' exists, but it is a transitive verb formed by combining the causative \ipa{sɯ}-- with the passive \ipa{ɤ}-- (see \ref{subsub:caus1.compatibility}), not a case of applicative. Only \ipa{tʰu} ``to ask'' has a human applicative \ipa{sɤtʰu}, but this verb has a distinct meaning, non-predictible from the base verb ``to ask in marriage'':
 
 \begin{exe}
\ex
\gll   \ipa{aʑo} 	\ipa{kɯ-sɤtʰu} 	\ipa{ɕe-a}     \\
 1\sg{} \nmlz{}:S/A-ask.in.marriage \npst{}:go-1\sg{}  \\
 \glt   I am going to find a wife. (Kun.bzang 02)
\end{exe} 


These three verbs seem to suggest that   the antipassive \ipa{rɤ}-- is the regular way to form the antipassive of  bitransitive verbs  when the resulting antipassive verb is semi-transitive (cf. \ref{sec:transitivity} on this notion).\footnote{\ipa{sɤtʰu} ``ask in marriage is a plain transitive verb.} However, the bitransitive verb    \ipa{sɯxɕɤt} ``to teach''   does not follow this rule, as their antipassive  \ipa{sɤ-sɯxɕɤt} ``to teach people'' a semi-transitive verb too, is formed with \ipa{sɤ}--:
 
 \begin{exe}
\ex
\gll   \ipa{tɤrtsɯz} \ipa{pjɯ-sɤ-sɯxɕat-a} \ipa{ŋu} \\
   mathematics \ipf{}-\apass{}:\textsc{hum}-1\sg{} \npst{}:be \\
 \glt    I teach mathematics. (el, Chen Zhen)
\end{exe} 


The verbf \ipa{fɕɤt} ``to tell'' is another special case. It is also a bitransitive verb, though the message, rather than the recipient, is encoded as the O:
  \begin{exe}
\ex
\gll   \ipa{a-χpi} \ipa{pa-fɕɤt}    \\
      1\sg{}.\poss{}-story \aor{}:3>3-tell   \\
 \glt   He told me a story. (el, Chen Zhen)
\end{exe} 

The antipassive of this verb \ipa{rɤ-fɕɤt} also means ``to tell'', but more in the sense of ``to report (a message)''. Although morphologically intransitive, and in spite of the fact that the semantic agent does not have ergative marking, \ipa{rɤ-fɕɤt} can appear with a nominal phrase corresponding to the message told:
  \begin{exe}
\ex
\gll  \ipa{tɕheme} 	\ipa{nɯ} 	\ipa{pɯ-pɯ-kɯ-fse} 		\ipa{nɯ} \ipa{ra} 	\ipa{ɲɤ-rɤ-fɕɤt,}    \\
     girl \topic{} \redp{}:all-\aor{}-\nmlz{}:S-be.in.such.a.way \topic{} \pl{} \evd{}-\evd{}-\apass{}-tell   \\
 \glt  The girl told (her parents) all that had happened.  (The flood, 70)
\end{exe} 
\ipa{rɤ-fɕɤt} is thus a semi-transitive verb, though its derivation is of a different nature from that of the three verbs above. Unlike them, the antipassive derivation of \ipa{rɤ-fɕɤt} does not demote the recipient, since the recipient was already an external argument. Its only effect  is to suppress ergative marking, transitive conjugation and cause a slight change of meaning.



%The resulting semi-transitive verbs can still be formed with a O-participle, unlike normal intransitive verbs:
%\begin{exe}
%\ex
% \gll \ipa{tɯ-ci-rqɯ} 	\ipa{kɤ-sɤ-jtshi} 	\ipa{ɯ́-tu}   \\
%\neu{}-water-cold \nmlz{}:O-\appl{}-give.to.drink \qu{}-\npst{}:be.there   \\
% \glt  Is there any cold water to give (us) to drink? (literally: ``cold water given to people'', Nyima Vodzer2, 35)
%\end{exe} 
%	\ipa{kɤ-sɤ-jtshi} 	here is a post-nominal relative clause: the relativized element here is the external argument (note that external arguments of bitransitive verbs can be relativized with \ipa{kɤ}--, see \ref{sec:relativisation.external}).
%
 




Antipassive can be used with labile verbs such as \ipa{sɤŋo} ``to listen'' and \ipa{nɤre} ``to laugh''. In these case, the meaning of the antipassive verb is slightly different from that of the base verb. Used transitively, \ipa{sɤŋo} and \ipa{nɤre} respectively mean ``to obey, listen to s.o.'s advice'' and ``to laugh at''. The antipassive verbs are clearly derived from the transitive forms: \ipa{sɤ-sɤŋo} ``to obey people, to heed people's advice'' and \ipa{sɤ-nɤre} ``to laugh at people''.
 

 \begin{exe}
\ex
\gll  Daihao 	\ipa{ɣɯ} 	\ipa{ɯ-ɕaχpu} 	\ipa{nɯ} 	\ipa{kɯ} 	\ipa{ɲɤ́-wɣ-sqar} 	\ipa{tɕe}  \ipa{pjɤ́-wɣ-z-rɤ-rɤt} 	\ipa{ɲɯ-ŋu.}   \\
  p.n. \gen{} 3\sg{}.\poss{}-friend \topic{} \erg{} \evd{}-\inv{}-ask.to.do \coord{} \evd{}-\inv{}-\caus{}-\apass{}-write \ipf{}-be  \\
 \glt A friend of Daihao's asked him to draw paintings.  (The painter and the shepherd boy, 10)
\end{exe} 

 


\subsubsection{Compatibility} \label{subsub:antipass.compat}
The antipassive is compatible with only few other derivations. 

Like most derivational processes (except anticausative), it can appear with the denominal prefixes in verbs such as \ipa{sɤ-nɯrɯtʂa} ``to envy people'': the transitive verb \ipa{nɯ-rɯtʂa} ``to envy'' is derived from the noun \ipa{rɯtʂa} ``envy''.

It can also be combined with the causative \ipa{sɯ}--, in examples such as \ipa{z-rɤ-rɤt} ``to cause to write/draw'':



 \begin{exe}
\ex
\gll  Daihao 	\ipa{ɣɯ} 	\ipa{ɯ-ɕaχpu} 	\ipa{nɯ} 	\ipa{kɯ} 	\ipa{ɲɤ́-wɣ-sqar} 	\ipa{tɕe}  \ipa{pjɤ́-wɣ-z-rɤ-rɤt} 	\ipa{ɲɯ-ŋu.}   \\
  p.n. \gen{} 3\sg{}.\poss{}-friend \topic{} \erg{} \evd{}-\inv{}-ask.to.do \coord{} \evd{}-\inv{}-\caus{}-\apass{}-write \ipf{}-be  \\
 \glt A friend of Daihao's asked him to draw paintings.  (The painter and the shepherd boy, 10)
\end{exe} 

%The verb \ipa{sɤ-jtsʰi} ``to give to drink to people'' could appear to be a counterexample

It is also compatible with the applicative in the example \ipa{sɤ-nɯ-rga} ``to like people''.

The antipassive never appears with any valency-decreasing affix.

\subsubsection{Semantics} \label{subsub:antipass.semantics}


With the exception of the three bitransitive verbs mentioned in \ref{subsub:antipass.syntax}, the distinction between the two antipassive prefixes is quite clear: \ipa{sɤ}-- appears when the suppressed O is human, and \ipa{rɤ}-- when it is non-human, as in Sun's description of Tshobdun.  The distinction cannot be argued to be animate vs. non-animate, as some verbs with \ipa{rɤ}--  antipassives have prototypically non-human animate Os, for instance \ipa{ftɕɤz} ``to castrate'' > \ipa{rɤ-ftɕɤz} ``to castrate (animals)'' . We only found one case where the antipassive \ipa{sɤ}--  is used with suppressed non-human animal patient:

 \begin{exe}
\ex
\gll  \ipa{thɯ-wxti} 	\ipa{ɯ-jɯja} 	\ipa{kɯ-sɤ-ndza} 	\ipa{kɯ-ŋu} 	\ipa{nɯ} 	\ipa{pjɯ-sɯχsɤl}   \\
  \aor{}-big 3\sg{}.\poss{}-following \nmlz{}:S-\apass{}-eat \nmlz{}:\stat{}-be \topic{} \ipf{}-realize\\
 \glt  As (the buzzard_{i}) grows bigger, (the bunting) realizes that it_{i} eats (other birds). (The Buzzard, 26)
\end{exe} 





Only a minority of verbs can occur with both prefixes, even when semantics would allow it. For instance, from the transitive verb \ipa{ndza} ``to eat'', one can derive  \ipa{sɤ-ndza} ``to eat people'', but not *rɤ-ndza (intended meaning: ``to eat things'' - the verb \ipa{rɯndzɤtsʰi} ``have a meal'' must be used instead). An example of verb compatible with both prefixes is \ipa{ɕar} ``to search'', from which one can form both \ipa{rɤ-ɕar} ``to search things'' and \ipa{sɤ-ɕar} ``to search people''. 
 
 The antipassive is used to express an indefinite patient:


 \begin{exe}
\ex
\gll \ipa{tɤ-rʑaβ} 	\ipa{nɯ} 	\ipa{pjɤ-rɤ-ɕphɤt}   \\
    \neu{}-wife \topic{} \evd{}-\apass{}-mend\\
 \glt    The wife mended (clothes). (The raven 19)
\end{exe} 
 
\begin{exe}
\ex
\gll    \ipa{nɤ-pi} 	\ipa{ɯ-tɯ-sɤ-nɤkhe} 	\ipa{nɯ}  \\
2\sg{}.\poss{}-elder.sibling 3\sg{}-\nmlz{}:\textsc{degree}-\apass{}-mistreat  \textsc{part}:exclamative   \\
 \glt  Your elder sister is really a bully! (Kun.bzang 373)
\end{exe} 

For labile verbs such as \ipa{taʁ}, indefinite patient can be expressed by simply stripping the verb of its transitive morphology, without recourse to the antipassive derivation (\ref{sub:ambitransitive}).

The agent participle of antipassive verbs is commonly used to form names of professions: \ipa{kɯ-rɤ-tʂɯβ} ``tailor'' (from \ipa{tʂɯβ} ``to sew''), \ipa{kɯ-rɤ-rkɤz} ``engraver'' (from \ipa{rkɤz} ``to carve'').

In some cases, such as \ipa{tʰu} ``to ask'' > \ipa{sɤtʰu} ``to ask in marriage'', \ipa{pɣaʁ} ``to turn over'' > \ipa{rɤ-pɣaʁ} ``to reclaim wasteland, to plough unused land'' as well as the two labile verbs mentioned in \ref{subsub:anticaus.syntax}, the semantic derivation between the base verb and its antipassive form is not predictable.

\subsubsection{diachronic origin} \label{subsub:antipass.dia}
  
XXXX add tWrACphAt ɯʑo tɯ-rɤɕphɤt to-βzu 他打了很多补丁 
\subsection{Passive} \label{sub:passive}
The Japhug passive is formed with the prefix \ipa{a}--, which as all contracting prefixes (see \ref{sec:contracting}) presents three allomorphs  \ipa{a}--,  \ipa{ɤ}--, and  \ipa{kɤ}-- depending on the preceding prefix. 

This prefix is etymologically related to the reciprocal (\ref{sub:reciprocal}) and the intransitive thematic element \ipa{a}-- (\ref{subsub:intransitive.det}), but these forms are best kept distinct in a synchronic description.



\subsubsection{Morphophonology} \label{subsub:passive.morph}
The passive is productive, but occurs only relatively rarely in traditional stories. The most common examples are the following:
\begin{table}[H]
\caption{Examples of the passive prefix \ipa{a}--}\label{tab:passive} \centering
\begin{tabular}{lllllllll} \toprule
basic verb  & &derived  verb &\\
\midrule
\ipa{mbi}   &	to give&  	\ipa{a-mbi}   &	 to be given \\  
\ipa{prɤt}   &	to cut &  	\ipa{a-prɤt}   &	 to be cut \\   
\ipa{mpʰɯr}   &	to wrap &  	\ipa{a-mpʰɯr}   &	 to be wrapped\\  
%\ipa{mto}   &	to see&  	\ipa{a-mto}   &	 to be seen \\ 
\ipa{pa}   &	to close &  	\ipa{a-pa}   &	 to become, to be closed \\  
\ipa{ta}   &	to put &  	\ipa{a-ta}   &	 to be put \\  
\ipa{βzu}   &	to do   &  	\ipa{a-βzu}   &	 to become, to grow \\  
\bottomrule
\end{tabular}
\end{table}


The passive \ipa{a}-- is a contracting prefix  like the progressive \ipa{asɯ}--. We will not describe here   in detail the distribution of its allomorphs, since this topic is discussed in \ref{sec:contracting}. It suffices to remind here that \ipa{a}-- can be realized as \textit{ɤ}-- when merging with some prefixes, and that with the evidential prefixes it becomes \textit{kɤ}--, and a suffix --\textit{chɯ} has to be added after the verb stem, as in the following example:
\begin{exe}
\ex
\gll  \ipa{nɯnɯ} 	\ipa{pjɤ-ɣɤŋgi} 	\ipa{ma} 	\ipa{ɯ-pa} 	\ipa{nɯ} 	\ipa{tɕu,} 	\ipa{rŋɯl} 	\ipa{tɯ-taɴɢoq} 	\ipa{ʑo} 	\ipa{\textbf{pjɤ}-kɤ-rkú-\textbf{chɯ},}   \\
\dem{}.\textsc{distal} \evd{}.\ipf{}-be.right because 3\sg{}.\poss{}-down \topic{} \loc{} silver one-basket \emphat{} \evd{}.\ipf{}-\pass{}-put.in-\evd{} \\
 \glt  That one was right, as a basketfull of silver was place under (the boulder).  (The divination, 93)
\end{exe} 
 
The aorist and past imperfective (\ipa{pɯ}-- prefix) forms of passive verbs are homophonous with that of third person direct transitive, though their morphological structure is different. Thus a phonetic form such as [pata] can be underlying analysed in two ways:
\begin{exe}
\ex
\gll   \ipa{pa-ta}  /  \ipa{pɯ-a-ta}\\
 \aor{}.3>3-put / \pst{}.\ipf{}-\pass{}-put \\
 \glt   He put it down / It was put.
\end{exe} 

However, number agreement clearly disambiguates between transitive direct and passive forms. For instance, in the following example, agreement  occurs with the S:

 
\begin{exe}
\ex
\gll  \ipa{tɤ-sta} 	\ipa{zɯ} 	\ipa{pɯ-atá-ndʑi} 	\ipa{nɤ} 	\ipa{pɯ-sí-ndʑi}  \\
\neu{}-place \loc{}  \pst{}.\ipf{}-\pass{}-put-\du{} \coord{} \aor{}-die-\du{} \\
 \glt    They had died, lying in their bed. (Lobzang, 78)
\end{exe} 

If [patandʑi] were analysed as a transitive form, it would mean ``they_{du} put it down''. 

%ɯngra pe mɤpe nɯ, kɤnɤma pe mɤpe arɤtɕha
%
%
%zrɤtɕʰa arɤtɕʰa
%
%nɤsɯm ɕe mɤɕe tɤzrɤtɕhe



 

\subsubsection{Syntax}  \label{subsub:passive.syntax}

The passive, like the anticausative (\ref{sub:anticausative}), changes a transitive verb into an intransitive stative verb whose S corresponds to the P of the original verb.
 
It is an agentless passive in that the agent cannot be specified in the same clause, and can only be used with a third person  S (generally, but not exclusively, inanimate), except for \ipa{aβzu} ``to become, to grow'' and \ipa{apa} ``to become'', two verbs that will be discussed in more detail in \ref{subsub:passive.become}. No overt agent can appear in the same sentence. 

This is not to say that the passive is forbidden when the agent is known. For instance, in one text we find the following sentence:
\begin{exe}
\ex
\gll     \ipa{tɯ-rdoʁ} 	\ipa{pa-qrɯ} 	\ipa{tɕe,} 	\ipa{ɯ-ŋgɯ} 	\ipa{nɯ} 	\ipa{tɕu} \ipa{rŋɯl} 	\ipa{qhoʁqhoʁ} 	\ipa{tɯ-rdoʁ} 	\ipa{pjɤ-kɤ-mphɯ́r-chɯ,} 
  \\
 one-piece	\aor{}-3>3-tear	\coord{}	3\sg{}.\poss{}-inside	\topic{}  \loc{} silver	ingot		one-piece	\ipf{}.\evd{}-\pass{}-wrap-\evd{} \\
 \glt   He opened one piece of (bread), and there was a wrapped silver ingot inside it.  (The Raven, 112)
\end{exe} 


Although not formally expressed in this sentence, the agent who put the silver ingots in the bread is known. A few sentences back in the same story, we read:
\begin{exe}
\ex
\gll  	\ipa{rŋɯl} 	\ipa{qhoʁqhoʁ} 	\ipa{tɯ-rdoʁ} 	\ipa{ntsɯ} 	\ipa{ko-mphɯr}  \\
silver	ingot		one-piece	always	\evd{}-wrap  \\
 \glt   She (a character named Lhamo) put a silver ingot in each (bread). (The Raven, 109)
\end{exe} 


In the case of bitransitive verbs whose O is the recipient, like \ipa{mbi} ``to give'', the S of the passive form sometimes corresponds to the external argument rather than to the recipient:

\begin{exe}
\ex
\gll  \ipa{ki} 	\ipa{tɤpotso} 	\ipa{saχsɯ} 	\ipa{ɣɯ} 	\ipa{ɯ-smɤn} 	\ipa{a-mbi,} 	\ipa{tɯrmɯ} 	\ipa{ɣɯ} 	\ipa{nɯ} 	\ipa{mɤ-a-mbi}     \\
 \dem{}.\textsc{prox} child lunch \gen{} 3\sg{}-medecine  \npst{}:\pass{}-give evening \gen{} \topic{} \negat{}-\npst{}:\pass{}-give\\
 \glt     This child's midday medicine has been given, but his evening medicine has not. (el. Chen Zhen, \wav{8_ambi})
\end{exe} 
 
 This reminds the fact that the O-participle of the verb \ipa{mbi} can refer not the recipient, but to the external argument too (\ref{sec:relativisation.external}). The aorist participle form  \ipa{nɯ-kɤ-mbi} for instance can mean ``the one that was given''. 
 
 
 Given the fact that the S-participle prefix \ipa{kɯ}-- contracted with the passive \ipa{a}-- should yield \ipa{kɤ}--, a form homophonous with the O-participle, one can wonder if indeed the O-participle is not the product of this fusion. In other word, should  forms such as \ipa{nɯ-kɤ-mbi} ``the one that was given'' or \ipa{tɤ-kɤ-ndza} ``the one that was eaten'' be analysed as:
 
  \begin{exe}
\ex
\gll      \ipa{nɯ-kɯ-ɤ-mbi} / \ipa{tɤ-kɯ-ɤ-ndza}  \\
?\aor{}-\nmlz{}:S-\pass{}-give  / ?\aor{}-\nmlz{}:S-\pass{}-eat\\
\end{exe} 
 
Such analyses would seem possible phonologically, but are clearly incorrect since the relative clauses where these participles are used can include the agent:
 \begin{exe}
\ex
\gll      [\ipa{qajɯ} 	\ipa{kɯ} 	\ipa{tɤ-kɤ-ndza}] 	\ipa{pjɯ-tu} 	\ipa{mɤ-jɤɣ,}   \\
worm \erg{} \aor{}-\nmlz{}:O-eat \ipf{}-be.there \negat{}-\npst{}:could  \\
 \glt You cannot have the ones that have been eaten by worms.    (The rTsampa 24)
\end{exe} 
  
 Since, as mentioned above, passive verbs do not accept an overt agent present in the same sentence, the form \ipa{tɤ-kɤ-ndza} in the sentence above cannot be analysed as a passive participle.
 
  

\subsubsection{Compatibility}  \label{subsub:passive.compat}
The passive cannot be used with many other derivations. In particular, it is incompatible with all other valency decreasing morphological processes. The only good examples of compounds involving the passive with another affix are the combination of the causative with the passive \ipa{sɯ}-- + \ipa{a}-- which after vowel contraction yields the form \ipa{sɤ}-- homophonous with the antipassive and the deexperiencer. 

Note also that the complex prefix \ipa{sɤ}-- can be combined with the reflexive \ipa{ʑɣɤ}-- in verbs such as \ipa{ʑɣɤ-sɯ-ɤ-βzu} ``to transform oneself into".

These complex derivations are studied in detail in \ref{subsub:caus1.compatibility}.

\subsubsection{Semantics}  \label{subsub:passive.semantics}
Passive and anticausative are derivations that change a transitive verb into an intransitive one whose S corresponds to the O of the base verb. These two derivation could appear to be very similar, but their semantics differ in two ways. 

First, although the semantic agent cannot appear overtly in the same sentence in the passive, the existence of a volitional agent is not excluded, unlike the case of the anticausative, where the action occurs spontaneously.

Second, the passive verbs are fundamentally stative. They can occur with past imperfective and evidential imperfective, like all stative verbs (see \ref{sec:past}), to express a former state that is no longer true:

 
\begin{exe}
\ex
\gll   \ipa{tɕheme} 	\ipa{nɯ} 	\ipa{ci} 	\ipa{thɯ-sta} 	\ipa{ri,} 	\ipa{ɯ-mphɯz} 	\ipa{thɤcu} 	\ipa{nɯ} 	\textbf{\ipa{pɯ-a-ta}} 	\ipa{ɕti}    \\
  girl \topic{} a.little \aor{}-wake.up but 3\sg{}.\poss{}-bottom downstream \topic{}    \pst{}.\ipf{}-\pass{}-put \npst{}:be.\emphat{}\\
 \glt   The girl woke up, and it (the horse embryo) was there / someone had put it under her bottom.  (Kun.bzang, 106)
\end{exe} 
 	 
\begin{exe}
\ex
\gll   \ipa{nɯ} 	\ipa{mbro} 	\ipa{ɯ-pɯ} 	\ipa{nɯnɯ} 	\ipa{ɯ-jme} 	\ipa{nɯ} 	\ipa{tɕu} 	\textbf{\ipa{pjɤ-kɤ-ta-chɯ}} 	\ipa{tɕe,} 	\ipa{ɯ-tɯ-nɤzraʁ,} 	\ipa{ɯ-tɯ-nɯzdɯɣ} 	\ipa{pjɤ-saχaʁ}     \\
 \dem{} horse 3\sg{}.\poss{}-baby \dem{}.\textsc{distal}  3\sg{}.\poss{}-tail \topic{} \loc{} \evd.\ipf{}-\pass{}-put-\evd{} \coord{} 3\sg{}-\nmlz{}:\textsc{degree}-be.ashamed 3\sg{}-\nmlz{}:\textsc{degree}-be.worried \evd.\ipf{}-extremely  \\
 \glt  A horse embryo had been put at her bottom, and she was very ashamed, very worried.  (The three sisters, 128)
\end{exe}  	 
 	 
 In the non-past, the passive also expresses either a future state or an ongoing state (that the speaker knows to be so not be sensory experience, but by intimate knowledge):
 
\begin{exe}
\ex
\gll      \ipa{ɯ-phɯŋgɯ} 	\ipa{nɯ} 	\ipa{tɕu} 	\ipa{qapɯtɯm} 	\ipa{ci} 	\ipa{na-rku} 	\ipa{ɲɯ-ŋu} 	\ipa{tɕetha} 	\ipa{qhuj} 	\ipa{tɕe} 	\ipa{ki} 	\ipa{a-phɯŋgɯ} 	\textbf{\ipa{a-rku}} 	\ipa{tɕe,} 	\ipa{tɕetha} 	\ipa{pɯ-mto-t-a} 	\ipa{ʑo} 	\ipa{tɕe,} 	\ipa{rɟɤlpu} 	\ipa{ɯ-phe} 	\ipa{tu-ti-a} 	\ipa{ɲɯ-ra}   \\
 3\sg{}.\poss{}-bosom \topic{} \loc{} pebble a \aor{}.3>3-put.in \ipf{}-be a.while this.evening \coord{} this  1\sg{}.\poss{}-bosom \npst{}:\pass{}-put.in \coord{}   a.while \aor{}-see-\pst{}-1\sg{} \emphat{} \coord{} king 3\sg{}-\dat{} \ipf{}-say-1\sg{} \ipf{}-have.to \\
 \glt   He put a pebble stone in his bosom, thinking: ``This evening, it will be in my bosom, and as I see it, I will certainly tell (the story) to the king. (Kunbzang, 279)
\end{exe} 
 
 \begin{exe}
\ex
\gll      \ipa{ɯ-rkɯ} 	\ipa{zɯ,} 	  	\ipa{xɕɤj} 	\ipa{tɤ-loʁ} 	\ipa{ci} 	\textbf{\ipa{a-ta}} 	\ipa{tɕe,} 	\ipa{nɯ} 	\ipa{ɯ-ŋgɯ} 	\ipa{a-tɤ-tɯ-rke}  \\
 3\sg{}.\poss{}-side \loc{} grass \neu{}-nest \indef{} \npst{}:\pass{}-put \coord{} \dem{}.\textsc{distal}  3\sg{}.\poss{}-inside \irr{}-pfv-2-put.in[III] \\
 \glt On its side, there is a nest made of grass, you will put (the bird) inside it.    (The fox, 29)
\end{exe} 
  
Using the passive here indicates that at the future time in question, the action of putting will be completed, and that only the result of the action will be witnessed. To express a meaning such as ``Someone will put it (and you will witness this action)'', the passive cannot be used.

%Passive verbs are not commonly found in the evidential and aorist forms in texts
\subsubsection{The two verbs ``to become''} \label{subsub:passive.become}
   \ipa{a-pa} ``to become'' and \ipa{a-βzu} ``to become, to grow''  stand apart other passives in several ways.
   
   First,  their meaning is not transparently related to that of their respective base verb. Although the evolution ``do'' > ``become'' is cross-linguistically well attested, the relation between the base and the passive verbs here are purely diachronic; synchronically, they are better treated as separated verbs. Note that \ipa{a-pa} also means ``to be closed'', the regular synchronic meaning expected from the passive of \ipa{pa} ``to close''.
   
   Second, used in the aorist or evidential, these two verbs do not even have a resultative meaning:

 
\begin{exe}
\ex
\gll     \ipa{tɕe} 	\ipa{ʁʑɯnɯ} 	\ipa{ci} 	\ipa{rcánɯ} 	\ipa{kɯ-wxtɯ-wxti} 	\ipa{ʑo} 	\ipa{nɯ-aβzu} 	\ipa{ɲɯ-ŋu.}        \\
       \coord{} young.man \indef{} \emphat{} \nmlz{}:\stat{}-\redp{}-big \empaht{} \aor{}-become \ipf{}-be        \\
 \glt    Then, he changed into a big young man. (The fox, 193)
\end{exe} 


Third, these verbs  appear with first and second person, unlike other passive verbs:

\begin{exe}
\ex
\gll  \ipa{ki} 	\ipa{kɯ-fse} 	\ipa{nɯ-aβzu-a} 	\ipa{tɕe} 	\ipa{kɤ-nɯɕe} 	\ipa{mɯ-ɲɤ-cha-a}        \\
          this \nmlz{}:\stat{}-be.in.such.a.way  \aor{}-become-1\sg{} \coord{} \inftv{}-go.back \negat{}-\evd{}.\perm{}-can-1\sg{}   \\
 \glt I became like this, and I cannot go back.   (The divination2, 31)
\end{exe} 
 
Fourth, the causative forms of these two verbs \ipa{sɤpa} and \ipa{sɤβzu} differ from that of the other passive verbs, as explained in \ref{subsub:caus1.compatibility}.

Thus, it can be concluded that these verb are not really passive synchronically, just as irregular causatives such as \ipa{jtsʰi} ``give to drink'' are not real causatives.


\subsection{Reciprocal}  \label{sub:reciprocal}
The reciprocal derivation in Japhug is distinct from both the reflexive and the passive. As the atelic and the reflexive-tropative derivations, it is generally formed by combining prefixes to the reduplicated verb stem.
\subsubsection{Morphophonology} \label{subsub:recip.morpho}
The reciprocal derivation is extremely productive. Regular reciprocals are formed by adding the contracting prefix \ipa{a}-- to the partially reduplicated verb stem. If the verb stem is more than one syllable long, only the last syllable is reduplicated.

Two other non-productive reciprocal formations also exist, one with a double prefix \ipa{a-mɯ}--, and one with the prefix \ipa{a-nɤ}--. The following table presents some of the most common reciprocal verbs, including all the examples of irregular reciprocal formations:

\begin{table}[H]
\caption{Examples of reciprocal verbs}\label{tab:reciprocal} \centering
\begin{tabular}{lllllllll} \toprule
basic verb  & &derived  verb &\\
\midrule
\ipa{ndza}   &	to eat&  	\ipa{a-ndzɯ-ndza}   &	 to eat each other \\  
\ipa{rqoʁ}   &	to hug&  	\ipa{a-rqɯ-rqoʁ}   &	 to hug each other \\  
\ipa{lɤt}   &	to throw&  	\ipa{a-lɯ-lɤt}   &	 to fight each other \\  
\ipa{nɯrɯtʂa}   &	to envy&  	\ipa{a-nɯrɯtʂɯ-tʂa}   &	 to envy each other \\  
\ipa{nɯsɯkʰo}   &	to extort  &  	\ipa{a-nɯsɯkhɯ-kʰo}   &	 to extort from each other \\  
\ipa{nɤqɤtsa}   &	to be adequate with  &  	\ipa{a-nɤqɤtsɯ-tsa}   &	 to be adequate together \\  
\ipa{χpjɤt}   &	to observe  &  	\ipa{a-χpɯ-χpjɤt}   &	 to observe each other \\  
 \midrule
 \ipa{stʰaβ}   &	to touch  &  	 \ipa{a-mɯ-stʰaβ}   &	to be one next to the other \\  
 \ipa{ti}   &	to say &  	 \ipa{a-mɯ-ti}   &	to  talk to each other\\ 
  \ipa{mtsʰɤm}   &	to hear &  	 \ipa{a-mɯ-mtsʰɤm}   &	to  hear each other \\ 
  \ipa{mto}   &	to  see&  	 \ipa{a-mɯ-mto}   &	to  see each other \\ 
  \ipa{tso}   &	to  understand (vi)&  	 \ipa{a-mɯ-tso}   &	to  understand each other \\ 
  \ipa{rpu}   &	to bump into &  	 \ipa{a-mɯ-rpu}   &	to  bump into each other \\ 
  \ipa{atɯɣ}   &	to  meet, to touch (vi)&  	 \ipa{a-mɯ-tɯɣ}   &	to  meet  each other \\ 
  \ipa{nɯfse}   &	to know (someone) &  	 \ipa{a-mɯ-fse}   &	to  know each other \\ 
  \ipa{arqʰi}   &	to be far (vi)&  	 \ipa{a-mɯ-rqʰi}   &	to  be far from each other \\ 
  \ipa{armbat}   &	to be near (vi)&  	 \ipa{a-mɯ-rmbat}   &	to  be close to each other \\ 
  \midrule
  \ipa{sɯso}   &	to think &  	 \ipa{a-nɤ-sɯso}   &	to miss each other \\ 
\bottomrule
\end{tabular}
\end{table}
All reciprocal verbs have the contracting prefix \ipa{a}--, which presents the allomorphs  \ipa{a}--,  \ipa{ɤ}-- and  \ipa{kɤ}--, depending on the preceding prefixes, as described in section \ref{sec:contracting}. In the evidential forms, an additional evidential suffix --\ipa{cʰɯ} is added, as shown by the following example of the verb \ipa{a-ndɯ-ndo} ``to attach to each other, to fight'' from \ipa{ndo} ``to take'':


\begin{exe}
\ex
\gll \ipa{ʑɯrɯʑari} 	\ipa{tɕe} 	\ipa{ko-kɤndɯndo-ndʑi-cʰɯ}         \\
       progressively \coord{} \evd{}-\recip{}:take-\du{}-\evd{}    \\
 \glt They progressively started to fight.   (The mouse and the sparrow, 51)
\end{exe} 
 
 

When the base verb is already derived, and contains other prefixes (\ref{subsub:recip.compat}), reduplication applies to the last syllable of the verb stem regardless whether a morpheme break is present:
\begin{exe}
\ex
 \glt \ipa{ɲaʁ} ``to be black'' > \ipa{sɯɣ-ɲaʁ} ``to blacken'' (causative)
  \glt > \ipa{a-sɯɣɲɯ\textbf{ɣ}ɲaʁ} ``to blacken each other''
 \glt \ipa{bɯɣ} ``to miss'' > \ipa{nɯɣ-bɯɣ} ``to miss s.o.'' (applicative)
  \glt > \ipa{a-nɯɣbɯ\textbf{ɣ}bɯɣ} ``to miss each other''
 \glt \ipa{tɯ-jaʁ} ``hand'' > \ipa{tɯ-jaʁndzu} ``finger''  > \ipa{sɯ-jaʁndzu} ``to point with the finger''  (denominal)
  \glt > \ipa{a-sɯja-ndzɯ\textbf{ʁ}ndzu} ``to miss each other''
\end{exe} 

In the forms above, the final consonant of the prefix --\ipa{ɣ}-- or the final consonant of the nominal root \textit{--jaʁ} ``hand'' is resyllabified to the next syllable, and included in the reduplicated syllable: reduplication is blind as to morphological structure.

In \ipa{a-mɯ}-- reciprocals, the final syllable can be optionally reduplicated. For instance, one can say both  \ipa{a-mɯ-tso} or  \ipa{a-mɯ-tsɯ-tso} ``to understand each other''.

Some \ipa{a-mɯ}-- reciprocals are derived from intransitive verbs, and especially from intransitives with the intransitive thematic element \ipa{a}--. For these verbs, the derivation is (at least synchronically) to insert the prefix \ipa{mɯ}-- between the thematic element \ipa{a}-- and the verb root, as in \ipa{arqʰi} ``to be far'' > \ipa{a-\textbf{mɯ}-rqʰi} ``to be far from each other''.

Some verbs present reciprocal morphology, but the verb base from which they are derived is not attested synchronically. This includes \ipa{aʑɯʑu} ``to wrestle'' from a non-attested root *ʑu, and \ipa{amɯmi} ``to be in good terms'' from a root *mi. For this last example, it is unclear whether it counts as a regular case of reciprocal, or as a \ipa{a-mɯ}-- reciprocal, as the reduplicated form of /mi/ is /mɯ/ anyway.

From a historical point of view, it is clear that the reciprocal and the passive are related, and the path of grammaticalization that led to the creation of the reciprocal form is relatively straightforward. It is instructive in this regard to compare \ipa{a}-- to valency-decreasing affixes in other languages which can assume both passive and reciprocal meaning.

The prefix  \textit{v}- (orthographical notation for [ə]) in Rawang precisely has these two functions.\footnote{In spite of their functional similarity, it is quite improbably that Japhug \ipa{a}, which originates from proto-Japhug *ŋa, and Rawang \ipa{v}--  are historically related.} It can derive passive-like verb forms, whose S corresponds to the O of the transitive verb (\citealt[288]{lapolla01valency}):
\begin{exe}
\ex
\gll \ipa{tv́l-ò-ē} > 	\ipa{v-tv́l-ē} 	        \\
roll-3>3-\npst{} >  \textsc{intransitive}-roll-\npst{}  \\
 \glt  to roll (vt) >		to roll (vi)
\end{exe} 
The base verb has the direct suffix --\ipa{o} which only appears with transitive verbs, while this suffix cannot appear with verbs prefixed with \ipa{v}--. LaPolla also points out that ``if the single direct argument of the derived intransitive is a plural animate argument, then the meaning is reciprocal'':
\begin{exe}
\ex
\gll \ipa{àngmaq}  	\ipa{v-shvt-ē} 	        \\
 they		 \textsc{intransitive}-hit/kill-\npst{}  \\
 \glt  They are fighting.
\end{exe} 
 
 The main difference between Japhug \ipa{a}-- and Rawang \ipa{v}-- is that in the former, reduplication is obligatory with the reciprocal use of this prefix. It is interesting in this regard to note that some Rgyalrongic languages reduplicate the stem to form plurals (\citealt[29-30]{huangbf91daofu}):

\begin{table}[H]
\caption{The paradigm of the verb ``to go'' in Rtau}\label{tab:rtau} \centering
\begin{tabular}{lll} \toprule
person  & singular & plural\\
\midrule
1& \ipa{ɕo-ŋ} &	\ipa{ɕə-ɕo-ŋ} \\
2& \ipa{ɕi-n} &	\ipa{ɕə-ɕi-n} \\
3 & \ipa{ɕə} &	\ipa{ɕə-ɕə} \\ 
\bottomrule
\end{tabular}
\end{table}
On the basis of the Rawang and Rtau data, it becomes possible to explain how  the reciprocal came into being in the modern Rgyalrong languages as well as its historical relationship with the passive.


We need to make two assumptions. First, that the proto-Rgyalrongic  ancestor\footnote{Since no real reconstruction of proto-Rgyalrongic is yet available, we will not attempt at providing a protoform here, but comparative evidence from other Rgyalrong languages suggest a shape *ŋa.} of the prefix \ipa{a}-- had a function similar to that of Rawang \ipa{v}--. Second, that the reduplication as a marker of plurality present in Rtau is a preservation from proto-Rgyalrongic, not a local innovation.

From these assumptions, it follows logically that when a verb prefixed in  \ipa{a}--  prefixed verb had a plural animate argument, the verb stem was reduplicated, and   reciprocal meaning was a secondary meaning of this form as in Rawang. After verb stem reduplication ceased to be a marker of plurality, reduplication on \ipa{a}--  prefixed verbs was reanalyzed as a marker of reciprocality. Rgyalrong languages therefore exemplify a gramamticalization path:

\begin{exe}
\ex
 \glt  \textsc{plural} + \textsc{passive} > \textsc{reciprocal}
\end{exe} 

This theory of course does not provide an explanation for the  \ipa{a-mɯ}-- reciprocals, which have a distinct origin.


\subsubsection{Syntax}
The reciprocal forms change a transitive verb into an intransitive, whose S corresponds both to the A and the O of the base verbs. Some  \ipa{a-mɯ}-- reciprocals however are derived from intransitive verbs, and will be treated separately.

Since reciprocal verbs always involve several entities interacting with each other, these verbs always have a dual or plural S, never a singular S. The S can be a long noun phrase, made of several constituents linked with the conjunction \ipa{cʰo}:


\begin{exe}
\ex
\gll 	 	\ipa{qʰlɯʁdɯɣpakɤrpu} 	\ipa{nɯnɯ} 	\ipa{ɣɯ} 	\ipa{ɯ-tɕɯ} 	\ipa{nɯ} 	\ipa{cʰo} 	\ipa{aʑo} 	\ipa{a-tɕɯ} 	\ipa{nɯ} 	\ipa{jisŋi} 	\ipa{tɕe} 	\ipa{to-kɤlɯlɤ́t-ndʑi,} 	\ipa{pjɯ-ɤsɯsát-ndʑi} 	\ipa{pɯ-ŋu} 	\ipa{ri,} \\
Klu.gdug.pa.dkar.po \dem{} \gen{} 3\sg{}.\poss{}-son \topic{} and I 1\sg{}.\poss{}-son \topic{} today \coord{} \evd{}-fight.each.other-\du{} \ipf{}-\textsc{recip}:kill-\du{} \pst{}.\ipf{}-be but \\
 \glt  Klu.gdugpa dkarpo's son and my son  fought each other today, they were about to kill each other,  (Smanmi, 40)
 
\end{exe} 
Although passive and reciprocals are historically related as presented in the previous subsection, reciprocal verbs strongly differ from passive verbs. First, reciprocal verbs are intrinsically dynamic: they appear with the periphrastic past imperfective. Second, they are not restricted to the third person, unlike passive verbs:

\begin{exe}
\ex
\gll \ipa{cʰɯ-ɤndzɯndzá-ndʑi} 	\ipa{pjɤ-ŋu} 	\ipa{ri,} 	\ipa{``ma-tɤ-ɤndzɯndzá-ndʑi} \ipa{maka} \ipa{ndʑi-kʰa} \ipa{pɯ-nɯɕé-ndʑi''} 	\ipa{to-ti} 	\ipa{ri,}          \\
 \ipf{}--\recip{}:eat-\du{} \evd{}.\ipf{}-be but \negat{}-\imp{}-\recip{}:eat-\du{} entirely 2\du{}.\poss{}-house  \imp{}:down-go.back-\du{} \evd{}-say but \\
 \glt  They were eating each other, and he said ``Do not eat each other, go back to your respective homes.'' (Smanmi, 25)
\end{exe} 

 The \ipa{a-mɯ}-- reciprocal construction, unlike \ipa{a}--+reduplication, can derive reciprocal verbs out of both transitive and intransitive verbs. Examples derived from intransitive verbs include \ipa{a-mɯ-tso} ``to understand each other'',  \ipa{a-mɯ-tɯɣ} ``to meet each other'', \ipa{a-mɯ-rqʰi} ``to be far from each other'' and \ipa{a-mɯ-rmbat} ``to be close from each other''. All four base verbs \ipa{tso} ``to understand'', \ipa{atɯɣ} ``to meet'', \ipa{arqʰi} ``to be far'' and \ipa{armbat} ``to be near'' can appear with an external argument:
 
  
 \begin{exe}
\ex
\gll    	\ipa{mbro} \textbf{	\ipa{tɯ-skɤt}} 	\ipa{kɯ-tso} 	\ipa{ci,} 	\ipa{a-mbro} 	\ipa{tɤ-rkú-nɯ} 	\ipa{ra}   \\
  horse \neu{}-speech \nmlz{}:S-understand \indef{} 1\sg{}.\poss{}-horse \imp{}-put.in-\pl{} \npst{}:have.to\\
 \glt  Prepare for me a horse who understands speech.  (The oath, 24)
\end{exe} 
 
 \begin{exe}
\ex
\gll   \ipa{tɤ-wɯ}, \textbf{\ipa{rgɤtpu}} \textbf{\ipa{tɤkɤwɣrum} }	\textbf{\ipa{nɯ}} 	\ipa{nɯ-atɯɣ} 	\ipa{ɲɯ-ŋu}    \\
   \neu{}-grand.father old.man with.a.white.head  \topic{} \aor{}-meet \ipf{}-be \\
 \glt  He met the old man whose hair was white.  (Divination2, 33)
\end{exe} 
 
 \begin{exe}
\ex
\gll  Beijing 	\ipa{nɯ} \textbf{	\ipa{kutɕu}} 	\ipa{pɤrthɤβ} 	\ipa{arqʰi} 	\ipa{ɕi} 	\ipa{kɯma?} \\
Beijing \topic{} here in.between \npst{}:far \qu{} \textsc{part}  \\
 \glt Is it far between here and Beijing?   (Watching the snow, 15)
\end{exe}  

 \begin{exe}
\ex
\gll   \ipa{tɕeri} 	\textbf{\ipa{nɯ-kʰa} }	\textbf{\ipa{ɯ-tʰɤcu} }	\ipa{kɯ-ɤrmbat} 	\ipa{nɯ} 	\ipa{tɕu,} 	\ipa{rɟɤlpu} 	\ipa{ci} 	\ipa{ɯ-kʰa} 	\ipa{pjɤ-tu} 	\ipa{tɕe,}  \\
  but 3\pl{}.\poss{}-house 3\sg{}.\poss{}-downstream \nmlz{}:\stat{}-near \topic{} \loc{} king \indef{} 3\sg{}.\poss{}-house \pst{}.\ipf{}-be.there \coord{}   \\
 \glt However, close to their house further downstream, there was the house of a king.   (Nyima vodzer 84)
\end{exe}  
  \ipa{arqʰi}  and \ipa{armbat} indicate the distance between two places or two persons. Used in their reciprocal forms, the expressions referring to the two places must belong to the same noun phrase (indicated between square brackets in the following example):
   \begin{exe}
\ex
\gll [\ipa{nɤj} 	\ipa{nɤ-kʰa} 	\ipa{cho} 	\ipa{aʑo} 	\ipa{a-kʰa}] 	\ipa{nɯ} 	\ipa{mɯ-ɲɯ-ɤmɯrmbat-ndʑi}    \\
you 2\sg{}.\poss{}-house  and I 1\sg{}.\poss{}-house \topic{} \negat{}-\const{}-\textsc{recip}:near-\du{} \\
 \glt Your house and my house are not close to each other.  (el., Chen Zhen\wav{amWrmbat})
\end{exe}  

While the  \ipa{a}--+reduplication construction is restricted to morphologically transitive verbs,  the  \ipa{a-mɯ}-- reciprocal construction can  be used with verbs that are semantically, but not morphologically, transitive.

\subsubsection{Compatibilities}  \label{subsub:recip.compat}

The  \ipa{a-mɯ}-- reciprocal only appears with the \ipa{sɯ}-- causative, as in the following example:
  \begin{exe}
\ex
\gll    \ipa{tʂu} 	\ipa{kɤ-βzu} 	\ipa{nɯ-sɯ-ɤmɯtɯɣ-i}   \\
  road \inftv{}-makr \aor{}-\caus{}-\textsc{recip}:meet-1\pl{} \\
 \glt  While constructing the road, we made (the two ends) meet each other. (el, Chen Zhen)
\end{exe}  


The reciprocal \ipa{a}--+\textsc{reduplication} on the other hand is compatible with all valency-increasing derivations. Both on the case of the \ipa{sɯ}-- causative, both orderings causative+reciprocal or reciprocal+causative are possible, depending on semantic scope.

In the ordering causative+reciprocal, the causative has scope over the reciprocal:
   \begin{exe}
\ex
\gll    \ipa{ʑimkʰɤm} 	\ipa{ʑo} 	\ipa{pjɤ-fse} 	\ipa{tɕendɤre} 	\ipa{tú-wɣ-sɯ-ɤlɯlɤ́t-ndʑi} 	\ipa{pjɤ-ŋu} 	\ipa{tɕɤn}    \\
    long.time \emphat{} \evd{}.\ipf{}-be.in.such.a.way \coord{} \ipf{}-\inv{}-\caus{}-fight.each.other-\du{}  \evd{}.\ipf{}-be \coord{}  \\
 \glt     It was like that for a long time, and (the one-foot demon) caused both of them to fight. (Rkang.rgyal01, 15)
\end{exe} 

In the ordering reciprocal+causative, the reciprocal has scope over the causative:

   \begin{exe}
\ex
\gll   \ipa{ɲɤ-kɤ-sɯɴqʰɯɴqʰi-ndʑi} \ipa{tɕe}, \ipa{ɲɤ-kɤsɯɣɲɯɣɲaʁ-ndʑi-cʰɯ} \ipa{ʑo}    \\
     \evd{}-\recip{}:\caus{}:be.dirty-\du{} \coord{} \evd{}-\recip{}:\caus{}:be.black-\du{}-\evd{} \emphat{} \\
 \glt They got each other dirty and smeared each other black.  (el., Chen Zhen \wav{asWGYWGYaR})
\end{exe} 

The form  of the reciprocal causative of \ipa{ɲaʁ} ``to be black'' is discussed in \ref{subsub:recip.morpho}. The combination of reciprocal and causative is not straightforward to cut down into morphemes, as the causative (and other derivational affixes) is sandwiched between the prefix \ipa{a}-- and the reduplicated syllable, which is the last syllable of the stem:
  \begin{exe}
\ex
\gll   \ipa{a-sɯ-ɴqʰɯ-ɴqʰi} \\
		\recip{}-\caus{}-dirty-\textsc{reduplicated.syllable} \\
 \glt to  cause each other to become dirty.
\end{exe} 


The reciprocal also appears with the applicative \ipa{nɯ(ɣ)}--, the tropative \ipa{nɤ}-- and the causative \ipa{ɣɤ}-- in forms such as:
  \begin{exe}
\ex
\gll  \ipa{bɯɣ} > \ipa{nɯɣ-bɯɣ} > \ipa{a-nɯɣ-bɯ-ɣbɯɣ} \\
		miss > \appl{}-miss > \recip{}-\appl{}-miss-\textsc{reduplicated.syllable} \\
 \glt to miss each other
 \end{exe}
  \begin{exe}
\ex
\gll  \ipa{mpɕɤr} > \ipa{nɤ-mpɕɤr} > \ipa{a-nɤ-mpɕɯ-mpɕɤr} \\
		be.beautiful > \trop{}-be.beautiful > \recip{}-\trop{}-be.beautiful-\textsc{reduplicated} \\
 \glt to consider each other to be beautiful  \wav{anAmpCWmpCAr}
\end{exe} 
  \begin{exe}
\ex
\gll  \ipa{rlaʁ} > \ipa{ɣɤ-rlaʁ} > \ipa{a-ɣɤ-rlɯ-rlaʁ} \\
		be.lost > \caus{}-be.lost > \recip{}-\caus{}-be.lost-\textsc{reduplicated} \\
 \glt to destroy each other  \wav{aGArlWrlaR}
\end{exe} 
 
The reverse order is impossible in the case of \ipa{ɣɤ}-- causative and \ipa{nɤ}-- tropative, as reciprocal verbs are dynamic, while these two derivations only apply to stative verbs. In the case of the applicative \ipa{nɯ(ɣ)}--, the order \textsc{applicative}+\textsc{reciprocal} is conceivable, though no good examples are found. The verb \ipa{nɯ-aʑɯʑu} ``to wrestle with'', the applicative of \ipa{aʑɯʑu} ``to wrestle'' however, constitutes at least historically such an example: the verb  \ipa{aʑɯʑu} is structurally a reciprocal, though no verb *ʑu is attested (\ref{subsub:appl.sem}). Attempt to create applicatives with regular reciprocal verbs however are rejected by native speakers.

The reciprocal \ipa{a}+\textsc{reduplication} construction is also compatible with denominal derivation, as in the case of \ipa{sɯjaʁndzu} ``to point with the finger'', a verb derived from \ipa{tɯ-jaʁndzu} ``finger'', whose reciprocal form is  \ipa{asɯjaʁndzɯʁndzu} ``to point each other with the finger''.
 

\subsubsection{Semantics}
The semantic derivation between reciprocal verbs and their base verbs is most of the time is most of the time transparent. Exceptions include  \ipa{lɤt} ``to throw'' > \ipa{alɯlɤt} ``to fight each other'' and \ipa{nɤrko} ``to endure" > \ipa{anɤrkɯrko} ``to force each other'', whose usage can be illustrated by the following example:


   \begin{exe}
\ex
\gll    \ipa{iɕqʰa} 	\ipa{tɯrme} 	\ipa{nɯ} 	\ipa{kɯ} 	  	\ipa{tʂʰa} \ipa{kɤ-tsʰi} \ipa{ra} 	\ipa{ɲɯ-ti} 	\ipa{ri,} 	\ipa{aʑo} 	\ipa{mɤ-tsʰi-a} 	\ipa{tɤ-tɯt-a} 	\ipa{ri,} 	\ipa{mɯ́j-khɯ} 	\ipa{tɕe} 	\ipa{tɤ-anɤrkɯrko-tɕi}   \\
the.aforementioned man \topic{} \erg{}  tea \imp{}-drin \npst{}:have.to \const{}-say but I  \negat{}-\npst{}:drink-1\sg{} \aor{}-say[II]-1\sg{} but \negat{}:\const{}-agree \coord{} \aor{}-force.each.other-1\du{}   \\
 \glt   This person told me to have some tea, I said I that I wouldn't but he would not agree, and we had a dispute. (el. Dpalcan)
 
\end{exe} 

The stative verb \ipa{amɯzɣɯt} ``to be homogeneous" seems to be a reciprocal derivation of \ipa{zɣɯt} ``to reach'' - the original meaning of this verb was perhaps ``to reach each other".

\subsection{Anticausative} \label{sub:anticausative}
Besides passive derivation, Japhug has another patient-preserving intransitivising derivation: anticausative prenalizalisation. This derivation was productive until recently, since at least one Tibetan loanword occurs among the examples of anticausative verb pairs: \ipa{χtɤr} ``to spill'' from Tibetan \ipa{gtor}, anticausative \ipa{ʁndɤr} ``to be spilled''.  The form \ipa{ʁndɤr} cannot be a loanword from Tibetan itself, since there is no corresponding Tibetan word and the phonological structure of its onset /fricative+prenasalized stop/ is never found in the Tibetan layer.

This example is also important, as it proves that the direction of derivation is from transitive to intransitive, not the other way round -- given the fact that anticausative is formed by a morphophonological process rather than an affix, the direction is not self-evident.

In the following subsections, we first list all attested examples of anticausative verbs and describe the morphophonological rules of the prenasalization process. Then, we study the syntactic properties of anticausative verbs and the compatibility of this process with other derivation affixes. Finally, we present the semantic properties of this derivation and its differences with passivization.

\subsubsection{Morphophonology} \label{subsub:anticaus.morph}
Anticausative prenasalization is one of the few non-concatenative derivational processes in Japhug. The anticausative verb is formed from the transitive verb by changing its unvoiced stop/affricate onset into the corresponding voiced prenasalized one. 





Only the following 24 verb pairs have been discovered up to now:

\begin{table}[H]
\caption{Examples of anticausative in Japhug}\label{tab:anticausative}
\begin{tabular}{lllllllll} \toprule
basic verb  & &derived  verb &\\
\midrule
\ipa{ftʂi}  &	to melt (vt)	&		\ipa{ndʐi}  &	to melt (vi)		\\
\ipa{kio}  &	to cause to drop	&		\ipa{ŋgio}  &	to slip		\\
\ipa{kra}  &		to cause to fall&		\ipa{ŋgra}  &	to fall		\\
\ipa{plɯt}  &	to destroy	&		\ipa{mblɯt}  &	to be destroyed		\\
\ipa{prɤt}  &	to cut	&		\ipa{mbrɤt}  &		to be cut	\\
\ipa{pɣaʁ}  &	to turn over (vt)	&		\ipa{mbɣaʁ}  &		to turn over (vi)	\\
\ipa{qɤt}  &	to separate	&		\ipa{ɴɢɤt}  &	to be separated		\\
\ipa{qʰrɯt}  &	to completely scratch	&		\ipa{ɴɢrɯt}  &	to be completely scratched		\\
\ipa{qrɯ}  &	to cut, to tear, to break	&		\ipa{ɴɢrɯ}  &	to break (vi), be torn		\\
\ipa{tɕɤβ}  &	to burn (vt)	&		\ipa{ndʑɤβ}  &	to be burned		\\
\ipa{tʰɯ}  &	to pitch (tent),  	&		\ipa{ndɯ}  &	to appear (rainbow), 	\\
 &	 to build (road, bridge)	&		   &	  to be built (road, bridge)		\\
\ipa{χtɤr}  &	 to spill	&		\ipa{ʁndɤr}  &		to be spilled	\\
\ipa{tʂaβ}  &	to cause to roll	&		\ipa{ndʐaβ}  &	to roll (vi)		\\
\ipa{qraʁ}  &	to tear	&		\ipa{ɴɢraʁ}  &		to be torn	\\
\ipa{qia}  &	to tear	&		\ipa{ɴɢia}  &		to get loose  	\\
\ipa{qlɯt}  &	to break	&		\ipa{ɴɢlɯt}  &		to be broken	\\
\ipa{sɤpʰɤr}  &	to shake off, to wipe off	&		\ipa{mbɤr}  &	wiped off	 	\\
 \ipa{pri}  &	 to tear	&		\ipa{mbri}  &	to be torn	 	\\
  \ipa{xtʰom}  &	 to put horizontally	&		\ipa{ndom}  &	 	to be horizontal 	\\
  \ipa{tɕɣaʁ}  &	 to squeeze out 	&		\ipa{ndʑɣaʁ}  &	 to be squeezed out	 	\\ 
   \ipa{kɤɣ}  &	 to bend 	&		\ipa{ŋgɤɣ}  &	 to be bent	 	\\ 
   \ipa{qrɤz}  &	 to shave 	&		\ipa{ɴɢrɤz}  &	 	to be shaved 	\\ 
   \ipa{cʰɤβ}  &	 to flatten, to crush 	&		\ipa{ɲɟɤβ}  &	to be crushed, flattened 	 	\\ 
   \ipa{cɯ}  &	 to open 	&		\ipa{ɲɟɯ}  &	 to be opened	 	\\ 
 \bottomrule
\end{tabular}
\end{table}



The prenasalization changes undergone by the \textsc{main consonants} of the onset are the following:
\begin{table}[H] \centering
\caption{Prenasalization} 
\begin{tabular}{lllllllll} \toprule 
basic    &  derived    &\\
\midrule
p(ʰ) & > mb\\
t(ʰ) & > nd\\
tʂ(ʰ) & > ndʐ\\
tɕ(ʰ) & > ndʑ\\
c(ʰ) & > ɲɟ\\
k(ʰ) & > ŋg\\
q(ʰ) & > ɴɢ \\
 \bottomrule
\end{tabular}
\end{table}

The rule of prenasalization changes both unaspirated stop and aspirated unvoiced stops and affricates into the corresponding voiced prenasalized one. The fact that the aspiration contrast is neutralized during the process constitutes additional evidence for the direction transitive > intransitive: the form of the intransitive verb is generally predictable from the transitive one, but that of the transitive verb is not predictable from the intransitive one. 

The \textsc{medial} consonants are not affected by the prenasalization process, but the \textsc{pre-onset} consonants are sometimes lost (in \ipa{xtʰom} > \ipa{ndom} and \ipa{ftʂi} > \ipa{ndʐi}). The only case of preserved \textsc{pre-onset} is that of the Tibetan loanword \ipa{χtɤr} > \ipa{ʁndɤr} mentioned above.

The only stops/affricates that are not attested among the anticausative pairs are the /kʰ/, /tʂʰ/, /tɕʰ/, /ts/ and /tsʰ/, but their absence is fortuitous, a consequence of the limited productivity of this derivation. On the other, the absence of fricatives and voiced segments results from a phonological constraint on the application of the prenasalization rule. While a set of rule converting unvoiced fricatives to prenasalized affricates of the same place of articulation (/s/ > /ndz/, /ɕ/ > /ndʑ/ etc) is conceivable, no such examples are found in Japhug. 

The only example that could remind of an anticausative derivation with a fricative consonant is \ipa{sɤndu} ``to exchange'' > \ipa{antsɤndu} ``to be exchanged by mistake'', but this verb is best explained as a lexicalized spontaneous-autobenefactive (cf \ref{sub:autoben}).


Prenasalization cannot occur with a polysyllabic verb stem: only the verb root can undergo this process. Therefore, no denominal or deideophonic verb can have an anticausative form, as these verbs are always prefixed.

The anticausative verbs have in some cases irregular nominal forms with a prefix \ipa{ɣ}-- historically related to the participle \ipa{kɯ}-- prefix. This includes \ipa{ɣndʑɤβ} ``devastating fire'' (from \ipa{ndʑɤβ} ``to burn'')\footnote{Notice the probable Tibetan cognate \ipa{gzhob} ``burning smell'' < *gV-ndʑop} and the second element of \ipa{ɯ-ɣɲɟɯ} ``opening'' (from \ipa{ɲɟɯ} ``to be opened''). (cf \ref{sec:fossil.participle}).

\subsubsection{Syntax}  \label{subsub:anticaus.syntax}
Like the passive, the anticausative derivation changes a transitive verb into an intransitive one, whose S corresponds to the O of the original verb. Anticausative derivation is mainly limited to verbs of handling and concrete action. However, the resulting verbs have distinct syntactic properties. Some, such as \ipa{ndom} ``horizontal'' are stative verbs. Others, such as \ipa{mbrɤt} ``to be cut'' or \ipa{ndʐi} ``to melt (vi)'' are dynamic verbs with non-human S. Their infinitive is in \ipa{kɯ}-- (see \ref{sub:infinitive}), as illustrated by the following example with an infinitive used as converb:

 \begin{exe}
\ex
\gll \ipa{ndʑi-kha} 	\ipa{ɣɯ} 	\ipa{ndʑi-rɟɯ} 	\ipa{mɤ-kɯ-mbrɤt} 	\ipa{tu-ɣɤji} 	\ipa{pjɤ-ŋu} \\
1\du{}.\poss{}-house \gen{}  1\du{}.\poss{}-fortune \negat{}-\inftv{}.\textsc{n.hum}-be.cut \ipf{}-increase[III] \evd.\ipf{}-be   \\
 \glt  He was constantly increasing  their house's fortune.  (The Raven, 12)
\end{exe} 

  Finally, some such as \ipa{ndʐaβ} ``to roll'' or \ipa{mbɣaʁ} ``to turn over (vi)'' are regular dynamic intransitive verbs.

Unlike passive verbs, the anticausative verbs can be used with SAP:

 \begin{exe}
\ex
\gll \ipa{nɤʑo} 	\ipa{jɤ-ɕe,} 	\ipa{tɕiʑo} 	\ipa{ni} 	\ipa{nɯ-ɴɢɤt-tɕi} 	\ipa{ma} 	\ipa{mɤ-jɤɣ}   \\
you \imp{}-go we_{di} \du{} \npst{}:\auto{}-be.separated-1\du{} apart.from \negat{}-\npst{}:be.possible \\
 \glt  Go, we_{di} have no choice but to separate.  (The Raven, 38)
\end{exe} 


\subsubsection{Compatibility}
 The anticausative is not compatible with either the  \ipa{sɯ}-- causative, the \ipa{ɣɤ}-- causative or the applicative. It is restricted to derivations which do not increase the valency of the verb: autobenefactive-spontaneous, deexperiencer and atelic.

The atelic derivation appears for instance with \ipa{mbɣaʁ} ``to roll (vi)'' > \ipa{nɤmbɣaʁlaʁ} ``to roll about, wallow about''. A example of anticausative with deexperiencer (\ref{subsub:anticaus.semantics}) is \ipa{ŋgio} ``to slip'' > \ipa{sɤŋgio} ``to be slippery''.
 
\subsubsection{Semantics}  \label{subsub:anticaus.semantics}


The  passive and the prenasalized anticausative are semantically quite different.  When the  passive form is used,  the agent is omitted  but semantically, the existence of an indefinite external agent is not excluded. In the case of prenasalized anticausative, no agent is present, and the action is viewed as taking place spontaneously. 

 \begin{exe}
\ex
\gll  \ipa{wo} 	\ipa{a-ʑi} 	\ipa{ra} 	\ipa{nɯ-mkɤɣɯr} 	\ipa{pɯ-mbrɤt} 	\ipa{ti-nɯ}   \\
      oh 1\sg{}.\poss{}-lady \pl{} 3\pl{}.\poss{}-necklace  \aor{}-\acaus{}:break \npst{}:say-\pl{}\\
 \glt  "Oh, my lady, your necklace broke!" they say.  (Kunbzang 214)
\end{exe} 
In this example, the character pronouncing this sentence believes that the necklace broke by itself,\footnote{This detail is actually relevant to the plot of the story, as the lady broke it herself on purpose, but wanted her servants to believe it broke spontaneously.} without an external agent. Therefore, he uses an anticausative form. If the passive \ipa{a-prɤt} had been used instead, it could have implied that someone had broken the necklace on purpose.

Another major difference with the passive is the fact that most anticausative are dynamic verbs, and do not normally have a resultative meaning when used with aorist or evidential:
\begin{exe}
\ex
\gll  \ipa{popo} 	\ipa{pjɤ-nɯ-ɕlɯɣ} 	\ipa{tɕe,} 	\ipa{pjɤ-ɴɢrɯ}    \\
       earthen.ware \evd{}-\auto{}-drop \coord{} \evd{}-\acaus{}:break \\
 \glt  She dropped the earthen ware and it broke. (Gesar 328)
\end{exe} 
In this example, \ipa{pjɤ-ɴɢrɯ} clearly can  not be understood as ``it was broken'' or ``it had been broken''.


The anticausative cannot be applied to all transitive verbs: only verbs expressing a concrete action such as moving or breaking an object can undergo this derivation. No verb of perception, speech  and even verbs of handling and using an object has an anticausative. For instance, there is no anticausative counterpart to \ipa{fkur} ``to bear (on the back)'' or \ipa{tsɯm} ``to take away'', though phonologically *ndur or *ndzɯm would have been possible. The derivation is only possible when the resulting verb occurs without an external agent: with a verb of handling such as ``to bear'' or ``to take'', the presence of a volitive agent is always implied.\footnote{Unless one is taken away by a natural force such as water. A form *dzɯm could have conceivably existed in the meaning ``to be (spontaneously) taken away'', but it is not attested.} 


\subsection{Deexperiencer} \label{sub:deexperiencer}

The deexperiencer prefix \ipa{sɤ}-- derives an intransitive stative verb out of an intransitive verb or a transitive perception verb. It is homophonous with the antipassive prefix \ipa{sɤ}-- studied in section \ref{sub:antipassive}.

The S of the derived verb denotes the stimulus of the state or action, and it has the meaning ``to be such that other people or things (meaning of the basic verb)''. 
 
In the case of intransitive verbs, the original S argument is suppressed and replaced by the stimulus. For transitive perception verbs, the A (corresponding to the experiencer) is suppressed and the original O (which is also the stimulus) becomes the S of the derived verb. The following examples illustrate this derivation:


\begin{table}[H]
\caption{Examples of the deexperiencer prefix \ipa{sɤ}-- }\label{tab:deexperiencer}
\begin{tabular}{lllllllll} \toprule
 
Basic verb	&meaning	&Derived verb	&meaning\\
\midrule
\ipa{ŋgio}  &	to slip 	& \ipa{sɤ-ŋgio}  &	  to be slippery (of the ground)\\
\ipa{scit}  &	  to be happy	& \ipa{sɤ-scit}  &	  to be nice (of a situation), \\ &&& be funny (of a person)\\
\ipa{ɕke}  &	   to be burned 	& \ipa{sɤ-ɕke}  &	  to be burning \\
\ipa{rga}  &	  to like (itr.)	& \ipa{sɤ-rga}  &	  to be nice\\
\ipa{nɤz}  &	  to dare (itr.)	& \ipa{sɤ-nɤz}  &	  to be such that one does not dare to do \\
\midrule
\ipa{mto}  &	 to  see	& \ipa{sɤ-mto}  &	  to be easy to see\\
\ipa{mtshɤm}  &	  to hear	& \ipa{sɤ-mtshɤm}  &	  to be easy to hear\\
\bottomrule
\end{tabular}
\end{table}

Since the deleted argument is always the experiencer (whether the verb is transitive or intransitive), we call this prefix “deexperiencer”.  Although in the case of intransitive verbs there is no decrease in valency, the addition of this prefix is nonetheless a demotion in the sense that a stimulus is lower than an experiencer in terms of agentivity (for instance, humans are less likely to be stimuli). 

For instance, the intransitive verb \ipa{ɕke} means ``to be burned'' (of a person or a body part):
\begin{exe}
\ex
\gll    \ipa{a-jaʁ} 	\ipa{pɯ-ɕke} 	   \\
1\sg{}.\poss{} \aor{}-burn      \\
 \glt My hand was burned.   (el., Chen Zhen)
\end{exe}  

The derived verb \ipa{sɤ-ɕke} means ``to be burning", in sense of being a source of heat (and causing burns to entities touching it):

\begin{exe}
\ex
\gll \ipa{qambɯt} 	\ipa{nɯ} 	\ipa{tɤ-sɤ-ɕke} 	\ipa{ʑo} 	\ipa{tɕe,} 	\ipa{tɕe} 	\ipa{tɤɕi} 	\ipa{chɯ́-wɣ-lɤt}    \\
      sand \topic{} \aor{}-\deexp{}-burn \emphat{} \coord{} \coord{} barley \ipf{}-\inv{}-throw   \\
 \glt When the sand becomes burning, one puts the barley.  (Rtsampa 114)
\end{exe}  

The modal verbs \ipa{cʰa} ``can'' and \ipa{nɤz} ``to dare'' (which can have a human S) present a peculiar case. One can derive from them \ipa{sɤ-cʰa} ``be possible'' and  \ipa{sɤ-nɤz} ``be such that one does not dare'', which can only be used with an infinitive verb or a verb in the generic form:

\begin{exe}
\ex
\gll \ipa{nɯ} 	\ipa{ra} 	\ipa{ɯ-tsʰɯɣa} 	\ipa{ra} 	\ipa{kɯ-mkhɤz} 	\ipa{tsa} 	\ipa{ɯ-kɯ-spa} 	\ipa{tsa} 	\ipa{pɯ-pɯ-kɯ-maʁ} 	\ipa{nɤ,} 	\ipa{kɤ-pʰɣo} 	\ipa{ra} 	\ipa{mɯ-pjɤ-sɤ-cʰa} 	\ipa{kʰi,} \\
\dem{} \pl{} 3\sg{}.\poss{}-method \pl{} \nmlz{}:\stat{}-expert a.little 3\sg{}-\nmlz{}:A-be.able a.little \textsc{if}-\pst{}.\ipf{}-\genr{}:S/O-not.be \coord{} \inftv{}-flee \pl{} \negat{}-\evd{}.\ipf{}-\deexp{}-can \textsc{hearsay} \\
\glt If one was not an expert in the method (how to escape the yeti), it was impossible to flee.  (Yeti2, 19)
\end{exe}  

\begin{exe}
\ex
\gll \ipa{ki}   	\ipa{kʰɯna}   	\ipa{kɤ-ɤtɯɣ}   	\ipa{mɯ́j-sɤ-nɤz}   	\ipa{ma}   	\ipa{ku-kɯ-mtsɯɣ}   	\ipa{ɲɯ-ɕti}    \\
this dog  \textsc{inf}-meet \textsc{neg:const}-\deexp{}-meet because \ipf{}-\genr{}-bite \const{}-be\textsc{.emph} \\
\glt One does not dare to meet this dog, as it bites people. (Aesop story 52)
\end{exe}  

This prefix should not be confused with the  agent-preserving antipassive \ipa{sɤ}-- in derivations such as \ipa{mtsɯɣ} ``bite'' > \ipa{sɤ-mtsɯɣ} ``bite people, be biting''. In some rare cases (such as \ipa{sɤ-nɯ-rga} studied in \ref{subsub:deexp:compat}), genuine ambiguity between the two prefixes exist. 

The resulting verbs, being stative, are compatible with the past imperfective:
\begin{exe}
\ex
\gll    \ipa{atu} 	\ipa{tɤʁaʁ} 	\ipa{ɯ-pɯ́-sɤ-scit?}   \\
up.there  gathering \qu{}-\pst{}.\ipf{}-\deexp{}-happy        \\
 \glt Was it nice, the gathering up there?    (conversation, tarrdo 74)
\end{exe}  


\subsubsection{Deexperiencer / Tropative pairs} \label{subsub:deexp.pairs}
Aside from the deexperiencer verbs derived from simple verbs, we also find several pairs of verb sharing the same root,  including one intransitive stative verb prefixed in \ipa{sɤ}--, and a transitive verb prefixed in \ipa{nɤ}--. The S of the intransitive counterpart and the O of the transitive one correspond  to the stimulus, as in regular deexperiencer and tropative verbs, and the A of the transitive verb corresponds to the experiencer.


The pairs of  verbs that lack a base verb, so that they cannot be considered to be normal deexperiencer or tropative verbs, but they  are related to a noun in most cases. Four pairs have been discovered:
\begin{table}[H]
\caption{Deexperiencer / Tropative pairs } 
\begin{tabular}{lllllllll} \toprule
 
Basic verb	&meaning	&Derived verb	&meaning &Noun	&meaning\\
\midrule
\ipa{sɤ-re}  &	to be ridiculous 	& \ipa{nɤ-re}  &	  to laugh &\ipa{tɤ-re}  &	  laughter\\
\ipa{sɤ-mtsʰɤr}  &	to be strange 	& \ipa{nɤ-mtsʰɤr}  &	  to consider  &\ipa{tɤ-mtsʰɤr}  &	 strange\\

&&& to be strange && event \\
\ipa{sɤ-ŋɤβ}  &	to be such that   	& \ipa{nɤ-ŋɤβ}  &	  to feel sorry  & \\
& people do not want   \\
\ipa{sɤ-ʑɯloʁ}  &	to be disgusting 	& \ipa{nɤ-ʑɯloʁ}  &	  to feel nausea   & \ipa{ɯ-ʑi} \ipa{loʁ} & feel nausea \\ 
\bottomrule
\end{tabular}
\end{table}
The first two verb pairs \ipa{sɤre} / \ipa{nɤre} and  \ipa{sɤmtsʰɤr} / \ipa{nɤmtsʰɤr} are similar, related to base nouns prefixed in \ipa{tɤ}--. Note that the prefix in these nouns is the indefinite possessive, which disappears when a possessive prefix is added:
 \begin{exe}
\ex
\gll  \ipa{ɯ-re} 	\ipa{ci} 	\ipa{ɕmɯɣ} 	\ipa{ɲɤ-ɕlɯɣ,}   \\
 3\sg{}.\poss{}-laughter a.little   \textsc{ideo}:1:laughter \evd{}-drop \\
 \glt She (could not resist and) laughed a little.  (The Frog, 100)
\end{exe}   

For the pair \ipa{sɤŋɤβ} / \ipa{nɤŋɤβ} on the other hand, no corresponding noun seem to exist, though it could have existed or be attested in other varities of Japhug.

\ipa{sɤʑɯloʁ} and \ipa{nɤʑɯloʁ} are related to the idiom meaning ``have nausea'':
 \begin{exe}
\ex
\gll  \ipa{ɯʑo} \ipa{ɯ-ʑi} \ipa{ɲɯ-loʁ}  \\
 he 3\sg{}.\poss{}-nausea \const{}-have.nausea \\
 \glt He has nausea. (elicitation, Dpalcan)
\end{exe}  
Neither  the noun \ipa{tɯ-ʑi} nor the verb \ipa{loʁ} can appear on their own, though they behave morphologically like independent words. This pair of verb is remarkable, as they constitute examples of incorporation (cf \ref{sub:incorporation}).

  
The nature of the derivation in the case of these pairs is unclear. It seems possible that the pairs of verbs above are derived from the nouns, and thus that the deexperiencer and tropative prefixes can be used as denominal markers alongside their regular use. However, from a comparative point of view, the root of \ipa{tɤ-re} is clearly verbal in origin (see for instance Tangut \tgf{4335} \ipa{rjijr²}, proto-LB *ray¹ #668, both ``to laugh''). Besides, \ipa{tɤ-mtsʰɤr} is a loanword from the Tibetan \ipa{mtsʰar.ba} ``to be fabulous, to be strange''. This suggests that \ipa{tɤ-re}  and \ipa{tɤ-mtsʰɤr} in their turn are deverbal nouns, so that unattested verbal roots *re and *mtsʰɤr must have existed. 
	
	The following scenario can be postulated to explain these forms. If we assume the existence of a root *are ``to laugh'' (intransitive) with a intransitive thematic element \ipa{a}--, \ipa{sɤ-re} would be its regular deexperiencer, and \ipa{tɤ-re} is a possible deverbal noun (type \ipa{aɕqʰe} ``to cough'' > \ipa{tɤ-ɕqʰe} ``cough (n)''), and \ipa{nɯ-ɤre} > \ipa{nɤre} ```laugh at'' its applicative (not tropative). This hypothetical form *are in turn would be derived from a root *re ``to laugh (tr)'', in the same way as \ipa{akʰu} ``to call'' must originate from a transitive root *kʰu ``to call (transitive)''.
	
	The disappearance of *are (and the other comparable forms) makes this formation quite opaque. For \ipa{sɤʑɯloʁ}, we need to suppose a different path, first incorporation of the noun \ipa{ɯ-ʑi} in a hypothetical form *a-ʑɯ-loʁ ``to have nausea (it)'', then deexperiencer and applicative derivations.


\subsubsection{Compatibility} \label{subsub:deexp:compat}
The deexperiencer prefix is compatible with several derivations, but examples are isolated. 


It co-occurs with the applicative \ipa{nɯ}-- in the example \ipa{sɤ-nɯ-rga}  ``to be likeable'', though in this case the semantic difference with the simple deexperiencer \ipa{sɤ-rga} is unclear (only the latter has been found in stories and spontaneous conversations).

The deexperiencer can apply to anticausative verbs such as \ipa{ŋgio} ``to slip'' > \ipa{sɤ-ŋgio} ``to be slippery''. \ipa{ŋgio} itself is the anticausative of \ipa{kio} ``to cause to slip''.

The tropative can be added to a deexperiencer verb (cf \ref{subsub:trop.compat}) such as \ipa{nɤ-sɤ-scit} ``to consider to be nice'' from \ipa{sɤ-scit} ``to be nice''.

Finally, we find one example of a combination of reflexive with deexperiencer: \ipa{ʑɣɤ-sɤ-ɕke} ``to burn oneself'' from \ipa{sɤ-ɕke} ``to be burning'' (cf \ref{subsub:reflexive.compat}).



\subsection{Reflexive} \label{sub:reflexive}
Japhug, as other Rgyalrong languages, has a specific prefix to express reflexive distinct from both the anticausative and the reciprocal. This reflexive prefix  \ipa{ʑɣɤ}-- has no equivalent in other Sino-Tibetan languages. Its history is studied in \citet{jacques10refl}, where  \ipa{ʑɣɤ}-- is argued to originate from an incorporated  third person pronoun.


The reflexive prefix \ipa{ʑɣɤ}-- is the first derivational prefix of the verbal template. It is highly productive, and can be added to most transitive verb that accept an animate as their O. The following table illustrates some examples:

\begin{table}[H]
\caption{Examples of the reflexive prefix \ipa{ʑɣɤ}-- }\label{tab:refl}
\begin{tabular}{lllllllll} \toprule
 
Basic verb	&meaning	&Derived verb	&meaning\\
\midrule
\ipa{mto}  &	  to see	& \ipa{ʑɣɤ-mto}  &	  to see oneself \\ 
\ipa{rku}  &	to put in	& \ipa{ʑɣɤ-rku}  &	  to put oneself in \\
\ipa{rɤβraʁ}  &	  to scratch	& \ipa{ʑɣɤ-rɤβraʁ}  &	  to scratch oneself \\  
\bottomrule
\end{tabular}
\end{table}
The prefix \ipa{ʑɣɤ}-- has no special allomorphs, and displays no irregular forms. Semantically, we observe to types of reflexive verbs. First, verbs whose S corresponds to both the A and the O of the transitive base verb, as in 
   \begin{exe}
\ex
\gll \ipa{nɯnɯ} 	\ipa{ɯ-ŋgɯ} 	\ipa{nɯ} 	\ipa{tɕu} 	\ipa{ɯʑo} 	\ipa{to-ʑɣɤ-rku}    \\
that 3\sg{}-inside \topic{} \loc{} he \evd{}-\refl{}-put.in    \\
  \glt He put himself inside it. (The Flood, 17)
   \end{exe}
 


Second, verbs whose S only corresponds to the O of the original verb, implying the existence of a distinct agent. For instance, the verb  \ipa{nɯsmɤn} ``to treat (a patient)''\footnote{A denominal verb from \ipa{smɤn} ``medecine''.} has the reflexive form\ipa{ʑɣɤ-nɯ-smɤn}, which does not mean ``to treat oneself'', but ``to have oneself treated (by a doctor)'', as in the following example:
   \begin{exe}
\ex
\gll  \ipa{nɤʑo} 	\ipa{ɕɯ-tɯ-ʑɣɤ-nɯsmɤn} 	\ipa{ɯ-ŋu?}   \\
you \transloc{}-2-\refl{}-treat \qu{}-\npst{}:be     \\
  \glt Are you going to see a doctor? (conversation, Chen Zhen, \wav{8_ZGAnWsmAn}
   \end{exe}
In other words, this verb behave semantically as if it were a combination of reflexive and causative, but no overt causative marker is present.

Moreover, a few reciprocal verbs has slightly unpredictable meanings. For instance, \ipa{ʑɣɤ-ɕtʰɯz}, the reflexive of \ipa{ɕtʰɯz} 	``to turn towards'' means   ``to reveal oneself'': 

   \begin{exe}
\ex
\gll  \ipa{a-tɕɯ}	\ipa{kɤ-ʑɣɤ-ɕtʰɯz}	\ipa{ma}	\ipa{dɯɣpa}  \\
    1\sg{}.\poss{}-son \imp{}-\refl{}-turn.toward because pitiful\\
  \glt My son, reveal (your real identity) to her, poor of her!  (Gesar 176)
   \end{exe}
Another such example is \ipa{ʑɣɤ-pa} ``to pretend, to be arrogant'' from the root \ipa{pa} ``to do'' (in Japhug restricted mainly to the meaning ``to open''). This verb requires to be used with a A/S participle:
   \begin{exe}
\ex
\gll  \ipa{ɯʑo} 	\ipa{kɯ-ngo} 	\ipa{to-ʑɣɤpa,} \\
     she \nmlz{}:S/A-sick \evd{}-pretend \\
  \glt She pretended to be sick. (Nyimawodzer1, 15)
   \end{exe}

In the case of transitive verbs with prenasalized anticausative forms, the reflexive form of the transitive verb is semantically almost identical to the anticausative. However, given the fact that most anticausatives verbs with do not normally allow a volitive human S, this property is only visible with \ipa{ʑɣɤ-pɣaʁ} ``to turn oneself over'' and \ipa{mbɣaʁ} ``to roll, to turn over (it)'' for \ipa{pɣaʁ} ``turn over (tr)'':

   \begin{exe}
\ex
\begin{xlist}[(ii)]
\exi{(i)} 
\gll  \ipa{aʑo} 	\ipa{stomku} 	\ipa{ri} 	\ipa{pɯ-rŋgɯ-a} 	\ipa{tɕe,} 	\ipa{tʰɯ-ʑɣɤ-pɣaʁ-a} 	\ipa{ma} 	\ipa{a-pa} 	\ipa{qajɯ} 	\ipa{ɣɤʑu}  \\
I lawn \loc{} \pst{}.\ipf{}-lay.down \coord{} \aor{}-\refl{}-turn.over-1\sg{} because 1\sg{}-under worm be.there.\mir{} \\
\exi{(ii)} 
\gll  \ipa{tʰɯ-mbɣaʁ-a} 	\ipa{ma} 	\ipa{apa} 	\ipa{qajɯ} 	\ipa{ɣɤʑu}  \\
 \aor{}-\aturn.over-1\sg{} because 1\sg{}-under worm be.there.\mir{} \\
  \glt  I was laying on the lawn, but I turn myself over, as there was a worm under me. (elicitation, Chen Zhen \wav{ZGApGaR})
  \end{xlist}
   \end{exe}
   
The reflexive cannot be used with transitive verbs whose O cannot refer to animate being. It is in particular impossible to used it with verbs whose O can refer to a body part but not a human. This is the case of \ipa{qrɤz} ``to shave'' for instance. One can say the following sentence with the autobenefactive-spontaneous:
       \begin{exe}
\ex
\gll \ipa{a-mtɕhi-rme} \ipa{pɯ-nɯ-qraz-a}\\
1\sg{}.\poss{}-mouth-hair \aor{}-\auto{}-shave-1\sg{}  \\
  \glt I shaved my beard (elicitation, Chen Zhen)
   \end{exe}
 On the other hand, \ipa{qrɤz} cannot receive the \ipa{ʑɣɤ}-- prefix. The sentence above is the only way to express the meaning of English ``I shaved myself".
 

\subsubsection{Compatibility}   \label{subsub:reflexive.compat}
The reflexive \ipa{ʑɣɤ}-- is commonly used with the causative \ipa{sɯ}--. The relevant examples are studied in \ref{subsub:caus1.compatibility}. 

Though examples are fewer, \ipa{ʑɣɤ}-- is also compatible with all valency-increasing prefixes: \ipa{ɣɤ}-- causative (cf. \ref{subsub:caus2:compat}), applicative and tropative.

Only one example with the applicative is known: \ipa{ʑɣɤ-nɤ-stu} ``to believe in oneself'', derived from \ipa{nɤ-stu} ``to believe in (someone)'', itself from \ipa{stu} ``to believe (intr)''.
   \begin{exe}
\ex
\gll \ipa{ɯ-zda} 	\ipa{mɯ́j-nɤ-ste} 	\ipa{ma} 	\ipa{ɯʑo} 	\ipa{ɲɯ-ʑɣɤ-nɤ-stu}   \\
3\sg{}.\poss{}-companion \negat{}.\const{}-\appl{}-believe[III] \textsc{adversative} he \const{}-\refl{}-\appl{}-believe      \\
  \glt He doesn't believe his companions, he only believes himself. (elicitation, Chen Zhen \wav{8_ZGAnAstu}) 
   \end{exe}
   
Other applicative verbs cannot be used with the reflexive; there are semantic reasons for this incompatibility, especially in the case of \ipa{nɯɣ-bɯɣ} ``to miss (someone)'': ``to miss oneself'' is hardly interpretable even figuratively. One would have expected that applicative verbs such as \ipa{nɯ-rga} ``to like (someone)'' or \ipa{nɯɣ-mu} ``to be afraid of (someone)'' could be derived with the reflexive, but native speakers reject these forms.

With the tropative derivation, the only clear example of reflexive that we have discovered is \ipa{ʑɣɤ-nɤ-mpɕɤr} ``to consider oneself beautiful'', derived from \ipa{nɤ-mpɕɤr} ``to consider beautiful'' and \ipa{mpɕɤr} ``to be beautiful''.



 ʑɣɤnɯsmɤn

derived from a stative verb ?
ʑɣɤnɤstu 





ʑɣɤɕɯrga

ɣɤtɕa > ʑɣɤɣɤtɕa
from intr
sɤɕke > ʑɣɤsɤɕke (mapɯtɯʑɣɤsɤɕke)



ʑɣɤpa
ʑɣɤχtɕi

jieshang karia tɕe, nɯʑɣɤsɤzɣɯta
\wav{ZGAsAzGWt}

 


\subsection{Reflexive-tropative} \label{sub:refl.trop}


   \begin{exe}
\ex
 \glt   \ipa{ɕqraʁ} ``be intelligent'' > \ipa{znɤ-ɕqraʁ-ɕqraʁ} ``consider onself intelligent''.
\end{exe} 
 
   \begin{exe}
\ex
\gll \ipa{tɯ-znɤɕqraʁɕqraʁ} 	\ipa{nɯɣe,} 	\ipa{kɯmtɕhɯ} 	\ipa{ra} 	\ipa{mɯ́j-ra} 	\ipa{ɣe}  \\
 2-\npst{}:\trop{}:\refl{}:intelligent isn't.it toy \pl{} \negat{}:\const{}-have.to isn't.it \\
  \glt   You think you are so smart, you don't need toys, don't you? (Conversation 2003, 68)
   \end{exe}
自以为

znɤjpɯjpe
znɤɕqɯɕqraʁ
znɤmpɕɯmpɕɤr
znɤchacha



\subsection{Relic prefixes} \label{sub:relic.val.decrease}

passive mɯ-, as in mɯ-rmbɯ
zɣɯt
\section{Modal derivational prefixes} \label{sec:derivation.modal}
\subsection{Atelic}  \label{sub:atelic}
\jg{nɤpɣaʁlaʁ}
\jg{nɤrɟɯrɟɯɣ}

applies after anticausative nɤmbɣaʁlaʁ
\subsection{Vertitive} \label{sub:vertitive}
\jg{nɯɕe}
\subsection{Spontaneous-Autobenefactive} \label{sub:autoben}

nɤʑo tɤtɯnɯβzut ɕti
The only example that could remind of an anticausative derivation with a fricative consonant is \ipa{sɤndu} ``exchange'' > \ipa{antsɤndu} ``exchanged by mistake'', a verb that can appear by itself or with the causative:

   \begin{exe}
\ex
\gll \ipa{tɕi-ŋga} \ipa{to-nɯ-sɯ-ɤntsɤndu-tɕi}  \\
   1\du{}.\poss{}-clothes \evd{}-\auto{}-\caus{}-exchanged.by.mistake-1\du{} \\
  \glt We exchanged our clothes by mistake.   (el., Dpalcan)
   \end{exe}
The change s > nts is however of a different nature from the prenasalization, and the presence of the intransitive thematic element \ipa{a}-- seems to indicate that this verb form is  

%A	25	ɕɤr	tɯ-tshot	sqamnɯs	ʑo	nɯtɕu a-jɤ-tɯ-z-nɯ-ɤtɯɣ	tɕe,
%	25	必须要赶到晚上12点正,
\begin{exe}
\ex
\gll  \ipa{alo} 	\ipa{iʑo} 	\ipa{i-kha} 	\ipa{nɤ-ku} 	\ipa{ko-tɯ-\textbf{nɯ}-rpu-t} 	\ipa{khi} 	\ipa{laβtɕi}  \\
  upstream we 1\pl{}.\poss{}-house 2\sg{}.\poss{}-head \evd{}-2-\textbf{\auto{}}-bump-\pst{} \textsc{hearsay} isn't.it \\ 
 \glt   Over there in our house, you bumped your head, isn't it ? (conversation 2010, autobenefactive-spontaneous)
\end{exe} 
 nɤtɯɣ > spontané autobenefactif
 
 kɯm apɯnɯɲɟɯ je : laisse la porte ouverte
 vi-YJW2
 
 
 
"tɤmɯmɯm nɯ arɟɤntɕa kɯmpɕɤr ci ɲɯŋu tɕe, khɯna nɯ ɯŋgɯz tɕe aʑo kɯfse kɯmpɕɤr me" ɲɯnɯsɯsɤm pjɤŋu.
53 histoires. 
 ɯʑo sɤz pɯ-nɯ-wxti nɯ tu-ndze, pɯ-nɯ-xtɕi nɯ tu-ndze ɲɯ-ɕti  
 qartshaz
 
 skɤmndʐi kɯ-fse, mbro-ndʐi kɯ-fse qartshaz ndʐi nɯra tɕe, pjɯ-χtsɤβ-nɯ tɕe,
ɯ-rme chɯ-phɯt-nɯ, ɯ-ɕa chɯ-phɯt-nɯ ra ri, 
qachɣa ɣɯ nɯ-rme pjɯ-nɯ-ndzoʁ ɲɯ-ra
qarchGa 238
 
\subsection{Abilitative} \label{sub:abilitative}
The abilitative \ipa{sɯ-} is formally identical with the causative  \ipa{sɯ-}, including the presence of an allomorph \ipa{z-} when the verb stem starts with a prefix in sonorant. This agent-oriented modality expresses the ability of the agent to perform a task, and is never used to express possibility, which is why we avoid the more common term ``potential''.

Most examples of abilitatives in texts are found in the negative form:

\begin{exe}
\ex 
\gll \ipa{khɤdi} 	\ipa{ʁnɯz-mɯz} 	\ipa{nɯ-antsɤndu} 	\ipa{mɤ-kɯ-nɯ-sɯ-rtoʁ} \\
place.of.the.wife two-sort \aor{}-be.exchanged \negat{}-\nmlz{}:S-\auto{}-\abil{}-see \\
 \glt  Two persons were exchanged at the place of the wife in the kitchen, but he is not able to notice it.  (The frog, 155)
   \end{exe}
   
 \begin{exe}
\ex 
\gll  	\ipa{li} 	\ipa{mɯ-na-z-nɤɕqa} 	\ipa{nɤ,} 	\ipa{ɯ-qom} 	\ipa{pɯ-ɣe} 	\ipa{ɕti} 	\ipa{ɲɯ-ŋu,}  \\
again \negat{}-\aor{}:3>3-\abil{}-bear \coord{} 3\sg{}.\poss{}-tear \aor{}:down-come[II] \npst{}:be.\emphat{} \ipf{}-be \\
 \glt  Again, she could not help and dropped a tear. (Kunbzang, 367)
   \end{exe}
The abilitative is only compatible with transitive verbs, and does not change their valency or even the status of their A and O.  The causative and the abilitative are formally similar, and the two forms are  homophonous. For instance from the transitive \ipa{ndza} ``eat'' one can derive \ipa{sɯ-ndza} ``cause s.o. to eat'', ``eat with s.t.'' or ``be able to eat''. 


It is therefore legitimate to wonder whether the abilitative is really distinct from the causative as a verbal affix. The difference between the two  is mainly semantic, as the abilitative does not add a causer to the core arguments of the verb. An additional argument is the presence of the lexicalized abilitative \ipa{spa}, a transitive verb meaning ``be able to do'', etymologically derived from the verb \ipa{pa} ``do, close'', though the link between the two verbs is not visible any more synchronically. In the case of this verb, the expected form would be \ipa{sɯ-pa}, a verb which actually exists but with a different meaning (see the preceding chapter). 

\ipa{spa} can be either used with a nominal O, or with a complement verb in the infinitive:
 \begin{exe}
\ex 
\gll  	     \ipa{ɕoŋβzu} \ipa{ko-spa} \\
    carpenter  \evd{}-be.able  \\
 \glt  He learned to be a carpenter  
   \end{exe}
 \begin{exe}
\ex 
\gll  	 \ipa{maka}   \ipa{kɤ-rɯɕmi} \ipa{mɯ́j-spe} \\
   not.at.all \inftv{}-speak \negat{}:\const{}-be.able[III] \\
 \glt She cannot speak at all. (Divination2005, 23) 
   \end{exe}


The fossilization of the abilitative prefix with the verb \ipa{spa} is likely to be ancient, as outside of the Rgyalrong languages, we find a cognate pair of verbs in Tangut: 
\tgz{5113} < Pre-Tangut *C-pja ``to do'' vs. \tgz{0385} < Pre-Tangut *C-S-pja ``to be able''. 
 
Another verbal form with a synchronically non-analyzable abilitative is the transitive verb \ipa{jqu} ``be able to lift'', whose base verb is not attested in Japhug, though it is unclear whether this verb is related to the abilitative derivation observed here.

sphɯt
 
\subsection{Facilitative} \label{sub:facilitative}
A third category of modal derivational prefixes in Japhug are the facilitative prefixes \ipa{ɣɤ-} and \ipa{nɯɣɯ-}. The former is homophonous with the causative of stative verbs \ipa{ɣɤ-} (see section \ref{sub:caus2}), while the latter is one of the rare bisyllabic prefixes in the language.

\ipa{ɣɤ-} derives an intransitive stative verb   into another stative verb, with the added meaning ``become X easily''. Only one counterexample, with a transitive experiencer verb is known: \ipa{ɣɤɕɯftaʁ} ``to have a good memory'' from \ipa{ɕɯftaʁ} ``memorise, remember''. Interestingly, the meaning here is not ``to be easy to remember''.
\begin{exe}
\ex 
\gll   \ipa{ɲɯ-ɕqraʁ} 	\ipa{tɕe} 	\ipa{ɲɯ-ɣɤ-ɕɯftaʁ} 	   \\
  \const{}-intelligent \coord{} \const{}-\facil{}-memorize \\
 \glt   He is intelligent, he has a good memory.
   \end{exe}
   XXXXXXXɣɤ-jmɯt
   
   
The following table illustrates several regular examples of the facilitative prefixes:

\begin{table}[H]
\caption{Examples of facilitative prefixes in Japhug}\label{tab:facilitative}
\begin{tabular}{lllllllll} \toprule
basic verb  & derived  verb &\\
\midrule
 \ipa{nɯndzɯlŋɯz} ``be sleepy'' &  \ipa{ɣɤ-nɯndzɯlŋɯz} ``get sleepy very easily'' \\
\ipa{mtsɯr} ``be hungry'' &  \ipa{ɣɤ-mtsɯr} ``get hungry very easily'' \\
\ipa{sɤmbrɯ} ``be angry'' &  \ipa{ɣɤ-sɤmbrɯ} ``get angry  easily'' \\
 \midrule
 \ipa{ŋga} ``wear (clothes)'' & \ipa{nɯɣɯ-ŋga} ``be easy to wear" \\
\ipa{ndza} ``eat'' &  \ipa{nɯɣɯ-ndza} ``be easy/nice to eat'' \\
\ipa{ntɕhoz} ``use'' & \ipa{nɯɣɯ-ntɕhoz} ``be easy to use'' \\
\bottomrule
\end{tabular}
\end{table}

 Unlike \ipa{ɣɤ-}, \ipa{nɯɣɯ-} is used with transitive action verbs. It changes the transitive verb into a stative intransitive verb whose S corresponds to the O of the original verb (thus a different situation from  \ipa{ɣɤ-ɕɯftaʁ}, which was agent-preserving). The following example illustrate the use of this prefix with \ipa{mto} ``see'' (which in Japhug is a transitive verb whose experiencer is encoded as the A):
 
 \begin{exe}
\ex 
\gll   \ipa{ɯ-loʁ} 	\ipa{nɯ} 	\ipa{tʂu} 	\ipa{ɯ-rkɯ} 	\ipa{kɯ-ɤrmbat} 	\ipa{ʑo} 	\ipa{sɯphɯ} 	\ipa{kɯ-xtɕi} 	\ipa{ɯ-qa,} 	\ipa{xɕɤj} 	\ipa{ɯ-qa} 	\ipa{ra} 	\ipa{ku-βze} 	\ipa{ŋu} 	\ipa{tɕe,} 	\ipa{nɯɣɯ-mto}  \\
3\sg{}.\poss{}-nest \topic{} road 3\sg{}.\poss{}-side \nmlz{}:\stat{}-near \emphat{} tree \nmlz{}:\stat{}-small 3\sg{}.\poss{}-foot grass 3\sg{}.\poss{}-foot \pl{} \ipf{}-do[III] \npst{}:be \coord{} \npst{}:\facil{}-see \\
 \glt   It builds its nest on the foot of bushes and small trees not far from the road, and it  is easy to see. (Story on Emberiza)
   \end{exe}
  

iʑo jiji pɯkɤzmɤku nɯ iʑo ɯphɯt kuzmɤkuj ma ɲɯɣɤmda 成熟得快


8_GAGAzbaR
ɲɯɣɤɣɤzbaʁ

nɯɣɯsɯɣɲaʁ
8_nWGWsWGYaR
\subsection{Intensive reduplication} \label{sub:intensive.redp}
iconic

\subsection{Relic prefixes}

ɕɯmthu ?

\section{Denominal and deideophonic} \label{sec:denominal}


tɤɕɤt > sɤɕɤt

\subsection{rɯ-}
rɯsɯso, not derived from sɯso, but from the noun tɯsɯso
\subsection{sɯ-}
sɯlaʁrdɤβ
sɯjaʁndzu
sɯndzɯpe

\subsection{Zero derivation} \label{sub:zero.derivation}


znde

ndzom

<> jpɣom, mkɯm

sɯjno
ɕpɯsɯjnoa

 
\section{Unclear derivations}


\subsubsection{Intransitive determiner}  \label{subsub:intransitive.det}
sɤʑa > kɤ-

ʑa > tɯ-


> rɤwum, rɤtshɤt etc, sans effet
\section{Compounds} \label{sec:verbal.composition}

\subsection{Compound verbs}

ngɤjtshi

ngu+jtshi
\subsection{Incorporation} \label{sub:incorporation}
Incorporated nouns include adjunct, O, but also S (as in \ipa{nɯ-mbrɤ-rɟɯɣ} ``to do a horse race''), a property that Rgyalrong shares with Athabaskan (\citet[215]{rice2000scope})
 
kɤtɯpa

kɤti pa
 aʑo tɤtɯta nɯ, ɯʑo kɯ nɤɕki kɤtɯpe \wav{kAtWpa}
\chapter{Relative clauses} \label{chapt:relative}


%8_relatives_nWsmAn

%akɯnɯsmɤn smɤnba nɯ lɤβzaŋ rmi
%???smɤnba tɤ́wɣnɯsmana nɯ lɤβzaŋ rmi
%???smɤnba kɯ akɯnɯsmɤn nɯ lɤβzaŋ rmi
%différence avec tɤpɤtso ci kɯ jaŋma ɯkɯnɯmbrɤpɯ ci jɤɣe > restrictive/nonrestrictive?
%smɤnba kɯ tɤ́wɣnɯsmana nɯ lɤβzaŋ rmi
%smɤnba ra nɯŋgɯ aʑo akɯnɯsmɤn nɯ lɤβzaŋ rmi
%akɤsqar smɤnba nɯ lɤβzaŋ rmi
%aʑo nɯsqara smɤnba nɯ lɤβzaŋ rmi
%aʑo smɤnba nɯsqara nɯnɯ lɤβzaŋ rmi
%

ndzɤpri nɯ koʁmɯz ɯɣɲɟɯ ŋgɯ tukɯnɯɬoʁ ci pjɤmtondʑi. 
 nɯ koʁmɯz ɯɣɲɟɯ ŋgɯ tukɯnɯɬoʁ ndzɤpri ci pjɤmtondʑi. 
Aesop adaptation, the two friends and the bear

tɕheme	kɯ-pe	mɤ-kɯ-pe	nɯ	ɣɯ	koŋla	ʑo	ɯ-<biaozhun>	ɲɯ-ŋu,
huadai.100


smɤnba kɯ tu-nɯsmɤn mɤ-kɯ-cha nɯ, nɯ tɕu ju-ɕe-nɯ tɕe ɕ-pjɯ-ʑɣɤɣɤla-nɯ
20-ldwgi 54
Those who cannot be healing by the doctors
\section{Restrictive}
Prenominal nominalized sentence (for A)

ndzɤpri ɯmto pɯkɯzmɤku tɯrme nɯ sɯku zɯ tonɯrʁɯrʁa
Aesop adaptation, the two friends and the bear
 si ɯtaʁ tɤkɯnɯrʁɯrʁa tɯrme nɯ pjɤnɯɬoʁ
 
 
 ɕnat	tu-kɯ-rɤɕi	ndʑu	nɯnɯ,	nɯkínɯ	ɕnat-ndʑu	rmi,
 Huadai 64
 
 iɕqha	nɯnɯ	ɯ-ɣɯ-jɤ-kɯ-qru	tɤ-tɕɯ	nɯnɯ
Three sisters 231

For P
chɤmdɤru	tɤ-kɤ-sɤzgɯr nɯ	ɲɤ-sɤstú-nɯ	qhe,	tɕe	to-mna
Gesar 	315
He put straight the straw that had been twisted, and (her son) healed.

tɕe ɯ-rʑaβ kɤ-kɤ-nɤsci nɯnɯ ɣɯ ɯ-tɕɯ lɤβzaŋ sɤs tɯ-xpa kɯ-wxti,
Blobzang 10

ta-ʁi	pɯ-kɤ-βde	nɯnɯ,	pɯ-kɯ-si	nɯnɯ	pɣɤtɕɯ	ci	to-sci,
Kunbzang 152

 tɕheme	kɯ-mpɕɤr	nɯ	kɯ-maʁ,	mɤ-kɯ-mpɕɤr	nɯ	kɯ-maʁ,	ɯ-ŋga	kɯ-pe	nɯ	kɯ-maʁ,	mɤ-kɯ-pe	nɯ	kɯ-maʁ	ci,	khri	ɯ-ʁɤri	zɯ	lo-kɤmdzɯ́-chɯ	tɕe,
Sras 31-2


kɯ-lɤɣ kɯ-ŋu kɯ-maʁ ʑo jo-ɣinɯ
无论
\subsection{Non-restrictive}
ɯʑo	nɯɕɯŋgɯ ɯ-nmaq		pɯ-kɯ-ŋu	tshɯraŋ	nɯ	pjɤ-mto
Raven 102

\section{Categories}
\subsection{Nominalized}
Internally headed
\subsection{Non-nominalized} \label{sub:relative.non.nmlz}
nɯ-sloχpɯn nɯ kɯ ɯʑo pɯ-xtɕi ɣɯ nɯ ra pjɤ-fɕɤt
histoire08.17

\subsection{Head nouns}
aʑo	staq-lu-pa	tɕe	kɤ-ɣɤrɤt	ɯ-spa	ɕti-a	nɤ́tɕi
Nyima vodzer 116

\section{S}
ʁlaŋsaŋtɕhin χsɯm ma mɯtɤkɯrʑaʁ nɯ
Gesar 82

mɤʑɯ	iɕqha	tɕe	pɣɤtɕɯ	kɤ-kɯ-nɯ-rɤloʁ	phoʁ-phoʁ	nɯ	ɣɯ	ɯ-loʁ	nɯ-ŋgɯ	nɯ	ra	ɯʑo	ɕ-tu-ndze,
Bzar 4
He goes to eat in their nest the birds that made a nice nest.

tɕheme	kɯ-kaβ	ju-kɯ-ɣi	ci	tu	tɕe,
Fox 64

ɯɣmbɤj zɯ tɕhi tukɯndɯ ci pɯtu ɲɯŋu
On the side, a ladder was present.
Solpdpon 55



\section{A}
tɤpɤtso ci kɯ jaŋma ɯkɯnɯmbrɤpɯ ci jɤɣe
pear-story.CZ 5 rel A

tɤfsaŋ kɯ ɯkɯsɯfsaŋ me, tɯci kɯ ɯkɯsɯχtɕi me


ɕu	kɯ	kɯ-mɯrkɯ	pɯ-ŋu	lo-kɤ-tɕɤt	pɯ-kɯ-cha	nɯ a-sci	rɟɤlpu	chɯ-ta-sɯ-ndo	ŋu
qachGa-4-5




\section{P}

aʑo stu akɤnɯrga tɯskɤt nɯ kɯrɯskɤt ŋu
aʑo tɯskɤt stu akɤnɯrga nɯ kɯrɯskɤt ŋu
\wav{gram-akAnWrga}



	lɤ-fsoʁ ɯ-jɯja nɯ pjɯ-ru tɕe ɯ-kɤ-nɯmbrɤpɯ nɯ khu pɯ-ɕti ɲɯ-ŋu,
Tiger 20



nɯŋa ɣɯ ɯkɤndza ɯspa tɯɣro ɯtaʁ nɯ tɕu konɯrŋgɯ. 
aesop46

ɯ-loʁ	ŋgɯ	nɯ	tɕu,	ɯ-ŋgɯm	thɯ-kɤ-nɯ-lɤt	pɯ-nɯ-ŋu, ɯ-pɯ	pɯ-nɯ-ŋu	nɯ	ra,	βʑar	nɯ	kɯ	ɕ-tu-ndze,
Bzar 14 无论是下的蛋

tɕheme	nɯ	kɯ	[qɤjɣi	kɤ-kɯ-ɕke]	nɯ	ra	kɯ-lɤk	nɯ	na-mbi,
Three sisters 70

tɕe	qapri	ɣɯ	ɯ-ku	nɤrwɯ	nɯ-kɤ-phɯt	nɯ	to-nɯ-ndo,
divination 118

ɯʑo kɯ ta-nɯrdoʁ tɤ-nɯ-ndó-t-a me
Sparrow and Mouse 55
 
\section{Possessed noun}
praʁkhaŋ zɯ tɤ-mu ci ɯ-ku to-kɯ-wɣrum zɯŋ-zɯŋ ci pjɤ-rɤʑi tɕe,

63 sras


a-kɯ zɯ qapri ci ɯ-kɤχcɤl ɯ-ʁrɯ kɯ-tu ci ɣɤʑu 
On the east, there was a snake who had a horn in the middle of his head.
divination 43


 iɕqha nɯ-me lʁɯba kɯ-ŋu ra ɣɯ nɯ-khɤru lɤ-nɯɬoʁ
 divination 55
 He when in the kitchen of those whose daughter was mute.
 
nɤ-mu	nɤ-wa	kɯ-tshos	tɯ-ŋu, aʑo	a-mu	kɯ-me	ŋu-a	tɕe,
Nyima vodzer 12
You are someone whose father and mother are all there, I am someone without a mother, 
 
 nɯ	ɕɯŋgɯ	pɯ-kɤ-ɣɤrɤt	nɯ	ra	ɣɯ	ɯ-ɲɯ́-ŋu
 Nyima vodzer 79
Where these (the bones) the ones who were thrown in (the lake) before ?
 
 [nɯmkɤqhu chɤmdɤru thɯkɤsɤtsa] ʁɟa ʑo pɯŋu ɲɯŋu 
 Slopdpon 59
All of them were such that drinking straws had been inserted in their neck.
 
 ci	nɯ	nɯ-rmi	pa	mɯ-tɤ-kɤ-tɕɤt	nɯ	kɯ-lɤɣ	to-ɕe	
 Gesar 136
\subsection{Benefactive}
kumpɣɤtɕɯ	kɯ	(ɯkɯr ɯŋgɯ) ɯ-smɤn	ɕ-pjɤ-lɤt	tu-kɤ-ti nɯ	ʁlaŋsaŋtɕhin	ɯ́-ŋu
Gesar 217
The one about whom it is said that a bird dropped a medicine into his mouth, it is Gesar?

\section{Time}
arɕo pɯ-ŋu ɯ-sŋi nɯ tɕu tɕe,

sras 31


\section{Place}
qaɕpa ɣɯ ɯzda ra tɯci kúwɣmɢla kɯra nɯtɕu pjɤrɤʑinɯ ɕti tɕe, aesop07

qala nɯ pɤjkhu mɯchosta ri, bɤlqhoʁ nɯnɯ juzɣɯtndʑi kɯra nɯtɕu jozɣɯt. aesop50

tɯci kɯ-tu, sakaβ kɯ-tu cɯz lo-rɤtʂhá-nɯ,
sras 52
Divination 2:
A	2	tɕendɤre tɤ-mu nɯ kɯ ti kɯ "a-tɕɯ, kɯɕɯŋgɯ tɕe to-ti kɯ tɕe  tɤ-ŋe sɤ-ɕqhlɯt pɕoʁ zɯ, rgɤtpu ci tu


 khɯtsa atʂha pɯkɤrku ɯsta nɯ tɕu atɯjno tɤrke
 在我原来喝过茶的那个碗给我装菜
\wav{8_Wsta}
\subsection{possessed}
tɯ-ci	rkɯ	ndzom	kɯ-tu	ɯ-rkɯ	chɯs	pjɤ-nɯná-ndʑi	ɲɯ-ŋu
120 Kubzang

sɤtɕha nɯ mi ʁɟa ʑo kɯ-ŋu chɯs nɯ-azɣɯt ɲɯ-ŋu
divination 7



\section{Degree}
tɤjpa kɯŋgɯrtsɤɣ kɯ-jaʁ ko-sɯ-lɤt

\section{Action}
kumpɣɤtɕɯ	kɯ	(ɯkɯr ɯŋgɯ) ɯ-smɤn	ɕ-pjɤ-lɤt	tu-kɤ-ti nɯ	ʁlaŋsaŋtɕhin	ɯ́-ŋu
Gesar 217
The one about whom it is said that a bird dropped a medicine into his mouth, it is Gesar?

tɤkɤtɯt nɯra kɤʑɣɤsɯɣtso ma tha nɤɕɯkɤrɤfɕɤt me
phone, 2011, Chen Zhen

tɕeri zlawi-ɕɤrɤβ kɯ tɕhos pɯ-asɯ-zgrɯβ ɯ-fɕɤt tu ma
 jɯm pɯ-asɯ-ɕar ɯ-fɕɤt me
sras 79-80

donggua cho qiezi ni tɕhi ʑo mɯ́jnatɕɯɣ ɣɯ ɯtɕha jɤtɯɣɯt ra
yici bi yici you jinbu 7

jɯm kɤ-ɕar ftɕaka ɣɯ tɯ-rɟaʁ sɯ-βzu-j 
sras.8

\section{Reduplicated nominalization}
nɤʑo tɯ-tɤ-tɯrɯɕmi nɯ aʑo lu pjɯ-nɯβze-a ra
\wav{ex-tWtAtWrWCmi}

\section{External argument of bitransitive verbs}  \label{sec:relativisation.external}

\begin{exe}
\ex
\gll      \ipa{tɕe} 	\ipa{ɬamu} 	\ipa{kɯ} 	\ipa{qɤjɣi} 	\ipa{nɯ-kɤ-mbi} 	\ipa{nɯ} 	\ipa{tu-ndze} 	\ipa{pjɤ-ŋu.}   \\
\coord{} Lhamo \erg{} bread \aor{}-\nmlz{}:O-give \topic{} \ipf{}-eat[III] \ipf.\evd{}-be  \\
 \glt    He was eating the bread that Lhamo had given him. (The Raven 111)
\end{exe} 



special case of the verb aro:
aʑo a-kɯ-ɤro nɯ lonba nɯ-kho-t-a

aʑo qaʑo aro-a nɯra kɯki ŋu
\wav{aro-relative} \wav{aro-relative2}
\section{Non-argument}
kɤnɯrɤɣo ɯskɤt ɲɯsna
他唱歌的声音
%8_kAnWrAGO-WskAt

maka tɕɤkɯ ɲɯ-kɯ-ru tɕe tɕɤndi smar ɯ-βzɯr tɕhi kɯ-fse mɤ-kɯ-sɤ-mto ʑo schɯs nɯ-azɣɯt ɲɯ-ŋu
divintation2 27
He arrived at river (that was so wide) that nothing was visible on the other side when one looked.

ɕɯ-kɤ-pjɤl a-ʁa me

\section{Relative pronoun}

tɕe stɤβtshɤt ɕu pɯ-kɯ-βʁa zɯ nɯ ɯ-jaʁ khám-a ŋu ma,
sras 88



\section{Negative indefinite}
Use of relative sentences where negative indefinite pronouns would be used in English: 
Mainly human

nɤʑo	nɯ-nɯ-ɣɤwu	ma,	nɤ-kɯ-nɯɣmu	me	ma	ma-ta-mbi
Frog 38
 

	kɤ-mɯnmu	kɯ́nɤ	mɯ-pjɤ-mɯnmu				
	Raven 58
	
	
	
\chapter{Complement clauses and complementation strategies}
ɕu ɯ-jaʁ ɣɯ a-nɯ-khám-a, ɕu ɯ-jaʁ a-mɤ-nɯ-khám-a ɲɯ-ɕti tɕe
sras 83


\section{Raising of directional prefix}
kɤnɤma tasthɯt
kɤβzjoz pasthɯt


kɤntʂu lɤsɤɲɤja
我赶紧地锄草了
 kɤnɯzʁe kasɤɲɤj
 他赶紧搬了东西
 kɤraχtɕi nɯsɤɲɤja
 我赶紧地洗了
kɤrɤrɤt pɯsɤɲɤja
 
\section{Infinitive kɤ-}
nɯra kɯ paχɕi ra kɤwum taqurnɯ
pear story

mɯ-ɲɯ-kɤ-mɤrʑaβ	kɯ-ra	to-nɯkrɤ́z-nɯ,
Mchodrten 5

Infinitive with causative of stative verbs:
kɤɣndʑɯr chɤsɤmɲɤm
 他磨得均匀

kɤrɤmbi nɯsɤmɯzɣɯt

tɯntʂu ɲɤsɤmɯzɣɯt



kɤ-nɯsmɤn	tɕe	tɕhi	tu-tɯ-ste	ŋu
Qala 37


???ɯʑo kɤmto pɯrɲota ɯŋu?

mɯpɯtɯrɲot
phone conversation nov 2011

\section{tɯ- infinitive} 
tɯ-ɕke ta-ʑa tɕe, 
tɤ-rŋi	ɲo-ʑa
aʁɤndɯndɤt	tɯ-nɤrɯra	ci	na-ʑa

tɯŋke nasɯɣʑaʁ = kɤŋke nasɯɣʑaʁ

alternative form with kɤ- always possible ?


aʑo kɤnɯʑɯβ kordala!
\section{Finite verb}
ɯ-ŋgɯ	rdɤstaq	tɤ-phɯ	mɤʑɯ	kɯ-maq	tɤ-ɲɟoʁɲɟi	ra,	qale	kɯ	lu-tɕɤt	mɤkɤcha	nɯ	ra	tɕe,	tɕɤn	tú-ɣ-ntshi,

+ra

nɯɕɯŋgɯ tɕiʑo laritɕi nɯ tɤŋketɕi pɯra ma tham tɕe qiche ɯŋgu tukɯɕe khɯ
\wav{8_tANketCipWra}
Phone conversation, Oct 2011, Chen Zhen


rgɤtpu kɯ tɤ-kɤ-tɯt nɯ fse-tɕi ra
smanmi4.70
We must be like the old man said.
\section{Causation} \label{sec:causation.complement}

tʰamaka nɯ-sɯ-ftɕat-a
a-wa tʰamaka ɯ-kɯ-ftɕɤt nɯ-sɤβzu-t-a
a-wa tʰamaka mɤ-kɯ-sko nɯ-tɕat-a
a-wa tʰamaka mɤ-kɯ-ra nɯ-tɕat-a
\wav{8_causatifs_complexes}

Aside from the causative prefixes, we also find syntactic causatives, based on the verb \jpg{βzu} ``to do'', or simply in some case by juxtaposition.

 iʑora kɤnɯʑɯβ mɤkɯkhɯ tutɯsɤβze ɲɯŋu

tɯŋga kɯxtɕi tɤβzuta > 故意 ɣɤxtɕi
 

qale kɯ kɯzbaʁ tosɤβzu

'to force' mɤkɯftshi

a ɯphɯ ɲɯwxti tɕe, nɯra thamtɕɤt matɤtɯɣɤwxti, aʑo nɯ tunɯχtia
nɯra sthɯci kɯwxti ɯphɯ mɤtɤtɯβze kɯtʂaŋ ci tɤβze
Bargaining, 11-12


kɤlɯlɤt mɤkɯra ɲɤtɯtɕɤt \wav{mAkWra}
你令他们不打架


There is no specific verb such as ``to hinder'' or ``to prevent'' in Japhug, so that one has to find another construction to express this meaning, sometimes with a negative causative:



mɯ-tu-kɯ-nɤtɯti	kɯ-ra	tɤ́-ɣ-sɯ-βzu-a-nɯ	ndʐa	ɕti
Bdud 158 
It is because they forced me not to tell everyone. > tɤ́wɣsɯβzuanɯ


aʑo	nɯsthɯ́ci	a-kɯ-sɤscit	tu-tɯ-βze	tɕe
Kubzang 249

ɯ-kɯr ɯ-ŋgɯ tɕe, tɯci pjɯ-kɯ-lɤt to-βzu
Nyima wodzer 2011.54


paʁ thɯ-ɣɤtshu-t-a > kɯ-tshɯtshu ʑo thɯ-sɤβzu-t-a
8_gram-caus-tshu

ju-kɤ-nɯɕe kɯme tusɤβze
gram-ju-kAnWCe-kWme

\section{Modal auxiliaries}
Obligation: zgɤt 应该 ra 要 ɬoʁ 必须 ntshi 只好

\subsection{ra}

 nɯ, tɯ-xtsa ɯ-pa nɯ mŋulɤn tu-kɯ-ti ŋu.
tɕe nɯnɯ tɕe konaʁ xtsa ɣɯ tɕi ra, 
komɤrxtsa ɣɯ tɕi ra,
30-komAr, 66-67

\subsection{jɤɣ}

jɤxtshi jɤ-tɯ-ɣe tɕe, bingguan ɯ-ŋgɯ kɤ-rɤʑi ɲɯ-ra ma siren kha ku-tɯ-rɤʑi mɯ́j-jɤɣ khi
\wav{mWjjAG}
\section{Reported speech}
nɤʑo ɯ-rɯz ɣɤʑu tɤ-tɯt-a tɕe
Nyima wodzer4.153


tɕe lɤ-jɣɤt tɕe tɕendɤre li ɯ-rʑaβ nɯ kɯ  "a-tɤ́-wɣ-nɤsna-a ra" ɲɤ-sɯso tɕe,
kWnWrmAZu.2.33

\chapter{Clause Linking}

Type of clause linking which are neither relative clauses nor complement clauses.
Focal clause (FC).

vs. Main clause/Non-main clause
 Supporting clause
\section{Serial verb construction} \label{sec:serial.verb}

tɤstu tɤmbat

atʂha ci pɯzmɤke pɯrke


si tasɤdʑɯdʑaŋ paβde


tu-zrɤre tɕe kɯ-rɤma ju-sɯxɕe pjɤ-ŋu.
thème56

pjɯ-tɯ-ɣi mɤ-ra
aʑo pjɯ-kɯ-nɯ-βde-a-nɯ mɤ-ra, aʑo pjɯ-nɯmtsaʁ-a jɤɣ
Nyima Vodzer4.137

\section{Temporal}

Temporal succession
Relative time

kuɣi ɕɯŋgɯ χsɯsŋi tɕe adianhua tulɤt ɯŋu?


tɕe jamwon ɯ-phe lɤβzaŋ ju-nɯzɣɯt ɕɯŋgɯ stɯnmɯ βzu-j tɕe
Blobzang30

pɤjkhu qala nɯ mɯchɯsta ɕɯŋgɯ tɕe bɤlqhoʁ nɯnɯ juzɣɯtndʑi kɯra nɯtɕu jozɣɯt, aesop 50 =qala nɯ pɤjkhu mɯchosta ri

tɤ-ŋe	pjɯ-ɕqhlɯt	ɕɯŋgɯ	mɯ-mɤ-jɤ-kɯ-phɣo	nɤ, tɕe	tɤ-ŋe	pɯ-ɕqhlɯt	ɯ-qhu	tɕe	ku-kɯ-ndo	pjɤ-ɕti
Mirgod22

pjɯ-ɣɤrɤt	ɕɯŋgɯ	nɯ	tɕu	tɕe,	tú-ɣ-z-rɯndzɤtshí-nɯ	ɲɯ-ŋgrɤl	
Nyma vodzer 102


nɯ	ɯ-qhu	tɕe	tɯ-wi	nɯ	kɯ-rɤma	ju-ɕe	tɕe,	ju-nɯɣi	ɕɯŋgɯ	tɕe,
	ɯ-ndzɤtshi	nɯ	tɕhɣaq-tɕhɣaq	ɯ-tu-kɯ-stu	ci,
ɯ-kha	nɯ	ra	tshɯn-tshɯn	ʑo	ɯ-tu-kɯ-stu	ci	pɯ-tu	ɲɯ-ŋu,


Kubzang 330-2


rɟɤlpu	lu-ɣi	ɕɯŋgɯ	ndɤre,	ɯ-rŋa	ta-ʁɟɤs	cicilɯk	ʑo	ta-mar	ɲɯ-ŋu,
Kunbzang 348


	kɤ-ari-a	ɯ-qhu	tɕe,	nɤ-sni	nɯ	mɤ-nɯ-tɯ-ɕlɯɣ	ra	ma
Gesar 337

tɕe	tɤ-ŋe	pɯ-ɕqhlɯt	ɯ-qhu	tɕe	ku-kɯ-ndo	pjɤ-ɕti
Mirgod 23

aʑo	pɯ-ari-a	ɯ-qhu	tɕe,	tɯ-mda	rɟɤlpu	ɣɯ	nɯ-mtshu	nɯ	a-pɯ-khrɯ	ra
Nyima2.73
\section{Consequence}

nɯ-phɣó-tɕi	pɯ-ra, a-ɬaq	mɤ-kɯ-sna	ndʐa	ŋu
Nyima wodzer41

\subsection{Conditional}

ʁnaʁna ʑo ndʑi-tɕɯ tɯ-tɤ-tu nɤ, kɤndʑɯ-xtɤɣ tu-kɤ-sɯ-pa,

nɤkɯmŋɤm ɯɣɤʑu nɤ, tutanɯsmɤn
Aesop 43

mo-mbroŋ	ci	tɯ-tɯ-ŋu	nɤ,	iɕqhánɯ,	tɤ-jpɣom	ɣɯ	[ɯ-thɤβ]	ɯ-χcɤl	tɕe	nɯ-ɣi,
Gesar41

tɯ-rme ɲɯ-ɲɯ-maʁ ŋu nɤ, a-ku ɲɯ-sɤphár-a ŋu tɕe, tɕe phɣó-tɕi ra
Tiger35

kɯki	ɕi-kɤ-ru	ɯ-tɯ-chá-nɯ	nɤ	a-sci	rɟɤlpu	chɯ-ta-sɯ-ndo	ŋu
Fox18	


Two conditions
tʂu	ku-xti	ju-nɯɬóʁ-a	nɤ,	tɯ-rme	kɯ-dɤn	kɯ	a-pɯ́-ɣ-mto-a-nɯ	nɤ, tɕhi	nɯ	mɤ-nɯ-nɯmŋa
Fox 120

tɤ-tɯ-ndza-t	nɯ	a-pɯ-ŋu	ma,	nɯ	ma	kɯ-me	mɤ-ŋgrɤl	manɤʑo	nɯ	thamtɕɤt	mɤ-tɯ-sɯ-ɕkɯt",	to-ti
moineau 42-3


ɯ@biaozhun mɯ́jnaχtɕɯɣ ndʐa ɲɯŋu
\wav{8_biaozhun}
conversation

"a a-tɕɯ, tɕe nɤ-ɬaʁ ɯ-kɯ-mŋɤm tu-mna maŋe tɕe, nɯ ɲɯ-tɯ-cha ndɤre, wuma ʑo ɲɯ-tʂɯn" to-ti
28, smanmi v4.

"otherwise"
a-wɯ cho a-ʁi nɯ pɯ-nɯ-khɯr-nɯ ra ma nɯ mɤɕtʂa mɤ-ɣi-a
Nyimawodzer4.126




mɤ́ɣrɤz nɤ 反而
\subsection{Resultative}

\subsection{Purposive}
chengdu kɯ-ɕe tɤ-aɣɯɣu-a
ɯ-tɕhɯβ
\xv fso tɕe atɤtɯnɯtɕi, ʑa atɤtɯrɤru tɕe, kɤnɤma amɤnɯkɯmaqhu ɯtɕhɯβ apɯŋu
\xn 明年早点起床,以便不要迟到
\xv tɯŋga kɯjaʁ tsa tɤŋge tɕe, mɯɲɯtɯnɤndʐo ɯtɕhɯβ apɯŋu (=nɤmɤɲɯznɤndʐɯndʐo apɯŋu)

amɤnɯtɯnɤndʐo apɯŋu

\wav{8_WtChWB}

nɤʑo nɯtɕu pɯrɤt tɕe, mɯɲɯtɯnɯjmɯt apɯŋu
nɤmɤɲɯsɤjmɯjmɯt apɯŋu
\wav{8_WtChWB2}
\subsection{Possible consequence}


\subsection{Counterfactual}
%8_counterfactual
jɯfɕɯr ʑatsa kutɯrŋgɯ apɯŋu tɕe, tham tɕe nɯtshɯci mɯpɯtɯɲat

ʑatsa akɤtsoa tɕe, kɯki laχtɕha ki mɯpɯχtɯta

\section{Addition}
Unordered addition and [no SC/FC distinction]
 Same-event addition and, moreover [both FC]
 Elaboration <apposition> [2nd clause is FC]
Contrast but [FC], although [SC]



aku pɯphaβa tɕe laria ma kɯm ɲɯmbɤr
8_phAB
\section{Concession}
Alternatives
Vd Disjunction or [no SC/FC distinction]
Vr Rejection instead of [SC]
Vs Suggestion rather than [SC]


nɯ kɯ nɯ-skɤlɤn nɯ tu-βze laʁsɯɣnɤma, nɯ-rŋa ci nɯ mɯ-pjɤ-ru ɲɯ-ŋu,
sras 19



ɯtɕhɯβ



pɯ-nɯɣi	ma	ʑɣɤsát-a	ŋu
mchodrten
Come, otherwise I will kill myself


instead of

nɯtɯsɤre nɯ mɤra, nɯʑo kɯrɯfsaʁfsoʁ jɤkɤri ra, 你们出去挣东西的
aʑo kha kɯrɤʑi nɯ ɣɯ phe, aʑɯɣ jutɯɣɯtnɯ ɯtshɤt kɯ aʑo aphe "kɯɲidi ɣɤʑu" tutɯtinɯ
Slobdpon 267



No only
75	tɕendɤre	kɤ-nɯsmɤn	ra	maq-kɯ,	kɯ-rtsɤk	nɯ	kɤ-ɬoq	ri	mɯ-pjɤ-khɯ	ma
qala


\section{Manner}
VIr Real like, in the way that [both SC]
VIh Hypothetical as if (, like) [both SC]


ɯro mɯɲɤta ʑo pjɤkra

\section{Comparative}
tɤ-khe	pɣɤtɕɯ	nɯ	a-pɯ-ŋu,	kɯ-mŋu	ci	tú-ɣ-sɯpa	jamar,
Bzar 25 It is considered to be approximately five times (as big) as a ...

 
kɯ-ɕqraʁ ɣɯ ɯ-kɯ-ɕqraʁ tu kɤ-ti ɲɯ-ŋu.
thème55
There is always someone smarter that oneself


lɯlu ɣɯ tɕe ɯʑo ɯ-phoŋbu tɕhi kɯ-zri jamar ɯ-jme nɯ kɯnɤ zri ri,
217, \wav{x-27-qartshaz} Fox.
As for the cat's (tail), its tail is as long as its body.


li tɯ-ɕtʂi ʑo pjɯ-tɕɤt tu-mŋɤm ŋu. 
25-KACAl, kroŋwa
It hurts to the point of making one sweat.



\begin{exe}
   \ex  \label{ex:ki.jamar1}
   \gll
[\ipa{kɯ-jpɯ-jpum}]   	\ipa{ki}   	\ipa{jamar}   	\ipa{ɣɤʑu.}   \\
   \textsc{nmlz:S-redp}-thick \textsc{dem.prox} about exist:\textsc{sensory} \\
\glt There are (yak horns) that are thick like this. (Wild yak, 25)
\end{exe}

This construction is very rare; this particular instance was chosen because in this particular story it is directly followed by two sentences of similar meaning,\footnote{The speaker repeated almost the same sentence several times to make herself understood.} which illustrate the more common comparative constructions using a simple noun (example \ref{ex:ki.jamar2}) or a degree nominalization in \ipa{tɯ}-- (\ref{ex:ki.jamar3}).

\begin{exe}
   \ex  \label{ex:ki.jamar2}
   \gll
\ipa{ki}   	\ipa{aʑo}   	\ipa{a-jaʁ}   	\ipa{ki}   	\ipa{jamar}   	\ipa{ɣɤʑu.}   \\
   \textsc{dem.prox}    \textsc{1sg} \textsc{1sg.poss}-hand   \textsc{dem.prox} about exist:\textsc{sensory} \\
\glt There are (yak horns) that are like this hand of mine. (Wild yak, 26)
\end{exe}
\begin{exe}
   \ex  \label{ex:ki.jamar3}
   \gll
\ipa{ɯ-tɯ-jpum}   	\ipa{nɯ}   	\ipa{kɯra}   	\ipa{jamar}   	\ipa{ɣɤʑu.}     \\
   \textsc{3sg-nmlz}:degree-thick \textsc{top} \textsc{dem.prox:pl} about    exist:\textsc{sensory} \\
\glt There are (yak horns) that are thick like this. (Wild yak, 26)
\end{exe}

\chapter{Topic and Focus} \label{chapt:discourse} 
This chapter deals with discourse structure, in particular the morphological and syntactic marking of topic and focus. The use of intonation in marking focus and topic in Japhug will not be discussed here (see \citet{lin09phd} for a study of intonation in Situ Rgyalrong).

In this chapter, we will first present topic markers, including both clitic markers, word order and whole topical constructions. Second, we will describe the focal constructions. Third, we will briefly examine right dislocation. Fourth, we will analyse the use of the inverse marker (see chapter \ref{chapt:flexional.agr}) as a discourse marker in Japhug.

\section{Topic} \label{sec:topic}
%accessibility ???

Topic can be marked in Japhug using various morphosyntactic devices, including various clitics, word order, and direct / inverse marking on the verb. The pragmatic use of inverse marking will  be discussed later in \ref{sec:inverse.discourse}, so that the present section will focus on the first two categories. 

Two subsections will be devoted to topic clitic markers: first, the general topic markers, and second the adversative ones. Then, we will describe  topicalization by left dislocation. Finally, we will present the switch topic construction.

\subsection{Topic markers} 
There are two main topic markers in Japhug, postnominal \ipa{nɯ}, and prenominal \ipa{iɕqha}, which originate, as we shall see, from different parts of speech.

The most common topic marker in Japhug, and arguably one of the most common words in running text, is the clitic \ipa{nɯ}. It is obviously related to the distal demonstratives \ipa{nɯ} and \ipa{nɯnɯ} (see \ref{sec:demonstratives}) , as well as the complementizer \ipa{nɯ} (see chapter \ref{chapt:relative}), and the question whether these  three homonymous markers are best treated separately from a synchronic point of view, or should be analysed as one morpheme, is far from trivial. 

In the following, we will show that \ipa{nɯ} cannot be analysed either as a distal demonstrative in cases, and that it cannot be seen as a marker of definiteness either.

The distinction between \ipa{nɯ} as a topic marker and \ipa{nɯ} as a demonstrative is not straightforward. As presented in \ref{sec:demonstratives}, the demonstrative \ipa{ki} ``this'' and \ipa{nɯ} ``that'' appears in two competing constructions: the noun can be either followed by the demonstrative, or preceding and followed by it, as in the following example:

\begin{exe}	
\ex	
\gll  \ipa{nɯ}  	\ipa{tɯ-ldʑa}  	\ipa{nɯ}  	\ipa{tsaʁ}  	\ipa{nɤ-βra}  	\ipa{a-pɯ-ŋu}  	\ipa{tɕe,}  	\ipa{ma-pɯ-tɯ-qlɯt}  \\
 \dist{}.\dem{} one-\cl{}:long.object \dist{}.\dem{} only 2\sg{}.\poss{}-share  \irr{}-\ipf{}-be \coord{} \negat{}-\imp{}-2-break   \\
 \glt 	  You only have that one as your share, do not break it. (Gesar 313)
\end{exe} 	
The second demonstrative can be reduplicated (\ipa{kɯki} and \ipa{nɯnɯ}). Both the simple and reduplicated demonstratives can be used alone without noun. 

However, the main reason for analyzing \ipa{nɯ} as a topic marker rather than as a demonstrative in some cases it that the clitic \ipa{nɯ} does not appears in answers to questions:

 
 \begin{exe}	
\ex	
\gll  \ipa{tshɤmbɤr}  	\ipa{nɯ}  	\ipa{tɕhi}  	\ipa{ɲɯ-ŋu?}     \\
   ???? \topic{} what \ipf{}-be   \\
     \gll  \ipa{rkɯwɯ}  	\ipa{kɯ-wxtɯ-wxti}  	   \\
  oil.lamp \nmlz{}.\stat{}-\emphat{}-big     \\
 \glt 	 What is a ``tshɤmbɤr''? It is a big oil lamp. (Gesar 381-382)
\end{exe} 	
 
A	74	ɯ-nmaq	kɯ	to-thu	tɕe	"ki	tɕhi	ŋu?"	toti	ri,			
B	74	3s:mari	ERG	MIP:3s>3:demander	CJC	DMP	quoi	NPA:3s:être	MIP:3s>3:dire	CJC
C	74	丈夫	施格	问	连词	那	什么	是	说	连词

A	75	ɬamu	kɯ	"ki	rŋɯl	qhoq-qhoq	ŋu"	toti,	
 
 
 
\begin{exe}	 \label{ex:who.dem}
\ex	
\gll  \ipa{ɕu}  	\ipa{nɯ}  	\ipa{kɯ}  	\ipa{to-ti} ?  \\
     who \dem{} \erg{} \evd{}-say \\
     \gll  \ipa{qachɣa}  	   \\
     fox \\
 \glt 	 Who (which of them) said it? The fox. (The fox,  41-42)
\end{exe} 	

% Note however that the proximal demonstrative \ipa{ki} is not found in such cases, and does not alternate with  \ipa{nɯ} here.

We observe nevertheless examples where \ipa{nɯ}  occurs in answers. We will argue that in these  cases  \ipa{nɯ} is not a topic marker, but should rather be analyzed as demonstrative or complementizer.

In the following sentence, \ipa{nɯ} is not a topic marker, but a complementizer used to relativize the finite sentence indicated in square brackets (concerning these relative clauses see \ref{sub:relative.non.nmlz}):
\begin{exe}	
\ex	
\gll\ipa{tɕe}  	\ipa{``nɤʑɯɣ}  	\ipa{tɕhi}  	\ipa{ra?"}  	\ipa{ti}  	\ipa{ɲɯ-ŋu}    \\
    \coord{} 2\sg{}:\gen{} what \npst{}:need \npst{}:say \ipf{}-be \\
\gll  [\ipa{a-wa}   	\ipa{tɕhi}	\ipa{tɤ-tɯ-rkɯ-rku-t}  	\ipa{ʑo}]  	\ipa{nɯ}  	\ipa{ra}  	\ipa{ti}  	\ipa{ɲɯŋu}    \\
   1\sg{}.\poss{}--father what \aor{}-2-\emphat{}-put.in-\pst{} \emphat{} \compl{} \npst{}:need \npst{}:say \ipf{}-be \\
 \glt 	``What do you need?", he_i said. ``I need what you prepared for my father", he_j said.   (Slob-spon Norbu bzang 167)
\end{exe} 	


With the postposition \ipa{tɕu}, we however find \ipa{nɯ} used in focus position:

\begin{exe}	
\ex	
\gll  \ipa{tʂu}  	\ipa{ŋotɕu}  	\ipa{nɯ}  	\ipa{tɕu}  	\ipa{chɤ-ɕe}  	\ipa{ɲɯ-ŋu?}     \\
    road  where \dem{} \loc{} \evd{}:downstream-go \ipf{}-be \\
     \gll    \ipa{tʂu}  	\ipa{kɯ-xtɕi}  	\ipa{nɯ}  	\ipa{tɕu}  	\ipa{ʑo}  	\ipa{jo-ɕe,}   \ipa{ɯ-pi}  	\ipa{ni}  	\ipa{tʂu}  	\ipa{kɯ-wxti}  	\ipa{nɯ}  	\ipa{tɕu}  	\ipa{jo-ɕe-ndʑi}    \\
    road \nmlz{}:\stat{}-small \dem{} \loc{} \emphat{} \evd{}-go  3\sg{}.\poss{}-elder.sibling \du{} road  \nmlz{}:\stat{}-big \dem{} \loc{} \evd{}-go-\du{} \\
 \glt 	 Which road did he take? He took the small road, and his brothers took the big road. (The fox,  21-22)
\end{exe} 	
 
 A possible explanation for these forms is the following.  As seen in \ref{sec:loc},  the locative markers are generally optional in Japhug, and never obligatory with movement verbs.  In the same story by the same speaker, we find the following sentence:

\begin{exe}	
\ex	
\gll   \ipa{tʂu}  	\ipa{kɯ-wxti}  	\ipa{jɤ-ari}  	\ipa{ri,}  \\
    road \nmlz{}:\stat{}-big \aor{}-go[II] \advers{} \\
 \glt 	Although he took the big road,    (The fox,  121)
\end{exe} 	


We can therefore are argue that [\ipa{tʂu} \ipa{kɯ-xtɕi}] and [\ipa{nɯ} \ipa{tɕu}]  constitute two distinct constituants meaning ``the small road'' and ``there''. \ipa{nɯ} is not a clitic belonging to the noun phrase whose head is \ipa{tʂu} ``road'', but rather a demonstrative.
 


 
 a-zda ra ʑara-sɯso tu-nɯ-nɤŋkɯŋké-nɯ ɲɯ-khɯ, nɤki, pjɯ-nɯ-rŋgɯ́-nɯ ɲɯ-khɯ ma aʑo nɯ sŋiɕɤr ʑo kɯtɕu ki fse-a ndzúr-a ntsɯ ɲɯ-ra tɕe,
 divination01.43-44

tɕendɤre	kɤndʑɯxtɤɣ	ʑo	kɯ-fse	nɯnɯ,	thɯ-nɯ-rɤʑi-ndʑi	ndɤre,
ɯʑo	kɯ	rɟɤlpu	ɯ-ma	tu-nɤme,	ci	nɯ	kɯ	ɯ-xtɤɣ	kɯ-fse	nɯ,	qachɣa	kɯ	tú-ɣ-qurɣa,
(The fox 178) 
thaχtsa	nɯ	iʑo	kɯrɯ	tɕheme	ra	nɯnɯ	mɤlɤn	ʑo	pjɯ-tu	kɯ-ra	tɕe,
Colored belt 92

jɯm	nɯ	mɯ-pjɤ-ɕar	ɲɯ-ŋu
The prince.2



iɕqha	       kɤtɯm	nɯ	ɯʑo	kɯ	ko-ndo
Gesar 272
325	iɕqha	tɯ-ku	nɯ	ra	z-ɲɤ-sɯ-ru	qhe,	tɕe	ʁlaŋ	ɣɯ	ɯ-rɟaβlun	nɯ	ra	to-ɣɤsna
 
%sqaβdɤ-rʑaq	tɤ-tsu	ndɤre,	tɕe	iɕqha	tɯ-ŋga	ɯ-spa	ɕ-pa-ɣɤrɤt	ɲɯ-ŋu, 329

iɕqha	tɯ-phu	koŋ-zoŋ-jaŋ-rtɤn-bu	na-nɯ-mɢla	ɲɯ-ŋu	
20


mɲɯrgɤt kɤ-ti ci pjɤ-tu tɕe,
Mirgod1.1

\subsection{Adversative Topic}


ʁo lɯski?

ndɤre

 tɕe nɯnɯ kɯ pjɯ́-wɣ-sɯ-tɤβ tɕe tɕɤn,
 pjɯ́-wɣ-sɯ-tɤβ pɯ-ra, jinde tɕe ʁo, tɕɤn, 
 mkhɯrlu tú-wɣ-ntɕhoz ɕti tɕe, wuma ʑo pe ma,
rtsampa2.81 
 
\begin{exe}	
\ex	
\gll 	\ipa{khu} 	\ipa{nɯ} 	\ipa{ʁo} 	\ipa{nɯ-βde} 	\ipa{wo}, 	\ipa{a-me} 	\ipa{nɯ} 	\ipa{sɤ́znɤ} 	\ipa{a-ftsaʁ} 	\ipa{ɯβrɤ-ɣi} 	\ipa{ma} \\
tiger \topic{} adversative \imp{}-throw.away \hort{} 1\sg{}-daughter \topic{} \comptv{} 1\sg{}-leak  \intrg{}-\npst{}:come \question{} \\
 \glt 	Don't worry about the tiger, daughter, rather than that I wonder whether the roof will leak. (Tiger, 5)
\end{exe} 	



\subsection{Left dislocation}
\begin{exe}								
\ex								
\gll 	\ipa{ɬandʐi} 	\ipa{wuma} 	\ipa{nɯ} 	\ipa{nɤʑo} 	\ipa{ɲɯ-tɯ-ŋu} 	\ipa{ma} 	\ipa{aʑo} 	\ipa{ɬandʐi} \ipa{ɲɯ-máʁ-a}  \\
demon true \topic{} you \ipf{}-2-be \concsv{} I demon \ipf{}-be-1\sg{} \\
 \glt 	You are the true demon, not me (Demon and monk, 9)
\end{exe} 

\begin{exe}	
\ex								
\gll 	\ipa{astaʁɬamu}	\ipa{tu-kɤ-ti}	\ipa{pjɤ-ŋu}	\ipa{ma}	\ipa{tɕe,}	\ipa{βdɯt}	\ipa{ɣɯ}	\ipa{ɯ-snom,}	\ipa{tʂu}	\ipa{ku-ndɤm}	\ipa{pjɤ-ŋu,}
 \\
  \\
 \glt 	 
\end{exe} 

\subsection{Switch-topic}

\begin{exe}	
\ex	
\gll \ipa{a-mu} 	\ipa{nɯ} 	\ipa{pɯ-pɯ-ŋu} 	\ipa{nɤ,} 	\ipa{qhlɯʁdɯxpa-kɤrpu} 	\ipa{ɣɯ} 	\ipa{ɯ-me} 	\ipa{stu} 	\ipa{kɯ-xtɕi} 	\ipa{nɯ} 	\ipa{a-mu} 	\ipa{ɲɯ-pe,} \\
1\sg{}-mother \topic{} \cond{}-\pst{}.\ipf{}-be if Klu.gdugpa.dkarpo \gen{} 3\sg{}-daughter most \nmlz{}:\stat{}-small \topic{} 1\sg{}-mother \const{}-good \\
 \glt 	 As far as my mother is concerned, the daughter of Klu Gdugpa Dkarpo would be good as my mother (Gesar 5)
\end{exe} 	


chɯ-ta-tɕɤt	pɯ-pɯ-ŋu	nɤ,	nɤki,	nɤ-jɯɣi	nɯ	ci	thɯ-khɤm	tɕe,	tu-nɤmɲám-a
Gesar 142

New topic introduced with \ipa{ɣɤʑu}:

aʑo	tɕheme	ɲɯ-ŋu-a	tɕe,	rŋɯl	qaɕpa	ci	ɣɤʑu	tɕe,	nɯ	thɯ-mqláʁ-a.
Nyima 144


tɕendɤre ɯ-fso tɕe nɤʑo sɤtɕha ɯ-taʁ kɤ-ɕe 
aʑo nɯnɯ tɯmɯ kɤrŋi ci ku-ɕe-a" to-ti
smanmi4.135
\section{Focus}

\subsection{Questions and answers}
tɯtshot thɤstɯɣ jamar tɯnɯsaχsɯnɯ ?
tɯtshot sqamnɯz tɕe tunɯsaχsɯj ŋu
(Dpalcan 2010)

ŋotɕu nɯ ?

kukutɕu ʑo ŋu
(Aesop, The horse and the Lion)
 
a-wi	ɯ-thɯrʑi	tɯ-rme	nɯ-aβzu-a	ŋu
It is thanks to my grand-mother's compassion that I became human.
372

\subsection{Focalization}
focalisation simple (ou informationnelle ) et focalisation contrastive
Creissels 2006b:120


%– focalisation de rejet, comme dans –Jean est allé au cinéma avec Marie. –Mais
%non, ce n’est pas avec Marie qu’il est allé au cinéma ;
> A tukɯti maʁ ma b tukɯti ɕti



%– focalisation de substitution, comme dans –Jean est allé au cinéma avec Marie.
%–Mais non, c’est avec Nathalie qu’il est allé au cinéma ; la focalisation de
%substitution apparaît souvent combinée dans la même phrase avec une focalisation
%de rejet, comme dans –Jean est allé au cinéma avec Marie. –Mais non, ce n’est pas
%avec Marie qu’il est allé au cinéma, c’est avec Nathalie ;
ɯʑo	kɯ	aʑo	kɤ-ndza	maʁ	kɯ, aʑo	kɯ	ɯʑo	tɤ-ndzá-t-a	ɕti	zɯ
demon 94

nɤʑo	tɤ-tɯ-nɯ-ndo-t	ŋu	ma	nɤʑo	nɯ	ma	i-ɕki	pɯ-kɯ-rɤʑi	me
Flood 59
%– focalisation d’expansion, comme dans –Jean est allé au cinéma avec Marie. –Il
%n’est pas seulement allé au cinéma avec Marie, il l’a aussi raccompagnée ;
125	smɤnmi-mitoʁ-kuɕana	mɤkɯjɤɣkɯ	ʁnɯ-ɕana,	χsɯ-ɕana	kɤ-nɯʁdán-a	ŋu	tɕe,

%– focalisation de restriction, comme dans –Jean est sorti avec Marie et avec
%Nathalie. –Non, il est seulement sorti avec Nathalie ;
nɤʑo	tɕhi	tɯ-nɯɣme"	to-ti	ri,
	nɤki,	"aʑo	ndɤre	a-lo	tshɤmbɤr	ki	nɯɣme-a	ma	nɯ	ma	nɯɣme-a	me",
Gesar 380



%– focalisation de sélection, comme dans –Je me demande si Jean sort avec Marie
%ou avec Nathalie. –Je peux te dire que c’est avec Nathalie


Not necessarily preverbal position for focus

\subsection{Pseudo-cleft}

tɕeri ɯ-kɤ-mtshi nɯ khu pɯ-ɕti ɲɯ-ŋu 
What he was leading was a tiger
Tiger 17

\subsection{Focus markers}
not even
tɕeri	tɕheme	nɯ	kɯ	ɯ-skɤlɤn	cínɤ	mɯ-ta-βzu	nɤ
25
not even one



\subsection{Presentative}
"Here is"
aʑo	aj	a-mbro	nɯ	nɤrwɯ-rɯnbotɕhi	ŋu	tɕe,	nɯ	tɤ-nɯ-ndɤm	tɕe,
smanmi 49


\section{Repetition}

iɕqʰa nɯ ŋu ŋu, nɯ ntsɯ tu-ste ŋgrɤl
他一贯都是这样



"wo, nɯ kɯ-fse ci tɯ-ŋu tɯ-ŋu nɯ ma  mɤ-kɯ-naχtɕɯɣ ci tɯ-ŋu nɯ-sɯso-t-a ɕti ma
smanmi4.51
只有你这种人才是这样


\section{right dislocation}

\citet{creissels06sgit2}[120]

tɕhi	tu-tɯ-ste	ŋu	kɤ-sɤfstɯn


kɤ-sɤfstɯn tɕhi	tu-tɯ-ste	ŋu	
Kunbzang 128

nɤʑo tɯ-nɯɕe ɕi, nɤ-mu ɯ-ɕki ?
(Chenzhen 2011, phone conversation)

mɤxsi matɕi pɤjkhu
\wav{mAxsi}

jiucai tu-ti-nɯ ŋu, kupa kɯ,
hist-07-Cku
148

tɕe nɯra pjɯ-tu ra wo, tɤ-sno ɣɯ.
87 tAsno.30
\section{The particle \ipa{rcánɯ}}
Important discourse particle
居然 ``contrary to expectation"

1. Expression of degree
168	maka	rdɤstaq	cho	si	rcánɯ	tɯ-ldʑa	cínɤ	kɯ-me	scɯ	kɤ-azɣɯ́t-ndʑi	ɲɯ-ŋu,
Il n'y avait pas le moindre arbre ou pierre


2. Topique (et quand à)

A	163	tɕheme	nɯ	tɯ-ɣjɤn	ci	kɯ́nɤ	nɯ-nɤre	kɤ-mtshɤm	pɯ-me,
A	164	pɣɤtɕɯ	rcánɯ	ɯ-loʁ	ŋgɯ	ntsɯ	ʁɟa-ʁɟa	ʑo	ku-nɯ-rŋgɯ	pɯ-ŋu	ma	nɯ	ma	kɯ-fse	pɯ-me,
A	165	mbro	rcánɯ	tɯ-ɣjɤn	cínɯ	nɯ-ntshɤr	pɯ-me	ɲɯ-ŋ


A	11	  jamwon ɯ-tɯ-mpɕɤr rcánɯ mɯntoʁ ʑo pɯ-fse ɲɯ-ŋu
A	12	lɤβzaŋ nɯ rcánɯ tɤ-ma ra ɯ-tɯ-pe pɯ-saχaʁ ʑo ɲɯ-ŋu,



193	tɕe	ʁʑɯnɯ	ci	rcánɯ	kɯ-wxtɯ-wxti	ʑo	nɯ-a-βzu	ɲɯ-ŋu.

 \section{Inverse marking} \label{sec:inverse.discourse}

\chapter{Alignment typology}

\section{Accusative alignment}

> paʁ ɲɯ́wɣrɤβraʁ tɕe rga

ɣɯ-tɤ́-wɣ-qur-a 
He came to help me > S = A \wav{8_GWtAwGqura}
\section{Ergative alignment}
> Case marking

> Generic wɣ vs. kɯ

> agreement suffixes on SAP

> 


\section{Tripartite alignment}
> Nmlz
\section{Bitransitive predicates} \label{sec:bitransitive}


\section{Inverse and alignment}  

tú-wɣ-nɯmbrɤpɯ ɲɯ-cha.
It can be ridden
19rNamoN, 44
\chapter{Final particles}


\section{kɯma}
会不会


ŋotɕu jamar janɯzɣɯt kɯma, tɕendɤre pɯndʐaβ qhe, (pear story)


rɟa	nɯ	tham	ku-pa	sɤtɕha	ɕti	woɣe?
>ɲɯ-ɕti.
Gesar 222

\section{nɯma}
smɤnmi mitoʁ nɯnɯ tɯmɯ kɤrŋi ɯ-taʁ nɯ tɕu (ko-rɟɯɣ) ko-ɕe
tɕe ko-nɯqambɯmbjom ɲɯ-ŋu nɯma, tɕe nɯ 
smanmi4.146
\section{khi}

fso tɕe chɯnɯɣi ɲɯkhɯ khi

\wav{8_chWnWGi}, phone conversation OCt 2011


\section{kɯ}

Remembering
 \begin{exe}
\ex
\gll    \ipa{nɯnɯ} 	\ipa{tɕʰi} 	\ipa{pɯ-rmi} 	\ipa{kɯ?}       \\
that what \pst{}.\ipf{}-be.called \textsc{qu} \\
\glt  What was he called, again? (Gesar, 249)
\end{exe} 
 
 
 Question indirecte:

\section{nɤ}


mɤʑɯ tutia ŋu nɤ?


\section{ɕi}
faguo tɕheme ra mpɕɤrnɯ ɕi, zhongguo tɕheme ra mpɕɤrnɯ
conversation

\section{thaŋ}
hypothèse

Est-ce que je le rencontrerai?
mɤxsi ma tɕe, fangjia βze thaŋ
\wav{8_BzethaN}
conversation 2012
%\printindex
\tableofcontents
\bibliographystyle{LSAlike}
\bibliography{bibliogj}
\end{document}
