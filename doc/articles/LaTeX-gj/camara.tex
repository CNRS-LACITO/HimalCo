\documentclass[oldfontcommands,oneside,a4paper,11pt]{article} 
\usepackage{fontspec}
\usepackage{natbib}
\usepackage{booktabs}
\usepackage{xltxtra} 
\usepackage{polyglossia} 
\usepackage[table]{xcolor}

\usepackage{multicol}
\usepackage{graphicx}
\usepackage{lineno}
\usepackage{float}
\usepackage{hyperref} 
\hypersetup{bookmarks=false,bookmarksnumbered,bookmarksopenlevel=5,bookmarksdepth=5,xetex,colorlinks=true,linkcolor=blue,citecolor=blue}
\usepackage[all]{hypcap}
\usepackage{memhfixc}
\usepackage{lscape}
\usepackage{tikz}
%
\usetikzlibrary{trees}
\usepackage{gb4e} 
\bibpunct[: ]{(}{)}{,}{a}{}{,}
 
%\setmainfont[Mapping=tex-text,Numbers=OldStyle,Ligatures=Common]{Charis SIL}  
\newfontfamily\phon[Mapping=tex-text,Ligatures=Common,Scale=MatchLowercase,FakeSlant=0.3]{Charis SIL} 
\newcommand{\ipa}[1]{{\phon #1}} %API tjs en italique
 
\newcommand{\grise}[1]{\cellcolor{lightgray}\textbf{#1}}
\newfontfamily\cn[Mapping=tex-text,Ligatures=Common,Scale=MatchUppercase]{MingLiU}%pour le chinois
\newcommand{\zh}[1]{{\cn #1}}
   
\newfontfamily\mleccha[Mapping=tex-text,Ligatures=Common,Scale=MatchLowercase]{Arial Unicode MS}%{Galatia SIL}%pour le grec
\newcommand{\grec}[1]{{\mleccha #1}}


\begin{document} 
\title{Sanskrit \ipa{camara--} `yak' et tibétain \ipa{ⁿbri} `yak femelle'}
\author{Guillaume Jacques}
\maketitle
L'étymologie du sanskrit \ipa{camara-} (m.), \ipa{camarī--} (f.) `yak' est restée jusqu'à présent obscure (\citealt[I, 375]{mayrhofer56kurz}: `nicht befriedigend erklärt'). Mayrhofer critique la comparaison avec le grec \grec{κεμάς} `jeune cerf sans corne', qui est impossible puisque le cognat sanskrit de ce nom, \ipa{śáma-} `sans cornes' (\citealt[II, 289]{mayrhofer56kurz}), montre qu'il remonte à un radical *\ipa{ḱem--} avec une palatale qui ne pourrait en aucun cas donner un \ipa{c--} en sanskrit. En outre, cette comparaison est absurde du point de vue du sens, le yak étant une animal plutôt connu pour sa vigueur et ses cornes imposantes.

%Principalement dans des listes
%Bhāgavatapurāṇa, 3, 10, 22.1
%kharo 'śvo 'śvataro gauraḥ śarabhaś camarī tathā 
%ete caikaśaphāḥ kṣattaḥ śṛṇu pañcanakhān paśūn

Ce terme n'est pas védique, et n'est pas attesté avant les épopées, où l'on trouve exclusivement   \ipa{camara--}.  Le féminin \ipa{camarī--} apparaît dans les textes plus tardifs, tels que le \ipa{Bhāgavatapurāṇa--} ou les pièces de théâtre. On mentionne cet animal étranger à l'Inde ancienne presque uniquement pour l'usage qui était fait de sa queue pour fabriquer des chasse-mouches royaux (le  \ipa{cāmara--}), comme dans l'exemple \ref{ex:camari}.

\begin{exe}
\ex \label{ex:camari}
\glt lāṅgūlavikṣepavisarpiśobhair itas tataś candramarīcigauraiḥ 
\glt yasyārthayuktaṃ girirājaśabdaṃ kurvanti vālavyajanaiś camaryaḥ 
\gll lāṅgūla-vikṣepa-visarpi-śobhais itas tatas candra-marīci-gaurais yasya artha-yuktam giri-rāja-śabdaṃ kurvanti vāla-vyajanais camaryas \\
queue-agitation-qui.s'étend.tout.autour-splendeur:\textsc{bahuv:m:instr.pl} ça là lune-rayon-blanc:\textsc{m:instr.pl} \textsc{dem:m.sg.gen} sens-attaché:\textsc{acc.sg} montagne-roi-mot:\textsc{acc.sg} faire:\textsc{3pl:pres} crin.de.la.queue-éventail\textsc{:instr.pl} yak.\textsc{nom.pl} \\
\glt  `Where the yaks prove his title, the ``Lord of the mountains,"  to be significant by (waving over him) the chauries (in the shape of their tails) as white as the rays of the moon, and having their beauty extended all round by the movements of the tails.' (\ipa{Kumārasaṃbhava--} 1, 13, \citealt[204]{kale17kumarasambhava})
\end{exe}

La deux dernières syllabes de la forme de féminin \ipa{camarī--} ressemblent à la proto-forme *\ipa{məri} qui donne le tibétain  \ipa{ⁿbri} `yak femelle' d'après la loi de Simon (\citealt[187]{simon29}, \citealt{hill11laws}). Ce mot tibétain lui-même est un Wanderwort attesté dans les langues rgyalronguiques (japhug \ipa{qambrɯ} `yak mâle', mot qui ne correspond pas régulièrement du point de vue du vocalisme, avec le préfixe \ipa{qa--} de noms d'animaux, voir   \citealt[158-9]{jacques14snom}) et même en chinois, où  `yak' est désigné dans des textes d'époque Han par le caractère \zh{犛} auquel on donne deux lectures remontant au chinois archaïque *\ipa{mrˁə} et *\ipa{rə} (système de \citealt{bs14oc}).

La première syllabe du sanskrit \ipa{ca--} n'est problématique qu'en apparence: on peut montrer indépendamment, au moyen de la reconstruction interne que la plupart des groupes de consonnes à trois éléments du proto-tibétain ont été réduits à deux consonnes par la chute de la première (\citealt{coblin76}, \citealt{hill11laws}), et cette loi s'appliquerait à un groupe du type *\ipa{Cəməri} $\rightarrow$ *\ipa{Cmri} $\rightarrow$ *\ipa{mri} $\rightarrow$ \ipa{ⁿbri}. Il n'y a donc pas d'obstacle phonétique à cette comparaison; il faut admettre une forme pré-tibétaine plus complexe contenant peut-être un préfixe de classe. 
  
L'exotisme de cet animal du point de vue indien justifie qu'il puisse s'agir d'un emprunt à une langue himâlayenne. Le masculin \ipa{camara--}, dans cette hypothèse, serait formé à partir du féminin \ipa{camarī--} (où le \ipa{--i} final du mot tibétain aurait été ré-interprété comme le thème du féminin). Il n'est pas absurde que le nom de l'espèce dérive du féminin, car la femelle du yak est plus utile dans l'économie quotidienne, dans la mesure où elle produit du lait et engendre des yak hybrides (\ipa{mdzo}) qui servent pour le labour et le transport. La seule faiblesse de cette hypothèse est l'attestation légèrement plus tardive du féminin, mais étant donné le faible nombre d'occurrences dans le corpus épique de \ipa{camara--}, il est possible qu'il s'agisse d'un hasard.

Il n'est pas certain que la langue donneuse au sanskrit (et au chinois) ait été le tibétain directement, car l'extension géographique de cette langue devait se limiter à quelques vallées du haut-plateau avant l'empire tibétain. 
  
\bibliographystyle{unified}
\bibliography{bibliogj}
\end{document}
