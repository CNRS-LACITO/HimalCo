\documentclass[oldfontcommands,oneside,a4paper,11pt]{article} 
\usepackage{fontspec}
\usepackage{natbib}
\usepackage{booktabs}
\usepackage{xltxtra} 
\usepackage{polyglossia} 
\usepackage[table]{xcolor}

\usepackage{multicol}
\usepackage{graphicx}
\usepackage{lineno}
\usepackage{float}
\usepackage{hyperref} 
\hypersetup{bookmarks=false,bookmarksnumbered,bookmarksopenlevel=5,bookmarksdepth=5,xetex,colorlinks=true,linkcolor=blue,citecolor=blue}
\usepackage[all]{hypcap}
\usepackage{memhfixc}
\usepackage{lscape}
\usepackage{tikz}
%
\usetikzlibrary{trees}
\usepackage{gb4e} 
\bibpunct[: ]{(}{)}{,}{a}{}{,}
 
%\setmainfont[Mapping=tex-text,Numbers=OldStyle,Ligatures=Common]{Charis SIL}  
\newfontfamily\phon[Mapping=tex-text,Ligatures=Common,Scale=MatchLowercase,FakeSlant=0.3]{Charis SIL} 
\newcommand{\ipa}[1]{{\phon #1}} %API tjs en italique
 
\newcommand{\grise}[1]{\cellcolor{lightgray}\textbf{#1}}
\newfontfamily\cn[Mapping=tex-text,Ligatures=Common,Scale=MatchUppercase]{MingLiU}%pour le chinois
\newcommand{\zh}[1]{{\cn #1}}
   
\newfontfamily\mleccha[Mapping=tex-text,Ligatures=Common,Scale=MatchLowercase]{Arial Unicode MS}%{Galatia SIL}%pour le grec
\newcommand{\grec}[1]{{\mleccha #1}}


\begin{document} 
\title{Sanskrit \ipa{camara--} `yak' et tibétain \ipa{'bri} `yak femelle'}
\author{Guillaume Jacques}
\maketitle
L'étymologie du sanskrit \ipa{camara-} `yak' est restée jusqu'à présent obscure (\citealt[I, 375]{mayrhofer56kurz}: `nicht befriendigend erklärkt'). Mayrhofer cite la comparaison avec le grec \grec{κεμάς} `jeune cerf sans corne', qui est impossible puisque le cognat sanskrit de ce nom, \ipa{śáma-} `sans cornes' (\citealt[II, 289]{mayrhofer56kurz}), montre qu'il remonte à un radical *\ipa{ḱem--} avec une palatale qui ne pourrait en aucun cas donner un \ipa{c--} en sanskrit. En outre, cette comparaison est absurde du point de vue du sens, le yak étant une animal plutôt connu pour sa vigueur et ses cornes imposantes.

Principalement dans des listes
Bhāgavatapurāṇa, 3, 10, 22.1
kharo 'śvo 'śvataro gauraḥ śarabhaś camarī tathā 
ete caikaśaphāḥ kṣattaḥ śṛṇu pañcanakhān paśūn


KumSaṃ, 1, 13
lāṅgūlavikṣepavisarpiśobhair itas tataś candramarīcigauraiḥ /
yasyārthayuktaṃ girirājaśabdaṃ kurvanti vālavyajanaiś camaryaḥ // 
%204
13. Where the yaks prove his title, the "Lord of the 
mountains, " to be significant by (waving over him) the 
chauries (in the shape of their tails) as white as the rays of 
the moon, and having their beauty extended all round by 
the movements of the tails


\ipa{camara-}  \ipa{camarī-}
  nepali \ipa{cauṃrī}
  
  \ipa{ɴbri}
  
  \citet{jacques14snom}
  
\bibliographystyle{unified}
\bibliography{bibliogj}
\end{document}
