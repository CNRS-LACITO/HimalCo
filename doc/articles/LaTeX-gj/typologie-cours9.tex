\documentclass[oldfontcommands,twoside,a4paper,12pt]{article} 
\usepackage{fontspec}
\usepackage{natbib}
\usepackage{booktabs}
\usepackage{xltxtra} 
\usepackage{polyglossia} 
 \usepackage{geometry}
 \geometry{
 a4paper,
 total={210mm,297mm},
 left=10mm,
 right=10mm,
 top=15mm,
 bottom=15mm,
 }
\usepackage[table]{xcolor}
\usepackage{color}
\usepackage{multirow}
\usepackage{gb4e} 
\usepackage{multicol}
\usepackage{graphicx}
\usepackage{float}
\usepackage{hyperref} 
\hypersetup{bookmarks=false,bookmarksnumbered,bookmarksopenlevel=5,bookmarksdepth=5,xetex,colorlinks=true,linkcolor=blue,citecolor=blue}
\usepackage{memhfixc}
\usepackage{lscape}
\usepackage[footnotesize,bf]{caption}
 

%%%%%%%%%%%%%%%%%%%%%%%%%%%%%%%
\setmainfont[Mapping=tex-text,Numbers=OldStyle,Ligatures=Common]{Charis SIL} 
\setsansfont[Mapping=tex-text,Ligatures=Common,Mapping=tex-text,Ligatures=Common,Scale=MatchLowercase]{Lucida Sans Unicode} 
 


\newfontfamily\phon[Mapping=tex-text,Ligatures=Common,Scale=MatchLowercase,FakeSlant=0.3]{Charis SIL} 
\newfontfamily\phondroit[Mapping=tex-text,Ligatures=Common,Scale=MatchLowercase]{Doulos SIL} 
\newcommand{\ipa}[1]{{\phon\textbf{#1}}} 
\newcommand{\ipab}[1]{{\phon #1}}
\newcommand{\ipapl}[1]{{\phondroit #1}} 
\newcommand{\captionft}[1]{{\captionfont #1}} 
\newfontfamily\cn[Mapping=tex-text,Ligatures=Common,Scale=MatchUppercase]{SimSun}%pour le chinois
\newcommand{\zh}[1]{{\cn #1}}
\newcommand{\tgf}[1]{{\large\mo{#1}}}
\newcommand{\rc}{}
\newcommand{\tete}{}
\newcommand{\topic}{\textsc{top}}
\newcommand{\racine}[1]{\begin{math}\sqrt{#1}\end{math}} 
\newcommand{\grise}[1]{\cellcolor{lightgray}\textbf{#1}} 
\newcommand{\tinynb}[1]{\tiny#1}

\newcommand{\ro}{$\Sigma$}
\newcommand{\siga}{$\Sigma_1$} 
\newcommand{\sigc}{$\Sigma_3$}   
\begin{document}
Exemples de katoukina tirés de Queixalós (2011, 2012)

 \begin{exe}
\ex  
\gll  (piya) daan \\
homme partir \\
 \glt   L'homme est parti.
\end{exe}
 
 \begin{exe}
\ex  
\gll  daan (ityaro-na=tyo)  \\
 partir femme-??= fille \\
 \glt   La fille de la femme est partie.
\end{exe}

 \begin{exe}
\ex  
\gll  daan (yo-tyo)  \\
 partir \textsc{1sg.poss}-fille \\
 \glt   Ma fille est partie.
\end{exe}

 \begin{exe}
\ex  
\gll  pi:da-na= ti (paiko)   \\
jaguar-??= tuer grand.père \\
 \glt   Le jaguar a tué le grand-père.
\end{exe}

 \begin{exe}
\ex  
\gll yo-pu tu (barahai) \\
\textsc{1sg.poss}-manger \textsc{neg} gibier \\
 \glt   Je n'ai pas mangé le gibier.
\end{exe}

 \begin{exe}
\ex  
\gll wa-pu tu adu \\
???-manger \textsc{neg} \textsc{1sg} \\
 \glt   Je n'ai pas mangé.
\end{exe}

 \begin{exe}
\ex  
\gll  (piya) wa-pu (barahai) \\
homme ???-manger gibier \\
 \glt   L'homme a mangé le gibier.
\end{exe}

 \begin{exe}
\ex  
\gll  yo-wahak (barahai)  \\
\textsc{1sg.poss}-cuire gibier \\
 \glt   J'ai cuit le gibier.
\end{exe}

 \begin{exe}
\ex  
\gll  bak tu yo-wahak nyan  \\
bon \textsc{neg} \textsc{1sg.poss}-cuire \textsc{dem} \\
 \glt   Ce que j'ai cuit n'est pas bon.
\end{exe}
 \begin{exe}
\ex  
\gll  
yo-hik nyan waokdyi-nin anyan piya \\
\textsc{1sg.poss}-avoir \textsc{dem}   arriver-\textsc{dépendent} cet homme \\
\glt Je connais l'homme qui est arrivé.
 \end{exe}
 \begin{exe}
\ex  
\gll  
yo-hik nyan Nodia-na= dahudyi-nin tukuna\\
\textsc{1sg.poss}-avoir \textsc{dem} n.p.-??= amener-\textsc{dépendent} homme \\
\glt Je connais l'homme que Nodia a amené.
 \end{exe}
 
  \begin{exe}
\ex  
\gll  
yo-hik nyan piya wa-dahudyi-nin Hiowai \\
\textsc{1sg.poss}-avoir \textsc{dem} homme ???-amener-\textsc{dépendent} n.p. \\
\glt Je connais l'homme qui a amené Hiowai.
 \end{exe}
 
 
\end{document}

