\documentclass[oldfontcommands,oneside,a4paper,11pt]{article} 
\usepackage{fontspec}
\usepackage{natbib}
\usepackage{booktabs}
\usepackage{xltxtra} 
\usepackage{polyglossia} 
\usepackage[table]{xcolor}
\usepackage{lineno}
\usepackage{multicol}
\usepackage{graphicx}
\usepackage{float}
\usepackage{hyperref} 
\hypersetup{bookmarks=false,bookmarksnumbered,bookmarksopenlevel=5,bookmarksdepth=5,xetex,colorlinks=true,linkcolor=blue,citecolor=blue}
\usepackage[all]{hypcap}
\usepackage{memhfixc}
\usepackage{lscape}
\usepackage{tikz}
\usetikzlibrary{trees}
\usepackage{gb4e} 
\bibpunct[: ]{(}{)}{,}{a}{}{,}
 
%\setmainfont[Mapping=tex-text,Numbers=OldStyle,Ligatures=Common]{Charis SIL}  
\newfontfamily\phon[Mapping=tex-text,Ligatures=Common,Scale=MatchLowercase,FakeSlant=0.3]{Charis SIL} 
\newcommand{\ipa}[1]{{\phon #1}} %API tjs en italique
 
\newcommand{\grise}[1]{\cellcolor{lightgray}\textbf{#1}}
\newfontfamily\cn[Mapping=tex-text,Ligatures=Common,Scale=MatchUppercase]{MingLiU}%pour le chinois
\newcommand{\zh}[1]{{\cn #1}}
\newcommand{\tld}{\textasciitilde{}}

\begin{document} 
 \title{The origin of comitative adverbs in Japhug}
 \author{Guillaume Jacques}
 \maketitle 
 \linenumbers 

\section{Introduction}
This paper discusses the origin of comitative adverbs in Japhug and other Gyalrong languages. These adverbs, only attested in the core Gyalrong languages, are a relatively recent innovations in these languages, and provide an interesting case study to investigate the origin of comitative constructions in the world's languages.


The paper contains three sections.  First, I provide background information on the morphological expression of possession in Japhug nouns, which must be taken into consideration in all types of denominal derivations, including that of comitative adverbs. Second, I describe the syntactic morphological and syntactic properties of comitative adverbs. Third, I propose a grammaticalization hypothesis to account for their origin, involving comparison with the closely related Tshobdun language, and show that the pathway in question has not been previously proposed for comitative markers.

 \section{Inalienably possessed nouns} 
Japhug nouns can be divided into inalienably possessed nouns (IPN) and non-inalienably possessed nouns (NIPN). IPN differ from NIPN in that they require the presence of one of the possessive prefixes (Table \ref{tab:possessive}), while NIPN can appear as their bare stem without any possessive prefix. The IPN / NIPN distinction is not completely predictable: although all body parts and kinship terms are IPN, we also find some (but not all) clothes, implements and abstract notions (like thought, debt, sleep etc).

When no definite possessor is present, IPN take one of the indefinite possessive prefixes \ipa{tɤ--} or \ipa{tɯ--}. This form is used as their citation form (\ipa{tɤ-lu} `milk', \ipa{tɯ-ŋga} `clothes', \ipa{tɤ-rpɯ} `uncle', \ipa{tɯ-ci} `water'). The distribution of the prefixes \ipa{tɤ--} vs \ipa{tɯ--} is lexically determined.  When a specific possessor is present, the indefinite prefix is replaced by the appropriate possessive prefix (\ipa{ɯ-lu} `her/its milk (from her nipple)', \ipa{a-ŋga} `my clothes', \ipa{nɤ-rpɯ} `your uncle', \ipa{ɯ-ci} `its juice'). 

Although the generic possessive prefix \ipa{tɯ--} is homophonous with one of the indefinite possessive prefixes, the two are semantically distinct (compare \ipa{tɤ-se} \textsc{indef.poss}-blood `blood' with \ipa{tɯ-se} \textsc{genr.poss}-blood `one's/people's blood').

\begin{table}[h] \centering
\caption{Possessive prefixes }\label{tab:possessive}
\begin{tabular}{lllllllll} 
\toprule
 Prefix & Person\\
\midrule
\ipa{a--}  & 1\textsc{sg} \\
\ipa{nɤ--}  & 2\textsc{sg}\\
\ipa{ɯ--}  & 3\textsc{sg}\\
\midrule
\ipa{tɕi--}  &  1\textsc{du} \\
\ipa{ndʑi--}  & 2\textsc{du} \\	
\ipa{ndʑi--}  & 3\textsc{du} \\	
\midrule
\ipa{i--}  & 1\textsc{pl} \\
\ipa{nɯ--}  & 2\textsc{pl} \\
\ipa{nɯ--}  & 3\textsc{pl} \\
\midrule
\ipa{tɯ--},  \ipa{tɤ--} & indefinite \\
\ipa{tɯ--}   &  generic \\
\bottomrule
\end{tabular}
\end{table}

It is possible to turn an IPN into an NIPN by prefixing a definite possessive prefix to the indefinite one, as in \ipa{ɯ-tɤ-lu} \textsc{3sg.poss-indef.poss}-milk `his milk (to drink)', \ipa{ɯ-tɯ-ci} \textsc{3sg.poss-indef.poss}-water `its water (of irrigated water, to a plant)'. NIPN cannot take indefinite possessive prefixes. However, they are compatible with the human generic possessor prefix \ipa{tɯ--}, as in example \ref{ex:tWlaXtCha}, where the nouns \ipa{kha} `house' and \ipa{laχtɕha} `thing' are NIPN.

\begin{exe}
\ex \label{ex:tWlaXtCha}
\gll  \ipa{wuma}  	\ipa{ʑo}  	\ipa{tɯ-kha}  	\ipa{cho}  	\ipa{tɯ-laχtɕha}  	\ipa{ra}  	\ipa{sɯ-ɴqhi}  \\
really \textsc{emph} \textsc{genr.poss}-house and \textsc{genr.poss}-thing \textsc{pl} \textsc{caus}-be.dirty:\textsc{fact} \\
\glt (Flies) make one's house and one's things dirty. (25 akWzgumba, 62)
\end{exe}


 
 \section{Comitative derivation} 
In Japhug, adverbs meaning `having X' or `together with X' can be productively built from various types of nouns.\footnote{Comitative adverbs in Japhug have been briefly mentioned in a previous publication (\citealt[51]{jacques08}), but this paper is the first detailed description of this derivation and its uses.} In this section, I first describe the morphological processes involved in the noun to adverb derivation, and then provide an overview of the use of these adverbs in context.

\subsection{Morphology}
Comitative adverbs are formed by reduplicating the last syllable of the noun stem and prefixing either \ipa{kɤ́--} or \ipa{kɤɣɯ--}, as in examples such as \ipa{χɕɤlmɯɣ} `glasses' $\Rightarrow$ \ipa{kɤ́-χɕɤlmɯ\tld{}lmɯɣ} / \ipa{kɤɣɯ-χɕɤlmɯ\tld{}lmɯɣ} `together with glasses'.\footnote{Japhug \ipa{χɕɤlmɯɣ} `glasses' is a loanword from Tibetan \ipa{ɕel.mig}; note that reduplication disregards morpheme boundaries (\ipa{χɕɤl} `glass' (Tibetan \ipa{ɕel}) is also attested in Japhug). } No semantic difference between the comitative adverbs in \ipa{kɤ́--} and those in \ipa{kɤɣɯ--} has been detected; both are fully productive and can be built from the same nouns.

When the base noun is an IPN, it is possible to build a comitative adverb with the indefinite possessor prefix or with the bare stem. For instance, from \ipa{tɤ-rte} `hat' one can derive both \ipa{kɤ́-rtɯ\tld{}rte} / \ipa{kɤɣɯ-rtɯ\tld{}rte} `with his/her hat' and \ipa{kɤ́-tɤ-rtɯ\tld{}rte} /  \ipa{kɤɣɯ-tɤ-rtɯ\tld{}rte} `with a hat' with the indefinite possessor prefix \ipa{tɤ--}. The inalienable/non-alienable distinction is present in these forms: \ipa{kɤ́-rtɯ\tld{}rte} / \ipa{kɤɣɯ-rtɯ\tld{}rte} means `wearing one's hat' (example \ref{ex:kAGWrtWrte}), while \ipa{kɤ́-tɤ-rtɯ\tld{}rte} /  \ipa{kɤɣɯ-tɤ-rtɯ\tld{}rte} implies that one is not wearing the hat (\ref{ex:kAGWtArtWrte}).



\begin{exe}
\ex \label{ex:kAGWrtWrte}
\gll \ipa{kɤɣɯ-rtɯ\tld{}rte} 	\ipa{ʑo} 	\ipa{kha} 	\ipa{ɯ-ŋgɯ} 	\ipa{lɤ-tɯ-ɣe} 	\\
\textsc{comit}-hat \textsc{emph} house \textsc{3sg}-inside \textsc{pfv}-2-come[II] \\
\glt You came inside the house with your hat. (You were expected to take it off before coming in)
\end{exe}

\begin{exe}
\ex \label{ex:kAGWtArtWrte}
\gll  \ipa{laχtɕha} 	\ipa{kɤɣɯ-tɤ-rtɯ\tld{}rte} 	\ipa{ʑo} 	\ipa{ta-ndo}  \\
thing \textsc{comit-indef.poss}-hat \textsc{emph} \textsc{pfv}:3$\rightarrow$3'-take \\
\glt He took the things together with the hat. (Not wearing it)
\end{exe}

Cognates of the Japhug comitative adverbs have been reported in other Gyalrong languages, in particular Tshobdun \ipa{ko--} (\citealt[107]{jackson98morphology}), and the comitative adverb derivation can thus be reconstructed back at least to proto-Gyalrong. However, given the dearth of data on languages other than Japhug (in particular in terms of text examples), little external data will be discussed in this paper. A full comparative assessment of the hypotheses laid out here will have to wait the publication of fully-fledged grammatical descriptions of all Gyalrong languages. 

Comitative adverbs, in any case, appear to be unattested outside of core Gyalrong languages (even in Khroskyabs, their closest relative, see \citealt{lai13affixale}), and is probably one of the many common Gyalrong  morphological innovations.

\subsection{Syntactic uses} 

The comitative adverb can either follow (\ref{ex:kAjWjaR}) or precede (\ref{ex:kArnWrna}, \ref{ex:kAthAlwWlwa}) the noun over which it has scope. Alternatively, a comitative adverb can occur without a corresponding overt noun (\ref{ex:kAsnWsno}). However, if the noun is overt, the comitative adverb is contiguous to the NP to which it belongs. 

The NP in question can either correspond to the P (\ref{ex:kAjWjaR}, \ref{ex:kAthAlwWlwa}), the S (\ref{ex:kArnWrna}) or even the A (\ref{ex:kArJWrJit.kW}). This last option is not attested in the text corpus, but speakers have no trouble producing sentences of this type.

\begin{exe}
\ex \label{ex:kAjWjaR}
\gll
\ipa{tɤ-sno}  	\ipa{kɤ́-jɯ\tld{}jaʁ}  	\ipa{nɯ}  	\ipa{lu-ta-nɯ}  \\
\textsc{indef.poss}-saddle \textsc{comit}-hand \textsc{dem} \textsc{ipfv}-put-\textsc{pl} \\
\glt (Then), they put the saddle with its handles.
\end{exe}

\begin{exe}
\ex \label{ex:kArnWrna}
\gll
\ipa{pɣɤkhɯ}  	\ipa{nɯ}  	\ipa{ɯ-ku}  	\ipa{nɯnɯ}  	\ipa{lɯlu}  	\ipa{tsa}  	\ipa{ɲɯ-fse,}  	\ipa{ɯ-mtsioʁ}  	\ipa{ɣɤʑu}  	\ipa{ma}  \ipa{kɤ́-rnɯ\tld{}rna}  	\ipa{lɯlu}  	\ipa{tu-fse}  	\ipa{ɲɯ-sɤre}  	\ipa{ʑo.}  \\
owl \textsc{dem} \textsc{3sg.poss}-head \textsc{dem} cat a.little \textsc{sens}-be.like \textsc{3sg.poss}-beak exist:\textsc{sens} a.part.from \textsc{comit}-ear cat \textsc{ipfv}-be.like \textsc{sens}-be.extremely/be.funny \textsc{emph} \\
\glt The owl's head looks a little like that of a cat, apart from the fact that it has a beak, it looks very much like a cat with its ears.
\end{exe}


\begin{exe}
\ex \label{ex:kAthAlwWlwa}
\gll \ipa{kɤ́-thɤlwɯ\tld{}lwa}  	\ipa{ɯ-zrɤm}  	\ipa{ra}  	\ipa{kɯnɤ}  	\ipa{chɯ́-wɣ-ɣɯt}  	\ipa{pjɯ́-wɣ-ji}  	\ipa{ri}  	\ipa{maka}  	\ipa{tu-ɬoʁ}  	\ipa{mɯ́j-cha}  \\
\textsc{comit}-earth \textsc{3sg.poss}-root \textsc{pl} also \textsc{ipfv-inv}-bring \textsc{ipfv-inv}-plant but at.all \textsc{ipfv}-come.out \textsc{neg:sens}-can \\
\glt Even if one takes its root with earth (around it) and plant it, it cannot grow.
\end{exe}


\begin{exe}
\ex \label{ex:kAsnWsno}
\gll \ipa{kɤ́-snɯ\tld{}sno}  	\ipa{ʑo}  	\ipa{kɤ-rŋgɯ}  \\
\textsc{comit}-saddle \textsc{emph} \textsc{pfv}-lie.down \\
\glt (The horse) slept with its saddle. (elicited)
\end{exe}

\begin{exe}
\ex \label{ex:kArJWrJit.kW}
\gll 
\ipa{lɯlu} 	\ipa{kɤ́-rɟɯ\tld{}rɟit} 	\ipa{ra} 	\ipa{kɯ} 	\ipa{ʑo} 	\ipa{βʑɯ} 	\ipa{to-ndza-nɯ.} \\
cat \textsc{comit}-offspring \textsc{pl}  \textsc{erg} \textsc{emph} mouse \textsc{ifr}-eat-\textsc{pl} \\
\glt The cat and its young ate the mouse. (elicited)
\end{exe}

Nouns incorporated into comitative adverbs lose their nominal status and cannot be determined by relative clauses (including attributive adjectives), numerals or demonstratives. In a sentence such as \ref{ex:kAGWNkhWNkhor} for instance, the attributive participial relative [\ipa{kɯ\tld{}kɯ-ŋɤn}] `all the ones who are evil' does not determine \ipa{kɤɣɯ-ŋkhɯ\tld{}ŋkhor} `with his subjects', a syntactic structure which would entail the translation `with all his evil subjects'. Rather, it determines the head noun together with the comitative adverb  \ipa{rɟɤlpu} \ipa{kɤɣɯ-ŋkhɯ\tld{}ŋkhor} `the king with his subjects', which implies the translation given below.

\begin{exe}
\ex \label{ex:kAGWNkhWNkhor}
\gll \ipa{rɟɤlpu}  	\ipa{kɤɣɯ-ŋkhɯ\tld{}ŋkhor}  	[\ipa{kɯ\tld{}kɯ-ŋɤn}]  	\ipa{ʑo}  	\ipa{to-ndo}  	\ipa{tɕe,}  	\ipa{tɕendɤre}  	\ipa{kɯ-mɤku}  	\ipa{nɯ}  	\ipa{sɤtɕha}  	\ipa{kɯ\tld{}kɯ-sɤ-scit}  	\ipa{ʑo}  	\ipa{jo-tsɯm}  	\ipa{ɲɯ-ŋu}  	\ipa{ri}  	\ipa{kɯ-maqhu}  	\ipa{tɕe,}  	\ipa{kɯ\tld{}kɯ-sɤɣ-mu}  	\ipa{ʑo}  	\ipa{jo-tsɯm}  	\ipa{tɕe}  \\
king \textsc{comit}-subjects \textsc{total}\tld{}\textsc{nmlz}:S/A-be.bad \textsc{emph} \textsc{ifr}-take \textsc{lnk}  \textsc{lnk} \textsc{nmlz}:S/A-be.before \textsc{dem} place \textsc{total}\tld{}\textsc{nmlz}:S/A-\textsc{deexp}-be.happy \textsc{emph} \textsc{ifr}-take.away \textsc{sens}-be \textsc{lnk} \textsc{nmlz}:S/A-be.after \textsc{lnk} \textsc{total}\tld{}\textsc{nmlz}:S/A-\textsc{deexp}-fear \textsc{emph} \textsc{ifr}-take.away \textsc{lnk} \\
\glt She took the king and his subjects, all the evil ones, in the beginning she took them to nice places, but later she took them to fearful places. (slobdpon)
\end{exe}


\section{Grammaticalization pathway} 
In this section, I first present the proprietive denominal derivation in \ipa{aɣɯ--} and the infinitival and participial prefix \ipa{kɯ--}. Then, I show that comitative adverbs are actually synchronically formally ambiguous with the infinive of proprietive denominal verbs in some contexts. Finally, I propose that comitative adverbs diachronically derive from the infinival of participial forms of proprietive denominal verbs, and was then extended to other contexts after reanalysis.


\subsection{Denominal derivation}
Japhug has a rich array of denominal prefixes (\citealt{jacques14antipassive}). One of these prefixes, \ipa{aɣɯ--}, derives stative intransitive verbs from either inalienably possessed and non-inalienably possessed nouns. As illustrated by the examples in Table \ref{tab:aGW}, verbs derived with the prefix have meaning such as `having X', `producing a lot of X' or `having the same X ' (with plural S). The noun stem is sometimes reduplicated, especially for the first type of meaning.
 
\begin{table}[h] \centering
\caption{The denominal prefix \ipa{aɣɯ--}}\label{tab:aGW}
\resizebox{\columnwidth}{!}{
\begin{tabular}{lllllllll} 
\toprule
Base noun & Meaning & Denominal verb & Meaning\\
\midrule
\ipa{tɯ-ɣli} & excrement, manure & \ipa{aɣɯ-ɣli} & producing a lot of manure (of pigs) \\
\ipa{tɤ-lu} & milk &\ipa{aɣɯ-lu} & producing a lot of milk (of cows) \\
\ipa{tɯ-mɲaʁ} & eye & \ipa{aɣɯ-mɲaʁ} & having a lot of holes \\
\ipa{tɯ-ɕnaβ} & snot & \ipa{aɣɯ-ɕnɯ\tld{}ɕnaβ} & be  slimy \\
\ipa{ɯ-mdoʁ} & colour & \ipa{aɣɯ-mdoʁ} & having the same colour \\
\ipa{tɯ-sɯm} & thought & \ipa{aɣɯ-sɯm} & get along well \\
\ipa{smɤn} & medicine &  \ipa{aɣɯ-smɤn} & have a medical effect \\
\ipa{tɯ-ɕna} & nose &  \ipa{aɣɯ-ɕnɯ\tld{}ɕna} & having a keen sense of smell \\
\bottomrule
\end{tabular}}
\end{table}

In some cases, the semantic relationship between the base noun and the the derived verb is more metaphorical and not predictable. For instance, from the noun \ipa{tɯ-jaʁ} `hand' one can derive either \ipa{aɣɯ-jɯ\tld{}jaʁ} `having a lot of hands' (of a bug), while the non-reduplicated form \ipa{aɣɯ-jaʁ} means `who steals anything (that comes near his hand)'.

\subsection{S/A participle and infinitive}

In Japhug, stative verbs (including the denominal verbs in \ipa{aɣɯ--} presented in the previous section) have two homophonous non-finite forms with a prefix \ipa{kɯ--}, the S/A-participle (\citealt[5]{jacques14antipassive}) and the infinitive.\footnote{The morphological evidence for distinguishing between participle and infinitive is clearer with dynamic verbs, whose infinitive in \ipa{kɤ--} differ from the S/A-participle. Cognates of the participle and the infitive \ipa{kɯ--} are found in all Gyalrong languages, with only milde differences (see in particular \citealt{sun14generic}).} The participle appears mainly in participial relatives (including all forms corresponding to attributive adjectives in European languages), as in example (\ref{ex:kWpWpe}).

\begin{exe}
\ex \label{ex:kWpWpe}
\gll
\ipa{tɕheme} 	\ipa{ci} 	\ipa{kɯ-pɯ\tld{}pe} 	\ipa{kɯ-mpɕɯ\tld{}mpɕɤr,} 	\ipa{nɤ-ɕɣa} 	\ipa{kɯ-xtɕɯ\tld{}xtɕi} 	\ipa{ʑo} 	\ipa{a-nɯ-tɯ-ɤβzu} 	\ipa{smɯlɤm}  \\
girl a \textsc{nmlz:S/A-emph}\tld{}-be.good \textsc{nmlz:S/A-emph}\tld{}-be.beautiful  \textsc{2sg.poss}-tooth \textsc{nmlz:S/A-emph}\tld{}-be.small \textsc{emph} \textsc{irr-pfv}-2-become prayer \\
\glt May you become a nice and beautiful girl with short teeth. (Slobdpon, 261)
\end{exe}

The infinitive is used (by some speakers) as the citation form of verbs, and appears in some types of complement clauses and manner subordinate clauses (\citealt[271-2; 321-5]{jacques14linking}), as in \ref{ex:kWpWpe2} where \ipa{kɯ-pɯ\tld{}pe}, meaning here `nicely', is a manner subordinate clause comprising a single verb.
 \begin{exe}
\ex \label{ex:kWpWpe2}
\gll \ipa{ɕɤr} 	\ipa{tɕe} 	\ipa{ʁzɤmi} 	\ipa{ni} 	\ipa{kɯ-pɯ\tld{}pe} 	\ipa{ʑo} 	\ipa{ɕ-ko-nɯ-rŋgɯ-ndʑi}  \\
evening \textsc{lnk} husband.and.wife \textsc{du} \textsc{inf:stat-emph}\tld{}good \textsc{emph} \textsc{transl-ifr-auto}-lie.down-\textsc{du} \\
\glt In the evening, the husband and the wife lay down in bed nicely.
\end{exe}

%  \begin{exe}
%\ex \label{ex:kWxtCi}
%\gll  \ipa{tʂu} 	\ipa{kɯ-xtɕi} 	\ipa{nɯ} 	\ipa{tɕu} 	\ipa{ʑo} 	\ipa{jo-ɕe} \\
%path \textsc{nmlz}:S/A-be.small \textsc{dem} \textsc{loc} \textsc{emph} \textsc{ifr}-go \\
%\glt He went by the small path. (the fox 03, 66)
%\end{exe}

\subsection{Potential ambiguity}
Due to the sandhi rule according to which \ipa{kɯ--} combined with \ipa{a--} initial verbs yields /\ipa{kɤ--}/ in Japhug (\citealt{jacques04these}), S/A-participles or infinitive forms of denominal verbs in \ipa{aɣɯ--} are formally homophonous with comitative adverbs in \ipa{kɤɣɯ--}. For example, the form \ipa{kɤɣɯrtɯrtaʁ}  `together with its branches' from \ipa{tɤ-rtaʁ} `branch' is identical to the participle \ipa{kɤɣɯrtɯrtaʁ} `the one which has many branches' found in example (\ref{ex:kAGWrtWrtaR}).

  \begin{exe}
\ex \label{ex:kAGWrtWrtaR}
\gll   
  \ipa{si} 	\ipa{kɯ-ɤɣɯrtɯrtaʁ} 	\ipa{ki} 	\ipa{kɯ-fse} 	\ipa{ɲɯ-ɕar-nɯ} \\
  tree \textsc{nmlz}:S/A-have.many.branches this \textsc{nmlz}:S/A-be.this.way \textsc{ipfv}-search-\textsc{pl} \\
\glt They search for a tree having a lot of branches like this. (NOT: `a tree with its branches' in this particular context)
\end{exe}

Examples \ref{ex:kAGWrJWrJit} and \ref{ex:kAGWrJWrJit2} present a minimal pair contrasting the comitative adverb  `with his/her children' on the one hand and the participle  `having many children' on the other hand (both derived from the possessed noun  \ipa{tɤ-rɟit} `child').

\begin{exe}
\ex \label{ex:kAGWrJWrJit}
\gll   
\ipa{iɕqha} 	\ipa{tɕʰeme} 	\ipa{nɯ} 	\ipa{kɯ-ɤɣɯrɟɯrɟit} 	\ipa{ci} 	\ipa{pɯ-ŋu}  \\
the.aforementioned woman \textsc{dem} \textsc{nmlz}:S/A-have.many.children \textsc{indef} \textsc{pst.ipfv}-be \\
\glt This woman had a lot of children.
\end{exe}

\begin{exe}
\ex \label{ex:kAGWrJWrJit2}
\gll   
\ipa{kɤɣɯ-rɟɯ\tld{}rɟit} 	\ipa{ʑo} 	\ipa{jo-nɯ-ɕe-nɯ} \\
\textsc{comit}-children \textsc{emph} \textsc{ifr-vert}-go-\textsc{pl} \\
\glt She/They went back with their children.
\end{exe}



\subsection{Grammaticalization pathway}
The formal ambiguity between the comitative adverbs on the one hand, and the participles and infinitives of \ipa{aɣɯ--} denominal verbs on the other hand, together with the semantic proximity of the two forms, raise the question of their potential historical relatedness.

An obvious possibility is that comitative adverb originate from the reanalysis of infinitival forms of reduplicated \ipa{aɣɯ--} denominal verbs. Ambiguous sentences like \ref{ex:kAGWrtWrtaR} actually constitute the pivot constructions which allow reanalysis in contexts where both proprietive (`having X') and a comitative (`with X') interpretations were possible.


  \begin{exe}
\ex \label{ex:kAGWrtWrtaR2}
\gll   
  \ipa{si} 	\ipa{kɤɣɯrtɯrtaʁ} \ipa{ɲɯ-ɕar-nɯ} \\
  tree \textsc{nmlz}:S/A-have.many.branches//\textsc{comit}-branch \textsc{ipfv}-search-\textsc{pl} \\
\glt `They search for a tree having a lot of branches' $\Rightarrow$ `They search for a tree and/with its branches'
\end{exe}

Starting from such ambiguous sentences, the comitative adverb was extended to nouns without a corresponding proprietive denominal verb. In addition, comitative adverbs incorporating the indefinite possessive prefix were created (such as \ipa{kɤɣɯ-tɤ-rtɯ\tld{}rte} `with his hat'). Forms of this type are clearly distinct from infinitives or participles of denominal verbs, as indefinite possessive prefixes are always deleted during  denominal derivation.


I therefore propose the pathway (\ref{kAGW-pathway}) to account for comitative adverbs in \ipa{kɤɣɯ--} in Japhug:

 \begin{exe}
\ex \label{kAGW-pathway}
 \glt  \textsc{noun} + \textsc{property denominal derivation} + infinitive/participle $\rightarrow$ \textsc{comitative}
\end{exe} 

This pathway differs from other attested origins of comitative markers, which include nouns meaning `comrade' or verbs such as `follow' and `take'(\citealt[91, 139, 287]{heine-kuteva02}). The core semantic shift is a reanalysis from proprietive to comitative, and itis conceivable that such a shift took place in other languages without the additional denominal and infinitival markers found in Japhug.


The pathway presented above accounts well for the \ipa{kɤɣɯ--} type comitative adverbs, but does not explain the \ipa{kɤ́--} variant, which is actually more common in the corpus.

Comitative adverbs in \ipa{kɤ́--} are anomalous in Japhug in being among the very few prefixes attracting stress, a feature that could indicate fusion of two syllables (for instance the negative sensory marker \ipa{mɯ́j-} probably historically results from the fusion of the negative \ipa{mɯ--} and the sensory prefix \ipa{ɲɯ--}).

If the sound laws of Japhug (\citealt{jacques04these}) are applied backwards, the prefix \ipa{kɤɣɯ--} would go back to pre-Japhug *\ipa{kɐwə--}. We know that in Tshobdun, *\ipa{wə} changes to \ipa{o}. It is in particular the case of the inverse prefix \ipa{o--} (\citealt{jackson02rentongdengdi}) which originates from proto-Gyalrong *\ipa{wə}). Through vowel fusion (which also occurs with the inverse prefix), \ipa{ko--}, the actual form of the comitative prefix (\citealt[107]{jackson98morphology}), is the expected outcome of *\ipa{kɐwə--}. We can therefore safely conclude that (1) the comitative prefixes \ipa{kɤɣɯ--} in Japhug and \ipa{ko--} in Tshobdun are cognate and (2) that the grammaticalization in \ref{kAGW-pathway} took place before the split of Japhug and Tshobdun, and show be reconstructed back at least to their common ancestor.

The comitative prefix \ipa{kɤ́--} in Japhug, on the other hand, makes no sense from a Japhug-internal perspective. A possible way to explain it however, it to suppose \textit{borrowing} from Tshobdun \ipa{ko--}. Japhug, and especially the Kamnyu variety described in the present paper, has borrowed a few nouns from Tshobdun, as shown by forms such as \ipa{qro} `ant', \ipa{qaliaʁ} `eagle' and \ipa{tɯɟo} `demon' instead of expected *\ipa{qroʁ}, *\ipa{qarɟaʁ} (attested in some dialects of Japhug) and *\ipa{tɯʑu}, following the sound laws set out in \citet{jacques04these}.

Borrowing of Tshobdun \ipa{ko--} as Japhug \ipa{kɤ́--} is not surprising phonologically. The stress on the prefix in Japhug is probably a trace of the stress on that prefix in pre-Tshobdun, lost due to the strong tendency of Gyalrong languages to stress the final or penultimate syllable (\citealt{jackson05yingao}). The vowel \ipa{ɤ} rather than \ipa{o} is a consequence of the fact that derivational prefixes in Japhug have strong phonolotactic constrainst: the only possible vowels are either \ipa{ɤ} or \ipa{ɯ} (and \ipa{a}, but only in the case of stem-initial \ipa{a--}).

The borrowing hypothesis also accounts for the complete homonymy of the two comitative prefixes in Japhug.

\section{Conclusion} 
The contribution of this paper is threefold. First, it provides the first detailed description of comitative adverbs in any Gyalrong language. Second, it documents a hitherto unknown origin for comitative markers. Third, it shows that language contact between Gyalrong languages is not restricted to the lexicon, but actually also involves clear cases of borrowing of grammatical morphemes.

\bibliographystyle{unified}
\bibliography{bibliogj}
\end{document}
