\documentclass[oldfontcommands,oneside,a4paper,11pt]{article} 
\usepackage{fontspec}
\usepackage{natbib}
\usepackage{booktabs}
\usepackage{xltxtra} 
\usepackage{polyglossia} 
\usepackage[table]{xcolor}
\usepackage{lineno}
\usepackage{multicol}
\usepackage{graphicx}
\usepackage{float}
\usepackage{hyperref} 
\hypersetup{bookmarks=false,bookmarksnumbered,bookmarksopenlevel=5,bookmarksdepth=5,xetex,colorlinks=true,linkcolor=blue,citecolor=blue}
\usepackage[all]{hypcap}
\usepackage{memhfixc}
\usepackage{lscape}
\usepackage{tikz}
\usetikzlibrary{trees}
\usepackage{gb4e} 
\bibpunct[: ]{(}{)}{,}{a}{}{,}
 
%\setmainfont[Mapping=tex-text,Numbers=OldStyle,Ligatures=Common]{Charis SIL}  
\newfontfamily\phon[Mapping=tex-text,Ligatures=Common,Scale=MatchLowercase,FakeSlant=0.3]{Charis SIL} 
\newcommand{\ipa}[1]{{\phon #1}} %API tjs en italique
 
\newcommand{\grise}[1]{\cellcolor{lightgray}\textbf{#1}}
\newfontfamily\cn[Mapping=tex-text,Ligatures=Common,Scale=MatchUppercase]{MingLiU}%pour le chinois
\newcommand{\zh}[1]{{\cn #1}}
\newcommand{\tld}{\textasciitilde{}}

\begin{document} 
 \title{The origin of comitative adverbs in Japhug}
 \author{Guillaume Jacques}
 \maketitle  
 \section{Introduction}
\linenumbers

 \section{Inalienably possessed nouns} 
Japhug nouns can be divided into inalienably possessed nouns (IPN) and non-inalienably possessed nouns (NIPN). IPN differ from NIPN in that they require the presence of one of the possessive prefixes (Table \ref{tab:possessive}), while NIPN do not.

When no definite possessor is present, IPN take one of the indefinite possessive prefixes \ipa{tɤ--} or \ipa{tɯ--}. This form is used as their citation form (\ipa{tɤ-lu} `milk', \ipa{tɯ-ŋga} `clothes', \ipa{tɤ-rpɯ} `uncle', \ipa{tɯ-ci} `water'). The distribution of the prefixes \ipa{tɤ--} vs \ipa{tɯ--} is lexically determined.  When a specific possessor is present, the indefinite prefix is replaced by the appropriate possessive prefix (\ipa{ɯ-lu} `her/its milk (from her nipple)', \ipa{a-ŋga} `my clothes', \ipa{nɤ-rpɯ} `your uncle', \ipa{ɯ-ci} `its juice'). 

Although the generic possessive prefix \ipa{tɯ--} is homophonous with one of the indefinite possessive prefixes, the two are semantically distinct (compare \ipa{tɤ-se} \textsc{indef.poss}-blood `blood' with \ipa{tɯ-se} \textsc{genr.poss}-blood `one's/people's blood').

\begin{table}[H] \centering
\caption{Possessive prefixes }\label{tab:possessive}
\begin{tabular}{lllllllll} 
\toprule
 Prefix & Person\\
\midrule
\ipa{a--}  & 1\textsc{sg} \\
\ipa{nɤ--}  & 2\textsc{sg}\\
\ipa{ɯ--}  & 3\textsc{sg}\\
\midrule
\ipa{tɕi--}  &  1\textsc{du} \\
\ipa{ndʑi--}  & 2\textsc{du} \\	
\ipa{ndʑi--}  & 3\textsc{du} \\	
\midrule
\ipa{i--}  & 1\textsc{pl} \\
\ipa{nɯ--}  & 2\textsc{pl} \\
\ipa{nɯ--}  & 3\textsc{pl} \\
\midrule
\ipa{tɯ--},  \ipa{tɤ--} & indefinite \\
\ipa{tɯ--}   &  generic \\
\bottomrule
\end{tabular}
\end{table}

It is possible to turn an IPN into an NIPN by prefixing a definite possessive prefix to the indefinite one, as in \ipa{ɯ-tɤ-lu} \textsc{3sg.poss-indef.poss}-milk `his milk (to drink)', \ipa{ɯ-tɯ-ci} \textsc{3sg.poss-indef.poss}-water `its water (of irrigated water, to a plant)'. NIPN cannot take indefinite possessive prefixes. However, they are compatible with the human generic possessor prefix \ipa{tɯ--}, as in example \ref{ex:tWlaXtCha}, where the nouns \ipa{kha} `house' and \ipa{laχtɕha} `thing' are NIPN.

\begin{exe}
\ex \label{ex:tWlaXtCha}
\gll  \ipa{wuma}  	\ipa{ʑo}  	\ipa{tɯ-kha}  	\ipa{cho}  	\ipa{tɯ-laχtɕha}  	\ipa{ra}  	\ipa{sɯ-ɴqhi}  \\
really \textsc{emph} \textsc{genr.poss}-house and \textsc{genr.poss}-thing \textsc{pl} \textsc{caus}-be.dirty:\textsc{fact} \\
\glt (Flies) make one's house and one's things dirty. (25 akWzgumba, 62)
\end{exe}


 
 \section{Comitative derivation} 
In Japhug, adverbs meaning `having X' or `together with X' can be productively built from various types of nouns. In this section, I first describe the morphological processes involved in the noun to adverb derivation, and then provide an overview of the use of these adverbs in context.

\subsection{Morphology}
Comitative are formed by reduplicating the last syllable of the noun stem and prefixing either \ipa{kɤ́--} or \ipa{kɤɣɯ--}, as in examples such as \ipa{χɕɤlmɯɣ} `glasses' $\Rightarrow$ \ipa{kɤ́-χɕɤlmɯ\tld{}lmɯɣ} / \ipa{kɤɣɯ-χɕɤlmɯ\tld{}lmɯɣ} `together with glasses'.\footnote{Japhug \ipa{χɕɤlmɯɣ} `glasses' is a loanword from Tibetan \ipa{ɕel.mig}; note that reduplication disregards morpheme boundaries (\ipa{χɕɤl} `glass' (Tibetan \ipa{ɕel}) is also attested in Japhug). } No semantic difference between the comitative adverbs in \ipa{kɤ́--} and those in \ipa{kɤɣɯ--} has been detected; both are fully productive and can be built from the same nouns.

When the base noun is an IPN, it is possible to build a comitative adverb with the indefinite possessor prefix or with the bare stem, 

\subsection{Syntactic uses} 

P or S:

\begin{exe}
\ex
\gll \ipa{rɟɤlpu}  	\ipa{kɤɣɯ-ŋkhɯ\tld{}ŋkhor}  	\ipa{kɯ\tld{}kɯ-ŋɤn}  	\ipa{ʑo}  	\ipa{to-ndo}  	\ipa{tɕe,}  	\ipa{tɕendɤre}  	\ipa{kɯ-mɤku}  	\ipa{nɯ}  	\ipa{sɤtɕha}  	\ipa{kɯ\tld{}kɯ-sɤ-scit}  	\ipa{ʑo}  	\ipa{jo-tsɯm}  	\ipa{ɲɯ-ŋu}  	\ipa{ri}  	\ipa{kɯ-maqhu}  	\ipa{tɕe,}  	\ipa{kɯ\tld{}kɯ-sɤɣ-mu}  	\ipa{ʑo}  	\ipa{jo-tsɯm}  	\ipa{tɕe}  \\
king \textsc{comit}-subjects \textsc{total}\tld{}\textsc{nmlz}:S/A-be.bad \textsc{emph} \textsc{ifr}-take \textsc{lnk}  \textsc{lnk} \textsc{nmlz}:S/A-be.before \textsc{dem} place \textsc{total}\tld{}\textsc{nmlz}:S/A-\textsc{deexp}-be.happy \textsc{emph} \textsc{ifr}-take.away \textsc{sens}-be \textsc{lnk} \textsc{nmlz}:S/A-be.after \textsc{lnk} \textsc{total}\tld{}\textsc{nmlz}:S/A-\textsc{deexp}-fear \textsc{emph} \textsc{ifr}-take.away \textsc{lnk} \\
\glt She took the king and his subjects, all the evil ones, in the beginning she took them to nice places, but later she took them to fearful places. (slobdpon)
\end{exe}

\begin{exe}
\ex 
\gll \ipa{kɤɣɯ-tɤ-snɯ\tld{}sno}  	\ipa{ci}  	\ipa{rcanɯ,}  	\ipa{tshɯntshɯn}  	\ipa{ʑo}  	\ipa{kɯ-pa}  	\ipa{jɤ-ɣe}  	\ipa{ɲɯ-ŋu,}\\
 \textsc{comit-indef.poss}-saddle \textsc{indef} \textsc{unexpected} \textsc{idph:stat}:complete  \textsc{emph} \textsc{nmlz}:S/A-light.verb \textsc{pfv}-come[II] \textsc{sens}-be \\
\glt (A horse) came, together with a saddle and everything. 
\end{exe}


\begin{exe}
\ex
\gll
\ipa{tɤ-sno}  	\ipa{kɤ́-jɯ\tld{}jaʁ}  	\ipa{nɯ}  	\ipa{lu-ta-nɯ}  \\
\textsc{indef.poss}-saddle \textsc{comit}-hand \textsc{dem} \textsc{ipfv}-put-\textsc{pl} \\
\glt (Then), they put the saddle with its handles.
\end{exe}

\begin{exe}
\ex
\gll
\ipa{pɣɤkhɯ}  	\ipa{nɯ}  	\ipa{ɯ-ku}  	\ipa{nɯnɯ}  	\ipa{lɯlu}  	\ipa{tsa}  	\ipa{ɲɯ-fse,}  	\ipa{ɯ-mtsioʁ}  	\ipa{ɣɤʑu}  	\ipa{ma}  \ipa{kɤ́-rnɯ\tld{}rna}  	\ipa{lɯlu}  	\ipa{tu-fse}  	\ipa{ɲɯ-sɤre}  	\ipa{ʑo.}  \\
owl \textsc{dem} \textsc{3sg.poss}-head \textsc{dem} cat a.little \textsc{sens}-be.like \textsc{3sg.poss}-beak exist:\textsc{sens} a.part.from \textsc{comit}-ear cat \textsc{ipfv}-be.like \textsc{sens}-be.extremely/be.funny \textsc{emph} \\
\glt The owl's head looks a little like that of a cat, apart from the fact that it has a beak, it looks very much like a cat with its ears.
\end{exe}


\begin{exe}
\ex
\gll \ipa{kɤ́-thɤlwɯ\tld{}lwa}  	\ipa{ɯ-zrɤm}  	\ipa{ra}  	\ipa{kɯnɤ}  	\ipa{chɯ́-wɣ-ɣɯt}  	\ipa{pjɯ́-wɣ-ji}  	\ipa{ri}  	\ipa{maka}  	\ipa{tu-ɬoʁ}  	\ipa{mɯ́j-cha}  \\
\textsc{comit}-earth \textsc{3sg.poss}-root \textsc{pl} also \textsc{ipfv-inv}-bring \textsc{ipfv-inv}-plant but at.all \textsc{ipfv}-come.out \textsc{neg:sens}-can \\
\glt Even if one takes its root with earth (around it) and plant it, it cannot grow.
\end{exe}


Inference: that the saddle should not have been left on the horse.
\begin{exe}
\ex
\gll \ipa{kɤ́-snɯ\tld{}sno}  	\ipa{ʑo}  	\ipa{kɤ-rŋgɯ}  \\
\textsc{comit}-saddle \textsc{emph} \textsc{pfv}-lie.down \\
\glt (The horse) slept with its saddle. 
\end{exe}

A ??

lɯlu kɤ́rɟɯrɟit kɯ βʑɯ to-ndza-nɯ.???

lɯlu kɯ kɤ́rɟɯrɟit βʑɯ to-ndza-nɯ.???


Impossible to determine a noun within a comitative adverb
\section{Grammaticalization pathway} 

\subsection{Denominal derivation}
Japhug has a rich array of denominal prefixes (\citealt{jacques14antipassive}). One of these prefixes, \ipa{aɣɯ--}, derives intransitive verbs meaning `having X' from either inalienably possessed and non-inalienably possessed nouns, 

\subsection{S/A participle}

\subsection{Potential ambiguity}
  
  \begin{exe}
\ex
\gll   
  \ipa{si} 	\ipa{kɯ-ɤɣɯrtɯrtaʁ} 	\ipa{ki} 	\ipa{kɯ-fse} 	\ipa{ɲɯ-ɕar-nɯ} \\
  tree \textsc{nmlz}:S/A-have.many.branches this \textsc{nmlz}:S/A-be.this.way \textsc{ipfv}-search-\textsc{pl} \\
\glt They search for a tree having a lot of branches (NOT: a tree with its branches)
\end{exe}

 \ipa{(tɤ)-rɟit} $\Rightarrow$ \ipa{kɤɣɯ-rɟɯ\textasciitilde{}rɟit} `with his children' or `having many children'
\begin{exe}
\ex
\gll   
\ipa{iɕqha} 	\ipa{tɕʰeme} 	\ipa{nɯ} 	\ipa{kɯ-ɤɣɯrɟɯrɟit} 	\ipa{ci} 	\ipa{pɯ-ŋu}  \\
the.aforementioned woman \textsc{dem} \textsc{nmlz}:S/A-have.many.children \textsc{indef} \textsc{pst.ipfv}-be \\
\glt This woman had a lot of children.
\end{exe}

\begin{exe}
\ex
\gll   
\ipa{kɤɣɯ-rɟɯ\tld{}rɟit} 	\ipa{ʑo} 	\ipa{jo-nɯ-ɕe-nɯ} \\
\textsc{comit}-children \textsc{emph} \textsc{ifr-vert}-go-\textsc{pl} \\
\glt She/They went back with their children.
\end{exe}
  

\subsection{Grammaticalization pathway}
 \begin{exe}
\ex
 \glt  \textsc{noun} + \textsc{property denominal derivation} + participle $\rightarrow$ \textsc{comitative}
\end{exe} 
\section{Conclusion} 

\bibliographystyle{unified}
\bibliography{bibliogj}
\end{document}
