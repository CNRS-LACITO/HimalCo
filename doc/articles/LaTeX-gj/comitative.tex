\documentclass[oldfontcommands,oneside,a4paper,11pt]{article} 
\usepackage{fontspec}
\usepackage{natbib}
\usepackage{booktabs}
\usepackage{xltxtra} 
\usepackage{polyglossia} 
\usepackage[table]{xcolor}
\usepackage{lineno}
\usepackage{multicol}
\usepackage{graphicx}
\usepackage{float}
\usepackage{hyperref} 
\hypersetup{bookmarks=false,bookmarksnumbered,bookmarksopenlevel=5,bookmarksdepth=5,xetex,colorlinks=true,linkcolor=blue,citecolor=blue}
\usepackage[all]{hypcap}
\usepackage{memhfixc}
\usepackage{lscape}
\usepackage{tikz}
\usetikzlibrary{trees}
\usepackage{gb4e} 
\bibpunct[: ]{(}{)}{,}{a}{}{,}
 
%\setmainfont[Mapping=tex-text,Numbers=OldStyle,Ligatures=Common]{Charis SIL}  
\newfontfamily\phon[Mapping=tex-text,Ligatures=Common,Scale=MatchLowercase,FakeSlant=0.3]{Charis SIL} 
\newcommand{\ipa}[1]{{\phon #1}} %API tjs en italique
 
\newcommand{\grise}[1]{\cellcolor{lightgray}\textbf{#1}}
\newfontfamily\cn[Mapping=tex-text,Ligatures=Common,Scale=MatchUppercase]{MingLiU}%pour le chinois
\newcommand{\zh}[1]{{\cn #1}}
   

\begin{document} 
 \title{The origin of comitative adverbs in Japhug}
 \author{Guillaume Jacques}
 \maketitle  
 \section{Introduction}
\linenumbers

 \section{Inalienably possessed nouns} 
 Japhug nouns can be divided into three sub-classes: simple nouns, inalienably possessed nouns and nouns with numeral prefixes (classifiers).  

The same set of possessive prefixes (see Table \ref{tab:possessive}) is used for all nouns, but inalienably possessed nouns cannot be used on their own without one of these prefixes. 

When there is no definite possessor, inalienably possessed nouns require one of the indefinite possessive prefixes \ipa{tɤ--} or \ipa{tɯ--}. This form is used as the citation form of inalienably possessed nouns (\ipa{tɤ-lu} `milk', \ipa{tɯ-ŋga} `clothes', \ipa{tɤ-rpɯ} `uncle', \ipa{tɯ-ci} `water'). The choice of the prefix \ipa{tɤ--} vs \ipa{tɯ--} is lexically determined.  When a specific possessor is present, the indefinite prefix is replaced by the appropriate possessive prefix (\ipa{ɯ-lu} `her/its milk (from her nipple)', \ipa{a-ŋga} `my clothes', \ipa{nɤ-rpɯ} `your uncle', \ipa{ɯ-ci} `its juice'). 

\begin{table}[H] \centering
\caption{Possessive prefixes }\label{tab:possessive}
\begin{tabular}{lllllllll} 
\toprule
 Prefix & Person\\
\midrule
	\ipa{a--}  &		1\textsc{sg} \\
	\ipa{nɤ--}  &			2\textsc{sg}\\
\ipa{ɯ--}  &			3\textsc{sg}\\
\midrule
	\ipa{tɕi--}  &			1\textsc{du} \\
	\ipa{ndʑi--}  &		2\textsc{du} \\	
	\ipa{ndʑi--}  &		3\textsc{du} \\	
\midrule
	\ipa{i--}  &			1\textsc{pl} \\
		\ipa{nɯ--}  &			2\textsc{pl} \\
	\ipa{nɯ--}  &			3\textsc{pl} \\
\midrule
 \ipa{tɯ--},  \ipa{tɤ--} & indefinite \\
 \ipa{tɯ--}   &  generic\\
\bottomrule
\end{tabular}
\end{table}

It is possible to turn an inalienably possessed noun into an alienably possessed one by prefixing a definite possessive prefix to the indefinite one (\ipa{ɯ-tɤ-lu} `his milk (to drink)', \ipa{ɯ-tɯ-ci} `its water (of irrigated water, to a plant)'). Simple nouns cannot take indefinite possessive prefixes, but can take the generic possessor prefix.
 
 \section{Comitative derivation} 
In Japhug, adverbs meaning `having X' or `together with X' can be productively built from various types of nouns. In this section, I first describe the morphological processes involved in the noun to adverb derivation, and then provide an overview of the use of these adverbs in context.

\subsection{Morphology}
Comitative are formed by reduplicating the last syllable of the noun stem and prefixing either \ipa{kɤ́--} or \ipa{kɤɣɯ--}, as in examples such as \ipa{χɕɤlmɯɣ} `glasses' $\Rightarrow$ \ipa{kɤ́-χɕɤlmɯ\textasciitilde{}lmɯɣ} / \ipa{kɤɣɯ-χɕɤlmɯ\textasciitilde{}lmɯɣ} `together with glasses'.\footnote{Japhug \ipa{χɕɤlmɯɣ} `glasses' is a loanword from Tibetan \ipa{ɕel.mig}; note that reduplication disregards morpheme boundaries (\ipa{χɕɤl} `glass' (Tibetan \ipa{ɕel}) is also attested in Japhug). } No semantic difference between the comitative adverbs in \ipa{kɤ́--} and those in \ipa{kɤɣɯ--} has been detected; both are fully productive and can be built from the same nouns.

When the base noun is an inalienably possessed noun, two possibilities are available

\subsection{Syntactic uses} 


\ipa{rɟɤlpu}  	\ipa{kɤɣɯ-ŋkhɯ\textasciitilde{}ŋkhor}  	\ipa{kɯ-kɯ-ŋɤn}  	\ipa{ʑo}  	\ipa{to-ndo}  	\ipa{tɕe,}  	\ipa{tɕendɤre}  	\ipa{kɯ-mɤku}  	\ipa{nɯ}  	\ipa{sɤtɕha}  	\ipa{kɯ-kɯ-sɤscit}  	\ipa{ʑo}  	\ipa{jo-tsɯm}  	\ipa{ɲɯ-ŋu}  	\ipa{ri}  	\ipa{kɯ-maqhu}  	\ipa{tɕe,}  	\ipa{kɯ-kɯ-sɤɣmu}  	\ipa{ʑo}  	\ipa{jo-tsɯm}  	\ipa{tɕe}  

\section{Grammaticalization pathway} 

\subsection{Denominal derivation}
Japhug has a rich array of denominal prefixes (\citealt{jacques14antipassive}). One of these prefixes, \ipa{aɣɯ--}, derives intransitive verbs meaning `having X' from either inalienably possessed and non-inalienably possessed nouns, 

\subsection{S/A participle}
  \subsection{Synchronic differences between the S/A participle and the comitative adverb}
\section{Conclusion} 

\bibliographystyle{unified}
\bibliography{bibliogj}
\end{document}
