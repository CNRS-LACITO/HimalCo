\documentclass[11pt]{article} 
\usepackage{fontspec}
\usepackage{natbib}
\usepackage{booktabs}
\usepackage{xltxtra} 
\usepackage{polyglossia} 
\usepackage[table]{xcolor}
\usepackage{gb4e} 
\usepackage{multicol}
\usepackage{graphicx}
\usepackage{float}
\usepackage{hyperref} 
\usepackage{lineno}
\hypersetup{bookmarks=false,bookmarksnumbered,bookmarksopenlevel=5,bookmarksdepth=5,xetex,colorlinks=true,linkcolor=blue,citecolor=blue}
\usepackage[all]{hypcap}
\usepackage{memhfixc}
%\usepackage{lscape}

%\bibpunct[: ]{(}{)}{,}{a}{}{,}

%\setmainfont[Mapping=tex-text,Numbers=OldStyle,Ligatures=Common]{Charis SIL} 
\newfontfamily\phon[Mapping=tex-text,Ligatures=Common,Scale=MatchLowercase]{Charis SIL} 
\newcommand{\ipa}[1]{{\phon\textit{#1}}} %API tjs en italique
\newcommand{\ipab}[1]{{\scriptsize \phon#1}} 

\newcommand{\grise}[1]{\cellcolor{lightgray}\textbf{#1}}
\newfontfamily\cn[Mapping=tex-text,Ligatures=Common,Scale=MatchUppercase]{SimSun}%pour le chinois
\newcommand{\zh}[1]{{\cn #1}}
\newcommand{\refb}[1]{(\ref{#1})}
\newcommand{\factual}[1]{\textsc{:fact}}
\newcommand{\rdp}{\textasciitilde{}}

\XeTeXlinebreaklocale 'zh' %使用中文换行
\XeTeXlinebreakskip = 0pt plus 1pt %
 %CIRCG
 \newcommand{\bleu}[1]{{\color{blue}#1}}
\newcommand{\rouge}[1]{{\color{red}#1}} 
\newcommand{\ro}{$\Sigma$}

\begin{document} 
\title{Egophoric marking and Person Indexation in Japhug\footnote{ The glosses follow the Leipzig glossing rules. Other abbreviations used here are: \textsc{auto}  autobenefactive-spontaneous, \textsc{anticaus} anticausative, \textsc{antipass} antipassive, \textsc{appl} applicative, \textsc{dem} demonstrative,  \textsc{dist} distal, \textsc{emph} emphatic, \textsc{fact} factual, \textsc{genr} generic, \textsc{ifr} inferential, \textsc{indef} indefinite, \textsc{inv} inverse,  \textsc{lnk} linker, \textsc{pfv} perfective, \textsc{poss} possessor, \textsc{pres} egophoric present, \textsc{prog} progressive, \textsc{sens} sensory. This research was funded by the HimalCo project (ANR-12-CORP-0006) and is related to the research strand LR-4.11 ‘‘Automatic Paradigm Generation and Language Description’’ of the Labex EFL (funded by the ANR/CGI). Acknowledgements will be added after editorial decision. %Marc Bavant, Lauren Gawne, Simeon Floyd, Nathan W. Hill, Theo Lap, Holger Markgraf, Alexis Michaud, Nicolas Tournadre
} }
\author{Guillaume Jacques}  
\maketitle
%\linenumbers

\textbf{Abstract}: Japhug, like other Gyalrong languages, is one of the very few languages with both a full-fledged person indexation system and an egophoric evidential category. A detailed account of the uses and meanings of this category is thus of interest to the typology of evidential systems.

This paper describes the uses of Egophoric marking in Japhug and of the two other evidential categories with which it contrasts (Factual and Sensory), as well as their interaction with person indexation. Due to the limited distribution of the Egophoric in Japhug (it only occurs in present imperfective), the present paper exclusively focuses on the uses of evidentials with stative verbs in present imperfective contexts, where minimal pairs are available in the corpus.

\textbf{Keywords}: Egophoric, Factual, Sensory, Mirative, Hybrid Indirect Speech, Conjunct/Disjunct, Person indexation


\end{document}