\documentclass[twoside,a4paper,11pt]{article} 
\usepackage{polyglossia}
\usepackage{natbib}
\usepackage{booktabs}
\usepackage{xltxtra} 
\usepackage{longtable}
 \usepackage{geometry}
\usepackage[usenames,dvipsnames,svgnames,table]{xcolor}
\usepackage{multirow}
%\usepackage{gb4e} 
\usepackage{multicol}
\usepackage{graphicx}
\usepackage{float}
\usepackage{hyperref} 
\hypersetup{colorlinks=true,linkcolor=blue,citecolor=blue}
\usepackage{memhfixc}
\usepackage{lscape}
\usepackage{lineno}
\usepackage[footnotesize,bf]{caption}


%%%%%%%%%%%%%%%%%%%%%%%%%%%%%%%
\setmainfont[Mapping=tex-text,Numbers=OldStyle,Ligatures=Common]{Junicode} 
%\setsansfont[Mapping=tex-text,Ligatures=Common,Mapping=tex-text,Ligatures=Common,Scale=MatchLowercase]{Ubuntu} 
\newfontfamily\phon[Mapping=tex-text,Ligatures=Common,Scale=MatchLowercase]{Charis SIL} 
%\newfontfamily\phondroit[Mapping=tex-text,Ligatures=Common,Scale=MatchLowercase]{Doulos SIL} 
%\newfontfamily\greek[Mapping=tex-text,Scale=MatchLowercase]{Galatia SIL} 
\newcommand{\ipa}[1]{{\phon\textit{#1}}} 
\newcommand{\ipab}[1]{{\phon #1}}
\newcommand{\ipapl}[1]{{\phondroit #1}}
\newcommand{\captionft}[1]{{\captionfont #1}} 
%\newfontfamily\cn[Mapping=tex-text,Scale=MatchUppercase]{IPAGothic}%pour le chinois
%\newcommand{\zh}[1]{{\cn #1}}
\newcommand{\tgf}[1]{\mo{#1}}
%\newfontfamily\mleccha[Mapping=tex-text,Ligatures=Common,Scale=MatchLowercase]{Galatia SIL}%pour le grec

\newcommand{\sg}{\textsc{sg}}
\newcommand{\pl}{\textsc{pl}}
\newcommand{\grise}[1]{\cellcolor{lightgray}\textbf{#1}} 
\newcommand{\Σ}{\greek{Σ}}
\newcommand{\ro}{$\Sigma$}
\newcommand{\ra}{$\Sigma--1$} 
\newcommand{\rc}{$\Sigma--3$}  

\begin{document}
%\linenumbers
\title{The Arapaho conjunct order }

\author{Guillaume JACQUES}
%\date{}
\maketitle
\section{Introduction}

\begin{itemize}

 \item Sound laws: \citet{goddard74arapaho}

 \item Arapaho paradigms from \citet{cowell06arapaho}

 \item Proto-Algonquian paradigms follow \citet{bloomfield46proto} and \citet{goddard00cheyenne}
\end{itemize}
\section{VAI paradigm}
Table \ref{tab:vai} present the main VAI paradigms in Arapaho (after long and short vowel stem respectively).

The 1s, 1pi, 2s and 2p form straightforwardly come from the generally reconstructed paradigm. 3s has -' or -t depending on the conjugation type; a similar distribution is observed for this form in other Algonquian languages (Ojibwe --d  vs --g). The absence of palatalization of the -t is puzzling (-θ would be expected from *-ci < *--ti); note the palatalization on the 3's form \ipa{--níθ} < *\ipa{--rici}  < *\ipa{--riti}.

\begin{table}[H]
\caption{The Arapaho VAI paradigms and its proto-Algonquian origin}
\centering \label{tab:vai}
\begin{tabular}{lllllll}
\toprule
Person &   1 &  2 (short vowel) & Proto-Algonquian\\
\midrule
1s & 	\ipa{--noo} & 	\ipa{--noo} & 		*\ipa{--yâni} & 	\\	
1pe & 	\ipa{--ni'} & 	\ipa{--ni'} & 			\\	
1pi & 	\ipa{--no'} & 	\ipa{--no'} & 		*\ipa{--yankwe} & 	\\	
\midrule
2s & 	\ipa{--n} & 	\ipa{--n} & 		*\ipa{--yani} & 	\\	
2p & 	\ipa{--nee} & 	\ipa{--nee} & 		*\ipa{--yêkwe} & 	\\	
\midrule
3s & 	\ipa{--t} & 	\ipa{--'} & 		*\ipa{--ti} / \ipa{--ki}& 	\\	
3's & 	\ipa{--níθ} & 	\ipa{--níθ} & 		*\ipa{--rici} & 	\\	
3p & 	\ipa{--θi'} & 	\ipa{--'i} & 			\\	
3'p & 	\ipa{--níθi} & 	\ipa{--níθi} & 			\\	
\bottomrule
\end{tabular}
\end{table}

However, the 1pe and 3p forms do not directly originate from the PA VAI paradigm. 1pe *--yânke would give *-noo, a form homophonous with the 1s, and maybe homophony avoidance explains its reshaping; the  form  	\ipa{--ni'} probably originates from the VII plural *-riki; the 1pe would originate from an impersonal form (like French on).

The 3p forms have no trace of the plural *-wâ suffix, and are reshaped from the corresponding singular forms by addition of  a suffix. A possibility (at least from a phonetic point of view would be \ipa{--θi'} < *ciki (with double third person marking) and  	\ipa{--í'i} < *kiki(ri), but this reconstruction is uncertain.

\section{VTI paradigm}

In the VTI paradigm (Table \ref{tab:vti}), only the singular forms are inherited. Plural forms have all been (probably recently) renewed by combining the corresponding VAI ending with the suffix \ipa{--owu-} or \ipa{--owi-} , which corresponds phonetically to the element \ipa{--amo--} found in the 3p VTI suffix in proto-Algonquian 	*\ipa{--amowâci}.
\begin{table}[H]
\caption{The Arapaho VTI paradigm and its proto-Algonquian origin}
\centering \label{tab:vti}
\begin{tabular}{llllll}
\toprule
person & Arapaho & PA\\
\midrule
1s & 					\ipa{--owoo} & 	*\ipa{--amâni} & 	\\	
1pe & 					\ipa{--owu-ni'} & 	X*\ipa{--amânke} & 	\\	
1pi & 					\ipa{--owú-no'} & 	X*\ipa{--amankwe} & 	\\	
\midrule
2s & 					\ipa{--ow} & 	*\ipa{--amani} & 	\\	
2p & 					\ipa{--owú-nee} & X	*\ipa{--amêkwe} & 	\\	
\midrule
3s & 					\ipa{--o'} & 	*\ipa{--anki} & 	\\	
3's & 					\ipa{--owi-níθ} &   & 	\\
3p & 					\ipa{--óú'u} &   	X*\ipa{--amowâci} 	\\	
3'p & 					\ipa{--owu-níθi} & 	  & 	\\	
\bottomrule
\end{tabular}
\end{table}

\section{VTA paradigm}
Table \ref{tab:vta} presents the VTA conjunct paradigm in Arapaho (without the further obviative 3'>3' forms).
\begin{table}[H]
\caption{The Arapaho VTA paradigm}
\centering \label{tab:vta}
\begin{tabular}{llllllllllll}
\toprule
 & 	1s & 	1i & 	1e & 	2s & 	2p & 	3s & 	3p & 	3' & 	\\
1s & \grise{} & 	\grise{} & 	\grise{} & 	\ipa{--éθen} & 	\ipa{--eθénee} & 	\ipa{--o'} & 	\ipa{} & 	\ipa{} & 	\\
1i & 	\grise{} & 	\grise{} & 	\grise{} & 	\grise{} & 	\grise{} & 	\ipa{--óóno'} & 	\ipa{} & 	\ipa{} & 	\\
1e & 	\grise{} & 	\grise{} & 	\grise{} & 	\ipa{--een} & 	\ipa{--eenee} & 	\ipa{--éét} & 	\ipa{} & 	\ipa{} & 	\\
2s & 	\ipa{--ún} & 	\grise{} & 	\ipa{--únee} & 	\grise{} & 	\grise{} & 	\ipa{--ót} & 	\ipa{} & 	\ipa{} & 	\\
2p & 	\ipa{--éi'een} & 	\grise{} & 	\ipa{--éi'éénee} & 	\grise{} & 	\grise{} & 	\ipa{--óónee} & 	\ipa{} & 	\ipa{} & 	\\
3s & 	\ipa{--éínoo} & 	\ipa{--éíno'} & 	\ipa{éi'éét} & 	\ipa{--éín} & 	\ipa{--éínee} & 	\grise{} & 	\grise{} & 	\ipa{--oot} & 	\\
3p & 	\ipa{} & 	\ipa{} & 	\ipa{} & 	\ipa{} & 	\ipa{} & 	\grise{} & 	\grise{} & 	\ipa{--óóθi'} & 	\\
3's & 	\ipa{} & 	\ipa{} & 	\ipa{} & 	\ipa{} & 	\ipa{} & 	\ipa{--éít} & 	\ipa{--éíθi'} &   & 	\\
\bottomrule
\end{tabular}
\end{table}

As shown in Table \ref{tab:vta.1}, the singular   direct forms   are inherited. It is puzzling that the final -t is not palatalized to θ in 1pe, 2s and 3s forms (see the same phenomenon in the VAI paradigm above). The plural forms have different origins: 1pi and 2p are either  combinations of the direct --aa suffix of the independent order, or generalization of the -aa element in 3s, with the corresponding VAI endings \ipa{--no'} and \ipa{--nee} seen above (almost the same occurred in Plains Cree, see \citealt{dahlstrom89change}).

The relationship between the 1pe \ipa{--éét} and the corresponding PA ending \ipa{--akenti} is more complex: the expected regular form would be *--oeet or *-oeeθ, which by vowel harmony would merge to *--ooot. The form 	\ipa{--éét} might the the result of the generalization of an allomorph of the suffix occurring after VTA stems ending in -e, where the vowel fusion *-eoeet might have given the expected outcome  \ipa{--éét}.

The 3p is remade (without trace of the *-wâ- suffix) and a possible origin would be *--âci(ki).

\begin{table}[H]
\caption{The Arapaho VTA paradigm direct forms and their PA origins}
\centering \label{tab:vta.1}
\begin{tabular}{lllll}
\toprule
 1s & 	\ipa{--o'} & 	\ipa{--aki} & 		\\		
1pe & 	\ipa{--éét} & 	\ipa{--akenti} & 		\\		
1pi & 	\ipa{--óó-no'} & X	\ipa{--ankwe} & 		\\		
\midrule
2s & 	\ipa{--ót} & 	\ipa{--ati} & 		\\		
2p & 	\ipa{--óó-nee} & 	\ipa{--êkwe} & 		\\		
\midrule
3s & 	\ipa{--oot} & 	\ipa{--âti} & 		\\		
3p & 	\ipa{--óóθi'} & 	X\ipa{--âwâti} & 		\\		
\bottomrule
\end{tabular}
\end{table}

Unlike the direct forms, the inverse forms are almost entirely remade. Only the 3s is inherited 	\ipa{--éít} <   	\ipa{--ekoti} < *\ipa{--ekweti}. The other forms are built by combibing the inverse 	\ipa{--éí} < *\ipa{--ekwe--} (either generalized form the independent order or from the 3s of the conjunct) with the corresponding VAI endings (the 1pe ending corresponds to the VTI suffix rather than the VAI one). Plains Cree has the same phenomenon, but limited to the plural SAP (\citealt{dahlstrom89change}).



\begin{table}[H]
\caption{The Arapaho VTA paradigm inverse forms and their PA origins}
\centering \label{tab:vta.2}
\begin{tabular}{lllll}
\toprule
1s & 	\ipa{--éí-noo} & 	X\ipa{--iti} & 		\\
1pe & 	\ipa{--éi-'éét} & X	\ipa{--iyamenti} & 		\\
1pi & 	\ipa{--éí-no'} & 	X\ipa{--eθankwe} & 		\\
\midrule
2s & 	\ipa{--éí-n} & X	\ipa{--eθki} & 		\\
2p & 	\ipa{--éí-nee} & X	\ipa{--eθâkwe} & 		\\
\midrule
3s & 	\ipa{--éít} & 	\ipa{--ekweti} & 		\\
3p & 	\ipa{--éí-θi'} & X	\ipa{--ekowâti} & 		\\
\bottomrule
\end{tabular}
\end{table}
 
 
 The local forms (Table \ref{tab:vta.3}) and entirely remade. The 1s>2s  \ipa{--éθen} originates from the independent order suffix  *\ipa{--eθene}; the 2s>1s \ipa{--ún} / \ipa{--ín} is related to the PA 2s>1s suffix \ipa{--i}  VAI suffix, in addition to other elements. The --n is reminiscent of the 2s>1s ending -in Cree, but the -n part might be from the 2s  VAI suffix.
 
 The 1pe>2s suffix \ipa{--een} can originate from the conjunct order 2p suffix \ipa{--êkwe} combined with a n-- initial element, maybe the same suffix as in  2s>1s \ipa{--ún} / \ipa{--ín}.
 
The 2p>1 \ipa{--éi'een} is formed by combining the inverse  \ipa{--ei} with the 1pe>2s \ipa{--een} suffix.  


\begin{table}[H]
\caption{The Arapaho VTA paradigm local forms and their PA origins}
\centering \label{tab:vta.3}
\begin{tabular}{lllll}
\toprule
Person & Arapaho & PA Conjunct    \\
\midrule 
1s>2s& \ipa{--éθen} &X \ipa{--eθâni}   \\
1s>2p&\ipa{--éθenee} &X \ipa{--eθakokwe} & \\
1pe>2s&\ipa{--een} & X\ipa{--eθânke} &   \\
1pe>2p&\ipa{--eenee} &X \ipa{--eθânke} &   \\
\midrule 
2s>1s&\ipa{--ún} / \ipa{--ín} &  X \ipa{--iyani}     \\
2s>1pe&\ipa{--únee} / \ipa{--ínee} &  X \ipa{--iyânkwe} &  \\
2p>1s&\ipa{--éi'een} & X\ipa{--iyêkwe} &   \\
2p>1pe&\ipa{--éi'een} &X \ipa{--iyânkwe} &   \\
\bottomrule
\end{tabular}
\end{table}
%

 \bibliographystyle{linquiry2}
 \bibliography{bibliogj}

\end{document}