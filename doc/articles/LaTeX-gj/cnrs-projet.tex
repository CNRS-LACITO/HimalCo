\documentclass[oldfontcommands,oneside,a4paper,11pt]{article} 
\usepackage{fontspec}
\usepackage{natbib}
\usepackage{booktabs}
\usepackage{xltxtra} 
\usepackage{polyglossia} 
\setdefaultlanguage{french} 
\usepackage[table]{xcolor}
\usepackage{multirow}
\usepackage{gb4e} 
\usepackage{graphicx}
\usepackage{float}
\usepackage{lscape}
\usepackage{hyperref} 
\hypersetup{bookmarks=false,bookmarksnumbered,bookmarksopenlevel=5,bookmarksdepth=5,xetex,colorlinks=true,linkcolor=blue,citecolor=blue}
\usepackage[all]{hypcap}
\usepackage{memhfixc}

\bibpunct[~: ]{(}{)}{,}{a}{}{,} 


\setmainfont[Mapping=tex-text,Numbers=OldStyle,Ligatures=Common]{Charis SIL} %ici on définit la police par défaut du texte


\newfontfamily\phon[Mapping=tex-text,Ligatures=Common,Scale=MatchLowercase,FakeSlant=0.3]{Charis SIL} 
\newfontfamily\phondroit[Mapping=tex-text,Ligatures=Common,Scale=MatchLowercase]{Charis SIL} 
\newcommand{\ipa}[1]{{\phon #1}} %API tjs en italique
\newcommand{\ipac}[1]{{\tiny #1}}
\newcommand{\ipapl}[1]{{\phondroit #1}} 
\newfontfamily\cn[Mapping=tex-text,Ligatures=Common,Scale=MatchUppercase]{MingLiU}%pour le chinois
\newcommand{\zh}[1]{{\cn #1}}
\newfontfamily\mleccha[Mapping=tex-text,Ligatures=Common,Scale=MatchLowercase]{Galatia SIL}%pour le grec
\newcommand{\grec}[1]{{\mleccha #1}}
\newcommand{\tgz}[1]{#1 \mo{#1} \tg{#1}}
\newcommand{\indextg}[1]{\index{\tge{#1}@\tgz{#1}}}
\newcommand{\tgb}[1]{\tgz{#1}\indextg{#1}}
\newcommand{\tgc}[1]{\tg{#1} \#1\indextg{#1}}
\newcommand{\tgd}[1]{\tge{#1}\indextg{#1}}
\newcommand{\tgf}[1]{\mo{#1}\indextg{#1}}
\newcommand{\petit}[1]{\tiny#1}
\newcommand{\sig}{\begin{math}\Sigma\end{math}}
\newcommand{\phone}{\begin{math}\Phi\end{math}}
\newcommand{\ra}{$\Sigma_1$} 
\newcommand{\rc}{$\Sigma_3$} 
\newcommand{\grise}[1]{\cellcolor{lightgray}\textbf{#1}}


\begin{document}
%\OnehalfSpacing
\title{Projet de recherche}
\author{Guillaume Jacques}
\maketitle

\sloppy
\tableofcontents
\section{Grammaire et corpus de référence du japhug}
Ma priorité à moyen terme est de finir une grammaire de référence du japhug, accompagnée d'un dictionnaire et d'un corpus de textes disponibles en ligne. Une première version du dictionnaire sera disponible en ligne d'ici mi-2015, et le corpus sera progressivement publié sur le site Pangloss. C'est surtout sur la rédaction de la grammaire que mes efforts vont porter.

Si j'ai déjà publié une grammaire en chinois (\citealt{jacques08zh}), celle-ci, du fait des contraintes de la collection où elle a été publiée, est relativement courte (472 pp -- j'avais dû raccourcir le manuscrit) et n'est pas utilisable par les typologues occidentaux à cause de la barrière de la langue. Par ailleurs, ma connaissance du japhug est considérablement plus avancée maintenant qu'en 2006-2007 (lorsque je rédigeais ce livre), et mon corpus actuel est six fois plus long qu'à cette époque, et il est nécessaire d'écrire une nouvelle grammaire, en anglais, destinée aux typologues et aux comparatistes.

Depuis 2010, j'ai publié une série d'articles en anglais sur plusieurs points de la morphosyntaxe du japhug, qui peuvent être remaniés comme chapitres de grammaire en ajoutant des exemples et en supprimant les redondances et en les intégrant à l'ensemble du manuscrit. Cela inclut les sections sur l'indexation (\citealt{jacques10inverse}), la valence et la voix (\citealt{jacques12demotion}, \citealt{jacques12incorp}, \citealt{jacques13tropative},   \citealt{jacques14antipassive}), les idéophones (\citealt{japhug14ideophones}), le mouvement associé et les constructions à verbe de mouvement (\citealt{jacques13harmonization}) et le chainage de propositions (\citealt{jacques14linking}). Si l'on inclut d'autres sections déjà partiellement rédigées, je dispose déjà en tout de 290 pages, mais ces matériaux n'ont pas encore été combinés en un  tout cohérent.

Il reste encore à écrire de nombreuses parties de la grammaire: les sections sur la nominalisation, la complémentation, la morphologie nominale, la structure de la phrase nominale, les classes de verbes, les pivots syntaxiques et l'alignement, les quantificateurs, les particules de fin de phrase, la structure informationnelle, et de compléter celles sur les marques de voix, la dérivation verbale et le TAM.

  J'ai soumis cinq articles qui pourront être intégrés sous forme remaniée à la grammaire sur la phonologie (\textit{illustration of the IPA}), les comparatives, le générique, les constructions causatives et les relatives. Je compte soumettre en 2015 au moins cinq articles sur la morphosyntaxe du japhug: les numéraux et les classificateurs (pour un Festschrift), le préfixe spontané / autobénéfactif, le discours indirect hybride, les complétives et l'évidentialité. Lorsque tous ces articles seront rédigés (fin 2015), je me consacrerai pleinement  à synthétiser toutes les sections déjà rédigées et à les compléter, et à écrire les sections qui peuvent difficilement mener à un article (dans le domaines de la grammaire où le japhug n'est pas typologiquement inhabituel).

J'estime à quatre ans, à raison de 150 pages par an, le délai nécessaire minimal pour compléter la première version complète de la grammaire (pour 2018-2019).

\section{Grammaire comparée des langues rgyalronguiques et des langues kiranti}
La grammaire du japhug sera la première pierre d'un édifice plus général -- la grammaire comparée (synchronique et diachronique) des langues rgyalronguiques et kiranti. J'ai abandonné l'idée d'une recherche comparative sur l'ensemble du sino-tibétain, et préfère me concentrer sur les deux branches de la famille dont la morphologie verbale est la plus complexe.

Ce projet de recherche sera nécessairement collectif, car la masse des données à traiter rend difficilement possible une approche isolée. Il comprendra quatre parties: descriptions des langues et collection de données, phonologie comparée du rgyalronguique, phonologie historique du khaling et enfin morphosyntaxe comparée.

\subsection{Collection de données}
Une recherche comparative requiert des données riches, fiables et complètes sur toutes les langues étudiées, et à ce titre cette entreprise n'est pas réalisable sur la base exclusive de la grammaire japhug.

Je ne compte pas d'ici 2020 me charger de rédiger de grammaire autre que celle du japhug. Néanmoins, je continuerai à travailler sur trois autres langues, le khaling, le stau et le situ, et écrirai des articles sur des points spécifiques de la morphosyntaxe de ces langues, probablement en collaboration (avec Aimée Lahaussois, Anton Antonov et Gong Xun), et continuerai à collecter des textes. 

En outre, j'envisage un rôle de supervision: diriger des thèses de façon à couvrir l'ensemble des langues (et les dialectes) rgyalronguiques et kiranti qui restent à décrire, ainsi qu'à approfondir des points spécifiques sur les langues mieux dotées.


\subsection{Phonologie et morphologie comparée du rgyalronguique}
Dans mes travaux précédents, en particulier \citet{jacques04these} et \citet{jacques14esquisse}, j'ai présenté une reconstruction préliminaire du proto-rgyalrong, et ai établi un certain nombre de lois phonétiques. Toutefois, ces résultats, basés sur des données incomplètes, devront être intégralement revus lorsque, dans cinq à dix ans, on disposera de lexiques fiables sur toutes les langues rgyalronguiques.

Le lexique comparatif des langues rgyalronguiques comprend actuellement plus de 700 groupes de cognats. Etant donné la proximité de ces langues, il sera sans doute possible de doubler le nombre de cognats avec des lexiques plus complets,  d'affiner les lois déjà connues et d'en découvrir de nouvelles. 

Outre la reconstruction phonologique proprement dite, il est nécessaire d'intégrer la morphologie dans ces reconstructions, en particulier les alternances de thèmes verbaux réguliers et irréguliers et les dérivations verbales et dénominales. Si certaines alternances vocaliques peuvent être expliquées comme résultant de la fusion du radical verbal avec des suffixes (\citealt[357-8]{jacques04these}) ce n'est pas le cas de la majorité d'entre elles en particulier en zbu (\citealt{jackson04showu}), et leur origine reste jusqu'à ce jour un mystère.

\subsection{Phonologie et morphologie historique du khaling}
En kiranti, je compte au moins dans un premier temps me focaliser sur la comparaison du khaling et du dumi (\citealt{driem93dumi}) et du (\citealt{lahaussois09}), car ces trois langues forment un sous-groupe clair au sein du kiranti et leur systèmes verbaux sont facilement comparables -- ce qui n'est pas le cas avec des langues kiranti plus éloignées comme le thulung ou le limbu.

Ce groupe présente une situation presque idéale pour le comparatiste. En effet, le khaling est très conservateur du point de vue des attaques (en particulier les groupes de consonnes initiaux) mais très innovateur en ce qui concerne les voyelles et les consonnes finales, alors qu'à l'inverse, le dumi a perdu les groupes de consonnes, mais préserve bien en revanche les consonnes finales. 


J'ai déjà étudié la tonogénèse en khaling, dans un article soumis à un volume collectif sur les alternances tonales dans les paradigmes verbaux (en cours d'évaluation). J'y montre que les tons du khaling ont deux origines: la simplification des consonnes finales et la réduction de dissyllabes (qui sont préservés en dumi). La linguistique historique permet en outre de rendre compte des alternances observées en synchronie.

Je compte par la suite préparer un lexique comparatif des verbes khaling et dumi, ainsi qu'une reconstruction détaillée du système verbal de leur ancêtre commun, en combinant méthode comparative et reconstruction interne.

\subsection{Morphosyntaxe comparée}
La grammaire comparée ne se limite pas à une liste de morphèmes reconstruits. Une fois les lois phonétiques établies, il devient possible d'effectuer une comparaison détaillée de l'ensemble des constructions grammaticales dans les langues étudiées.

Il est difficile, à ce stade de nos connaissance des langues rgyalronguiques autres que le japhug, de savoir quels seront les sujets d'étude les plus prometteurs dans ce domaine, mais je compte au moins étudier en détail l'histoire des système d'indexation et de voix dans les langues rgyalronguiques d'une part, et kiranti d'autre part. Ce travail permettra d'évaluer jusqu'à quel point les systèmes verbaux des langues rgyalronguiques et kiranti sont cognats, et de répondre à la question de l'antiquité de la morphologie verbale en sino-tibétain.


\section{Typologie panchronique des systèmes d'indexation et des systèmes de voix }
 
\subsection{Évolution des systèmes d'indexation} 
 Mes travaux sur la morphologie comparée des langues rgyalronguiques et kiranti pourront être intégrés dans un cadre plus large, celui des principes généraux des changements diachroniques dans les systèmes verbaux à indexation polypersonnelle.  
 
 Ce programme de recherche, non limité au sino-tibétain, intégrera des données d'autres familles à indexation complexe, en particulier algonquien, sahaptien, sioux. 

Une première étape de ce travail est \citet{jacques15directionality}, dans lequel nous avons proposé certains principes généraux de changements analogiques sur la base de données de plusieurs langues algonquiennes. Il sera intéressant de tester si les conclusions de cet article restent valides en entreprenant une étude typologique sur un échantillon de langues plus important.
 
 Une autre question importante est la genèse des morphèmes portemanteaux  (notamment pour les marques `locales' $1\rightarrow2$ et $2\rightarrow1$, voir \citealt{heath98skewing}). J'ai déjà contribué à cette question dans le cas du rgyalronguique (\citealt{jacques15generic}), et compte l'aborder dans les langues algonquiennes et sioux, dont la phonologie historique est suffisamment bien élucidée pour pouvoir entreprendre ce type de recherches. 
 
 
 \subsection{Origine diachronique des marques de voix} 
 Monographie 
 
 
 Le graphe orienté des chemins de grammaticalisations possibles
 (\citealt{haspelmath90passive})
 
Dénominaux: (\citealt{jacques14antipassive})

constructions à verbe support : alguique  (\citealt{garrett04stem.structure})
 
 instrumental: sioux
 
 \subsection{Typologie des marques de voix et de l'incorporation} 
  
\section{Méthodes formelles en linguistique historique}

\subsection{Évaluation automatique d'hypothèses concurrentes}


\citet{rissanen84}, \citet{walther14compactness}, 
\subsubsection{Changements phonétiques}
application sur des dictionnaires étymologiques: \citet{csd2006}, \citet{hewson93proto}, \citet{hewson11roots}
\subsubsection{Analogie}

\subsection{Classification phylogénétique}

\bibliographystyle{unified}
\bibliography{bibliogj}
 
\end{document}
