\documentclass[oldfontcommands,oneside,a4paper,12pt]{article} 
\usepackage{fontspec}
\usepackage{natbib}
\usepackage{booktabs}
\usepackage{xltxtra} 
\usepackage{longtable}
\usepackage{tangutex2} 
\usepackage{tangutex4} 
\usepackage{polyglossia} 
\usepackage[table]{xcolor}
\usepackage{color}
\usepackage{multirow}
\usepackage{gb4e} 
\usepackage{multicol}
\usepackage{graphicx}
\usepackage{float}
\usepackage{hyperref} 
\hypersetup{bookmarks=false,bookmarksnumbered,bookmarksopenlevel=5,bookmarksdepth=5,xetex,colorlinks=true,linkcolor=blue,citecolor=blue}
\usepackage{memhfixc}
\usepackage{lscape}
\usepackage[footnotesize,bf]{caption}


%%%%%%%%%%%%%%%%%%%%%%%%%%%%%%%
\setmainfont[Mapping=tex-text,Numbers=OldStyle,Ligatures=Common]{Charis SIL} 
\setsansfont[Mapping=tex-text,Ligatures=Common,Mapping=tex-text,Ligatures=Common,Scale=MatchLowercase]{Lucida Sans Unicode} 
 


\newfontfamily\phon[Mapping=tex-text,Ligatures=Common,Scale=MatchLowercase,FakeSlant=0.3]{Charis SIL} 
\newfontfamily\phondroit[Mapping=tex-text,Ligatures=Common,Scale=MatchLowercase]{Doulos SIL} 
\newfontfamily\greek[Mapping=tex-text,Ligatures=Common,Scale=MatchLowercase]{Doulos SIL} 
\newcommand{\ipa}[1]{{\phon#1}} 
%\newcommand{\ipab}[1]{{\phon #1}}
\newcommand{\ipapl}[1]{{\phondroit #1}} 
\newcommand{\captionft}[1]{{\captionfont #1}} 
\newfontfamily\cn[Mapping=tex-text,Ligatures=Common,Scale=MatchUppercase]{MingLiU}%pour le chinois
\newcommand{\zh}[1]{{\cn #1}}

\newcommand{\racine}[1]{\begin{math}\sqrt{#1}\end{math}} 
\newcommand{\grise}[1]{\cellcolor{lightgray}\textbf{#1}} 
\newcommand{\redp}{$\mathtt{\sim}$}


\begin{document}
\title{The morphology of numerals and classifiers  in Japhug\footnote{Glosses follow the Leipzig rules, to which the following are added:  \textsc{fact} factual, \textsc{genr} generic,   \textsc{emph} emphatic, \textsc{ifr} inferential, \textsc{inv} inverse, \textsc{lnk} linker, \textsc{sens} sensory. The Tibetan transcription is based on \citet{jacques12transcription}.} }
\author{Guillaume Jacques}
\maketitle

\sloppy

\textbf{Abstract}: This paper describes the morphology and syntactic uses of numerals and classifiers in Japhug, and discusses the Burmo-Qiangic origins of the numeral prefixal paradigm.

\textbf{Keywords}: Japhug, numerals, classifiers, Naish, Pumi, Lolo-Burmese, analogy, \textit{status constructus}

\section{Introduction}
In many Burmo-Qiangic languages, including Lolo-Burmese (\citealt{bradley05numerals}) and Naish (\citealt{michaud11cl, michaud13numeral}), the combination of numerals with classifier is the one (in some cases the only one) area of grammar where morphological irregularities and complex alternations are attested.\footnote{For want of space, we do not present here data on languages other than Japhug and Stau; the reader is invited to refer to the cited sources for comparison.}

Somewhat paradoxically, in Rgyalrong languages, otherwise known for their polysynthetic and irregular verbal morphology (\citealt{jackson14morpho, jacques12incorp}), numerals and classifiers present relatively simple and predictable alternations. 

In this paper, we first present a  description of the morphology and morphosyntax of numerals and classifiers in Japhug and other Rgyalrongic  languages (going beyond the account in   \citealt{jacques08}) based on both corpus data and elicitation for some paradigms. 

Then, we evaluate several competing analyses to account for the observed data. First,   Japhug may never have developed these irregular systems: as shown by \citet{bradley05numerals}, most alternations in Lolo-Burmese and other languages  are the indirect effects of lost final obstruents. Since Japhug preserved all final obstruents as distinct segments, the basic conditions for the alternations to develop might not have been present in the first place. Second, the system found in Japhug could have recently been completely innovated. Third,  it could be cognate with the numeral + classifier paradigms in Lolo-Burmese and Naish but have been thoroughly simplified by analogical levelling.   

\section{Numerals and classifiers in Japhug}

In this section, I present a synchronic description of the syntax and morphology of numerals and classifiers in Japhug, with some additional data on Stau, another Rgyalrongic language, for comparison. 

First, I describe the structure of the noun phrase and the place of numerals and classifiers in it. Second, I provide an account of the morphology of numerals up to one hundred.  Third, I compare the plain numerals with the numeral prefixes used on classifiers. Fourth, I discuss the numerals above one hundred. Fifth, I briefly mention the approximate numerals, which appear to be specific to Rgyalrong languages.

\subsection{Word order}
The noun phrase in Japhug presents the following word order:\footnote{Note that attributive adjectives are all relative clauses; pre-nominal attributive adjective are rare, but not completely unattested.}
\begin{exe}
\ex \label{ex:noun.template}
\glt \textsc{dem-noun^{modifier}-noun^{head}-adj-num-dem}
\end{exe}

Numerals and classifiers appear after nouns and adjectives\footnote{Note that in Japhug; attributive adjective are in fact nominalized stative verbs; the noun phrases constitutes a head-internal relative.} as in examples \ref{ex:three.two} and  \ref{ex:Wmat.tWrdoR}.

\begin{exe}
\ex \label{ex:three.two}
\gll 
 \ipa{tɤpɤtso} 	\ipa{χsɯm,} 	\ipa{rgargɯn} 	\ipa{ʁnɯz,} 	\ipa{nɯ} \ipa{ra} 	\ipa{kɤ-fstɯn} 	\ipa{pɯ-ra} \\
 child three old.people two \textsc{dem} \textsc{pl} \textsc{inf}-serve \textsc{pst.ipfv}-have.to \\
\glt She had to take care of three children and two old people (on her own). (Relatives, 27)
\end{exe}

\begin{exe}
\ex \label{ex:Wmat.tWrdoR}
\gll 
\ipa{tɕe} 	\ipa{ɯ-mat} 	\ipa{tɯ-rdoʁ} 	\ipa{ɯ-ŋgɯ} 	\ipa{ɯ-rɣi} 	\ipa{tɯ-rdoʁ} 	\ipa{ma} 	\ipa{me.} \\
\textsc{lnk} \textsc{3sg.poss}-fruit  one-piece \textsc{3sg}-inside textsc{3sg.poss}-see  one-piece apart.from not.exist:\textsc{fact} \\
\glt There is only one seed in (each one) of its fruit. (Cherry, 89)
\end{exe}

The use of classifiers in Japhug is quite restricted in comparison with languages such as Naish, Lolo-Burmese or Lizu/Ersu (see \citealt[216-224]{lidz10na}, \citealt{zhang14classifiers}). There are a few classifiers specific for particular shapes such as \ipa{tɯ-ldʑa} `one long object', \ipa{tɯ-pʰɯ} `one tree', or \ipa{tɯ-mpɕar} `one sheet'. However, most nouns (including nouns with animate or inanimate referent) use the generic classifier \ipa{tɯ-rdoʁ} `one piece' (from Tibetan \ipa{rdog}).  

Classifiers are not used in Japhug to express indefinite reference (unlike in languages such as Na, cf \citealt[206]{lidz10na}). The indefinite determiner and numeral \ipa{ci} `one, a' is used for this purpose.

Some classifiers express a specific quantity,  size or number of individuals (such as \ipa{tɯ-boʁ} `one group' or \ipa{tɯ-spra} `a handful of'). Most classifiers, however, are used either to convey a distributive meaning, as in the example \ref{ex:Wmat.tWrdoR}, or to single out of individual from a group, as in \ref{ex:tWrdoR.cinA}.
 
\begin{exe}
\ex \label{ex:Wmat.tWrdoR}
\gll 
\ipa{tɯrme} 	\ipa{tɯ-rdoʁ} 	\ipa{kɯ} 	\ipa{cʰɤmdɤru} 	\ipa{tɯ-ldʑa} 	\ipa{tu-nɯ-ndɤm} \\
people one-\textsc{cl} \textsc{erg} drinking.straw one-\textsc{cl} \textsc{ipfv-auto}-take[III] \\
\glt Each person takes one straw. (Alcohol, 37)
\end{exe}
 
\begin{exe}
\ex \label{ex:tWrdoR.cinA}
\gll 
\ipa{ɯ-tɕɯ} 	\ipa{kɯβde} 	\ipa{pɯ-tu} 	\ipa{ri,} 	\ipa{ɯʑo} 	\ipa{kɯ-fse} 	\ipa{kɯ-ɕqraʁ} 	\ipa{tɯ-rdoʁ} 	\ipa{cinɤ} 	\ipa{pɯ-me} 	\ipa{ɲɯ-ŋu} 	\\
\textsc{3sg.poss}-son four \textsc{pst.ipfv}-exist but \textsc{3sg} \textsc{nmlz}:S/A-be.like \textsc{nmlz}:S/A-be.intelligent one-piece even \textsc{pst.ipfv}-not.exist \textsc{sens}-be \\
\glt He had four sons, but not even one of them was as smart as he. (The smart one, 3)
\end{exe} 
 
 Numerals and classifiers can be used without nouns. Repetition of a classifier with the same numeral prefix  expresses a regular distribution as in \ref{ex:tWrdoR.tWrdoR}.
 
 \begin{exe}
\ex \label{ex:tWrdoR.tWrdoR}
\gll 
 \ipa{tɯ-rdoʁ} 	\ipa{tɯ-rdoʁ} 	\ipa{kɯ-fse} 	\ipa{tu-ɬoʁ} 	\ipa{ŋu} 	\ipa{ma} 	\ipa{mɤ-arɤkʰɯmkʰɤl.}  \\
  one-\textsc{cl}  one-\textsc{cl} \textsc{nmlz}:S/A-be.like  be:\textsc{fact} \textsc{ipfv}-come.out \textsc{lnk} \textsc{neg}-be.in.patches:\textsc{fact} \\
 \glt (Each of the mushrooms) grows at regular intervals, (they do not grow all) in patches. (\ipa{tɤqiaβjmɤɣ}, 129)
\end{exe}
 
 \subsection{The morphology of plain numerals}
 
 Rgyalrongic languages differ from otherwise closely related languages such as Naish (\citealt{michaud11cl}) or Pumi (\citealt[141]{daudey14grammar}, \citealt[91-2]{ding14grammar}) in that the numerals 11, 12, 13, 16  and in some languages 14, present a labial linker element between the root for `ten' and that of the unit.  
 This labial linker  is variously realized as a stop, a labio-dental fricative or the nasal \ipa{m} depending on the following consonant.
 
 The linker appears whenever  the root of the unit does not contain an initial cluster. Note that the bare root of the numeral does not always correspond to the simple numeral. In Japhug \ipa{ʁnɯz}  `two' and 	\ipa{χsɯm} `three'  have a uvular prefix; in Stau \ipa{ɣni} `two' , \ipa{xsʚ} `three', \ipa{ɣɮdə} `four' and 	\ipa{xtɕʰu} `six'  have a velar fricative prefix which is lost in the numerals between 11 and 20, and which appears to correspond  to  to the presyllable \ipa{kɯ--} found in the numerals from four to nine in Japhug.\footnote{On the simplification of presyllables in Japhug and their sensitivity to onset complexity, see \citet{jacques14antipassive}.}
 
Since the numerals which do not have the linker element (15, 17, 18 and 19) are also the ones whose bare root contains an initial cluster, the labial linker can thus be considered to appear between the two numerals roots in numerals between 11 and 19, whenever no cluster is present in the second root.
 
It is unclear to what extent this linker is a Rgyalrongic innovation, or an archaism, lost in other languages due to  analogy, but its complete absence outside of Rgyalrongic suggests that the first option is more probable.


Numerals between  twenty  and 99 in Japhug can be generated by combining the tens with the units, replacing the \ipa{--sqi} element with the appropriate teen form, as indicated for numerals between 21 and 29 in Table \ref{tab:num.simple}.

\begin{table}[H]
\caption{Comparison of basic numerals in Japhug and Stau}  \label{tab:num.simple} \centering
\begin{tabular}{lllllll}
\toprule
Numeral & Japhug & Stau \\
\midrule
1	&	\ipa{tɤɣ} / \ipa{ci} &	\ipa{ru}  &	\\
2	&	\ipa{ʁ-nɯz}  &	\ipa{ɣ-ni}  &	\\
3	&	\ipa{χ-sɯm}  &	\ipa{x-sʚ}  &	\\
4	&	\ipa{kɯ-βde}  &	\ipa{ɣ-ɮdə}  &	\\
5	&	\ipa{kɯ-mŋu}  &	\ipa{mbe}  &	\\
6	&	\ipa{kɯ-tʂɤɣ}  &	\ipa{x-tɕʰu}  &	\\
7	&	\ipa{kɯ-ɕnɯz}  &	\ipa{zɲi}  &	\\
8	&	\ipa{kɯ-rcat}  &	\ipa{rje}  &	\\
9	&	\ipa{kɯ-ngɯt}  &	\ipa{ŋɡə}  &	\\
10	&	\ipa{sqi}  &	\ipa{zʁa}  &	\\
\midrule
11	&	\ipa{sqa-\textbf{p}-tɯɣ} \grise &	\ipa{ʁa-\textbf{v}-ru}  \grise&	\\
12	&	\ipa{sqa-\textbf{m}-nɯz} \grise &	\ipa{ʁa-\textbf{m}-ɲi}  \grise&	\\
13	&	\ipa{sqa-\textbf{f}-sum}  \grise&	\ipa{ʁa-\textbf{f}-sʚ} \grise &	\\
14	&	\ipa{sqa-βde}  &	\ipa{ʁa-\textbf{v}-ɮdə}  \grise&	\\
15	&	\ipa{sqa-mŋu}  &	\ipa{ʁa-mbe}  &	\\
16	&	\ipa{sqa-\textbf{p}-rɤɣ}  \grise&	\ipa{ʁa-\textbf{p}-tɕʰu}  \grise&	\\
17	&	\ipa{sqa-ɕnɯz}  &	\ipa{ʁa-zɲi}  &	\\
18	&	\ipa{sqa-rcat}  &	\ipa{ʁa-rje}  &	\\
19	&	\ipa{sqa-ngɯt}  &	\ipa{ʁa-ŋɡə}  &	\\
20	&	\ipa{ɣnɤ-sqi}  &	\ipa{ɣnə-sqʰa}  &	\\
\midrule
21	&	\ipa{ɣnɤ-sqaptɯɣ}  &	\ipa{nə-ɣru}  &	\\	
22	&	\ipa{ɣnɤ-sqamnɯz}  &	\ipa{nə-ɣni}  &	\\	
23	&	\ipa{ɣnɤ-sqafsum}  &	\ipa{nə-xsʚ}  &	\\	
24	&	\ipa{ɣnɤ-sqaβde}  &	\ipa{nə-ɣɮdə}  &	\\	
25	&	\ipa{ɣnɤ-sqamŋu}  &	\ipa{nə-mbe}  &	\\	
26	&	\ipa{ɣnɤ-sqaprɤɣ}  &	\ipa{nə-xtɕʰu}  &	\\	
27	&	\ipa{ɣnɤ-sqaɕnɯz}  &	\ipa{nə-zɲi}  &	\\	
28	&	\ipa{ɣnɤ-sqarcat}  &	\ipa{nə-rje}  &	\\	
29	&	\ipa{ɣnɤ-sqangɯt}  &	\ipa{nə-ŋgə}  &	\\	
\midrule					
30	&	\ipa{fsu-sqi}  &	\ipa{xsʚ-sqʰa}  &	\\	
40	&	\ipa{kɯβdɤ-sqi}  &	\ipa{ɣɮə-sqʰa}  &	\\	
50	&	\ipa{kɯmŋɤ-sqi}  &	\ipa{mbe-sqʰa}  &	\\	
60	&	\ipa{kɯtʂɤ-sqi}  &	\ipa{xtɕʰu-sqʰa}  &	\\	
70	&	\ipa{kɯɕnɤ-sqi}  &	\ipa{zɲi-sqʰa}  &	\\	
80	&	\ipa{kɯrcɤ-sqi}  &	\ipa{rje-sqʰa}  &	\\	
90	&	\ipa{kɯngɯ-sqi}  &	\ipa{ŋɡə-sqʰa}  &	\\	
\bottomrule
\end{tabular}
\end{table}
		


 \subsection{Numeral prefixes} \label{sec:prefixes.japhug}
 
 In contrast with the relatively complex forms of the numerals 11 to 19, the combinations of numerals and classifiers in Japhug and Stau are relatively simple. 
 
Table \ref{tab:num.prefix} illustrates the numeral prefix paradigm in Japhug: the final consonants of the numeral root are lost, the vowels \ipa{a} and \ipa{i} change to \ipa{ɤ} and \ipa{ɯ} respectively, but no other change takes place. In the case of the numerals above ten, prefixal form is optional; it is possible to use the free form instead. Prefixal forms for other numerals under 100 can be generated with the same rules.\footnote{The final stop \ipa{--t} in \ipa{kɯngɯt} `nine' is unexpected (it is not even found in the closely related Situ language where we have \ipa{kəngu} `nine'), and most probably due to analogy with the coda of \ipa{kɯrcat}  `eight'. }
 
The numerals 	\ipa{kɯβde}  `four'   and \ipa{kɯmŋu} `five' have two variants \ipa{ kɯβde-- / kɯβdɤ--} and \ipa{ kɯmŋu-- / kɯmŋɤ--} in the prefixal paradigm, the first of which is most common.
 
 \begin{table}[H]
\caption{Numeral prefixes in Japhug}  \label{tab:num.prefix} \centering
\begin{tabular}{lllllll}
\toprule
Numeral & Free form &  \ipa{--sŋi} `day' &  \ipa{--rʑaʁ} `night' \\
\midrule
 1	&	\ipa{tɤɣ}  &	\ipa{tɯ-sŋi}  &	\ipa{tɤ-rʑaʁ}  &	\\
2	&	\ipa{ʁnɯz}  &	\ipa{ʁnɯ-sŋi}  &	\ipa{ʁnɤ-rʑaʁ}  &	\\
3	&	\ipa{χsɯm}  &	\ipa{χsɯ-sŋi}  &	\ipa{χsɤ-rʑaʁ}  &	\\
4	&	\ipa{kɯβde}  &	\ipa{kɯβde-sŋi}  &	\ipa{kɯβdɤ-rʑaʁ}  &	\\
5	&	\ipa{kɯmŋu}  &	\ipa{kɯmŋu-sŋi}  &	\ipa{kɯmŋɤ-rʑaʁ}  &	\\
6	&	\ipa{kɯtʂɤɣ}  &	\ipa{kɯtʂɤ-sŋi}  &	\ipa{kɯtʂɤ-rʑaʁ}  &	\\
7	&	\ipa{kɯɕnɯz}  &	\ipa{kɯɕnɯ-sŋi}  &	\ipa{kɯɕnɤ-rʑaʁ}  &	\\
8	&	\ipa{kɯrcat}  &	\ipa{kɯrcɤ-sŋi}  &	\ipa{kɯrcɤ-rʑaʁ}  &	\\
9	&	\ipa{kɯngɯt}  &	\ipa{kɯngɯ-sŋi}  &	\ipa{kɯngɤ-rʑaʁ}  &	\\
10	&	\ipa{sqi}  &	\ipa{sqɯ-sŋi}  &\ipa{sqɤ-rʑaʁ}  &	\\
\midrule
11	&	\ipa{sqaptɯɣ}  &	\ipa{sqaptɯ-sŋi}  &	\\
12	&	\ipa{sqamnɯz}  &	\ipa{sqamnɯ-sŋi}  &	\\
13	&	\ipa{sqafsum}  &	\ipa{sqafsum-sŋi}  &	\\
14	&	\ipa{sqaβde}  &	\ipa{sqaβde-sŋi}  &	\\
15	&	\ipa{sqamŋu}  &	\ipa{sqamŋu-sŋi}  &	\\
16	&	\ipa{sqaprɤɣ}  &	\ipa{sqaprɤ-sŋi}  &	\\
17	&	\ipa{sqaɕnɯz}  &	\ipa{sqaɕnɯ-sŋi}  &	\\
18	&	\ipa{sqarcat}  &	\ipa{sqarcɤ-sŋi}  &	\\
19	&	\ipa{sqangɯt}  &	\ipa{sqangɯ-sŋi}  &	\\
20	&	\ipa{ɣnɤsqi}  &	\ipa{ɣnɤsqɯ-sŋi}  &	\\
\bottomrule
\end{tabular}
\end{table}
In addition,  \ipa{kɤntɕʰɯ} `many, several' (the participle of \ipa{antɕʰɯ} `be many') and the interrogative pronom \ipa{tʰɤstɯɣ} `how many' have the prefixal forms \ipa{kɤntɕʰɯ--} and \ipa{tʰɤstɯ--} (as in \ipa{kɤntɕʰɯ-xpa} `many years' and \ipa{tʰɤstɯ-tɯrpa} `how many pounds').

All classifiers in Japhug except \ipa{--rʑaʁ} `night' follow the paradigm of \ipa{--sŋi} `day'. The classifier \ipa{--rʑaʁ} `night' is the only one with irregular forms, and even  \ipa{--rʑaʁ} can be used with the regular paradigm.
 

 
 In Stau, only the numeral prefixes  `one' \ipa{e--} and `two' \ipa{ɣnə--} have a special form, the rest is identical to the free numerals.
 
 \subsection{Other numerals}
Numerals above one hundred present less morphological alternations than the units and tens.
 
 There are two ways of expressing `one hundred' in Japhug. First, the noun-like numeral \ipa{ɣurʑa}   `one hundred' can be employed as in \ref{ex:hundred}.

\begin{exe}
\ex \label{ex:hundred}
\gll \ipa{aʑo} 	\ipa{kɯ-fse} 	\ipa{kɯ-cʰɯ\redp{}cʰa} 	\ipa{ʑo} 	\ipa{ʁʑɯnɯ} 	\ipa{ɣurʑa} 	\ipa{kɯrcat} 	\ipa{ra} \\
\textsc{1sg} \textsc{nmlz}:S/A-be.like  \textsc{nmlz}:S/A-\textsc{emph}\redp{}can \textsc{emph} young.man hundred eight need:\textsc{fact} \\
\glt I need one hundred and eight able young men like me. (Slobdpon, 16)
\end{exe}

The numeral \ipa{ɣurʑa}  cannot be combined with unit numerals to express numbers between 200 and 900. The  the classifier \ipa{tɯ-ri} `one hundred' is used for this purpose, as in \ref{ex:three.hundreds} (see section \ref{sec:prefixes.japhug} for an account of the numeral prefixes).
\begin{exe}
\ex \label{ex:three.hundreds}
\gll
\ipa{χsɯ-ri} 	\ipa{jamar} 	\ipa{ndɤre} 	\ipa{tu-nɯ} 	\ipa{ko,} 	\ipa{tɯ-tɯphu} 	\ipa{nɯ} \\
three-hundred about \textsc{lnk} exist:\textsc{fact-pl} \textsc{sfp} one-hive \textsc{dem} \\
\glt There are about three hundred of them, in one hive. (Bees, 48)
\end{exe}
 
  

Numerals above the hundreds are all borrowed from Tibetan: \ipa{stoŋtsu} `thousand', \ipa{kʰrɯtsu} `ten thousand', \ipa{mbɯmχtɤr} `hundred thousand' from \ipa{stŋ}, \ipa{kʰri} and \ipa{ɴbum.tʰer} respectively. These numerals appear after the noun they qualify like \ipa{ɣurʑa} `hundred'.  
 
 \subsection{Approximate Numerals}
There are three strategies in Japhug to express an approximate number.

First, there is a restricted set of approximate numerals for numerals under ten (Table \ref{tab:approx.num}).

\begin{table}[H]
\caption{Approximate numerals in Japhug} \label{tab:approx.num} \centering
\begin{tabular}{llllll}
\toprule
&Numeral &Numeral prefixes (with \ipa{--sŋi} `day')\\
\midrule
a few &\ipa{laʁnɯz}  & 	\ipa{laʁnɯ-sŋi}  & 	\\	
2-3&\ipa{laʁnɯχsɯm}  & 	\ipa{laʁnɯχsɯ-sŋi}  & 	\\	
4-5&\ipa{lɤβdelɤŋu}  & 	\ipa{lɤβdelɤŋu-sŋi}  & 	\\	
5-6&\ipa{lɤŋɤtʂɤɣ}  & 	\ipa{lɤŋɤtʂɤ-sŋi}  & 	\\	
7-8&\ipa{ɕnɤcat}  & 	\ipa{ɕnɤcɤ-sŋi }  & 	\\	
9-10&\ipa{kɯngɯsqi}  & 	\ipa{kɯngɯsqɯ-sŋi}  & 	\\	
\bottomrule
\end{tabular}
\end{table}

Second, it is possible to repeat the same classifier with a different numeral prefix, as in \ref{ex:kWBde.tWrpa}.
 
 
 \begin{exe}
\ex \label{ex:kWBde.tWrpa}
\gll 
\ipa{kɯβde-tɯrpa} 	\ipa{kɯmŋu-tɯrpa} 	\ipa{jamar} 	\ipa{ma} 	\ipa{tu-zɣɯt} 	\ipa{mɤ-cha} \\
four-pound five-pound about apart.from \textsc{ipfv}-reach \textsc{neg}-can:\textsc{fact} \\
\glt It can only reach three or four pounds. (Hen, 14)
\end{exe}


Third, for numerals above 99, it is possible to add a third person singular possessive prefix to  express an approximate value, as in  \ipa{ɯ-ɣurʑa} `several hundreds', \ipa{ɯ-kʰrɯtsu} `several dozen of thousand' etc.


\section{Possible pathways of development for the numeral prefix paradigms  in Rgyalrongic }

There three logical possibilities to account for the regularity of numeral/classifiers paradigms in Rgyalrongic. 

First, it could be a conservative feature, namely the non-development of complex alternations due to the fact that Rgyalrong languages preserve final obstruents, unlike Pumi or Naish languages. Second, it could be due to the fact that the whole system of numeral prefixes was recently innovated. Third, the system itself could be cognate to the one found in Naish and Pumi, but have been renewed by analogy. 

\subsection{Archaism}

Rgyalrong languages, and Japhug in particular, preserve the final obstruents fairly well. This is obvious in the case of words borrowed from Tibetan (see Table \ref{tab:tib.borrow}) and also in the inherited vocabulary. The final stops \ipa{--b}, \ipa{--d}, \ipa{--g} or Old Tibetan correspond to Japhug \ipa{--β}, \ipa{--t}, \ipa{--ɣ} / \ipa{--ʁ} respectively: the dental stop is preserved as a stop, and the other stops appear as fricatives. 

\begin{table}[H]
\caption{Preservation of final obstruents in Tibetan loanwords in Japhug} \label{tab:tib.borrow} \centering
\begin{tabular}{llllll}
\toprule
Japhug &Tibetan & Meaning\\
\midrule
 \ipa{rɟɤlkʰɤβ}  & 	\ipa{rgʲal.kʰab}  & country 	\\
 \ipa{βdɯt}  & 	\ipa{bdud}  & demon 	\\
 \ipa{tɯɣ}  & 	\ipa{dug}  & poison 	\\
 \ipa{praʁ}  & 	\ipa{brag}  & cliff 	\\
 \ipa{sŋaʁspa}  & 	\ipa{sŋags.pa}  & sorcerer 	\\
\bottomrule
\end{tabular}
\end{table}

In Burmo-Qiangic languages other than Rgyalrongic (except the Burmish branch), final stops are invariably lost. In the case of Naish loss of final obstruents had already happened at the proto-Naish stage (\citealt{jacques.michaud11naish}).

There is some evidence that the final stops in pre-proto-Naish may have left a trace in the patterning of tonal alternation in the numeral+classifier paradigms. As shown by \citet[16-17]{michaud11cl}, the comparison of the three Naish languages Na, Laze and Naxi reveals that numerals under 10 can be classified into the several groups based on their tonal alternations. The numerals 3, 7, 9 and 10 have specific alternations, but {1, 2}, {4, 5} and {6, 8} respectively always have the same tonal class. The group {6, 8} is particularly significant, as it is the only group of non-contiguous numerals, and both 6 and 8 have final obstruents in conservative languages (Tibetan \ipa{drug} and \ipa{brgʲad}, for instance).

Thus, it can be hypothesized that (1) although final stops were lost, they were partially transphonologized as tonal contrasts and (2) the development of the classifier system in Naish predates the loss of final stops.

This also suggests that languages that have not lost final stops, like Japhug, would be unlikely to have developed complex tonal or segmental alternations in their classifier system, since the transphonologization of obstruent codas  did not occur.  Japhug, in this view, would be conservative in preserving a regular system with little phonetic accidence. Yet, such a hypothesis is untenable.  

While Japhug does preserve final stops in isolation, these final stops are lost in classifier+numeral combinations, and always in the same way. If the Japhug paradigm were really conservative, the final stops of the numerals should combine in complex ways with the onset of the classifier. For instance, we know from comparison that the proto-Rgyalrong group *\ipa{pk} recently changed to Japhug \ipa{βɣ} (as in \ipa{βɣaza} `fly' cognate with Situ \ipa{kəpos tsa}, see \citealt[272]{jacques04these}). This implies that classifiers with initial \ipa{p--} should  \textit{lautgesetzlich} have an allomorph \ipa{βɣ} following numerals with a coda coming from *\ipa{--k} (ie `one' and `six').  For instance, the classifier \ipa{--pɤrme} `year (of life)'  should have had the form  *\ipa{tɤβɣɤrme} from proto-Japhug *\ipa{tek-pɐrme}  instead of regular \ipa{tɯ-pɤrme} `one year (old)' if the whole form had been inherited. The fact that not a single classifier presents any alternation of this type proves that the system as such cannot be archaic.



\subsection{Innovation}
An alternative possibility would be that the numeral+classifier systems found among Burmo-Qiangic languages are only superficially similar: it could be proposed that these paradigms are analogous rather than homologous, and result from independent parallel grammaticalizations.

In this hypothesis, it would not be surprising that Japhug and Stau have few irregular alternations: it might just imply that these systems are very young and have not yet had the time to develop irregular alternations.

Yet,  there is  evidence that the numeral+classifier systems in Rgyalrongic languages are actually cognate to the systems of at least some of other languages of the Burmo-Qiangic group, and thus have some degree of antiquity. The only type of evidence that can show that the classifier systems are not independently innovated is to find irregular or suppletive patterns in the paradigms common between Rgyalrongic and non-Rgyalrongic languages.

Although, as mentioned above, the numeral+classifiers paradigms in Rgyalrongic languages are very regular, there are nevertheless a few cases of suppletion found across Burmo-Qiangic, showing that the numeral+classifier paradigms are not mere parallel developments, but should be reconstructed back to  an intermediate node of the group.

The first such evidence concern the numerals for `hundred'. We saw that in Japhug two roots are used to express `one hundred',  \ipa{ɣurʑa}  and the classifier \ipa{--ri} `one hundred'. The former appears for numerals up to 199, while the second is used to express the hundreds from 200 to 900; for `one hundred'. This particularity is shared with  other Burmo-Qiangic languages. Pumi  Pumi   has the noun-like \ipa{ɕí}  hundred' and the classifier \ipa{--ɻɛj} (\citealt[101]{daudey14grammar}) with distributions very similar to the Japhug etyma. Moreover,  note that the correspondences \ipa{--i} : \ipa{--a}\footnote{Exclusively before palatalized onsets; \ipa{--ə} : \ipa{--a} in other contexts.} and \ipa{--ɛj} : \ipa{--i} between Wadu Pumi and Japhug are widely attested (see examples in Table \ref{tab:jpg.pumi}).


Hence, there is little doubt that the pair of roots for hundred in Japhug and Pumi are cognate. Since in both languages one of the member is a classifier, obligatorily taking a numeral prefix, it is unlikely that the classifier system of Pumi and Japhug were independently grammaticalized. Rather, it suggest that the two roots corresponding to Japhug \ipa{ɣurʑa} and \ipa{--ri} can both be reconstructed back to the common ancestor of Rgyalrong languages and Pumi, and that the ancestral form of \ipa{--ri} was already a classifier in the proto-language; hence, the classifier system was already in existence at that time.

\begin{table}[H]
\caption{Correspondences between Japhug and Pumi} \label{tab:jpg.pumi} \centering
\begin{tabular}{llllll}
\toprule
Japhug & Meaning & Pumi & Meaning \\
\midrule
 \ipa{--sla} &month &\ipa{ʑí} & month\\
\ipa{χtʂɯɣdʑa} &butter tea  & \ipa{dʑǐ} & tea  \\
&(from Tibetan \ipa{dkrug.dʑa})&&(from Tibetan \ipa{dʑa})&\\
\midrule
 \ipa{--pi} &elder sibling &\ipa{pɛ̌j} & elder sibling\\
  \ipa{wxti} &be big & \ipa{tɛ́j}  & be big\\
\bottomrule
\end{tabular}
\end{table}
 
The second piece of evidence for the antiquity of the classifier system is the suppletion in the word for `year' (on which, see 
\citealt{jacques.michaud11naish} and \citealt{jacques14tangoute}). Naish and Qiangic languages (but not Lolo-Burmese) share a suppletion, whereby a root with a labial onset is used in the year ordinals `last year, this year, next year' (in Stau \ipa{--və}) and a root with a velar onset is used as a classifier (in Stau \ipa{--fku}), as illustrated in Table \ref{tab:year}.\footnote{Rgyalrong languages, including Japhug, are an exception in that they have generalized the labial root to the ordinal too, but this must be a late common Rgyalrong innovation since the closely related Khroskyabs and Stau languages preserve the two roots.}

\begin{table}[H]
\caption{Suppletion in the forms of `year' in Burmo-Qiangic.}  \label{tab:year}
\resizebox{\columnwidth}{!}{
\begin{tabular}{llllllll} \toprule
Meaning&	Tangut& Japhug& Stau &	Pumi (Shuiluo)&	Muya & Proto-Naish\\
\midrule
Last year&	\mo{5168}\mo{2712} \ipa{.jɨ².wji¹}&	\ipa{ja\textbf{pa}}& \ipa{ja\textbf{və}}& 	\ipa{ʑɛ́\textbf{pə}}&	\ipa{jø³³zɑ²⁴}\ & *\textbf{C-ba}\\
This year&	\mo{0748}\mo{2712} \ipa{pjɨ¹.wji¹}&	\ipa{ɣɯj\textbf{pa}}&\ipa{pə\textbf{və}}& 	\ipa{pə\textbf{pə́}}&	\ipa{pə³³\textbf{βə⁵³}}& *\textbf{C-ba} \\
Next year&	\mo{5500}\mo{2712} \ipa{sjij¹.wji¹}&	\ipa{fsaqhe}&	\ipa{se\textbf{və}}& \ipa{ʑɛkhiú}&	\ipa{sæ³³\textbf{βə⁵³}} &*\textbf{C-ba} \\
One year&	\mo{5981}\mo{3305} \ipa{.a-kjiw¹}&	\ipa{tɯ-xpa}&\ipa{e-\textbf{fku}}& 	\ipa{tɜ́-\textbf{kó}}&	\ipa{tɐ⁵⁵-\textbf{kui⁵³}} &*\textbf{kʰu}\\
Two years&	\mo{4027}\mo{3305} \ipa{ njɨɨ¹-kjiw¹}&	\ipa{ʁnɯ-xpa}&\ipa{ɣnə-\textbf{fku}}&	\ipa{ɲí-\textbf{kó}}&	&*\textbf{kʰu}\\
\bottomrule
\end{tabular}}
\end{table}

This implies that at the stage of the common ancestor of Naish and Qiangic, the root ancestral to Stau \ipa{--fku} was already a classifier and thus confirms the idea that the classifier system with numeral prefixes already existed.

These data show that for some subbranch of Burmo-Qiangic (but perhaps not at the Burmo-Qiangic level), a numeral+classifier paradigm has already been grammaticalized, and that the present systems have not been independently re-created, but are at least partially inherited from it.



\subsection{Analogical levelling}
The third possibility to explain the simplicity of the morphology of numeral+classifier paradigms in Rgyalrongic is that although their origin goes back to the common ancestor of Naish, Pumi and Rgyalrongic, they have undergone several layers of analogical levelling which have erased irregular alternations.

The  regular alternations between numerals and their prefixal forms, involving  centralization of vowels (\ipa{--u}  $\rightarrow$ \ipa{--ɤ}, \ipa{--i}  $\rightarrow$ \ipa{--ɯ}) are similar to the \textit{status constructus} alternations that apply to the first member of compounds in Japhug (see \citealt{jacques12incorp}), as in Table \ref{tab:status}. The loss of final consonants  (including \ipa{--ɣ}, \ipa{--z},  \ipa{--t},  \ipa{--m}) found in the numeral prefixes is not generally observed in \textit{status constructus} forms.

\begin{table}[H]
\caption{Regular \textit{status constructus} forms in Japhug.}  \label{tab:status} \centering
\begin{tabular}{lllll}
\toprule
First element & Second element & Compound \\
\midrule
 \ipa{--ku} `head' & \ipa{--rme} `(body) hair'  &\ipa{\textbf{kɤ}-rme} `(head) hair'   \\
  \ipa{si} `tree, wood' & \ipa{--rtaʁ} `branch'  &\ipa{\textbf{sɯ}-rtaʁ} `tree branch'   \\
  \midrule
  \ipa{zrɯɣ} `louse' & \ipa{ndza} `eat'  &\ipa{\textbf{zrɯɣ}-ndza} `praying mantis'   \\  
\bottomrule
\end{tabular}
\end{table}

  However, a few examples of compounds whose first element loses its coda are attested, mainly, but not exclusively, where the second element has a complex cluster (Table \ref{tab:status2}). None of these alternations are productive.


\begin{table}[H]
\caption{Loss of final consonants in status constructus forms in Japhug.}  \label{tab:status2} \centering
\resizebox{\columnwidth}{!}{
\begin{tabular}{lllll}
\toprule
Coda &First element & Second element & Compound \\
\midrule
\ipa{--β}& \ipa{ɴqiaβ} `be bitter' & \ipa{zwɤr} `mugwort'  &\ipa{ɴqia-zwɤr} `Artemisia sp.'  \\
\midrule
\ipa{--t}&\ipa{xtɯt} `be short' & \ipa{rɲɟi} `be long' &\ipa{xtɯ-rɲɟi} `length (n)'  \\
 &\ipa{tsʰɤt} `goat' & \ipa{--ʁrɯ} `horn' &\ipa{tsʰɤ-ʁrɯ} `goat horn'  \\
\midrule
\ipa{--z}&\ipa{qartsʰaz} `deer' & \ipa{--ndʐi} `skin' &\ipa{qartsʰɤ-ndʐi} `deer hide'  \\
\midrule
\ipa{--r}&\ipa{zwɤr} `mugwort' & \ipa{wɣrum} `be white' &\ipa{zwɤ-ɣrum} `Artemisia sp.'  \\
&\ipa{ɕɤr} `night' & \ipa{--χcɤl} `middle' &\ipa{ɕɤ-χcɤl} `middle of the night'  \\
\midrule
\ipa{--ɣ}&\ipa{tɤjmɤɣ} `mushroom' & \ipa{--sti} `alone'  &\ipa{jmɤ-tɤsti} `species of mushroom'  \\
&\ipa{tɯ-mtʰɤɣ} `waist' & \ipa{rŋgɤβ} `attach'  &\ipa{--mtʰɤ-rɴɢɤβ} `(tucking into \\
&&&one's) trousers'  \\
\midrule
\ipa{--ʁ}&\ipa{ɕoʁ} `buckwheat' & \ipa{wɣrum} `be white'  &\ipa{ɕɤ-ɣrum} `buckwheat sp'  \\
&\ipa{paʁ} `pig' & \ipa{ɯ-qa} `foot'  &\ipa{pɤ-qa} `stuffed pig feet'  \\
\bottomrule
\end{tabular}}
\end{table}
 

%stɤtoŋ stod.thuŋ
%thɯrʑi tʰugs.rdʑe

The limited amount of phonological alternations observed in numeral prefixes can thus be treated as a particular case of \textit{status constructus}, generalized to all classifiers, although it originally probably was  restricted to classifiers with a particular type of onset (in particular those with complex consonant clusters).\footnote{Note however that not a single of t*loss of \ipa{--m} could be found in Japhug compounds. Table \ref{tab:status2} is almost completely exhaustive.}

The  fact that some numerals have two competing prefixal forms (for instance \ipa{kɯmŋu--} vs \ipa{kɯmŋɤ--} for \ipa{kɯmŋu} `five') shows that analogy is still synchronically at work in the system, and therefore that a massive generalization of one particular allomorph is probable to have occurred several times in the history of Japhug and other Rgyalrongic languages, on the basis of phonological alternations otherwise attested in the language.

\section{Conclusion}
The present paper contributes in two ways to the comparative linguistics of Burmo-Qiangic languages. 

First, it provides detailed information on numerals and classifiers in Japhug, complementing the brief description  in \citet[185-194]{jacques08}.


Second, it presents a model explaining how the numeral prefixal system found in Japhug came to be the way it is, and documents irregular cases of \textit{status constructus} involving loss of final consonants.


\bibliographystyle{unified}
\bibliography{bibliogj}
\end{document}