\documentclass[xcolor=table]{beamer}
\usepackage{fontspec} 
\usepackage{natbib} 
\usepackage{gb4e} 
\usepackage[table]{xcolor} 
\usepackage{booktabs} 
%\usepackage{color}
\usepackage{graphicx}

 \setmainfont[Mapping=tex-text]{Charis SIL}
\let\sfdefault\rmdefault
%\newcommand{\racine}[1]{\begin{math}\sqrt{#1}\end{math}} 
\newfontfamily\phon[Mapping=tex-text,Ligatures=Common,Scale=MatchLowercase,FakeSlant=0.3]{Charis SIL} 
\newcommand{\ipa}[1]{{\phon \mbox{#1}}} %API tjs en italique
\newcommand{\grise}[1]{\cellcolor{lightgray}\textbf{#1}} 
\newcommand{\ra}{$\Sigma_1$} 
\newcommand{\rc}{$\Sigma_3$} 
\newcommand{\ro}{$\Sigma$} 

\usepackage{tkz-kiviat}
\usetikzlibrary{arrows,automata,positioning}

\tikzset{
    state/.style={
           rectangle,
           rounded corners,
           draw=black, very thick,
           minimum height=2em,
           inner sep=2pt,
           text centered,
           },
}


 \begin{document}

 \title{Paradigm generation in PERL}
 \author{Guillaume Jacques}
 \maketitle

 
  \begin{frame} 
 \frametitle{Programming languages} 
 \begin{enumerate}
\item PERL
\item Python
\item XFST (\citealt{bessley03fsm})
\end{enumerate}

Implementations: \citet{jacques12khaling}, \citet{walther14compactness}
   
  \end{frame}   

\begin{frame}
 \frametitle{Chomsky hierarchy} 
 
 \begin{table}
\centering
\resizebox{\textwidth}{!}{
 \begin{tabular}{llllll}
 \toprule
Language & Automaton & Rules\\
\midrule
Regular	&Finite state automaton	&$A \rightarrow aB$, $A \rightarrow a$\\
Context-free	&Pushdown automaton	&$A \rightarrow \gamma$ \\
Context-sensitive	&Linear-bounded &	$\alpha A \beta \rightarrow \alpha \gamma \beta$\\
	&non-deterministic Turing machine\\
Recursively enumerable	&Turing machine	& (no restrictions)\\
	\bottomrule
 \end{tabular}}
 \end{table}
\end{frame}


\begin{frame}
 \frametitle{Example of formal languages} 


\begin{itemize}[<+->] 
\item Grammar 1

$S \rightarrow A$

$A \rightarrow aA$

$A \rightarrow bB$

$B \rightarrow b$


\item Grammar 2

$S \rightarrow A$

$A \rightarrow aAb$

$A \rightarrow ab$


\item Grammar 3


$S \rightarrow A$

$A \rightarrow aAa$

$A \rightarrow bAb$

$A \rightarrow cAc$

$A \rightarrow \epsilon$

\end{itemize}


\end{frame}
\begin{frame}%{Information distribution within both descriptions}
\frametitle{Evaluating the descriptive economy of both descriptions}
\begin{center}
\mbox{
%\begin{figure}[htbp]
\centering
\begin{tikzpicture}[scale=0.27, every node/.style={scale=0.8}]
\tkzKiviatDiagram[scale=2,label space=5cm,
 radial style/.style ={-},
 gap = 2,
 step = 1,
 lattice = 2]
 {{~~~~~features},{patterns+structure},{\hspace*{-0.65cm} phono\\\hspace*{-1cm}\mbox{+morphono}\mbox{~~~}},{morphological\\operations}}
 \tkzKiviatLine[thick,color=red,mark=none,
 fill=red!20,opacity=.5](1.114,2.182,1.901,1.017)
\tkzKiviatLine[thick,color=blue,mark=none,
 fill=blue!20,opacity=.5](1.380,1.198,1.874,0.988)
)
\draw[thick,red] (5,3) --(6,3);
\draw[thick,blue] (5,2) --(6,2);
\node[text width=3cm] at (8.6,3) {{standard}};
\node[text width=3cm] at (8.6,2) {{morphomic}};
\node at (-1,-1) {{\scriptsize 1~Kbit}};
\node at (-1,-3) {{\scriptsize 2~Kbits}};
\end{tikzpicture}
%\caption{Compared lengths of our two descriptions}
%\label{fig:khaling-kiviat}
%\end{figure}
}
\end{center}
\end{frame}




 \begin{frame} 
 \frametitle{References}
 \tiny
 \bibliographystyle{Linquiry2}
\bibliography{bibliogj}
 \end{frame}
\end{document}