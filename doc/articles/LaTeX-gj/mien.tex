\documentclass[oneside,a4paper,11pt]{article} 
\usepackage{fontspec}
\usepackage{natbib}
\usepackage{booktabs}
\usepackage{xltxtra} 
\usepackage{polyglossia} 
\usepackage[table]{xcolor}
\usepackage{tikz}
\usetikzlibrary{trees}
\usepackage{gb4e} 
\usepackage{multicol}
\usepackage{graphicx}
\usepackage{float}
\usepackage{hyperref} 
\hypersetup{bookmarksnumbered,bookmarksopenlevel=5,bookmarksdepth=5,colorlinks=true,linkcolor=blue,citecolor=blue}
\usepackage[all]{hypcap}
\usepackage{memhfixc}
\usepackage{lscape}
\usepackage{amssymb}
 
\bibpunct[: ]{(}{)}{,}{a}{}{,}

%\setmainfont[Mapping=tex-text,Numbers=OldStyle,Ligatures=Common]{Charis SIL} 
\newfontfamily\phon[Mapping=tex-text,Scale=MatchLowercase]{Charis SIL} 
\newcommand{\ipa}[1]{\textbf{{\phon\mbox{#1}}}} %API tjs en italique
%\newcommand{\ipab}[1]{{\scriptsize \phon#1}} 

\newcommand{\grise}[1]{\cellcolor{lightgray}\textbf{#1}}
\newfontfamily\cn[Mapping=tex-text,Ligatures=Common,Scale=MatchUppercase]{SimSun}%pour le chinois
\newcommand{\zh}[1]{{\cn #1}}
\newfontfamily\mleccha[Mapping=tex-text,Ligatures=Common,Scale=MatchLowercase]{Galatia SIL}%pour le grec
\newcommand{\grec}[1]{{\mleccha #1}}


\newcommand{\sg}{\textsc{sg}}
\newcommand{\pl}{\textsc{pl}}
\newcommand{\ro}{$\Sigma$}
\newcommand{\ra}{$\Sigma_1$} 
\newcommand{\rc}{$\Sigma_3$}  
\newcommand{\dhatu}[2]{|\ipa{#1}| `#2'}
\newcommand{\dhat}[1]{|\ipa{#1}|}
\newcommand{\change}[2]{*\ipa{#1} $\rightarrow$ \ipa{#2}}
 

\XeTeXlinebreakskip = 0pt plus 1pt %
 %CIRCG
 
\newcommand{\zhc}[2]{\zh{#1} \ipa{#2}} 
\newcommand{\mien}[5]{\ipa{#1}^{#2} `#3' (\zh{#4}, p.#5)} 

\begin{document}

\title{The Hmong-Mien contribution to Old Chinese reconstruction}
\author{Guillaume Jacques}
\maketitle

\section*{Introduction}
\citet{sagart03prenasalized}
\citet{maozw92mien}
\citet{ratliff10protohm}
\citet{wang95protomy}
%江底
\section{Anticausative prenasalization}
\citet[14-16]{downer73loanwords}

\citet{wanglz12jieci}
%http://max.book118.com/html/2014/0421/7812829.shtm
 
\mien{khoːi}{1}{open (tr)}{打开}{131} 

\mien{goːi}{1}{open (it)}{开}{131}  pour 开, et
\zh{多作补语祸自然开裂、开放}

\mien{tshɛʔ}{7}{pull down (house)}{拆}{120}
 
\mien{dzɛʔ}{7}{be cracked (of earth, of skin)}{拆开;皴}{151}

\mien{pɛːŋ}{1}{to stretch tight, to pull tight}{绷;拉紧;拔}{120}
\mien{bɛːŋ}{1}{to crack}{裂开}{150-1}
 
\ipa{dei^6} \ipa{ɲwo^4} \ipa{ɲei^1} \ipa{top^8} \ipa{jet^8} \ipa{khai^5} \ipa{phuːi^1} \ipa{bɛːŋ^1} \ipa{ɲa^6}
 
 \zh{地理的豆子全部晒裂了}

\mien{tshɛːŋ}{1}{ssss}{撑、支}{153} 

\mien{dzip}{7}{continue}{接、继续}{121}
\mien{tsip}{7}{receive}{接、承受、接受、迎接}{121}


Autres:
\mien{beu}{1}{toss}{抛}{124} 
\mien{dzɛːŋ}{1}{fight over}{争、争夺}{127} 
\mien{dzaːt}{7}{smear}{擦、抹、涂}{121} 

Anticausative:
\mien{dap}{7}{ssss}{坍、塌}{151} 
\mien{dzaːn}{5}{be scattered}{散}{sss}
\mien{baːŋ}{7}{xxxxx}{崩}{151} 
\mien{duːi}{1}{pile up}{堆、堆放}{153} 
\mien{bjoːt}{8}{boil}{沸腾}{146} 
\subsection{Anticausative prenasalization}
\citet[193-4]{jacques15causative}
\citet[285-288]{jacques15spontaneous}
\citet{jacques15derivational.khaling}

\section{How Old is prenasalization in Mien?}

\section*{Conclusion}

\bibliographystyle{unified}
\bibliography{bibliogj}
\end{document}
