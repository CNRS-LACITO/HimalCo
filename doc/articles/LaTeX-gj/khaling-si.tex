\documentclass[twoside,a4paper,11pt]{article} 
\usepackage{polyglossia}
%\usepackage{fontspec}
\usepackage{natbib}
\usepackage{booktabs}
\usepackage{xltxtra} 
\usepackage{longtable}
 \usepackage{geometry}
\usepackage[table]{xcolor}
\usepackage{color}
\usepackage{multirow}
\usepackage{gb4e} 
\usepackage{multicol}
\usepackage{graphicx}
\usepackage{float}
\usepackage{hyperref} 
\hypersetup{colorlinks=true,linkcolor=blue,citecolor=blue}
\usepackage{amssymb} 
\usepackage{memhfixc}
\usepackage{lscape}
\usepackage[footnotesize,bf]{caption}

%%%%%%%%%%%%%%%%%%%%%%%%%%%%%%%
%\setmainfont[Mapping=tex-text,Numbers=OldStyle,Ligatures=Common]{Charis SIL} 
\newfontfamily\phon[Mapping=tex-text,Ligatures=Common,Scale=MatchLowercase,FakeSlant=0.3]{Charis SIL} 
 \newcommand{\ipa}[1]{{\phon#1}} 
\newcommand{\ipab}[1]{{\phon #1}}
\newcommand{\ipapl}[1]{{\phondroit #1}} 
\newcommand{\captionft}[1]{{\captionfont #1}} 
\newfontfamily\cn[Mapping=tex-text,Scale=MatchUppercase]{MingLiU}%pour le chinois
\newcommand{\zh}[1]{{\cn #1}}

 
\newcommand{\racine}[1]{\begin{math}\sqrt{#1}\end{math}} 
\newcommand{\grise}[1]{\cellcolor{lightgray}\textbf{#1}} 
\newcommand{\grisf}[1]{\cellcolor{gray}\textbf{#1}} 
\newcommand{\ra}{$\Sigma_1$} 
\newcommand{\rbb}{$\Sigma_2$} 
\newcommand{\rc}{$\Sigma_3$} 
\newcommand{\ro}{$\Sigma$} 
\let\eachwordone=\textbf

\begin{document}



\title{Reflexive paradigms in Khaling\footnote{This research was funded by the HimalCo project (ANR-12-CORP-0006) and is related to the research strand LR-4.11 ‘‘Automatic Paradigm Generation and Language Description’’ of the Labex EFL (funded by the ANR/CGI). We would like to thank two anonymous reviewers and the editors for insightful suggestions. Glosses follow the Leipzig glossing rules, to which we add \textsc{habit} habitual, \textsc{impers} impersonal, \textsc{sens} sensory.} } 

\author{Guillaume Jacques\\Aimée Lahaussois\\Dhan Bahadur Rai}

\maketitle

\textbf{Abstract}: Khaling, like other Kiranti languages, has a Reflexive / Middle suffix -si, the main function of which is to reduce the valence of verbs. The present paper describes the complex morphological alternations observed in the reflexive paradigms, and provides a detailed account of the various meanings of the -si derivation in Khaling.  


\textbf{Keywords}: Khaling, Kiranti, reflexive,  autobenefactive, middle, antipassive, anticausative, reciprocal


 \section{Introduction}
Like all Kiranti languages, Khaling presents a reflexive conjugation distinct from the simplex intransitive conjugation   described in \citet{jacques12khaling}.\footnote{\citet{toba84khaling} includes one reflexive paradigm, but without any analysis of the stem alternations and of the semantics of the \ipa{--si} construction.}  This paper describes the reflexive conjugation and presents the set of rules necessary to generate all regular forms. It is divided into three sections. 

First, we present the set of affixes in the reflexive conjugation and compare it with the intransitive vowel-final  conjugation. We also provide comparative data from  closely related Kiranti languages and discuss the historical origin of the idiosyncrasies observed in these paradigms.

Second, we analyse the stem alternations observed in the reflexive paradigm, and show that the rules necessary to generate the forms have already been proposed in  \citet{jacques12khaling}. Third, we describe the semantic values of   reflexive derivation.

\section{Affixal paradigm}
The conjugation of reflexive verbs is close to that of vowel-final intransitive verbs. As in other Kiranti languages (a more detailed comparison with Dumi, Wambule and Limbu is provided below), the paradigm includes a reflexive suffix  whose main allomorph is \ipa{--si} in Khaling. 

Table (\ref{tab:nyamsi}) lists the forms of the reflexive paradigm of   |\ipa{nɛm-si}|\footnote{The verbs  given in this format are the underlying roots used to generate paradigms. Conjugated verb forms exhibit stem alternations in the vowel, tone and final consonant.}
 \ipa{nɛmsinɛ}  ``to immerse oneself, to dive", a verb chosen for its lack of stem alternations.

\begin{table}[h]
\caption{The conjugation of |\ipa{nɛm-si}|  ``to immerse oneself"  } \centering \label{tab:nyamsi}
\begin{tabular}{l|l|l|l}  
\toprule
& non-past & past & imperative\\
\midrule
\textsc{1s}  &  \ipa{nɛm-si-ŋʌ}   &  \ipa{nɛm-\textbf{tʌsu}} / \ipa{nɛm-siŋʌtʌ}  \grise{}\\ 
\textsc{1di}  &  \ipa{nɛm-si-ji}   &  \ipa{nɛm-sî-jti} \\
\textsc{1de}  &  \ipa{nɛm-si-ju}   &  \ipa{nɛm-sî-jtu} \\ 
\textsc{1pi}  &  \ipa{nɛm-si-ki}   &  \ipa{nɛm-si-ktiki} \\ 
\textsc{1pi}  &  \ipa{nɛm-si-kʌ}   &  \ipa{nɛm-si-ktʌkʌ} \\ 
\midrule
\textsc{2s}  &  \ipa{ʔi-nɛ̂m-si}   &  \ipa{ʔi-nɛm-tɛ-si}   &  \ipa{nɛm-si-je} \\ 
\textsc{2d}  &  \ipa{ʔi-nɛm-si-ji}   &  \ipa{ʔi-nɛm-sî-jti}   &  \ipa{nɛ̄m-sî-jje} \\
\textsc{2p}  &  \ipa{ʔi-nɛm-si-ni}   &  \ipa{ʔi-nɛm-\textbf{tɛ̂nnu}} \grise{}  &  \ipa{nɛm-\textbf{nuje}} \grise{}\\ 
\midrule
\textsc{3s}  &  \ipa{nɛ̂m-si}   &  \ipa{nɛm-tɛ-si} \\ 
\textsc{3d}  &  \ipa{nɛm-si-ji}   &  \ipa{nɛm-sî-jti} \\ 
\textsc{3p}  &  \ipa{nɛm-si-nu}   &  \ipa{nɛm-\textbf{tɛ̂nnu}} \grise{}\\ 
\bottomrule
\end{tabular}
\end{table}
This paradigm presents overabundance, as two competing variants are attested for all first person past forms: \ro{}\ipa{--tʌsu} and \ro{}\ipa{--siŋʌtʌ}.



 The personal suffixes and prefixes of reflexive verbs are essentially the same as those of  intransitive verbs with a vowel-final root, as a comparison with Table (\ref{tab:pi}) shows. All non-past forms are fully predictable once the reflexive suffix \ipa{--si} is taken into account. In the rest of the paradigm,  only three suffixes   differ  from those for non-reflexive verbs, and they are indicated in bold in Table (\ref{tab:nyamsi}).
 
   The   suffix \ipa{--tʌsu} for \textsc{1sg} past  is non-predictable from either the \textsc{1sg} non-past form or from the rest of the past tense paradigm. The variant \ipa{--siŋʌtʌ} on the other hand can be generated from the non-past suffix \ipa{--siŋʌ} following the same rule as simplex intransitive verbs (\ipa{pi-ŋʌ} `I come', \ipa{pi-ŋʌtʌ} `I came'). This   suggests that  \ipa{--tʌsu} is the inherited form, while  \ipa{--siŋʌtʌ} is analogically renewed from the rest of the paradigm, following four-part analogy (\citealt[167-175]{hock91principles}), see Table \ref{tab:four}.

 \begin{table}[h]
\caption{Four-part analogy}  \label{tab:four} \centering
\begin{tabular}{l|l|ll}
\toprule
  & simplex verb `come' & reflexive verb `immerse oneself' \\
 \midrule
\textsc{1sg.n.pst} &\ipa{pi-ŋʌ} & \ipa{nɛm-siŋʌ} \\
\textsc{1sg.pst}  &\ipa{pi-ŋʌ-\textbf{tʌ}} & \ipa{nɛm-tʌsu} $\rightarrow$ \ipa{nɛm-siŋʌ-\textbf{tʌ}} \\
\bottomrule			
\end{tabular}
\end{table} 

 
 The second and third person past \ipa{--tɛ̂nnu} and  the imperative plural \ipa{--nuje} do not present vowel syncope as do the regular  suffixes of vowel final verbs \ipa{-tnu} and \ipa{--nje}. These reflexive suffixes are more similar to the suffixes found in   consonant-final paradigms (\ipa{--tɛnu} and \ipa{--nuje} respectively for \textsc{2/3pl} past and imperative respectively) which do not have vowel loss.
 


\begin{table}[h] 
\caption{Intransitive verb \ipa{pi} ``come (level)" } \centering \label{tab:pi}
\begin{tabular}{l|l|l|l}  
 \toprule
& non-past & past & imperative\\
\midrule
\textsc{1s}  &  \ipa{pi-ŋʌ}  &  \ipa{pi-ŋʌtʌ}   \\
\textsc{1di}  &  \ipa{pi-ji}  &  \ipa{pî-jti}   \\
\textsc{1de}  &  \ipa{pi-ju}  &  \ipa{pî-jtu}   \\
\textsc{1pi}  &  \ipa{pi-ki}  &  \ipa{pi-ktiki}   \\
\textsc{1pe}  &  \ipa{pi-kʌ}  &  \ipa{pi-ktʌkʌ}   \\
\midrule
\textsc{2s}  &  \ipa{ʔi-pi}  &  \ipa{ʔi-pūː-tɛ}  &  \ipa{pūː-je}   \\
\textsc{2d}  &  \ipa{ʔi-pi-ji}  &  \ipa{ʔi-pî-jti}  &  \ipa{pî-jje}   \\
\textsc{2p}  &  \ipa{ʔi-pi-ni}  &  \ipa{ʔi-pūː-tnu}  &  \ipa{pû-nje}   \\
\midrule
\textsc{3s}  &  \ipa{pi}  &  \ipa{puː-tɛ}   \\
\textsc{3d}  &  \ipa{pi-ji}  &  \ipa{pî-jti}   \\
\textsc{3p}  &  \ipa{pi-nu}  &  \ipa{pūː-tnu}   \\
\bottomrule
\end{tabular}
\end{table}


In most forms, the reflexive suffix \ipa{--si} appears between the verb stem and the personal suffixes. There are only three sets of exceptions to this rule:

\begin{itemize}
\item The second and third person singular forms, where \ipa{--si} occurs after the  non-first person past marker \ipa{--tɛ}, resulting in \ipa{-tɛ-si} where we would have expected †\ipa{--si-tɛ}.
\item The   second and third person plural forms, where \ipa{--si} has an allomorph  \ipa{n} occurring between the  non-first person past marker \ipa{--tɛ}  and the plural suffix --\ipa{nu}.  The suffix in these forms is thus \ipa{--tɛ̂nnu} instead of expected †\ipa{--si-tɛ-nu}. It is unclear if a sound change *\ipa{sn} > --\ipa{nn}-- ever existed in Khaling; in any case the group \ipa{--sn--} is attested across morpheme boundaries, as in the imperative plural of transitive \ipa{--t} final verbs (\ipa{ses-nuje} ``kill him" \textsc{2p>3},  \citealt[1118]{jacques12khaling}). 
\item The second plural imperative \ipa{--nuje}, which lacks an  overt reflexive marker.
\end{itemize}

These particularities of the Khaling reflexive paradigm appear to be isolated among Kiranti languages. Even in Dumi, the closest relative of Khaling, the forms corresponding to Khaling \textsc{1sg} past, \textsc{2sg/pl} past and \textsc{3sg} past are regular, as shown in Table (\ref{tab:dumi-khyal}).\footnote{Note the suffix --\ipa{t}, the past marker in most other Kiranti langauges, is the Dumi non-past marker. A past / non-past flip-flop has occurred. } 

\begin{table}[H] \centering
\caption{Comparison of the Dumi and Khaling paradigms of the verb ``to wash oneself"} \label{tab:dumi-khyal}
\begin{tabular}{llll}
\toprule
&	Dumi &	Khaling	\\
\midrule
root & |\ipa{sir-si}| & |\ipa{sur-si}|\\
\midrule
\textsc{1sg} &	\ipa{sir-\textbf{s}-tɨ} &	\ipa{sʌr-\textbf{si}-ŋʌtʌ / sʌr-tʌ-\textbf{su}} &	\\
\textsc{1pi} &	\ipa{sir-\textbf{si}-kti} &	\ipa{sʌr-\textbf{si}-ktiki} &	\\
\textsc{2sg} &	\ipa{a-sir-\textbf{s}-ti} &	\ipa{ʔi-sʌr-tɛ-\textbf{si}} &	\\
\textsc{2pl} &	\ipa{a-sir-\textbf{s}-ti-ni} &	\ipa{ʔi-sʌr-tɛ̂\textbf{n}nu} &	\\
\textsc{3sg} &	\ipa{sir-\textbf{s}-ti} &	\ipa{sʌr-tɛ-\textbf{si}} &	\\
\bottomrule
\end{tabular}
\end{table}


Table \ref{tab:dumi-khyal} compares our Khaling data (the verb |sur-si| ``to wash oneself") with the paradigm in \citet[125;362-3]{driem93dumi}. In Dumi, the reflexive has two allomorphs \ipa{--s} and \ipa{--si}, the selection of which depends on the following suffix. 

Apart from this allomorphy, the reflexive suffix in Dumi does not merge with personal suffixes to form portmanteau morphemes, and it always occurs to the left of all personal suffixes. 

 
 

%\citet[306]{opgenort04wambule}


\section{Stem alternations}
In comparison with consonant-final paradigms, which can present up to five distinct stems for intransitive verbs (see for instance Table \ref{tab:khot}) and twelve for transitive verbs, no more than three distinct stems need to be posited for any reflexive verb.


\begin{table}[H]
\caption{Stem alternation of  the consonant-final intransitive conjugation  of |\ipa{kʰot}| ``to go"} \label{tab:khot} \centering
\begin{tabular}{llll}
\toprule
form&stem\\
\midrule
\textsc{1sg.npst}&\ipa{kʰoɔ̂j-ŋʌ} \\
\textsc{1di.npst}&\ipa{kʰɵʦ-i} \\
\textsc{1pi.npst}&\ipa{kʰoɔç-ki} \\
\textsc{3p.npst}&\ipa{kʰoɔ̂n-nu} \\
\textsc{1sg.pst}&\ipa{kʰɵs-tʌ} \\
\textsc{2sg.imp}&\ipa{kʰɵʦʦ-e} \\
\bottomrule
\end{tabular}
\end{table}

The stem  alternations observed in reflexive verbs can, with a few exceptions, be generated by the rules  described in  \citet[1108-1111]{jacques12khaling}. Verb stems in Khaling can be derived from roots following the syllabic template $C(r,l)VC_f(t)$ for which only six vowels (|\ipa{a}|, |\ipa{ɛ}|, |\ipa{e}|, |\ipa{i}|, |\ipa{o}|, |\ipa{u}|) and eight final consonants $C_f$ (|\ipa{-p}|, |\ipa{-t}|, |\ipa{-k}|, |\ipa{-m}|, |\ipa{-n}|, |\ipa{-ŋ}|, |\ipa{-r}|, |\ipa{-l}|) need to be distinguished. Some verbs have roots with final clusters in |\ipa{-Ct}| (hence the $C_ft$ in the template, cf \citealt[1119-1122]{jacques12khaling}), but the reflexive derivation neutralizes the contrast between $CVC$ and $CVCt$ roots: the final cluster |\ipa{-Ct}| is simplified to |\ipa{-C}| when a verb receives the suffix --\ipa{si}.

In reflexive verbs, we find three distinct stems: a weak stem (\ra{}), a strong stem (\rbb{}), and a nasalized strong stem (\rc{}). Their distribution is shown in Table \ref{tab:reflstems}: 
the weak stem is found in all dual forms, the strong stem is restricted to first person plural, and the nasalized strong stem appears with all other forms (all singular forms and non-first person plural).

\begin{table}[h]
\caption{ Distribution of verbal stems in the Khaling reflexive paradigm}
\label{tab:reflstems} \centering 
\begin{tabular}{l|l|l|l|l|l|l|l|l|l|l|l|l}  \toprule
\textsc{1s}  &  \ipa{\rc{}-si-ŋʌ}   &  \ipa{\rc{}-\textbf{tʌsu}} / \ipa{\rc{}-siŋʌtʌ}   \\ 
\textsc{1di}  &  \ipa{\ra{}-si-ji}  \grise{} &  \ipa{\ra{}-sî-jti} \grise{} \\
\textsc{1de}  &  \ipa{\ra{}-si-ju} \grise{}   &  \ipa{\ra{}-sî-jtu} \grise{} \\ 
\textsc{1pi}  &  \ipa{\rbb{}-si-ki} \grisf{}  &  \ipa{\rbb{}-si-ktiki} \grisf{}  \\ 
\textsc{1pi}  &  \ipa{\rbb{}-si-kʌ}  \grisf{}   &  \ipa{\rbb{}-si-ktʌkʌ} \grisf{} \\ 
\midrule
\textsc{2s}  &  \ipa{ʔi-\rc{}-si}   &  \ipa{ʔi-\rc{}-tɛ-si}   &  \ipa{\rc{}-si-je} \\ 
\textsc{2d}  &  \ipa{ʔi-\ra{}-si-ji}  \grise{}  &  \ipa{ʔi-\ra{}-sî-jti} \grise{}   &  \ipa{\ra{}-sî-jje} \grise{} \\
\textsc{2p}  &  \ipa{ʔi-\rc{}-si-ni}   &  \ipa{ʔi-\rc{}-\textbf{tɛ̂nnu}}   &  \ipa{\rc{}-\textbf{nuje}}  \\ 
\midrule
\textsc{3s}  &  \ipa{\rc{}-si}   &  \ipa{\rc{}-tɛ-si} \\ 
\textsc{3d}  &  \ipa{\ra{}-si-ji} \grise{}   &  \ipa{\ra{}-sî-jti}\grise{} \\ 
\textsc{3p}  &  \ipa{\rc{}-si-nu}   &  \ipa{\rc{}-\textbf{tɛ̂nnu}}  \\ 
\bottomrule
\end{tabular}
\end{table}

In reflexive verbs, the following three rules are needed to derive the weak stem from the root:

\begin{enumerate}
\item The final dentals |\ipa{-t}| and |\ipa{-n}| change to --\ipa{s}, with lengthening of the preceding  vowel in the case of |\ipa{-n}|. The other final consonants are not affected.
\item The back vowels |\ipa{o}| and |\ipa{u}| are fronted to \ipa{ɵ} and \ipa{ʉ} when not followed by a velar.
\item In the case of vowel-final roots, an --\ipa{s} is inserted between the stem and the suffixes.
\end{enumerate}
 
For the strong stem, the rules are the following:

\begin{enumerate}
\item Final  |\ipa{-t}|  changes to --\ipa{s}.
\item Final  |\ipa{-n}|  changes to --\ipa{j}.
\item |\ipa{i}| in closed syllables changes to \ipa{ʌ}.
\item The back vowels  |\ipa{o}|  and |\ipa{u}|  change  to \ipa{oɔ} and \ipa{ʌ} in roots with a non-velar coda.
\item The back vowels |\ipa{o}| and |\ipa{u}| are fronted to \ipa{ɵ} and \ipa{ʉ} in vowel-final roots.
\item In the case of vowel-final roots, an --\ipa{s} is inserted between the stem and the suffixes.
\end{enumerate}
The nasalized strong stem shares  rules with the strong stem 1-5 (but not rule 6), to which are added the following:
\begin{enumerate}
\item Root-final  |\ipa{-p}|, |\ipa{-t}|, |\ipa{-k}| change to the corresponding nasals \ipa{--m}, \ipa{--n}, \ipa{--ŋ} and the stem receives falling tone.
\item Root-final |\ipa{-ŋ}| changes to \ipa{-n}.
\item In the case of vowel-final roots, an --\ipa{n} is inserted between the stem and the suffixes and the stem receives falling tone.
\end{enumerate}
The nasalization observed in the nasalized strong stem has been reported in most other Kiranti languages, and \citet[127]{driem93dumi} accounts for a similar phenomenon in Dumi by positing an underlying form |\ipa{--nsi}| for the cognate suffix. Although such an analysis may be valid from a diachronic point of view, it is unclear how to explain the absence of nasalization in dual and first person plural forms in Khaling, which unlike in Dumi are not subject to nasalization.

The stem alternations of reflexive paradigms have two particularities not found in the simple paradigms studied in \citet{jacques12khaling}. 

First, the strong nasalized stems of sonorant-final roots do  not receive a falling tone when followed by the past tense suffix --\ipa{t}; thus, while \ipa{nɛ̂m-tɛ} `he immersed it' ( \textsc{3sg} past) has a falling tone, the reflexive \ipa{nɛm-tɛ-si} does not.

Second, there is no irregular conjugation for |\ipa{--ak}| roots (unlike for simple intransitive verbs, cf \citealt[1115]{jacques12khaling}) .

These morphophonological alternations have been implemented in PERL and the paradigms generated by the script have been thoroughly revised with several native speakers. This computational application and its  verification convincingly suggest that the analysis provided here can correctly be used to predict  all forms. 


However, not all speakers of Khaling, especially in the younger generation, feel confident about all the forms of the paradigms.  %We observed analogical levelling  in two cases.
In particular, in the case of vowel-final roots, the application of the above rules generates a paradigm with alternating \ipa{n} and \ipa{s}, as in the case of |\ipa{no-si}| ``to rest" (Table \ref{tab:no}). Some speakers generalize the \ipa{n}-forms to the whole paradigm, resulting in a conjugation without any stem alternation.

\begin{table}[H]
 \centering 
\caption{ \ipa{nɵ̂nsinɛ}  ``to rest"  } \label{tab:no}
\begin{tabular}{l|l|l|l|l|l|l|l|l|l|l|l|l}  \toprule
&original paradigm& analogical paradigm \\
\midrule
\textsc{1s} & \ipa{nɵ̂nsiŋʌ} &\ipa{nɵ̂nsiŋʌ} \\ 
\textsc{1di} & \ipa{nɵssiji} &\ipa{nɵ̂nsiji} \grise{}  \\
\textsc{1de} & \ipa{nɵssiju} & \ipa{nɵ̂nsiju} \grise{}  \\ 
\textsc{1pi} & \ipa{nɵssiki} & \ipa{nɵ̂nsiki}  \grise{} \\ 
\textsc{1pe} & \ipa{nɵssikʌ} & \ipa{nɵ̂nsikʌ} \grise{}  \\ 
\textsc{2s} & \ipa{ʔinɵ̂nsi} & \ipa{ʔinɵ̂nsi}   \\ 
\textsc{2d} & \ipa{ʔinɵssiji} & \ipa{ʔinɵ̂nsiji}  \grise{}  \\
\textsc{2p} & \ipa{ʔinɵ̂nsini}  & \ipa{ʔinɵ̂nsini}    \\ 
\textsc{3s} & \ipa{nɵ̂nsi} & \ipa{nɵ̂nsi}   \\ 
\textsc{3d} & \ipa{nɵssiji} & \ipa{nɵ̂nsiji}  \grise{} \\ 
\textsc{3p} & \ipa{nɵ̂nsinu}  & \ipa{nɵ̂nsinu} \\ 
\bottomrule
\end{tabular}
\end{table}


\section{Syntax and semantics}
Detransitive \ipa{–si}  verbs, aside from the morphology described above, present common morphosyntactic properties: they are intransitive both from the point of view of verbal morphology (absence of transitive direct or inverse markers) and from the point of view of case marking (no ergative on the S). However,  the relationship between the S of the \ipa{–si}  verb and the A and P of the base verb is not straightforward, and several subtypes can be distinguished.

In this section, we first study the various subtypes of detransitive \ipa{--si} derivation (reflexive, autobenefactive, reciprocal, antipassive, and impersonal). Then, we present some additional semantic changes caused by this derivation. Although the diverse semantic effects associated with the \ipa{--si} derivation are reminiscent of the oft-cited category of `middle', this fuzzy concept is of little help in describing in detail  the specific uses of this derivation and we prefer to avoid it, opting for a more precise terminology.

\subsection{Categories} \label{sec:categories}
There are five main subtypes of \ipa{--si} derivations. The reflexive, autobenefactive and impersonal subject interpretations of the \ipa{--si} derivation are productive, while the two remaining ones (reciprocal and antipassive) are each restricted to a few verbs.  


\subsubsection{Reflexive}
The most common effect of \ipa{–si} is to derive a reflexive verb from a transitive one. The S of the reflexive verb corresponds to both the A and the P of the base verb; this  can be represented by  formula \ref{ex:refl}.

\begin{exe}
\ex \label{ex:refl}
\glt $A = P \rightarrow S$
\end{exe}

Example \ref{ex:sursi} illustrates three reflexive verbs derived from the transitive |\ipa{sent}| `look at', |\ipa{sur}| `wash' and |\ipa{ɦur}| `wash hair'. The emphatic pronoun \ipa{--tāːp} is optional and its presence does not  imply a reflexive interpretation. The transitive use of the base verb |\ipa{ɦur}| `wash hair'  is shown in \ref{ex:sur}, with the A marked in the ergative.

\begin{exe}
\ex \label{ex:sursi}
\gll \ipa{gʰole}  	\ipa{kɛ̄m-ʔɛ}  	\ipa{ʔu-tāːp}  	\ipa{sēi-si-nɛ,}  	\ipa{sʌ̄r-si-nɛ,}  	\ipa{ɦʌ̄r-si-nɛ}  	\ipa{mu-jʌt-w-ɛ}  \\
much work-\textsc{erg} \textsc{3sg}-self  look.at\textsc{-refl-inf} wash\textsc{-refl-inf} wash.hair\textsc{-refl-inf} \textsc{neg}-have.time-\textsc{irr}-2/3 \\
\glt Because of all this work, she did not have time to look after herself, to wash (her body) and her hair. (Solme Lamalit 31)
\end{exe}

\begin{exe}
\ex \label{ex:sur}
\gll
 \ipa{ʔu-mɛ̂m-ʔɛ}  	\ipa{ʦɵʦʦɵ}  	\ipa{ɦʉ̂r-t-ɛ.}  \\
\textsc{3sg.poss}-mother-\textsc{erg} child wash.hair-\textsc{pst}-2/3 \\
\glt The mother washed her child's hair.
\end{exe}
 
 In some cases, the reflexive is derived from a verb with a historical \ipa{--t} applicative post-final whose base form has disappeared. A good example is provided by |ʔipt| ``put to sleep (vt)".  *|ʔip| does not exist in Khaling (though cognates of it can be found even outside of Kiranti). The simplex verb has been replaced by the reflexive form of |ʔipt|, |ʔipt-si| `to sleep (vi)' (literally `put oneself to sleep'). Interestingly, this replacement appears to be very old, as all Kiranti languages appear to form their verb `to sleep' with a reflexive form, suggesting that in proto-Kiranti,  the reflexive had already replaced the base intransitive verb. This process has typological parallels in other language families (for instance, French \textit{se coucher} `to go to bed' from \textit{coucher} `to put to bed').
 
\subsubsection{Autobenefactive}

Another very productive interpretation of the  \ipa{--si} derivation is the autobenefactive, whereby the A of the base verb is converted to S, with  P remaining as an unmarked adjunct.   It corresponds to  `indirect reflexive' in \citet{kemmer93middle}'s terminology.
 
\begin{exe}
\ex \label{ex:autoben}
\glt $A  \rightarrow S$
\glt $P  \rightarrow adjunct$
\end{exe}
 
This  productive formation can be illustrated with |\ipa{lom-si}| `search for oneself' as in \ref{ex:lomsi}; the transitive use of the base verb |\ipa{lom}| `search, look for' is shown in \ref{ex:lom}.

\begin{exe}
\ex \label{ex:lomsi} 
\gll
\ipa{ʔûŋ}  	\ipa{sʌ̄ŋ}  	\ipa{loɔ̄m-si-ŋʌ.}  \\
\textsc{1sg} wood search-\textsc{refl-1sg:S/P} \\
\glt I search for firewood for myself.
 \end{exe}
 
 \begin{exe}
\ex \label{ex:lom} 
\gll
 \ipa{kʰɵle-ʔɛ}  	\ipa{pɛp}  	\ipa{lɵ̂m-t-ɛ-nu}  \\
 all-erg father search-\textsc{pst-2/3-pl} \\
 \glt They looked for the father. (Khamdime)
 \end{exe}
 

\subsubsection{Impersonal subject} 
The impersonal subject interpretation of the \ipa{--si} derivation differs from the previous ones in that it can not only be applied to transitive verbs, but also to intransitive ones.

With transitive verbs, the impersonal subject \ipa{--si} construction implies either  (i) a facilitative meaning, implying that the action takes place easily (with or without an agent) (ii) a passive meaning, with an unspecified agent that is still present semantically. Example \ref{ex:kriptsi} illustrates that the same surface sentence can have different interpretations depending on the context. 
   
   \begin{exe}
\ex \label{ex:kriptsi} 
\gll 
 \ipa{sâː}  	\ipa{krʌ̂m-si.}  \\
 vegetable cut-\textsc{impers.S/A} \\
\glt Passive: The green vegetables are cut (by someone).
\glt Facilitative: The green vegetables cut easily.
\end{exe}
 
 With intransitive verbs, this form expresses an unspecified S argument, as in \ref{ex:khoantyasi}.
 
    \begin{exe}
\ex \label{ex:khoantyasi} 
\gll 
 \ipa{kʰoɔ̂n-tɛ-si.}  \\
go-\textsc{pst-impers.S/A} \\
\glt (People) went (there).
\end{exe}

The story `The hunter and the old man' (based on true events) provided in the appendix of this paper illustrates a playful use of the impersonal subject forms. An old man asks a hunter about a hunt, and the young man, reluctant to give a clear answer, repeatedly uses impersonal forms of intransitive and transitive verbs to hide the identity of who did what during the hunt, in particular whether he or someone else was responsible for spotting a deer and killing it.

\subsubsection{Reciprocal}
The reciprocal  interpretation of the \ipa{--si} derivation, like the reflexive one, entails a conversion of both the A and P of the transitive verb into the S of the intransitive one. However, in a reciprocal event, for two participants (or groups of participants) $x$ and $y$, a reciprocal event will involve $x$ acting on $y$ and $y$  acting on $x$ simultaneously or in alternation.  Thus, such an interpretation is only possible with a non-singular S. As shown in \ref{ex:recip}, while both the A and P of the transitive verb correspond to the S of the  \ipa{--si} verb, there is no identification of A and P.

 \begin{exe}
\ex \label{ex:recip}
\glt $A \neq P  \rightarrow S$ \textsc{ns}

\end{exe}

Reciprocal interpretation is attested with verb whose semantics belong to \citet{kemmer93middle}'s categories of `naturally reciprocal events' and `naturally collective actions'.


An example of the reciprocal use of \ipa{--si} is found in the verb |\ipa{pʰak-si}| `to separate', derived from |\ipa{pʰak}| `to separate (vt)', as in examples 


\begin{exe}
\ex 
\gll 
\ipa{daʣubhai}  	\ipa{ʔok}  	\ipa{pʰak-si-ktʌkʌ.}  \\
brothers \textsc{1pe} separate-\textsc{refl-pst:1pe} \\
\glt We brothers separated.
\end{exe}

\begin{exe}
\ex 
\gll \ipa{ʔʌ̄m-ʔɛ} 	\ipa{ki-pɛ} 	\ipa{ɦʌs} 	\ipa{phâːk-t-ɛ-su.} \\ 
\textsc{3sg-erg} fight-\textsc{nmlz}:S/A person separate-\textsc{pst}-2/3-\textsc{du} \\
\glt He separated the fighting people.
\end{exe}


\subsubsection{Antipassive}
In the antipassive  interpretation of the \ipa{--si} derivation, the A of the transitive base verb becomes S of the  \ipa{--si} verb, while the P is deleted, but remains syntactically recoverable. The  examples with such an interpretation  group semantically around cognition and emotion verbs. The A of the transitive base verb and the S of the derived \ipa{--si} verb correspond to the experiencer, and the P of the base verb to the stimulus.

 \begin{exe}
\ex \label{ex:antipass}
\glt $A  \rightarrow S$  
\glt $P \rightarrow \varnothing  $  
\end{exe}

Example \ref{ex:ghryamtsi} illustrates this type of interpretation, in comparison with the base verb |\ipa{gʰrɛmt}| `be disgusted by' in \ref{ex:ghryamt}.

 \begin{exe}
\ex \label{ex:ghryamtsi} 
\gll \ipa{gʰrɛ̄m-si-ŋʌ}\\
 be.disgusted.by-\textsc{refl-1sg:S/P} \\
\glt  I feel disgust.
\end{exe}

\begin{exe}
\ex \label{ex:ghryamt} 
\gll 
  	\ipa{lokpei}  	\ipa{ghrɛ̄md-u.}  \\
leech  be.disgusted.by-\textsc{1sg$\rightarrow$3} \\
 \glt  I am disgusted by leeches.
\end{exe}

The patient demoted during derivation can be recovered as an adjunct marked with the ablative \ipa{--kʌʔʌ}, as in \ref{ex:ghryamtsi-kA}.

 \begin{exe}
\ex \label{ex:ghryamtsi-kA} 
\gll 	\ipa{lokpei-kʌʔʌ} \ipa{ghrɛ̄m-si-ŋʌ}\\
leech-\textsc{abl} be.disgusted.by-\textsc{refl-1sg:S/P} \\
 \glt  I feel disgust because of leeches.
\end{exe}
 

The verb |\ipa{mimt}| `to think about' (example \ref{ex:mamta}) and its derived form |\ipa{mimt-si}| `to think' present a case of antipassive derivation distinct from the previous one. 

\begin{exe}
\ex \label{ex:mamta}
\gll
\ipa{ʔuŋʌ} 	\ipa{ʔʌ-jɛʦhɛ} 	\ipa{mʌ̂m-t-ʌ.} \\
\textsc{1sg:erg} \textsc{1sg.poss}-lover think.about-\textsc{pst-1sg} \\
\glt I thought about my lover.
\end{exe}

The derived verb |\ipa{mimt-si}| `to think', like all antipassives, has the original A as its S, but the original P is not recoverable with an ablative postpositional phrase. Instead, the stimulus can be expressed as direct speech,\footnote{Direct speech is morphosyntactically recognizable as such because of the lack of nominalization, unlike other complement clauses.} with a specific intonation, as in \ref{ex:mimtsi}.

\begin{exe}
\ex \label{ex:mimtsi}
\gll ``\ipa{ʦɵʦʦɵ}  	\ipa{mu-ŋʌ-t-ʌ-lo}  	\ipa{bʰʌ̄ŋte}  	\ipa{ʦei-w-ʌsu-kʰo}  	\ipa{nōː-w-ɛ}  	\ipa{rʌ̄iʦʰʌ}"  	\ipa{mʌm-si-ŋʌ.}  \\
child be-\textsc{1sg:S/P-pst-1sg} well teach-\textsc{irr-refl:1sg}-if good-\textsc{irr-2/3} \textsc{sens} remember-\textsc{refl-1sg} \\
\glt  I think to myself "It would have been good if I had studied hard as a child".
\end{exe}

When |\ipa{mimt-si}| `to think' appears with an NP as in  \ref{ex:mimtsi2}, even though it might look like an unmarked adjunct corresponding to the original P, the intonation makes it clear that here too, the NP in question is a nominal predicate expressed as direct speech.

\begin{exe}
\ex \label{ex:mimtsi2}
\gll 
``\ipa{ʔʌ-mɛ̂m-po} 	\ipa{ʔu-gɵ}" 	\ipa{mʌm-si-ŋʌ-t-ʌ-nʌ} 	\ipa{tû-ŋ-t-ʌ.}
 \\
 \textsc{1sg.poss}-mother-\textsc{gen} \textsc{3sg.poss}-clothes think.about-\textsc{refl-1sg-pst-1sg-sequ} put-\textsc{1sg-pst-1sg} \\
\glt I thought: "These were my mother's clothes" and kept them.
\end{exe}
 
 
  
 
 \subsubsection{Multiple readings}  
No verb allows all five possible  distinct readings, but as a rule nearly all verbs can have multiple readings. The following examples illustrate the possible interpretation of the \ipa{--si} form with the root |\ipa{went}| `cut, slice';  reciprocal and antipassive interpretations are not possible for this verb.


\begin{exe}
\ex \label{ex:wendu} 
\gll 
\ipa{ʔuŋʌ}  	\ipa{sɵ}  	\ipa{wēnd-u.}  \\
\textsc{1sg:erg} meat cut-\textsc{1sg$\rightarrow$3} \\
\glt I cut the meat.
\end{exe}

\begin{exe}
\ex \label{ex:weisi} 
\gll 
 \ipa{sɵ}  	\ipa{wêi-si.}  \\
 meat cut-\textsc{refl} \\
\glt \textit{Impersonal subject}: `The meat is cut (by someone)' OR `The meat cuts easily.'
\end{exe}


\begin{exe}
\ex \label{ex:weisi2} 
\gll 
 \ipa{ʔʌ̄m} \ipa{sɵ}  	\ipa{wêi-si.}  \\
\textsc{3sg} meat cut-\textsc{refl} \\
\glt \textit{Autobenefactive}: `He cuts meat for himself.'
\end{exe}
 
\begin{exe}
\ex \label{ex:weiwasu} 
\gll 
	\ipa{mu-wei-w-ʌsu.}  \\
\textsc{neg}-cut-\textsc{irr}-\textsc{refl:1sg:pst} \\
\glt \textit{Reflexive}: `I did not cut myself.'
\end{exe}  

 \subsection{Irregularities}
The  \ipa{--si} derivation, while very productive and regular, presents semantic ideosyncrasies involving lexical aspect or general meaning restricted to a few lexical items.
  
\subsubsection{Defective verbs}
Khaling has defective \ipa{--si} verbs, in other words verbs that do not have a corresponding transitive base form. There are three such verbs in our data, two of which are nontranslational motion verbs  (|\ipa{pʰop-si}| `to bend', |\ipa{luk-si}|, `to bend one’s body’), and another a natural reciprocal (|\ipa{braŋ-si}|, `to separate’). We can surmise that the hypothetical base forms used to exist at an earlier stage (and traces may be found in related languages) but have disappeared in modern-day Khaling.

\subsubsection{Intransitive} 
Derivation in \ipa{--si}  on intransitive verbs is strongly restricted to the impersonal subject interpretation. Yet, we find two cases where the \ipa{--si} derivation applies to an intransitive verb replacing the stimulus by the experiencer (Table \ref{tab:stimulus}).

\begin{table}[H]
 \centering 
\caption{Stimulus to experiencer  } \label{tab:stimulus}
\begin{tabular}{lllll}
\toprule
Base verb & Derived verb \\
\midrule
\ipa{ʦɛʔi} `to taste bad, to be unpleasant'.    & \ipa{ʦɛʔi-si} `to be embarrassed'  \\
\ipa{ʦɛʔnʉ} `be nice, be pleasant' &   \ipa{ʦɛʔnʉ-si} `feel nice' \\
\bottomrule
\end{tabular}
\end{table}
 
\subsubsection{Unpredictable semantics}

Some \ipa{--si} detransitive verbs are semantically quite removed from the base verb, to the extent that it is not always easy to determine whether we have two homophonous but unrelated roots or whether an etymological relationship can be hypothesized.

An interesting example is provided by the transitive verb |\ipa{lunt}| whose meaning includes `repeat, put a second layer of clothes, overlap' as in \ref{ex:lunt}.
\begin{exe}
\ex \label{ex:lunt}
\gll 
\ipa{ʣʰūŋ}  	\ipa{lōː-t-ɛ-nʌ}  	\ipa{gɵ}  	\ipa{lʌ̂n-t-ʌ.}  \\
wind feel-\textsc{pst-2/3}-then clothes overlap-\textsc{pst-1sg} \\ 
\glt As it was cold, I put a second layer of clothes.
\end{exe}

Its derived form |\ipa{lunt-si}| has very restricted meanings: `put on a shaman garb' or the potentially metaphorical extension `adopt a family name', as in \ref{ex:luntsi}.

\begin{exe}
\ex \label{ex:luntsi}
\gll 
\ipa{ʔʌ̄mɦɛm}  	\ipa{doɔ̄snʌŋ}  	\ipa{lʌi-t-ɛ-n-nu.}  \\
\textsc{3pl} family.name overlap-\textsc{pst-2/3-refl-pl} \\
\glt They adopted a family name.
\end{exe}

While the meanings of the detransitive verb appears to be indirectly related to that of the transitive one, the exact path of semantic derivation is not recoverable without a deeper description of the local culture.

\section{Conclusion}
Reflexive derivation in \ipa{--si} in Khaling is quite rich and productive. It  presents a few idiosyncratic phonological alternations in comparison with non-reflexive verbs, and its slot in the suffixal chain undergoes metathesis depending on the other suffixes present,  resulting in some cases in verb forms such as \ipa{sʌr-t-ɛ-si} wash-\textsc{pst-2/3-refl} `he washed himself' where the inflectional elements \ipa{-t-} and \ipa{-ɛ} are sandwiched between the stem and the derivational suffix \ipa{--si}.

From the point of view of morphosyntax, the meaning of  \ipa{--si}  is not fully predictable. The basic meaning of this derivation is clearly reflexive, as this is the meaning observed for the largest number of examples. Yet, other detransitive interpretations are also attested for the suffix \ipa{--si}. 

One possible interpretation is that the meaning of this suffix in proto-Kiranti (and earlier, as cognates of it appear in Kham and Dulong/Rawang) was specifically reflexive, and that all other interpretations secondarily  derived from it through various pathways. Indeed, derivation from reflexive to anticausative, passive, or antipassive is widely attested (see \citealt{haspelmath90passive}, \citealt{nedjalkov07reciprocal}, \citealt{say09antipassive}), while the opposite pathway is not documented.


Another possibility is that the function of this suffix in the proto-language was much broader and already encompassed all types of detransitive  derivation, but that only the reflexive interpretation was maintained solidly in Kiranti, while the other subtypes became restricted to a few items. This scenario is attractive as a very similar polysemy is found with \ipa{-ɕɯ}, the cognate of \ipa{--si} in Dulong/Rawang (see in particular \citealt{lapolla05reflexive}). Yet, it is perhaps less probable in view of the fact that other valency decreasing derivations are reconstructible to proto-Kiranti, especially  anticausative prenasalization (see \citealt{jacques13derivational.khaling}). 

\section{Appendix}

\begin{exe}
\ex 
\gll \ipa{ʔʌnʌ̄m}  	\ipa{phêi}  \ipa{dʌbu}  	\ipa{mʉk-bi}  	\ipa{khoɔ̂n-nɛ-po}  	\ipa{ʦʌlʌn}  	\ipa{gōː-ther-t-ɛ.}  	\\
long.ago about hunt do-\textsc{loc} go-\textsc{inf-gen} tradition exist-\textsc{habit-pst-2/3} \\
\glt Long ago, people used to hunt.
\end{exe}
 

\begin{exe}
\ex 
\gll \ipa{mʌnʌ}  	\ipa{tûː-bʌ}  	\ipa{dēl-bî-m}  	\ipa{dʉspɛ-ɦɛm-ʔɛ}  	\ipa{blɛt-t-ɛ-nû-m}  	\ipa{tûː}  	\ipa{kʌtha}  	\ipa{ʦɛnʉ-pɛ}  	\ipa{gɵ.}   \\
then one-\textsc{cl} village-\textsc{loc-nmlz} elder-\textsc{pl-erg} tell-\textsc{pst-2/3-pl-nmlz} one story be.nice-\textsc{nmlz}:S/A exist \\
\glt There is a nice story told by the elders of one village.
\end{exe}

 
\begin{exe}
\ex 
\gll \ipa{mʌnʌ}  	\ipa{tûː-bʌ}  	\ipa{dʉspɛ,}  	\ipa{ɦʌs}  	\ipa{dēl-bi}  	\ipa{mōː-ther-t-ɛ.}   \\
then one-\textsc{cl} elder man village-\textsc{loc} exist-\textsc{habit-pst-2/3} \\
\glt There used to be an elder man in the village.
\end{exe}
 
 
\begin{exe}
\ex 
\gll \ipa{mʌnʌ}  	\ipa{tûː-bʌ}  	\ipa{juba} \ipa{mōː-ther-t-ɛ.}  	\ipa{dēl-bi}  	\ipa{mōː-t-ɛ.}  	\ipa{gɵ.}   \\
then one-\textsc{cl} youth exist-\textsc{habit-pst-2/3}  village-\textsc{loc} exist-\textsc{pst-2/3} exist \\
\glt And there used to be a young man in the village.
\end{exe}
 
 \begin{exe}
\ex 
\gll \ipa{mʌnʌ}  	\ipa{ʔʌ̄m}  	\ipa{ʦʌ̄i}  	\ipa{dʌbu}  	\ipa{mʉk-ther-pɛ.}  \\
then \textsc{3sg} \textsc{top} hunt do-\textsc{habit-nmlz:S/A} \\
\glt He used to hunt.
\end{exe}  
 
 \begin{exe}
\ex 
\gll \ipa{mʌnʌ}  	\ipa{tu-nɵ̂l}  	\ipa{mɛ}  	\ipa{bʉre-ɦʌs-kolo}  	\ipa{mɛ}  	\ipa{sala-kolo}  	\ipa{dʉm-i-t-i-nʌ,}  \\
then one-day that elder-man-\textsc{comit} that young.man-\textsc{comit} meet-\textsc{2/3du-pst-2/3du}-and \\
\glt Then, one day, that elder and that young man met.
\end{exe} 

 
 \begin{exe}
\ex 
\gll \ipa{brâː}  	\ipa{mʉ-s-su,}  	\ipa{sʌ̄ŋ}  	\ipa{lʉ̂-i-t-i,}  \\
word make-\textsc{pst-du} ask \textsc{recip}-\textsc{2/3du-pst-2/3du} \\
\glt They had a conversation, they asked each other questions.
\end{exe} 
 
 \begin{exe}
\ex 
\gll \ipa{mʌnʌ}  	\ipa{mɛ}  	\ipa{dʉspɛ,}  	\ipa{mɛ}  	\ipa{bʉre-ɦʌs-ʔɛ}  	\ipa{ʦʌ̄i,}  
\ipa{mâŋ}  	\ipa{ʦhʉk-t-ɛ}  	\ipa{ʔɛ̂n-nɛ}  	\ipa{ʔu-nûː-bɵjo}  	\ipa{ghole}  	\ipa{brâː}  	\ipa{khɵ̂ŋ-t-ɛ,}    \ipa{mʌnʌ}  	\ipa{mɛ}  	\ipa{sala}  	\ipa{sîŋ-t-ɛ-ʔe.}   \\
then that elder that elder-man-\textsc{erg} \textsc{top} what happen-\textsc{pst}-2/3 say-\textsc{inf} \textsc{3sg.poss}-mind-\textsc{loc:level} a.lot word come.up-\textsc{pst}-2/3  then that young.man ask-\textsc{pst-2/3-hearsay}\\
\glt The old man was wondering what had happened, and he asked the young man.
\end{exe} 

 \begin{exe}
\ex 
\gll \ipa{mʌnʌ}  	\ipa{dʉspɛ}  	\ipa{mɛ}  	\ipa{bʉre-ʔɛ}  	\ipa{sala}  	\ipa{sîŋ-t-ɛ-lo}   \\
then elder that old.man-\textsc{erg} young.man ask-\textsc{pst}-2/3-when \\
\glt The old man asked the young man:
\end{exe} 


 \begin{exe}
\ex 
\gll \ipa{dʌbu-bi}  	\ipa{ʔi-khɵs-t-ɛ-nu}  	\ipa{ɦɛi}?   \ipa{lʉ̂ː-t-ɛ-nʌ}  	\ipa{sîŋ-t-ɛ-ʔe.}   \\
hunt-\textsc{loc} 2-go-\textsc{pst-2/3-pl} \textsc{qu} tell-\textsc{pst-2/3}-and ask-\textsc{pst-2/3-hearsay} \\
\glt He asked him: `Did you_{pl} go hunting?'
\end{exe} 


 \begin{exe}
\ex 
\gll  \ipa{mʌnʌ}  	\ipa{mɛ}  	\ipa{sala-ʔɛ}  	\ipa{lʉ̂ː-t-ɛ-ʔe}  	\ipa{khoɔ̂n-t-ɛ-si}  \\
then that young.man-\textsc{erg} tell-\textsc{pst-2/3-hearsay} go-\textsc{pst-2/3-impers:S/A} \\
\glt Then the young man told him: `(People) went'.
\end{exe} 

 \begin{exe}
\ex 
\gll \ipa{mʌnʌ}  	\ipa{ʔu-brâː}  	\ipa{ʦhoʈo}  	\ipa{lʉ̂ːtɛʔe}  	\ipa{mʌnʌ}  	\ipa{pheri}  	\ipa{mɛ}  	\ipa{dʉspɛ-ʔɛ}  	\ipa{ʦʌ̄i}  	\ipa{mɛ}  	\ipa{sala}  	\ipa{sîŋ-t-ɛ.}   \\
then \textsc{3sg.poss}-word short  tell-\textsc{pst-hearsay} then again that elder-\textsc{erg} \textsc{top} that young.man ask-\textsc{pst}-2/3\\
\glt His answer (words) was short, and the old man asked the young man again:
\end{exe} 



 \begin{exe}
\ex 
\gll \ipa{mʌnʌ}  	\ipa{dʌbu-bi}  	\ipa{ʔi-khɵs-t-ɛ-nu,}  	\ipa{mʌnʌ}  	\ipa{sikar}  	\ipa{ʔi-thōː-t-nu}  	\ipa{ɦɛi?}  \ipa{lʉ̂ː-t-ɛ-nʌ}  	\ipa{sîŋ-t-ɛ-lo,}    \\
then hunt-\textsc{loc} 2-go-\textsc{pst-2/3-pl} then game \textsc{2/inv}-see-\textsc{pst-pl} 
\textsc{qu} tell-\textsc{pst-2/3}-and ask-\textsc{pst-2/3}-when \\
\glt He asked him `So you went hunting, did you see game?'
\end{exe} 

 \begin{exe}
\ex 
\gll \ipa{mʌnʌ}  	\ipa{sala-ʔɛ}  	\ipa{lʉ̂ː-t-ɛ-ʔe}  	\ipa{thɵ̂-n<t-ɛ>si}    \\
then young.man-\textsc{erg} tell-\textsc{pst-2/3-hearsay} see-\textsc{<pst-2/3>impers:S/A} \\
\glt The young man told him: `(It) was seen'.
\end{exe} 


 \begin{exe}
\ex 
\gll \ipa{mʌnʌ}  	\ipa{mɛ}  	\ipa{sikar}  	\ipa{bhîr}  	\ipa{ni}  	\ipa{thɵ-t-nu,} \ipa{mʌnʌ}  	\ipa{mâŋ}  	\ipa{ʦhʉk-t-ɛ}  	\ipa{mɛ}  	\ipa{dʉspɛ-ʔɛ}  	\ipa{ʔodi}  	\ipa{mu-thɵ-wɛ,}   \\
then that game deer \textsc{top} see-\textsc{pst-pl} then what happen-\textsc{pst}-2/3 that elder-\textsc{erg} idea \textsc{neg}-see-\textsc{irr}  \\
\glt The old man (still) did not have a (clear) idea of what had happened; was the game that they saw a deer (or something else)?
\end{exe} 


 \begin{exe}
\ex 
\gll \ipa{mʌnʌ}  	\ipa{pheri}  	\ipa{sîŋ-t-ɛ-ʔe}  	\ipa{mʌnʌ}  	\ipa{mɛ}  	\ipa{bhîr}  	\ipa{ʔi-ʔɵp-t-ɛ-nu}  	\ipa{ɦɛi}  	\ipa{lʉ̂ː-t-ɛ-lo,}  \\
then again ask-\textsc{pst-2/3-hearsay} then that deer \textsc{2/inv}-shoot-\textsc{pst-2/3-pl} \textsc{qu} tell-\textsc{pst}-2/3-when \\
\glt Then he asked again: `Did you shoot that deer?'
\end{exe} 


 \begin{exe}
\ex 
\gll \ipa{ʔoɔ̂m-t-ɛ-si,}  	\ipa{ʔɛs-t-ɛ-ʔe}    \\
shoot-\textsc{pst-2/3-impers:S/A} say-\textsc{pst-2/3-hearsay} \\
\glt `(It) was shot'.
\end{exe} 


 \begin{exe}
\ex 
\gll \ipa{mʌnʌ}  	\ipa{ʔɵ̂ːp-t-ɛ-nu,}  	\ipa{ʔʌbʌ}  	\ipa{mis-tɛ-ʔo}  	\ipa{mu-mis-w-ɛ-ʔo}  	\ipa{thāː}  	\ipa{ŋʌ}  	\ipa{mu-ʦhʉk-w-ɛ.}    \\
then shoot-\textsc{pst-2/3-pl} now die-\textsc{pst-2/3-qu}  \textsc{neg}-die-\textsc{irr-2/3-qu} know \textsc{foc} \textsc{neg}-happen-\textsc{irr}-2/3 \\
\glt So they had shot it, but now, did it die or not? (the old man still) did not know.
\end{exe} 
 
 
 \begin{exe}
\ex 
\gll \ipa{mʌnʌ}  	\ipa{pheri}  	\ipa{sîŋ-t-ɛ-ʔe}  	\ipa{mʌnʌ}  	\ipa{sala-ʔɛ}  	\ipa{blɛt-t-ɛ.}  
\ipa{mʌnʌ}  	\ipa{ʔi-ɦɵs-t-ɛ-nu}  	\ipa{ɦɛi}  	\ipa{lʉ̂ː-t-ɛ}  	\\
then again ask-\textsc{pst-2/3-hearsay} then young.man-\textsc{erg} tell-\textsc{pst}-2/3 then \textsc{2/inv}-bring.back-\textsc{pst-2/3-pl} \textsc{qu} tell-\textsc{pst}-2/3 \\
\glt  Then he asked him again, and the young man told him; `Did  you bring it back?', he said.
\end{exe} 

\begin{exe}
\ex 
\gll \ipa{mʌnʌ}  	\ipa{sala-ʔɛ}  	\ipa{ɦoɔ̂n-t-ɛ-si}  	\ipa{lʉ̂ː-t-ɛ-ʔe.}\\
then young.man-\textsc{erg} bring.back-\textsc{pst-2/3-impers:S/A} tell-\textsc{pst-2/3-hearsay}\\
\glt  Then the young man said:  `It was brought back.'
\end{exe} 


\bibliographystyle{unified}
\bibliography{bibliogj}
\end{document}