\documentclass[11pt]{article} 
\usepackage{fontspec}
\usepackage{natbib}
\usepackage{booktabs}
\usepackage{xltxtra} 
\usepackage{polyglossia} 
\usepackage[table]{xcolor}
%\usepackage{tikz}
%\usetikzlibrary{trees}
\usepackage{gb4e} 
\usepackage{multicol}
\usepackage{graphicx}
\usepackage{float}
\usepackage{hyperref} 
\hypersetup{bookmarks=false,bookmarksnumbered,bookmarksopenlevel=5,bookmarksdepth=5,xetex,colorlinks=true,linkcolor=blue,citecolor=blue}
\usepackage[all]{hypcap}
\usepackage{memhfixc}
 
%\setmainfont[Mapping=tex-text,Numbers=OldStyle,Ligatures=Common]{Charis SIL} 
\newfontfamily\phon[Mapping=tex-text,Ligatures=Common,Scale=MatchLowercase]{Charis SIL} 
\newcommand{\ipa}[1]{{\phon\textit{#1}}} 
\newcommand{\grise}[1]{\cellcolor{lightgray}\textbf{#1}}
\newfontfamily\cn[Mapping=tex-text,Ligatures=Common,Scale=MatchUppercase]{SimSun}%pour le chinois
\newcommand{\zh}[1]{{\cn #1}}
\newcommand{\Y}{\Checkmark} 
\newcommand{\N}{} 
\newcommand{\dhatu}[2]{|\ipa{#1}| `#2'}
\newcommand{\jpg}[2]{\ipa{#1} `#2'}  
\newcommand{\refb}[1]{(\ref{#1})}
\newcommand{\tld}{\textasciitilde{}}
\newfontfamily\mleccha[Mapping=tex-text,Ligatures=Common,Scale=MatchLowercase]{Galatia SIL}%pour le grec
\newcommand{\grec}[1]{{\mleccha #1}}


 \begin{document} 
\title{Étymologies germaniques: *wambô-, XXX}
\author{Guillaume Jacques\\ CNRS-CRLAO-INALCO}
\maketitle

\section{*wambō-}

Le nom féminin proto-germanique *\ipa{wambō-}  `ventre, entrailles' reconstructible sur la base du gothique \ipa{wamba} f. `\grec{κοιλία}', du vieux norrois \ipa{vǫmb} f. et du vieil anglais \ipa{wamb}, n'a pas d'étymologie acceptée (\citealt[W28,393]{lehmann86gothic}).

\citet[572]{kroonen13dict} considère comme possible le rapprochement avec le védique \ipa{gabhá-} m. `vagin', un mot qui apparaît dans un passage très cité du rituel de l'Aśvamedha (\ref{ex:gabhe}), proposant une proto-forme \ipa{*gʷʰṃbʰ-ó-} au degré zéro (avec loi de Grassmann) impliquant donc pour le germanique un degré *\ipa{o} \ipa{*gʷʰómbʰ-eh_2-}.


\begin{exe}
\ex \label{ex:gabhe}
\glt \ipa{iyáṃ yakā́ śakuntikā́hálam íti sárpati}
\glt \ipa{ā́hataṃ gabhé páso ní jalgulīti dhā́ṇikā}
\glt This little bird which creeps around saying `âhalam'
knocks the penis into the slit; the vulva devours it. (TS 7.4.19.3m, \citealt[275]{watkins95})
\end{exe}


Toutefois, cette comparaison souffre de plusieurs problèmes.

Premièrement, comme le note Kroonen lui-même, il est plus plausible de rapprocher \ipa{gabhá-} de la famille de \ipa{gámbha-} n. `profondeur', \ipa{gabhīrá-} `profond' suivant \citet[324]{mayrhofer56kurz}, une comparaison incompatible avec le germanique du simple point de vue sémantique.

%(KSAśv 5.4.8:165.7 et VS 23.22cd)

% (que le \citealt[206]{liv} rapproche de la racine *\ipa{gʷeh_2bʰ}).

Deuxièmement, le nom *\ipa{wambō-} ne désigne pas nécessairement le ventre en tant qu'organe reproducteur (comme c'est le cas avec son réflexe en anglais moderne \ipa{womb}); en gothique il peut faire référence au ventre en tant que partie du système digestif, comme l'illustre (\ref{ex:guth}), et son usage dans le nom d'un roi Visigoth d'Espagne, Wamba (\textit{regnauit} 672−680, voir \citealt{bronisch06wamba}), qui doit être un sobriquet signifiant `le ventru'.

\begin{exe}
\ex \label{ex:guth}
\glt \ipa{þize guþ wamba ist jah wulþus in skandai ize}
\glt \grec{ὧν ὁ θεὸς ἡ κοιλία καὶ ἡ δόξα ἐν τῇ αἰσχύνῃ αὐτῶν}
\glt `Dont le dieu est leur ventre, et dont la gloire est dans la honte.' (Philippiens 3:19)
\end{exe}

Il n'est donc pas illégitime de rechercher une solution étymologique alternative.

Je propose de partir de la racine \ipa{*webʰ} `tisser' (\citealt[653]{liv}), bien attestée en germanique (\citealt[540-2]{seebold70starken}) par un verbe fort \ipa{*weban-} (vieux norrois \ipa{vefa} `tisser', vieil anglais \ipa{wefan} `id') et une multitude de noms dérivés.

Parmi ceux-ci se trouve le vieux norrois \ipa{vaf} n. $\leftarrow$ \ipa{*waba-} $\leftarrow$ \ipa{*wobʰo-}, dont le sens précis selon \citet[653]{cleasby1874icelandic} est `a wrapping, winding round' et en islandais `the piece of skin wound round a quill for infants to suck is called \ipa{vaf}'. On peut partir d'un étymon ayant un sens proche de celui de \ipa{*waba-}, tel que `pièce de tissu enroulée' ou `bobine', d'où `protubérance' et par désignation familière, le terme de partie du corps `ventre'.

Pour dériver la forme \ipa{*wambō} de la racine \ipa{*webʰ}, supposer un dérivé à infixe \ipa{*-n-} du type \ipa{*wemba-} n'est pas absolument impossible étant donné l'existence de formes de présent de la classe IX en sanskrit et en grec pour cette racine (voir \citealt[653]{liv}), mais l'absence totale d'autre attestation de nasale dans les nombreux dérivés de cette racine en germanique milite plutôt contre une telle hypothèse.

Pour expliquer la présence de la nasale dans \ipa{*wambō}, il faut prendre en compte l'existence d'une autre désignation du `ventre' en germanique, \ipa{*amban-} m. attesté par le vieux saxon \ipa{ambon} `ventre' et le vieil haut allemand \ipa{amban} `id', que \citet[24]{kroonen13dict} reconstruit \ipa{*h_3émbʰ-on-}, et analyse comme un dérivé de la racine du nom du `nombril' \ipa{*h_3nóbʰ-s}.

Je propose une contamination entre un mot \ipa{*wabō} `pièce de tissu enroulée, protubérance, ventre' et \ipa{*amban-} donnant la forme \ipa{*wambō} `ventre'; la nasale serait donc non-étymologique.

\bibliographystyle{unified}
\bibliography{bibliogj}

 \end{document}
 