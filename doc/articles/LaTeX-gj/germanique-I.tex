\documentclass[11pt]{article} 
\usepackage{fontspec}
\usepackage{natbib}
\usepackage{booktabs}
\usepackage{xltxtra} 
\usepackage{polyglossia} 
\usepackage{gb4e} 
\usepackage{multicol}
\usepackage{graphicx}
\usepackage{float}
\usepackage{hyperref} 
\hypersetup{bookmarks=false,bookmarksnumbered,bookmarksopenlevel=5,bookmarksdepth=5,xetex,colorlinks=true,linkcolor=blue,citecolor=blue}
\usepackage[all]{hypcap}
\usepackage{memhfixc}
 
%\setmainfont[Mapping=tex-text,Numbers=OldStyle,Ligatures=Common]{Charis SIL} 
\newfontfamily\phon[Mapping=tex-text,Ligatures=Common,Scale=MatchLowercase]{Charis SIL} 
\newcommand{\ipa}[1]{{\phon\textit{#1}}} 
\newcommand{\grise}[1]{\cellcolor{lightgray}\textbf{#1}}
\newfontfamily\cn[Mapping=tex-text,Ligatures=Common,Scale=MatchUppercase]{SimSun}%pour le chinois
\newcommand{\zh}[1]{{\cn #1}}
\newcommand{\Y}{\Checkmark} 
\newcommand{\N}{} 
\newcommand{\dhatu}[2]{|\ipa{#1}| `#2'}
\newcommand{\jpg}[2]{\ipa{#1} `#2'}  
\newcommand{\refb}[1]{(\ref{#1})}
\newcommand{\tld}{\textasciitilde{}}
\newfontfamily\mleccha[Mapping=tex-text,Ligatures=Common,Scale=MatchLowercase]{Galatia SIL}%pour le grec
\newcommand{\grec}[1]{{\mleccha #1}}


 \begin{document} 
\title{Contribution à l'étude des réflexes de l'indoeuropéen *\ipa{gʷʰ} à l'initiale en germanique}
\author{Guillaume Jacques\\ CNRS-CRLAO-INALCO}
\maketitle

\sloppy

\section{Le traitement de *\ipa{gʷʰ} initial en germanique}
Selon \citet{seebold67ghw}, le traitement régulier de l'indo-européen \ipa{*gʷʰ} à l'initiale de mot en germanique n'est ni \ipa{*w-} ni \ipa{*g-} (\citealt[122-3]{streitberg1900urgermanische}), mais \ipa{*b-}, sauf devant \ipa{*-u-} où il devient bien \ipa{g-} (y compris les \ipa{*u} provenant de sonantes syllabiques).\footnote{Les réflexes de \ipa{*gʷʰ} à l'intérieur des mots sont en revanche plus compliqués, et comprennent \ipa{*w} et \ipa{*g}; cette question n'est pas traitée dans le présent article. }

Seebold présente cinq exemples de \ipa{*gʷʰ-} à \ipa{*b-}, qui en particulier comprennent toutes les racines à \ipa{*gʷʰ-} initial recensées dans \citet{liv},\footnote{Voir aussi \citet[note 46]{garnier16dybo}.} à l'exception de \ipa{*gʷʰel} `verzehren, grasen', que l'on ne trouve qu'en celtique.

Le seul exemple de \ipa{*gʷʰ-} à \ipa{g-} devant \ipa{*u} qu'il juge certain est \ipa{*gundiō-} f. `Krieg, Kampf' de \ipa{*gʷʰṇ-tjéh_2}, cf Skt. \ipa{°hatyā́-}\footnote{Notons que la forme sanskrite est refaite, sans quoi on attendrait $\dagger$\ipa{ghatyā-}.} (\citealt[105]{seebold67ghw}).

La loi de Seebold, si elle est mentionnée par tous les travaux sur le germanique (et systématiquement référencée dans le LIV), est très controversée. En particulier, \citet[xxviii]{kroonen13dict} juge que `since all instances [of proto-IE initial \ipa{*gʷʰ-}] with \ipa{*b} except \ipa{*bedjan-} have alternative etymologies, whereas the ones with \ipa{*w} have not, it seems best to suspend the implementation of this change until further notice.'

Une telle évaluation de l'article Seebold est peu justifiée, car celui-ci a proposé des étymologies alternatives pour dix mots à initiale \ipa{*w} en germanique comparés à des formes dont la consonne initiale remonte à \ipa{*gʷʰ} dans d'autres langues, y compris pour \ipa{*warma-} `chaud', le \textit{Paradebeispiel} de la correspondance \ipa{*gʷʰ-} $\rightarrow$ \ipa{*w-}, qu'il explique par la racine \ipa{*werH} `heiss sein' (\citealt[689]{liv}) et rejette l'étymologie traditionnelle par \ipa{*gʷʰor-mo-} (Skt \ipa{gharma-}`chaleur', la seule étymologie mentionnée par \citealt[575]{kroonen13dict}).

Dans ce travail, je propose une étymologie alternative pour un exemple proposé de \ipa{*gʷʰ-} à \ipa{w-} non traité par Seebold, le proto-germanique *\ipa{wambō-}  f. `ventre', et un nouvel example du traitement \ipa{*gʷʰ} $\rightarrow$ \ipa{*g-} devant \ipa{*u}, \ipa{*gunda-} `gangrène'. 

\section{*wambō-}

Le nom proto-germanique *\ipa{wambō-} f. `ventre, entrailles' reconstructible sur la base du gothique \ipa{wamba} f. `\grec{κοιλία}', du vieux norrois \ipa{vǫmb} f. et du vieil anglais \ipa{wamb}, n'a pas d'étymologie acceptée (\citealt[W28,393]{lehmann86gothic}).

\citet[572]{kroonen13dict} considère comme possible le rapprochement avec le védique \ipa{gabhá-} m. `vagin', un mot qui apparaît dans un passage très cité du rituel de l'Aśvamedha (\ref{ex:gabhe}), proposant une proto-forme \ipa{*gʷʰṃbʰ-ó-} au degré zéro (avec loi de Grassmann) impliquant donc pour le germanique un degré *\ipa{o} \ipa{*gʷʰómbʰ-eh_2-}.


\begin{exe}
\ex \label{ex:gabhe}
\glt \ipa{iyáṃ yakā́ śakuntikā́hálam íti sárpati}
\glt \ipa{ā́hataṃ gabhé páso ní jalgulīti dhā́ṇikā}
\glt `This little bird which creeps around saying `âhalam'
knocks the penis into the slit; the vulva devours it.' (TS 7.4.19.3m, \citealt[275]{watkins95})
\end{exe}


Toutefois, cette comparaison souffre de plusieurs problèmes.

Premièrement, comme le note Kroonen lui-même, il est plus plausible de rapprocher \ipa{gabhá-} de la famille de \ipa{gámbha-} n. `profondeur', \ipa{gabhīrá-} `profond' suivant \citet[324]{mayrhofer56kurz}, une comparaison incompatible avec le germanique du simple point de vue sémantique.

%(KSAśv 5.4.8:165.7 et VS 23.22cd)

% (que le \citealt[206]{liv} rapproche de la racine *\ipa{gʷeh_2bʰ}).

Deuxièmement, le nom *\ipa{wambō-} ne désigne pas nécessairement le ventre en tant qu'organe reproducteur (comme c'est le cas avec son réflexe en anglais moderne \ipa{womb}); en gothique il peut faire référence au ventre en tant que partie du système digestif, comme l'illustre (\ref{ex:guth}), et son usage dans le nom d'un roi Visigoth d'Espagne, Wamba (\textit{regnauit} 672−680, voir \citealt{bronisch06wamba}), qui doit être un sobriquet signifiant `le ventru'.

\begin{exe}
\ex \label{ex:guth}
\glt \ipa{þize guþ wamba ist jah wulþus in skandai ize}
\glt \grec{ὧν ὁ θεὸς ἡ κοιλία καὶ ἡ δόξα ἐν τῇ αἰσχύνῃ αὐτῶν}
\glt `Dont le dieu est leur ventre, et dont la gloire est dans la honte.' (Philippiens 3:19)
\end{exe}

Il n'est donc pas illégitime de rechercher une solution étymologique alternative.

Je propose de partir de la racine \ipa{*webʰ} `tisser' (\citealt[653]{liv}), bien attestée en germanique (\citealt[540-2]{seebold70starken}) par un verbe fort \ipa{*weban-} (vieux norrois \ipa{vefa} `tisser', vieil anglais \ipa{wefan} `id') et une multitude de noms dérivés.

Parmi ceux-ci se trouve le vieux norrois \ipa{vaf} n. $\leftarrow$ \ipa{*waba-} $\leftarrow$ \ipa{*wobʰo-}, dont le sens précis selon \citet[653]{cleasby1874icelandic} est `a wrapping, winding round' et en islandais `the piece of skin wound round a quill for infants to suck is called \ipa{vaf}'. On peut partir d'un étymon ayant un sens proche de celui de \ipa{*waba-}, tel que `pièce de tissu enroulée' ou `bobine', d'où `protubérance' et par désignation familière, le terme de partie du corps `ventre'.

Pour dériver la forme \ipa{*wambō} de la racine \ipa{*webʰ}, supposer un dérivé à infixe \ipa{*-n-} du type \ipa{*wemba-} n'est pas absolument impossible étant donné l'existence de formes de présent de la classe IX en sanskrit et en grec pour cette racine (voir \citealt[653]{liv}), mais l'absence totale d'autre attestation de nasale dans les nombreux dérivés de cette racine en germanique milite plutôt contre une telle hypothèse.

Pour expliquer la présence de la nasale dans \ipa{*wambō}, il faut prendre en compte l'existence d'une autre désignation du `ventre' en germanique, \ipa{*amban-} m. attesté par le vieux saxon \ipa{ambon} `ventre' et le vieil haut allemand \ipa{amban} `id', que \citet[24]{kroonen13dict} reconstruit \ipa{*h_3émbʰ-on-}, et analyse comme un dérivé de la racine du nom du `nombril' \ipa{*h_3nóbʰ-s}.

Je propose une contamination entre un mot \ipa{*wabō} `pièce de tissu enroulée, protubérance, ventre' et \ipa{*amban-} donnant la forme \ipa{*wambō} `ventre'; la nasale serait donc non-étymologique.

\section{*gunda-}

Le gothique \ipa{gund} n. `gangrène' (hapax attesté dans \ref{ex:gund}), apparenté au norwégien dialectal \ipa{gund} `skin crust', au vieil anglais \ipa{gund} m. `pus' et au vieil haut allemand \ipa{gund} de même sens, remonte à un proto-germanique *\ipa{gunda-} qui n'a pas jusqu'ici reçu d'étymologie satisfaisante (\citealt[195]{kroonen13dict}, \citealt[163, G116]{lehmann86gothic}).

\begin{exe}
\ex \label{ex:gund}
\glt \ipa{jah waurd ize swe gund wuliþ}
\glt \grec{καὶ ὁ λόγος αὐτῶν ὡς γάγγραινα νομὴν ἕξει}
\glt `Et leur parole se répandra comme la gangrène.' (Timothée II 2:17)
\end{exe}

La forme germanique est toutefois candidate à une comparaison avec des formes indo-iraniennes qui sont elles-mêmes restées sans explication: le sanskrit \ipa{gandha-} m. `odeur', l'avestique \ipa{gaiṇti-} `puanteur' et le vieux perse \ipa{gasta-} `répugnant' (données des autres cognats iraniens dans \citealt[103-104]{cheung07dictionary}). 

Mis à part le vieux perse où l'adjectif \ipa{gasta-} a développé un sens abstrait (voir \ref{ex:gasta}), les formes des autres langues indo-iraniennes désignent une perception olfactive. 

\begin{exe}
\ex \label{ex:gasta}
\glt \ipa{martiyā} \ipa{hyā} \ipa{Auramazdāh} \ipa{ā} \ipa{framānā} \ipa{hauvtaiy} \ipa{gastā} \ipa{mā} \ipa{θadaya}
\glt `O man, that which is the command of Ahuramazda, let this not seem repugnant to thee.' (DNa.56-58 \citealt[137]{kent53op})
\end{exe}

Une telle comparaison soulève trois questions qu'il est nécessaire de discuter: 

\begin{enumerate}
\item L'incompatibilité phonétique entre le sanskrit \ipa{gandha-} (qui suppose une proto-forme \ipa{*g(ʷʰ)ondʰo-} qui ne peut être directement comparée aux formes de l'iranien qui impliquent une racine de la forme \ipa{*g(ʷʰ)ent-}, ainsi que la différence de sens en sanskrit et iranien.
\item Le problème de la structure d'une  racine ayant la forme \ipa{*g(ʷ)ʰent-} (la co-occurence d'une voisée aspirée et d'une sourde)
\item L'évolution sémantique et phonétique précise ayant donné la forme germanique \ipa{*gunda-}.
\end{enumerate}

Pour rendre compte des formes iraniennes, peut poser soit une racine verbale de forme \ipa{*g(ʰʷ)ent-} `avoir une mauvaise odeur', soit un nom-racine  \ipa{*g(ʷ)ʰont- \textasciitilde *g(ʷ)ʰnt-} `puanteur'. Les formes \ipa{gaiṇti-} et  \ipa{gasta-} seraient dérivées par suffixe \ipa{*-i-} sur degré \ipa{*o} (type de sankrit \ipa{bodhi-} `éveil' et avestique \ipa{baoiδi-} `odeur' de \ipa{*bʰowdʰ-i-}) et par un suffixe \ipa{*-tó-} sur degré zéro (un participe si la racine est verbale).

Dans la synchronie du sanskrit, le nom-racine venant de \ipa{*g(ʷ)ʰont- \textasciitilde *g(ʷ)ʰnt-} `puanteur' aurait présenté dans son paradigme certaines formes telles que le locatif pluriel \ipa{*ghatsú} $\leftarrow$ \ipa{*g(ʷ)ʰṇt-sú} `dans la puanteur' (type \ipa{pṛt-sú}) où l'opposition entre les quatre modes d'articulation des occlusives est neutralisé, et qui se prêtaient donc à une ré-interprétation comme provenant d'une racine plus normale du point de vue de la phonotactique du sanskrit \ipa{*gandh} par hyper-Grassmannisation (par analogie avec des nom-racines tels que \ipa{°búdh-} en tant que partie finale de composé nominal, qui donne au nominatif \ipa{°bhut} et au locatif pluriel \ipa{°bhutsú}).\footnote{Comme la racine \ipa{*g(ʷ)ʰent} n'est attestée que par des formations nominales (ou verbes dénominaux), il serait controuvé de rechercher la forme pivot dans la conjugaison (un futur $\dagger$\ipa{ghantsya-ti} ou un désidératif $\dagger$\ipa{jighantsa-ti}). }

D'une telle forme-pivot, on a ensuite refait un ancien $\dagger$\ipa{ghánti-} `puanteur' en \ipa{*gándhi-}, forme qui est attestée en second membre de composé (\ipa{su-gándhi-} `ayant une bonne odeur').

Notons qu'en iranien, du fait de la confusion des voisées aspirées et des voisées simples, une telle réinterprétation n'aurait pas pu avoir lieu.

L'avestique \ipa{duž-gaiṇti-} `puanteur' suggère de lui-même une explication à la différence sémantique entre sanskrit et iranien: le préfixe \ipa{duž-}, originellement utilisé pour intensifier le sens (`forte puanteur'), a dans le cadre du sanskrit été réinterprété comme exprimant un sens péjoratif (\ipa{durgandhi-} `ayant une mauvaise odeur'), et donc pourvu d'une doublet en \ipa{su-} (\ipa{sugandhi-} `ayant une bonne odeur').\footnote{A titre d'illustration de l'extrême productivité de cette formation, notons que même le nom de personne Duryodhana a  une dénomination alternative euphémistique \textit{Suyodhana} dans le Mahâbhârata.} Dans la synchronie du sanskrit, la pseudo-racine \ipa{gandh-} a donc été associée au sens neutre d'odeur, et un dérivé thématique \ipa{gándha-} `odeur' en a été tiré.\footnote{Notons malgré l'absence de relation étymologique entre ce nom \ipa{gandha-} `odeur' et celui des créatures mythologiques \ipa{gandharva-}, l'équivalent tibétain de ce nom \ipa{dri.za} `mangeur d'odeur' suggère que les traducteurs de ce terme ont peut-être imaginé que \ipa{gandharva-} était une forme tronquée tirée d'un \ipa{*gandha-bharva-}.}

Une racine verbale \ipa{*g(ʷ)ʰent-} contreviendrait à la règle de co-occurrence des voisées aspirées et des sourdes (voir par exemple \citealt{kuemmel12pie}) et ne peut pas être indo-européen, et il est donc plus vraisemblable que le nom-racine \ipa{*g(ʷ)ʰṇt-} `puanteur' posé ci-dessus pour rendre compte des formes indo-iraniennes soit secondaire.

Une étymologie possible pour ce nom serait un dérivé  archaïque en \ipa{*-t-} de la racine \ipa{*gʷʰen} `schlagen' (\citealt[218]{liv}), un \ipa{*gʷʰont-\textasciitilde *gʷʰṇt-} `blessure' (type \ipa{*per} $\rightarrow$ \ipa{pṛt-} `combat') d'où `blessure purulente', puis `puanteur' en proto-indo-iranien.

En germanique, ce nom-racine aurait été thématique en \ipa{*gʷʰṇt-ó-}, donnant \ipa{*gunda-} par voisement Verner et loi de Seebold. Du point de vue sémantique le germanique serait donc plus archaïque que l'indo-iranien.

\section{Conclusion}
Ce travail contribue à confirmer la loi de Seebold concernant le traitement phonétique de \ipa{*gʷʰ} à l'initiale de mot, en proposant une étymologie alternative au seul exemple de \ipa{*gʷʰ} à \ipa{*w} initial non discuté par Seebold, et en apportant un nouvel exemple du traitement \ipa{*gʷʰ} à \ipa{*g} devant \ipa{*u}. 


\bibliographystyle{unified}
\bibliography{bibliogj}

 \end{document}
 