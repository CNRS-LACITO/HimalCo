\documentclass[oldfontcommands,oneside,a4paper,11pt]{article} 
\usepackage{fontspec}
\usepackage{natbib}
\usepackage{booktabs}
\usepackage{xltxtra} 
\usepackage{longtable}
\usepackage{polyglossia} 
\usepackage[table]{xcolor}
\usepackage{lineno}
\usepackage{gb4e} 
\usepackage{multicol}
\usepackage{graphicx}
\usepackage{float}
\usepackage{hyperref} 
\hypersetup{bookmarks=false,bookmarksnumbered,bookmarksopenlevel=5,bookmarksdepth=5,xetex,colorlinks=true,linkcolor=blue,citecolor=blue}
\usepackage[all]{hypcap}
\usepackage{memhfixc}
\usepackage{lscape}
\bibpunct[: ]{(}{)}{,}{a}{}{,}
%%%%%%%%%quelques options de style%%%%%%%%
%\setsecheadstyle{\SingleSpacing\LARGE\scshape\raggedright\MakeLowercase}
%\setsubsecheadstyle{\SingleSpacing\Large\itshape\raggedright}
%\setsubsubsecheadstyle{\SingleSpacing\itshape\raggedright}
%\chapterstyle{veelo}
%\setsecnumdepth{subsubsection}
%%%%%%%%%%%%%%%%%%%%%%%%%%%%%%%
\setmainfont[Mapping=tex-text,Numbers=OldStyle,Ligatures=Common]{Charis SIL} 
\newfontfamily\phon[Mapping=tex-text,Ligatures=Common,Scale=MatchLowercase,FakeSlant=0.3]{Charis SIL} 
\newcommand{\ipa}[1]{{\phon \mbox{#1}}} %API tjs en italique
 
 
 
\newcommand{\grise}[1]{\cellcolor{lightgray}\textbf{#1}}
\newfontfamily\cn[Mapping=tex-text,Ligatures=Common,Scale=MatchUppercase]{MingLiU}%pour le chinois
\newcommand{\zh}[1]{{\cn #1}}

\newcommand{\jg}[1]{\ipa{#1}\index{Japhug #1}}
\newcommand{\wav}[1]{#1.wav}
\newcommand{\tgz}[1]{\mo{#1} \tg{#1}}

\XeTeXlinebreaklocale 'zh' %使用中文换行
\XeTeXlinebreakskip = 0pt plus 1pt %
 %CIRCG
\begin{document} 
\linenumbers

\title{Non-propositional evidentiality}
\author{Guillaume Jacques}
\maketitle

\citet{launey94}
Not all evidential markers within noun phrases excludes evidential markers in relatives 




As a modifier \ipa{tikî-m} can be used to modify   a noun with a prenominal     relative clause  in between (example \ref{ex:salpu}).

\begin{exe}
\ex \label{ex:salpu}
\gll    	 	\ipa{tikî-m}   	\ipa{kɵ̂m-go-jo}   	\ipa{ʣe-pɛ}   	\ipa{sʌ̄lpu-ʔɛ}   	\ipa{ʔʌnɵ̂l-ni}   	\ipa{mâŋ-go}   	\ipa{blɛtt-ʉ}   	\ipa{ɦolʌ}   
 \\
 there:\textsc{aud}-\textsc{nmlz} cloud-\textsc{inside-locative.level} speak-\textsc{nmlz:}S/A bird-\textsc{erg} today-\textsc{top} what-\textsc{foc} tell-\textsc{3sg$\rightarrow$3} maybe \\
\glt The bird that is singing in the clouds, what might it be telling today? (excerpt from a song by the Khaling songwriter Urmila)
\end{exe}
 

 
Examples \ref{ex:radio} and \ref{ex:lel} above illustrate uses of the auditory demonstratives with visible referents; likewise,   \ref{ex:kogu}, uttered by a person watching a song contest on the television, makes it clear that the visibility or non-visibility of the referent is not a relevant factor in using this demonstrative.

\begin{exe}
\ex \label{ex:kogu}
\gll  	\ipa{tikî-m-kʌ}   	\ipa{ʦʌ̄i} \ipa{ʔuŋʌ} \ipa{tūŋ }   	\ipa{kog-u}   \\
there:\textsc{aud}-\textsc{nmlz}-from \textsc{top} \textsc{1sg:erg} more be.able-\textsc{1sg$\rightarrow$3sg}	  \\
\glt I can (sing) better than that one. (Heard in context)
\end{exe}

 

In all of the examples above, non-auditory demonstratives could also have been used. The choice of \ipa{tikî-m}  highlights the  fact that the  speakers' perception is primarily via the auditory channel.
 
 

 

\begin{table}
\caption{Systems     including auditory demonstratives } \label{tab:attested}
\resizebox{\columnwidth}{!}{
\begin{tabular}{lcccl}
\toprule
  &	Connection  & 	Indexation of   & 	Indexation of distinctions  	&References\\
&  with proximal / distal &visual perception &between audition and other senses\\
\midrule  
Santali  & 	no  & 	yes  & 	no  & 	\citet[42-44]{neukom01santali}\\
Nyelayu  & 	yes  & 	yes  & 	no  & 	\citet[98]{ozanne97spatial}\\
Southern Pomo  & 	unknown  & 	unclear  & 	no  & 	\citet[37, ft]{oswalt86evidential}\\
Muna  & 	yes  & 	yes  & 	no  & 	\citet{berg97deixis.muna}\\
Dyirbal  & 	yes   & 	yes  & 	no  & 	\citet{dixon72dyirbal}\\
\midrule
Khaling  & 	no  & 	no  & 	yes  & 	\\
\bottomrule
\end{tabular}}
\end{table}	


\citet{aikhenvald14knowledge}
\citet{dixon14nonvisible}
\citet{storch14maaka}
\citet{gutierrez11evidentiality}
\citet{yang00hearsay}
\citet{yang00tsou.case}
\citet{matthewson07epistemic}
\citet{gutierrez12determiners}
 


\bibliographystyle{unified}
\bibliography{bibliogj}
\end{document}
