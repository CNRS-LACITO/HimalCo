\documentclass[oneside,a4paper,11pt]{article} 
\usepackage{fontspec}
\usepackage{natbib}
\usepackage{booktabs}
\usepackage{xltxtra} 
\usepackage{polyglossia} 
\usepackage[table]{xcolor}
\usepackage{lineno}
\usepackage{gb4e} 
\usepackage{multicol}
\usepackage{graphicx}
\usepackage{float}
\usepackage{hyperref} 
\hypersetup{bookmarks=false,bookmarksnumbered,bookmarksopenlevel=5,bookmarksdepth=5,xetex,colorlinks=true,linkcolor=blue,citecolor=blue}
\usepackage[all]{hypcap}
\usepackage{memhfixc}
\usepackage{lscape}
\bibpunct[: ]{(}{)}{,}{a}{}{,}
 
\setmainfont[Mapping=tex-text,Numbers=OldStyle,Ligatures=Common]{Charis SIL} 
\newfontfamily\phon[Mapping=tex-text,Ligatures=Common,Scale=MatchLowercase,FakeSlant=0.3]{Charis SIL} 
\newcommand{\ipa}[1]{{\phon \mbox{#1}}} %API tjs en italique
 
 
 
\newcommand{\grise}[1]{\cellcolor{lightgray}\textbf{#1}}
\newfontfamily\cn[Mapping=tex-text,Ligatures=Common,Scale=MatchUppercase]{MingLiU}%pour le chinois
\newcommand{\zh}[1]{{\cn #1}}

\newcommand{\jg}[1]{\ipa{#1}\index{Japhug #1}}
\newcommand{\wav}[1]{#1.wav}
\newcommand{\tgz}[1]{\mo{#1} \tg{#1}}

\XeTeXlinebreaklocale 'zh' %使用中文换行
\XeTeXlinebreakskip = 0pt plus 1pt %
 %CIRCG
\begin{document} 
%\linenumbers
\sloppy

\title{Non-propositional evidentiality}
\author{Guillaume Jacques}
\maketitle

\section{Introduction}
In most languages of the worlds where evidentiality is grammaticalized, it is expressed either by verbal morphology, sentential markers or adverbs, which have scope over the entire proposition. A minority of languages have evidential-like distinctions on markers (mainly deictic, see \citealt[130]{aikhenvald06}) whose scope is limited to a noun phrase. The present study focuses on these non-propositional evidential markers; evidential markers present in relative clauses embedded within noun phrases are not considered in this study: although some languages have restriction on the use of evidential markers in relatives,\footnote{For instance in Japhug Gyalrong, the inferential cannot be used in relative clauses with finite verb.} many languages allow evidential markers on the verb in relative clauses (\citealt[253-6]{aikhenvald06}). This exclusion concerns in particular predicative evidential markers in omnipredicative languages.\footnote{I omnipredicative languages such as Nahuatl or most Salish languages, nouns, although distinct from verbs, are predicative and must be relativized to be used in argument position (\citealt{launey94}).}

 All languages with non-propositional evidential markers in the sample also have propositional evidentials, though often exhibiting completely distinct meanings. XXXX


\section{Locus of non-propositional evidential marking} \label{sec:loc}
Non-propositional evidential systems as defined above include evidential distinctions on nominal determiners (either affixes, clitics or independent words), case or discourse markers, and also on demonstrative pronouns and adverbs.

In some languages, noun modifiers can be derived from demonstrative adverbs by means of a nominalization morpheme. In such systems, demonstrative determiners and demonstrative adverbs are completely parallel and show the same evidential contrasts. For instance, in Khaling, the demonstrative determiner/pronoun \ipa{tikî-m} `this (audible)' is derived from the demonstrative adverb \ipa{tikí} `there (audible)' by means of the all-purpose nominalizer \ipa{--m}.

However, some languages may present distinct non-propositional evidential sub-systems. For instance, Lillooet/St'at'imcets has a visible / non-visible distinction in demonstrative pronouns and adverbs (\citealt[169, 171]{eijk97lillooet}), but determiners rather mark the reliability of evidence for assuming that the referent exists or not (\citealt{matthewson98determiners, gutierrez12determiners}, see section \ref{sec:UT}).

Although non-propositional evidential markers are in some languages independent of any other morphosyntactic parameter (as in Khaling for instance, see section \ref{sec:auditory}), they can also be combined with  proximal / distal distinctions, case marking and topicalization.

Evidential markers with proximal / distal distinctions are particularly common in the case of sensory evidentials, especially those marking visiblity, as exemplified by Tsou in Table \ref{tab:tsou} (\citealt{tung64tsou, yang00tsou.case}; see also sections \ref{sec:proximal}, \ref{sec:auditory} and \ref{tab:attested} below). 

Evidential meanings can also be combined with case within a single marker. The data Tsou in Table (\ref{tab:tsou}) incidentally also illustrate this interaction between evidentiality and case. Note that the case markers in Tsou are portmanteau morphemes: it is not possible to decompose them into two morphemes (evidential marker and case marker), at least synchronically.  
 
\begin{table}[H]
 \caption{Tsou case markers, adapted from \citet[54]{yang00tsou.case}} \centering \label{tab:tsou}
\begin{tabular}{llllllll}
\toprule
	 & 	 & 	\multicolumn{2}{c}{case markers } 	 \\	
	 & 	 & 	nominative & 	oblique \\	
\midrule
	 & 	proximal & 	\ipa{'e} & 	\ipa{ta} \\ 	
visual	 & 	medial & 	\ipa{si} & 	\ipa{ta}  \\ 	
	 & 	distal & 	\ipa{ta} & 	\ipa{ta}  \\ 	
\multicolumn{2}{l}{non-visual sensory}  	 & 	\ipa{co} & 	\ipa{nca/ninca} \\ 	
	 \midrule
\multicolumn{2}{l}{hearsay}	 & 	\ipa{'o} & 	\ipa{to} \\ 	
\multicolumn{2}{l}{belief / inference} 	 & 	\ipa{na} & 	\ipa{no} \\ 	
\bottomrule
\end{tabular}
\end{table} 

In other languages where case and evidential markers interact, such as Dyirbal, the morphology is more concatenative. Note however that even in Dyirbal the case paradigms of the evidential demonstratives are not completely predictable (see \citealt{dixon14nonvisible}).

Aside from proximal / distal distinction and case, a third parameter has been shown to interact with non-propositional evidentiality: topicality. For instance,  the Chadic language Maaka has three evidential markers \ipa{--mú} `eye-witnessed', \ipa{--dìyà} `joint-perception', \ipa{--kà} `assumption' occurring on noun phrases. They can be used with referents which are `hardly core participants, but rather topicalized peripheral participants that motivate an action or event' (\citealt[195-7]{storch14maaka}), as exemplified by (\ref{ex:maaka}).

\begin{exe}
\ex \label{ex:maaka}
\gll \ipa{yáayà} \ipa{círòmà-mú} \ipa{nín-nì} \ipa{gùu=ɓálɓìyá} \ipa{tà-lòwó} \ipa{gàamôɗí} \ipa{bòɲcéttí} \\
\textsc{name} \textsc{title-vis} mother-\textsc{3sg:poss:masc} person=\textsc{toponym} \textsc{3sg:fem}-deliver:\textsc{pfv} once \textsc{dem:ref} \\
\glt Yaaya Ciroma [eye-witnessed]: her mother is from Balbiya town, she once gave birth there.
\end{exe}

Non-propositional evidential system display a considerable diversity in terms of morphology, and it would not be surprising if future fieldwork brings to light previously unknown types of evidential markers in noun phrases.

\section{Sensory evidentials} \label{sec:sensory}
Nearly all non-propositional evidential systems described in the literature involve sensory evidential meanings, rather than other types of evidentials. 

This section first discusses the visual vs non-visual contrast, which has been described for almost all languages with non-propositional evidentials. Second, it addresses the issue of non-visual sensory or auditory evidentials, which are considerably rarer. Third, it presents attested cases of non-propositional evidentials encoding the verbal source rather than the sensory access to information.\footnote{On the distinction between `source' and `access' in the description of evidential systems, see \citet{tournadre14evidentiality}.}

\subsection{Visual evidential} \label{sec:visible}
The first type of non-propositional evidential distinction to have been described  is that between visible and non-visible demonstratives in Kwak'wala (\citealt[527-531]{boas11kwakiutl}). Similar systems have been reported for various languages of the Pacific Northwest (see for instance \citealt{bach06deixis.wakashan}), and most languages with non-propositional evidential markers appear to include one or several visual evidential marker (see for example Tsou in section \ref{sec:loc}).  

 

\subsubsection{Proximal / distal and visual evidentials} \label{sec:proximal}
While in Kwak'wala (as well as other Wakashan and Salishan languages), the visual / non-visual contrast is independent of the proximal / distal distinction, it is not the case of some languages with non-propositional evidentials. For instance, in Dyirbal (\citealt[45]{dixon72dyirbal}, \citealt{dixon14nonvisible}) we find three series of demonstratives \ipa{ya--} `here and visible', \ipa{ba--} `there and visible' and \ipa{ŋa--} `not visible'. In this system, the proximal / distal contrast is neutralized for non-visible referents.\footnote{This is a common phenomenon, see exactly the same neutralization in the Tsou paradigm above, Table \ref{tab:tsou}.} The non-visible \ipa{ŋa--} demonstratives are used either when the referent is not perceivable, or perceivable through senses other than vision.\footnote{\citet{dixon14nonvisible} presents a detailed account of the non-vsible marker  \ipa{ŋa--}, which occurs in five distinct contexts: (1) audible but not visible; (2) previously visible but now just audible; (3) neither visible or audible (and not perceivable through other senses) (4) spirits (5) remembered.  } 

It is of utmost importance, when dealing with systems where evidential distinctions are not independent from proximal / distal contrasts, not to rely exclusively on elicitation and to use data from traditional stories and conversations, as speakers can have unreliable intuitions. Khaling for instance has a three-degree proximal / distal contrast; the `further distal' markers are spontaneously described as `non-visible' by speakers (in Nepali \ipa{adṛśya} `invisible') and proximal ones as `visible', though clear examples of further distal demonstratives with visible referents, and of proximal demonstratives with invisible ones can be found in stories (\citealt[399]{jacques14auditory}). There is thus a potential for `false positives' in the description of visible demonstratives.

\subsubsection{Extended meanings} \label{sec:extended.visible}
In some languages, visual non-propositional evidentials have extended uses that depart from direct sensory access. For instance, in Tsou (\citealt[55-8]{yang00tsou.case}) the visual evidential markers \ipa{'e},  \ipa{si},  \ipa{ta} can be used with a variety of non-sensory meanings. 

First, the proximal visual marker \ipa{'e} can be used when `a speaker is so involved in telling a story that he feels as if the narrated event or object were visible'. In addition the visual evidentials can be used to refer to words that the speaker has just heard, in which case the proximal visual evidential indicates high involvement (example \ref{ex:eknuyu}) whereas the medial visual evidential marks lesser involvement, when for instance the speaker is `an outsider in the conversation' (\ref{ex:siknuyu}).

\begin{exe}
\ex \label{ex:eknuyu}
\gll \ipa{'e} \ipa{knuyu} \\
\textsc{nom:prox:vis} lie \\
\ex \label{ex:siknuyu}
\gll \ipa{si} \ipa{knuyu} \\
\textsc{nom:med:vis} lie \\
\glt That is a lie!
\end{exe}

The proximal visual evidential can used be used to refer to express intimacy: when referring to a close friend, only the proximal visual marker can be used, regardless of the visibility of that person at the time of speaking.

Finally, the proximal visual evidential is used when the speaker takes responsibility for the reliability of information coming from dreams or visions.

It is likely that non-propositional visual evidentials may have similar extended or metaphorical uses in other languages.

\subsubsection{Vision vs best sensory evidence}
Some languages have non-propositional evidentials that encode not specifically visual perception, but, to use \citet{gutierrez11evidentiality, gutierrez12determiners}'s terminology, \textit{best sensory evidence}. While best sensory evidence is nearly always equivalent with visual perception, there are specific cases in 


  
  
  
\subsection{Non-visual sensory evidentials}  \label{sec:auditory}
In languages with non-propositional evidential, referents that are perceptible through senses other than vision are treated in some languages in the same way as referent which are absent or non-perceptible, for example with the marker \ipa{ŋa--} in Dyirbal (\citealt{{dixon14nonvisible}}).

Several languages, including Southern Pomo (Pomoan, \citealt[37, ft]{oswalt86evidential}), Santali (Austroasiatic, \citealt[42-44]{neukom01santali}), Tsou Austronesian, (\citealt{yang00tsou.case}, see Table \ref{tab:tsou}), Nyelayu and Yuanga (Oceanic, \citealt[42-44]{neukom01santali}), Muna  (Oceanic, \citealt{berg97deixis.muna}), have been described as having  demonstratives used to refer to participants that are invisible but perceptible through senses other than vision. Some of the descriptions cited above refer to these markers as \textit{auditory} or \textit{auditive} demonstratives. 

In the case of Tsou, as shown by \citet[50-1]{yang00tsou.case}, the marker \ipa{co} is used with participants that are not visible but perceptible through audition, touch, smell or any non-visual sensation, including endopathic feelings such as hunger:


\begin{exe}
\ex 
\gll \ipa{mi-cu} \ipa{tazvo'hi} \ipa{co} \ipa{f'UsU-'u} \\
\textsc{aux-pfv} be.long \textsc{nom:sens:n.vis} hair-\textsc{1sg} \\
\glt My hair has grown long. (Not visible, since it is on my head)
\end{exe}

\begin{exe}
\ex 
\gll \ipa{mo} \ipa{mema'congo} \ipa{co} \ipa{poepe} \\
\textsc{aux} fly:strong \textsc{nom:sens:n.vis} wind \\
\glt I felt the wind was strong.
\end{exe}

Like visual evidentials in this language (see \ref{sec:extended.visible}), the non-visual sensory marker \ipa{co} has extended uses. It can appear in sentences such as \ref{ex:tueafa}, in a situation where the money in question is not perceivable, but the speaker feels that the addressee has money.

\begin{exe}
\ex  \label{ex:tueafa}
\gll \ipa{tueafa}   \ipa{co} \ipa{peisu-'mu} \\
give \textsc{nom:sens:n.vis} money-\textsc{2sg:poss} \\
\glt Give (me) your money.
\end{exe}

It cannot be excluded that some of the markers described as auditory demonstratives should in fact be analyzed as non-visual sensory; recent field data from Yuanga (\citealt{bril-yuanga}), in particular, where the non-visual marker  \ipa{--ili} is used to refer to a liquid only perceptible through its taste, suggest that the cognate marker \ipa{--ili} in the closely related Nyelayu language too might not be an auditory demonstrative \textit{stricto sensu}.


Table \ref{tab:attested} summarizes all known cases of language with non-visual sensory non-propositional evidential markers.  Santali stands out in being the only language with a proximal / distinction on non-visual sensory demonstratives; in all other languages, the proximal / distal contrast is neutralized with non-visual evidentials. 

\begin{table}[H]
\caption{Non-propositional evidential systems with non-visual sensory evidentials } \label{tab:attested}
\resizebox{\columnwidth}{!}{
\begin{tabular}{lcccl}
\toprule
  &	proximal / distal   & 	Indexation of   & 	Indexation of   	&References\\
&  contrast &visual perception &auditory perception \\
\midrule  
Southern Pomo  & 	unknown  & 	unclear  & 	no  & 	\citet[37, ft]{oswalt86evidential}\\
Santali  & 	yes  & 	yes  & 	no  & 	\citet[42-44]{neukom01santali} \\
Tsou & no & yes & no & \citet{yang00tsou.case}\\ 
Nyelayu  & 	no  & 	yes  & 	no  & 	\citet[98]{ozanne97spatial}\\
Muna  & 	no  & 	yes  & 	no  & 	\citet{berg97deixis.muna}\\
Khaling  & 	no  & 	no  & 	yes  & 	\citet{jacques14auditory} \\
\bottomrule
\end{tabular}}
\end{table}	
 
The only language for which we have positive evidence of the existence of an auditory demonstrative is Khaling (Kiranti, Nepal), \citealt{jacques14auditory}). Khaling has no visual demonstratives (see section \ref{sec:proximal}), but has an demonstrative adverb \ipa{tikí} `there (audible)' used to refer to an entity that is perceivable by its sound. Its nominalized form \ipa{tikî-m} `this (audible)' can be used either as a nominal modifier as in  example (\ref{ex:salpu}) or occur on its own (\ref{ex:what}). 

\begin{exe}
\ex \label{ex:salpu}
\gll    	 	\ipa{tikî-m}   	\ipa{kɵ̂m-go-jo}   	\ipa{ʣe-pɛ}   	\ipa{sʌ̄lpu-ʔɛ}   	\ipa{ʔʌnɵ̂l-ni}   	\ipa{mâŋ-go}   	\ipa{blɛtt-ʉ}   	\ipa{ɦolʌ}   
 \\
 there:\textsc{aud}-\textsc{nmlz} cloud-\textsc{inside-locative.level} speak-\textsc{nmlz:}S/A bird-\textsc{erg} today-\textsc{top} what-\textsc{foc} tell-\textsc{3sg$\rightarrow$3} maybe \\
\glt The bird that is singing in the clouds, what might it be telling today? (excerpt from a song by the Khaling songwriter Urmila)
\end{exe}

\begin{exe}
\ex \label{ex:what}
\gll  	\ipa{mâŋ}  	 	\ipa{tikî-m?}   \\
what  there:\textsc{aud}-\textsc{nmlz} \\
\glt What is that ? (of a something making a sound in another room of the house, not visible at the time of utterance, heard in context)
\end{exe}

The determiner \ipa{tikî-m} is nearly always used to refer to objects, animals or persons  that are audible but invisible; it cannot be used to refer to sensory access through other senses, including taste, touch or pain without auditory perception. Two native speakers independently explained the meaning of  \ipa{tikî-m} as (\ref{ex:def}).\footnote{This native gloss for the meaning of \ipa{tikî-m} actually provides a possible hint as to its etymology; as pointed out by Aimée Lahaussois (p.c.), \ipa{tikî-m} could be derived from a fusion of the proximal demonstrative \ipa{tɛ} `this' with the nominalized form of the verb `hear' in the first inclusive plural/generic form \ipa{ŋi-kî-m}, literally `this one that we/people (can) hear'. In this hypothesis, the demonstrative adverb \ipa{tikí} `there' would have been back-formed from the \ipa{tikî-m}, not an impossible assumption given the fact that  \ipa{tikî-m} occur with considerably greater frequency in conversation that  \ipa{tikí}. }

\begin{exe}
\ex \label{ex:def}
\gll  	 	 \ipa{mu-toɔç-pɛ,} \ipa{ŋi-kî-m} \ipa{tʌ̂ŋ}   \\
\textsc{neg}-be.visible-\textsc{nmlz:S/A} hear-\textsc{1pi-nmlz:O} only  \\
\glt (It refers to something) invisible, which we only hear.
\end{exe}

Yet, there are specific contexts where \ipa{tikî-m}  can be used with visible referents, as in example  (\ref{ex:kogu}), uttered by a person watching a song contest on the television. The use of \ipa{tikî-m} highlights the  fact that the  speakers' perception of a referent is primarily or exclusively via the auditory channel.

\begin{exe}
\ex \label{ex:kogu}
\gll  	\ipa{tikî-m-kʌ}   	\ipa{ʦʌ̄i} \ipa{ʔuŋʌ} \ipa{tūŋ }   	\ipa{kog-u}   \\
there:\textsc{aud}-\textsc{nmlz}-from \textsc{top} \textsc{1sg:erg} more be.able-\textsc{1sg$\rightarrow$3sg}	  \\
\glt I can (sing) better than that one. (Heard in context)
\end{exe}

 The case of Khaling \ipa{tikî-m} `this (audible)' is interesting for two reasons. First, it shows that specifically auditory evidential are possible, even when no corresponding visual evidential exists in the system.  Second it suggests that non-propositional evidential systems may include categories that are not attested in propositional evidentials. There is only one reported case of auditory evidential in verbal morphology, namely perhaps Yuchi as described by \citet{linn01euchee} (see also \citealt[37]{aikhenvald06}). However, the data in this source are not sufficient to exclude the possibility that the affix in question is a non-visual sensory evidential rather than a specifically auditory one, and the only other available source on this language (\citealt{wagner38yuchi}) cannot settle this matter.

\subsection{Joint perception}
While most sensory evidentials only encode the perception of the speaker, some languages have two degree of sensory access, namely a contrast between single eye-witness and joint perception, as in Maaka (\citealt[195-7]{storch14maaka}, see section \ref{sec:loc}).

 The joint-perception marker \ipa{-dìyà} in Maaka is used when both speaker and hearer see the referent in question, as in (\ref{ex:diya}).
 
 \begin{exe}
\ex \label{ex:diya}
\gll \ipa{làa} 	\ipa{nàmáa-dìyà} 	\ipa{sáy} 	\ipa{mìnè-póɗí-ní} 	\ipa{gè-gòrkù-wà}  \\
child this-\textsc{joint:vis} must \textsc{1pl}-remove:\textsc{tel-obj:3sg:masc} \textsc{loc}-village-\textsc{def} \\
\glt This child (whom we can both see), we must him from the village.
\end{exe}

\subsection{Sensory evidentials and utterance time} \label{sec:UT}
While most non-propositional sensory evidentials described in the literature indicate the sensory (visual or non-visual) perceptibility of a particular referent at utterance time (this is the case for instance for the evidentials in Khaling  and Tsou seen above), some languages rather encode sensory observations having occurred at any point during the lifetime of the speaker (including the time of utterance).

  The Nivacle language (Matacoan-Mataguayan) offers a clear example of a system of this type. Nivacle has four determiners \ipa{na}, \ipa{ja}, \ipa{ca}  and \ipa{pa} encoding evidential meanings, as summarized in Table (\ref{tab:nivakle}) taken from \citet{gutierrez11evidentiality}.\footnote{The system was first described by \citet[363]{stell89niwakle}. All Nivacle examples in this paper come from \citet{gutierrez11evidentiality}.}  
  
  
\begin{table}[H]
\caption{Nivacle determiner system (\citealt{gutierrez11evidentiality}) } \centering \label{tab:nivakle}
\begin{tabular}{l|c|cc|cc}
\toprule
&\multicolumn{2}{c}{Best sensory evidence} &&Non-Best sensory evidence \\
\hline
Present at UT & \multicolumn{2}{c}{\ipa{na}}&& \\
\cline{1-4}
Absent at UT &\ipa{ja} & \ipa{ca}&&\ipa{pa}& \\
&&ceased to exist&\\
\bottomrule
\end{tabular}
\end{table}
   
  The determiner \ipa{pa} is used with referents that the speaker has never seen in his life and are not physically present at the moment of utterance, even if the speaker is sure of their existence (\ref{ex:chita}) or even has talked to them on the phone without ever seeing them (\ref{ex:leon}).
  
  \begin{exe}
\ex \label{ex:chita}
\gll   \ipa{caaj}  \ipa{lhôn}  \ipa{lh-pa}  \ipa{ve'lha}  \ipa{chita'} \\
have \textsc{rep} \textsc{fem-not.best.sens} one elder.sister \\
\glt `I have an elder sister' (I have been told, I never met her)
\end{exe}


\begin{exe}
\ex  \label{ex:leon}
\gll \ipa{ja-yasinôy-esh} \ipa{pa}  \ipa{León} \\
\textsc{1sg}-talk.to-\textsc{com} \textsc{not.best.sens} \textsc{name} \\
\glt  `I talked to León.' (only on the phone, but I never met him)
\end{exe}

If the speaker has seen the referent even once in his lifetime, \ipa{pa} cannot be used and one of the other determiner \ipa{na}, \ipa{ja}, \ipa{ca} must appear  depending on whether the referent is  spatially present, absent (or non-visible) or deceased. For instance, example (\ref{ex:baby1}) can be used if the speakers hears a baby crying if he has never seen it before.

\begin{exe}
\ex  \label{ex:baby1}
\gll   \ipa{yip-'in}  \ipa{pa} \ipa{taôĉlaj} \\
cry-\textsc{ipfv} \textsc{not.best.sens} baby \\
\glt  `A baby is crying.'
\end{exe}

Example (\ref{ex:baby2}), on the other hand, implies that the speaker has seen the baby before in his life, and the determiner  \ipa{ja} (rather than \ipa{na}) indicates the speaker can only hear the baby and not see it at utterance time.

\begin{exe}
\ex  \label{ex:baby2}
\gll    \ipa{ja} \ipa{lh-ôôs} \ipa{lh-ja} \ipa{Patricia} \ipa{yip-'in} \\
\textsc{best.sens:absent} \textsc{3sg.poss}-child \textsc{fem-best.sens:absent}
\textsc{name} cry-\textsc{ipfv}  \\
\glt  `Patricia's child is crying.'
\end{exe}

The contrast between \ipa{na} `spatially present' and \ipa{ja/ca} `absent', judging from data in \citet{gutierrez11evidentiality}, is also a sensory evidential one, as \ipa{na} cannot be used if the referent is only audible and not visible (example \ref{ex:baby2}). It seems possible to provide the alternative interpretation of the Nivaclé system in Table (\ref{tab:nivakle2}): the determiners \ipa{na} vs \ipa{ja/ca} encode whether the speaker has best sensory information at utterance time about a referent on which they had best sensory information at some point in their lifetime. In this view, Nivaclé has the same sensory evidential contrast, but distinguishes several possible temporal XXXXX
  
  
\begin{table}[H]
\caption{Nivaclé determiner system (alternative interpretation) } \centering \label{tab:nivakle2}
\begin{tabular}{l|c|cc|cc}
\toprule
&\multicolumn{2}{c}{BSE at some point in lifetime} &&Non-BSE \\
\hline
BSE at UT& \multicolumn{2}{c}{\ipa{na}}&& \\
\cline{1-4}
BSE at UT&\ipa{ja} & \ipa{ca}&&\ipa{pa}& \\
&&ceased to exist&\\
\bottomrule
\end{tabular}
\end{table}

  
  
  Maaka (see section \ref{sec:loc}) h
 
 



 %\citet{gutierrez12determiners}'s words 
%whether or not the speaker has the best possible grounds to believe in the existence of an individual
 
\section{Non-sensory} \label{sec:nonsens}
Non-sensory evidential distinctions are very rare in non-propositional evidential systems. I know of only two cases, Tsou (\citealt{yang00tsou.case} see Table \ref{tab:tsou} above) and Nambiquara (\citealt{lowe99nambiquara}, see Table \ref{tab:nambiquara}).\footnote{Note that Lakondê, a language closely related to Nambiquara, has been reported to have sensory non-propositional evidentials (\citealt[248-9]{wetzels06lakonde}), but it is unclear whether it also has inferential or hearsay evidentials on nouns.}  These two systems present some commonalities. 


 \begin{table}[H]
 \caption{Nambiquara nominal evidential markers, \citet[282]{lowe99nambiquara} } \centering \label{tab:nambiquara}
\begin{tabular}{llllllll}
\toprule
\ipa{-a^2} & definite, unmarked \\
\ipa{-ai^2na^2} & definite, current \\
\ipa{-in^3ti^2} & observational, recent past, given \\
\ipa{-ait^3ta^3li^2} & observational, mid-past, given \\
\ipa{-ait^3tã^2} & observational, mid-past, new \\
\ipa{-nũ^1tã} & inferential, definite, unmarked \\
\ipa{-nũ^1tai^2na^2} & inferential, current \\
\ipa{-au^3tẽʔ^1tã^2} & quotative, mid-past, given \\
\bottomrule
\end{tabular}
\end{table}


First, in addition to non-sensory evidentials, they also have sensory (including visual evidentials).This may be suggestive of an implicational universal, namely that in non-propositional evidential systems, the presence of non-sensory evidential markers implies that of sensory evidentials, though the size of the sample does not allow reaching any firm conclusion at this stage.
 
 Second, both systems have distinct inferential and hearsay evidentials. 
 
 They differ however in that Nambiquara combines nominal tense with evidentiality, at least for the observational (sensory evidential) and inferential, as shown by example (\ref{ex:walinsunti}).

\begin{exe}
\ex \label{ex:walinsunti}
\gll  \ipa{hĩ^1na^2su^2} \ipa{wa^3lin^3-su^3-nti^2} \ipa{ḭ̃^3-a^1-ra^2} \\
today manioc-\textsc{cl.bone.like-observ.rec.pst.given} plant-\textsc{1sg-pfv} \\
\glt Today I planted the manioc roots that we both saw earlier today. (\citealt[290, ex 62.]{lowe99nambiquara} ) 
  \end{exe}
 
 negation
  
   \citet[193]{eijk97lillooet}

St'at'imcets
\citet{gutierrez11evidentiality, gutierrez12determiners}

\section{Non-propositional evidentiality and nominal tense}

Non-propositional evidentiality are much rarer than nominal tense (on which see \citealt{nordlinger04nominal}, \citealt[132]{francois05overview}), and are not incompatible with it.



\section{Conclusion}


\bibliographystyle{unified}
\bibliography{bibliogj}
\end{document}
