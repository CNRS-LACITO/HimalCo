\documentclass[oneside,a4paper,11pt]{article} 
\usepackage{fontspec}
\usepackage{natbib}
\usepackage{booktabs}
\usepackage{xltxtra} 
\usepackage{polyglossia} 
\usepackage[table]{xcolor}
\usepackage{lineno}
\usepackage{gb4e} 
\usepackage{multicol}
\usepackage{graphicx}
\usepackage{float}
\usepackage{hyperref} 
\hypersetup{bookmarks=false,bookmarksnumbered,bookmarksopenlevel=5,bookmarksdepth=5,xetex,colorlinks=true,linkcolor=blue,citecolor=blue}
\usepackage[all]{hypcap}
\usepackage{memhfixc}
\usepackage{lscape}
\bibpunct[: ]{(}{)}{,}{a}{}{,}
 
%\setmainfont[Mapping=tex-text,Numbers=OldStyle,Ligatures=Common]{Charis SIL} 
\newfontfamily\phon[Mapping=tex-text,Ligatures=Common,Scale=MatchLowercase,FakeSlant=0.3]{Charis SIL} 
\newcommand{\ipa}[1]{{\phon \mbox{#1}}} %API tjs en italique
 
 
 
\newcommand{\grise}[1]{\cellcolor{lightgray}\textbf{#1}}
\newfontfamily\cn[Mapping=tex-text,Ligatures=Common,Scale=MatchUppercase]{SimSun}%pour le chinois
\newcommand{\zh}[1]{{\cn #1}}

\newcommand{\jg}[1]{\ipa{#1}\index{Japhug #1}}
\newcommand{\wav}[1]{#1.wav}
\newcommand{\tgz}[1]{\mo{#1} \tg{#1}}

\XeTeXlinebreaklocale 'zh' %使用中文换行
\XeTeXlinebreakskip = 0pt plus 1pt %
 %CIRCG
\begin{document} 
%\linenumbers
\sloppy

\title{Non-propositional evidentiality\footnote{I am grateful to Alexandra Aikhenvald, Benjamin Brosig, Gong Xun, Zev Handel, Annie Montaut, Amos Teo, Alice Vittrant for useful comments on previous versions of this paper.}}
 
\author{Guillaume Jacques}
\maketitle


\section{Introduction}
In most languages of the world where evidentiality is grammaticalized, it is expressed either by verbal morphology, sentential markers or adverbs, which have scope over the entire proposition. A minority of languages have evidential-like distinctions on markers (mainly deictic, see \citealt[130]{aikhenvald06}) whose scope is limited to a noun phrase. The present study focuses on these non-propositional evidential markers.

Evidential markers present in relative clauses embedded within noun phrases are not considered in this study: although some languages have restrictions on the use of evidential markers in relative clauses and other non-main clauses,\footnote{For instance in Japhug Gyalrong, the inferential cannot be used in relative clauses with finite verb (\citealt{jacques16relatives}).} many languages allow evidential markers on the verb in relative clauses (\citealt[253-6]{aikhenvald06}; see for instance Nivaclé in section \ref{sec:prop}). Such evidential markers, although their scope is limited to the noun phrase that includes the relative (from the point of view of the main), are not strictly non-propositional, since they also at the same time have scope over the entire relative clause.

%\footnote{I omnipredicative languages such as Nahuatl or most Salish languages, nouns, although distinct from verbs, are predicative and must be relativized to be used in argument position (\citealt{launey94}). Thus, an omnipredicative language with allowing evidential marking in relative clauses would potentially allow}

Excluded from this survey are likewise evidential markers that can combine with nouns phonologically, but have scope over the whole sentence, such as the reportative  \ipa{--si} in Quechua (on which see for instance \citealt{faller02cuzco}).

 
This paper is divided into eight sections.  First, I present the different types of   non-propositional evidential markers, including demonstrative pronouns and adverbs, determiners or various types of affixes. Second,  I show how non-propositional evidential markers can encode  morphosyntactic parameters such as case or topicality in addition to evidentiality. Third, I describe the different types of non-propositional sensory evidentials attested in the world's languages. Fourth, I briefly mention a few rare cases of non-propositional non-sensory evidentials. Fifth, I discuss how non-propositional evidentiality and nominal tense can interact in some languages. Finally, I present some general observations on propositional and non-propositional evidential systems.

\section{Subtypes of non-propositional evidentials} \label{sec:subtypes}
Non-propositional evidential systems as defined above include evidential distinctions on nominal determiners (either affixes, clitics or independent words),  on demonstrative pronouns and adverbs, on case or discourse markers and also directly on nouns.

In languages with evidential marking on demonstratives, the  non-propositional evidential sub-systems may be embedded within the proximal / distal system.  This is particularly common in the case of sensory evidentials, especially those marking visibility, as exemplified by Lillooet (Salish) in Table \ref{tab:statimcets.pro} and Tsou (Austronesian) in Table \ref{tab:tsou} (\citealt{tung64tsou, yang00tsou.case} and discussion below; see also sections \ref{sec:proximal}, \ref{sec:auditory} and \ref{tab:attested}).  Some languages combine visible / invisible and proximal / distal with other contrasts, such as elevation (\citealt{schapper14elevation}).

\begin{table}[H]
\caption{Demonstrative pronouns in Lillooet (\citealt[168-9]{eijk97lillooet})} \label{tab:statimcets.pro} \centering 
\begin{tabular}{l|lll|llllll}
\toprule
& \multicolumn{3}{c}{visible} & \multicolumn{3}{c}{invisible} \\
&\textsc{prox} & \textsc{mid} & \textsc{dist}&\textsc{prox} & \textsc{mid} & \textsc{dist} \\
\midrule
\textsc{sg} & \ipa{cʔa} & \ipa{tiʔ} & \ipa{tʔu} & \ipa{kʷʔa} & \ipa{niʔ} & \ipa{kʷuʔ}  \\
\textsc{pl} & \ipa{ʔizá} & \ipa{ʔiz'} & \ipa{ʔizú} & \ipa{kʷɬa} & \ipa{nəɬ} & \ipa{kʷɬ} \\
\bottomrule
\end{tabular}
\end{table}

Lillooet determiners, on the other hand, have a much more fine-grained system, which encodes two degrees of sensory  evidential distinctions, but lacks the proximal/distal distinction. Table \ref{tab:statimcets.det}  presents \citet{eijk97lillooet}'s analysis of the system. 

 \begin{table}[H]
\caption{Articles in Lillooet (\citealt[192]{eijk97lillooet})} \label{tab:statimcets.det} \centering 
\begin{tabular}{l|lllllllll}
\toprule
& \multicolumn{2}{c}{known} &\multicolumn{2}{c}{unknown} \\
& present & absent & present & absent \\
\midrule
\textsc{sg} & \ipa{ti...a} & \ipa{ni...a} & \ipa{kʷu...a} &\ipa{kʷu} \\
\textsc{pl} &  \ipa{ʔi...a} & \ipa{nəɬ...a} & \ipa{kʷɬ...a} &\ipa{kʷɬ} \\
\bottomrule
\end{tabular}
\end{table}
 
 The determiner `present, known' \ipa{ti...a} or its plural form is used to refer to person or things visible to the speaker at utterance time (as in \ref{ex:tia}), or in specific cases to entities that the speaker saw in the past at an unspecified moment (\citealt[193]{eijk97lillooet}).\footnote{For glossing \ipa{=a}, I adopt \citet{matthewson98determiners}'s analysis as an `assertion of evidence'. }
 
\begin{exe}
\ex \label{ex:tia}
\gll \ipa{pún-ɬkan} 	\ipa{ti=n-ɬk'ʷál'us=a} \\
find-\textsc{1sg.A} \textsc{det:vis=1sg.poss}-basket=\textsc{assertion.of.existence} \\
\glt I found my basket (when just mentioning the fact, or when showing the backet to the addressee)
\end{exe}

The determiners `unknown, present' (\ipa{kʷu...a} and \ipa{kʷɬ...a}) on the other hand is used for entities that are not visible but perceptible through another sense, in particular audition or smell (\citealt[195]{eijk97lillooet}), as in example \ref{ex:c7as}.\footnote{On the meaning of the propositional evidential marker \ipa{lákʷʔa}, see \citet{matthewson10lakw7a}.}

\begin{exe}
\ex \label{ex:c7as}
\gll \ipa{cʔas} 	\ipa{lákʷʔa} 	\ipa{ɬlákʷu} 	\ipa{kʷu=sƛʼaɬáləm=a} \\
 come \textsc{n.vis} there:\textsc{n.vis} \textsc{det:n.vis.sens}=grizzly=\textsc{assertion.of.existence} \\
\glt There is a grizzly coming from there (used by a person who hears a grizzly).
\end{exe}

The other determiners \ipa{ni...a} and \ipa{kʷu} are used for referents that are not perceptible.

In some languages, noun modifiers can be derived from demonstrative adverbs by means of a nominalizing morpheme. In such systems, demonstrative determiners and demonstrative adverbs are completely parallel and show the same evidential contrasts. For instance, in Khaling (Sino-Tibetan, Kiranti, Nepal), the demonstrative determiner/pronoun \ipa{tikî-m} `this (audible)' is derived from the demonstrative adverb \ipa{tikí} `there (audible)' by means of the all-purpose nominalizer \ipa{--m}.\footnote{See \citet{bickel99nmlz} on this type of nominalizers and their various uses in the syntax of most Sino-Tibetan languages.}

Cases of languages where nouns can directly take the same set of evidential markers as verbs (with semantic scope on the noun phrase) are extremely rare; Jarawara (Arawan, \citealt[88, ex 3.19]{aikhenvald06}) however offers such an example, as in (\ref{ex:mone}) where the noun phrase \ipa{Banawaa} \ipa{batori} `the mouth of the Banawá' takes the reported evidential suffix \ipa{--mone}.

\begin{exe}
\ex \label{ex:mone}
\gll  \ipa{Banawaa} \ipa{batori-tee-mone} \ipa{jaa} \ipa{faja} \ipa{otaa} \ipa{ka-waha-ro} \ipa{otaa-ke} \\
Banawá mouth-\textsc{customary-reported.f} at then \textsc{1nsg.excl}.S \textsc{appl}-become.dawn-\textsc{remote.pst.firsthand-f} \textsc{1nsg-declarative.f} \\
\glt ‘Then the day dawned on us (\textsc{firsthand}) (lit. we with-dawned) at the
place \textsc{reported} to be (customarily) the mouth of the Banawá river’
\end{exe}

 \section{Non-propositional evidentials and other morphosyntactic parameters} \label{sec:parameter}
Non-propositional evidential markers  can be combined with  case marking and topicalization.

The Tsou data in Table (\ref{tab:tsou})   illustrate markers encoding both evidentiality and case. Note that the case markers in Tsou are portmanteau morphemes: it is not possible to decompose them into two morphemes (evidential marker and case marker), at least synchronically.  
 
\begin{table}[H]
 \caption{Tsou case markers, adapted from \citet[54]{yang00tsou.case}} \centering \label{tab:tsou}
\begin{tabular}{llllllll}
\toprule
	 & 	 & 	\multicolumn{2}{c}{case markers } 	 \\	
	 & 	 & 	nominative & 	oblique \\	
\midrule
	 & 	proximal & 	\ipa{'e} & 	\ipa{ta} \\ 	
visual	 & 	medial & 	\ipa{si} & 	\ipa{ta}  \\ 	
	 & 	distal & 	\ipa{ta} & 	\ipa{ta}  \\ 	
\multicolumn{2}{l}{non-visual sensory}  	 & 	\ipa{co} & 	\ipa{nca/ninca} \\ 	
	 \midrule
\multicolumn{2}{l}{hearsay}	 & 	\ipa{'o} & 	\ipa{to} \\ 	
\multicolumn{2}{l}{belief / inference} 	 & 	\ipa{na} & 	\ipa{no} \\ 	
\bottomrule
\end{tabular}
\end{table} 

In other languages where case and evidential markers interact, such as Dyirbal, the morphology is more concatenative. Note however that even in Dyirbal the case paradigms of the evidential demonstratives are not completely predictable (see \citealt{dixon14nonvisible}).

Aside from proximal / distal distinction and case, a third parameter has been shown to interact with non-propositional evidentiality: topicality. For instance,  the Chadic language Maaka has three evidential markers \ipa{--mú} `eye-witnessed', \ipa{--dìyà} `joint-perception', \ipa{--kà} `assumption' occurring on noun phrases (see section \ref{sec:joint}). They can be used with referents which are `hardly core participants, but rather topicalized peripheral participants that motivate an action or event' (\citealt[195-7]{storch14maaka}).

 

Non-propositional evidential system display a considerable diversity in terms of morphology, and it would not be surprising if future fieldwork brings to light previously unknown types of evidential markers in noun phrases.

\section{Sensory evidentials} \label{sec:sensory}
Nearly all non-propositional evidential systems described in the literature involve sensory evidential meanings, rather than other types of evidentials such as hearsay or inferential. 

This section first discusses the visual vs non-visual contrast, which has been described for almost all languages with non-propositional evidentials. Second, it addresses the issue of non-visual sensory or auditory evidentials, which are considerably rarer.  Third, it mentions the existence of evidential markers encoding joint perception of speaker and addressee. Fourth, it discusses the time frame of sensory perception, in particular the distinction between utterance time sensory evidentials \textit{vs} lifespan sensory evidentials.

\subsection{Visual evidential} \label{sec:visible}
The first type of non-propositional evidential distinction to have been described  is that between visible and non-visible demonstratives in Kwak'wala (\citealt[527-531]{boas11kwakiutl}). 

System of demonstratives encoding a visible / invisible contrast are not particularly rare cross-linguistically, and are found on all continents. In the Sino-Tibetan family alone, for instance, visible / invisible contrasts on demonstratives have been reported for Kham (\citealt{watters02grammar}), some varieties of Wu Chinese (\citealt[89]{yue03dialects}) and Darma (\citealt{willis15deictic}). The present paper does not attempt to systematically survey all systems of this type, but will mention some of their most conspicuous features.

 
\subsubsection{Proximal / distal and visual evidentials} \label{sec:proximal}
While in Kwak'wala (as well as other Wakashan and Salishan languages), the visual / non-visual contrast is independent of the proximal / distal distinction, it is not the case of some languages with non-propositional evidentials. For instance, in Dyirbal (\citealt[45]{dixon72dyirbal}, \citealt{dixon14nonvisible}) we find three series of demonstratives \ipa{ya--} `here and visible', \ipa{ba--} `there and visible' and \ipa{ŋa--} `not visible'. In this system, the proximal / distal contrast is neutralized for non-visible referents.\footnote{This is a common phenomenon, see exactly the same neutralization in the Tsou paradigm above, Table \ref{tab:tsou}.} The non-visible \ipa{ŋa--} demonstratives are used either when the referent is not perceivable, or perceivable through senses other than vision.\footnote{\citet{dixon14nonvisible} presents a detailed account of the non-visible marker  \ipa{ŋa--}, which occurs in five distinct contexts: (1) audible but not visible; (2) previously visible but now just audible; (3) neither visible or audible (and not perceivable through other senses) (4) spirits (invisible beings) (5) remembered.  } 

It is of utmost importance, when dealing with systems where evidential distinctions are not independent from proximal / distal contrasts, not to rely exclusively on elicitation and to use data from traditional stories and conversations, as speakers can have unreliable intuitions. Khaling for instance has a three-degree proximal / distal contrast; the `further distal' markers are spontaneously described as `non-visible' by speakers (in Nepali \ipa{adṛśya} `invisible') and proximal ones as `visible', though clear examples of further distal demonstratives with visible referents, and of proximal demonstratives with invisible ones can be found in stories (\citealt[399]{jacques14auditory}). There is thus a potential for falsely interpreting proximal / distal contrasts as visible / invisible ones.

\subsubsection{Extended meanings} \label{sec:extended.visible}
In some languages, visual non-propositional evidentials have extended uses that depart from direct sensory access. For instance, in Tsou (\citealt[55-8]{yang00tsou.case}) the visual evidential markers \ipa{'e},  \ipa{si},  \ipa{ta} can be used with a variety of non-sensory meanings. 

First, the proximal visual marker \ipa{'e} can be used when `a speaker is so involved in telling a story that he feels as if the narrated event or object were visible'. In addition the visual evidentials can be used to refer to words that the speaker has just heard, in which case the proximal visual evidential indicates high involvement (example \ref{ex:eknuyu}) whereas the medial visual evidential marks lesser involvement, when for instance the speaker is `an outsider in the conversation' (\ref{ex:siknuyu}).

\begin{exe}
\ex \label{ex:eknuyu}
\gll \ipa{'e} \ipa{knuyu} \\
\textsc{nom:prox:vis} lie \\
\glt That is a lie!
\ex \label{ex:siknuyu}
\gll \ipa{si} \ipa{knuyu} \\
\textsc{nom:med:vis} lie \\
\glt That is a lie!
\end{exe}

The proximal visual evidential can used be used to refer to express intimacy: when referring to a close friend, only the proximal visual marker can be used, regardless of the visibility of that person at the time of speaking.

Finally, the proximal visual evidential is used when the speaker takes responsibility for the reliability of information coming from dreams or visions.

It is likely that non-propositional visual evidentials may have similar extended or metaphorical uses in other languages, though more descriptive work is needed to ascertain this.

\subsubsection{Vision \textit{vs} `best' sensory evidence}
Some languages have non-propositional evidentials that encode not specifically visual perception, but, to use \citet{gutierrez11evidentiality}'s terminology, \textit{`best' sensory evidence}. While best sensory evidence is nearly always equivalent with visual perception, it can also be used for non-visual perception in specific cases. 

In Nivaclé for instance, the `best' sensory evidential \ipa{na}, while mainly used to refer to visible entities, is also compatible with tactile or gustatory perception. For instance if the speaker is blindfolded and asked to guess by touch or taste the nature of an object as in examples (\ref{ex:book}) and (\ref{ex:inoot}), it is still possible to use the determiner \ipa{na} rather than other determiners such as \ipa{ja} or \ipa{pa}, which indicate non-best sensory evidence at utterance time (see \ref{sec:UT} for an account of the determiner system of Nivaclé).


\begin{exe}
\ex \label{ex:book}
\gll  \ipa{na}  \ipa{vat-qu'is-jayan-ach}  \\
\textsc{det:best.sens:present} \textsc{indef.poss}-write-\textsc{caus-nmlz} \\
\glt `(It is) a book.'
\end{exe}


\begin{exe}
\ex  \label{ex:inoot}
\gll  \ipa{c'a-yôji} \ipa{na}   \ipa{inôôt}    \\
\textsc{1sg}-drink \textsc{det:best.sens:present} \ water \\
\glt  `I am drinking water.'
\end{exe}

In the case of blind persons, for whom touch is the best available sensory evidence, \ipa{na} is likewise used for tactile perception. 

One case however where \ipa{na} in Nivaclé can be used when the participant is not perceptible at utterance time is with nouns such as \ipa{jônshaja}  `night', \ipa{nalhu}  `world, sky, day' which refer to phenomena know to everybody (A. Fabre, p.c.). For instance, in example (\ref{ex:jonshaja}), even though \ipa{na} is used with \ipa{jônshaja} `obscurity, night', the sentence is not uttered during the night (and the night is therefore not `visible'). In such contexts, \ipa{pa} is also possible.

\begin{exe}
\ex \label{ex:jonshaja}
\gll \ipa{a-t’itan-jan} \ipa{lhôn} \ipa{na} \ipa{jônshaja-clai} \\
\textsc{2sg.irr}-twist-\textsc{apass} \textsc{report} \textsc{det:best.sens:present} \ obscurity-\textsc{dur} \\
\glt `You draw (fibres from the \textit{caraguatá} plant) during the night.' (\citealt[292]{fabre14nivacle})
  \end{exe}
  
  
\subsection{Non-visual sensory evidentials}  \label{sec:auditory}
In languages with non-propositional evidential, referents that are perceptible through senses other than vision are treated in some languages in the same way as referent which are absent or non-perceptible, for example with the marker \ipa{ŋa--} in Dyirbal (\citealt{{dixon14nonvisible}}).

Several languages, including Southern Pomo (Pomoan, \citealt[37, ft]{oswalt86evidential}), Santali (Austroasiatic, \citealt[42-44]{neukom01santali}), Tsou Austronesian, (\citealt{yang00tsou.case}, see Table \ref{tab:tsou}), Nyelayu and Yuanga (Oceanic,  \citealt{ozanne97spatial, bril-yuanga}), Muna  (Oceanic, \citealt{berg97deixis.muna}), Lillooet (\citealt[192-6]{eijk97lillooet}, see section \ref{sec:parameter}) have been described as having  demonstratives used to refer to participants that are invisible but perceptible through senses other than vision. Some of the descriptions cited above refer to these markers as \textit{auditory} or \textit{auditive} demonstratives. 

In the case of Tsou, as shown by \citet[50-1]{yang00tsou.case}, the marker \ipa{co} is used with participants that are not visible but perceptible through hearing, touch, smell or any non-visual sensation, including endopathic feelings such as hunger:


\begin{exe}
\ex 
\gll \ipa{mi-cu} \ipa{tazvo'hi} \ipa{co} \ipa{f'UsU-'u} \\
\textsc{aux-pfv} be.long \textsc{nom:sens:n.vis} hair-\textsc{1sg} \\
\glt My hair has grown long. (Not visible, since it is on my head)
\end{exe}

\begin{exe}
\ex 
\gll \ipa{mo} \ipa{mema'congo} \ipa{co} \ipa{poepe} \\
\textsc{aux} fly:strong \textsc{nom:sens:n.vis} wind \\
\glt I felt the wind was strong.
\end{exe}

Like visual evidentials in this language (see \ref{sec:extended.visible}), the non-visual sensory marker \ipa{co} has extended uses. It can appear in sentences such as \ref{ex:tueafa}, in a situation where the money in question is not perceivable, but the speaker feels that the addressee has money.

\begin{exe}
\ex  \label{ex:tueafa}
\gll \ipa{tueafa}   \ipa{co} \ipa{peisu-'mu} \\
give \textsc{nom:sens:n.vis} money-\textsc{2sg:poss} \\
\glt Give (me) your money.
\end{exe}

It cannot be excluded that some of the markers described as auditory demonstratives should in fact be analyzed as non-visual sensory; indeed, \citet[42]{neukom01santali} explicitely states that what the `auditive demonstratives' in Santali `may also refer to taste, feeling and smell.' The same is true of Yuanga; recent field data from \citet{bril-yuanga} show that the non-visual marker  \ipa{--ili} can be used to refer to a liquid only perceptible through its taste. This is possibly also the case in the closely related Nyelayu language whose non-visual marker  \ipa{--ili} is cognate to that of Yuanga, and more research on the other languages with reported auditory demonstrative might reveal similar cases.

Table \ref{tab:attested} summarizes all known cases of languages with non-visual sensory non-propositional evidential markers.  Santali stands out in being the only language with a proximal / distinction on non-visual sensory demonstratives; in all other languages, the proximal / distal contrast is neutralized with non-visual evidentials. 

\begin{table}[H]
\caption{Non-propositional evidential systems with non-visual sensory evidentials } \label{tab:attested}
\resizebox{\columnwidth}{!}{
\begin{tabular}{lcccl}
\toprule
  &	proximal / distal   & 	Indexation of   & 	Indexation of   	&References\\
&  contrast &visual perception &auditory perception \\
\midrule  
Southern Pomo  & 	unknown  & 	unclear  & 	no  & 	\citet[37, ft]{oswalt86evidential}\\
Santali  & 	yes  & 	yes  & 	no  & 	\citet[42-44]{neukom01santali} \\
Lillooet & yes &yes & no & \citet[171-196]{eijk97lillooet} \\
\midrule
Tsou & no & yes & no & \citet{yang00tsou.case}\\ 
Nyelayu  & 	no  & 	yes  & 	no  & 	\citet[98]{ozanne97spatial}\\
Muna  & 	no  & 	yes  & 	no  & 	\citet{berg97deixis.muna}\\
\midrule
Khaling  & 	no  & 	no  & 	yes  & 	\citet{jacques14auditory} \\
\bottomrule
\end{tabular}}
\end{table}	
 
The only language for which we have positive evidence of the existence of an auditory demonstrative is Khaling (Kiranti, Nepal, \citealt{jacques14auditory}). Khaling has no visual demonstratives (see section \ref{sec:proximal}), but has an demonstrative adverb \ipa{tikí} `there (audible)' used to refer to an entity that is perceivable by its sound. Its nominalized form \ipa{tikî-m} `this (audible)' can be used either as a nominal modifier as in  example (\ref{ex:salpu}) or occur on its own (\ref{ex:what}). 

\begin{exe}
\ex \label{ex:salpu}
\gll    	 	\ipa{tikî-m}   	\ipa{kɵ̂m-go-jo}   	\ipa{ʣe-pɛ}   	\ipa{sʌ̄lpu-ʔɛ}   	\ipa{ʔʌnɵ̂l-ni}   	\ipa{mâŋ-go}   	\ipa{blɛtt-ʉ}   	\ipa{ɦolʌ}   
 \\
 there:\textsc{aud}-\textsc{nmlz} cloud-\textsc{inside-locative.level} speak-\textsc{nmlz:}S/A bird-\textsc{erg} today-\textsc{top} what-\textsc{foc} tell-\textsc{3sg$\rightarrow$3} maybe \\
\glt The bird that is singing in the clouds, what might it be telling today? (excerpt from a song by the Khaling songwriter Urmila)
\end{exe}

\begin{exe}
\ex \label{ex:what}
\gll  	\ipa{mâŋ}  	 	\ipa{tikî-m?}   \\
what  there:\textsc{aud}-\textsc{nmlz} \\
\glt What is that ? (of a something making a sound in another room of the house, not visible at the time of utterance, overheard through participant -observation)
\end{exe}

The determiner \ipa{tikî-m} is nearly always used to refer to objects, animals or persons  that are audible but invisible; it cannot be used to refer to sensory access through other senses, including taste, touch or pain without auditory perception. Two native speakers independently explained the meaning of  \ipa{tikî-m} as (\ref{ex:def}).\footnote{This native gloss for the meaning of \ipa{tikî-m} actually provides a possible hint as to its etymology; as pointed out by Aimée Lahaussois (p.c.), \ipa{tikî-m} could be derived from a fusion of the proximal demonstrative \ipa{tɛ} `this' with the nominalized form of the verb `hear' in the first inclusive plural/generic \ipa{ŋi-kî-m}, literally `this one that we/people (can) hear'. In this hypothesis, the demonstrative adverb \ipa{tikí} `there' would have been back-formed from \ipa{tikî-m}, not an impossible assumption given the fact that  \ipa{tikî-m} occurs with considerably greater frequency in conversation that  \ipa{tikí}. }

\begin{exe}
\ex \label{ex:def}
\gll  	 	 \ipa{mu-toɔç-pɛ,} \ipa{ŋi-kî-m} \ipa{tʌ̂ŋ}   \\
\textsc{neg}-be.visible-\textsc{nmlz:S/A} hear-\textsc{1pi-nmlz:O} only  \\
\glt (It refers to something) invisible, which we only hear.
\end{exe}

Yet, there are specific contexts where \ipa{tikî-m}  can be used with visible referents, as in example  (\ref{ex:kogu}), uttered by a person watching a song contest on the television, and commenting on the signing abilities of one of the participants. The use of \ipa{tikî-m} highlights the  fact that the  speakers' perception of a referent is primarily or exclusively via the auditory channel.

\begin{exe}
\ex \label{ex:kogu}
\gll  	\ipa{tikî-m-kʌ}   	\ipa{ʦʌ̄i} \ipa{ʔuŋʌ} \ipa{tūŋ }   	\ipa{kog-u}   \\
there:\textsc{aud}-\textsc{nmlz}-from \textsc{top} \textsc{1sg:erg} more be.able-\textsc{1sg$\rightarrow$3sg}	  \\
\glt I can (sing) better than that one. (overheard through participant -observation)
\end{exe}

  

\subsection{Joint perception} \label{sec:joint}
While most sensory evidentials only encode the perception of the speaker, some languages have two degrees of sensory access, namely a contrast between single eye-witness and joint perception, as in Maaka (\citealt[195-7]{storch14maaka}, see section \ref{sec:parameter}).

 The joint-perception marker \ipa{-dìyà} in Maaka is used when both speaker and hearer see the referent in question, as in (\ref{ex:diya}), while the eye-witness marker \ipa{--mú} occurs if only the speaker, not the hearer, has seen it (\ref{ex:baayaamu}).
 
 \begin{exe}
\ex \label{ex:diya}
\gll \ipa{làa} 	\ipa{nàmáa-dìyà} 	\ipa{sáy} 	\ipa{mìnè-póɗí-ní} 	\ipa{gè-gòrkù-wà}  \\
child this-\textsc{joint:vis} must \textsc{1pl}-remove:\textsc{tel-obj:3sg:masc} \textsc{loc}-village-\textsc{def} \\
\glt This child (whom we can both see), we must him from the village.
\end{exe}

 \begin{exe}
\ex \label{ex:baayaamu}
\gll \ipa{tò} 	\ipa{báayà-à-mú} 	\ipa{mìnè-ʔákkó} 	\ipa{ɓà} 	\ipa{máy=ʔàŋgùwà}  \\
\textsc{top} \textsc{name-def-vis} \textsc{1pl}-do:\textsc{pfv} \textsc{cnj} chief-\textsc{title} \\
\glt As for Baaye (eye-witnessed), we dealt with Mai Anguwa.
\end{exe}

It is unclear if a language can have a joint perception sensory evidential without a corresponding single witness sensory evidential, as no such system has yet been reported.

\subsection{Sensory evidentials and utterance time} \label{sec:UT}
While most non-propositional sensory evidentials described in the literature indicate the sensory (visual or non-visual) perceptibility of a particular referent at utterance time (this is the case for instance for the evidentials in Khaling  and Tsou seen above), some languages rather encode sensory observations having occurred at any point during the lifetime of the speaker (including the time of utterance). 

  The Nivaclé language (Matacoan-Guaycuruan) offers a clear example of a system with lifespan non-propositional evidentials. Nivaclé has four determiners \ipa{na}, \ipa{ja}, \ipa{ca}  and \ipa{pa} encoding evidential meanings, as summarized in Table (\ref{tab:nivakle}) taken from \citet{gutierrez14determiners}.\footnote{The system was first described by \citet[363]{stell89niwakle}. All Nivaclé examples in this section come from \citet{gutierrez11evidentiality} and \citet{gutierrez14determiners}.}  
  
  
\begin{table}[H]
\caption{Nivaclé determiner system (\citealt{gutierrez14determiners}) } \centering \label{tab:nivakle}
\begin{tabular}{l|c|cc|cc}
\toprule
&\multicolumn{2}{c}{Best sensory evidence} &&Non-Best sensory evidence \\
\hline
Present at UT & \multicolumn{2}{c}{\ipa{na}}&& \\
\cline{1-4}
Absent at UT &\ipa{ja} & \ipa{ca}&&\ipa{pa}& \\
&&ceased to exist&\\
\bottomrule
\end{tabular}
\end{table}
   
  The determiner \ipa{pa} is used with referents that the speaker has never seen in his life and are not physically present at the moment of utterance, even if the speaker is sure of their existence (\ref{ex:chita}) or even has talked to them on the phone without ever seeing them (\ref{ex:leon}).
  
  \begin{exe}
\ex \label{ex:chita}
\gll   \ipa{caaj}  \ipa{lhôn}  \ipa{lh-pa}  \ipa{ve'lha}  \ipa{chita'} \\
have \textsc{rep} \textsc{fem-det:n.best.sens} one elder.sister \\
\glt `I have an elder sister' (I have been told, I never met her)
\end{exe}


\begin{exe}
\ex  \label{ex:leon}
\gll \ipa{ja-yasinôy-esh} \ipa{pa}  \ipa{León} \\
\textsc{1sg}-talk.to-\textsc{com} \textsc{det:n.best.sens} \textsc{name} \\
\glt  `I talked to León.' (only on the phone, but I never met him)
\end{exe}

If the speaker has seen the referent even once in his lifetime, \ipa{pa} cannot be used and one of the other determiner \ipa{na}, \ipa{ja}, \ipa{ca} must appear  depending on whether the referent is  spatially present, absent (or non-visible) or deceased. For instance, example (\ref{ex:baby1}) can be used if the speakers hears a baby crying if he has never seen it before.

\begin{exe}
\ex  \label{ex:baby1}
\gll   \ipa{yip-'in}  \ipa{pa} \ipa{taôĉlaj} \\
cry-\textsc{ipfv} \textsc{det:n.best.sens} baby \\
\glt  `A baby is crying.'
\end{exe}

Example (\ref{ex:baby2}), on the other hand, implies that the speaker has seen the baby before in his life, and the determiner  \ipa{ja} (rather than \ipa{na}) indicates the speaker can only hear the baby and not see it at utterance time.

\begin{exe}
\ex  \label{ex:baby2}
\gll    \ipa{ja} \ipa{lh-ôôs} \ipa{lh-ja} \ipa{Patricia} \ipa{yip-'in} \\
\textsc{det:best.sens:absent} \textsc{3sg.poss}-child \textsc{fem-det:best.sens:absent}
\textsc{name} cry-\textsc{ipfv}  \\
\glt  `Patricia's child is crying.'
\end{exe}

The contrast between \ipa{na} `spatially present' and \ipa{ja/ca} `absent', judging from data in \citet{gutierrez14determiners}, is also a sensory evidential one, as \ipa{na} cannot be used if the referent is nearby but only audible and not visible (example \ref{ex:baby2}). It seems possible to provide the alternative interpretation of the Nivaclé system in Table (\ref{tab:nivakle2}): the determiners \ipa{na} vs \ipa{ja/ca} encode whether the speaker has best sensory information at utterance time about a referent on which they had best sensory information at some point in their lifetime.\footnote{The determiner \ipa{ca} is not only used with referent that are deceased or have ceased to exist (\citealt[63-4]{fabre14nivacle}), but a detailed account of this marker goes beyond the scope of this paper. } In this view, Nivaclé determiners encode same sensory evidential contrast at two distinct time frames.
  
 
\begin{table}[H]
\caption{Nivaclé determiner system (alternative interpretation) } \centering \label{tab:nivakle2}
\begin{tabular}{l|c|cc|cc}
\toprule
&\multicolumn{2}{c}{BSE at some point in lifetime} && no BSE \\
\hline
BSE at UT& \multicolumn{2}{c}{\ipa{na}}&& \\
\cline{1-4}
no BSE at UT&\ipa{ja} & \ipa{ca}&&\ipa{pa}& \\
&&ceased to exist&\\
\bottomrule
\end{tabular}
\end{table}

The nominal evidential markers of Maaka (see sections \ref{sec:parameter} and \ref{sec:joint}) are also clearly encode sensory (perhaps only visual) perception at some point in the lifetime, rather than at utterance time, a shown by example \ref{ex:ciroma}, about a person not present at the time, but whose life-story had been witnessed by the speaker (\citealt[196]{storch14maaka}).

\begin{exe}
\ex \label{ex:ciroma}
\gll \ipa{yáayà} \ipa{círòmà-mú} \ipa{nín-nì} \ipa{gùu=ɓálɓìyá} \ipa{tà-lòwó} \ipa{gàamôɗí} \ipa{bòɲcéttí} \\
\textsc{name} \textsc{title-vis} mother-\textsc{3sg:poss:masc} person=\textsc{toponym} \textsc{3sg:fem}-deliver:\textsc{pfv} once \textsc{dem:ref} \\
\glt Yaaya Ciroma [eye-witnessed]: her mother is from Balbiya town, she once gave birth there.
\end{exe}

 The Maaka example proves that a language can have lifespan non-propositional evidentials without corresponding utterance time evidentials.
  
\section{Non-sensory evidentials} \label{sec:nonsens}
Non-propositional evidential systems encoding non-sensory evidential meanings are extremely uncommon, and all known systems also include sensory evidentials.

Languages with non-sensory non-propositional evidentials include Tsou (\citealt{yang00tsou.case}, see Table \ref{tab:tsou} above) and Nambiquara (\citealt{lowe99nambiquara}, see Table \ref{tab:nambiquara}).\footnote{Note that Lakondê, a language closely related to Nambiquara, has been reported to have sensory non-propositional evidentials (\citealt[248-9]{wetzels06lakonde}), but it is unclear whether it also has inferential or hearsay evidentials on nouns (this language may have lost non-propositional evidentiality markers, as it is known to have a simplified evidential system, see \citet[274-5]{aikhenvald12amazon}). Other Nambikwara varieties, such as Mamaindê, lack non-propositional evidentials (\citealt{eberhard09nambikwara} and p.c. from D. Eberhard).}  


 \begin{table}[H]
 \caption{Nambiquara nominal evidential markers, \citet[282]{lowe99nambiquara} } \centering \label{tab:nambiquara}
\begin{tabular}{llllllll}
\toprule
\ipa{-a^2} & definite, unmarked \\
\ipa{-ai^2na^2} & definite, current \\
\ipa{-in^3ti^2} & observational, recent past, given \\
\ipa{-ait^3ta^3li^2} & observational, mid-past, given \\
\ipa{-ait^3tã^2} & observational, mid-past, new \\
\ipa{-nũ^1tã} & inferential, definite, unmarked \\
\ipa{-nũ^1tai^2na^2} & inferential, current \\
\ipa{-au^3tẽʔ^1tã^2} & quotative, mid-past, given \\
\bottomrule
\end{tabular}
\end{table}

In addition to sensory evidentials, the rich non-propositional evidential systems of these two languages have distinct inferential and hearsay markers. Detailed descriptions of the use of these markers are not yet available.

By contrast with the rarity of non-sensory non-propositional evidentials, many languages without grammaticalized  non-propositional evidentiality commonly present evidential strategies expressing reportative meaning in noun phrases, such as the adjectives \ipa{alleged} or \ipa{so-called} in English. Typically, adjectives of this type have dubitative overtones and are not pure evidentials, and occur in highly marked situations.

\section{Non-propositional evidentiality and nominal tense} \label{sec:tense}

Non-propositional evidentiality is much rarer than nominal tense (on which see \citealt{nordlinger04nominal}, \citealt{haude04tense}, \citealt[132]{francois05overview}), and is not incompatible with it.


Some languages, such as Nambiquara, combine nominal tense with evidentiality within the same paradigm (see Table \ref{tab:nambiquara}). In Nambiquara,  the observational (sensory evidential)  distinguishes between recent past \ipa{--(i)n^3ti^2} and mid-past \ipa{-ait^3ta^3li^2} /
\ipa{-ait^3tã^2}, as in example (\ref{ex:walinsunti}). Other evidentials, such as the inferential and the quotative, appear to lack this distinction.

\begin{exe}
\ex \label{ex:walinsunti}
\gll  \ipa{hĩ^1na^2su^2} \ipa{wa^3lin^3-su^3-nti^2} \ipa{ḭ̃^3-a^1-ra^2} \\
today manioc-\textsc{cl.bone.like-observ.rec.pst.given} plant-\textsc{1sg-pfv} \\
\glt Today I planted the manioc roots that we both saw earlier today. (\citealt[290, ex 62.]{lowe99nambiquara} ) 
  \end{exe}

Non-propositional evidential contrasts may in specific contexts have readings that may lead fieldwork to analyze them as nominal tense markings. For instance, \citet[631]{campbell12chaco} (cited in \citealt{gutierrez14determiners}) interpret the contrast between (\ref{ex:tovoc1}) and (\ref{ex:tovoc2}) as nominal tense rather than non-propositional evidentiality.

\begin{exe}
\ex \label{ex:tovoc1}
\gll \ipa{tsej} \ipa{na} \ipa{tovôc} \\
  grow \textsc{det:best.sens:present} river \\
\glt `The river is rising.'
\end{exe}


\begin{exe}
\ex \label{ex:tovoc2}
\gll \ipa{tsej} \ipa{ja} \ipa{tovôc} \\
 grow \textsc{det:best.sens:absent} river \\
\glt `The river was rising.'
\end{exe}

\citet{gutierrez14determiners} however points out that this is a contextual reading of the evidential distinction (visible vs non-visible, see section \ref{sec:UT}) due to the lack of overt tense marking on the verb, which can be avoided if a temporal adverb is introduced. As shown by examples (\ref{ex:tovoc3}) and (\ref{ex:tovoc4}), \ipa{ja} is compatible with past or future contexts, and even with present contexts, as seen in section \ref{sec:UT}, if no visual or other best possible sensory evidence is available.


\begin{exe}
\ex \label{ex:tovoc3}
\gll \ipa{j-ovalh-ei} \ipa{ja} \ipa{tovôc} \\
 \textsc{1sg}-look.at-\textsc{dir} \textsc{det:best.sens:absent} river \\
\glt `I looked at the river.'
\end{exe}

\begin{exe}
\ex \label{ex:tovoc4}
\gll \ipa{j-ovalh-ei}  \ipa{jayu}   \ipa{ja} \ipa{tovôc} \\
 \textsc{1sg}-look.at-\textsc{dir} \textsc{prosp} \textsc{det:best.sens:absent} river \\
\glt `I will observe the river.'
\end{exe}

The existence of specific contexts where both non-propositional evidential and nominal tense would be compatible suggest that  pathways of diachronic evolution linking the two might exist: minimal pairs such as (\ref{ex:tovoc1}) and (\ref{ex:tovoc2}) could become pivot construction through which reanalysis from non-propositional evidential to nominal tense would be possible.

\section{Non-propositional \textit{vs} propositional evidentiality} \label{sec:prop}
 All languages with non-propositional evidential markers in the sample also have propositional evidentials. To illustrate how propositional and non-propositional evidentials interact within a single language, I draw here data from Nivaclé.
 
 Nivaclé has several markers of propositional evidentiality, including the reportative \ipa{lhôn}, the inferential/dubitative \ipa{t'e}  and the mirative \ipa{ma'lhan} (\citealt[256-257]{fabre14nivacle}). Available sources on this language do not report the existence of a sensory propositional evidential.

As shown by example (\ref{ex:cajojo}), the best sensory evidential determiner \ipa{na} and its feminine form \ipa{lha} are compatible with the phrasal indirect evidentials \ipa{lhôn} and \ipa{t'e} in a context where the referent is visible, but the property or ability ascribed to him has not been witnessed by the speaker. In this sentence, we have the marker \ipa{t'e} in the main clause (which however here rather has an epistemic modal meaning, expressing doubt) and the reportative \ipa{lhôn} in the complement clause, on a predicate whose sole argument is \ipa{lha} \ipa{cajôjô} `the frog', marked with the best sensory evidence determiner.

\begin{exe}
\ex \label{ex:cajojo}
\gll \ipa{y-ijô’} \ipa{t’e} \ipa{lha} \ipa{cajôjô} \ipa{ti} \ipa{Ø-pôtsej} \ipa{lhôn} \\
\textsc{3sg}-be.true \textsc{ifr} \textsc{fem:det:best.sens:present} frog \textsc{sub} \textsc{3sg}-be.fast \textsc{report} \\
\glt `Is it true that the frog (that I saw now) (jumps) fast?' (A. Fabre, p.c.)
\end{exe}

Indirect propositional evidentials are also possible in relative clauses, as in (\ref{ex:utisichat}).  Relative clauses of this type are thus doubly marked for evidentiality.

\begin{exe}
\ex \label{ex:utisichat}
\gll
[\ipa{pa} 	[\ipa{Ø-vaf-’e} 	\ipa{lhôn}] 	\ipa{yi-tsaat}] 	\ipa{lh-ei} 	\ipa{Utsichat} \\
\textsc{masc:det:n.best.sens} \textsc{3sg}-die-\textsc{appl:prox} \textsc{report} \textsc{indef.poss}-village \textsc{3sg.poss}-name \textsc{name} \\
\glt `The village where he died is called Utsichat.' (\citealt[157]{fabre14nivacle})
\end{exe}

These Nivaclé data illustrate two important facts about evidential systems. First, completely distinct propositional and non-propositional evidential systems can co-occur in the same language; in the case of Nivaclé, while propositional evidentials lack sensory markers, non-propositional evidentials all involve sensory meanings. Second, propositional and non-propositional evidentials can be combined in the same sentence and their meanings can complement each other, as in example (\ref{ex:cajojo}).
 

%\begin{exe}
%\ex \label{ex:jpoyich}
%\gll 
%\ipa{pa} 	\ipa{jpôyich} 	\ipa{pa-n} 	\ipa{ta-c’uiy-e-i} 	\ipa{lhôn} \\
%\textsc{masc:non.sens} house \textsc{masc:non.sens-dem} \textsc{3sg}-move-3-\textsc{dist} \textsc{report} \\
%\glt (\citealt[268]{fabre14nivacle})
%\end{exe}

Another important typological issue about non-propositional evidential systems is whether they may include categories that are not attested in propositional evidential system among languages of the world.

The auditory demonstrative in Khaling (see section \ref{sec:auditory}) might provide such an example, as it attests a specifically auditory evidential (not non-visual sensory, as it is not compatible with sensory perceptions other than audition).

   There is only one reported case of auditory evidential in verbal morphology, namely  Yuchi as described by \citet{linn01euchee} (see also \citealt[37]{aikhenvald06}). However, the data in this source are not sufficient to exclude the possibility that the affix in question is a non-visual sensory evidential rather than a specifically auditory one, and the only other available source on this language (\citealt{wagner38yuchi}) cannot settle this matter. Since further fieldwork on Yuchi might be difficult, only a detailed investigation of Yuchi texts (\citealt{wagner31tales}) might hope to bring an answer to this question.

\section{Conclusion}
Though relatively marginal from the point of view of the world's languages, non-propositional evidentials are relatively widespread in the languages of the Northwest Coast of Northern America, in particular Wakashan and Salish, the Amazon, and also in the  Austronesian language family.\footnote{Since many of these languages are also omnipredicative (\citealt{launey94, francois03predicat}), this raises the question whether a possible typological correlation can be drawn between the presence of non-propositional evidential and omnipredicativity. } In other areas of the world such as Australia, the Himalaya and South India, their presence is more diffuse, though, as suggested in section (\ref{sec:tense}), one cannot exclude the possibility that some systems described as having nominal tense might be reanalysable  as non-propositional evidentiality in some cases.

Non-propositional evidentiality is much rarer than propositional evidentiality, but there are cases of languages, like Dyirbal and Santali, with non-propositional evidentiality without clausal evidential markers. In most languages, propositional and non-propositional evidentials form completely distinct systems; the only exception appears to be Jarawara. 



Non-propositional evidential markers are overwhelming sensory evidentials.  Non-sensory non-propositional evidentials, though not completely unattested, are fairly rare, and  in view of the data available, it seems that the implicational universal (\ref{ex:non.sensory.univ}) can be proposed, as all the languages discussed in section (\ref{sec:nonsens}) also have non-propositional sensory evidentials.

\begin{exe}
\ex \label{ex:non.sensory.univ}
\glt \textsc{non-sensory non-propositional evidential} $\Rightarrow$ \textsc{sensory non-propositional evidential}
\end{exe}

\bibliographystyle{unified}
\bibliography{bibliogj}
\end{document}
