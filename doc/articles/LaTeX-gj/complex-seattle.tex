\documentclass[oldfontcommands,oneside,a4paper,11pt]{article} 
\usepackage{fontspec}
\usepackage{natbib}
\usepackage{booktabs}
\usepackage{xltxtra} 
\usepackage{polyglossia} 
\usepackage[table]{xcolor}
\usepackage{gb4e} 
\usepackage{multicol}
\usepackage{graphicx}
\usepackage{float}
\usepackage{hyperref} 
\hypersetup{bookmarks=false,bookmarksnumbered,bookmarksopenlevel=5,bookmarksdepth=5,xetex,colorlinks=true,linkcolor=blue,citecolor=blue}
\usepackage[all]{hypcap}
\usepackage{memhfixc}
\usepackage{lscape}
 \usepackage{bbding}
 
%\setmainfont[Mapping=tex-text,Numbers=OldStyle,Ligatures=Common]{Charis SIL} 
\newfontfamily\phon[Mapping=tex-text,Ligatures=Common,Scale=MatchLowercase]{Charis SIL} 
\newcommand{\ipa}[1]{{\phon\textbf{#1}}} 
\newcommand{\grise}[1]{\cellcolor{lightgray}\textbf{#1}}
\newfontfamily\cn[Mapping=tex-text,Ligatures=Common,Scale=MatchUppercase]{SimSun}%pour le chinois
\newcommand{\zh}[1]{{\cn #1}}
\newcommand{\Y}{\Checkmark} 
\newcommand{\N}{} 
\newcommand{\dhatu}[2]{|\ipa{#1}| `#2'}
 \newcommand{\jpg}[2]{\ipa{#1} `#2'}  
\newcommand{\refb}[1]{(\ref{#1})}

 \begin{document} 
\title{The life cycle of multiple indexation and bipartite verbs in Sino-Tibetan}
\author{Guillaume Jacques\\ CNRS-CRLAO-INALCO}
\maketitle

Fourth Workshop on Sino-Tibetan Languages of Southwest China, 

University of Washington, Seattle

 \section*{Introduction}
Bipartite verbs, though common in some areas of the world (\citealt{delancey96bipartite}), are relatively rare in Eurasia. In the Trans-Himalayan/Sino-Tibetan family, bipartite verbs are found in Kiranti and Gyalrongic, and present another uncommon typological characteristic, 
multiple argument indexation (\citealt{denk15multiple}).

This paper presents the first description of bipartite verbs in Gyalrongic, and show how the genesis of this relatively recent construction sheds light on how bipartite verbs were independently grammaticalized in Kiranti.

\section{Bipartite verbs in Kiranti}
Kiranti languages are well-known for having a class of complex predicates combining a lexical verb with an auxiliary, which share some of their person indexation and TAM markers. Example (\ref{ex:ityodzyoyi}) from Khaling shows the verb root \dhatu{tA}{put}\footnote{See \citet{jacques12khaling} for an account of stem alternations in Khaling.} combined with the auxiliary \dhatu{-dzA}{keep on} grammaticalized from the verb \dhatu{dzA}{eat}. The verb and the auxiliary share the second person prefix \ipa{ʔi-}, but the suffixes remain distinct: the auxiliary receives the full form of the suffix \ipa{-ji}, while the lexical verb has a reduced form \ipa{-j-}: dual number is thus redundantly indexed two times in this form.

\begin{exe}
\ex \label{ex:ityodzyoyi}
\gll \ipa{ʔi-tɵ-j-dzɵ-ji}\\
2/\textsc{inv}-put-\textsc{du.incl}-keep.on-\textsc{du.incl} \\
\glt You_{du} keep on putting it.
\end{exe}

While affixes in Khaling follow a rigid order, other languages such as Chintang (\citealt{bickel07chintang}) and Bantawa present variability in affix ordering in complex verb forms, as shown by the following examples (\citealt[170-172]{doornenbal09}):

\begin{exe}
\ex 
\gll 
\ipa{tɨ-man-kʰat-da} \\
2-\textsc{neg}-go-\textsc{pst} \\
\ex
\gll 
\ipa{man-tɨ-kʰat-da} \\
\textsc{neg}-2-go-\textsc{pst} \\
\ex 
\gll 
\ipa{man-kʰat-tɨ-da} \\
\textsc{neg}-go-2-\textsc{pst} \\
\glt `You did not go.'
\end{exe}

As in Khaling, some suffixes are obligatorily redundantly replicated in bipartite verb forms, such as the 1/2 plural \ipa{-in}, while other ones such as the exclusive \ipa{-ka} are not, as illustrated by \refb{ex:kharin}.

 \begin{exe}
\ex  \label{ex:kharin}
\gll \ipa{kʰar-in} \ipa{lont-in-ka} \\
go-\textsc{1/2pl} come.out-\textsc{1/2pl-excl} \\
\glt `We shall rise again.' (\citealt[254]{doornenbal09})
\end{exe}



\section{Bipartite verbs in Japhug} \label{sec:japhug.bipart}

Bipartite verbs in Japhug and other Gyalrong languages are a small class (only ten such verbs have been discovered up to now). 
The most common bipartite verb, \jpg{stu=mbat}{try hard, do one's best} can be conjugated in four distinct ways. Table \ref{tab:four} illustrates these four options with the imperative second dual form `try hard (the two of you)'.


\begin{table}[h]
\caption{Four degrees of integration} \centering \label{tab:four}
\begin{tabular}{lllllll}
\toprule
Type & Example & V_1 suffix & V_2 prefix \\
\midrule
A& \ipa{tɤ-stu-ndʑi} \ipa{tɤ-mbat-ndʑi} &\Y &\Y \\
 &\textsc{imp}-V_1-\textsc{du}  \textsc{imp}-V_2-\textsc{du} \\
B& \ipa{tɤ-stu=tɤ-mbat-ndʑi} &\N  &\Y \\
 &\textsc{imp}-V_1-\textsc{imp}-V_2-\textsc{du} \\
C& \ipa{tɤ-stu-ndʑi=mbat-ndʑi} &\Y  &\N \\
 &\textsc{imp}-V_1-\textsc{du}-V_2-\textsc{du} \\
B& \ipa{tɤ-stu-mbat-ndʑi} &\N  &\N \\
 &\textsc{imp}-V_1-V_2-\textsc{du} \\
\bottomrule
\end{tabular}
\end{table}

In type A forms, the two verb stems are not phonologically integrated, and each of them takes both prefixes and suffixes. This form is a type of lexicalized serial verb construction (cf section \ref{sec:serial}).

In type B and C forms, the two conjugated verb forms merge phonologically, and either the suffixe(s  of the first verb or the prefixe(s) of the second one are removed. Prefixal or suffixal chains are either preserved or completely removed; it is not possible to have intermediate forms. For instance, the type B form in example \refb{ex:atAtWstu} cannot be changed to something like $\dagger$\ipa{a-tɤ-tɯ-stu=tɯ-mbat-ndʑi} with only the second person \ipa{tɯ-} prefix without the irrealis prefixes.

\begin{exe}
\ex \label{ex:atAtWstu}
\gll \ipa{a-tɤ-tɯ-stu=a-tɤ-tɯ-mbat-ndʑi} \\
\textsc{irr-pfv-2}-try.hard(1)=\textsc{irr-pfv-2}-try.hard(2)-\textsc{du} \\
\glt `(while I am gone),  may the two of you do your best.' (Smanmi0, 56)
\end{exe}

In type D forms, the two verb come to share the same prefixal and suffixal chain, with no intervening affix between the two verb stems.

Table \ref{tab:bipartite} summarizes all bipartite verbs discovered up to now in Japhug. Note the important proportion of Tibetan loanwords (\ipa{=raŋ}, \ipa{zdɯɣ=sŋɤl}, \ipa{ntsʰɤβ=}, \ipa{rga=}, respectively from \ipa{riŋ}, \ipa{sdug.bsŋal}, \ipa{ⁿtsʰab}, \ipa{dga}).

The stative bipartite verbs \ipa{zdɯɣ=sŋɤl} , \ipa{rga=le}  and \ipa{rga=χi}  are attested in A form only in non-finite form.

XXX tropative verb forms

\begin{table}[h]
\caption{Bipartite verbs in Japhug} \label{tab:bipartite} \centering
\begin{tabular}{lllllllllll}
\toprule
Compound verb& Meaning	 & 	A & 	B & 	C & 	D & \\
\midrule
\ipa{stu=mbat} & 	try hard & 	\Y & 	\Y & 	\Y & 	\Y & 	\\	
\ipa{mu=cɯɣ} & 	be  terrified  & 	\Y & 	 & 	 & 	 & 	\\	
\ipa{χɕu=rnaʁ} & 	thank a lot & 	\Y & 	\Y & 	 & 	 & 	\\	
\ipa{ntsʰɤβ=rlu} & 	in a hurry & 	 & 	\Y & 	 & 	\Y & 	\\	
\ipa{fse=raŋ} & 	happen so many things & 	\Y & 	 & 	 & 	 & 	\\	
\ipa{kʰrɯ=jɤβ} & 	be extremely dry & 	\Y & 	 & 	 & 	\Y & 	\\	
\ipa{zdɯɣ=sŋɤl} & 	suffer extremely & 	\Y? & 	 & 	 & 	\Y & 	\\	
\ipa{rga=le} & 	be extremely happy & 	\Y? & 	\Y & 	 & 	 & 	\\	
\ipa{rga=χi} & 	be extremely happy & 	\Y? & 	\Y & 	 & 	 & 	\\	
\midrule
\ipa{spa=rka tu/me} & 	be guilty/innocent & 	\Y & 	 & 	 & 	\Y & 	\\	
\bottomrule
\end{tabular}
\end{table}

There is only one bipartite transitive verb, \ipa{spa=rka}, which is actually even tripartite, since its always occur with an existential verb -- it means `be guilty' when used with the positive existential verb \jpg{tu}{exist} and  `be innocent' with the negative one \jpg{me}{not exist}. Although morphologically transitive (see for instance the presence of stem III \ipa{a} $\rightarrow$ \ipa{e} alternation in \textsc{sg}$\rightarrow$3 non-past forms, in examples \ref{ex:mAspea} to \ref{ex:mAtWspe2}), this verb has an expletive object which cannot be overt. Examples \refb{ex:mAspea} and \refb{ex:mAspea} illustrate the A form, and \refb{ex:mAtWsparkandZi} and \refb{ex:mAtWspe2} the D form. Note than even though the suffixes of the V_1 cannot be realized in D-form, stem III alternation still appears.

\begin{exe}
\ex \label{ex:mAspea}
\gll \ipa{mɤ-spe-a} \ipa{mɤ-rke-a} \ipa{me} \\
\textsc{neg}-be.innocent(1)[III]:\textsc{fact}-\textsc{1sg} \textsc{neg}-be.innocent(2)[III]:\textsc{fact}-\textsc{1sg} not.exist:\textsc{fact} \\
\glt I am innocent.
\end{exe} 


\begin{exe}
\ex \label{ex:mAtWspe1}
\gll \ipa{mɤ-tɯ-spe} \ipa{mɤ-tɯ-rke} \ipa{me} \\
\textsc{neg}-2-be.innocent(1)[III]:\textsc{fact} \textsc{neg}-2-be.innocent(2)[III]:\textsc{fact} not.exist:\textsc{fact} \\
\glt You_{sg} are innocent.
\end{exe} 


\begin{exe}
\ex \label{ex:mAtWsparkandZi}
\gll \ipa{mɤ-tɯ-spa=rka-ndʑi} \ipa{me} \\
\textsc{neg}-2-be.innocent(1)-be.innocent(2):\textsc{fact}-\textsc{du} not.exist:\textsc{fact} \\
\glt You_{du} are innocent.
\end{exe} 


\begin{exe}
\ex \label{ex:mAtWspe2}
\gll \ipa{mɤ-tɯ-spe=rke} \ipa{me} \\
\textsc{neg}-2-be.innocent(1)[III]-be.innocent(2)[III]:\textsc{fact} not.exist:\textsc{fact} \\
\glt You_{sg} are innocent.
\end{exe} 

\section{Compound verbs} \label{sec:}
Apart from bipartite verbs, we find in Japhug another class of verbs combining two verb roots into one lexeme: compound verbs.

Unlike bipartite verbs,  compound verbs do not allow multiple indexation (such as inflectional morphemes intervening between the two roots) and the first member of the compound undergoes \textit{status constructus} vowel change (see  \citealt[1215]{jacques12incorp}, \ipa{a/u} $\rightarrow$ \ipa{ɤ} in examples 1 and 4 below). In addition, an additional derivational prefix is often present (\ipa{a-} and \ipa{rɤ-/a-} in examples 1, 2 and 3 below).

\begin{enumerate}
\item \jpg{pa}{do} + \jpg{mbat}{be easy} $\Rightarrow$ \jpg{a-pɤ-mbat}{easy to do}
\item \jpg{χtɯ}{buy} + \jpg{ntsɣe}{sell} $\Rightarrow$ \jpg{ra-χtɯ-tsɣe}{do commerce (\zh{做买卖})} 
\item \jpg{joʁ}{raise} + \jpg{βzɯr}{move}   $\Rightarrow$  \jpg{rɤ-joʁ-βzɯr}{tidy up}
\item \jpg{ngu}{feed} + \jpg{jtsʰi}{give to drink} $\Rightarrow$ \jpg{ngɤ-jtsʰi}{feed}
\end{enumerate}

The first three examples are actually historically denominal verbs, derived from action noun compounds made of two verb roots, in a way similar to incorporating verbs, which originate from the denominal derivation of noun-verb compounds (\citealt{jacques12incorp}). The noun compound \jpg{joʁ-βzɯr}{tidying up} from which \jpg{rɤ-joʁ-βzɯr}{tidy up} is derived is still attested, and can be used in a light verb construction:

\begin{exe}
\ex 
 \gll \ipa{joʁβzɯr} \ipa{tɤ-βzu-t-a} \\
 tidying.up \textsc{pfv}-do-\textsc{pst:tr-1sg} \\
 \glt `I did some tidying up.'
\end{exe}

The other action nominal *\ipa{pɤ-mbat} `action that is easy to do' and ?\ipa{χtɯ-tsɣe} `commerce' have not been recorded, but it is unproblematic to assume that they used to exist and that the corresponding verbs derive from them.

Note the antipassive-like value of the derivation in examples 2 and 3, where the compound verb derived from two transitive verb is intransitive. Note that the denominal prefix \ipa{rɤ-} is the same from which the antipassive derivation was grammaticalized (see  \citealt{jacques14antipassive}).

Example 4 \jpg{ngɤ-jtsʰi}{feed} differs from the preceding one by having no denominal prefix. Such compound verbs are a minority (only four are attested, see Table \ref{tab:compound.verbs}), but all start in a cluster whose first element is a nasal. It is thus possible that a reduced form of the denominal prefix (as in \jpg{ngo}{be sick} from denominal \ipa{n(ɯ)}+\jpg{--ŋgo}{disease}) has been absorbed by the verb stem.\footnote{For examples of such a development in Khroskyabs, see \citet{jacques12incorp, lai13affixale}. }

\begin{table}[h]
\caption{Compound verbs in Japhug} \label{tab:compound.verbs} 
 \resizebox{\columnwidth}{!}{
\begin{tabular}{lllllllll}
\toprule
Compound verb &&$V_1$ & & $V_2$ \\
\midrule
\ipa{andʑɤmstu} &	well-ironed &	\ipa{ndʑɤm} &	warm &	\ipa{astu} &	straight &	\\	
\ipa{argɤle} &	be extremely happy &	\ipa{rga} &	be happy &	\ipa{le} &	-- &	\\	
\ipa{apɤmbat} &	easy to do &	\ipa{pa} &	do &	\ipa{mbat} &	easy &	\\	
\ipa{arɟumtɕɤr} &	having uneven wideness &	\ipa{rɟum} &	wide &	\ipa{tɕɤr} &	narrow &	\\	
\ipa{nɤrtoχpjɤt} &	observe &	\ipa{rtoʁ} &	look, watch &	\ipa{χpjɤt} &	observe &	\\	
\ipa{nɤscɤlɤt} &	take to somewhere &	\ipa{sco} &	see off &	\ipa{lɤt} &	get so. back home &	\\	
&and back home \\
\ipa{nɤtsɯmɣɯt} &	take away and  &	\ipa{tsɯm} &	take away &	\ipa{ɣɯt} &	bring &	\\	
&bring back\\
\ipa{nɯndzɤmbɣom} &	in a hurry to eat &	\ipa{ndza} &	eat &	\ipa{mbɣom} &	be in a hurry &	\\	
\ipa{nɯndzɤqɤr} &	eat on one's own &	\ipa{ndza} &	eat &	\ipa{qɤr} &	separate &	\\	
\ipa{nɯrkorlɯt} &	be obstinate &	\ipa{rko} &	be hard &	\ipa{arlɯt} &	be many &	\\	
\ipa{nɯrŋgɯmbri} &	make noise in the bed &	\ipa{rŋgɯ} &	lie down &	\ipa{mbri} &	make noise &	\\	
\ipa{rɤjoʁβzɯr} &	tidy up &	\ipa{joʁ} &	raise &	\ipa{βzɯr} &	move &	\\	
\midrule
\ipa{mpɯmnu} &	soft and smooth &	\ipa{mpɯ} &	soft &	\ipa{mnu} &	smooth &	\\	
\ipa{ngɤjtsʰi} &	feed &	\ipa{ngu} &	feed &	\ipa{jtsʰi} &	give to drink &	\\	
\ipa{mbɯjtsʰi} &	go to eat and drink &	\ipa{mbi} &	give &	\ipa{jtsʰi} &	give to drink &	\\	
\ipa{mtsɯrɕpaʁ} &	be hungry and thursty &	\ipa{mtsɯr} &	be hungry &	\ipa{ɕpaʁ} &	be thursty &	\\	
\bottomrule
\end{tabular}}
\end{table}

Most compound verbs in Japhug are \textit{dvandva}-like (`do $V_1$ and $V_2$')  but in a few examples, like \jpg{a-pɤ-mbat}{easy to do} or \jpg{nɯndzɤmbɣom}{be in a hurry to eat}, the semantic relationship between $V_1$  and $V_2$ is the same as that of a main verb and its complement (`$V_2$ to $V_1$').

The verb \jpg{a-rgɤ-le}{be extremely happy} is related to the bipartite verb \ipa{rga=le} mentioned in Table \ref{tab:bipartite}, showing that bipartite verbs too can be converted into compound verb verbs by means of denominal derivation.

\section{Serial verb constructions} \label{sec:serial}
The A form of bipartite verbs is formally similar to serial verb constructions (\citealt{sun12complementation, jacques13harmonization}). In these constructions, both verbs share the same arguments, transitivity, TAM, polarity and associated motion markers. Unlike serial constructions in other languages, various linkers and emphatic markers are possible between the two verbs (examples \ref{ex:totChW2} and \ref{ex:kuWGstuanW}).

Serial verb constructions mainly occur with deideophonic verbs (example \ref{ex:totChW2}) and action deixis verbs (transitive \ref{ex:kuWGstuanW} or intransitive \ref{ex:ki.fsea}).

\begin{exe}
\ex \label{ex:totChW2}
\gll 	\ipa{srɯnmɯ} 	\ipa{nɯ} 	\ipa{to-nɯdrɯβ} 	\ipa{ʑo} 	 	\ipa{to-tɕʰɯ} \\
 râkshasî \textsc{dem}  \textsc{ifr}-repeatedly.gore  \textsc{emph}  \textsc{ifr}-gore \\
 \glt `(The rhinoceros) gored the râkshasî repeatedly and killed her.' 
\end{exe}	

\begin{exe}
\ex \label{ex:kuWGstuanW}
\gll 	
 \ipa{aʑo} 	\ipa{kɯki} 	\ipa{ntsɯ} 	\ipa{kú-wɣ-stu-a-nɯ} 	\ipa{tɕe,} 	\ipa{kú-wɣ-znɯkʰrɯm-a-nɯ} \\
 \textsc{1sg} \textsc{dem:prox} always \textsc{ipfv-inv}-do.like-\textsc{1sg-pl} \textsc{lnk} \textsc{ipfv-inv}-punish-\textsc{1sg-pl} \\
 \glt `They punished me like this.' (Gesar, 278)
\end{exe}	

\begin{exe}
\ex \label{ex:ki.fsea}
\gll \ipa{aʑo} 	\ipa{nɯ} 	\ipa{sŋiɕɤr} 	\ipa{ʑo} 	\ipa{kutɕu} 	\ipa{ki} 	\ipa{fse-a} 	\ipa{ndzur-a} 	\ipa{ntsɯ} 	\ipa{ɲɯ-ra} 	\ipa{tɕe,} \\
\textsc{1sg} \textsc{dem} night.and.day \textsc{emph} here \textsc{dem:prox} be.like:\textsc{fact-1sg} stand:\textsc{fact-1sg} always \textsc{sens}-have.to like \\
\glt `I have to stand like this night and day.' (The divination, 2002, 44)
\end{exe}

It is very likely that bipartite verbs originate from serial verb constructions,\footnote{Note that for instance associated motion prefixes are likely to have originated in a serial verb construction too (see \citealt{jacques13harmonization}).} especially since in the case of \jpg{mu=cɯɣ}{be  terrified}, \jpg{ntsʰɤβ=rlu}{in a hurry}, \jpg{kʰrɯ=jɤβ}{be extremely dry}, \jpg{rga=le}{be extremely happy} and \jpg{rga=χi}{be extremely happy} , the V_2 is clearly ideophonic-like. 

The only other possible construction from which bipartite verb could have originated are finite complements, as in example \ref{ex:tundzxi}.

\begin{exe}
\ex \label{ex:tundzxi}
\gll 
\ipa{tɤjpa} 	\ipa{kɯ-xtɕɯ\tld{}xtɕi} 	\ipa{ka-lɤt} 	\ipa{ri,} 	\ipa{mɯ́j-ʁdɯɣ,} 	\ipa{pɤjkʰu} 	\ipa{tu-ndʐi} 	\ipa{ɲɯ-cʰa} \\
snow \textsc{inf:stat-emph}\tld{}be.small \textsc{pfv}:3$\rightarrow$3'-throw but \textsc{neg:sens}-be.serious still \textsc{ipfv}-melt \textsc{sens}-can \\
\glt `There was a little snow, but it doesn't matter, it can still melt.' (conversation, 2015/12/17)
\end{exe}

However, none of V_2 of bipartite verbs in Japhug are complement taking verbs, so that such hypothesis is highly unlikely. 

\section{A model for proto-Kiranti}
Bipartite verbs in Japhug are a marginal class, still barely grammaticalized from a serial verb construction.  

While some Kiranti language do allow some remnant of variable affix ordering, none present the variability of tbipartite verbs in Japhug as described in section \ref{sec:japhug.bipart}. Thus, The comparatively more recent and less grammaticalized construction found in Japhug may provide a useful model to understand how bipartite verbs in Kiranti came into being.

\citealt[168]{doornenbal09} 
\begin{exe}
\ex
\gll
\ipa{tɨ-man-nin} \ipa{kʰan-nin} \\
2-lose-\textsc{1ns$\leftrightarrow$2} send.away-\textsc{1ns$\leftrightarrow$2} \\
\glt ‘you_s have forgotten us_{pe}’
\end{exe}


\begin{exe}
\ex \label{ex:timankhatda2}
\gll 
\ipa{tɨ-man-kʰat-da} \\
2-\textsc{neg}-go-\textsc{pst} \\
\ex
\gll 
\ipa{man-tɨ-kʰat-da} \\
\textsc{neg}-2-go-\textsc{pst} \\
\ex 
\gll 
\ipa{man-kʰat-tɨ-da} \\
\textsc{neg}-go-2-\textsc{pst} \\
\glt `You did not go.'
\end{exe}

% \begin{exe}
%\ex \label{ex:ityodzyoyi2}
%\gll \ipa{ʔi-tɵ-j-dzɵ-ji}\\
%2/\textsc{inv}-put-\textsc{du.incl}-keep.on-\textsc{du.incl} \\
%\glt You_{du} keep on putting it.
%\end{exe}

\bibliographystyle{unified}
\bibliography{bibliogj}

 \end{document}
 