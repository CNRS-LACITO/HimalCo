\documentclass[oldfontcommands,oneside,a4paper,11pt]{article} 
\usepackage{fontspec}
\usepackage{natbib}
\usepackage{booktabs}
\usepackage{xltxtra} 
\usepackage{polyglossia} 
\usepackage[table]{xcolor}
\usepackage{gb4e} 
\usepackage{multicol}
\usepackage{graphicx}
\usepackage{float}
\usepackage{hyperref} 
\hypersetup{bookmarks=false,bookmarksnumbered,bookmarksopenlevel=5,bookmarksdepth=5,xetex,colorlinks=true,linkcolor=blue,citecolor=blue}
\usepackage[all]{hypcap}
\usepackage{memhfixc}
\usepackage{lscape}
 \usepackage{bbding}
 
%\setmainfont[Mapping=tex-text,Numbers=OldStyle,Ligatures=Common]{Charis SIL} 
\newfontfamily\phon[Mapping=tex-text,Ligatures=Common,Scale=MatchLowercase]{Charis SIL} 
\newcommand{\ipa}[1]{{\phon\textbf{#1}}} 
\newcommand{\grise}[1]{\cellcolor{lightgray}\textbf{#1}}
\newfontfamily\cn[Mapping=tex-text,Ligatures=Common,Scale=MatchUppercase]{SimSun}%pour le chinois
\newcommand{\zh}[1]{{\cn #1}}
\newcommand{\Y}{\Checkmark} 
\newcommand{\N}{} 
 

 \begin{document} 
\title{The life cycle of doubly conjugated verbs in Sino-Tibetan}
\author{Guillaume Jacques}
\maketitle

\section{Kiranti}
Kiranti languages are well-known for having a class of complex predicates combining a lexical verb with an auxiliary, which share some of their person indexation and TAM markers. Example (\ref{ex:ityodzyoyi}) from Khaling shows the verb root |tA| `put'\footnote{See \citealt{jacques12khaling} for an account of stem alternations in Khaling.} combined with the auxiliary |-dzA| `keep on' grammaticalized from the verb |dzA| `eat'. The verb and the auxiliary share the second person prefix \ipa{ʔi-}, but the suffixes remain distinct: the auxiliary receives the full form of the suffix \ipa{-ji}, while the lexical verb has a reduced form \ipa{-j-}. This kind of verb form will henceforth be referred to as a \textit{doubly conjugated verb}.

\begin{exe}
\ex \label{ex:ityodzyoyi}
\gll \ipa{ʔi-tɵ-j-dzɵ-ji}\\
2/\textsc{inv}-put-\textsc{du.incl}-keep.on-\textsc{du.incl} \\
\glt You_{du} keep on putting it.
\end{exe}


Prefix permutability:

\citet{bickel07chintang}, \citet[170-172]{doornenbal09}

\begin{exe}
\ex 
\gll 
\ipa{tɨ-man-kʰat-da} \\
2-\textsc{neg}-go-\textsc{pst} \\
\ex
\gll 
\ipa{man-tɨ-kʰat-da} \\
\textsc{neg}-2-go-\textsc{pst} \\
\ex 
\gll 
\ipa{man-kʰat-tɨ-da} \\
\textsc{neg}-go-2-\textsc{pst} \\
\glt `you did not go'
\end{exe}

\section{Serial verb constructions in Rgyalrongic}

Deideophonic and action deixis verbs (\citealt{sun12complementation}).

Share arguments, TAM and person marking.

\begin{exe}
\ex \label{ex:totChW2}
\gll 	\ipa{srɯnmɯ} 	\ipa{nɯ} 	\ipa{to-nɯdrɯβ} 	\ipa{ʑo} 	 	\ipa{to-tɕʰɯ} \\
 râkshasî \textsc{dem}  \textsc{ifr}-repeatedly.gore  \textsc{emph}  \textsc{ifr}-gore \\
 \glt `(The rhinoceros) gored the râkhsasî repeatedly and killed her.' 
\end{exe}	

\begin{exe}
\ex \label{ex:kuWGstuanW}
\gll 	
 \ipa{aʑo} 	\ipa{kɯki} 	\ipa{ntsɯ} 	\ipa{kú-wɣ-stu-a-nɯ} 	\ipa{tɕe,} 	\ipa{kú-wɣ-znɯkʰrɯm-a-nɯ} \\
 \textsc{1sg} \textsc{dem:prox} always \textsc{ipfv-inv}-do.like-\textsc{1sg-pl} \textsc{lnk} \textsc{ipfv-inv}-punish-\textsc{1sg-pl} \\
 \glt `They punished me like this.' (Gesar, 278)
\end{exe}	

\begin{exe}
\ex \label{ex:ki.fsea}
\gll \ipa{aʑo} 	\ipa{nɯ} 	\ipa{sŋiɕɤr} 	\ipa{ʑo} 	\ipa{kutɕu} 	\ipa{ki} 	\ipa{fse-a} 	\ipa{ndzur-a} 	\ipa{ntsɯ} 	\ipa{ɲɯ-ra} 	\ipa{tɕe,} \\
\textsc{1sg} \textsc{dem} night.and.day \textsc{emph} here \textsc{dem:prox} be.like:\textsc{fact-1sg} stand:\textsc{fact-1sg} always \textsc{sens}-have.to like \\
\glt `I have to stand like this night and day.' (The divination, 2002, 44)
\end{exe}

\section{Doubly conjugated verbs}

There are in Japhug a few complex predicates comprising two verbs roots which must appear together in a fixed order. The most comon such complex predicate, \ipa{-stu -mbat} `try hard, do one's best'


\begin{table}[h]
\caption{Four degrees of integration} \centering \label{tab:four}
\begin{tabular}{lllllll}
\toprule
Type & Example & V_1 suffix & V_2 prefix \\
\midrule
A& \ipa{tɤ-stu-ndʑi} \ipa{tɤ-mbat-ndʑi} &\Y &\Y \\
 &\textsc{imp}-V_1-\textsc{du}  \textsc{imp}-V_2-\textsc{du} \\
B& \ipa{tɤ-stu-tɤ-mbat-ndʑi} &\N  &\Y \\
 &\textsc{imp}-V_1-\textsc{imp}-V_2-\textsc{du} \\
C& \ipa{tɤ-stu-ndʑi-mbat-ndʑi} &\Y  &\N \\
 &\textsc{imp}-V_1-\textsc{du}-V_2-\textsc{du} \\
B& \ipa{tɤ-stu-mbat-ndʑi} &\N  &\N \\
 &\textsc{imp}-V_1-V_2-\textsc{du} \\
\bottomrule
\end{tabular}
\end{table}


\subsection{A triple complex predicate: `be innocent'}

`be innocent', which combines the morphologically transitive roots \ipa{spa} and \ipa{rka} in the negative form with the auxiliary \ipa{me} `not exist', as in \ref{ex:mAspea}.\footnote{The stem \ipa{spa} independently exists as a transitive verb meaning `be able', but \ipa{rka} is not attested outside of this complex predicate.}

\begin{exe}
\ex \label{ex:mAspea}
\gll \ipa{mɤ-spe-a} \ipa{mɤ-rke-a} \ipa{me} \\
\textsc{neg}-be.innocent(1)[III]:\textsc{fact}-\textsc{1sg} \textsc{neg}-be.innocent(2)[III]:\textsc{fact}-\textsc{1sg} not.exist:\textsc{fact} \\
\glt I am innocent.
\end{exe} 

In the second person form however, we find two possible realizations, either (\ref{ex:mAtWspe1}) or (\ref{ex:mAtWspe2}). 

\begin{exe}
\ex \label{ex:mAtWspe1}
\gll \ipa{mɤ-tɯ-spe} \ipa{mɤ-tɯ-rke} \ipa{me} \\
\textsc{neg}-2-be.innocent(1)[III]:\textsc{fact} \textsc{neg}-2-be.innocent(2)[III]:\textsc{fact} not.exist:\textsc{fact} \\
\glt You_{sg} are innocent.
\end{exe} 

\begin{exe}
\ex \label{ex:mAtWspe2}
\gll \ipa{mɤ-tɯ-spe-rke} \ipa{me} \\
\textsc{neg}-2-be.innocent(1)[III]-be.innocent(2)[III]:\textsc{fact} not.exist:\textsc{fact} \\
\glt You_{sg} are innocent.
\end{exe} 

\section{Conclusion}
Doubly conjugated verbs may have been an intermediate stage between a serial verb construction and full bleaching of the auxiliary, for instance in the case of the associated motion verb construction in Rgyalrong languages (\citealt{jacques13harmonization}). 

\bibliographystyle{unified}
\bibliography{bibliogj}

 \end{document}
 