\documentclass[oldfontcommands,oneside,a4paper,11pt]{article} 
\usepackage{fontspec}
\usepackage{natbib}
\usepackage{booktabs}
\usepackage{xltxtra} 
\usepackage{polyglossia} 
\usepackage[table]{xcolor}
\usepackage{gb4e} 
\usepackage{multicol}
\usepackage{graphicx}
\usepackage{float}
\usepackage{hyperref} 
\hypersetup{bookmarks=false,bookmarksnumbered,bookmarksopenlevel=5,bookmarksdepth=5,xetex,colorlinks=true,linkcolor=blue,citecolor=blue}
\usepackage[all]{hypcap}
\usepackage{memhfixc}
\usepackage{lscape}
\bibpunct[: ]{(}{)}{,}{a}{}{,}
 
%\setmainfont[Mapping=tex-text,Numbers=OldStyle,Ligatures=Common]{Charis SIL} 
\newfontfamily\phon[Mapping=tex-text,Ligatures=Common,Scale=MatchLowercase,FakeSlant=0.3]{Charis SIL} 
\newcommand{\ipa}[1]{{\phon \mbox{#1}}} %API tjs en italique
 \newcommand{\ipab}[1]{{\phon \mbox{#1}}} %API tjs en italique
\newcommand{\grise}[1]{\cellcolor{lightgray}\textbf{#1}}
\newfontfamily\cn[Mapping=tex-text,Ligatures=Common,Scale=MatchUppercase]{MingLiU}%pour le chinois
\newcommand{\zh}[1]{{\cn #1}}

 

 \begin{document} 
\title{The life cycle of doubly conjugated verbs in Sino-Tibetan}
%\author{Guillaume Jacques}
\maketitle


Kiranti languages are well-known for having a class of complex predicates combining a lexical verb with an auxiliary, which share some of their person indexation and TAM markers. Example (\ref{ex:ityodzyoyi}) from Khaling shows the verb root |tA| `put'\footnote{See \citealt{jacques12khaling} for an account of stem alternations in Khaling.} combined with the auxiliary |-dzA| `keep on' grammaticalized from the verb |dzA| `eat'. The verb and the auxiliary share the second person prefix \ipa{ʔi-}, but the suffixes remain distinct: the auxiliary receives the full form of the suffix \ipa{-ji}, while the lexical verb has a reduced form \ipa{-j-}. This kind of verb form will henceforth be referred to as a \textit{doubly conjugated verb}.

\begin{exe}
\ex \label{ex:ityodzyoyi}
\gll \ipa{ʔi-tɵ-j-dzɵ-ji}\\
2/\textsc{inv}-put-\textsc{du.incl}-keep.on-\textsc{du.incl} \\
\glt You_{du} keep on putting it.
\end{exe}

Doubly conjugated verbs may have been an intermediate stage between a serial verb construction and full bleaching of the auxiliary, for instance in the case of the associated motion verb construction in Rgyalrong languages (\citealt{jacques13harmonization}). However, we also find in Rgyalrong a nascent complex predicate construction that offers a model of how doubly conjugated verbs may come into being.

There are in Japhug a few complex predicates with two verbs roots which must appear together in a fixed order, and between which no conjuction can be inserted, as `be innocent', which combines the morphologically transitive roots \ipa{spa} and \ipa{rka} in the negative form with the auxiliary \ipa{me} `not exist', as in \ref{ex:mAspea}.\footnote{The stem \ipa{spa} independently exists as a transitive verb meaning `be able', but \ipa{rka} is not attested outside of this complex predicate.}

\begin{exe}
\ex \label{ex:mAspea}
\gll \ipa{mɤ-spe-a} \ipa{mɤ-rke-a} \ipa{me} \\
\textsc{neg}-be.innocent(1)[III]:\textsc{fact}-\textsc{1sg} \textsc{neg}-be.innocent(2)[III]:\textsc{fact}-\textsc{1sg} not.exist:\textsc{fact} \\
\glt I am innocent.
\end{exe} 

In the second person form however, we find two possible realizations, either (\ref{ex:mAtWspe1}) or (\ref{ex:mAtWspe2}). 

\begin{exe}
\ex \label{ex:mAtWspe1}
\gll \ipa{mɤ-tɯ-spe} \ipa{mɤ-tɯ-rke} \ipa{me} \\
\textsc{neg}-2-be.innocent(1)[III]:\textsc{fact} \textsc{neg}-2-be.innocent(2)[III]:\textsc{fact} not.exist:\textsc{fact} \\
\glt You_{sg} are innocent.
\end{exe} 

\begin{exe}
\ex \label{ex:mAtWspe2}
\gll \ipa{mɤ-tɯ-spe-rke} \ipa{me} \\
\textsc{neg}-2-be.innocent(1)[III]-be.innocent(2)[III]:\textsc{fact} not.exist:\textsc{fact} \\
\glt You_{sg} are innocent.
\end{exe} 

Examples such as (\ref{ex:mAtWspe2}) show that the verb-verb complex predicates in Japhug are in the process of becoming doubly conjugated verbs like those observed in Kiranti; the two verb roots share all their prefixes, and each still keeps its own stem alternation, as observed in Kiranti languages.

The study of this marginal and little investigated phenomenon (the only references discussing such complex predicates in Rgyalrong languages are \citealt{sun12complementation} and  \citealt{jacques13harmonization}) can shed light on the genesis of doubly conjugated verbs in Kiranti and beyond (examples from Siouan will be discussed). 

\bibliographystyle{unified}
\bibliography{bibliogj}

 \end{document}
 