\documentclass[oldfontcommands,oneside,a4paper,11pt]{article} 
\usepackage{fontspec}
\usepackage{natbib}
\usepackage{booktabs}
\usepackage{xltxtra} 
\usepackage{polyglossia} 
\usepackage[table]{xcolor}
\usepackage{gb4e} 
\usepackage{multicol}
\usepackage{graphicx}
\usepackage{float}
\usepackage{hyperref} 
\hypersetup{bookmarks=false,bookmarksnumbered,bookmarksopenlevel=5,bookmarksdepth=5,xetex,colorlinks=true,linkcolor=blue,citecolor=blue}
\usepackage[all]{hypcap}
\usepackage{memhfixc}
\usepackage{lscape}

\bibpunct[: ]{(}{)}{,}{a}{}{,}

%\setmainfont[Mapping=tex-text,Numbers=OldStyle,Ligatures=Common]{Charis SIL} 
\newfontfamily\phon[Mapping=tex-text,Ligatures=Common,Scale=MatchLowercase,FakeSlant=0.3]{Charis SIL} 
\newcommand{\ipa}[1]{{\phon \mbox{#1}}} %API tjs en italique
\newcommand{\ipab}[1]{{\scriptsize \phon#1}} 

\newcommand{\grise}[1]{\cellcolor{lightgray}\textbf{#1}}
\newfontfamily\cn[Mapping=tex-text,Ligatures=Common,Scale=MatchUppercase]{MingLiU}%pour le chinois
\newcommand{\zh}[1]{{\cn #1}}
\newcommand{\refb}[1]{(\ref{#1})}
\newcommand{\factual}[1]{\textsc{:fact}}
\newcommand{\rdp}{\textasciitilde{}}

\XeTeXlinebreaklocale 'zh' %使用中文换行
\XeTeXlinebreakskip = 0pt plus 1pt %
 %CIRCG
 \newcommand{\bleu}[1]{{\color{blue}#1}}
\newcommand{\rouge}[1]{{\color{red}#1}} 
\newcommand{\ch}[3]{\zh{#1} \ipa{#2} `#3'} 

\begin{document} 
\title{How many *-s suffixes in Old Chinese? }
\author{Guillaume Jacques}
\maketitle
 


\section{Introduction}
 \citet[262]{downer59}:
`The present writer holds the opinion that with our present knowledge of Classical Chinese, it is better to regard chiuhsheng derivation not as a remnant of a former inflectional system of the Indo-European type, but simply as a system of derivation and nothing more. When new words were needed, they were created by pronouncing the basic word in the chiuhsheng. The grammatical regularity found in many cases would then be in a way fortuitous, being the result not of grammatical inflection, but of the need to create new words.'

\textbf{Differing opinion}:

Qusheng derivation comes (at least in part) from *\ipa{--s} suffixes (\citealt{haudricourt54chinois}, \citealt{forrest60occlusives}, \citealt{sagart99roc, bs14oc}). The vast array of function of the *-s is due to the merger of many independent dental suffixes in Chinese, which are however distinct in more conservative languages.



\section{Old Chinese}
\citet{downer59}'s classification of qusheng derivations:

\begin{enumerate}
\item verb $\rightarrow$ noun 


\ch{高}{kaw}{be high} $\rightarrow$ \ch{高}{kaw^H}{height} 

\ch{處}{tɕʰjo^X}{be at} $\rightarrow$ \ch{處}{tɕʰjo^H}{place} 

\item noun $\rightarrow$ verb 

\ch{家}{kæ}{family} $\rightarrow$ \ch{嫁}{kæ^H}{marry} 

\ch{枕}{tɕim^X}{pillow} $\rightarrow$ \ch{枕}{tɕim^H}{use as a pillow} 

\item causative

\ch{飲}{ʔim^X}{drink} $\rightarrow$ \ch{飲}{ʔim^H}{give to drink} 

\ch{買}{mɛ^X}{buy} $\rightarrow$ \ch{賣}{mɛ^H}{sell} 


+ tropative (\ch{好}{xaw^X}{be good} $\rightarrow$ \ch{好}{xaw^H}{like} (distinct from causative and applicative, \citealt{jacques13tropative})
\item applicative (Downer's ``effective")

\ch{渴}{kʰat}{be thirsty} $\rightarrow$ \ch{愒}{kʰaj^H}{long for} 

\item `restricted meaning'
\ch{少}{ɕjew^X}{be few} $\rightarrow$ \ch{少}{ɕjew^H}{be young} 

\item Passive
\ch{散}{san^X}{scatter} $\rightarrow$ \ch{散}{san^H}{be loose} 

+ antipassive \ch{射}{ʑjek}{shoot at} $\rightarrow$ \ch{射}{ʑjæ^H}{practise archery} 

\item Adverb
\ch{三}{sam}{three} $\rightarrow$ \ch{三}{sam^H}{thrice} 

\item Form in compounds
\end{enumerate}

\section{Comparisons}
Proposed sound law (C stands for a particular subset of codas, perhaps restricted to obstruents).

\begin{exe}
\ex \label{ex:ts}
\glt *\ipa{--t} $\rightarrow$ *\ipa{--s} /C\_\# 
\end{exe}

Similar to the rule undergone by the present tense \ipa{--d} suffix in Tibetan, which is realized as \ipa{--s} after final stops and \ipa{-m}, as in \ipa{ɴdeb-s, btab}   `plant' vs \ipa{ɴde-d, bdas}   `chase' (\citealt[52-53]{coblin76}). The *\ipa{--s} allomorph of *\ipa{--t} suffixes generated by this rule is then extended analogically to other contexts, in particular open syllables (this idea in particular is necessary to explain the *\ipa{--s} applicatives, see section \ref{sec:causative}).


Very few languages in ST clearly preserve final \ipa{--s} suffixes, and complex final clusters: Tibetan, Kiranti, Gyalrongic, Dulong/Rawang (other languages can only serve to confirm data from these three conservative branches). 

\subsection{Nominalization}
The most obvious parallel with Tibetan, first pointed out by \citet{forrest60occlusives}.

Well attested in Tibetan, \citet[43]{conrady1896}, \citet[624-5]{hill14derivational} (\ipa{ɴgro} `go'  $\rightarrow$ \ipa{ɴgros} `gait' etc). Some nouns derived from verbs are cognate between Chinese and Tibetan, eg \ipa{ɴtʰag, btags} ``weave" $\rightarrow$ \ipa{tʰags} ``textile", \ch{織}{tɕik}{weave} $\rightarrow$ \ch{織}{tɕi^H}{cloth} 

Vestigial in Gyalrongic languages, but several good examples cf \citet{jacques03s.houzhui}

 

\subsection{Causative/applicative} \label{sec:causative}
The applicative and causative suffixes are most clearly preserved in Kiranti, \citet{michailovsky85dental, jacques15derivational.khaling}

Distinct \ipa{-s} causative and \ipa{--t} applicative suffixes in Limbu, \citealt{michailovsky02dico}:
\begin{itemize}
\item \ipa{haːp-} \textit{vi} `weep'
\item \ipa{haːps-} \textit{vt} `cause to weep' (causative)
\item \ipa{haːpt-} \textit{vt} `mourn for, weep for' (applicative)
\end{itemize}

No clear example in Tibetan \citet[630]{hill14derivational}, only a few isolated examples in Gyalrongic (Japhug \ipa{ɣi} `come' $\rightarrow$ \ipa{ɣɯt} `bring'). 

In Chinese, one direct trace of the applicative *\ipa{--t} (\citealt{sagart04directions}), the other examples are due to the generalisation of the *\ipa{--s} allomorph (generated by the sound change \ref{ex:ts}) to open syllable verbs. 

\begin{exe}
\ex 
\gll \zh{行} \zh{或} \zh{使} \zh{之}, \zh{止} \zh{或} \zh{尼} \zh{之} \\
 \ipa{Cə.ɡˁ<r>aŋ}  \ipa{ɢʷˁək}  \ipa{s-rəʔ}  \ipa{tə}  \ipa{təʔ}  \ipa{ɢʷˁək}  \ipa{n<r>əl-t}  \ipa{tə} \\
\glt `A man's advancement is effected, it may be, by others, and the stopping him is, it may be, from the efforts of others.' (Legge)
\end{exe}


Fanqie \zh{女乙反}, MC \ipa{ɳit} vs \zh{女夷切} MC \ipa{ɳij} $\leftarrow$ *\ipa{n<r>ɨl}, cognate with Tibetan \ipa{ɲal} `lie down, sleep' and Japhug \ipa{nɯna} `rest' (NB: loss of final *\ipa{-l} after *\ipa{a} is a common innovation of Burmo-Qiangic languages). 

This example shows that the rule (\ref{ex:ts}) does not apply after *\ipa{--l}; cases of *\ipa{--s} in open syllables and after *\ipa{--l} corresponding to \ipa{--t} in conservative languages are due to analogical levelling. Analogical levelling of the *\ipa{--s} allomorph nearly everywhere in Chinese is likely to be due to the presence of the causative *\ipa{--s}: after application of (\ref{ex:ts}), the *\ipa{--s} was reinterpreted as a mixed causative/applicative.

\subsection{Passive/Antipassive}
Apparent cases of passive or antipassive *\ipa{--si} can be accounted for as a trace of the sibilant reflexive suffix, still productive in Kiranti and Dulong/Rawang.

In Khaling, both passive/anticausative and antipassive meanings are attested for \ipa{--si} derivation (\citealt{jacques16si}).
\subsubsection{Impersonal subject vs reflexive} 

\begin{exe}
\ex \label{ex:wendu} 
\gll 
\ipa{ʔuŋʌ}  	\ipa{sɵ}  	\ipa{wēnd-u.}  \\
\textsc{1sg:erg} meat cut-\textsc{1sg$\rightarrow$3} \\
\glt I cut the meat.
\end{exe}

\begin{exe}
\ex \label{ex:weisi} 
\gll 
 \ipa{sɵ}  	\ipa{wêi-si.}  \\
 meat cut-\textsc{refl} \\
\glt \textit{Impersonal subject}: `The meat is cut (by someone)' OR `The meat cuts easily.'
\end{exe}

\begin{exe}
\ex \label{ex:weisi2} 
\gll 
 \ipa{ʔʌ̄m} \ipa{sɵ}  	\ipa{wêi-si.}  \\
\textsc{3sg} meat cut-\textsc{refl} \\
\glt \textit{Autobenefactive}: `He cuts meat for himself.'
\end{exe}
 
\begin{exe}
\ex \label{ex:weiwasu} 
\gll 
	\ipa{mu-wei-w-ʌsu.}  \\
\textsc{neg}-cut-\textsc{irr}-\textsc{refl:1sg:pst} \\
\glt \textit{Reflexive}: `I did not cut myself.'
\end{exe}  
\subsubsection{Antipassive} 
\begin{exe}
\ex \label{ex:ghryamtsi} 
\gll \ipa{gʰrɛ̄m-si-ŋʌ}\\
 be.disgusted.by-\textsc{refl-1sg:S/P} \\
\glt  I feel disgust.
\end{exe}

\begin{exe}
\ex \label{ex:ghryamt} 
\gll 
  	\ipa{lokpei}  	\ipa{ghrɛ̄md-u.}  \\
leech  be.disgusted.by-\textsc{1sg$\rightarrow$3} \\
 \glt  I am disgusted by leeches.
\end{exe}

\subsection{Denominal}
Denominal morphology in Gyalrongic is exclusively prefixal (\citealt{jacques14antipassive}), examples of \ipa{--t} or \ipa{--s} suffixes in morphologically conservative languages are very rare:

Limbu \ipa{thiːn} `egg' $\rightarrow$ \ipa{thiːnt-} `lay an egg'

The origin of the denominal \ipa{--s} obscure, maybe some cases are in fact causative or applicatives applied to a intransitive verb that was derived from the noun by either zero-derivation or a prefixal derivation process that left no visible traces.

\subsection{Adverb}

 
Locative Situ \ipa{-s} (\citealt{linxr93jiarong}), Japhug \ipa{zɯ} (\citealt[167-9]{jacques08zh}), \ipa{-s} element in Tibetan case markers (\citealt{hill12bas}).

Oblique case $\rightarrow$ adverbial marker


\subsection{Second member of compounds}


\citet{uebach08rjeblas}, \ipa{--s} suffix in the second member of compounds in Old Tibetan, eg \ipa{lag} + \ipa{riŋ} $\rightarrow$ \ipa{lag.riŋs} `long arms'.

 However, in OC, examples of this type are probably cases of nominalized verbs.
 
\subsection{Perfective?}
Some authors (in particular \citealt{jinlx06}) have claimed that some examples of *\ipa{--s} suffixes can be accounted for as traces of the *\ipa{--s} perfective suffix found in Tibetan (\ipa{bʲed, bʲas} `do') or in Gyalrong languages (Japhug \ipa{--t} or \ipa{--z} \textsc{1/2sg}$\rightarrow$3 perfective suffix, Situ third person intransitive perfective \ipa{--s}; see \citealt{huangbf97s.houzhui} for other potential examples, though many in that article come from languages that do not preserve final *\ipa{--s} and hence canot be cognate to the Tibetan or Gyalrong suffix).

Tantalizing, but hardly compelling: if the OC verb had inflexional morphology, it is not reflected in the Jingdian shiwen glosses in a way that allows systematic investigation.

\section{Conclusion}

\bibliographystyle{unified}
\bibliography{bibliogj}

\end{document}