\documentclass[oneside,a4paper,11pt]{article} 
\usepackage{fontspec}
\usepackage{natbib}
\usepackage{booktabs}
\usepackage{xltxtra} 
\usepackage{polyglossia} 
\usepackage[table]{xcolor}
\usepackage{gb4e} 
\usepackage{multicol}
\usepackage{graphicx}
\usepackage{float}
\usepackage{hyperref} 
\hypersetup{bookmarksnumbered,bookmarksopenlevel=5,bookmarksdepth=5,colorlinks=true,linkcolor=blue,citecolor=blue}
\usepackage[all]{hypcap}
\usepackage{memhfixc}
\usepackage{lscape}

%\bibpunct[: ]{(}{)}{,}{a}{}{,}

%\setmainfont[Mapping=tex-text,Numbers=OldStyle,Ligatures=Common]{Charis SIL} 
\newfontfamily\phon[Mapping=tex-text,Ligatures=Common,Scale=MatchLowercase]{Charis SIL} 
\newcommand{\ipa}[1]{{\phon\textbf{\mbox{#1}}}} 
\newcommand{\ipab}[1]{{\scriptsize \phon#1}} 

\newcommand{\grise}[1]{\cellcolor{lightgray}\textbf{#1}}
\newfontfamily\cn[Mapping=tex-text,Ligatures=Common,Scale=MatchUppercase]{SimSun}
\newcommand{\zh}[1]{{\cn#1}}
\newcommand{\refb}[1]{(\ref{#1})}
\newcommand{\factual}[1]{\textsc{:fact}}
\newcommand{\rdp}{\textasciitilde{}}

\XeTeXlinebreaklocale 'zh' %使用中文换行
\XeTeXlinebreakskip = 0pt plus 1pt %
 %CIRCG
 \newcommand{\bleu}[1]{{\color{blue}#1}}
\newcommand{\rouge}[1]{{\color{red}#1}} 
\newcommand{\ch}[3]{\zh{#1} \ipa{#2} `#3'} 
\newcommand{\dhatu}[2]{|\ipa{#1}| `#2'}
\newcommand{\tib}[2]{\ipa{#1} `#2'}

\begin{document} 
\title{How many *-s suffixes in Old Chinese?\footnote{Acknowledgments will be added after editorial decision. This paper is the revised version of a talk presented at the conference `Recent Advances in Old Chinese Historical Phonology'. I gratefully acknowledge the grant `Beyond Boundaries: Religion, Region, Language and the State' by the ERC. The Japhug examples are taken from a corpus that is progressively being made available on the Pangloss archive (\citealt{michailovsky14pangloss}).} } %Nathan W. Hill, Laurent Sagart, Scott DeLancey, Bettina Zeisler
\author{Guillaume Jacques}
\maketitle
 
\section{Introduction}
While qusheng \zh{去聲} derivation is one of the most prominent trace of morphology in Old Chinese, it is probably also the least understood one, as it presents diverse and even contradictory functions, to the extent that \citet[262]{downer59}, in his seminal article, argued that it was simply a way of creating new words, not a derivation with a well-defined grammatical function.\footnote{`The present writer holds the opinion that with our present knowledge of Classical Chinese, it is better to regard chiuhsheng derivation not as a remnant of a former inflectional system of the Indo-European type, but simply as a system of derivation and nothing more. When new words were needed, they were created by pronouncing the basic word in the chiuhsheng. The grammatical regularity found in many cases would then be in a way fortuitous, being the result not of grammatical inflection, but of the need to create new words.'}
 
Yet, we know thanks to the work of scholars such as \citet{haudricourt54chinois}, \citet{forrest60occlusives, schuessler85qusheng} and \citet{sagart99roc} that qusheng derivation comes (at least in part) from *\ipa{--s} suffixes. Since \ipa{-s} suffixes with functions similar to those which have been reconstructed for Old Chinese are attested and even are still productive in more conservative languages of the Trans-Himalayan family, it is worthwhile to explore the exact opposite hypothesis to Downer's ultrascepticism, namely that the vast array of functions of the *-s is due to the merger of many independent dental suffixes, and constitute indeed obscured traces of a former inflectional system.

In this paper, I first present Downer's work and adapt it to modern-day terminology. Second, I propose a new sound law for pre-Old Chinese. Third, drawing on first-hand data from Japhug (Gyalrongic), Khaling (Kiranti) as well as Tibetan, I show that seven out of Downer's eight functions of the qusheng derivation have potential comparanda in conservative languages of the family.

Old Chinese is given in Middle Chinese (in an IPA adaptation of Baxter's \citeyear{baxter92} system) rather than OC, because all modern old Chinese reconstruction models agree on *\ipa{-s} as the origin of the qusheng, and the discussion in this paper is therefore independent of any particular system (except for the reconstruction of an applicative *\ipa{-t} suffix, see section \ref{sec:causative}).

\section{Downer's eight functions}
Despite his suggestion that qusheng derivation was basically random, \citet{downer59} provides a very useful classification of attested derivations, which I take as basic data for this study. It is possible that other functions for the qusheng can be found which are not included in Downer's work, but I leave the examination of those to other scholars.

He counts the following eight categories, here illustrated by one or two representative examples.

\begin{enumerate}
\item \textbf{Nominalization}

The qusheng can be used derive a noun of property out of an stative adjectival verb, and a patient noun out of a transitive verb.

\ch{高}{kaw}{be high} $\rightarrow$ \ch{高}{kaw^H}{height} 

\ch{處}{tɕʰjo^X}{be at} $\rightarrow$ \ch{處}{tɕʰjo^H}{place} 

\item \textbf{Verbalization} 

The semantics of verbs derived from nouns with the Qusheng is very varied, but includes in particular the meaning `use X as ...'

\ch{家}{kæ}{family} $\rightarrow$ \ch{嫁}{kæ^H}{marry} 

\ch{枕}{tɕim^X}{pillow} $\rightarrow$ \ch{枕}{tɕim^H}{use as a pillow} 

\item \textbf{Causative}

\ch{飲}{ʔim^X}{drink} $\rightarrow$ \ch{飲}{ʔim^H}{give to drink} 

\ch{買}{mɛ^X}{buy} $\rightarrow$ \ch{賣}{mɛ^H}{sell} 

In addition to this causative function, the qusheng is also used with a tropative meaning `consider to be X' (\citealt{jacques13tropative}), as the famous case of \ch{好}{xaw^X}{be good} $\rightarrow$ \ch{好}{xaw^H}{like}.

\item \textbf{Applicative} 

What Downer meant by `effective' corresponds to what typologists now call `applicative', namely a patient-adding derivation which preserves the syntactic status of the S/A subject. The examples found in Old Chinese typically involve experiencer verbs, whose applicative form adds a stimulus.

\ch{渴}{kʰat}{be thirsty} $\rightarrow$ \ch{愒}{kʰaj^H}{long for} 

\item \textbf{`Restricted meaning'}
This category appears a catch-all for cases of qusheng derivations not otherwise classifiable into any of the well-understood categories. It includes examples such as:

\ch{少}{ɕjew^X}{be few} $\rightarrow$ \ch{少}{ɕjew^H}{be young} 

Yet, some of the pairs in this category are better reanalysed as particular cases of other categories; I will show in section \ref{sec:passive} that at least some examples may have the same origin as passive and antipassive values of the qusheng derivation.

\item \textbf{Passive}

Qusheng derivation is used for argument demoting derivations. Examples of passive (A-demoting) derivations are found:

\ch{散}{san^X}{scatter} $\rightarrow$ \ch{散}{san^H}{be loose} 

In addition, although Downer does not set up a special category, some of his examples rather attest an antipassive (P-demoting) meaning, as the following:

\ch{射}{ʑjek}{shoot at} $\rightarrow$ \ch{射}{ʑjæ^H}{practise archery} 

\item \textbf{Adverbialization}

A few restricted examples appear to be interpretation as derivation from verbs or numerals to adverbs, as:

\ch{三}{sam}{three} $\rightarrow$ \ch{三}{sam^H}{thrice} 

\item \textbf{Form in compounds}

Downer notes many case of qusheng derivation in the first or second element of compounds; most of his examples however appear to be interpretable as special instances of denominal or deverbal derivations.
\end{enumerate}

\section{New sound law}
A few languages of the Trans-Himalayan family, most notably Kiranti and West Himalayish, have a contrast between final clusters -Cs and -Ct in verb roots (\citealt{michailovsky85dental}).\footnote{These clusters are not word-final in these languages, as they always surface before inflectional suffixes.} 

In Old Chinese, no *-Ct final clusters have ever been proposed. Yet, the final *-s as reconstructed in all modern systems is much too common in comparison with languages of the family which allow final clusters. In Tibetan for instance, there are 7398 words ending in -s in  \citet{bodrgya} out of 53922 (14\%; this figure includes many -s originating from *\ipa{-t} after grave codas). By contrast, in Baxter and Sagart's (\citeyear{bs14oc}) online list of reconstructions, we find 1160 words with final -s out of 4968 (23\%). This raises the possibility that final *-s is not the only origin of the qusheng.

To account for the absence of *-Ct clusters and for the excess of final *-s in present reconstruction models, I propose the following sound law (C stands for a particular subset of codas, perhaps restricted to labial and velar consonants):\footnote{A similar idea was suggested by \citet[42]{schuessler07}.}

\begin{exe}
\ex \label{ex:ts}
\glt *\ipa{--t} $\rightarrow$ *\ipa{--s} /C\_\# 
\end{exe}

This sound change is similar to the rule undergone by the present tense \ipa{--d} suffix in Tibetan, which is realized as \ipa{--s} after final stops and \ipa{-m}, as in \ipa{ɴdeb-s, btab}   `plant' vs \ipa{ɴde-d, bdas}   `chase' (\citealt[52-53]{coblin76}). 

This hypothesis allows comparison of Chinese qusheng derivations with not only \ipa{-s} suffixes, but also \ipa{-t} suffixes, at least following grave final consonants. In addition, in the following comparisons, I assume that the *\ipa{--s} allomorph of *\ipa{--t} suffixes generated by this rule is then extended analogically to other contexts, in particular open syllables (this idea in particular is necessary to explain the *\ipa{--s} applicatives, see section \ref{sec:causative}).


\section{Comparisons}

Very few languages in ST clearly preserve final \ipa{--s} suffixes, and complex final clusters: only data from Tibetan, Kiranti, Gyalrongic, West-Himalayish and Dulong/Rawang can be used without reconstruction, and this paper therefore focuses on these languages.\footnote{Phonologically less conservative languages may provide crucial confirmatory evidence, but at the present stage it would be premature to investigate languages where final \ipa{-s} is only accessible through the comparative method. }

Seven of Downer's eight functions of the qusheng are discussed here (the causative and applicative derivations are discussed in the same subsection), and in addition the suggestion of a perfective value for the qusheng in Chinese and its possible cognates is briefly touched upon.

\subsection{Nominalization} \label{sec:nmlz}
As early as \citet{forrest60occlusives}, scholars have noted that the nominalization function of the qusheng derivation could be compared with the \ipa{-s} nominalization found in Tibetan (on which see \citealt[43]{conrady1896}, \citealt[624-5]{hill14derivational}). The \ipa{-s} derives patient nouns, nouns of manner and nouns of characteristic, as illustrated by the following examples. 

\begin{itemize}
\item Patient: \ipa{za} `eat' $\rightarrow$ \ipa{zas} `food' 
\item Characteristic: \ipa{zab.mo} `deep'  $\rightarrow$ \ipa{zabs} `depth'.
\item Manner: \ipa{ɴgro} `go'  $\rightarrow$ \ipa{ɴgros} `gait' 
 \end{itemize}
 
Incidentally, the three same categories are attested in the West-Himalayish language Bunan (\citealt[179-180]{widmer14bunan}).

The similarity with Old Chinese, where qusheng derivation is used exactly with the first two meanings, is striking (note for instance \ch{深}{ɕim}{deep} $\rightarrow$ \ch{深}{ɕim^H}{depth}).

Moreover, some nouns derived from verbs are cognate between Chinese and Tibetan, eg \ipa{ɴtʰag, btags} ``weave" $\rightarrow$ \ipa{tʰags} ``textile", \ch{織}{tɕik}{weave} $\rightarrow$ \ch{織}{tɕi^H}{cloth}.

Nominalization by \ipa{-s} suffixation was already only very marginally productive in Old Tibetan. The only clear case of a neologism formed in historical times using the \ipa{-s} suffix is \ipa{gzuŋs} `dhâraṇî' (a type of Buddhist magical formula). This word is derived from the root \ipa{|zuŋ|} of \ipa{bzuŋ} `seize', by calque of `dhâraṇî-', which derives from the root dhṛ- `seize, hold' in Sanskrit. It actually uses a circumfix \ipa{g-...-s}, the prefixal element of which has cognates in Gyalrongic languages and beyond (\citealt{konnerth16gV, jacques14snom}).

Outside of Tibetan and West-Himalayish, we find vestigial traces of *\ipa{-s} nominalization in Gyalrongic languages (\citealt{jacques03s.houzhui, jackson14morpho}.

 \subsection{Causative/Applicative} \label{sec:causative}
The applicative and causative suffixes are most clearly preserved in Kiranti (\citealt{michailovsky85dental, jacques15derivational.khaling}), though isolated traces can be found in most branches of the family.

Limbu stands out among Kiranti languages with clearly distinct \ipa{-s} causative and \ipa{--t} applicative suffixes, as illustrated by the following triplet (\citealt{michailovsky02dico}):
\begin{itemize}
\item \ipa{haːp-} \textit{vi} `weep'
\item \ipa{haːps-} \textit{vt} `cause to weep' (causative)
\item \ipa{haːpt-} \textit{vt} `mourn for, weep for' (applicative)
\end{itemize}

No clear example of causative or applicative \ipa{-d} or \ipa{-s} are found in Tibetan (\citealt[630]{hill14derivational}). In Gyalrongic, only isolated examples are found, mainly with motion verbs. For instance Japhug has \ipa{ɣi} < *\ipa{wi} `come' $\rightarrow$ \ipa{ɣɯt} <  *\ipa{wit} `bring'; note however that despite  the dearth of examples, this pair actually has cognates in Kiranti (Khaling \ipa{|pi|} `come' $\rightarrow$ \ipa{|pit|} `bring'), a fact which strongly supports the idea that this is an archaic feature, reconstructible to the common ancestor of these languages.

In Chinese, \citet{sagart04directions} discovered one direct trace of the applicative *\ipa{-t}  in the following example:\footnote{I reconstruct *\ipa{-l} where \citet{bs14oc} have *\ipa{-r} in Old Chinese, as (1) this is a way to avoid syllables with medial *\ipa{-r-} and final *\ipa{-r}, a structure disallowed in monomorphemic words in both Tibetic and Gyalrongic languages (\citealt{jacques04these}) and (2) this coda corresponds to \ipa{-l}  in Tibetan in many examples, while correspondences with \ipa{-r} are few and less convincing. (\citealt[101-2]{hill14jrn}).  } 

 

\begin{exe}
\ex 
\gll \zh{行} \zh{或} \zh{使} \zh{之}, \zh{止} \zh{或} \zh{尼} \zh{之} \\
 \ipa{Cə.ɡˁ<r>aŋ}  \ipa{ɢʷˁək}  \ipa{s-rəʔ}  \ipa{tə}  \ipa{təʔ}  \ipa{ɢʷˁək}  \ipa{n<r>əl-t}  \ipa{tə} \\
\glt `A man's advancement is effected, it may be, by others, and the stopping him is, it may be, from the efforts of others.' (Legge)
\end{exe}

The character \zh{尼} is read here with the fanqie \zh{女乙反}, corresponding to a Middle Chinese pronunciation \ipa{ɳit}. This reading can be opposed to the usual pronunciation of this character \ipa{ɳij} $\leftarrow$ *\ipa{n<r>ɨl}  (fanqie \zh{女夷切}), cognate with Tibetan \ipa{ɲal} `lie down, sleep' and Japhug \ipa{nɯna} `rest',\footnote{Loss of final *\ipa{-l} after *\ipa{a} is a common innovation of Burmo-Gyalrongic languages (\citealt{jacques.michaud11naish}).} and suggests a reconstruction *\ipa{n<r>əl-t} in Old Chinese. This example shows that rule (\ref{ex:ts}) does not apply after *\ipa{--l}.

 Yet, examples of  applicative qusheng are found in syllables with final *\ipa{-l} or even open syllables in Old Chinese. These examples can be accounted for by supposing that rule (\ref{ex:ts}) generated *\ipa{-s}/\ipa{-t} allomorphy (depending on the final consonant of the verb root), and due to the existence of a causative *\ipa{--s} suffix, the *\ipa{--s} was reinterpreted as a mixed causative/applicative, and overgeneralized to contexts where rule (\ref{ex:ts}) does not apply.
 
%Second, examples of  applicative qusheng are due to an *\ipa{-s} to begin with, cognate to the \ipa{-s} causative found in Limbu. Applicative and causative derivations may be difficult to distinguish when the base verb is intransitive, and the applicative value of this *\ipa{-s} suffix would thus derive from its causative use.

\subsection{Passive/Antipassive} \label{sec:passive}
Cases of passive or antipassive value for the qusheng derivation can be accounted for as a trace of the sibilant reflexive suffix, still productive in Kiranti, Dulong/Rawang and Kham (\citealt[320]{driem93agreement}).\footnote{Gyalrongic languages have innovated reflexive prefixes (\citealt{jacques10refl}, \citealt[89]{lai13affixale}), which replaced the \ipa{-si} derivation.} 

In the following I use first-hand data on Khaling (\citealt{jacques16si}) as representative of the functions of the \ipa{-si} reflexive/middle suffix in Kiranti languages.



In Khaling, the  \ipa{--si} derivation has three common identifiable functions: reflexive (\ref{ex:weiwasu}), autobenefactive (\ref{ex:weisi2}) and impersonal subject (\ref{ex:weisi}), the last of which resembles a passive or an anticausative when used with transitive verbs (use with intransitive verbs are also found, but less common). Given the fact that Khaling verbal morphology has a complex morphophonology that is not directly relevant to the present paper, stem alternations will not be commented upon, and verbs are indicated according to their root, an abstract form from which the whole paradigm can be mechanically derived (\citealt{jacques12khaling, jacques16si}).

Note that in the case of the autobenefactive value of the \ipa{-si} derivation (\ref{ex:weisi2}), both agents and patients can still be overt, the agent is syntactically an S, and cannot take the ergative suffix \ipa{-ʔɛ}, and the patient becomes an unmarked adjunct and cannot be indexed on the verb.
 
\begin{exe}
\ex \label{ex:wendu} 
\gll 
\ipa{ʔuŋʌ}  	\ipa{sɵ}  	\ipa{wēnd-u.}  \\
\textsc{1sg:erg} meat cut-\textsc{1sg$\rightarrow$3} \\
\glt I cut the meat.
\end{exe}

\begin{exe}
\ex \label{ex:weiwasu} 
\gll 
	\ipa{mu-wei-w-ʌsu.}  \\
\textsc{neg}-cut-\textsc{irr}-\textsc{refl:1sg:pst} \\
\glt \textit{Reflexive}: `I did not cut myself.'
\end{exe}  

\begin{exe}
\ex \label{ex:weisi2} 
\gll 
 \ipa{ʔʌ̄m} \ipa{sɵ}  	\ipa{wêi-si.}  \\
\textsc{3sg} meat cut-\textsc{refl} \\
\glt \textit{Autobenefactive}: `He cuts meat for himself.'
\end{exe}
 
\begin{exe}
\ex \label{ex:weisi} 
\gll 
 \ipa{sɵ}  	\ipa{wêi-si.}  \\
 meat cut-\textsc{refl} \\
\glt \textit{Impersonal subject}: `The meat is cut (by someone)' OR `The meat cuts easily.'
\end{exe}

It is this last use as \textit{impersonal subject} which may be compared to the passive value of the qusheng derivation in Chinese, in examples such as \ch{聞}{mjun}{hear, smell} $\rightarrow$ \ch{聞}{mjun^H}{be heard, be smelt}.

In addition, the \ipa{-si} derivations has an antipassive value when applied to transitive verbs expressing a feeling (whose A and P are experiencers and stimuli respectively). As shown by examples (\ref{ex:ghryamt})  and (\ref{ex:ghryamtsi}), the \ipa{-si} derivation removes the P (the stimulus) and changes the A of the base verb into an S. The stimulus is still recoverable, but must be assigned oblique case (the ablative \ipa{-kʌ}). This type of example offers a parallel to the antipassive use of the qusheng in Chinese.

\begin{exe}
\ex \label{ex:ghryamt} 
\gll 
  	\ipa{lokpei}  	\ipa{ghrɛ̄md-u.}  \\
leech  be.disgusted.by-\textsc{1sg$\rightarrow$3} \\
 \glt  I am disgusted by leeches.
\end{exe}

\begin{exe}
\ex \label{ex:ghryamtsi} 
\gll \ipa{gʰrɛ̄m-si-ŋʌ}\\
 be.disgusted.by-\textsc{refl-1sg:S/P} \\
\glt  I feel disgust.
\end{exe}

Finally, some of the pairs which Downer classified as examples of the `restricted meaning' category appear to be compatible with the analysis of the qusheng as a fossilized reflexive/middle marker. In particular the semantics of the qusheng in \ch{憶}{ʔik}{remember} $\rightarrow$ \ch{意}{ʔi^H}{think} is identical to that of \ipa{-si} in the pair \ipa{|mimt|} `think of, miss, remember' $\rightarrow$ \ipa{|mimt-si|} `think (that)'. Even if the verb root is different, the meaning of the derivation is clearly identical in Chinese and Khaling.

If the comparison between Old Chinese *\ipa{-s} and Khaling \ipa{-si} is indeed valid, it suggests that syllabic suffixes may have lost their vowel in Chinese, leaving only the consonant as a coda. This raises the question whether other examples of the same type can be proposed, but we defer this question to future investigations.

Another language with a possible trace of the \ipa{-si} derivation is Tibetan. As pointed out by \citet{hill14voicing} (see also \citealt{uray53morphology}, \citealt[864]{zeisler04}), we find in Tibetan several triplets of verbs (A, B and C). A-type verbs are intransitive and have voiced initial stops or affricates. B-types verbs are transitive and present a voicing alternation (voiced initial in the Present and Future forms, unvoiced initial in Past and Imperative). In the present form some B-type verbs have a \ipa{-d} suffix that causes fronting of \ipa{a} to \ipa{o}, while other verbs have \ipa{a} to \ipa{o} Umlaut). C-type verbs are intransitive, have unvoiced initials, and \ipa{-s} suffix (when the phonolotactic constraints of the language allows it to surface) and \ipa{a} to \ipa{e} vowel alternation. The following list includes the most representative examples:

\begin{enumerate}
\item 
A: \tib{ɴgag}{be stopped, break off}

B: \tib{ɴgegs, bkag}{hinder, prohibit}

C: \tib{kʰegs}{be hindered, be prohibited}

\item 
A: \tib{gaŋ}{fill intr.}

B: \tib{ɴgeŋs, bkaŋ}{fill tr.}

C: \tib{kʰeŋs}{be full}

\item 
A: \tib{gab}{hide intr.}

B: \tib{ɴgebs, bkab}{cover tr.}

C: \tib{kʰebs}{be covered over}

\item 
A: \tib{grol}{be free}

B: \tib{ɴgrol  bkrol}{liberate}

C: \tib{kʰrol}{unravel}

\item 
A: \tib{dul}{be tame}

B: \tib{ɴdul, btul}{tame, subdue}

C: \tib{tʰul}{be tame}

\item 
A: \tib{zug}{pierce, penetrate}

B: \tib{ɴdzugs, btsugs}{plant, establish, insert}

C: \tib{tsʰugs}{go into, begin}
\end{enumerate}

\citet{jacques12internal} argues that verbs of type B are the base forms, and that the voicing in the Present and Future stems are due to the a nasal prefix, while type A verbs are derived by anticausative derivation (on which see \citealt{jacques15spontaneous, jacques15causative}). No explanation for type C verbs, however, has ever been proposed.  

Yet, if we accept Jacques's (\citeyear{jacques12internal}) idea that the verb roots in these triplets have unvoiced initials, and that voicing in type A and in Present and Future stem is due to morphological alternations, a solution offers itself: type C verb could be remnants of *\ipa{-si} suffixed middle verbs. This hypothesis explains the three morphological features of type C verbs:

\begin{enumerate}
\item The unvoiced initial is simply the bare stem without any prefix.
\item The \ipa{-s} suffix is a direct segmental trace of the \ipa{-si} suffix; note that the loss of vowel is not unexpected, given the constraint of verbs stems to remain monosyllabic.
\item The vowel alternation \ipa{a} $\rightarrow$ \ipa{e} can be explain as Umlaut due to the lost *\ipa{i} in the suffix. 
\end{enumerate}

Thus, in this hypothesis, a verb form such as \ipa{kʰegs} ‘be hindered, be prohibited’ would originate from pre-Tibetan *\ipa{kak-si}. To confirm this idea, a detailed research on the use of type C verbs in Old Tibetan texts will be necessary, in particular how they semantically differ from type A verbs.

 
\subsection{Denominal verbalization}
While the use of the qusheng to build verbs out of nouns in Chinese is quite common, there are barely any potential cognates in the rest of the family. In Gyalrongic languages, denominal morphology is exclusively prefixal (\citealt{jacques14antipassive}), and Tibetan adopts zero-derivation (adding the TAM markers N- in the present and \textit{b-...-s} in the past, \citealt[29]{jacques14esquisse}). 


Denominal \ipa{-t} derivation is most clearly attested in West-Himalayish. In Bunan, \citet[426]{widmer14bunan} describes the following examples of verbalizing -t suffix: 

\begin{itemize}
\item \ipa{ken} `birth' $\rightarrow$ \ipa{ken-t} `give birth (animal)'
\item \ipa{kur} `load (n)' $\rightarrow$ \ipa{kur-t} `carry a load'
\item \ipa{len} `work (n)' $\rightarrow$ \ipa{len-t} `work (v)'
\item \ipa{sur} `weed (n)' $\rightarrow$ \ipa{sur-t} `weed (v)'
\item *\ipa{ti} `water' (lost in Bunan but attested in closely related languages) $\rightarrow$ \ipa{ti-t} `irrigate'
\end{itemize}

A few examples of denominal \ipa{-t} are also found in various Kiranti languages:
\begin{itemize}
\item Limbu \ipa{thiːn} and Bantawa \ipa{din} `egg' $\rightarrow$ \dhatu{thiːnt-}{lay an egg} and \dhatu{dint}{lay eggs} respectively.  
\item The Limbu transitive verb \dhatu{khaːnt}{wound} <*\ipa{kwɑːr-t} derives with a denominal \ipa{-t} suffix from a noun not attested in Limbu, but found in Wambule \ipa{ɓari} `wound (n)' and Khaling \ipa{koɔ̄r} `wound (n)'
 \item Limbu \dhatu{sokt}{to aim, to point}, Khaling \dhatu{tsukt}{point (with a finger)}, Limbu \ipa{cok} `toe, finger' (Limbu \ipa{c} comes however from proto-Kiranti \ipa{dz}, implying that the noun underwent voicing of the initial).
 \item The Khaling \dhatu{kakt}{hoe} corresponds to the Japhug noun \ipa{qaʁ} `hoe' outside of Kiranti (perhaps also to Limbu \ipa{kaːŋ} `hoe', with unexplained voicing of initial as in \ipa{cok} `toe, finger').
\end{itemize}

\subsection{Adverbialization}
The use of the qusheng  to derive adverbs from verbs or nouns may originate from a locative suffix *\ipa{-s} that is well attested in Tibetan and Gyalrongic languages.

In Tibetan, this suffix is not attested as such, but exists as the \ipa{-s} element in several case markers, including the ergative \ipa{-s}/\ipa{-kʲis}, the comparative \ipa{-bas}, the ablative \ipa{-las} and the elative \ipa{-nas}, the last two of which are compound cases combining the \ipa{-s} element with the allative \ipa{-la} and the locative \ipa{-na} (\citet{konow09intro},  \citealt{delancey82ergative}, \citealt[282]{zeisler11kenhat}, \citealt{hill12bas}), which are all used to build various types of subordinate clauses (\citealt{tournadre10cases}). It is also found in some adverbs, such as \ipa{jas} `from above'.\footnote{There is some evidence that this \ipa{-s} element had an alternative form \ipa{-se} in Old Tibetan and some modern Tibetan languages (\citealt[280-284]{zeisler11kenhat}), but I defer this question to further research.}

In Gyalrongic, we find a locative \ipa{-s} in Situ (\citealt{linxr93jiarong}) and Tshobdun (\citealt[129]{jackson98morphology}), and a locative clitic \ipa{zɯ} Japhug, which was degrammaticalized from the locative suffix (\citealt[167-9]{jacques08}). In Japhug, this clitic \ipa{zɯ}, in addition to is function as a general locative marker, is commonly used to build subordinate clauses with various semantics (\citealt[275;293]{jacques14linking}), and can directly appear after verbs, sometimes with only a very mild subordinating meaning as in (\ref{ex:anWtWCar}).

\begin{exe}
\ex \label{ex:anWtWCar}
\gll \ipa{kɯki} 	\ipa{sthɯci} 	\ipa{ji-kha} 	\ipa{mɯ́j-pe,} 	\ipa{ɲɯ-xtɕi,} 	\ipa{ɲɯ-ŋgɤr} 	\ipa{zɯ} 	\ipa{nɤʑo} 	\ipa{kɯ,}  \ipa{jɤ-ɕe} 	\ipa{tɕe} 	\ipa{a-nɯ-tɯ-ɕar} 	\ipa{tɕe,}  \\
this such \textsc{1pl.poss}-house \textsc{neg:sens}-be.good  \textsc{sens}-be.small  \textsc{sens}-be.narrow \textsc{loc} \textsc{2sg} \textsc{erg} \textsc{imp}-go \textsc{lnk}  \textsc{irr-pfv}-2-search \textsc{lnk} \\
\glt Our house is so bad, so small, so narrow, go there and look for (the fish to ask him for something). (140430 yufu he ta de qizi, 55)
\end{exe}
The reanalysis of a clausal linker as an adverbializer is straightforward, and thus the examples of adverbialization collected by Downer might be cases of a former locative *\ipa{-s} which underwent the same extension as Japhug, and further grammaticalized as a derivational morpheme.
 

\subsection{Second member of compounds}
In Old Tibetan, \citet{uebach08rjeblas} have brought to light examples of a \ipa{--s} suffix in the second member of nominal compounds, as for instance \ipa{lag} `arm, hand' + \ipa{riŋ} `long' $\rightarrow$ \ipa{lag.riŋs} `long arms'. These examples may be analyzable as particular instances of the nominalization suffix treated in section \ref{sec:nmlz}.

Likewise, in Old Chinese, compounds with qusheng in the first or second element, as for instance the alternative reading \ch{雙生}{ʂæwŋ ʂæŋ^H}{twin} for \ch{生}{ʂæŋ}{live}, are best analyzed as cases of nominalization, and hardly constitute a distinct derivational category.
 
\subsection{Perfective?} \label{sec:perf}
Some authors (in particular \citealt{jinlx06}) have claimed that some examples of *\ipa{--s} suffixes can be accounted for as traces of the *\ipa{--s} perfective suffix found in Tibetan (\ipa{bʲed, bʲas} `do') or in Gyalrong languages (Japhug \ipa{--t} or \ipa{--z} \textsc{1/2sg}$\rightarrow$3 perfective suffix, Situ third person intransitive perfective \ipa{--s}).\footnote{See \citet{huangbf97s.houzhui} for other potential examples, though many in that article come from languages that do not preserve final *\ipa{--s} and hence cannot be cognate to the Tibetan or Gyalrong suffix, unless degrammaticalization took place.}

This idea is tantalizing, as this would be the only inflectional use of a *\ipa{-s} suffix in Old Chinese, as opposed to the previous cases, which all concern derivational functions.

However, this idea is hardly compelling, as alternative readings are not listed systematically enough in the \textit{Jingdian shiwen} to allow easy reverification. Judgement on this matter must be deferred until the Chinese evidence has been clearly sifted through on the basis of an exhaustive analysis of available sources.

\section{Conclusion}
This paper presents all potential comparative evidence known to me in Trans-Himalayan languages of suffixes comparable to the qusheng derivation. While not all comparisons are equally compelling, postulating several unrelated origins for the qusheng derivation solves two unrelated issues, namely (1) the excess of *\ipa{-s} in all reconstructions of Old Chinese, which is explained here due to the fact that *\ipa{-s} comes from the merger of *\ipa{-s}, *\ipa{-t} and *\ipa{-sV} suffixes, and perhaps also other codas such as primary *\ipa{-h}. (2) the numerous and contradictory functions attributed to the qusheng derivation.

Future progress in the study of Old Chinese morphology can only come from collaboration between philologists familiar with the texts and the alternative readings therein, and linguists with a first-hand knowledge of conservative Trans-Himalayan languages where morphology is still visible without any need for reconstruction.

\bibliographystyle{unified}
\bibliography{bibliogj}

\end{document}