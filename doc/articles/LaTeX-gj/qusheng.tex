\documentclass[oldfontcommands,oneside,a4paper,11pt]{article} 
\usepackage{fontspec}
\usepackage{natbib}
\usepackage{booktabs}
\usepackage{xltxtra} 
\usepackage{polyglossia} 
\usepackage[table]{xcolor}
\usepackage{gb4e} 
\usepackage{multicol}
\usepackage{graphicx}
\usepackage{float}
\usepackage{hyperref} 
\hypersetup{bookmarks=false,bookmarksnumbered,bookmarksopenlevel=5,bookmarksdepth=5,xetex,colorlinks=true,linkcolor=blue,citecolor=blue}
\usepackage[all]{hypcap}
\usepackage{memhfixc}
\usepackage{lscape}

\bibpunct[: ]{(}{)}{,}{a}{}{,}

%\setmainfont[Mapping=tex-text,Numbers=OldStyle,Ligatures=Common]{Charis SIL} 
\newfontfamily\phon[Mapping=tex-text,Ligatures=Common,Scale=MatchLowercase,FakeSlant=0.3]{Charis SIL} 
\newcommand{\ipa}[1]{{\phon \mbox{#1}}} %API tjs en italique
\newcommand{\ipab}[1]{{\scriptsize \phon#1}} 

\newcommand{\grise}[1]{\cellcolor{lightgray}\textbf{#1}}
\newfontfamily\cn[Mapping=tex-text,Ligatures=Common,Scale=MatchUppercase]{MingLiU}%pour le chinois
\newcommand{\zh}[1]{{\cn #1}}
\newcommand{\refb}[1]{(\ref{#1})}
\newcommand{\factual}[1]{\textsc{:fact}}
\newcommand{\rdp}{\textasciitilde{}}

\XeTeXlinebreaklocale 'zh' %使用中文换行
\XeTeXlinebreakskip = 0pt plus 1pt %
 %CIRCG
 \newcommand{\bleu}[1]{{\color{blue}#1}}
\newcommand{\rouge}[1]{{\color{red}#1}} 
\newcommand{\ch}[3]{\zh{#1} \ipa{#2} `#3'} 

\begin{document} 
\title{How many *-s suffixes in Old Chinese? }
\author{Guillaume Jacques}
\maketitle
 
\section{Introduction}
While qusheng-derivation is one of the most prominent trace of morphology in Old Chinese, it is probably also the least understood one, as it present diverse and even contradictory functions, to the extent that \citet[262]{downer59}, in his seminal article, argued that it was simply a way of creating new words, not a derivation with a well-defined grammatical function.\footnote{`The present writer holds the opinion that with our present knowledge of Classical Chinese, it is better to regard chiuhsheng derivation not as a remnant of a former inflectional system of the Indo-European type, but simply as a system of derivation and nothing more. When new words were needed, they were created by pronouncing the basic word in the chiuhsheng. The grammatical regularity found in many cases would then be in a way fortuitous, being the result not of grammatical inflection, but of the need to create new words.'}
 
Yet, we know thanks to the work of scholars such as \citet{haudricourt54chinois}, \citet{forrest60occlusives} and \citet{sagart99roc, bs14oc} that Qusheng derivation comes (at least in part) from *\ipa{--s} suffixes. Since \ipa{-s} suffixes with functions similar to those which have been reconstructed for Old Chinese are attested and even still productive in more conservative languages of the Trans-Himalayan family, it is worthwhile to explore the exact opposite hypothesis to Downer's ultrascepticism, namely that the vast array of function of the *-s is due to the merger of many independent dental suffixes, and constitute indeed obscured traces of a former inflectional system.

In this paper, I first present Downer's work and adapt to to modern-day terminology. Second, I propose a new sound law for pre-Old Chinese. Third, drawing on first-hand data from Japhug (Gyalrongic), Khaling (Kiranti) and Tibetan, I show that seven out of Downer's eight functions of the qusheng derivation have potential comparanda in conservative languages of the family.

Old Chinese is given in Middle Chinese (in an IPA adaptation of Baxter's \citeyear{baxter92} system) rather than OC, because all modern old Chinese reconstruction models agree on *\ipa{-s} as the origin of the qusheng, and the discussion in this paper is therefore independent of any particular system.

\section{Downer's eight functions}
Despite his suggestion that qusheng derivation was basically random, \citet{downer59} provide a very useful classification of attested derivations, which I take as basic data for this study. It is possible that other functions for the qusheng can be found which are not included in Downer's work, but I leave the examination of those to other scholars.

He counts the following eight categories, here illustrated by one or two representative examples.

\begin{enumerate}
\item \textbf{nominalization}

The Qusheng can be used derive a noun of property out of an stative adjectival verb, and a patient noun out of a transitive verb.

\ch{高}{kaw}{be high} $\rightarrow$ \ch{高}{kaw^H}{height} 

\ch{處}{tɕʰjo^X}{be at} $\rightarrow$ \ch{處}{tɕʰjo^H}{place} 

\item \textbf{verbalization} 

The semantics of verbs derived from nouns with the Qusheng is very varied, but includes in particular the meaning `use X as ...'

\ch{家}{kæ}{family} $\rightarrow$ \ch{嫁}{kæ^H}{marry} 

\ch{枕}{tɕim^X}{pillow} $\rightarrow$ \ch{枕}{tɕim^H}{use as a pillow} 

\item \textbf{causative}

\ch{飲}{ʔim^X}{drink} $\rightarrow$ \ch{飲}{ʔim^H}{give to drink} 

\ch{買}{mɛ^X}{buy} $\rightarrow$ \ch{賣}{mɛ^H}{sell} 

In addition to causative derivation, the qusheng is also used with a tropative meaning `consider to be X' (\citealt{jacques13tropative}), as the famous case of \ch{好}{xaw^X}{be good} $\rightarrow$ \ch{好}{xaw^H}{like}.

\item \textbf{applicative} 

What Downer meant by `effective' corresponds to what typologists now call `applicative', namely a patient-adding derivation which preserves the syntactic status of the S/A subject. The examples found in Old Chinese typically involve experiencer verb, whose applicative form adds a stimulus.

\ch{渴}{kʰat}{be thirsty} $\rightarrow$ \ch{愒}{kʰaj^H}{long for} 

\item \textbf{`restricted meaning'}
This category appears a catch-all for cases of qusheng derivations not otherwise classifiable into any of the well-understood categories. It includes examples such as:

\ch{少}{ɕjew^X}{be few} $\rightarrow$ \ch{少}{ɕjew^H}{be young} 

This category, given its absence of clarity, is not discussed in the comparative section of this paper.

\item \textbf{passive}

Qusheng derivation is used for argument demoting derivations. Examples of passive (agent-demoting) derivations are found:

\ch{散}{san^X}{scatter} $\rightarrow$ \ch{散}{san^H}{be loose} 

In addition, although Downer does not set up a special category, some of his examples rather attest an antipassive (patient-demoting) meaning, as the following:

\ch{射}{ʑjek}{shoot at} $\rightarrow$ \ch{射}{ʑjæ^H}{practise archery} 

\item \textbf{adverbialization}

A few restricted examples appear to be interpretation as derivation from noun or numeral to adverb, as:

\ch{三}{sam}{three} $\rightarrow$ \ch{三}{sam^H}{thrice} 

\item \textbf{form in compounds}

Downer notes many case of qusheng derivation in the first or second element of compounds; most of his examples however appear to be interpretable as special instances of denominal or deverbal derivations.
\end{enumerate}

\section{New sound law}
A few languages of the Trans-Himalayan family, most notably Kiranti, have a contrast between final clusters -Cs and -Ct in verb roots.\footnote{These clusters are not word-final in these languages, as they always surface before inflectional suffixes.} 

In Old Chinese, no *-Ct final clusters have even been proposed. Yet, the final *-s as reconstructed in all modern systems is much too common in comparison with languages of the family which allow final clusters. In Tibetan for instance, there are 7398 words ending in -s in  \citet{bodrgya} out of 53922 (14\%). By contrast, in Baxter and Sagart's (\citeyear{bs14oc}) online list of reconstructions, we find 1160 words with final -s out of 4968 (23\%). This raises the possibility that final *-s is not the only origin of qusheng.

To account for the absence of *-Ct clusters and for the excess of final *-s in present reconstruction models, I propose the following sound law (C stands for a particular subset of codas, perhaps restricted to labial and velar consonants):

\begin{exe}
\ex \label{ex:ts}
\glt *\ipa{--t} $\rightarrow$ *\ipa{--s} /C\_\# 
\end{exe}

This rule is similar to the rule undergone by the present tense \ipa{--d} suffix in Tibetan, which is realized as \ipa{--s} after final stops and \ipa{-m}, as in \ipa{ɴdeb-s, btab}   `plant' vs \ipa{ɴde-d, bdas}   `chase' (\citealt[52-53]{coblin76}). 

This hypothesis allows comparison of Chinese qusheng derivations with not only \ipa{-s} suffixes, but also \ipa{-t} suffixes, at least following grave final consonants. In addition, in the following comparisons, I assume that the *\ipa{--s} allomorph of *\ipa{--t} suffixes generated by this rule is then extended analogically to other contexts, in particular open syllables (this idea in particular is necessary to explain the *\ipa{--s} applicatives, see section \ref{sec:causative}).


\section{Comparisons}

Very few languages in ST clearly preserve final \ipa{--s} suffixes, and complex final clusters: only data from Tibetan, Kiranti, Gyalrongic, Dulong/Rawang can be used without reconstruction.\footnote{Phonologically less conservative languages may provide crucial confirmatory evidence, but at the present stage it would be premature to investigate languages where final \ipa{-s} is only accessible through the comparative method. }

\subsection{Nominalization}
As early as \citet{forrest60occlusives}, scholars have noted that the nominalization function of the qusheng derivation could be compared with the \ipa{-s} nominalization found in Tibetan (on which see \citealt[43]{conrady1896}, \citealt[624-5]{hill14derivational}). The \ipa{-s} derives patient nouns, nouns of manner and nouns of characteristic, as illustrated by the following examples.

\begin{itemize}
\item Patient: \ipa{za} `eat' $\rightarrow$ \ipa{zas} `food' 
\item Characteristic: \ipa{zab.mo} `deep'  $\rightarrow$ \ipa{zabs} `depth'.
\item Manner: \ipa{ɴgro} `go'  $\rightarrow$ \ipa{ɴgros} `gait' 
 \end{itemize}

The similarity with Old Chinese, where qusheng derivation is used exactly with the first to meanings, is striking (note for instance \ch{深}{ɕim}{deep} $\rightarrow$ \ch{深}{ɕim^H}{depth}).

Moreover, some nouns derived from verbs are cognate between Chinese and Tibetan, eg \ipa{ɴtʰag, btags} ``weave" $\rightarrow$ \ipa{tʰags} ``textile", \ch{織}{tɕik}{weave} $\rightarrow$ \ch{織}{tɕi^H}{cloth}.

Nominalization by \ipa{-s} suffixation was already only very marginally productive in Old Tibetan. The only clear case of a neologism formed in historical times using the \ipa{-s} suffix is \ipa{gzuŋs} `dhâraṇî' (a type of Buddhist magical formula). This word is derived from the root \ipa{|zuŋ|} of \ipa{bzuŋ} `seize', by calque of `dhâraṇî-', which derives from the root dhṛ- `seize, hold' in Sanskrit. It actually uses a circumfix \ipa{g-...-s}, the prefixal element of which has cognates in Gyalrongic languages and beyond (\citealt{jacques14snom}).

Outside of Tibetan, we find vestigial traces of *\ipa{-s} nominalization in Gyalrongic languages (\citealt{jacques03s.houzhui}).

 \subsection{Causative/applicative} \label{sec:causative}
The applicative and causative suffixes are most clearly preserved in Kiranti (\citealt{michailovsky85dental, jacques15derivational.khaling}).

Limbu stands out among Kiranti languages with clearly distinct \ipa{-s} causative and \ipa{--t} applicative suffixes, as illustrated by the following triplet (\citealt{michailovsky02dico}):
\begin{itemize}
\item \ipa{haːp-} \textit{vi} `weep'
\item \ipa{haːps-} \textit{vt} `cause to weep' (causative)
\item \ipa{haːpt-} \textit{vt} `mourn for, weep for' (applicative)
\end{itemize}

No clear example of causative or applicative \ipa{-d} or \ipa{-s} are found in Tibetan (\citealt[630]{hill14derivational}). In Gyalrongic, only isolated examples are attested, mainly with motion verbs. For instance Japhug has \ipa{ɣi} < *\ipa{wi} `come' $\rightarrow$ \ipa{ɣɯt} < *\ipa{wit} `bring'; note however that despite  the dearth of examples, this pair actually has cognates in Kiranti (Khaling \ipa{|pi|} `come' $\rightarrow$ \ipa{|pit|} `bring';

In Chinese, \citet{sagart04directions} discovered one direct trace of the applicative *\ipa{--t}  in the following example:

\begin{exe}
\ex 
\gll \zh{行} \zh{或} \zh{使} \zh{之}, \zh{止} \zh{或} \zh{尼} \zh{之} \\
 \ipa{Cə.ɡˁ<r>aŋ}  \ipa{ɢʷˁək}  \ipa{s-rəʔ}  \ipa{tə}  \ipa{təʔ}  \ipa{ɢʷˁək}  \ipa{n<r>əl-t}  \ipa{tə} \\
\glt `A man's advancement is effected, it may be, by others, and the stopping him is, it may be, from the efforts of others.' (Legge)
\end{exe}

The character \zh{尼} is read here with the fanqie \zh{女乙反}, corresponding to a Middle Chinese pronunciation \ipa{ɳit}. This reading can be opposed to the usual pronunciation of this character \ipa{ɳij} $\leftarrow$ *\ipa{n<r>ɨl}  (fanqie \zh{女夷切}), cognate with Tibetan \ipa{ɲal} `lie down, sleep' and Japhug \ipa{nɯna} `rest',\footnote{Loss of final *\ipa{-l} after *\ipa{a} is a common innovation of Burmo-Gyalrongic languages.} and suggests a reconstruction *\ipa{n<r>əl-t} in Old Chinese. This example shows that rule (\ref{ex:ts}) does not apply after *\ipa{--l}.

 Yet, examples of  applicative qusheng are found in syllables with final *\ipa{-l} or even open syllables in Old Chinese. These examples can be accounted for in two ways.
 
 First, rule (\ref{ex:ts}) generated *\ipa{-s}/\ipa{-t} allomorphy (depending on the final consonant of the verb root), and due to the existence of a causative *\ipa{--s} suffix, the *\ipa{--s} was reinterpreted as a mixed causative/applicative, and overgeneralized to contexts where rule (\ref{ex:ts}) does not apply.
 
Second, examples of  applicative qusheng are due to an *\ipa{-s} to begin with, cognate to the \ipa{-s} causative found in Limbu. Applicative and causative derivations may be difficult to distinguish when the base verb is intransitive, and the applicative value of this *\ipa{-s} suffix would thus derive from its causative use.

\subsection{Passive/Antipassive}
Cases of passive or antipassive qusheng can be accounted for as a trace of the sibilant reflexive suffix, still productive in Kiranti and Dulong/Rawang.

In Khaling, both passive/anticausative \textit{and} antipassive meanings are attested for \ipa{--si} derivation (\citealt{jacques16si}).
\subsubsection{Impersonal subject vs reflexive} 
\begin{exe}
\ex \label{ex:wendu} 
\gll 
\ipa{ʔuŋʌ}  	\ipa{sɵ}  	\ipa{wēnd-u.}  \\
\textsc{1sg:erg} meat cut-\textsc{1sg$\rightarrow$3} \\
\glt I cut the meat.
\end{exe}

\begin{exe}
\ex \label{ex:weisi} 
\gll 
 \ipa{sɵ}  	\ipa{wêi-si.}  \\
 meat cut-\textsc{refl} \\
\glt \textit{Impersonal subject}: `The meat is cut (by someone)' OR `The meat cuts easily.'
\end{exe}

\begin{exe}
\ex \label{ex:weisi2} 
\gll 
 \ipa{ʔʌ̄m} \ipa{sɵ}  	\ipa{wêi-si.}  \\
\textsc{3sg} meat cut-\textsc{refl} \\
\glt \textit{Autobenefactive}: `He cuts meat for himself.'
\end{exe}
 
\begin{exe}
\ex \label{ex:weiwasu} 
\gll 
	\ipa{mu-wei-w-ʌsu.}  \\
\textsc{neg}-cut-\textsc{irr}-\textsc{refl:1sg:pst} \\
\glt \textit{Reflexive}: `I did not cut myself.'
\end{exe}  
\subsubsection{Antipassive} 
\begin{exe}
\ex \label{ex:ghryamtsi} 
\gll \ipa{gʰrɛ̄m-si-ŋʌ}\\
 be.disgusted.by-\textsc{refl-1sg:S/P} \\
\glt  I feel disgust.
\end{exe}

\begin{exe}
\ex \label{ex:ghryamt} 
\gll 
  	\ipa{lokpei}  	\ipa{ghrɛ̄md-u.}  \\
leech  be.disgusted.by-\textsc{1sg$\rightarrow$3} \\
 \glt  I am disgusted by leeches.
\end{exe}

\subsection{Denominal}
Denominal morphology in Gyalrongic is exclusively prefixal (\citealt{jacques14antipassive}), examples of \ipa{--t} or \ipa{--s} suffixes in morphologically conservative languages are very rare:

Limbu \ipa{thiːn} `egg' $\rightarrow$ \ipa{thiːnt-} `lay an egg'

The origin of the denominal \ipa{--s} obscure, maybe some cases are in fact causative or applicatives applied to a intransitive verb that was derived from the noun by either zero-derivation or a prefixal derivation process that left no visible traces.

\subsection{Adverbialization}

 
Locative Situ \ipa{-s} (\citealt{linxr93jiarong}), Japhug \ipa{zɯ} (\citealt[167-9]{jacques08zh}), \ipa{-s} element in Tibetan case markers (\citealt{hill12bas}).

Oblique case $\rightarrow$ adverbial marker


\subsection{Second member of compounds}


\citet{uebach08rjeblas}, \ipa{--s} suffix in the second member of compounds in Old Tibetan, eg \ipa{lag} + \ipa{riŋ} $\rightarrow$ \ipa{lag.riŋs} `long arms'.

 However, in OC, examples of this type are probably cases of nominalized verbs.
 
\subsection{Perfective?}
Some authors (in particular \citealt{jinlx06}) have claimed that some examples of *\ipa{--s} suffixes can be accounted for as traces of the *\ipa{--s} perfective suffix found in Tibetan (\ipa{bʲed, bʲas} `do') or in Gyalrong languages (Japhug \ipa{--t} or \ipa{--z} \textsc{1/2sg}$\rightarrow$3 perfective suffix, Situ third person intransitive perfective \ipa{--s}; see \citealt{huangbf97s.houzhui} for other potential examples, though many in that article come from languages that do not preserve final *\ipa{--s} and hence cannot be cognate to the Tibetan or Gyalrong suffix, unless degrammaticalization took place).

Tantalizing, but hardly compelling: if the OC verb had inflexional morphology, it is not reflected in the Jingdian shiwen glosses in a way that allows systematic investigation.

\section{Conclusion}

\bibliographystyle{unified}
\bibliography{bibliogj}

\end{document}