\documentclass[oldfontcommands,oneside,a4paper,11pt]{article} 
\usepackage{fontspec}
\usepackage{natbib}
\usepackage{booktabs}
\usepackage{xltxtra} 
\usepackage{polyglossia} 
\usepackage[table]{xcolor}
\usepackage{multicol}
\usepackage{graphicx}
\usepackage{float}
\usepackage{hyperref} 
\hypersetup{bookmarks=false,bookmarksnumbered,bookmarksopenlevel=5,bookmarksdepth=5,xetex,colorlinks=true,linkcolor=blue,citecolor=blue}
\usepackage[all]{hypcap}
\usepackage{memhfixc}
\usepackage{lscape}
\usepackage{tikz}
\usetikzlibrary{trees}
\usepackage{lineno}
\usepackage{gb4e} 
\bibpunct[: ]{(}{)}{,}{a}{}{,}
 
%\setmainfont[Mapping=tex-text,Numbers=OldStyle,Ligatures=Common]{Charis SIL}  
\newfontfamily\phon[Mapping=tex-text,Ligatures=Common,Scale=MatchLowercase,FakeSlant=0.3]{Charis SIL} 
\newcommand{\ipa}[1]{{\phon #1}} %API tjs en italique
 \newcommand{\khyal}[1]{\ipa{|#1|}}
\newcommand{\grise}[1]{\cellcolor{lightgray}\textbf{#1}}
\newfontfamily\cn[Mapping=tex-text,Ligatures=Common,Scale=MatchUppercase]{MingLiU}%pour le chinois
\newcommand{\zh}[1]{{\cn #1}}
\newcommand{\tld}{\textasciitilde{}}
\newcommand{\ch}[4]{\zh{#1} \ipa{#2} $\leftarrow$ \ipa{*#3} `#4'} 
   \linenumbers

\begin{document} 
 \title{On basic vocabulary in ST}
 \author{Guillaume Jacques}
 \maketitle  
 
  \section{fly}
  
  \ipa{bʲer} `flee' \khyal{bʰer}
  %cIn shi shis bong bu sha cur ' byerte pyin ma dag brdzangs bar bgyis nI'
 \section{eat}
 
 
 \subsection{`lick' $\rightarrow$  eat'}
\ch{食}{ʑik}{mlək}{eat}
 \subsection{`chew' $\rightarrow$  eat'}
 
 \begin{exe}
\ex 
\gll \ipa{ʑɯmkhɤm}  	\ipa{ʑo}  	\ipa{tu-ndza-nɯ}  	\ipa{mɤɕtʂa}  	\ipa{kɤ-mqlaʁ}  	\ipa{mɯ́j-βze}   \\
long.time \textsc{emph} \textsc{ipfv}-chew-\textsc{pl} until \textsc{inf}-swallow \textsc{neg:sens}-do[III] \\
\glt They only swallow it after having chewed it for a long time. (19 qachGa mWntoR, 204)
\end{exe}
 
 \ipa{tɯ-ŋka}, \ipa{nɤŋka}

\khyal{ʣA} 
 \khyal{kA}
 
\bibliographystyle{unified}
\bibliography{bibliogj}
\end{document}
