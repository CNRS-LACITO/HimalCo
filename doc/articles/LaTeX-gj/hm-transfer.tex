\documentclass[oneside,a4paper,11pt]{article} 
\usepackage{fontspec}
\usepackage{natbib}
\usepackage{booktabs}
\usepackage{xltxtra} 
\usepackage{polyglossia} 
\usepackage[table]{xcolor}
\usepackage{gb4e} 
\usepackage{multicol}
\usepackage{graphicx}
\usepackage{float}
\usepackage{hyperref} 
\hypersetup{bookmarksnumbered,bookmarksopenlevel=5,bookmarksdepth=5,colorlinks=true,linkcolor=blue,citecolor=blue}
\usepackage[all]{hypcap}
\usepackage{memhfixc}
\usepackage{lscape}
\usepackage{amssymb}
 
\bibpunct[: ]{(}{)}{,}{a}{}{,}

%\setmainfont[Mapping=tex-text,Numbers=OldStyle,Ligatures=Common]{Charis SIL} 
\newfontfamily\phon[Mapping=tex-text,Scale=MatchLowercase]{Charis SIL} 
\newcommand{\ipa}[1]{\textbf{{\phon\mbox{#1}}}} %API tjs en italique
%\newcommand{\ipab}[1]{{\scriptsize \phon#1}} 

\newcommand{\grise}[1]{\cellcolor{lightgray}\textbf{#1}}
\newfontfamily\cn[Mapping=tex-text,Ligatures=Common,Scale=MatchUppercase]{SimSun}%pour le chinois
\newcommand{\zh}[1]{{\cn #1}}
\newfontfamily\mleccha[Mapping=tex-text,Ligatures=Common,Scale=MatchLowercase]{Galatia SIL}%pour le grec
\newcommand{\grec}[1]{{\mleccha #1}}



\newcommand{\change}[2]{*\ipa{#1} $\rightarrow$ \ipa{#2}}
 

\XeTeXlinebreakskip = 0pt plus 1pt %
 %CIRCG
 
\newcommand{\zhc}[2]{\zh{#1} \ipa{#2}} 
\newcommand{\mien}[5]{\ipa{#1}^{#2} `#3' (\zh{#4}, p.#5)} 
\newcommand{\phm}[3]{\ipa{#1} `#2' (#3)} 
\begin{document}

\title{Transfer of nasality from onset to rhyme in Hmongic}
\author{Guillaume Jacques}
\maketitle

\section*{Introduction} 
A gap is observable in the reconstructed proto-Hmong phonological system (\citealt{wang94hmong}, \citealt{wang95protomy}, \citealt{ratliff10protohm}): reconstructed etyma with aspirated nasal initial and non-nasal vowels are extremely few, and the contrast between oral and nasal vowels, clearly present in Mien, is almost completely neutralized. 

While some authors have hypothesized in some cases secondary prenasalization in Hmongic (\citealt[57]{ratliff10protohm}), no work until now has attempted to account for this gap in terms of a sound law. 

%\citet{maozw92mien}
 
\citet{michaud-jacques12nasalite}

\section{Sound law} 
\ipa{*hmV} $\rightarrow$ \ipa{*hmṼ} , a proto-Hmongic sound law (does not apply if final stop)


\citealt[57]{ratliff10protohm}

proto-Hmong \phm{*hnuŋ^B}{forget}{2.8/27} = proto-Mien \phm{*hɲou^B}{forget}{4.8/13}

proto-Hmong \phm{*hnɛŋ^A}{sun/day}{2.8/22} = proto-Mien \phm{*hnʷɔi^A}{sun/day}{2.8/11}, Situ \ipa{tə-sɲî} `day'

proto-Hmong  \phm{*hnæn^B}{crossbow}{2.8/19} = proto-Mien \phm{*hnək}{crossbow}{2.8/7} 
Common Thai \ipa{*ʰna^B}, \zh{弩} \ipa{nuX} < \ipa{*nˁaʔ}
 
From KD:
\phm{*hmaŋ^C}{wild dog}{1.8/24}
\citet{ferlus96kamsui}
Lakkia \ipa{kʰwõ¹} < \ipa{*kʰma^A}, Common Thai \ipa{*ʰma^A}

  

\section{Exceptions}
\subsection{Stop codas}
\phm{*hnɔp}{cough}{2.8/29} - nasality in 3, 4, not in 5

\subsection{Alternative reconstruction}
\phm{*hmeiH}{taste}{1.8/29}, p40-1, only 1 \ipa{mhu^5}, Mien 10 \ipa{hmi^5}

Compatible with a different analysis, \ipa{mhu^5} from proto-Hmongic \ipa{*hmʷəŋH} (29h, same rhyme as \phm{*hɲʷəŋH}{year}{2.8/29}, Mien \ipa{hɲaŋ^5}, from \ipa{*-iŋ})
\ipa{*-iŋ} > \ipa{*-jaŋ} (cf Amdo Tibetan, \citet{gong16amdo})

Proto-HM \ipa{-iŋ} otherwise only in \phm{*ljiŋ}{field}{2.42.1/18}, borrowed from Chinese \zh{田} \ipa{den} < \ipa{*lˁiŋ} `field'

\phm{*ŋiuŋ}{water buffalo/cow}{5.9/23}, \citet(21){downer73loanwords}
\citet[88]{ratliff10protohm}
\section{Dissimilation} 
\citet[172]{ratliff10protohm}


\section{On the origins of aspirated nasals in Hmong-Mien}
1.54 \ipa{*S-mruɔŋH}, to be reconstructed \ipa{*rmɔŋ}, > \ipa{hm-} in Mien, metathesis in Hmong.
\bibliographystyle{unified}
\bibliography{bibliogj}
\end{document}
