\documentclass[oneside,a4paper,11pt]{article} 
\usepackage{fontspec}
\usepackage{natbib}
\usepackage{booktabs}
\usepackage{xltxtra} 
\usepackage{polyglossia} 
\usepackage[table]{xcolor}
\usepackage{tikz}
\usetikzlibrary{trees}
\usepackage{gb4e} 
\usepackage{multicol}
\usepackage{graphicx}
\usepackage{float}
\usepackage{hyperref} 
\hypersetup{bookmarksnumbered,bookmarksopenlevel=5,bookmarksdepth=5,colorlinks=true,linkcolor=blue,citecolor=blue}
\usepackage[all]{hypcap}
\usepackage{memhfixc}
\usepackage{lscape}
\usepackage{amssymb}
 
\bibpunct[: ]{(}{)}{,}{a}{}{,}

%\setmainfont[Mapping=tex-text,Numbers=OldStyle,Ligatures=Common]{Charis SIL} 
\newfontfamily\phon[Mapping=tex-text,Scale=MatchLowercase]{Charis SIL} 
\newcommand{\ipa}[1]{\textbf{{\phon\mbox{#1}}}} %API tjs en italique
%\newcommand{\ipab}[1]{{\scriptsize \phon#1}} 

\newcommand{\grise}[1]{\cellcolor{lightgray}\textbf{#1}}
\newfontfamily\cn[Mapping=tex-text,Ligatures=Common,Scale=MatchUppercase]{SimSun}%pour le chinois
\newcommand{\zh}[1]{{\cn #1}}
\newfontfamily\mleccha[Mapping=tex-text,Ligatures=Common,Scale=MatchLowercase]{Galatia SIL}%pour le grec
\newcommand{\grec}[1]{{\mleccha #1}}
\newfontfamily\tibetain{Microsoft Himalaya} 

\newcommand{\sg}{\textsc{sg}}
\newcommand{\pl}{\textsc{pl}}
\newcommand{\ro}{$\Sigma$}
\newcommand{\ra}{$\Sigma_1$} 
\newcommand{\rc}{$\Sigma_3$}  
\newcommand{\dhatu}[2]{|\ipa{#1}| `#2'}
\newcommand{\tibet}[3]{{\tibetain#1} \textit{\phon#2} `#3'}  
\newcommand{\tibetan}[1]{{\tibetain#1}}
\newcommand{\dhat}[1]{|\ipa{#1}|}
\newcommand{\change}[2]{*\ipa{#1} $\rightarrow$ \ipa{#2}}
 

\XeTeXlinebreakskip = 0pt plus 1pt %
 %CIRCG
 
\newcommand{\zhc}[2]{\zh{#1} \ipa{#2}} 


\begin{document}

\title{Fossil nominalization prefixes in Tibetan and Chinese\footnote{I would like to thank Nathan W. Hill for useful comments. The glosses follow the Leipzig Glossing Rules. Other abbreviations used here are:  \textsc{auto} autobenefactive / spontaneous,  \textsc{dem} demonstrative,  \textsc{emph} emphatic, \textsc{inv} inverse,  \textsc{lnk} linker, \textsc{pfv} perfective, \textsc{poss} possessor, \textsc{fact} factual,  \textsc{sens} sensory. The Japhug examples are taken from a corpus that is progressively being made available on the Pangloss archive (\citealt{michailovsky14pangloss},  
 \url{http://lacito.vjf.cnrs.fr/pangloss/corpus/list\textunderscore rsc.php?lg=Japhug}). Middle Chinese is presented in an IPAnized version of Baxter's (\citeyear{baxter92}) transcription.}}
\author{Guillaume Jacques}
\maketitle

\section{Introduction}
  In the Trans-Himalayan family, unlike Indo-European, the morphologically most complex languages, Rgyalrong and Kiranti, are endangered languages without a written literary tradition. A growing body of evidence suggests that this complex morphology is at least in part archaic (\citealt{jacques12agreement, delancey14second, jacques16ssuffixes, gong17xingtaixue}), in particular because affixes that are non-productive or even accessible through reconstruction (such as the sigmatic causative) in Tibetan and Chinese are still productive in Rgyalrong languages.

The present paper deals with another case comparable of recessive morphology in the literary languages, the sigmatic and velar nominalization prefixes. These prefixes, which are completely productive in Rgyalrong languages such as Japhug, are only attested in a handful of words in Tibetan and Old Chinese. Using evidence from Japhug to ascertain the precise meaning of these prefixes, this study proposes new etymologies and evaluates older proposals.

\section{Participles in Rgyalrongic}
Rgyalrongic languages have a set of prefixes deriving non-finite verb forms, including participles, converbs and infinitives (\citealt{jacques14linking}, \citealt{jackson14morpho}, \citealt{jacques16relatives}). In the present paper, two sets of participles, the velar (core argument) and sigmatic (oblique) participles are discussed.


\subsection{Velar participles} \label{sec:rgyalrong.velar}
All Rgyalrong languages (Japhug, Tshobdun, Zbu and Situ) all have a set participle prefixes in \ipa{kV-} used in particular to build participial relative clauses with subject or object relativization. There are slight difference between the languages (\citealt{jackson06guanxiju, jacksonlin07, jacques16relatives, zhang16bragdbar}); this paper only includes data from Japhug, which are sufficient to illustrate the constructions shared by all Rgyalrong languages.\footnote{The participle prefixes are historically related to generic person and infinitive prefixes (\citealt{sun12complementation}, \citealt{sun14generic}, \citealt{jacques16complementation}, \citealt{jacques18generic}), but these are not discussed in this paper. }

The core argument participle prefixes in Japhug are \ipa{kɯ-} for subject (S/A) participle and \ipa{kɤ-} for object participle, and can be illustrated by examples (\ref{ex:kWrAZi}) (intransitive subject, in a non-restrictive head-internal relative) (\ref{ex:WkWtsxWB}) (transitive subject)  and (\ref{ex:pWkABde}) (object and intransitive subject).

 \begin{exe}
\ex \label{ex:kWrAZi}
 \gll [\ipa{nɯ}	\ipa{ɯ-ŋgɯ}	\ipa{ɴɢoɕna}	\ipa{kɯ-rɤʑi}]	\ipa{nɯ}	\ipa{kɯ}	\ipa{pjɯ-mtsʰɤm}	\ipa{tɕe,}  \\
 \textsc{dem} \textsc{3sg}.\textsc{poss}-inside spider \textsc{nmlz}:S/A-remain \textsc{dem} \textsc{erg} \textsc{ipfv}-hear \textsc{lnk}  \\
 \glt  `The spider, which stays inside, feels it.' (26-mYaRmtsaR, 64)
\end{exe}

 \begin{exe}
\ex \label{ex:WkWtsxWB}
 \gll [\ipa{tɯ-xtsa}	\ipa{ɯ-kɯ-tʂɯβ}]	\ipa{nɯ}	\ipa{ɯ-pʰe}  \\
 \textsc{indef}.\textsc{poss}-shoe \textsc{3sg}.\textsc{poss}-\textsc{nmlz}:S/A-sew \textsc{dem} \textsc{3sg}.\textsc{poss}-\textsc{dat} \\
 \glt  `(He told) the shoe sewer.' (2003tWxtsa, 12)
\end{exe}

 \begin{exe}
\ex \label{ex:pWkABde}
 \gll [\ipa{ta-ʁi}	\ipa{pɯ-kɤ-βde}]	\ipa{nɯnɯ,}	[\ipa{pɯ-kɯ-si}]	\ipa{nɯnɯ}	\ipa{pɣɤtɕɯ}	\ipa{ci}	\ipa{to-sci,} \\
 \textsc{indef}.\textsc{poss}-younger.sibling \textsc{pfv}-\textsc{nmlz}:P-throw \textsc{dem} \textsc{pfv}-\textsc{nmlz}:S/A-die \textsc{dem} bird \textsc{indef} \textsc{ifr}-be.born \\ 
\glt  `The younger sister, who had been throw away, who had died, was reborn as a bird.' (qaCpa2002, 152) 
\end{exe}

Adjectival stative verbs need to take the participle \ipa{kɯ-} prefix to be used as attributes. The noun they modifiy is the head of the participial relative, which is generally head-internal as \ipa{wuma ʑo kʰa kɯ-ʑru} `very splendid house' (`house that is very splendid/luxurious') in (\ref{ex:kha.kWZru}), as shown by the place of the intensifier \ipa{wuma ʑo}.

 \begin{exe}
\ex \label{ex:kha.kWZru}
 \gll  [\ipa{wuma}	\ipa{ʑo}	\ipa{kʰa}	\ipa{kɯ-ʑru}]	\ipa{ɯ-ŋgɯ}	\ipa{nɯtɕu}	\ipa{tó-wɣ-tsɯm}	\ipa{tɕe} \\
 really \textsc{emph} house \textsc{nmlz}:S/A-be.strong \textsc{3sg}.\textsc{poss}-inside \textsc{dem}:\textsc{loc} \textsc{ifr}:\textsc{up}-\textsc{inv}-take.way \textsc{lnk} \\
\glt `The bird took her away to a splendid house (in heaven).' (2003zrAntCW tWrme, 74)
\end{exe}

In addition, there is handful of lexicalized participles, where the vowel in the prefix has been lost and which appear as fricativized preinitials \ipa{x-} or \ipa{ɣ-} in Japhug (see Table \ref{tab:irr.nmlz}  from \citealt[5]{jacques14antipassive}). Among these, \ipa{--ɣɲɟɯ} `opening, orifice'  has cognates in Stau and Khroskyabs (\citealt[609]{jacques17stau}).

\begin{table}[H]
\caption{Irregular nominalizations in \ipa{ɣ}-- and \ipa{x}--} \label{tab:irr.nmlz} \centering
\begin{tabular}{llll}
\toprule
 noun & meaning &base verb & meaning\\
\midrule
\ipa{\textbf{ɣ}ndʑɤβ} & `disastrous fire' & \ipa{ndʑɤβ} & `burn' \\
\ipa{--\textbf{ɣ}ɲaʁ}   &`disaster'& \ipa{ɲaʁ} & `be black' \\
\ipa{--\textbf{ɣ}ɲɟɯ}   & `orifice' & \ipa{ɲɟɯ} &  `open (vi)' \\
\ipa{--\textbf{x}so}   &  `empty thing' &\ipa{so} & `be empty' \\
\bottomrule
\end{tabular}
\end{table}


The velar participle prefixes found in Rgyalrong languages have cognate prefixes in many Trans-Himalayan languages, including Bodo-Garo, Jinghpo, Kuki-Chin, Karbi and Kiranti (\citealt{konnerth16gV, delancey15adjectival}).   In some languages, such as Limbu, the velar nominalization prefixes co-occur with a nominalization suffix. The active participle \ipa{kɛ-...-pa} serves as active participle (\citealt[199-202]{driem87}), as in (\ref{ex:kEdengba}), is an example of this type of construction.

\begin{exe}
\ex \label{ex:kEdengba}
 \gll \ipa{kɛ-de:ŋ-ba} \ipa{te:ʔl-in} \ipa{thund-u} \\
 \textsc{nmlz}-tear-\textsc{nmlz} clothes-\textsc{abs} mend-3P \\
 \glt `He mends torn clothes.' (\citealt[201]{driem87})
\end{exe}

\subsection{Sigmatic participles} \label{sec:rgyalrong.sigmatic}
In addition to the velar prefixes, we also find in Rgyalrong languages oblique participle prefixes (in Japhug \ipa{sɤ-}, \ipa{sɤz-} or \ipa{z-}) and related converbial forms (\citet{yanmuchu05sa}, \citealt{jackson14morpho}, \citealt{jacques16relatives}).


The oblique participles are fully productive, and can be applied to any verb, even stative verbs (\ref{ex:WsAdAn}). They can be converted into nouns (in particular for instruments, such as \ipa{sɤcɯ} `key' from the verb \ipa{cɯ} `open') even placenames (there is for instance in Kamnyu village a place called \ipa{znɤrɣɤma}, a lexicalized locative participle from the verb \ipa{nɤrɣɤma} `to call the rain'), but are most commonly used to build participial relatives, relativizing oblique arguments and adjuncts.

\begin{exe}
\ex \label{ex:WsAdAn}
\gll \ipa{stu} 	\ipa{ɯ-sɤ-dɤn} 	\ipa{nɯ} 	\ipa{stɤmku} 	\ipa{nɯra} 	\ipa{ŋu-nɯ}  \\
most \textsc{3sg}.\textsc{poss}-\textsc{nmlz}:\textsc{oblique}-be.many \textsc{dem} grassland \textsc{dem}:\textsc{pl} be:\textsc{fact}-\textsc{pl} \\
\glt `The (places) where they are the most numerous are the grasslands.' (19-qachGa mWntoR, 24)
\end{exe}

Among the syntactic roles that can be relativized with the oblique participle are instrumental adjuncts (\ref{ex:thongthar}), dative arguments (\ref{ex:WsAfCAt}), comitative arguments (\ref{ex:WsAmWmi}), time adjuncts (\ref{ex:WsAji}) and locative adjuncts / goals (\ref{ex:WsAdAn}, \ref{ex:asAGi}). Japhug text examples from each category are presented here to facilitate comparison with Tibetan and Chinese examples in the following sections.

 \begin{exe}
\ex \label{ex:thongthar}
\gll [\ipa{qandʑi}   	\ipa{cʰɯ-sɤ-ɣnda}]   	\ipa{nɯ}   	\ipa{tʰoŋtʰɤr}   	  	\ipa{ɲɯ-rmi}    \\
bullet \textsc{ipf}-\textsc{nmlz:oblique}-ram   \textsc{dem} ramrod \textsc{testim}-call \\
 \glt `What is used to ram a bullet (into the muzzle of the gun) is called a ramrod.' (28-CAmWGdW, 55)
 \end{exe}
 
 \begin{exe}
\ex \label{ex:WsAfCAt}
\gll
[\ipa{ɯ-sɤ-fɕɤt}] \ipa{pjɤ-me} 	\ipa{qʰe} 	\ipa{tɕe} 	\ipa{tɤ-pɤtso} 	\ipa{ɯ-ɕki} 	\ipa{nɯ} 	\ipa{tɕu} 	\ipa{nɯra} 	\ipa{tɕʰi} 	\ipa{pɯ-kɯ-fse} 	\ipa{nɯra} 	\ipa{pjɤ-fɕɤt.} \\
\textsc{3sg-nmlz:oblique}-tell \textsc{ipfv.ifr}-not.exist \textsc{lnk} \textsc{lnk} \textsc{indef.poss}-child \textsc{3sg-dat} \textsc{dem} \textsc{loc} \textsc{dem:pl} what \textsc{pst-nmlz:S}-be.like  \textsc{dem:pl} \textsc{ifr}-tell \\
\glt `She had no one (else) to tell it to, so she told the boy everything that had happened.' (140515 congming de wusui xiaohai, 77)
\end{exe} 

\begin{exe}
   \ex \label{ex:WsAmWmi}
 \gll 
\ipa{tɕe}   	[\ipa{ɯʑo}   	\ipa{ɯ-sɤ-ɤmɯmi}]  	\ipa{nɯ}   	\ipa{dɤn}   	\ipa{ma}   	\ipa{ca}   	\ipa{kɯ-fse}   	\ipa{qaʑo}   	\ipa{kɯ-fse,}   	\ipa{tsʰɤt}   	\ipa{kɯ-fse,}   	 \ipa{ɯʑo}   	\ipa{cʰo}   	\ipa{kɯ-naχtɕɯɣ}   	\ipa{sɯjno,}   	\ipa{xɕaj}   	\ipa{ma}   	\ipa{mɤ-kɯ-ndza}   	\ipa{nɯ} \ipa{ra}   	\ipa{cʰo}   	\ipa{nɯ}   	\ipa{amɯmi-nɯ}   	\ipa{tɕe,}   \\
\textsc{lnk} it \textsc{3sg-nmlz:oblique}-be.in.good.terms \textsc{dem} be.many:\textsc{fact} because musk.deer \textsc{nmlz:S}-be.like sheep \textsc{nmlz:S}-be.like goat  \textsc{nmlz:S}-be.like it with  \textsc{nmlz:S}-be.identical herbs grass apart.from \textsc{neg-nmlz:A}-eat \textsc{dem} \textsc{pl} with \textsc{dem} be.in.good.term:\textsc{fact}-\textsc{pl} \textsc{lnk} \\
\glt `The (animals) that are in good terms with the rabbit are many, it is in good terms with those that only eat grass, like musk deer, sheep or goats.' (04 qala1, 33-4)
\end{exe}

\begin{exe}
   \ex \label{ex:WsAji}
   \gll
   \ipa{tɕe} 	\ipa{nɯnɯ} 	\ipa{ʑaka} 	[\ipa{ɯ-sɤ-ji}] 	\ipa{ɲɯ-ŋu} 	\ipa{tɕe}\\
   \textsc{lnk} \textsc{dem} each \textsc{3sg-nmlz:oblique}-plant \textsc{sens}-be \textsc{lnk}\\
\glt `These are the (periods) when people plant each of these (crops).' (15 tChWma, 19)
\end{exe}

\begin{exe}
   \ex \label{ex:asAGi}
 \gll
\ipa{kɯki}   	\ipa{tɯ-ci}   	\ipa{ki}   	\ipa{ɯ-tɯ-rnaʁ}   	\ipa{mɯ́j-rtaʁ}   	\ipa{tɕe,}   	\ipa{aʑo}   	[\ipa{a-sɤ-ɣi}]   	\ipa{mɯ́j-kʰɯ}   \\
this \textsc{indef.poss}-water this \textsc{3sg-nmlz:degree}-deep \textsc{neg:sens}-deep \textsc{lnk} I \textsc{1sg-nmlz:oblique}-come \textsc{neg:sens}-be.able \\
\glt `The water is not deep enough, there is not (enough) place for me to come.' (2010-03, 4)
\end{exe}


There are several clues that the oblique participle prefixes in Rgyalrong languages are not recently innovated. First, outside of core Rgyalrong, there are fossilized traces of sigmatic nominalization prefixes with instrumental or locative value in Khroskyabs (\citealt[511]{lai17khroskyabs}) and Tangut (\citealt[256-257]{jacques14esquisse}). Second, within Rgyalrong, these prefixes have many allophones and there are several lexicalized nouns derived from oblique participles, even as first element of compounds, like \ipa{sɤqrɤcʰa} `alcohol to treat the guests' from \ipa{cʰa} `alcohol' and \ipa{sɤqrɤ-}, the status constructus of the oblique participle \ipa{sɤ-qru} of the verb \ipa{qru} `meet, greet (\zh{迎接})'. Third, there is no plausible source for this prefix, as if it were from a relator noun meaning `place' for instance, it should have been grammaticalized as a suffix.

\subsection{Sigmatic converbs} \label{sec:s.converbs}

Forms related to the sigmatic participles include purposive  converbs (\ref{ex:WmAYWsAjmWjmWt})  and gerunds (\ref{ex:sAlhWlhWoR} and \ref{ex:sAmtsWmtsWr}), but there are recently derived from the participles (\citealt[272-273]{jacques14linking}).

\begin{exe}
\ex \label{ex:WmAYWsAjmWjmWt}
\gll
[\ipa{kɯ-lɤɣ}   	\ipa{acɤβ}   	\ipa{nɯ}   	\ipa{kɯ}   	\ipa{\textbf{ɯ-mɤ-sɤ-jmɯ\textasciitilde{}jmɯt}},]   	\ipa{ɯ-pʰɯŋgɯ}   	\ipa{nɯ}   	\ipa{tɕu}   	\ipa{rdɤstaʁ-pɯpɯ}   	\ipa{tɕʰɯrdu}   	\ipa{ci}  \ipa{ɲɤ-rku,}\\
 \textsc{nmlz}:S/A-herd Askyabs \textsc{dem} \textsc{erg}  \textsc{3sg-neg-purp:conv}-forget \textsc{3sg.poss}-inside.clothes \textsc{dem} \textsc{loc} stone-little pebble \textsc{indef}
 \textsc{ifr}-put.in\\
\glt `The shepherd Askyabs put a little pebble inside his clothes so that he would not forget it.' (qaCpa, 166)
\end{exe}

The gerund generally means simultaneous action, without any obligatory argument coreference between the gerund clause and the main clause, as shown by (\ref{ex:sAlhWlhWoR}), where the verb in the gerund clause is intransitive, and its subject \ipa{ɯ-qom} `her tears' is not even an argument of the main clause (however, the transitive subject of the main clause is coreferent with the third person possessor of \ipa{ɯ-qom}).  

\begin{exe}
\ex \label{ex:sAlhWlhWoR}
\gll  \ipa{ɯ-qom} \ipa{sɤ-ɬɯ\textasciitilde{}ɬoʁ} \ipa{kɯ} \ipa{ɲɤ-mɟa} \ipa{tɕe} \\
\textsc{3sg}.\textsc{poss}-tear  \textsc{gerund}-come.out \textsc{erg} \textsc{ifr}-take \textsc{lnk} \\
\glt  `She took it (from hear mother's hand) while her tears flowed.' (140428 mu e guniang, 31)
\end{exe}


Most commonly however, the subject of the greund clause is coreferent with that of the finite verb in the main clause, as in (\ref{ex:sAmtsWmtsWr}). Note the absence of \textsc{1sg} person indexation on the gerund, as opposed to the finite verb \ipa{ku-rɤʑit-a}.

\begin{exe}
\ex \label{ex:sAmtsWmtsWr}
\gll  \ipa{kutɕu}  	\ipa{sɤ-mtsɯ\textasciitilde{}mtsɯr}  	\ipa{ku-rɤʑit-a}  	\ipa{tɕe,}  	\ipa{jisŋi}  	\ipa{ndɤ}  	\ipa{tɯmɯkɯmpɕi}  	\ipa{kɯ}  	\ipa{pɯ́-wɣ-nɯ-mbi-a}  	\ipa{ɕti}  \\
 here \textsc{gerund}-be.hungry \textsc{ipfv}-remain-\textsc{1sg} \textsc{lnk} today however heavens \textsc{erg} \textsc{pfv:down-inv-auto}-give-\textsc{1sg} be.\textsc{affirm}:\textsc{fact} \\
\glt `While I am staying here in hunger, today heavens have sent it (down) for  me (to eat).' (Norbzang 2005, 253)
\end{exe}

The gerund is also used to describe a background situation, as in (\ref{ex:sarkWrkWn}).

 \begin{exe}
\ex \label{ex:sarkWrkWn}
\gll 
\ipa{nɯ} \ipa{sɤ-rkɯ\textasciitilde{}rkɯn} \ipa{ʑo} \ipa{tɤ-pɤtso} \ipa{cʰɯ́-wɣ-tɕɤt} \ipa{pjɤ-ra} \ipa{tɕe},  \\
\textsc{dem} \textsc{gerund}-be.few \textsc{emph} \textsc{indef}.\textsc{poss}-child \textsc{ipfv}-\textsc{inv}-take.out \textsc{ifr}.\textsc{ipfv}-have.to \textsc{lnk} \\
\glt `One had to raise children with little (while food and clothes were few).' (140426 tApAtso kAnWBdaR, 5)
\end{exe}

The gerund also occurs in lexicalized expression with a specific meaning, for instance the gerund \ipa{sɤ-xtɕɯ-xtɕi} from \ipa{xtɕi} `be small' can mean `when X was young, since childhood'.



\section{Tibetan}
Tibetan,\footnote{This section presupposes accepted knowledge concerning Tibetan historical phonology and morphology (\citealt{lifk33, coblin76, hill11laws, hill14derivational}), and obvious alternations (such as aspiration) are not commented on. The transcription of Tibetan adopted follows \citet{jacques12transcription}.} like Khroskyabs, is a language where the prefixes corresponding to syllabic prefixes in core Rgyalrong languages have become simple consonants, without phonemic vowel. As a result of the dramatic syllabic contraction that occurred in proto-Tibetan, much of the archaic morphology has become obscured. 

\subsection{Velar nominalization} \label{sec:tibetan.velar}
The complementary distribution of velar \ipa{g-} and dental \ipa{d-} preradicals in Tibetan, an observation which \citet{hill11laws} ascribes to Saskya Paṇḍita but which was first pointed out in modern scholarship by \citet{lifk33}, suggests that velar presyllables have been dissimilated to dentals (\ipa{*kə-} $\rightarrow$ \ipa{d-}) before velars and labials, and that dental presyllables have been dissimilated to velars (\ipa{*tə-} $\rightarrow$ \ipa{g-}) before dentals. There is no way to distinguish between \ipa{*kə-} and \ipa{*tə-} presyllables from Tibetan alone, except before the initial \ipa{r-} (and perhaps \ipa{l-}), where no dissimilation took place.

A certain number of examples of \ipa{g-} or \ipa{d-} preinitials, often together with a \ipa{-n}, \ipa{-d} or \ipa{-s} suffix (forming a circumfix like the active participle in Limbu in §\ref{sec:rgyalrong.velar}) derive nouns from verbs or adjectives, with either action nominal or subject nominal meaning (\citealt{jacques14snom}). 

\begin{itemize}
\item \tibet{ནག་པོ་}{nag.po}{black} (root \dhat{nag}) $\rightarrow$	\tibet{གནག་}{gnag.pa}{black ox} 	
\item \tibet{ངུ་}{ŋu, ŋus}{cry} (root \dhat{ŋu}) $\rightarrow$	\tibet{དངུད་མོ་}{dŋud.mo}{sob, wail (n)} 	
\item \tibet{ངན་}{ŋan}{evil} (root \dhat{ŋan}) $\rightarrow$	\tibet{དངན་པ་}{dŋan.pa}{sorcery, evil} 	
\item \tibet{འཁྱིལ་}{ⁿkʰʲil}{gather (of water), whirl, twist round} (root \dhat{ŋan}) $\rightarrow$	\tibet{དཀྱིལ་}{dkʲil}{center}\footnote{The meaning `center' would be derived from a older meaning `confluence'. The verb  \tibet{འཁྱིལ་}{ⁿkʰʲil}{gather, whirl} is used for instance to refer to water gathering into a pond. }
\item \tibet{བླུ་}{blu, blus}{buy off, ransom} (root \dhat{lu}) $\rightarrow$	\tibet{གླུད་}{glud}{ransom ritual}\footnote{Concerning this ritual, see \citet{karmay91glud}.}
\item \tibet{ཡོ་}{jo}{crooked} (root \dhat{jo}) $\rightarrow$	\tibet{གཡོ་}{gjo}{deceit}
\item \tibet{ཉེ་}{ɲe}{near} (root \dhat{n(j)e}) $\rightarrow$	\tibet{གཉེན་}{gɲen}{relative; friend}
\item \tibet{ཉོ་}{ɲo}{buy} (root \dhat{ɲo}) $\rightarrow$	\tibet{གཉོད་}{gɲod}{price (of a bride)}
\item \tibet{མང་}{maŋ}{many} (root \dhat{maŋ}) $\rightarrow$	\tibet{དམངས་}{dmaŋs}{people}
\item \tibet{འཛིན་}{ⁿdzin, bzuŋ}{seize, grasp}(root \dhat{zuŋ}) $\rightarrow$	\tibet{གཟུངས་}{gzuŋs}{dhāraṇī}
\end{itemize}
 
The example \tibet{གཟུངས་}{gzuŋs}{dhāraṇī} is evidence that the \ipa{d/g-...-s} circumfix was still marginally productive in the imperial period, as it is calque from the Sanskrit \ipa{dhṛ} (the root from which \ipa{dhāraṇī} derives),  and must postdate the introduction of Buddhism.

It is conceivable that some of the \ipa{g-/d-} prefixes found in the Tibetan verbal system, in particular in the future tense, may be participial form that entered the finite system. This question goes however beyond the scope of this paper.

\subsection{Sigmatic nominalization}
Examples of oblique nominalization by sigmatic prefix in Tibetan are not many. The clearest ones, unsurprisingly, involve the instrumental nouns (the function illustrated by \ref{ex:thongthar} in Japhug). As in the case of the velar nominalization prefix, as noted above, the sigmatic nominalization prefix generally occurs together with a suffix \ipa{-d} or \ipa{-s}, as in \tibet{སྣོད་}{s-no-d}{vessel}.\footnote{Another derivation involving \ipa{s-...-d}   circumfixes are found in Tibetan: the collective noun derivation found with kinship terms such as \tibet{སྐུད་}{skud}{husband's male relatives} from  \tibet{ཁུ་}{kʰu}{father's brother}, \tibet{སྤུན་}{spun}{brothers} from  \tibet{ཕུ་ནུ་}{pʰu.nu}{elder and younger brothers} (\citealt{nagano94khu}). This formation is unrelated to the sigmatic nominalization described in this section.}

\begin{itemize}
\item \tibet{ནོད་}{nod, mnos}{receive} (root \dhat{no}) $\rightarrow$	\tibet{སྣོད་}{snod}{vessel} 	
\item \tibet{འབུད་}{ⁿbud, bus}{blow} (root \dhat{bu}) $\rightarrow$	\tibet{སྦུད་པ་}{sbud.pa}{bellows} 
\item \tibet{འགེལ་}{ⁿgel, bkal}{load on} (root \dhat{gal/kal}) $\rightarrow$	\tibet{སྒལ་}{sgal}{load, back} 
\item \tibet{ཉན་}{ɲan}{hear} $\rightarrow$	(root \dhat{ɲan}) \tibet{སྙན་}{sɲan}{ear (honorific)} 
\end{itemize}

The examples above are not problematic; note that in the case of honorific \tibet{སྙན་}{sɲan}{ear}, we possibly have a calque from Sanskrit \ipa{śrava-}, which is attested in the sense of `ear' in Classical Sanskrit. The adjective \tibet{སྙན་}{sɲan}{euphonious} is also derived from \tibet{ཉན་}{ɲan}{hear}, but through another \ipa{s-} prefix, the cognate of the deexperiencer prefix \ipa{sɤ-} in Japhug, which is found in examples such as \ipa{mtsʰɤm} `hear' $\rightarrow$  \ipa{sɤ-mtsʰɤm} `audible' (\citealt{jacques12demotion}).

Nouns of location derived by the prefix \ipa{s-} (corresponding to Japhug examples such as \ref{ex:WsAdAn} and \ref{ex:asAGi}) include the following:

\begin{itemize}
\item \tibet{འདིང་}{ⁿdiŋ, btiŋ}{lay out, spread out} (root \dhat{diŋ/tiŋ}) $\rightarrow$	\tibet{སྡིངས་}{sdiŋs}{flat surface} 
\item \tibet{འཁོར་}{dgar, bkar}{pitch (tent)} (as in \ipa{gur bkar} `pitch a tent', root \dhat{gar/kar}) $\rightarrow$	\tibet{སྒར་}{sgar}{encampment} 
\item \tibet{འཆད་}{ⁿtɕʰad, tɕʰad}{be cut, be broken off} (root \dhat{tɕʰad}) $\rightarrow$	\tibet{ཤད་}{ɕad}{division stroke} 
\item \tibet{འཁོར་}{ⁿkʰor}{turn, circumambulate} $\rightarrow$	\tibet{སྐོར་}{skor}{circle} 
\item \tibet{དམའ་}{dma}{low} $\rightarrow$	\tibet{སྨད་}{smad}{lower part}\footnote{In this example, the base adjective \tibet{དམའ་}{dma}{low} contains a \ipa{d-} from a velar presyllable \ipa{*kə-} which underwent dissimilation. Note also the adverb \tibet{མར་}{mar}{down} from the same root, with a terminative \ipa{-r} suffix. }
\item \tibet{མཐོ་}{mtʰo}{high} $\rightarrow$	\tibet{སྟོད་}{stod}{upper part}\footnote{This example suggests that the pre-Tibetan cluster \ipa{*smt-} was simplified as \ipa{st-}. }
\end{itemize}
 
In this list, note that the noun \tibet{སྐོར་}{skor}{circle} may be derived from the causative \tibet{སྐོར་}{skor}{cause to turn, surround, circumambulate} and therefore not be an example of sigmatic nominalization prefix. The same explanation is not possible for the other examples, however. 

The derivation of \tibet{ཤད་}{ɕad}{division stroke} from \tibet{འཆད་}{ⁿtɕʰad, tɕʰad}{be cut, be broken off} as proposed by \citet[141]{lifk33}\footnote{Note that while Li Fang-kuei's sound law *\ipa{s-tɕ-} $\rightarrow$ \ipa{ɕ-} is certainly correct (as pointed out by Abel Zadoks in an unpublished manuscript for instance, the numeral \ipa{ɲi.ɕu} `twenty' can be explained as an instance of this sound change, the proto-form being \ipa{*ɲis-tɕu}, perfectly parallel to \ipa{sum.tɕu} `thirty'), some of his examples have to be abandoned. For instance, he argues that \tibet{གཤམ་}{gɕam}{lower part, under} derives from  \ipa{tɕʰam} in the expression \tibet{ཆམ་ལ་འབེབས་}{tɕʰam-la ⁿbebs}{defeat completely}, a collocation containing the verb \tibet{འབེབས་}{ⁿbebs, pʰab}{cast down}). However, the syllable \ipa{tɕʰam} here is more plausibly related to the verb \tibet{འཇོམས་}{ⁿndʑoms, btɕoms}{subdue, destroy}. The noun \tibet{གཤམ་}{gɕam}{lower part, under} is more likely to be the exact cognate of Japhug \ipa{tɤ-zrɤm} `root' and Chinese \zh{參} \ipa{ʂim} $\leftarrow$ \ipa{*srəm} `plant root' (see \citealt{jacques15sr}), with the sound laws \ipa{*sr-} $\rightarrow$ \ipa{ɕ-} and \ipa{*tə-} $\rightarrow$ \ipa{g-} before coronal consonants (on the latter, see the discussion in § \ref{sec:tibetan.velar}); the \ipa{g-} would reflect the indefinite possessor prefix (the noun `root' being an inalienably possessed noun). } with a proto-form *\ipa{s-tɕad} $\rightarrow$ \ipa{ɕad} is a non-trivial example of sigmatic nominalization; this word would mean literally `breaking place (when reading)'.  

In addition to these two main categories, there are two potential isolated examples of sigmatic nominalization. First, the noun \tibet{སྐྱོན་}{skʲon}{fault} is clearly derived from the verb \tibet{བཀྱོན་}{bkʲon}{scold, reprimand} (whose \ipa{b-} may be a frozen past tense prefix); the semantic relation between the verb and the noun may be that of cause, a type not attested in Japhug. 

Another example is \tibet{སྐག་}{skag}{(astrological) hindrance, obstacle}, which could derive from the verb \tibet{འགོག་}{ⁿgog, bkag}{hinder},\footnote{Another possibility would be the adjective \tibet{ཁག་པོ་}{kʰag.po}{difficult, hard}, if ancient attestations can be brought to light. \citet[109]{uebach06woerterbuch02} provides numerous attestations of  \tibetan{ཀེག་} \ipa{keg}, \tibetan{ཀག་} \ipa{kag}, \tibetan{སྐེག་} \ipa{skeg}, \tibetan{སྐག་} \ipa{skag}, \tibetan{སྐྱེག་} \ipa{skʲeg} `kalendarisch ungünstige, gefährlische Zeit, Hindernis, drohendes Unglück'. The s-less forms are found as second members of compounds, as in \tibet{དགུང་ཀེག་}{dguŋ.keg}{(astrological) hindrance (honorific)}. Uebach cites \zhc{忌}{giH} `taboo, abstain from' apparently suggesting that the Tibetan word could be related; this is impossible for phonological reasons (voiced initial and absence of coda). } a temporal nominalization, like the Japhug example (\ref{ex:WsAji}). This is the only example from verbs with voicing alternation (on which see \citealt{jacques12internal} and \citealt{hill14voicing}) where the oblique nominalization is based on the voiced alternant.

 

\section{Old Chinese}
Reconstruction of morphology in Old Chinese is a much more delicate enterprise that in Tibetan, since even the mere existence of clusters has to be reconstructed (\citealt{gong17clusters}). Nevertheless, there are potential cases of nominalization prefixes in Old Chinese, whose interpretation depends on the reconstruction system followed.

\subsection{Velar nominalization}
Evidence for velar nominalization in Old Chinese is slim, but not non-existent. \citet[57]{bs14oc} suggest the following possibilities:

\begin{enumerate}
\item \zhc{方}{pjaŋ} ‘square’ (\ipa{*C-paŋ})  $\rightarrow$ \zhc{匡}{kʰjwaŋ} ‘square basket’   (\ipa{*k-pʰaŋ})
\item \zhc{明}{mjæŋ} ‘bright’ (\ipa{*mraŋ})  $\rightarrow$ \zhc{囧}{kjwæŋX} ‘bright window’   (\ipa{*k-mraŋʔ})
\item \zhc{威}{ʔjwɨj} ‘awe-inspiring’ (\ipa{*ʔuj})  $\rightarrow$ \zhc{鬼}{kjwɨjX} ‘ghost’   (\ipa{*k-ʔujʔ})
\end{enumerate}

It is however likely that some Old Chinese \ipa{*kə-} presyllables disappear without observable traces in Middle Chinese, as shown by the data in Table 4.47 in \citet[153]{bs14oc}, where data from ancient loanwords into Vietic and Lakkia demonstrate the presence of a velar preinitial element in words such has \zhc{賊}{dzok} `bandit' (Ruc \ipa{kəcʌk}). By consequence, most of potential traces of velar nominalization prefixes may have been lost by the effects of sound change, though the study of OC loanwords in Kra-Dai, Vietic and Hmong-Mien may provide us examples.

\subsection{Sigmatic nominalization}
\citet[73]{sagart99roc} proposed the following examples of \ipa{*s-} nominalization prefix (I keep here the origin reconstruction, not converting to the system of \citealt{bs14oc}, as none of these examples has become invalid in the new reconstruction).

\begin{enumerate}
\item \zhc{蒸}{tɕiŋ} `to steam' (\ipa{^b*tɨŋ}) $\rightarrow$ \zhc{甑}{tsiŋH} `earthenware pot for steaming rice' (\ipa{^b*s-tɨŋ-s})
\item \zhc{抴}{jet, jejH} `to pull' (\ipa{^b*lat(-s)}) $\rightarrow$ \zhc{靾}{sjet} `leading string' (\ipa{^b*s-lat})
\item \zhc{囓}{ŋet} `to bite, gnaw' (\ipa{^a*ŋet}) $\rightarrow$ \zhc{楔}{set} `wedge, piece of wood between the teeth of a corpse' (\ipa{^a*s-ŋet})
\item \zhc{射}{ʑek, ʑæH} `to shoot' (\ipa{^b*m-lak(-s)}; on the function of the \ipa{*-s} suffix here, see \citealt{jacques18antipass}) $\rightarrow$ \zhc{榭}{zjæH} `open hall for archery exercises' (\ipa{^b*s-lak-s}) 
\item \zhc{侍}{dʑiH} `to accompany, wait upon' (\ipa{^b*dɨ(ʔ)-s}) $\rightarrow$ \zhc{寺}{ziH} `servant, eunuch' (\ipa{^b*s-dɨ(ʔ)-s})
\item \zhc{食}{ʑik} `to eat' (\ipa{^b*m-lɨk}) $\rightarrow$ \zhc{食}{ziH} `food' (\ipa{^b*s-lɨk-s})
\end{enumerate}

Additional examples are presented in \citet[56]{bs14oc} and \citealt{sagart12sprefix}. Some of these nouns, for instance \zhc{星}{seŋ} `star’ and \zhc{席}{zjek} `mat’, are derived from a root that is only attested in derived verbal forms.

\begin{enumerate}
\item \zhc{屰}{ŋjæk} ‘go against, reverse’ (\ipa{*ŋrak})  $\rightarrow$ \zhc{朔}{ʂæwk} ‘first day of month’  (\ipa{*s-ŋrak})
\item \zhc{通}{tʰuŋ} ‘penetrate’ (\ipa{*l̥ˤoŋ}) $\rightarrow$ \zhc{窗}{tʂʰæwŋ} ‘window’ (\ipa{*s-l̥ˤ<r>oŋ})
\item \zhc{亡}{mjaŋ} ‘flee; disappear; die’ (\ipa{*maŋ}) $\rightarrow$ \zhc{喪}{saŋ} ‘mourning, burial’ (\ipa{*s-mˤaŋ})
\item \zhc{以}{jiX} ‘take, use’ (\ipa{*ləʔ}) $\rightarrow$ \zhc{鈶}{ziX} ‘handle of plow or sickle’ (\ipa{*sə.ləʔ})
\item \zhc{晴}{dzjeŋ} ‘clear (weather)’ (\ipa{*N-tsʰeŋ}), \zhc{清}{tsʰjeŋ} ‘clear'  $\rightarrow$ \zhc{星}{seŋ} `star’ (\ipa{*s-tsʰˤeŋ},  \citealt[139]{bs14oc})\footnote{Even if one accepts the sound change \ipa{*s-tsʰ-} $\rightarrow$ \ipa{s-}, analyzing \zhc{星}{seŋ} `star’ as a locative nominalization \ipa{*s-tsʰˤeŋ} from a root \ipa{*tsʰeŋ} attested by \zhc{晴}{dzjeŋ} ‘clear (weather)’ and \zhc{清}{tsʰjeŋ} ‘clear' is problematic. Although the graphs belong to the same phonetic series and the affricate onset is supported by Min data, the semantic difference is considerable, since these adjectives never mean `bright' and are not associated to stars. The only way to salvage the hypothesis would be to suppose a locational/temporal noun `clearing (in the night sky)' $\Rightarrow$ `starry sky' (night sky without any cloud), from which `star' would be a singulative.
The meaning `starry sky' for \zhc{星}{seŋ} may be attested, as in the following passage from the poem \zh{定之方中} Ding Zhi Fang Zhong (50) in the Shijing: \zh{靈雨既零、命彼倌人。星言夙駕、說于桑田。} `When the good rains had fallen, He would order his groom, \textbf{By starlight}, in the morning, to yoke his carriage, And would then stop among the mulberry trees and fields.' (translation by Legge). \citet[33]{karlgren74odes} translates it as `when it cleared during the night, early he yoked his carriage'. } 
\item \zhc{署}{dʑoH} ‘to place; position’ (\ipa{*m-taʔ-s}),  \zhc{緒}{zjoX} ‘arrange in order’ (\ipa{*s-m-taʔ}),  \zhc{著}{ʈjak} ‘to place’ (\ipa{**t<r>ak }) $\rightarrow$  \zhc{席}{zjek} `mat’ (\ipa{*s-m-tAk},  \citealt[61]{bs14oc})
\item \zhc{除}{ɖrjo} ‘remove’ (\ipa{*[l]<r>a}) $\rightarrow$ \zhc{鋤}{dʐjo} ‘hoe’ (\ipa{*s-[l]<r>a}, \citealt[81]{bs14oc})
\item \zhc{匀}{jwin} ‘even, uniform’ (\ipa{*[N-q]ʷi[n]}) $\rightarrow$ \zhc{旬}{zwin} ‘ten-day cycle’ (\ipa{*s-N-qʷi[n]}, \citealt[127]{bs14oc})
\item \zhc{處}{tɕʰoX} ‘be at’ (\ipa{*t.qʰaʔ}) $\rightarrow$ \zhc{所}{ʂjoX} ‘place, nominalizer’ (\ipa{*s-qʰ<r>aʔ}, \citealt[130]{bs14oc})
\item \zhc{圓}{hjwen} ‘round’ (\ipa{*ɢʷ<r>en}) $\rightarrow$ \zhc{旋}{zjwenH} ‘whorl of hair on the head’ (\ipa{*s-ɢʷen-s}, \citealt[141]{bs14oc})\footnote{The word \zhc{旋}{zjwenH} is more likely to be a nominalization by \textit{qùshēng} from the verb \zhc{旋}{zjwen} `revolve, turn' (on which see \citealt{downer59, jacques16ssuffixes}). }
\item \zhc{尼}{nejX} ‘to stop’ (\ipa{*nˤərʔ}) $\rightarrow$ \zhc{西}{sej} ‘west’, \zhc{棲}{sej} ‘bird’s nest’  (\ipa{*s-nˤər}, \citealt[147]{bs14oc})\footnote{This example, also discussed in \citet{sagart04directions}, would have a direct Japhug equivalent \ipa{sɤz-nɯna} `resting place'. These are however not real cognate, since the Japhug word can be productively formed from the verb  \ipa{nɯna} `to rest'.}
\end{enumerate}
 
  
While some of sound changes involved are not universally accepted, in particular \ipa{*s-tsʰ-} $\rightarrow$ \ipa{s-} and Li Fang-kuei's (\citealt{lifk71shanggu}) idea of \ipa{*sN-} $\rightarrow$ \ipa{*s-} (see the debate between  \citealt{mei12caus} and \citealt{sagart12sprefix}), the change \ipa{*sl-} $\rightarrow$ \ipa{z-} in B type syllables is least controversial.\footnote{On the voicing of preinitial \ipa{*s-} in contact with a voiced lateral, see the typological discussion in \citet{gong16ld}, with evidence from Rgyalrongic and Tibetic languages.} Even if we exclude all examples with controversial onsets for the sake of argument, we still have good examples of locative (\zhc{榭}{zjæH}), comitative (\zhc{寺}{ziH}) and instrumental (\zhc{鈶}{ziX}) nominalizations, comparable to Japhug examples (\ref{ex:WsAdAn}, \ref{ex:asAGi}), (\ref{ex:WsAmWmi}) and (\ref{ex:thongthar}) respectively. The case of \zhc{食}{ziH} `food' is more doubtful because there is a causative \zhc{食}{ziH} `feed' from which it could derive by zero derivation, and also because an oblique nominalization of `eat' should rather mean `eating place' or `instrument used for eating', not `food'.

These three examples of sigmatic nominalization  (\zhc{榭}{zjæH}, \zhc{寺}{ziH},  \zhc{鈶}{ziX}) are not the only ones that seem relatively straightforward. Without committing to a particular reconstruction system, if we accept Li Fang-kuei's (\citealt{lifk71shanggu}) reconstruction of s + nasal onsets, and Bodman's (\citeyear{bodman69sdud}) hypothesis of OC dental affricates originating from clusters in \ipa{*s-} (\ipa{*s-T-} $\rightarrow$ \ipa{*TS-}), three examples of instrumental nominalization (\zhc{楔}{set},   \zhc{甑}{tsiŋH} and \zhc{鋤}{dʐjo} ‘hoe’) and three examples of temporal/locative nominalization (\zhc{朔}{ʂæwk}, \zhc{喪}{saŋ}, \zhc{棲}{sej}) are fairly convincing.

 A possible example of sigmatic converb (parallel to the gerund in Japhug, § \ref{sec:s.converbs} in examples such as \ref{ex:sAlhWlhWoR} to \ref{ex:sarkWrkWn}) in Old Chinese is the conjunction \zhc{雖}{swij} `although'. This conjunction is already attested in the Shijing as in (\ref{ex:sui}),  where it occurs in opposition to the copula \zhc{維}{jwij}  in the main clause.
 
 \begin{exe}
\ex \label{ex:sui}
 \glt \zh{周雖舊邦,其命維新} 
 \glt `Though Chow is an old state, its (heavenly) appointment is new.' (235; Daya, Wenwang, \citealt[185]{karlgren74odes})
\end{exe}

An etymological relationship between \zhc{雖}{swij} `although' and \zhc{維}{jwij} `be' is likely given their occurrence in the same phonetic series. A copula with a converbial \ipa{*s-} prefix would mean `while being XXX', a form that can have a concessive meaning in appropriate context (in the same way that English `while', originally a temporal conjunction, has also become concessive) and could have become restricted in this usage.

Baxter reconstruct \zhc{雖}{swij} as \ipa{*s-qʷij} and \zhc{維}{jwij} as  \ipa{*ɢʷij}, and do not imply a morphological reconstruction between the two; in their system, the expected outcome of \ipa{*s-ɢʷij} would be $\dagger$\ipa{zwij}. There are several ways around this problem; \citet{jacques00ywij} reconstructs \zhc{雖}{swij} as \ipa{*s-tə-wuj} and  \zhc{維}{jwij} as  \ipa{*tə-wuj}, using a system based on \citet{sagart99roc}; in this hypothesis, the presence of a \ipa{*tə-} preinitial, reconstructed here to account for the xiesheng relationship with \zhc{推}{tʰwoj} (from \ipa{*tʰˁuj}), accounts for the absence of voicing. If the reconstruction of a preinitial in this word is not accepted, it remains possible to suppose that, given the fact that this converbial prefix was probably not lexicalized at the same time as the other \ipa{*s-} prefixes, different sound laws apply.

 Although the existence of sigmatic nominalization in Old Chinese is less immediately obvious than in Tibetan, the number of examples is of a comparable order.

\section{Conclusion}
The Tibetan and Chinese data presented in this paper support the idea that the sigmatic prefixes of Rgyalrong languages are not a Rgyalrongic innovation, but rather the preservation of  prefixes that used to exist in the literary languages, but of which only fossil traces remain in the earliest attested stages of these languages. It is likely that traces of the same prefixes can be found elsewhere in the Trans-Himalayan family, though only few branches preserve unequivocal traces of the preinitials.\footnote{In particular, a possibly related nominalization prefix, which has the allomorphs \ipa{chya-}, \ipa{sha-} and \ipa{sa-}, is found in Jinghpo (\citealt[3-4]{dai90yufa}). However, it does not derive oblique nominals, and this comparison requires more investigations. }

It is hoped that the Japhug data provided in this paper will be useful to researchers of Old Chinese and Tibetan to look for additional examples of sigmatic oblique nominalization, and better evaluate the precise semantics of proposed etymologies: since the sigmatic participles are fully productive in Japhug, this language allows a finer-grained understanding of the use of this derivation than language where only fossilized remnants remain. 

The morphology-rich Rgyalrong languages have a role in the reconstruction of Trans-Himalayan morphology comparable to that of Sanskrit in Indo-European and Arabic in Semitic: their exuberant productivity offers an living model to build hypotheses on the traces of morphology in lesser-endowed languages.

\bibliographystyle{unified}
\bibliography{bibliogj}
\end{document}
