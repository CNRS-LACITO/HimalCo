\documentclass[oneside,a4paper,11pt]{article} 
\usepackage{fontspec}
\usepackage{natbib}
\usepackage{booktabs}
\usepackage{xltxtra} 
\usepackage{polyglossia} 
\usepackage[table]{xcolor}
\usepackage{tikz}
\usetikzlibrary{trees}
\usepackage{gb4e} 
\usepackage{multicol}
\usepackage{graphicx}
\usepackage{float}
\usepackage{hyperref} 
\hypersetup{bookmarksnumbered,bookmarksopenlevel=5,bookmarksdepth=5,colorlinks=true,linkcolor=blue,citecolor=blue}
\usepackage[all]{hypcap}
\usepackage{memhfixc}
\usepackage{lscape}
\usepackage{amssymb}
 
\bibpunct[: ]{(}{)}{,}{a}{}{,}

%\setmainfont[Mapping=tex-text,Numbers=OldStyle,Ligatures=Common]{Charis SIL} 
\newfontfamily\phon[Mapping=tex-text,Scale=MatchLowercase]{Charis SIL} 
\newcommand{\ipa}[1]{\textbf{{\phon\mbox{#1}}}} %API tjs en italique
%\newcommand{\ipab}[1]{{\scriptsize \phon#1}} 

\newcommand{\grise}[1]{\cellcolor{lightgray}\textbf{#1}}
\newfontfamily\cn[Mapping=tex-text,Ligatures=Common,Scale=MatchUppercase]{SimSun}%pour le chinois
\newcommand{\zh}[1]{{\cn #1}}
\newfontfamily\mleccha[Mapping=tex-text,Ligatures=Common,Scale=MatchLowercase]{Galatia SIL}%pour le grec
\newcommand{\grec}[1]{{\mleccha #1}}
\newfontfamily\tibetain{Microsoft Himalaya} 

\newcommand{\sg}{\textsc{sg}}
\newcommand{\pl}{\textsc{pl}}
\newcommand{\ro}{$\Sigma$}
\newcommand{\ra}{$\Sigma_1$} 
\newcommand{\rc}{$\Sigma_3$}  
\newcommand{\dhatu}[2]{|\ipa{#1}| `#2'}
\newcommand{\tibet}[3]{{\tibetain#1} \textit{\phon#2} `#3'}  
\newcommand{\dhat}[1]{|\ipa{#1}|}
\newcommand{\change}[2]{*\ipa{#1} $\rightarrow$ \ipa{#2}}
 

\XeTeXlinebreakskip = 0pt plus 1pt %
 %CIRCG
 
\newcommand{\zhc}[2]{\zh{#1} \ipa{#2}} 


\begin{document}

\title{Fossil nominalization prefixes in Tibetan and Chinese}
\author{Guillaume Jacques}
\maketitle

\section{Introduction}
%The fact that Gyalrong languages preserve many morphological archaisms is relatively (\citealt{gong17xingtaixue})

\section{Participles in Rgyalrongic}
Gyalrongic languages have a set of prefixes deriving non-finite verb forms, including participles, converbs and infinitives (\citealt{jacques14linking}, \citealt{jackson14morpho}, \citealt{jacques16relatives}). In the present paper, two sets of participles, the velar (core argument) and sigmatic (oblique) participles are discussed.


\subsection{Velar participles}
All Gyalrong languages (Japhug, Tshobdun, Zbu and Situ) all have a set participle prefixes in \ipa{kV-} used in particular to build participial relative clauses with subject or object relativization. There are slight difference between the languages (\citealt{jackson06guanxiju, jacksonlin07, jacques16relatives, zhang16bragdbar}); this paper only includes data from Japhug, which are sufficient to illustrate the constructions shared by all Rgyalrong languages.\footnote{The participle prefixes are historically related to generic person and infinitive prefixes (\citealt{sun12complementation}, \citealt{sun14generic}, \citealt{jacques16complementation}, \citealt{jacques18generic}), but these are not discussed in this paper. }

The core argument participle prefixes in Japhug are \ipa{kɯ-} for subject (S/A) participle and \ipa{kɤ-} for object participle, and can be illustrated by examples (\ref{ex:kWrAZi}) (intransitive subject, in a non-restrictive head-internal relative) (\ref{ex:WkWtsxWB}) (transitive subject)  and XXX

 \begin{exe}
\ex \label{ex:kWrAZi}
 \gll [\ipa{nɯ}	\ipa{ɯ-ŋgɯ}	\ipa{ɴɢoɕna}	\ipa{kɯ-rɤʑi}]	\ipa{nɯ}	\ipa{kɯ}	\ipa{pjɯ-mtsʰɤm}	\ipa{tɕe,}  \\
 \textsc{dem} \textsc{3sg}.\textsc{poss}-inside spider \textsc{nmlz}:S/A-remain \textsc{dem} \textsc{erg} \textsc{ipfv}-hear \textsc{lnk}  \\
 \glt  `The spider, which stays inside, feels it.' (26-mYaRmtsaR, 64)
\end{exe}

 \begin{exe}
\ex \label{ex:WkWtsxWB}
 \gll tɯ-xtsa ɯ-kɯ-tʂɯβ nɯ ɯ-pʰe,  \\
 \textsc{indef}.\textsc{poss}-shoe \textsc{3sg}.\textsc{poss}-\textsc{nmlz}:S/A-sew \textsc{dem} \textsc{3sg}.\textsc{poss}-\textsc{dat} \\
 \glt  `(He told) the shoe sewer.' (2003tWxtsa, 12)
\end{exe}

\begin{table}[H]
\caption{Irregular nominalizations in \ipa{ɣ}-- and \ipa{x}--} \label{tab:irr.nmlz} \centering
\begin{tabular}{llll}
\toprule
 noun & meaning &base verb & meaning\\
\midrule
\ipa{\textbf{ɣ}ndʑɤβ} & disastrous fire & \ipa{ndʑɤβ} & burn \\
\ipa{--\textbf{ɣ}ɲaʁ}   &disaster& \ipa{ɲaʁ} & be black \\
\ipa{--\textbf{ɣ}ɲɟɯ}   & orifice & \ipa{ɲɟɯ} & be opened \\
\ipa{--\textbf{x}so}   &  empty thing &\ipa{so} & be empty \\
\bottomrule
\end{tabular}
\end{table}

 

which have cognate prefixes in many Trans-Himalayan languages, including Karbi and Kiranti (\citealt{konnerth16gV}) and  

\subsection{Sigmatic participles}
In addition to the velar prefixes, we also find in Gyalrong languages oblique participle prefixes (in Japhug \ipa{sɤ-}, \ipa{sɤz-} or \ipa{z-}) and related converbial forms (\citet{yanmuchu05sa}, \citealt{jackson14morpho}, \citealt{jacques16relatives}).


The oblique participles are fully productive, and can be applied to any verb, even stative verbs (\ref{ex:WsAdAn}). They can be converted into nouns (in particular for instruments, such as \ipa{sɤcɯ} `key' from the verb \ipa{cɯ} `open') even placenames (there is for instance in Kamnyu village a place called \ipa{znɤrɣɤma}, a lexicalized locative participle from the verb \ipa{nɤrɣɤma} `to call the rain'), but are most commonly used to build participial relatives, relativizing oblique arguments and adjuncts.

\begin{exe}
\ex \label{ex:WsAdAn}
\gll \ipa{stu} 	\ipa{ɯ-sɤ-dɤn} 	\ipa{nɯ} 	\ipa{stɤmku} 	\ipa{nɯra} 	\ipa{ŋu-nɯ}  \\
most \textsc{3sg}.\textsc{poss}-\textsc{nmlz}:\textsc{oblique}-be.many \textsc{dem} grassland \textsc{dem}:\textsc{pl} be:\textsc{fact}-\textsc{pl} \\
\glt `The (places) where they are the most numerous are the grasslands.' (19-qachGa mWntoR, 24)
\end{exe}

Among the syntactic roles that can be relativized with the oblique participle are instrumental adjuncts (\ref{ex:thongthar}), dative arguments (\ref{ex:WsAfCAt}), comitative arguments (\ref{ex:WsAmWmi}), time adjuncts (\ref{ex:WsAji}) and locative adjuncts / goals (\ref{ex:WsAdAn}, \ref{ex:asAGi}). Japhug text examples from each category are presented here to facilitate comparison with Tibetan and Chinese examples in the following sections.

 \begin{exe}
\ex \label{ex:thongthar}
\gll [\ipa{qandʑi}   	\ipa{cʰɯ-sɤ-ɣnda}]   	\ipa{nɯ}   	\ipa{tʰoŋtʰɤr}   	  	\ipa{ɲɯ-rmi}    \\
bullet \textsc{ipf}-\textsc{nmlz:oblique}-ram   \textsc{dem} ramrod \textsc{testim}-call \\
 \glt `What is used to ram a bullet (into the muzzle of the gun) is called a ramrod.' (28-CAmWGdW, 55)
 \end{exe}
 
 \begin{exe}
\ex \label{ex:WsAfCAt}
\gll
[\ipa{ɯ-sɤ-fɕɤt}] \ipa{pjɤ-me} 	\ipa{qʰe} 	\ipa{tɕe} 	\ipa{tɤ-pɤtso} 	\ipa{ɯ-ɕki} 	\ipa{nɯ} 	\ipa{tɕu} 	\ipa{nɯra} 	\ipa{tɕʰi} 	\ipa{pɯ-kɯ-fse} 	\ipa{nɯra} 	\ipa{pjɤ-fɕɤt.} \\
\textsc{3sg-nmlz:oblique}-tell \textsc{ipfv.ifr}-not.exist \textsc{lnk} \textsc{lnk} \textsc{indef.poss}-child \textsc{3sg-dat} \textsc{dem} \textsc{loc} \textsc{dem:pl} what \textsc{pst-nmlz:S}-be.like  \textsc{dem:pl} \textsc{ifr}-tell \\
\glt `She had no one (else) to tell it to, so she told the boy everything that had happened.' (140515 congming de wusui xiaohai, 77)
\end{exe} 

\begin{exe}
   \ex \label{ex:WsAmWmi}
 \gll 
\ipa{tɕe}   	[\ipa{ɯʑo}   	\ipa{ɯ-sɤ-ɤmɯmi}]  	\ipa{nɯ}   	\ipa{dɤn}   	\ipa{ma}   	\ipa{ca}   	\ipa{kɯ-fse}   	\ipa{qaʑo}   	\ipa{kɯ-fse,}   	\ipa{tsʰɤt}   	\ipa{kɯ-fse,}   	 \ipa{ɯʑo}   	\ipa{cʰo}   	\ipa{kɯ-naχtɕɯɣ}   	\ipa{sɯjno,}   	\ipa{xɕaj}   	\ipa{ma}   	\ipa{mɤ-kɯ-ndza}   	\ipa{nɯ} \ipa{ra}   	\ipa{cʰo}   	\ipa{nɯ}   	\ipa{amɯmi-nɯ}   	\ipa{tɕe,}   \\
\textsc{lnk} it \textsc{3sg-nmlz:oblique}-be.in.good.terms \textsc{dem} be.many:\textsc{fact} because musk.deer \textsc{nmlz:S}-be.like sheep \textsc{nmlz:S}-be.like goat  \textsc{nmlz:S}-be.like it with  \textsc{nmlz:S}-be.identical herbs grass apart.from \textsc{neg-nmlz:A}-eat \textsc{dem} \textsc{pl} with \textsc{dem} be.in.good.term:\textsc{fact}-\textsc{pl} \textsc{lnk} \\
\glt `The (animals) that are in good terms with the rabbit are many, it is in good terms with those that only eat grass, like musk deer, sheep or goats.' (04 qala1, 33-4)
\end{exe}

\begin{exe}
   \ex \label{ex:WsAji}
   \gll
   \ipa{tɕe} 	\ipa{nɯnɯ} 	\ipa{ʑaka} 	[\ipa{ɯ-sɤ-ji}] 	\ipa{ɲɯ-ŋu} 	\ipa{tɕe}\\
   \textsc{lnk} \textsc{dem} each \textsc{3sg-nmlz:oblique}-plant \textsc{sens}-be \textsc{lnk}\\
\glt `These are the (periods) when people plant each of these (crops).' (15 tChWma, 19)
\end{exe}

\begin{exe}
   \ex \label{ex:asAGi}
 \gll
\ipa{kɯki}   	\ipa{tɯ-ci}   	\ipa{ki}   	\ipa{ɯ-tɯ-rnaʁ}   	\ipa{mɯ́j-rtaʁ}   	\ipa{tɕe,}   	\ipa{aʑo}   	[\ipa{a-sɤ-ɣi}]   	\ipa{mɯ́j-kʰɯ}   \\
this \textsc{indef.poss}-water this \textsc{3sg-nmlz:degree}-deep \textsc{neg:sens}-deep \textsc{lnk} I \textsc{1sg-nmlz:oblique}-come \textsc{neg:sens}-be.able \\
\glt `The water is not deep enough, there is not (enough) place for me to come.' (2010-03, 4)
\end{exe}

Forms related to the oblique participles include gerunds (\ref{ex:sAmtsWmtsWr}) and purposive  (\ref{ex:sAmtsWmtsWr}) converbs, but there are recently derived from the participles (\citealt[272-273]{jacques14linking}).

\begin{exe}
\ex \label{ex:sAmtsWmtsWr}
\gll  \ipa{kutɕu}  	\ipa{sɤ-mtsɯ\textasciitilde{}mtsɯr}  	\ipa{ku-rɤʑit-a}  	\ipa{tɕe,}  	\ipa{jisŋi}  	\ipa{ndɤ}  	\ipa{tɯmɯkɯmpɕi}  	\ipa{kɯ}  	\ipa{pɯ́-wɣ-nɯ-mbi-a}  	\ipa{ɕti}  \\
 here \textsc{gerund}-be.hungry \textsc{ipfv}-remain-\textsc{1sg} \textsc{lnk} today however heavens \textsc{erg} \textsc{pfv:down-inv-auto}-give-\textsc{1sg} be.\textsc{affirm}:\textsc{fact} \\
\glt `I am very hungry here, but heavens have sent it (down) for  me (to eat).' (Norbzang 2005, 253)
\end{exe}

\begin{exe}
\ex \label{ex:WmAYWsAjmWjmWt}
\gll
[\ipa{kɯ-lɤɣ}   	\ipa{acɤβ}   	\ipa{nɯ}   	\ipa{kɯ}   	\ipa{\textbf{ɯ-mɤ-sɤ-jmɯ\textasciitilde{}jmɯt}},]   	\ipa{ɯ-pʰɯŋgɯ}   	\ipa{nɯ}   	\ipa{tɕu}   	\ipa{rdɤstaʁ-pɯpɯ}   	\ipa{tɕʰɯrdu}   	\ipa{ci}  \ipa{ɲɤ-rku,}\\
 \textsc{nmlz}:S/A-herd Askyabs \textsc{dem} \textsc{erg}  \textsc{3sg-neg-purp:conv}-forget \textsc{3sg.poss}-inside.clothes \textsc{dem} \textsc{loc} stone-little pebble \textsc{indef}
 \textsc{ifr}-put.in\\
\glt `The shepherd Askyabs put a little pebble inside his clothes so that he would not forget it.' (qaCpa, 166)
\end{exe}

In Gyalrong languages other than Japhug, the oblique participle prefixes have very similar functions, though a comparative survey is still lacking. 

There are several clues that the oblique participle prefixes in Gyalrong languages are not recently innovated. First, outside of core Gyalrong, there are fossilized traces of sigmatic nominalization prefixes with instrumental or locative value in Khroskyabs (\citealt[511]{lai17khroskyabs}) and Tangut (\citealt[256-257]{jacques14esquisse}). Second, within Gyalrong, these prefixes have many allophones and there are several lexicalized nouns derived from oblique participles, even as first element of compounds, like \ipa{sɤqrɤcʰa} `alcohol to treat the guests' from \ipa{cʰa} `alcohol' and \ipa{sɤqrɤ-}, the status constructus of the oblique participle \ipa{sɤ-qru} of the verb \ipa{qru} `meet, greet (\zh{迎接})'. Third, there is no plausible source for this prefix, as if it were from a relator noun meaning `place' for instance, it should have been grammaticalized as a suffix.

\section{Tibetan}
Tibetan,\footnote{This section presupposes accepted knowledge concerning Tibetan historical phonology and morphology (\citealt{lifk33, coblin76, hill11laws, hill14derivational}), and obvious alternations (such as aspiration) are not commented on. The transcription of Tibetan adopted follows \citet{jacques12transcription}.} like Khroskyabs, is a language where the prefixes corresponding to syllabic prefixes in core Gyalrong languages have become simple consonants, without phonemic vowel. As a result of the dramatic syllabic contraction that occurred in proto-Tibetan, much of the archaic morphology has become obscured. Fortunately, sigmatic prefixes have been less affected by Tibetan-internal changes than velar prefixes.

\subsection{Velar nominalization}




have fossil reflexes in Tibetan (as \ipa{g-} or \ipa{d-} preinitials, see \citealt{jacques14snom}).Circumfixal nominalization with a velar nominalization prefix  is a feature also found in Limbu, where the circumfixal form \ipa{kɛ-...-pa} serves as active participle (\citealt[199]{driem87}), as in (\ref{ex:kEdengba}).

\begin{exe}
\ex \label{ex:kEdengba}
 \gll \ipa{kɛ-de:ŋ-ba} \ipa{te:ʔl-in} \ipa{thund-u} \\
 \textsc{nmlz}-tear-\textsc{nmlz} clothes-\textsc{abs} mend-3P \\
 \glt `He mends torn clothes.' (\citealt[201]{driem87})
\end{exe}
\begin{itemize}
\item \tibet{ནག་པོ་}{nag.po}{black} (root \dhat{nag}) $\rightarrow$	\tibet{གནག་}{gnag.pa}{black ox} 	
\item \tibet{ངུ་}{ŋu, ŋus}{cry} (root \dhat{ŋu}) $\rightarrow$	\tibet{དངུད་མོ་}{dŋud.mo}{sob, wail (n)} 	
\item \tibet{ངན་}{ŋan}{evil} (root \dhat{ŋan}) $\rightarrow$	\tibet{དངན་པ་}{dŋan.pa}{sorcery, evil} 	
\item \tibet{འཁྱིལ་}{ⁿkʰʲil}{gather, whirl, twist round} (root \dhat{ŋan}) $\rightarrow$	\tibet{དཀྱིལ་}{dkʲil}{center}
\item \tibet{བླུ་}{blu, blus}{buy off, ransom} (root \dhat{lu}) $\rightarrow$	\tibet{གླུད་}{glud}{ransom ritual}
\item \tibet{ཡོ་}{jo}{crooked} (root \dhat{jo}) $\rightarrow$	\tibet{གཡོ་}{gjo}{deceit}
\item \tibet{ཉེ་}{ɲe}{near} (root \dhat{n(j)e}) $\rightarrow$	\tibet{གཉེན་}{gɲen}{relative; friend}
\item \tibet{ཉོ་}{ɲo}{buy} (root \dhat{ɲo}) $\rightarrow$	\tibet{གཉོད་}{gɲod}{price (of a bride)}
\item \tibet{མང་}{maŋ}{many} (root \dhat{maŋ}) $\rightarrow$	\tibet{དམངས་}{dmaŋs}{people}
\item \tibet{འཛིན་}{ⁿdzin, bzuŋ}{seize, grasp}(root \dhat{zuŋ}) $\rightarrow$	\tibet{གཟུངས་}{gzuŋs}{dhāraṇī}
\end{itemize}
 
The example \tibet{གཟུངས་}{gzuŋs}{dhāraṇī} is evidence that the d/g- prefix was still mariginally productive in the imperial period, as it is calque from the Sanskrit \ipa{dhṛ} (the root from which \ipa{dhāraṇī} derives),  and must postdate the introduction of Buddhism.


\subsection{Sigmatic nominalization}
Examples of oblique nominalization by sigmatic prefix in Tibetan are not many. The clearest ones, unsurprisingly, involve the instrumental nouns (the function illustrated by \ref{ex:thongthar} in Japhug). As in the case of the velar nominalization prefix, as noted above, the sigmatic nominalization prefix generally occurs together with a suffix \ipa{-d} or \ipa{-s}, as in \tibet{སྣོད་}{s-no-d}{vessel}.

\begin{itemize}
\item \tibet{ནོད་}{nod, mnos}{receive} (root \dhat{no}) $\rightarrow$	\tibet{སྣོད་}{snod}{vessel} 	
\item \tibet{འབུད་}{ⁿbud, bus}{blow} (root \dhat{bu}) $\rightarrow$	\tibet{སྦུད་པ་}{sbud.pa}{bellows} 
\item \tibet{འགེལ་}{ⁿgel, bkal}{load on} (root \dhat{gal/kal}) $\rightarrow$	\tibet{སྒལ་}{sgal}{load, back} 
\item \tibet{ཉན་}{ɲan}{hear} $\rightarrow$	(root \dhat{ɲan}) \tibet{སྙན་}{sɲan}{ear(honorific)} 
\end{itemize}

The examples above are not problematic; note that in the case of honorific \tibet{སྙན་}{sɲan}{ear}, we possibly have a calque from Sanskrit \ipa{śrava-}, which is attested in the sense of `ear' in Classical Sanskrit. The adjective \tibet{སྙན་}{sɲan}{euphonious} XXX

mtho, stod dma, smad

Nouns of location derived by the prefix \ipa{s-} (corresponding to Japhug examples such as \ref{ex:WsAdAn} and \ref{ex:asAGi}) include the following:
\begin{itemize}
\item \tibet{འདིང་}{ⁿdiŋ, btiŋ}{lay out, spread out} (root \dhat{diŋ/tiŋ}) $\rightarrow$	\tibet{སྡིངས་}{sdiŋs}{flat surface} 
\item \tibet{འཁོར་}{dgar, bkar}{pitch (tent)} (as in \ipa{gur bkar} `pitch a tent', root \dhat{gar/kar}) $\rightarrow$	\tibet{སྒར་}{sgar}{encampment} 
\item \tibet{འཆད་}{ⁿtɕʰad, tɕʰad}{be cut, be broken off} (root \dhat{tɕʰad}) $\rightarrow$	\tibet{ཤད་}{ɕad}{division stroke} 
\item \tibet{འཁོར་}{ⁿkʰor}{turn, circumambulate} $\rightarrow$	\tibet{སྐོར་}{skor}{circle} 
\end{itemize}

In this list, note that the noun \tibet{སྐོར་}{skor}{circle} may be derived from the causative \tibet{སྐོར་}{skor}{cause to turn, surround, circumambulate} and therefore not be an example of sigmatic nominalization prefix. The same explanation is not possible for the other examples, however. 

The derivation of \tibet{ཤད་}{ɕad}{division stroke} from \tibet{འཆད་}{ⁿtɕʰad, tɕʰad}{be cut, be broken off} uses Li's (\citeyear[141]{lifk33}) sound law *\ipa{s-tɕad} $\rightarrow$ \ipa{ɕad}; this word would mean literally `breaking place (when reading)'. XXXX gsham

In addition to these two main categories, there are two potential isolated examples of sigmatic nominalization. First, the noun \tibet{སྐྱོན་}{skʲon}{fault} is clearly derived from the verb \tibet{བཀྱོན་}{bkʲon}{scold, reprimand} (whose \ipa{b-} may be a frozen past tense prefix); the semantic relation between the verb and the noun may be that of cause, a type not attested in Japhug. 

Another example is \tibet{ལོ་སྐག་}{lo.skag}{unlucky year} (also spelled \tibet{ལོ་སྐག་}{lo.skeg}{unlucky year}), which could be from the adjective \tibet{ཁག་པོ་}{kʰag.po}{difficult, hard}, or alternatively from the verb \tibet{འགོག་}{ⁿgog, bkag}{hinder}. In both cases, it would be a temporal nominalization, like the Japhug example (\ref{ex:WsAji}). This is the only example from verbs with voicing alternation (on which see \citealt{jacques12internal} and \citealt{hill14voicing}) where the oblique nominalization is based on the voiced alternant.


\section{Old Chinese}
Reconstruction of morphology in Old Chinese is a much more delicate enterprise that in Tibetan, since even the mere existence of clusters has to be reconstructed (\citealt{gong17clusters}). Nevertheless, there are potential case of sigmatic nominalization in Old Chinese, whose interpret naturally depend on the reconstruction system followed.

\citet[73]{sagart99roc} proposed the following examples of \ipa{*s-} nominalization prefix (I keep here the origin reconstruction, not converting to the system of \citealt{bs14oc}; Middle Chinese is in an IPAnized version of Baxter's \citealt{baxter92} transcription).

\begin{enumerate}
\item \zhc{蒸}{tɕiŋ} `to steam' (\ipa{^b*tɨŋ}) $\rightarrow$ \zhc{甑}{tsiŋH} `earthenware pot for steaming rice' (\ipa{^b*s-tɨŋ-s})
\item \zhc{抴}{jet, jejH} `to pull' (\ipa{^b*lat(-s)}) $\rightarrow$ \zhc{靾}{sjet} `leading string' (\ipa{^b*s-lat})
\item \zhc{囓}{ŋet} `to bite, gnaw' (\ipa{^a*ŋet}) $\rightarrow$ \zhc{楔}{set} `wedge, piece of wood between the teeth of a corpse' (\ipa{^a*s-ŋet})
\item \zhc{射}{ʑek, ʑæH} `to shoot' (\ipa{^b*m-lak(-s)}; on the function of the \ipa{*-s} suffix here, see \citealt{jacques18antipass}) $\rightarrow$ \zhc{榭}{zjæH} `open hall for archery exercises' (\ipa{^b*s-lak-s}) 
\item \zhc{侍}{dʑiH} `to accompany, wait upon' (\ipa{^b*dɨ(ʔ)-s}) $\rightarrow$ \zhc{寺}{ziH} `servant, eunuch' (\ipa{^b*s-dɨ(ʔ)-s})
\item \zhc{食}{ʑik} `to eat' (\ipa{^b*m-lɨk}) $\rightarrow$ \zhc{食}{ziH} `food' (\ipa{^b*s-lɨk-s})
\end{enumerate}
%席 *s-m-tAk > zjek > xí ‘mat’ (< ‘where one puts things’?)61
%署 *m-taʔ-s > dzyoH > shǔ ‘to place; position’
%緒 *s-m-taʔ > zjoX > xù ‘arrange in order’
%著 *t<r>ak > trjak > zhuó ‘to place’
%鋤 *s-[l]<r>a > *s-d<r>a > *dzra > dzrjo > chú ‘hoe’; cf.
%除 *[l] <r>a > *dra > drjo > chú ‘remove’ 81
%匀 *[N-q]ʷi[n] > ywin > yún ‘even, uniform’
%旬 *s-N-qʷi[n] > zwin > xún ‘ten-day cycle’ 127
%所 *s-qʰ<r>aʔ > srjoX > suǒ ‘place (n.); that which’
%處 *t.qʰaʔ > tsyhoX > chǔ ‘be at’ 130
%旋 *s-ɢʷen-s > (*zɢʷen-s >) zjwenH > xuàn ‘whorl of hair on the head’;
%the root is
%圜, 圓 *ɢʷ<r>en > hjwen > yuán ‘round’ 141
%尼 *nˤərʔ > nejX > ní ‘to stop’ (intransitive?)
%尼 *nˤərʔ-s > nejH > ní ‘to stop, obstruct’ (transitive?); cf.
%柅 *n<r>[ə]rʔ > nrijX > nǐ ‘a stopper for carriages’
%西 *s-nˤər > *sˤər > sej > xī ‘(place for stopping:) west’; the same
%word as
%棲 *s-nˤər > sej > qī ‘bird’s nest’ 147

Additional examples are presented in \citet[56, 139]{bs14oc} and \citealt{sagart12sprefix}:

\begin{enumerate}
\item \zhc{屰}{ŋjæk} ‘go against, reverse’ (\ipa{*ŋrak})  $\rightarrow$ \zhc{朔}{ʂæwk} ‘first day of month’  (\ipa{*s-ŋrak})
\item \zhc{通}{tʰuŋ} ‘penetrate’ (\ipa{*l̥ˤoŋ}) $\rightarrow$ \zhc{窗}{tʂʰæwŋ} ‘window’ (\ipa{*s-l̥ˤ<r>oŋ})
\item \zhc{亡}{mjaŋ} ‘flee; disappear; die’ (\ipa{*maŋ}) $\rightarrow$ \zhc{喪}{saŋ} ‘mourning, burial’ (\ipa{*s-mˤaŋ})
\item \zhc{以}{jiX} ‘take, use’ (\ipa{*ləʔ}) $\rightarrow$ \zhc{鈶}{ziX} ‘handle of plow or sickle’ (\ipa{*sə.ləʔ})
\item \zhc{晴}{dzjeŋ} ‘clear (weather)’ (\ipa{*N-tsʰeŋ}) $\rightarrow$ \zhc{星}{seŋ} `star’ (\ipa{*s-tsʰˤeŋ})\footnote{Even if one accepts the sound change \ipa{*s-tsʰ-} $\rightarrow$ \ipa{s-}, analyzing \zhc{星}{seŋ} `star’ as a locative nominalization \ipa{*s-tsʰˤeŋ} from a root \ipa{*tsʰeŋ} attested by \zhc{晴}{dzjeŋ} ‘clear (weather)’ and \zhc{清}{tsʰjeŋ} ‘clear' is problematic. Although the graphs belong to the same phonetic series, the semantic difference is considerable, since these adjectives never mean `bright' and are not associated to stars. The only way to salvage the hypothesis would be to suppose a locational noun `the clear place' $\Rightarrow$ `starry sky' (night sky without any cloud), from which `star' would be a singulative.} 
%灵雨既零、命彼倌人。
%星言夙驾、说于桑田。
%匪直也人、秉心塞渊、騋牝三千。
%Ding Zhi Fang Zhong:	
%When the good rains had fallen,
%He would order his groom,
%By starlight, in the morning, to yoke his carriage,
%And would then stop among the mulberry trees and fields.
%But not only thus did he show what he was; -
%Maintaining in his heart a profound devotion to his duties,
%His tall horses and mares amounted to three thousand.
\end{enumerate}

While some of sound changes involved are not universally accepted, in particular \ipa{*s-ts-} $\rightarrow$ \ipa{s-} and Li Fang-kuei's (\citealt{lifk71shanggu}) idea of \ipa{*sN-} $\rightarrow$ \ipa{*s-} (see the debate between  \citealt{mei12caus} and \citealt{sagart12sprefix}), the change \ipa{*sl-} $\rightarrow$ \ipa{z-} in B type syllables is least controversial.\footnote{On the voicing of preinitial \ipa{*s-} in contact with a voiced lateral, see the typological discussion in \citet{gong16ld}, with evidence from Gyalrongic and Tibetic languages.} Even if we exclude all examples with controversial onsets for the sake of argument, we still have good examples of locative (\zhc{榭}{zjæH}), comitative (\zhc{寺}{ziH}) and instrumental (\zhc{鈶}{ziX}) nominalizations, comparable to Japhug examples (\ref{ex:WsAdAn}, \ref{ex:asAGi}), (\ref{ex:WsAmWmi}) and (\ref{ex:thongthar}) respectively. 

The case of \zhc{食}{ziH} `food' is more doubtful because there is a causative \zhc{食}{ziH} `feed' from which it could derive by zero derivation, and also because an oblique nominalization of `eat' should rather mean `eating place' or `instrument used for eating', not `food'.

If we accept Li Fang-kuei's (\citealt{lifk71shanggu}) reconstruction of s + nasal onsets, we have additional examples of instrumental nominalization (\zhc{楔}{set}) and temporal nominalization (\zhc{朔}{ʂæwk}, \zhc{喪}{saŋ}), further strengthening this hypothesis.



 Although the existence of sigmatic nominalization in Old Chinese is less immediately obvious than in Tibetan, the number of examples is of a comparable order.

\section{Conclusion}
The Tibetan and Chinese data presented in this paper support the idea that the sigmatic oblique particle prefix of Rgyalrong languages is not a Rgyalrongic innovation, but rather the preservation of a prefix that used to exist in the literary languages, but of which only fossil traces remain in the earliest attested stages of these languages. It is likely that traces of the same prefix can be found elsewhere in the Trans-Himalayan family, though only few branches preserve unequivocal traces of the preinitials.\footnote{In particular, a possibly related nominalization prefix, which has the allomorphs \ipa{chya-}, \ipa{sha-} and \ipa{sa-}, is found in Jinghpo (\citealt[3-4]{dai90yufa}). However, it does not derive oblique nominals, and this comparison requires more investigations. }

It is hoped that the Japhug data provided in this paperwill be useful to researchers of Old Chinese and Tibetan to look for additional examples of sigmatic oblique nominalization, and better evaluate the precise semantics of proposed etymologies: since the sigmatic participles are fully productive in Japhug, this language allows a finer-grained understanding of the use of this derivation than language where only fossilized remnants remain.

\bibliographystyle{unified}
\bibliography{bibliogj}
\end{document}
