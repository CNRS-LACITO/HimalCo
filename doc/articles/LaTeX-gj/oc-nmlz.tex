\documentclass[oneside,a4paper,11pt]{article} 
\usepackage{fontspec}
\usepackage{natbib}
\usepackage{booktabs}
\usepackage{xltxtra} 
\usepackage{polyglossia} 
\usepackage[table]{xcolor}
\usepackage{tikz}
\usetikzlibrary{trees}
\usepackage{gb4e} 
\usepackage{multicol}
\usepackage{graphicx}
\usepackage{float}
\usepackage{hyperref} 
\hypersetup{bookmarksnumbered,bookmarksopenlevel=5,bookmarksdepth=5,colorlinks=true,linkcolor=blue,citecolor=blue}
\usepackage[all]{hypcap}
\usepackage{memhfixc}
\usepackage{lscape}
\usepackage{amssymb}
 
\bibpunct[: ]{(}{)}{,}{a}{}{,}

%\setmainfont[Mapping=tex-text,Numbers=OldStyle,Ligatures=Common]{Charis SIL} 
\newfontfamily\phon[Mapping=tex-text,Scale=MatchLowercase]{Charis SIL} 
\newcommand{\ipa}[1]{\textbf{{\phon\mbox{#1}}}} %API tjs en italique
%\newcommand{\ipab}[1]{{\scriptsize \phon#1}} 

\newcommand{\grise}[1]{\cellcolor{lightgray}\textbf{#1}}
\newfontfamily\cn[Mapping=tex-text,Ligatures=Common,Scale=MatchUppercase]{SimSun}%pour le chinois
\newcommand{\zh}[1]{{\cn #1}}
\newfontfamily\mleccha[Mapping=tex-text,Ligatures=Common,Scale=MatchLowercase]{Galatia SIL}%pour le grec
\newcommand{\grec}[1]{{\mleccha #1}}
\newfontfamily\tibetain{Microsoft Himalaya} 

\newcommand{\sg}{\textsc{sg}}
\newcommand{\pl}{\textsc{pl}}
\newcommand{\ro}{$\Sigma$}
\newcommand{\ra}{$\Sigma_1$} 
\newcommand{\rc}{$\Sigma_3$}  
\newcommand{\dhatu}[2]{|\ipa{#1}| `#2'}
\newcommand{\tibet}[3]{{\tibetain#1} \textit{\phon#2} `#3'}  
\newcommand{\dhat}[1]{|\ipa{#1}|}
\newcommand{\change}[2]{*\ipa{#1} $\rightarrow$ \ipa{#2}}
 

\XeTeXlinebreakskip = 0pt plus 1pt %
 %CIRCG
 
\newcommand{\zhc}[2]{\zh{#1} \ipa{#2}} 


\begin{document}

\title{Fossil nominalization prefixes in Tibetan and Chinese}
\author{Guillaume Jacques}
\maketitle

\section{Introduction}

\citet{gong17xingtaixue}
\citet{konnerth16gV}
\citet{jacques14snom}

\section{Rgyalrongic}
\citet{jacques16relatives}
\citet{jacques14linking}
\citet{jackson14morpho}

\section{Tibetan}
\citet{lifk33, coblin76, hill11laws, hill14derivational}


Instruments:
\begin{itemize}
\item \tibet{ནོད་}{nod, mnos}{receive} $\rightarrow$	\tibet{སྣོད་}{snod}{vessel} 	(\ipa{bhâjana-})
\item \tibet{འབུད་}{ⁿbud, bus}{blow} $\rightarrow$	\tibet{སྦུད་པ་}{sbud.pa}{bellows} 
\item \tibet{འགེལ་}{ⁿgel, bkal}{load on} $\rightarrow$	\tibet{སྒལ་}{sgal}{load, back} 
\item \tibet{ཉན་}{ɲan}{hear} $\rightarrow$	\tibet{སྙན་}{sɲan}{ear(honorific)} 
\end{itemize}
%ⁿdiŋ, btiŋ	étendre	/tiŋ/	sdiŋs  	cavité	endroit
%dgar, bkar 	monter une tente	/kar/	sgar 	tente	endroit
%ⁿkʰor 	tourner	/kor/	skor	région 
%(< encercler)	endroit
%kʰag(-po)	difficile, dur	/kag/	skag 	calamit¨¦ (lo.skag = ann¨¦e de malheur)	temps
%bkʲon 	reprocher	/kʲon/	skʲon 	faute	raison

\section{Old Chinese}
\citet[73]{sagart99roc}, \citet{sagart12sprefix}

\begin{itemize}
\item \zhc{蒸}{tɕiŋ} `to steam' (\ipa{^b*tɨŋ}) $\rightarrow$ \zhc{甑}{tsiŋH} `earthenware pot for steaming rice' (\ipa{^b*s-tɨŋ-s})
\item \zhc{抴}{jet, jejH} `to pull' (\ipa{^b*lat(-s)}) $\rightarrow$ \zhc{靾}{sjet} `leading string' (\ipa{^b*s-lat})
\item \zhc{囓}{ŋet} `to bite, gnaw' (\ipa{^a*ŋet}) $\rightarrow$ \zhc{楔}{set} `wedge' (\ipa{^a*s-ŋet})
\item \zhc{射}{ʑek, ʑæH} `to shoot' (\ipa{^b*m-lak}) $\rightarrow$ \zhc{榭}{zjæH} `open hall for archery exercises' (\ipa{^b*s-lak-s}) (\citealt{jacques18antipass})
\item \zhc{侍}{dʑiH} `to accompany, wait upon' (\ipa{^b*dɨ(ʔ)-s}) $\rightarrow$ \zhc{寺}{ziH} `servant, eunuch' (\ipa{^b*s-dɨ(ʔ)-s})
\item \zhc{食}{ʑik} `to eat' (\ipa{^b*m-lɨk}) $\rightarrow$ \zhc{食}{ziH} `food' (\ipa{^b*s-lɨk-s})
\end{itemize}

\section{Conclusion}

\bibliographystyle{unified}
\bibliography{bibliogj}
\end{document}
