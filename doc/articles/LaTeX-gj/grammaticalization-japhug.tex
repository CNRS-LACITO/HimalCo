\documentclass[oldfontcommands,oneside,a4paper,11pt]{article} 
\usepackage{fontspec}
\usepackage{natbib}
\usepackage{booktabs}
\usepackage{xltxtra} 
\usepackage{polyglossia} 
\usepackage[table]{xcolor}

\usepackage{multicol}
\usepackage{graphicx}
\usepackage{float}
\usepackage{hyperref} 
\hypersetup{bookmarks=false,bookmarksnumbered,bookmarksopenlevel=5,bookmarksdepth=5,xetex,colorlinks=true,linkcolor=blue,citecolor=blue}
\usepackage[all]{hypcap}
\usepackage{memhfixc}
\usepackage{lscape}
\usepackage{tikz}
%
\usetikzlibrary{trees}
\usepackage{gb4e} 
\bibpunct[: ]{(}{)}{,}{a}{}{,}
 
%\setmainfont[Mapping=tex-text,Numbers=OldStyle,Ligatures=Common]{Charis SIL}  
\newfontfamily\phon[Mapping=tex-text,Ligatures=Common,Scale=MatchLowercase,FakeSlant=0.3]{Charis SIL} 
\newcommand{\ipa}[1]{{\phon #1}} %API tjs en italique
 
\newcommand{\grise}[1]{\cellcolor{lightgray}\textbf{#1}}
\newfontfamily\cn[Mapping=tex-text,Ligatures=Common,Scale=MatchUppercase]{MingLiU}%pour le chinois
\newcommand{\zh}[1]{{\cn #1}}
   

\begin{document} 
 \title{Grammaticalization in Japhug and Gyalrongic languages}
 \author{Guillaume Jacques}
 \maketitle  
 \section{Introduction}

 \section{Nominal categories} 
 
 No class/gender
 %d.	Determiners
\subsection{Number} 
Japhug has two number clitics, dual \ipa{ni} and plural \ipa{ra}. These clitics are not obligatory for non-singular arguments (even in the case of human referents), and do not necessary trigger plural or dual agreement on the verb. The dual \ipa{ni} is obviously related to the numeral \ipa{ʁnɯz} < *\ipa{q-nis} `two', exemplifying the well-attested pathway \textsc{two} $\rightarrow$ \textsc{dual}. The etymology of the plural clitic \ipa{ra} is unknown.
 
 
 The plural clitic \ipa{ra} can express plurality or collective meaning, as in example \ref{ex:ra1}; it is not incompatible with numerals, as shown in \ref{ex:ra2}.
 
\begin{exe}
\ex \label{ex:ra1}
\gll \ipa{ɯ-kha} 	\ipa{ra} 	\ipa{nɯ-mɤ-kɤ-sɯz} 	\ipa{nɤ} 	\ipa{ɯʑo} 	\ipa{kɯ} 	\ipa{qɤjɣi} 	\ipa{χsɯm} 	\ipa{lo-βzu} \\
\textsc{3sg.poss}-house \textsc{pl} \textsc{3pl.poss-neg-inf}-know \textsc{lnk} \textsc{3sg} \textsc{erg} bread three \textsc{ifr}-make \\
\glt She made three pieces of bread without her relatives knowing. (the raven, 108)
\end{exe}

\begin{exe}
\ex \label{ex:ra2}
\gll 
 \ipa{rɟɤlpu} 	\ipa{nɯ} 	\ipa{kɯ} 	\ipa{kɯki} 	\ipa{tɯrme} 	\ipa{kɯtʂɤɣ} 	\ipa{ra} 	\ipa{nɯ-ɕki}  	\ipa{to-ti} \\
 king \textsc{dem} \textsc{erg} this people six \textsc{pl} \textsc{3pl-dat} \textsc{ifr}-say  \\
 \glt The king told these six men. (Liuhaohan zoubian tianxia, 200)
\end{exe}

In addition, the marker \ipa{ra} can approximative time or location  (see section  \ref{sec:loc}).
 


 \subsection{Possession and pronouns} 
Japhug  nouns can be divided into two main categories, inalienably possessed nouns and alienably possessed nouns. The former must take a possessive prefix, even when the possessor is non-specific. In that case, the indefinite possessor \ipa{tɯ--/tɤ--} or generic possessor \ipa{tɯ--} prefixes are used.

Pronouns  and possessive prefixes are very similar (see Table \ref{tab:pronoun}), but it appears that in Japhug pronouns are derived from possessive prefixes rather than the opposite: pronouns other than \textsc{3du} and \textsc{3pl} are build by combining the possessive prefix with the pronominal root \ipa{-ʑo}.  

The \textsc{3du} and \textsc{3pl} combine the \textit{status constructus} form of that pronominal root \ipa{ʑɤ--} with the dual and plural clitic \ipa{ni} and \ipa{ra}.
 
 
\begin{table}[H] \centering
\caption{Pronouns and possessive prefixes }\label{tab:pronoun}
\begin{tabular}{lllllllll} 
\toprule
 Free pronoun & Prefix & Person\\
\midrule
 \ipa{a-ʑo},    \ipa{ɤj} &	\ipa{a--}  &		1\textsc{sg} \\
\ipa{nɤ-ʑo},  \ipa{nɤj} &	\ipa{nɤ--}  &			2\textsc{sg}\\
\ipa{ɯ-ʑo}  &	\ipa{ɯ--}  &			3\textsc{sg}\\
\midrule
\ipa{tɕi-ʑo}  &	\ipa{tɕi--}  &			1\textsc{du} \\
\ipa{ndʑi-ʑo}  &	\ipa{ndʑi--}  &		2\textsc{du} \\	
\ipa{ʑɤ-ni}  &	\ipa{ndʑi--}  &		3\textsc{du} \\	
\midrule
\ipa{i-ʑo}    &	\ipa{i--}  &			1\textsc{pl} \\
\ipa{nɯ-ʑo}   &	\ipa{nɯ--}  &			2\textsc{pl} \\
\ipa{ʑa-ra}  &	\ipa{nɯ--}  &			3\textsc{pl} \\
\midrule
&  \ipa{tɯ--},  \ipa{tɤ--} & indefinite \\
\ipa{tɯ-ʑo} & \ipa{tɯ--}   &  generic\\
\bottomrule
\end{tabular}
\end{table}
 
 

 
 \subsection{Ergative and genitive}  \label{sec:erg}
Japhug has borrowed its ergative  and genitive clitics \ipa{kɯ} and \ipa{ɣɯ} from Amdo Tibetan.  The ergative is highly polyfunctional,  and appears in several types of subordinate clauses as a linker (see \ref{sec:adverbial}). In the comparative construction, the ergative is unexpectedly used to mark the \textit{comparee} rather than the standard, as in \ref{ex:comp1} (\citealt{jacques15comparative}). This is all the more surprising that the ergative \ipa{kə/ɣə} in Amdo Tibetan from which the Japhug ergative \ipa{kɯ} was borrowed is used for the standard.

\begin{exe}
\ex \label{ex:comp1}
\gll  [\ipa{ɯ-ʁi}]_{standard}   	\ipa{sɤz}   	[\ipa{ɯ-pi}   	\ipa{nɯ}]_{comparee}   	\ipa{\textbf{kɯ}}   	\ipa{mpɕɤr}     \\
\textsc{3sg.poss}-younger.sibling \textsc{comparative} \textsc{3sg.poss}-elder.sibling \textsc{dem} \textsc{erg}  be.beautiful:\textsc{fact} \\
\glt `The elder one is more beautiful than the young one.' (elicited)
\end{exe}

The genitive \ipa{ɣɯ}, in addition to possessor, is the normal way to mark beneficiaries and can be used to mark recipient  (although for the latter, the dative is more often used), as in \ref{ex:aZWG.nWkhAm}
\begin{exe}
\ex \label{ex:aZWG.nWkhAm}
\gll
\ipa{ki} 	\ipa{kɯra} 	\ipa{ɲɯ-kham-a} 	\ipa{tɕe} 	\ipa{ki} 	\ipa{nɤki} 	\ipa{nɯ} 	\ipa{aʑɯɣ} 	\ipa{nɯ-khɤm.} \\
this \textsc{dem:pl} \textsc{ipfv}-give[III]-\textsc{1sg} \textsc{lnk} this \textsc{dem} \textsc{dem} \textsc{1sg:gen} \textsc{imp}-give[III] \\
\glt I will give you this, and you will give me that. (slobdpon, 130)
\end{exe}

 \subsection{Dative}  \label{sec:dat}
The dative \ipa{ɯ-phe} or \ipa{ɯ-ɕki} derives from a relator noun 
 \subsection{Location}   \label{sec:loc}
In Japhug, there are four distinct (non mutually exclusive) ways of marking locative adjuncts. First, locative and temporal adjunct are commonly left unmarked. Second, they can  take the locative case markers \ipa{zɯ} or \ipa{tɕu}. Third, relator (possessed) nouns such as \ipa{ɯ-ŋgɯ} `inside', \ipa{ɯ-taʁ} `on' etc can be used for more specific locations. 

Fourth, the plural clitic \ipa{ra} can indicate approximate location, as in \ref{ex:khara}. This use of \ipa{ra} is reminiscent of plural markers in Kirghiz and Old Japanese, which combine collective, hypocoristic and approximate locative meanings see \citealt[195]{antonov07ra}).

\begin{exe}
\ex \label{ex:khara}
\gll
\ipa{tɯ-zda} 	\ipa{nɯ} 	\ipa{ma} 	\ipa{kɯmaʁ} 	\ipa{tɯrme} 	\ipa{a-pɯ-me} 	\ipa{tɕe,} 	\ipa{kha} 	\ipa{ra} 	\ipa{aʁɤndɯndɤt} \ipa{ɲɯ-ɤnɯɣro} 	\ipa{ɲɯ-ŋu} 	\ipa{ɲɯ-ti.} \\
\textsc{indef.poss}-companion \textsc{dem} apart.from other people \textsc{irr-ipfv}-not.exist \textsc{lnk} house \textsc{pl} everywhere \textsc{sens}-play \textsc{sens}-be \textsc{sens}-say \\
\glt He says that (the young monkey) would play everywhere in the house whenever there are no other people (apart from members of the family). (19 GZW2, 10)
\end{exe}

 

 \subsection{Topic and focus} 

The temporal adverb \ipa{jiɕqha} `just before' has become grammaticalized as a pre-nominal determiner `the aforementioned' expressing that the nominal in question has been referred to previously in the discourse, though not in the last few sentences. In example \ref{ex:jiCqha}, for instance, the leaf is mentioned four sentences before.

\begin{exe}
\ex \label{ex:jiCqha}
 \gll \ipa{tɯmɯ} 	\ipa{ci} 	\ipa{tɕhɤrnaʁ} 	\ipa{ci} 	\ipa{tɕhɯmtɕhɯm} 	\ipa{ko-lɤt.} 	\ipa{tɕendɤre} 	\ipa{jiɕqha} 	\ipa{tɤ-jwaʁ} 	\ipa{nɯ} 	\ipa{pjɤ-nɯndzom} 	\ipa{tɕe,} 	\ipa{ɯ-ʁi} 	\ipa{ɯ-kɯr} 	\ipa{ɯ-ŋgɯ} 	\ipa{nɯ} 	\ipa{tɕu} 	\ipa{tɯ-ci} 	\ipa{χsɯ-ntɕhaʁ} 	\ipa{jamar} 	\ipa{pjɤ-ɕe.} \\
 rain \textsc{indef}  rain \textsc{indef} \textsc{idph}.II:little.rain \textsc{ifr}-throw \textsc{lnk} the.aforementioned \textsc{indef.poss}-leaf \textsc{dem} \textsc{ifr:down}-flow.along \textsc{lnk} \textsc{3sg.poss}-younger.sibling  \textsc{3sg.poss}-mouth \textsc{3sg}-inside \textsc{dem} \textsc{loc} \textsc{indef.poss}-water three-drop about \textsc{ifr:down}-go \\
 \glt There was a little rain, and (the water) flowed along the leaf (that the elder brother had placed) and three drops of water flowed into his younger brother's mouth. (Smanmi 11, 61)
\end{exe}

 
The clitic \ipa{rcanɯ} emphasizes the unexpectedness of the situation or event described by the phrase that follows. It is often used to express high degree
The noun phrase preceding \ipa{rcanɯ} is topical.
\begin{exe}
\ex
\gll \ipa{tɕe} 	\ipa{nɯ} 	\ipa{ɕoŋtɕa} 	\ipa{rcanɯ} 	\ipa{kɯ-dɯ\textasciitilde{}dɤn} 	\ipa{ʑo} 	\ipa{pjɤ-sɯ-phɯt-nɯ.} \\
\textsc{lnk} wood \textsc{foc:unexpected} \textsc{nmlz}:S/A-\textsc{emph}\textasciitilde{}-be.a.lot \textsc{emph} \textsc{ifr-caus}-chop-\textsc{pl}\\
\glt And they had (people) chop a lot of wood (for them) (28 qAjdo,  103)
\end{exe}

This focus clitic is derived from the possessed noun  \ipa{ɯ-rca} `following, together with' (see example \ref{ex:arca}) together with the distal demonstrative \ipa{nɯ} `that'.
\begin{exe}
\ex \label{ex:arca}
\gll
\ipa{aʑo} 	\ipa{a-rca} 	\ipa{kɤ-ɣi} 	\ipa{mɤ-tɯ-cha} \\
\textsc{1sg} \textsc{1sg}-following \textsc{inf}-come \textsc{neg}-2-can:\textsc{fact} \\
\glt You cannot come with me.
\end{exe}
 

  \section{Verbal categories} 
  %a.	Voice/valency
%b.	Aspect
%c.	Modality (dynamic)
%d.	Tense
%e.	Mood
%f.	Agreement (subject/object agreement)
%g.	other

\subsection{Associated motion}
Japhug has a simple associated motion system, with one translocative / andative prefix \ipa{ɕɯ--} and a cislocative / venitive prefix \ipa{ɣɯ--} transparently grammaticalized from the verbs \ipa{ɕe} `go' and \ipa{ɣi} `come' respectively. These prefixes are morphologically fully integrated, as illustrated by example \ref{maCthWtWZGABde}, where the translocative (in the allomorph \ipa{ɕ--}) appears closer to the root than the negation marker, and cannot bear any TAM or person marker.

\begin{exe}
\ex \label{maCthWtWZGABde}
\gll \ipa{ma-ɕ-thɯ-tɯ-ʑɣɤ-βde} 	\ipa{ma} 	\ipa{nɤ-wa} 	\ipa{ɲɯ-ɤkhu}   \\
\textsc{neg-transloc-imp-2-refl}-throw because \textsc{2sg.poss}-father \textsc{sens}-call \\
\glt `Don't throw yourself (in the river), your father is calling you' 
\end{exe}

Grammaticalization of motion verbs as prefixes is unexpected in a strict verb-final language like Japhug, especially since purposive complements of motion verbs are always preverbal. These prefixes therefore originate from a construction where the motion verbs appeared before the main verb, either in a serial verb construction or simple parataxis (\citealt{jacques13harmonization})

\subsection{Voice}

\subsection{Voice and denominal derivations}

\begin{table}[H]
\begin{tabular}{lllllllll} \toprule
Form& Voice & Corresponding denominal prefix \\
\midrule
\ipa{rɤ}-- & Antipassive &    \ipa{rɤ}-- (intransitive dynamic verbs)\\
\ipa{nɯ}-- & Applicative &    \ipa{nɯ}-- (transitive dynamic verbs)\\
\ipa{sɯ}-- & Causative &    \ipa{sɯ}-- (instrumental transitive verb )\\
\ipa{a}-- & Agentless Passive &    \ipa{a}-- (stative verb )\\
\ipa{sɤ}--  & Deexperiencer &    \ipa{sɤ}-- (stative verb expressing a property)\\
    \bottomrule
\end{tabular}
\end{table}
\citet{jacques14antipassive}
\citet{jacques15causative}


\subsection{Reflexive}
\citet{jacques10refl}



\subsection{Incorporation}
\citet{jacques12incorp}

\subsection{Agreement}
\citet{jacques15generic}
\citet{jacques12agreement}

\subsection{TAME}
TAME is marked by directional prefixes and Stem alternations.


\begin{table}[H]
\caption{Directional prefixes in Japhug Rgyalrong} \label{tab:directional}
\resizebox{\columnwidth}{!}{
\begin{tabular}{llllll}
\toprule
   &  	perfective  (A) &  	imperfective  (B)  &  	perfective 3$\rightarrow$3' (C)  &  	inferential  (D) \\  	
   \midrule
up   &  	\ipa{tɤ--}   &  	\ipa{tu--}   &  	\ipa{ta--}   &  	\ipa{to--}   \\  	
down   &  	\ipa{pɯ--}   &  	\ipa{pjɯ--}   &  	\ipa{pa--}   &  	\ipa{pjɤ--}   \\  	
upstream   &  	\ipa{lɤ--}   &  	\ipa{lu--}   &  	\ipa{la--}   &  	\ipa{lo--}   \\  	
downstream   &  	\ipa{tʰɯ--}   &  	\ipa{cʰɯ--}   &  	\ipa{tʰa--}   &  	\ipa{cʰɤ--}   \\  	
east   &  	\ipa{kɤ--}   &  	\ipa{ku--}   &  	\ipa{ka--}   &  	\ipa{ko--}   \\  	
west   &  	\ipa{nɯ--}   &  	\ipa{ɲɯ--}   &  	\ipa{na--}   &  	\ipa{ɲɤ--}   \\  	
no direction &\ipa{jɤ--}   &  	\ipa{ju--}   &  	\ipa{ja--}   &  	\ipa{jo--}   \\  	
\bottomrule
\end{tabular}}
\end{table}


\begin{table}[H]
\caption{Finite verb categories in Japhug Rgyalrong} \label{tab:finite.forms} \centering
\begin{tabular}{lllllll}
\toprule
&	&	stem&	prefixes\\
\midrule
factual&	\textsc{fact} &	1 or 3&	no prefix\\
imperfective&	\textsc{ipfv} &	1 or 3&	B\\
perfective&	\textsc{pfv} &	2&	A or C\\
past imperfective&	\textsc{pst.ipfv} &	2&	\ipa{pɯ--}\\
inferential&	\textsc{ifr} &	1&	D\\
inferential imperfective&	\textsc{ifr.ipfv} &	1&	\ipa{pjɤ--}\\
sensory&	\textsc{sens} &	1 or 3&	\ipa{ɲɯ--}\\
egophoric present&	\textsc{pres} &	1 or 3&	\ipa{ku--}\\
irrealis&	\textsc{irr} &	1 or 3&	\ipa{a--} + A\\
imperative&	\textsc{imp} &	1 or 3&	A\\
\bottomrule
\end{tabular}
\end{table}

\subsection{Progressive}
\ipa{asɯ--}


    \section{Complex constructions} 
    
   \subsection{Complement clauses}
   
   \subsection{Relative clauses}    
   
\subsection{Adverbial clauses} \label{sec:adverbial} 
\citet{jacques14linking}

 \citet{jacques15comparative}
 
    \section{Comparative outlook} 
%5.	Other (prominent) patterns of grammaticalization and reanalysis 
%6.	Comparative outlook (comparison with related languages to determine to what extent grammaticalizations paths are representative of a family )
%7.	Discussion /conclusion (summary of grammaticalization paths, comments on the processes and mechanism operating in individual languages and, maybe most importantly, remarks on typologically unusual aspects of grammaticalization in individual languages ). 


\bibliographystyle{unified}
\bibliography{bibliogj}
\end{document}
