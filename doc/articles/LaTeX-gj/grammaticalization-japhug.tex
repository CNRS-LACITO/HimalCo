\documentclass[oldfontcommands,oneside,a4paper,11pt]{article} 
\usepackage{fontspec}
\usepackage{natbib}
\usepackage{booktabs}
\usepackage{xltxtra} 
\usepackage{polyglossia} 
\usepackage[table]{xcolor}
\usepackage{multicol}
\usepackage{graphicx}
\usepackage{float}
\usepackage{hyperref} 
\hypersetup{bookmarks=false,bookmarksnumbered,bookmarksopenlevel=5,bookmarksdepth=5,xetex,colorlinks=true,linkcolor=blue,citecolor=blue}
\usepackage[all]{hypcap}
\usepackage{memhfixc}
\usepackage{lscape}
\usepackage{tikz}
\usetikzlibrary{trees}
\usepackage{lineno}
\usepackage{gb4e} 
\bibpunct[: ]{(}{)}{,}{a}{}{,}
 
%\setmainfont[Mapping=tex-text,Numbers=OldStyle,Ligatures=Common]{Charis SIL}  
\newfontfamily\phon[Mapping=tex-text,Ligatures=Common,Scale=MatchLowercase,FakeSlant=0.3]{Charis SIL} 
\newcommand{\ipa}[1]{{\phon #1}} %API tjs en italique
 
\newcommand{\grise}[1]{\cellcolor{lightgray}\textbf{#1}}
\newfontfamily\cn[Mapping=tex-text,Ligatures=Common,Scale=MatchUppercase]{MingLiU}%pour le chinois
\newcommand{\zh}[1]{{\cn #1}}
\newcommand{\tld}{\textasciitilde{}}
   \linenumbers

\begin{document} 
 \title{Grammaticalization in Japhug and Gyalrongic languages}
 \author{Guillaume Jacques}
 \maketitle  
 \section{Introduction}

 \section{Nominal categories} 
 
 No class/gender
 %d.	Determiners
\subsection{Number} 

Japhug has two number clitics, dual \ipa{ni} and plural \ipa{ra}. These clitics are not obligatory for non-singular arguments (even in the case of human referents), and do not necessary trigger plural or dual agreement on the verb. The dual \ipa{ni} is obviously related to the numeral \ipa{ʁnɯz} < *\ipa{q-nis} `two', exemplifying the well-attested pathway \textsc{two} $\rightarrow$ \textsc{dual}. The etymology of the plural clitic \ipa{ra} is unknown.
 
  The plural clitic \ipa{ra} can express plurality or collective meaning, as in example \ref{ex:ra1}; it is not incompatible with numerals, as shown in \ref{ex:ra2}.
 
\begin{exe}
\ex \label{ex:ra1}
\gll \ipa{ɯ-kha} 	\ipa{ra} 	\ipa{nɯ-mɤ-kɤ-sɯz} 	\ipa{nɤ} 	\ipa{ɯʑo} 	\ipa{kɯ} 	\ipa{qɤjɣi} 	\ipa{χsɯm} 	\ipa{lo-βzu} \\
\textsc{3sg.poss}-house \textsc{pl} \textsc{3pl.poss-neg-inf}-know \textsc{lnk} \textsc{3sg} \textsc{erg} bread three \textsc{ifr}-make \\
\glt She made three pieces of bread without her relatives knowing. (the raven, 108)
\end{exe}

\begin{exe}
\ex \label{ex:ra2}
\gll 
 \ipa{rɟɤlpu} 	\ipa{nɯ} 	\ipa{kɯ} 	\ipa{kɯki} 	\ipa{tɯrme} 	\ipa{kɯtʂɤɣ} 	\ipa{ra} 	\ipa{nɯ-ɕki}  	\ipa{to-ti} \\
 king \textsc{dem} \textsc{erg} this people six \textsc{pl} \textsc{3pl-dat} \textsc{ifr}-say  \\
 \glt The king told these six men. (Liuhaohan zoubian tianxia, 200)
\end{exe}

In addition, the marker \ipa{ra} can approximative time or location  (see section  \ref{sec:loc}).
 


 \subsection{Possession and pronouns} 
 
Japhug  nouns can be divided into two main categories, inalienably possessed nouns and alienably possessed nouns. The former must take a possessive prefix, even when the possessor is non-specific. In that case, the indefinite possessor \ipa{tɯ-/tɤ-} or generic possessor \ipa{tɯ-} prefixes are used.

Pronouns  and possessive prefixes are very similar (see Table \ref{tab:pronoun}), but it appears that in Japhug pronouns are derived \textit{from} possessive prefixes rather than the opposite: pronouns other than \textsc{3du} and \textsc{3pl} are build by combining the possessive prefix with the root \ipa{-ʑo} meaning originally `self' (a common source for pronouns, see \citealt{heine11pronoms}). Japhug thus exemplifies a pathway \textsc{pronominal affix} $\rightarrow$ \textsc{pronoun}. It is not however a case of degrammaticalization in the strict sense, since the bound pronominal prefixes have not become free morphemes by themselves.

The \textsc{3du} and \textsc{3pl} are formed differently from the rest of the pronouns, by combining the \textit{status constructus} form \ipa{ʑɤ-} of the pronominal root /\ipa{ʑo}/ with the dual and plural clitic \ipa{ni} and \ipa{ra}, with regular vowel harmony \ipa{ɤ} $\rightarrow$ \ipa{a} / \_Ca in the case of \ipa{ʑara} `they'.
 
 \begin{table}[H] \centering
\caption{Pronouns and possessive prefixes }\label{tab:pronoun}
\begin{tabular}{lllllllll} 
\toprule
 Free pronoun & Prefix & Person\\
\midrule
 \ipa{a-ʑo},    \ipa{ɤj} &	\ipa{a-}  &		1\textsc{sg} \\
\ipa{nɤ-ʑo},  \ipa{nɤj} &	\ipa{nɤ-}  &			2\textsc{sg}\\
\ipa{ɯ-ʑo}  &	\ipa{ɯ-}  &			3\textsc{sg}\\
\midrule
\ipa{tɕi-ʑo}  &	\ipa{tɕi-}  &			1\textsc{du} \\
\ipa{ndʑi-ʑo}  &	\ipa{ndʑi-}  &		2\textsc{du} \\	
\ipa{ʑɤ-ni}  &	\ipa{ndʑi-}  &		3\textsc{du} \\	
\midrule
\ipa{i-ʑo}    &	\ipa{i-}  &			1\textsc{pl} \\
\ipa{nɯ-ʑo}   &	\ipa{nɯ-}  &			2\textsc{pl} \\
\ipa{ʑa-ra}  &	\ipa{nɯ-}  &			3\textsc{pl} \\
\midrule
&  \ipa{tɯ-},  \ipa{tɤ-} & indefinite \\
\ipa{tɯ-ʑo} & \ipa{tɯ-}   &  generic\\
\bottomrule
\end{tabular}
\end{table}
 
 

 
 \subsection{Beneficiary}  \label{sec:benef}
The genitive \ipa{ɣɯ} (borrowed from Amdo Tibetan), in addition to marking the possessor, is the normal way to mark beneficiaries and can be used to mark recipient of some ditransitive verbs such as \ipa{kʰo} `give, pass over', as in \ref{ex:aZWG.nWkhAm}.

\begin{exe}
\ex \label{ex:aZWG.nWkhAm}
\gll
\ipa{ki} 	\ipa{kɯra} 	\ipa{ɲɯ-kʰam-a} 	\ipa{tɕe} 	\ipa{ki} 	\ipa{nɤki} 	\ipa{nɯ} 	\ipa{aʑɯɣ} 	\ipa{nɯ-kʰɤm.} \\
this \textsc{dem:pl} \textsc{ipfv}-give[III]-\textsc{1sg} \textsc{lnk} this \textsc{dem} \textsc{dem} \textsc{1sg:gen} \textsc{imp}-give[III] \\
\glt I will give you this, and you will give me that. (slobdpon, 130)
\end{exe}

Alternatively, when the predicate is a transitive verb with an overt object, the beneficiary can be marked as a possessive prefix on the object, as 	\ipa{a-tɯ-ci} \textsc{1sg-indef.poss}-water `water for me' in \ref{ex:atWci}.

\begin{exe}
\ex \label{ex:atWci}
\gll \ipa{χsɤr}  	\ipa{khɯtsa}  	\ipa{ɯ-ŋgɯ}  	\ipa{nɯ}  	\ipa{tɕu}  	\ipa{a-tɯ-ci}  	\ipa{ci}  	\ipa{tɤ-rke}  	\ipa{ma}  	\ipa{wuma}  	\ipa{ɲɯ-ɕpaʁ-a}  \\
gold bowl \textsc{3sg}-inside \textsc{dem} \textsc{loc} \textsc{1sg-indef.poss}-water \textsc{indef} \textsc{imp}-put.in[III] because really \textsc{sens}-be.thirsty-\textsc{1sg} \\
\glt Pour some water for me in the golden bowl, I am very thirsty.
\end{exe}

\subsection{Dative}  \label{sec:dat}
The dative \ipa{ɯ-ɕki}, used to mark the recipient of indirective ditransitive verbs as in \ref{ex:tathu}, derives from a relator noun  meaning `side', a meaning still marginally present in Japhug in examples like \ref{ex:WCki.loc}.

 \begin{exe}
   \ex   \label{ex:tathu}
 \gll \ipa{tɤ-pɤtso}  	\ipa{ra}  	\ipa{kɯ}  	\ipa{nɯ-sloχpɯn}  	\ipa{ɯ-ɕki}  	\ipa{to-tʰu-nɯ}  \\
\textsc{indef.poss}-child \textsc{pl} \textsc{erg} \textsc{3pl.poss}-teacher \textsc{3sg-dat} \textsc{ifr}-ask-\textsc{pl} \\
\glt The children asked their teacher. (Looking at the snow, 11)
   \end{exe}  

\begin{exe}
\ex \label{ex:WCki.loc}
\gll \ipa{ɯ-rte} 	\ipa{nɯ} 	\ipa{ɯ-rna} 	\ipa{ɯ-ɕki} 	\ipa{pɯ-kɯ-ɴqoʁ} 	\ipa{nɯnɯ} 	\ipa{pjɤ-mɟa} 	\ipa{tɕe} 	\ipa{ɯ-ku} 	\ipa{ɯ-taʁ} 	\ipa{to-ta} \\
\textsc{3sg.poss}-hat \textsc{dem} \textsc{3sg.poss}-ear \textsc{3sg}-dat \textsc{pfv:down-nmlz}:S/A-hang \textsc{dem} \textsc{ifr:down}-take \textsc{lnk} \textsc{3sg.poss}-head \textsc{3sg}-on \textsc{ifr}-put \\
\glt He took the hard that was hanging on his ear and put it on his head.(140505 liuhaohan zoubian tianxia, 164)
\end{exe}

\subsection{Comitative adverbs}  \label{sec:comit}

\ipa{tɤ-rte} `hat' $\rightarrow$ \ipa{kɤɣɯ-rtɯ\tld{}rte}

\begin{exe}
\ex \label{ex:kAGWrtWrte}
\gll \ipa{kɤɣɯ-rtɯ\tld{}rte} 	\ipa{ʑo} 	\ipa{kha} 	\ipa{ɯ-ŋgɯ} 	\ipa{lɤ-tɯ-ɣe} 	\\
\textsc{comit}-hat \textsc{emph} house \textsc{3sg}-inside \textsc{pfv}-2-come[II] \\
\glt You came inside the house with your hat. (You were expected to take it off before coming in)
\end{exe}

\begin{table}[h] \centering
\caption{The denominal prefix \ipa{aɣɯ-}}\label{tab:aGW}
\resizebox{\columnwidth}{!}{
\begin{tabular}{lllllllll} 
\toprule
Base noun & Meaning & Denominal verb & Meaning\\
\midrule
\ipa{tɯ-ɣli} & excrement, manure & \ipa{aɣɯ-ɣli} & producing a lot of manure (of pigs) \\
\ipa{tɤ-lu} & milk &\ipa{aɣɯ-lu} & producing a lot of milk (of cows) \\
\ipa{tɯ-mɲaʁ} & eye & \ipa{aɣɯ-mɲaʁ} & having a lot of holes \\
\ipa{tɯ-ɕnaβ} & snot & \ipa{aɣɯ-ɕnɯ\tld{}ɕnaβ} & be  slimy \\
\ipa{ɯ-mdoʁ} & colour & \ipa{aɣɯ-mdoʁ} & having the same colour \\
\bottomrule
\end{tabular}}
\end{table}


\subsection{Comparee and standard}
In the comparative construction, both the comparee and the standard are marked, respectively by the postpositions \ipa{kɯ} and \ipa{sɤz}  (example \ref{ex:comp1}). The marker \ipa{kɯ} on the comparee is obligatory only if the standard is not overt, otherwise it is optional.

\begin{exe}
\ex \label{ex:comp1}
\gll  [\ipa{ɯ-ʁi}]_{standard}   	\ipa{sɤz}   	[\ipa{ɯ-pi}   	\ipa{nɯ}]_{comparee}   	\ipa{\textbf{kɯ}}   	\ipa{mpɕɤr}     \\
\textsc{3sg.poss}-younger.sibling \textsc{comparative} \textsc{3sg.poss}-elder.sibling \textsc{dem} \textsc{erg}  be.beautiful:\textsc{fact} \\
\glt `The elder one is more beautiful than the young one.' (elicited)
\end{exe}

The mark \ipa{kɯ} on the comparee is etymologically related to the ergative \ipa{kɯ} (borrowed from Amdo Tibetan). The complex grammaticalization pathway leading from ergative to comparee marker is presented in \citet{jacques15comparative}. 

This unusual pathway \textsc{ergative} $\rightarrow$ \textsc{comparee} rather than the more common \textsc{ergative} $\rightarrow$ \textsc{standard} is all the more surprising that the ergative \ipa{kə/ɣə} in Amdo Tibetan from which the Japhug ergative \ipa{kɯ} was borrowed is used for the \textit{standard} in the comparative construction.

The marker \ipa{sɤz} contains the locative suffix \ipa{-z} (which also appears as a tautosyllabic clitic \ipa{zɯ}), but the etymology of the first element \ipa{sɤ-} is unknown. 

 \subsection{Location}   \label{sec:loc}
In Japhug, there are four distinct (non mutually exclusive) ways of marking locative adjuncts. First, locative and temporal adjunct are commonly left unmarked. Second, they can  take the locative case markers \ipa{zɯ} or \ipa{tɕu}. Third, relator (possessed) nouns such as \ipa{ɯ-ŋgɯ} `inside', \ipa{ɯ-taʁ} `on' etc can be used for more specific locations. 

Fourth, the plural clitic \ipa{ra} can indicate approximate location, as in \ref{ex:khara}. This use of \ipa{ra} is reminiscent of plural markers in Kirghiz and Old Japanese, which combine collective, hypocoristic and approximate locative meanings see \citealt[195]{antonov07ra}).

\begin{exe}
\ex \label{ex:khara}
\gll
\ipa{tɯ-zda} 	\ipa{nɯ} 	\ipa{ma} 	\ipa{kɯmaʁ} 	\ipa{tɯrme} 	\ipa{a-pɯ-me} 	\ipa{tɕe,} 	\ipa{kha} 	\ipa{ra} 	\ipa{aʁɤndɯndɤt} \ipa{ɲɯ-ɤnɯɣro} 	\ipa{ɲɯ-ŋu} 	\ipa{ɲɯ-ti.} \\
\textsc{indef.poss}-companion \textsc{dem} apart.from other people \textsc{irr-ipfv}-not.exist \textsc{lnk} house \textsc{pl} everywhere \textsc{sens}-play \textsc{sens}-be \textsc{sens}-say \\
\glt He says that (the young monkey) would play everywhere in the house whenever there are no other people (apart from members of the family). (19 GZW2, 10)
\end{exe}

\subsection{Topic and focus} 

The temporal adverb \ipa{jiɕqha} `just before' has become grammaticalized as a pre-nominal determiner `the aforementioned' expressing that the nominal in question has been referred to previously in the discourse, though not in the last few sentences. In example \ref{ex:jiCqha}, for instance, the leaf is mentioned four sentences before.

\begin{exe}
\ex \label{ex:jiCqha}
 \gll \ipa{tɯmɯ} 	\ipa{ci} 	\ipa{tɕhɤrnaʁ} 	\ipa{ci} 	\ipa{tɕhɯmtɕhɯm} 	\ipa{ko-lɤt.} 	\ipa{tɕendɤre} 	\ipa{jiɕqha} 	\ipa{tɤ-jwaʁ} 	\ipa{nɯ} 	\ipa{pjɤ-nɯndzom} 	\ipa{tɕe,} 	\ipa{ɯ-ʁi} 	\ipa{ɯ-kɯr} 	\ipa{ɯ-ŋgɯ} 	\ipa{nɯ} 	\ipa{tɕu} 	\ipa{tɯ-ci} 	\ipa{χsɯ-ntɕhaʁ} 	\ipa{jamar} 	\ipa{pjɤ-ɕe.} \\
 rain \textsc{indef}  rain \textsc{indef} \textsc{idph}.II:little.rain \textsc{ifr}-throw \textsc{lnk} the.aforementioned \textsc{indef.poss}-leaf \textsc{dem} \textsc{ifr:down}-flow.along \textsc{lnk} \textsc{3sg.poss}-younger.sibling  \textsc{3sg.poss}-mouth \textsc{3sg}-inside \textsc{dem} \textsc{loc} \textsc{indef.poss}-water three-drop about \textsc{ifr:down}-go \\
 \glt There was a little rain, and (the water) flowed along the leaf (that the elder brother had placed) and three drops of water flowed into his younger brother's mouth. (Smanmi 11, 61)
\end{exe}

 
The clitic \ipa{rcanɯ} focalizes the preceding noun phrase and emphasizes the unexpectedness of the situation or event described by the phrase that follows, as in \ref{ex:YAwGsWGYaRnW}, where the blackening of the sparrows surprised (and amused) the person telling the story.

%\begin{exe}
%\ex \label{ex:tAmdzArgi}
%\gll 	\ipa{tɤmdzɤrgi} 	\ipa{nɯ} 	\ipa{ɣɯ} 	\ipa{ɯ-di} 	\ipa{ci} 	\ipa{tu} 	\ipa{tɕe,} 	\ipa{nɯ} 	\ipa{ɯ-di} 	\ipa{nɯnɯ} 	\ipa{rcanɯ} 	\ipa{tɯ-ku} 	\ipa{ʑo} 	\ipa{ɕɯ-mŋɤm.}  \\
%thistle \textsc{dem} \textsc{gen} \textsc{3sg.poss}-smell \textsc{indef} exist:\textsc{fact} \textsc{lnk} \textsc{dem} 
%\textsc{3sg.poss}-smell \textsc{dem}  \textsc{foc:unexpected} \textsc{genr:poss}-head \textsc{emph}  \textsc{caus}-hurt:\textsc{fact} \\
%\glt The thistle has a smell, and this smell makes one's head hurt. (15 babW, 82)
%\end{exe}

\begin{exe}
\ex \label{ex:YAwGsWGYaRnW}
\gll 
\ipa{tɕendɤre} 	\ipa{thɯ-kɤ-βlɯ} 	\ipa{nɯ} 	\ipa{ɲɯ-ɕti} 	\ipa{tɕe,} 	<yancong>	 \ipa{ɯ-ŋgɯ} 	\ipa{ɲɯ-ɲaʁ} 	\ipa{rcanɯ} 	\ipa{kumpɣɤtɕɯ} 	\ipa{ra} 	\ipa{ɲɤ́-wɣ-sɯɣ-ɲaʁ-nɯ} 	\ipa{ʑo} 	\\
\textsc{lnk} \textsc{pfv-nmlz}:P-burn \textsc{dem} \textsc{sens}-be:\textsc{assert} \textsc{lnk} chimney \textsc{3sg}-inside \textsc{sens}-be.black \textsc{foc:unexpected} sparrow \textsc{pl} \textsc{ifr-inv-caus}-be.black-\textsc{pl} \textsc{emph} \\
\glt Because there has been burning going on, the inside of the chimney is black, and it made the sparrows (who had build a nest inside it) become (completely) black! (22 kWmpGAtCW, 72)
\end{exe}
When it occurs before an adjectival verb, whether in finite or non-finite form as \ipa{kɯ-dɯ\textasciitilde{}dɤn} `numerous' in \ref{ex:kWdWdAn}, or before an ideophone (\ref{ex:RYJliRYJli}), \ipa{rcanɯ} indicates high degree. Adjectival verbs in this case often have emphatic reduplication.

\begin{exe}
\ex \label{ex:kWdWdAn}
\gll \ipa{tɕe} 	\ipa{nɯ} 	\ipa{ɕoŋtɕa} 	\ipa{rcanɯ} 	\ipa{kɯ-dɯ\textasciitilde{}dɤn} 	\ipa{ʑo} 	\ipa{pjɤ-sɯ-phɯt-nɯ.} \\
\textsc{lnk} wood \textsc{foc:unexpected} \textsc{nmlz}:S/A-\textsc{emph}\textasciitilde{}-be.a.lot \textsc{emph} \textsc{ifr-caus}-chop-\textsc{pl}\\
\glt And they had (people) chop quite lot of wood (for them) (28 qAjdo,  103)
\end{exe}
\begin{exe}
\ex \label{ex:RYJliRYJli}
\gll
\ipa{ɯ-phoŋbu} 	\ipa{nɯ} 	\ipa{rcanɯ} 	\ipa{ʁɲɟliʁɲɟli} 	\ipa{ʑo} 	\ipa{ɲɯ-pa} \\
\textsc{3sg.poss}-body \textsc{dem}  \textsc{foc:unexpected} \textsc{idph}:II:huge;massive \textsc{emph} \textsc{sens}-\textsc{aux} \\
\glt Its body is huge. (20 sWNgi,  16)
\end{exe}

This focus clitic is derived from the possessed noun  \ipa{ɯ-rca} `following, together with' (see example \ref{ex:arca}) together with the distal demonstrative \ipa{nɯ} `that'.

\begin{exe}
\ex \label{ex:arca}
\gll
\ipa{aʑo} 	\ipa{a-rca} 	\ipa{kɤ-ɣi} 	\ipa{mɤ-tɯ-cha} \\
\textsc{1sg} \textsc{1sg}-following \textsc{inf}-come \textsc{neg}-2-can:\textsc{fact} \\
\glt You cannot come with me.
\end{exe}
 
 

  \section{Verbal categories} 

\subsection{Person indexation} 
Japhug and other Rgyalrongic languages have a polypersonal indexation system that is partially cognate to that of Kiranti languages, and appears to be at least in part of proto-Sino-Tibetan origin, though this issue is controversial (see \citealt{delancey11prefixes, jacques12agreement}).

The Japhug transitive conjugation includes two portanteau prefixes for local scenarios \ipa{ta-} 1$\rightarrow$2 and \ipa{kɯ-} 2$\rightarrow$1.  The non-local forms taking these prefixes in Gyalrong languages have suffixes coreferent with the P, as illustrated by examples (\ref{ex:2.1sg}) and (\ref{ex:1.2du}).

\begin{exe}
\ex \label{ex:2.1sg}
\gll \ipa{pɯ-kɯ-nɤjo-a} \\
imp-2$\rightarrow$1-wait-\textsc{1sg} \\
\glt Wait for me (heard in context).
\end{exe}

\begin{exe}
\ex \label{ex:1.2du}
\gll  \ipa{maka} 	\ipa{ʑo} 	\ipa{mɤ-ta-βde-ndʑi} \\
at.all \textsc{emph} \textsc{neg}-1$\rightarrow$2-leave-\textsc{du} \\
\glt I will never abandon you two. (140507 tangguowu, 166)
\end{exe}

\citet{jacques15generic}  proposes Gyalrong-internal etymologies for these prefixes. 

 

\subsection{Associated motion}
Japhug has a simple associated motion system, with one translocative / andative prefix \ipa{ɕɯ-} and a cislocative / venitive prefix \ipa{ɣɯ-} transparently grammaticalized from the verbs \ipa{ɕe} `go' and \ipa{ɣi} `come' respectively. These prefixes are morphologically fully integrated, as illustrated by example \ref{maCthWtWZGABde}, where the translocative (in the allomorph \ipa{ɕ-}) appears closer to the root than the negation marker, and cannot bear any TAM or person marker.

\begin{exe}
\ex \label{maCthWtWZGABde}
\gll \ipa{ma-ɕ-tʰɯ-tɯ-ʑɣɤ-βde} 	\ipa{ma} 	\ipa{nɤ-wa} 	\ipa{ɲɯ-ɤkʰu}   \\
\textsc{neg-transloc-imp-2-refl}-throw because \textsc{2sg.poss}-father \textsc{sens}-call \\
\glt `Don't throw yourself (in the river), your father is calling you' 
\end{exe}

Grammaticalization of motion verbs as prefixes is unexpected in a strict verb-final language like Japhug, especially since purposive complements of motion verbs are always preverbal. These prefixes therefore originate from a construction where the motion verbs appeared before the main verb, either in a serial verb construction or simple parataxis (\citealt{jacques13harmonization}).

\subsection{Voice}
The main sources for voice markers in Japhug are denominal prefixes. Five of the voice derivation prefixes, namely the Antipassive, the Applicative, the Causative, the Passive and the Deexperiencer, are homophonous with denominal derivations with similar meanings, as shown in Table \ref{tab:denom}.

\begin{table}[H] \caption{Voice markers and corresponding denominal derivations} \label{tab:denom} \centering
\begin{tabular}{lllllllll} \toprule
Form& Voice & Corresponding denominal prefix \\
\midrule
\ipa{rɤ}- & Antipassive &    \ipa{rɤ}- (intransitive dynamic verbs)\\
\ipa{nɯ(ɣ)}- & Applicative &    \ipa{nɯ(ɣ)}- (transitive dynamic verbs)\\
\ipa{sɯ(ɣ)}- & Causative &    \ipa{sɯ(ɣ)}- (verb meaning `use X' or \\
&& `cause to have X') \\
\ipa{a}- & Agentless Passive &    \ipa{a}- (stative verb)\\
\ipa{sɤ}-  & Deexperiencer &    \ipa{sɤ}- (stative verb expressing a property)\\
    \bottomrule
\end{tabular}
\end{table}

These five voice derivations and their corresponding denominal origin are discussed in the following. In addition, voice derivations originating from markers other than denominal prefixes (in particular, the reflexive \ipa{ʑɣɤ-}) are briefly analyzed.

\subsubsection{Antipassive}
The relationship between voice and derivation prefixes was first explained in the case of the Antipassive prefix \ipa{rɤ-} (\citealt{jacques14antipassive}), a prefix attested only in Gyalrong languages and not even found in Khroskyabs, their closest relative (\citealt{lai13affixale}).


The Antipassive derives from the intransitive denominal prefix \ipa{rɯ-/rɤ-} by a two-stage pathway.

First, an action (or patient) nominal is derived from a transitive verb. This action nominal has the same form as the bare root of the verb, but is a possessed noun requiring a possessive prefix. For instance, from \ipa{ɕphɤt} `patch' one derives the possessed noun 
\ipa{-ɕphɤt} `a patch', which, in the absence of a definite possessor, must occur with the indefinite possessor prefix \ipa{tɤ-} (\ipa{tɤ-ɕphɤt}).

Second, intransitive derivation in \ipa{rɯ-/rɤ-} is applied to this possessed noun, yielding the form \ipa{rɤ-ɕphɤt} `to do patching'. Following the regular pattern, possessive prefixes are lost during denominal derivations, so that a form such as *\ipa{rɤ-tɤ-ɕphɤt} with the indefinite possessor prefix would not be expected.

The end form \ipa{rɤ-ɕphɤt} `to do patching' can then be reanalyzed as being directly derived from the base transitive verb \ipa{ɕphɤt} `patch', and since the S of this intransitive verb corresponds to the A of the transitive verb \ipa{ɕphɤt}, and the P is lost, this originally denominal prefix is reinterpreted as being an Antipassive marker. Then, this prefix is overgeneralized to most transitive verbs.

This reanalysis probably occurred recently in Japhug, as forms such as \ipa{rɤ-ɕphɤt} are still synchronically ambiguous between an Antipassive and a Denominal verb. Further evidence for this pathway can be found in irregular nominal forms, as there are several verb for which a semantic or morphological irregularity is shared between the Antipassive verb and the corresponding action/patient noun, but not the base transitive verbs, showing that the Antipassive form derives from the patient. For example, the intransitive verb \ipa{rɤ-nŋa}`owe money' is an irregular Antipassive form of \ipa{ŋa} `owe X'; the additional \ipa{-n-} is also found in the noun \ipa{-nŋa} `debt', showing that this irregular Antipassive historically derives from the noun \ipa{-nŋa} `debt' rather than directly from the transitive verb \ipa{ŋa} `owe X'. 

The pathway presented here can be summarized as (\ref{ex:pathway}):

\begin{exe}
\ex \label{ex:pathway}
\glt \textsc{action nominalization} of transitive verb + \textsc{intransitive denominal derivation} $\Rightarrow$ \textsc{antipassive}
\end{exe}

The general mechanism is that the action nominalization neutralizes the transitivity of the base verb, and that a new transitivity and argument structure is allocated by the denominal prefix. The same applies to the four other voice derivations originating from denominal prefixes described below.

\subsubsection{Causative}
The causative prefix \ipa{sɯ(ɣ)-} is one of the most productive derivation prefixes in japhug, and can be applied to nearly all transitive and intransitive verbs (the detailed meaning of this prefix and the constructions that it can be used in are described in \citealt{jacques15causative}). 

 The homophonous denominal prefix \ipa{sɯ(ɣ)-} derives verbs meaning `use X' or `cause to have X', such as \ipa{sɯ-ɕtʂi} `cause to sweat' (from the possessed noun \ipa{-ɕtʂi} `sweat'), or \ipa{sɯɣ-tshaʁ} `to sieve (=to use a sieve)' from \ipa{tshaʁ} `sieve'. 

The causative \ipa{sɯ(ɣ)-} can thus be explained as the result of reanalysis from the denominal derivation `cause to X' from a possessed action nominal deriving from the base verb. 

Even synchronically, transitive forms such as \ipa{sɯ-ndza} `cause to eat' from the transitive verb \ipa{ndza} `eat' are between a plain causative and a denominal verb from the possessed noun \ipa{ɯ-ndza} `its food' (= `cause to have food'). 



\subsubsection{Other derivations}
The same two-step pathway of grammaticalization proposed to account for the origin of the antipassive and causative prefixes above can also be applied to three other voice derivation prefixes: the deexperiencer \ipa{sɤ--}, the passive \ipa{a-} and the applicative \ipa{nɯ-}.

The deexperiencer prefix \ipa{sɤ-} derives stative verbs from intransitive verbs whose S is an experiencer or any non-agentive semantic role. The S of the deexperiencer verb corresponds to the stimulus. Examples include \ipa{rga} `like' $\rightarrow$ \ipa{sɤ-rga} `be lovable' or \ipa{ŋgio} `slip (of a human)' $\rightarrow$ \ipa{sɤ-ŋgio} `be slippery (of the ground)'(\citealt{jacques12demotion}).

There are a few examples of a denominal prefix \ipa{sɤ-} expressing a property related to the base noun, such as \ipa{-ndɤɣ} `poison' $\rightarrow$ \ipa{sɤ-ndɤɣ} `be poisonous' or \ipa{-mbrɯ} `anger'  $\rightarrow$ \ipa{sɤ-mbrɯ} `be angry'. The semantics of the deexperiencer derivation is closely related to that of the verb  \ipa{sɤ-ndɤɣ} `be poisonous': the property of an object that has effects on surrounding people, and here again can be proposed to deriva from it through a two-step pathway like:

\begin{exe}
\ex \label{ex:pathway}
\glt \textsc{action nominalization} of verb + \textsc{property denominal derivation} $\Rightarrow$ \textsc{deexperiencer}
\end{exe}

The passive \ipa{a-} is an agentless passive, which derives intransitive verbs whose S corresponds to the P of the base verb, as \ipa{ata}  `be put on' from \ipa{ta} `put'. The corresponding denominal prefix \ipa{a-} is used to derive a stative verb describing a shape related to the noun, or a visible / perceptible concrete property, as in \ipa{-ci} `water' $\rightarrow$  \ipa{aci} `be wet' or  \ipa{ʑɤwu} `lame' $\rightarrow$  \ipa{aʑɤwu} `be lame'.

The applicative \ipa{nɯ(ɣ)--} derives a transitive verb from an intransitive one; unlike in the causative derivation, the A of the applicative verb corresponds to the S of the intransitive one, and a P argument is added (\citealt{jacques13tropative}). The P of applicative verbs refers to either the stimulus in the case of cognition verbs (\ipa{mu} `be afraid (intr)' $\rightarrow$ \ipa{nɯɣ-mu} `fear (tr)') or the the addressee (\ipa{akhu} `shout (intr)'  $\rightarrow$ \ipa{nɯ-ɤkhu} `shout at (tr)'). The corresponding denominal derivation \ipa{nɯ(ɣ)-} has many different meaning, but its most productive one is to create a transitive verb from a noun, especially when one has a pair with an intransitive verb in \ipa{rɯ--}; for instance, from a noun such as \ipa{ftɕaka} `manner' one can derive the intransitive \ipa{rɯftɕaka} `make preparations' and the transitive verb \ipa{nɯftɕaka} `prepare (vt)'. 

XXX

\begin{exe}
\ex \label{ex:pathway}
\glt \textsc{action nominalization} of intransitive verb + \textsc{transitive denominal derivation} $\Rightarrow$ \textsc{applicative}
\end{exe}

\subsubsection{Reflexive}
The reflexive \ipa{ʑɣɤ-} (and its cognates in other Gyalrong languages) differs from all other derivations in that it does not derive from a denominal prefix. Two hypotheses have been proposed to account for its origin. 

\citet{jacques10refl} proposed that \ipa{ʑɣɤ-} from proto-Gyalrong *\ipa{wjɐ-} results from the incorporation of the third person full pronoun *\ipa{wəjaŋ}, (Japhug \ipa{ɯʑo}) with phonological reduction. \citet{jackson14morpho} argued that it originates from the fusion of the pronominal root *\ipa{jaŋ} with the verb stem, to which the inverse prefix *\ipa{wə-} is added.


\subsection{Incorporation}
Japhug has an incorporation-like construction in which noun-verb nominal compounds are turned into verbs by means of a denominal prefix (\citealt{jacques12incorp}). For instance, from the noun \ipa{cɯ} `stone' and the verb \ipa{pʰɯt} `pluck, take out' one can derive an action nominal   \ipa{cɯpʰɯt} `clearing the stones (from a field, before ploughing)', which can in turn be made into an incorporating verb by denominal derivation  \ipa{ɣɯ-cɯpʰɯt } `take out stones (out of the field)'. 

\begin{exe}   
\ex
\begin{xlist}[(ii)]
\exi{(i)} 
\gll     \ipa{cɯ} \ipa{nɯ-pʰɯt-a}  \\
  stone \textsc{pfv}-take.out-\textsc{1sg} \\
  \exi{(ii)} 
\gll     \ipa{cɯ-pʰɯt} \ipa{nɯ-βzu-t-a}  \\
  stone-clearing \textsc{pfv}-do-\textsc{pst}-\textsc{1sg} \\
\exi{(iii)} 
\gll     \ipa{nɯ-ɣɯ-cɯ-pʰɯt-a}  \\
  \textsc{pfv-denominal}-stone-take.out-\textsc{1sg} \\
  \end{xlist}
  \glt   I cleared the stones (from the field). 
\end{exe}   

The construction (iii) has further become a full incorporating construction in the closely related Khroskyabs language, where the denominal prefix has in some cases disappeared due to phonological attrition (\citealt{lai13affixale}).

Gyalrongic languages thus offer a third possible origin for incorporating constructions, after coalescence of noun and verb and backformation (\citealt{mithun84incorp}): reanalysis of denominal verbs derived from noun-verb nominal compounds.


\subsection{TAME}
TAME is marked by directional prefixes and Stem alternations.

\subsubsection{Directional prefixes}

\begin{table}[H]
\caption{Directional prefixes in Japhug Rgyalrong} \label{tab:directional}
\resizebox{\columnwidth}{!}{
\begin{tabular}{llllll}
\toprule
   &  	perfective  (A) &  	imperfective  (B)  &  	perfective 3$\rightarrow$3' (C)  &  	inferential  (D) \\  	
   \midrule
up   &  	\ipa{tɤ-}   &  	\ipa{tu-}   &  	\ipa{ta-}   &  	\ipa{to-}   \\  	
down   &  	\ipa{pɯ-}   &  	\ipa{pjɯ-}   &  	\ipa{pa-}   &  	\ipa{pjɤ-}   \\  	
upstream   &  	\ipa{lɤ-}   &  	\ipa{lu-}   &  	\ipa{la-}   &  	\ipa{lo-}   \\  	
downstream   &  	\ipa{tʰɯ-}   &  	\ipa{cʰɯ-}   &  	\ipa{tʰa-}   &  	\ipa{cʰɤ-}   \\  	
east   &  	\ipa{kɤ-}   &  	\ipa{ku-}   &  	\ipa{ka-}   &  	\ipa{ko-}   \\  	
west   &  	\ipa{nɯ-}   &  	\ipa{ɲɯ-}   &  	\ipa{na-}   &  	\ipa{ɲɤ-}   \\  	
no direction &\ipa{jɤ-}   &  	\ipa{ju-}   &  	\ipa{ja-}   &  	\ipa{jo-}   \\  	
\bottomrule
\end{tabular}}
\end{table}


\begin{table}[H]
\caption{Finite verb categories in Japhug Rgyalrong} \label{tab:finite.forms} \centering
\begin{tabular}{lllllll}
\toprule
&	&	stem&	prefixes\\
\midrule
Factual&	\textsc{fact} &	1 or 3&	no prefix\\
Imperfective&	\textsc{ipfv} &	1 or 3&	B\\
Perfective&	\textsc{pfv} &	2&	A or C\\
Past Imperfective&	\textsc{pst.ipfv} &	2&	\ipa{pɯ-}\\
Inferential&	\textsc{ifr} &	1&	D\\
Inferential imperfective&	\textsc{ifr.ipfv} &	1&	\ipa{pjɤ-}\\
Sensory&	\textsc{sens} &	1 or 3&	\ipa{ɲɯ-}\\
Egophoric present&	\textsc{pres} &	1 or 3&	\ipa{ku-}\\
Irrealis&	\textsc{irr} &	1 or 3&	\ipa{a-} + A\\
Imperative&	\textsc{imp} &	1 or 3&	A\\
\bottomrule
\end{tabular}
\end{table}

\subsubsection{Progressive}
\ipa{asɯ-}


\section{Complex constructions} 

    
\subsection{Alternative}
 Japhug has a conjunction \ipa{me}  `whether ... or' repeated after each noun or phrase in the alternative correlative construction, as in example \ref{ex:saCW}.
 
\begin{exe}
\ex \label{ex:saCW}
\gll  \ipa{saɕɯ} 	\ipa{nɯnɯ} 	\ipa{ɯ-qa} 	\ipa{me,} 	\ipa{ɯ-ru} 	\ipa{me,} 	\ipa{ɯ-jwaʁ} 	\ipa{me} 	\ipa{nɯra} 	\ipa{tɯrgi} 	\ipa{cho} 	\ipa{naχtɕɯɣ} \\
 larch \textsc{dem} \textsc{3sg.poss}-root whether \textsc{3sg.poss}-trunk whether \textsc{3sg.poss}-leave whether \textsc{dem:pl} fir  \textsc{comit} be.similar:\textsc{fact} \\
\glt Whether its root, its trunk or its leaves, the larch is identical to the fir. (08 saCW, 5)
\end{exe} 

This conjunction is obviously grammaticalized from the negative existential copula \ipa{me} `not exist' through an alternative  concessive conditional `whether ... exists or' involving originally the affirmative and negative existential verbs \ipa{tu} vs \ipa{me} as in \ref{ex:pWnWme}

\begin{exe}
\ex \label{ex:pWnWme}
\gll \ipa{tɤ-ʁa} 	\ipa{me} 	\ipa{tɕe,} 	\ipa{nɯ} 	\ipa{pɯ-nɯ-tu} 	\ipa{pɯ-nɯ-me} 	\ipa{kɯ-khɯ} 	\ipa{nɯ} 	\ipa{kɯ-rga} 	\ipa{me.} \\
\textsc{indef.poss}-free.time not.exist:\textsc{fact} \textsc{lnk} \textsc{dem} \textsc{pst.ipfv-auto}-not.exist \textsc{pst.ipfv-auto}-exist \textsc{nmlz}:S/A-be.possible \textsc{dem} \textsc{nmlz}:S/A-like not.exist:\textsc{fact} \\
\glt (Nobody gathers wild strawberries), because (we) don't have time, it is fine whether or not (we) have it, nobody likes it. (11 paRzwamWntoR, 92)
\end{exe}
 
 In its grammaticalized form, \ipa{me} has lost all person and TAME marking.  In \ref{ex:aZo.me}, we see that the conjunction \ipa{me} does not take first or second person singular indexation when used with a pronoun, as would be expected if it still were a verb and the construction an alternative  concessive conditional.
 
 \begin{exe}
\ex \label{ex:aZo.me}
\gll 
 \ipa{aʑo} 	\ipa{me,} 	\ipa{nɤʑo} 	\ipa{me,} 	\ipa{ɯʑo} 	\ipa{me,} 	\ipa{kɤsɯfse} 	\ipa{ɕe-j} 	\ipa{ra} \\
\textsc{1sg} whether \textsc{2sg} whether \textsc{3sg} whether all go:\textsc{fact}-\textsc{1pl} have.to:\textsc{fact} \\
\glt Whether I, you or he, we all have to go. (elicited)
\end{exe}

Expressing futility

\ipa{ndza} \ipa{mɤ-ndza} \ipa{me-a}

\subsection{Adversative}

`not only ... but also'


\ipa{mɤ-ra ma}

\ipa{mɤ-kɯ-jɤɣ kɯ}

\ipa{ɯ-tɤjɯ}
\ipa{ʁo alala ri}

   \subsection{Complement clauses}
   \ipa{ɯ-spa} `matter' > purposive \citet[212]{heine-kuteva02}
   
 
    \section{Comparative outlook} 
%5.	Other (prominent) patterns of grammaticalization and reanalysis 
%6.	Comparative outlook (comparison with related languages to determine to what extent grammaticalizations paths are representative of a family )
%7.	Discussion /conclusion (summary of grammaticalization paths, comments on the processes and mechanism operating in individual languages and, maybe most importantly, remarks on typologically unusual aspects of grammaticalization in individual languages ). 


\bibliographystyle{unified}
\bibliography{bibliogj}
\end{document}
