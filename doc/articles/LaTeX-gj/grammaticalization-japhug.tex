\documentclass[oneside,a4paper,11pt]{article} 
\usepackage{fontspec}
\usepackage{natbib}
\usepackage{booktabs}
\usepackage{xltxtra} 
\usepackage{polyglossia} 
\usepackage[table]{xcolor}
\usepackage{multicol}
\usepackage{graphicx}
\usepackage{float}
\usepackage{hyperref} 
\hypersetup{bookmarksnumbered,bookmarksopenlevel=5,bookmarksdepth=5,colorlinks=true,linkcolor=blue,citecolor=blue}
\usepackage[all]{hypcap}
\usepackage{memhfixc}
\usepackage{lscape}
\usepackage{tikz}
\usetikzlibrary{trees}
\usepackage{gb4e} 
\bibpunct[: ]{(}{)}{,}{a}{}{,}
 
%\setmainfont[Mapping=tex-text,Numbers=OldStyle,Ligatures=Common]{Charis SIL}  
\newfontfamily\phon[Mapping=tex-text,Ligatures=Common,Scale=MatchLowercase]{Charis SIL} 
\newcommand{\ipa}[1]{\mbox{\phon\textbf{#1}}} %API tjs en italique
 
\newcommand{\grise}[1]{\cellcolor{lightgray}\textbf{#1}}
\newfontfamily\cn[Mapping=tex-text,Ligatures=Common,Scale=MatchUppercase]{SimSun}%pour le chinois
\newcommand{\zh}[1]{{\cn #1}}
\newcommand{\tld}{\textasciitilde{}}
\newcommand{\change}[2]{*\ipa{#1} $\rightarrow$ \ipa{#2}}


\begin{document} 
 \title{Grammaticalization in Japhug\footnote{Acknowledgements to be added after editorial decision. The examples are taken from a corpus that is progressively being made available on the Pangloss archive (\citealt{michailovsky14pangloss}). This research was funded by the HimalCo project (ANR-12-CORP-0006) and is related to the research strand LR-4.11 ‘‘Automatic Paradigm Generation and Language Description’’ of the Labex EFL (funded by the ANR/CGI). }}
 %Alec Coupe, Linda Konnerth, Nat Krause, Alexis Michaud, Mark W. Post, Amos Teo.
 \author{Guillaume Jacques}
 \maketitle  
 \section{Introduction}
Japhug and other Gyalrong languages are among the languages with the richest morphology of all the Sino-Tibetan family. While the ultimate lexical origin of most grammatical markers in Japhug is unknown and probably unrecoverable, many affixes are nevertheless analyzable as being derived from other independent words or other grammatical markers (for instance, denominal prefixes to voice markers). Although the latter are not cases of grammaticalization \textit{stricto sensu}, they are nevertheless  highly relevant to grammaticalization theory, as they potentially provide examples of `missing links' in pathways of grammaticalization. Therefore, they are systematically included in this survey.

All grammatical elements whose origin can be traced without overdue speculation are treated in this paper. First, I discuss the noun phrase, in particular nominal morphology, pronouns, case markers are discourse markers. Second, I analyze the verbal template. Third, I present the historical origin of a selection of complex constructions. Fourth, I study two cases of degrammaticalization in Japhug.

 \section{Nominal categories} 
 
The inflection of Japhug nouns is quite limited in comparison with that of the verbs. There are no genders or noun classes, number (section \ref{sec:number}) and case (sections \ref{sec:benef} to \ref{sec:loc}) are marked by clitics, and the only inflectional category is that of possession (see section \ref{sec:pronouns}).  The only productive nominal derivation not involving compounding is that of comitative adverbs (section \ref{sec:comit}).

Topic and focus markers are also discussed in this section (\ref{sec:topic}), as their scope is nearly always on a noun phrase rather than a verb phrase.

\subsection{Number} \label{sec:number}

Japhug has two number markers, dual \ipa{ni} and plural \ipa{ra}. These clitics are not obligatory for non-singular arguments (even in the case of human referents), and do not necessary trigger plural or dual agreement on the verb. The dual \ipa{ni} is obviously related to the numeral \ipa{ʁnɯz} < *\ipa{q-nis} `two', exemplifying the well-attested pathway \textsc{two} $\rightarrow$ \textsc{dual}. The etymology of the plural marker \ipa{ra} is unknown; a potential cognate exists in Pumi (\ipa{=ɹə}, cf \citealt[135]{daudey14grammar}; Japhug \ipa{-a} regularly corresponds to Pumi \ipa{-ə}).
 
  The plural marker \ipa{ra} can express plurality or collective meaning, as in example \ref{ex:ra1}; it is not incompatible with numerals, as shown in \ref{ex:ra2}.
 
\begin{exe}
\ex \label{ex:ra1}
\gll \ipa{ɯ-kʰa} 	\ipa{ra} 	\ipa{nɯ-mɤ-kɤ-sɯz} 	\ipa{nɤ} 	\ipa{ɯʑo} 	\ipa{kɯ} 	\ipa{qɤjɣi} 	\ipa{χsɯm} 	\ipa{lo-βzu} \\
\textsc{3sg.poss}-house \textsc{pl} \textsc{3pl.poss-neg-inf}-know \textsc{lnk} \textsc{3sg} \textsc{erg} bread three \textsc{ifr}-make \\
\glt She made three pieces of bread without her relatives knowing. (the raven, 108)
\end{exe}

\begin{exe}
\ex \label{ex:ra2}
\gll 
 \ipa{rɟɤlpu} 	\ipa{nɯ} 	\ipa{kɯ} 	\ipa{kɯki} 	\ipa{tɯrme} 	\ipa{kɯtʂɤɣ} 	\ipa{ra} 	\ipa{nɯ-ɕki}  	\ipa{to-ti} \\
 king \textsc{dem} \textsc{erg} this people six \textsc{pl} \textsc{3pl-dat} \textsc{ifr}-say  \\
 \glt The king told these six men. (Liuhaohan zoubian tianxia, 200)
\end{exe}

In addition, the marker \ipa{ra} can indicate approximative time or location  (see section  \ref{sec:loc}).
 


 \subsection{Possession and pronouns} \label{sec:pronouns}
 
Japhug  nouns can be divided into two main categories, inalienably possessed nouns and alienably possessed nouns. The former must take a possessive prefix, even when the possessor is non-specific. In that case, the indefinite possessor \ipa{tɯ-/tɤ-} or generic possessor \ipa{tɯ-} prefixes are used.

Pronouns  and possessive prefixes are very similar (see Table \ref{tab:pronoun}), but it appears that in Japhug pronouns are derived \textit{from} possessive prefixes rather than the opposite: pronouns other than \textsc{3du} and \textsc{3pl} are build by combining the possessive prefix with the root \ipa{-ʑo} meaning originally `self' (a common source for pronouns, see \citealt{heine11pronoms}). Japhug thus exemplifies a pathway \textsc{pronominal affix} $\rightarrow$ \textsc{pronoun}. It is not however a case of degrammaticalization in the strict sense, since the bound pronominal prefixes have not become free morphemes by themselves.

The \textsc{3du} and \textsc{3pl} are formed differently from the rest of the pronouns, by combining the \textit{status constructus} form \ipa{ʑɤ-} of the pronominal root /\ipa{ʑo}/ with the dual and plural markers \ipa{ni} and \ipa{ra}, with regular vowel harmony \ipa{ɤ} $\rightarrow$ \ipa{a} / \_Ca in the case of \ipa{ʑara} `they'.
 
 \begin{table}[H] \centering
\caption{Pronouns and possessive prefixes }\label{tab:pronoun}
\begin{tabular}{lllllllll} 
\toprule
 Free pronoun & Prefix & Person\\
\midrule
 \ipa{a-ʑo},    \ipa{ɤj} &	\ipa{a-}  &		1\textsc{sg} \\
\ipa{nɤ-ʑo},  \ipa{nɤj} &	\ipa{nɤ-}  &			2\textsc{sg}\\
\ipa{ɯ-ʑo}  &	\ipa{ɯ-}  &			3\textsc{sg}\\
\midrule
\ipa{tɕi-ʑo}  &	\ipa{tɕi-}  &			1\textsc{du} \\
\ipa{ndʑi-ʑo}  &	\ipa{ndʑi-}  &		2\textsc{du} \\	
\ipa{ʑɤ-ni}  &	\ipa{ndʑi-}  &		3\textsc{du} \\	
\midrule
\ipa{i-ʑo}    &	\ipa{i-}  &			1\textsc{pl} \\
\ipa{nɯ-ʑo}   &	\ipa{nɯ-}  &			2\textsc{pl} \\
\ipa{ʑa-ra}  &	\ipa{nɯ-}  &			3\textsc{pl} \\
\midrule
&  \ipa{tɯ-},  \ipa{tɤ-} & indefinite \\
\ipa{tɯ-ʑo} & \ipa{tɯ-}   &  generic\\
\bottomrule
\end{tabular}
\end{table}
 
 

 
 \subsection{Beneficiary}  \label{sec:benef}
The genitive \ipa{ɣɯ} (borrowed from Amdo Tibetan), in addition to marking the possessor, is the normal way to mark beneficiaries and can be used to mark recipient of some ditransitive verbs such as \ipa{kʰo} `give, pass over', as in \ref{ex:aZWG.nWkhAm}.

\begin{exe}
\ex \label{ex:aZWG.nWkhAm}
\gll
\ipa{ki} 	\ipa{kɯra} 	\ipa{ɲɯ-kʰam-a} 	\ipa{tɕe} 	\ipa{ki} 	\ipa{nɤki} 	\ipa{nɯ} 	\ipa{aʑɯɣ} 	\ipa{nɯ-kʰɤm.} \\
this \textsc{dem:pl} \textsc{ipfv}-give[III]-\textsc{1sg} \textsc{lnk} this \textsc{dem} \textsc{dem} \textsc{1sg:gen} \textsc{imp}-give[III] \\
\glt I will give you this, and you will give me that. (slobdpon, 130)
\end{exe}

Alternatively, when the predicate is a transitive verb with an overt object, the beneficiary can be marked as a possessive prefix on the object, as 	\ipa{a-tɯ-ci} \textsc{1sg-indef.poss}-water `water for me' in \ref{ex:atWci}.

\begin{exe}
\ex \label{ex:atWci}
\gll \ipa{χsɤr}  	\ipa{khɯtsa}  	\ipa{ɯ-ŋgɯ}  	\ipa{nɯ}  	\ipa{tɕu}  	\ipa{a-tɯ-ci}  	\ipa{ci}  	\ipa{tɤ-rke}  	\ipa{ma}  	\ipa{wuma}  	\ipa{ɲɯ-ɕpaʁ-a}  \\
gold bowl \textsc{3sg}-inside \textsc{dem} \textsc{loc} \textsc{1sg-indef.poss}-water \textsc{indef} \textsc{imp}-put.in[III] because really \textsc{sens}-be.thirsty-\textsc{1sg} \\
\glt Pour some water for me in the golden bowl, I am very thirsty.
\end{exe}

\subsection{Dative}  \label{sec:dat}
The dative \ipa{ɯ-ɕki}, used to mark the recipient of indirective ditransitive verbs as in (\ref{ex:tathu}), derives from a relator noun  meaning `side', a meaning still marginally present in Japhug in examples like (\ref{ex:WCki.loc}).

 \begin{exe}
   \ex   \label{ex:tathu}
 \gll \ipa{tɤ-pɤtso}  	\ipa{ra}  	\ipa{kɯ}  	\ipa{nɯ-sloχpɯn}  	\ipa{ɯ-ɕki}  	\ipa{to-tʰu-nɯ}  \\
\textsc{indef.poss}-child \textsc{pl} \textsc{erg} \textsc{3pl.poss}-teacher \textsc{3sg-dat} \textsc{ifr}-ask-\textsc{pl} \\
\glt The children asked their teacher. (Looking at the snow, 11)
   \end{exe}  

\begin{exe}
\ex \label{ex:WCki.loc}
\gll \ipa{ɯ-rte} 	\ipa{nɯ} 	\ipa{ɯ-rna} 	\ipa{ɯ-ɕki} 	\ipa{pɯ-kɯ-ɴqoʁ} 	\ipa{nɯnɯ} 	\ipa{pjɤ-mɟa} 	\ipa{tɕe} 	\ipa{ɯ-ku} 	\ipa{ɯ-taʁ} 	\ipa{to-ta} \\
\textsc{3sg.poss}-hat \textsc{dem} \textsc{3sg.poss}-ear \textsc{3sg}-\textsc{dat} \textsc{pfv:down-nmlz}:S/A-hang \textsc{dem} \textsc{ifr:down}-take \textsc{lnk} \textsc{3sg.poss}-head \textsc{3sg}-on \textsc{ifr}-put \\
\glt He took the hard that was hanging on his ear and put it on his head. (140505 liuhaohan zoubian tianxia, 164)
\end{exe}


Japhug thus attests a grammaticalization pathway \textsc{side} $\Rightarrow$ \textsc{locative} $\Rightarrow$ \textsc{dative}.

\subsection{Comitative adverbs}  \label{sec:comit}
Japhug and other Gyalrong languages have a productive derivation whereby a comitative adverb can be derived from a noun by removing all possessive prefixes, adding the prefix \ipa{kɤɣɯ-} and partially reduplication the last syllable of the noun stem, as in \ipa{tɤ-rte} `hat' $\Rightarrow$ \ipa{kɤɣɯ-rtɯ\tld{}rte} `together with (his) hat', as illustrated by (\ref{ex:kAGWrtWrte}).

\begin{exe}
\ex \label{ex:kAGWrtWrte}
\gll \ipa{kɤɣɯ-rtɯ\tld{}rte} 	\ipa{ʑo} 	\ipa{kha} 	\ipa{ɯ-ŋgɯ} 	\ipa{lɤ-tɯ-ɣe} 	\\
\textsc{comit}-hat \textsc{emph} house \textsc{3sg}-inside \textsc{pfv}-2-come[II] \\
\glt You came inside the house with your hat. (You were expected to take it off before coming in)
\end{exe}

As shown in \citet{jacques16comitative}, these adverbs originate from the combination of the S/A-participle \ipa{kɯ-} with the denominal derivation prefix \ipa{aɣɯ-}, which derives proprietive stative verbs from nouns, as shown in Table (\ref{tab:aGW}).

\begin{table}[h] \centering
\caption{The denominal prefix \ipa{aɣɯ-}}\label{tab:aGW}
\resizebox{\columnwidth}{!}{
\begin{tabular}{lllllllll} 
\toprule
Base noun & Meaning & Denominal verb & Meaning\\
\midrule
\ipa{tɯ-ɣli} & excrement, manure & \ipa{aɣɯ-ɣli} & producing a lot of manure (of pigs) \\
\ipa{tɤ-lu} & milk &\ipa{aɣɯ-lu} & producing a lot of milk (of cows) \\
\ipa{tɯ-mɲaʁ} & eye & \ipa{aɣɯ-mɲaʁ} & having a lot of holes \\
\ipa{tɯ-ɕnaβ} & snot & \ipa{aɣɯ-ɕnɯ\tld{}ɕnaβ} & be  slimy \\
\ipa{ɯ-mdoʁ} & colour & \ipa{aɣɯ-mdoʁ} & having the same colour \\
\bottomrule
\end{tabular}}
\end{table}

Comitative adverbs are actually homophonous with the S/A-participle of such verbs, as shown by examples \ref{ex:kAGWrJWrJit} and \ref{ex:kAGWrJWrJit2}, which present a minimal pair contrasting the comitative adverb  `with his/her children' on the one hand and the participle  `having many children' on the other hand (both derived from the possessed noun  \ipa{tɤ-rɟit} `child'), both pronounced \ipa{kɤɣɯrɟɯrɟit}.

\begin{exe}
\ex \label{ex:kAGWrJWrJit}
\gll   
\ipa{iɕqha} 	\ipa{tɕʰeme} 	\ipa{nɯ} 	\ipa{kɯ-ɤɣɯrɟɯrɟit} 	\ipa{ci} 	\ipa{pɯ-ŋu}  \\
the.aforementioned woman \textsc{dem} \textsc{nmlz}:S/A-have.many.children \textsc{indef} \textsc{pst.ipfv}-be \\
\glt This woman had a lot of children.
\end{exe}

\begin{exe}
\ex \label{ex:kAGWrJWrJit2}
\gll   
\ipa{kɤɣɯ-rɟɯ\tld{}rɟit} 	\ipa{ʑo} 	\ipa{jo-nɯ-ɕe-nɯ} \\
\textsc{comit}-children \textsc{emph} \textsc{ifr-vert}-go-\textsc{pl} \\
\glt She/They went back with their children.
\end{exe}

Ambiguous sentences like \ref{ex:kAGWrtWrtaR2} actually constitute the pivot constructions which allow reanalysis in contexts where both proprietive (`having X') and a comitative (`with X') interpretations were possible.


  \begin{exe}
\ex \label{ex:kAGWrtWrtaR2}
\gll   
  \ipa{si} 	\ipa{kɤɣɯrtɯrtaʁ} \ipa{ɲɯ-ɕar-nɯ} \\
  tree \textsc{nmlz}:S/A-have.many.branches//\textsc{comit}-branch \textsc{ipfv}-search-\textsc{pl} \\
\glt `They search for a tree having a lot of branches' $\Rightarrow$ `They search for a tree and/with its branches'
\end{exe}

This is thus a particular instance of a pathway \textsc{proprietive} $\Rightarrow$ \textsc{comitative}, which may be attested in other language families (\citealt{stassen00and, stolz06comitative, arkhipov09comitative}).

\subsection{Comparee and standard}
In the comparative construction, both the comparee and the standard are marked, respectively by the postpositions \ipa{kɯ} and \ipa{sɤz}  (example \ref{ex:comp1}). The marker \ipa{kɯ} on the comparee is obligatory only if the standard is not overt, otherwise it is optional.

\begin{exe}
\ex \label{ex:comp1}
\gll  [\ipa{ɯ-ʁi}]_{standard}   	\ipa{sɤz}   	[\ipa{ɯ-pi}   	\ipa{nɯ}]_{comparee}   	\ipa{\textbf{kɯ}}   	\ipa{mpɕɤr}     \\
\textsc{3sg.poss}-younger.sibling \textsc{comparative} \textsc{3sg.poss}-elder.sibling \textsc{dem} \textsc{erg}  be.beautiful:\textsc{fact} \\
\glt `The elder one is more beautiful than the young one.' (elicited)
\end{exe}

The mark \ipa{kɯ} on the comparee is etymologically related to the ergative \ipa{kɯ} (borrowed from Amdo Tibetan). The complex grammaticalization pathway leading from ergative to comparee marker is presented in \citet{jacques16comparative}. 

This unusual pathway \textsc{ergative} $\Rightarrow$ \textsc{comparee} rather than the more common \textsc{ergative} $\Rightarrow$ \textsc{standard} is all the more surprising as the ergative \ipa{kə/ɣə} in Amdo Tibetan from which the Japhug ergative \ipa{kɯ} was borrowed is used for the \textit{standard} in the comparative construction.

The marker \ipa{sɤz} contains the locative suffix \ipa{-z} (which also appears as a tautosyllabic postposition \ipa{zɯ}), but the etymology of the first element \ipa{sɤ-} is unknown. 

 \subsection{Location}   \label{sec:loc}
In Japhug, there are four distinct (non mutually exclusive) ways of marking locative adjuncts. First, locative and temporal adjunct are commonly left unmarked. Second, they can  take the locative postpositions \ipa{zɯ} or \ipa{tɕu}.  Third, relator (possessed) nouns such as \ipa{ɯ-ŋgɯ} `inside', \ipa{ɯ-taʁ} `on', \ipa{ɯ-pa} `under', \ipa{ɯ-rkɯ} `side' can be used for more specific locations. They can be followed by the locative \ipa{zɯ} or \ipa{tɕu} as in (\ref{ex:WrkW.zW}).

\begin{exe}
\ex \label{ex:WrkW.zW}
\gll \ipa{kha} 	\ipa{ɯ-rkɯ} 	\ipa{zɯ} 	\ipa{nɯnɯ} 	\ipa{qajɯ} 	\ipa{pɯ-nnɯ-ŋu,} 	\ipa{tɯrdoʁ} 	\ipa{kɯ-fse,} 	\ipa{tɤ-rɤku} 	\ipa{pɯ-kɯ-ʁndɤr} 	\ipa{kɯ-fse} 	\ipa{nɯra} 	\ipa{ɣɯ-tu-ndze} 	\ipa{ɲɯ-ŋu} \\
house \textsc{3sg.poss}-side \textsc{loc} \textsc{dem} worm \textsc{pst.ipfv-auto}-be grain \textsc{nmlz}:S/A-be.like \textsc{indef.poss}-cereals \textsc{pfv-nmlz}:S/A-be.spilled \textsc{nmlz}:S/A-be.like \textsc{cisloc-ipfv}-eat[III] \textsc{sens}-be \\
\glt (During winter,) it comes near the house to eat worms or grains that have been spilled (on the ground). (23-pGAYaR, 94)
\end{exe}

These postpositions are however optional, as shown by the following example from the same story as (\ref{ex:WrkW.zW}).

\begin{exe}
\ex \label{ex:kha.WrkW}
\gll
\ipa{kha} 	\ipa{ɯ-rkɯ} 	\ipa{kɤ-ɣi} 	\ipa{wuma} 	\ipa{ʑo} 	\ipa{rga} \\
house \textsc{3sg.poss}-side  \textsc{inf}-come really \textsc{emph} \textsc{like}:fact \\
\glt It likes to come near the house. (23-pGAYaR, 95)
\end{exe}

Fourth, the plural marker \ipa{ra} can indicate approximate location, as in \ref{ex:khara}. This use of \ipa{ra} is reminiscent of plural markers in Kirghiz and Old Japanese, which combine collective, hypocoristic and approximate locative meanings see \citealt[195]{antonov07ra}).

\begin{exe}
\ex \label{ex:khara}
\gll
\ipa{tɯ-zda} 	\ipa{nɯ} 	\ipa{ma} 	\ipa{kɯmaʁ} 	\ipa{tɯrme} 	\ipa{a-pɯ-me} 	\ipa{tɕe,} 	\ipa{kha} 	\ipa{ra} 	\ipa{aʁɤndɯndɤt} \ipa{ɲɯ-ɤnɯɣro} 	\ipa{ɲɯ-ŋu} 	\ipa{ɲɯ-ti.} \\
\textsc{indef.poss}-companion \textsc{dem} apart.from other people \textsc{irr-ipfv}-not.exist \textsc{lnk} house \textsc{pl} everywhere \textsc{sens}-play \textsc{sens}-be \textsc{sens}-say \\
\glt He says that (the young monkey) would play everywhere in the house whenever there are no other people (apart from members of the family). (19 GZW2, 10)
\end{exe}


   \subsection{Topic and focus} \label{sec:topic}
Topic and focus markers do not strictly belong to nominal markers, since they can have scope over verb phrases or even entire sentences, but since they are mainly used with noun phrases, they are nevertheless included in this section. Three of the discourse markers in Japhug have clear etymologies: the delimitative topic \ipa{pɯpɯŋunɤ} `as for ...', the aforementioned topic \ipa{iɕqha} and the unexpected focus \ipa{rcanɯ}.

  \subsubsection{Delimitative topic}
The delimitative topic marker \ipa{pɯpɯŋunɤ} `as for ...' is transparently derived from the conditional past imperfective form be the verb `be' (as in \ref{ex:pWpWNunA1}), meaning originally `if it/he/she is ...'.

\begin{exe}
\ex \label{ex:pWpWNunA1}
\gll \ipa{pɯ\tld{}pɯ-ŋu} \ipa{nɤ} \\
\textsc{cond}\tld{}-\textsc{pst.ipfv}-be if \\
\glt If it is ...
\end{exe}

However, the grammaticalized status of this marker is obvious when the element marked with \ipa{pɯpɯŋunɤ} `as for ...' is a first or second person pronoun, as in example (\ref{ex:pWpWNunA}).

\begin{exe}
\ex \label{ex:pWpWNunA}
\gll \ipa{nɤʑo}	\ipa{pɯpɯŋunɤ,}	\ipa{ɬɤndʐi}	\ipa{ra}	\ipa{ɣɯ}	\ipa{nɯ-kɯ-βʁa,}	\ipa{nɯ-rɟɤlpu}	\ipa{tɯ-ŋu} \\
\textsc{2sg} as.for demon \textsc{pl} \textsc{gen} \textsc{3pl.poss}-\textsc{nmlz}:S/A-be.victorious \textsc{3pl.poss}-king 2-be:\textsc{fact} \\
\glt You, you are the king of the demons. (hist140512 fushang he yaomo, 61)
\end{exe}
Here, the form of the topic markers remains \ipa{pɯpɯŋunɤ}, without agreement: if this still were a conjugated verb, the second person singular form shown in (\ref{ex:pWpWtWNunA1}) would have been expected instead.

\begin{exe}
\ex \label{ex:pWpWtWNunA1}
\gll \ipa{a-slama} \ipa{pɯ\tld{}pɯ-tɯ-ŋu} \ipa{nɤ} \\
\textsc{1sg.poss}-student \textsc{cond}\tld{}-\textsc{pst.ipfv}-2-be if \\
\glt If you were my student...
\end{exe}

This shows that \ipa{pɯpɯŋunɤ} is not a proper verb form anymore, and has been fully grammaticalized.

\subsubsection{Aforementioned topic}
The temporal adverb \ipa{jiɕqha} `just before' has become grammaticalized as a pre-nominal determiner `the aforementioned' expressing that the nominal in question has been referred to previously in the discourse, though not in the last few sentences. In example \ref{ex:jiCqha}, for instance, the leaf is mentioned four sentences before.

\begin{exe}
\ex \label{ex:jiCqha}
 \gll \ipa{tɯmɯ} 	\ipa{ci} 	\ipa{tɕhɤrnaʁ} 	\ipa{ci} 	\ipa{tɕhɯmtɕhɯm} 	\ipa{ko-lɤt.} 	\ipa{tɕendɤre} 	\ipa{jiɕqha} 	\ipa{tɤ-jwaʁ} 	\ipa{nɯ} 	\ipa{pjɤ-nɯndzom} 	\ipa{tɕe,} 	\ipa{ɯ-ʁi} 	\ipa{ɯ-kɯr} 	\ipa{ɯ-ŋgɯ} 	\ipa{nɯ} 	\ipa{tɕu} 	\ipa{tɯ-ci} 	\ipa{χsɯ-ntɕhaʁ} 	\ipa{jamar} 	\ipa{pjɤ-ɕe.} \\
 rain \textsc{indef}  rain \textsc{indef} \textsc{idph}.II:little.rain \textsc{ifr}-throw \textsc{lnk} the.aforementioned \textsc{indef.poss}-leaf \textsc{dem} \textsc{ifr:down}-flow.along \textsc{lnk} \textsc{3sg.poss}-younger.sibling  \textsc{3sg.poss}-mouth \textsc{3sg}-inside \textsc{dem} \textsc{loc} \textsc{indef.poss}-water three-drop about \textsc{ifr:down}-go \\
 \glt There was a little rain, and (the water) flowed along the leaf (that the elder brother had placed) and three drops of water flowed into his younger brother's mouth. (Smanmi 11, 61)
\end{exe}

 \subsubsection{Unexpectedness}
The marker \ipa{rcanɯ} topicalizes the preceding noun phrase and emphasizes the unexpectedness of the situation or event described by the phrase that follows, as in \ref{ex:YAwGsWGYaRnW}, where the blackening of the sparrows surprised (and amused) the person telling the story.

%\begin{exe}
%\ex \label{ex:tAmdzArgi}
%\gll 	\ipa{tɤmdzɤrgi} 	\ipa{nɯ} 	\ipa{ɣɯ} 	\ipa{ɯ-di} 	\ipa{ci} 	\ipa{tu} 	\ipa{tɕe,} 	\ipa{nɯ} 	\ipa{ɯ-di} 	\ipa{nɯnɯ} 	\ipa{rcanɯ} 	\ipa{tɯ-ku} 	\ipa{ʑo} 	\ipa{ɕɯ-mŋɤm.}  \\
%thistle \textsc{dem} \textsc{gen} \textsc{3sg.poss}-smell \textsc{indef} exist:\textsc{fact} \textsc{lnk} \textsc{dem} 
%\textsc{3sg.poss}-smell \textsc{dem}  \textsc{unexpected} \textsc{genr:poss}-head \textsc{emph}  \textsc{caus}-hurt:\textsc{fact} \\
%\glt The thistle has a smell, and this smell makes one's head hurt. (15 babW, 82)
%\end{exe}

\begin{exe}
\ex \label{ex:YAwGsWGYaRnW}
\gll 
\ipa{tɕendɤre} 	\ipa{thɯ-kɤ-βlɯ} 	\ipa{nɯ} 	\ipa{ɲɯ-ɕti} 	\ipa{tɕe,} 	<yancong>	 \ipa{ɯ-ŋgɯ} 	\ipa{ɲɯ-ɲaʁ} 	\ipa{rcanɯ} 	\ipa{kumpɣɤtɕɯ} 	\ipa{ra} 	\ipa{ɲɤ́-wɣ-sɯɣ-ɲaʁ-nɯ} 	\ipa{ʑo} 	\\
\textsc{lnk} \textsc{pfv-nmlz}:P-burn \textsc{dem} \textsc{sens}-be:\textsc{assert} \textsc{lnk} chimney \textsc{3sg}-inside \textsc{sens}-be.black \textsc{unexpected} sparrow \textsc{pl} \textsc{ifr-inv-caus}-be.black-\textsc{pl} \textsc{emph} \\
\glt Because there has been burning going on, the inside of the chimney is black, and it made the sparrows (who had build a nest inside it) become (completely) black! (22 kWmpGAtCW, 72)
\end{exe}
When it occurs before an adjectival verb, whether in finite or non-finite form as \ipa{kɯ-dɯ\textasciitilde{}dɤn} `numerous' in \ref{ex:kWdWdAn}, or before an ideophone (\ref{ex:RYJliRYJli}), \ipa{rcanɯ} indicates high degree. Adjectival verbs in this case often have emphatic reduplication.

\begin{exe}
\ex \label{ex:kWdWdAn}
\gll \ipa{tɕe} 	\ipa{nɯ} 	\ipa{ɕoŋtɕa} 	\ipa{rcanɯ} 	\ipa{kɯ-dɯ\textasciitilde{}dɤn} 	\ipa{ʑo} 	\ipa{pjɤ-sɯ-phɯt-nɯ.} \\
\textsc{lnk} \textsc{dem} wood \textsc{unexpected} \textsc{nmlz}:S/A-\textsc{emph}\textasciitilde{}-be.a.lot \textsc{emph} \textsc{ifr-caus}-chop-\textsc{pl}\\
\glt And they had (people) chop quite lot of wood (for them) (28 qAjdo,  103)
\end{exe}

\begin{exe}
\ex \label{ex:RYJliRYJli}
\gll
\ipa{ɯ-phoŋbu} 	\ipa{nɯ} 	\ipa{rcanɯ} 	\ipa{ʁɲɟliʁɲɟli} 	\ipa{ʑo} 	\ipa{ɲɯ-pa} \\
\textsc{3sg.poss}-body \textsc{dem}  \textsc{unexpected} \textsc{idph}:II:huge;massive \textsc{emph} \textsc{sens}-\textsc{aux} \\
\glt Its body is huge. (20 sWNgi,  16)
\end{exe}

This marker is derived from the possessed noun  \ipa{ɯ-rca} `following, together with' (see example \ref{ex:arca}) together with the distal demonstrative \ipa{nɯ} `that'.

\begin{exe}
\ex \label{ex:arca}
\gll
\ipa{aʑo} 	\ipa{a-rca} 	\ipa{kɤ-ɣi} 	\ipa{mɤ-tɯ-cha} \\
\textsc{1sg} \textsc{1sg}-following \textsc{inf}-come \textsc{neg}-2-can:\textsc{fact} \\
\glt You cannot come with me.
\end{exe}
 
The evolution from `following, together with' to `unexpectedness' is not completely straightforward. The pathway \textsc{together with} $\Rightarrow$ \textsc{also} (non-scalar additive) $\Rightarrow$ \textsc{even} (scalar additive) $\Rightarrow$ \textsc{unexpectedness} can be proposed.\footnote{This pathway was suggested by Nat Krause and Linda Konnerth.} It involves intermediate stages that are all attested: in particular, many languages have the same morpheme for scalar and non-scalar additives (for instance Karbi, see \citealt{konnerth14additive}), and the directionality is clearly from non-scalar to scalar additives.

However, this hypothesis can only be confirmed if traces of the proposed intermediate stages are discovered in other Gyalrong languages.

  \section{Verbal categories} 

\subsection{Person indexation} 
Japhug and other Rgyalrongic languages have a polypersonal indexation system that includes several morphemes with cognates in Kiranti languages (for instance, the second person *\ipa{tə-} prefix, the inverse *\ipa{wə-} and the direct third person object *\ipa{-w}), and appears to be at least in part of proto-Sino-Tibetan origin, though this issue is controversial (see \citealt{delancey11prefixes, jacques12agreement}), and thus cannot be included in this paper.


I focus here on generic person markers and portmanteau prefixes for local scenario 1$\rightarrow$2 and 2$\rightarrow$1, which have relatively transparent etymologies, and briefly discuss possible origins for the inverse prefix.


\subsubsection{Generic person prefixes}

Japhug has a system of generic person marking with ergative alignment (\citealt{jacques12demotion}), in which generic S/P are marked by the prefix \ipa{kɯ-} (examples \ref{ex:pWkWNu} and \ref{ex:tukWCWngo}), while generic A is marked by the prefix \ipa{wɣ-} (example \ref{ex:genr.tWrme}).


\begin{exe}
\ex \label{ex:pWkWNu}
\gll
\ipa{tɕeri} 	\ipa{tɤ-pɤtso} 	\ipa{pɯ-kɯ-ŋu} 	\ipa{tɕe,} 	\ipa{nɯ} 	\ipa{kɤ-ndza} 	\ipa{wuma} 	\ipa{ʑo} 	\ipa{pɯ-kɯ-rga.} \\
but \textsc{indef.poss}-child \textsc{pst.ipfv-genr}:S/P-be \textsc{lnk} \textsc{dem} \textsc{inf}-eat really \textsc{emph} \textsc{pst.ipfv-genr}:S/P-like \\
\glt When (we) were children, (we) liked it a lot. (12: ndZiNgri, 135)
\end{exe}


\begin{exe}
\ex \label{ex:tukWCWngo}
\gll  \ipa{tɕe} 	\ipa{ʁja} 	\ipa{nɯnɯ} 	\ipa{tɯ-qʰoχpa} 	\ipa{a-mɤ-tʰɯ-ɕe} 	\ipa{ra} 	\ipa{ma} 	\ipa{tu-kɯ-ɕɯ-ngo} 	\ipa{ɲɯ-ɕti} \\
\textsc{lnk} rust \textsc{top} \textsc{genr.poss}-inner.organ \textsc{irr-neg-pfv:downstream}-go have.to:\textsc{fact} \textsc{lnk} \textsc{ipfv-genr:S/P-caus}-be.sick  \textsc{testim}-be:\textsc{assert} \\
\glt Rust should not go into one's organs, otherwise it would cause one to get sick. (30: Com, 86)
\end{exe}

\begin{exe}
\ex  \label{ex:genr.tWrme}
\gll
\ipa{tɯrme} 	\ipa{kɯ} 	\ipa{tú-wɣ-ndza} 	\ipa{mɤ-sna.} \\
people \textsc{erg} \textsc{ipfv-inv}-eat \textsc{neg}-be.good:\textsc{fact} \\
\glt It is not edible. (11: paRzwamWntoR, 39)
\end{exe}

The generic human S/P prefix \ipa{kɯ-} is homophonous with the S/A participle prefix, illustrated by examples (\ref{ex:part.S}) and (\ref{ex:part.A}).

\begin{exe}
\ex \label{ex:part.S}
\gll \ipa{spjaŋkɯ}	\ipa{kɤ-kɯ-nɯʑɯβ}	\\
wolf \textsc{pfv-nmlz}:S/A-sleep \\
\glt The wolf which had fallen asleep.
\end{exe}

\begin{exe}
\ex \label{ex:part.A}
\gll \ipa{ɯ-kɯ-ndza} \\
\textsc{3sg-nmlz}:S/A-eat \\
\glt The one who eats it.
\end{exe}

The generic human \ipa{kɯ-} arose most probably due to the reanalysis of participles as finite verbs. The exact scenario for this grammatical change is too complex to be presented here in detail (see \citealt{jacques17generic} for a complete account), but the general lines are as follows.

There is evidence that the ancestor of \ipa{kɯ-} in proto-Gyalrong could be used to mark generic human for all S, A and P. First, there are two irregular verbs in Japhug, \ipa{ti} `say' and \ipa{sɯz} `know' which form their generic A with \ipa{kɯ-} rather than \ipa{wɣ-}. Second, other Gyalrong  languages (Tshobdun and Situ, see \citealt{sun14generic}) use the cognates of \ipa{kɯ-} for all generic human core arguments.
 
The ancestor of Japhug \ipa{kɯ-}, proto-Gyalrong *\ipa{kə-}, was a general nominalizer that could be used for S-, A- and P-participles, and which was reanalyzed as a generic human marker for these three syntactic roles already in proto-Gyalrong. In Japhug \ipa{kɯ-} became replaced by the inverse \ipa{wɣ-} to mark generic human in A function.\footnote{The generic human A prefix  \ipa{wɣ-} in Japhug is homophonous with the inverse marker \ipa{wɣ-}, and presents exactly the same morphological alternations (in particular, it is one of the very few prefixes to attract stress and to be infixable with the progressive \ipa{asɯ-}). It is possible to synchronically analyze it  as a particular instance of the use of the inverse, supposing an Empathy Hierarchy where generic human are lower than inanimates (see \citealt{jacques10inverse, jacques12demotion} for more details on the use of the inverse in Japhug). 

\begin{exe}
\ex 
\glt 1/2 > 3 animate > 3 inanimate > 3 generic human
\end{exe}
}

Japhug thus attests the following two paths of grammatical change:

\begin{exe}
\ex 
\glt \textsc{nominalization} > \textsc{generic} 
\ex 
\glt \textsc{inverse} > \textsc{generic A}
\end{exe}

\subsubsection{Portmanteau prefixes}
The Japhug transitive conjugation includes two portmanteau prefixes for local scenarios \ipa{ta-} 1$\rightarrow$2 and \ipa{kɯ-} 2$\rightarrow$1.  The non-local forms taking these prefixes in Gyalrong languages have suffixes coreferent with the P, as illustrated by examples (\ref{ex:2.1sg}) and (\ref{ex:1.2du}).

\begin{exe}
\ex \label{ex:2.1sg}
\gll \ipa{pɯ-kɯ-nɤjo-a} \\
\textsc{imp}-2$\rightarrow$1-wait-\textsc{1sg} \\
\glt Wait for me (heard in context).
\end{exe}

\begin{exe}
\ex \label{ex:1.2du}
\gll  \ipa{maka} 	\ipa{ʑo} 	\ipa{mɤ-ta-βde-ndʑi} \\
at.all \textsc{emph} \textsc{neg}-1$\rightarrow$2-leave-\textsc{du} \\
\glt I will never abandon you two. (140507 tangguowu, 166)
\end{exe}

The portmanteau prefixes for 1$\rightarrow$2 and 2$\rightarrow$1 are nearly identical in Situ, Tshobdun and Zbu, as presented in Table \ref{tab:local.rgy} (data from \citealt[218]{linxr93jiarong}, \citealt{jackson02rentongdengdi}, \citealt{jacques12agreement}, \citealt{gongxun14agreement}).

\begin{table}[H]
\caption{Local scenario prefixes in Rgyalrong languages} \centering \label{tab:local.rgy} 
\begin{tabular}{lllll}
\toprule
& $1\rightarrow2$ & $2\rightarrow1$ \\
\midrule
Japhug &  \ipa{ta-} & \ipa{kɯ-} \\
Tshobdun &  \ipa{tɐ-} & \ipa{kə-o-}, \ipa{tə-o-} \\
Zbu &  \ipa{tɐ-} &\ipa{kə-w-}, \ipa{tə-w-} \\
Situ &  \ipa{ta-} & \ipa{kə-w-} \\
\bottomrule
\end{tabular}
\end{table}

Other Gyalrong languages only differ from Japhug in two regards: Japhug does not have the inverse \ipa{wɣ-} prefix in the  2$\rightarrow$1 form, and Zbu and Tshobdun allow an alternative form with the second person prefix \ipa{tə-} and the inverse prefix. In all four languages, the verb receives suffixes coreferent with the patient (second person in 1$\rightarrow$2 and first person in  2$\rightarrow$1).\footnote{All languages apart from Situ allow double suffixation in 2$\rightarrow$\textsc{1sg}, with the dual or plural suffixes stacked after the first person, as in Japhug \ipa{ɲɯ-kɯ-mbi-a-nɯ} \textsc{ipfv}-2$\rightarrow$1-give-\textsc{1sg-pl} `you^{\textsc{pl}} (will have to) give (her) to me.'}


  A possible explanation for the 1$\rightarrow$2  prefix is a combination between the second person prefix \ipa{tɯ-} and the agentless passive \ipa{a-}, which yields the expected form in all four languages. In this view, a form such as \ipa{ta-no-n} 1$\rightarrow$2-chase-\textsc{2sg} `I will chase you^{\textsc{sg}}' (\citealt[219]{linxr93jiarong}) would have developped through the following stages:\footnote{Proto-Rgyalrong follows the preliminary sound laws presented in \citet{jacques04these}.}

\begin{itemize}
\item *\ipa{tə-ŋa-naŋ-nə}  2-\textsc{pass}-chase-\textsc{2sg} `you will be chased' (Passive form)
\item *\ipa{ta-naŋ-nə}  2:\textsc{pass}-chase-\textsc{2sg} (Regular phonological fusion between the person marker and the passive prefix, attested in all four Rgyalrong languages)  
\item  *\ipa{ta-naŋ-nə}  1$\rightarrow$2-chase-\textsc{2sg} `I will chase you' (reanalysis of the fused form as a portmanteau prefix; the unspecified agent of the passive construction is construed as being first person)
\item  \ipa{ta-no-n} 1$\rightarrow$2-chase-\textsc{2sg} `I will chase you^{\textsc{sg}}' (regular sound changes)
\end{itemize}


In the case of 2$\rightarrow$1, the phonetic identity of this prefix with the nominalizer / generic in all four languages is striking. If, as suggested above, the grammaticalization of the nominalizer \ipa{kə-} as a generic person marker goes back to the common ancestor of all four Rgyalrong languages and not simply that of Japhug and Tshobdun, we may interpret the origin of a form such as \ipa{kə-w-no-ŋ} `you will chase me' in the following way:


\begin{itemize}
\item *\ipa{kə-wə-naŋ-ŋa}  \textsc{genr}-\textsc{inv}-chase-\textsc{1sg} `someone will chase me' (generic form, with inverse since the SAP argument is patient)
\item  *\ipa{kə-wə-naŋ-ŋa}  2$\rightarrow$1-\textsc{inv}-chase-\textsc{1sg} `You will chase me' (reanalysis of the fused form as a portmanteau prefix; the unspecified agent of the passive construction is construed as being second person, ie, the SAP participant not otherwise indexed in the verb form)
\item  \ipa{kə-w-no-ŋ} 2$\rightarrow$1-chase-\textsc{1sg} `You will chase me'  (regular sound changes)
\end{itemize}

Note that in the Situ Gyalrong language, unlike Japhug and Tshobdun, nominalized forms in \ipa{kə-} (the cognate of Japhug \ipa{kɯ-}) are compatible with person affixes in particular conditions (see \citealt[11-12]{jacksonlin07}), as in \ref{ex:situ.tokEpEntS}, where the verb  come' bears the dual suffix \ipa{-ntʃ}. It is impossible to nominalize a verb in this way in the other Gyalrong languages.

 \begin{exe}
\ex \label{ex:situ.tokEpEntS}
\gll
\ipa{tərmî}  	\ipa{to-kə́-pə-ntʃ=tə}  	\ipa{tʂaʃī}  	\ipa{nɐrə}  	\ipa{ɬamō}  	\ipa{na-ŋôs-ntʃ}  	\\
person \textsc{pfv-nmlz}-come:\textsc{pst}-\textsc{du=det} Trashi and Lhamo \textsc{pst.ipfv}-be:\textsc{pst}-\textsc{du} \\
\glt The people who came were Trashi and Lhamo.
  \end{exe}

In this view, the absence of inverse marker in the  2$\rightarrow$1 form in Japhug is an innovation, which can be explained by the fact that the inverse is redundant in this form. This redundancy is solved in a different way in Zbu and Tshobdun, where at least speaker accept forms replacing the portmanteau \ipa{kə-} by the second person \ipa{tə-} (see \citealt{jackson02rentongdengdi} and \citealt{gongxun14agreement}).



 \subsubsection{Inverse prefix}
The inverse prefix \ipa{wɣ-} (proto-Gyalrong *\ipa{wə}) has cognates in many other languages of the family (\citealt{jacques12agreement}), in particular Kiranti (Bantawa \ipa{ɨ-}), and is not a Gyalrong-specific innovation. Given its antiquity, attempts at etymologizing this marker are necessarily speculative. 

There are two possibilities to account for the origin of this prefix, if it is indeed etymologizable. 

First, it could derive from the verb `to come' (Japhug \ipa{ɣi} < *\ipa{wi}, a verb widely attested in the Sino-Tibetan family), through the well-established pathway \textsc{come} > \textsc{venitive} > \textsc{cislocative} > \textsc{inverse} (\citealt{jacques14inverse}; see also \citealt{konnerth15cisloc} for a potential counterexample).

Second, it could originate from the third person possessive marker -- although in Japhug the two prefixes are dissimilar (\ipa{wɣ-} vs \ipa{ɯ-}), they are homophonous in all other languages (for instance, in Bantawa and Zbu). Such grammaticalization could have taken place through a nominalized form without nominalization prefix taking a third person possessive prefix, used in subordinate clauses that are later reinterpreted as main clauses.  Japhug indeed has a non-finite verb form of this type, the bare infinitive (a form discussed in particular in \citealt{jacques14antipassive}), which could be analogous to the hypothesized construction from which the inverse could have been developed.

These hypotheses must be considered to be preliminary until full reconstructions of proto-Gyalrongic and proto-Kiranti become available.

\subsection{Associated motion} \label{sec:motion}
Japhug has a simple associated motion system, with one translocative / andative prefix \ipa{ɕɯ-} and a cislocative / venitive prefix \ipa{ɣɯ-} transparently grammaticalized from the verbs \ipa{ɕe} `go' and \ipa{ɣi} `come' respectively. These prefixes are morphologically fully integrated, as illustrated by example \ref{maCthWtWZGABde}, where the translocative (in the allomorph \ipa{ɕ-}) appears closer to the root than the negation marker, and cannot bear any TAM or person marker.  

\begin{exe}
\ex \label{maCthWtWZGABde}
\gll \ipa{ma-ɕ-tʰɯ-tɯ-ʑɣɤ-βde} 	\ipa{ma} 	\ipa{nɤ-wa} 	\ipa{ɲɯ-ɤkʰu}   \\
\textsc{neg-transloc-imp-2-refl}-throw because \textsc{2sg.poss}-father \textsc{sens}-call \\
\glt `Don't throw yourself (in the river), your father is calling you' 
\end{exe}

In Situ, \citet{youjing03zhuokeji} notices that the cislocative has been further grammaticalized as marking prospective aspect.

Grammaticalization of motion verbs as prefixes is unexpected in a strict verb-final language like Japhug, especially since purposive complements of motion verbs are always preverbal. These prefixes therefore originate from a construction where the motion verbs appeared before the main verb, either in a serial verb construction or simple parataxis (\citealt{jacques13harmonization}).

\subsection{Voice} \label{sec:voice}
The main sources for voice markers in Japhug are denominal prefixes. Five of the voice derivation prefixes, namely the Antipassive, the Applicative, the Causative, the Passive and the Deexperiencer, are homophonous with denominal derivations with similar meanings, as shown in Table \ref{tab:denom}.

\begin{table}[H] \caption{Voice markers and corresponding denominal derivations} \label{tab:denom} \centering
\begin{tabular}{lllllllll} \toprule
Form& Voice & Corresponding denominal prefix \\
\midrule
\ipa{rɤ}- & Antipassive &    \ipa{rɤ}- (intransitive dynamic verbs)\\
\ipa{nɯ(ɣ)}- & Applicative &    \ipa{nɯ(ɣ)}- (transitive dynamic verbs)\\
\ipa{sɯ(ɣ)}- & Causative &    \ipa{sɯ(ɣ)}- (verb meaning `use X' or \\
&& `cause to have X') \\
\ipa{a}- & Agentless Passive &    \ipa{a}- (stative verb)\\
\ipa{sɤ}-  & Deexperiencer &    \ipa{sɤ}- (stative verb expressing a property)\\
    \bottomrule
\end{tabular}
\end{table}

These five voice derivations and their corresponding denominal origin are discussed in the following. In addition, voice derivations originating from markers other than denominal prefixes (in particular, the reflexive \ipa{ʑɣɤ-}) are briefly analyzed.

\subsubsection{Antipassive} \label{sec:apass}
The relationship between voice and derivation prefixes was first explained in the case of the Antipassive prefix \ipa{rɤ-} (\citealt{jacques14antipassive}), a prefix attested only in Gyalrong languages and not even found in Khroskyabs, their closest relative (\citealt{lai13affixale}).


The Antipassive derives from the intransitive denominal prefix \ipa{rɯ-/rɤ-} through a two stage pathway.

First, an action (or patient) nominal is derived from a transitive verb. This action nominal has the same form as the bare root of the verb, but is a possessed noun requiring a possessive prefix. For instance, from \ipa{ɕphɤt} `patch' one derives the possessed noun 
\ipa{-ɕphɤt} `a patch', which, in the absence of a definite possessor, must occur with the indefinite possessor prefix \ipa{tɤ-} (\ipa{tɤ-ɕphɤt}).

Second, intransitive derivation in \ipa{rɯ-/rɤ-} is applied to this possessed noun, yielding the form \ipa{rɤ-ɕphɤt} `to do patching'. Following the regular pattern, possessive prefixes are lost during denominal derivations, so that a form such as *\ipa{rɤ-tɤ-ɕphɤt} with the indefinite possessor prefix would not be expected.

The end form \ipa{rɤ-ɕphɤt} `to do patching' can then be reanalyzed as being directly derived from the base transitive verb \ipa{ɕphɤt} `patch', and since the S of this intransitive verb corresponds to the A of the transitive verb \ipa{ɕphɤt}, and the P is lost, this originally denominal prefix is reinterpreted as being an Antipassive marker. Then, this prefix is overgeneralized to most transitive verbs.

This reanalysis probably occurred recently in Japhug, as forms such as \ipa{rɤ-ɕphɤt} are still synchronically ambiguous between an Antipassive and a Denominal verb. Further evidence for this pathway can be found in irregular nominal forms, as there are several verb for which a semantic or morphological irregularity is shared between the Antipassive verb and the corresponding action/patient noun, but not the base transitive verbs, showing that the Antipassive form derives from the patient. For example, the intransitive verb \ipa{rɤ-nŋa} `owe money' is an irregular Antipassive form of \ipa{ŋa} `owe X'; the additional \ipa{-n-} is also found in the noun \ipa{-nŋa} `debt', showing that this irregular Antipassive historically derives from the noun \ipa{-nŋa} `debt' rather than directly from the transitive verb \ipa{ŋa} `owe X'.\footnote{See \citet{jacques14antipassive} for additional examples of common idiosyncrasies between action noun and antipassive verb.}

The pathway presented here can be summarized as (\ref{ex:pathway}):

\begin{exe}
\ex \label{ex:pathway}
\glt \textsc{action nominalization} of transitive verb + \textsc{intransitive denominal derivation} $\Rightarrow$ \textsc{antipassive}
\end{exe}

The general mechanism is that the action nominalization neutralizes the transitivity of the base verb, and that a new transitivity and argument structure is allocated by the denominal prefix.

 The same two-step pathway of grammaticalization proposed to account for the origin of the antipassive prefix can also be applied to four other voice derivation prefixes: the causative \ipa{sɯ-}, the deexperiencer \ipa{sɤ--}, the passive \ipa{a-} and the applicative \ipa{nɯ-}. Although for these derivations, unlike the antipassive case, we lack common (semantic or morphological) irregularities between action nouns and derived verbs, the semantics of the voicing markers and corresponding denominal derivations is very close.

\subsubsection{Causative}  \label{sec:causative}
The causative prefix \ipa{sɯ(ɣ)-} is one of the most productive derivation prefixes in Japhug, and can be applied to nearly all transitive and intransitive verbs (the detailed meaning of this prefix and the constructions in which it can be used in are described in \citealt{jacques15causative}). 

 The homophonous denominal prefix \ipa{sɯ(ɣ)-} derives verbs meaning `use X' or `cause to have X', such as \ipa{sɯ-ɕtʂi} `cause to sweat' (from the possessed noun \ipa{-ɕtʂi} `sweat'), or \ipa{sɯɣ-tshaʁ} `to sieve (=to use a sieve)' from \ipa{tshaʁ} `sieve'. 

The causative \ipa{sɯ(ɣ)-} can thus be explained as the result of reanalysis from the denominal derivation `cause to X' from a possessed action nominal deriving from the base verb:

\begin{exe}
\ex \label{ex:pathway2}
\glt \textsc{action nominalization} of verb + \textsc{causative denominal derivation} $\Rightarrow$ \textsc{causative}
\end{exe} 

 


\subsubsection{Deexperiencer} \label{sec:deexp}
 
 The deexperiencer prefix \ipa{sɤ-} derives stative verbs from intransitive verbs whose S is an experiencer or any non-agentive semantic role. The S of the deexperiencer verb corresponds to the stimulus. Examples include \ipa{rga} `like' $\rightarrow$ \ipa{sɤ-rga} `be lovable' or \ipa{ŋgio} `slip (of a human)' $\rightarrow$ \ipa{sɤ-ŋgio} `be slippery (of the ground)'(\citealt{jacques12demotion}).

There are a few examples of a denominal prefix \ipa{sɤ-} expressing a property related to the base noun, such as \ipa{-ndɤɣ} `poison' $\rightarrow$ \ipa{sɤ-ndɤɣ} `be poisonous' or \ipa{-mbrɯ} `anger'  $\rightarrow$ \ipa{sɤ-mbrɯ} `be angry'. The semantics of the deexperiencer derivation is closely related to that of the verb  \ipa{sɤ-ndɤɣ} `be poisonous': the property of an object that has effects on surrounding people. Here again, the deexperiencer can be hypothesized to derive from the denominal derivation in \ipa{sɤ-} a possessed action nominal deriving from the base verb, as in (\ref{ex:pathway3}).

\begin{exe}
\ex \label{ex:pathway3}
\glt \textsc{action nominalization} of verb + \textsc{property denominal derivation} $\Rightarrow$ \textsc{deexperiencer}
\end{exe}
\subsubsection{Passive} \label{sec:passive}
The passive \ipa{a-} is an agentless passive, which derives intransitive verbs whose S corresponds to the P of the base verb, as \ipa{ata}  `be put on/in' from \ipa{ta} `put'. The corresponding denominal prefix \ipa{a-} is used to derive a stative verb describing a shape related to the noun, or a visible / perceptible concrete property, as in \ipa{-ci} `water' $\rightarrow$  \ipa{aci} `be wet',  \ipa{ʑɤwu} `lame' $\rightarrow$  \ipa{aʑɤwu} `be lame' 
or \ipa{scaʁa} `magpie' $\rightarrow$  \ipa{ascaʁa} `be white and black (like a magpie)'.\footnote{The last two examples are nouns borrowed from Tibetan, showing that this derivation is fully productive.} 

The passive is mainly used in the text corpus with concrete action verbs (\ipa{a-ta} `be put on', \ipa{a-rku} `be (put) in', \ipa{a-mphɯr} `be wrapped' etc),  which are generally used (though not exclusively) with a resultative meaning, thus basically stative like the denominal in \ipa{a-}. Hence, as in the case of all preceding derivations, it is possible to hypothesize taht the passive originates from the reanalysis of the denominal derivation in \ipa{a-} of  the action nominalization of a transitive verb, following the pathway indicated in \ref{ex:pathway4}.


\begin{exe}
\ex \label{ex:pathway4}
\glt \textsc{action nominalization} of transitive verb + \textsc{stative denominal derivation} $\Rightarrow$ \textsc{agentless resultative passive}
\end{exe}

\subsubsection{Applicative} \label{sec:appl}
The applicative \ipa{nɯ(ɣ)--} derives a transitive verb from an intransitive one; unlike in the causative derivation, the A of the applicative verb corresponds to the S of the intransitive one, and a P argument is added (\citealt{jacques13tropative}). The P of applicative verbs refers to either the stimulus in the case of cognition verbs (\ipa{mu} `be afraid (intr)' $\rightarrow$ \ipa{nɯɣ-mu} `fear (tr)') or the the addressee (\ipa{akhu} `shout (intr)'  $\rightarrow$ \ipa{nɯ-ɤkhu} `shout at (tr)'). 

The corresponding denominal derivation \ipa{nɯ(ɣ)-} has many different meaning, but its most productive one is to create a transitive verb from a noun, especially when one has a pair with an intransitive verb in \ipa{rɯ--}. For instance, from a noun such as \ipa{ftɕaka} `manner' one can derive the intransitive \ipa{rɯftɕaka} `make preparations' and the transitive verb \ipa{nɯftɕaka} `prepare (vt)'. Supposing an action noun such as `fear' from the verb \ipa{mu}  `be afraid', applying this \ipa{nɯ(ɣ)-} derivation would predictably yield a transitive verb with the meaning `be afraid of, fear' of the applicative verb \ipa{nɯɣmu}. It is thus possible here again to suppose that the applicative derivation in Japhug came into being through the pathway in (\ref{ex:pathway5}).

\begin{exe}
\ex \label{ex:pathway5}
\glt \textsc{action nominalization} of intransitive verb + \textsc{transitive denominal derivation} $\Rightarrow$ \textsc{applicative}
\end{exe}

Although for the causative, applicative, passive and deexperiencer derivations, no common irregularities between action noun and derived verb have been brought to light up to now, the case for reanalysis of the denominal marker as a voice marker is strong, as they not only have compatible semantics and phonological shape, they also share identical allomorphs (\ipa{nɯ-/nɯɣ-/nɤ-}, \ipa{sɯ-/sɯɣ-/sɤ-} and \ipa{a-/ɤ-}, see \citealt{jacques13tropative, jacques15causative, jacques07passif} for more details). 


\subsubsection{Reflexive} \label{sec:refl}
The reflexive \ipa{ʑɣɤ-} (and its cognates in other Gyalrong languages) differs from all other derivations in that it does not derive from a denominal prefix. Two hypotheses have been proposed to account for its origin. 

\citet{jacques10refl} proposed that \ipa{ʑɣɤ-} from proto-Gyalrong *\ipa{wjɐ-} results from the incorporation of the third person full pronoun *\ipa{wəjaŋ}, (Japhug \ipa{ɯʑo}) with phonological reduction. \citet{jackson14morpho} argued that it originates from the fusion of the pronominal root *\ipa{-jaŋ} with the verb stem, to which the inverse prefix *\ipa{wə-} is added.

These two hypotheses agree in any case that this prefix is partly derived from the bound pronominal root *\ipa{-jaŋ} `oneself', and that its shape in proto-Gyalrong was *\ipa{wəjaŋ}, disagreement between the two hypotheses concerns the interpretation of the nature of the element *\ipa{wə-} in this form, since both the inverse marker and the third person singular possessive prefix have the same shape.

\subsection{Incorporation} \label{sec:incorp}
Japhug has an incorporation-like construction in which noun-verb nominal compounds are turned into verbs by means of a denominal prefix (\citealt{jacques12incorp}). For instance, from the noun \ipa{cɯ} `stone' and the verb \ipa{pʰɯt} `pluck, take out' one can derive an action nominal   \ipa{cɯpʰɯt} `clearing the stones (from a field, before ploughing)', which can in turn be made into an incorporating verb by denominal derivation  \ipa{ɣɯ-cɯpʰɯt } `take out stones (out of the field)'. 

\begin{exe}   
\ex
\begin{xlist}[(ii)]
\exi{(i)} 
\gll     \ipa{cɯ} \ipa{nɯ-pʰɯt-a}  \\
  stone \textsc{pfv}-take.out-\textsc{1sg} \\
  \exi{(ii)} 
\gll     \ipa{cɯ-pʰɯt} \ipa{nɯ-βzu-t-a}  \\
  stone-clearing \textsc{pfv}-do-\textsc{pst}-\textsc{1sg} \\
\exi{(iii)} 
\gll     \ipa{nɯ-ɣɯ-cɯ-pʰɯt-a}  \\
  \textsc{pfv-denominal}-stone-take.out-\textsc{1sg} \\
  \end{xlist}
  \glt   I cleared the stones (from the field). 
\end{exe}   

The construction (iii) has further become a full incorporating construction in the closely related Khroskyabs language, where the denominal prefix has in some cases disappeared due to phonological attrition (\citealt{lai13affixale}).

Gyalrongic languages thus offer a third possible origin for incorporating constructions, after coalescence of noun and verb and backformation (\citealt{mithun84incorp}): reanalysis of denominal verbs derived from noun-verb nominal compounds.


\subsection{TAME}
Tense-Aspect-Modality-Evidentiality in the Japhug verb is mainly marked by orientation prefixes and stem alternations. The diachronic origin of stem alternations is completely opaque, so that the present section focus on orientation prefixes. 

In addition, I discuss the progressive prefix \ipa{asɯ-}, for which a Japhug-internal etymology can be proposed.

\subsubsection{Orientation prefixes} \label{sec:orientation}
With one exception (the non-past factual), all finite verbs forms in Japhug obligatorily take one and only one orientation prefix. As shown by Table (\ref{tab:directional}) orientation prefixes encode seven different directions, and come in four distinct sets, here marked as A to D.

\begin{table}[H]
\caption{Orientation prefixes in Japhug Rgyalrong} \label{tab:directional}
\resizebox{\columnwidth}{!}{
\begin{tabular}{llllll}
\toprule
   &  	perfective  (A) &  	imperfective  (B)  &  	perfective 3$\rightarrow$3' (C)  &  	inferential  (D) \\  	
   \midrule
up   &  	\ipa{tɤ-}   &  	\ipa{tu-}   &  	\ipa{ta-}   &  	\ipa{to-}   \\  	
down   &  	\ipa{pɯ-}   &  	\ipa{pjɯ-}   &  	\ipa{pa-}   &  	\ipa{pjɤ-}   \\  	
upstream   &  	\ipa{lɤ-}   &  	\ipa{lu-}   &  	\ipa{la-}   &  	\ipa{lo-}   \\  	
downstream   &  	\ipa{tʰɯ-}   &  	\ipa{cʰɯ-}   &  	\ipa{tʰa-}   &  	\ipa{cʰɤ-}   \\  	
east   &  	\ipa{kɤ-}   &  	\ipa{ku-}   &  	\ipa{ka-}   &  	\ipa{ko-}   \\  	
west   &  	\ipa{nɯ-}   &  	\ipa{ɲɯ-}   &  	\ipa{na-}   &  	\ipa{ɲɤ-}   \\  	
no direction &\ipa{jɤ-}   &  	\ipa{ju-}   &  	\ipa{ja-}   &  	\ipa{jo-}   \\  	
\bottomrule
\end{tabular}}
\end{table}

Finite verb forms are built by combining a specific orientation prefix with the appropriate verb stem, as indicated in Table \ref{tab:finite.forms}. With the exception of motion verbs and concrete action verbs, which are compatible with all directions, most verbs have only one or two lexically determined orientation, which appears in the Imperfective, Past Perfective, Past Inferential, Irrealis and Imperative. Thus for  instance the verb \ipa{ndza} `eat' appears with the `upwards'' orientation prefixes, as shown by the \textsc{3sg$\rightarrow$3} Imperfective \ipa{tu-ndze} `He eats it' (series B prefix, \ipa{ndza} $\rightarrow$ stem 3 \ipa{ndze}) or the imperative \textsc{2sg$\rightarrow$3} \ipa{tɤ-ndze} `Drink it!' (series A prefix).

However, three TAM categories, namely Egophoric, Sensory and Past Imperfective require always the same orientation prefix (respectively `towards east' (B) \ipa{ku-}, `towards west' (B) \ipa{ɲɯ-} and `downwards' (A, D) \ipa{pɯ-}/\ipa{pjɤ-}), regardless of the orientation lexically selected by the verb in question. For instance, the sensory form of `eat' is \ipa{ɲɯ-ndze} \textsc{sens}-eat[III] `he/it eats it' with the `towards west' series B orientation prefix \ipa{ɲɯ-} instead of an `upward' orientation prefix.

\begin{table}[H]
\caption{Finite verb categories in Japhug Rgyalrong} \label{tab:finite.forms} \centering
\begin{tabular}{lllllll}
\toprule
&	&	stem&	prefixes\\
\midrule
Non-past Factual Non-Past&	\textsc{fact} &	1 or 3&	no prefix\\
Imperfective&	\textsc{ipfv} &	1 or 3&	B\\
Past Perfective &	\textsc{pfv} &	2&	A or C\\
Past Imperfective &	\textsc{pst.ipfv} &	2&	\ipa{pɯ-}\\
Past Inferential Perfective &	\textsc{ifr} &	1&	D\\
Past Inferential Imperfective&	\textsc{ifr.ipfv} &	1&	\ipa{pjɤ-}\\
Sensory Imperfective&	\textsc{sens} &	1 or 3&	\ipa{ɲɯ-}\\
Egophoric Present Imperfective&	\textsc{pres} &	1 or 3&	\ipa{ku-}\\
Irrealis&	\textsc{irr} &	1 or 3&	\ipa{a-} + A\\
Imperative&	\textsc{imp} &	1 or 3&	A\\
\bottomrule
\end{tabular}
\end{table}

The pathway of grammaticalization \textsc{downwards} $\rightarrow $ \textsc{past imperfective} is relatively straightforward and has been the topic of a specific study which does not need to be repeated here (\citealt{lin11direction}). Given he fact that the `downwards' orientation prefix is used to build the past imperfective category in all Gyalrong languages, grammaticalization most probably took place at the proto-Gyalrong stage.

\subsubsection{Egophoric and Sensory evidential prefixes} \label{sec:egoph}

For the remaining two categories, Sensory and Egophoric, note that in other Gyalrong languages, including Situ, the `towards east' \ipa{ko-}, `towards west' \ipa{nə-} prefixes are etymologically related to Japhug \ipa{ku-} and \ipa{ɲɯ-} respectively, and that they also have the Egophoric and Sensory Evidential functions attested in Japhug (\citealt{lin02dimension}). These data data could seem to provide evidence for the pathways indicated in (\ref{ex:sens.egoph}).


\begin{exe}
\ex \label{ex:sens.egoph}
\glt \textsc{east} $\rightarrow $ \textsc{sensory evidential} 
\glt \textsc{west} $\rightarrow $ \textsc{egophoric evidential}
\end{exe}

Yet, the functional link between evidentiality and solar orientation system is not obvious, and it is by no means certain that the orientation system of proto-Gyalrong, when the Sensory and Egophoric markers were grammaticalized, indeed included an East/West solar-based dimension, or whether the prefixes ancestral to Japhug \ipa{ɲɯ-} and \ipa{ku-} expressed something different.  It should be noted that the prefixes  \ipa{ɲɯ-} and \ipa{ku-} in Japhug and their cognates in other Gyalrong languages have additional meanings than `towards west' and `towards east'. In particular, \citet[228-9]{linxr93jiarong} argues that in Situ, the `towards west' prefix \ipa{nə-} expresses in some cases `centrifuge' or `towards outside' directions (\zh{离心、向外扩散}), while the `towards east' prefix \ipa{ko-} expresses `centripete' direction (\zh{向心}). 

There is some evidence that the same is true in Japhug too; for instance, in example (\ref{ex:YWqAt}) we see that the imperfective of the verb `to separate' (in this particular context, `spread wings') takes the prefix \ipa{ɲɯ-} `toward east' (expressing thus motion \textit{away from} oneself) while that of the verb `to put together, to gather' (here `to fold wing') takes \ipa{ku-}  towards west' (motion \textit{towards} oneself).

\begin{exe}
\ex \label{ex:YWqAt}
\gll \ipa{ji-kɯ-nɯqambɯmbjom} 	\ipa{tɤ-kɯ-rɤŋgat} 	\ipa{nɯ} 	\ipa{kɯ-fse,} 	\ipa{ɯ-ʁar} 	\ipa{nɯ} 	\ipa{ɲɯ-qɤt} 	\ipa{nɤ} 	\ipa{ku-wum}, 	\ipa{ɲɯ-qɤt} 	\ipa{nɤ} 	\ipa{ku-wum} \ipa{ŋu} \\
\textsc{conat-nmlz:S/A}-fly \textsc{pfv-nmlz:S/A}-prepare \textsc{dem} \textsc{inf:stat}-be.like \textsc{3sg.poss}-wing \textsc{dem}  \textsc{ipfv:west}-separate \textsc{lnk} \textsc{ipfv:east}-put.together \textsc{ipfv:west}-separate \textsc{lnk} \textsc{ipfv:east}-put.together be:\textsc{fact} \\
\glt (When it tweets), it does as if it were about to fly, it spreads its wings and then folds them, spreads its wings and then folds them. (24-ZmbrWpGa, 121)
\end{exe}

It makes more sense that centripetal orientation, rather than `towards east' direction, would be grammaticalized as an egophoric marker. In Japhug, the egophoric indicates that the speaker has an  intimate knowledge of a state of affair due to his direct participation in the event, as in example (\ref{ex:kutaRa}). It is mainly restricted to first person forms in assertive sentences, though it it also compatible with third persons in the case of third person referents possessed by the first person (`my son', `my work' etc). 

\begin{exe}
\ex \label{ex:kutaRa}
\gll 
<kuabao> 	\ipa{ɯ-spa}  	\ipa{ci}  	\ipa{ku-taʁ-a}  \\
bag \textsc{3sg.poss}-material \textsc{indef} \textsc{prs:egoph}-weave-\textsc{1sg} \\
\glt `I am weaving (cloth to make) a bag.' (conversation, 14.10)
\end{exe}

In interrogative sentences, due to the rule of anticipation (using the TAM category one expects the addressee with employ in his answer, see \citealt{tournadre14evidentiality}), Egophoric marking appears in second person forms, or in third person forms in case of referents possessed by a second person (see example \ref{ex:Wkupe} below).

While no direct pathway of grammaticalization \textsc{cislocative} $\rightarrow$ \textsc{egophoric} has ever been reported, there are clear cases of cislocatives becoming 2/3$\rightarrow$1 portmanteau person markers, in particular in hierarchical indexation systems (see \citealt{jacques14inverse}). While the exact pivot construction which could allow reanalysis from centripetal/cislocative to egophoric is still unclear, it is valuable to explore in more detail this hypothesis.

Accounting for the pathway \textsc{centripete} $\rightarrow$ \textsc{sensory} might be more difficult at first glance. However, it should be noted that the Sensory evidential is used in direct opposition to the Egophoric in most contexts. 

In assertive sentences, it is very rarely used with a first person verb form (only if the speaker forgot something or lost consciousness at a certain stage) and is mainly restricted to second and third person forms. In interrogative sentences, it appears with first or third persons, very rarely in second). There is almost complete complementary distribution with the Egophoric. 
 
Sentences (\ref{ex:Wkupe}) and (\ref{ex:WYWpendZi}) illustrate the difference of use of the Egophoric and Sensory forms in present third person forms, the only context where the two TAM categories are commonly in contrast to each other. These questions expect answers in the Egophoric and Sensory forms respectively. Question (\ref{ex:WYWpendZi}) was asked when I phoned from my parents' home (when I came for the holidays). The Sensory is used because I only seldom meet with my parents, and the expectation is that I had just realized whether or not they were well after having arrived at their place. Question (\ref{ex:Wkupe}) on the other hand, asked about my son, expects an answer in the Egophoric because since I live with him in the same house, I always know whether he is fine or not (I did not `discover' whether he was fine at a certain point). No other TAM category could be appropriate in this context.\footnote{For instance, the factual could only be used for state of affairs that are part of commonly accepted knowledge.}
 
\begin{exe}
\ex \label{ex:Wkupe}
\gll \ipa{nɤ-tɕɯ} \ipa{ɯ-kú-pe?}\\
\textsc{2sg.poss}-son \textsc{qu-egoph}-be.good\\
\glt `Is your son well?' (conversation 2014.02)
\end{exe}

\begin{exe}
\ex \label{ex:WYWpendZi}
\gll 
\ipa{nɤ-mu}  	\ipa{nɤ-wa}  	\ipa{ni}  	\ipa{ɯ-ɲɯ́-pe-ndʑi?}  \\
\textsc{2sg.poss}-mother \textsc{2sg.poss}-father \textsc{du} \textsc{qu-sens}-be.good-\textsc{du} \\
\glt `Are your parents well?' (conversation 2014.12)
\end{exe}

The Egophoric and the Sensory are thus in near-complementary distribution, and in the few cases where both are possible with a verb in the same person form, the contrast is nearly always binary. The opposition between Egophoric (personally experienced knowledge) and  Sensory (knowledge mediated through observation or second hand report)  thus appears to have been grammaticalized as a metaphorical extension of that between motion \textit{towards} vs \textit{away} from the speaker.
 
\subsubsection{Progressive} \label{sec:prog}
The progressive \ipa{asɯ-} / \ipa{ɤsɯ-} / \ipa{az-} / \ipa{ɤz-} differs from most TAM markers in being disyllabic (at least some of its allomorphs) and by the fact that two verb prefixes (the inverse and the autobenefactive, see example \ref{ex:pjAkAwGznAjoci} and \citealt{jacques15spontaneous}) can be infixed \textit{within} it, suggesting that this prefix should be etymologically analyzed as a combination of two elements.

\begin{exe}
\ex \label{ex:pjAkAwGznAjoci}
\gll
\ipa{tɕe} 	\ipa{pjɤ-ɣi} 	\ipa{tɕe} 	\ipa{qala} 	\ipa{kɯ} 	\ipa{pjɤ-k-ɤ́<wɣ>z-nɤjo-ci} 	\ipa{tɕe,} \\
\textsc{lnk} \textsc{ifr:down}-come \textsc{lnk} rabbit \textsc{erg} \textsc{ifr.ipfv-evd<inv>}-wait-\textsc{evd} \textsc{lnk} \\
\glt (The leopard) came down, and the rabbit was waiting for him there. (The smart rabbit.2014, 60)
\end{exe}

It can only be used with transitive verbs, and removes all markers of morphological transitivity (stem three alternation, past 1/3$\rightarrow$3 \ipa{-t} suffix) on the verb forms, as in (\ref{ex:asWndo}), where in the factual stem 3 \ipa{ndɤm} instead of stem 1 \ipa{ndo} would be expected. The verb remains however syntactically transitive, and the A still takes the ergative marker (as in example \ref{ex:pjAkAwGznAjoci}).


\begin{exe}
\ex \label{ex:asWndo}
\gll
\ipa{sɯjno} 	\ipa{ɯ-mdoʁ} 	\ipa{ʑo} 	\ipa{asɯ-ndo.} \\
grass \textsc{3sg.poss}-colour \textsc{emph} \textsc{prog}-hold:\textsc{fact} \\
\glt It has the colour of grass. (25 rtchWRjW, 69)
\end{exe}

\citet{jacques16prog} proposes to account for these two features by assuming that \ipa{asɯ-} originates from the combination of the agentless passive \ipa{a-} (on which see section \ref{sec:passive}) with the causative \ipa{sɯ-} (section \ref{sec:causative}). First, the causative derivation was applied (the \ipa{sɯ-} element is closer to the verb stem). The causative turned the transitive base verb into a ditransitive one. Then, the passive turned it back to two-argument valency, suppressing the causer, and removing all morphological transitivity marking. In addition, as mentioned in section \ref{sec:passive}, the passive in Japhug  has a stative overtone, which, applied to a dynamic transitive verb, became a progressive reading. The combination of passive and causative became common enough to change from a combination of derivations into an inflexional marker.

This hypothesis does not account for all the data; in particular, it does not explain the presence of the ergative on the A: according to the model proposed here, the A of the base transitive verb, turned into a causee by the causative derivation, should be changed to an S by the passive one and would not be expected to take the ergative: the resulting verb form should have a zero-marked S corresponding to the A of the base verb, and a zero-marked adjunct, not indexable in the verb morphology, corresponding to the P of the base verb.  

It is possible that sentences such as (\ref{ex:asWndo}) with zero-marked 3$\rightarrow$3 form and non-overt A were the pivot allowing reinterpretation from a stative intransitive construction into a syntactically transitive construction. Suppose that we accept the historical hypothesis proposed above. At an earlier stage,\footnote{Cognates of the \ipa{asɯ-} prefix are only found in Tshobdun and Zbu, not in Situ; this is probably a Northern Gyalrong innovation; this earlier stage would correspond to the exclusive common ancestor of Japhug, Tshobdun and Zbu.} in the progressive construction, the referent corresponding to the A of the basic transitive construction was not marked with the ergative, and the one corresponding to the P of the basic transitive construction could not be indexed on the verb. 

A sentence such as (\ref{ex:asWndo}), where the first referent is not overt (and thus the presence or absence of ergative not explicitely manifested), and the second referent is third person (zero-marked),  could be reinterpreted as a syntactically transitive one by analogy with other transitive constructions, keeping the surface form but modifying the underlying analysis.

\section{Complex constructions} 
The present section focuses on a selection of complex constructions involving linking elements for which a straightforward etymology can be proposed.  Very few constructions are discussed in this section; those borrowed from Tibetan (such as conditional in \ipa{nɤ} `if'), subordinate clauses with finite main verb and no overt subordinator   (most complement clauses, some relative clauses, see \citealt{jacques16relatives}) and subordinate clauses with a main verb in participial form are not treated here.

\subsection{Alternative}
 Japhug has a conjunction \ipa{me}  `whether ... or' repeated after each noun or phrase in the alternative correlative construction, as in example \ref{ex:saCW}.
 
\begin{exe}
\ex \label{ex:saCW}
\gll  \ipa{saɕɯ} 	\ipa{nɯnɯ} 	\ipa{ɯ-qa} 	\ipa{me,} 	\ipa{ɯ-ru} 	\ipa{me,} 	\ipa{ɯ-jwaʁ} 	\ipa{me} 	\ipa{nɯra} 	\ipa{tɯrgi} 	\ipa{cho} 	\ipa{naχtɕɯɣ} \\
 larch \textsc{dem} \textsc{3sg.poss}-root whether \textsc{3sg.poss}-trunk whether \textsc{3sg.poss}-leave whether \textsc{dem:pl} fir  \textsc{comit} be.similar:\textsc{fact} \\
\glt Whether its root, its trunk or its leaves, the larch is identical to the fir. (08 saCW, 5)
\end{exe} 

This conjunction is obviously grammaticalized from the negative existential copula \ipa{me} `not exist' through an alternative  concessive conditional `whether ... exists or' involving originally the affirmative and negative existential verbs \ipa{tu} vs \ipa{me} as in \ref{ex:pWnWme}

\begin{exe}
\ex \label{ex:pWnWme}
\gll \ipa{tɤ-ʁa} 	\ipa{me} 	\ipa{tɕe,} 	\ipa{nɯ} 	\ipa{pɯ-nɯ-tu} 	\ipa{pɯ-nɯ-me} 	\ipa{kɯ-khɯ} 	\ipa{nɯ} 	\ipa{kɯ-rga} 	\ipa{me.} \\
\textsc{indef.poss}-free.time not.exist:\textsc{fact} \textsc{lnk} \textsc{dem} \textsc{pst.ipfv-auto}-not.exist \textsc{pst.ipfv-auto}-exist \textsc{nmlz}:S/A-be.possible \textsc{dem} \textsc{nmlz}:S/A-like not.exist:\textsc{fact} \\
\glt (Nobody gathers wild strawberries), because (we) don't have time, it is fine whether or not (we) have it, nobody likes it. (11 paRzwamWntoR, 92)
\end{exe}
 
 In its grammaticalized form, \ipa{me} has lost all person and TAME marking.  In \ref{ex:aZo.me}, we see that the conjunction \ipa{me} does not take first or second person singular indexation when used with a pronoun, as would be expected if it still were a verb and the construction an alternative  concessive conditional.
 
 \begin{exe}
\ex \label{ex:aZo.me}
\gll 
 \ipa{aʑo} 	\ipa{me,} 	\ipa{nɤʑo} 	\ipa{me,} 	\ipa{ɯʑo} 	\ipa{me,} 	\ipa{kɤsɯfse} 	\ipa{ɕe-j} 	\ipa{ra} \\
\textsc{1sg} whether \textsc{2sg} whether \textsc{3sg} whether all go:\textsc{fact}-\textsc{1pl} have.to:\textsc{fact} \\
\glt Whether I, you or he, we all have to go. (elicited)
\end{exe}


\subsection{Adversative}
There are four different  adversative constructions in Japhug whose meaning can all be translated as English  `not only/ not just ... but also', and appear to be semantically very close and interchangeable. All four constructions are recently grammaticalized and etymologically transparent.

First, \ipa{mɤra ma} `not just, ...'  occurs after either noun phrases (example  \ref{ex:marama}) or clauses. It is grammaticalized from the negative form of the verb \ipa{ra} `need, have to' followed by the linker \ipa{ma}, a construction that still exists in the language, as in example (\ref{ex:marama2}).

\begin{exe}
\ex \label{ex:marama}
\gll \ipa{nɤʑo} 	\ipa{mɤrama} 	\ipa{rɟɤlpu} 	\ipa{ɕɯŋarɯra} 	\ipa{kɯ} 	\ipa{ta-thu-nɯ} 	\ipa{ɕti} 	\ipa{ri,} 	\ipa{mɯ-tɤ-nɤla-j} 	\ipa{ɕti} 	\ipa{tɕe} 	\ipa{mɤ-jɤɣ}  \\
\textsc{2sg} not.just king each.better.than.the.other \textsc{erg} \textsc{pfv}:3$\rightarrow$3'-ask-\textsc{pl} be.\textsc{assert:fact} \textsc{lnk} \textsc{neg-pfv}-agree-\textsc{1pl} be.\textsc{assert:fact} \textsc{lnk} \textsc{neg}-be.possible:\textsc{fact} \\
\glt `Not just you, many kings, each better than the other (came) to ask (for our daughter in marriage), but we did not agree, so it is not possible.' (The fox, 72-73)
\end{exe}

\begin{exe}
\ex \label{ex:marama2}
\gll
\ipa{tɯ-nɯzdɯɣ-nɯ} 	\ipa{mɤ-ra} 	\ipa{ma} 	\ipa{a-βlu} 	\ipa{tu}\\
2-worry.about:\textsc{fact-pl} \textsc{neg}-have.to \textsc{lnk} \textsc{1sg.poss}-idea exist:\textsc{fact}\\
\glt `You don't need to worry about that, I have an idea.' (hist140505 liuhaohan zoubian tianxia, 217)
\end{exe}
 

Second, the linker \ipa{ɯtɤjɯ} `not only ...', mainly used after finite verbs, as in (\ref{ex:WtAjW}), is originally a relator noun meaning `something added', `some more ...', as in (\ref{ex:WtAjW2}). It can be compared to constructions such as English `in addition to being X, it is also Y'.

\begin{exe}
\ex \label{ex:WtAjW}
\gll \ipa{ɕɯ-mŋɤm} 	\ipa{ɯtɤjɯ} 	\ipa{ɲɯ-sɤzoŋzoŋ} 	\ipa{ʑo} 	\ipa{ŋu}  \\
\textsc{caus}-avoir.mal:\textsc{fact} not.only \textsc{sens}-tingle \textsc{emph} be:\textsc{fact} \\
\glt Not only does it (nettle) hurt, it also causes a tingling sensation. (hist140428 mtshalu, 6)
\end{exe}

\begin{exe}
\ex \label{ex:WtAjW2}
\gll \ipa{ki} 	\ipa{nɤ-ŋga} 	\ipa{ɯ-tɤjɯ} 	\ipa{a-pɯ-ŋu} 	\ipa{ma} 	\ipa{tɯ-nɤndʐo} \\
this \textsc{2sg.poss}-clothes \textsc{3sg.poss}-added \textsc{irr-ipfv}-be \textsc{lnk} 2-feel.cold:\textsc{fact} \\
\glt Have some more clothes, otherwise you will be cold. (\citealt{jacques15japhug})
\end{exe}

Third, the form \ipa{mɤkɯjɤɣ kɯ} `not only' used after finite verbs (example \ref{ex:mAkWjAG}) is the negative form S/A participle of the verb \ipa{jɤɣ} `be possible, be allowed' followed by the ergative \ipa{kɯ}. Example (\ref{ex:mAkWjAG2}) illustrate the same verb form in its non-grammaticalized use.

 \begin{exe}
\ex \label{ex:mAkWjAG}
\gll  \ipa{tɤ-mthɯm} 	\ipa{nɯra} 	\ipa{tu-ndze} 	\ipa{mɤkɯjɤɣ kɯ,} 	\ipa{ɯ-di} 	\ipa{ɲɯ-ɕɯmnɤm}  \\
\textsc{indef.poss}-meat \textsc{dem:pl} \textsc{ipfv}:eat[III] not.only \textsc{3sg.poss}-smell \textsc{sens}-cause.to.have.a.smell \\
\glt  (The mouse) does not only eat meat, it also makes it stinky. (27-spjaNkW, 198)
\end{exe}


 \begin{exe}
\ex \label{ex:mAkWjAG2}
\gll
<baohu> 	\ipa{kɯ-ra} 	\ipa{tɕe} 	\ipa{pjɯ́-wɣ-sat} 	\ipa{mɤ-kɯ-jɤɣ} \ipa{ŋu} \\
protect \textsc{nmlz}:S/A-have.to \textsc{lnk} \textsc{ipfv-inv}-kill \textsc{neg-nmlz}:S/A-be.allowed be:fact \\ 
\glt It has to be protected and is not to be killed. (27-kikakCi, 88)
\end{exe}

Fourth, the linker \ipa{ʁo alala ri} `not only', used after noun phrases as in \ref{ex:alala}, is the combination of the adversative marker \ipa{ʁo} (\zh{倒}  \textit{dào}), the adverb \ipa{alala} `of course' and the locative \ipa{ri}.

 \begin{exe}
\ex \label{ex:alala}
\gll  \ipa{tɕe} 	\ipa{nɯnɯra} 	\ipa{ʁo alala ri} 	\ipa{ɯʑo} 	\ipa{sɤz} 	\ipa{kɯ-xtɕi} 	\ipa{pɣa} 	\ipa{nɯra} 	\ipa{kɯnɤ} 	\ipa{ku-ndɤm} 	\ipa{qhe} 	\ipa{tu-ndze} \\
\textsc{lnk} \textsc{dem:pl} not.only \textsc{3sg} \textsc{comp} \textsc{nmlz}:S/A-be.small bird \textsc{dem:pl} also \textsc{ipfv}-catch[III] \textsc{lnk} \textsc{ipfv}-eat[III] \\
\glt In addition to these, it also eats birds that are smaller than itself. (19-qandZGi, 58)
\end{exe}

 \subsection{Purposive clauses}
There are three purposive constructions in Japhug, and their origin is transparent.

   First, purposive clauses can be build with a verb in participial form followed by the linker \ipa{ɯ-spa} as in ({ex:Wspa}).  The form \ipa{ɯ-spa} is etymologically a noun meaning `matter' (at an earlier stage the irregular oblique participle of the verb \ipa{pa} `do'). This construction is one more example of the well-attested pathway \textsc{matter} $\rightarrow$ \textsc{purposive} (\citealt[212]{heine-kuteva02}).


\begin{exe}
\ex \label{ex:Wspa}
\gll \ipa{paʁ} 	\ipa{nɯra} 	\ipa{ʁo} 	\ipa{lɯski,} 	\ipa{ɕa} 	\ipa{kɤ-ndza} 	\ipa{ɯ-spa} 	\ipa{ku-χsu-nɯ} 	\ipa{pjɤ-ŋu}  \\
pig \textsc{dem:pl} \textsc{adv} of.course meat \textsc{nmlz}:P-eat \textsc{3sg.poss}-purposive \textsc{ipfv}-raise-\textsc{pl} \textsc{ifr.ipfv}-be \\
\glt 
\end{exe}
 
Second, the verb \ipa{nɯmga} `do (on purpose), (to have)', either in a finite or infinitive form (as \ipa{kɤ-nɯmga}  in example \ref{ex:kAnWmga}) can mark a purposive clause.

\begin{exe}
\ex \label{ex:kAnWmga}
\gll 
\ipa{tɕe} 	\ipa{kɯpɤz} 	\ipa{nɯ} 	\ipa{mɯ-ɲɯ-kɤ-βzu} 	\ipa{kɤ-nɯmga,} 	\ipa{iʑɤra,} \ipa{ji-mthɯm} 	\ipa{nɯra} 	<binggui> 	\ipa{tu-χtɯ-j} \ipa{nɯ} 	\ipa{ɯ-ŋgɯ} 	\ipa{ri} 	\ipa{pjɯ-nɯ-rku-j} 	\ipa{ɕti} 	\ipa{ma} \ipa{maka} 	\ipa{ɯ-pɕi} 	\ipa{tú-wɣ-ɕɯɴqoʁ} 	\ipa{qhe,} \ipa{ŋotɕu} 	\ipa{nɯ́-wɣ-tɯ\textasciitilde{}ta} 	\ipa{ʑo} 	\ipa{kɯpɤz} 	\ipa{ɲɯ-βze} 	\ipa{ɲɯ-ɕti}\\
\textsc{lnk} type.of.bug \textsc{dem} \textsc{neg-ipfv-inf}-grow \textsc{inf}-do.on.purpose \textsc{1sg} \textsc{1sg.poss}-meat \textsc{dem:pl} fridge \textsc{ipfv}-buy-\textsc{1pl} \textsc{dem} \textsc{3sg.poss}-inside \textsc{loc} \textsc{ipfv-auto}-put.in-\textsc{1pl} be.\textsc{assert:fact} \textsc{lnk} at.all \textsc{3sg.poss}-outside \textsc{ipfv-inv}-hang \textsc{lnk} where \textsc{ipfv-inv}-\textsc{indefinite}\textasciitilde{}put \textsc{emph} \textsc{emph} \textsc{ipfv}-grow \textsc{sens}-be.\textsc{assert}\\
\glt In order not to have \ipa{kɯpɤz} bugs, our meat, we bought a fridge and put it in there, as if one hangs it outside, bugs will grow wherever you put it. (28-kWpAz)
\end{exe}
 

Third, Core Gyalrong languages, including Japhug (\citealt{jacques14linking}) and Tshobdun (\citealt{sun12complementation}) have purposive converbs in \ipa{sɤ-}, mainly used in the negative form as in (\ref{ex:WmAYWsAjmWjmWt}), which appear to be derived from the oblique participle \ipa{sɤ-} (on which see \citealt{jacques16relatives}).
 
\begin{exe}
\ex \label{ex:WmAYWsAjmWjmWt}
\gll
[\ipa{kɯ-lɤɣ}   	\ipa{acɤβ}   	\ipa{nɯ}   	\ipa{kɯ}   	\ipa{\textbf{ɯ-mɤ-sɤ-jmɯ\textasciitilde{}jmɯt}},]   	\ipa{ɯ-pʰɯŋgɯ}   	\ipa{nɯ}   	\ipa{tɕu}   	\ipa{rdɤstaʁ-pɯpɯ}   	\ipa{tɕʰɯrdu}   	\ipa{ci}  \ipa{ɲɤ-rku,}\\
 \textsc{nmlz}:S/A-herd Askyabs \textsc{dem} \textsc{erg}  \textsc{3sg-neg-purp:conv}-forget \textsc{3sg.poss}-inside.clothes \textsc{dem} \textsc{loc} stone-little pebble \textsc{indef}
 \textsc{evd}-put.in\\
\glt The cowboy Askyabs put a little pebble inside his clothes so that he would not forget it. (The frog, 166)
\end{exe}
   \subsection{Causal clauses}
Among the causal clauses in Japhug (\citealt{jacques14linking}), one presents a clear grammaticalization pathway:  clausal clauses in \ipa{núndʐa} `for this reason' as in (\ref{ex:qapGAmtWmtW}). This linker is the fusion of the demontrative \ipa{nɯ} `that' with the possessed noun \ipa{ɯ-ndʐa} `reason', a word borrowed from Tibetan \ipa{ɴdra}.
   
   
\begin{exe}
\ex \label{ex:qapGAmtWmtW}
\gll
[\ipa{tɕe}  	\ipa{ɯ-mtɯ}  	\ipa{ɣɤʑu}]  	\ipa{tɕe,}  	\ipa{tɕe}  	\textbf{\ipa{núndʐa}}  	\ipa{qapɣɤmtɯmtɯ}  	\ipa{tu-ti-nɯ}  	\ipa{ɲɯ-ŋu}   \\
\textsc{lnk} \textsc{3sg.poss}-crest \textsc{sensory}:exist \textsc{lnk} \textsc{lnk} for.this.reason hoopoe \textsc{ipfv}-say-\textsc{pl} \textsc{testim}-be \\
\glt It has a crest, and this is the reason why it is called `hoopoe'. (Hoopoe, 20)
\end{exe}

%Second, it is possible to use the ergative \ipa{kɯ} as a clausal 
%
%\citet{jacques16comparative}
   
\subsection{Other constructions}
The verb \ipa{me} `not exist' is used in addition in a construction expressing that an action is futile -- that the result will be the same whether or not it takes place. As shown by example (\ref{ex:mAndza}), in this construction the verb \ipa{me} `not exist' follows two bare infinitives of the same verb, in affirmative and in negative forms, and \ipa{me} agrees with the P if the verb is transitive.\footnote{This construction is very rare and is not attested with intransitive verbs; according to some consultants, in the case of intransitive complement verbs there is not agreement on \ipa{me}.}


\begin{exe}
\ex \label{ex:mAndza}
\gll \ipa{ndza} \ipa{mɤ-ndza} \ipa{me-a} \\
eat:\textsc{bare.inf} \textsc{neg}-eat:\textsc{bare.inf} not.exist-\textsc{1sg} \\
\glt (I am so lean that) whether you eat me or not, the result will be the same.
\end{exe}
 
 \section{Degrammaticalization}
 This overview of grammaticalziation would not be compelete without an account of attested cases of degrammaticalization in Japhug, which include a suffix  becoming an independent word, and a relator noun of location (equivalent to a a postposition) becoming a common noun.
 
 
 \subsection{Suffix to clitic}
The locative postposition  \ipa{zɯ} in Japhug is related to the allative suffix \ipa{-s} found in Situ (\citealt[330]{linxr93jiarong}).  

Yet, the phonetic correspondence of Japhug \ipa{z} to Situ \ipa{s} is anomalous: in word-initial position without cluster, Situ \ipa{s-} always corresponds to Japhug \ipa{s-}  (\citealt[317-8]{jacques04these}).

However, a sound change common to Japhug, Tshobdun and Zbu is the voicing of final \change{-s}{-z}.\footnote{Final \ipa{-z} is realized as voiced in these three languages except utterance-finally and when preceding a word beginning with an unvoiced obstruent, as was first recognized by  \citet{jackson05yingao} about Tshobdun.} 

The Japhug form \ipa{zɯ} can be accounted for in the following way. After the regular sound change \change{-s}{-z}, the locative suffix \ipa{-z} was degrammaticalized as a an enclitic, and a postthetic vowel \ipa{ɯ} was added to it as in all case marking clitics (Ergative \ipa{kɯ}, Genitive \ipa{ɣɯ}). 

 The opposite possibility, namely that the allative marker was an independent word or clitic in proto-Gyalrong and that it became phonologically fused with the preceding noun in Situ, cannot account for the presence of voicing in the Japhug form \ipa{zɯ}. Moreover, there is evidence that the proto-Gyalrong allative suffix reconstructed here as *\ipa{-s} is cognate to the \ipa{-s} element found in several case markers in Tibetan (on which see \citealt{hill12bas}).
  

 \subsection{Relator noun of location to common noun}

The Japhug noun \ipa{ɯ-thoʁ} `the ground' is highly anomalous in having an obligatory third person singular possessive suffix \ipa{ɯ-}. In addition, this word has no known cognates in other Gyalrongic languages, while it is a perfect match for a being a borrowing from a TIbetan word with the shape \ipa{tʰog} (compare the other borrowed noun \ipa{thoʁ} `thunder' from Tibetan \ipa{tʰog}). To account for the etymology of this word, I propose the following scenario in four stages.

First, Japhug borrowed the Tibetan relator noun  \ipa{tʰog(tu)} `on' as \ipa{ɯ-tʰoʁ} *`on' (not attested), adding a third person possessive prefix like all relator nouns (see section \ref{sec:loc}). This relator noun was in competition with the existing native equivalent \ipa{ɯ-taʁ} `on'.\footnote{It is not surprising in Japhug to have several competing relator noun for the same functional slot. The dative \ipa{ɯ-ɕki} discussed in section \ref{sec:dat} is itself in competition with another form \ipa{ɯ-phe}, probably borrowed from another Gyalrong variety.  }
  
  Second, it  became restricted to the collocation *\ipa{sɤtɕha ɯ-thoʁ zɯ} `on the ground' (not attested), with the native locative \ipa{zɯ} and the  noun of Tibetan origin \ipa{sɤtɕha} `earth, ground, place'.
  
    Third, the collocation *\ipa{sɤtɕha ɯ-thoʁ zɯ} `on the ground' became reduced as \ipa{ɯ-thoʁ zɯ} `on the ground' (attested).
 
 Fourth, the noun \ipa{ɯ-thoʁ} `ground' was created by backformation from the locative phrase \ipa{ɯ-thoʁ zɯ} `on the ground'. The fact that the locative postposition \ipa{zɯ} is always optional (section \ref{sec:loc}) undoubtedly made this step easier.
 
 Thus, Japhug attests an example of degrammaticalization from a relator noun meaning `on' (with or without motion) to a common noun meaning `ground'.
 
    \section{Discussion} 



\subsection{The verbal template} \label{sec:template}
 Japhug and other Gyalrong languages have elaborate verbal templates, with more than ten prefixal slots (\citealt{jacques13harmonization}). It is possible to find complex verb forms with eight prefixes and an incorporated noun, such as \ref{ex:amaCtAtWwGznWsnWYaR}.
 
\begin{exe}
\ex  \label{ex:amaCtAtWwGznWsnWYaR}
\gll
	\ipa{a-mɤ-ɕ-tɤ-tɯ́-wɣ-z-nɯ-snɯ-ɲaʁ}  	\ipa{ra}  \\
  \textsc{irr-neg-transloc-pfv-2-inv-caus-denom}-heart-black have.to:\textsc{fact} \\
\glt Don't let them go to harm you!
\end{exe} 

Some of these prefixes are of proto-Sino-Tibetan provenance (see \citealt{delancey11prefixes, delancey14second, jacques12agreement}), but others are Gyalrong innovations, grammaticalized from either denominal prefixes (see section \ref{sec:denom}), nouns, pronouns, adverbs or verbs that can still be identified.

The only verbal prefixes coming from verbs in Japhug are the associated motion prefixes (\ref{sec:motion}). This grammaticalization is shared by all core Gyalrong languages (Japhug, Tshobdun, Zbu and Situ), but not shared by Khroskyabs and Stau, which have apparently never developped these prefixes (\citealt{lai13affixale}).

The reflexive \ipa{ʑɣɤ-} is the only prefix of pronominal origin (\ref{sec:refl}). Here again the grammaticalization is a common innovation of Core Gyalrong, while Khroskyabs has innvoted a slightly different reflexive prefix  (\citealt[156-7]{lai13affixale}).

While Japhug and all other Gyalrongic languages have incorporation, there are very few prefixes whose origin can be unambiguously traced to nouns. The orientation prefixes (section \ref{sec:orientation}) can be either from locative nouns or locative adverbs. The date of the grammaticalization of these prefixes is a vexed matter, as similar systems are found in the region. \citet{sun83liujiang} proposed the existence of a `Qiangic' subgroup of Sino-Tibetan on the basis of the presence of these prefixes, but the fact that some varieties of Tibetan have developed orientation prefixes too (\citealt{jackson07khalong}) shows that this typological feature is of little value for establishing the phylogeny.

The orientation systems of Core Gyalrong languages show too many commonalities to be the result of independent grammaticalization, and which cannot be due to language contact. In particular, all Core Gyalrong languages except Zbu (which has a simplified orientation system, and probably lost many features) have developed a present egophoric marker from the orientation prefix meaning `toward east', probably through grammaticalization of its use as a centripete motion marker (section \ref{sec:egoph}).
 
\subsection{Denominal derivations} \label{sec:denom}
Denominal verbalizing derivations are involved in the pathways of grammaticalization of many nominal and verbal affixes in Japhug, in particular valency-changing prefixes (antipassive, causative etc, see section \ref{sec:voice}), incorporation (section \ref{sec:incorp}) and also comitative adverbs (section \ref{sec:comit}).

Grammaticalization pathways based on denominal derivation are in many ways comparable to pathways  based on light verbs (as in the case of the Antipassive, cf \citealt{creissels12antip}). The only real  difference is in fact that the historical origin of denominal verbalizing prefixes in Japhug is unknown. It is conceivable that they originate from verbs too, but the grammaticalization took place in a past so remote that it may not be recoverable, as cognates of these prefixes exist in other branches of the Sino-Tibetan family. If they indeed originate from verbs, it is interesting that they are prefixes and not suffixes, in a language with strict verb-final word order (on which see \citealt{jacques13harmonization}).  
 

\section{Conclusion} 
The study of grammaticalization in Gyalrong languages and Japhug in particular is a very rich topic, and the present paper is but a mere sketch of the most obvious phenomena observed in this language. While some of the grammaticalization pathways found in Japhug are quite common crosslinguistically (\textsc{matter} $\Rightarrow$ \textsc{purposive}, \textsc{go} $\Rightarrow$ \textsc{cislocative/andative}), other appear to be unique to Japhug or Gyalrong languages, for instance \textsc{east} $\Rightarrow$ \textsc{centripete} $\Rightarrow$ \textsc{egophoric} or \textsc{together with} $\Rightarrow$ \textsc{unexpectedness}.


\bibliographystyle{unified}
\bibliography{bibliogj}
\end{document}
