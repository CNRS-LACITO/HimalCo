\documentclass[oldfontcommands,oneside,a4paper,11pt]{article} 
\usepackage{fontspec}
\usepackage{natbib}
\usepackage{booktabs}
\usepackage{xltxtra} 
\usepackage{longtable}
\usepackage{tangutex2} 
\usepackage{tangutex4} 
\usepackage{polyglossia} 
\setdefaultlanguage{french} 
\usepackage[table]{xcolor}
\usepackage{multirow}
\usepackage{gb4e} 
\usepackage{graphicx}
\usepackage{float}
\usepackage{lscape}
\usepackage{hyperref} 
\hypersetup{bookmarks=false,bookmarksnumbered,bookmarksopenlevel=5,bookmarksdepth=5,xetex,colorlinks=true,linkcolor=blue,citecolor=blue}
\usepackage[all]{hypcap}
\usepackage{memhfixc}

\bibpunct[~: ]{(}{)}{,}{a}{}{,}
%%%%%%%%%quelques options de style%%%%%%%%
%\nouppercaseheads
%\pagestyle{Ruled}
%\setsecheadstyle{\SingleSpacing\LARGE\scshape\raggedright\MakeLowercase}
%\setsubsecheadstyle{\SingleSpacing\Large\itshape\raggedright}
%\setsubsubsecheadstyle{\SingleSpacing\itshape\raggedright}
%\setsecnumdepth{subsubsection}
%%%%%%%%%%%%%%%%%%%%%%%%%%%%%%%
\setmainfont[Mapping=tex-text,Numbers=OldStyle,Ligatures=Common]{Charis SIL} %ici on définit la police par défaut du texte


\newfontfamily\phon[Mapping=tex-text,Ligatures=Common,Scale=MatchLowercase,FakeSlant=0.3]{Charis SIL} 
\newfontfamily\phondroit[Mapping=tex-text,Ligatures=Common,Scale=MatchLowercase]{Charis SIL} 
\newcommand{\ipa}[1]{{\phon #1}} %API tjs en italique
\newcommand{\ipac}[1]{{\tiny #1}}
\newcommand{\ipapl}[1]{{\phondroit #1}} 
\newfontfamily\cn[Mapping=tex-text,Ligatures=Common,Scale=MatchUppercase]{MingLiU}%pour le chinois
\newcommand{\zh}[1]{{\cn #1}}
\newfontfamily\mleccha[Mapping=tex-text,Ligatures=Common,Scale=MatchLowercase]{Galatia SIL}%pour le grec
\newcommand{\grec}[1]{{\mleccha #1}}
\newcommand{\tgz}[1]{#1 \mo{#1} \tg{#1}}
\newcommand{\indextg}[1]{\index{\tge{#1}@\tgz{#1}}}
\newcommand{\tgb}[1]{\tgz{#1}\indextg{#1}}
\newcommand{\tgc}[1]{\tg{#1} \#1\indextg{#1}}
\newcommand{\tgd}[1]{\tge{#1}\indextg{#1}}
\newcommand{\tgf}[1]{\mo{#1}\indextg{#1}}
\newcommand{\petit}[1]{\tiny#1}
\newcommand{\sig}{\begin{math}\Sigma\end{math}}
\newcommand{\phone}{\begin{math}\Phi\end{math}}
\newcommand{\ra}{$\Sigma_1$} 
\newcommand{\rc}{$\Sigma_3$} 
\newcommand{\grise}[1]{\cellcolor{lightgray}\textbf{#1}}


\begin{document}
%\OnehalfSpacing
\title{Rapport sur les travaux effectués}
\author{Guillaume Jacques}
\maketitle

\sloppy
%
\section{Introduction}

\section{Activités de recherche}
Une question fondamentale sous-tend l'ensemble de mes travaux: quels sont les principes généraux de changements linguistiques, indépendamment d'une famille de langue particulière, et dans quelle mesure  ces principes permettent de rendre compte des limites de la diversité des langues et des `universaux' linguistiques?

Pour contribuer de la façon la plus efficace possible à cette perspective panchronique,  aussi bien phonologie qu'en morphosyntaxe, j'ai préféré me spécialiser tout d'abord dans une seule famille de langue afin d'éviter une dispersion prématurée. Le choix du sino-tibétain comme objet d'étude principal a été motivé par le fait qu'il s'agit d'une immense famille comprenant des centaines de langues, d'une grande diversité typologique (c'est une des seule famille dans laquelle on trouve à la fois des langues isolantes et des langues polysynthétiques), et aussi pour laquelle on dispose de plusieurs langues anciennes.

Mes activités de recherches peuvent se regrouper en six thèmes principaux.

Premièrement, mon travail a d'abord porté sur l'étude philologique des trois langues les plus anciennement attestées de la famille (le chinois, le tibétain, puis le tangoute). 

Deuxièmement, à partir de ma thèse, ma recherche s'est focalisée sur la description détaillée de langues non-écrites sur le terrain, d'abord avec le japhug, puis d'autres langues telles que pumi, le khaling, le stau et plusieurs variétés de tibétain. 

Troisièmement, outre l'aspect proprement linguistique de ma description du japhug (grammaire, dictionnaire, recueil de textes), je me suis également intéressé à la contribution que ces données pouvaient apporter à l'ethnologie, en particulier l'étude des systèmes de parenté.

Quatrièmement, sur la base de mes descriptions du japhug, du pumi et de langues tibétaines, j'ai publié un certain nombre de travaux sur la phonologie et la morphologie comparées de sous-branches du sino-tibétain. J'ai également écrit des articles des changements phonétiques ou des étymologies particulières dans d'autres familles de langues (indo-européen, sémitique, algonquien, sioux, turcique) afin d'élargir ma perspective typologique.

Cinquièmement, depuis mon recrutement au CNRS en 2009 comme CR1, je me suis orienté vers la typologie morphosyntaxique (dans une perspective panchronique),  en particulier dans mes articles sur les marques de voix, l'incorporation et le mouvement associé.

Sixièmement, depuis 2012 j'ai commencé à utiliser des méthodes computationnelles pour analyser la morphologie de langues étudiées sur le terrain, en particulier le khaling.

\subsection{Philologie} \label{sec:philologie}
Ma première publication (\citealt{jacques00ywij}) a porté sur la linguistique historique et la philologie du chinois, et cette orientation continue jusqu'à mon plus récent ouvrage (\citealt{jacques14esquisse}) sur le tangoute.

Mon étude des langues anciennes est motivée par mon intérêt pour les principes généraux des changements linguistiques: en effet, sans documents historiques, les états anciens des langues ne sont accessibles que sous la forme d'hypothèses bâties en combinant méthode comparative et reconstruction interne. Les langues tibétaines en particulier, attestée depuis le 8^e siècle, offrent un terrain d'étude des changements phonologiques, morphologiques et syntaxiques particulièrement fructueux, auquel j'ai contribué en particulier dans mon article sur le tibétain de Tchoné (\citealt{jacques14cone}).

Le tangoute, langue ancienne attestée entre le 10^{e} et le 15^e siècles sans descendant direct, est d'une importance cruciale pour l'ensemble de la famille sino-tibétaine pour une raison différente. En effet, c'est la langue la plus anciennement attestée qui présente un système d'indexation personnelle sur le verbe. A ce titre, elle a un rôle prééminent à jouer dans le débat sur l'antiquité du système d'accord personnel en sino-tibétain (\citealt{jacques09tangutverb}, \citealt{jacques10zos}, \citealt{jacques11tangut.verb}), un débat dont l'intérêt méthodologique dépasse cette seule famille (voir section \ref{sec:diachronie}). Son écriture logographique très opaque est un obstacle à son utilisation par les typologues et les diachroniciens non-spécialistes, et dans mes travaux, en particulier \citet{jacques07textes} et \citet{jacques14esquisse}, j'ai fait en sorte de faciliter l'accès aux textes tangoutes et de rendre ainsi mes analyses de la morphosyntaxe de cette langue indépendamment revérifiables par un public plus large que les seuls tangoutologues.


Le chinois pose une question différente: c'est l'une de quatre langues (avec le grec, l'araméen et l'égyptien/copte) à être attestée sur plus de trois mille ans, et elle est donc d'un intérêt  fondamental pour toute théorie générale des changements linguistiques. Toutefois, du fait du caractère  logographique de son écriture, la phonologie et la morphologie du chinois de l'antiquité ne sont pas directement attestés -- ils doivent être reconstruits. Mes travaux dans ce domaine, basés sur ceux de \citet{sagart99roc}, ont été essentiellement d'ordre pédagogique (des articles d'encyclopédie comme \citealt{jacques2015traditional} et  \citealt{jacques2015genetic} et des documents de cours), mais j'ai également écrit quelques travaux utilisant des données comparatives pour évaluer certaines hypothèses de reconstruction (\citealt{jacques00ywij}, \citealt{jacques05}).

\subsection{Documentation de langues en danger} \label{sec:documentation}
 
La description de langues en danger est depuis le début de ma thèse le sujet qui occupe la majeure partie de mon temps de recherche.  J'ai tout d'abord étudié le japhug, langue rgyalronguique parlée au Sichuan, parlée par moins de 10000 personnes.

Ce choix avait été motivé d'une part par la complexité morphologique et la singularité typologique du système verbal des langues rgyalronguiques (connues essentiellement à cette époque par l'ouvrage \citealt{linxr93jiarong} et l'article \citealt{jackson00sidaba}), et par leur phonologie conservatrice. J'ai décidé de travaillé sur le japhug car c'était alors la langue la moins connue du groupe rgyalronguique:  seules quelques pages de description phonologique (sur un dialecte différent) avaient été publiées lorsque j'ai commencé mes travaux.

Après ma thèse (\citealt{jacques04these}), j'ai tout d'abord publié en chinois une courte grammaire du japhug (\citealt{jacques08}) afin de rendre mon travail plus accessible aux collègues chinois et aux locuteurs de cette langue. J'ai également publié un recueil de textes glosés (\citealt{jacques10gesar}) dont certains sont disponibles sur l'archive Pangloss. J'ai recueilli au fil des années (et en particulier en 2012 et 2014) un corpus de textes d'environ soixante heures, dont plus de 47h sont transcrites et recherchable automatiquement ainsi qu'un dictionnaire comprenant plus de 6200 entrées. Ces documents seront prochainement publiés en ligne dans le cadre de l'ANR HimalCo (voir section  \ref{sec:projets}).

Ensuite, j'ai écrit une série d'articles portant sur des aspects particuliers de la morphosyntaxe du japhug: le marquage direct/inverse (\citealt{jacques10inverse}), le réfléchi (\citealt{jacques10refl}), l'incorporation  (\citealt{jacques12incorp}), les marques de voix (\citealt{jacques12demotion}),  le mouvement associé (\citealt{jacques13harmonization}), l'applicatif (\citealt{jacques13tropative}), les idéophones (\citealt{japhug14ideophones}), l'antipassif (\citealt{jacques14antipassive}) et enfin le chainage de propositions  (\citealt{jacques14linking}), certains pour des revues aréales (LTBA, BSOAS, Language and Linguistics), d'autres pour des revues plus généralistes (Lingua, Linguistic typology, Anthropological linguistics). 

J'ai également soumis des articles sur la relativisation, les constructions causatives, les constructions comparatives, le générique ainsi qu'une `illustration of the IPA', pour lesquels j'attends les réponses des évaluateurs. 

J'ai le projet, lorsque j'aurai publié des articles sur tous les sujets principaux de la morphosyntaxe du japhug, de les synthétiser et de les remanier pour en faire une grammaire de référence en anglais. Segmenter ce travail en articles a plusieurs avantages. D'un part, il est possible dans un article de mettre en valeur les données de la langue étudiée en apportant une discussion typologique qui n'auraient pas sa place dans une grammaire de la langue. D'autre part, ce choix permet de bénéficier de la lecture attentive de relecteurs variés, afin de rendre la terminologie employée et la description les plus claires et les plus riches possibles. Enfin, il rend possible d'affiner progressivement la description de la langue, et de confronter la validité des généralisations proposées dans les articles à un corpus qui croît régulièrement. 

Outre le japhug, j'ai effectué des travaux de terrains sur neuf langues de la famille sino-tibétaines: le zbu et le stau (deux autres langues rgyalronguiques), le tanhai et le chang (langues naga du nord, parlées en Inde du nord-est), le kami et le tchoné (deux langues tibétaines, parlées au Sichuan et au Gansu respectivement), le pumi et le muya (deux langues `qianguiques' du Sichuan), et le khaling (kiranti, Népal).

Parmi ces langues, c'est essentiellement sur le khaling et le stau que j'ai recueilli le plus de données. Je prépare des recueils de textes et des  dictionnaires pour ces deux langues.

Mon travail sur le khaling a donné lieu à l'article \citet{jacques12khaling}, dans lequel je présente un modèle de la morphologie flexionnelle des verbes simples de cette langues, ainsi qu'une liste de paradigmes générés automatiquement à partir des règles décrites dans l'article (voir section \ref{sec:morpho}). Dans \citet{jacques14auditory}, nous décrivons le système de démonstratifs en khaling, et en particulier l'usage du démonstratif auditif, qui se distingue de tous les phénomènes similaires décrits dans d'autres familles de langues. J'ai également fait une présentation invitée sur la morphologie dénotationnelle du khaling (\citealt{jacques13derivational.khaling}) qui sera publiée dans un ouvrage collectif dirigé par les organisateurs de la conférence.

L'étude du stau, commencée en 2012, n'a pour le moment mené qu'à un article (\citealt{jacques14rtau}) et à quelques présentations (\citealt{antonov14rtau}), mais plusieurs articles utilisant les données de cette langue sont en cours de rédaction (sur le discours indirect hybride et sur les classificateurs).
 
\subsection{Ethnolinguistique}  \label{sec:ethno}
Le corpus de données recueilli en japhug et en khaling, outre son aspect proprement linguistique, a un intérêt non-négligeable pour l'ethnologie, en particulier pour l'étude de la parenté et de la mythologie comparée.

Le japhug a un système de type Omaha, dans lequel on observe un même terme pour l'oncle maternel et le fils de l'oncle maternel, et réciproquement, un même terme pour le fils de la tante paternelle et le fils de la sœur). Ce type de systèmes, s'il sont bien connus en Amérique du nord, n'avaient pas jusqu'ici   été décrit dans une langue de la région tibétaine. Cette question est discutée dans l'article \citet{jacques11kinship}, où je propose une comparaison entre les systèmes du japhug, du tangoute et du pumi, ainsi qu'une hypothèse historique pour rendre compte de leurs ressemblances et de leurs différences.

En ce qui concerne la mythologie,    la culture des locuteurs de langues rgyalronguiques est fortement tibétanisée et il est difficile de trouver des traces de folklore proprement local. En revanche, les locuteurs des langues kiranti, en particulier du khaling, ont préservé un mythologie originale complètement distincte de l'hindouisme et du bouddhisme tibétaine, et une tradition chamanique encore vivante. Durant mes séjours de terrain au népal, j'ai pu recueillir et analyser des histoires mythologiques khaling et des récitations rituelles.

\subsection{Linguistique historique}  \label{sec:diachronie}
Mes travaux en linguistique historique portent sur trois questions principales: reconstruction de sous-branches du sino-tibétain (lois phonétiques et étymologies), la question de la reconstructibilité de la morphologie en sino-tibétain, et la typologie des changements phonétiques, morphosyntaxiques et sémantiques.
 
La reconstruction de sous-branches du sino-tibétain, par la mise en évidence de correspondances et de lois phonétiques et la découverte d'étymologies, est le travail de base sur lequel tous mes travaux plus généraux de typologie diachronique sont basés. 

J'ai commencé à contribuer à cette recherche dans ma thèse (\citealt{jacques04these}) où je propose une reconstruction préliminaire du proto-rgyalrong. Cette reconstruction, la première pour ce groupe de langue,   n'est toujours pas obsolète dix ans plus tard, et aucune des lois phonétiques proposées n'a été réfutée, malgré l'afflux de nouvelles données durant la dernière décennie. 

J'ai par la suite contribué à reconstruire le proto-naish, un groupe de langue proche du rgyalronguique, en collaboration avec Alexis Michaud (\citealt{jacques.michaud11naish}) et ai proposé une nouvelle classification de la famille sino-tibétaine dans ce même article. Par ailleurs, j'ai  mis au jour une série de nouvelles lois phonétiques en tibétain (\citealt{jacques09wazur},   \citealt{jacques09e},  \citealt{jacques12internal},   \citealt{jacques13yod}, \citealt{jacques14snom}). Enfin, dans  \citet{jacques14esquisse}, je présente une comparaison systématique entre tangoute et langues rgyalronguiques modernes enrichie d'exemples tirés de textes, où j'expose pour la première fois la phonologie historique du tangoute, et les innovations lexicales et morphologiques qui la distinguent des autres langues du même groupe.

En dehors du sino-tibétain, j'ai également écrit, parfois en collaboration, des articles sur la phonologie historique et l'étymologie de langues indo-européennes (\citealt{jacques13vama}),   turciques (\citealt{antonov12kumush}), algonquiennes (\citealt{jacques13arapaho}) et sémitiques (\citealt{rg-gj12yod}). Ces travaux m'ont permis de me familiariser avec la pratique de l'étymologie, des lois phonétiques et de l'analyse morphologique dans des familles autres que le sino-tibétain, de me confronter à d'autres traditions de recherche, et m'ont ainsi été indirectement utiles pour développer une approche plus typologique de la linguistique historique.

 La deuxième orientation majeure de ma recherche en linguistique historique porte sur la question controversée de reconstruction de la morphologie en sino-tibétain. Dans cette famille, on trouve en effet d'une part des langues à morphologie complexe (rgyalronguique et kiranti, sur lesquelles j'effectue un travail de terrain), et d'autre part des langues isolantes (langues chinoises modernes, lolo-birmanes, karen etc). 
 
Il est donc possible que les langues à morphologie riche l'ont développée à partir d'une proto-langue isolante, ou au contraire que la proto-langue avait une morphologie riche, et que certaines branches l'ont perdu. L'approche traditionnelle sur le sino-tibétain, favorisant les langues écrites et en particulier le chinois et le birman au dépend des langues à tradition orale, privilégie la première hypothèse. 

Néanmoins, il existe des traces de morphologie même en chinois (\citealt{sagart99roc}), et la ressemblance frappante des systèmes verbaux des langues rgyalronguiques et kiranti, pourtant éloignées géographiquement et phylogénétiquement, ne peut s'expliquer comme des développements parallèles (\citealt{jacques10zos}, \citealt{jacques12agreement}). Pour évaluer l'hypothèse d'un proto-sino-tibétain à morphologie riche, la documentation détaillée des langues rgyalronguiques et kiranti et la reconstruction de ces deux sous-groupes est un préliminaire nécessaire. Ainsi, même si cette question motive une grande partie de mon travail, j'y ai encore peu contribué en terme de quantité d'articles, préférant d'abord bien maîtriser les données et décrire les langues que je soupçonne être les plus conservatrices du point de vue du système verbal, avant de mener de front une comparaison sur l'ensemble de la famille.
   
 La troisième orientation en linguistique diachronique a une portée plus générale: les principes des changements linguistiques, indépendamment d'une famille particulière. Mon travail dans ce domaine a porté sur la phonologie et la morphologie. 
 
 En phonologie, j'ai écrit des travaux sur les fricatives aspirées (\citealt{jacques11lingua}) et sur les transferts de nasalité entre consonnes initiales et voyelles (\citealt{michaud-jacques12nasalite}) en utilisant des données issues de mon terrain (en particulier pumi et japhug), dans lesquels je propose des chemins généraux d'évolution phonétiques. Je montre en particulier que pas moins de huit chemins différents peuvent créer des fricatives aspirées, et que la rareté de ces sons n'est pas due à un manque d'origines possibles, mais à la difficulté de maintenir ce contraste difficile à percevoir surtout pour les fricatives labiales et dorsales.  
 
En morphologie, je me suis intéressé à l'origine des marques de voix (en particulier l'antipassif) et à celle de l'incorporation, et ai démontré, sur la base des données du japhug, d'autres langues rgyalronguiques et de l'algonquien, l'existence de nouveaux chemins de grammaticalisation  (\citealt{jacques12incorp} et \citealt{jacques14antipassive}). J'y montre que l'une des origine des marques de voix est une dérivation en deux étapes, dans lequel le verbe de base est d'abord nominalisé et sa valence neutralisée, et  un affixe dénominal appliqué à cette forme nominalisée crée un nouveau verbe avec une structure argumentale différente du verbe originel. 

 
 
\subsection{Typologie}  \label{sec:typologie}
En typologie morphosyntaxique, j'ai contribué à quatre domaines distincts: la voix et la valence, le mouvement associé (avec une discussion de la question de l'harmonie transcatégorielle, une question importante pour certaines approches théoriques), les constructions possessives et les systèmes  `hiérarchiques' et direct / inverse.

Mon travail sur les marques de voix et l'incorporation n'est pas exclusivement d'un intérêt pour la typologie diachronique, mais également pour la morphosyntaxe générale: dans l'article \citet{jacques14antipassive} précédemment cité, outre la discussion de l'origine diachronique de l'antipassif, j'étudie aussi dans une perspective typologie sa combinaison avec d'autres marques de voix, et aussi de son emploi avec les verbes ditransitifs, sujet rarement abordé dans la littérature. En outre, dans \citet{jacques13tropative}, je présente un début de typologie des constructions tropatives (`beau' $\rightarrow$ `trouver beau') sur la base de données du japhug, du turc et du lakota. 

Mon deuxième sujet d'intérêt en typologie, le mouvement associé, a fait l'objet d'un article,  \citet{jacques13harmonization}. Ce travail décrit le système  de mouvement associé en japhug, présente ses différences avec les systèmes précédemment documentés, et aborde également la question de l'harmonie transcatégorielle de l'ordre des mots et des affixes. 

Si  le japhug n'a pas un système de mouvement associé d'une complexité comparable à celle des langues tacananes ou arandiques (\citealt{guillaume09mouv.assoc}), c'est une langue d'une grande importance pour décrire correctement ces systèmes, car on y trouve à la fois des marqueurs directionnels sur le verbe et des affixes de mouvement associé sur le verbe. Le mouvement associé, qui exprime  un déplacement ayant lieu avant, pendant ou après l'action décrite par le verbe, se distingue clairement en japhug  des directionnels, qui indiquent la façon dont l'action a lieu sans impliquer de déplacement. Ces deux notions sont confondues dans la plupart des langues ayant des marques de direction sur le verbe, et le japhug est l'une des rares langues où la distinction peut s'observer.

 Par ailleurs, mon travail est la première référence décrivant la différence sémantique entre l'emploi du mouvement associé et celui d'une construction de but à verbe de mouvement -- il pourrait être utile de tester si la même différence s'observe en tacana ou en arandique, et je compte par la suite étudier la typologie du mouvement associé dans d'autres langues (algonquien et toungouse).

Cet article montre par ailleurs que les morphèmes de mouvement associé ont été grammaticalisés à partir des verbes `aller' et `venir' comme préfixes, alors que le japhug et une langue strictement à verbe final, et que la construction à verbe de mouvement a le verbe de mouvement qui suit le verbe lexical. Le japhug montre donc, comme les langues athabasques, qu'une langue à verbe final peut privilégier le développement de préfixes, et que le \textit{Head Ordering Principle} proposé par \citet{hawkins88prefixing} n'a donc pas d'effet observable aussi bien en diachronie qu'en synchronie dans ces langues -- son caractère universel mérite donc d'être remis en question. 

Ma troisième contribution à la typologie porte sur les constructions possessives. Dans \citet{antonov14need}, coécrit avec Anton Antonov, nous évaluons la validité de l'universel proposé par \citet{harves12need}. Selon ces auteurs, les langues ayant un verbe transitif dont le sens correspond à `avoir besoin de' ont nécessairement aussi une construction possessive dans laquelle l'objet possédé est traité syntaxiquement de la même façon que l'objet (P) d'une construction transitive prototypique. Nous montrons que des contre-exemples à cette généralisation peuvent s'observer dans des langues diverses (arabe, bantou, quechua et estonien -- le japhug constitue aussi un contre-exemple, mais nous avons décidé de l'omettre pour des raisons de place). 
 
 La quatrième orientation de mes travaux en morphosyntaxe est la typologie des systèmes direct / inverse. Parmi les langues que j'ai étudiées, le japhug (\citealt{jacques10inverse}), le khaling (\citealt{jacques12khaling}) et le stau  (\citealt{jacques14rtau}) présentent des systèmes verbaux dont l'indexation polypersonnelle n'est ni de type nominatif/accusatif, ni de type absolutif/ergatif: certains index de personne  peuvent référer soit à l'agent soit au patient dans les conjugaisons transitives, selon la présence ou non d'autres affixes.
 
 L'indexation des arguments sur le verbe en khaling est très opaque et difficile à interpréter en termes fonctionnels -- seule une analyse morphomique permet de décrire ce système de façon efficace (\citealt{walther14compactness}). Le système d'indexation personnel du  japhug est en revanche plus transparent, et peut se décrire à partir de règles simples: c'est un système direct / inverse qui ressemble fortement à l'\textit{independent order} des langues algonquiennes.  
 
Afin de faciliter la diffusion de mes travaux sur les langues rgyalronguiques et kiranti auprès des américanistes et des typologues, j'ai produit  plusieurs travaux en collaboration (\citealt{jacques14inverse}, \citealt{jacques15directionality},   \citealt{walther14inv.canon}) portant sur les systèmes direct / inverse dans une perpective canonique, en confrontant les données sino-tibétaines à celle des langues  algonquiennes et sahaptiennes.   Dans un article en cours de relecture, j'étudie également l'effet du marquage direct / inverse sur les pivots syntaxique dans les relatives et les complétives en japhug, en comparaison avec d'autres langues comme le movima.
  
\subsection{Morphologie computationnelle}  \label{sec:morpho}
  
 \citet{jacques12khaling}
 \citet{walther14compactness}
 
\section{Encadrement et enseignement}

\subsection{Enseignement} \label{sec:enseignement}

\subsection{Encadrement d'étudiants en mastaire et en thèse} \label{sec:theses}
\citet{gongxun14agreement}
\citet{gong14prosodic.tibetan}
\citet{gong14prenasalized}

\citet{lai14caus}
\citet{lai14person}
\citet{lai13fuyin}
\citet{lai13affixale}
\section{Animation de la recherche} \label{sec:animation}

\subsection{Organisation de conférences} \label{sec:conf}

\subsection{Activités éditoriales}\label{sec:editorial}

\subsection{Projets de recherches} \label{sec:projets}


\bibliographystyle{linquiry2}
\bibliography{bibliogj}
 
\end{document}
