\documentclass[oneside,a4paper,11pt]{article} 
\usepackage{fontspec}
\usepackage{natbib}
\usepackage{booktabs}
\usepackage{xltxtra} 
\usepackage{polyglossia} 
\usepackage[table]{xcolor}
\usepackage{gb4e} 
\usepackage{multicol}
\usepackage{graphicx}
\usepackage{float}
\usepackage{tangutex2} 
\usepackage{tangutex4}
\usepackage{hyperref} 
\hypersetup{bookmarks=false,bookmarksnumbered,bookmarksopenlevel=5,bookmarksdepth=5,xetex,colorlinks=true,linkcolor=blue,citecolor=blue}
\usepackage[all]{hypcap}
\usepackage{memhfixc}
\usepackage{lscape}
\usepackage{amssymb}
\bibpunct[: ]{(}{)}{,}{a}{}{,}

%\setmainfont[Mapping=tex-text,Numbers=OldStyle,Ligatures=Common]{Charis SIL} 
\newfontfamily\phon[Mapping=tex-text,Scale=MatchLowercase]{Charis SIL} 
\newcommand{\ipa}[1]{\textbf{{\phon\mbox{#1}}}} %API tjs en italique

\newcommand{\grise}[1]{\cellcolor{lightgray}\textbf{#1}}
\newfontfamily\cn[Mapping=tex-text,Ligatures=Common,Scale=MatchUppercase]{SimSun}%pour le chinois
\newcommand{\zh}[1]{{\cn{#1}}}

\newcommand{\sg}{\textsc{sg}}
\newcommand{\pl}{\textsc{pl}}
\newcommand{\ro}{$\Sigma$}
\newcommand{\ra}{$\Sigma_1$} 
\newcommand{\rc}{$\Sigma_3$}  
\newcommand{\dhatu}[2]{|\ipa{#1}| `#2'}
\newcommand{\dhat}[1]{|\ipa{#1}|}
\newcommand{\change}[2]{*\ipa{#1} $\rightarrow$ \ipa{#2}}
\newcommand{\tgf}[1]{\mo{#1}}

\XeTeXlinebreakskip = 0pt plus 1pt %
 %CIRCG
 


\begin{document}

\title{The labial causative in Trans-Himalayan}
\author{Guillaume Jacques}
\maketitle
\sloppy

\section{Introduction}


 

\section{North-East India}

\subsection{Bodo-Garo}
Bodo \ipa{pʰV-},  \citealt{delancey15adjectival},
\subsection{Karbi}
 \citealt{konnerth15cisloc})  
Karbi \ipa{pa-} 
\subsection{Kuki-Chin}
\subsection{Historical accounts}
Previous authors have interpreted the presence of a labial stop causative prefix in various Trans-Himalayan languages of North-Eastern India as  grammaticalization from the root of the verb `to give'\footnote{The verb in question is attested throughout the family, cf Japhug \ipa{mbi} `give', Tibetan \ipa{sbʲin} `give' etc.}
(\citealt[132]{matisoff03}) or as borrowing from some Austroasiatic language (\citealt{konnerth15cisloc, delancey15adjectival}). Both hypotheses are problematic. 

First, since all languages in question are strictly verb-final, grammaticalization of a causative construction with the verb `to give' into a causative affix would be expected to yield a prefix; while exceptions to this well-known tendency are attested (see in particular \citealt{jacques13harmonization}), the hypothesis that unrelated branches of Trans-Himalayan such as Bodo-Garo, Karbi, Angami-Pochuri and Kuki-Chin would undergo such an unusual grammaticalization process independently is untenable. 

Second, borrowing of derivational morphology is attested, but only occurs in cases of extreme contact situations involving heavy lexical borrowing. Since Austroasiatic lexical influence on Trans-Himalayan languages of North-Eastern India is marginal, it is unlikely that the labial stop causative prefix in these languages was borrowed.

Both hypotheses depart from the common assumption that the labial stop causative prefix is an innovation. Yet, labial causative prefixes are found in languages outside of North-Eastern India. 

\section{Macro-Rgyalrongic}


\subsection{Core Rgyalrong}


\citet{jacques15causative}
\citet{jackson14morpho}
\citet{jackson06paisheng}
\citet{lai13affixale}


\begin{exe}
\ex
\gll 
\ipa{nɯnɯ} 	\ipa{ɕquwa} 	\ipa{nɯra} 	\ipa{nɯ-mɲaʁ} 	\ipa{ɯ-taʁ} 	\ipa{nɯtɕu,} 	\ipa{si} 	\ipa{kɯ-xtɕi} 	\ipa{ɣɯ} 	\ipa{ɯ-jwaʁ} 	\ipa{nɯnɯ} 	\ipa{a-pɯ-tu} 	\ipa{tɕe,} 	\ipa{nɯnɯ} 	\ipa{kɯ} 	\ipa{maka} 	\ipa{nɯ-mɲaʁ} 	\ipa{tu-ɣɤ-mtɤm} 	\ipa{cʰa} 	\ipa{ri} \\
\textsc{dem} blind.person \textsc{dem:pl} \textsc{3pl.poss}-eye \textsc{3sg}-on \textsc{dem:loc} tree \textsc{nmlz}:S/A-be.small \textsc{gen} \textsc{3sg.poss}-leave \textsc{dem} \textsc{irr-ipfv}-exist \textsc{lnk} \textsc{dem} \textsc{erg} at.all \textsc{3pl.poss}-eye \textsc{ipfv-caus}-have.sight[III] can:\textsc{fact} but \\
\glt `If one (puts) leaves of the small trees on the eyes of these blind people, it can cause their eye to recover sight. (hist140517 mogui de jing, 82-4)
\end{exe}


\subsection{Tangut}
Evidence for a labial causative is also found in Tangut as an \ipa{-w-} infix  which can originate from a labial stop prefix by application of regular sound changes (\citealt[253-4]{jacques14esquisse}).

Table \ref{tab:prefixep}  provides the clearest examples (from \citealt[45-6]{gong88alternations}) of causative \ipa{-w-} in Tangut.\footnote{Note that the character \tgf{0735} \ipa{dʑjwij¹} `make cold' is incorrectly reconstructed as $\dagger$\ipa{dʑjij¹} in \citet[144]{lifw97}; that a \ipa{-w-} medial must be restaured is proven by the \textit{fanqie} \tgf{2397}\tgf{5237} \ipa{dʑjwi²}\ipa{tśjwij¹} in the Wenhai dictionary, both of whose characters have  \ipa{-w-} medial.}

\begin{table}[H]
\caption{The causative -w- infix in Tangut }\label{tab:prefixep} 
\resizebox{\columnwidth}{!}{
\begin{tabular}{lllllllllll} 
\toprule
\multicolumn{4}{c}{base verb} &\multicolumn{4}{c}{causative form} & \\
\midrule
\tgf{3259} &\ipa{dzji} & 1.11 & be calm & \tgf{3411} &\ipa{dzjwi} & 1.11 & make calm \\
\tgf{1829} &\ipa{tshja} & 1.20 & hot, burn & \tgf{1825} &\ipa{tshjwa} & 1.20 & roast \\
\tgf{4033} &\ipa{dʑjij} & 2.32 & be cold & \tgf{0735} &\ipa{dʑjwij} & 1.35 & make cold \\
%\tgf{4186} &\ipa{khej} & 1.33 & lush & \tgf{1231} &\ipa{khwej} & 1.33 & develop, open (field)  \\
 \bottomrule
\end{tabular}}
\end{table}

 Although text examples of these pairs have not yet been identified, the definitions in the Wenhai monolingual dictionary (though not in the Chinese and English glosses in \citealt{lifw97}) clearly indicate that these verbs are causative forms, as their definitions include a stative verb followed by the causative \tgf{0749} \ipa{phji¹}, as shown below:

\begin{itemize}
\item  \tgf{3411}  \ipa{dzjwi¹}: \tgf{2518}\tgf{3098}\tgf{0749} \ipa{njiij¹djɨj²phji¹} `cause to the mind to settle'
\item \tgf{1825} \ipa{tshjwa¹}: \tgf{1829}\tgf{0749} \ipa{tshja¹phji¹} `cause to become hot'
\item  \tgf{0735} \ipa{dʑjwij¹}:  \tgf{4052}\tgf{0749} \ipa{dạ²phji¹} `cause to become cold'
\end{itemize}

As with the previous examples from Bodo-Garo and Rgyalrongic, the labial causative in Tangut is restricted to causativization of stative verbs.

 \section{Conclusion}
 
 
 
\bibliographystyle{unified}
\bibliography{bibliogj}
\end{document}

