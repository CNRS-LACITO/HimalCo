\documentclass[oneside,a4paper,11pt]{article} 
\usepackage{fontspec}
\usepackage{natbib}
\usepackage{booktabs}
\usepackage{xltxtra} 
\usepackage{polyglossia} 
\usepackage[table]{xcolor}
\usepackage{gb4e} 
\usepackage{multicol}
\usepackage{graphicx}
\usepackage{float}
\usepackage{tangutex2} 
\usepackage{tangutex4}
\usepackage{hyperref} 
\hypersetup{bookmarks=false,bookmarksnumbered,bookmarksopenlevel=5,bookmarksdepth=5,xetex,colorlinks=true,linkcolor=blue,citecolor=blue}
\usepackage[all]{hypcap}
\usepackage{memhfixc}

%\setmainfont[Mapping=tex-text,Numbers=OldStyle,Ligatures=Common]{Charis SIL} 
\newfontfamily\phon[Mapping=tex-text,Scale=MatchLowercase]{Charis SIL} 
\newcommand{\ipa}[1]{\textbf{{\phon\mbox{#1}}}} %API tjs en italique

\newcommand{\grise}[1]{\cellcolor{lightgray}\textbf{#1}}
\newfontfamily\cn[Mapping=tex-text,Ligatures=Common,Scale=MatchUppercase]{SimSun}%pour le chinois
\newcommand{\zh}[1]{{\cn{#1}}}

\newcommand{\sg}{\textsc{sg}}
\newcommand{\pl}{\textsc{pl}}  
\newcommand{\forme}[2]{\ipa{#1} `#2'} 
\newcommand{\change}[2]{*\ipa{#1} $\rightarrow$ \ipa{#2}}
\newcommand{\tgf}[1]{\mo{#1}}
\newcommand{\refb}[1]{(\ref{#1})}
\XeTeXlinebreakskip = 0pt plus 1pt %
\newcommand{\tld}{\textasciitilde{}}

\begin{document}

\title{The labial causative in Trans-Himalayan}
%\author{Guillaume Jacques}
\maketitle
\sloppy
\textbf{Abstract}: This paper proposes that the labial causative prefixes found in various Trans-Himalayan languages of North-Eastern India are not innovations as is generally assumed. Instead, it is argued that they are related to labial causative prefixes found in Rgyalrongic languages, whose traces are perhaps attested in other branches of the family.
%acknow: Linda Konnerth

\section{Introduction}
Most Trans-Himalayanists since \citet{conrady1896} agree that a sibilant causative prefix (corresponding to Tibetan \ipa{s-}) is reconstructible to the proto-language (for instance \citealt{wolfenden29outlines, matisoff03}). Direct or indirect traces of a sibilant causative are indeed widespread in most groups of the Trans-Himalayan family (also known as Sino-Tibetan or Tibeto-Burman), though voicing alternations originating from anticausative derivation are often mistaken for traces of the sibilant causative (see \citealt{jacques15causative}).

In this paper, I argue that another labial stop causative prefix might be potentially reconstructible to proto-Trans-Himalayan. This study is divided into four sections. First,  I discuss evidence for bilabial stop causative prefixes in languages of North-Eastern India. Second, I describe the uses of labial causative prefixes in Rgyalrongic languages and in Tangut and their reconstruction in proto-Rgyalrongic. Third, I briefly examine data from other groups of Trans-Himalayan. Fourth, I discuss the relationship between labial causatives and denominal labial prefixes in several Trans-Himalayan languages.

\section{North-East India}
Labial stop causatives are attested in at least six groups spoken in North-Eastern India and neighbouring areas, Bodo-Garo, Karbi, Kuki-Chin, Mru and several Naga languages, with various degrees of productivity.

\subsection{Bodo-Garo}
In addition to the sibilant causative \ipa{sV-}, Bodo and other Bodo-Garo languages have a non-productive labial prefix, used to derive causative verbs out of adjectives (\citealt[90]{mazo04st}, \citealt{delancey15adjectival}).

Table \ref{tab:bodo.phV} show representative examples of the labial causative  \ipa{pʰV-} in Bodo.

\begin{table}[H]
\caption{Example of \ipa{phV-} causative in Bodo, \citet[54]{delancey15adjectival}} \label{tab:bodo.phV}
\begin{tabular}{lllllll}
\toprule
Stative verb & Adjective & Causative \\
\midrule
\forme{dér}{be, get big} & \forme{ge-der}{big} & \forme{phe-der}{make big} \\
\forme{sung}{be, get short} & \forme{go-sung}{short} & \forme{pho-sung}{shorten} \\
\forme{zam}{become old (of things)} & \forme{gw-zam}{old, worn out} & \forme{phw-zam}{wear out} \\
\bottomrule
\end{tabular}
\end{table}

\subsection{Karbi} \label{sec:karbi}
Karbi has a fully productive \ipa{pa- \tld{} pe-} causative prefix (\citealt[93-4]{gruessner78mikir}, \citealt[238-9]{konnerth14karbi}) which can be applied to any verb. It can even be doubled, as in \ipa{pa-pe-mē} \textsc{caus-caus}-be.good `make sb improve smth'. 

The allomorph \ipa{pe-} occurs when preceding a monosyllabic stem (\citealt[105]{konnerth14karbi}; the distribution of the allomorphs \ipa{pe-} vs \ipa{pa-} differs between dialects). Some verbs present unpredictable tonal alternations when causatized with \ipa{pa- \tld{} pe-} (\citealt[103]{konnerth14karbi}). 

The causative \ipa{pa- \tld{} pe-} occurs in a construction where a causativized adjective describes the manner in which the action takes place, and a complement verb expresses the action itself, as in example \refb{ex:peme} (see section \ref{sec:japhug} where a similar construction is described in Japhug in example \ref{ex:akAtWGABdi}).

\begin{exe}
\ex \label{ex:peme}
\gll 
\ipa{lapènte} 	\ipa{menthū=tā} 	\ipa{ékdóm} 	\ipa{langpōng} 	\ipa{tòk} 	\ipa{pe-mé} \\
after.this dried.fish=\textsc{add} \textsc{exclamative} small.bamboo.container pound \textsc{caus}-be.good\\
\glt `After that, you need to pound the dried fish in the Langpong well.' (\citealt[124]{konnerth14karbi}).
\end{exe}

The causative \ipa{pa- \tld{} pe-} prefix is related to, and perhaps even synchronically identical to the homophonous denominal \ipa{pa- \tld{} pe-}, a question investigated in more detail in section \refb{sec:karbi.denom}.

\subsection{Kuki-Chin}  \label{sec:kc}
Labial causative prefixes are found in several Kuki-Chin languages, in particular \ipa{pə-} in Maraa (\citealt[139]{hartmann01prenasalization}, Table \ref{tab:maraa}), \ipa{p-} in Khumi (\citealt[99]{peterson10elaborate}) and \ipa{pə-} \tld{} \ipa{pu-} \tld{} \ipa{pər-} in in in Lamkang (\citealt[52-4]{chelliah07lamkang}).

Note that Khumi has another causative prefix \ipa{t-} (\citealt[12]{hartmann13valence}). It perhaps originates from the sibilant causative. Although pre-Kuki-Chin *\ipa{s-} changes to aspirated *\ipa{th-} in proto-Kuki-Chin (\citealt[16]{vanbik09pkc}), presyllables have more reduced phonological contrasts and it is thus possible that *\ipa{th-} lost its aspiration in this position (see \citealt{jacques12agreement} for examples of this phenomenon elsewhere in Trans-Himalayan). Although aspiration alternation in Kuki-Chin languages are explained as traces of the sibilant causative (see \citealt[220;259]{vanbik09pkc}), this prefix could have had several allomorphs in the ancestor of Kuki-Chin as it still has in Rgyalrongic languages.\footnote{In all Rgyalrongic languages, sibilant causative prefixes present several regular as well as irregular allomorphs, see for instance \citet{jackson07shangzhai} on Stodsde, \citet{jacques15causative} on Japhug and \citet{lai16caus} on Khroskyabs. In Japhug we find both \ipa{z-} (before sonorant prefixes) and \ipa{sɯ-} (in most contexts).} yielding radically different reflexes.


\begin{table}[H]
\caption{Examples of the causative prefix \ipa{pə-} In Maraa (\ipa{pa-} in local orthography)} \centering \label{tab:maraa}
\begin{tabular}{lllllll}
\toprule
Base verb & Causative verb \\
\midrule
\forme{a-rhei}{lives} & \forme{a-pa-rhei}{causes to live} \\
\forme{a-thi}{dies} & \forme{a-pa-thi}{causes to die} \\
\forme{a-chi}{is bad} & \forme{a-pa-chi}{makes bad} \\
\bottomrule
\end{tabular}
\end{table}

Other Kuki-Chin languages such as Daai Chin and Mro have a causative prefix \ipa{m-} in corresponding forms, as shown in Table \ref{tab:kukichin}  (data from \citealt[139]{hartmann01prenasalization}).

\begin{table}[H]
\caption{Labial causative prefixes in several Kuki-Chin languages} \centering \label{tab:kukichin}
\begin{tabular}{lllllll}
\toprule
Daai & Mro & Maraa & Meaning \\
\midrule
\ipa{m-thoh}  &\ipa{m-thau}  & \ipa{(a)pa-thao} & `cause to wake up' \\
\ipa{m-thu}  &  & \ipa{(a)pa-thu} & `cause to rot' \\
&\ipa{m-xin}  & \ipa{(a)pa-rhei} & `cause to live' \\
\bottomrule
\end{tabular}
\end{table}

Since Maraa also has \ipa{m-} prefixes (such as the reciprocal \ipa{ma-}, as in \forme{a-thei}{kills} $\rightarrow$ \forme{a-ma-thei}{kill each other}), while Daai Chin lacks \ipa{pV-} prefixes (none is mentioned in \citealt{hartmann09grammar}), it can be concluded that Maraa is more conservative, and that in Daai Chin and Mro labial stop prefixes have become nasal -- a parallel sound change also affecting presyllables is attested in Buyang, a Kra-Dai language of Guangxi (\citealt{jacques17buyang}).

\subsection{Angami-Pochuri}
Angami has a causative prefix \ipa{pə-} of great productivity, occurring with both dynamic verbs and adjectives (\citealt[132-3]{matisoff03}, \citealt[66-67]{giridhar80angami}), for instance \forme{ŋū}{see} $\rightarrow$ \forme{pə-ŋū}{show}.

\subsection{Tangkhul} \label{sec:tangkhul}
A non-productive labial causative \ipa{mə-} is attested in Tangkhul Naga and its sister languages, as in \forme{kə̀.thaw}{be fat} $\rightarrow$ \forme{khə̀.mə̀.thaw}{fatten} and \forme{kə̀.theŋ}{be dry}  $\rightarrow$ \forme{kə̀.mə̀.then}{make dry}  (\citealt[23]{mortensen03tangkhul}). 

\citet[23]{mortensen03tangkhul} suggests a relationship to the denominal \ipa{mə-} (\forme{thèj}{fruit}  $\rightarrow$ \forme{kə̀.mə̀.thèj}{bear fruit}) and the verb `give' (Proto-Tangkhul *\ipa{mi}) but also points out the similarity with the Kuki-Chin data mentioned in section \ref{sec:kc}.

\subsection{Mru} \label{sec:mru}
Mru is the only one among Trans-Himalayan languages spoken to the West of Burma with SVO basis word order (\citealt{peterson05mru}). It has a rich array of prefixes (\citealt{williams08directionals}), including a causative prefix \ipa{p'-} which can be added to transitive verbs, as in \refb{ex:mru.caus}:\footnote{In this transcription, \ipa{p'-} represents a reduced syllable [\ipa{pə-}]. The clitic \ipa{=’ö’} is glossed as \textsc{emot} in the source, but no explanation is given for this abbreviation.}
 
\begin{exe}
\ex \label{ex:mru.caus}
\gll
\ipa{öta=mi=pe} \ipa{rik-tüng} \ipa{o} \ipa{öpa=pe} \ipa{p’-rik-tüng=’ö’} \\
older.brother=\textsc{dem-evidential} \textsc{first}-cut \textsc{interjection} father=\textsc{evidential} \textsc{caus-first}-cut-?? \\
\glt `Then they let the elder brother cut first, er, they let the father go first.' (\citealt[52]{williams08directionals})
\end{exe}
 
\subsection{The labial causative: innovation or archaism} \label{sec:innovation}
Previous authors have interpreted the presence of labial stop causative prefixes in various Trans-Himalayan languages of North-Eastern India as  grammaticalization from the root of the verb `to give'\footnote{The verb in question is attested throughout the family, cf Japhug \forme{mbi}{give}, Tibetan \forme{sbʲin}{give} etc.}
(\citealt[132]{matisoff03}, \citealt{jenny15give}) or as borrowing from some Austroasiatic language (\citealt{maspero46, diffloth08parallels, konnerth15cisloc, delancey15adjectival}). Both hypotheses are problematic. 

First, since all languages in question (except Mru) are strictly verb-final, grammaticalization of a causative construction with the verb `to give' into a causative affix would be expected to yield a suffix. Exceptions to this well-known tendency are attested in the Trans-Himalayan family (see in particular \citealt{jacques13harmonization}), and one cannot exclude the possibility that some of the labial causative prefixes are indeed grammaticalized from a verb.\footnote{The verb `give' is not the only possibility. An alternative would be a verb `to do' (found in Japhug \forme{pa}{make, do, open} and Tibetan \forme{bʲed, bʲas, bʲa}{do}), for instance.} 

However, since  Bodo-Garo, Karbi, Angami-Pochuri, Tangkhul and Kuki-Chin belong to different primary branches of Trans-Himalayan,\footnote{\citet{delancey15central} has suggested, on the basis of morphological evidence, for a close relationship between Kuki-Chin and Jinghpo (together with other Sal languages, \citealt{burling83sal}); this hypothesis remains to be confirmed by evidence from lexical innovations.} the hypothesis of such an unusual grammaticalization process independently occurring five times is untenable.

As for the second hypothesis, borrowing of derivational morphology is attested, but only occurs in cases of extreme contact situations involving heavy lexical borrowing. Since Austroasiatic lexical influence on Trans-Himalayan languages of North-Eastern India has never been systematically documented and appears to be marginal, it is unlikely that the labial stop causative prefix in these languages could be borrowed.

Both hypotheses depart from the common assumption that the labial stop causative prefix is an innovation. Yet, labial causative prefixes are found in Trans-Himalayan languages outside of North-Eastern India. 

\section{Macro-Rgyalrongic}
Labial causative prefixes are well-attested in Rgyalrongic languages and Tangut, where they coexist with sibilant causative prefixes.

\subsection{Rgyalrong} \label{sec:japhug}
All Rgyalrongic languages\footnote{On the definition of the Rgyalrongic subgroup and its internal classification, see \citet{jackson00sidaba}, \citet{jackson00puxi}, \citet{jacques14esquisse} and \citet{lai15person}.} have at least two causative prefixes, a sibilant prefix and a labial prefix whose form goes back to proto-Rgyalrong *\ipa{wɐ-} (Japhug \ipa{ɣɤ-}, Tshobdun \ipa{wɐ-}, Khroskyabs \ipa{v-}, see \citealt[322]{jacques04these}, \citealt{jackson06paisheng} and \citealt[136]{lai13affixale}).\footnote{The labial causative only has limited allomorphy. In Japhug a few verbs with nasal initial have a causative allomorph in \ipa{m-}, for instance \ipa{mɲo} ($\leftarrow$*\ipa{w-ɲaŋ}) `prepare' from \forme{ɲo}{be ready, be prepared} (I owe this observation to Gong Xun, p.c.). Another irregular labial causative is \ipa{βri} `protect' from the intransitive \ipa{ri} `remain'.}

The labial causative is only used to derive stative verbs, including adjectives. It can be considered to be  productive at least in Japhug, since it can be applied to adjectives of Tibetan origin, such as \forme{dɤn}{be many} (from \forme{ldan}{having X}) $\rightarrow$ \forme{ɣɤdɤn}{increase the number of} or \forme{βdi}{be good, be well} (from \forme{bde}{good, well, peaceful}) $\rightarrow$ \forme{ɣɤβdi}{repair, fix, make better}.

In addition to its use as a plain causative (example \ref{ex:tuGABdi}), the prefix \ipa{ɣɤ-} occurs in a manner construction, where a causativized adjective expressing the manner takes a complement verb in infinitival form describing the action (example \ref{ex:akAtWGABdi}). Note that a similar construction involving a labial causative prefix has been described in Karbi (see section \ref{sec:karbi}).

\begin{exe}
\ex \label{ex:tuGABdi}
 \gll \ipa{a-ʁi} 	\ipa{kɯ} 	\ipa{nɯ} 	\ipa{ma} 	\ipa{spe} 	\ipa{me} 	\ipa{ri,} 	<tuolaji> 	\ipa{kɯ-fse,} 	\ipa{mkʰɯrlu} 	\ipa{nɯra} 	\ipa{tu-ɣɤβdi} 	\ipa{spe} \\
 \textsc{1sg.poss}-younger.sibling \textsc{erg} \textsc{dem} apart.from be.able[III]:\textsc{fact} not.exist:\textsc{fact} but tractor \textsc{nmlz}:S/A-be.like.this machine \textsc{dem:pl} \textsc{ipfv-caus}-be.good be.able[III]:\textsc{fact} \\
 \glt `My brother is only able to do one thing, repair tractors and cars.' (14-tApitaRi, 166)
\end{exe}

\begin{exe}
\ex \label{ex:akAtWGABdi}
\gll \ipa{kʰa} 	\ipa{ɯ-ʁɤri} 	\ipa{nɯtɕu} 	\ipa{ɯ-fkrɤm} 	\ipa{a-kɤ-tɯ-ɣɤ-βdi} 	\ipa{tɕe,} 	\ipa{ɕ-pɯ-sɤtse} \\
house \textsc{3sg.poss}-front.of \textsc{dem:loc} \textsc{3sg.poss-bare.inf}:place \textsc{irr-pfv-2-caus}-be.good \textsc{lnk} \textsc{transloc-imp}-stick.into[III] \\
\glt `Place these in front of your house in orderly fashion and stick them (into the ground).' (Smanmi 2003, 129)
\end{exe}


Some stative verbs take the sibilant causative (for instance adjectives of colour, \citealt[183]{jacques15causative}), and quite a few stative verbs can have both sibilant and labial causatives. In Tshobdun, \citet{jackson06paisheng, jackson14morpho} reports a difference of meaning between the two, as illustrated by examples \refb{ex:sEGchi} and \refb{ex:wAchi}.

\begin{exe}
\ex \label{ex:sEGchi}
\gll \ipa{cʰɐ́ɟi}	\ipa{ne-kɐ-səɣ-cʰiʔ=nəʔ}	\ipa{mimʔ=cə} \\
beer \textsc{ipfv-genr-caus}-be.sweet=\textsc{subord} be.tasty=\textsc{mediative} \\
\glt `Beer is tasty when one allows it to sweeten (naturally and gradually).'
\end{exe}

 \begin{exe}
\ex \label{ex:wAchi}
\gll \ipa{cʰɐ́ɟi}	\ipa{ne-kɐ-wɐ-cʰiʔ=nəʔ}	\ipa{mimʔ=cə} \\
beer \textsc{ipfv-genr-caus}-be.sweet=\textsc{subord} be.tasty=\textsc{mediative} \\
\glt `Beer is tasty when one sweetens it (e.g. by adding sugar).'
\end{exe}

As shown by this minimal pair, in Tshobdun the sibilant causative \ipa{sə(ɣ)-} implies `an increase in the degree  of the predicated state' (eg, `make something sweeter'), while the labial causative \ipa{wɐ-} is used to express the `causation of a changed state' (eg, `make something sweet').

Although a cognate pair exists in Japhug \ipa{sɯx-cʰi} vs \ipa{ɣɤ-cʰi} `sweeten', no semantic contrast could be ascertained with my Japhug consultants. The only minimal pair in Japhug with an observable meaning is that between \forme{sɯ-mto}{cause to see, show} (example \ref{ex:sWmto}) and \forme{ɣɤ-mto}{cause to recover sight (of the eyes of a blind man)} (example \ref{ex:GAmto}), which derive from the labile verb from \forme{mto}{see} (meaning `have clear sight (of eyes)' when used as an intransitive stative verb).


\begin{exe}
\ex \label{ex:sWmto}
\gll
\ipa{mɯntoʁ} 	\ipa{nɯstʰɯci} 	\ipa{kɯ-mpɕɤr} 	\ipa{nɯ,} 	\ipa{nɯ} 	\ipa{mɤɕtʂa} 	\ipa{kɤ-mto} 	\ipa{mɯ-pɯ-rɲo-j} 	\ipa{ri,} 	\ipa{pɯ-kɯ-sɯ-mto-j} 	\ipa{tɕe,} \\
flower so.much \textsc{nmzl}:S/A-be.beautiful \textsc{dem} \textsc{dem}  until \textsc{inf}-see \textsc{neg-pfv}-experience-\textsc{1pl} but \textsc{pfv}-2$\rightarrow$1-\textsc{caus}-see-\textsc{1pl} \textsc{lnk} \\
\glt `We had never such a beautiful flower, but you showed it to us.' (150824 yuanding, 179)
\end{exe}

\begin{exe}
\ex \label{ex:GAmto}
\gll 
\ipa{nɯnɯ} 	\ipa{ɕquwa} 	\ipa{nɯra} 	\ipa{nɯ-mɲaʁ} 	\ipa{ɯ-taʁ} 	\ipa{nɯtɕu,} 	\ipa{si} 	\ipa{kɯ-xtɕi} 	\ipa{ɣɯ} 	\ipa{ɯ-jwaʁ} 	\ipa{nɯnɯ} 	\ipa{a-pɯ-tu} 	\ipa{tɕe,} 	\ipa{nɯnɯ} 	\ipa{kɯ} 	\ipa{maka} 	\ipa{nɯ-mɲaʁ} 	\ipa{tu-ɣɤ-mtɤm} 	\ipa{cʰa} 	\ipa{ri} \\
\textsc{dem} blind.person \textsc{dem:pl} \textsc{3pl.poss}-eye \textsc{3sg}-on \textsc{dem:loc} tree \textsc{nmlz}:S/A-be.small \textsc{gen} \textsc{3sg.poss}-leave \textsc{dem} \textsc{irr-ipfv}-exist \textsc{lnk} \textsc{dem} \textsc{erg} at.all \textsc{3pl.poss}-eye \textsc{ipfv-caus}-have.sight[III] can:\textsc{fact} but \\
\glt `If one (puts) leaves of the small trees on the eyes of these blind people, it can cause their eyes to recover sight. (hist140517 mogui de jing, 82-4)
\end{exe}

While it is yet unclear whether the subtle semantic contrast observed in Tshobdun is an archaism or an innovation of this language, there is no doubt that a labial causative *\ipa{wɐ-} of stative verbs has to be reconstructed to proto-Rgyalrongic. 

\subsection{Tangut}
Evidence for a labial causative is also found in Tangut as an \ipa{-w-} infix  which can originate from a labial stop prefix by application of regular sound changes (\citealt[253-4]{jacques14esquisse}).

Table \ref{tab:prefixep}  provides the clearest examples (from \citealt[45-6]{gong88alternations}) of causative \ipa{-w-} in Tangut.\footnote{Note that the character \tgf{0735} \forme{dʑjwij¹}{make cold} is incorrectly reconstructed as $\dagger$\ipa{dʑjij¹} in \citet[144]{lifw97}; that a \ipa{-w-} medial must be restaured is proven by the \textit{fanqie} \tgf{2397}\tgf{5237} \ipa{dʑjwi²}\ipa{tśjwij¹} in the Wenhai dictionary, both of whose characters have  \ipa{-w-} medial.}

\begin{table}[H]
\caption{The causative -w- infix in Tangut }\label{tab:prefixep} 
\resizebox{\columnwidth}{!}{
\begin{tabular}{lllllllllll} 
\toprule
\multicolumn{4}{c}{base verb} &\multicolumn{4}{c}{causative form} & \\
\midrule
\tgf{3259} &\ipa{dzji} & 1.11 & be calm & \tgf{3411} &\ipa{dzjwi} & 1.11 & make calm \\
\tgf{1829} &\ipa{tshja} & 1.20 & hot, burn & \tgf{1825} &\ipa{tshjwa} & 1.20 & roast \\
\tgf{4033} &\ipa{dʑjij} & 2.32 & be cold & \tgf{0735} &\ipa{dʑjwij} & 1.35 & make cold \\
%\tgf{4186} &\ipa{khej} & 1.33 & lush & \tgf{1231} &\ipa{khwej} & 1.33 & develop, open (field)  \\
 \bottomrule
\end{tabular}}
\end{table}

 Although text examples of these pairs have not yet been identified, the definitions in the Wenhai monolingual dictionary (though not in the Chinese and English glosses in \citealt{lifw97}) clearly indicate that these verbs are causative forms, as their definitions include a stative verb followed by the causative \tgf{0749} \ipa{phji¹}, as shown below:

\begin{itemize}
\item  \tgf{3411}  \ipa{dzjwi¹}: \tgf{2518}\tgf{3098}\tgf{0749} \forme{njiij¹djɨj²phji¹}{cause to the mind to settle}
\item \tgf{1825} \ipa{tshjwa¹}: \tgf{1829}\tgf{0749} \forme{tshja¹phji¹}{cause to become hot}
\item  \tgf{0735} \ipa{dʑjwij¹}:  \tgf{4052}\tgf{0749} \forme{dạ²phji¹}{cause to become cold}
\end{itemize}

As with the previous examples from Bodo-Garo and Rgyalrongic, the labial causative in Tangut is restricted to causativization of stative verbs.

\subsection{Stop lenition}

The labial causative prefixes found in Macro-Rgyalrongic presents one important commonality with Bodo-Garo: they are specifically used to derive causative forms of adjectives and other stative verbs.

On the other hand, they differ from the causative prefixes found in languages of North-Eastern India by having labial approximants (or segments regularly originating from labial approximants such as \ipa{ɣ} in Japhug). 

Although this phonological difference could appear to an insuperable obstacle to comparison between the labial causative prefixes of Bodo-Garo and Rgyalrongic, we have to take into account several facts. 

First, there is a recurrent correspondence of *\ipa{w} in Rgyalrongic languages in some words with labial stops in languages outside of Rgyalrongic, as exemplified by the verb `come' (Japhug \forme{ɣi}{come}, Tshobdun \forme{wi}{come}, \citealt[237]{sun14generic}, Khroskyabs \ipa{vjî}, \citealt[276]{lai15person}), whose cognates in Bodo-Garo and Kuki-Chin have labial stops (Bodo \forme{pay}{come}, Proto-Kuki-Chin *\forme{paay}{go} \citealt{vanbik09pkc}). While the exact conditioning of this lenition in Rgyalrong languages (and its complex distribution across Rgyalrongic languages, not all which have lenition in the same words), the correspondence of stop to labial approximant observed with the labial causative is by no means exceptional.

Second, there is a strong phonological constraint on the shape of derivational prefixes in Rgyalrong languages. Despite the richness of derivational prefixes (see \citealt{jackson14morpho, jacques14antipassive}), they are all built from a restricted section of the phonological inventory.

In Japhug for instance, out of 50 consonantal phonemes, only nine of them can occur in derivational prefixes in Japhug (\ipa{m}, \ipa{n}, \ipa{r}, \ipa{j}, \ipa{ɣ}, \ipa{s}, \ipa{z}, \ipa{ɕ} and \ipa{ʑ}):\footnote{Moreover, voicing in fricative prefixes is always predictable from the phonological context, so that strictly phonologically speaking, only seven distinct phonemes are used to build derivational prefixes in Japhug.} some sonorants and fricatives (all continuant consonants). Stops prefixes are found in person indexation, TAM and participle prefixes, in other words all prefixes outside of the verb stem. 

It is thus possible to propose that stops systematically underwent lenition in derivational prefixes already in proto-Rgyalrongic (including *\ipa{p/b} $\rightarrow$ *\ipa{w}). This lenition does not need to be formulated as a grammatically conditioned sound change (which is not acceptable in a Neogrammarian approach, see \citealt{hill14conditioned}). 

Prefixes (and all syllables but the last of the verb stem) rarely receive stress and never have tonal contrasts (see \citealt{jackson05yingao}) in Rgyalrongic languages. Derivational prefices, unlike person and TAM prefixes, are very rarely word-initial (since they occur between TAM affixes and the verb root, see the verbal templates of Japhug and Khroskyabs in \citealt{jacques13harmonization} and \citealt{lai15person} respectively). Thus, lenition of stops can be defined as having happened to intervocalic stops in unaccented position at the proto-Rgyalrongic stage.\footnote{This sound law would have created some degree of allomorphy, soon suppressed by analogy.}

There is thus no phonological obstacle against comparing Rgyalrongic *\ipa{wɐ-} to labial stop prefixes in other languages.

 \section{Other Trans-Himalayan languages} 
Evidence for a labial stop causative  prefix in Trans-Himalayan languages outside of Macro-Rgyalrongic and North-Eastern India exist, but are less compelling.

\subsection{Chinese}
 \citet[593]{maspero52} proposed to reconstruct a prefix *\ipa{p-} in Old Chinese, and postulated causative as one of its values. However,  while there is some evidence for the existence of *\ipa{p-} prefixes in Old Chinese, their grammatical function are still poorly understood (\citealt[87-9]{sagart99roc}, \citealt{behr10ocp}).
 
 \citet[154]{bs14oc} cite one potential example of causative \ipa{*p-} prefix in Old Chinese \zh{廢} *\ipa{[p-k]aps} $\rightarrow$ \ipa{pjojH} (fèi) `cast aside', if derived from 
 \zh{去} *\ipa{kʰ(r)ap-s} $\rightarrow$ \ipa{kʰjoH} (qù) `depart'.\footnote{ \citet[153]{bs14oc} suggest a dialectal development *\ipa{-ps} $\rightarrow$ *\ipa{-ks} instead of regular *\ipa{-ps} $\rightarrow$ *\ipa{-ts} in this word. } This tantalizing hypothesis needs to be confirmed by philological investigations and by more examples of this putative prefix.
  
\subsection{Tibetan}
A few apparent examples of \ipa{b-} causative prefixes are found in Tibetan, such as for instance \forme{ɴgril}{roll down (it), gathered together} $\rightarrow$ \forme{bgril}{cause to roll down} (`\ipa{ɴgril-du ɴdʑug.pa}' in \citealt{bodrgya}), alongside the sibilant causative \forme{sgril}{roll up, wrap, combine}.

While such examples could be interpreted as traces of a Trans-Himalayan labial causative, it is necessary to take into account an alternative possibility. Past and future tense can be marked with a transitive \ipa{b-} prefix in Old and Classical Tibetan (using the traditional terminology, regardless of its actual TAM value), and cases have been documented of past prefixes being reinterpreted as part of the stem (especially verbs whose stem starts with \ipa{r-} or \ipa{l-}, see  \citet{hill05vbri, jacques10ndr, hill15lan}).

It is therefore conceivable that  \ipa{bgril} is the generalized past tense of a transitive verb whose present could be either $\dagger$\ipa{ɴgril} or $\dagger$\ipa{dgril};\footnote{Philological research is needed to ascertain whether such forms could be attested.} this verb would be related to \forme{ɴgril}{roll down (it)}, but the \ipa{b-} would not have a derivational function. 
 
For this reason, Tibetan evidence should be used with circumspection, and thorough philological studies of the use of verb stems in Old Tibetan texts must be undertaken before any such forms is adduced as a an example of labial causative in Tibetan.

\subsection{Dulong-Rawang}
While available descriptions of Dulong and Rawang (\citealt{sunhk82dulong, lapolla01valency}) do not describe any labial causative prefix, \citet[91]{sunhk05anong} mention a few examples which could be interpreted as  such in Anong (Table \ref{tab:anong}).

\begin{table}[H]
\caption{Labial causative prefix in Anong} \centering \label{tab:anong}
\begin{tabular}{lllllllllll}
\toprule
Meaning & Base verb& Causative \\
\midrule
collapse (wall) & 	\ipa{dím} & 	\ipa{bɯ̀ dím} \\ 
be flat & 	\ipa{ɑ̀ dʑɑ̀} & 	\ipa{pʰɑ́ dʑɑ̀} \\ 
be sour & 	\ipa{m̀ tɕʰɯ́m} & 	\ipa{pʰɑ̀m tɕʰɯ́m} \\ 
\bottomrule
\end{tabular}
\end{table}

Anong is known for being highly endangered and phonologically innovative in comparison with Dulong and Rawang (\citealt[133-152]{sunhk05anong}), so that if these pairs are indeed the trace of a causative prefix lost in its sister languages, this would be one of the few archaisms of this language. 


\section{Causative and denominal} \label{sec:denominal}
Given the known historical relationship between the voice derivations and denominal derivations in general (\citealt{jacques14antipassive}), and the sibilant causative prefix and the causative denominal in particular (\citealt{jacques15causative}), the question of whether labial causative prefixes in Trans-Himalayan could originate from denominal derivations deserves to be taken into account.

I focus in this section on Tibetan, Karbi and Rgyalrongic, three branches where denominal labial prefixes have been documented.\footnote{A relationship between the two derivations has been suggested about other languages such as Tangkhul (see section \ref{sec:tangkhul}), but data are too limited for a detailed discussion.  }

\subsection{Tibetan} \label{sec:tib.denom}
There is some evidence for a denominal \ipa{b-} in Tibetan (\citealt[100]{mazo04st}) \citealt[250-1]{zhang09cizu}). Examples of denominal \ipa{b-} deriving transitive verbs suffer from the same problem as the hypothesized causative \ipa{b-}: it is unclear whether the \ipa{b-} is a real denominal prefix or a generalized past tense prefix reanalyzed as part of the stem.

A common way of building denominal verbs is by applying the prefix \ipa{ɴ-} in the present and \ipa{b-...-s} in the past, as in \forme{tɕʰu}{water} $\rightarrow$ \forme{ɴtɕʰu, btɕus}{to scoop water} or \forme{ʑo}{curd}\footnote{The meaning `curd' is secondary, see from `milk', see \citet[29-30]{jacques14esquisse} and \citet{tournadre15chocha}.} $\rightarrow$ \forme{ɴdʑo, bʑos}{to milk}.

Some apparent examples of denominal \ipa{b-} in Tibetan are undeniably verbs derived using this pattern which have undergone subsequent paradigm leveling. A good example of this phenomenon is the verb `butcher' whose paradigm is given as \ipa{bɕa, bɕas} in \citet{bodrgya}. This verb is obviously denominal from \forme{ɕa}{meat}. Its expected present would be $\dagger$\ipa{ɴtɕʰa} (from *\ipa{n-ɕa}, see \citealt{lifk33}); this form is not attested in written Tibetan texts (as far as I know), but reflected in the old borrowing \forme{ntɕʰa}{to butcher} in Japhug and thus must have existed in some ancient Tibetan language.\footnote{I am grateful to Nathan W. Hill for this observation (p.c., 2012).} Such indirect evidence is not available for all potential examples of denominal \ipa{b-}, but a detailed case-by-case evaluation is necessary before any conclusion can be taken.

There are in addition a few examples of intransitive verbs derived with \ipa{b-}, for  instance \forme{ro}{smell (n)} $\rightarrow$ \forme{bro}{arise, experience, taste, desire}). Such examples cannot be explained as analogical levelling, and thus represent genuine cases of denominal \ipa{b-}. However, their semantics does not allow any comparison with the labial causative.


\subsection{Karbi} \label{sec:karbi.denom}
\citet[238]{konnerth14karbi} shows that the  \ipa{pa- \tld{} pe-} causative prefix in Karbi is related to the denominal  \ipa{pa- \tld{} pe-} (\citealt[205]{konnerth14karbi}). This relationship can be interpreted in three ways. 

First, both prefixes could be grammaticalized from a light verb (though this raises the question of how a light verb would become a prefix in a verb final language, see \ref{sec:innovation}). 

Second, the causative could derive from the denominal (a scenario proposed for another causative prefix in Japhug in \citealt{jacques15causative}). 

Third, as suggested by Konnerth, the denominal use could derive from the causative in this language, since nouns can function as predicates.

Without further data from Karbi's closest relatives, it is difficult for now to draw any definite conclusion as to the direction of grammaticalization. The labial causative in Karbi could derive from the denominal prefix, but the alternative direction is also conceivable in this case.

\subsection{Rgyalrongic} \label{sec:rgyalrong.denom}
In Japhug, in addition to the causative \ipa{ɣɤ-}, we find an homophonous deideophonic \ipa{ɣɤ-} deriving intransitive verbs from ideophones (see \citealt{japhug14ideophones}, and also \citealt{jackson04zhuangmaoci, jackson14morpho} on the cognate prefix \ipa{wɐ-} in Tshobdun).

We also find examples of denominal \ipa{ɣɤ-} deriving either stative proprietive verbs (X $\rightarrow$ `having X') or transitive verbs with an unpredictable meaning (see Table \ref{tab:denominal.GA}).

\begin{table}[H]
\caption{Denominal \ipa{ɣɤ-} prefixes in Japhug} \label{tab:denominal.GA} \centering
\begin{tabular}{llllll}
\toprule
 & Example (Base noun) & (Denominal verb)\\
\midrule
(stative)& \forme{tɤ-di}{smell} & \forme{ɣɤ-di}{smell (badly)} \\
(stative) & \forme{tɤ-jwaʁ}{leaf} & \forme{ɣɤ-jwaʁ}{growing leaves} \\
\midrule
(tr) & \forme{tɤpra}{messenger} & \forme{ɣɤ-xpra}{order, send} \\
(tr) & \forme{tɯ-jmŋo}{dream} & \forme{ɣɤ-jmŋo}{see X in a dream} \\
\bottomrule
\end{tabular}
\end{table}

There strong evidence in Japhug and other Rgyalrong languages than several voice prefixes are parallel to, and originate from denominal prefixes (see Table \ref{tab:denom}, \citealt{jacques14antipassive}). As shown in \citet{jacques14antipassive} and \citet{jacques15causative}, the derivation from denominal to voice marker occurs in two stages.

First, the verb is nominalized into a bare action nominal consisting of the verb stem with a possessive prefix, without any nominalization marking (for instance, the transitive verb \forme{ɕpʰɤt}{patch} is nominalized as \forme{(tɤ-)ɕpʰɤt}{a patch}, a possessed noun). The transitivity of the verb is neutralized in this process. 

Second, the bare infinitive undergoes denominal derivation into a new verb, whose transitivity and argument structure is determined by the denominal prefix. For instance, the antipassive \forme{rɤ-ɕpʰɤt}{patch things} from  \forme{ɕpʰɤt}{patch} can be historically (and perhaps even still synchronically) analyzed as  derived from \forme{(tɤ-)ɕpʰɤt}{a patch} by addition of \ipa{rɤ-} intransitive dynamic denominal prefix. 

\begin{table}[H] \caption{Voice markers and corresponding denominal derivations} \label{tab:denom} \centering
\begin{tabular}{lllllllll} \toprule
Form& Voice & Corresponding denominal prefix \\
\midrule
\ipa{rɤ}- & Antipassive &    \ipa{rɤ}- (intransitive dynamic verbs)\\
\ipa{nɯ(ɣ)}- & Applicative &    \ipa{nɯ(ɣ)}- (transitive dynamic verbs)\\
\ipa{sɯ(ɣ)}- & Causative &    \ipa{sɯ(ɣ)}- (verb meaning `use X' or \\
&& `cause to have X') \\
\ipa{a}- & Agentless Passive &    \ipa{a}- (stative verb)\\
\ipa{sɤ}-  & Deexperiencer &    \ipa{sɤ}- (stative verb expressing a property)\\
    \bottomrule
\end{tabular}
\end{table}

However, in the case of the causative \ipa{ɣɤ-}, evidence for reanalysis of a denominal prefix as a voice marker is almost non-existant.

Denominal \ipa{ɣɤ-} mainly derive stative verbs. Transitive denominal verbs in \ipa{ɣɤ-} are very few examples (none from loanwords) and their meaning is never causative-like unlike denominal \ipa{sɯ(ɣ)- / ɕɯ(ɣ)-} (from which the sibilant causative originates), which shows pairs such as \forme{(tɯ)-ɕtʂi}{sweat (n)} $\rightarrow$ \forme{sɯ-ɕtʂi}{to cause to sweat}. 

Thus, Rgyalrong-internal evidence does not suggest a denominal origin for the labial causative  prefix.

 \section{Conclusion}
Given our imperfect understanding of sound laws in Trans-Himalayan, this research is inevitably of a preliminary nature. I argued that the labial stop causative prefixes found in languages of North-Eastern India are unlikely to be parallel developments in every one of these highly diverse branches, and that a historical relationship with labial causative prefixes in Rgyalrongic languages should be considered. Evidence from other languages is more difficult to interpret, in particular in Tibetan, due to the confusion with the past \ipa{b-} prefix in many paradigms. Although denominal prefixes are attested in some Trans-Himalayan languages, the labial causative prefixes do not appear to originate from  them, except possibly in Karbi.

If the hypothesis that a labial causative prefix does go back to proto-Trans-Himalayan is valid, indirect evidence should be found in less conservative languages where prefixes are only recoverable through reconstruction. In Tibetan, detailed philological studies of verb paradigms should be undertaken to evaluate whether genuine examples of \ipa{b-} causatives exist. 
 
 \phon
\bibliographystyle{unified}
\bibliography{bibliogj}
\end{document}

