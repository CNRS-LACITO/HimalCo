\documentclass[oneside,a4paper,11pt]{article} 
\usepackage{fontspec}
\usepackage{natbib}
\usepackage{booktabs}
\usepackage{xltxtra} 
\usepackage{polyglossia} 
\usepackage[table]{xcolor}
\usepackage{gb4e} 
\usepackage{multicol}
\usepackage{graphicx}
\usepackage{float}
\usepackage{tangutex2} 
\usepackage{tangutex4}
\usepackage{hyperref} 
\hypersetup{bookmarks=false,bookmarksnumbered,bookmarksopenlevel=5,bookmarksdepth=5,xetex,colorlinks=true,linkcolor=blue,citecolor=blue}
\usepackage[all]{hypcap}
\usepackage{memhfixc}
\usepackage{lscape}
\usepackage{amssymb}
\bibpunct[: ]{(}{)}{,}{a}{}{,}

%\setmainfont[Mapping=tex-text,Numbers=OldStyle,Ligatures=Common]{Charis SIL} 
\newfontfamily\phon[Mapping=tex-text,Scale=MatchLowercase]{Charis SIL} 
\newcommand{\ipa}[1]{\textbf{{\phon\mbox{#1}}}} %API tjs en italique

\newcommand{\grise}[1]{\cellcolor{lightgray}\textbf{#1}}
\newfontfamily\cn[Mapping=tex-text,Ligatures=Common,Scale=MatchUppercase]{SimSun}%pour le chinois
\newcommand{\zh}[1]{{\cn{#1}}}

\newcommand{\sg}{\textsc{sg}}
\newcommand{\pl}{\textsc{pl}}
\newcommand{\ro}{$\Sigma$}
\newcommand{\ra}{$\Sigma_1$} 
\newcommand{\rc}{$\Sigma_3$}  
\newcommand{\dhatu}[2]{|\ipa{#1}| `#2'}
\newcommand{\bodo}[2]{\ipa{#1} `#2'}
\newcommand{\dhat}[1]{|\ipa{#1}|}
\newcommand{\change}[2]{*\ipa{#1} $\rightarrow$ \ipa{#2}}
\newcommand{\tgf}[1]{\mo{#1}}
\newcommand{\refb}[1]{(\ref{#1})}
\XeTeXlinebreakskip = 0pt plus 1pt %
 %CIRCG
 


\begin{document}

\title{The labial causative in Trans-Himalayan}
\author{Guillaume Jacques}
\maketitle
\sloppy

\section{Introduction}


 

\section{North-East India}

\subsection{Bodo-Garo}

\begin{table}[H]
\caption{Example of \ipa{phV-} causative in Bodo, \citet[54]{delancey15adjectival}}
\begin{tabular}{lllllll}
\toprule
Stative verb & Adjective & Causative \\
\midrule
\bodo{dér}{be, get big} &\bodo{ge-der}{big} & \bodo{phe-der}{make big} \\
\bodo{sung}{be, get short} & \bodo{go-sung}{short} & \bodo{pho-sung}{shorten} \\
\bodo{zam}{become old (of things)} & \bodo{gw-zam}{old, worn out} & \bodo{phw-zam}{wear out} \\
\bottomrule
\end{tabular}
\end{table}
 

Bodo \ipa{pʰV-},  \citealt{delancey15adjectival},
\subsection{Karbi}
 \citealt{konnerth15cisloc})  
Karbi \ipa{pa-} 
\subsection{Kuki-Chin}
\subsection{Historical perspective}
Previous authors have interpreted the presence of a labial stop causative prefix in various Trans-Himalayan languages of North-Eastern India as  grammaticalization from the root of the verb `to give'\footnote{The verb in question is attested throughout the family, cf Japhug \ipa{mbi} `give', Tibetan \ipa{sbʲin} `give' etc.}
(\citealt[132]{matisoff03}) or as borrowing from some Austroasiatic language (\citealt{konnerth15cisloc, delancey15adjectival}). Both hypotheses are problematic. 

First, since all languages in question are strictly verb-final, grammaticalization of a causative construction with the verb `to give' into a causative affix would be expected to yield a prefix; while exceptions to this well-known tendency are attested (see in particular \citealt{jacques13harmonization}), the hypothesis that unrelated branches of Trans-Himalayan such as Bodo-Garo, Karbi, Angami-Pochuri and Kuki-Chin would undergo such an unusual grammaticalization process independently is untenable. 

Second, borrowing of derivational morphology is attested, but only occurs in cases of extreme contact situations involving heavy lexical borrowing. Since Austroasiatic lexical influence on Trans-Himalayan languages of North-Eastern India is marginal, it is unlikely that the labial stop causative prefix in these languages was borrowed.

Both hypotheses depart from the common assumption that the labial stop causative prefix is an innovation. Yet, labial causative prefixes are found in languages outside of North-Eastern India. 

\section{Macro-Rgyalrongic}


\subsection{Rgyalrong}
All Rgyalrongic languages have two causative prefixes, a sibilant prefix\footnote{In all Rgyalrongic languages, the sibilant causative prefix presents several regular as well as irregular allomorphs, see for instance \citealt{jackson07shangzhai} on Stodsde and \citealt{jacques15causative} on Japhug.} and a labial prefix whose form goes back to proto-Rgyalrong *\ipa{wɐ-} (Japhug \ipa{ɣɤ-}, Tshobdun \ipa{wɐ-}, Khroskyabs \ipa{v-}, see \citealt[322]{jacques04these}, \citealt{jackson06paisheng} and \citealt[136]{lai13affixale}).\footnote{The labial causative only has limited allomorphy. In Japhug a few verbs with nasal initial have a causative allomorph in \ipa{m-}, for instance \ipa{mɲo} ($\leftarrow$*\ipa{w-ɲaŋ}) `prepare' from \ipa{ɲo} `be ready, be prepared' (I owe this observation to Gong Xun, p.c.).}

The labial causative is only used to derive stative verbs, including adjectives. It can be considered to be  productive at least in Japhug, since it can be applied to adjectives of Tibetan origin, such as \ipa{dɤn} `be many' (from \ipa{ldan} `having X') $\rightarrow$ \ipa{ɣɤdɤn} `increase the number of'.

Some stative verbs take the sibilant causative (for instance adjectives of colour, \citealt[183]{jacques15causative}), and quite a few stative verbs can have both sibilant and labial causatives. In Tshobdun, \citet{jackson06paisheng, jackson14morpho} reports a difference of meaning between the two, as illustrated by examples \refb{ex:sEGchi} and \refb{ex:wAchi}.

\begin{exe}
\ex \label{ex:sEGchi}
\gll \ipa{cʰɐ́ɟi}	\ipa{ne-kɐ-səɣ-cʰiʔ=nəʔ}	\ipa{mimʔ=cə} \\
beer \textsc{ipfv-genr-caus}-be.sweet=\textsc{subord} be.tasty=\textsc{mediative} \\
\glt `Beer is tasty when one allows it to sweeten (naturally and gradually).'
\end{exe}

 \begin{exe}
\ex \label{ex:wAchi}
\gll \ipa{cʰɐ́ɟi}	\ipa{ne-kɐ-wɐ-cʰiʔ=nəʔ}	\ipa{mimʔ=cə} \\
beer \textsc{ipfv-genr-caus}-be.sweet=\textsc{subord} be.tasty=\textsc{mediative} \\
\glt `Beer is tasty when one sweetens it (e.g. by adding sugar).'
\end{exe}

As shown by this minimal pair, in Tshobdun the sibilant causative \ipa{sə(ɣ)-} implies `an increase in the degree  of the predicated state' (eg, `make something sweeter'), while the labial causative \ipa{wɐ-} is used to express the `causation of a changed state' (eg, `make something sweet').

Although a cognate pair exists in Japhug \ipa{sɯxcʰi} vs \ipa{ɣɤcʰi}, no semantic contrast could be ascertained with my Japhug consultants. The only minimal pair in Japhug with an observable meaning is that between \ipa{sɯ-mto} `cause to see, show' and \ipa{ɣɤ-mto} `cause to recover sight (of the eyes of a blind man), whose use can be illustrated by example \refb{ex:GAmto}, which derive from the labile verb from \ipa{mto} `see' (meaning `have clear sight (of eyes)' when used as an intransitive stative verb).

\begin{exe}
\ex \label{ex:GAmto}
\gll 
\ipa{nɯnɯ} 	\ipa{ɕquwa} 	\ipa{nɯra} 	\ipa{nɯ-mɲaʁ} 	\ipa{ɯ-taʁ} 	\ipa{nɯtɕu,} 	\ipa{si} 	\ipa{kɯ-xtɕi} 	\ipa{ɣɯ} 	\ipa{ɯ-jwaʁ} 	\ipa{nɯnɯ} 	\ipa{a-pɯ-tu} 	\ipa{tɕe,} 	\ipa{nɯnɯ} 	\ipa{kɯ} 	\ipa{maka} 	\ipa{nɯ-mɲaʁ} 	\ipa{tu-ɣɤ-mtɤm} 	\ipa{cʰa} 	\ipa{ri} \\
\textsc{dem} blind.person \textsc{dem:pl} \textsc{3pl.poss}-eye \textsc{3sg}-on \textsc{dem:loc} tree \textsc{nmlz}:S/A-be.small \textsc{gen} \textsc{3sg.poss}-leave \textsc{dem} \textsc{irr-ipfv}-exist \textsc{lnk} \textsc{dem} \textsc{erg} at.all \textsc{3pl.poss}-eye \textsc{ipfv-caus}-have.sight[III] can:\textsc{fact} but \\
\glt `If one (puts) leaves of the small trees on the eyes of these blind people, it can cause their eye to recover sight. (hist140517 mogui de jing, 82-4)
\end{exe}

While it is yet unclear whether the subtle semantic contrast observed in Tshobdun is an archaism or an innovation of this language, it is clear that a labial causative *\ipa{wɐ-} of stative verbs has to be reconstructed to proto-Rgyalrongic. 

\subsection{Tangut}
Evidence for a labial causative is also found in Tangut as an \ipa{-w-} infix  which can originate from a labial stop prefix by application of regular sound changes (\citealt[253-4]{jacques14esquisse}).

Table \ref{tab:prefixep}  provides the clearest examples (from \citealt[45-6]{gong88alternations}) of causative \ipa{-w-} in Tangut.\footnote{Note that the character \tgf{0735} \ipa{dʑjwij¹} `make cold' is incorrectly reconstructed as $\dagger$\ipa{dʑjij¹} in \citet[144]{lifw97}; that a \ipa{-w-} medial must be restaured is proven by the \textit{fanqie} \tgf{2397}\tgf{5237} \ipa{dʑjwi²}\ipa{tśjwij¹} in the Wenhai dictionary, both of whose characters have  \ipa{-w-} medial.}

\begin{table}[H]
\caption{The causative -w- infix in Tangut }\label{tab:prefixep} 
\resizebox{\columnwidth}{!}{
\begin{tabular}{lllllllllll} 
\toprule
\multicolumn{4}{c}{base verb} &\multicolumn{4}{c}{causative form} & \\
\midrule
\tgf{3259} &\ipa{dzji} & 1.11 & be calm & \tgf{3411} &\ipa{dzjwi} & 1.11 & make calm \\
\tgf{1829} &\ipa{tshja} & 1.20 & hot, burn & \tgf{1825} &\ipa{tshjwa} & 1.20 & roast \\
\tgf{4033} &\ipa{dʑjij} & 2.32 & be cold & \tgf{0735} &\ipa{dʑjwij} & 1.35 & make cold \\
%\tgf{4186} &\ipa{khej} & 1.33 & lush & \tgf{1231} &\ipa{khwej} & 1.33 & develop, open (field)  \\
 \bottomrule
\end{tabular}}
\end{table}

 Although text examples of these pairs have not yet been identified, the definitions in the Wenhai monolingual dictionary (though not in the Chinese and English glosses in \citealt{lifw97}) clearly indicate that these verbs are causative forms, as their definitions include a stative verb followed by the causative \tgf{0749} \ipa{phji¹}, as shown below:

\begin{itemize}
\item  \tgf{3411}  \ipa{dzjwi¹}: \tgf{2518}\tgf{3098}\tgf{0749} \ipa{njiij¹djɨj²phji¹} `cause to the mind to settle'
\item \tgf{1825} \ipa{tshjwa¹}: \tgf{1829}\tgf{0749} \ipa{tshja¹phji¹} `cause to become hot'
\item  \tgf{0735} \ipa{dʑjwij¹}:  \tgf{4052}\tgf{0749} \ipa{dạ²phji¹} `cause to become cold'
\end{itemize}

As with the previous examples from Bodo-Garo and Rgyalrongic, the labial causative in Tangut is restricted to causativization of stative verbs.

\subsection{Historical perspective}

The labial causative prefixes found in Macro-Rgyalrongic presents one important commonality with Bodo-Garo: they are specifically used to derive causative forms of adjectives and other stative verbs.

On the other hand, they differ from the causative prefixes found in languages of North-Eastern India by having labial approximants (or segments regularly originating from labial approximants such as \ipa{ɣ} in Japhug). 

Although this phonological difference could appear to an insuperable obstacle to comparison between the labial causative prefixes of Bodo-Garo and Rgyalrongic, we have to take into account several facts. 

First, there is a recurrent correspondence of *\ipa{w} in Rgyalrongic languages in some words with labial stops in other languages, as exemplified by the verb `come' (Japhug \ipa{ɣi} `come', Tshobdun \ipa{wi} `come', \citealt[237]{sun14generic}, Khroskyabs \ipa{vjî}, \citealt[276]{lai15person}), whose cognates in Bodo-Garo and Kuki-Chin have labial stops (Bodo \ipa{pay} `come', Proto-Kuki-Chin *\ipa{paay} `go' \citealt{vanbik09pkc}). The exact conditioning for the lenition of stops in Rgyalrongic is extremely complex and imperfectly understood.

Second, the appear to be a phonological constraint on the shape of derivational prefixes in Rgyalrong languages: despite the richness of derivational prefixes in Rgyalrong languages (see \citealt{jackson14morpho, jacques14antipassive}), out of 50 consonantal phonemes, only nine can occur in derivational prefixes in Japhug (\ipa{m}, \ipa{n}, \ipa{r}, \ipa{j}, \ipa{ɣ}, \ipa{s}, \ipa{z}, \ipa{ɕ} and \ipa{ʑ}): some sonorants and fricatives (all continuant consonants). Stops prefixes are found in person indexation, TAM and participle prefixes, all outside of the verb stem. It is thus possible to propose that labial stops were lenited to *\ipa{w} in proto-Rgyalrong in derivational prefixes XXX

 \section{Conclusion}
 
 
 
\bibliographystyle{unified}
\bibliography{bibliogj}
\end{document}

