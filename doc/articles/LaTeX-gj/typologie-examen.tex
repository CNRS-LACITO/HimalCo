\documentclass[oldfontcommands,twoside,a4paper,12pt]{article} 
\usepackage{fontspec}
\usepackage{natbib}
\usepackage{booktabs}
\usepackage{xltxtra} 
\usepackage{polyglossia} 
\usepackage[table]{xcolor}
\usepackage{color}
\usepackage{multirow}
\usepackage{gb4e} 
\usepackage{multicol}
\usepackage{graphicx}
\usepackage{float}
\usepackage{hyperref} 
\hypersetup{bookmarks=false,bookmarksnumbered,bookmarksopenlevel=5,bookmarksdepth=5,xetex,colorlinks=true,linkcolor=blue,citecolor=blue}
\usepackage{memhfixc}
\usepackage{lscape}
\usepackage[footnotesize,bf]{caption}
\usepackage{multicol}
 
 
\setmainfont[Mapping=tex-text,Numbers=OldStyle,Ligatures=Common]{Charis SIL} 
\setsansfont[Mapping=tex-text,Ligatures=Common,Mapping=tex-text,Ligatures=Common,Scale=MatchLowercase]{Lucida Sans Unicode} 
 


\newfontfamily\phon[Mapping=tex-text,Ligatures=Common,Scale=MatchLowercase,FakeSlant=0.3]{Charis SIL} 
\newcommand{\ipa}[1]{{\phon\textbf{#1}}} 

\begin{document}
\title{Typologie : examen du 09 janvier 2015}
\date{}
\maketitle
\section*{Question de cours}
Expliquez la différence entre anticausatif et passif à partir d'exemples tirés de langues que vous connaissez.

\section*{Exercice 1}
En vous servant des phrases ci-dessous:
\begin{enumerate}
 \item Décrivez le marquage dans cette langue.
 \item Décrivez l'indexation dans cette langue.
 \item Traduisez ``Il me frappe'', ``Tu me frappes'', ``L'oiseau court''.
 \item Segmentez et glosez les exemples.
\end{enumerate}
\vspace{0.25cm}

 
\begin{exe}
\ex
\glt \ipa{Bayramdi} \ipa{Murad} \ipa{urččwura.} 
\glt Bayram frappe Murad.
\end{exe}

\begin{exe}
\ex
\glt \ipa{Bayramdi} \ipa{žaqw'} \ipa{ubččwura.} 
\glt Bayram frappe l'oiseau.
\end{exe}

\begin{exe}
\ex
\glt \ipa{Uzu} \ipa{urččwuraza.}
\glt Je le frappe.
\end{exe}

\begin{exe}
\ex
\glt \ipa{Uzu} \ipa{uvu} \ipa{urččwurazavu.} 
\glt Je te frappe.
\end{exe}

\begin{exe}
\ex
\glt \ipa{Bayram} \ipa{žarğura.} 
\glt Bayram court.
\end{exe}

\begin{exe}
\ex
\glt \ipa{Uzu} \ipa{žarğuraza.} 
\glt Je cours.
\end{exe}

\begin{exe}
\ex
\glt \ipa{Uzu} \ipa{aldarkurazu.}  
\glt Je tombe.
\end{exe}

\begin{exe}
\ex
\glt \ipa{žaqw'} \ipa{aldabkurazu.} 
\glt L'oiseau tombe.
\end{exe}
 

\section*{Exercice 2}
\begin{enumerate}
\item Placez les formes verbales des exemples ci-dessous dans le tableau du paradigme transitif.
\item Décrivez l'indexation dans cette langue.
 \item Segmentez et glosez les exemples \ref{ex:obv} et \ref{ex:antipass} en justifiant vos décisions.
\item (Question bonus) Proposez une traduction alternative en ojibwe des phrases ``Elle poursuit le chien'' et ``Le chien la poursuit'' et commentez.
\end{enumerate}
\begin{multicols}{2}
\begin{exe}
\ex \label{ex:obv}
\glt animoshan obiminizha'ogoon
\glt Le chien la poursuit.
\end{exe} 

\begin{exe}
\ex 
\glt animoshan obiminizha'waan 
\glt Elle poursuit le chien.
\end{exe} 
 
\begin{exe} 
\ex 
\glt imbiminizha'og animosh.
\glt Le chien me poursuit.
\end{exe}

\begin{exe}
\ex 
\glt imbiminizha'waa
\glt Je le poursuis
\end{exe}

\begin{exe}
\ex 
\glt imbiminizha'ogoog.
\glt Ils me poursuivent.
\end{exe}

\begin{exe}
\ex 
\glt Geget ina go gibiminizha'ogoog?
\glt Ils te poursuivent vraiment?
\end{exe}

\begin{exe}
\ex \label{ex:antipass}
\glt imbiminizha'ige
\glt Je poursuis des gens.
\end{exe}

 
\end{multicols}


\end{document}

