\documentclass[oldfontcommands,oneside,a4paper,11pt]{memoir} 
\usepackage{xunicode}%packages de base pour utiliser xetex
\usepackage{fontspec}
\usepackage{natbib}
\usepackage{booktabs}
\usepackage{xltxtra} 
\usepackage{longtable}
\usepackage{tangutex2} 
\usepackage{polyglossia} 
\setdefaultlanguage{french} 
\usepackage{multicol}
\usepackage{gb4e} 
\usepackage{graphicx}
\usepackage{float}
\usepackage{hyperref} 
\hypersetup{bookmarks=false,bookmarksnumbered,bookmarksopenlevel=5,bookmarksdepth=5,xetex,colorlinks=true,linkcolor=blue,citecolor=blue}
\usepackage[all]{hypcap}
\usepackage{memhfixc}

%%%%%%%%%quelques options de style%%%%%%%%
\chapterstyle{veelo}
\nouppercaseheads
\pagestyle{Ruled}
\setsecheadstyle{\SingleSpacing\LARGE\scshape\raggedright\MakeLowercase}
\setsubsecheadstyle{\SingleSpacing\Large\itshape\raggedright}
\setsubsubsecheadstyle{\SingleSpacing\itshape\raggedright}
\setsecnumdepth{subsubsection}
%%%%%%%%%%%%%%%%%%%%%%%%%%%%%%%
\setmainfont[Mapping=tex-text,Numbers=OldStyle,Ligatures=Common]{Charis SIL} %ici on définit la police par défaut du texte
\setsansfont[Mapping=tex-text,Ligatures=Common,Mapping=tex-text,Ligatures=Common,Scale=MatchLowercase]{Lucida Sans Unicode:language='FRA'} %définition de la police sans sérif
\setmonofont[Mapping=tex-text,Scale=MatchLowercase]{Courier New:language='FRA'} %définition de la police monospace
\SetSymbolFont{letters}{normal}{\encodingdefault}{\rmdefault}{m}{rm}
\setmathrm{MinionPro-Regular}

\newfontfamily\phon[Mapping=tex-text,Ligatures=Common,Scale=MatchLowercase,FakeSlant=0.3]{Charis SIL} 
\newfontfamily\phondroit[Mapping=tex-text,Ligatures=Common,Scale=MatchLowercase]{Doulos SIL} 
\newcommand{\ipa}[1]{{\phon #1}} %API tjs en italique
\newcommand{\ipapl}[1]{{\phondroit #1}} 
\newfontfamily\cn[Mapping=tex-text,Ligatures=Common,Scale=MatchUppercase]{MingLiU}%pour le chinois
\newcommand{\zh}[1]{{\cn #1}}
\newcommand{\py}[2]{\ipa{#1} \zh{#2} \index{#1 \zh{#2}}}

\title{Histoire Tangoute} 
\author{Guillaume Jacques }
\date{\today}
\makeindex
\begin{document}


%āáǎà
%īíǐì
%ēéěè
%ōóǒò
%ūúǔù
%ǖǘǚǜ
\chapter{Suíshū 83} 

Les  \ipapl{Dǎngxiàng} descendent des trois \ipapl{Miáo}. Leur race est apparentée à celle des \ipapl{Dàngchāng} et des \ipapl{Báiláng} ; ils disent tous être de la race des singes. Leur territoire s'étend de Líntáo\footnote{Dans l'actuel Gānsù} et Xīpíng\footnote{Actuelle Xīníng au Qīnghǎi.} à l'est jusqu'à celui du \py{Yabgu}{葉護}\footnote{L'empire des Göktürk.} à l'ouest, il recouvre plusieurs milliers de lieues du nord au sud. Il se situe dans les montagnes et les vallées. 

A chaque nom de famille correspond une tribu, les plus grandes comptent plus de cinq milles cavaliers, les plus petites seulement un peu plus de mille cavaliers. Ils tressent les poils de queue de yak et de mouton noir pour  fabriquer leurs maisons. Ils s'habillent de manteaux de cuir et portent du feutre comme décoration. Par leurs coutumes, ils respectent la force militaire, et n'ont pas de loi. Ils vivent séparément, mais se rassemblent en cas de guerre. Ils n'ont ni corvées ni impôts, et ont peu d'interactions les uns avec les autres. Il élèvent et font paître des bovins, des ovins et des cochons pour se nourrir, et ignorent l'agriculture. Leurs mœurs sont répugnantes même pour des barbares: ils pratiquent l'inceste avec des membres de la génération supérieure. Ils n'ont pas d'écriture, et se contentent d'employer des herbes ou des bouts de bois pour compter les années. Ils se réunissent tous les trois ans, et sacrifient des bovins et des ovins aux dieux du ciel.

Lorsqu'un homme a plus de quatre-vingt ans lorsqu'il meurt, ils considèrent que c'est une belle mort, et les parents du défunt ne pleurent pas. En revanche, si un jeune meurt, ils estiment que c'est là une grande injustice, et pleurent son décès tous ensemble avec tristesse. Ils ont des guitares pípá, des flûtes horizontales, et frappent sur des jarres pour accompagner le rythme. 

Entre Les Wèi du nord (386-534) et les Zhōu du nord (557–581), ils venaient en nombre important causer des troubles à la frontière. Lorsque Gāozǔ\footnote{Yáng Jiān, fondateur de la dynastie Suí} servait comme premier ministre (sous la dynastie Zhōu du nord),  ils y avait de nombreux troubles dans la plaine centrale, et ils en ont profité pour commettre des pillages. Liáng Ruì, le seigneur de Jiǎng, avait alors déjà pacifié Wáng Qiān\footnote{ (En 580.)} , et il demanda la permission de revenir avec son armée pour les attaquer, mais Gāozǔ ne le lui permit pas. En 584, plus de mille familles se soumirent spontanément. 

En 585, Tuòbá Níngcóng conduisit son peuple vers Xùzhōu\footnote{Dans l'actuel Gānsù.} pour demander son intégration à l'Empire, et reçu la charge de grand général. Ses subordonnés obtinrent également diverses fonctions. 

En 596, ils pillèrent à nouveau Huìzhōu\footnote{Dans l'actuel Sìchuān occidental, département Qiang de Màoxiàn}, et l'on ordonna d'envoyer des soldats de Lǒngxī\footnote{Actuel Gānsù.} pour les attaquer. Leur armée fut vaincue. Il vinrent donner leur capitulation tribu après tribu, acceptèrent de devenir sujets de l'Empire et envoyèrent enfants et jeunes frères à la capitale comme otage en signe d'excuse. 


Gāozǔ leur dit : ``Retournez dire à vos parents et aînés qu'un homme pour vivre a besoin de s'installer dans une résidence fixe afin de subvenir aux besoins des personnes âgées et permettre aux jeunes de grandir. Vous, au contraire, allez et venez en permanence, n'avez-vous pas honte ?''. Suite à cela, ils vinrent apporter leur tribut les uns après les autres. >>


\chapter{Traduction du Sòngshū 491}


Les tangoutes \zh{党項}, qui se trouvent là ou étaient les \py{Xīzhǐ}{析支}, sont une race des Qiāng occidentaux de l'époque Han. Il ont commencé à devenir puissants à partir des Zhōu postérieurs (951-960), et l'on compte les tribus \py{Xìfēng}{細風}, \py{Fèitīng}{費聽}, \py{Wǎnglì}{往利}, \py{Pōchāo}{頗超},  \py{Yěluàn,}{野亂}, \py{Fángdāng}{房當}, \py{Láiqín}{來禽} et \py{Tuòbá}{拓跋}, la plus puissante d'entre elles. A l'époque Táng, de l'ère \py{Zhēnguān}{貞觀} (627-650) à l'ère \py{Shàngyuán}{上元} (674-676), ils étaient soumis aux terres de l'intérieur\footnote{les régions chinoises}, et étaient dispersés dans les terres proches de la frontières du nord-ouest. A partir de l'ère \py{Yuánhé}{元和} (806-821), ils se mirent à venir piller en grand nombre par vagues successives. Au début de l'ère \py{Huìchāng}{會昌} (841-847), l'empereur \py{Wǔzōng}{武宗} nomma trois envoyés pour les diriger, l'un dans les régions de \py{Bīn}{邠}, de \py{Níng}{寧} et de \py{Yán}{延}, l'autre dans celles de \py{Yán}{鹽}, de \py{Xià}{夏} et de \py{Chángzé}{長澤} et le dernier dans celles de \py{Língwǔ}{靈武}, de  \py{Lín}{麟} et de \py{Shèng}{勝}. Durant les cinq dynasties, il sont venus apporter le tribut. A notre époque, les régions de \py{Língzhōu}{靈州}, \py{Xiàzhōu}{夏州}, \py{Suízhōu}{绥州}, \py{Línzhōu}{麟州}, \py{Fǔzhōu}{府州}, \py{Huánzhōu}{環州}, \py{Qìngzhōu}{慶州}, \py{Fēngzhōu}{豐州}, \py{Zhènróng}{鎮戎}, \py{Tiāndé}{天德}, \py{Zhènwǔjūn}{振武軍} sont peuplées par ce peuple.

Durant la deuxième année de l'ère \py{Jiànlóng}{建隆} de \py{Tàizǔ}{太祖} (961), le \py{cìshǐ}{刺史} de \py{Dàizhóu}{代州}, \py{Zhé Miēmái}{折乜埋} vint à la cour. \py{Miēmái}{乜埋} est une grande famille tangoute, qui depuis des générations habite à \py{Héyòu}{河右}; comme il s'était illustré dans la défense de la frontière, on lui donna une charge administrative dans la région de \py{Fāngzhōu}{方州}, l'invita en audience à la cour, puis le renvoya.

La première année de l'ère \py{Kāibǎo}{開寶} (968), \py{Chuài Jí}{啜佶} le chef de la tribu \py{Zhídàng}{直蕩} et d'autres emmenèrent les gens de \py{Bìng}{并} pour piller \py{Fǔzhōu}{府州}, mais ils furent vaincus par l'armée impériale. On ordonna que les tribus Qiāng soumises aux régions intérieures de \py{Qū Yù}{屈遇}, chef de seize départements et \py{Luó Yá}{羅崖}, chef de douze départements, attaquent \py{Chuài Jí}{啜佶}. Ce dernier prit peur, et se rendit avec sa tribu. 

\py{Qū Yù}{屈遇} fut nommé \py{jiǎnjiào tàibǎo}{檢校太保} et \py{guīshùn jiāngjūn}{歸順將軍}. \py{Luó Yá}{羅崖} et \py{Chuài Jí}{啜佶} furent nommés \py{jiǎnjiào sītú}{檢校司徒} et \py{huáihuà jiāngjūn}{懷化將軍}

Au deuxième mois de la deuxième année de l'ère \py{Tàipíng Xīngguó}{太平興國} (977), le gouvernement de \py{Língzhōu}{靈州} envoya (vers la capitale) son tribut annuel de chevaux destinés à la vente dans les marchés étatiques. (Les émissaires chargés du transport des chevaux), passant par les régions habitées par des tribus Qiang, ne leur offrirent que des cadeaux de mauvaise qualité. Les Qiang, furieux, les refusèrent. Le \py{zhīzhōu}{知州} et \py{bǐbùlángzhōng}{比部郎中} \py{Zhāng Quáncāo}{張全操} fit arrêter dix-huit (des rebelles) et les exécuta. Il réquisitionna leurs armes, leurs moutons et leurs chevaux. Suite à cela, les barbares se rebellèrent. La rébellion  ne fut apaisée que lorsque l'empereur manda un émissaire pour leur offrir des présents d'or et de soie et établir une alliance avec eux.

On convoqua  Quáncāo pour être interrogé par le \py{yǒusī}{有司}, et il fut condamné à recevoir de coups de bâtons et à être exilé à l'île de \py{Shāmén}{沙門} à \py{Dēngzhōu}{登州}. La même année, à la frontière entre \py{Língzhōu}{靈州} et \py{Tōngyuǎnjūn}{通遠軍}, les tribus de \py{Sǎngmiē}{嗓咩 }, 
\py{Zhésì}{折四},
\py{Tǔfāncūn}{吐蕃村}, 
\py{Nàiwāi sānjiā}{柰㖞三家} 
\py{Wěiluò}{尾落} 
\py{Nàijiā}{柰家}
\py{Sǎngní}{嗓泥}
vinrent piller l'office chargé du transports des marchandises (\py{guāngāng}{官綱}). Il fut ordonné à 
\py{Ān Shǒuzhōng}{安守忠} de \py{Língzhōu}{靈州}	 et à \py{Dǒng Zūnhuì}{董遵誨} de \py{Tōngyuǎnjūn}{通遠軍} de les pacifier. 


Lors de la sixième année (981), \py{Láidū}{來都} , chef de la tribu \py{Wàilàng}{外浪} de \py{Fǔzhōu}{府州} et d'autres virent pour offrir des chevaux en tribut.

Lors de la septième année (982), le chef de \py{Fēngzhōu}{豐州}, \py{Huángluó}{黃羅}, accompagné de son frère \py{Qǐbàng}{乞蚌} et d'autres, vinrent offrir des chevaux en tribut. \py{Tuòbá Yù}{拓跋遇}, des tributs Qiang de \py{Yínzhōu}{銀州}, vint se plaindre que les impôts et les corvées dans cette région étaient excessifs, et demanda de pouvoir s'installer dans les régions intérieures. Il lui fut ordonné de rester dans le territoire de sa tribut.  \py{Bǎoxì}{保細}	tenta d'établir des alliances avec les autres tributs Qiang et de les inciter à la révolte, mais \py{Liáng Jiǒng}{梁迥}, le \py{xúnjiǎnshǐ}{巡檢使}	de \py{Xiàzhōu}{夏州} les pacifia avec ses troupes.

Au début de l'ère \py{Yōngxī}{雍熙} (984), les chef de tributs Qiang se révoltèrent à la suite de \py{Lǐ Jìqiān}{李繼遷}. Il fut ordonné au juge du \py{sìfāngguǎn}{四方館}, \py{Tiān Rénlǎng}{田仁朗}, au \py{géménshǐ}{閣門使} \py{Wáng Shēn}{王侁} l'un après l'autre de lancer des expéditions punitives. Des édits impériaux furent transmis aux régions de \py{Línzhōu}{麟州}, de \py{Fǔzhōu}{府州}, de \py{Yínzhōu}{銀州}, de \py{Xiàzhōu}{夏州}, de \py{Fēngzhōu}{豐州} ainsi qu'au tribus de \py{Rìlì}{日利} et \py{Yuèlì}{月利} pour leur octroyer une amnistie.

Au quatrième mois de la deuxième année (985), \py{Wáng Shēn}{王侁} vainquit \py{Xīlì}{悉利} et d'autres tribus au nord de \py{Yínzhōu}{銀州}. Il fit décapiter plus de 3600 personnes, fit prisonniers 80 personnes, captura plus de 1400 enfants et personnes âgées, et saisit 186 armures et armes. 
Il exécuta le faux \py{cìshǐ}{刺史} de \py{Dàizhōu}{代州} \py{Zhéluóyù}{折羅遇} et son frère \py{Màiqǐ}{埋乞}, et confisqua en tout 30000 chevaux, bœufs et moutons. 

Le cinquième mois, à l'est de \py{Kāiguāngǔ}{開光谷},  \py{Xìng Zǐpíng}{杏子平} vainquit les tribus de \py{Bǎosì}{保寺} et \py{Bǎoxiāng}{保香}, les poursuivit sur plus de 20 lieues, fit décapiter plus de 80 d'entre eux, et exécuta leur chef \py{Máimiēsí}{埋乜已} et 57 autres personnes, fit prisonniers 49 d'entre eux, captura plus de 300 vieillards et enfants, et confisqua plus de 4000 bœufs, moutons, chevaux et ânes. Il vainquit aussi les tribus de  \py{Bǎo}{保} et \py{Xǐ}{洗}, fit prisonniers 3000 personnes, soumit 55 tribus, et confisqua en tout 8000 bœufs et moutons.

\py{Wáng Shēn}{王侁}  et les autres rapportèrent que plus de 2000 familles de Qiang de \py{Línzhōu}{麟州} et \py{Sānzúzhài}{三族寨} s'étaient rendus. Le chef \py{Zhéyùmiē}{折御乜} et 64 autres personnes se livrèrent aux autorités, auxquelles ils offrirent des chevaux. Pour s'amender de leurs 
%侁等又言,麟州及三族砦羌人二千餘戶皆降,酋長折御乜等六十四人獻馬首罪,願改圖自效,爲國討賊,遂與部下兵入濁輪川,斬賊首五十級、酋豪二十人,李繼遷及三族砦監押折御乜皆遁去。旋命內客省使郭守文自三交乘驛亟往,與王侁等同領邊事。五月,王侁、李繼隆等又破銀州杏子平東北山谷內沒邵、浪悉訛等族,及濁輪川東、兔頭川西諸族,生擒七十八人,梟五十九人,俘二百三十六口,牛羊驢馬千二百六十,招降千四百五十二戶。

\chapter{Traduction du Sòngshū 486}
\py{Lǐ Bǐngcháng}{李秉常}, fils aîné de Yìzōng, avait pour mère l'impératrice \py{Gōngsùzhāngxiàn}{恭肅章憲} de la famille \py{Liáng}{梁}. Il monta sur le trône l'hiver de la quatrième année de l'ère Zhiping (1067), alors qu'il avait sept ans, et l'impératrice Liáng devint régente.

Durant le troisième mois de la première année de l'ère \py{Xīníng}{熙寧} (1068), (la cour tangoute) envoya \py{Xuē Zōngdào}{薛宗道}, le \py{zhuǎnyùnshǐ}{轉運使} nouvellement nommé du Héběi et \py{xíngbù lángzhōng}{刑部郎中} à la cour (de Song) pour annoncer le décès (de l'empereur). L'empereur Shénzōng l'interrogea à propos du meurtre de \py{Yáng Dìng}{楊定}, et Zōngdào répondit que le meurtrier avait été arrêté et renvoyé aux autorités de Song, 




%\begin{multicols}{3}  
%\end{multicols}
\printindex
\bibliographystyle{plainnat}
\bibliography{bibliogj}
\end{document}
