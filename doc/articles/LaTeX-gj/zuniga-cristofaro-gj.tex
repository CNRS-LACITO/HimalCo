\documentclass[twoside,a4paper,11pt]{article} 
\usepackage{polyglossia}
\usepackage{natbib}
\usepackage{booktabs}
\usepackage{xltxtra} 
\usepackage{longtable}
 \usepackage{geometry}
\usepackage[usenames,dvipsnames,svgnames,table]{xcolor}
\usepackage{multirow,slashbox}
%\usepackage{gb4e} 
\usepackage{multicol}
\usepackage{graphicx}
\usepackage{float}
\usepackage{hyperref} 
\hypersetup{colorlinks=true,linkcolor=blue,citecolor=blue}
\usepackage{memhfixc}
\usepackage{lscape}
\usepackage{lineno}
\usepackage[footnotesize,bf]{caption}


%%%%%%%%%%%%%%%%%%%%%%%%%%%%%%%
\setmainfont[Mapping=tex-text,Numbers=OldStyle,Ligatures=Common]{Junicode} 
%\setsansfont[Mapping=tex-text,Ligatures=Common,Mapping=tex-text,Ligatures=Common,Scale=MatchLowercase]{Ubuntu} 
\newfontfamily\phon[Mapping=tex-text,Ligatures=Common,Scale=MatchLowercase]{Charis SIL} 
%\newfontfamily\phondroit[Mapping=tex-text,Ligatures=Common,Scale=MatchLowercase]{Doulos SIL} 
%\newfontfamily\greek[Mapping=tex-text,Scale=MatchLowercase]{Galatia SIL} 
\newcommand{\ipa}[1]{{\phon\textit{#1}}} 
\newcommand{\ipab}[1]{{\phon #1}}
\newcommand{\ipapl}[1]{{\phondroit #1}}
\newcommand{\captionft}[1]{{\captionfont #1}} 
%\newfontfamily\cn[Mapping=tex-text,Scale=MatchUppercase]{IPAGothic}%pour le chinois
%\newcommand{\zh}[1]{{\cn #1}}
\newcommand{\tgf}[1]{\mo{#1}}
%\newfontfamily\mleccha[Mapping=tex-text,Ligatures=Common,Scale=MatchLowercase]{Galatia SIL}%pour le grec

\newcommand{\sg}{\textsc{sg}}
\newcommand{\pl}{\textsc{pl}}
\newcommand{\grise}[1]{\cellcolor{lightgray}\textbf{#1}} 
\newcommand{\Σ}{\greek{Σ}}
\newcommand{\ro}{$\Sigma$}
\newcommand{\ra}{$\Sigma_1$} 
\newcommand{\rc}{$\Sigma_3$}  

\begin{document}
\linenumbers
\title{Directionality of change and   (non-)canonical direct/inverse systems \footnote{We would like to thank XXX. We are responsible for any remaining errors. This research was funded by the HimalCo project (ANR-12-CORP-0006) and is related to the research strand LR-4.11 'Automatic paradigm generation and language description' of the Labex EFL (funded by the ANR/CGI). } }

\author{Guillaume JACQUES, Anton ANTONOV\\ CNRS-INALCO-EHESS, CRLAO}
%\date{}
\maketitle
\section{Introduction}

\section{The (re)shaping of a direct-inverse system: the Plains Cree conjunct order }

\subsection{Independent vs conjunct order in modern Plains Cree}



\subsubsection{Independent order}

\begin{table}[h]
\caption{Plains Cree present paradigms. TA \ipa{wâpam--} ``see" and IA \ipa{pimipahtâ--}``run" (\citealp{wolfart73})}
\label{tab:cree.ind} \centering
\resizebox{\textwidth}{!}{
\begin{tabular}{lllllllll}
\toprule
 \backslashbox{A}{P}  & 	1\sg  & 1\textsc{pi} & 1\textsc{pe} &  2\sg & 2\pl  &  3\sg & 3\pl &	3' \\ 
\midrule
1\sg   & 	\grise{}   & 	\grise{} &  \grise{} &	\ipa{kiwâpamitin}  & \ipa{kiwâpamitinâwâw}	& \cellcolor{Dandelion}\ipa{niwâpamâw}   & 	\cellcolor{Dandelion}\ipa{niwâpamâwak}  & \cellcolor{Dandelion}	\ipa{niwâpamimâwa}   \\ 
1\textsc{pi} & \grise{}   &\grise{} & \grise{} & \multicolumn{2}{c}{\grise{}}  & \cellcolor{Dandelion}\ipa{kiwâpamânaw} & \cellcolor{Dandelion}\ipa{kiwâpamânawak}  & \cellcolor{Dandelion}	\ipa{kiwâpamimânawa}   \\ 
1\textsc{pe} & \grise{}   &\grise{} & \grise{} & \multicolumn{2}{c}{\ipa{kiwâpamitinân}}   & \cellcolor{Dandelion}\ipa{niwâpamânân} & \cellcolor{Dandelion}\ipa{niwâpamânânak}  & \cellcolor{Dandelion}	\ipa{niwâpamimânâna}   \\ 
2\sg   & 	\ipa{kiwâpamin}   & \grise{}& \multirow{2}{*}{\ipa{kiwâpaminân}}	&	\grise{}   &  \grise{} & \cellcolor{Dandelion}\ipa{kiwâpamâw}  & \cellcolor{Dandelion}\ipa{kiwâpamâwak} &\cellcolor{Dandelion} 	\ipa{kiwâpamimâwa}   \\ 
2\pl  & 	\ipa{kiwâpaminâwâw} & \grise{}& \multirow{-2}{*}{ } & \grise{}  & 	\grise{}   & 	\cellcolor{Dandelion}\ipa{kiwâpamâwâw}  & \cellcolor{Dandelion}\ipa{kiwâpamâwâwak} &\cellcolor{Dandelion} 	\ipa{kiwâpamimâwâwa}   \\
3\sg   & 	\cellcolor{green}\ipa{niwâpamik}   & \cellcolor{green}\ipa{kiwâpamikonaw} & \cellcolor{green}\ipa{niwâpamikonân} & \cellcolor{green}	\ipa{kiwâpamik}  & \cellcolor{green}	\ipa{kiwâpamikowâw} & \cellcolor{Dandelion}	\grise{}  & \grise{}	 & \cellcolor{Dandelion}	\ipa{wâpam(im)êw}   \\ 
3\pl   & 	\cellcolor{green}\ipa{niwâpamikwak}&  \cellcolor{green}\ipa{kiwâpamikonawak} & \cellcolor{green}\ipa{niwâpamikonânak}   & \cellcolor{green}	\ipa{kiwâpamikwak}   & \cellcolor{green}	\ipa{kiwâpamikowâwak} & \cellcolor{Dandelion}	\grise{} &	\grise{}  & \cellcolor{Dandelion}	\ipa{wâpam(im)êwak}   \\ 
\multirow{2}{*}{3'}   & \multirow{2}{*}{\cellcolor{green}}  &  \multirow{2}{*}{\cellcolor{green}}  & \multirow{2}{*}{\cellcolor{green}} &\cellcolor{green} &  \multirow{2}{*}{\cellcolor{green}}  &\multirow{2}{*}{\cellcolor{green}}   & \multirow{2}{*}{\cellcolor{green}} & \cellcolor{Dandelion} \ipa{wâpamêyiwa} \\ 
 \multirow{-2}{*}{} & \multirow{-2}{*}{\cellcolor{green}\ipa{niwâpamikoyiwa}} & \multirow{-2}{*}{\cellcolor{green}\ipa{kiwâpamikonawa}}   &  \multirow{-2}{*}{\cellcolor{green}\ipa{niwâpamikonâna}} &  \multirow{-2}{*}{\cellcolor{green}\ipa{kiwâpamikoyiwa}} &  \multirow{-2}{*}{\cellcolor{green}\ipa{kiwâpamikowâwa}}& \multirow{-2}{*}{\cellcolor{green}\ipa{wâpamik}}  & \multirow{-2}{*}{\cellcolor{green}\ipa{wâpamikwak}} & \cellcolor{green} \ipa{wâpamikoyiwa}  \\ 
\bottomrule
\textsc{intr} & \ipa{nipimipahtân} & \ipa{ kipimipahtâ(nâ)naw} & \ipa{nipimipahtânân} &\ipa{ kipimipahtân} &\ipa{ kipimipahtânâwâw} & \ipa{pimipahtâw} & \ipa{pimipahtâwak} & \ipa{pimipahtâyiwa} \\
\bottomrule
\end{tabular}
}
\end{table}


\subsubsection{Conjunct order}

\begin{table}[h]
\caption{Plains Cree present paradigms. TA \ipa{wâpam--} ``see" and IA \ipa{pimipahtâ--}``run" (\citealp{wolfart73})}
\label{tab:cree.conj} \centering
\resizebox{\textwidth}{!}{
\begin{tabular}{lllllllll}
\toprule
 \backslashbox{A}{P}  & 	1\sg  & 1\textsc{pi} & 1\textsc{pe} &  2\sg & 2\pl  &  3\sg & 3\pl &	3' \\ 
\midrule
1\sg   & 	\grise{}   & 	\grise{} &  \grise{} &	\cellcolor{pink}\ipa{ê-wâpam-it-ân}  & \cellcolor{pink}\ipa{ê-wâpam-it-ako-k}	& \cellcolor{Turquoise}\ipa{ê-wâpam-ak}   & 	\cellcolor{Turquoise}\ipa{ê-wâpam-ak-ik}  & \cellcolor{Turquoise}	\ipa{ê-wâpam-im-ak}   \\ 
1\textsc{pi} & \grise{}   &\grise{} & \grise{} & \multicolumn{2}{c}{\grise{}}  & \cellcolor{Dandelion}\ipa{ê-wâpam-â-yahk} & \cellcolor{Dandelion}\ipa{ê-wâpam-â-yahko-k}  & \cellcolor{Dandelion}	\ipa{ê-wâpam-im-â-yahk}   \\ 
1\textsc{pe} & \grise{}   &\grise{} & \grise{} & \multicolumn{2}{c}{\cellcolor{pink}\ipa{ê-wâpam-it-âhk}}   & \cellcolor{Dandelion}\ipa{ê-wâpam-â-yâhk} & \cellcolor{Dandelion}\ipa{ê-wâpam-â-yâhk-ik}  & \cellcolor{Dandelion}	\ipa{ê-wâpam-im-â-yâhk}   \\ 
2\sg   & 	\cellcolor{cyan}\ipa{ê-wâpam-i-yan}   & \grise{}& \multirow{2}{*}{\cellcolor{cyan}}	&	\grise{}   &  \grise{} & \cellcolor{Aquamarine}\ipa{ê-wâpam-at}  & \cellcolor{Aquamarine}\ipa{ê-wâpam-at-ik} &\cellcolor{Aquamarine}\ipa{ê-wâpam-im-at}   \\ 
2\pl  & 	\cellcolor{cyan}\ipa{ê-wâpam-i-yêk} & \grise{}& \multirow{-2}{*}{\cellcolor{cyan} \ipa{ê-wâpam-i-yâhk}} & \grise{}  & 	\grise{}   & 	\cellcolor{Dandelion}\ipa{ê-wâpam-â-yêk}  & \cellcolor{Dandelion}\ipa{ê-wâpam-â-yêko-k} &\cellcolor{Dandelion} 	\ipa{ê-wâpam-im-â-yêk}   \\
3\sg   & 	\cellcolor{cyan}\ipa{ê-wâpam-i-t}   & \cellcolor{green}\ipa{ê-wâpam-iko-yahk} & \cellcolor{green}\ipa{ê-wâpam-iko-yâhk} & \cellcolor{SkyBlue}\ipa{ê-wâpam-isk}  & \cellcolor{green}	\ipa{ê-wâpam-iko-yêk} & \cellcolor{Dandelion}	\grise{}  & \grise{}	 & \cellcolor{Dandelion}	\ipa{ê-wâpam-(im)-â-t}   \\ 
3\pl   & 	\cellcolor{cyan}\ipa{ê-wâpam-i-c-ik}&  \cellcolor{green}\ipa{ê-wâpam-iko-yahko-k} & \cellcolor{green}\ipa{ê-wâpam-iko-yâhk-ik}   & \cellcolor{SkyBlue}	\ipa{ê-wâpam-isk-ik}   & \cellcolor{green}	\ipa{ê-wâpam-iko-yêko-k} & \cellcolor{Dandelion}	\grise{} &	\grise{}  & \cellcolor{Dandelion}	\ipa{ê-wâpam-(im)-â-c-ik}   \\ 
\multirow{2}{*}{3'}   & \multirow{2}{*}{\cellcolor{cyan}}  &  \multirow{2}{*}{\cellcolor{green}}  & \multirow{2}{*}{\cellcolor{green}} &\cellcolor{SkyBlue} &  \multirow{2}{*}{\cellcolor{green}}  &\multirow{2}{*}{\cellcolor{green}}   & \multirow{2}{*}{\cellcolor{green}} & \cellcolor{Dandelion} \ipa{ê-wâpam-â-yi-t} \\ 
 \multirow{-2}{*}{} & \multirow{-2}{*}{\cellcolor{cyan}\ipa{ê-wâpam-iy-i-t}} & \multirow{-2}{*}{\cellcolor{green}\ipa{ê-wâpam-ikow-â-yahk}}   &  \multirow{-2}{*}{\cellcolor{green}\ipa{ê-wâpam-ikow-â-yâhk}} &  \multirow{-2}{*}{\cellcolor{SkyBlue}\ipa{ê-wâpam-iy-isk}} &  \multirow{-2}{*}{\cellcolor{green}\ipa{ê-wâpam-ikow-â-yêk}}& \multirow{-2}{*}{\cellcolor{green}\ipa{ê-wâpam-iko-t}}  & \multirow{-2}{*}{\cellcolor{green}\ipa{ê-wâpam-iko-c-ik}} & \cellcolor{green} \ipa{ê-wâpam-iko-yi-t}  \\ 
\bottomrule
\textsc{intr} & \ipa{ê-pimipahtâ-yân} & \ipa{ ê-pimipahtâ-yahk} & \ipa{ê-pimipahtâ-yâhk} &\ipa{ ê-pimipahtâ-yan} &\ipa{ ê-pimipahtâ-yêk} & \ipa{ê-pimipahtâ-t} & \ipa{ê-pimipahtâ-c-ik} & \ipa{ê-pimipahtâ-yi-t} \\
\bottomrule
\end{tabular}
}
\end{table}

\subsection{The diachrony of the Plains Cree conjunct order inflections}


\begin{table}[h]
\caption{Plains Cree present paradigms. TA \ipa{wâpam--} ``see" and IA \ipa{pimipahtâ--}``run" (\citealp{wolfart73})}
\label{tab:cree.conj} \centering
\resizebox{\textwidth}{!}{
\begin{tabular}{lllllllll}
\toprule
 \backslashbox{A}{P}  & 	1\sg  & 1\textsc{pi} & 1\textsc{pe} &  2\sg & 2\pl  &  3\sg & 3\pl &	3' \\ 
\midrule
1\sg   & 	\grise{}   & 	\grise{} &  \grise{} &	\cellcolor{pink}\ipa{ê-wâpam-it-ân}  & \cellcolor{pink}\ipa{ê-wâpam-it-ako-k}	& \cellcolor{Turquoise}\ipa{ê-wâpam-ak}   & 	\cellcolor{Turquoise}\ipa{ê-wâpam-ak-ik}  & \cellcolor{Turquoise}	\ipa{ê-wâpam-im-ak}   \\ 
1\textsc{pi} & \grise{}   &\grise{} & \grise{} & \multicolumn{2}{c}{\grise{}}  &  \ipa{ê-wâpam-ahk} & \ipa{ê-wâpam-ahko-k} & \cellcolor{Dandelion}	\ipa{ê-wâpam-im-â-yahk}   \\ 
1\textsc{pe} & \grise{}   &\grise{} & \grise{} & \multicolumn{2}{c}{\cellcolor{pink}\ipa{ê-wâpam-it-âhk}}   &  \cellcolor{Turquoise}\ipa{ê-wâpam-ak-iht} & \cellcolor{Turquoise} \ipa{ê-wâpam-ak-iht-ik}   & \cellcolor{Dandelion}	\ipa{ê-wâpam-im-â-yâhk}   \\ 
2\sg   & 	\cellcolor{cyan}\ipa{ê-wâpam-i-yan}   & \grise{}& \multirow{2}{*}{\cellcolor{cyan}}	&	\grise{}   &  \grise{} & \cellcolor{Aquamarine}\ipa{ê-wâpam-at}  & \cellcolor{Aquamarine}\ipa{ê-wâpam-at-ik} &\cellcolor{Aquamarine}\ipa{ê-wâpam-im-at}   \\ 
2\pl  & 	\cellcolor{cyan}\ipa{ê-wâpam-i-yêk} & \grise{}& \multirow{-2}{*}{\cellcolor{cyan} \ipa{ê-wâpam-i-yâhk}} & \grise{}  & 	\grise{}   & 	\ipa{ê-wâpam-êk}  & \ipa{ê-wâpam-êko-k} &\cellcolor{Dandelion} 	\ipa{ê-wâpam-im-â-yêk}   \\
3\sg   & 	\cellcolor{cyan}\ipa{ê-wâpam-i-t}   & \ipa{ê-wâpam-it-ahk} & \ipa{ê-wâpam-i-yam-iht}  & \cellcolor{SkyBlue}\ipa{ê-wâpam-isk}  & \cellcolor{pink}	\ipa{ê-wâpam-it-êk} & \cellcolor{Dandelion}	\grise{}  & \grise{}	 & \cellcolor{Dandelion}	\ipa{ê-wâpam-(im)-â-t}   \\ 
3\pl   & 	\cellcolor{cyan}\ipa{ê-wâpam-i-c-ik}&  \ipa{ê-wâpam-it-ahko-k} & \ipa{ê-wâpam-i-yam-iht-ik}   & \cellcolor{SkyBlue}	\ipa{ê-wâpam-isk-ik}   & \cellcolor{pink}	\ipa{ê-wâpam-it-êko-k} & \cellcolor{Dandelion}	\grise{} &	\grise{}  & \cellcolor{Dandelion}	\ipa{ê-wâpam-(im)-â-c-ik}   \\ 
\multirow{2}{*}{3'}   & \multirow{2}{*}{\cellcolor{cyan}}  &  \multirow{2}{*}{\cellcolor{green}}  & \multirow{2}{*}{\cellcolor{green}} &\cellcolor{SkyBlue} &  \multirow{2}{*}{\cellcolor{green}}  &\multirow{2}{*}{\cellcolor{green}}   & \multirow{2}{*}{\cellcolor{green}} & \cellcolor{Dandelion} \ipa{ê-wâpam-â-yi-t} \\ 
 \multirow{-2}{*}{} & \multirow{-2}{*}{\cellcolor{cyan}\ipa{ê-wâpam-iy-i-t}} & \multirow{-2}{*}{\cellcolor{green}\ipa{ê-wâpam-ikow-â-yahk}}   &  \multirow{-2}{*}{\cellcolor{green}\ipa{ê-wâpam-ikow-â-yâhk}} &  \multirow{-2}{*}{\cellcolor{SkyBlue}\ipa{ê-wâpam-iy-isk}} &  \multirow{-2}{*}{\cellcolor{green}\ipa{ê-wâpam-ikow-â-yêk}}& \multirow{-2}{*}{\cellcolor{green}\ipa{ê-wâpam-iko-t}}  & \multirow{-2}{*}{\cellcolor{green}\ipa{ê-wâpam-iko-c-ik}} & \cellcolor{green} \ipa{ê-wâpam-iko-yi-t}  \\ 
\bottomrule
\textsc{intr} & \ipa{ê-pimipahtâ-yân} & \ipa{ ê-pimipahtâ-yahk} & \ipa{ê-pimipahtâ-yâhk} &\ipa{ ê-pimipahtâ-yan} &\ipa{ ê-pimipahtâ-yêk} & \ipa{ê-pimipahtâ-t} & \ipa{ê-pimipahtâ-c-ik} & \ipa{ê-pimipahtâ-yi-t} \\
\bottomrule
\end{tabular}
}
\end{table}



\begin{table}[h]
\caption{Plains Cree present paradigms. TA \ipa{wâpam--} ``see" and IA \ipa{pimipahtâ--}``run" (\citealp{wolfart73})}
\label{tab:cree.conj} \centering
\resizebox{\textwidth}{!}{
\begin{tabular}{lllllllll}
\toprule
 \backslashbox{A}{P}  & 	1\sg  & 1\textsc{pi} & 1\textsc{pe} &  2\sg & 2\pl  &  3\sg & 3\pl &	3' \\ 
\midrule
1s & 	 & 	 & 	 & 	\ipa{-eθâni} & 	\ipa{-eθakokwe} & 	\ipa{-aki} & 	\ipa{-akwâwi} & 	\ipa{-emaki} \\ 	
1i & 	 & 	 & 	 & 	 & 	 & 	\ipa{-ankwe} & 	\ipa{} & 	\ipa{-emankwe} \\ 	
1e & 	 & 	 & 	 & 	\ipa{-eθânke} & 	\ipa{} & 	\ipa{-akenti} & 	\ipa{} & 	\ipa{-emakenti} \\ 	
2s & 	\ipa{-iyani} & 	 & 	\ipa{-iyânkwe} & 	 & 	 & 	\ipa{-ati} & 	\ipa{-atwâwi} & 	\ipa{-emati} \\ 	
2p & 	\ipa{-iyêkwe} & 	 & 	\ipa{} & 	 & 	 & 	\ipa{-êkwe} & 	\ipa{} & 	\ipa{-emêkwe} \\ 	
3s & 	\ipa{-iti} & 	\ipa{-eθankwe} & 	\ipa{-iyamenti} & 	\ipa{-eθki} & 	\ipa{-eθâkwe} & 	 & 	 & 	\ipa{-âti} \\ 	
3p & 	\ipa{-iwâti} & 	\ipa{} & 	\ipa{} & 	\ipa{-eθkwâwi} & 	\ipa{} & 	 & 	 & 	\ipa{-âwâti} \\ 	
3's & 	\ipa{-iriti} & 	\ipa{} & 	\ipa{} & 	\ipa{-emeθki} & 	\ipa{} & 	\ipa{-ekweti} & 	\ipa{-ekowâti} & 	\ipa{} \\ 	
3'p & 	& 	 & 	 & 	 & 	 & 	 & 	 &  \\ \bottomrule
\end{tabular}
}
\end{table}	
\subsection{Arapaho}

The paradigm reshaping that has occurred in Cree is not isolated in that language. Among Algonquian languages, Arapaho provide an example of a language which reshaped the conjunct order even further than Plains Cree. Before discussing the Arapaho VTA paradigm, we provide some information on the VAI   paradigm, which are necessary to understands what happened in the VTA. Proto-Algonquian reconstructions are systematically given in these section, as the drastic sound changes of Arapaho (see \citealt{goddard74arapaho}) have rendered the cognate forms barely recognizable. We cannot provide here a detailed account of Arapaho historical phonology, and defer the reader to Goddard's works for further information. Arapaho data used in this section are taken from \citet{salzmann67arapaho.verb} and \citet{cowell06arapaho}.

The Arapaho VAI conjunct order paradigm, as shown by \citet[16-7]{goddard65arapaho}, regularly derives from the proto-Algonquian Conjunct Order participle. Had they originated from the indicative Conjunct Order forms, the third person forms shoudl have been different: for instance, the third singular would have been **-θ $\leftarrow$ *\ipa{--ci}.

Table \ref{tab:arapaho.vai} shows the main  allomorphs for the Conjunct order suffixes in Arapaho and their Proto-Algonquian origins. The first plural exclusive \ipa{--'} originates from the indefinite S form *\ipa{--nki} (\citealt{goddard98morphology.arapaho}), replacing the \textsc{1pe} ending.

%a possible way to account for it is proposed by \citet[22]{goddard65arapaho}, who argues that possibly in the allomorph *\ipa{--ânke} the \ipa{â}


\begin{table}[H]
\caption{The Arapaho VAI paradigms and its proto-Algonquian origin}
\centering \label{tab:arapaho.vai}
\begin{tabular}{lllllll}
\toprule
Person &   Arapaho    & Proto-Algonquian\\
\midrule
1s & 	\ipa{--noo} &  	*\ipa{--yâni} & 	\\	
1pe & 	\ipa{--ni'} /  	\ipa{--'}  & 		 *\ipa{--yânke}	\grise{}\\	
1pi & 	\ipa{--no'} & 	 		*\ipa{--yankwe} & 	\\	
\midrule
2s & 	\ipa{--n} & 	 	*\ipa{--yani} & 	\\	
2p & 	\ipa{--nee} & 	  		*\ipa{--yêkwe} & 	\\	
\midrule
3s & 	\ipa{--t} /	\ipa{--'} & 		*\ipa{--ta} / \ipa{--ki}& 	\\	
3's & 	\ipa{--níθ} &  		*\ipa{--riciri} & 	\\	
3p & 	\ipa{--θi'} &  		*\ipa{--ciki} 	\\	
3'p & 	\ipa{--níθi} & 	 		*\ipa{--ricihi} 	\\	
\bottomrule
\end{tabular}
\end{table}

In comparison with the VAI paradigm, which is almost entirely inherited from proto-Algonquian, the VTA paradigms presents considerable reshaping; the account proposed here and the Proto-Algonquian reconstructions are largely based on  \citet[19-24]{goddard65arapaho} (in combination with  \citealt{goddard00cheyenne} for some details of the Proto-Algonquian paradigms). Table \ref{tab:arapaho.vta}   presents the regular endings of the VTA paradigm in Arapaho, taken from  \citet[487-490]{cowell06arapaho}. The  further obviative 3'$\rightarrow$3' direct and inverse forms are not included.

\begin{table}[H]
\caption{The Arapaho VTA paradigm}
\centering \label{tab:arapaho.vta}
\begin{tabular}{llllllllllll}
\toprule
 & 	1s & 	1i & 	1e & 	2s & 	2p & 	3s & 	3p & 	3' & 	\\
1s & \grise{} & 	\grise{} & 	\grise{} & 	\ipa{--éθen} & 	\ipa{--eθénee} & 	\ipa{--o'} & 	 & 	 & 	\\
1i & 	\grise{} & 	\grise{} & 	\grise{} & 	\grise{} & 	\grise{} & 	\ipa{--óóno'} & 	 & 	 & 	\\
1e & 	\grise{} & 	\grise{} & 	\grise{} & 	\ipa{--een} & 	\ipa{--eenee} & 	\ipa{--éét} & 	 & 	 & 	\\
2s & 	\ipa{--ín} / \ipa{--ún}& 	\grise{} & \ipa{--ínee} /	\ipa{--únee} & 	\grise{} & 	\grise{} & 	\ipa{--ót} & 	 & 	 & 	\\
2p & 	\ipa{--éi'een} & 	\grise{} & 	\ipa{--éi'éénee} & 	\grise{} & 	\grise{} & 	\ipa{--óónee} & 	 & 	 & 	\\
3s & 	\ipa{--éínoo} & 	\ipa{--éíno'} & 	\ipa{éi'éét} & 	\ipa{--éín} & 	\ipa{--éínee} & 	\grise{} & 	\grise{} & 	\ipa{--oot} & 	\\
3p & 	 & 	 & 	 & 	 & 	 & 	\grise{} & 	\grise{} & 	\ipa{--óóθi'} & 	\\
3' & 	 & 	 & 	 & 	 & 	 & 	\ipa{--éít} & 	\ipa{--éíθi'} &   & 	\\
\bottomrule
\end{tabular}
\end{table}

Given the complexity of the paradigm in 	Table \ref{tab:arapaho.vta}, we split the discussion  in three parts, analyzing the direct, inverse and local forms separately.

The direct forms of the VTA paradigm are compared with the corresponding reconstructed Proto-Algonquian forms in Table \ref{tab:arapaho.vta.1}; the Proto-Algonquian forms that do not match Arapaho are indicated in grey. This table shows that as in Plains Cree, while the singular direct forms are inherited, the SAP plural ones are reshaped by reanalyzing the third person ending \ipa{--oot} as \ipa{--oo-} + the VAI ending \ipa{--t}\footnote{This is a case of \citealt{watkins62celtic}'s law, whereby analogy takes place from the third person to the other forms, by reanalyzing the third person ending as part of the verb stem.} and generalizing this structure to the first and second person plural: \ipa{--óó-no'} \textsc{1pi} are \ipa{--óó-nee} \textsc{2p} built by combining the direct marker \ipa{--oo--} with the regular VAI endings.

The \textsc{1pe} \ipa{--éét} probably does not originate from inherited  *\ipa{--akenta}. This form should have yielded  either *\ipa{--ooot} or *\ipa{--eeet}. While it is not entirely impossible that vowel shortening would have happened, it is more satisfying to derive  \ipa{--éét}  from the indefinite X-3 form of the conjunct participle  *\ipa{--enta} (\citealt{goddard98morphology.arapaho}).

\begin{table}[H]
\caption{The Arapaho VTA paradigm direct forms and their PA origins}
\centering \label{tab:arapaho.vta.1}
\begin{tabular}{lllll}
\toprule
Form& Arapaho & Proto-Algonquian \\
\midrule
 \textsc{1s}$\rightarrow$3 & 	\ipa{--o'} & 	*\ipa{--aka} & 		\\		
\textsc{1pe}$\rightarrow$3 & 	\ipa{--éét} & 	\grise{} *\ipa{--akenta} & 		\\		
\textsc{1pi}$\rightarrow$3 & 	\ipa{--óó-no'} & \grise{}*\ipa{--ankwe} & 		\\		
\midrule
\textsc{2s}$\rightarrow$3 & 	\ipa{--ót} & 	*\ipa{--ata} & 		\\		
\textsc{2p}$\rightarrow$3 & 	\ipa{--óó-nee} & \grise{}*\ipa{--êkwe} & 		\\		
\midrule
\textsc{3s}$\rightarrow$3' & 	\ipa{--oot} & 	*\ipa{--âta} & 		\\		
\textsc{3p}$\rightarrow$3' & 	\ipa{--óóθi'} & 	*\ipa{--âciki} & 		\\		
\bottomrule
\end{tabular}
\end{table}

By contrast with the direct paradigm, the inverse   VTA paradigm is almost entirely remade: only the third person forms are inherited, as can be seen in Table \ref{tab:arapaho.vta.2}. As in the direct paradigm, the third person ending \ipa{--éít} was reanalyzed as \ipa{--ei--} + the VAI ending \ipa{--t} and all other forms were rebuilt on that model, replacing the inherited forms. All inverse forms follow this pattern, except the \textsc{3$\rightarrow$1pe} suffix, where *\ipa{--éi'} would have been expected. The attested \textsc{3$\rightarrow$1pe} form \ipa{--éi-'-éét} combines the expected form with the direct ending \ipa{--eet}.

\begin{table}[H]
\caption{The Arapaho VTA paradigm inverse forms and their PA origins}
\centering \label{tab:arapaho.vta.2}
\begin{tabular}{lllll}
\toprule
\textsc{3$\rightarrow$1s} & 	\ipa{--éí-noo} & 	\grise{}*\ipa{--iti} & 		\\
\textsc{3$\rightarrow$1pe} & 	\ipa{--éi-'-éét} & \grise{}*\ipa{--iyamenti} & 		\\
\textsc{3$\rightarrow$1pi} & 	\ipa{--éí-no'} & 	\grise{}*\ipa{--eθankwe} & 		\\
\midrule
\textsc{3$\rightarrow$2s} & 	\ipa{--éí-n} & \grise{}*\ipa{--eθki} & 		\\
\textsc{3$\rightarrow$2p} & 	\ipa{--éí-nee} & \grise{}*\ipa{--eθâkwe} & 		\\
\midrule
3'\textsc{$\rightarrow$3s} & 	\ipa{--éít} & 	*\ipa{--ekweti} & 		\\
3'\textsc{$\rightarrow$3p} & 	\ipa{--éíθi'} & 	*\ipa{--ekociki} & 		\\
\bottomrule
\end{tabular}
\end{table}

As the inverse paradigm, the local paradigm has undergone considerable analogy. Only the \textsc{2s$\rightarrow$1s} and \textsc{2p$\rightarrow$1s} are inherited and have escaped reshaping. 



\begin{table}[H]
\caption{The Arapaho VTA paradigm local forms and their PA origins}
\centering \label{tab:vta.3}
\begin{tabular}{lllll}
\toprule
Person & Arapaho & PA Conjunct    \\
\midrule 
\textsc{1s$\rightarrow$2s}& \ipa{--éθen} & \grise{}*\ipa{--eθâni}   \\
\textsc{1s$\rightarrow$2p}&\ipa{--eθénee} & \grise{}*\ipa{--eθakokwe} & \\
\textsc{1pe$\rightarrow$2s}&\ipa{--een} & \grise{}*\ipa{--eθânke} &   \\
\textsc{1pe$\rightarrow$2p}&\ipa{--eenee} & \grise{}*\ipa{--eθânke} &   \\
\midrule 
\textsc{2s$\rightarrow$1s}&\ipa{--ún} / \ipa{--ín} &   \ipa{--iyani}     \\
\textsc{2s$\rightarrow$1pe}& \ipa{--éi'een}& \grise{}*\ipa{--iyânkwe} &  \\
\textsc{2p$\rightarrow$1s}&\ipa{--únee} / \ipa{--ínee} &  \ipa{--iyêkwe} &   \\
\textsc{2p$\rightarrow$1pe}&\ipa{--éi'eenee} & \grise{}*\ipa{--iyânkwe} &   \\
\bottomrule
\end{tabular}
\end{table}

\citet[23]{goddard65arapaho} explains the forms \textsc{1pe$\rightarrow$2s} \ipa{--een} and \textsc{3$\rightarrow$1pe} \ipa{--éi-'-één} by proportional analogy, after the reshaping of the inverse paradigm had taken place: as direct and inverse forms were rebuilt by adding VAI endings to the first part of the third person endings \ipa{--oo--} and \ipa{--ei--} reanalyzed as direction markers, the final consonants \ipa{--t} and \ipa{--n} became   respectively \textsc{3sg} and \textsc{2sg} markers not only for  S, but also for P.

After that, even in forms where the \ipa{--t} was not a third person marker, in particular  \textsc{1pe$\rightarrow$3}   \ipa{--éét} and   \textsc{3$\rightarrow$1pe}   \ipa{--éi'éét}, it became reanalyzed as such and the forms  \textsc{1pe$\rightarrow$2}   \ipa{--één} and   \textsc{2$\rightarrow$1pe}   \ipa{--éi'één} were built by changing the final \ipa{--t} to \ipa{--n} on the model of the VAI and VTA inverse forms (see Table  \ref{tab:arapaho.analogy.local}).

\begin{table}[H]
\caption{Proportional analogy in the Arapaho local forms}
\centering \label{tab:arapaho.analogy.local}
\begin{tabular}{lllll}
\toprule
 Person &  Form &  Person &  Form\\
\midrule 
 VAI \textsc{3s} & \ipa{--t} &  VAI \textsc{2s} & \ipa{--n} \\
  \textsc{3'$\rightarrow$3s} & \ipa{--éí-\textbf{t}} &   \textsc{3$\rightarrow$2s} & \ipa{--éí-\textbf{n}} \\
  \midrule 
    \textsc{1pe$\rightarrow$3} & \ipa{--éé-\textbf{t}} &  \textsc{1pe$\rightarrow$2s} &  \grise{}\ipa{--éé-\textbf{n}} \\
  \textsc{3$\rightarrow$1pe} & \ipa{--éi'éé-\textbf{t}} &  \textsc{2s$\rightarrow$1pe} &  \grise{}\ipa{--éi'éé-\textbf{n}} \\
\bottomrule
\end{tabular}
\end{table}

From there, the \textsc{1s$\rightarrow$2s}  \ipa{--éθen} (instead of expected *\ipa{eθoon}) is likely to originate from the independent order \textsc{1s$\rightarrow$2s} ending \ipa{--éθ} $\leftarrow$ *\ipa{--eθe} to which the second person suffix \ipa{--n} from the VAI paradigm was added. 

The second plural forms \textsc{1s$\rightarrow$2p} \ipa{--eθénee}, \textsc{1pe$\rightarrow$2p} \ipa{--eenee}  and \textsc{2p$\rightarrow$1pe} \ipa{--éi'eenee} were built from the corresponding second singular forms by replacing the \textsc{2s}  \ipa{--n} marker with the \textsc{2p} one \ipa{--nee}, as shown in Table \ref{tab:arapaho.analogy.local2}.
 
 
 \begin{table}[H]
\caption{Proportional analogy in the Arapaho local forms -- second plural}
\centering \label{tab:arapaho.analogy.local2}
\begin{tabular}{lllll}
\toprule
 Person &  Form &  Person &  Form\\
\midrule 
 VAI \textsc{2s} & \ipa{--n} &  VAI \textsc{2p} & \ipa{--nee} \\
  \textsc{3$\rightarrow$2s} & \ipa{--éí-\textbf{n}} &   \textsc{3$\rightarrow$2p} & \ipa{--éí-\textbf{nee}} \\
\textsc{2s$\rightarrow$1s}&  \ipa{--í-\textbf{n}} & \textsc{2p$\rightarrow$1s}&  \ipa{--í-\textbf{nee}} \\
   \midrule 
    \textsc{1s$\rightarrow$2s}& \ipa{--éθe-\textbf{n}} & \textsc{1s$\rightarrow$2p}&\ipa{--eθé-\textbf{nee}} \grise{} \\
    \textsc{1pe$\rightarrow$2s}&\ipa{--ee-\textbf{n}} & \textsc{1pe$\rightarrow$2p}&\ipa{--ee-\textbf{nee}}\grise{} \\
    \textsc{2s$\rightarrow$1pe}& \ipa{--éi'ee-\textbf{n}}&\textsc{2p$\rightarrow$1pe}&\ipa{--éi'ee-\textbf{nee}}\grise{}\\
\bottomrule
\end{tabular}
\end{table}
 
The restructuring that took place in the Arapaho conjunct order goes one step further than that observed in the Cree paradigms: while the extent of reshaping in the direct paradigm is comparable, all inverse forms, and   all local forms except \textsc{2s$\rightarrow$1s} have been remade. The direct \ipa{--oo--} and inverse \ipa{--éí--} theme signs, which originally were restricted to non-local forms, were generalized to all inverse forms and even to the local \textsc{2$\rightarrow$1pe} forms.

Arapaho thus proves that a language can develop a near-canonical direct-inverse system from a tripartite one by generalizing the direct and inverse markers of the non-local forms to the mixed and local ones. 

\subsection{From tripartite alignment to direct-inverse?}

\section{From or towards a canonical direct-inverse system?}

\subsection{Direct-inverse systems in Sino-Tibetan}

\subsubsection{Zbu Rgyalrong}

\begin{table}[h]
\caption{Zbu Rgyalrong transitive paradigm (data adapted from \citealt{gongxun14agreement})}\label{tab:zbu.tr}
\resizebox{\columnwidth}{!}{
\begin{tabular}{l|l|l|l|l|l|l|l|l|l|l}
\textsc{} & 	\textsc{1sg} & 	  \textsc{1du} & 	\textsc{1pl} & 	\textsc{2sg} & 	\textsc{2du} & 	\textsc{2pl} & 	\textsc{3sg} & 	\textsc{3du} & 	\textsc{3pl} & 	\textsc{3'} \\ 	
\hline
\textsc{1sg} & \multicolumn{3}{c|}{\grise{}} &	\ipa{} & 	\ipa{} & 	\ipa{} &\cellcolor[wave]{600} 	\ipa{\rc{}-ŋ}   & 	\cellcolor[wave]{600} \ipa{\rc{}-ŋ-ndʑə} & 	\cellcolor[wave]{600} \ipa{\rc{}-ŋ-ɲə} & 	\grise{} \\	
\cline{8-10}
\textsc{1du} & 	\multicolumn{3}{c|}{\grise{}} &	\ipa{tɐ-\ra{}} & 	\ipa{tɐ-\ra{}-ndʑə} & 	\ipa{tɐ-\ra{}-ɲə} & 	\multicolumn{3}{c|}{ \ipa{\ra{}-tɕə}}  & 	\grise{} \\	
\cline{8-10}
\textsc{1pl} & 	\multicolumn{3}{c|}{\grise{}} & 	  & 	&  & 	\multicolumn{3}{c|}{ \ipa{\ra{}-jə}}  & 	\grise{} \\	
\cline{1-10}
\textsc{2sg} & 	\cellcolor[wave]{500}\ipa{tə-wə-\ra{}-ŋ} & 	\cellcolor[wave]{500} & 	\cellcolor[wave]{500} & 	\multicolumn{3}{c|}{\grise{}}&	\multicolumn{3}{c|}{\cellcolor[wave]{600}\ipa{tə-\rc{}}} & 	\grise{} \\	
\cline{2-2}
\cline{8-10}
\textsc{2du} & \cellcolor[wave]{500}	\ipa{tə-wə-\ra{}-ŋ-ndʑə} & \cellcolor[wave]{500}	\ipa{tə-wə-\ra{}-tɕə} & 	\cellcolor[wave]{500}\ipa{tə-wə-\ra{}-jə} & 	\multicolumn{3}{c|}{\grise{}} &	\multicolumn{3}{c|}{\ipa{tə-\ra{}-ndʑə}} & 	\grise{} \\	
\cline{2-2}
\cline{8-10}
\textsc{2pl} &\cellcolor[wave]{500} 	\ipa{tə-wə-\ra{}-ŋ-ɲə} & 	\cellcolor[wave]{500} & \cellcolor[wave]{500} & 	\multicolumn{3}{c|}{\grise{}}&	\multicolumn{3}{c|}{\ipa{tə-\ra{}-ɲə}} & 	\grise{} \\	
\hline
\textsc{3sg} & \cellcolor[wave]{500} 	\ipa{wə-\ra{}-ŋ} & 	\cellcolor[wave]{500} & 	\cellcolor[wave]{500} & 	\cellcolor[wave]{500} & 	\cellcolor[wave]{500} & 	\cellcolor[wave]{500} & \multicolumn{3}{c|}{\grise{}} &	\cellcolor[wave]{600}\ipa{\rc{}} \\ 	
\cline{2-2}
\cline{11-11}
\textsc{3du} &  \cellcolor[wave]{500}	\ipa{wə-\ra{}-ŋ-ndʑə} & 	\cellcolor[wave]{500} \ipa{wə-\ra{}-tɕə} & \cellcolor[wave]{500}		\ipa{wə-\ra{}-jə} & \cellcolor[wave]{500}	\ipa{tə-wə-\ra{}} &\cellcolor[wave]{500}	\ipa{tə-wə-\ra{}-ndʑə} & 	\cellcolor[wave]{500}\ipa{tə-wə-\ra{}-ɲə} & 	\multicolumn{3}{c|}{\grise{}} &	\ipa{\ra{}-ndʑə} \\ 
\cline{2-2}	
\cline{11-11}
\textsc{3pl} &  \cellcolor[wave]{500}	\ipa{wə-\ra{}-ŋ-ɲə} & 	\cellcolor[wave]{500} & \cellcolor[wave]{500} & 	\cellcolor[wave]{500} & 	\cellcolor[wave]{500} & 	\cellcolor[wave]{500} & \multicolumn{3}{c|}{\grise{}} &	\ipa{\ra{}-ɲə} \\ 	
\hline
\textsc{3'} & 	\multicolumn{6}{c|}{\grise{}} &\cellcolor[wave]{500}	\ipa{wə-\ra{}} & 	\cellcolor[wave]{500}\ipa{wə-\ra{}-ndʑə} & \cellcolor[wave]{500}	\ipa{wə-\ra{}-ɲə} & 	\grise{} \\	
	\hline	\hline
\textsc{intr}&\ipa{\ra{}-ŋ}&\ipa{\ra{}-tɕə}&\ipa{\ra{}-jə}&\ipa{tə-\ra{}}&\ipa{tə-\ra{}-ndʑə}&\ipa{tə-\ra{}-ɲə}&\ipa{\ra{}}&\ipa{\ra{}-ndʑə} &\ipa{\ra{}-ɲə}& 	\grise{} \\	
	\hline
\end{tabular}}
\end{table}

\subsubsection{Bantawa}

\begin{table}[b]
\caption{The Bantawa non-past affirmative transitive paradigm (\citealt[145-8]{doornenbal09})}\label{tab:bantawapos}
\resizebox{\textwidth}{!}{
\begin{tabular}{llllllllllll}
%\cline{1-12}
\toprule
\backslashbox{A}{P}  & 	\textsc{1sg} & 	\textsc{1di} & 	\textsc{1de} & 	\textsc{1pi} & 	\textsc{1pe} & 	\textsc{2sg} & 	\textsc{2du} & 	\textsc{2pl} & 	\textsc{3sg} & 	\textsc{3du} & \textsc{3pl}\\
% \cline{1-1}
% \cline{7-12}
\midrule
\textsc{1sg} &  \multicolumn{5}{c}{\grise{}}	 	&	\ipa{\ro{}-na} & 	\ipa{\ro{}-naci} & 	\ipa{\ro{}-nanin}& 	\ipa{\ro{}-uŋ}\cellcolor[wave]{600} & 	\multicolumn{2}{c}{\ipa{\ro{}-uŋcɨŋ}\cellcolor[wave]{600}} 	\\
\textsc{1di} & \multicolumn{8}{c}{\grise{}}	 		&	\ipa{\ro{}-cu}\cellcolor[wave]{600} & 	\multicolumn{2}{c}{\ipa{\ro{}-cuci}\cellcolor[wave]{600}}	\\
\textsc{1de} & 	 \multicolumn{5}{c}{\grise{}}	&	 	 \multicolumn{3}{c}{\ipa{\ro{}-ni}}     & 	\ipa{\ro{}-cuʔa}\cellcolor[wave]{600} & 	\multicolumn{2}{c}{\ipa{\ro{}-cuciʔa}\cellcolor[wave]{600}} 	\\
\textsc{1pi} & 	 \multicolumn{8}{c}{\grise{}}	&	\ipa{\ro{}-um}\cellcolor[wave]{600} & \multicolumn{2}{c}{	\ipa{\ro{}-umcɨm}\cellcolor[wave]{600}} 	\\
\textsc{1pe} & 	 \multicolumn{5}{c}{\grise{}}&		 \multicolumn{3}{c}{\ipa{\ro{}-ni}} & 	\ipa{\ro{}-umka}\cellcolor[wave]{600} & 	\multicolumn{2}{c}{\ipa{\ro{}-umcɨmka}\cellcolor[wave]{600}} 	\\
%\cline{10-12}
\textsc{2sg} & 	\ipa{tɨ-\ro{}-ŋa} & 	\grise{}&	\ipa{} & \grise{}	&	\ipa{} & 	 \multicolumn{3}{c}{\grise{}}	&	\ipa{tɨ-\ro{}-u}\cellcolor[wave]{600} & 	\multicolumn{2}{c}{\ipa{tɨ-\ro{}-uci}\cellcolor[wave]{600}} \\
\textsc{2du} & 	\ipa{tɨ-\ro{}-ŋaŋcɨŋ} & \grise{}&	\ipa{tɨ-\ro{}-ni(n)} & \grise{}	&	\ipa{tɨ-\ro{}-ni(n)} & 	 \multicolumn{3}{c}{\grise{}}	&	\ipa{tɨ-\ro{}-cu}\cellcolor[wave]{600} & 	\multicolumn{2}{c}{\ipa{tɨ-\ro{}-cuci}\cellcolor[wave]{600}}\\
\textsc{2pl} & 	\ipa{tɨ-\ro{}-ŋaŋnɨŋ} & \grise{}	&	\ipa{} & \grise{}	&	\ipa{} & 	 \multicolumn{3}{c}{\grise{}}&	\ipa{tɨ-\ro{}-um}\cellcolor[wave]{600} & 	\multicolumn{2}{c}{\ipa{tɨ-\ro{}-umcum}\cellcolor[wave]{600}} \\
%\cline{2-12}
\textsc{3sg} &\cellcolor[wave]{500}	 	\ipa{ɨ-\ro{}-ŋa} & 	 \cellcolor[wave]{500}	 & 	\cellcolor[wave]{500}	\ipa{(n)ɨ-\ro{}-aciʔa} &    \cellcolor{red}	& \cellcolor[wave]{500}		\ipa{(n)ɨ-\ro{}-inka} & 	\cellcolor[wave]{500}	 & 	\cellcolor[wave]{500}	& 	\cellcolor[wave]{500}	& 	\ipa{\ro{}-u}\cellcolor[wave]{600} & 	\multicolumn{2}{c}{\ipa{\ro{}-uci}\cellcolor[wave]{600}} \\
\textsc{3du} &\ipa{ɨ-\ro{}-ŋaŋcɨŋ}\cellcolor[wave]{500} & 	  \ipa{nɨ-\ro{}-ci}\cellcolor[wave]{500} 	& 	\cellcolor[wave]{500}{\multirow{2}{*}{\ipa{nɨ-\ro{}-aciʔa}}}	 & 	 \ipa{mɨ-\ro{}}\cellcolor{red} 	 & \multirow{2}{*}{\ipa{nɨ-\ro{}-inka}\cellcolor[wave]{500}} & 	\cellcolor[wave]{500}	\ipa{nɨ-\ro{}} & \ipa{nɨ-\ro{}-ci}\cellcolor[wave]{500} & 	\ipa{nɨ-\ro{}-in}\cellcolor[wave]{500} & \ipa{ɨ-\ro{}-cu} \cellcolor[wave]{550}& \multicolumn{2}{c}{\ipa{ɨ-\ro{}-cuci}\cellcolor[wave]{550}}	\\
\textsc{3pl} &	 \ipa{nɨ-\ro{}-ŋa}\cellcolor[wave]{500} & \cellcolor[wave]{500}	 &  \multirow{-2}{*}{\ipa{nɨ-\ro{}-aciʔa}\cellcolor[wave]{500}}	  & 	\cellcolor{red} &  \multirow{-2}{*}{\ipa{nɨ-\ro{}-inka}\cellcolor[wave]{500}}	 &   \cellcolor[wave]{500}		  & 	 \cellcolor[wave]{500}	  & \cellcolor[wave]{500}	   & 	\ipa{ɨ-\ro{}} \cellcolor[wave]{500}	& \multicolumn{2}{c}{\ipa{mɨ-\ro{}-uci}\cellcolor[wave]{550}} 	\\
%\cline{1-12}
\textsc{intr}	&\ipa{\ro{}-ŋa}&\ipa{\ro{}-ci}&\ipa{\ro{}-ca}&\ipa{\ro{}-in}&\ipa{\ro{}-inka}&\ipa{tɨ-\ro{}}& \ipa{tɨ-\ro{}-ci}& \ipa{tɨ-\ro{}-in}& \ipa{\ro{}}  & \ipa{\ro{}-ci} &\ipa{mɨ-\ro{}} \\
\bottomrule
\end{tabular}}
\end{table}

\begin{table}[b]
\caption{The Bantawa non-past negative transitive paradigm (\citealt[145-8]{doornenbal09})}\label{tab:bantawaneg}
\resizebox{\textwidth}{!}{
\begin{tabular}{llllllllllll}
\toprule
\backslashbox{A}{P}  & 	\textsc{1sg} & 	\textsc{1di} & 	\textsc{1de} & 	\textsc{1pi} & 	\textsc{1pe} & 	\textsc{2sg} & 	\textsc{2du} & 	\textsc{2pl} & 	\textsc{3sg} & 	\textsc{3du} & \textsc{3pl}	\\
%\cline{1-1}
%\cline{7-12}
\midrule
\textsc{1sg} &  \multicolumn{5}{c}{\grise{}}	 	&	{\ipa{ɨ-\ro{}-nan}} & 	\ipa{ɨ-\ro{}-nancin} & 	\ipa{ɨ-\ro{}-naminin}& 	\cellcolor{red}\ipa{ɨ-\ro{}-nɨŋ}& 	\multicolumn{2}{c}{\ipa{ɨ-\ro{}-nɨŋcɨŋ}\cellcolor{red}} 	\\
\textsc{1di} & \multicolumn{8}{c}{\grise{}}	 		&	\cellcolor{red}\ipa{ɨ-\ro{}-cun} & 	\multicolumn{2}{c}{\ipa{ɨ-\ro{}-cuncin}\cellcolor{red}}	\\
\textsc{1de} & 	 \multicolumn{5}{c}{\grise{}}	&	 	 \multicolumn{3}{c}{\ipa{ɨ-\ro{}-nin}}     & 	\cellcolor{red}\ipa{ɨ-\ro{}-cunka}& 	\multicolumn{2}{c}{\ipa{ɨ-\ro{}-cuncinka}\cellcolor{red}} 	\\
\textsc{1pi} & 	 \multicolumn{8}{c}{\grise{}}	&	\cellcolor{red}\ipa{ɨ-\ro{}-imin}& \multicolumn{2}{c}{\ipa{ɨ-\ro{}-imincin}\cellcolor{red}} 	\\
\textsc{1pe} & 	 \multicolumn{5}{c}{\grise{}}&		 \multicolumn{3}{c}{\ipa{ɨ-\ro{}-nin}} & 	\cellcolor{red}\ipa{ɨ-\ro{}-iminka} & 	\multicolumn{2}{c}{\ipa{ɨ-\ro{}-imincinka}\cellcolor{red}} 	\\
%\cline{10-12}
\textsc{2sg} & 	\ipa{tɨ-\ro{}-nɨŋ} & 	\grise{}&	\ipa{} & \grise{}	&	\ipa{} & 	 \multicolumn{3}{c}{\grise{}}	&	\ipa{tɨ-\ro{}-nan} & 	\multicolumn{2}{c}{\ipa{tɨ-\ro{}-nancin}} \\
\textsc{2du} & 	\ipa{tɨ-\ro{}-ŋɨŋcɨŋ} & \grise{}&	\ipa{tɨ-\ro{}-niminin} & \grise{}	&	\ipa{tɨ-\ro{}-niminin} & 	 \multicolumn{3}{c}{\grise{}}	&	\ipa{tɨ-\ro{}-nancin} & 	\multicolumn{2}{c}{\ipa{tɨ-\ro{}-nancinan}}\\
\textsc{2pl} & 	\ipa{tɨ-\ro{}-ŋɨŋmɨnɨŋ} & \grise{}	&	\ipa{} & \grise{}	&	\ipa{} & 	 \multicolumn{3}{c}{\grise{}}&	\ipa{tɨ-\ro{}-naminin}& 	\multicolumn{2}{c}{\ipa{tɨ-\ro{}-nannimincin}} \\
%\cline{2-12}
\textsc{3sg} & \ipa{ɨ-\ro{}-nɨŋ} & 	& &  &  & & & & \ipa{ɨ-\ro{}-un} & 	\multicolumn{2}{c}{\ipa{ɨ-\ro{}-uncin}} \\
\textsc{3du} & \ipa{ɨ-\ro{}-ŋɨŋcɨŋ} &   \ipa{nɨ-\ro{}-cin} 	& 	 \ipa{nɨ-\ro{}-cinka}	 & 	 \ipa{mɨ-\ro{}-nin} 	 &	\ipa{nɨ-\ro{}-iminka} & 	\ipa{nɨ-\ro{}-nan} & 	\ipa{nɨ-\ro{}-nancin} & 		\ipa{nɨ-\ro{}-naminin} & \ipa{ɨ-\ro{}-cun}& \multicolumn{2}{c}{\ipa{ɨ-\ro{}-cuncin}}	\\
\textsc{3pl} & 	\ipa{nɨ-\ro{}-nɨŋ} &  	  & 	  & 	 & 	 & 	  & 	 	  & 	   & 	\ipa{nɨ-\ro{}-un} 	& \multicolumn{2}{c}{\ipa{nɨ-\ro{}-uncin}} 	\\
%\cline{1-12}
\textsc{intr}	&\cellcolor{red}\ipa{ɨ-\ro{}-nɨŋ}&\cellcolor{red}\ipa{ɨ-\ro{}-cin}&\cellcolor{red}\ipa{ɨ-\ro{}-cinka}&\cellcolor{red}\ipa{ɨ-\ro{}-imin}&\cellcolor{red}\ipa{ɨ-\ro{}-iminka}&\ipa{tɨ-\ro{}-nan}& \ipa{tɨ-\ro{}-nanci}& \ipa{tɨ-\ro{}-naminin}& \cellcolor{red}\ipa{ɨ-\ro{}-nin}  & \cellcolor{red}\ipa{ɨ-\ro{}-cin} &\ipa{nɨ-\ro{}-nin} \\
\bottomrule
\end{tabular}}
\end{table}


\subsection{A comparative perspective on Rgyalrong and Kiranti}

While all specialists of Sino-Tibetan languages agree that the Rgyalrong and Kiranti verbal systems are at least partially cognate (\citealt{lapolla03}, \citealt{delancey10agreement} and \citealt{jacques12agreement}),\footnote{There authors differ in their interpretation of the data: LaPolla considers the Rgyalrong / Kiranti commonalities to be common innovations, while DeLancey and Jacques think that they go back to proto-Sino-Tibetan. This controversy will not be dealt with in this paper.} there is not consensus as to exactly which type of system should be reconstructed for the common ancestor of Rgyalrong and Kiranti.

Since Rgyalrong and Kiranti languages, unlike Cree, lack ancient attestations,\footnote{There are texts in Situ Rgyalrong, some of them dating back from the eighteenth century (\citealt{ngagdbang10gtamdpe}), but these texts are difficult to date, not fully understood and contain few   conjugated verbal forms; until a systematic study of verbal morphology in these texts has been undertaken, historical studies on Rgyalrong languages will have to be exclusively based on the comparative method.} former stages of these languages are only recoverable by reconstruction. The historical phonology of these languages are still imperfectly understood (see \citealt{jacques04these} and \citealt{opgenort05jero}), and the reconstruction of morphology in the Sino-Tibetan family is still in infancy -- this state of affair is due to the fact that these languages have not been described in detail until recently.

Given the absence of historical data, all logical possibilities will have to be explored to explain the commonalities and differences between the Bantawa and the Zbu verbal systems, taking also into account other Kiranti languages.




\subsubsection{From a canonical towards a less canonical system}

The first possibility is that the common ancestor of Zbu and Bantawa had a canonical direct-inverse close to that of Zbu, that evolved into the opaque system found in Bantawa. This hypothesis is possible in that Zbu, like other Rgyalrong languages, is   phonologically conservative as far as syllable onsets and prefixes are concerned generally in comparison to other languages. For instance, while it has been known since \citet{conrady1896} that most if not all Sino-Tibetan languages have  traces of a causative \ipa{s--} prefix, this causative prefix only remains fully productive in Rgyalrong languages (where it can be applied to recent loanwords from Chinese). While some degree of paradigm levelling has to be posited in Rgyalrong languages in any case, otherwise the person agreement morphology would be highly irregular, it is possible that the overall structure of the system goes back earlier than proto-Rgyalrong.

The focus of this section and of the following one will be on the prefixes; although most personal suffixes in Rgyalrong and Kiranti appear to be cognate, their similarity to pronouns, especially in Rgyalrong, raises the suspicion that they might have been recently grammaticalized from pronouns at least in part.   

Zbu Rgyalrong only three prefixes in its personal agreement system:   inverse   \ipa{wə--},   second person   \ipa{tə--}  and the portmanteau 1$\rightarrow$2   \ipa{tɐ--}. Bantawa has four distinct prefixes in the positive paradigm:   second person \ipa{tɨ--},   3$\rightarrow$SAP \ipa{nɨ--}, 3$\rightarrow$\textsc{1pi} and \textsc{3pl} \ipa{mɨ--} and 3$\rightarrow$1 / \textsc{3du/pl}$\rightarrow$3 \ipa{ɨ}. Phonologically,  the Zbu inverse  \ipa{wə--}, and the  second person   \ipa{tə--}  are matches for Bantawa \ipa{tɨ--},  and \ipa{ɨ--} (\citealt{jacques12agreement}), but there are no equivalent for the other prefixes in Zbu.


Since no rigorous phonological reconstruction is yet possible, only semi-reconstruction (written with \# instead of *) are provided for the common ancestor of Zbu and Bantawa, using the following symbols: \#T for the proto-form ancestral to Zbu \ipa{tə--} and Bantawa \ipa{tɨ--}, \#W for the ancestral form of Zbu \ipa{wə--} and Bantawa \ipa{ɨ--}, \#M   from the origin  of Bantawa \ipa{mɨ--}. Since the 1$\rightarrow$2 forms are not comparable between Rgyalrong and Bantawa, this form will not be discussed further in this section or the next (On the propensity of local forms to be regularly remade, see \citealt{heath98skewing}).


The basic distribution of prefixes in the common ancestor of Rgyalrong and Kiranti is assumed to be as in Table \ref{tab:proto.st1}, without any equivalent of the prefixes\ipa{mɨ--} and \ipa{nɨ--} of Bantawa. The proto-system is assumed, like Zbu (and all Rgyalrong languages) and unlike Bantawa (and all Kiranti languages), to have had two 3$\rightarrow$3 forms, a direct 3$\rightarrow$3' and an inverse one 3'$\rightarrow$3, with a function similar to that still attested in   Rgyalrong languages. The Zbu paradigm derives from the model in Table \ref{tab:proto.st1} almost unchanged, except for the local forms 1$\rightarrow$2 and 2$\rightarrow$1, which have undergone special evolutions.\footnote{In most Rgyalrong languages, the 2$\rightarrow$1 form has a  prefix distinct from the second person prefix. For instance, Japhug has   \ipa{tɯ--} in the second person (intransitive, 3$\rightarrow$2 and 2$\rightarrow$3) but \ipa{kɯ--} in 2$\rightarrow$1. This issue is not discussed in this paper.}


\begin{table}[H] 
\caption{A model of the proto-Rgyalrong/Kiranti prefixal paradigm, hypothesis 1} 
 \centering \label{tab:proto.st1}
\begin{tabular}{l|lllll} 
\toprule
&1 & 2 &3 & 3'\\
\hline
1 &\grise{} & ? &  \#\ro{} \\
2&\#T-W-\ro{} & \grise{} &\#T-\ro{}\\
3&\#W-\ro{} &\#T-W-\ro{} & \grise{} &\#\ro{} \\
3'&&&\#W-\ro{} \\
\hline
\textsc{intr}&\#\ro{}&\#T-\ro{}&\#\ro{}\\
\bottomrule
\end{tabular}
\end{table}
After the stage represented in Table \ref{tab:proto.st1}, we posit six steps to obtain the Bantawa prefixal paradigm.

\begin{enumerate}


\item  Kiranti innovates an indefinite person marker \#M--, possibly grammaticalized from the noun `man' (cognate with Japhug \ipa{tɯrme} and Tibetan \ipa{mi} `man', see \citealt{driem93agreement}).  

\item At a later stage, this form is first used in intransitive third person and transitive non-local forms; it is then reanalyzed as a third person plural marker on the one hand (explaining its restriction to \textsc{intr3p} and \textsc{3p$\rightarrow$3p} in the Bantawa paradigms), and as the inverse first plural inclusive on the other hand. The second reanalysis is straightforward: first plural inclusive is the form generally used in most Kiranti languages (for instance, in Khaling) to express generic human referents.

\item  *\ipa{u-} in initial position is centralized to \ipa{ɨ} in Bantawa; this causes the merger of \#T-W- and \#T- into \ipa{tɨ--}.

\item  \#M-- is generalized to \textsc{3pl}$\rightarrow$SAP forms;  the combination \#M-T- undergoes phonological merger and changes to \ipa{nɨ--} (the postulated sound change was *mɨtɨ-- $\rightarrow$ *mtɨ-- $\rightarrow$ *ntɨ-- $\rightarrow$  \ipa{nɨ--}. 

\item the prefix \ipa{nɨ--} is generalized to all second person inverse forms (including the first person inclusive dual), and the change is in progress of replacing \ipa{ɨ--} by \ipa{nɨ--} also in  first person dual and plural inverse forms (in \textsc{3sg$\rightarrow$1de/pe} forms, \ipa{ɨ--} and  \ipa{nɨ--} are in competition). 

\item Bantawa lost the direct vs inverse contrast in non-local forms, but rather than generalizing either direct or inverse forms, the inverse was generalized in \textsc{3du/pl}$\rightarrow$3, while direct forms were preserved in \textsc{3sg}$\rightarrow$3; the \textsc{3pl$\rightarrow$3pl} remained marked with the plural (originally indefinite) \ipa{mɨ-} prefix (This particular change may have occurred at any time after 1).
\end{enumerate}

\subsubsection{From a non-canonical towards a more canonical system}

The hypothesis presented in the previous section is but one possibility among many others. The Cree and Arapaho examples in Algonquian show that a regular direct-inverse system can be build by analogy from a more opaque system (with tripartite marking). Hence, it is not necessary to consider that the more regular Zbu Rgyalrong paradigm is closer in structure to that of the  common ancestor of Rgyalrong and Kiranti.

The Algonquian example shows that analogy follows a directionality from third person to SAP. This phenomenon is reminiscent of, though distinct from, Watkins' law. \citet[96]{watkins62celtic} shows that in many Indo-European languages, the third person (including the stem and the endings) is re-interpreted a a zero form, and new first and second person forms are build by adding affixes to the third person. In Algonquian, the stem together with third person endings *\ipa{--ekwet} and *\ipa{--aat} was not reinterpreted as a zero form, but the third person endings nevertheless spread through the direct and inverse mixed forms (and even up to the local domain in Arapaho).

In this view, it is possible that the regular system observed in Rgyalrong languages results from generalization of the inverse marker and that the only inherited use of the inverse is in the 3'$\rightarrow$3 slot. In this view, of all the hypotheses proposed in the previous section,  the only one which is necessary is n°6, the confusion of direct and inverse forms in Bantawa non-local scenario.

The Rgyalrong direct-system could have originated from a different type of system such as that schematically indicated in Table \ref{tab:proto.st2}, where only the direct mixed and the non-local are   reflected by Rgyalrong (and Bantawa) while the inverse and local slots are not known.  The original system could have either had  accusative or tripartite alignment, and was later restructured as a direct-inverse system like the conjunct order  in Cree and Arapaho by combining the inverse marker with the affixes of the intransitive paradigm.


\begin{table}[H] 
\caption{A model of the proto-Rgyalrong/Kiranti prefixal paradigm, hypothesis 2} 
 \centering \label{tab:proto.st2}
\begin{tabular}{l|lllll} 
\toprule
&1 & 2 &3 & 3'\\
\hline
1 &\grise{} & ? &  \#\ro{} \\
2&? & \grise{} &\#T-\ro{}\\
3&?& ? & \grise{} &\#\ro{} \\
3'&&&\#W-\ro{} \\
\hline
\textsc{intr}&\#\ro{}&\#T-\ro{}&\#\ro{}\\
\bottomrule
\end{tabular}
\end{table}

In this view, prefixes like Bantawa \ipa{nɨ--} and   \ipa{mɨ--} in some parts of the inverse paradigms could be archaisms that have been regularized in Rgyalrong. The presence of \ipa{ɨ--} in some 3$\rightarrow$1 forms in this view is innovative, and represents a parallel development with Rgyalrong. As a general rule, analogy   applies first on the plural and dual forms and only later in the singular, so it is not likely that \ipa{nɨ--} is being replaced by \ipa{ɨ--} in 3$\rightarrow$1 forms (since \ipa{ɨ--} is only found in \textsc{3sg/du}$\rightarrow$1, not in  \textsc{3pl}$\rightarrow$1). This however does not preclude the possibility that \ipa{nɨ--} was  originally a \textsc{3pl}$\rightarrow$SAP prefix, and that more opaque portmanteau prefixes (and suffixes) existed in the paradigm of the proto-language from which the Bantawa and the Zbu Rgyalrong paradigms originated.


\section{Conclusion}


 \bibliographystyle{linquiry2}
 \bibliography{biblioinverse}

\end{document}