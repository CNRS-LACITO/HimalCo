\documentclass[oldfontcommands,oneside,a4paper,11pt]{article} 
\usepackage{fontspec}
\usepackage{natbib}
\usepackage{booktabs}
\usepackage{xltxtra} 
\usepackage{polyglossia} 
\usepackage[table]{xcolor}
\usepackage{gb4e} 
\usepackage{tangutex2} 
\usepackage{tangutex4}
\usepackage{multicol}
\usepackage{graphicx}
\usepackage{float}
\usepackage{hyperref} 
%\usepackage{lineno}
\hypersetup{bookmarks=false,bookmarksnumbered,bookmarksopenlevel=5,bookmarksdepth=5,xetex,colorlinks=true,linkcolor=blue,citecolor=blue}
\usepackage[all]{hypcap}
\usepackage{memhfixc}
%\usepackage{lscape}
\usepackage{amssymb}
 
\bibpunct[: ]{(}{)}{,}{a}{}{,}

%\setmainfont[Mapping=tex-text,Numbers=OldStyle,Ligatures=Common]{Charis SIL} 
\newfontfamily\charis[Mapping=tex-text,Numbers=OldStyle,Ligatures=Common]{Charis SIL} 
\newfontfamily\phon[Mapping=tex-text,Ligatures=Common,Scale=MatchLowercase]{Charis SIL} 
\newcommand{\ipa}[1]{{\phon \mbox{#1}}} %API tjs en italique


\newcommand{\grise}[1]{\cellcolor{lightgray}\textbf{#1}}
\newfontfamily\cn[Mapping=tex-text,Ligatures=Common,Scale=MatchUppercase]{MingLiU}%pour le chinois
\newcommand{\zh}[1]{{\cn #1}}
\newcommand{\refb}[1]{(\ref{#1})}
\newcommand{\factual}[1]{\textsc{:fact}}
\newcommand{\sg}{\textsc{sg}}
\newcommand{\pl}{\textsc{pl}}

\XeTeXlinebreaklocale 'zh' %使用中文换行
\XeTeXlinebreakskip = 0pt plus 1pt %
 %CIRCG
 \newcommand{\bleu}[1]{{\color{blue}#1}}
\newcommand{\rouge}[1]{{\color{red}#1}} 
\newcommand{\tgf}[1]{\begin{tabular}{l}\mo{#1}\\{\tiny #1}\end{tabular}}
\newcommand{\tinynb}[1]{\tiny#1}


\begin{document} 
\title{Tangut, Gyalrongic and the nature of person indexation in Sino-Tibetan/Trans-Himalayan\footnote{I would like to thank Scott DeLancey, Nathan Hill, Boyd Michailovsky, Alexis Michaud, Jackson T.S. Sun and two anonymous reviewers for useful comments on previous versions of this paper. I am responsible for any error that remain in this work. The Japhug examples are taken from a corpus that is progressively being made available on the Pangloss archive (\citealt{michailovsky14pangloss}).This research was funded by the HimalCo project (ANR-12-CORP-0006) and is related to the research strand LR-4.11 ‘‘Automatic Paradigm Generation and Language Description’’ of the Labex EFL (funded by the ANR/CGI). Glosses follow the Leipzig glosses rules, to which the following are added: \textsc{antierg} antiergative, \textsc{dir} direct, \textsc{hon} honorific, \textsc{irr} irrealis, \textsc{inv} inverse, \textsc{pot} potentialis, \textsc{sens} sensory.}}
\author{Guillaume Jacques}
\maketitle
%\linenumbers 

\textbf{Abstract}: The aim of this paper is to dispel some misconceptions that tend to creep in when discussing  the person indexation systems of Trans-Himalayan languages. First, it is argued that the view that personal affixes are derived from pronouns, rather than the other way round, is not as obvious as it seems. Second, we show that person indexation in Tangut, the oldest Trans-Himalayan language with person indexation, is not optional, as has often been assumed. Third, we demonstrate that person indexation in Gyalrongic and Kiranti is sensitive to grammatical relations, a finding which is at variance with its analysis as marking `speech act participant involvement'.

\textbf{Keywords}: Gyalrongic, Kiranti, Tangut, Person indexation, Agreement, Grammaticalization, Sino-Tibetan

\section{Introduction}
Trans-Himalayan / Sino-Tibetan is probably one of the typologically most diverse of the world's language families, perhaps the only one where both prototypically isolating languages (such as Lolo-Burmese or Chinese) and polysynthetic languages (Gyalrongic and Kiranti) are found. 

There is intense debate as to how to interpret this diversity. A line of argument (pursued in particular by \citealt{bauman75}, \citealt{delancey89agreement}, \citealt{driem93agreement}, \citealt{delancey10agreement}, \citealt{delancey11prefixes}, \citealt{jacques12agreement}) is that the complex verbal morphology of Gyalrongic and Kiranti should be reconstructed back (at least in part) to proto-Trans-Himalayan. Another line of argument (\citealt{lapolla92},  \citealt{lapolla03}, \citealt{lapolla12comments}, \citealt{zeisler15eat}) is that little verbal morphology can be reconstructed back to proto-Trans-Himalayan, and that the complex person indexation systems of Gyalrongic and Kiranti were grammaticalized from the accretion of pronouns, a view that had already been expressed in Hodgson's pioneering research on these languages in the mid-nineteenth century, for which he coined the term `pronominalizing languages' (see for instance  \citealt{hodgson57sifan}).

 

First, I address the issue of the term `Pronominalizing language', taking the strong stand that this term should be avoided, as it tends to convey an overly simplistic and linear view of grammaticalization. Second, I discuss the person indexation system of Tangut, and show that it is neither optional nor transparent, as has been assumed in various publications. Third, I present counterarguments to the idea that person indexation in Gyalrongic and Kiranti languages is based on `person involvement' rather than syntactic relations. The analyses proposed here are based on first-hand linguistic fieldwork on these two groups of languages, which has yielded more reliable documentation than had been available heretofore.

\section{Pronominalization: a descriptive label, or an explanation?}
`Pronominalization' has been used by \citet{hodgson57sifan} as a label to describe the presence of person indexation in some Trans-Himalayan languages like Kiranti languages, on which he had first-hand fieldwork experience. This pioneering insight attracted attention to an important feature of these languages. On the other hand, the analysis encapsulated in this term (that person indexation is the result of the accretion of pronouns on the verb complex) may need to be re-examined in full detail, making use of newly available evidence. To preview the result of the re-analysis set out below, it would appear useful to recognize that the term `pronominalized' is now outdated and potentially misleading, and should be retired for at least three reasons.

First, whatever the actual antiquity of the person indexation systems in Trans-Himalayan, it is clear that an important part of person markers on verbs have been grammaticalized from sources other than pronouns, in particular nominalized forms (see for instance \citealt{jacques15generic} on the origin of the portmanteau 2$\rightarrow$1 and 1$\rightarrow$2 prefixes in Gyalrong languages).

Second, although in some Trans-Himalayan languages such as Kuki-Chin productive person markers are identical to (and probably derive from) possessive prefixes or pronouns, the situation is different in Kiranti and Gyalrong languages where, in particular, the second person prefixes cannot be shown to derive from pronouns (\citealt{jacques12agreement}, \citealt{delancey11prefixes} and \citealt{delancey14second}).

Third, the hypothesis that, whenever pronouns, possessive markers and person indexation markers present resemblances, the latter must be derived from the former requires careful examination.  Cases of degrammaticalization of person markers into pronouns are admittedly very rare (though not unattested, see \citealt{norde09degrammaticalization, hyman11macrosudan}). However, in some languages pronouns are build from a conjugated verbal stem, or a possessed nominal stem. 
 

For instance, in Ainu, the \textsc{1sg} pronoun \ipa{kuani} is in fact etymologically the nominalization of the \textsc{1sg} form of the existential copula (\ipa{ku-an-i} \textsc{1sg}-exist-\textsc{nmlz}, \citealt[31]{shibatani90japan}). Likewise, in Lakota the pronouns \textsc{1sg} \ipa{miyé}, \textsc{2sg} \ipa{niyé} and \textsc{3sg} \ipa{iyé} are in fact conjugated verb forms meaning `it is him, it is you, it is him' (\citealt{deloria41}, \citealt[707;754]{ullrich08}). 

Pronouns derived from possessed nouns are found in Algonquian (as in Ojibwe \ipa{n-iin} `I', \ipa{g-iin} `you' and \ipa{w-iin} `he', see \citealt{valentine01grammar}), and in the Trans-Himalayan family Japhug provides such an example. As can be seen in Table \ref{tab:pronoun} below, pronouns are built by combining possessive prefixes with the root \ipa{-ʑo} `oneself' (the same is true of other Gyalrongic languages, such as Tshobdun, see \citealt[113]{jackson98morphology}).

 \begin{table}[H] \centering
\caption{Pronouns and possessive prefixes in Japhug}\label{tab:pronoun}
\begin{tabular}{lllllllll} 
\toprule
 Free pronoun & Prefix & Person\\
\midrule
 \ipa{a-ʑo}  &	\ipa{a-}  &		1\textsc{sg} \\
\ipa{nɤ-ʑo}  &	\ipa{nɤ-}  &			2\textsc{sg}\\
\ipa{ɯ-ʑo}  &	\ipa{ɯ-}  &			3\textsc{sg}\\
\midrule
\ipa{tɕi-ʑo}  &	\ipa{tɕi-}  &			1\textsc{du} \\
\ipa{ndʑi-ʑo}  &	\ipa{ndʑi-}  &		2\textsc{du} \\	
\midrule
\ipa{i-ʑo}    &	\ipa{i-}  &			1\textsc{pl} \\
\ipa{nɯ-ʑo}   &	\ipa{nɯ-}  &			2\textsc{pl} \\
\midrule
\ipa{tɯ-ʑo} & \ipa{tɯ-}   &  generic\\
\bottomrule
\end{tabular}
\end{table}

 Grammaticalization pathways exist between person indexation markers and pronouns, but these pathways are not as linear and unidirectional as the use of a term such as `pronominalizing languages' can suggest.  

\section{Person indexation in Tangut}
The role of Tangut data in the debate of the antiquity of person indexation systems in Trans-Himalayan is critical, as of all the languages with ancient attestation (Tangut is attested from the 11^{th} to the 16^{th} centuries), Tangut is the only one with a fully fledged person indexation system.


%Given Kepping \citeyear{kepping94conjugation} lucid response to \citet{lapolla92} published more than twenty years ago, it is surprising that \citet{zeisler15eat} still quotes the conclusions of the latter article concerning Tangut.

\citet{lapolla92} argues that (1) there is a one-to-one relationship between the pronouns and the suffixes, (2) person agreement in Tangut is optional and that for these reasons (3) person indexation suffixes have been recently grammaticalized from pronouns and show no sign of age.  

LaPolla's three claims are evaluated here on the basis of a corpus comprising major narrative texts (see \citealt[8-9]{jacques14esquisse}), in particular the collection of short stories called  `The Grove of Categories', from which most examples in this section are taken.

%\begin{tabular}{lllllllllll}
%	\tgf{4028}&	\tgf{3986}&	\tgf{4893}&	\tgf{1139}&	\tgf{1526}&	\tgf{5880}&	\tgf{0524}&	\tgf{2590}&	\tgf{5591}&	\tgf{4601}&\tgf{3916}\\
%\tinynb{4028}&	\tinynb{3986}&	\tinynb{4893}&	\tinynb{1139}&	\tinynb{1526}&	\tinynb{5880}&	\tinynb{0524}&	\tinynb{2590}&	\tinynb{5591}&	\tinynb{4601}&\tinynb{3916}\\
%\end{tabular}

\subsection{Person indexation suffixes}
As pointed out by \citet{kepping94conjugation}, while agreement suffixes in Tangut do present  resemblances with pronouns, as shown in Table \ref{tab:pronoms.suffixes}, pronouns and agreement markers cannot be equated at a synchronic level.

\begin{table}[H]
\caption{Pronouns and person suffixes in Tangut (\citealt{kepping75agreement, kepping85})}\label{tab:pronoms.suffixes} \centering
\begin{tabular}{llllll} 
\toprule
\multicolumn{3}{c}{Pronoun} &\multicolumn{3}{c}{Suffix} \\
\midrule
\tgf{2098} & \ipa{ŋa²}  & 1\textsc{sg} & \tgf{2098} & \ipa{ŋa²}  &1\textsc{sg} \\
\tgf{3926} & \ipa{nja²} & 2\textsc{sg} & \tgf{4601} & \ipa{nja²} &2\textsc{sg} \\
\tgf{4028} &  \ipa{nji²} & 2\textsc{sg}  honorific or 2\textsc{pl} & \tgf{4884} & \ipa{nji²} & 1\textsc{pl} and 2\textsc{pl} \\
\bottomrule
\end{tabular}
\end{table}

First, for the first person singular, while the pronoun \mo{2098} \ipa{ŋa²} is given the same reading as the \textsc{1sg} suffix,\footnote{It should be emphasized here that both the \textsc{1sg} pronoun and the suffix are not directly cognate with forms such as Tibetan \ipa{ŋa}, as pre-Tangut *\ipa{-(j)a} regularly becomes \ipa{-e} or \ipa{-ji}, see \citet{jacques14esquisse}.
} another \textsc{1sg} pronoun \mo{0261} \ipa{mjo²} is also very common. In  books 3 to 6 of the  \textit{Grove of Categories}, there are 31 occurrences of the pronoun \mo{2098} \ipa{ŋa²}, 60 occurrences of \mo{2098} \ipa{ŋa²} as a \textsc{1sg} suffix, and 20 occurrences of  \mo{0261} \ipa{mjo²}. This latter pronoun always triggers agreement with  the \mo{2098} \ipa{ŋa²} suffix when occurring as a core argument, as in example \ref{ex:tg:mjo}. Thus, despite the homophony and homography of the \textsc{1sg} pronoun and the \textsc{1sg} suffix,\footnote{On this topic, it should be noted that it is common in Tangut to write unrelated but homophonous (or perhaps, near homophonous) morphemes with the same character, see several examples in \citet{jacques11tangut.verb}.} it is a fact of the synchronic grammar of Tangut that the two are distinct - otherwise, we would not expect agreement between \mo{0261} \ipa{mjo²} and \mo{2098} \ipa{ŋa²}. 

\begin{exe}
\ex \label{ex:tg:mjo}  
\glll 
\tgf{0261} 	\tgf{2541} 	\tgf{1139} 	\tgf{3104} 	\tgf{0046}\tgf{0749} 	\tgf{2620}\tgf{2098} \\
\ipa{mjo²} 	\ipa{dzjwo²} 	\ipa{.jij¹} 	? 	\ipa{ljij²-phji¹} 	\ipa{njwi²-ŋa¹} \\
\textsc{1sg} man \textsc{antierg} ghost see[A]-cause[A] can-\textsc{1sg} \\
\glt `I can make people see ghosts.' (The Grove of Categories, 05.21B.4)
\end{exe} %吾能令人见鬼

It is however with the honorific pronoun \mo{4028} \ipa{nji²} and the SAP plural suffix \mo{4884} \ipa{nji²} that the difference is most telling. \mo{4028} \ipa{nji²} is singular, and when serving as core argument, the verb takes the \textsc{2sg} \mo{4601} \ipa{-nja²} suffix, as in \ref{ex:tg:toi}. No example of  \mo{4028} \ipa{nji²} used with \mo{4884} \ipa{nji²} has been found in the corpus under study.
 
\begin{exe}
\ex \label{ex:tg:toi}  
\glll   \tgf{4028} 	\tgf{3986}\tgf{4893} 	\tgf{1139} 	\tgf{1526} 	\tgf{5880} 	\tgf{0524} \tgf{2590}\tgf{5591}\tgf{4601}\tgf{3916} \\
\ipa{nji²}	\ipa{njɨ¹.wjɨ¹}	\ipa{.jij¹}	\ipa{tshji²}	\ipa{ŋwu²}	\ipa{dzju¹}	\ipa{.wjɨ²-lhjị²-nja²-sji²} \\
\textsc{2sg:hon} mother.in.law  \textsc{antierg} serve \textsc{instr} order \textsc{pfv}-receive[B]-\textsc{2sg-ifr} \\
\glt `It is you_{sg} who served (our) mother-in-law and received her instructions.' (Newly Collected Biographies of Affection and Filial Piety, 33.4, \citealt{jacques07textes})
\end{exe}

The suffix \mo{4884} \ipa{nji²}, on the other hand, appears with first or second person plural, as in \ref{ex:tg:1pi}.

\begin{exe}
\ex \label{ex:tg:1pi}  
\glll 
\tgf{2248}\tgf{2065} 	\tgf{1413}\tgf{5970} 	\tgf{1139} 	\tgf{1567}\tgf{0239} 	\tgf{0508}\tgf{4884} \\
\ipa{gjɨ²mji²} 	\ipa{tej¹pie¹} 	\ipa{.jij¹} 	\ipa{gji²lhjɨ¹} 	\ipa{ŋwu²-nji²} \\
\textsc{1pi} Taibo \textsc{gen} grandchildren be-\textsc{1/2pl} \\
\glt `We are the descendants of Taibo.'  (The Grove of Categories, 04.33A.4) %我等是太伯之子孙
\end{exe}

The idea that the suffix \mo{4884} \ipa{nji²} is historically derived from the ancestor of the pronoun \mo{4028} \ipa{nji²} implies that the singular honorific \textsc{2sg} originates from a \textsc{2pl} (a reasonable assumption), and that in verbal indexation the first vs second person plural distinction became neutralized at the expense of the first person, a rare phenomenon, but not altogether unthinkable. The exact mechanism for this neutralization in itself is an interesting question, and may first have occurred for the first plural inclusive before affecting the first plural exclusive.

The resemblance between person indexation suffixes and some of the pronouns in Tangut (Table \ref{tab:pronoms.suffixes}) calls for an explanation, and the most likely one is that these suffixes were ultimately  grammaticalized from pronouns, or at least remade after the pronouns. Given however the functional difference between pronouns and the corresponding suffixes, this grammaticalization cannot have occurred immediately before the time when Tangut was put to writing.

%Yet, this hypothesis has many implications that have never been dealt within a systematic way. 
%
%Table \ref{tab:macrorgy} presents correspondences between person indexation suffixes in various Macro-Gyalrongic languages, including Tangut. While most of these markers do present similarities, it is striking that the correspondences between these forms are \textit{unique}: within Gyalrongic; the first person suffix is the only case of \ipa{a} to \ipa{ŋ} correspondence, and the second plural the only case of \ipa{n} to \ipa{ɲ} between Japhug and Zbu. Moreover, final \ipa{-ŋ} has been lost in the inherited vocabulary of Japhug, Zbu and Situ, and its preservation of
%
%It is not possible to reconstruct the forms of these suffixes in a linear way. However, in some Gyalrongic languages, the person indexation suffixes have some degree of allomorphy. For instance, in Zbu the \textsc{1sg} can be realized as \ipa{-ɑŋ}, \ipa{-ŋ} or nasalization of the final stop + \ipa{ɑŋ} (\citet[46]{gongxun14agreement}).  This allomorphy offers an explanation for the irregular correspondences: in proto
%
% \begin{table}[H]
%\caption{SAP suffixes in Macro-Gyalrongic languages } \centering \label{tab:macrorgy}
%\begin{tabular}{lllllll}
%\toprule
%Language &\textsc{1sg} & \textsc{2sg} &\textsc{1pl} & \textsc{2pl} & Source\\
%\midrule
%Japhug & \ipa{-a} & zero & \ipa{-j} & \ipa{-nɯ} & \citet{jacques10inverse} \\
%Zbu & \ipa{-ŋ} & zero& \ipa{-jə} & \ipa{-ɲə} & \citet{gongxun14agreement} \\
%Situ & \ipa{-ŋ} & \ipa{-n}& \ipa{-j} & \ipa{-ɲ} & \citet{linxr93jiarong} \\
%Khroskyabs & \ipa{-ŋ} & \ipa{-n}& \ipa{-j} & \ipa{-n} & \citet{lai15person} \\
%Stau & \ipa{-u} /  \ipa{-ã} & \ipa{-j}  & \ipa{-ã} & \ipa{-j} & \citet{jacques14rtau} \\
%Muya & *\ipa{-ø} & *\ipa{-ɛ} & *\ipa{-e} & *\ipa{-e} & \citet{gao15menya} \\
%Tangut & \ipa{-ŋa²} &\ipa{-nja²}& \ipa{-nji²}& \ipa{-nji²}& \\
%\bottomrule
%\end{tabular}
%\end{table}


\subsection{Stem Alternations}
The most compelling piece of evidence that person indexation in Tangut is not recent, however, does not come from the person indexation suffixes. \citet{gong01huying}, \citet{jacques09tangutverb} and \citet{jacques14esquisse} have shown the existence of a closed class of irregular verbs with two stems, called A and B. Stem B is restricted to \textsc{2sg}$\rightarrow$3, \textsc{1sg}$\rightarrow$3 forms (as well as the reflexive \textsc{1sg} and \textsc{2sg} forms), as illustrated by examples \ref{ex:1>3} and \ref{ex:see:1>3}.\footnote{The presence of stem A, and the absence of \textsc{1sg} indexation suffix in example \ref{ex:see:1>3} is explained in section \ref{sec:optional}.}

\begin{exe}
\ex \label{ex:1>3}
\glll
\tgf{1542}	\tgf{3508}	\tgf{0100}	\tgf{2798}	\tgf{2987}	\tgf{5481}	\tgf{4547}\tgf{2098}\\
   \ipa{ku¹}	\ipa{bji²}	\ipa{lew¹}	\ipa{.jir²}	\ipa{lhjɨ̣¹}	\ipa{bo²}	\ipa{dzjo¹-ŋa²} \\
therefore subject one hundred strike staff \bleu{eat[B]-\textsc{1sg}} \\
\glt `Then I, your subject, will take (eat) a hundred blows.'   (The Grove of Categories 06.13A.5) %则拷臣一百杖
\end{exe}

\begin{exe}
\ex  \label{ex:see:1>3}
\glll  	\tgf{3133}	\tgf{0261}	\tgf{1531}	\tgf{1139}	\tgf{0795}\tgf{0676}	\tgf{0046}	\tgf{5643}\tgf{3092}	\tgf{2912}\tgf{3092}	\tgf{5643}\tgf{1374}\tgf{4803}\tgf{2098} \\
 \ipa{sjij¹}	\ipa{mjo²}	\ipa{gja¹}	\ipa{.jij¹}	\ipa{rjɨr²-.wjij¹}	\ipa{ljij²} \ipa{mjɨ¹djij²}	\ipa{lhjwo¹-djij²}	\ipa{mjɨ¹-tɕhjɨ¹-lji²-ŋa²} \\
today \textsc{1sg} army \textsc{antierg} pfv-leave see[A] but come.back-\textsc{dur} \bleu{\textsc{neg-pot}-see[B]-\textsc{1sg}} \\
\glt  `Today I see the army leave, but I will not see it return.' (The Grove of Categories, 3.16B.6-7) %今吾见军之出,不见军之入
\end{exe}

In other configurations with the suffixes  \mo{2098} \ipa{ŋa²}  and \mo{3926} \ipa{nja²}, including 1$\rightarrow$\textsc{2sg}, 2$\rightarrow$\textsc{1sg}, 3$\rightarrow$\textsc{2sg} and 3$\rightarrow$\textsc{1sg}, stem A is used, as in \ref{ex:2>1}. It is also the form which appears with SAP plural or third person arguments.

\begin{exe}
\ex \label{ex:2>1} 
\glll	\tgf{2447}	\tgf{1519}\tgf{5165}	\tgf{2590}\tgf{4517}\tgf{2098}\\
\ipa{ljo²}	\ipa{ɣu¹twụ¹}	\ipa{wjɨ²-dzji¹-ŋa²} \\
brother instead \rouge{\textsc{imp}-eat[A]-\textsc{1sg}} \\
\glt `Eat me instead of my brother!' (Newly Collected Biographies of Affection and Filial Piety 17.7, \citealt[55-6]{jacques07textes})
\end{exe}



\begin{exe}
\ex \label{ex:3>1}
\glll \tgf{4689}	\tgf{1531}	\tgf{0866}	\tgf{1139}	\tgf{2393}	\tgf{3456}	\tgf{0705}	\tgf{2219}\tgf{0046}\tgf{2098}\\
  \ipa{.jwar¹}	\ipa{gja¹}	\ipa{ɣu¹}	\ipa{.jij¹}	\ipa{ljiij²}	\ipa{lja¹}	\ipa{zjịj¹}	\ipa{kjij¹-ljij²-ŋa²} \\
Yue army Wu \textsc{antierg} destroy come time \rouge{\textsc{irr}-see[A]-\textsc{1sg}} \\
\glt `When the Yue army will come to destroy Wu, it will see me.' (The Grove of Categories, 03.21B.4-5) %悬我之头于城东门,以视越军来灭吴
\end{exe}

Table \ref{tab:paradigm} summarizes the distribution of stems and suffixes in the forms of the transitive paradigm attested in the corpus. Unattested forms are indicated by a question mark. Regardless of the historical interpretation of these alternations (on which see \citealt{jacques09tangutverb} and \citealt{jacques14esquisse}), they are synchronically non-productive, securely attested for less than a fifty common verbs, and stem A is not predictable from stem B and vice-versa. %Given the complexity of the Tangut script, it is possible, though unprovable, that stem alternation was more developed, but that stem A and stem B characters were created only for the most common verbs, and that other verbs used the same character with distinct readings. 

Even if the suffixes were recently grammaticalized (i.e. only a few centuries before the Tangut script was created), it is unclear how irregular stem alternations could come into existence in such a short time span, and also requires an explanation as to their historical origin.

\begin{table}[H]
\caption{Attested forms of the ditransitive paradigm in Tangut}\centering  \label{tab:paradigm}
\begin{tabular}{lllll}
\toprule
	&	1\sg{}	&	2\sg{}	&	1/2\pl{}	&	3	\\
	\midrule
1\sg{}	&	?	&	A-\ipa{nja²}	&	?	&	 B-\ipa{ŋa²}	\\
2\sg{}	&	A-\ipa{ŋa²}	&	B-\ipa{nja²}	&	A-\ipa{nji²}	&	 B-\ipa{nja²}	\\
1/2\pl{}	&	 A-\ipa{ŋa²}	& ?	&	?	&	A-\ipa{nji²}	\\
3	&	A-\ipa{ŋa²}	&	A-\ipa{nja²}	&	?	&	A 	\\
\bottomrule
\end{tabular}
\end{table}

At that time when LaPolla wrote his article, Tangut texts were not easily accessible for independent study, and stem alternations had not yet been fully described. Although \citet{nishida75} first suggested their existence, this discovery was not taken into account by other scholars (even \citealt{kepping85} and \citealt{driem91tangut}), and it was not until \citet{gong01huying} that it became widely known.

 What we need most at the present moment would be an investigation of stem alternations in the whole Tangut corpus, based on an exhaustive database including all verbal forms. In particular, it seems that in addition to the A/B stem alternation described above, Tangut also had other types of stem alternation whose morphosyntactic function is still unclear (see \citealt{jacques14esquisse}).

\subsection{Is person marking in Tangut optional?} \label{sec:optional}
As part of an argument that Tangut person indexation is recent, 
\citet{lapolla92} argues that person indexation in Tangut is optional, and interprets this as a sign of it not being fully grammaticalized. He cites  \citet{ahrens90tangut} (whose corpus was based on the \textit{Grove of Categories} like the present study) according to whom:

\begin{itemize}
\item `Verb agreement only occurs in quoted speech.'
\item `Agreement is usually with the A and S argument, not with the P argument.'
\item `When there are two SAPs involved in a clause, agreement is not necessarily with the P argument.'
\end{itemize}

Generalizations n°2 and n°3 do not match my observations about Tangut texts. Table \ref{tab:1sg} lists the number of attestations of the suffix \mo{2098} \ipa{-ŋa²} in chapters 3 to 6 of the Grove of Categories.\footnote{This corpus is available as a supplementary file to this article, so that any scholar may recheck the data.} Leaving aside intransitive verbs, which are not relevant to the present debate, we see that \ipa{-ŋa²} appears in \textsc{1sg}$\rightarrow$3, 3$\rightarrow$\textsc{1sg} and 2$\rightarrow$\textsc{1sg} (in conformity with Table \ref{tab:paradigm}), never in \textsc{1sg}$\rightarrow$2 forms. Stem B appears  in \textsc{1sg}$\rightarrow$3 forms, and stem A in all other transitive forms. 

Agreement with the P argument is the rule in 3$\rightarrow$SAP and SAP$\rightarrow$SAP to configurations; the only case when agreement with the A occurs is in SAP$\rightarrow$3 forms. Since however SAP$\rightarrow$3 are by far more common in texts than all SAP$\rightarrow$SAP and 3$\rightarrow$SAP forms put together, the absolute number of attestations of P argument indexation (here 10 out of 60) is lower than that of A argument indexation (36); this fact may explain how Ahrens came to propose generalization n°2.

When both arguments are SAPs, the indexation suffix always refer to the P, as is the case in all Gyalrongic languages (\citealt{jackson03caodeng, jacques10inverse, gongxun14agreement, lai15person}). 

\begin{table}[H]
\caption{Distribution of the \textsc{1sg} suffix \ipa{-ŋa²} in chapters 3-6 of the \textit{Grove of Categories}} \label{tab:1sg} \centering
\begin{tabular}{lll}
\toprule
Form& Nb of attestations \\
\midrule
\textsc{1sg} (intransitive verb) & 	13 \\ 
\textsc{1sg}$\rightarrow$3  (no stem alternation) & 	29 \\ 
\textsc{1sg}$\rightarrow$3 (stem B)& 	7 \\ 
3$\rightarrow$\textsc{1sg} (no stem alternation) & 	2 \\ 
3$\rightarrow$\textsc{1sg} (stem A)& 	2 \\ 
2$\rightarrow$\textsc{1sg} (no stem alternation)& 	1 \\ 
2$\rightarrow$\textsc{1sg}  (stem A)& 	5 \\ 
\textsc{1sg} possessor & 	1 \\ 
\midrule
Total & 	60 \\ 
\bottomrule
\end{tabular}
\end{table}

%Cases of indexation of P argument are common in the \textit{Grove of Categories} (see for instance example \ref{ex:3>1} above, and Table \ref{tab:paradigm}). When both argument are SAP, it is always the P argument which is marked by the person  indexation suffix (example \ref{ex:2>1}), a property shared with Gyalrongic languages. 

As for claim n°1, third person is zero marked in Tangut, so that a verb whose core arguments are all third person cannot receive person marking. A slight bias here is that the whole text of the \textit{Grove of Categories} is a series of short stories translated from Chinese from a third-person point of view, and clauses other than those in quoted speech only have third person arguments. In the absence of a first-person narrator or first-person comments on the text, it is no surprise that first or second person will only occur in reported speech; this would be true in any language.  A translation of this text in any language where third persons are unmarked for person would yield the same result.

 


%\begin{exe}
%\ex \label{ex:tg:non.actant.indicie.tuer}
%\glll
%\tgf{5354}	\tgf{3583}	\tgf{1326}\tgf{2833}	\tgf{2635}\tgf{5218}	\tgf{5604}\tgf{5113}	\tgf{4028}	\tgf{1139} \tgf{2455}\tgf{2129}	\tgf{4342}\tgf{4225}\tgf{5113}\tgf{4601}\tgf{3916}\\
%  \ipa{thjɨ²}	\ipa{tja¹}	\ipa{kjɨ¹djɨj²}	\ipa{xjow²tɕhjwo¹}	\ipa{dʑjɨ.wji¹}	\ipa{nji²}	\ipa{.jij¹}	\ipa{gji²bjij²}	\ipa{dja²-sja¹-.wji¹-nja²-sji²} \\
%  \textsc{dem} \textsc{top} certainly Feng.Chang \textsc{erg} \textsc{2sg:hon} \textsc{gen} wife \textsc{pfv}-kill-cause[A]-\textsc{2sg-ifr} \\
%\glt Given all of this, it is certainly Feng Chang (\zh{馮昌}) who had your wife killed. (The Grove of Categories, 06.17A.6)
%\end{exe}

%The phenomenon, already described in \citet{kepping85}, is in no way indicative that Tangut person indexation is recent. Among Gyalrongic languages, some languages like Japhug completely exclude indexation of non-arguments (see section \ref{sec:rgyalrong}), while other languages like Khroskyabs allow it in very specific circumstances (see \citealt{lai15person}). It is unclear which of Khroskyabs or Japhug is more conservative here: pathways in both directions are conceivable, and likewise this property of Tangut person indexation might be an innovation (a type of extreme possessor raising) rather than a sign of recent grammaticalization.

Actually, Tangut texts do show cases where verbs with a clear SAP core argument neither show indexation suffixes nor stem alternation. Although the proponents of the hypothesis that Tangut person indexation is optional have never used such examples and appear to be unaware of their existence, it is a Tangutulogist's duty at this point to mention their existence and explain them.

Example \ref{ex:hongnong} is a typical example of absence of person indexation of some verbs in Tangut. Although the S/A of all verbs in this paragraph is \textsc{2sg}, the verbs \mo{5113} \ipa{.wji¹},  \mo{749} \ipa{phji¹} and \mo{4039} (pronunciation unknown) have neither the suffix \mo{4601} \ipa{nja²} nor stem B as would be expected (there are indicated in red below). Only the last verbal form  \mo{4342}\mo{2511}\mo{4601}\mo{3916} \ipa{dja²}-\ipa{rjɨr²}-\ipa{nja²}-\ipa{sji²} has person indexation (in blue).

\begin{exe}
\ex \label{ex:hongnong}
\glll
\tgf{4028} 	\tgf{2104} 	\tgf{2780}\tgf{5267}\tgf{3831} 	\tgf{5113} 	\tgf{4861} 	\tgf{2302}-\tgf{5880} 	\tgf{4408} 	\tgf{4663}\tgf{749} 	\tgf{2503} 	\tgf{1402}\tgf{3567} 	\tgf{289}\tgf{3266} 	\tgf{795}\tgf{5113} 	\tgf{1477} 	\tgf{4039} 	\tgf{5880} 	\tgf{4342}\tgf{2511}\tgf{4601}\tgf{3916} 	\tgf{3583} 	\tgf{5688} 	\tgf{290} \\
\ipa{nji²} 	\ipa{ɕji¹} 	\ipa{kiow²ljɨj¹}\ipa{ljij²} 	\ipa{.wji¹} 	\ipa{zjọ²} 	\ipa{ljɨ¹}-\ipa{ŋwu²} 	\ipa{məə¹} 	\ipa{lha¹}-\ipa{phji¹} 	\ipa{kụ¹} 	\ipa{xũ¹}\ipa{luu²} 	\ipa{.we²}-\ipa{dzju²} 	\ipa{rjɨr²}-\ipa{.wji¹} 	\ipa{le²} 	\ipa{0} 	\ipa{ŋwu²} 	\ipa{dja²}-\ipa{rjɨr²}-\ipa{nja²}-\ipa{sji²} 	\ipa{tja¹} 	\ipa{.wa²} 	\ipa{sju²}  \\
\textsc{2sg} first Jiangling.Ling \rouge{make[A]} time wind-\textsc{instr} fire extinguish-\rouge{cause[A]} after Hongnong city-lord \textsc{pfv}-\rouge{make[A]} tiger \rouge{ride} \textsc{instr} \bleu{\textsc{pfv}-leave-\textsc{2sg-ifr}} \textsc{top} what be.like \\
\glt `First, when you were Jiangling Ling, you extinguished the fire with the wind, later, when you became Taishou of Hongnong, you went away riding a tiger. How did (you do that)?' (The Grove of Categories, 04.13B.4)
\end{exe} %卿初为江陵令,以风灭火;后为弘农太守,乘虎而去者,是因何德也?

While such example might appear to strike a damaging blow to the hypothesis that Tangut person indexation was \textit{not} optional, it is important to read and understand Tangut texts in the context of its better understood modern relatives, in particular the Gyalrongic languages.\footnote{The Gyalrongic group includes three branches (\citealt{jackson00sidaba, jackson00puxi}), Core Gyalrong (comprising Japhug, Tshobdun, Zbu and Situ), Khroskyabs and Horpa (a subgroup which includes Stau). There is evidence of lexical common innovations linking this group with Tangut and other languages of area (\citealt{jacques14esquisse}), including Naish and Lolo-Burmese, languages that have lost person indexation (\citealt{jacques.michaud11naish}).}

All Gyalrongic languages described up to now have converbial forms that can bear some TAM markers, but no person indexation and evidential markers. In Japhug, these forms are   marked by the nominalization prefixes \ipa{kɤ-} or \ipa{sɤ-}, as in example \ref{ex:mAkApa.kW} (see \citealt{jacques14linking}).

\begin{exe}
\ex \label{ex:mAkApa.kW}
\gll
\ipa{tɕe}   	\ipa{ɯ-ŋgɯ}   	\ipa{nɯ} \ipa{tɕu}   	\ipa{paʁndza}   	\ipa{ɲɤ-raʁ}   	\ipa{tɕe,}   	\ipa{tɕendɤre}   	[<dian>   	<guan>   	\ipa{mɤ-kɤ-βzu}] 	\ipa{\textbf{kɯ}}   	\ipa{mɤ-kɤ-pa}   	\ipa{\textbf{kɯ}}   	\ipa{ɯ-jaʁ}   	\ipa{lo-tsɯm}   \\
\textsc{lnk} \textsc{3sg}-inside \textsc{dem} \textsc{loc} pig.fodder \textsc{evd}-be.stuck \textsc{lnk}
\textsc{lnk} electricity turn.off \textsc{neg-inf}-make \textsc{erg}  \textsc{neg-inf}-close \textsc{erg}  \textsc{3sg.poss}-hand \textsc{evd:upstream}-take.away \\
\glt `Some pig fodder got stuck inside (the machine) she reached her hand into it without turning it off,' (Relatives, 372-3)
\end{exe} 

In Stau, where nominalization prefixes have disappeared (only  fossilized traces of them remain), the converbial forms are marked with suffixes such as \ipa{-dʑə} as in example \ref{tEphjidZE}. Note that the verb form \ipa{tə-pʰji-dʑə} \textsc{pfv}-run.away-\textsc{conv}(1) does not take the first person marker (otherwise \ipa{tə-pʰjã} would be expected, see \citealt{jacques14rtau}) (2) keeps the perfective directional prefix \ipa{tə-}, though cases without directional prefixes are also attested in converbial forms.

\begin{exe}
\ex \label{tEphjidZE}
\gll \ipa{ŋa} 	\ipa{tə-pʰji-dʑə} 	\ipa{arədadu} 	\ipa{ɞrɞ} 	\ipa{la} 	\ipa{ʁə} 	\ipa{rə-ɕã} \\
 \textsc{1sg} \textsc{pfv}-run.away-\textsc{conv} up up pass head \textsc{pfv:up}-go:\textsc{1pl} \\
\glt `I ran away and went up toward the mountain pass up there.' (The hybrid yak, 86) 
\end{exe}

Returning back to example \ref{ex:hongnong}, we see that of the four verb forms without person marking, one takes a directional prefix, one is followed by the instrumental suffix \mo{5880} \ipa{ŋwu²} (actually, a probable cognate of the ergative \ipa{-w} in Stau), and one followed by the relator noun \mo{4861} \ipa{zjọ²} `time; when'. 

What we have here is a \textit{converbial chain}, with four non-finite verbs lacking person markers but taking various TAM and subordination markers (although zero marked converbs are also possible) and the last verb, which has all the person and TAME markers, is the only finite one in the sentence.

My claim is that Tangut person indexation (by which I mean suffixes AND stem alternation) is \textit{not} optional. This claim can be falsified if one finds counterexamples with the following characteristics:

\begin{itemize}
\item The sentence is complete (not from a fragment missing characters, or a sentence at the end of an isolated page).
\item The sentence has a SAP core argument marked by an overt pronoun.
\item The sentence is at the end of a quotation, and cannot be interpreted as a converb.
\end{itemize}

\section{Person indexation in Gyalrong and Kiranti} \label{sec:rgyalrong}
\citet[53]{zeisler15eat} argues  that `It should also be noted that several pronominalising Tibeto-Burman languages do not mark syntactic relations or semantic roles but simply person or, more precisely, speech act participant (1P and/ or 2P) involvement: Tangut, Gyarong, Nocte, Muya, Dulong, Kiranti, Hayu.' 

The notion of SAP involvement refers to LaPolla's (\citeyear[308]{lapolla92}) statement that `agreement in Gyarong is with 1st person any time a 1st person is involved, regardless of its semantic or syntactic function'. In other words, a SAP referent, independently of its syntactic function (argument, adjunct or possessor of an argument), would trigger agreement. 

In this section,  I adduce fresh data from fieldwork on Gyalrongic and Kiranti languages that strongly suggest that this does not reflect the full story, at least as far as Gyalrongic and Kiranti languages are concerned.

It is a fact that \textit{some} person affixes in Gyalrongic and Kiranti languages have the same form whether they refer to S, A or P. For instance, in Japhug the same \textsc{1sg} \ipa{-a} suffix appears to mark S (\ref{ex:pWnWZWBa}), A (\ref{ex:pWmtota}) or P (\ref{ex:pWwGmtoa}).

\begin{exe}
\ex \label{ex:pWnWZWBa}
\gll \ipa{pɯ-nɯʑɯβ-a} \\
\textsc{pfv}-sleep-\textsc{1sg} \\
\glt `I slept'.
\end{exe}

\begin{exe}
\ex \label{ex:pWmtota}
\gll \ipa{pɯ-mto-t-a} \\
\textsc{pfv}-see-\textsc{pst:tr}-\textsc{1sg} \\
\glt `I saw him'.
\end{exe}

\begin{exe}
\ex \label{ex:pWwGmtoa}
\gll \ipa{pɯ́-wɣ-mto-a} \\
\textsc{pfv}-\textsc{inv}-see-\textsc{1sg} \\
\glt `He saw me'.
\end{exe}

However, this does not warrant the general conclusion that  the system of person indexation as a whole does not mark syntactic relations. If this were the case, none of the languages with direct-inverse systems  (including Algonquian, Mapuche and Movima) would mark syntactic relations. In Ojibwe, in the independent order, for instance, the same prefix \ipa{ni-} occurs in exactly the same contexts as the suffix \ipa{-a} in Japhug in examples (\ref{ex:ninibaa}) to (\ref{ex:niwaabamig}). 

\begin{exe}
\ex  \label{ex:ninibaa}
\gll \textit{ni-nibaa} \\
1-sleep  \\
\glt `I sleep.'
\end{exe} 

\begin{exe}
\ex  \label{ex:niwaabamaa}
\gll \textit{ni-waabam-aa} \\
1-see-\textsc{dir} \\
\glt `I see him.'
\end{exe} 

\begin{exe}
\ex   \label{ex:niwaabamig}
\gll \textit{ni-waabam-ig} \\
1-see-\textsc{inv} \\
\glt `He sees me.'
\end{exe} 
 
However, there is clear evidence that Ojibwe morphosyntax is very sensitive to grammatical relations. First, in the conjunct order, the \textsc{1sg} is marked by entirely distinct suffixes (-\textit{yaan}, -\textit{ag} and -\textit{id} respectively for \textsc{1sg.intr}, \textsc{1sg}$\rightarrow$3 and  3$\rightarrow$\textsc{1sg}, see \citealt[295]{valentine01grammar}). Second, strict accusative and ergative syntatic pivots are attested in Ojibwe (\citealt{rhodes94valency}, \citealt[119-126]{zuniga06}). No author has ever concluded from examples such as (\ref{ex:ninibaa}) to (\ref{ex:niwaabamig}) that agreement with \textsc{1sg} occurs in Ojibwe  regardless of syntactic function.

When analyzing complex polypersonal systems, it is important not to focus on individual affixes, whose distribution may in some cases not even be describable in functional terms, but rather to study the whole set of affixes and stem alternations as a complete system. 

In the following, I present evidence showing that person indexation systems in Gyalrongic and Kiranti, as in Ojibwe, do mark grammatical relations, and not simply SAP involvement. First, indexation systems in these languages present numerous affixes which are restricted to S, A or P, or portmanteau markers used in specific configurations. Second, even in the case of most ditransitive verbs in Japhug, the recipient (even if SAP) is not indexed on the verb, and instead the theme (rarely human) is indexed. Third, agreement of the verb with possessors is not attested at least in Japhug and Khaling.

\subsection{Unambiguous marking of S, A and P}
Several affixes or morphological processes in Gyalrongic and Kiranti only occur  in specific configurations of A and P. Here is a list of the most important ones in Japhug (and the same are true of Zbu and Tshobdun, see \citealt{jackson00sidaba, jackson02rentongdengdi} and \citealt{gongxun14agreement}).

\begin{enumerate}
\item The past transitive \ipa{-t} suffix only occurs with a \textsc{1sg} or \textsc{2sg} A and a third person P in the perfective form. It is a portmanteau morpheme indicating both the person of the A and the P as well as TAM.
\item The portmanteau \ipa{kɯ-} and \ipa{ta-} prefixes mark 2$\rightarrow$1 and 1$\rightarrow$2 respectively; if the indexation system only marked SAP involvement these two forms would not be distinguished.
\item In the perfective, when A and P are both third person and no inverse prefix is present, directional prefixes take a special form.
\item Stem III is used in forms with a singular agent and a third person patient in non-past TAM categories.
\end{enumerate}

In Kiranti languages in general, recent work by Balthasar Bickel has brought out the existence of strict ergative \textit{and} accusative alignment in the person indexation systems of various languages. Most spectacularly, \citet{bickel08scope} (see Table \ref{tab:silverstein}, including only singular forms) concludes that in Puma (a Southern Kiranti language) \textsc{1sg} has ergative alignment, \textsc{2sg} neutral alignment, and \textsc{3sg} accusative alignment, the exact opposite of what  Silverstein's Hierarchy (\citealt{silverstein76}) predicts.

\begin{table}[H]
\caption{Counterexample to the Silverstein Hierarchy in Puma} \label{tab:silverstein} \centering
\begin{tabular}{l|c|c|c|l}
\toprule
& A & S & P & alignement\\
\hline
\textsc{1sg}& \ipa{-ŋ} (1$\rightarrow$3)& \multicolumn{2}{|c|}{\ipa{-ŋa} \grise{}(\textsc{npst})}  &ergative\\
 & \ipa{-na} (1$\rightarrow$2)& \multicolumn{2}{|c|}{\ipa{-oŋ} \grise{}(\textsc{pst})} &\\
\hline
 \textsc{2sg} & \multicolumn{3}{c|}{\ipa{tʌ-}}& neutral\\
 \cline{4-4}
 &\multicolumn{2}{c|}{}&\ipa{-na} (1$\rightarrow$2)&\\
 \hline
  \textsc{3sg} & \multicolumn{2}{c|}{$\varnothing$} & \ipa{-u} &accusative\\
   \cline{2-2}
  &\ipa{pʌ-}  (3$\rightarrow$1) & &\\
  \bottomrule
\end{tabular}
\end{table}

Other Kiranti languages like Khaling for instance abound with affixes and stem alternations that mark not only the presence of an SAP argument, but also unambiguously indicate syntactic relations:

\begin{enumerate}
\item The portmanteau \textsc{1sg$\rightarrow$2} non past \ipa{-nɛ} and past \ipa{-tɛni} suffixes are not found in 2$\rightarrow$1 forms.
\item The suffix \ipa{-u}  combined with stem 1 (\citealt[1104]{jacques12khaling}) specifically expresses \textsc{1sg$\rightarrow$3}, and completely differs from the corresponding inverse or intransitive forms.
\item The suffix \ipa{-ʉ}, combined with stem 5, indicates \textsc{3sg} P in the non-past.
\end{enumerate}


\subsection{Ditransitive verbs}
In the case of ditransitive verbs, in Japhug the indexation of the T vs R is not dependent on person hierarchy, but on the configuration of each verb individually. Some ditransitive verbs, like \ipa{mbi}, treat the R like the P of a monotransitive verb, and it is indexed on the verb.\footnote{The noun phrase \ipa{nɤ-rʑaβ} `(as) your wife' is a functive phrase, syntactically an adjunct (see \citealt{creissels14functive}).}

\begin{exe}
\ex \label{ex:YWtambi}
\gll \ipa{a-me} 	\ipa{nɯ} 	\ipa{nɤ-rʑaβ} 	\ipa{ɲɯ-ta-mbi} 	\ipa{ŋu} \\
\textsc{1sg.poss}-daughter \textsc{dem} \textsc{2sg.poss}-wife \textsc{ipfv}-1$\rightarrow$2-give be:\textsc{fact} \\
\glt `I will give you my daughter in marriage.'
\end{exe}

Other ditransitive verbs, which follow indirective alignment, treat the T like the P, while the R cannot be indexed on the verb and receives oblique case, as in \ref{ex:YWkhama}. Using the 1$\rightarrow$2 form \ipa{ɲɯ-ta-kho} \textsc{ipfv}-1$\rightarrow$2-give would mean `I will give you (to him)'.

\begin{exe}
\ex \label{ex:YWkhama}
\gll \ipa{nɤʑɯɣ} 	\ipa{ɲɯ-kham-a} \ipa{ŋu} \\
\textsc{2sg:gen} \textsc{ipfv}-give[III]-\textsc{1sg} be:\textsc{fact} \\
\glt `I will give it to you.'
\end{exe}

The same is true of most ditransitive verbs; another example is \ipa{thu} `ask', as in \ref{ex:tuthea}; the form treating the \textsc{2sg} as the P, \ipa{tu-ta-thu} \textsc{ipfv}-1$\rightarrow$2-give can only mean `I asked you (in marriage, to your father)'.

\begin{exe}
\ex \label{ex:tuthea}
\gll \ipa{nɤ-ɕki} 	\ipa{tu-the-a} \\
\textsc{2sg-dat} \textsc{ipfv}-ask[III]-\textsc{1sg} \\
\glt `I ask you (about it).'
\end{exe}

In examples such as \ref{ex:YWkhama} and \ref{ex:tuthea}, we have a third person T and a second person R. If syntactic relations had no effect on person indexation in Japhug, and only the SAP > 3 hierarchy determined the use of a particular form, we would expect the \textsc{2sg} to be indexed on the verb regardless whether it is T or R.

\subsection{Agreement restrictions}
Verbs in languages such as Japhug and Khaling only agree with core arguments, not with possessor of arguments. Thus, in (\ref{ex:YWmNAm}), the verb \ipa{mŋɤm} cannot take the \textsc{1sg} suffix \ipa{-a} (such a form would be nonsensical).

\begin{exe}
\ex \label{ex:YWmNAm}
\gll \ipa{a-xtu} 	\ipa{ɲɯ-mŋɤm} \\
\textsc{1sg.poss}-belly \textsc{sens}-hurt \\
\glt `My belly hurts.' (Japhug)
\end{exe}

In Khaling, although the verb \ipa{|ŋet|} can be used in the first person form, the third person must be used in example (\ref{ex:amupu}):

\begin{exe}
\ex \label{ex:amupu}
\gll \ipa{ʔʌ-mupu} 	\ipa{ŋêj} \\
\textsc{1sg.poss}-belly hurt:\textsc{3sg:npst} \\
\glt `My belly hurts.' (Khaling)
\end{exe}

Other Kiranti and Gyalrongic languages present similar, but slightly different systems (see for instance Khroskyabs, \citealt{lai15person}), and I would not be surprised to find languages differing from Japhug and Khaling by all three properties described above. 

To labour the point: it appears preferable to avoid lumping all languages with person indexation into one category on the basis of partial paradigms. A promising direction for future work consists in making use of the text corpora that are becoming available in these languages. This method holds great promise for gradual progress in unraveling the complex history of  the tremendous morphosyntactic diversity of the Trans-Himalayan family.

\section{Conclusion}

The increasing body of in-depth linguistic documentation on an ever greater number of Sino-Tibetan languages holds promise of decisive progress, in the next few years, on the debate on the antiquity of person indexation in Trans-Himalayan. One may reasonably hope that Kiranti and Gyalrongic languages will soon be well-documented, the proto-languages of these groups reconstructed, and the mass of data subjected to investigations by a growing part of  the community of Trans-Himalayan linguists.

This will allow the debate to take place on solid ground,  on the basis of \textit{text corpora}, as is done in Indo-European studies, which to this day remain an inspiring example for studies of other language families.

\charis
\bibliographystyle{unified}
\bibliography{bibliogj}

\end{document}