%ɯ-jwaʁ nɯnɯ tɯ-xpɤlɤskɤr anɯrŋi ɕti.
%pɯ-nɯ-mpja nɯ arŋi, pɯ-nɯ-ɣɤndʐo nɯre arŋi, 
%tɤjpa ka-nɯ-lɤt kɯnɤ, mɤ-naʁdɯɣ ma arŋi ɕti qhe, pjɯ-nɯɕɯrɲɟo mɤ-cha
%A 269 ɯ-rŋa ra pjɤ-nɯ-pɕɯz qhe, pjɤ-nɯ-χtɕi qhe,

\documentclass[oldfontcommands,oneside,a4paper,11pt]{article} 
\usepackage{fontspec}
\usepackage{natbib}
\usepackage{booktabs}
\usepackage{xltxtra} 
\usepackage{polyglossia} 
\usepackage[table]{xcolor}
\usepackage{gb4e} 
\usepackage{multicol}
\usepackage{graphicx}
\usepackage{float}
\usepackage{hyperref} 
\hypersetup{bookmarks=false,bookmarksnumbered,bookmarksopenlevel=5,bookmarksdepth=5,xetex,colorlinks=true,linkcolor=blue,citecolor=blue}
\usepackage[all]{hypcap}
\usepackage{memhfixc}
\usepackage{lscape}
 \usepackage{lineno}
\bibpunct[: ]{(}{)}{,}{a}{}{,}
 
%\setmainfont[Mapping=tex-text,Numbers=OldStyle,Ligatures=Common]{Charis SIL} 
\newfontfamily\phon[Mapping=tex-text,Ligatures=Common,Scale=MatchLowercase,FakeSlant=0.3]{Charis SIL} 
\newcommand{\ipa}[1]{{\phon \mbox{#1}}} %API tjs en italique
 \newcommand{\ipab}[1]{{\phon \mbox{#1}}} %API tjs en italique
\newcommand{\grise}[1]{\cellcolor{lightgray}\textbf{#1}}
\newfontfamily\cn[Mapping=tex-text,Ligatures=Common,Scale=MatchUppercase]{MingLiU}%pour le chinois
\newcommand{\zh}[1]{{\cn #1}}

\newcommand{\ro}{$\Sigma$} 

 \begin{document} 
 \title{ The spontaneous-autobenefactive prefix in Japhug Rgyalrong
} 
\author{Guillaume Jacques}
\maketitle
\linenumbers
\sloppy

\section{Introduction}
There is a well-attested cross-linguistic tendency for the same marker to be used for reflexive or passive and for the expression of spontaneous events without a volitional agent or anticausative (\citealt[142-144]{kemmer93middle}). This type of `middle' marking has been described in Romance and Slavic languages and also in various language families, for instance Athabaskan (\citealt{thompson96middle}). 

Japhug presents a marker \ipa{nɯ--} that can be used in situations referring to  spontaneous events or autobenefactive actions. This marker differs from the reflexive as observed in Romance and Slavic in several important ways.

First, Japhug also as a  reflexive prefix (see for instance \citealt{jacques10refl} concerning Japhug Rgyalrong)   as well as distinct passive, anticausative and antipassive prefixes (\citealt{jacques12demotion}) and a reciprocal construction all entirely different from the autobenefactive.


Second, although Japhug  \ipa{nɯ--}  semantically belongs to the ‘middle domain’ as defined by \citet[15]{kemmer93middle}, this marker can be applied to both intransitive and transitive verbs,  and does not modify the valency of the verb.
 
 Third, in addition to marking spontaneous and autobenefactive actions, it also has a permansive aspectual value.
 
 
 This paper is a description of the spontaneous-autobenefactive  \ipa{nɯ--}  in Japhug, and a contribution to the typology of ‘middle’ marking systems. It comprises five sections. First, we present the morphological marking  of transitivity in Japhug, and show that the spontaneous-autobenefactive   marker has no influence on it. Second, we discuss the position of this prefix  in the Japhug verbal template. Then, we describe the three main functions of this prefix: the expression of spontaneous events, autobenefactive and grooming activities and  permansive aspect.
 

\section{Morphological transitivity}

Japhug verbs have two conjugations, transitive and intransitive. The intransitive conjugation indexes the person and number (singular, dual, plural) of the S, while the transitive conjugation indexes the person and number of both A and P. The indexation of arguments on transitive verb follows a quasi-canonical direct-inverse system (see \citealt{jacques10inverse}, \citealt{jacques14inverse}). The person marking prefixes and suffixes of the intransitive conjugation can be combined with either direct marking (via stem alternation), inverse marking (the \ipa{wɣ-} prefix) or portmanteau prefixes (the local scenario markers \ipa{kɯ--} $2\rightarrow1$ and \ipa{ta--} $1\rightarrow2$).

Transitive verbs can be unambiguously distinguished from intransitive ones by three morphological criteria. First, stem alternation in the non-past  \textsc{1/2/3sg}$\rightarrow$3 direct forms (including factual, imperfactive, testimonial, present, imperative and irrealis). Second, presence of a past transitive prefix \ipa{--t--} in the past \textsc{1/2sg}$\rightarrow$3 forms. Third, in perfective 3$\rightarrow$3 direct forms (without the inverse \ipa{wɣ--} prefix), the perfective prefixes have a distinct form.

There is in addition a small class of labile verbs, which can be conjugated either transitively or intransitively (\citealt{jacques12demotion}). All examples exhibit agent-preserving  lability.

In addition, Japhug has ergative alignment on all nominal arguments:  S and P are unmarked (examples \ref{ex:abs} and \ref{ex:erg}), while the A of transitive verbs receives the clitic \ipa{kɯ} (example \ref{ex:erg}). This clitic is obligatory with noun phrases and third person pronouns, but in the case of first and second person pronouns it is optional.  

 \begin{exe}
\ex \label{ex:abs}
\gll
\ipa{rɟɤlpu}  	\ipa{nɯ}  	\ipa{mɯ-pjɤ-rɤʑit}  \\
king \textsc{dem} \textsc{neg-evd.ipfv}-be.there \\
 \glt Le king was not there. (Nyima wodzer2003.1, 18)
\end{exe}

 \begin{exe}
\ex \label{ex:erg}
\gll 
\ipa{rɟɤlpu}  	\ipa{nɯ}  	\ipa{kɯ}  	\ipa{li}  	\ipa{ci}  	\ipa{ɯ-rʑaβ}  	\ipa{kɯ-ɕɤɣ}  	\ipa{ci}  	\ipa{ɲɤ-ɕar.}  	 \\
king \textsc{dem} \textsc{erg} again \textsc{indef} \textsc{3sg.poss}-wife \textsc{nmlz}:S/A-be.new \textsc{indef}  \textsc{evd}-look.for \\
\glt The king married a new wife. (Snow-white, 15)
\end{exe}

 The presence of the autobenefactive/spontaneous prefix has no incidence on  morphological or syntactic transitivity. 
 
 Examples \ref{ex:tAnWndAm} and \ref{ex:ki.tAndAm} show that the prefix \ipa{nɯ--} does not affect stem alternation. In these two examples, stem III   \ipa{--ndɤm} is found instead of stem I \ipa{--ndo}, as expected for a verb is in the imperative with a \textsc{2sg} A and a third person P, even in \ref{ex:tAnWndAm} with the autobenefactive/spontaneous.
 
%ɕoŋtɕa na-fsoʁ qhe tɕe kha ta-nɯ-sɯβzu qhe, kha kɯ-wxtɯ-wxti kɯ-ʑɯ-ʑru ʑo ta-nɯ-sɯβzu
%pjɤ-kɯ-nɯβlu-a tɕe, nɯnɯ kɯ-phɤn sɤznɤ, mɤ́ɣrɤz a-mphɯz ɯ-ntɕhɯr pa-nɯ-phɯt 
%92 tɕe mɤlɤjaʁ nɯ pa-phɯt-nɯ ɲɯ-ŋu,

 \begin{exe}
\ex \label{ex:tAnWndAm}
\gll
\ipa{laʁjɯɣ} 	\ipa{nɤʑo} 	\ipa{tɤ-nɯ-ndɤm} 	\ipa{je} 	\ipa{tɕe,} 	\ipa{aʑo} 	\ipa{jɤɣɤt} 	\ipa{ci} 	\ipa{lu-ɕe-a} 	\\
staff \textsc{2sg} \textsc{imp-auto}-take[III] \textsc{hort} \textsc{lnk} \textsc{1sg} toilet \textsc{indef} \textsc{ipfv:upstream}-go-\textsc{1sg} \\
\glt Take the staff (to hit the animal), I am going to the toilet. (The tiger, 13) 
\end{exe}

 \begin{exe}
\ex \label{ex:ki.tAndAm}
\gll
\ipa{ki}  	\ipa{tɤ-ndɤm}  	\ipa{tɕe,}  	\ipa{koŋla}  	\ipa{ʑo}   \ipa{ɯ-pɯ}  	\ipa{ɯ-pa}  	\ipa{a-tɤ-tɯ-ɣɤ-βdi}  	\ipa{ma}  \\
this \textsc{imp}-take[III] \textsc{lnk} really \textsc{emph} \textsc{3sg.poss}-CP:keep \textsc{3sg.poss}-\textsc{bare.inf}:CP:keep \textsc{irr-pfv-2-caus}-be.well \textsc{lnk} \\
\glt Take this, you will have to keep it well because... (die Gänsemagd, 24)
\end{exe}

Example \ref{ex:tAnWndo} shows that ergative case marking on the A is not lost when the verb has the autobenefactive / spontaneous prefix, and comparison \ref{ex:tAnWndo} between  \ref{ex:tando} reveals that the perfective transitive 3$\rightarrow$3' prefix \ipa{ta--}\footnote{The  form \ipa{ta--} is only found in transitive non-local scenarios without inverse marking. In all other forms, including  intransitive perfective forms (such as \ipa{tɤ-nɯna} \textsc{pfv}-rest `I had a rest'), forms including a SAP argument or inverse, we find \ipa{tɤ--}.  } is found even in cases when \ipa{nɯ--} is prefixed. The verb has stem I \ipa{--ndo} in these two examples as stem III is restricted to non-past tenses, and is not compatible with the perfective.

 \begin{exe}
\ex \label{ex:tAnWndo}
\gll
\ipa{kɯ-nɯmbrɤpɯ} 	\ipa{nɯ} 	\ipa{kɯ} 	\ipa{laʁjɯɣ} 	\ipa{ɯʑo} 	\ipa{ta-nɯ-ndo} 	\ipa{ɲɯ-ŋu.} \\
\textsc{nmlz}:S/A-ride \textsc{dem} \textsc{erg} staff \textsc{3sg} \textsc{pfv}:3$\rightarrow$3'-take  \\ 
\glt The one who rode took the staff (for himself). (The tiger, 14) 
\end{exe}

 \begin{exe}
\ex \label{ex:tando}
\gll 
\ipa{qaʁ} 	\ipa{kɯ-fse} 	\ipa{tsuku} 	\ipa{ta-ndo}  \\
hoe  \textsc{nmlz}:S/A-be.like several \textsc{pfv}:3$\rightarrow$3'-take  \\ 
 \glt He took a hoe and (other tools). (The fox, 79)
\end{exe} 

Finally, examples \ref{ex:kunWrAZia} and \ref{ex:tuixiu.nWBzuta} show that the transitive past \ipa{--t} suffix appears both in verb forms with and without the \ipa{nɯ--} prefix.

 \begin{exe}
\ex \label{ex:kunWrAZia}
\gll 
\ipa{aʑo} 	<tuixiu> 	\ipa{nɯ-nɯ-βzu-t-a} 	\ipa{tɕe} 	\ipa{sɲikuku} 	\ipa{kɯre} 	\ipa{ku-nɯ-rɤʑi-a} \\
\textsc{1sg} retire \textsc{pfv-auto}-do-\textsc{pst:tr-1sg} \textsc{lnk} everyday here \textsc{pres-auto}-stay-\textsc{1sg} \\
\glt Iretired, and (now) stay (at home) here everyday. (Conversation 2013)
\end{exe} 

 \begin{exe}
\ex \label{ex:tuixiu.nWBzuta}
\gll 
\ipa{aʑo} 	\ipa{mɤ-kɯ-mda} 	<tuixiu> 	\ipa{nɯ-βzu-t-a} \ipa{tɕe} 	\ipa{tɤ-nɯna-a} \\
\textsc{1sg} \textsc{neg-inf:stat}-reach retire \textsc{pfv}-do-\textsc{pst:tr-1sg} \textsc{lnk} \textsc{pfv}-rest-\textsc{1sg} \\
\glt I retired early (before the time was reached). (Lhazgron, 73)
\end{exe} 

The examples above prove that verb forms with the autobenefactive / spontaneous present all the morphological  properties of transitive verbs.

\section{Position in the template}
The Japhug verbal template follows the general structure in Table \ref{tab:template:derivational} (see \citealt{jacques12incorp} and \citealt{jacques13harmonization}). 

The autobenefactive / spontaneous prefix is peculiar in that it can occupy two distinct slots depending on the verb form: it can occur  in slot 12, after the passive or denominal \ipa{a--} prefix, but in the case of verbs without \ipa{a--} prefix, it occurs to the left of the reflexive prefix, as in example \ref{ex:nWZGA}.

\begin{exe}
\ex \label{ex:nWZGA}
\gll 
\ipa{tɕe}  	\ipa{ji-kɤ-nɯ-raχtɕɤz}  	\ipa{ra}  	\ipa{jɤ-ɣe-nɯ,}  	\ipa{iʑora}  	\ipa{tu-nɯ-ʑɣɤ-raχtɕɤz-i,}  \\
\textsc{lnk} \textsc{1pl-nmlz:P-auto}-cherish \textsc{pl} \textsc{pfv}-come[II]-\textsc{pl} \textsc{1pl} \textsc{ipfv-auto-refl}-cherish-\textsc{1pl} \\
\glt When people we cherish come, or when we (wish) to treat ourselves, (Tea, 74)
\end{exe}

The prefix \ipa{nɯ--} occurs in slot 12 for all verbs whose stem begins in \ipa{a--}, even when   \ipa{a--} is neither the passive or the denominal at least synchronically. For instance, in verbs such as \ipa{arŋi} `be blue', whose stem is disyllabic (the \ipa{a--} element is not a prefix), the spontaneous prefix in infixed after the \ipa{a--}, as in  example \ref{ex:anWrŋi}.

\begin{exe}
\ex \label{ex:anWrŋi}
\gll 
\ipa{tɤtʰo}  	\ipa{nɯnɯ}  	\ipa{qartsɯmɤftɕar}  	\ipa{ʑo}  	\ipa{a<nɯ>rŋi}  	\ipa{ɕti}  \\
pine \textsc{dem} winter.and.summer \textsc{emph} <\textsc{auto}>be.blue be.\textsc{affirm:fact} \\
\glt The pine (remains) green the whole year. (Pine, 51)
\end{exe}
 
 
 There is one exception to this rule: the verb \ipa{atɯɣ} `meet, run into' (conjugated intransitively), which has two different autobenefactive forms (also intransitive), a regular one \ipa{a<nɯ>tɯɣ} `meet by oneself' (as in \ref{ex:nanWtWGndZi}), and  \ipa{nɯ-ɤtɯɣ}  `run into (by mistake), happen to be in'   (\ref{ex:YWnAtWG}).
 
 \begin{exe}
\ex \label{ex:nanWtWGndZi}
\gll 
\ipa{ndʑi-stɯnmɯ}  	\ipa{nɯ}  	\ipa{pʰama}  	\ipa{pɯ-βgoz}  	\ipa{pɯ-ŋu}  	\ipa{ma}  	\ipa{ʑɤni}  	\ipa{nɯ-a<nɯ>tɯɣ-ndʑi}  	\ipa{pɯ-maʁ}  \\
\textsc{3du.poss}-marriage \textsc{dem} parent \textsc{pfv}-organize \textsc{pst.ipfv}-be \textsc{lnk} \textsc{3du} \textsc{pfv}-<\textsc{auto}>meet-\textsc{du}  \textsc{pst.ipfv}-not.be \\
\glt Their marriage was arranged by they parents, they did not get together by themselves. (elicited)
\end{exe}

 \begin{exe}
\ex \label{ex:YWnAtWG}
\gll 
\ipa{tɕɤndi}  	\ipa{zɯ}  	\ipa{mɤ-kɯ-βɟɤt}  	\ipa{ɯ-rca}  	\ipa{ntsɯ}  	\ipa{ɲɯ-nɯ-ɤtɯɣ}  	\ipa{pjɤ-ŋu.}  \\
west \textsc{loc} \textsc{neg-nmlz}:S/A-obtain \textsc{3sg}-among always \textsc{ipfv-auto}-meet \textsc{evd.ipfv}-be \\
\glt On the other side, he   always happened to be among those who did not get anything (of the food being distributed). (The raven 4.138)
\end{exe}

The verb \ipa{nɯ-ɤtɯɣ}  `run into (by mistake)'  has the autobenefactive prefix \ipa{nɯ--} in the same slot as the homophonous applicative \ipa{nɯ--} (see \citealt{jacques13tropative}). Its meaning   is not completely predictable from the base verb.

Another example of irregularity related to verbs in \ipa{a--}, is the verb \ipa{antsɤndu} `to be exchanged by mistake' (intransitive), which derives from   \ipa{sɤndu} `exchange' (transitive) by a combination of the passive \ipa{a--} and an allomorph \ipa{nt--} of the autobenefactive/spontaneous \ipa{nɯ--}. This verb is anomalous in two regards. First, there is no corresponding simple passive verb *\ipa{asɤndu} `be exchanged' : \ipa{sɤndu} `exchange' is morphologically the causative of \ipa{andu} `be exchanged' (mainly used about money). Second, the allomorph \ipa{nt--} is not found in any other verb; the \ipa{--t--} is epenthetic here, since the cluster \ipa{--ns--} is not attested in Japhug.

Examples like \ipa{nɯ-ɤtɯɣ}  `run into (by mistake)' and \ipa{antsɤndu} `to be exchanged by mistake'   suggest the autobenefactive/spontaneous \ipa{nɯ--}, despite its high productivity, is better treated as a derivational rather than an inflectional morpheme, since (i) the meaning of the derived verb is not always predictable and (ii) there is not always, at least synchronically, a corresponding base verb whose only difference with the derived verb is the absence of autobenefactive/spontaneous prefix.
 
Verbs in \ipa{a--} are not the only ones where the \ipa{nɯ--} prefix is infixed. The irregular existential verbs \ipa{ɣɤʑu} `be there, exist (sensory)'  and \ipa{maŋe} `not exist (sensory)' also take  the spontaneous marker as an infix rather than as a prefix as in \ref{ex:GAnWZu}; note that all prefixes, including the second person \ipa{tɯ--} and the generic \ipa{kɯ--} are infixed in the conjugation of these verbs (see \citealt{jacques12agreement, jacques15generic}).
\begin{exe}
\ex \label{ex:GAnWZu}
\gll 
 \ipa{pakuku}  	\ipa{ʑo}  	\ipa{ju-nɯɕe-nɯ}  	\ipa{tɕe}  	\ipa{nɯtɕu}  	\ipa{li}  	\ipa{ɣɤ<nɯ>ʑu}  	\ipa{ɕti.}  	\\
 every.year \textsc{emph} \textsc{ipfv}-come.back-\textsc{pl} \textsc{lnk} there again <\textsc{auto}>exist:\textsc{sens} be.\textsc{affirm:fact} \\
 \glt They come back every year, and it is still there. (Matsutake, 51)
\end{exe}

  \begin{landscape}
\begin{table}[H]
\caption{The Japhug verbal template }\label{tab:template:derivational}
\begin{tabular}{llllll|llllllll|lllll} \toprule
 
\ipab{a-}  &  	\ipab{mɯ- }   &  	\ipab{ɕɯ-}   &\ipab{tɤ-} &  	\ipab{tɯ-}  &  	\ipab{wɣ-}   &

  	 \grise{\ipab{ʑɣɤ-}}  &  	\grise{\ipab{sɯ-}}  & \grise{\ipab{rɤ-}}& \grise{\ipab{nɤ-}} &   	 \grise{\ipab{a-}}   &  	\grise{\ipab{nɯ-}}  &  	\grise{\ipab{ɣɤ-}}  &  	\grise{\ipab{noun}}    &  	 \begin{math}\Sigma\end{math}    &  	\ipab{-t}  &  	\ipab{-a}  &  	\ipab{-nɯ}   &  \\
   &  	\ipab{mɤ-}   &  	\ipab{ɣɯ-}   &\ipab{pɯ-}&  	  &  	 
    & \grise{ }	  &  	 \grise{ }	  &  	  \grise{ }	  &  	   \grise{ }	&  	\grise{\ipab{sɤ-}}&  \grise{ }	 &  	\grise{\ipab{rɯ-}}  &  	 \grise{ }	  &  	  &  	  &  	  &  	\ipab{-ndʑi} &  \\
  &  	   &     &  etc.	  & & 	  &  	  &  	 & &  	  &  	 & &  etc.	  &  	  &  	  &  	  &  	  &  	  &  \\
1  &  	2  &  	3  &  	4  &  	5  &  	6  &  	7  &  	8  &  	9  &  	10  &  	11  &  	12  &  	13  &  	14  &  	15  & 16 &17&18\\
\bottomrule
\end{tabular}
\end{table}
\begin{multicols}{2}
\begin{enumerate}


\item Irrealis  \ipa{a}--, Interrogative \ipa{ɯ́}--, conative \ipa{jɯ}--
\item negation \ipa{ma}-- / \ipa{mɤ}-- / \ipa{mɯ}-- / \ipa{mɯ́j}--
\item \textbf{Translocative / Cislocative \ipa{ɕɯ}-- and \ipa{ɣɯ}--}
\item Directional prefixes (tɤ- pɯ- lɤ- thɯ- kɤ- nɯ- jɤ-, tu- pjɯ- lu- chɯ- ku- ɲɯ- ju-) permansive nɯ-, apprehensive ɕɯ-
\item Second person (\ipa{tɯ}--, \ipa{kɯ}-- 2>1 and ta- 1>2)
\item Inverse -\ipa{wɣ}- / Generic S/O prefix \ipa{kɯ}-, Progressive \ipa{asɯ}-. 
\item Reflexive \ipa{ʑɣɤ}-- 
\item Causative \ipa{sɯ}--, Abilitative \ipa{sɯ}--
\item  Antipassive  \ipa{sɤ}-- / \ipa{rɤ}--
\item Causative sɯ-/z-/sɯɣ-/ɕɯ-/ɕ-/ɕɯɣ-/ʑ-/ɣɤ-, tropative \ipa{nɤ}--, applicative \ipa{nɯ}--
\item Passive or Intransitive thematic marker \ipa{a}-- / Deexperiencer \ipa{sɤ}--
\item Autobenefactive-spontaneous \ipa{nɯ}--
\item Other derivation prefixes \ipa{nɯ}-- \ipa{ɣɯ}-- \ipa{rɯ}-- \ipa{nɤ}-- \ipa{ɣɤ}-- \ipa{rɤ}--
\item Noun root
\item Verb root 
\item Past 1sg/2sg transitive -\ipa{t} (aorist and evidential)
\item 1sg --\ipa{a}
\item Personal agreement suffixes (--\ipa{tɕi}, --\ipa{ji}, --\ipa{nɯ}, --\ipa{ndʑi})
\end{enumerate}


\end{multicols}
  \end{landscape}
 
 
 
 
 
\section{Spontaneous}
The prefix \ipa{nɯ--} marks spontaneous actions, either  actions occurring without any external cause, or actions occurring against or independently of the volition of the subject S/A.

In the case of animals, plants and inanimate beings, \ipa{nɯ--} can be used to express  their apparent spontaneous growth, as in \ref{ex:YWkWnWBze}.
\begin{exe}
\ex \label{ex:YWkWnWBze}
\gll 
\ipa{tɕe} 	\ipa{zrɯɣ} 	\ipa{nɯ} 	\ipa{tɕe,} 	\ipa{tsuku} 	\ipa{kɯ} 	\ipa{tɯ-pɤcʰaʁ} 	\ipa{ɯ-ŋgɯ} 	\ipa{tu-nɯ-ɬoʁ} 	\ipa{ŋu} 	\ipa{tu-ti-nɯ} 	\ipa{ŋu} 	\ipa{tɕe} 	\ipa{mɤxsi} 	\ipa{ma} 	\ipa{ɯʑo} 	\ipa{ɲɯ-kɯ-nɯ-βze} 	\ipa{ci} 	\ipa{ɲɯ-ɕti} 	\ipa{tɕe,} 	\\
\textsc{lnk} louse \textsc{dem} \textsc{lnk} some \textsc{erg} \textsc{indef.poss}-navel \textsc{3sg}-inside \textsc{ipfv-auto}-come.out be:\textsc{fact} \textsc{ipfv}-say-\textsc{pl}  be:\textsc{fact} \textsc{lnk} \textsc{neg:genr:}A:know \textsc{lnk} \textsc{3sg} \textsc{ipfv-nmlz:S/A-auto}-grow \textsc{indef} \textsc{testim}-be.\textsc{affirm} \textsc{lnk} \\
\glt The louse, some say that it comes from the navel, I don't know, it grows by itself. (louse, 54-55)
\end{exe}

With a human S/A, it can indicate a action performed of one's own volition, without being forced by anything or anyone, as in \ref{ex:pjWnWmtsaRa}.

\begin{exe}
\ex
\gll \label{ex:pjWnWmtsaRa}
\ipa{aʑo} 	\ipa{pjɯ-kɯ-ɣɤrat-a-nɯ} 	\ipa{mɤ-ra} 	\ipa{ma} 	\ipa{aʑo} 	\ipa{pjɯ-nɯ-mtsaʁ-a} 	\ipa{jɤɣ} \\
\textsc{1sg} \textsc{ipfv:down}-2$\rightarrow$1-throw-\textsc{1sg-pl} \textsc{neg-fact}:need because \textsc{1sg} \textsc{neg-ipfv:down-auto}-jump-\textsc{1sg} \textsc{fact}:be.possible \\
\glt You don't need to throw me in there, I will jump of my own free will.
\end{exe}

Somewhat paradoxically, \ipa{nɯ--} can also indicate that an action occurs against the volition of the S/A, as in \ref{ex:panWClWG}. 
% a-mthɯm ʁnɯ-ɣjɤn nɯ-nɯβde-t-a

\begin{exe}
\ex \label{ex:panWClWG}
\gll \ipa{ɯ-qom} 	\ipa{ci} 	\ipa{pa-nɯ-ɕlɯɣ} 	\ipa{ɲɯ-ŋu,} \\
\textsc{3sg.poss}-tear \textsc{indef} \textsc{pfv:3$\rightarrow$3'-auto}-drop \textsc{testim}-be \\
\glt She shed a tear (unvoluntarily).(Kunbzang 228)
\end{exe}

The verb \ipa{jmɯt} `forget' almost always appears with \ipa{nɯ--} in the corpus (in 23 examples out of 28). In the first person, the autobenefactive / spontaneous can be combined with the evidential to insist on the non-volitionality of the action, as in  \ref{ex:mAxsi2}.

\begin{exe}
 \ex \label{ex:mAxsi2}
 \gll
\ipa{mɤ-xsi}  	\ipa{ko,}  	\ipa{nɯra}  	\ipa{ɲɤ-nɯ-jmɯt-a}  \\
\textsc{neg-genr}:know \textsc{sfp} \textsc{dem:pl} \textsc{evd-auto}-forget-\textsc{1pl} \\
\glt I don't know, I forgot those things. (Conversation)
\end{exe}

It could appear contradictory that a single marker has such opposite semantic values. However, in both cases the action takes place against or independently of the will of a particular referent. This referent can be an argument of the sentence, as in examples \ref{ex:panWClWG} and \ref{ex:mAxsi2}, or can be an external referent, without syntactic function in the sentence, as in \ref{ex:pjWnWmtsaRa}. 

 
\section{Autobenefactive}

\begin{exe}
\ex
\gll 
\ipa{nɤ-ku} 	\ipa{pɯ-nɯ-χtɕi} \\
\textsc{2sg.poss}-head \textsc{imp-auto}-wash \\
\glt Wash your head.
\end{exe}

Use of the third person pronouns without the ergative

   \begin{exe}
\ex \label{ex:tunWndzandZi}
\gll  \ipa{nɤ-pi}   	\ipa{ni}   	\ipa{kɯ}   	...   	\ipa{qɤjɣi}   	\ipa{nɯra}   	\ipa{kɯ-mɯm}   	\ipa{ʑɤni}   	\ipa{tu-nɯ-ndza-ndʑi,}   	\ipa{ɯ-rkɯ}   	\ipa{kɤ-kɯ-ɕke}   	\ipa{ra}   	\ipa{aʑo}   	\ipa{ɲɯ́-wɣ-\textbf{mbi}-a-ndʑi,}   	\ipa{cʰa}   	\ipa{ra}   	\ipa{ʑɤni}   	\ipa{ku-nɯ-tsʰi-ndʑi,}   	\ipa{aʑo}   	\ipa{ɯ-ʁɟo}   	\ipa{ɲɯ́-wɣ-\textbf{jtsʰi}-a-ndʑi}   	\ipa{pɯ-ɕti}        \\
\textsc{2sg.poss}-elder.sister \textsc{du} \textsc{erg} ... bread \textsc{top:pl} \textsc{nmlz:stative}-tasty they.\textsc{du} \textsc{ipf}-\textsc{auto}-eat-\textsc{du} \textsc{3sg.poss}-side \textsc{pfv}-\textsc{nmlz:S}-burn \textsc{pl}  I \textsc{pfv}-\textsc{inv}-\textbf{give}-\textsc{1sg}-\textsc{du} alcohol \textsc{pl} they.\textsc{du} \textsc{ipf}-\textsc{auto}-drink-\textsc{du} I \textsc{3sg.poss}-diluted.alcohol \textsc{pfv}-\textsc{inv}-\textbf{give.to.drink}-\textsc{1sg}-\textsc{du} \textsc{pst.ipf}-be.\textsc{assert}  \\
 \glt    `Your two sisters (...) ate the tasty food and gave me the burned part of the bread, drank the alcohol and gave me diluted alcohol to drink.'  (The three sisters, 68).
\end{exe} 

   \begin{exe}
\ex \label{ex:kAnWBzu.mArtaRtCi}
\gll
\ipa{tɕiʑo} 	\ipa{ʁnɯz} 	\ipa{ma} 	\ipa{maŋe-tɕi} 	\ipa{tɕe,} 	\ipa{ʑaka} 	\ipa{kɤ-nɯ-βzu} 	\ipa{mɤ-rtaʁ-tɕi} \\
\textsc{1du} two apart.from not.exist:\textsc{testim}-\textsc{1du} \textsc{lnk} each \textsc{inf-auto}-do \textsc{neg}-be.enough-\textsc{1du} \\
\glt We are only two, we are not enough people to act separately. (The three sisters, 74)
\end{exe} 

pronoun+ \ipa{ʑo}

\begin{exe}
\ex \label{ex:zYWnWrua}
\gll
\ipa{aʑo} 	\ipa{ʑo} 	\ipa{z-ɲɯ-nɯ-ru-a} 	\ipa{ɲɯ-ntshi} \\
\textsc{1sg} \textsc{emph} \textsc{transloc-ipfv-auto}-look-\textsc{1sg} \textsc{testim}-have.to \\
\glt I have to go to have a look by myself. (Hansel und Gretel, 139)
\end{exe} 

possessive:
\begin{exe}
\ex \label{ex:atAtWnWndAm}
\gll
\ipa{nɤʑo} 	\ipa{ɣɯ} 	\ipa{nɤ-nmaʁ} 	\ipa{nɯ} 	\ipa{a-tɤ-tɯ-nɯ-ndɤm} 	\ipa{tɕe,} 	\ipa{a-jɤ-tɯ-nɯɣi} 	\ipa{khɯ}   \\
\textsc{2sg} \textsc{gen} \textsc{2sg.poss}-husband \textsc{dem} \textsc{irr-pfv-2-auto}-take[III]  \textsc{lnk} \textsc{irr-pfv-2}-come.home  be.possible:\textsc{fact} \\
\glt It will be possible for you to take your husband and come back home. (Das singende springende Löweneckerchen, 199)
\end{exe} 


% (possessive)

with imperative: opposite meanings

\section{Permansive}
`still'
\begin{exe}
\ex \label{ex:pjAnWGAwu}
\gll
\ipa{tɕʰeme} 	\ipa{nɯ} 	\ipa{ɲɤ-nɯkʰɤda} 	\ipa{ri,} 	\ipa{mɯ-pjɤ-pʰɤn,} 	\ipa{tɕʰeme} 	\ipa{nɯ} 	\ipa{pjɤ-nɯ-ɣɤwu} 	\ipa{ɕti,} \\
girl \textsc{dem} \textsc{evd}-convince \textsc{lnk} \textsc{neg-evd}-be.efficient girl \textsc{dem} \textsc{evd.ipfv-auto}-cry  be.\textsc{affirm:fact} \\
\glt She (tried to) comfort the girl, but it was for nothing, the girl was still crying. (Bean and linen, 48)
\end{exe} 

\begin{exe}
\ex \label{ex:anWmphWr}
\gll
\ipa{nɯnɯ} 	\ipa{pjɯ-ŋgra} 	\ipa{ɕɯŋgɯ} 	\ipa{tɕe} 	\ipa{tɕe} 	\ipa{ɲɯ-rom} 	\ipa{ɕti} 	\ipa{tɕe,} 	\ipa{ɯ-rɣi} 	\ipa{nɯnɯ} 	\ipa{tɕu} 	\ipa{a-nɯ-mphɯr} 	\ipa{ɕti} \\
\textsc{dem} \textsc{ipfv-anticaus}:make.fall before \textsc{lnk}  \textsc{lnk} \textsc{ipfv}-be.dry be\textsc{:affirm:fact} \textsc{lnk} \textsc{3sg.poss}-seed \textsc{dem} \textsc{loc} \textsc{pass-auto}-wrap:\textsc{fact} be\textsc{:affirm:fact} \\
\glt Before (the flower) falls down, it dries up, and its seed is still wrapped in it. (\ipa{tɤtɕɯβraʁ}, 59)
\end{exe} 

%还是在里面裹着


This permansive value of the autobenefactive/spontaneous probably explains the presence of this prefix the protasis of alternative  (\citealt{ex:pannWri}) and scalar (\citealt{ex:pWnnWtu.kWnA})concessive conditionals (see \citet{jacques14linking}.

\begin{exe}
\ex  \label{ex:pannWri}
\gll
\ipa{tɕe}  	[\ipa{tɯ-sɯm}  	\textbf{\ipa{pɯ-a<nɯ>ri}}]  	\ipa{nɤ}  	\ipa{ju-kɯ-ɕe,}  	[\textbf{\ipa{mɯ-pɯ-a<nɯ>ri}}]  	\ipa{nɤ}  	\ipa{ju-kɯ-ɕe}  	\ipa{pɯ-ra}  \\
\textsc{lnk} \textsc{indef.poss}-mind  \textsc{pfv-<auto>}go[II] \textsc{lnk} \textsc{ipfv-genr}:S/P-go \textsc{neg-pfv-<auto>}go[II] \textsc{lnk} \textsc{ipfv-genr}:S/P-go \textsc{pst.ipfv}-have.to \\
\glt Whether one liked it or not, one had to go. (Relatives, 212)
\end{exe}


 \begin{exe}
\ex  \label{ex:pWnnWtu.kWnA}
\gll
\ipa{nɯ}    	\ipa{li}    	\ipa{ɯ-qa}    	\ipa{ɲɯ-βze}    	\ipa{ɲɯ-ɕti}    	\ipa{ma}    	[\ipa{ɯ-mɯntoʁ}    	\ipa{pɯ-nnɯ-tu}]    	\ipa{kɯnɤ,}    	\ipa{ɯ-rɣi}    	\ipa{ra}    	\ipa{kɤ-mto}    	\ipa{maŋe.}    \\
\textsc{dem} again \textsc{3sg.poss}-foot \textsc{ipfv}-do[III] \textsc{testim}-be:\textsc{affirm} \textsc{lnk} \textsc{3sg.poss}-flower \textsc{pst.ipfv-auto}-exist also \textsc{3sg.poss}-seed \textsc{pl} \textsc{inf}-see not.exist:\textsc{sensory} \\
\glt This one also grows by its root, as even if it has flowers, (I) have never seen its seeds. (\ipa{paʁtsa rna}, 155)
\end{exe}



\begin{exe}
 \ex \label{ex:YWnWNWNu}
 \gll
\ipa{ʑmbɯlɯm}	\ipa{chondɤre}  	\ipa{grɯβgrɯβ}  	\ipa{kɯ-fse}  	\ipa{tɤ-ɬoʁ}  	\ipa{tɕe}  	\ipa{χploʁχploʁ}  	\ipa{kɯ-pa}  
\ipa{tɕe}  	\ipa{ʑɯrɯʑɤri}  	\ipa{ɲɯ-kɯ-nɤwɤt}  	\ipa{nɯ}  	\ipa{ɲɯ-maʁ.}  
\ipa{tɤ-ɬoʁ}  	\ipa{jɤznɤ}  	\ipa{ɲɯ-xtɕi}  	\ipa{laʁma}  	\ipa{nɯ}  	\ipa{kɯ-fse}  	\ipa{ɲɯ-nɯ-ŋɯ\textasciitilde{}ŋu}  	\ipa{qhe}  	\\
type.of.mushroom \textsc{comit} Matsutake \textsc{nmlz}:S/A-be.like \textsc{pfv}-come.out \textsc{lnk} \textsc{ideo:II:}spherical \textsc{nmlz}:S/A-auxiliary \textsc{lnk} progressively \textsc{ipfv}-\textsc{nmlz}:S/A-open.towards.the.exterior \textsc{dem} \textsc{testim}-not.be  \textsc{pfv}-come.out at.the.moment \textsc{testim}-be.small only \textsc{dem} \textsc{nmlz}:S/A-be.like \textsc{testim-auto}-\textsc{emph}\textasciitilde{}be \textsc{lnk} \\
\glt It is not like the \ipa{ʑmbɯlɯm} and the Matsutake, which are spherical when they come out and progressively open towards the exterior. It is just that it is small when it comes out, (otherwise) it is already like that.
 (\ipa{zwɤrqhɤjmɤɣ} 18, 19)
 \end{exe}


\section{Autobenefactive and other derivations}

Japhug has a rich array of derivations (see in particular \citealt{jacques14antipassive}), and several are homophonous with the spontaneous-autobenefactive (\citealt{jacques13tropative}): the applicative \ipa{nɯ--} / \ipa{nɯɣ--}, the denominal \ipa{nɯ--} and the vertitive \ipa{nɯ--}. The first two are unlikely to be historically related to the autobenefactive, due to the wide semantic difference between them.

The vertitive \ipa{nɯ--}\footnote{θhis term is adopted from Siouan linguistics, cf \citet{taylor76motion}.} is exclusively attested with a restricted set of motion verbs, indicated in Table \ref{tab:vertitive}. It implies a motion back to the origin point.
 
\begin{table}
\caption{The vertitive \ipa{nɯ--} prefix in Japhug} \centering \label{tab:vertitive}
\begin{tabular}{lllllllll}
\toprule
Base verb & Meaning & Derived verb & Meaning& \\
\midrule
\ipa{ɕe} & go & \ipa{nɯɕe} & go back (home) & \\
\ipa{ɣi} & come & \ipa{nɯɣi} & come back (home)& \\
\ipa{tsɯm} & take away & \ipa{nɯtsɯm} & take back  (home)& \\
\ipa{ɣɯt} & bring & \ipa{nɯɣɯt} & bring back  (home)& \\
\ipa{no} & chase (cattle) & \ipa{nɯno} & chase back  (home)& \\
\bottomrule
\end{tabular}
\end{table}



vertitive:

\begin{exe}
\ex
\gll
\ipa{iɕqha} 	\ipa{rɟɤlpu} 	\ipa{ɯ-tɕɯ} 	\ipa{nɯ} 	\ipa{kɯ} 	\ipa{tɤɕime} 	\ipa{nɯ,} 	\ipa{ɯʑo} 	\ipa{ɯ-rɟɤlkhɤβ} 	\ipa{nɯ} 	\ipa{tɕu} 	\ipa{jo-nɯ-tsɯm} 	\ipa{qhe} \\
the.aforementioned king \textsc{3sg.poss}-son \textsc{dem} \textsc{erg} girl \textsc{dem} \textsc{3sg} \textsc{3sg.poss}-kingdom \textsc{dem} \textsc{loc} \textsc{evd-vert}-take.away \textsc{lnk} \\
\glt The prince took the girl back to his kingdom. (Snowwhite, 232)
\end{exe}

spontaneous:
\ipa{aʑɯɣ} 	\ipa{nɯ-nɯ-βde-t-a} 	\ipa{nɯ} 	\ipa{ɯ́-ŋu} 	\ipa{tɯ-ci} 	\ipa{kɯ} 	\ipa{tha-nɯ-tsɯm} 	\ipa{nɯ} 	\ipa{ɯ́-ŋu} 


 anticausative
 
 
\begin{table}[H]
\caption{Examples of anticausative in Japhug}\label{tab:anticausative}
\begin{tabular}{lllllllll} \toprule
basic verb  & &derived  verb &\\
\midrule
\ipa{ftʂi}  &	to melt (vt)	&		\ipa{ndʐi}  &	to melt (vi)		\\
\ipa{kio}  &	to cause to drop	&		\ipa{ŋgio}  &	to slip		\\
\ipa{kra}  &		to cause to fall&		\ipa{ŋgra}  &	to fall		\\
\ipa{plɯt}  &	to destroy	&		\ipa{mblɯt}  &	to be destroyed		\\
\ipa{prɤt}  &	to cut	&		\ipa{mbrɤt}  &		to be cut	\\
\ipa{pɣaʁ}  &	to turn over (vt)	&		\ipa{mbɣaʁ}  &		to turn over (vi)	\\
\ipa{qɤt}  &	to separate	&		\ipa{ɴɢɤt}  &	to be separated		\\
\ipa{qʰrɯt}  &	to completely scratch	&		\ipa{ɴɢrɯt}  &	to be completely scratched		\\
\ipa{qrɯ}  &	to cut, to tear, to break	&		\ipa{ɴɢrɯ}  &	to break (vi), be torn		\\
\ipa{tɕɤβ}  &	to burn (vt)	&		\ipa{ndʑɤβ}  &	to be burned		\\
\ipa{tʰɯ}  &	to pitch (tent),  	&		\ipa{ndɯ}  &	to appear (rainbow), 	\\
 &	 to build (road, bridge)	&		   &	  to be built (road, bridge)		\\
\ipa{χtɤr}  &	 to spill	&		\ipa{ʁndɤr}  &		to be spilled	\\
\ipa{tʂaβ}  &	to cause to roll	&		\ipa{ndʐaβ}  &	to roll (vi)		\\
\ipa{qraʁ}  &	to tear	&		\ipa{ɴɢraʁ}  &		to be torn	\\
\ipa{qia}  &	to tear	&		\ipa{ɴɢia}  &		to get loose  	\\
\ipa{qlɯt}  &	to break	&		\ipa{ɴɢlɯt}  &		to be broken	\\
\ipa{sɤpʰɤr}  &	to shake off, to wipe off	&		\ipa{mbɤr}  &	wiped off	 	\\
 \ipa{pri}  &	 to tear	&		\ipa{mbri}  &	to be torn	 	\\
  \ipa{xtʰom}  &	 to put horizontally	&		\ipa{ndom}  &	 	to be horizontal 	\\
  \ipa{tɕɣaʁ}  &	 to squeeze out 	&		\ipa{ndʑɣaʁ}  &	 to be squeezed out	 	\\ 
   \ipa{kɤɣ}  &	 to bend 	&		\ipa{ŋgɤɣ}  &	 to be bent	 	\\ 
   \ipa{qrɤz}  &	 to shave 	&		\ipa{ɴɢrɤz}  &	 	to be shaved 	\\ 
   \ipa{cʰɤβ}  &	 to flatten, to crush 	&		\ipa{ɲɟɤβ}  &	to be crushed, flattened 	 	\\ 
   \ipa{cɯ}  &	 to open 	&		\ipa{ɲɟɯ}  &	 to be opened	 	\\ 
 \bottomrule
\end{tabular}
\end{table}



The prenasalization changes undergone by the \textsc{main consonants} of the onset are the following:
\begin{table}[H] \centering
\caption{Prenasalization} 
\begin{tabular}{lllllllll} \toprule 
basic    &  derived    &\\
\midrule
p(ʰ) & > mb\\
t(ʰ) & > nd\\
tʂ(ʰ) & > ndʐ\\
tɕ(ʰ) & > ndʑ\\
c(ʰ) & > ɲɟ\\
k(ʰ) & > ŋg\\
q(ʰ) & > ɴɢ \\
 \bottomrule
\end{tabular}
\end{table}

The rule of prenasalization changes both unaspirated stop and aspirated unvoiced stops and affricates into the corresponding voiced prenasalized one. The fact that the aspiration contrast is neutralized during the process constitutes additional evidence for the direction transitive > intransitive: the form of the intransitive verb is predictable from the transitive one, but the reverse is not true. 

The \textsc{medial} consonants are not affected by the prenasalization process, but the \textsc{pre-onset} consonants are sometimes lost (in \ipa{xtʰom} > \ipa{ndom} and \ipa{ftʂi} > \ipa{ndʐi}). The only case of preserved \textsc{pre-onset} is that of the Tibetan loanword \ipa{χtɤr} > \ipa{ʁndɤr} mentioned above.


\bibliographystyle{linquiry2}
\bibliography{bibliogj}

 \end{document}
 