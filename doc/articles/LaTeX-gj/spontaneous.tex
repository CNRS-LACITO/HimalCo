%ɯ-jwaʁ nɯnɯ tɯ-xpɤlɤskɤr anɯrŋi ɕti.
%pɯ-nɯ-mpja nɯ arŋi, pɯ-nɯ-ɣɤndʐo nɯre arŋi, 
%tɤjpa ka-nɯ-lɤt kɯnɤ, mɤ-naʁdɯɣ ma arŋi ɕti qhe, pjɯ-nɯɕɯrɲɟo mɤ-cha
%A 269 ɯ-rŋa ra pjɤ-nɯ-pɕɯz qhe, pjɤ-nɯ-χtɕi qhe,

\documentclass[oldfontcommands,oneside,a4paper,11pt]{article} 
\usepackage{fontspec}
\usepackage{natbib}
\usepackage{booktabs}
\usepackage{xltxtra} 
\usepackage{polyglossia} 
\usepackage[table]{xcolor}
\usepackage{gb4e} 
\usepackage{multicol}
\usepackage{graphicx}
\usepackage{float}
\usepackage{hyperref} 
\hypersetup{bookmarks=false,bookmarksnumbered,bookmarksopenlevel=5,bookmarksdepth=5,xetex,colorlinks=true,linkcolor=blue,citecolor=blue}
\usepackage[all]{hypcap}
\usepackage{memhfixc}
\usepackage{lscape}
 \usepackage{lineno}
\bibpunct[: ]{(}{)}{,}{a}{}{,}
 
\setmainfont[Mapping=tex-text,Numbers=OldStyle,Ligatures=Common]{Charis SIL} 
\newfontfamily\phon[Mapping=tex-text,Ligatures=Common,Scale=MatchLowercase,FakeSlant=0.3]{Charis SIL} 
\newcommand{\ipa}[1]{{\phon \mbox{#1}}} %API tjs en italique
 \newcommand{\ipab}[1]{{\phon \mbox{#1}}} %API tjs en italique
\newcommand{\grise}[1]{\cellcolor{lightgray}\textbf{#1}}
\newfontfamily\cn[Mapping=tex-text,Ligatures=Common,Scale=MatchUppercase]{MingLiU}%pour le chinois
\newcommand{\zh}[1]{{\cn #1}}

\newcommand{\ro}{$\Sigma$} 

 \begin{document} 
 \title{ The spontaneous-autobenefactive prefix in Japhug Rgyalrong
} 
\author{Guillaume Jacques}
\maketitle
\linenumbers


\section{Introduction}
There is a well-attested cross-linguistic tendency for the same marker to be used for reflexive or passive and for the expression of spontaneous events without a volitional agent or anticausative (\citealt[142-144]{kemmer93middle}). This type of `middle' marking has been described in Romance and Slavic languages and also in various language families, for instance Althabaskan (\citealt{thompson96middle}). 

Rgyalrong languages present a marker which can be used in situations referring to  spontaneous events or autobenefactive actions. This marker differs from the reflexive as observed in Romance and Slavic in several important ways.

First, Rgyalrong languages also present specifically reflexive (see for instance \citealt{jacques10refl} concerning Japhug Rgyalrong)   as well as distinct
14 passive, anticausative and antipassive prefixes (\citealt{jacques12demotion}) all entirely different from the autobenefactive.


Rgyalrong belong to the ‘middle domain’ as defined by \citet[15]{kemmer93middle}, this marker can be applied to both intransitive and transitive verbs,  and does not modify the valency of the verb.
 
 This paper is a description of the spontaneous-autobenefactive in Japhug Rgyalrong, and a contribution to the typology of ‘middle’ marking systems. It comprises six section. First, we present the morphological marking  of transitivity in Japhug, and show that the spontaneous-autobenefactive   marker has no influence on it. Second, we discuss the position of this prefix   in the Japhug verbal template, and provide a comparison with the closely   related Khroskyabs language. Third, we illustrate the use of the prefix   to express spontaneous events. Fourth, we describe its use for autobenefactive and grooming activities. Finally, we show that the spontaneous autobenefactive is used in various types of clause linking.


\section{Morphological transitivity}


%\citet{jacques13harmonization}

%In Khroskyabs: \citet{lai13affixale}
\section{Position in the template}
The Japhug verbal template follows the general structure in Table \ref{tab:template:derivational} (see \citealt{jacques12incorp} and \citealt{jacques13harmonization}). 

The autobenefactive / spontaneous prefix is peculiar in that it can occupy two distinct slots depending on the verb 
  \begin{landscape}
\begin{table}[H]
\caption{The Japhug verbal template }\label{tab:template:derivational}
\begin{tabular}{llllll|llllllll|lllll} \toprule
 
\ipab{a-}  &  	\ipab{mɯ- }   &  	\ipab{ɕɯ-}   &\ipab{tɤ-} &  	\ipab{tɯ-}  &  	\ipab{wɣ-}   &

  	 \grise{\ipab{ʑɣɤ-}}  &  	\grise{\ipab{sɯ-}}  & \grise{\ipab{rɤ-}}& \grise{\ipab{nɤ-}} &   	 \grise{\ipab{a-}}   &  	\grise{\ipab{nɯ-}}  &  	\grise{\ipab{ɣɤ-}}  &  	\grise{\ipab{noun}}    &  	 \begin{math}\Sigma\end{math}    &  	\ipab{-t}  &  	\ipab{-a}  &  	\ipab{-nɯ}   &  \\
   &  	\ipab{mɤ-}   &  	\ipab{ɣɯ-}   &\ipab{pɯ-}&  	  &  	 
    & \grise{ }	  &  	 \grise{ }	  &  	  \grise{ }	  &  	   \grise{ }	&  	\grise{\ipab{sɤ-}}&  \grise{ }	 &  	\grise{\ipab{rɯ-}}  &  	 \grise{ }	  &  	  &  	  &  	  &  	\ipab{-ndʑi} &  \\
  &  	   &     &  etc.	  & & 	  &  	  &  	 & &  	  &  	 & &  etc.	  &  	  &  	  &  	  &  	  &  	  &  \\
1  &  	2  &  	3  &  	4  &  	5  &  	6  &  	7  &  	8  &  	9  &  	10  &  	11  &  	12  &  	13  &  	14  &  	15  & 16 &17&18\\
\bottomrule
\end{tabular}
\end{table}
\begin{multicols}{2}
\begin{enumerate}


\item Irrealis  \ipa{a}--, Interrogative \ipa{ɯ́}--, conative \ipa{jɯ}--
\item negation \ipa{ma}-- / \ipa{mɤ}-- / \ipa{mɯ}-- / \ipa{mɯ́j}--
\item \textbf{Translocative / Cislocative \ipa{ɕɯ}-- and \ipa{ɣɯ}--}
\item Directional prefixes (tɤ- pɯ- lɤ- thɯ- kɤ- nɯ- jɤ-, tu- pjɯ- lu- chɯ- ku- ɲɯ- ju-) permansive nɯ-, apprehensive ɕɯ-
\item Second person (\ipa{tɯ}--, \ipa{kɯ}-- 2>1 and ta- 1>2)
\item Inverse -\ipa{wɣ}- / Generic S/O prefix \ipa{kɯ}-, Progressive \ipa{asɯ}-. 
\item Reflexive \ipa{ʑɣɤ}-- 
\item Causative \ipa{sɯ}--, Abilitative \ipa{sɯ}--
\item  Antipassive  \ipa{sɤ}-- / \ipa{rɤ}--
\item Causative sɯ-/z-/sɯɣ-/ɕɯ-/ɕ-/ɕɯɣ-/ʑ-/ɣɤ-, tropative \ipa{nɤ}--, applicative \ipa{nɯ}--
\item Passive or Intransitive thematic marker \ipa{a}-- / Deexperiencer \ipa{sɤ}--
\item Autobenefactive-spontaneous \ipa{nɯ}--
\item Other derivation prefixes \ipa{nɯ}-- \ipa{ɣɯ}-- \ipa{rɯ}-- \ipa{nɤ}-- \ipa{ɣɤ}-- \ipa{rɤ}--
\item Noun root
\item Verb root 
\item Past 1sg/2sg transitive -\ipa{t} (aorist and evidential)
\item 1sg --\ipa{a}
\item Personal agreement suffixes (--\ipa{tɕi}, --\ipa{ji}, --\ipa{nɯ}, --\ipa{ndʑi})
\end{enumerate}


\end{multicols}
  \end{landscape}
\section{Non-productive}
 
 \ipa{atɯɣ} $\rightarrow$ \ipa{nɤtɯɣ}
 
 ndi na-ʑa-nɯ qhe,
tɕɤndi zɯ mɤ-kɯ-βɟɤt tɤ-rca ntsɯ ɲɯ-nɤtɯɣ pjɤ-ŋu. (750)

 \ipa{sɤndu}  $\rightarrow$ \ipa{antsɤndu} `to be exchanged by mistake'
\section{Spontaneous}

not on purpose:


\begin{exe}
\ex
\gll \ipa{ɯ-qom} 	\ipa{ci} 	\ipa{pa-nɯ-ɕlɯɣ} 	\ipa{ɲɯ-ŋu,} \\
\textsc{3sg.poss}-tear \textsc{indef} \textsc{pfv:3$\rightarrow$3'-auto}-drop \textsc{testim}-be \\
\glt She shed a tear (unvoluntarily).(Kunbzang 228)
\end{exe}


on its own (of one's own volition:

\begin{exe}
\ex
\gll 
\ipa{aʑo} 	\ipa{pjɯ-kɯ-ɣɤrat-a-nɯ} 	\ipa{mɤ-ra} 	\ipa{ma} 	\ipa{aʑo} 	\ipa{pjɯ-nɯ-mtsaʁ-a} 	\ipa{jɤɣ} \\
\textsc{1sg} \textsc{ipfv:down}-2$\rightarrow$1-throw-\textsc{1sg-pl} \textsc{neg-fact}:need because \textsc{1sg} \textsc{neg-ipfv:down-auto}-jump-\textsc{1sg} \textsc{fact}:be.possible \\
\glt You don't need to throw me in there, I will jump of my own free will.
\end{exe}


\section{Autobenefactive}

\begin{exe}
\ex
\gll 
\ipa{nɤ-ku} 	\ipa{pɯ-nɯ-χtɕi} \\
\textsc{2sg.poss}-head \textsc{imp-auto}-wash \\
\glt Wash your head.
\end{exe}

Use of the third person pronouns without the ergative

   \begin{exe}
\ex \label{ex:secundative}
\gll  \ipa{nɤ-pi}   	\ipa{ni}   	\ipa{kɯ}   	...   	\ipa{qɤjɣi}   	\ipa{nɯra}   	\ipa{kɯ-mɯm}   	\ipa{ʑɤni}   	\ipa{tu-nɯ-ndza-ndʑi,}   	\ipa{ɯ-rkɯ}   	\ipa{kɤ-kɯ-ɕke}   	\ipa{ra}   	\ipa{aʑo}   	\ipa{ɲɯ́-wɣ-\textbf{mbi}-a-ndʑi,}   	\ipa{cʰa}   	\ipa{ra}   	\ipa{ʑɤni}   	\ipa{ku-nɯ-tsʰi-ndʑi,}   	\ipa{aʑo}   	\ipa{ɯ-ʁɟo}   	\ipa{ɲɯ́-wɣ-\textbf{jtsʰi}-a-ndʑi}   	\ipa{pɯ-ɕti}        \\
\textsc{2sg.poss}-elder.sister \textsc{du} \textsc{erg} ... bread \textsc{top:pl} \textsc{nmlz:stative}-tasty they.\textsc{du} \textsc{ipf}-\textsc{auto}-eat-\textsc{du} \textsc{3sg.poss}-side \textsc{pfv}-\textsc{nmlz:S}-burn \textsc{pl}  I \textsc{pfv}-\textsc{inv}-\textbf{give}-\textsc{1sg}-\textsc{du} alcohol \textsc{pl} they.\textsc{du} \textsc{ipf}-\textsc{auto}-drink-\textsc{du} I \textsc{3sg.poss}-diluted.alcohol \textsc{pfv}-\textsc{inv}-\textbf{give.to.drink}-\textsc{1sg}-\textsc{du} \textsc{pst.ipf}-be.\textsc{assert}  \\
 \glt    `Your two sisters (...) ate the tasty food and gave me the burned part of the bread, drank the alcohol and gave me diluted alcohol to drink.'  (The , 68).
\end{exe} 


qɤjɣi kɯnɤ kɯ-mpɯ ra ɯʑo tu-nɯ-ndze,
ɯ-rkɯ kɤ-kɯ-ɕke ra qaɕpa nɯ ɲɯ-mbi pjɤ-ɕti,
49 the frog

tɕiʑo ʁnɯz ma maŋe-tɕi tɕe, ʑaka kɤ-nɯ-βzu mɤ-rtaʁ-tɕi
74
\section{Causal clause linking}


\section{Iterated action}

non-goal oriented iterated action, aimless motion

βɟi > nɤβɟɯβɟi


\bibliographystyle{linquiry2}
\bibliography{bibliogj}

 \end{document}
 