\documentclass[oneside,a4paper,11pt]{article} 
\usepackage{fontspec}
\usepackage{natbib}
\usepackage{booktabs}
\usepackage{xltxtra} 
 \usepackage{geometry}
 \geometry{
 a4paper,
 total={210mm,297mm},
 left=20mm,
 right=20mm,
 top=20mm,
 bottom=20mm,
 }
\usepackage{polyglossia} 
\usepackage[table]{xcolor}
\usepackage{gb4e} 
\usepackage{multicol}
\usepackage{graphicx}
\usepackage{float}
\usepackage{hyperref} 
\hypersetup{bookmarks=false,bookmarksnumbered,bookmarksopenlevel=5,bookmarksdepth=5,xetex,colorlinks=true,linkcolor=blue,citecolor=blue}
\usepackage[all]{hypcap}
\usepackage{memhfixc}
\usepackage{lscape} 
 \usepackage{multicol}
 \usepackage{amssymb}
%\setmainfont[Mapping=tex-text,Numbers=OldStyle,Ligatures=Common]{Charis SIL} 
\newfontfamily\phon[Mapping=tex-text,Ligatures=Common,Scale=MatchLowercase]{Charis SIL} 
\newcommand{\ipa}[1]{{\phon\textit{#1}}} 
\newcommand{\grise}[1]{\cellcolor{lightgray}\textbf{#1}}
\newfontfamily\cn[Mapping=tex-text,Ligatures=Common,Scale=MatchUppercase]{SimSun}%pour le chinois
\newcommand{\zh}[1]{{\cn #1}}
\newcommand{\Y}{\Checkmark} 
\newcommand{\N}{} 
\newcommand{\refb}[1]{(\ref{#1})}
\newcommand{\tld}{\textasciitilde{}}
\XeTeXlinebreaklocale "zh" %使用中文换行 
\XeTeXlinebreakskip = 0pt plus 1pt % 

 \begin{document} 
\title{Associated motion in Sino-Tibetan/Trans-Himalayan}
%\author{Guillaume Jacques\\ CNRS-CRLAO-INALCO}
\maketitle
\textbf{Abstract}: While the term `associated motion' has up to now only been applied to a relatively limited number of Sino-Tibetan languages (Japhug and Karbi, see \citealt{jacques13harmonization} and \citealt{konnerth15cisloc}), the phenomenon itself is not rare in this family. This paper comprises three sections. 

First, I provide an overview of associated motion morphology in Sino-Tibetan, focusing especially on Kiranti and Gyalrongic, the two subgroups of the family where this morphology is most complex,  on the basis of fieldwork data and published literature (in particular \citealt{driem87}, \citealt{driem93dumi}, \citealt{doornenbal09} and \citealt{schackow15yakkha}). 

Second, I present a preliminary typological comparison of Associated Motion in Sino-Tibetan with South America (\citealt{guillaume16am}) and Australia (\citealt{koch84associated.motion}) where this grammatical category has been identified and studied for a longer time. 

Third, I discuss the sources of associated motion affixes in Sino-Tibetan, in particular the question whether these affixes have origins other than motion verbs.


\bibliographystyle{unified}
\bibliography{bibliogj}

 \end{document}
 