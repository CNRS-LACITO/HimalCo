\documentclass[oneside,a4paper,11pt]{article}  
\usepackage{fontspec}
\usepackage{natbib}
\usepackage{booktabs}
\usepackage{xltxtra} 
 \usepackage{geometry}
 \geometry{
 a4paper,
 total={210mm,297mm},
 left=20mm,
 right=20mm,
 top=30mm,
 bottom=30mm,
 }
\usepackage{polyglossia} 
\usepackage[table]{xcolor}
\usepackage{gb4e} 
\usepackage{multicol}
\usepackage{graphicx}
\usepackage{float}
\usepackage{hyperref} 
\hypersetup{bookmarks=false,bookmarksnumbered,bookmarksopenlevel=5,bookmarksdepth=5,xetex,colorlinks=true,linkcolor=blue,citecolor=blue}
\usepackage[all]{hypcap}
\usepackage{memhfixc}
\usepackage{lscape} 
 \usepackage{multicol}
 \usepackage{amssymb}
 \usepackage{bbding}
 
\setmainfont[Mapping=tex-text,Numbers=OldStyle,Ligatures=Common]{Charis SIL} 
\newfontfamily\phon[Mapping=tex-text,Ligatures=Common,Scale=MatchLowercase]{Charis SIL} 
\newcommand{\ipa}[1]{{\phon\textit{#1}}} 
\newcommand{\grise}[1]{\cellcolor{lightgray}\textbf{#1}}
\newfontfamily\cn[Mapping=tex-text,Ligatures=Common,Scale=MatchUppercase]{STHeiti}%pour le chinois
\newcommand{\zh}[1]{{\cn #1}}
\newcommand{\Y}{\Checkmark} 
\newcommand{\N}{\XSolidBrush} 
\newcommand{\tld}{\textasciitilde{}}
\XeTeXlinebreaklocale "zh" %使用中文换行 
\XeTeXlinebreakskip = 0pt plus 1pt % 
\newcommand{\forme}[1]{\textit{\phon#1}}  
\newcommand{\japhug}[2]{\textit{\phon#1} `#2'}  
 \newcommand{\redp}{\textasciitilde}
   \newcommand{\sens}[1]{‘#1’}
    \newcommand{\bleu}[1]{{\color{blue}#1}}
\newcommand{\rouge}[1]{{\color{red}#1}} 
 \newcommand{\fl}{$\rightarrow$}
 \newcommand{\dhatu}[2]{|\ipa{#1}| `#2'}
 \newcommand{\gray}[1]{\cellcolor{lightgray!30}{#1}}
%  \newcommand{\lightgray}[1]{\cellcolor{lightgray!15}{#1}}
  \bibpunct[: ]{(}{)}{,}{a}{}{,}
  
 \begin{document} 
\title{Associated motion in Sino-Tibetan/Trans-Himalayan\footnote{The Japhug, Situ, Khaling and Thulung data are from the authors' fieldwork. The Japhug and Khaling examples are taken from corpora that are progressively being made available on the Pangloss archive (\citealt{michailovsky14pangloss},  
 \url{http://lacito.vjf.cnrs.fr/pangloss/corpus/list\textunderscore rsc.php?lg=Japhug} and  \url{http://lacito.vjf.cnrs.fr/pangloss/corpus/list\textunderscore rsc.php?lg=Khaling}).}}
\author{Guillaume Jacques (CNRS-CRLAO)\\ Aimée Lahaussois (CNRS-HTL) \\ Zhang Shuya (INALCO-CRLAO)}
\maketitle
%\textbf{Abstract}: While the term `associated motion' has up to now only been applied to a relatively limited number of Sino-Tibetan languages (Japhug and Karbi, see \citealt{jacques13harmonization} and \citealt{konnerth15cisloc}), the phenomenon itself is not rare in this family. This paper comprises three sections. 
%
%First, I provide an overview of associated motion morphology in Sino-Tibetan, focusing especially on Kiranti and Gyalrongic, the two subgroups of the family where this morphology is most complex,  on the basis of fieldwork data and published literature (in particular \citealt{driem87}, \citealt{driem93dumi}, \citealt{doornenbal09} and \citealt{schackow15yakkha}). 
%
%Second, I present a preliminary typological comparison of Associated Motion in Sino-Tibetan with South America (\citealt{guillaume16am}) and Australia (\citealt{koch84associated.motion}) where this grammatical category has been identified and studied for a longer time. 
%
%Third, I discuss the sources of associated motion affixes in Sino-Tibetan, in particular the question whether these affixes have origins other than motion verbs.


\section{Introduction}

This article aims to examine associated motion (henceforth AM) in the Sino-Tibetan languages and to determine the distribution of the phenomenon across a number of language groups of the family.  We also aim to look into the contribution that the study of the phenomenon in Sino-Tibetan languages can bring to a wider typological discussion of this linguistic category, through the consideration of additional parameters and questions that have not yet been taken up in the existing literature.  
%The language groups considered in this study are Gyalrongic, Kiranti, Sinitic, as well as a few additional languages where AM has been found.  For each of the language groups, we examine the main characteristics of AM marking, highlighting what is distinct from the presentation of the phenomenon in the other ST groups in the sample.  We find that AM is present in a number of groups within the ST family, but that the configurations of AM are quite different from one group to the next.  The comparison grid that we apply in looking at the various language groups is based on  previously identified parameters of AM which have been described in the literature for different linguistic areas (in particular \citealt{koch84associated.motion}, \citealt{wilkins91associated.motion} and \citealt{guillaume16am}), namely temporal relation, deixis and argument of motion, in addition to a few extra parameters which are particularly relevant for the Sino-Tibetan languages (Vertical dimension, Mono- vs pluriactionality and Volitionality)
%
%The vertical dimension is a parameter which is particularly relevant to the Kiranti subgroup, with most of the languages in the group have AM markers which encode  'upwards', `horizontal' and 'downwards' orientation into motion verbs.
%%The non-AM meanings are of interest for languages which do not have devoted markers; in a number of languages (again, particularly in the Kiranti group), the same marker can sometimes encode motion while at other times encoding other meanings, including orientation without motion, aspect, person configuration and even voice.
%
%Another parameter we have found to be relevant in the Sino-Tibetan languages is whether the verb+AM marker unit is mono- or pluri-actional, namely whether the motion event can be separated out from the main action or not.  A number of tests are proposed to determine the inseparability of motion event and verbal action in different languages in § \ref{sec:am.japhug}. Volitionality is considered in § \ref{sec:volitionality}.
%
%In this survey, we consider all markers for which examples show AM, even if the primary function of the marker is to express some other feature, most commonly aspect or orientation.  We do not, however, take into account markers which do not encode any AM -- even though these may in many cases come from source verbs which indicate motion --, or those for which the examples are questionable as to the interpretation that should be given to the marker in question.  In working through the grammars of languages in our sample, we have identified a number of issues in the presentation of material making it difficult to determine whether or not AM is present in a language, and we thus provide a list of suggestions for fieldworkers in order to better capture the phenomenon during future data collection.

\section{Definition of Associated Motion}

The term `Associated' motion was originally proposed to describe a morphological category in Arandic languages (\citealt{koch84associated.motion} and \citealt{wilkins91associated.motion}) and was more recently extended to the description of various languages of South America, in particular Tacanan and Arawakan (\citealt{guillaume09mouv.assoc}, \citealt{rose15am}, \citealt{guillaume16am}) and also to some Sino-Tibetan languages (\citealt{jacques13harmonization}, \citealt{konnerth15cisloc}).

The present work adopts Guillaume's (\citeyear[13]{guillaume16am}) definition of associated motion in (\ref{ex:AM.def}).


\begin{exe}
\ex \label{ex:AM.def}
 \glt An AM marker is a grammatical morpheme that is associated with the verb and that has among its possible functions the coding of translational motion.
\end{exe}

A clear example of AM marker in a ST language is provided for instance by the cislocative \ipa{ɣɯ-} prefix in Japhug (colored in red), as in (\ref{ex:GWtundze}) and (\ref{ex:amAGWkWsWzdWGa}). 
 
\begin{exe}
\ex \label{ex:GWtundze}
 \gll  \ipa{qajɯ}   \ipa{ra}  \ipa{tu-ndze} \ipa{ma} \ipa{tɤ-rɤku}  \ipa{\rouge{ɣɯ}-tu-ndze} \ipa{mɤ-ŋgrɤl} \\
bugs \textsc{pl} \textsc{ipfv}-eat[III] \textsc{lnk} \textsc{indef}.\textsc{poss}-crops   \textsc{\rouge{come\&do}}-\textsc{ipfv}-eat[III] \textsc{neg}-be.usually.the.case:\textsc{fact} \\
\glt `It eats bugs, and does not \rouge{come and} eats the crops.' (24-ZmbrWpGa, 129) (Japhug)
\end{exe}

\begin{exe}
\ex \label{ex:amAGWkWsWzdWGa}
 \gll     \ipa{nɯ}   \ipa{ma}   \ipa{aʑo}   \ipa{a-mɤ-\rouge{ɣɯ}-kɯ-sɯzdɯɣ-a}   \ipa{ra} \\
 \textsc{dem} apart.from \textsc{1sg} \textsc{irr}-\textsc{neg}-\textsc{\rouge{cisloc}}-2$\rightarrow$1-bother-\textsc{1sg} have.to:\textsc{fact} \\
 \glt `Don't \rouge{come to} bother me again!' (150901 changfamei, 166)
\end{exe}

This prefix expresses a motion event linked to the action of the main verb, but nevertheless distinct from it, that involves not simply a body part of the referent (unlike orientation markers, see below), but the translation of its whole body from one place to the other. It is distinct from orientation markers (which occupy a different slot, see \ref{sec:orientation.japhug}) and from a motion verb in a purposive construction.

While the definition in (\ref{ex:AM.def}) does not specifically exclude motion auxiliaries and clitics, the present work, like \citet[19-20]{guillaume16am}, focuses on AM markers that are verbal affixes. Potential examples of AM clitics and auxiliaries are briefly treated in section XXX.

%My search for AM systems in South American languages abides by the definition formulated in (9), with one restriction, namely that I have only been concerned with AM expressed by affixes, i. e., excluding clitics, particles, or auxiliaries (which are not excluded by the definition, since they also represent grammatical categories).

In existing grammars of ST languages, AM markers are almost never described as a distinct class, but treated in chapters discussing morphemes expressing orientation or even aspect. The terms `directionals' or `directives' (see for instance \citealt{delancey85analysis}) encompass the domain of AM, but do not distinguish it from other notions such as orientation.

In order to clarify the sampling method used in this paper, and also to provide a terminology that may be useful for fieldwork linguistics, we provide here a list of criteria to sort out what markers and constructions are analyzed as AM, in particular purposive motion verb construction (MVC) and orientation markers.

\subsection{Purposive motion verb construction}
Most, if not all languages of the ST family, have at least one supine/purposive motion verb construction (henceforth MVC), with the main verb in non-finite form, and the motion verb is the syntactic head of the clause. As an example, in the case of Japhug, compare the AM forms in  (\ref{ex:GWtundze})  and (\ref{ex:amAGWkWsWzdWGa}) above with the MVC as in (\ref{ex:kWrABzjoz}). 

\begin{exe}
\ex \label{ex:kWrABzjoz}
 \gll \ipa{tɕe} \ipa{mbarkʰom} \ipa{kɯ-rɤ-βzjoz} \ipa{tʰɯ-ɣe-a} \\
\textsc{lnk} Mbarkham \textsc{nmlz}:S/A-\textsc{antipass}-\textsc{learn} \textsc{pfv}:\textsc{downstream}-come[II]-\textsc{1sg} \\
 \glt `I came to Mbarkham to study.' (140501 tshering skyid, 49) (Japhug)
 \end{exe}
 
In (\ref{ex:kWrABzjoz}), the verb expressing the main action is a distinct word in non-finite form (in this particular case, a subject participle), and the motion verb is the main verb of the sentence and takes the full TAME and person inflexion, in particular here the first singular suffix \ipa{-a}. In (\ref{ex:amAGWkWsWzdWGa}), by contrast, the indexation affixes (\textsc{1sg} suffix and 2$\rightarrow$1 portmanteau prefix) are directly adjacent to the verb stem, and the AM marker has no verbal properties, in particular no indexation affixes.

In some languages, the motion verb and the main verb of the purposive clause, when adjacent, may become  a single phonological constituent and undergo sandhi phenomena. This is not however a criterion in itself sufficient to analyze the motion verb as an AM marker, if that motion verb retains full verbal morphology, and if the verb in the purposive clause is non-finite.

For instance, in Northern Pumi, \citet[§ 5.2]{jacques11pumi.tone} notes sandhi phenomena specifically affecting verbs followed by the verb `go', and analyzes the two as one phonological constituent. \citet{daudey14grammar} treats the motion verb `go' as a distinct word, though she also notes the existence of sandhi. In any case, even if the group [purposive verb + motion verb] is treated as a phonological word, the facts that (i) prefixes of the motion verb can occur between the two verb roots, as in (\ref{ex:miCE}) and that (ii) person and TAME inflection only occurs on the motion verb, never on the verb in the purposive clause, as in (\ref{ex:khEtCi}), preclude to analyze the motion verb here as an AM marker, since it does not count as a grammatical morpheme.
 

\begin{exe}
\ex \label{ex:miCE}
 \gll   \ipa{ɐ́=bú} \ipa{ə́-wù}  \ipa{tʰútù} \ipa{tú} \ipa{mí=ɕə̂} \\
 \textsc{1sg}=\textsc{top} that-in   immediately look \textsc{neg}:\textsc{pfv}=go \\
 \glt ‘I did not go to look' (\citealt[364]{daudey14grammar})
\end{exe}

\begin{exe}
\ex \label{ex:khEtCi}
 \gll  \ipa{ɡʉtɑpǽŋ=ɡæ}  \ipa{qʰu}  \ipa{kʰə̌-tɕì}  \ipa{ɕîŋ}  \ipa{sæ̂} \\
flat.stone=\textsc{gen} \textsc{top} \textsc{out}-pour go:\textsc{imp}:\textsc{pl} \textsc{confirm}   \\
\glt `Go and pour them on top of that flat stone' (\citealt[568]{daudey14grammar})
\end{exe}

While the distinction between AM and motion verb phonologically fused with an adjacent non-finite verb is quite trivial in the case of languages with inflectional morphology, it can be less clear with more isolating languages, and borderline cases are presented in section XXX.

\subsection{Orientation}
The terms `directional' has been used in the literature to refer to quite distinct phenomena, including AM, but also static orientation and even person configuration (\citealt[20]{zuniga06}). 

In some languages, in particular Gyalrongic and neighbouring languages, directional prefixes mark orientation  and TAME, but cannot express motion by themselves. In Japhug for instance, all finite verb forms require a directional prefix from six distinct orientations (up, down, upstream, downstream, east, west; in addition, there is a series of unspecified orientation prefixes used with some verbs, see \citealt{jacques17sketch}). Motion verbs and verbs of manipulation are compatible with all directional prefixes, but other verbs generally have one or two fixed possible lexicalized orientations. For instance, the verb \ipa{tsʰi} `drink' is generally used with the `towards east; centripete' prefixes, though it does occur with the `downstream' prefixes in the meaning `drink from a straw' and with the `down' prefixes to express the meaning `drink from below (as a animal)' as in (\ref{ex:WtAlu}). 

  \begin{exe}
\ex \label{ex:WtAlu}
 \gll \ipa{tɕe} 	\ipa{ɯ-tɤ-lu} 	\ipa{pjɯ́-wɣ-rku} 	\ipa{tɕe} 	\ipa{nɯnɯ} 	\ipa{pjɯ-tsʰi} 	\ipa{qʰe,} 	\ipa{tɯ-sŋi} 	\ipa{tɕe} 	\ipa{tɯ-kʰɯtsa} 	\ipa{jamar} 	\ipa{tɯ-rdoʁ} 	\ipa{kɯ} 	\ipa{pjɯ-tsʰi} 	\ipa{ɲɯ-cʰa.}  \\
\textsc{lnk} \textsc{3sg.poss}-\textsc{indef.poss}-milk \textsc{ipfv}:\textsc{down}-\textsc{inv}-put.in  \textsc{lnk} \textsc{dem} \textsc{ipfv}:\textsc{down}-drink \textsc{lnk} one-day \textsc{lnk} one-bowl about one-\textsc{cl} \textsc{erg} \textsc{ipfv}:\textsc{down}-drink \textsc{sens}-can \\
\glt `People pour drink for it (the cat) to drink, and it drinks it, one cat can drink about a bowl of milk per day.' (21-lWLU, 45)
  \end{exe}

The fact that the form \ipa{pjɯ-tsʰi} absolutely cannot be interpreted as meaning `it goes down and drinks it' shows that the directional prefixes in Japhug exclusively express orientation and not AM. To test the difference between the two, it is important \textsc{not} to use motion verbs or verb of manipulation that have an intrinsic motion, like  \ipa{ɣɯt} `bring', which means `bring down' when used with the same prefix as in (\ref{ex:pjWGWta}).

\begin{exe}
\ex \label{ex:pjWGWta}
 \gll \ipa{sci} 	\ipa{tɕe} 	\ipa{pjɯ-ɣɯt-a} 	\ipa{ŋu}  \\
 be.born:\textsc{fact}  \textsc{lnk} \textsc{ipfv}:\textsc{down}-bring-\textsc{1sg} be:\textsc{fact} \\
\glt `When he will be born, I will bring him down (from heaven).'(150828 donglang)
\end{exe}
 
  
\section{Gyalrongic}

\subsection{General overview}
Associated motion prefixes are reported in Japhug (\citealt{jacques13harmonization}), Zbu (\citealt{gong18these}), Tshobdun (\citealt{jackson14morpho}) and Situ (\citealt[200-204]{zhang16bragdbar}, \citealt[497-500]{prins16kyomkyo}, \citealt{linyj17space}), but not found in other Gyalrongic languages (for example in Khroskyabs, cf \citealt{lai17khroskyabs}).
 
%The main characteristics which set apart AM in Gyalrongic languages within the Sino-Tibetan context are the following:
%
%\begin{enumerate}
%\item The languages have devoted AM prefixes.
%\item AM prefixes and orientation prefixes co-occur, filling distinct templatic slots.
%\item AM marking is associated with echo phenomena, with multiple marking coding a single motion event.
%\item Verbs with AM marking are mono-actional (for Japhug).
%\end{enumerate}
%
%These characteristics will be discussed in detail after an overview presenting how the main parameters of temporal relation, deixis and argument of motion manifest for Gyalrongic languages.

 \subsubsection{Temporal Relation}
AM prefixes in Japhug and other Gyalrongic languages refer to a motion event occurring before the action of the main verb, resulting in a prior temporal relation with respect to the main verb, as in  (\ref{ex:GWpjWnWtshinW}) and (\ref{ex:CpjAnWtshi}).

\begin{exe}
\ex \label{ex:GWpjWnWtshinW}
\gll \ipa{tɕe}	\ipa{tɯ-ci}	\ipa{\rouge{ɣɯ}-pjɯ-nɯ-tsʰi-nɯ} \\
\textsc{lnk} \textsc{indef}.\textsc{poss}-water \rouge{\textsc{come\&do}}-\textsc{ipfv}-\textsc{auto}-drink-\textsc{pl} \\
\glt \sens{(The wild yaks) come to drink water there.} (20-RmbroN, 46) (Japhug)
\end{exe}

\begin{exe}
\ex \label{ex:CpjAnWtshi}
\gll \ipa{tɕe}	\ipa{tɯ-ci}	\ipa{\rouge{ɕ}-pjɤ-nɯ-tsʰi}. \\
\textsc{lnk} \textsc{indef}.\textsc{poss}-water \rouge{\textsc{go\&do}}-\textsc{ifr}-\textsc{auto}-drink  \\
\glt \sens{She went to drink water.} (140428 mu e guniang, 72) (Japhug)
\end{exe}

\subsubsection{Deixis}

All four Gyalrongic languages  distinguish cislocative and translocative AM (see examples \ref{ex:GWpjWnWtshinW}, \ref{ex:CpjAnWtshi} and \ref{ex:vEtEtwi}).  They mark the category through two prefixes each (see Table \ref{tab:am-gyalrong} from \citealt[200]{zhang16bragdbar}, Tshobdun data from \citealt{sun12complementation}, Zbu from \citealt{gong18these}), grammaticalized from the motion verbs `go' and `come' (except the cislocative prefixe of Brag-dbar Situ). The fact that motion verbs were grammaticalized as prefixes rather than suffixes in strict verb-final languages like Japhug and Situ can be accounted for by assuming that they comes from the first member of a former serial verb construction (\citealt{jacques13harmonization}). AM markers in Gyalrong do not have redundant person and TAM markers, as they do in Kiranti languages. 

\begin{exe}
\ex \label{ex:vEtEtwi}
\gll  \ipa{ɐ-kómʔ}	\ipa{\rouge{və}-tə-twíʔ} \\
\textsc{1sg}-door  \rouge{\textsc{come\&do}}-\textsc{imp}-open \\
\glt  \sens{Come open the door for me!} (Zbu, \citealt{gong18these})
\end{exe}

\begin{table}[H]
\caption{Associated motion prefixes in Gyalrong languages} \centering \label{tab:am-gyalrong}
\begin{tabular}{lllll}
\toprule
&come & \textsc{cisloc} & go & \textsc{transloc} \\
\midrule
Japhug &  \ipa{ɣi} &\ipa{ɣɯ-} &\ipa{ɕe} &\ipa{ɕɯ-, ɕ-, ʑ-,z- } \\
Kyom-kyo (Situ) &\ipa{vi} &\ipa{və-} &\ipa{tʃʰi} &\ipa{ʃi-} \\
Brag-dbar (Situ) &\ipa{βʑê, və} &\ipa{ɟɐ-} &\ipa{tɕʰê} &\ipa{ɕɐ-} \\
Tshobdun & \ipa{wî}& \ipa{o-} &\ipa{ʃɐ̂} &\ipa{ʃə-} \\
Zbu & \ipa{və̂}& \ipa{və-} &\ipa{xwéʔ} &\ipa{ɕə-} \\
\bottomrule
\end{tabular}
\end{table}

Note that in Situ, the cislocative can be used with a prospective aspectual  value (\citealt{linyj03tense}, \citealt[204]{zhang16bragdbar}), whereas in the other Gyalrong languages, it only marks associated motion.

\subsubsection{Argument of motion}

The argument undergoing the motion event is always the subject (S/A), except in the case of causative constructions, when it can be either causer or causee (as in \ref{ex:GWchWsWXtWnW} from Japhug and \ref{ex:mESEtosnAmYiaN} from Tshobdun).

\begin{exe}
\ex \label{ex:GWchWsWXtWnW}
\gll
\ipa{tɕe} 	\ipa{kupa} 	\ipa{cʰu} 	\ipa{nɯra} 	\ipa{atʰi} 	\ipa{pɕoʁ} 	\ipa{nɯra,} 	\ipa{ɯ-pɕi} 	\ipa{nɯra} 	\ipa{kɯ} 	\ipa{kɯreri} 	\ipa{\rouge{ɣɯ}-cʰɯ-sɯ-χtɯ-nɯ} 	\ipa{ŋu.}  \\
\textsc{lnk} Chinese \textsc{loc} \textsc{dem:pl} downstream direction \textsc{dem:pl} \textsc{3sg}-outside  \textsc{dem:pl}  \textsc{erg} here \textsc{\rouge{come\&do}-ipfv:downstream-caus}-buy-\textsc{pl} be:\textsc{fact} \\cd \
\glt \sens{People from the Chinese areas, people from outside send people to come here to buy (matsutake and sell them in the areas downstream).} (20 grWBgrWB 58) (Japhug)
  \end{exe} 

\begin{exe}
\ex \label{ex:mESEtosnAmYiaN}
\gll \ipa{tʃone=nəʔ}	\ipa{ɐɟiʔ}	\ipa{\rouge{ʃə}-te-nɐ́mɲi-aŋ}	\ipa{nɐ-zgɐ̂t}	\ipa{ʃənəʔ}	\ipa{mə-\rouge{ʃə}-tə-o-s-nɐmɲi-aŋ} \\
 show=\textsc{det} \textsc{1sg} \rouge{\textsc{go\&do}}-\textsc{ipfv}-watch-\textsc{1sg} \textsc{ipfv}.\textsc{pst}-be.justified[II] but \textsc{neg}-\rouge{\textsc{go\&do}}-\textsc{pfv}-\textsc{inv}-\textsc{caus}-watch[II]-\textsc{1sg} \\
\glt  \sens{I deserved to go and watch the show, but s/he did not let me go watch it.} (Tshobdun, \citealt[478]{sun12complementation})
  \end{exe} 
 

\subsubsection{Motion verbs and AM prefixes}
In Gyalrongic, there is no constraint on AM prefixes occurring on motion verbs with the same deixis. Examples (\ref{ex:GWjuGinW}) and (\ref{ex:CpjACe}) respectively illustrate the cislocative on the verb \japhug{ɣi}{come} and the translocative on the verb \japhug{ɕe}{go}. Such examples are not common enough to allow a clear analysis of the semantic value of the redundant AM in these examples.

\begin{exe}
\ex \label{ex:GWjuGinW}
 \gll <jiazhang> \ipa{ra}	\ipa{ju-ɣi-nɯ}	\ipa{tɕe}  <laoshi> \ipa{ɯ-ɕki,}	\ipa{tɯ-ɕki}	\ipa{ʑo}	\ipa{\rouge{ɣɯ}-ju-ɣi-nɯ}	\ipa{ɕti}	\ipa{netɕi?}  \\
 parents \textsc{pl} \textsc{ipfv}-come-\textsc{pl} \textsc{lnk} teacher \textsc{3sg}.\textsc{poss}-\textsc{dat} \textsc{genr}.\textsc{poss}-\textsc{dat} \textsc{emph} \rouge{\textsc{come\&do}}-\textsc{ipfv}-come-\textsc{pl} be.\textsc{affirm}:\textsc{fact} \textsc{sfp} \\
 \glt `The parents come, come to the teachers (us).' (conversation140501 01, 60) (Japhug)
\end{exe}

\begin{exe}
\ex \label{ex:CpjACe}
 \gll \ipa{li}	\ipa{nɤki}	\ipa{iɕqʰa}	\ipa{nɯ}	\ipa{tɤjlu}	\ipa{kɤ-rku}	\ipa{ɯ-ŋgɯ}	\ipa{zɯ}	\ipa{\rouge{ɕ}-pjɤ-ɕe} \\
 again \textsc{dem} the.aforementioned \textsc{dem} flour \textsc{nmlz}:P-put.in \textsc{3sg}.\textsc{poss}-inside \textsc{loc} \rouge{\textsc{go\&do}}-\textsc{ifr}:\textsc{down}-go \\
 \glt `He went into the bag of flour.' (140519 chou xiaoya-zh, 145) (Japhug)
\end{exe}

The opposite combinations, namely cislocative with \japhug{ɕe}{go} and translocative with \japhug{ɣi}{come}, are not grammatical. 



\subsection{Orientation and AM} \label{sec:orientation.japhug}
The first characteristic of note is that Gyalrongic languages have devoted AM markers.  These markers are only used for AM, and occur in a distinct slot within the verbal template.  Orientation markers can thus be reliably distinguished from AM markers, clarifying the function of the two markers when they co-occur.


Non-orientable verbs (verbs expressing actions other than motion, manipulation, sight or actions with a single direction) select one or two lexicalized orientation prefixes. For instance, in Japhug the verb \japhug{mɯrkɯ}{steal} occurs with the orientation `up' (with the orientation prefixes \forme{tɤ-}, \forme{ta-}, \forme{tu-}, \forme{to-}). 

When non-orientable verbs occurs with AM, the verb normally keeps the lexicalized orientation prefix, as in \ref{ex:CtumWrki}, where \japhug{mɯrkɯ}{steal} is used with the \forme{tu-} `up' prefix; the orientation prefix is thus irrelevant to the motion event. 

\begin{exe}
\ex \label{ex:CtumWrki}
 \gll \ipa{kɯ-nŋo}	\ipa{nɯ}	\ipa{qʰe}	\ipa{ci}	\ipa{ci}	\ipa{\rouge{ɕ}-tu-mɯrki}	\ipa{kɯ-fse}	\ipa{ma}	\ipa{nɯ}	\ipa{ma}	\ipa{mɯ-ɲɯ-ɤʁe.} \\
\textsc{nmlz}:S/A-be.defeated \textsc{dem} \textsc{lnk} one one \rouge{\textsc{go\&do}}-\textsc{ipfv}-steal[III] \textsc{nmlz}:S/A-be.like apart.from \textsc{dem} apart.from \textsc{neg}-\textsc{sens}-have.to.eat \\
\glt `The (lion) which is defeated steals a little out of it, but apart from that has nothing to eat.' (20-sWNgi, 65) (Japhug)
\end{exe}

However, some verbs with an AM marker select the possessive prefix not on the basis of the lexicalized orientation, but of the motion expressed by that AM marker, as in (\ref{ex:pjWGWta}), where the `down' orientation (instead of expected `west, centrifuge' orientation selected by the verb \forme{ntsɣe} `sell') represents the motion from Gyalrong areas to Chinese areas.

\begin{exe}
\ex \label{ex:pjWGWta}
\gll \ipa{pot} 	\ipa{ɣɯ} 	\ipa{ɯ-laχtɕʰa} 	\ipa{pjɯ-ɣɯt-a} 	\ipa{tɕe,} 	\ipa{rɟa} 	\ipa{zɯ} 	\ipa{ɕ-pjɯ-ntsɣe-a.}  \\
Tibet gen \textsc{3sg}.\textsc{poss}-thing  \textsc{ipfv}:\textsc{down}-bring-\textsc{1sg} \textsc{lnk} China \textsc{loc} \textsc{transloc}-\textsc{ipfv}:\textsc{down}-sell-\textsc{1sg} \\
\glt `I bring things (down) from central Tibet, and sell them to China.' (28-qAjdoskAt)
\end{exe}


\subsection{Echo phenomena} \label{sec:AM.echo}
Previous literature on AM has reported the existence of `echo phenomena' in the use of AM markers (\citealt[251]{wilkins91associated.motion}, \citealt[681-683]{vuillermet12eseejja}, \citealt[128-130]{rose15am}, \citealt[11]{guillaume16am}), namely that the same motion event can be expressed by more than one AM marker. This phenomenon is common in Japhug narratives. Two subtypes of AM echo can be distinguished.

First, in examples such as (\ref{ex:CtAru}) and (\ref{ex:GWYWsloR}), a motion verb (\japhug{ɕe}{go} and \japhug{ɣi}{come} respectively) is followed by a verb with an AM prefix with the same deixis, though there is a single motion event.

\begin{exe}
\ex \label{ex:CtAru}
\gll \ipa{tɕʰi}	\ipa{ɯ-taʁ}	\ipa{to-ɕe}	\ipa{tɕe}	\ipa{\rouge{ɕ}-tɤ-ru}   \\
stairs \textsc{3sg}.\textsc{poss}-on \textsc{ifr}:\textsc{up}-go \textsc{lnk}  \rouge{\textsc{transloc}}-\textsc{up}:\textsc{pfv}-look \\
\glt `He went up the stairs and looked up.'  (08-kWqhi, 18) (Japhug)
\end{exe}

\begin{exe}
\ex \label{ex:GWYWsloR}
\gll \ipa{kʰa}	\ipa{mɯ-pɯ-rɤʑi}	\ipa{tɕe}	\ipa{tɕe,}	\ipa{ftɕar}	\ipa{nɯ}	\ipa{wuma}	\ipa{ʑo}	\ipa{βɣɯz}	\ipa{pjɤ-rɯŋɯŋɤn}	\ipa{tɕe}	\ipa{maka,}	\ipa{kɯmtʰoʁ}	\ipa{ra}	\ipa{kɯnɤ}	\ipa{ju-ɣi}	\ipa{\rouge{ɣɯ}-ɲɯ-sloʁ}	\ipa{pjɤ-ŋu.} \\
house \textsc{neg}-\textsc{pst}.\textsc{ipfv}-stay \textsc{lnk} \textsc{lnk} summer \textsc{dem} really \textsc{emph} badger \textsc{ifr}.\textsc{ipfv}-cause.damage \textsc{lnk} completely threshold \textsc{pl} also \textsc{ipfv}-come \rouge{\textsc{come\&do}}-\textsc{ipfv}-dig.up \textsc{ifr}.\textsc{ipfv}-be \\
\glt `He was not home, and that summer badgers were causing a lot of damages, they came and even dug up  the threshold of the house.'  (27-spjaNkW, 107) (Japhug)
\end{exe}

Second, we also find cases, such as (\ref{ex:GWtaBzu}), without a motion verb, but with two verbs redundantly prefixed with the same AM marker (here \forme{ɣɯ-}).

\begin{exe}
\ex \label{ex:GWtaBzu}
\gll \ipa{tɕe}	\ipa{a-kʰa}	\ipa{ra}	\ipa{\rouge{ɣɯ}-ta-rɤroʁrɯz,}	\ipa{a-mgo}	\ipa{ra}	\ipa{\rouge{ɣɯ}-ta-βzu}	\ipa{ŋu}	\ipa{ɕi} \\
\textsc{lnk} \textsc{1sg}.\textsc{poss}-house \textsc{pl} \rouge{\textsc{come\&do}}-\textsc{pfv}:3$\rightarrow$3'-tidy 
 \textsc{1sg}.\textsc{poss}-food \textsc{pl} \rouge{\textsc{come\&do}}-\textsc{pfv}:3$\rightarrow$3'-make be:\textsc{fact} \textsc{qu} \\ 
\glt `Is it (the neighbour's wife who took pity on me) and came to tidy my house and make food for me?'  (150827 tianluo, 76) (Japhug)
\end{exe}


Echo AM is required in serial verb constructions (\citealt[253-255]{jacques16complementation}), as shown in (\ref{ex:CkunWrtCe}), where the verbs \japhug{stu}{do like} and the \japhug{nɯrtɕa}{tease} share the same person (3$\rightarrow$3'), TMA (imperfective) and AM (translocative) markers.

\begin{exe}
\ex \label{ex:CkunWrtCe}
\gll \ipa{kɯra}	\ipa{\rouge{ɕ}-tu-ste}	\ipa{tɕe}	\ipa{\rouge{ɕ}-ku-nɯrtɕe}	\ipa{ra}	\ipa{pjɤ-ŋu.} \\
\textsc{dem}:\textsc{prox}:\textsc{pl} \rouge{\textsc{go\&do}}-\textsc{ipfv}-do.like[III] \textsc{lnk}  \rouge{\textsc{go\&do}}-\textsc{ipfv}-tease[III] \textsc{pl} \textsc{ifr}.\textsc{ipfv}-be \\
\glt `(The mouse) went and teased (the cat) like that.' (150902 dashu, 31) (Japhug)
\end{exe}


 \subsection{Pluri- vs monoactionality} \label{sec:am.japhug}
Another topic of interest in Gyalrong is the issue of the separability of the motion event and the verbal action. Unlike other languages with AM such as Kiranti or Tungusic (\citealt{stojnova16nda, fuente18am}), in Japhug the scope of negation, interrogation, conditionals and complement-taking verbs applies to the entire verbal event (motion+main action). In cases where the verbal action is outside their scope, an alternative purposive construction with a motion verb (the Motion Verb Construction, henceforth MVC) is used. This is however not common to all Gyalrong languages: in the closely related Situ language, the semantic distinction between AM and the corresponding purposive construction is less clear (§ \ref{sec:am.situ}). 
 
To express the meaning of motion prior to an action, associated motion prefixes are nearly two times as common as corresponding MVCs in the Japhug corpus. There is however a clear semantic difference between the two constructions, which was briefly described in \citet{jacques13harmonization}, but is presented here in more detail.

AM and MVC differ from each other in that in the former, the completion of both motion event and verbal action is presupposed (AM is monoactional), whereas in the case of the latter, the two can be separated. This mono- vs. pluractionality contrast is most conspicuous in past perfective forms, and can be observed in four types of constructions: concessives (with negation of the verbal action), interrogatives, conditionals and complement clauses.

\subsubsection{Concessive} \label{sec:am.concessive}
A MVC  with the motion verb in perfective form can be followed by a clause negating the purposive action, as in (\ref{ex:nAkWrtoR}). In this example, only the motion is realized, while the action expressed by the verb \japhug{rtoʁ}{look} could not be accomplished.

\begin{exe}
\ex \label{ex:nAkWrtoR}
\gll \ipa{nɤ-kɯ-rtoʁ}	\ipa{jɤ-ɣe-a}	\ipa{ri,}	\ipa{mɯ-nɯ-atɯɣ-tɕi,}	\ipa{mɯ-pɯ-ta-mto.} \\
\textsc{1sg.poss}-\textsc{nmlz}:S/A-see \textsc{pfv}-come[II]-\textsc{1sg} \textsc{lnk} \textsc{neg}-\textsc{pfv}-meet-\textsc{1du} \textsc{neg}-\textsc{pfv}-1\fl2-see \\
\glt `I came to see you but I did not see you.' (Japhug)
\end{exe}

With the corresponding AM verb form \japhug{ɣɯ-jɤ-ta-rtoʁ}{I came to see you}, negating the action of the verb is self-contradictory and nonsensical, and a sentence such as (\ref{ex:GWjAtartoR}) is incorrect.

\begin{exe}
\ex \label{ex:GWjAtartoR}
\gll $\dagger$ɣɯ-jɤ-ta-rtoʁ ri mɯ-pɯ-ta-mto \\
\textsc{cisloc}-\textsc{pfv}-1\fl2-look \textsc{lnk} \textsc{neg}-\textsc{pfv}-1\fl2-see \\
\glt Intended meaning: `I came to see you but I did not see you.'  (Japhug)
\end{exe}

Additional minimal pairs of the same type are presented in \citet[202-203]{jacques13harmonization}.


\subsubsection{Interrogative} \label{sec:am.interrogative}
In interrogative clauses, MVCs are required to express meanings such as `What/who have you come/gone to X', as in example (\ref{ex:tChi.WkWpa}), an example which occurs nine times in the corpus.

\begin{exe}
\ex \label{ex:tChi.WkWpa}
\gll \ipa{tɕʰi}	\ipa{ɯ-kɯ-pa}	\ipa{jɤ-tɯ-ɣe?}\\
what \textsc{3sg.poss}-do \textsc{pfv}-2-come[II] \\
\glt `What did you come to do?' (nine examples in the corpus) (Japhug)
\end{exe}

The difference between MVC and AM in interrogatives can be illustrated by comparing the minimal pair  (\ref{ex:tChi.WkWpa}) and (\ref{ex:tChi.GWtAtWpat}). Example (\ref{ex:tChi.WkWpa}),  implies that the addressee has not done anything yet, while (\ref{ex:tChi.GWtAtWpat}) with AM can only be used if the action is already done, and requires a different translation.

\begin{exe}
\ex \label{ex:tChi.GWtAtWpat}
\gll \ipa{tɕʰi}	\ipa{\rouge{ɣɯ}-tɤ-tɯ-pa-t}  \\
what \rouge{\textsc{come\&do}}-\textsc{pfv}-2-do-\textsc{pst:tr}    \\
\glt `What did you do upon coming here?' (elicited) (Japhug)
\end{exe}

 

\subsubsection{Conditional} \label{sec:am.conditional}
The presuppositional difference between MVC and AM is also perceptible in the protasis of conditional clauses. 

With MVC in the protasis as in (\ref{ex:mWmAjAtWGe}), there is no presupposition that the verbal action took place, only the motion event constitutes a condition to the state of affair described in the apodosis.

\begin{exe}
\ex \label{ex:mWmAjAtWGe}
\gll \ipa{nɤ-wa}	\ipa{ɯ-kɯ-rtoʁ}	\ipa{mɯ\redp{}mɤ-jɤ-tɯ-ɣe}	\ipa{nɤ}	\ipa{aʑo}	\ipa{mɯ-pɯ-kɯ-mto-a.}
 \\
\textsc{1sg.poss}-father \textsc{3sg.poss-}\textsc{nmlz}:S/A-look \textsc{cond}\redp{}\textsc{neg}-\textsc{pfv}-2-come[II] \textsc{lnk} \textsc{1sg} \textsc{neg}-\textsc{pfv}-2\fl{}1-\textsc{1sg} \\
\glt `If you had not come to see your father, you would not have seen me.' (you saw me, but your father was not here) (Japhug)
\end{exe}

By contrast, with AM, the verbal action necessarily took place, as in example (\ref{ex:mWmAGWjAtWrtoR}).

\begin{exe}
\ex \label{ex:mWmAGWjAtWrtoR}
\gll \ipa{nɤ-wa}	\ipa{mɯ\redp{}mɤ-\rouge{ɣɯ}-jɤ-tɯ-rtoʁ}	\ipa{nɤ}	\ipa{pɯ-sɤzdɯxpa} \\
\textsc{1sg.poss}-father \textsc{cond}\redp{}\textsc{neg}-\rouge{\textsc{come\&do}}-\textsc{pfv}-2-look \textsc{lnk} \textsc{pst.ipfv}-be.pitiful \\ 
\glt `If you had not come to see your father, he would have felt sorry.' (but you did saw him, so he does not feel sorry) (Japhug)
\end{exe}

\subsubsection{Complement clauses} \label{sec:am.complement}
In complement clauses, verbs with AM prefixes are attested, and complement taking verbs always have scope over both the action of the verb and motion event.

 
In (\ref{ex:mACWkAtshi}), the modal verb \japhug{cʰa}{can} and the double negations (with the specific meaning `cannot help') have scope over both the motion event and the verbal action -- this example is taken from a passage in a story where the king reproaches a small child, who just returned from a mission he himself send him to accomplish, not to have first come to greet him on his return home; the child says these words to justify why he first went to see his mother before greeting the king -- from this context it is clear that both the motion event (to him mother's house, explaining the child's failure to go to see the king) and the action `drink milk' (the reason for that motion event) are equally important to the plot and inseparable. 

\begin{exe}
\ex \label{ex:mACWkAtshi}
\gll  \ipa{tɯ-nɯ}	\ipa{ɯ-kɯ-tsʰi}	\ipa{ɲɯ-ɕti-a}	\ipa{tɕe,}	\ipa{jɤ-azɣɯt-a}	\ipa{tɕe,}	\ipa{tɯ-nɯ}	\ipa{ci}	\ipa{mɤ-\rouge{ɕɯ}-kɤ-tsʰi}	\ipa{nɯ}	\ipa{mɯ́j-cʰa-a}  \\
\textsc{indef}.\textsc{poss}-breast \textsc{3sg}.\textsc{poss}-\textsc{nmlz}:S/A-drink \textsc{sens}-be.\textsc{affirm}-\textsc{1sg} \textsc{lnk} \textsc{pfv}-arrive-\textsc{1sg} \textsc{lnk} \textsc{indef}.\textsc{poss}-breast  \textsc{indef} \textsc{neg}-\rouge{\textsc{go\&do}}-\textsc{inf}-drink \textsc{dem} \textsc{neg}:\textsc{sens}-can-\textsc{1sg} \\
\glt `I am (a toddler) who (still) drinks (his mother's) milk, when I arrived, I could not help but go to drink milk.'  (Norbzang, 262) (Japhug)
 \end{exe}
 
 In (\ref{ex:CWkAmWrkW.mAtWcha}), the negated modal verb has also on the action of both the main verb and the motion event -- the guards would prevent the main character not only to steal, but also to go to the place where the object to be stolen is found.
 
\begin{exe}
\ex \label{ex:CWkAmWrkW.mAtWcha}
\gll \ipa{ʁmaʁ}	\ipa{χsɯ-tɤkʰar}	\ipa{kɯ}	\ipa{ɲɯ-ɤz-nɤkʰar-nɯ}	\ipa{ɕti}	\ipa{tɕe,}	\ipa{\rouge{ɕɯ}-kɤ-mɯrkɯ}	\ipa{mɤ-tɯ-cʰa}  \\
solider three-rounds \textsc{erg} \textsc{sens}-\textsc{prog}-surround-\textsc{pl} be.\textsc{affirm}:\textsc{fact} \textsc{lnk}  \rouge{\textsc{go\&do}}-\textsc{inf}-steal \textsc{neg}-2-can:\textsc{fact} \\
\glt `Three rounds of soldiers will be surrounding it, you will not be able to (go there and) steal it.' (2003qachga, 55) (Japhug)
   \end{exe}
 
The same observation also applies to  verbs with AM in complement clauses selected by a verb in the protasis, as in(\ref{ex:CWkAmWrkW}): the realization of the verbal action (in addition to that of the motion event) belongs to the condition.

\begin{exe}
\ex \label{ex:CWkAmWrkW}
\gll \ipa{nɤʑo}	\ipa{\rouge{ɕɯ}-kɤ-mɯrkɯ}	\ipa{a-pɯ-tɯ-cʰa}	\ipa{nɤ}	\ipa{aʑo}	\ipa{cʰɯ-sɯ-jɣat-a}	\ipa{jɤɣ} \\
\textsc{2sg} \rouge{\textsc{go\&do}}-\textsc{inf}-steal \textsc{irr}-\textsc{ipfv}-2-can \textsc{lnk} \textsc{1sg} \textsc{ipfv}-\textsc{caus}-go.back-\textsc{1sg} be.agreed:\textsc{fact} \\
\glt `If you succeed stealing it (after having gone there), I can cause him to go back there.' (02-montagnes-kamnyu, 46) (Japhug)
\end{exe}

By contrast, in  (\ref{ex:kWrAma.kACe}), in the case of the infinitival complement \forme{kɯ-rɤma kɤ-ɕe} `go to work' with a purposive clause (\forme{kɯ-rɤma}), the main verb \japhug{mda}{be time to} only has scope over the motion event expressed by the verb \japhug{ɕe}{go} -- the time that is indicated by the stars refers to the beginning of the journey to work, not the start of the work itself.
 
 \begin{exe}
\ex \label{ex:kWrAma.kACe}
\gll  \ipa{tɕe}	\ipa{kɯɕɯŋgɯ}	\ipa{tɕe}	\ipa{tɯtsʰot}	\ipa{pɯ-me}	\ipa{tɕe}	\ipa{nɯnɯ}	\ipa{cʰɯ-ɬoʁ}	\ipa{lu-ɕqʰlɤt}	\ipa{nɯra}	\ipa{ɕ-tu-kɯ-ru}	\ipa{tɕe,}	\ipa{nɯnɯ}	\ipa{kɤ-rɤru}	\ipa{mda}	\ipa{mɤ-mda}	\ipa{cʰondɤre}	\ipa{kɯ-rɤma}	\ipa{kɤ-ɕe}	\ipa{mda}	\ipa{mɤ-mda}	\ipa{nɯtɕu}	\ipa{ɕ-tu-kɯ-ru}	\ipa{pɯ-ŋgrɤl.} \\
 \textsc{lnk} long.ago \textsc{lnk} clock \textsc{pst}.\textsc{ipfv}-not.exist \textsc{lnk} \textsc{dem} \textsc{ipfv}:\textsc{downstream}-come.out \textsc{ipfv}:\textsc{upstream}-disappear \textsc{dem}:\textsc{pl} \textsc{go\&do}-\textsc{ipfv}:up-\textsc{genr}:S/P-look \textsc{lnk} \textsc{dem} \textsc{inf}-get.up be.time:\textsc{fact} \textsc{neg}-be.time:\textsc{fact} \textsc{comit} \textsc{nmlz}:S/A-work \textsc{inf}-go be.time:\textsc{fact} \textsc{neg}-be.time:\textsc{fact} \textsc{dem}:\textsc{loc} \textsc{go\&do}-\textsc{ipfv}:up-\textsc{genr}:S/P-look \textsc{pst}.\textsc{ipfv}-be.usually.the.case  \\
 \glt  `In former times, there was no clock, and people used to watch when (these stars) came out or disappeared (to know) whether it was time to get up or go to work.' (29-LAntshAm, 66) (Japhug)
  \end{exe}
  
\subsubsection{Pluri- vs monoactionality in Situ} \label{sec:am.situ}
Minimal pairs similar to those presented above in Japhug have been elicited in Situ. However, it appears that in this language, the use of verbs in perfective form with AM markers does not presuppose that both the verbal action and the motion event have been accomplished. 


In (\ref{ex:nEJAtanaston}), the action of the verb \ipa{nə-ɟɐ-ta-natsô-n}  `I came to see you' with the AM prefix \ipa{ɟɐ-} is negated by the following clauses, while the motion event clearly took place; the corresponding Japhug example in § \ref{sec:am.concessive} is considered to be not only agrammatical, but also non-sensical. Example (\ref{ex:nEJAtanaston}) has little discernible semantic difference with the corresponding MVC in (\ref{ex:nEkEnatso}).
 

\begin{exe}
\ex \label{ex:nEJAtanaston}
\gll \ipa{nə-\rouge{ɟɐ}-ta-natsô-n} \ipa{rɐ}, \ipa{nəɟə̂} \ipa{nə-ˈtə-mɐ-n}, \ipa{majnə} \ipa{ma-na-ta-mətɐ̂-n}.\\
\textsc{pfv}:downwards-\textsc{\rouge{come\&do}}-2→1-look\textsubscript{II}-2 \textsc{conj} \textsc{2sg} \textsc{sens}-2-do.not.exist\textsubscript{I}-\textsc{2sg} \textsc{conj} \textsc{neg}-\textsc{pfv}-2→1-see\textsubscript{II}-\textsc{2sg}\\
\glt  \sens{I came to see you, but you were not here, so I didn't see you.} (Situ)
\end{exe}

\begin{exe}
\ex  \label{ex:nEkEnatso}
\gll \ipa{nə-kə-natsō} \ipa{nə-və̄-ŋ} \ipa{rɐnə}, \ipa{nəɟə̂} \ipa{nə-ˈtə-mɐ-n}, \ipa{majnə} \ipa{ma-na-ta-mətɐ̂-n}.\\
\textsc{poss.2sg}-\textsc{nmlz}-see \textsc{pfv}:downwards-come\textsubscript{II}-\textsc{1sg} \textsc{conj} \textsc{2sg} \textsc{sens}-2-do.not.exist\textsubscript{I}-\textsc{2sg} \textsc{conj} \textsc{neg}-\textsc{pfv}-2→1-see\textsubscript{II}-\textsc{2sg}\\
\glt  \sens{I came to see you, but you were not here, so I didn't see you.} (Situ)
\end{exe}

Similarly, in (\ref{ex:nEJAtəVan}), the verb form  \ipa{nə-ɟɐ-ˈtə-va-n}  \sens{What did you come to do?} with the AM prefix \ipa{ɟɐ-} does not imply that the action of the verb has already taken place, unlike its corresponding Japhug in § \ref{sec:am.interrogative}.

 

\begin{exe}
\ex \label{ex:nEJAtəVan}
\gll
\ipa{nəɟə̂} \ipa{tɕɐ̄} \ipa{thə̂} \ipa{nə-\rouge{ɟɐ}-ˈtə-va-n}? \\
\textsc{2sg} here what  \textsc{pfv}:downwards-\textsc{\rouge{come\&do}}-2-do\textsubscript{II}-\textsc{2sg} \\
\glt  \sens{What did you come to do?} (Situ)
\end{exe}

\begin{exe}
\ex \label{ex:nEtEvEn}
\gll
\ipa{nəɟə̂} \ipa{tɕɐ̄} \ipa{thə̂} \ipa{kə-viɛ̂} \ipa{nə-ˈtə-və-n} \ipa{kɐ}? \\
\textsc{2sg} here what \textsc{inf}-do \textsc{pfv}:downwards-2-come\textsubscript{II}-\textsc{2sg} \textsc{part} \\
\glt  \sens{What did you come to do?} (Situ)
\end{exe}

This difference between Japhug and Situ, two otherwise relatively closely related languages, suggests that features such as monoactionality of motion event and verbal action may not be diachronically stable.

\subsection{Lack of volitionality and control } \label{sec:volitionality}
An additional difference between AM and MVC has to do with volitionality and/or controllability. In the case of an MVC, the verb in the purposive clause, whose action follows the motion event, is always necessarily volitional and controllable. By contrast, in the case of AM, it is possible to find examples where the verbal action expresses a non-controllable event, such as the action of finding a lost object in example (\ref{ex:CpjAmto}) with echo phenomenon (cf § \ref{sec:AM.echo}). Note that there are no examples of the non-volitional verb \japhug{mto}{see} with the MVC in the corpus (the volitional \japhug{rtoʁ}{see, look} or \japhug{ru}{look} occur  instead). 

\begin{exe}
\ex  \label{ex:CpjAmto}
\gll  \ipa{nɯɕɯmɯma}	\ipa{ʑo}	\ipa{tɯ-ci}	\ipa{ɯ-ŋgɯ}	\ipa{pjɤ-ɕe}	\ipa{qʰe}	\ipa{iɕqʰa}	\ipa{tɤɕime}	\ipa{kɯ}	\ipa{ɯ-sɤcɯ}	\ipa{pɯ-kɤ-nɯ-ɕlɯɣ}	\ipa{nɯ}	\ipa{\rouge{ɕ}-pjɤ-mto}. \\
immediately \textsc{emph} \textsc{indef}.\textsc{poss}-water \textsc{3sg}.\textsc{poss}-inside \textsc{ifr}:\textsc{down}-go \textsc{lnk} the.aforementioned lady \textsc{erg} \textsc{3sg}.\textsc{poss}-key \textsc{pfv}:\textsc{down}-\textsc{nmlz}:P-\textsc{auto}-drop \textsc{dem} \rouge{\textsc{go\&do}}-\textsc{ifr}-see \\
\glt `He went immediately into the water and saw there the key that the lady had dropped by mistake.' (140510 fengwang, 118) (Japhug)
\end{exe}

\section{Kiranti}  
The most complex AM systems of the Sino-Tibetan family are found in Kiranti, with systems with up to seven distinct markers (Khaling; Yamphu, \citealt[137-194]{rutgers98yamphu}).
The main characteristics of AM marking in Kiranti are the following:

\begin{enumerate}
\item AM systems of greatly differing complexity across the subgroup
\item Vertical dimension must be described
\item AM markers are labile: they can be either intransitive or transitive (usually matching the main verb, but not always) ; when transitive, the argument of motion is not only the A but also includes the P
\item AM markers often have multiple uses, with the same marker variously also marking aspect, as well as orientation, personal configuration, and even voice.
\item Verbs with AM marking are pluri-actional (in Khaling; unclear of other languages)
\end{enumerate}


In Kiranti, AM markers are ancient auxiliaries (variously called `aspectivizers', `bound roots', `motionalizers',   `general motion auxiliaries', `V_2s', `vector verbs', `oriented motion verbs' and `modal verbs' in different descriptions of Kiranti languages) which come from motion verbs, though in many cases the source verb does not exist any more. 
 
%One crucial distinction from AM in Gyalrongic is that these are not devoted markers.  

There is evidence that such complex verb forms originate from serial verb constructions, in which a motion verb (as well as other auxiliaries) occupied the second position, but became progressively integrated into a single word (though with differences across Kiranti; see \citealt{bickel07chintang}, \citealt{schiering10prosodic}). For some languages, the second verb has grammaticalized into an aspectual marker (see \citealt{bickel96aspect}), but some of these languages still show, within the examples given in grammars, some situations where the marker clearly indicates AM and not aspect.  In other cases, it is not aspect but orientation which is the main function of the markers which can also occasionally mark AM.  However, some languages have AM markers which are devoted to motion.  These different degrees of grammaticalization of auxiliary motion verbs explains one of the interesting features of AM in Kiranti: even with cognate morphemes, the languages range from 0 to 7 AM markers. 


Example (\ref{ex:ingyoklestyanu}) illustrates a typical AM-marked verb in Khaling, with the circumambulative \textbf{-le-} `go around doing'. Note that the \textsc{2pl:n.pst} suffix \ipa{-ni} occurs in reduced form \ipa{-n-} between the stem of the verb \ipa{-ŋô-} and  the AM marker \ipa{-le-}, suggesting that at an earlier stage both the main verb and the AM marker had the full array of person indexation and TAM marker.

\begin{exe}
\ex \label{ex:ingyoklestyanu}
\gll \ipa{ʔi-ŋô-n-\rouge{le}-ni} \\
2/\textsc{inv}-cry-\textsc{2pl:n.pst}-\textsc{\rouge{go.around.doing}}-\textsc{2pl:n.pst} \\
\glt `You_{pl} went around crying.'  (Khaling)
\end{exe}


\subsection{Morphosyntactic parameters of AM in Kiranti}
The Khaling AM system, which ranks among the most complex in the Kiranti branch, can be used to illustrate the general typological characteristics of AM systems in this branch. Table \ref{tab:system} summarizes the data concerning four parameters (temporal relation, deixis, vertical dimension and argument of motion) relating to the AM marker.

\begin{landscape}
\begin{table}%[H]
\caption{The Khaling AM system} \label{tab:system} \centering
\begin{tabular}{lllllll}
\toprule
Source &V_2 & Temp.& Deixis & Vertical  & Argument \\
\midrule
?& \dhatu{-le(t)-}{go around doing X} &C& \textsc{circum} & $\varnothing$ & S/A \\
\dhatu{kʰot}{go} &\dhatu{-kʰo(t)-}{do X and go} &S& \textsc{trans} & $\varnothing$ &  S/A(+P) \\
?&\dhatu{-pɛ(t)-}{go and do X} &P& \textsc{trans} & $\varnothing$ & S/A \\
\dhatu{ɦo}{come} &\dhatu{-ɦo(t)-}{do X and come} &S& \textsc{cis} & $\varnothing$ & S/A(+P) \\
\dhatu{kʰoŋ}{come up}&\dhatu{-kʰoŋ-}{do X and come up} &S& \textsc{cis} &up & S/A \\
 \dhatu{pi}{come (same level)}&\dhatu{-pi(t)-}{do X and come (same level)} &S& \textsc{cis} &same level & S/A(+P) \\
\dhatu{tɛn}{fall} & \dhatu{-tɛ(nt)-}{do X and come/bring down} &S/P & \textsc{cis} &down & S/A(+P) \\
\bottomrule
\end{tabular}
\end{table}
\end{landscape}

 \subsubsection{Temporal relation} \label{sec:temporal.khaling}
As seen in Table \ref{tab:system}, AM markers in Kiranti languages can indicate prior, concurrent or subsequent motion (abbreviated as P, C and S respectively) with respect to the main verb.  This is exemplified in (\ref{ex:tyungkhatya}) from Khaling and (\ref{ex:thuNdhaNNittu}) from Belhare, where the AM markers \dhatu{-kʰot-}{do and go} and  \dhatu{-thaŋ-}{go up and do} respectively express subsequent and prior motion.

\begin{exe}
\ex \label{ex:tyungkhatya}
 \gll   \ipa{tʉ̂ŋ-\rouge{kʰʌ}-tɛ} \\
 drink-\textsc{away/\rouge{do\&go}}-\textsc{2/3:pst} \\
 \glt `He drank it up' or `He drank it and left.'  (Khaling)
 \end{exe}
 
 \begin{exe}
\ex \label{ex:thuNdhaNNittu}
 \gll   \ipa{cia}	\ipa{thuŋ-\rouge{dhaŋŋ}-itt-u!} \\
 tea set.up.to.cook-\rouge{go.\textsc{upwards}}-\textsc{accelerative}-3U \\
\glt \sens{Go up and cook up some tea!} (Belhare, \citealt[73]{bickel99spatial})
 \end{exe}

In some cases, pragmatics will trump the usual temporal relation for a given AM marker.  This is seen in the set of examples below, where  \dhatu{-tɛ(nt)-}{do X and come/bring down}, which  generally implies subsequent motion (as in \ref{ex:ryaptyandyu1}), can instead be used to express prior motion if the scenario triggers that interpretation.  In (\ref{ex:ryaptyandyu1}), there is no obstacle to a default subsequent motion interpretation (buckwheat is cultivated at the same altitude as the villages, and the grain can be beaten without Sirise needing to ascend or descend first), but in (\ref{ex:ryaptyandyu2}), the fact that the rice is cultivated lower in altitude than the household means that only an interpretation whereby Jirise first descends to the level of the rice field before beating the rice makes sense to Khaling speakers. 

 \begin{exe}
\ex \label{ex:ryaptyandyu1}
  \gll   \ipa{siriseʔ-ɛ} \ipa{bʰrɛ̂m}  \ipa{rɛp-\rouge{tɛnd}-ʉ} \\
   p.n.-\textsc{erg} buckwheat beat-\textsc{\rouge{come.down}}-\textsc{3sg$\rightarrow$3.n.pst} \\
\glt `Sirise beats the buckwheat and comes down.'  (Khaling)
 \end{exe}
 
 
 \begin{exe}
\ex \label{ex:ryaptyandyu2}
 \gll   \ipa{dzirise-ʔɛ} \ipa{rɵ̂ː} \ipa{rɛp-\rouge{tɛnd}-ʉ} \\
 p.n.-\textsc{erg} rice beat-\textsc{\rouge{come.down}}-\textsc{3sg$\rightarrow$3.n.pst} \\
\glt `Jirise comes down, beats the rice (and returns).'  (Khaling)
 \end{exe}

%Interestingly, while most AM markers, at least in Khaling, seems to have a clearly identified temporal relation, this feature is often not described for Kiranti languages and sometimes difficult to retrieve from the examples given.


 \subsubsection{Deixis and vertical dimension} \label{sec:vertical.khaling}
  Most AM systems in Kiranti languages systems have a circumambulative `go around doing' (see (\ref{ex:hetughongwaga} from Yakkha), although its form is cognate only in very closely related languages, such as Khaling and Dumi.   
  
  \begin{exe}
\ex \label{ex:hetughongwaga}
 \gll  \ipa{ŋkha}	\ipa{i=ya}	\ipa{het-u-\rouge{ghond}-wa-ga?} \\
that what=NMLZ.SG cut-3P-\rouge{V_2\textsc{roam}}-\textsc{npst}-2  \\
\glt \sens{What are you cutting (at various places)?} (Yakkha, \citealt[326]{schackow15yakkha})
\end{exe}
 
The circumambulative marker is not necessarily always interpreted as AM in all languages; for instance in Belhare, \citet[164]{bickel96aspect} shows that the marker which he glosses as `spatially distributed temporary' \textsc{sdt}  has both AM and non-AM uses (compare \ref{ex:chapkonu} and \ref{ex:chigon}). 
 
 \begin{exe}
\ex \label{ex:chapkonu}  
\gll \ipa{rot-de} \ipa{i-baŋ-ŋa} \ipa{chap-\rouge{kon}-u}  \\
road-\textsc{loc} one-\textsc{hum}-\textsc{obl} write-\textsc{\rouge{sdt}}-\textsc{3u} \\
\glt \sens{Somebody is walking around taking notes on the road.}  (Belhare)
\end{exe}

 \begin{exe}
\ex \label{ex:chigon}
\gll \ipa{thali} \ipa{khore} \ipa{wat} \ipa{chi-gon}    \\
plat cup clean clean-\textsc{sdt} \\
\glt \sens{He is cleaning plates and cups.} (Belhare)
\end{exe}


The traditional deictic categories of cis- and translocative are clearly present in Kiranti languages, and always identifiable through the gloss.  With the circumambulative, this results in a three-way deictic system for the languages in the subgroup.

Kiranti languages also mark the vertical dimension with cislocative (or more rarely, translocative) deixis (motion down, up, and on the same level), as shown by (\ref{ex:semluuging}) and (\ref{ex:ikkaettu}) from Yamphu.
 
\begin{exe}
\ex \label{ex:semluuging}
 \gll \ipa{mo.ba}	\ipa{ka}	\ipa{sem.so}	\ipa{semlu-\rouge{ʔug}-iŋ} \\
that-\textsc{ela} I sing.too sing-\textsc{\rouge{come\_down}}-\textsc{exps} \\
\glt \sens{I came down a-singing} (Yamphu, \citealt[143]{rutgers98yamphu})
\end{exe}

\begin{exe}
\ex \label{ex:ikkaettu}
 \gll \ipa{i.doʔ}	\ipa{ik-\rouge{kætt}-u} \\
this.like grind-\textsc{\rouge{bring\_up}}-3P \\
\glt \sens{He ground and brought up [the chutney]} (Yamphu, \citealt[144]{rutgers98yamphu})
\end{exe}

 Since the AM markers are transparently grammaticalized from these motion verbs (and their applicative/causative counterparts), it is hardly surprising that this contrast is also found in the AM system.
Note that a distinction is often made in Kiranti languages between verbs and AM markers for `same level' and `unspecified for level'.

%Interestingly, AM markers expressing translocative motion have no vertical dimension contrast (the contrast is not found in motion verbs either).
 

  \subsubsection{Argument of motion} \label{sec:argument.khaling}
With all AM markers for Kiranti languages, the argument undergoing the motion event always includes the subject (S/A).  The AM markers tend, however, to be labile, having both intransitive and transitive forms.  When the AM is marked on a transitive verb, in many cases the motion is to be interpreted transitively, resulting in a `manipulative' meaning whereby translocatives are interpreted as `take away' and cislocatives as `bring', the latter exemplified in (\ref{ex:chikturana}).

The result of this lability is that when the AM is transitive, the P is an argument of motion, along with the A .
We have indicated this with the formula `S/A(+P)' in tables presenting Kiranti data.
 
\begin{exe}
\ex \label{ex:chikturana}
 \gll  \ipa{eko}	\ipa{phuŋ}	\ipa{chikt-u-\rouge{ra}=na} \\
one flower pluck:\textsc{pst}-3.P-\rouge{V_2.\textsc{bring}}=\textsc{nmlz.sg} \\
\glt \sens{She plucked a flower and brought it} (Yakkha, \citealt[312]{schackow15yakkha})
\end{exe}

While it is never possible for an intransitive main verb to be accompanied by a transitive version of the AM marker, it must be noted that the AM marker is not always interpreted transitively with a transitive main verb as in (\ref{ex:hungkhondu}). 

\begin{exe}  
\ex \label{ex:hungkhondu}
 \gll  \ipa{ʔīn} \ipa{kʰɵs-tʰer-e} \ipa{uŋʌ} \ipa{ʔʌ-dʌrʌm} \ipa{ɦûŋ-\rouge{kʰond}-u} \\
 2sg go-\textsc{habit}-\textsc{imp}:\textsc{2sg} \textsc{1sg}.\textsc{erg} \textsc{1sg}.\textsc{poss}-friend wait-\rouge{\textsc{do\&come.up}}-\textsc{1sg.A.n.pst} \\
\glt `You keep going, I will wait for my friends and come up then.' [NOT `brought them up'] (Khaling)
\end{exe}

 There are a number of examples of an intransitive interpretation for an AM marker on a transitive verb, even though the actual form of the AM marker will in such a case be transitive (in other words, it will be based on a form of the motion verb with an applicative/causative suffix).  Considering that the interpretation for the transitivity of the AM marker is contextual, this is another instance (as with the temporal relation being affected by context) of default interpretations being trumped by pragmatics.
  

 \subsubsection{Non-AM meanings} \label{sec:khaling.non.am}
As mentioned earlier, what accounts for the diversity of AM systems in Kiranti languages is the fact that different languages have undergone different degrees of grammaticalization of motion verbs in the second position of what were presumably originally serial verb constructions.  In some languages, the motion verbs have all grammaticalized to the point that none can be used to encode AM (this is the case with Bantawa, for example); in other languages, most markers have undergone considerably grammaticalization, but occasionally some uses of the V_2 encode AM (an example of this is provided by examples  \ref{ex:chapkonu} and \ref{ex:chigon} above from Belhare).  In yet other cases, most of the motion verb V_2's are used for AM (this is the case of Khaling, where only one AM marker -- \ipa{-kʰot-} -- is also used to mark completive aspect/telicity, as shown by \ref{ex:tyungkhatya} above, while the others are used exclusively for AM.)  
The primary non-AM meaning found is aspect.
%egs

Other non-AM meanings are however found.  One of interest is the voice reading found in Hayu, where with transitive verbs, \ipa{-la(t)}, which is otherwise an AM marker, is used to express the passive voice as in (\ref{ex:jeŋ.la}).

\begin{exe}
\ex \label{ex:jeŋ.la}
 \gll \ipa{ma}	\ipa{jeŋ}	\ipa{la}	\ipa{mima}	\ipa{jeksa} \\
\textsc{neg} voir il-va ainsi nuit \\
\glt \sens{Cela ne se verra pas comme ça de nuit.} (\citealt[153]{michailovsky88}, Hayu)
\end{exe}

In yet other situations, what are sometimes AM markers can also be used to encode orientation without motion.  One of the great difficulties in decoding the examples found in grammars is that AM markers are most often given with verbs of motion or of transfer, where it is not clear whether the marker in question merely encodes orientation or whether it also carries motion. For instance, since the verb \dhatu{yok}{seek} implies an intrinsic motion, an example such as (\ref{ex:yoktusing}) is not sufficient to prove that the V_2 \ipa{-tus-} can be analyzed as AM rather than orientation.
 
\begin{exe}
\ex \label{ex:yoktusing}
 \gll  \ipa{mo-ba}	\ipa{pa:tro}	\ipa{yok-tus-iŋ} \\
that-\textsc{ela} patro seek-around-\textsc{exps} \\
\glt \sens{So then I went around looking for a calendar} (\citealt[153]{rutgers98yamphu}, Yamphu)
\end{exe}


  
\subsubsection{Mono- vs pluriactionality} \label{sec:khaling.pluriactionality}

As discussed for Gyalrongic languages, the mono- vs pluri-actionality of the verb+AM unit is interesting to consider in Kiranti.  Again, one of the difficulties in broadening this discussion to Kiranti as a whole results from the fact that it is rarely possible to determine the extent of pluri-actionality from the examples found in many grammars.  Nonetheless the first-hand data we have collected on Khaling suggests that AM markers form pluri-actional verb events: when examples are negated, the scope of the negation is limited to the main event and does not cover the AM. This can be seen in example (\ref{ex:mahunkhonya}) (compare with \ref{ex:hungkhondu} above). 

\begin{exe}
\ex \label{ex:mahunkhonya}
 \gll
\ipa{ʔuŋʌ} \ipa{ʔʌ̄m} \ipa{mʌ-ɦû-n-\rouge{kʰōː}-nɛ-ʔɛ} \ipa{ʔu-nûː} \ipa{ŋes-tɛ} \\
\textsc{1sg}.\textsc{erg} \textsc{3sg} \textsc{neg}-wait-\textsc{inf}-\rouge{\textsc{do\&come.up}}-\textsc{inf}-\textsc{erg} \textsc{3sg}.\textsc{poss}-mind hurt-2/3:\textsc{pst} \\
\glt `I did not wait for him before going up and he was sad. (I was already gone up by the time he arrived at the waiting place).'  (Khaling)
\end{exe}

 

The pluri-actionality of the verb+AM in Khaling has only been tested with negation, but presumably the tests described for Gyalrongic languages would work equally well with Kiranti languages.

\subsection{Survey of AM systems in Kiranti}

A large number of grammars of Kiranti languages have been consulted for the present survey, but not all languages have been included, as shorter grammars  (\citealt{ebert97athpare}, \citealt{ebert97camling}, \citealt{opgenort05jero}, \citealt{tolsma06kulung}) do not always include enough data on V_2. 

Not all Kiranti languages have AM. Although Bantawa (\citealt{doornenbal09}) is among the better described languages, no example of V_2 clearly analyzable as AM could be found in the available publications.

The simplest AM system is that of Thulung (\citealt{lahaussois02thulung} and additional fieldwork) with only one AM marker, the circumambulative \ipa{-bal} (table \ref{tab:thulung.am}).

Dumi, the closest relative of Khaling, has at least the four markers indicated in Table \ref{tab:dumi.am}, possibly more (\citealt[199-214]{driem93dumi}). Note that all the V_2 in this Table have Khaling cognates, with similar functions, and that the AM system in these two languages may go back to their common ancestor.

Wambule (\citealt{opgenort04wambule}) has five AM markers, all of which can have a manipulative `bring/take' meaning (Table   \ref{tab:wambule.am}). The verb form (\ref{ex:seilwasta}) with  \ipa{-lwa-} `go/take and do' provides a clear example of the AM value of this V_2.
 
\begin{exe}
\ex \label{ex:seilwasta}
\gll \ipa{sei-\rouge{lwa}-s-ta} \\
kill.self-\rouge{go/take}-\textsc{detr}-\textsc{imp}:sAS   \\
\glt \sens{go and kill yourself} (\citealt[439]{opgenort04wambule})  (Wambule)
\end{exe}

Yamphu has a particularly rich AM system, with at least seven markers (\citealt[137-194]{rutgers98yamphu}). Among the V_2 listed as AM markers in table \ref{tab:yamphu.am}, note that \ipa{-phæt(t)-} essentially has non-AM functions, but does occur with the meaning `do and leave' as in example (\ref{ex:uŋbhaetto}), hence its analysis as a subsequent translocative AM marker.  
 
 \begin{exe}
\ex \label{ex:uŋbhaetto}
\gll \ipa{uŋ-\rouge{bhæ:tt}-o}	\ipa{uŋ-\rouge{bhæ:tt}-o}	\ipa{sapphi} \\
drink-\rouge{away}-\textsc{arq} drink-\rouge{away}-\textsc{arq}  abundantly\\
\glt \sens{Drink before you leave, drink before you leave, [drink] as much as you can.} (\citealt[150]{rutgers98yamphu}). (Yamphu)
\end{exe}

The Yamphu V_2 \ipa{-las-} `go, do and come back' implies two motion events, and may be similar to the `interrupted motion' category proposed by \citet[123]{rose15am} to describe `situation where the realization of the lexical event occurs between two stretches of motion and where the motion is encoded by a marker distinct from that of `concurrent motion'.


\begin{landscape}	

\begin{table}
\caption{The Thulung AM system} \label{tab:thulung.am} \centering
\begin{tabular}{llllllllll}
\toprule
Source &V_2 & Temp.& Deix. & Vert.& Arg. \\
\midrule
\dhatu{bal}{wind around} &	\ipa{-bal-} `go around doing' &	C &	\textsc{circum}&$\varnothing$&	S/A+P & \\
\bottomrule
\end{tabular}
\end{table}			

\begin{table}
\caption{The Dumi AM system (\citealt[199-214]{driem93dumi})} \label{tab:dumi.am} \centering
\begin{tabular}{llllllllll}
\toprule
V_2 & Gloss& Temp.& Deix. & Vert.& Arg. \\
\midrule
\ipa{-pad-} `to go off to do something' &	Allative&P &		\textsc{trans} &$\varnothing$&	S/A & \\
\ipa{-li(lɨt)-} `to be up and about while doing' &	Frolicsome&C &	\textsc{circum} &$\varnothing$&	S/A & \\
\ipa{-hu:d-} `do and bring back here' &	fetch&S &		\textsc{cis} &$\varnothing$&	S/A+P & \\
\ipa{-pid-} `do and bring it over here' &	bring&S &		\textsc{cis} &$\varnothing$&	S/A+P & \\
\bottomrule
\end{tabular}
\end{table}				

\begin{table}
\caption{The Wambule AM system (\citealt{opgenort04wambule})} \label{tab:wambule.am} \centering
\begin{tabular}{llllllllll}
\toprule
Source &V_2 & Gloss &Temp.& Deix. & Vert.& Arg. \\
\midrule
\dhatu{pi}{come (horizontal)} &	\ipa{-phi-} `come/bring and do' &come/bring:\textsc{hrz}&	P &		\textsc{cisl} &same level&	S/A+P \\
\dhatu{phit}{bring (horizontal)} & \\
\dhatu{ga}{come up} &	\ipa{-kha-} `come/bring up' &?&	P &		\textsc{cisl} & up&	S/A+P \\
\dhatu{khat}{bring up} &	 \\
\dhatu{ywa}{come down} &	\ipa{-ywa,hywa-} `come/bring down' &come/bring:\textsc{down}&	P/I &		\textsc{cisl} & down&	S/A+P \\
\dhatu{hywat}{bring down} &	 \\
\dhatu{di}{go and come back} &	\ipa{-di,du-} `go/take (and come back)' &go/take/come&	P &		\textsc{transl} & $\varnothing$&	S/A+P \\
\dhatu{lwa}{go} &	\ipa{-lwa-} `go/take and do' &go/take &	P &		\textsc{cisl} &same level&	S/A+P \\
\dhatu{lyat}{take away} &	\\
\bottomrule
\end{tabular}
\end{table}	

\begin{table}
\caption{The Yamphu AM system  (\citealt[137-194]{rutgers98yamphu}).} \label{tab:yamphu.am} \centering
\begin{tabular}{llllllllll}
\toprule
Source &V_2 & Gloss &Temp.& Deix. & Vert.& Arg. \\
\midrule
  &	\ipa{-ca/cæt-} `come and do' &come/bring &	P &		\textsc{cisl} & $\varnothing$&	S/A+P \\
\dhatu{ap}{come (horizontal)}    &	\ipa{-ap(t)-} `come and do' &come/bring &	P &		\textsc{cisl} & same level &	S/A+P \\
\dhatu{uks}{come down}    &	\ipa{-uk(t)-} `come down and do' &come/bring down &	P/C &		\textsc{cisl} & same level &	S/A+P \\
\dhatu{kæt}{come up}    &	\ipa{-kæt(t)-} `do and come/bring up' &come/bring up &S &		\textsc{cisl} & same level &	S/A+P \\
   &	\ipa{-phæt(t)-} `do and leave' &away &S &		\textsc{transloc} & $\varnothing$ &	S/A  \\
\dhatu{las}{go and come back}    &	\ipa{-las-} `go, do and come back' &go come &S+P &		\textsc{cisl} & $\varnothing$ &	S/A+P \\
   &	\ipa{-tus/tit-} `go around doing' &around &C &		\textsc{circum}& $\varnothing$ &	S/A  \\
\bottomrule
\end{tabular}
\end{table}	


\begin{table}
\caption{The Yakkha AM system  (\citealt[283-328]{schackow15yakkha}).} \label{tab:yakkha.am} \centering
\begin{tabular}{llllllllll}
\toprule
Source &V_2 & Gloss &Temp.& Deix. & Vert.& Arg. \\
\midrule
  &	\ipa{-kheʔ/t-} `do and go/carry' &V_2.\textsc{go/carry.off} &	S &		\textsc{trans} & $\varnothing$&	S/A, A+P  \\
  &	\ipa{-uks/t-} `do and come/bring down' &V_2.\textsc{come.down}/\textsc{bring.down} &	S &		\textsc{cis} & down&	S/A+P  \\
  &	\ipa{-ap(t)-} `do and come/bring' &V_2.\textsc{come}/\textsc{bring} &	S &		\textsc{cis} & same level&	S/A+P  \\
%  \dhatu{phes}{bring}    &	\ipa{-ap(t)-} `come and do' &bring &	P &		\textsc{cisl} & same level &	S/A+P \\
  &	\ipa{-ghond-} `go around doing' &V_2.\textsc{roam} &	C &	 \textsc{circum}&  $\varnothing$&	S/A   \\
\bottomrule
\end{tabular}
\end{table}	

\begin{table}
\caption{The Belhare AM system  (\citealt{bickel96aspect, bickel97spatial, bickel17belhare}).} \label{tab:belhare.am} \centering
\begin{tabular}{llllllllll}
\toprule
Source &V_2 & Gloss &Temp.& Deix. & Vert.& Arg. \\
\midrule
  &	\ipa{-itt-} `go up and do' &go.\textsc{upwards} &	S &		\textsc{cis} & down&	S/A+P  \\
 
  &	\ipa{-ap(t)-} `do and come/bring' &bring.\textsc{across} &	P&		\textsc{trans} & up&	S/A+P  \\
%  \dhatu{phes}{bring}    &	\ipa{-ap(t)-} `come and do' &bring &	P &		\textsc{cisl} & same level &	S/A+P \\
  &	\ipa{-kon-} `go around doing' &spatially distributed temporary &	C &	 \textsc{circum} &  $\varnothing$&	S/A   \\
\bottomrule
\end{tabular}
\end{table}	

\begin{table}
\caption{The Hayu AM system (\citealt[151]{michailovsky88}} \label{tab:hayu.am} \centering
\begin{tabular}{llllllllll}
\toprule
Source &V_2 & Temp.& Deix. & Vert.& Arg. \\
\midrule
\dhatu{lat}{go} &	\ipa{-la(t)-} `go and do' &	P &	\textsc{trans}&$\varnothing$&	S/A & \\
\bottomrule
\end{tabular}
\end{table}			
\end{landscape}


\section{Non-affixal AM}
The term `AM' has rarely been used to describe the grammar of ST languages other than Gyalrongic and Kiranti  -- to our knowledge, only the recent grammars of Karbi (\citet{konnerth14karbi}) and Hakhun Tangsa (\citet{boro17hakhun}), and some recent works on Sinitic (such as  \citealt{lamarre17motion.cum}) make use of this term. These languages in any case lack dedicated affixal AM markers.

\subsection{Sinitic}
AM has been rarely addressed in linguistic studies of the Sinitic languages. It has not been described until in \cite{lamarre17motion.cum} and \cite{lamarre17deictic} where the sequences [VP+\zh{去/来} \ipa{qu/lai}] \sens{go/come to do} in Northern Sinitic languages are analyzed as \textsc{am} encoding. In (\ref{distribution2}) and (\ref{distribution1}), the andative enclitic \zh{去} \ipa{qu} and the ventive enclitic \zh{来} \ipa{lai} respectively encode a translocative and a cislocative motion in addition to the action denoted by the VP.

\begin{exe}
\ex Standard Mandarin \citep{lamarre17deictic} \label{distribution2}
\glll
\zh{喝} \zh{点儿} \zh{水}=\zh{去} \\
\ipa{Hē} \ipa{diǎnr} \ipa{shuǐ}=\ipa{qu} \\
drink  some water=\textsc{go}$\&$ \\
\glt \sens{Go and drink some water!}.
\end{exe}

\begin{exe}
\ex Standard Mandarin \citep{lamarre17motion.cum} \label{distribution1}
\glll
\zh{你} \zh{干} \zh{嘛}=\zh{来} \zh{了} \\
\ipa{Nǐ} \ipa{gàn} \ipa{má}=\ipa{lai} \ipa{le} \\
\textsc{2sg} do what=\textsc{come}$\&$ \textsc{part} \\
\glt \sens{What are you coming for?}
\end{exe}

Note that unlike Gyalrongic and Kiranti, the host of the AM marker is not the verb itself, but the verb phrase, and occurs after the object as in (\ref{distribution2}) and (\ref{distribution1}). Its is thus a borderline case of AM; note that \citet{guillaume16am}, which is restricted to affixal AM markers, does not include markers that are  clitics. However, it does conform to Guillaume's definition (see § XXX) and given the problematic and controversial nature of word boundaries in Sinitic, it is unsurprising that these languages lack clearly affixal AM markers.

The [VP+\zh{去/来} \ipa{qu/lai}] construction was previously described as a type of \sens{purpose construction} \citetext{\citealp{lu1985vpqu}; \citealp{yang2012mudi}}, the final \zh{去/来} \ipa{qu/lai} analyzed as `particles of purpose' \citep[479]{chao68chinese}. The construction is considered to be interchangeable to some degree with the serial verb construction (henceforth SVC) [\zh{去/来} \ipa{qu/lai}+VP]. We can compare (\ref{distribution2}) and (\ref{distribution1}) with (\ref{inventory2}) and (\ref{inventory1}) below. Noted that as \textsc{am} markers (in \ref{distribution2} and \ref{distribution1}), the post-verbal \zh{去/来} \ipa{qu/lai} lose their verbality to some degree and show tendency to become bound morphemes, with phonetic weakening compared to their source verbs (in \ref{inventory2} and \ref{inventory1}). \citealp[479]{chao68chinese}; \citealp{lu1985vpqu}; \cite{lamarre17motion.cum} ; \cite{lamarre17deictic}).

\begin{exe}
\ex Standard Mandarin \citep{lamarre17deictic} \label{inventory2}
\glll
\zh{去} \zh{喝} \zh{点儿} \zh{水} \\
\ipa{Qù} \ipa{hē} \ipa{dianr} \ipa{shuǐ} \\
go drink some water \\
\glt \sens{Go and drink some water!}
\end{exe}

As shown by (\ref{distribution1}) and (\ref{inventory1}), AM markers in Mandarin allow pluri-actional reading, unlike those found in Japhug (see § \ref{sec:am.interrogative}).

\begin{exe}
\ex Standard Mandarin \label{inventory1}
\glll
\zh{你} \zh{来} \zh{干} \zh{什么}  \\
\ipa{Nǐ} \ipa{lái} \ipa{gàn} \ipa{shénme}  \\
\textsc{2sg} come do what \\
 \glt \sens{What are you coming for?}
\end{exe}

\begin{table} [H]
\caption{\textsc{am} markers in Mandarin} \centering
\begin{tabular}{llllllllll}
\toprule
Source &Enclitic & Temp.& Deix. & Vert.& Arg. \\
\midrule
\zh{去} \ipa{qù} &	\ipa{=qu} `go and X' &	P &		$\varnothing$ &$\varnothing$&	S/A & \\
 \zh{来}  \ipa{lái} &	\ipa{=lai} `come and X' &	P &		$\varnothing$ &$\varnothing$&	S/A & \\
\bottomrule
\end{tabular}
\end{table}

%Based on Chappell's (\citeyear{chappell15areas}) typological and geographical classification of the Sinitic languages in China and a preliminary survey (based on elicitation) on the acceptance of  the [VP+\zh{去/来} \ipa{qu/lai}] construction in representative varieties of each of these languages, Table (\ref{inventorytable1}) shows the general distribution of \textsc{am} marking in the Sinitic languages. The grayed cells correspond to the languages where AM marking  is found. The lighter the cells, the less likely that AM is present.

No Sinitic language has  dedicated AM markers. Depending on the variety, AM enclitics can also mark orientation (in particular when used with motion verbs), aspect and imperative.

%\begin{table} [H] \label{inventorytable1}
%\caption{Geographic distribution of \textsc{am} markers of the Sinitic languages of China} \centering
%%\resizebox{\columnwidth}{!}{
%\begin{tabular}{lllllllll}
%\toprule
%& Language branch & Dialect & \textsc{am} markers  \\
%\midrule
%\grise{I}. & \grise{Mandarin \zh{官话}} & \grise{}&  \grise{} \\
%\grise{} & \small \grise{Standard Mandarin} & \grise{} &  \grise{\zh{去} \zh{来} } \\
% \grise{}& \small \grise{Beijing \zh{北京}}  &  \grise{Beijing \zh{北京}} & \grise{\zh{去} \zh{来} }  \\
% \grise{}& \small \grise{Northwestern (Lanyin \zh{兰银})} & \grise{Yinchuan \zh{银川}} &\grise{\zh{去} \zh{来} \zh{走}} \\
%  \grise{}& \small \grise{} & \grise{Dachuan \zh{达川}} & \grise{\zh{去} \zh{来}} \\
% \grise{}& \small \grise{Jilu \zh{冀鲁}}  & \grise{Jizhou \zh{冀州}} &   \\
% \grise{} & \small \grise{Jiaoliao \zh{胶辽}}&   \grise{Qingdao \zh{青岛}}   & \\
% \grise{} & \small  \grise{Dongbei \zh{东北}}  &  \grise{Changchun \zh{长春}}  &  \\
% \grise{} & \small \grise{Central plains (Zhongyuan \zh{中原})} & \grise{Xi'an \zh{西安} }  &  \grise{\zh{去} \zh{来} }  \\
%  \grise{} & \small \grise{Southwestern \zh{西南}}  & \grise{Chengdu \zh{成都}} & \grise{\zh{去} \zh{来} }  \\
%   \grise{} & \small \grise{}  & \grise{Wuhan \zh{武汉}} &  \grise{\zh{去} \zh{来} } \\
% \grise{} & \small \grise{Jianghuai \zh{江淮} (Southern)}  & \grise{Nanjing \zh{南京}}&  \\
%  \grise{} & \small \grise{}  & \grise{Anqing \zh{安庆}}& \grise{\zh{去} \zh{来} }  \\
%\grise{II.} & \grise{Jin  \zh{晋}} & \grise{Shenmu \zh{神木}} &  \grise{\zh{去} \zh{来} }  \\
%\grise{} & \grise{} & \grise{Datong \zh{大同}} &  \grise{\zh{去} \zh{来} }  \\
%\grise{III.} &\grise {Xiang \zh{湘}} & \grise{Changsha \zh{长沙}} &   \grise{\zh{去} \zh{来} }  \\
%\grise{} &\grise {} & \grise{Changde \zh{常德}} & \grise{\zh{去} \zh{来}? }    \\
%\gray{IV.} & \gray{Wu \zh{吴}} &  \gray{Shanghai \zh{上海}} & \gray{\zh{去} \zh{来}} \\
%\gray{} & \gray{} &  \gray{Hangzhou \zh{杭州}} &  \gray{\zh{去}}  \\
%\gray{} & \gray{} &  \gray{Suzhou \zh{苏州}} & \gray{None} \\
%\gray{V.} & \gray{Hui \zh{徽}} &  \gray{Huizhou \zh{徽州}} & \gray{\zh{去}}   \\
%\lightgray{VI.} & \lightgray{Gan \zh{赣}} & \lightgray{Northeastern Gan} & \lightgray{\zh{去} \zh{来}} \\
% & &  Fuzhou \zh{抚州} & None  \\
%VII. & Min  \zh{闽} & Yilan \zh{宜兰}  & None  \\
%VIII. & Kejia \zh{客家} (Hakka) & &  None  \\
%IX. & Yue \zh{粤} (Cantonese) & & None   \\
%X. & Pinghua \zh{平话} and Tuhua \zh{土话}   & Rucheng \zh{汝城} & None \\
%\bottomrule
%\end{tabular}%}
%\end{table}
% 

 
 


\subsection{Karbi}
\citet{konnerth14karbi, konnerth15cisloc} reports the existence of a cislocative AM proclitic \ipa{nang=} in Karbi, whose use can be exemplified by (\ref{ex:karbi.AM}).

\begin{exe}
\ex \label{ex:karbi.AM}
\gll \ipa{alàng-lì=ke}	\ipa{là-tūm}	\ipa{a-hēm=si}	\ipa{\rouge{nang}=vùr-si}	\ipa{sá}	\ipa{aját}	\ipa{\rouge{nang}=jùn-lò} \\
 3-\textsc{hon}=\textsc{top} this-\textsc{pl} \textsc{poss}-house=\textsc{foc} \rouge{\textsc{cis}}=drop.in-\textsc{nf}:\textsc{rl} tea \textsc{genex} \rouge{\textsc{cis}}=drink-\textsc{rl} \\
 \glt `...it was him, at their house we stopped by and had tea [{come} and drink tea] and everything.'
\end{exe}

In addition, this proclitic occurs as a person indexation marker, used for first or second person non-subject (including local scenarios 1$\rightarrow$2  and 2$\rightarrow$1)as in (\ref{ex:karbi.2}) and as a marker of cislocative orientation with (\ref{ex:karbi.orientation}) or without motion (\ref{ex:karbi.orientation2}).

\begin{exe}
\ex \label{ex:karbi.2}
\gll \ipa{nang-phān=ke}	\ipa{nang=kV-pòn-pò}\\
 you-\textsc{nsubj}=\textsc{top} 1/2:\textsc{nsubj}=\textsc{ipfv}-take.away-\textsc{irr1} \\
 \glt ‘[…] (I) will carry you away’ 
\end{exe}

\begin{exe}
\ex \label{ex:karbi.orientation}
\gll \ipa{lasō}	\ipa{a-hūt}	\ipa{amāt}	\ipa{[e-nūt}	\ipa{a-kV-prék}	\ipa{a-monít}	\ipa{abàng=ke]}	\ipa{saikél}	\ipa{nang=ardòn-si}	\ipa{vàng-lò}… \\
 this \textsc{poss}-during and.then one-\textsc{clf}:\textsc{hum}.\textsc{sg} \textsc{poss}-\textsc{nmlz}-be.different  \textsc{poss}-man \textsc{npdl}=\textsc{top} bicycle(<Eng) \textsc{cis}=ride-\textsc{nf}:\textsc{rl} come-\textsc{rl} \\
 \glt ‘in this moment, another person riding on a bicycle came, 
\end{exe}

 \begin{exe}
\ex \label{ex:karbi.orientation2}
\gll \ipa{angsóng=pen=si}	\ipa{phén}	\ipa{nang=jāng-lìng} \\
 up.high=from=\textsc{foc} fan(<Eng) \textsc{cis}=hang-small:\textsc{s} \\
\glt  ‘the fan is hanging down from up high (from the ceiling)’ (Karbi)
 \end{exe}

 
\subsection{Hakhun Tangsa}

 \citet[311-312]{boro17hakhun} analyses the construction in (\ref{ex:hakhun}) as AM. It remains however unclear from the description whether an alternative analysis as MVC  is not possible.
 
 \begin{exe}
\ex \label{ex:hakhun}
\gll   \ipa{inɤ́}	\ipa{ʒuk}	\ipa{\rouge{kà}}	\ipa{l-oʔ} \\
there drink \rouge{go} \textsc{imp}-\textsc{2sg} \\
\glt `{Go} and drink (tea) there.’ (Hakhun Tangsa)
\end{exe}

%\subsection{Burmese}
%Burmese has four preverbal (\ipa{la} `come', \ipa{θwà}  `go', \ipa{laiʔ} `follow, go around and', \ipa{cauʔ} `walk, go around and') and two postverbal auxiliaries (\ipa{-la} `come' and \ipa{-θwà/ðwà} `go') that can be used to express a motion event (\citealt[185-190;203-5]{Jenny16grammar}) and are candidates to be analyzed as AM.
%
%Preverbal auxiliaries are used for prior and concurrent motion events (\ref{ex:laiq}) and postverbal ones to express subsequent motion (\ref{ex:saladay}).
%
% \begin{exe}
%\ex \label{ex:laiq}
%\gll   \ipa{θu}	\ipa{ywa-hma}	\ipa{paiʔsʰan}	\ipa{\rouge{laiʔ}-tàun-dɛ}  \\
%3 village-at  money \rouge{follow}-ask.for-\textsc{nfut} \\
%\glt `He went around the village begging for money' (\citealt[189]{Jenny16grammar}, Burmese).
%\end{exe}
%
% \begin{exe}
%\ex \label{ex:saladay}
%\gll   \ipa{θu-dó}	\ipa{tʰəmìn}	\ipa{sà-\rouge{la}-dɛ} \\
%3-\textsc{pl} cooked.rice eat-\rouge{come}-\textsc{nfut}  \\
%\glt ‘They have eaten (\rouge{before coming here}).' (\citet[388]{Jenny16grammar}, Burmese)
%\end{exe}
 

\section{Summary}

\resizebox{\columnwidth}{!}{
\begin{tabular}{llllllllll}
\toprule
Subgroup&Language  &Number &Temp. & Deixis & Vertical & Other & Mono- &Arg.\\
&&&relation &&dimension&functions& actional & \\
\midrule
Gyalrong&Japhug & 2&P & \textsc{cis}/\textsc{trans} & $\varnothing$ &  $\varnothing$ & \Y &S/A \\
&Tshobdun & 2&P & \textsc{cis}/\textsc{trans} & $\varnothing$ &  $\varnothing$ & ? &S/A \\
&Zbu & 2&P & \textsc{cis}/\textsc{trans} & $\varnothing$ &  $\varnothing$ & ? &S/A \\
&Situ &2&P & \textsc{cis}/\textsc{trans}  & $\varnothing$ & Aspect&\N &S/A \\
\midrule
Kiranti &Khaling & 7&P, C, S &\textsc{cis}/\textsc{trans}/\textsc{circum} & \Y & Aspect,  &  \N  &S/A, S/A+P \\
&&&&&&Orientation&&\\
&Dumi & 4? &P, C, S &\textsc{cis}/\textsc{trans}/\textsc{circum}  &  $\varnothing$ & Aspect,  &  ?&S/A, S/A+P \\
  &&&&&&Orientation&&\\
&Thulung & 1 & C  &\textsc{circum} &  $\varnothing$ & $\varnothing$ &?  &S/A  \\
&Wambule & 5  & P  &\textsc{cis}/\textsc{trans}/\textsc{circum}  &  \Y & Aspect,  &?  &S/A, S/A+P \\
     &&&&&&Orientation&&\\
%    &Jero & 2 & C, P?  & Trans? &  \Y & ?  &?  &S/A  \\
&Yamphu & 7  & P, C, S  &\textsc{cis}/\textsc{trans}/\textsc{circum}  &  \Y & Aspect,  &?  &S/A, S/A+P \\
        &&&&&&Orientation&&\\
&Yakkha & 7? & P, C, S  & \textsc{cis}/\textsc{trans}/\textsc{circum}  &  \Y & Aspect,  &?  &S/A, S/A+P \\
        &&&&&&Orientation&&\\
&Belhare & 3? & P, C, S  & \textsc{cis}/\textsc{trans}/\textsc{circum}  &  \Y & Aspect,  &?  &S/A, S/A+P \\
        &&&&&&Orientation&&\\
%&Hayu & 1  & P  & \textsc{trans} &   $\varnothing$& Aspect,  &?  &S/A  \\
%        &&&&&&Voice&&\\
%\midrule
%Sinitic & Mandarin &2& P & \textsc{cis}/\textsc{trans} & $\varnothing$ &Aspect,  Modality&\N   &S/A  \\
%        &&&&&&Orientation&&\\
%\midrule
%Karbi&&1&S&\textsc{cis}&  $\varnothing$ &Person, Orientation&?  &S/A +P \\
%\midrule
%Sal&Tangsa&1&P&\textsc{trans}&  $\varnothing$ & ?&?  &S/A  \\
%\midrule
%LB &Burmese & 6?&P, C, S &\textsc{cis}/\textsc{trans}/\textsc{circum} & $\varnothing$  & Aspect,  &  ?  &S/A, S/A+P \\
%&&&&&&Orientation&&\\
\bottomrule
 \end{tabular}}
 
\section{Conclusion}

%AM systems have been grammaticalized from motion verbs independently several times in the ST family, and present considerable typological diversity. Each of the three subgroups discussed in detail in this work contributes to the typology of AM in a different way.
%
%Gyalrong languages have simple AM systems with only two members, but the high frequency of the markers in texts makes it possible to describe in greater detail the semantic properties of AM in these languages. A corpus study on Japhug shows in particular that AM markers differ in many ways from the corresponding purposive construction with motion verbs (MVC): several syntactic tests show that the motion event cannot be separated from the action of the main verbs in the case of AM. Moreover, AM can be used with non-volitional verbs, unlike MVC.
%
%Kiranti languages have richer AM systems, but their AM markers are less common in texts, and lack echo phenomena (unlike Gyalrong, \ref{sec:AM.echo}) and monoactionality. They do however encode more parameters, in particular the temporal relation (AM markers for prior, concurrent and subsequent motion event are found), the vertical dimension and also include manipulative AM markers (`bring/take and X'). The languages with the highest number of AM markers in the ST family (Khaling and Yamphu) are found in this branch.
%
%Sinitic languages have borderline cases of AM systems, as AM markers in these languages are clitics attached on the verb phrase rather than verbal affixes. The diversity of this branch and the existence of continuous written attestations potentially allows to follow in greater detail the grammaticalization process of AM. The study of AM in this branch however offers particular challenged, in particular distinguishing between AM and directional complements.
%
%The present work contributes to the typology of AM by presenting in more detail a range of tests and criteria that can be used to refine the study of the semantic contrast between AM and other constructions. Previous descriptive work on AM in ST, even in the most detailed grammars, rarely provides explicit examples making it possible to determine whether pluriactional interpretation of the motion event + action is possible. The sharp contrast between Japhug on the one hand, and Khaling with regards to this parameter shows that fieldworkers should devote specific attention to this providing examples which make clear the presence or absence of this phenomenon.

\bibliographystyle{unified}
\bibliography{bibliogj}

 \end{document}
 