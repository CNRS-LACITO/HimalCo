\documentclass[oneside,a4paper,11pt]{article}  
\usepackage{fontspec}
\usepackage{natbib}
\usepackage{booktabs}
\usepackage{xltxtra} 
% \usepackage{geometry}
% \geometry{
% a4paper,
% total={210mm,297mm},
% left=20mm,
% right=20mm,
% top=30mm,
% bottom=30mm,
% }
\usepackage{polyglossia} 
\usepackage[table]{xcolor}
\usepackage{gb4e} 
\usepackage{multicol}
\usepackage{graphicx}
\usepackage{float}
\usepackage{hyperref} 
\hypersetup{bookmarks=false,bookmarksnumbered,bookmarksopenlevel=5,bookmarksdepth=5,xetex,colorlinks=true,linkcolor=blue,citecolor=blue}
\usepackage[all]{hypcap}
\usepackage{memhfixc}
\usepackage{lscape} 
 \usepackage{multicol}
 \usepackage{amssymb}
 \usepackage{bbding}
 
\setmainfont[Mapping=tex-text,Numbers=OldStyle,Ligatures=Common]{Charis SIL} 
\newfontfamily\phon[Mapping=tex-text,Ligatures=Common,Scale=MatchLowercase]{Charis SIL} 
\newcommand{\ipa}[1]{{\phon\textit{#1}}} 
\newcommand{\grise}[1]{\cellcolor{lightgray}\textbf{#1}}
\newfontfamily\cn[Mapping=tex-text,Ligatures=Common,Scale=MatchUppercase]{SimSun}%pour le chinois
\newcommand{\zh}[1]{{\cn #1}}
\newcommand{\Y}{\Checkmark} 
\newcommand{\N}{\XSolidBrush} 
\newcommand{\tld}{\textasciitilde{}}
\XeTeXlinebreaklocale "zh" %使用中文换行 
\XeTeXlinebreakskip = 0pt plus 1pt % 
\newcommand{\forme}[1]{\textit{\phon#1}}  
\newcommand{\japhug}[2]{\textit{\phon#1} `#2'}  
 \newcommand{\redp}{\textasciitilde}
   \newcommand{\sens}[1]{‘#1’}
 \newcommand{\fl}{$\rightarrow$}
 \newcommand{\dhatu}[2]{|\ipa{#1}| `#2'}
 \newcommand{\gray}[1]{\cellcolor{lightgray!30}{#1}}
 \begin{document} 
\title{Associated motion in Sino-Tibetan/Trans-Himalayan}
\author{Guillaume Jacques\\ Aimée Lahaussois \\ Zhang Shuya}
\maketitle
%\textbf{Abstract}: While the term `associated motion' has up to now only been applied to a relatively limited number of Sino-Tibetan languages (Japhug and Karbi, see \citealt{jacques13harmonization} and \citealt{konnerth15cisloc}), the phenomenon itself is not rare in this family. This paper comprises three sections. 
%
%First, I provide an overview of associated motion morphology in Sino-Tibetan, focusing especially on Kiranti and Gyalrongic, the two subgroups of the family where this morphology is most complex,  on the basis of fieldwork data and published literature (in particular \citealt{driem87}, \citealt{driem93dumi}, \citealt{doornenbal09} and \citealt{schackow15yakkha}). 
%
%Second, I present a preliminary typological comparison of Associated Motion in Sino-Tibetan with South America (\citealt{guillaume16am}) and Australia (\citealt{koch84associated.motion}) where this grammatical category has been identified and studied for a longer time. 
%
%Third, I discuss the sources of associated motion affixes in Sino-Tibetan, in particular the question whether these affixes have origins other than motion verbs.

\section{Introduction}

Following the existing literature on Associated motion (in particular \citealt{koch84associated.motion}, \citealt{wilkins91associated.motion} and \citealt{guillaume16am}), the following six parameters have been considered when analyzing the data relating to AM systems in ST.

\begin{enumerate}
\item Temporal relation (prior, concurrent, subsequent)
\item Deixis (cis- vs translocative)
\item Vertical dimension
\item Argument of motion (S/A vs P)
\item Non-AM meanings (orientation without motion, aspect, person configuration, voice)
\item Mono- vs pluriactionality
\end{enumerate}

In this survey, we have excluded markers of aspect and orientation XXXX

\section{Gyalrongic}

\subsection{General overview}
Associated motion prefixes are reported in Japhug (\citealt{jacques13harmonization}), Zbu (Gong Xun, p.c.), Tshobdun (\citealt{jackson14morpho}) and Situ (\citealt[200-204]{zhang16bragdbar}, \citealt[497-500]{prins16kyomkyo}), but not found in other Gyalrongic languages (for example in Khroskyabs, cf \citealt{lai17khroskyabs}).

All four Gyalrong languages have two associated motion prefixes (see Table \ref{tab:am-gyalrong} from \citealt[200]{zhang16bragdbar}) grammaticalized from the motion verbs `go' and `come'. The fact that motion verbs were grammaticalized as prefixes rather than suffixes in strict verb-final languages like Japhug and Situ can be accounted for by assuming that they comes from the first member of a former serial verb construction (\citealt{jacques13harmonization}). Unlike in Kiranti languages, AM markers in Gyalrong do not have redundant person and TAM markers.

\begin{table}[H]
\caption{Associated motion prefixes in Gyalrong languages} \centering \label{tab:am-gyalrong}
\begin{tabular}{lllll}
\toprule
&come & \textsc{cisloc} & go & \textsc{transloc} \\
\midrule
Japhug &  \ipa{ɣi} &\ipa{ɣɯ-} &\ipa{ɕe} &\ipa{ɕɯ-, ɕ-, ʑ-,z- } \\
Kyom-kyo (Situ) &\ipa{vi} &\ipa{və-} &\ipa{tʃʰi} &\ipa{ʃi-} \\
Brag-dbar (Situ) &\ipa{βʑē, və} &\ipa{ɟɐ-} &\ipa{tɕʰê} &\ipa{ɕɐ-} \\
\bottomrule
\end{tabular}
\end{table}

AM prefixes in Japhug and other Gyalrong languages refer to a motion event occurring before the action of the main verb. The argument undergoing the motion event is always the subject (S/A), except in the case of causative constructions, when it can be either causer or causee (as in \ref{ex:GWchWsWXtWnW}).

\begin{exe}
\ex \label{ex:GWchWsWXtWnW}
\gll
\ipa{tɕe} 	\ipa{kupa} 	\ipa{chu} 	\ipa{nɯra} 	\ipa{atʰi} 	\ipa{pɕoʁ} 	\ipa{nɯra,} 	\ipa{ɯ-pɕi} 	\ipa{nɯra} 	\ipa{kɯ} 	\ipa{kɯreri} 	\ipa{ɣɯ-cʰɯ-sɯ-χtɯ-nɯ} 	\ipa{ŋu.}  \\
\textsc{lnk} Chinese \textsc{loc} \textsc{dem:pl} downstream direction \textsc{dem:pl} \textsc{3sg}-outside  \textsc{dem:pl}  \textsc{erg} here \textsc{cisloc-ipfv:downstream-caus}-buy-\textsc{pl} be:\textsc{fact} \\cd \
\glt \sens{People from the Chinese areas, people from outside send people to come here to buy (matsutake and sell them in the areas downstream).} (20 grWBgrWB 58)
  \end{exe} 

In Situ the cislocative can be used with a prospective aspectual  value (\citealt{linyj03tense}, \citealt[204]{zhang16bragdbar}), but in all other Gyalrong languages, including Japhug, only marks associated motion.

While the AM systems found in Gyalrong languages are much simpler and less diverse than those found in Kiranti, a topic of interest in Gyalrong is the issue of the separability of the motion event and the verbal action. Unlike other languages with AM such as Kiranti or Tungusic (\citealt{stojnova16nda, fuente18am}), in Japhug the scope of negation, interrogation, conditionals and complement-taking verbs applies to the entire verbal event (motion+main action), as shown in § \ref{sec:am.japhug}. A purposive construction with a motion verb must be used to express cases when the verbal action is outside of their scope. This is however not a characteristic common to all Gyalrong languages. In the closely related Situ language, the semantic distinction between AM and the corresponding purposive construction is less clear (§ \ref{sec:am.situ}).

 \subsection{Pluri- vs monoactionality in Japhug} \label{sec:am.japhug}
To express the meaning of motion prior to an action, associated motion prefixes are nearly two times as common as corresponding motion verb constructions (henceforth MVC) in the Japhug corpus. There is however a clear semantic difference between the two constructions, which was briefly described in \citet{jacques13harmonization}, but is presented here in more detail.

AM and MVC differ from each other in that in the former, the completion of both motion event and verbal action is presupposed (AM is monoactional), whereas in the case of the latter, the two can be separated. This mono- vs. pluractionality contrast is most conspicuous in past perfective forms, and can be observed in four types of constructions: concessives (with negation of the verbal action), interrogatives, conditionals and complement clauses.

\subsubsection{Concessive} \label{sec:am.concessive}
A MVC  with the motion verb in perfective form can be followed by a clause negating the purposive action, as in (\ref{ex:nAkWrtoR}). In this example, only the motion is realized, while the action expressed by the verb \japhug{rtoʁ}{look} could not be accomplished.

\begin{exe}
\ex \label{ex:nAkWrtoR}
\gll nɤ-kɯ-rtoʁ jɤ-ɣe-a ri, mɯ-nɯ-atɯɣ-tɕi, mɯ-pɯ-ta-mto. \\
\textsc{1sg.poss}-\textsc{nmlz}:S/A-see \textsc{pfv}-come[II]-\textsc{1sg} \textsc{lnk} \textsc{neg}-\textsc{pfv}-meet-\textsc{1du} \textsc{neg}-\textsc{pfv}-1\fl2-see \\
\glt `I came to see you but I did not see you.' 
\end{exe}

With the corresponding AM verb form \japhug{ɣɯ-jɤ-ta-rtoʁ}{I came to see you}, negating the action of the verb is self-contradictory and nonsensical, and a sentence such as (\ref{ex:GWjAtartoR}) is incorrect.

\begin{exe}
\ex \label{ex:GWjAtartoR}
\gll $\dagger$ɣɯ-jɤ-ta-rtoʁ ri mɯ-pɯ-ta-mto \\
\textsc{cisloc}-\textsc{pfv}-1\fl2-look \textsc{lnk} \textsc{neg}-\textsc{pfv}-1\fl2-see \\
\glt Intended meaning: `I came to see you but I did not see you.' 
\end{exe}

Additional minimal pairs of the same type are presented in \citet[202-203]{jacques13harmonization}.


\subsubsection{Interrogative} \label{sec:am.interrogative}
In interrogative clauses, MVCs are required to express meanings such as `What/who have you come/gone to X', as in example (\ref{ex:tChi.WkWpa}), an example which occurs nine times in the corpus.

\begin{exe}
\ex \label{ex:tChi.WkWpa}
\gll tɕʰi ɯ-kɯ-pa jɤ-tɯ-ɣe? \\
what \textsc{3sg.poss}-do \textsc{pfv}-2-come[II] \\
\glt `What did you come to do?' (nine examples in the corpus)
\end{exe}

The difference between MVC and AM in interrogatives can be illustrated by comparing the minimal pair  (\ref{ex:tChi.WkWndza}) and (\ref{ex:tChi.GWtAtWndzat}). example (\ref{ex:tChi.WkWndza}), which has the same structure as (\ref{ex:tChi.WkWpa}), implies that the addressee has not eaten yet, while (\ref{ex:tChi.GWtAtWndzat}) with associated motion can only be used if the food ingestion has already taken place, and requires a different translation.

\begin{exe}
\ex \label{ex:tChi.WkWndza}
\gll tɕʰi ɯ-kɯ-ndza jɤ-tɯ-ɣe? \\
what \textsc{3sg.poss}-eat \textsc{pfv}-2-come[II] \\
\glt `What have you come to eat?' (elicited)
\end{exe}

\begin{exe}
\ex \label{ex:tChi.GWtAtWndzat}
\gll tɕʰi ɣɯ-tɤ-tɯ-ndza-t \\
what \textsc{cisloc}-\textsc{pfv}-2-eat-\textsc{pst:tr}    \\
\glt `What did you eat, after you came here?' (elicited)
\end{exe}

\subsubsection{Conditional} \label{sec:am.conditional}
The presuppositional difference between MVC and AM is also perceptible in the protasis of conditional clauses. 

With MVC in the protasis as in (\ref{ex:mWmAjAtWGe}), there is no presupposition that the verbal action took place, only the motion event constitutes a condition to the state of affair described in the apodosis.

\begin{exe}
\ex \label{ex:mWmAjAtWGe}
\gll nɤ-wa ɯ-kɯ-rtoʁ mɯ\redp{}mɤ-jɤ-tɯ-ɣe nɤ aʑo mɯ-pɯ-kɯ-mto-a. \\
\textsc{1sg.poss}-father \textsc{3sg.poss-}\textsc{nmlz}:S/A-look \textsc{cond}\redp{}\textsc{neg}-\textsc{pfv}-2-come[II] \textsc{lnk} \textsc{1sg} \textsc{neg}-\textsc{pfv}-2\fl{}1-\textsc{1sg} \\
\glt `If you had not come to see your father, you would not have seen me.' (you saw me, but your father was not here)
\end{exe}

By contrast, with AM, the verbal action necessarily took place, as in example (\ref{ex:mWmAGWjAtWrtoR}).

\begin{exe}
\ex \label{ex:mWmAGWjAtWrtoR}
\gll nɤ-wa  mɯ\redp{}mɤ-ɣɯ-jɤ-tɯ-rtoʁ nɤ pɯ-sɤzdɯxpa \\
\textsc{1sg.poss}-father \textsc{cond}\redp{}\textsc{neg}-\textsc{cisloc}-\textsc{pfv}-2-look \textsc{lnk} \textsc{pst.ipfv}-be.pitiful \\ 
\glt `If you had not come to see your father, he would have felt sorry.' (but you did saw him, so he does not feel sorry)
\end{exe}

\subsubsection{Complement clauses} \label{sec:am.conditional}
In complement clauses, verbs with AM prefixes are attested, and complement taking verbs always have scope over both the action of the verb and motion event.

 
In (\ref{ex:mACWkAtshi}), the modal verb \japhug{cʰa}{can} and the double negations (with the specific meaning `cannot help') have scope over both the motion event and the verbal action -- this example is taken from a passage in a story where the king reproaches a small child, who just returned from a mission he himself send him to accomplish, not to have first come to greet him on his return home; the child says these words to justify why he first went to see his mother before greeting the king -- from this context it is clear that both the motion event (to him mother's house, explaining the child's failure to go to see the king) and the action `drink milk' (the reason for that motion event) are equally important to the plot and inseparable. 

\begin{exe}
\ex \label{ex:mACWkAtshi}
\gll  tɯ-nɯ ɯ-kɯ-tsʰi ɲɯ-ɕti-a tɕe, jɤ-azɣɯt-a tɕe, tɯ-nɯ ci mɤ-ɕɯ-kɤ-tsʰi nɯ mɯ́j-cʰa-a \\
\textsc{indef}.\textsc{poss}-breast \textsc{3sg}.\textsc{poss}-\textsc{nmlz}:S/A-drink \textsc{sens}-be.\textsc{affirm}-\textsc{1sg} \textsc{lnk} \textsc{pfv}-arrive-\textsc{1sg} \textsc{lnk} \textsc{indef}.\textsc{poss}-breast  \textsc{indef} \textsc{neg}-\textsc{transloc}-\textsc{inf}-drink \textsc{dem} \textsc{neg}:\textsc{sens}-can-\textsc{1sg} \\
\glt `I am (a toddler) who (still) drinks (his mother's) milk, when I arrived, I could not help but go to drink milk.'  (Norbzang, 262)
 \end{exe}
 
 In (\ref{ex:CWkAmWrkW.mAtWcha}), the negated modal verb has also on the action of both the main verb and the motion event -- the guards would prevent the main character not only to steal, but also to the where the object to be stolen is found.
 
\begin{exe}
\ex \label{ex:CWkAmWrkW.mAtWcha}
\gll ʁmaʁ χsɯ-tɤkʰar kɯ ɲɯ-ɤz-nɤkʰar-nɯ ɕti tɕe, ɕɯ-kɤ-mɯrkɯ mɤ-tɯ-cʰa  \\
solider three-rounds \textsc{erg} \textsc{sens}-\textsc{prog}-surround-\textsc{pl} be.\textsc{affirm}:\textsc{fact} \textsc{lnk}  \textsc{transloc}-\textsc{inf}-steal \textsc{neg}-2-can:\textsc{fact} \\
\glt `Three rounds of soldiers will be surrounding it, you will not be able to (go there and) steal it.' (2003qachga, 55)
   \end{exe}
 
The same observation also applies to  verbs with AM in complement clauses selected by a verb in the protasis, as in(\ref{ex:CWkAmWrkW}): the realization of the verbal action (in addition to that of the motion event) belongs to the condition.

\begin{exe}
\ex \label{ex:CWkAmWrkW}
\gll nɤʑo ɕɯ-kɤ-mɯrkɯ a-pɯ-tɯ-cʰa nɤ aʑo cʰɯ-sɯ-jɣat-a jɤɣ \\
\textsc{2sg} \textsc{transloc}-\textsc{inf}-steal \textsc{irr}-\textsc{ipfv}-2-can \textsc{lnk} \textsc{1sg} \textsc{ipfv}-\textsc{caus}-go.back-\textsc{1sg} be.agreed:\textsc{fact} \\
\glt `If you succeed stealing it (after having gone there), I can cause him to go back there.' (02-montagnes-kamnyu, 46)
\end{exe}

By contrast, in  (\ref{ex:kWrAma.kACe}), in the case of the infinitival complement \forme{kɯ-rɤma kɤ-ɕe} `go to work' with a purposive clause \forme{kɯ-rɤma} (§ XXX), the main verb \japhug{mda}{be time to} only has scope over the motion event expressed by the verb \japhug{ɕe}{go} -- the time that is indicated by the stars refers to the beginning of the journey to work, not the start of the work itself.
 
 \begin{exe}
\ex \label{ex:kWrAma.kACe}
\gll  tɕe kɯɕɯŋgɯ tɕe tɯtsʰot pɯ-me tɕe  nɯnɯ cʰɯ-ɬoʁ lu-ɕqʰlɤt nɯra ɕ-tu-kɯ-ru tɕe, nɯnɯ kɤ-rɤru mda mɤ-mda cʰondɤre kɯ-rɤma kɤ-ɕe mda mɤ-mda nɯtɕu ɕ-tu-kɯ-ru pɯ-ŋgrɤl. \\
 \textsc{lnk} long.ago \textsc{lnk} clock \textsc{pst}.\textsc{ipfv}-not.exist \textsc{lnk} \textsc{dem} \textsc{ipfv}:\textsc{downstream}-come.out \textsc{ipfv}:\textsc{upstream}-disappear \textsc{dem}:\textsc{pl} \textsc{transloc}-\textsc{ipfv}:up-\textsc{genr}:S/P-look \textsc{lnk} \textsc{dem} \textsc{inf}-get.up be.time:\textsc{fact} \textsc{neg}-be.time:\textsc{fact} \textsc{comit} \textsc{nmlz}:S/A-work \textsc{inf}-go be.time:\textsc{fact} \textsc{neg}-be.time:\textsc{fact} \textsc{dem}:\textsc{loc} \textsc{transloc}-\textsc{ipfv}:up-\textsc{genr}:S/P-look \textsc{pst}.\textsc{ipfv}-be.usually.the.case  \\
 \glt  `In former times, there was no clock, and people used to watch when (these stars) came out or disappeared (to know) whether it was time to get up or go to work.' (29-LAntshAm, 66)
  \end{exe}
  
\subsection{Pluri- vs monoactionality in Situ} \label{sec:am.situ}
Minimal pairs similar to those presented above in Japhug have been elicited in Situ. However, it appears that in this language, the use of verbs in perfective form with AM markers does not presuppose that both the verbal action and the motion event have been accomplished. 


In (\ref{ex:nEJAtanaston}), the action of the verb \ipa{nə-ɟɐ-ta-natsô-n}  `I came to see you' with the AM prefix \ipa{ɟɐ-} is negated by the following clauses, while the motion event clearly took place; the corresponding )Japhug example in § \ref{sec:am.concessive} is considered to be not only agrammatical, but also non-sensical. Example (\ref{ex:nEJAtanaston}) has little discernible semantic difference with the corresponding MVC in (\ref{ex:nEkEnatso}).
 

\begin{exe}
\ex \label{ex:nEJAtanaston}
\gll \ipa{nə-ɟɐ-ta-natsô-n} \ipa{rɐ}, \ipa{nəɟə̂} \ipa{nə-ˈtə-mɐ-n}, \ipa{majnə} \ipa{ma-na-ta-mətɐ̂-n}.\\
\textsc{pfv}:downwards-\textsc{cisloc}-2→1-look\textsubscript{II}-2 \textsc{conj} \textsc{2sg} \textsc{sens}-2-do.not.exist\textsubscript{I}-\textsc{2sg} \textsc{conj} \textsc{neg}-\textsc{pfv}-2→1-see\textsubscript{II}-\textsc{2sg}\\
\glt  \sens{I came to see you, but you were not here, so I didn't see you.}
\end{exe}

\begin{exe}
\ex  \label{ex:nEkEnatso}
\gll \ipa{nə-kə-natsō} \ipa{nə-və̄-ŋ} \ipa{rɐnə}, \ipa{nəɟə̂} \ipa{nə-ˈtə-mɐ-n}, \ipa{majnə} \ipa{ma-na-ta-mətɐ̂-n}.\\
\textsc{poss.2sg}-\textsc{nmlz}-see \textsc{pfv}:downwards-come\textsubscript{II}-\textsc{1sg} \textsc{conj} \textsc{2sg} \textsc{sens}-2-do.not.exist\textsubscript{I}-\textsc{2sg} \textsc{conj} \textsc{neg}-\textsc{pfv}-2→1-see\textsubscript{II}-\textsc{2sg}\\
\glt  \sens{I came to see you, but you were not here, so I didn't see you.}
\end{exe}

Similarly, in (\ref{ex:nEJAtəVan}), the verb form  \ipa{nə-ɟɐ-ˈtə-va-n}  \sens{What did you come to do?} with the AM prefix \ipa{ɟɐ-} does not imply that the action of the verb has already taken place, unlike its corresponding Japhug in § \ref{sec:am.interrogative}.

 

\begin{exe}
\ex \label{ex:nEJAtəVan}
\gll
\ipa{nəɟə̂} \ipa{tɕɐ̄} \ipa{thə̂} \ipa{nə-ɟɐ-ˈtə-va-n}? \\
\textsc{2sg} here what  \textsc{pfv}:downwards-\textsc{cisloc}-2-do\textsubscript{II}-\textsc{2sg} \\
\glt  \sens{What did you come to do?}
\end{exe}

\begin{exe}
\ex \label{ex:nEtEvEn}
\gll
\ipa{nəɟə̂} \ipa{tɕɐ̄} \ipa{thə̂} \ipa{kə-viɛ̂} \ipa{nə-ˈtə-və-n} \ipa{kɐ}? \\
\textsc{2sg} here what \textsc{inf}-do \textsc{pfv}:downwards-2-come\textsubscript{II}-\textsc{2sg} \textsc{part} \\
\glt  \sens{What did you come to do?}
\end{exe}

This difference between Japhug and Situ, two otherwise relatively closely related languages, suggests that features such as monoactionality of motion event and verbal action may not be diachronically stable.


\section{Kiranti} 
The most complex AM systems of the Sino-Tibetan family are found in Kiranti. Systems with up to seven distinct markers have been described (Khaling and Yamphu from \citealt[137-194]{rutgers98yamphu}).

Existing grammars are not always explicitly enough to determine whether a particular marker has AM values, rather than being an orientation marker (without additional motion). In the present survey, all unclear cases have been discarded from the data.  

\subsection{Morphological status}

In Kiranti, AM markers are ancient auxiliaries (variously called `aspectivizers', `bound roots', `motionalizers',   general motion auxiliaries', V_'2s, vector verbs, `oriented motion verbs' and `modal verbs' in grammar of Kiranti languages), though in many cases the source verb does not exist any more.  Example (\ref{ex:ingyoklestyanu}) illustrates a typical AM marker in Khaling, the circumambulative \textbf{-le-} `go around doing'. Note that the \textsc{2pl:n.pst} suffix \ipa{-ni} occurs in reduced form \ipa{-n-} between the stem of the verb \ipa{-ŋô-} and  the AM marker \ipa{-le-}, suggesting that at an earlier stage both the main verb and the AM marker had the full array of person indexation and TAM marker.

\begin{exe}
\ex \label{ex:ingyoklestyanu}
\gll \ipa{ʔi-ŋô-n-\textbf{le}-ni} \\
2/\textsc{inv}-cry-\textsc{2pl:n.pst}-\textsc{go.around.doing}-\textsc{2pl:n.pst} \\
\glt `You_{pl} went around crying.' 
\end{exe}

There is evidence that such complex verb forms originate from serial verb constructions, in which a motion verb (as well as other auxiliary) occupied the second position, but became progressively integrated into a single word (though with differences across Kiranti, see \citealt{bickel07chintang}, \citealt{schiering10prosodic}). 

In the following discussion, complex verbs such as (\ref{ex:ingyoklestyanu}) are referred to as bipartite verbs, where V_1 refers to the lexical verb and V_2 to the AM marker.

Table \ref{tab:kurledu} provides partial paradigms for the Khaling circumambulative forms of the intransitive verb \dhatu{ŋok}{cry} (`go around crying') and of the transitive \dhatu{kur}{carry on the back} (`carry around'). Note that the stem alternation of the circumambulative V_2 \ipa{-le-}, while presenting commonalities with several conjugation classes, are unlike those of any independent verb.

\begin{table}%[H]
\caption{Paradigm of Khaling bipartite verbs (with the circumambulative \ipa{-le-} as V2)} \label{tab:kurledu}  \centering
\begin{tabular}{lllllll}
\toprule
&\multicolumn{2}{c}{\textsc{n.pst}} &  \multicolumn{2}{c}{\textsc{pst}} \\
&Simple verb &  Bipartite verb &  Simple verb&  Bipartite verb \\
\midrule
\textsc{1s} &  \ipa{ŋôŋ-ŋʌ} & \ipa{ŋô-ŋ-\textbf{le}-ŋʌ} & \ipa{ŋɵk-ʌtʌ} & \ipa{ŋɵk-\textbf{les}-tʌ} \\
\textsc{1di} &  \ipa{ŋɵk-i} & \ipa{ŋɵk-\textbf{leʦ}-i} & \ipa{ŋɵk-iti} & \ipa{ŋɵk-\textbf{les}-ti} \\
\textsc{1pi} &  \ipa{ŋok-ki} & \ipa{ŋok-\textbf{le}-ki} & \ipa{ŋok-tiki} & \ipa{ŋok-\textbf{le}-ktiki} \\
\textsc{2s} &  \ipa{ʔi-ŋôː} & \ipa{ʔi-ŋôː-\textbf{le}} & \ipa{ʔi-ŋɵk-tɛ} & \ipa{ʔi-ŋɵk-\textbf{les}-tɛ} \\
\textsc{2p} &  \ipa{ʔi-ŋôː-ni} & \ipa{ʔi-ŋô-n-\textbf{le}-ni} & \ipa{ʔi-ŋɵk-tɛnu} & \ipa{ʔi-ŋɵk-\textbf{les}-tɛnu}  \\
\textsc{3s} &  \ipa{ŋôː} & \ipa{ŋôː-\textbf{le}} & \ipa{ŋɵk-tɛ} & \ipa{ŋɵk-\textbf{les}-tɛ} \\
\textsc{3p} &  \ipa{ŋôː-nu} & \ipa{ŋô-n-\textbf{le}-nu} & \ipa{ŋɵk-tɛnu} & \ipa{ŋɵk-\textbf{les}-tɛnu} \\
\midrule
\textsc{1s$\rightarrow$3} &  \ipa{kur-u} & \ipa{kûr-\textbf{led}-u} & \ipa{kur-utʌ} & \ipa{kûr-\textbf{le}-tʌ} \\
\textsc{1di$\rightarrow$3} &  \ipa{kʉr-i} & \ipa{kʉ̂r-\textbf{leʦ}-i} & \ipa{kʉr-iti} & \ipa{kʉ̂r-\textbf{les}-ti} \\
\textsc{1pi$\rightarrow$3} &  \ipa{kʌ̄r-ki} & \ipa{kʌ̄r-\textbf{le}-ki} & \ipa{kʌ̄r-tiki} & \ipa{kʌ̄r-\textbf{le}-ktiki} \\
\textsc{2s$\rightarrow$3} &  \ipa{ʔi-kʉ̄ːr-ʉ} & \ipa{ʔi-kʉ̂r-\textbf{led}-ʉ} & \ipa{ʔi-kʉ̂r-tɛ} & \ipa{ʔi-kʉ̂r-\textbf{le}-tɛ} \\
\textsc{2p$\rightarrow$3} &  \ipa{ʔi-kʌ̄r-ni} & \ipa{ʔi-kʌ̄r-\textbf{le}-ni} & \ipa{ʔi-kʉr-tɛnu} & \ipa{ʔi-kʉr-\textbf{les}-tɛnu} \\
\textsc{3s$\rightarrow$3} &  \ipa{kʉ̄ːr-ʉ} & \ipa{kʉ̂r-\textbf{led}-ʉ} & \ipa{kʉ̂r-tɛ} & \ipa{kʉ̂r-\textbf{le}-tɛ} \\
\textsc{3p$\rightarrow$3} &  \ipa{kʉ̂r-nu} & \ipa{kʉ̂r-\textbf{let}-nu} & \ipa{kʉ̂r-tɛnu} & \ipa{kʉ̂r-\textbf{le}-tɛnu} \\
\bottomrule
\end{tabular}
\end{table}

\subsection{Morphosyntactic parameters of AM in Kiranti}
The Khaling AM system, among the most complex in the Kiranti branch, can be used to illustrate the general typological characteristics of AM systems in this branch. Table \ref{tab:system} summarizes the data concerning four parameters (temporal relation, deixis, vertical dimension and argument of motion) relating to V_2.

\begin{landscape}
\begin{table}%[H]
\caption{The Khaling AM system} \label{tab:system} \centering
\begin{tabular}{lllllll}
\toprule
Source &V_2 & Temp.& Deixis & Vertical  & Argument \\
\midrule
?& \dhatu{-le(t)-}{go around doing X} &C& $\varnothing$ & $\varnothing$ & S/A \\
\dhatu{kʰot}{go} &\dhatu{-kʰo(t)-}{do X and go} &S& \textsc{trans} & $\varnothing$ &  S/A(+P) \\
?&\dhatu{-pɛ(t)-}{go and do X} &P& \textsc{trans} & $\varnothing$ & S/A \\
\dhatu{ɦo}{come} &\dhatu{-ɦo(t)-}{do X and come} &S& \textsc{cis} & $\varnothing$ & S/A(+P) \\
\dhatu{kʰoŋ}{come up}&\dhatu{-kʰoŋ-}{do X and come up} &S& \textsc{cis} &up & S/A \\
 \dhatu{pi}{come (same level)}&\dhatu{-pi(t)-}{come (same level)} &S& \textsc{cis} &same level & S/A(+P) \\
\dhatu{tɛn}{fall} & \dhatu{-tɛ(nt)-}{do X and come/bring down} &S/P & cis &down & S/A(+P) \\
\bottomrule
\end{tabular}
\end{table}
\end{landscape}

 \subsubsection{Temporal relation} \label{sec:temporal.khaling}
As shown by Table \ref{tab:system}, prior, concurrent and subsequent AM markers (abbreviated as P, C and S respectively) are found in Khaling.  For instance, the \dhatu{-kʰot-} in (\ref{ex:tyungkhatya}) exclusively expresses subsequent motion.

\begin{exe}
\ex \label{ex:tyungkhatya}
 \gll   \ipa{tʉ̂ŋ-kʰʌ-tɛ} \\
 drink-\textsc{away/do\&go}-\textsc{2/3:pst} \\
 \glt `He drank it up' or `He drank it and left.'
 \end{exe}

Another marker,  \dhatu{-tɛ(nt)-}{do X and come/bring down}, also generally implies subsequent motion (as in \ref{ex:ryaptyandyu1}), but it can also express prior motion if pragmatics forces the interpretation: for instance, in (\ref{ex:ryaptyandyu2}), the rice field being lower in altitude than the household, only a prior motion interpretation makes sense, while in (\ref{ex:ryaptyandyu1}), the buckwheat field being at a higher altitude, there is no obstacle to adopt the default subsequent motion interpretation.

 \begin{exe}
\ex \label{ex:ryaptyandyu1}
  \gll   \ipa{siriseʔ-ɛ} \ipa{bʰrɛ̂m}  \ipa{rɛp-tɛnd-ʉ} \\
   p.n.-\textsc{erg} buckwheat beat-\textsc{come.down}-\textsc{3sg$\rightarrow$3.n.pst} \\
\glt `Sirise beats the buckwheat and comes down.'
 \end{exe}
 
 
 \begin{exe}
\ex \label{ex:ryaptyandyu2}
 \gll   \ipa{dzirise-ʔɛ} \ipa{rɵ̂:} \ipa{rɛp-tɛnd-ʉ} \\
 p.n.-\textsc{erg} rice beat-\textsc{come.down}-\textsc{3sg$\rightarrow$3.n.pst} \\
\glt `Jirise comes down, beats the rice (and returns).'
 \end{exe}


 \subsubsection{Vertical dimension and deixis} \label{sec:vertical.khaling}
 The vertical dimension (cislocative motion up, down, and on same level) in the Khaling AM system directly reflects a feature of the motion verbs that is common to all Kiranti languages: the fact that not less than four distinct verbs exist to express cislocative motion, \dhatu{ɦo}{come} that is neutral as the vertical dimension, and \dhatu{pi}{come (same level)}, \dhatu{kʰoŋ}{come up}, and \dhatu{je}{come down}. 
 
 Since the AM markers are transparently grammaticalized from these motion verbs (and their applicative/causative counterparts), it is hardly surprising that this contrast is also found in the AM system.
 
 AM markers expressing translocative motion have no vertical dimension contrast, as the corresponding motion verbs.
 
  \subsubsection{Argument of motion} \label{sec:argument.khaling}
With a few AM markers in Khaling, the argument undergoing the motion event is the subject (S/A), but all the V_2 expressing subsequent motion, a manipulative meaning `do X and bring' is sometimes observed when the V_1 is transitive. 

A possible way of analysing this phenomenon is to consider that for these two verbs not only the transitive subject, but also the object undergoes the motion event (hence the notation A+P in Table \ref{tab:system}).


 \subsubsection{Non-AM meanings} \label{sec:khaling.non.am}
Five of the seven AM markers in Khaling exclusively mark AM when used with non-motion V_1: \dhatu{-le(t)-}{go around doing X}, \dhatu{-pɛ(t)-}{go and do X},  \dhatu{-kʰoŋ-}{do X and come up}, \dhatu{-pi(t)-}{come (same level)} and \dhatu{-tɛ(nt)-}{do X and come/bring down}.

 The V_2   \dhatu{-kʰo(t)-}{do X and go} most commonly occurs in non-AM use to mark aspect (completed action), and is also attested in a considerable number of lexicalized bipartite verbs. The V_2  \dhatu{-ɦo(t)-}{do X and come} only has lexicalized non-AM uses. 
  
\subsubsection{Mono- vs pluriactionality} \label{sec:khaling.pluriactionality}

Unlike other languages with AM such as Japhug (see § \ref{sec:am.japhug}), in Khaling negation does not necessarily have scope over both the motion event and the action of the V_1. In example (\ref{ex:mahunkhonya}) (compare with \ref{ex:hungkhondu}), only the action of the V_1 is negated, not the motion event.  

\begin{exe}
\ex \label{ex:mahunkhonya}
 \gll
\ipa{ʔuŋʌ} \ipa{ʔʌ̄m} \ipa{mʌ-ɦû-n-kʰōː-nɛ-ʔɛ} \ipa{ʔu-nûː} \ipa{ŋes-tɛ} \\
\textsc{1sg}.\textsc{erg} \textsc{3sg} \textsc{neg}-wait-\textsc{inf}-\textsc{do\&come.up} \textsc{3sg}.\textsc{poss}-mind hurt-2/3:\textsc{pst} \\
\glt `I did not wait for him before going up and he was sad. (I was already gone up by the time he arrived at the waiting place)'
\end{exe}

\begin{exe}  
\ex \label{ex:hungkhondu}
 \gll  \ipa{ʔīn} \ipa{kʰɵs-tʰer-e} \ipa{uŋʌ} \ipa{ʔʌ-dʌrʌm} \ipa{ɦûŋ-kʰond-u} \\
 2sg go-\textsc{habit}-\textsc{imp}:\textsc{2sg} \textsc{1sg}.\textsc{erg} \textsc{1sg}.\textsc{poss}-friend wait-\textsc{do\&come.up}-\textsc{1sg.A.n.pst} \\
\glt `You keep going, I will wait for my friends and come up then.'
\end{exe}



\subsection{Survey of AM systems in Kiranti}
A large number of grammars of Kiranti languages have been consulted for the present survey, but not all languages have been included, as shorter grammars  (\citealt{ebert97athpare}, \citealt{ebert97camling}, \citealt{opgenort05jero}, \citealt{tolsma06kulung}) do not always include enough data on V_2. 

Not all Kiranti languages have AM. Although Bantawa (\citealt{doornenbal09}) is among the better described languages, no example of V_2 clearly analyzable as AM could be found inn the available publications.

The simplest AM system is that of Thulung (\citealt{lahaussois02thulung} and additional fieldwork) with only one AM marker, the circumambulative \ipa{-bal} (table \ref{tab:thulung.am}).

Dumi, the closest relative of Khaling, has at least the four markers indicated in Table \ref{tab:dumi.am}, possibly more (\citealt[199-214]{driem93dumi}). Note that all the V_2 in this Table have Khaling cognates, with similar functions, and that the AM system in these two languages may go back to their common ancestor.

Wambule (\citealt{opgenort04wambule}) has five AM markers, all of which can have a manipulative `bring/take' meaning (Table   \ref{tab:wambule.am}). The verb form (\ref{ex:seilwasta}) with  \ipa{-lwa-} `go/take and do' provides a clear example of the AM value of this V_2.
 
\begin{exe}
\ex \label{ex:seilwasta}
\gll sei-lwa-s-ta \\
kill.self-go/take-\textsc{detr}-\textsc{imp}:sAS   \\
\glt \sens{go and kill yourself} (\citealt[439]{opgenort04wambule}) 
\end{exe}

Yamphu has a particularly rich AM system, with at least seven markers (\citealt[137-194]{rutgers98yamphu}). Among the V_2 listed as AM markers in table \ref{tab:yamphu.am}, note that \ipa{-phæt(t)-} essentially has not-AM functions, but does occur with the meaning `do and leave' as in example (\ref{ex:uŋbhaetto}), hence its analysis as a subsequent translocative AM marker.  
 
 \begin{exe}
\ex \label{ex:uŋbhaetto}
\gll uŋ-bhæ:tt-o uŋ-bhæ:tt-o sapphi \\
drink-away-\textsc{arq} drink-away-\textsc{arq}  abundantly\\
\glt \sens{Drink before you leave, drink before you leave, [drink] as much as you can.} (\citealt[150]{rutgers98yamphu}).
\end{exe}

The V_2 \ipa{-las-} `go, do and come back' implies two motion events, and may be similar to the `interrupted motion' category proposed by \citet[123]{rose15am} to describe `situation where the realization of the lexical event occurs between two stretches of motion and where the motion is encoded by a marker distinct from that of `concurrent motion''.


The AM systems in Kiranti languages stand out by the following characteristics:

\begin{itemize}
\item Most systems have a circumambulative `go around doing', though its form is cognate only in very closely related languages, such as Khaling and Dumi. The circumambulative marker is not necessarily always interpreted as AM in all languages; for instance in Belhare, \citet[164]{bickel96aspect} shows that the marker which he glosses as `spatially distributed temporary' \textsc{sdt}  has both AM and non-AM uses (compare \ref{ex:chapkonu} and \ref{ex:chigon}).
 \begin{exe}
\ex \label{ex:chapkonu}
\gll \ipa{rot-de} \ipa{i-baŋ-ŋa} \ipa{chap-kon-u}  \\
road-\textsc{loc} one-\textsc{hum}-\textsc{obl} write-\textsc{sdt}-\textsc{3u} \\
\glt \sens{Somebody is walking around taking notes on the road.}
\end{exe}

 \begin{exe}
\ex \label{ex:chigon}
\gll \ipa{thali} \ipa{khore} \ipa{wat} \ipa{chi-gon}    \\
plat cup clean clean-\textsc{sdt} \\
\glt \sens{He is cleaning plates and cups.}
\end{exe}

\item Most systems allow manipulative interpretation of AM markers, with motion of both A and P.
\item In the same way as the cislocative motion verbs ( `come') have a vertical dimension in Kiranti (come up, come down and come at the same level), AM markers also have the same distinction, even if the verbs from which the AM markers are grammaticalized are not always attested.
\end{itemize}
\begin{landscape}	

\begin{table}
\caption{The Thulung AM system} \label{tab:thulung.am} \centering
\begin{tabular}{llllllllll}
\toprule
Source &V_2 & Temp.& Deix. & Vert.& Arg. \\
\midrule
\dhatu{bal}{wind around} &	\ipa{-bal-} `go around doing' &	C &	\textsc{circum}&$\varnothing$&	S/A+P & \\
\bottomrule
\end{tabular}
\end{table}			

\begin{table}
\caption{The Dumi AM system (\citealt[199-214]{driem93dumi})} \label{tab:dumi.am} \centering
\begin{tabular}{llllllllll}
\toprule
V_2 & Gloss& Temp.& Deix. & Vert.& Arg. \\
\midrule
\ipa{-pad-} `to go off to do something' &	Allative&P &		\textsc{trans} &$\varnothing$&	S/A & \\
\ipa{-li(lɨt)-} `to be up and about while doing' &	Frolicsome&C &	\textsc{circum} &$\varnothing$&	S/A & \\
\ipa{-hu:d-} `do and bring back here' &	fetch&S &		\textsc{cis} &$\varnothing$&	S/A+P & \\
\ipa{-pid-} `do and bring it over here' &	bring&S &		\textsc{cis} &$\varnothing$&	S/A+P & \\
\bottomrule
\end{tabular}
\end{table}				

\begin{table}
\caption{The Wambule AM system (\citealt{opgenort04wambule})} \label{tab:wambule.am} \centering
\begin{tabular}{llllllllll}
\toprule
Source &V_2 & Gloss &Temp.& Deix. & Vert.& Arg. \\
\midrule
\dhatu{pi}{come (horizontal)} &	\ipa{-phi-} `come/bring and do' &come/bring:\textsc{hrz}&	P &		\textsc{cisl} &same level&	S/A+P \\
\dhatu{phit}{bring (horizontal)} & \\
\dhatu{ga}{come up} &	\ipa{-kha-} `come/bring up' &?&	P &		\textsc{cisl} & up&	S/A+P \\
\dhatu{khat}{bring up} &	 \\
\dhatu{ywa}{come down} &	\ipa{-ywa,hywa-} `come/bring down' &come/bring:\textsc{down}&	P/I &		\textsc{cisl} & down&	S/A+P \\
\dhatu{hywat}{bring down} &	 \\
\dhatu{di}{go and come back} &	\ipa{-di,du-} `go/take (and come back)' &go/take/come&	P &		\textsc{transl} & $\varnothing$&	S/A+P \\
\dhatu{lwa}{go} &	\ipa{-lwa-} `go/take and do' &go/take &	P &		\textsc{cisl} &same level&	S/A+P \\
\dhatu{lyat}{take away} &	\\
\bottomrule
\end{tabular}
\end{table}	

\begin{table}
\caption{The Yamphu AM system  (\citealt[137-194]{rutgers98yamphu}).} \label{tab:yamphu.am} \centering
\begin{tabular}{llllllllll}
\toprule
Source &V_2 & Gloss &Temp.& Deix. & Vert.& Arg. \\
\midrule
  &	\ipa{-ca/cæt-} `come and do' &come/bring &	P &		\textsc{cisl} & $\varnothing$&	S/A+P \\
\dhatu{ap}{come (horizontal)}    &	\ipa{-ap(t)-} `come and do' &come/bring &	P &		\textsc{cisl} & same level &	S/A+P \\
\dhatu{uks}{come down}    &	\ipa{-uk(t)-} `come down and do' &come/bring down &	P/C &		\textsc{cisl} & same level &	S/A+P \\
\dhatu{kæt}{come up}    &	\ipa{-kæt(t)-} `do and come/bring up' &come/bring up &S &		\textsc{cisl} & same level &	S/A+P \\
   &	\ipa{-phæt(t)-} `do and leave' &away &S &		\textsc{transloc} & $\varnothing$ &	S/A  \\
\dhatu{las}{go and come back}    &	\ipa{-las-} `go, do and come back' &go come &S+P &		\textsc{cisl} & $\varnothing$ &	S/A+P \\
   &	\ipa{-tus/tit-} `go around doing' &around &C &		\textsc{circum}& $\varnothing$ &	S/A  \\
\bottomrule
\end{tabular}
\end{table}	


\begin{table}
\caption{The Yakkha AM system  (\citealt[283-328]{schackow15yakkha}).} \label{tab:yakkha.am} \centering
\begin{tabular}{llllllllll}
\toprule
Source &V_2 & Gloss &Temp.& Deix. & Vert.& Arg. \\
\midrule
  &	\ipa{-kheʔ/t-} `do and go/carry' &V_2.\textsc{go/carry.off} &	S &		\textsc{trans} & $\varnothing$&	S/A, A+P  \\
  &	\ipa{-uks/t-} `do and come/bring down' &V_2.\textsc{come.down}/\textsc{bring.down} &	S &		\textsc{cis} & down&	S/A+P  \\
  &	\ipa{-ap(t)-} `do and come/bring' &V_2.\textsc{come}/\textsc{bring} &	S &		\textsc{cis} & same level&	S/A+P  \\
%  \dhatu{phes}{bring}    &	\ipa{-ap(t)-} `come and do' &bring &	P &		\textsc{cisl} & same level &	S/A+P \\
  &	\ipa{-ghond-} `go around doing' &V_2.\textsc{roam} &	C &	 \textsc{circum}&  $\varnothing$&	S/A   \\
\bottomrule
\end{tabular}
\end{table}	

\begin{table}
\caption{The Belhare AM system  (\citealt{bickel96aspect, bickel97spatial, bickel17belhare}).} \label{tab:belhare.am} \centering
\begin{tabular}{llllllllll}
\toprule
Source &V_2 & Gloss &Temp.& Deix. & Vert.& Arg. \\
\midrule
  &	\ipa{-itt-} `go up and do' &go.\textsc{upwards} &	S &		\textsc{cis} & down&	S/A+P  \\
 
  &	\ipa{-ap(t)-} `do and come/bring' &bring.\textsc{across} &	P&		\textsc{trans} & up&	S/A+P  \\
%  \dhatu{phes}{bring}    &	\ipa{-ap(t)-} `come and do' &bring &	P &		\textsc{cisl} & same level &	S/A+P \\
  &	\ipa{-kon-} `go around doing' &spatially distributed temporary &	C &	 \textsc{circum} &  $\varnothing$&	S/A   \\
\bottomrule
\end{tabular}
\end{table}	

\begin{table}
\caption{The Hayu AM system (\citealt[151]{michailovsky88}} \label{tab:hayu.am} \centering
\begin{tabular}{llllllllll}
\toprule
Source &V_2 & Temp.& Deix. & Vert.& Arg. \\
\midrule
\dhatu{lat}{go} &	\ipa{-la(t)-} `go and do' &	P &	\textsc{trans}&$\varnothing$&	S/A & \\
\bottomrule
\end{tabular}
\end{table}			
\end{landscape}



\section{Sinitic}
The phenomenon of associated motion (henceforth AM) has been rarely addressed in the linguistic studies of the Sinitic languages. It has not been described until in \cite{lamarre17motion.cum} and \cite{lamarre17deictic} where the sequences [VP+\zh{去/来}] \sens{go/come to do} in Northern Sinitic languages are analyzed as \textsc{am} encoding. In (\ref{distribution2}) and (\ref{distribution1}), the andative enclitic \zh{去} and the ventive enclitic \zh{来} respectively encode an translocative and a cislocative motion to the action denoted by the VP.


\begin{exe}
\ex Standard Mandarin \citep{lamarre17deictic} \label{distribution2}
\glll
\zh{喝} \zh{点儿} \zh{水}=\zh{去} \\
\ipa{Hē} \ipa{diǎnr} \ipa{shuǐ}=\ipa{qu} \\
drink  some water=\textsc{go}$\&$ \\
\glt \sens{Go and drink some water!}.
\end{exe}

\begin{exe}
\ex Standard Mandarin \citep{lamarre17motion.cum} \label{distribution1}
\glll
\zh{你} \zh{干} \zh{嘛}=\zh{来} \zh{了} \\
\ipa{Nǐ} \ipa{gàn} \ipa{má}=\ipa{lai} \ipa{le} \\
\textsc{2sg} do what=\textsc{come}$\&$ \textsc{part} \\
\glt \sens{What are you coming for?}
\end{exe}

[VP+\zh{去/来}] was previously described as a type \sens{purpose construction} \citetext{\citealp{lu1985vpqu}; \citealp{yang2012mudi}}, the final \zh{去/来} as particles of purpose \citep[479]{chao68chinese}, and is considered to be interchangeable to some degree with the serial verb construction (henceforth SVC) [\zh{去/来}+VP]. We can compare (\ref{distribution2}) and (\ref{distribution1}) with (\ref{inventory2}) and (\ref{inventory1}) below. Noted that as \textsc{am} markers (in \ref{distribution2} and \ref{distribution1}), the post-verbal \zh{去/来} lose their verbality to some degree and show tendancy to become bound morphemes, such as phonetic weakening comparing to their source verbs (in \ref{inventory2} and \ref{inventory1}). \citealp[479]{chao68chinese}; \citealp{lu1985vpqu}; \cite{lamarre17motion.cum} ; \cite{lamarre17deictic}).

\begin{exe}
\ex Standard Mandarin \citep{lamarre17deictic} \label{inventory2}
\glll
\zh{去} \zh{喝} \zh{点儿} \zh{水} \\
\ipa{Qù} \ipa{hē} \ipa{dianr} \ipa{shuǐ} \\
go drink some water \\
\glt \sens{Go and drink some water!}
\end{exe}


\begin{exe}
\ex Standard Mandarin \label{inventory1}
\glll
\zh{你} \zh{来} \zh{干} \zh{什么}  \\
\ipa{Nǐ} \ipa{lái} \ipa{gàn} \ipa{shénme}  \\
\textsc{2sg} come do what \\
 \glt \sens{What are you coming for?}
\end{exe}
Based on \cite{chappell15areas} about the typo-geographic division of the Sinitic languages in China and our preliminary survey on the acceptance of [VP+\zh{去/来}] in representative varieties of each of these langues, Table (\ref{inventorytable1}) shows the general distribution of \textsc{am} marking in the Sinitic Languages in China. The gray parts correspond to the places where the marking of the associated movement is found. The lighter the cells, the more likely the \textsc{am} marking is absent. As there is not much work on these places regarding this aspect, a detailed job would be worthy doing. 

No Sinitic language has a dedicated AM markers. Depending on the variety, AM enclitics can also mark orientation (in particular when used with motion verbs), aspect and imperative.

The \textsc{am} enclitics in Sinitic languages in some cases can imply pluriactionality.  Sentences such as (\ref{mono1}) can have two interpretations. In most cases, it implies a monoactionality reading, the motion and the action are both realized. However, the plurilactionality interpretation is also possible. Imagine that the speaker is making a phone call to a friend who has gone to another place to do something, and asks him \sens{What did you go to do?}. In this case, the motion has been realized, while the action has not. 

\begin{exe}
\ex \label{mono1}
\glll
\zh{你} \zh{干} \zh{嘛}=\zh{去} \zh{了} \\
\ipa{Nǐ} \ipa{gàn} \ipa{má}=\ipa{qu} \ipa{le} \\
\textsc{2sg} do what=\textsc{go}$\&$ \textsc{part} \\
\glt \sens{What did you go to do?}
\end{exe}

 


\begin{table} [H] \label{inventorytable1}
\caption{Geographic distribution of \textsc{am} markers of the Sinitic languages of China}
\resizebox{\columnwidth}{!}{
\begin{tabular}{lllllllll}
\toprule
& Language branch & Dialect & \textsc{am} markers  \\
\midrule
\grise{I}. & \grise{Mandarin \zh{官话}} & \grise{}&  \grise{} \\
\grise{} & \small \grise{Standard Mandarin} & \grise{} &  \grise{\zh{去} \zh{来} } \\
 \grise{}& \small \grise{Beijing \zh{北京}}  &  \grise{Beijing \zh{北京}} & \grise{\zh{去} \zh{来} }  \\
 \grise{}& \small \grise{Northwestern (Lanyin \zh{兰银})} & \grise{Yinchuan \zh{银川}} &\grise{\zh{去} \zh{来} \zh{走}} \\
  \grise{}& \small \grise{} & \grise{Dachuan \zh{达川}} & \grise{\zh{去} \zh{来}} \\
 \grise{}& \small \grise{Jilu \zh{冀鲁}}  & \grise{Jizhou \zh{冀州}} &   \\
 \grise{} & \small \grise{Jiaoliao \zh{胶辽}}&   \grise{Qingdao \zh{青岛}}   & \\
 \grise{} & \small  \grise{Dongbei \zh{东北}}  &  \grise{Changchun \zh{长春}}  &  \\
 \grise{} & \small \grise{Central plains (Zhongyuan \zh{中原})} & \grise{Xi'an \zh{西安} }  &  \grise{\zh{去} \zh{来} }  \\
  \grise{} & \small \grise{Southwestern \zh{西南}}  & \grise{Chengdu \zh{成都}} & \grise{\zh{去} \zh{来} }  \\
   \grise{} & \small \grise{}  & \grise{Wuhan \zh{武汉}} &  \grise{\zh{去} \zh{来} } \\
 \grise{} & \small \grise{Jianghuai \zh{江淮} (Southern)}  & \grise{Nanjing \zh{南京}}&  \\
  \grise{} & \small \grise{}  & \grise{Anqing \zh{安庆}}& \grise{\zh{去} \zh{来} }  \\
\grise{II.} & \grise{Jin  \zh{晋}} & \grise{Shenmu \zh{神木}} &  \grise{\zh{去} \zh{来} }  \\
\grise{} & \grise{} & \grise{Datong \zh{大同}} &  \grise{\zh{去} \zh{来} }  \\
\grise{III.} &\grise {Xiang \zh{湘}} & \grise{Changsha \zh{长沙}} &   \grise{\zh{去} \zh{来} }  \\
\grise{} &\grise {} & \grise{Changde \zh{常德}} & \grise{\zh{去} \zh{来}? }    \\
\gray{IV.} & \gray{Wu \zh{吴}} &  \gray{Shanghai \zh{上海}} & \gray{\zh{去} \zh{来}} \\
\gray{} & \gray{} &  \gray{Hangzhou \zh{杭州}} &  \gray{\zh{去}}  \\
\gray{} & \gray{} &  \gray{Suzhou \zh{苏州}} & \gray{None} \\
\gray{V.} & \gray{Hui \zh{徽}} &  \gray{Huizhou \zh{徽州}} & \gray{\zh{去}}   \\
\lightgray{VI.} & \lightgray{Gan \zh{赣}} & \lightgray{Northeastern Gan} & \lightgray{\zh{去} \zh{来}} \\
 & &  Fuzhou \zh{抚州} & None  \\
VII. & Min  \zh{闽} & Yilan \zh{宜兰}  & None  \\
VIII. & Kejia \zh{客家} (Hakka) & &  None  \\
IX. & Yue \zh{粤} (Cantonese) & & None   \\
X. & Pinghua \zh{平话} and Tuhua \zh{土话}   & Rucheng \zh{汝城} & None \\
\bottomrule
\end{tabular}}
\end{table}


\begin{table} [H]
\caption{\textsc{am} markers in Sinitic} \centering
\begin{tabular}{llllllllll}
\toprule
Source &Enclitic & Temp.& Deix. & Vert.& Arg. \\
\midrule
\zh{去} \ipa{qù} &	\ipa{=qu} `go and X' &	P &		$\varnothing$ &$\varnothing$&	S/A & \\
 \zh{来}  \ipa{lái} &	\ipa{=lai} `come and X' &	P &		$\varnothing$ &$\varnothing$&	S/A & \\
\bottomrule
\end{tabular}
\end{table}
 
 

\section{Other languages}

\citet{konnerth14karbi, konnerth15cisloc}

\citet{boro17hakhun}

\citet{Jenny16grammar}

\section{Typological comparison}
\resizebox{\columnwidth}{!}{
\begin{tabular}{llllllllll}
\toprule
Subgroup&Language  &Number &Temp. & Deixis & Vertical & Other & Mono- &Arg.\\
&&&relation &&dimension&functions& actional & \\
\midrule
Gyalrong&Japhug & 2&P & Cis/Trans & $\varnothing$ &  $\varnothing$ & \Y &S/A \\
&Situ &2&P & Cis/Trans & $\varnothing$ & Aspect&\N &S/A \\
\midrule
Kiranti &Khaling & 7&P, C, S &Cis/Trans & \Y & Aspect,  &  \N  &S/A, S/A+P \\
&&&&&&Orientation&&\\
&Dumi & 4? &P, C, S &Cis/Trans &  $\varnothing$ & Aspect,  &  ?&S/A, S/A+P \\
  &&&&&&Orientation&&\\
&Thulung & 1 & C  &Circum &  $\varnothing$ & $\varnothing$ &?  &S/A  \\
&Wambule & 5  & P  &Cis/Trans &  \Y & Aspect,  &?  &S/A, S/A+P \\
     &&&&&&Orientation&&\\
%    &Jero & 2 & C, P?  & Trans? &  \Y & ?  &?  &S/A  \\
&Yamphu & 7  & P, C, S  & Cis/Trans &  \Y & Aspect,  &?  &S/A, S/A+P \\
        &&&&&&Orientation&&\\
&Yakkha & 7? & P, C, S  & Cis/Trans &  \Y & Aspect,  &?  &S/A, S/A+P \\
        &&&&&&Orientation&&\\
&Belhare & 3? & P, C, S  & Cis/Trans &  \Y & Aspect,  &?  &S/A, S/A+P \\
        &&&&&&Orientation&&\\
&Hayu & 1  & P  & Trans &   $\varnothing$& Aspect,  &?  &S/A  \\
        &&&&&&Voice&&\\
\midrule
Sinitic & Mandarin &2& P & Cis/Trans & $\varnothing$ &Aspect,  Modality&\N   &S/A  \\
        &&&&&&Orientation&&\\
\midrule
Karbi&&1&S&cis&  $\varnothing$ &Person, Orientation&?  &S/A +P \\
\midrule
Sal&Tangsa&1&P&transl&  $\varnothing$ & ?&?  &S/A  \\
\bottomrule
 \end{tabular}}
 
\section{Conclusion}

\bibliographystyle{unified}
\bibliography{bibliogj}

 \end{document}
 