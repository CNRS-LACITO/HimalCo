\documentclass[oldfontcommands,twoside,a4paper,12pt]{article} 
\usepackage{fontspec}
\usepackage{natbib}
\usepackage{booktabs}
\usepackage{xltxtra} 
\usepackage{polyglossia} 
 \usepackage{geometry}
 \geometry{
 a4paper,
 total={210mm,297mm},
 left=10mm,
 right=10mm,
 top=15mm,
 bottom=15mm,
 }
\usepackage[table]{xcolor}
\usepackage{color}
\usepackage{multirow}
\usepackage{gb4e} 
\usepackage{multicol}
\usepackage{graphicx}
\usepackage{float}
\usepackage{hyperref} 
\hypersetup{bookmarks=false,bookmarksnumbered,bookmarksopenlevel=5,bookmarksdepth=5,xetex,colorlinks=true,linkcolor=blue,citecolor=blue}
\usepackage{memhfixc}
\usepackage{lscape}
\usepackage[footnotesize,bf]{caption}
\usepackage{multicol}
 
 
\setmainfont[Mapping=tex-text,Numbers=OldStyle,Ligatures=Common]{Charis SIL} 
\setsansfont[Mapping=tex-text,Ligatures=Common,Mapping=tex-text,Ligatures=Common,Scale=MatchLowercase]{Lucida Sans Unicode} 
 


\newfontfamily\phon[Mapping=tex-text,Ligatures=Common,Scale=MatchLowercase]{Charis SIL} 
\newcommand{\ipa}[1]{{\phon\textbf{#1}}} 

\begin{document}
\title{Typologie : examen du 8 janvier 2016}
\date{}
\maketitle
 

\section*{Exercice 1 (Mapudungun)}

\begin{enumerate}
\item Décrivez l'indexation dans cette langue et son alignement (tableau obligatoire). Les formes locales obéissent-elles aux mêmes règles que le reste du paradigme?
\item Proposez une segmentation et une glose des morphèmes dans les exemples. (NB: on considèrera qu'il n'y a pas de morphème de temps-aspect-mode dans ces formes).
\item Traduisez la phrase `Vous deux l'avez vu' en mapudungun.
\end{enumerate}
Le mapudungun est transcrit en orthographe standard, où <ü> représente [ə], <y> [j], <ñ> [ɲ], <ng> [ŋ]

Notez les règles morphophonologiques suivantes: i-n\# > iñ; Cn\# >Cün; V-i > Vy; Ci-eC- > CeC, Ce-iC- > CeC, -i-i- > -i- (V, C et \# représentent une voyelle quelconque, une consonne quelconque, et la fin de mot respectivement). Il est possible que certains -i- correspondent à la fusion de deux \textit{i}. Les autres voyelles, en particulier \textit{e}, ne sont pas affectées par cette règle.


\begin{multicols}{2}
\begin{exe}
\ex 
\glt konün 
\glt Je suis venu.
\end{exe} 

\begin{exe}
\ex 
\glt konimi 
\glt Tu es venu.
\end{exe} 

\begin{exe}
\ex 
\glt koni 
\glt Il est venu.
\end{exe} 

\begin{exe}
\ex 
\glt koniyu
\glt Nous deux sommes venus.
\end{exe} 

\begin{exe}
\ex 
\glt konimu
\glt Vous deux êtes venus.
\end{exe} 

\begin{exe}
\ex 
\glt peen
\glt Tu m'as vu.
\end{exe} 

\begin{exe}
\ex 
\glt peeyu
\glt Je t'ai vu.
\end{exe} 

\begin{exe}
\ex 
\glt pefiñ
\glt Je l'ai vu.
\end{exe} 

\begin{exe}
\ex 
\glt pefimi
\glt Tu l'as vu.
\end{exe} 

\begin{exe}
\ex \label{ex:inv}
\glt pefi
\glt Il l'a vu.
\end{exe} 

\begin{exe}
\ex   \label{ex:inv2}
\glt peeyew
\glt Il l'a vu.  
\end{exe} 

\begin{exe}
\ex 
\glt peemew
\glt Il t'a vu. 
\end{exe} 
 
 \begin{exe}
\ex 
\glt peenew
\glt Il m'a vu. 
\end{exe} 

\end{multicols}


La différence entre les formes (\ref{ex:inv}) et (\ref{ex:inv2}) est traitée dans l'exercice 2.

\section*{Exercice 2 (Mapudungun)}

Sachant que les six phrases suivantes signifient toutes `la femme a tué l'homme', expliquez les conditions d'utilisation des formes  verbales \textit{langümfi} et \textit{langümeyew} et analysez-les.

 \begin{exe}
\ex 
\gll domo langümfi wentru \\
femme a.tué homme \\
\ex 
\glt domo wentru langümfi 
\ex 
\glt langümfi wentru domo
\ex 
\glt wentru langümeyew domo
\ex 
\glt wentru domo langümeyew 
\ex 
\glt langümeyew domo wentru
\end{exe} 

\section*{Exercice 3 (Khaling)}
Quel est l'alignement du marquage des arguments dans cette langue? (ne pas prendre en compte l'indexation)
 \begin{exe}
\ex 
\glt ɦʌsʔɛ nɵ̂r sêːtɛ.
\glt L'homme a tué le tigre. 
\ex 
\glt nɵ̂rʔɛ ghrôːt kʉtɛ.
\glt Le tigre a mangé la chèvre.
\ex 
\glt nɵ̂r krʉktɛ.
\glt Le tigre a rugit.
\ex 
\glt ɦʌsʔɛ brâː sîŋtɛ.
\glt L'homme (lui) a posé une question (brâː = parole).
\ex 
\glt salaʔɛ ɦʌs sîŋtɛ.
\glt Le jeune homme a demandé à l'homme: "..."
\end{exe} 
\section*{Exercice 4 (Question de cours)}
Expliquez avec des exemples la différence entre passif et anticausatif.

\end{document}

