\documentclass[oldfontcommands,oneside,a4paper,11pt]{article} 
\usepackage{fontspec}
\usepackage{natbib}
\usepackage{booktabs}
\usepackage{xltxtra} 
\usepackage{longtable}
\usepackage{polyglossia} 
\usepackage[table]{xcolor}
\usepackage{gb4e} 
\usepackage{multicol}
\usepackage{graphicx}
\usepackage{float}
\usepackage{hyperref} 
\usepackage{lineno}
\hypersetup{bookmarks=false,bookmarksnumbered,bookmarksopenlevel=5,bookmarksdepth=5,xetex,colorlinks=true,linkcolor=blue,citecolor=blue}
\usepackage[all]{hypcap}
\usepackage{memhfixc}
\usepackage{lscape}

\bibpunct[: ]{(}{)}{,}{a}{}{,}

%\setmainfont[Mapping=tex-text,Numbers=OldStyle,Ligatures=Common]{Charis SIL} 
\newfontfamily\phon[Mapping=tex-text,Ligatures=Common,Scale=MatchLowercase,FakeSlant=0.3]{Charis SIL} 
\newcommand{\ipa}[1]{{\phon \mbox{#1}}} %API tjs en italique
\newcommand{\ipab}[1]{{\scriptsize \phon#1}} 

\newcommand{\grise}[1]{\cellcolor{lightgray}\textbf{#1}}
\newfontfamily\cn[Mapping=tex-text,Ligatures=Common,Scale=MatchUppercase]{MingLiU}%pour le chinois
\newcommand{\zh}[1]{{\cn #1}}
\newcommand{\refb}[1]{(\ref{#1})}
\newcommand{\factual}[1]{\textsc{:fact}}

\XeTeXlinebreaklocale 'zh' %使用中文换行
\XeTeXlinebreakskip = 0pt plus 1pt %
 %CIRCG
 \newcommand{\bleu}[1]{{\color{blue}#1}}
\newcommand{\rouge}[1]{{\color{red}#1}} 


\begin{document} 
\title{Evidentiality in Japhug\footnote{ Acknowledgements to be added after editorial decision.
} }
\author{Guillaume Jacques}
\maketitle
\linenumbers

\section{Introduction}
Japhug, like other Gyalrongic languages and nearly all languages of the Tibetosphere, has a complex system of evidential marking. While the basic structure of the system is quite similar to that found in Tibetic languages, the presence of obligatory polypersonal indexation on the verb in Japhug makes the interaction between person marking and evidentials much clearer that in Tibetan.

While previous research has partially described the evidential systems of other Gyalrong languages (\citealt{linyj03tense}), this paper is the first attempt to present a complete account of evidentiality in a Gyalrong language on the exclusive basis of natural data, i.e. example sentences from either conversation or traditional stories.

The paper contains five sections; First, I present general background information on verbal morphology. In section 2 and 3, I describe the evidential contrasts in present (Egophoric vs Factual vs Sensory) and past (Perfective vs Inferential) situations. Then, I discuss the hearsay sentence final particle and its interaction with verbal evidentiality. Finally, I critically review the previous analyses of egophoricity as  a `conjunct / disjunct' systems and present data from reported speech in Japhug that support \citet{tournadre08conjunct}'s analysis of egophoric markers in Tibetan.


\section{Background information}
In this section, a general account of verbal morphology in Japhug is presented to provide the reader will all the information required to understand evidential marking.

First, I describe directional prefixes and the rules of stem alternation, which are the building blocks of all TAME categories in Japhug. Then, I list all basic TAME categories, of which only those involved in evidential contarsts (excluding Imperative and Irrealis) are discussed in the paper. Finally, I present two facts about evidential marking in Japhug, namely its neutralization in some subordinate clauses and the anticipation rule found in questions.

\subsection{Directional prefixes} \label{sec:directional}
All finite verbal forms in Japhug (except the Factual) and some nominalized forms have a directional prefix that contains information on TAM, transitivity and  (in the case of motion and concrete action verbs) the direction of the action.

With the exception of contracting verbs whose stem starts in \ipa{a--} and which present special alternations, Japhug intransitive verbs have three series of prefixes (A, B and D) and transitive ones four series, as shown in Table \refb{tab:directional}. The distribution of these four series will be explained in more detail in section \refb{sec:finite.TAM}.

\begin{table}[H]
\caption{Directional prefixes in Japhug Rgyalrong} \label{tab:directional}
\resizebox{\columnwidth}{!}{
\begin{tabular}{llllll}
\toprule
   &  	perfective  (A) &  	imperfective  (B)  &  	perfective 3$\rightarrow$3' (C)  &  	inferential  (D) \\  	
   \midrule
up   &  	\ipa{tɤ--}   &  	\ipa{tu--}   &  	\ipa{ta--}   &  	\ipa{to--}   \\  	
down   &  	\ipa{pɯ--}   &  	\ipa{pjɯ--}   &  	\ipa{pa--}   &  	\ipa{pjɤ--}   \\  	
upstream   &  	\ipa{lɤ--}   &  	\ipa{lu--}   &  	\ipa{la--}   &  	\ipa{lo--}   \\  	
downstream   &  	\ipa{tʰɯ--}   &  	\ipa{cʰɯ--}   &  	\ipa{tʰa--}   &  	\ipa{cʰɤ--}   \\  	
east   &  	\ipa{kɤ--}   &  	\ipa{ku--}   &  	\ipa{ka--}   &  	\ipa{ko--}   \\  	
west   &  	\ipa{nɯ--}   &  	\ipa{ɲɯ--}   &  	\ipa{na--}   &  	\ipa{ɲɤ--}   \\  	
no direction &\ipa{jɤ--}   &  	\ipa{ju--}   &  	\ipa{ja--}   &  	\ipa{jo--}   \\  	
\bottomrule
\end{tabular}}
\end{table}

Most verbs have one intrinsic direction which is lexically determined. For instance, the verb \ipa{sat} `kill' selects the direction `down' for all its forms: \textbf{perfective} \textsc{1sg$\rightarrow$3sg} \ipa{pɯ-sat-a}, \textbf{imperfective} \ipa{pjɯ-sat}, \textbf{perfective} \textsc{3sg$\rightarrow$3'} \ipa{pa-sat} and \textbf{evidential} \ipa{pjɤ-sat}. 

Some verbs may allow several directions with slightly different semantics. Thus, \ipa{ndza} `eat'   normally   selects the  `up' direction (\textbf{perfective} \textsc{3sg$\rightarrow$3'} \ipa{ta-ndza} `he ate it'), but when applied to carnivorous animals we also find the `downstream' direction. This can lead to further aspectual distinctions. For instance, the direction `downstream', when used with stative verbs, indicates a progressive development. 

Verbs of motion and some verbs of concrete action can be associated with all seven series of prefixes to indicate the direction of the motion. The  `no direction' series of prefixes only occurs with motion verbs. 

Only three verbs have defective paradigms and never occur with directional prefixes: the Sensory existential copulas \ipa{ɣɤʑu} `exist' and \ipa{maŋe} `not exist' and the verb \ipa{kɤtɯpa} `speak' (see the paradigm of the latter in \citealt[1215]{jacques12incorp}).

\subsection{Stem alternation} \label{sec:stem}
The existence of stem alternations in Gyalrong languages was first reported by \citet{jackson00puxi}, who distinguishes three stems: the base stem (Stem 1), the perfective stem (Stem 2) and the non stem (Stem 3). In Kamnyu Japhug, only four verbs have a perfective stem distinct from the base stem; the list is provided in Table \refb{tab:stem2}. 


 \begin{table}[H]
\caption{Stem 2 alternation in Japhug Rgyalrong} \label{tab:stem2} \centering
\begin{tabular}{llllll}
\toprule
Stem 1 & meaning &Stem 2 \\
\midrule
\ipa{ɕe}& to go (vi)&  \ipa{ari} \\
\ipa{sɯxɕe}& to sent (vt)  &\ipa{sɤɣri} \\
\ipa{ɣi}& to come (vi)  &\ipa{ɣe} \\
\ipa{ti}& to say (vt)  &\ipa{tɯt} \\
\bottomrule
\end{tabular}
\end{table}

By contrast, Stem 3 is completely productive. The rules of vowel alternation in Table \refb{tab:stem3} apply to all finite transitive verbs in the forms \textsc{1sg}$\rightarrow$3, \textsc{2sg}$\rightarrow$3 and \textsc{3sg}$\rightarrow$3'; stem 3 does not appear in verb forms with the inverse marker.

 \begin{table}[H]
\caption{Stem 3 alternation in Japhug Rgyalrong} \label{tab:stem3} \centering
\begin{tabular}{llllll}
\toprule
Stem 1 & Stem 3 \\
\midrule
\ipa{--a} & \ipa{--e} \\
\ipa{--u} & \ipa{--e} \\
\ipa{--ɯ} & \ipa{--i} \\
\ipa{--o} & \ipa{--ɤm} \\
\bottomrule
\end{tabular}
\end{table}

Following the Leipzig glossing rules, we indicate stem 2 as [II] and stem 3 as [III] in the glosses in this paper.

\subsection{Simple TAM categories} \label{sec:finite.TAM}
There are nine basic finite TAM categories in Japhug, as represented in Table \refb{tab:finite.forms}. All finite forms except the Factual require one and only one directional prefix. All forms can be generated  by combining the appropriate derivational prefixes and stems.\footnote{For the TAM categories requiring stem 3, it is restricted to  \textsc{1sg}$\rightarrow$3, \textsc{2sg}$\rightarrow$3 and \textsc{3sg}$\rightarrow$3' forms; all other forms take the base stem.  The person affixes  are not discussed here; for  more information on this topic, see \citet{jacques10inverse}.}

\begin{table}[H]
\caption{Finite verb categories in Japhug Rgyalrong} \label{tab:finite.forms} \centering
\begin{tabular}{lllllll}
\toprule
&	&	stem&	prefixes\\
\midrule
Factual&	\textsc{fact} &	1 or 3&	no prefix\\
Imperfective&	\textsc{ipfv} &	1 or 3&	B\\
Perfective&	\textsc{pfv} &	2&	A or C\\
Past Imperfective&	\textsc{pst.ipfv} &	2&	\ipa{pɯ--}\\
Inferential&	\textsc{evd} &	1&	D\\
Inferential Imperfective&	\textsc{evd.ipfv} &	1&	\ipa{pjɤ--}\\
Sensory&	\textsc{sensory} &	1 or 3&	\ipa{ɲɯ--}\\
Egophoric present&	\textsc{egoph} &	1 or 3&	\ipa{ku--}\\
Irrealis&	\textsc{irr} &	1 or 3&	\ipa{a--} + A\\
Imperative&	\textsc{imp} &	1 or 3&	A\\
\bottomrule
\end{tabular}
\end{table}

The present paper discussed nearly all categories in this Table \ref{tab:finite.forms}, except the Irrealis, the Imperative and the Imperfective, which do not directly intervene in the evidential system of the languages, as they occur in distinct contexts.\footnote{The imperfective is mainly used in periphrastic TAM categories and in certain types of subordinate clauses.}

\subsection{Neutralization of evidential marking} \label{sec:neutralization}
Evidential marking in Japhug is neutralized in most subordinate clauses, in particular relative, temporal, conditional, and purposive subordinate clauses (\citealt{jacques14linking}), following a well-established tendency among languages with evidential marking on the verb (\citealt[253-6]{aikhenvald06}).

Most relatives in Japhug are of the participial type, and participles can only take type A and type B directional prefixes, never type D prefixes. In \ref{ex:pWkWNu}, the relative modifying the personal name \ipa{tshɯraŋ} `Tshering' takes the past imperfective participle \ipa{pɯ-kɯ-ŋu}, while the main verb is in the Inferential with the type D prefix \ipa{pjɤ--}.\footnote{The Inferential appears here because the speaker tells a story that he has not directly witnessed.} If evidential marking was not neutralized in participle forms, a type D prefix **\ipa{pjɤ-kɯ-ŋu} would have been expected here, but such form is completely impossible.


\begin{exe}
\ex \label{ex:pWkWNu}
\gll [\ipa{ɯʑo} 	\ipa{nɯ} \ipa{ɕɯŋgɯ} 	\ipa{ɯ-nmaʁ}  \ipa{pɯ-kɯ-ŋu}] 	\ipa{tshɯraŋ} 	\ipa{nɯ} 	\ipa{pjɤ-mto} 	\\
\textsc{3sg} \textsc{dem} before \textsc{3sg.poss}-husband \textsc{pst.ipfv-nmlz}:S/A-be \textsc{name} \textsc{dem} \textsc{ifr}-see \\
\glt She saw Tshering, who used to be her husband before. (la parole du corbeau)
\end{exe}

Relatives with finite verbs also exist, but limited to the relativization of the P, the theme of ditransitive verbs, the recipient of secundative transitive verbs, the goal of motion verbs and also the semi-object of semi-transitive verbs (see \citealt{jacques16relatives}; an account of relativization in Japhug would go beyond the scope of this paper). In example \ref{ex:paBde}, the head-internal relative, whose relativized element is the P \ipa{rdɤstaʁ}  `stone' has is verb in the perfective form, while the verb of the main clause is in the inferential. Here, replacing the perfective form \ipa{pa-βde} \textsc{pfv:down}:3$\rightarrow$3'-throw `he threw it down' by the equivalent inferential \ipa{pjɤ-βde}  \textsc{ifr:down}-throw would result in an incorrect sentence.

\begin{exe}
\ex  \label{ex:paBde}
\gll  [\ipa{tɤ-tɕɯ-pɯ} 	\ipa{kɯ} 	\ipa{rdɤstaʁ} 	\ipa{pa-βde}] 	\ipa{nɯ} 	\ipa{jo-nɯɴqhu-ndʑi}  \\
\textsc{indef.poss}-boy-small \textsc{erg} stone \textsc{pfv:down}:3$\rightarrow$3'-throw \textsc{dem} \textsc{ifr}-follow-\textsc{du} \\
\glt They followed the stones that the little boy had thrown down (along the way). (Tangguowu)
\end{exe}

Regardless of the relativization type (whether the verb is finite or not, and what element is relativized), the contrast between Inferential and  Perfective (and their imperfective equivalents) is neutralized in relative clauses in Japhug. Similarly in present contexts, Egophoric and Sensory verb forms cannot be used in relatives, only Factual and Imperfective forms appear.

\subsection{Anticipation Rule} \label{sec:anticipation}
A generalization valid for all verbal forms in Japhug is the anticipation rule, already documented in Tibetic languages (\citealt[244]{tournadre14evidentiality}): in questions,  the speaker anticipates the answer of the addressee and uses the form that he expect the addressee will choose to respond to the question. 

As a result of this rule, constraints between person and evidential markers are always reversed between affirmative and interrogative sentences: if a particular form is restricted to first person in affirmative sentences, it will be restricted to second person in interrogatives, and vice-versa.

\section{Present} \label{sec:evd:prs}
In the present situations, there is a contrast between three basic evidential categories, namely Factual, Sensory and Egophoric. This system, although superficially similar to the one observed in Tibetan between the Imperfective Factual \ipa{gi.yod.pa.red}, Sensory \textit{gis} and Egophoric \ipa{gi.yod} (\citealt[295+]{tournadre08conjunct}) presents in fact important differences, as will be apparent throughout the discussion.

\subsection{Factual} \label{sec:fact}

The Factual is the only finite TAM category in Japhug without a directional prefix.\footnote{The term Factual', taken from \citet{oisel13aux}'s study of modern Lhasa Tibetan, corresponds to the category referred to as `assertive' in older publications on Tibetan languages (Lhasa Tibetan \ipa{yod.pa.red}).} In the case of verbs without stem III alternation (i.e. transitive verbs with a non-alternating rhyme and intransitive verbs), it is realized as the bare stem. In spite of the absence of overt marking for most verbs, I still indicate `Factual' in all glosses of finite verb forms. The gloss is indicated as a suffix (verb\factual{}), rather than as a prefix, to avoid confusion in the case of verb forms with several derivation prefixes.

The Factual has three main functions when used in an independent clause without an auxiliary verb.\footnote{The Factual when combined with the auxiliary  `be' in the past imperfective form \ipa{pɯ-ŋu} has a conative or frustrative meaning, see \citealt[292]{jacques14linking}.}

First,  in the case of stative verbs, whether adjectival stative verbs or existential verbs/copulas, the Factual is used to describe facts considered to belong to everybody's common knowledge. Example \ref{ex:kumpGAtCW} illustrate five examples of the use of the Factual in this way, including copulas and adjectives. 

\begin{exe}
\ex \label{ex:kumpGAtCW}
\gll
\ipa{tɕe} 	\ipa{kumpɣɤtɕɯ} 	\ipa{nɯnɯ} 	\ipa{pɣɤtɕɯ} 	\ipa{nɯ-rca,} 
 \ipa{kɯ-xtɕi} 	\ipa{ci} 	\ipa{zdoʁzdoʁ} 	\ipa{\textbf{ŋu}} 	\ipa{tɕe,}  \ipa{ɯʑo} 	\ipa{\textbf{xtɕi}} 	\ipa{ri} 	\ipa{wuma} 	\ipa{ʑo} 	\ipa{\textbf{ɕqraʁ}} \ipa{tɕe}  	\ipa{ɯ-mɲaʁ} 	\ipa{ɯ-rkɯ} 	\ipa{nɯnɯ} 	\ipa{ra} \ipa{kɯ-ɲaʁ} 	\ipa{kɯ} 	\ipa{tú-wɣ-fskɤr,} 	 	\ipa{nɯ} 	\ipa{ɯ-taʁ} 	\ipa{ri,} 	\ipa{χanɯni,} 	\ipa{kɯ-xtɕɯ\textasciitilde{}xtɕi} 	\ipa{kɯ-ɣɯrni} 	\ipa{kɯ-fse} 	\ipa{\textbf{tu},}  	\ipa{ɯ-xtɤpa} 	\ipa{nɯ} \ipa{ra,} 	\ipa{ɯ-rqopa} 	\ipa{pjɯ-ʑe} 	\ipa{tɕe,} 	\ipa{nɯ} \ipa{ra,}  \ipa{ɯ-jme} 	\ipa{mɯ-thɯ-nɯɬoʁ} 	\ipa{mɤɕtʂa} 	\ipa{nɯ} 	\ipa{\textbf{wɣrum}} \\
\textsc{lnk} sparrow \textsc{dem} bird \textsc{3pl}-among \textsc{nmlz}:S/A-be.small \textsc{indef} \textsc{ideo:stat}:small.and.cute \textbf{be}\factual{} \textsc{lnk}  \textsc{3sg} \textbf{be.small}\factual{} but really \textsc{emph} \textbf{be.smart}\factual{}  \textsc{lnk}  \textsc{3sg:poss}-eye \textsc{3sg:poss}-border \textsc{dem}  \textsc{pl}  \textsc{nmlz}:S/A-be.black \textsc{erg} \textsc{ipfv-inv}-surround \textsc{dem} \textsc{3sg}-on \textsc{loc} a.little \textsc{nmlz}:S/A-\textsc{emph}\textasciitilde{}be.small \textsc{nmlz}:S/A-be.red \textsc{nmlz}:S/A-be.like \textbf{exist}\factual{} \textsc{3sg:poss}-belly \textsc{dem}  \textsc{pl}  \textsc{3sg:poss}-throat \textsc{ipfv}-begin[III] \textsc{lnk}   \textsc{dem}  \textsc{pl}  \textsc{3sg:poss}-tail \textsc{neg-pfv-auto}-come.out until \textsc{dem} \textbf{be.white}\factual{} \\
\glt Among the birds, the sparrow is tiny and cute. Although it is small it is very intelligent. Its eyes are surrounded by black (feathers), and above that there are some red (dots). Its belly is white from the throat until the tail. (Sparrow, 2-7)
\end{exe}

 Dynamic verbs are more often in the imperfective rather than the Factual in such contexts with a generic (such as \ipa{tú-wɣ-fskɤr}  \textsc{ipfv-inv}-surround  `it surrounds it' above)  or an expletive subject (S/A) (\ipa{pjɯ-ʑe} \textsc{ipfv}-begin[III] `it begins' in the example).\footnote{With stative verbs, using the imperfective rather than the Factual induces an inchoative meaning; thus \ipa{chɯ-wɣrum} \textsc{ipfv}-be.white actually means `it is becoming white' as opposed to \ipa{wɣrum} be.white\factual{} `it is white'.}
 
 The Factual is more commonly used than imperfective with dynamic verbs in general knowledge sentences with overt or definite subjects, as in \ref{ex:mAndze}.  

\begin{exe}
\ex \label{ex:mAndze}
\gll
   	\ipa{ɯ-ku}  	\ipa{kɯ-mpɯ}  	\ipa{nɯ}  	\ipa{ɲɯ́-wɣ-pʰɯt}  	\ipa{tɕe,}  \ipa{nɯŋa}  	\ipa{ra}  	\ipa{kɯ}  	\ipa{ndza-nɯ,}  	\ipa{paʁ}  	\ipa{kɯ}  	\ipa{mɤ-ndze}  
(Vine, 20) \\
\textsc{3sg.poss}-head \textsc{nmlz}:S/A-be.soft \textsc{dem} \textsc{ipfv-inv}-pluck \textsc{lnk} cow \textsc{pl} \textsc{erg} eat:\textsc{fact-pl} pig \textsc{erg} \textsc{neg}-eat\factual{} \\
\glt  One plucks the (leaves) on the extremities, the soft ones, the cows eat it, the pigs don't. (vine, 20)
\end{exe}

 
Second, with dynamic verbs, the Factual can express an assertion or an intention in the first person in assertive sentences (example \ref{ex:kWsAthu.Cea}) or in the second in interrogatives (\ref{ex:tWGi}).

\begin{exe}
\ex \label{ex:kWsAthu.Cea}
\gll
\ipa{aʑo}  	\ipa{kɯ-sɤ-tʰu}  	\ipa{ɕe-a}  \\
\textsc{1sg} \textsc{nmlz}:S/A-\textsc{antipass:human}-ask go\factual{}-\textsc{1sg} \\
\glt I am going (there) to ask (a girl in marriage). (Kunbzang, 5)
\end{exe}

\begin{exe}
\ex \label{ex:tWGi}
\gll
\ipa{mbarkhom} \ipa{tʰɤjtɕu} \ipa{tɯ-ɣi}?\\
Mbarkham when 2-come\factual{} \\
\glt When are you coming to Mbarkham? (conversation, 2014)
\end{exe}

Third, the Factual is also used for future events, when the speaker has reasonable reasons for assuming that they will take place as in \ref{ex:GWsata}. Verbs in the Factual in this usage are often combined with auxiliaries such as the affirmative copula \ipa{ɕti} `be' (\ref{ex:Gi.Cti}) or epistemic modality sentence final particles with  such as \ipa{tʰaŋ} `maybe, probably' (\ref{ex:Gi.thaN}).

\begin{exe}
\ex \label{ex:GWsata}
\gll
\ipa{si-a}   \ipa{ɲɤ-sɯso,} \ipa{tʰa}  	\ipa{ɣɯ-sat-a}  \ipa{ɲɤ-sɯso} \\
die:\textsc{fact-1sg} \textsc{ifr}-think in.a.moment \textsc{inv}-kill:\textsc{fact-1sg} \textsc{ifr}-think \\
\glt He thought ``I will die", he thought ``It will kill me".
\end{exe}

\begin{exe}
\ex \label{ex:Gi.Cti}
\gll
 	\ipa{tʰa} 	\ipa{ɣi} 	\ipa{ɕti} 	\ipa{tɕe,} 	\ipa{sɲikuku} 	\ipa{ʑo} 	\ipa{ju-ɣi} 	\ipa{tɕe} 	\ipa{nɯ} 	\ipa{ntsɯ} 	\ipa{tu-ti} 	\ipa{ɲɯ-ɕti} 	\\
 in.a.moment come\factual{} be:\textsc{affirm}\factual{} \textsc{lnk} every.day \textsc{emph} \textsc{ipfv}-come \textsc{lnk} \textsc{dem} always \textsc{ipfv}-say \textsc{sens}-be:\textsc{affirm} \\
\glt It will come  soon: it comes everyday and each times says this. (Kubzang 185-186)
\end{exe}

\begin{exe}
\ex \label{ex:Gi.thaN}
\gll
 \ipa{a-mu} 	\ipa{tɕi-scawa} 	\ipa{ma} 	\ipa{jɯɣmɯr} 	\ipa{tɕe} 	\ipa{iɕqha} 	\ipa{kʰu} 	\ipa{ɣi} 	\ipa{tʰaŋ} 	\ipa{nɤ} 	\\
 \textsc{1sg.poss}-mother  \textsc{1du.poss}-poor.of because today.evening \textsc{lnk} the.aforementioned tiger  come\factual{} probably \textsc{lnk} \\
\glt Mother, poor of us, today the tiger is probably coming (for us). (The tiger, 12)
 \end{exe}

 

%\begin{exe}
%\ex \label{ex:saRjArtCi}
%\gll
%  \ipa{ɕe-tɕi}  	\ipa{ma}  	\ipa{ci}  	\ipa{ni}  	\ipa{saʁjɤr-tɕi}  \\
%  go\factual{}-\textsc{1du} \textsc{lnk} \textsc{indef} \textsc{du}   delay\factual{}-\textsc{1du} \\
%\glt  We (have to) go, we are disturbing them. (conversation 14/05/10, 575)
%\end{exe}
 
 \subsection{Sensory } \label{sec:sens}
The Sensory evidential is built by combining stem III or stem I with the prefix \ipa{ɲɯ--} `towards west'.  For verbs whose intrinsic direction is `west', it is identical to the imperfective in the case of verbs. For instance \ipa{ɲɯ-ɤ<nɯ>ɣro} can either be analysed as \textsc{sens-<auto>}play or \textsc{ipfv-<auto>}play (both translatable as `he is playing').


The existential copulas \ipa{tu} `exist' and \ipa{me} `not exist' have suppletive Sensory forms \ipa{ɣɤʑu} `exist'  and \ipa{maŋe} `not exist', that are not compatible with any directional prefix.\footnote{The form \ipa{ɲɯ-me} exists, but it is the imperfective, not the Sensory form of \ipa{me} `not exist', and has a dynamic sense `disappear'. }

In affirmative sentences, the Sensory is generally found with second or third person verb forms. First person is limited to  cases like \ref{ex:YWrAZia} when the speaker discovers something about himself.

Sentences \ref{ex:WYWpendZi} and \ref{ex:Wkupe} illustrate the difference of use of the Sensory and egophoric forms in third person contexts. These questions expect answers in the Sensory and egophoric forms respectively. Question \ref{ex:WYWpendZi} was asked when I phoned from my parents' home (when I came for the holidays). The Sensory is used because I only seldom meet with my parents, and the expectation is that I had just realized whether or not they were well after having arrived at their place. 

Question \ref{ex:Wkupe} on the other hand, asked about my son, expects an answer in the egophoric because since I live with him in the same house, I always know whether he is fine or not.

The Factual would not be appropriate in these contexts because neither involve a permanent state part of common knowledge.

\begin{exe}
\ex \label{ex:WYWpendZi}
\gll 
\ipa{nɤ-mu}  	\ipa{nɤ-wa}  	\ipa{ni}  	\ipa{ɯ-ɲɯ́-pe-ndʑi?}  \\
\textsc{2sg.poss}-mother \textsc{2sg.poss}-father \textsc{du} \textsc{qu-sens}-be.good-\textsc{du} \\
\glt Are your parents well? (2014.12 conversation, Chenzhen)
\end{exe}


\begin{exe}
\ex \label{ex:Wkupe}
\gll \ipa{nɤ-tɕɯ} \ipa{ɯ-kú-pe?}\\
\textsc{2sg.poss}-son \textsc{qu-egoph}-be.good\\
\glt Is your son well? (2014.08 conversation, Dpalcan)
\end{exe}

 
 As in other languages with sensory evidentials, this form can be used to in contexts where surprise is implied (hence the term `mirative', on which see \citealt{delancey97mirative} and \citealt{hill12mirativity}). For instance \ref{ex:YWrAZia} shows Sensory used with first person, where the speaker, after having been revived, discover herself in an unfamiliar place. The Sensory here marks not so much surprise \textit{per se} (the interjection \ipa{ama} expresses this meaning) but the discovery of a new situation.

\begin{exe}
\ex \label{ex:YWrAZia}
\gll 
\ipa{ama,}  	\ipa{kɯki}  	\ipa{aʑo}  	\ipa{ki}  	\ipa{ŋotɕu}  	\ipa{ɲɯ-rɤʑi-a}  	\ipa{ɲɯ-ŋu?}  \\
\textsc{intrj:surprise} \textsc{dem:prox} \textsc{1sg} \textsc{dem:prox} where \textsc{sens}-stay-\textsc{1sg} \textsc{sens}-be \\
\glt Where am I? (The three leaves, 105)
\end{exe}


The Sensory is also used concerning information that is somehow part of common knowledge, but that the speaker has not had the opportunity to personally confirm. For instance, it is commonly used instead of the Factual for describing facts about animals that do not live in Rgyalrong areas and that the speaker only knows through indirect channels. Compare for instance the forms of the stative verbs \ipa{sɤɣ-mu} `be terrifying' and \ipa{mpɕɤr} `be beautiful': they appear in the Factual when referring to  spiders or flowers found in the area (\ref{ex:sAGmu} and \ref{ex:mpCAr}) and in the Sensory when referring to lions and zebras, which the speaker has only seen in zoos or in the television  (\ref{ex:YWsAGmu} and \ref{ex:YWmpCAr}).

 
 
 \begin{exe}
\ex \label{ex:sAGmu}
\gll 
\ipa{ŋgoŋpu}  	\ipa{ɴɢoɕna}  	\ipa{kɤ-ti}  	\ipa{ci}  	\ipa{tu}  	\ipa{tɕe,}  	\ipa{nɯnɯ}  	\ipa{wxti}  	\ipa{nɯ}  	\ipa{stoʁ}  	\ipa{jamar}  	\ipa{tu.}  	\ipa{kú-wɣ-rtoʁ}  	\ipa{tɕe}  	\ipa{sɤɣ-mu.}  \\
disaster spider \textsc{nmlz}:P-say \textsc{indef} exist\factual{} \textsc{lnk} \textsc{dem} be.big\factual{} \textsc{dem} bean about exist\factual{} \textsc{ipfv-inv}-look.at \textsc{lnk} \textsc{deexp}-be.afraid\factual{}  \\
\glt There is one that is  called `disaster spider', it is big, like the size of a bean. It is terrifying to look at it. (spiders, 128)
\end{exe}

\begin{exe}
\ex \label{ex:YWsAGmu}
\gll 
\ipa{sɯŋgi}  	\ipa{nɯ}  	\ipa{ɲɯ-sɤɣ-mu.}  \\
tiger \textsc{dem} \textsc{sens-deexp}-be.afraid \\
\glt The tiger  is terrifying. (Tigers, 64)
\end{exe}


\begin{exe}
\ex \label{ex:mpCAr}
\gll
\ipa{nɯnɯ}  	\ipa{ɯ-mɯntoʁ}  	\ipa{nɯ}  	\ipa{mpɕɤr.}  \\
\textsc{dem} \textsc{3sg.poss}-flower \textsc{dem} be.beautiful\factual{} \\
\glt Its flower is beautiful (\ipa{qarɣɤpɤt}, 105)
\end{exe}


\begin{exe}
\ex \label{ex:YWmpCAr}
\gll 
<banma> 	\ipa{nɯ}  	\ipa{ɲɯ-mpɕɤr}  \\
zebra \textsc{dem} \textsc{sens}-be.beautiful \\
\glt The zebra is beautiful. (Zebra, 128)
\end{exe}

As in other languages of the area, the Sensory form is used for endopathic sensations (pain, itch, cold etc)  relating to the speaker of affirmative sentences/addressee of interrogatives (\citealt{tournadre14evidentiality}), as in examples \ref{ex:YWmNAm} and \ref{ex:mWjmtshama}. Endopathic verbs may either be stative verbs or transitive tropative verbs (on which see \citealt{jacques13tropative}).

\begin{exe}
\ex \label{ex:YWmNAm}
\gll
\ipa{tʰam} 	\ipa{tɕe} 	\ipa{mɯ́j-cʰa-a,} 	\ipa{a-mi} 	\ipa{ɲɯ-mŋɤm.} \\
now \textsc{lnk} \textsc{neg:sens}-can-\textsc{1sg} \textsc{1sg.poss}-foot \textsc{sens}-hurt \\
\glt Now I can't, my foot hurts. (\ipa{ʑmbɯlɯm}, 24)
\end{exe}

\begin{exe}
\ex \label{ex:mWjmtshama}
\gll
\ipa{tɕe} 	\ipa{ɯ-qiɯ} 	\ipa{ɲɯ-mtsʰam-a,} 	\ipa{ɯ-qiɯ} 	\ipa{mɯ́j-mtsʰam-a} 	\ipa{qʰe,} 	\ipa{ɕe} 	\ipa{mɤ-ɕe} 	\ipa{maŋe.} \\
\textsc{lnk} \textsc{3sg.poss}-half \textsc{sens}-hear-\textsc{1sg}   \textsc{3sg.poss}-half \textsc{neg:sens}-hear-\textsc{1sg}  \textsc{lnk}  \textsc{bare.inf}:go \textsc{neg}-\textsc{bare.inf}:go \textsc{sens}:not.exist \\
\glt I hear half of it, and don't hear the rest, whether I go or not the result is the same. (conversation, 2014)
\end{exe}


Unlike in Lhasa Tibetan where the Sensory \ipa{ɴdug} cannot be used for non-personal endopathic feelings (\citealt{tournadre14evidentiality}), this possibility is available in Japhug. The speaker can use the Factual combined with an auxiliary such as \ipa{ŋgrɤl}  `be usually the case' or the hypothetical \ipa{tʰaŋ} (\ref{ex:rAZa}), but the Sensory is also possible, as in \ref{ex:tWCGa.YWmNAm} to \ref{Wmi.YWmNAm}.


\begin{exe}
\ex \label{ex:rAZa}
\gll
\ipa{ɯ-rni}  	\ipa{ɯ-stu}  	\ipa{nɯ}  	\ipa{rɤʑa}  	\ipa{tɕe}  	\ipa{tɕe}  	\ipa{nɤ-sɤɣ-dɯɣ}  	\ipa{ŋgrɤl}  	\ipa{loβ}  \\
\textsc{3sg.poss}-gum \textsc{3sg.poss}-place \textsc{dem} itch\factual{} \textsc{lnk} \textsc{lnk} \textsc{trop-deexp}-be.fed.up\factual{} be.usually.the.case\factual{} \textsc{sfp} \\
\glt His gums itch, and he can't bear it. (about a baby whose teeths are growing, conversation, 2014-10)
\end{exe}

In \ref{ex:tWCGa.YWmNAm}, the Sensory is used in a generic sentence, when the speaker has experienced himself the feeling and recounts his experience while presenting it as a generic fact, and thus do not count as a real example of Sensory with non-first person.

\begin{exe}
\ex \label{ex:tWCGa.YWmNAm}
\gll
\ipa{kɯ-maqʰu}  	\ipa{qʰe}  	\ipa{tɯ-ɕɣa}  	\ipa{ɲɯ-mŋɤm}  \\
\textsc{nmlz}:S/A-be.after \textsc{lnk} \textsc{genr.poss}-tooth \textsc{sens}-hurt \\
\glt Afterwards tooths hurt. (toothache, 66)
\end{exe}

Such an example is provided by examples \ref{ex:YWNWGAtChom} and \ref{ex:nWrqo.YWmNAm},  with the Sensory is used for endopathic sensations with non-first person referents. In example \ref{ex:nWrqo.YWmNAm}, which describes the effects of foot and mouth disease on cattle, the speaker  infers that the cattle suffering from the disease are in pain (because of their whining), yet uses the Sensory due to the fact that she describes an event she has directly witnessed. 

\begin{exe}
\ex \label{ex:nWrqo.YWmNAm}
\gll \ipa{nɯ-mci} 	\ipa{kɤ-rɤwum} 	\ipa{maka} 	\ipa{mɯ́j-cʰa-nɯ} 	\ipa{tɕe} 	\ipa{nɯ-mci} 	\ipa{tu-ɣɤrɯβrɯβ} 	\ipa{ʑo} 	\ipa{ɲɯ-ŋu.}  
\ipa{tɕe} 	\ipa{nɯ-rqo} 	\ipa{ɲɯ-mŋɤm} 	\ipa{rca,} \\
\textsc{3pl.poss}-saliva \textsc{inf}-collect at.all \textsc{neg:sens}-can-\textsc{pl} \textsc{lnk} \textsc{3pl.poss}-saliva \textsc{ipfv}-flow.continuously \textsc{emph} \textsc{sens}-be \textsc{lnk} \textsc{3pl.poss}-throat \textsc{sens}-hurt \textsc{top}  \\
\glt They cannot keep the saliva in their mouth, and it flows continuously. Their throat hurt. (Mouth-foot disease, 6)
\end{exe}

In \ref{Wmi.YWmNAm} likewise we have the Sensory used with \ipa{mŋɤm} `hurt' to describe an event directly witnessed by the speaker.

\begin{exe}
\ex \label{Wmi.YWmNAm}
\gll
\ipa{kɯɕnɤsqi} 	\ipa{tʰɯ-azɣɯt} 	\ipa{ri,} \ipa{tɕe} 	\ipa{pɤjkʰu} 	\ipa{ɯ-mi} 	\ipa{ɲɯ-mŋɤm} 	\ipa{tɕe} 	\ipa{ri,} 	\ipa{nɯ} 	\ipa{kɯnɤ} 	\ipa{kʰa} 	\ipa{tsʰitsuku} 	\ipa{ɲɯ-nɤme} 	\ipa{ɕti.} \\
seventy \textsc{pfv}-reach but \textsc{lnk} already \textsc{3sg.poss}-foot \textsc{sens}-hurt \textsc{lnk} but \textsc{dem} also house some.things \textsc{sens}-work[III] be:\textsc{affirm}\factual{} \\
\glt He is seventy, his foot hurts already, but even like that he does all sorts of work at home. (Relatives, 49-50)
\end{exe}

Example \ref{ex:YWNWGAtChom} presents a fact belonging to generic knowledge, but the Sensory rather than the Factual is used for the same reason as the examples \ref{ex:YWsAGmu} and \ref{ex:YWmpCAr} above.

\begin{exe}
\ex \label{ex:YWNWGAtChom}
\gll
\ipa{ɯ-ndzɤtshi}  	\ipa{nɯ}  	\ipa{ɲɯ-nɯ-ɣɤ-tɕʰom}  	\ipa{tɕe,}  	\ipa{ɯ-xtu}  	\ipa{ɲɯ-mŋɤm}  	\ipa{tɕe,}  	\ipa{pjɯ-kɯ-si}  	\ipa{ɣɤʑu.}  \\
\textsc{3sg.poss}-food \textsc{dem} \textsc{sens-auto-caus}-be.too.much \textsc{lnk} \textsc{3sg.poss}-belly \textsc{sens}-hurt \textsc{lnk} \textsc{ipfv-nmlz}:S/A-die exist:\textsc{sens} \\
\glt (The monkey eats) too much food, its belly aches, and some die of it. (Monkeys, 56)
\end{exe}
  
Perception or feelings relating to the speaker are not necessarily expressed with the sensory, as they can be construed as part of common knowledge, as in \ref{ex:mAmtshama}. The Factual  \ipa{mɤ-mtsʰam-a} `I can't hear' is used in this example because the audition issues of the speaker are a permanent state (not recently discovered), and is known by everybody.

\begin{exe}
\ex \label{ex:mAmtshama}
\gll
\ipa{kukukuku kukukuku} 	\ipa{tu-ti.} 	\ipa{nɯ} 	\ipa{tu-ti} 	\ipa{ɲɯ-ŋu} 	\ipa{tɕe,} 	\ipa{ɯ-skɤt} 	\ipa{mɯ́j-wxti} 	\ipa{kʰi.} 	\ipa{a-pɯ-ŋu,} \ipa{mɤ-mtsʰam-a} 	\ipa{woma} \\
{ } \textsc{ipfv}-say \textsc{dem}  \textsc{ipfv}-say \textsc{sens}-be \textsc{lnk}  \textsc{3sg.poss}-voice \textsc{neg:sens}-be.big \textsc{hearsay} \textsc{irr-ipfv}-be \textsc{neg}-\textbf{hear}:\textsc{fact-1sg} \textsc{sfp} \\
\glt (The snowcock) calls `kukuku', it calls like that. It is said that its voice is not big. Maybe it is like that, I can't hear (well) anyway. (Snowcock, 30-32)
\end{exe}

 With intransitive verbs, the Sensory is not normally used in the first person in affirmative sentences, except in two cases. First, in the case of the discovery of a new situation relating to oneself, after a state of lessened consciousness (as in \ref{ex:YWrAZia} above). Second, when the speaker refers to himself through the eyes of the addressee, as in \ref{ex:mtChi.kWfse.rJAlpu} and \ref{ex:thWmqlaRa}. In \ref{ex:mtChi.kWfse.rJAlpu} , the king asks a question about himself, and expects an answer in the Sensory form; the presence of Sensory on \ipa{ɲɯ-ŋu-a} is a case of anticipation of evidential form in questions.


\begin{exe}
\ex \label{ex:mtChi.kWfse.rJAlpu}
\gll
\ipa{aʑo}  	\ipa{tɕʰi}  	\ipa{kɯ-fse}  	\ipa{rɟɤlpu}  	\ipa{ɲɯ-ŋu-a?}  \\
\textsc{1sg} what \textsc{nmlz}:S/A-be.like king \textsc{sens}-be-\textsc{1sg} \\
\glt What kind of a king am I? (The ape  king, 23)
\end{exe}


\begin{exe}
\ex \label{ex:thWmqlaRa}
\gll
\ipa{a-zda}  	\ipa{tɤ-tɕɯ}  	\ipa{ɲɯ-ŋu}  	\ipa{tɕe,}  	\ipa{χsɤr}  	\ipa{qaɕpa}  	\ipa{ci}  	\ipa{ɣɤʑu}  	\ipa{tɕe,}  	\ipa{nɯ}  	\ipa{tʰa-mqlaʁ.} \ipa{aʑo}  	\ipa{tɕʰeme}  	\ipa{ɲɯ-ŋu-a}  	\ipa{tɕe,}  	\ipa{rŋɯl}  	\ipa{qaɕpa}  	\ipa{ci}  	\ipa{ɣɤʑu}  	\ipa{tɕe,}  	\ipa{nɯ}  	\ipa{tʰɯ-mqlaʁ-a.}  \\
\textsc{1sg.poss}-companion \textsc{indef.poss}-son \textsc{sens}-be  \textsc{lnk} gold frog \textsc{indef} exist:\textsc{sens} \textsc{lnk} \textsc{dem} \textsc{pfv}:3$\rightarrow$3'-swallow \textsc{1sg} girl \textsc{sens}-be-\textsc{1sg}  \textsc{lnk} silver frog \textsc{indef} exist:\textsc{sens} \textsc{lnk} \textsc{dem} \textsc{pfv}-swallow-\textsc{1sg} \\
\glt (As you can see) my companion (present at the moment) is a boy, there was a golden frog and he swallowed it, I am a girl, there was a silver frog and I swallowed it. (Nyima Wodzer, 144)
\end{exe}

In example \ref{ex:thWmqlaRa}, the speaker uses the Sensory form to refer to himself and her companion, ie. the form the addressee would be expected to use. 


\subsection{Egophoric present } \label{sec:egoph}
The egophoric present  is built by prefixing \ipa{ku--} `towards east' to the stem III or stem I of the verb. It is thus homophonous with the imperfective in the case of verbs whose intrinsic direction is `east'.  For instance the form \ipa{ku-rɤʑi} is ambiguous between \textsc{egoph}-remain and \textsc{ipfv}-remain (both could be translated as `he is there' or he is remaining there' depending on the context).

There are no distinct egophoric existentials verbs; \ipa{tu} `exist' and \ipa{me}  `not exist' simply take the prefix \ipa{ku--} like regular verbs, as in \ref{ex:kume}.

\begin{exe}
\ex \label{ex:kume}
\gll 
\ipa{aʑo}  	\ipa{kɯre}  	\ipa{a-ʁa}  	\ipa{ku-me}  \ipa{tɯ-mgo} 	\ipa{ku-osɯ-βzu-a} 	\ipa{ɕti} 	  	\\
\textsc{1sg} here \textsc{1sg.poss}-free.time \textsc{prs:egoph}-not.exist \textsc{indef.poss}-food \textsc{prs:egoph}-\textsc{prog}-make-\textsc{1sg} be.\textsc{affirm}\factual{}  \\
\glt I don't have time, I am making food.  (Rkangrgyal, 47)
\end{exe}

The egophoric  is rather rare in narratives, but very common in conversations. Like the factual, it is used to express intimate knowledge of an event or state on the part of the speaker, not resulting from guess or recent information mediated through the senses. It cannot express a general or gnomic state of affair, it is only used to refer to an ongoing state or action (as in \ref{ex:kutaRa}; often with the transitive progressive prefix \ipa{asɯ--} as in \ref{ex:kume}). It is the only TAME category in Japhug to be exclusively restricted to present situations, as the Factual is also used for future actions and the Sensory appears in past contexts.

\begin{exe}
\ex \label{ex:kutaRa}
\gll 
<kuabao> 	\ipa{ɯ-spa}  	\ipa{ci}  	\ipa{ku-taʁ-a}  \\
bag \textsc{3sg.poss}-material \textsc{indef} \textsc{prs:egoph}-weave-\textsc{1sg} \\
\glt I am weaving (the cloth for) a bag. (conversation, 14.10)
\end{exe}


It appears mainly  in second person form in questions (examples \ref{ex:WkutWscitnW} and \ref{ex:kutWnAme}) and in first person form in declarative sentences (as in the first clause in \ref{ex:kusciti}, the answer to the question in \ref{ex:WkutWscitnW}) due to the rule of anticipation.

\begin{exe}
\ex \label{ex:WkutWscitnW}
\gll  \ipa{`a-ʁi} 			\ipa{ɯ-kú-tɯ-scit-nɯ?}' 	\ipa{ra} 	\ipa{to-ti,} \\
\textsc{1sg.poss}-younger.sibling  \textsc{qu-prs:egoph}-2-be.happy-\textsc{pl} \textsc{pl} \textsc{evd}-say \\
\glt She said: `Are you (and your husband) happy, my sister?' (The frog 2002, 121)
\end{exe}

\begin{exe}
\ex \label{ex:kutWnAme}
\gll \ipa{nɯtɕu}  \ipa{tɕʰi} \ipa{ku-tɯ-nɤme?}\\
what there \textsc{prs:egoph}-2-work[III] \\
\glt What are you doing there? (The smart rabbit 2012, 8)
\end{exe}
 
 
\begin{exe}
\ex \label{ex:kusciti}
\gll
\ipa{tɕʰeme} 	\ipa{nɯ} 	\ipa{kɯ} 	\ipa{`wuma} 	\ipa{ʑo} 	\ipa{ku-scit-i,} \ipa{rɟɤlpu} 	\ipa{ri} 	\ipa{a-taʁ} 	\ipa{wuma} 	\ipa{ku-sna} \ipa{ʁjoʁ} 	\ipa{ra} 	\ipa{ri} 	\ipa{wuma} 	\ipa{ʑo} 	\ipa{ku-pe-nɯ'} \ipa{to-ti,} \\
girl \textsc{dem} \textsc{erg} really \textsc{emph} \textsc{prs:egoph}-be.happy-\textsc{1pl}  roi also \textsc{1sg}-on really \textsc{prs:egoph}-be.kind servant \textsc{pl} also really \textsc{emph}   \textsc{prs:egoph}-be.good \textsc{ifr}-say \\
\glt The girl said `We are very happy, the king is very kind with me, the servants are very nice.'
(The frog 2002, 122-4)
\end{exe}

It is used also in either declarative (example \ref{ex:kusciti}) or  interrogative sentences (\ref{ex:Wkupe} and \ref{ex:WkudAn}) to refer to third persons, in the case of things belonging to or persons attached to the speaker (in declarative sentences) or the addressee (in the case of interrogatives). In example \ref{ex:kusciti}, the  use of present Factual is motivated by the facts that (1) the speaker is affected by the state of the persons she refers to (2) these persons are member or her household (her husband (the king) and her servant). Similar uses of the Egophoric are also found in Tibetan (\citealt[297]{tournadre08conjunct}).

 
  \begin{exe}
\ex \label{ex:WkudAn}
\gll \ipa{nɤ-kɤ-nɤma} 	\ipa{ɯ-kú-dɤn?}  \\
 \textsc{2sg.poss-nmlz:P}-work \textsc{qu-prs:egoph}-be.many \\
\glt Do you have a lot of work? (2014.10 conversation, Chenzhen)
\end{exe}

As seen above, a question in the present Factual expects an answer in the same form. However, answering in the periphrastic imperfective is also possible in the case of dynamic verbs, as in \ref{ex:pjWrAGrWa}, the answer to \ref{ex:kutWnAme}.\footnote{In another version of the same story by the same speaker however, the present Factual is also found in the answer.}

 \begin{exe}
\ex \label{ex:pjWrAGrWa}
\gll
 \ipa{a-pi} 	\ipa{kɯrtsɤɣ} 	\ipa{ma-tɯ-ɤrju} 	\ipa{ma,} 	\ipa{maka} 	\ipa{aʑo} 	\ipa{a-xtu} 	\ipa{ɯ-tɯ-mŋɤm} 	\ipa{ɲɯ-sɤre} 	\ipa{ʑo} 	\ipa{tɕe,} 	\ipa{kukutɕu} 	\ipa{pjɯ-rɤɣrɯ-a} 	\ipa{ŋu} 	\\
 \textsc{1sg.poss}-elder.sibling leopard \textsc{neg:imp}-2-say because at.all \textsc{1sg} \textsc{1sg.poss}-belly \textsc{3sg-nmlz:degree}-hurt \textsc{sens}-be.funny \textsc{emph} \textsc{lnk} here \textsc{ipfv}-treat.with.heat-\textsc{1sg} be\factual{}  \\
\glt Brother leopard, don't talk, my belly hurts terribly, and I am treating it. (The smart rabbit 2012, 10-2)
 \end{exe}
 
Japhug and other Gyalrongic languages stand out among languages with egophoric markers, from Tibetan (\citealt{tournadre08conjunct}) to Akhvakh (\citealt{creissels08akhvakh}), in having  a complex person indexation system. This confirms the idea that egophoric (or `conjunct-disjunct') markers are radically distinct from person indexation, though they may be historically related to them.\footnote{For instance, the non-egophoric volitional \ipa{-w-} infix in Pumi (\citealt{daudey14volition}) is potentially related to the inverse markers found in Gyalrongic languages (on which see \citealt{delancey81direction}, \citealt{jackson02rentongdengdi}, \citealt{jacques10inverse}, \citealt{gongxun14agreement}), which have become generalized to all third person agent forms in Khroskyabs and Stau (\citealt{jacques14rtau}, \citealt{lai14person}).}

%Modal verbs expressing an ability or knowledge such as \ipa{cʰa} `can', \ipa{spa} `be able to' or \ipa{sɯz} `know' are more often used in the factual (except in negative forms), others  such as \ipa{sɯso} `think'
 
\section{Past} \label{sec:evd:pst}
In past situations, neither the Factual nor the Egophoric  can be used. The main evidential contrast is between the Inferential and the Perfective on the one end, and between Imperfective Inferential and Past Imperfective on the other hand. In addition, the Sensory may appear in some past contexts.


\subsection{Inferential}  \label{sec:ifr}
The Inferential in Japhug is build by combining type D directional prefixes (see Table \ref{tab:directional} above) with Stem I of the verb. In addition, the transitive \ipa{--t} suffix appears in \textsc{2sg}$\rightarrow$3 forms. The Inferential is used in three main contexts. 


First, the Inferential appears when reporting information learned through other people without direct observation on the part of the speaker. For this reason, it is the TAME category with the highest frequency of occurrence in traditional stories, and one can find long stretches of text with all verbs in the Inferential as in example \ref{ex:pjAnWntChanW}.

\begin{exe}
\ex \label{ex:pjAnWntChanW}
\gll  \ipa{nɯ-nɯŋa} 	\ipa{pjɤ-nɯ-ntɕha-nɯ} 	\ipa{tɕe,} \ipa{tɕendɤre} 	\ipa{tɤ-pi} 	\ipa{ni} 	\ipa{kɯ} 	\ipa{ndʑi-kɯ-ra} 	\ipa{nɯra} 	\ipa{jo-nɯ-tsɯm-ndʑi} 	\ipa{tɕe} \ipa{tɤpi} 	\ipa{kɯ-wxti} 	\ipa{nɯ} 	\ipa{kɯ} 	\ipa{nɯŋa} 	\ipa{ɣɯ} 	 \ipa{ɯ-lu} 	\ipa{nɯ} 	\ipa{chondɤre} \ipa{ɯ-ɕa} 	\ipa{nɯ} 	\ipa{to-nɯ-ndo,} \\
\textsc{3pl.poss}-cow \textsc{ifr}-\textsc{auto}-kill-\textsc{pl} \textsc{lnk}  \textsc{lnk} \textsc{indef.poss}-elder.sibling \textsc{du} \textsc{erg} \textsc{3du.poss}-\textsc{nmlz}:S/A-need \textsc{dem:pl} \textsc{ifr}-\textsc{auto}-take.away-\textsc{du} \textsc{lnk} \textsc{indef.poss}-elder.sibling \textsc{nmlz}:S/A-be.big \textsc{dem} \textsc{erg} cow \textsc{gen} \textsc{3sg.poss}-milk \textsc{dem} \textsc{comit}  \textsc{3sg.poss}-meat  \textsc{dem} \textsc{ifr}-\textsc{auto}-take \\
\glt They killed their cow, and the two elder brothers took what they needed, the eldest took the cow's milk and meat, (Histoire 02 : deluge2012)
\end{exe}



Second, the Inferential is used when only the result of an action has been observed by the speaker, not the action itself, as in \ref{ex:pjAsi} or \ref{ex:totWwGndza}. In example \ref{ex:pjAsi}, the speaker has seen the dead turtle (its death was not reported by someone else), but using the Perfective \ipa{pɯ-si}  here would imply that he has witnessed the moment of the death of the animal. 

\begin{exe}
\ex  \label{ex:pjAsi}
\gll \ipa{li} 	\ipa{pjɯ-χsu-j} 	\ipa{pɯ-ŋu} 	\ipa{ri} 	\ipa{tɕe,} \ipa{kɯ-maqhu} 	\ipa{tɕe} 	\ipa{tɯ-rdoʁ} 	\ipa{nɯ} 	\ipa{pjɤ-si} 	\ipa{tɕe,} \\
again \textsc{ipfv}-raise-\textsc{1pl} \textsc{pst.ipfv}-be but \textsc{lnk} \textsc{inf:stat}-be.after \textsc{lnk}  one-\textsc{cl} \textsc{dem} \textsc{ifr}-die \textsc{lnk} \\
\glt We raised them (the turtles) for a time but later one of them died. (14.05.10 - wugui, 45)
\end{exe}


\begin{exe}
\ex \label{ex:totWwGndza}
\gll \ipa{wo}	\ipa{jmɤrtaʁ} 	\ipa{kɯ} 	\ipa{to-tɯ́-wɣ-ndza} 	\ipa{ɲɯ-ŋu!} \\
\textsc{interj} forked-tail \textsc{erg} \textsc{evd-2-inv}-eat \textsc{sens}-be\\
\glt O, you have been bitten by a forked-tail bug! (seeing someone itching; 25 qro, 128)
\end{exe}

Third, the Inferential can be used to describe an involuntary action, often used with the spontaneous-autobenefactive prefix. This use is especially common with first person in affirmative sentences or the second person in questions. Verbs expressing intrinsically involuntary action like \ipa{jmɯt} `forget' are only very rarely used in the Perfective form.

\begin{exe}
\ex 
\gll \ipa{kɤ-ti} \ipa{ɲɤ-nɯ-jmɯt-a} \\
\textsc{inf}-say  \textsc{ifr-auto}-forget-\textsc{1sg} \\
\glt I forgot (how) to say it. (kunbzang 2003, 89)
\end{exe}

For stative verbs, in addition to the evidential meanings described above, the Inferential implies a change of state as in \ref{ex:YAGWrni} or an increase of degree as in \ref{ex:towxti}.


\begin{exe}
\ex \label{ex:YAGWrni}
\gll 
\ipa{iɕqha} 	\ipa{tɤɕime} 	\ipa{ɣɯ} 	\ipa{ɯ-rŋa} 	\ipa{nɯnɯ} 	\ipa{ɯ-mdoʁ} 	\ipa{nɯ} 	\ipa{nɯɕɯmɯma} 	\ipa{ʑo} 	\ipa{ɲɤ-ɣɯrni} 	\ipa{sɯrsɯr} 	\ipa{qhe} 	\ipa{ɯ-mɲaʁ} 	\ipa{ɲɤ-cɯ}\\
the.aforementioned girl \textsc{gen} \textsc{3sg.poss}-face \textsc{dem} \textsc{3sg.poss}-colour \textsc{dem} immediately \textsc{emph} \textsc{ifr}-be.red \textsc{ideo}:II:round \textsc{lnk} \textsc{3sg.poss}-eye \textsc{ifr}-open \\
\glt The girl's face immediately became red and she opened her eyes. (14.05.10 sanpian sheye, 104)
\end{exe}

\begin{exe}
\ex \label{ex:towxti}
\gll \ipa{mɤʑɯ} 	\ipa{ʑo} 	\ipa{ɯ-kha} 	\ipa{ra} 	\ipa{to-wxti} 	\ipa{tɕe,} \\
even.more \textsc{emph} \textsc{3sg.poss}-house \textsc{pl} \textsc{ifr}-be.big \textsc{lnk} \\
\glt Her house became even bigger. (yufu he ta de qizi)
\end{exe}



\subsection{Perfective}  \label{sec:pfv}
The Perfective is the only TAME category in Japhug using Stem 2, in combination with either type A or type C directional prefixes (the latter occur in 3$\rightarrow$3 forms of transitive verbs when the inverse prefix is not present). As the Inferential, the Perfective requires a past transitive \ipa{-t} suffix in  \textsc{1sg}$\rightarrow$3 and \textsc{2sg}$\rightarrow$3 forms, as in example \ref{ex:tAndzata}.

\begin{exe}
\ex \label{ex:tAndzata}
\gll \ipa{aʑo} 	\ipa{kɯ} 	\ipa{ɯʑo} 	\ipa{tɤ-ndza-t-a} 	\ipa{ɕti} \\
\textsc{1sg} \textsc{erg} \textsc{3sg} \textsc{pfv}-eat-\textsc{pst:tr-1sg} be:\textsc{affirm:fact} \\
\glt It is I who ate him. (The demon, 95)
\end{exe}

When used in an independent clause,  the Perfective stands in opposition with the Inferential, and implies that the speaker has directly (with his own senses) witnessed the event, as in  \ref{ex:taXtW}.

\begin{exe}
\ex \label{ex:taXtW}
\gll \ipa{akɯ} 	\ipa{ri} 	\ipa{χpaltɕin} 	\ipa{kɯ} 	\ipa{nɯ-kha} 	\ipa{ci} 	\ipa{ta-χtɯ} \\
east \textsc{loc} Dpalcan \textsc{erg} \textsc{3pl.poss}-house \textsc{indef} pfv:3$\rightarrow$3'-buy \\
\glt Dpalcan bought a house for them there in the east. (hist 12 BzaNsa)
\end{exe}

In combination with various postpositions and relator nouns, the Perfective form is widely used in temporal subordinate clauses (see \citealt[284-93]{jacques14linking}). Used on its own in a subordinate clause, it is used to fix a point in time serving as the background for the situation described by the following clause even in future situations, as in \ref{ex:thWwxti}.

\begin{exe}
\ex \label{ex:thWwxti}
\gll 
\ipa{nɯnɯ} 	\ipa{ɯ-tɕɯ} 	\ipa{nɯ} 	\ipa{fso} 	\ipa{thɯ-wxti} 	\ipa{tɕe} 	\ipa{tha} 	\ipa{aʑɯɣ} 	\ipa{mɤ-sɤ-scit}  \\
\textsc{dem} \textsc{3sg.poss}-son \textsc{dem} tomorrow \textsc{pfv}-be.big \textsc{lnk} in.a.moment \textsc{1sg:gen} neg-deexp-be.happy\factual{} \\
\glt In the future, when his son will grow up, things will not be nice for me. (hist 28 smanmi)
\end{exe}

In subordinate clauses, the perfective does not imply direct witness on the part of the speaker, and can be combined with clauses in the inferential. In example \ref{ex:kazGWtndZi} for instance, the narrator switches from Inferential to Perfective with the same verb (\ipa{ko-zɣɯt-ndʑi} `they arrived' vs \ipa{kɤ-azɣɯt-ndʑi} `when they arrived')\footnote{The verb \ipa{zɣɯt} `arrive' is irregular, as it presents the stem \ipa{azɣɯt} in some parts of its paradigm, see \citet[424]{jacques04these}. } in order to make the action (the arrival) the reference point for the beginning of a new sequence in the story.

\begin{exe}
\ex \label{ex:kazGWtndZi}
\gll \ipa{tɕendɤre} 	\ipa{ko-ɣi-ndʑi} 	\ipa{tɕe,} 	\ipa{tɕendɤre} 	\ipa{ɯkɤcu} 	\ipa{tɕe} 	\ipa{iɕqha} 	\ipa{mtshu} 	\ipa{kɯ-wɣrum} 	\ipa{nɯ} 	\ipa{ɣɯ} 	\ipa{ɯ-taʁ} 	\ipa{nɯ} \ipa{tɕu} 	\ipa{ko-zɣɯt-ndʑi.} \ipa{tɕe} 	\ipa{nɯ} 	\ipa{kɯ-wɣrum} 	\ipa{nɯ} 	\ipa{ɯ-taʁ} 	\ipa{nɯtɕu} 	\textbf{\ipa{kɤ-azɣɯt-ndʑi}} 	\ipa{tɕe} 	\ipa{tɕendɤre,} 
\ipa{tɤ-wɯ} 	\ipa{ci} 	\ipa{li} 	\ipa{ɯ-ku} 	\ipa{tɤ-kɯ-wɣrɯ\textasciitilde{}wɣrum} 	\ipa{ci} 	\ipa{pjɤ-rɤʑi} \ipa{tɕe,} \ipa{`ndzaʁlaŋ} 	\ipa{tɯrme,} 	\ipa{ŋotɕu} 	\ipa{tɯ-ɕe?'} 	\ipa{to-ti} 	\ipa{ri,} \\
\textsc{lnk} \textsc{ifr:east}-come-\textsc{du} \textsc{lnk}  \textsc{lnk} \textsc{3sg}-in.the.east  \textsc{lnk} the.aforementioned lake \textsc{nmlz}:S/A-be.white \textsc{dem} \textsc{gen} \textsc{3sg}-on \textsc{ifr}-arrive-\textsc{du} \textsc{lnk} \textsc{dem} \textsc{nmlz}:S/A-be.white \textsc{dem} \textsc{gen} \textsc{3sg}-on \textsc{dem} \textsc{loc} \textsc{pfv}-arrive-\textsc{du}  \textsc{lnk}  \textsc{lnk} \textsc{indef.poss}-grandfather \textsc{indef} again \textsc{3sg.poss}-head \textsc{pfv}-\textsc{nmlz}:S/A-\textsc{emph}\textasciitilde{}be.white  \textsc{indef} \textsc{ifr.ipfv}-stay   \textsc{lnk} Jambudvîpa man where 2-go\factual{} \textsc{ifr}-say \textsc{lnk} \\
\glt They came (back towards the Eastern land, from the West), on the east of there, they arrived at the white lake. When they arrived at the white lake, (they saw there) an old man whose hair was all white who said, `Where are you going, man from Jambudvîpa?' (hist 28 smanmi)
\end{exe}

The Perfective can also be used in subordinate clauses referring to future events, as in \ref{ex:tAGApWplWG}; its use is restricted to past events only in independent clauses.

\begin{exe}
\ex \label{ex:tAGApWplWG}
\gll \ipa{ku-nɯ-rŋgɯ} 	\ipa{ŋu} 	\ipa{tɕe,} 	\ipa{ɯ-ro} 	\ipa{tɤ-ɣɤpɯplɯɣ} 	\ipa{ʑo} 	\ipa{tɕe,} 	\ipa{nɯ} 	\ipa{tɕu} 	\ipa{ʑo} 	\ipa{mdaʁʑɯɣ} 	\ipa{a-tɤ-tɯ-lɤt} 	\ipa{ma}  \\
\textsc{ipfv-auto}-lie.down be:\factual{} \textsc{lnk} \textsc{3sg.poss}-chest \textsc{pfv}-glitter \textsc{emph} \textsc{lnk} \textsc{dem} \textsc{loc} bow.and.arrow \textsc{irr-pfv}-2-throw \textsc{lnk} \\
\glt When he will be lying down, his chest will glitter, and at that time shoot an arrow (The demon, 34-5)
\end{exe}


This absence of evidential meaning in subordinate clauses shows that the perfective is not specified for evidentiality.\footnote{Likewise, in relative clauses, the Perfective is used in a context where the contrast between Perfective and Inferential is neutralized, see section \ref{sec:neutralization}.} Instead, it shows that the evidential meaning of `direct witness' in independent sentences results from its contrast with the inferential, which becomes neutralized in subordinate clauses.


\subsection{Past imperfective and imperfective evidential}  \label{sec:pst:ipfv}
The Inferential and Perfective verb forms have imperfective counterparts, the Inferential Imperfective and the  Past imperfective. These forms are marked by the prefixes \ipa{pjɤ--} and \ipa{pɯ--} respectively, identical to the 
corresponding `down' directional prefixes of series D and A respectively (\citealt{lin11direction}). These forms are refer to a past state that is no longer valid, as in example \ref{ex:pjAtundZi}. 

\begin{exe}
\ex \label{ex:pjAtundZi}
\gll  \ipa{tɤ-mu} 	\ipa{kɤtsa} 	\ipa{ci} 	\ipa{pjɤ-tu-ndʑi} 	\ipa{tɕe,} 	\ipa{wuma} 	\ipa{ʑo} 	\ipa{pjɤ-ŋgɯ-ndʑi} 	\ipa{tɕe}  \\
\textsc{indef.poss}-mother with.child \textsc{indef} \textsc{ifr:ipfv}-exist-\textsc{du} \textsc{lnk} really \textsc{emph}  \textsc{ifr:ipfv}-be.poor-\textsc{du} \\
\glt The was a mother and her son, and they were very poor.
\end{exe}

The evidential contrast between the two forms is the same as that found between Inferential and Perfective in independent clauses. The Past Imperfective does have special uses in concessive clauses (see \citealt[298]{jacques14linking}), where the evidential meaning is neutralized.

With stative verbs, the Past Imperfective and Inferential Imperfective forms can be used without any constraints. With dynamic verbs, this form is restricted to two contexts: in concessive or counterfactual clauses (see \citealt[298]{jacques14linking}), and in combination with the progressive prefix \ipa{asɯ--}, as in \ref{ex:paznWkhrWa}.

\begin{exe}
\ex \label{ex:paznWkhrWa}
\gll 
\ipa{ma} 	Achun 	\ipa{pɯ-az-nɯskhrɯ-a} 	\ipa{ri} 	\ipa{kɤ-tshi-t-a} \\
\textsc{lnk} \textsc{name} \textsc{pst.ipfv-prog}-be.pregnant.with-\textsc{1sg} \textsc{lnk} \textsc{pfv}-drink-\textsc{pst:tr-1sg} \\
\glt I drank (some of it) when I was pregnant with Achun. (histoire 27 qartshaz, 113)
\end{exe}

Otherwise, a periphrastic tense combining the imperfective form with the auxiliary `be' in the Past Imperfective \ipa{pɯ-ŋu} or Inferential Imperfective \ipa{pjɤ-ŋu}  is used.   A single auxiliary  can have scope over several verbs in the imperfective, as in \ref{ex:shangke}.

\begin{exe}
\ex \label{ex:shangke}
\gll  \ipa{sloχpɯn} 	\ipa{pɯ-ŋu-a} 	\ipa{tɕe} 	\ipa{tɕe} 	\ipa{tu-fkur-a} 	\ipa{tɕe} 	\ipa{tɕe} 	 <shangke> 	\ipa{lu-βze-a} 	\ipa{pɯ-ŋu} \\
teacher \textsc{pst.ipfv}-be-\textsc{1sg} \textsc{lnk}  \textsc{lnk} \textsc{ipfv}-carry.on.the.back-\textsc{1sg} \textsc{lnk}  \textsc{lnk}  teach.classes \textsc{ipfv}-make-\textsc{1sg} \textsc{pst.ipfv}-be \\
\glt I was a teacher, and I would carry him on the back while I was teaching classes. (14.04.26 tApAtso)
\end{exe}


\subsection{Sensory}  \label{sec:pst:sens}
While the Factual and egophoric are not possible in past situations, the Sensory does appear in past imperfective contexts instead of the Past Imperfective to highlight the fact that the speaker  has directly witnessed (through any of his senses) the event in question, as in \ref{ex:jAwGsWGWta}. 

\begin{exe}
\ex \label{ex:jAwGsWGWta}
\gll  \ipa{kɯki} 	\ipa{nɤʑo} 	\ipa{ɣɯ} 	\ipa{nɤ-tɤ-fkɯm} 	\ipa{kɯki} 	\ipa{kho-kɯm} 	\ipa{zɯ} 	\ipa{ɲɯ-ɤ-ta} 	\ipa{tɕe} 	\ipa{tɕendɤre,} \ipa{aʑo} 	\ipa{a-βdaχpu} 	\ipa{nɯ} 	\ipa{kɯ} 	\ipa{pjɤ-nɯmto} 	\ipa{tɕe} 	\ipa{nɤ-ɕki} 	\ipa{jɤ́-wɣ-sɯ-ɣɯt-a} 	\ipa{ŋu}  \\
 \textsc{dem:prox}  \textsc{2sg} \textsc{gen} \textsc{2sg.poss-indef.poss}-bag \textsc{dem:prox} house-door \textsc{loc} \textsc{sens-pass}-put \textsc{lnk} \textsc{lnk} \textsc{1sg} \textsc{1sg.poss}-master \textsc{dem} \textsc{erg} \textsc{ifr}-find.on.the.ground  \textsc{lnk} \textsc{2sg-dat} \textsc{pfv-inv-caus}-bring-\textsc{1sg} be\factual{} \\
\glt This bag of yours was on (our doorstep), my master found it, and sent me to bring it to you. (14 qianshang he xiaotou, 66)
\end{exe}

With dynamic transitive verbs, the Sensory in past contexts is most often used with the progressive \ipa{asɯ--}, as in \ref{ex:YAsWlAt}.

\begin{exe}
\ex \label{ex:YAsWlAt}
\gll  \ipa{a}-<dianhua> 	\ipa{ɲɯ-ɤsɯ-lɤt} 	\ipa{ri,} 	\ipa{kɯre} 	\ipa{ɕti-a} 	\ipa{tɤ-tɯt-a} 	\ipa{tɕe,} 	\ipa{qhe} 	\ipa{khapa} 	<zuoye> 	\ipa{pjɯ-nɯ-βze-a} 	\ipa{ŋu} 	\ipa{ɲɯ-ti} \\
\textsc{1sg.poss}-telephone \textsc{sens-prog}-throw \textsc{lnk} here be.affirm\factual{}-\textsc{1sg} \textsc{pfv}-say[II]-\textsc{1sg} \textsc{lnk} \textsc{lnk} downstairs homework \textsc{ipfv-auto}-make[III]-\textsc{1sg} be:\factual{} \textsc{sens}-say \\
\glt We were on the phone, I told her that I was here, and she was telling me that she was doing her homework downstairs. (conversation, 14.05.10)
\end{exe}

\section{Hearsay}
An evidential sentence final particle \ipa{kʰi}, expressing hearsay, can be combined with the evidential verbal system described in sections \ref{sec:evd:prs} and \ref{sec:evd:pst}. The particle \ipa{kʰi} does not mark reported speech: there is no implication that the wording reproduces that of the original speaker, and in any case no shift of point of view occurs. 

For reporting past events, the marker \ipa{kʰi} is always combined with the Inferential, as in \ref{ex:pjAsi}, never with the Perfective.

\begin{exe}
\ex \label{ex:pjAsi}
\gll \ipa{tɕɤtu} 	\ipa{ʑara} 	\ipa{nɯ-ɴɢarmɯ} 	\ipa{nɯ} 	\ipa{pjɤ-si} 	\ipa{kʰi} 	\ipa{tɕe} \\
over.there \textsc{3pl} \textsc{3pl.poss}-mongrel.cow  \textsc{dem} \textsc{ifr}-die \textsc{hearsay} \textsc{lnk} \\
\glt The people over there, their cow died, they say.
\end{exe} 
 
 The hearsay marker \ipa{kʰi} also appears with sentences expressing a general state of affair, in which case in can be combined with either a verb in the Factual or the Sensory. The Factual is used to refer to events or states that are part of common knowledge, but that the speaker only knows from a third party, as in \ref{ex:pGAkhW}.
 
 \begin{exe}
\ex \label{ex:pGAkhW}
\gll   \ipa{pɣɤkhɯ}  	\ipa{nɯ}  	\ipa{kɯ}  	\ipa{qaɲi}  	\ipa{kɤ-sat}  	\ipa{wuma}  	\ipa{ʑo}  	\ipa{cʰa} 	\ipa{kʰi.}  \\
owl \textsc{dem} \textsc{erg} mole \textsc{inf}-kill really \textsc{emph} can:\factual{} \textsc{hearsay} \\
\glt The owl is very good at killing moles, they say.
\end{exe} 
  
  The Sensory with the hearsay \ipa{kʰi} expresses a fact that the speakers knows from someone else, from  a book or from television, that cannot be considered to be part of the expected common knowledge of members of the local community, as in \ref{ex:elephant}, where the Sensory existential verb \ipa{ɣɤʑu} is used instead of the Factual \ipa{tu}.
  
   \begin{exe}
\ex \label{ex:elephant}
\gll   \ipa{tɕeri}  	\ipa{ʁloŋbutɕʰi}  	\ipa{nɯ}  	\ipa{ʁnɯ-tɯpʰu}  	\ipa{ɣɤʑu}  	\ipa{kʰi}  \\
  but elephant \textsc{dem} two-sort exist:\textsc{sens} \textsc{hearsay} \\
\glt But there are two species of elephants, they say.
\end{exe} 

%hesitation:
%  tɕe nɯreri nɯ-kha tɯtɯrca /taβzunɯ/ to-βzu-nɯ khi ma ɤj mɯ-lɤ-ari-a
The distinction between \textit{source} and \textit{access}  proposed by \citet{tournadre14evidentiality} is thus most relevant to describe the evidential system of Japhug, and probably of other Gyalrongic languages. Verb morphology encodes \textit{access} to information, namely whether knowledge of the facts described by the sentence come from personal Sensory experience or general knowledge, while the particle \ipa{kʰi} indicates the \textit{source}: whether the information comes from the speaker himself, or mediated through verbal or written communication.

Unlike verbal evidentials, the hearsay particle is not obligatory; it may allows the speaker to defer responsibility of the truthfulness of the assertion,  though it does not necessary carry dubitative overtone.

\section{Reported speech and egophoricity}  
Early description of egophoric systems have referred to these as `conjunct-disjunct'  (see \citet{hale80conjunct} and \citet{delancey90erg} in particular). In this section, following insights from   \citet{tournadre08conjunct}, I will show why this terminology is inappropriate for describing the evidential system of Japhug.

 Table \ref{tab:conjunct} summarizes the conjunct / disjunct model as developed for Newar and Lhasa Tibetan. 

\begin{table}[H]
\caption{The conjunct / disjunct model} \label{tab:conjunct} \centering
\begin{tabular}{lllllll}
\toprule
& Conjunct & Disjunct \\
\midrule
Affirmative & 1st person & 2/3nd person \\
Interrogative & 2st person & 1/3nd person \\
Complement clause & same subject & distinct subject \\
\bottomrule
\end{tabular}
\end{table}
  
  The conjunct form (in Tibetan for instance the copula \ipa{yin}) is used in affirmative sentences with first person subject, in interrogative sentences with second person subject, and in complement clauses with a verb of speech when the subject of the complement clause is coreferent with that of the main clause, as in example \ref{ex:yin}, taken from (\citealt[295]{delancey90erg}).

\begin{exe}
\ex \label{ex:yin}
\gll   \ipa{kho-s} 	\ipa{kho} 	\ipa{bod=pa} 	\ipa{yin} 	\ipa{zer}-\ipa{gyis} \\
He-\textsc{erg} he Tibetan be:\textsc{conjunct}  say-\textsc{impf/disjunct} \\
\glt “He_i says that he_i is a Tibetan”  
   \end{exe}

The \textit{disjunct} form (in Tibetan, the copula \ipa{red}) appears in all other situations, in particular when the subject of the complement clauses is not coreferent with that of the main clause, as in \ref{ex:red}.
\begin{exe}
\ex \label{ex:red}
\gll \ipa{kho-s} 	\ipa{kho} 	\ipa{bod=pa} 	\ipa{red} 	\ipa{zer}-\ipa{gyis} \\
He-\textsc{erg} he Tibetan be:\textsc{disjunct}  say-\textsc{impf/disjunct}\\
\glt “He_i says that he_j is a Tibetan”
   \end{exe}
   
The term `conjunct / disjunct' itself was chosen by  \citet{hale80conjunct} specifically to refer to  this particular contrast in complement clauses. The term conjunct / disjunct describes the use of these markers in syntactic terms (person indexation and coreference). Languages such as Japhug, where both egophoric markers (corresponding to `conjunct' in the Hale/DeLancey terminology) and person indexation are found, represent an ideal test ground for the analysis of these markers in a typological perspective.


In present tense contexts, where there is a contrast between egophoric and other categories in Japhug, it is true that the use of egophoric vs Sensory is affected by person. The prototypical situation is indeed that the egophoric forms appears in affirmative sentences mainly with first person, and in interrogatives with second person (as a consequence of the rule of anticipation, see section \ref{sec:anticipation}), while the Sensory appears most commonly in all other situations. Yet, this distribution is only a tendency, and is by no means definitory of egophoricity.
 
As we have seen in section \ref{sec:egoph}, the egophoric marker is also compatible with third person referents in both affirmative and interrogative sentences (see for instance example \ref{ex:kume} and \ref{ex:WkudAn}), and the Sensory form is also used with first person referents in affirmative sentences in specific contexts.

In complement clauses with a verb of speech, the egophoric marker can appear  when the subject of the complement clause is coreferent with that of the main clause, as in sentence \ref{ex:kupea2}. 

\begin{exe}
\ex \label{ex:kupea2}
\gll \ipa{ku-pe-a} \ipa{ɲɯ-ti} \\
  \textsc{prs:egoph}-be.good-\textsc{1sg} \textsc{sens}-say \\ 
\glt  He says `I am fine'.' = `He_i says that he_i is fine.'%revérifier
\end{exe}

However, the function of the egophoric \ipa{ku--} prefix here is not to \textit{indicate} coreference between the two subjects, as the term `conjunct' would suggest. Pure indirect speech is almost non-attested in Japhug (except in texts translated from Chinese calquing the indirect speech in the original), and the egophoric is used here simply because it is the verbal form found in the original sentence. Since first person subjects have a tendency of being marked with the egophoric in specific contexts in assertive sentences, a speaker reporting his own words would use the egophoric (together with obligatory first person marking) in the complement clause, but the \textit{correlation} between the semantic identity between the subject of the complement clause and that of the main clause and the use of egophoric is not a \textit{causation}. 

In fact,  this correlation is quite weak in Japhug, and we find plenty of counterexamples. First, as illustrated by example \ref{ex:kupenW}  (from \ref{ex:kusciti} above), the egophoric can be used in reported speech to refer to third person (in a context explained in section \ref{sec:egoph}).

\begin{exe}
\ex \label{ex:kupenW}
\gll \ipa{ʁjoʁ} 	\ipa{ra} 	\ipa{ri} 	\ipa{wuma} 	\ipa{ʑo} 	\ipa{ku-pe-nɯ} \ipa{to-ti} \\
 \textsc{prs:egoph}-be.kind servant \textsc{pl} also really \textsc{emph}   \textsc{prs:egoph}-be.good \textsc{ifr}-say \\ 
\glt `The servants are very nice (to me).' (The frog 2002, 122-4)
\end{exe}

Second, the Sensory or the Factual rather than the egophoric appear in the reported sentence with the first person if the original speech used these forms, as for instance \ref{ex:YWCpaRa} with the Sensory form used to describe an endopathic sensation.

\begin{exe}
\ex \label{ex:YWCpaRa}
\gll 
\ipa{χsɤr} 	\ipa{kʰɯtsa} 	\ipa{ɯ-ŋgɯ} 	\ipa{nɯ} 	\ipa{tɕu} 	\ipa{a-tɯ-ci} 	\ipa{ci} 	\ipa{tɤ-rke} 	\ipa{ma} 	\ipa{wuma} 	\ipa{ɲɯ-ɕpaʁ-a} 	\ipa{to-ti.} \\
gold bowl \textsc{3sg}-inside \textsc{dem} \textsc{loc} 1\textsc{sg.poss-indef.poss}-water \textsc{indef} \textsc{imp}-put.in[III] \textsc{lnk} really \textsc{sens}-be.thirsty-\textsc{1sg} \textsc{ifr}-say \\
\glt He said `Pour me some water in the golden bowl, I am very thirsty.'
\end{exe}

Thus, in Japhug, the simple rule that the verbal form in reported speech always correspond to that in the original sentence explains the distribution of Egophoric vs non-egophoric forms without any need to refer to coreference or absence thereof between the subjects of the complement and the main clauses.

The reason why the same analysis cannot be straightforwardly applied to Tibetan is that a sentence such as \ref{ex:yin2} cannot be direct speech, since the pronoun  \ipa{kho} `he' represents the point of the of the current speaker, the original sentence being \ref{ex:yin3}.

\begin{exe}
\ex \label{ex:yin2}
\gll   \ipa{kho-s} 	\ipa{kho} 	\ipa{bod=pa} 	\ipa{yin} 	\ipa{zer}-\ipa{gyis} \\
He-\textsc{erg} he Tibetan be:\textsc{conjunct}  say-\textsc{impf/disjunct} \\
\glt “He_i says that he_i is a Tibetan.”  
   \end{exe}

\begin{exe}
\ex \label{ex:yin3}
\gll     	\ipa{nga} 	\ipa{bod=pa} 	\ipa{yin} 	  \\
I Tibetan be:\textsc{conjunct}    \\
\glt I am a Tibetan.
   \end{exe}
   
However,  \citet{tournadre08conjunct} argues that Tibetic languages present in fact Hybrid Indirect Speech (also called Semi-Indirect Speech, \citealt{aikhenvald08semidirect}), whereby the predicate (\ipa{yin} in example \ref{ex:yin2}) preserves the form used by the original speaker, while the pronouns are shifted to the point of view of the current speaker (\ipa{ŋanga} \textsc{1sg} changed to \ipa{kho} \textsc{3sg}). In this view, the distribution of \ipa{yin} vs \ipa{red} in complement clauses has nothing to do with a syntactic coreference constraint; it is a by-product of Semi-Indirect Speech.

Overt pronouns are very rare in Japhug (like in Kiranti languages, see \citealt{bickel01deictic}), as grammatical relations and person are mainly indicated by verb morphology, and the presence of Semi-Indirect Speech is not as immediately obvious as it is in Tibetan. Yet, the Japhug corpus\footnote{Eliciting Semi-Indirect Speech is possible, but the value of those data is questionable and  it is not used in this paper.} reveals several cases of discrepancy between pronouns and person markers on the verb in reported speech.\footnote{Some examples of Semi-Indirect Speech involve the verb \ipa{sɯso} `think'. Although  \ipa{sɯso} `think' is a verb of cognition rather than a verb of speech, the complement clauses translate into words what the current speaker thinks that the original speaker is thinking, so it is not essentially distinct from reported speech in the proper sense.}


In example \ref{ex:nWGi.kAsWso}, the verb \ipa{nɯɣi} `he comes/will come back (home)' in the complement clause of the verb \ipa{kɤ-sɯso} `think' is in the Factual third person singular form. In the same clause, we find however the second person singular pronoun \ipa{nɤʑo} `you_s'; there is no pause between the pronoun and the verb, and no indication from the prosody that \ipa{nɤʑo} is left-dislocated. 

This type of mismatch  between pronouns and indexation on the verb is anomalous and never found in independent sentences. Here the verb form corresponds to the point of view of the original speaker (indicated in blue in all following examples), whose original sentence would have been \ipa{ɯʑo nɯɣi} \textsc{2sg} {come.back:\textsc{fact}} `he is coming back'. The pronoun reflects the point of view of the current speaker (in red), for whom the equivalent sentence would be converted to \ipa{nɤʑo tɯ-nɯɣi} \textsc{2sg} {2-come.back:\textsc{fact}} `you are coming back', since the addressee of the current situation corresponds to the S of the original situation.

Three distinct translations are proposed here: a direct speech translation, reproducing the words pronounced by the original speaker, an indirect speech translation, and an attempt at representing Japhug Semi-Indirect Speech in English.


\begin{exe}
\ex \label{ex:nWGi.kAsWso}
\gll 
\ipa{nɤ-wa}  	\ipa{kɯ}  	[\rouge{\ipa{nɤʑo}} 	\bleu{\ipa{nɯɣi}}]  	\ipa{kɤ-sɯso}  	\ipa{kɯ}  	\ipa{kʰa}  	\ipa{ɯ-rkɯ}  	\ipa{ʁmaʁ}  	\ipa{χsɯ-tɤxɯr}  	\ipa{pa-sɯ-lɤt}  	\ipa{ɕti}  	\ipa{tɕe}  \\
\textsc{2sg.poss}-father \textsc{erg} \textsc{2sg} {come.back:\textsc{fact}}  \textsc{inf}-think \textsc{erg} house \textsc{3sg.poss}-side soldier three-circle \textsc{pfv:3$\rightarrow$3'-caus}-throw be.\textsc{affirm}:\textsc{fact} \textsc{lnk}\\
\glt \textbf{Direct}: Your father, thinking `\bleu{He is coming back}',   put three circles of soldiers around the house. 
\glt  \textbf{Indirect}: Your father, thinking that \rouge{you are coming back},
\glt  \textbf{Semi-Direct}: Your father, thinking that `\rouge{you}' \bleu{is coming back}, 
\end{exe}
   
Example \ref{ex:tunAmea} and \ref{ex:juGWta} illustrate the case of possessive prefixes on nouns, which undergo the same shift towards the point of the view of the current speaker, while the verb remains in the same form that was either thought or uttered by the original speaker.

\begin{exe}
\ex \label{ex:tunAmea}
\gll  \ipa{tɕe}  	\ipa{ta-ʁi}  	\ipa{nɯ}  	\ipa{kɯ}  	[\rouge{\ipa{ɯ-pi}}  	\ipa{ɣɯ}  	\ipa{ɯ-sci}  	\bleu{\ipa{tu-nɤme-a}}  	\ipa{ra}] 	\ipa{ɲɤ-sɯso}  	\ipa{tɕe,}  	\\
\textsc{lnk}  \textsc{indef.poss}-younger.sibling \textsc{dem} \textsc{erg}  {\textsc{3sg.poss}-elder.sibling}  \textsc{gen} \textsc{3sg.poss}-revenge {\textsc{ipfv}-make[III]-\textsc{1sg}} have.to:\textsc{fact} \textsc{ifr}-think \textsc{lnk} \\
\glt  \textbf{Direct}: The (younger) sister thought ``\bleu{I have to get revenge} on {my brother}".
\glt  \textbf{Indirect}:  The (younger) sister$_i$ \rouge{wanted to get revenge on her$_i$ brother}.
\glt  \textbf{Semi-direct}:  The (younger) sister$_i$ thought \bleu{I_i have to get revenge} on \rouge{her$_i$ brother}".
  \end{exe}
  
The original sentence corresponding to the complement clause in \ref{ex:tunAmea} is presented in \ref{ex:tunAmea2}: the possessive pronoun was first person (coreferent with the A of the main verb) and undergoes a shift to third person in \ref{ex:tunAmea} (representing the point of view of the person telling the tale).
  \begin{exe}
\ex \label{ex:tunAmea2}
\gll \bleu{\ipa{a-pi}}  	\ipa{ɣɯ}  	\ipa{ɯ-sci}  	\bleu{\ipa{tu-nɤme-a}}  	\ipa{ra}	\\
 {\textsc{1sg.poss}-elder.sibling}  \textsc{gen} \textsc{3sg.poss}-revenge {\textsc{ipfv}-make[III]-\textsc{1sg}} have.to:\textsc{fact}  \\
\glt  I have to get revenge on my brother.
  \end{exe}
  
  Example \ref{ex:juGWta} illustrates the same phenomenon as \ref{ex:tunAmea}, but with the verb of speech \ipa{ti} `say' instead of \ipa{sɯso} `think'.
 
\begin{exe}
\ex \label{ex:juGWta}
\gll   \ipa{tɤɕime}  	\ipa{nɯ}  	\ipa{kɯ}  	\ipa{pjɯ-tɯ-mtshɤm}  	\ipa{tɕe,}  	[\ipa{nɯnɯ}  \rouge{\ipa{ɯ-kɯmtɕhɯ}}  	\ipa{nɯ}  	\bleu{\ipa{ju-ɣɯt-a}}  	\ipa{ŋu}]  		\ipa{ɯ-kɯ-ti}  	\ipa{pjɤ-tu}  	\ipa{ndɤre,}  \\
girl \textsc{dem} \textsc{erg} \textsc{ipfv-conv:imm}-hear \textsc{lnk} \textsc{dem} {\textsc{3sg.poss}-toy} \textsc{dem} {\textsc{ipfv}-bring-\textsc{1sg}}  be:\textsc{fact} \textsc{3sg-nmlz}:S/A-say \textsc{evd.ipfv}-exist \textsc{lnk} \\
\glt   \textbf{Direct}: As soon as the girl heard that there was someone saying ``\bleu{I will bring your toy}".
\glt   \textbf{Indirect}:  As soon as the girl heard that there was someone saying that \rouge{he would bring her toy}.
\glt   \textbf{Semi-direct}: As soon as the girl$_i$ heard that there was someone saying ``\bleu{I will bring} \rouge{her$_i$ toy}".
  \end{exe}
  
Semi-Indirect Speech in Japhug is only detectable when the sentence contains pronouns or possessive prefixes, and in addition when the shift of perspective from the original speaker to the current speaker entails a change of person. The latter is rare. In most examples there is no shift since the referent is third person for both the current and the original speaker, and Semi-Indirect Speech is not distinguishable from Direct Speech.

%Examples such as \ref{ex:nWGi.kAsWso}, \ref{ex:tunAmea} and \ref{ex:juGWta} cannot be accounted for 
Coming back to Tibetan, in view of Semi-Indirect Speech in Japhug, Tournadre's analysis makes perfect sense. In example \ref{ex:red2}, the original speaker used the non-egophoric copula \ipa{red} to refer to a person different than himself. The third person pronoun \ipa{kho} `he' is shifted to the point of view of the current speaker, but in this case the referent in question is third person for both the current and the original speaker, so that the shift is not visible (it remains third person).

\begin{exe}
\ex \label{ex:red2}
\gll   \ipa{kho-s} 	[\rouge{\ipa{kho}} 	\ipa{bod=pa} 	\bleu{\ipa{red}}] 	\ipa{zer}-\ipa{gyis} \\
He-\textsc{erg} he Tibetan be:\textsc{n.egoph}  say-\textsc{ipfv:n.egoph} \\
\glt “He_i says that he_i is a Tibetan”  
   \end{exe}
 
 
In \ref{ex:yin4} however, the original speaker refers to himself and uses the Egophoric copula, and this form is kept in the reported sentence. The pronoun however undergoes shift of perspective to third person (just like the possessive prefixes in examples \ref{ex:tunAmea} and \ref{ex:juGWta} in Japhug).

\begin{exe}
\ex \label{ex:yin4}
\gll   \ipa{kho-s} 	\rouge{\ipa{kho}} 	\ipa{bod=pa} 	\bleu{\ipa{yin}} 	\ipa{zer}-\ipa{gyis} \\
He-\textsc{erg} he Tibetan be:\textsc{egoph}  say-\textsc{ipfv:n.egoph} \\
\glt \textbf{Direct}: He says `\bleu{I am Tibetan}'.
\glt \textbf{Indirect}: He_i says that \rouge{he_i is a Tibetan}.
\glt  \textbf{Semi-direct}:  He says `\rouge{he} \bleu{am} Tibetan.'
   \end{exe}

Coreference or absence thereof between the subject of the main clause and that of the complement clause in reported speech is therefore only a secondary effect of Egophoric markers derived from the way reported speech works in languages of the Tibetan cultural area, not a syntactic property definitory of egophoricity. 
 
\section{Conclusion}
This paper has three main contributions. First, it provides the first complete account of evidentiality in Japhug, on the basis of non-elicited data. Second, it illustrates how Egophoric markers behave in a language with obligatory person indexation, and confirms that Egophoric markers are completely distinct from person markers. Third, it describes Semi-Indirect Speech in Japhug and shows how a proper understanding of reported speech in languages of the Tibetosphere allows us to better understand the phenomena previously refer to as `conjunct / disjunct' in languages such as Tibetan.

\bibliographystyle{unified}
\bibliography{bibliogj}

\end{document}