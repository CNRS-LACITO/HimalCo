
\documentclass[oldfontcommands,oneside,a4paper,11pt]{article} 
\usepackage{fontspec}
\usepackage{natbib}
\usepackage{booktabs}
\usepackage{xltxtra} 
\usepackage{longtable}
\usepackage{polyglossia} 
\usepackage[table]{xcolor}
\usepackage{gb4e} 
\usepackage{multicol}
\usepackage{graphicx}
\usepackage{float}
\usepackage{hyperref} 
\hypersetup{bookmarks=false,bookmarksnumbered,bookmarksopenlevel=5,bookmarksdepth=5,xetex,colorlinks=true,linkcolor=blue,citecolor=blue}
\usepackage[all]{hypcap}
\usepackage{memhfixc}
\usepackage{lscape}
 \usepackage{lineno}
\bibpunct[: ]{(}{)}{,}{a}{}{,}

\setmainfont[Mapping=tex-text,Numbers=OldStyle,Ligatures=Common]{Charis SIL} 
\newfontfamily\phon[Mapping=tex-text,Ligatures=Common,Scale=MatchLowercase,FakeSlant=0.3]{Charis SIL} 
\newcommand{\ipa}[1]{{\phon \mbox{#1}}} %API tjs en italique
\newcommand{\ipab}[1]{{\scriptsize \phon#1}} 

\newcommand{\grise}[1]{\cellcolor{lightgray}\textbf{#1}}
\newfontfamily\cn[Mapping=tex-text,Ligatures=Common,Scale=MatchUppercase]{MingLiU}%pour le chinois
\newcommand{\zh}[1]{{\cn #1}}



\XeTeXlinebreaklocale 'zh' %使用中文换行
\XeTeXlinebreakskip = 0pt plus 1pt %
 %CIRCG
 


\begin{document} 
\linenumbers
\title{Is there a class of adjectives in Japhug Rgyalrong ? }
\author{Guillaume Jacques}
\maketitle
%\linenumbers

 
\section{Introduction}

\citet{dixon04adjective}
\citet{dixaikhenvald04adjectives}


Criteria for distinguishing `verb-like' adjectives from verbs (\citealt[15-22]{dixon04adjective}

\begin{enumerate}
\item Different morphological or syntactic properties when used as a predicate
\item Transitivity and voice alternations
\item Different use as a modifier within a NP
\item Comparative constructions
\item Forming adverbs
\end{enumerate}

All criteria, except the last one are examined; moreover, we treat  reduplication patterns in a separate section.

Not assume the existence of an adjectival class: \textit{property verbs} and \textit{property nouns}, whose meaning correspond to adjective classes in European languages, and belong to the   semantic categories in \citet[3-5]{dixon04adjective}. Heuristic before determining whether a morphosyntactic category of adjectives exist. 

\section{Use as a predicate}
Property verbs in Japhug are a subclass of intransitive verbs, and present personal and TAM morphology (see \ref{ex:WkutWpe} and \ref{ex:kupea}). 

\begin{exe}
\ex \label{ex:WkutWpe}
\gll \ipa{ɯ-kú-tɯ-pe?} \\
\textsc{intrg-pres}-2-be.good \\
\glt How are you doing?
\ex \label{ex:kupea}
\gll \ipa{ku-pe-a} \\
\textsc{pres}-be.good-\textsc{1sg}   \\
\glt I am fine. (Common greetings, heard in situation)
\end{exe}

Alternatively, it is possible to nominalize   property verbs with the prefix \ipa{kɯ--} and use the copula \ipa{ŋu} `be' as in \ref{ex:kAmtCoR}, but this is a very rare and marked construction.

\begin{exe}
\ex \label{ex:kAmtCoR}
\gll
\ipa{tɕe} 	\ipa{ɯ-ku} 	\ipa{kɯ-ɤmtɕoʁ} 	\ipa{ɲɯ-ŋu} 	\ipa{tɕe,} 	\ipa{tɯ-taʁ} 	\ipa{ku-otsa} 	\ipa{tɕe} 	\ipa{ɲɯ-ɕɯ-mŋɤm.} \\
\textsc{lnk} 3sg.poss-head \textsc{nmlz:S}-be.pointy \textsc{const}-be \textsc{lnk} \textsc{genr}-on \textsc{ipfv}-be.planted \textsc{lnk} \textsc{const-caus}-hurt \\
\glt The top (of its leaves) is pointy, and if they get planted on people, it hurts.
(Fir, 26)
\end{exe}

Depending on the TAM category in which they appear, finite property verbs have either an inchoative or purely stative  meaning. While they differ from dynamic verbs in this regard, this feature, as we will see, is not sufficient to distinguish  them from other stative verbs.

\subsection{Stative} \label{sec:stative}
There are five TAM categories used with property verbs with purely stative meaning: the non-past (unprefixed), the present \ipa{ku--} (illustrated by examples 
 \ref{ex:WkutWpe} and \ref{ex:kupea}), the constative \ipa{ɲɯ--}, the past imperfective \ipa{pɯ--} and the past evidential \ipa{pjɤ--}.

The first three can occur with all regular verbs, and are not treated in the present paper - they cannot be used to determine whether property verbs in Japhug form an identifiable part of speech.

The past imperfective \ipa{pɯ--} and the past evidential \ipa{pjɤ--} on the other hand, have a more restricted distribution. As explained in section \ref{sec:infinitive}, Japhug verbs can be divided into two classes depending on how their infinitive is formed: stative and dynamic verbs. Past imperfective and evidential forms are compatible with all stative verbs, as in \ref{ex:pWxtCi}. This category is not restricted to property verbs, and includes copulas such as \ipa{ŋu} `be', existential verbs such as \ipa{tu} `exist' and a series of auxiliaries such as \ipa{ra} `have to' (see section \ref{sec:infinitive} for more detail).

\begin{exe}
\ex \label{ex:pWxtCi}
\gll  \ipa{a-wa} 	\ipa{pɯ-xtɕi} \ipa{tɕe}	\ipa{wuma} \ipa{ʑo}	\ipa{pjɤ-qhi} 
 \\
\textsc{1sg.poss}-father \textsc{pst.ipfv}-be.small \textsc{lnk} really \textsc{emph} \textsc{evd.ipfv}-be.naughty  \\
\glt When my father was small, he was very naughty.  (Naughty, 1)
\end{exe}

With most dynamic verbs, however, and a different strategy is used: combining the imperfective form of the verb (see section \ref{sec:inchoative}, Table \ref{tab:directional}) with the copula in the past imperfective form \ipa{pɯ-ŋu} as in (see \citealt[261-7]{jacques08zh}, \citealt{lin11direction}).

\begin{exe}
\ex \label{ex:tundzanW}
\gll 
\ipa{ɯ-di} 	\ipa{mnɤm.} 	\ipa{kɯɕɯŋgɯ} 	\ipa{tɕe} 	\ipa{tu-ndza-nɯ} 	\ipa{pɯ-ŋu} 	\\
\textsc{3sg;poss}-smell \textsc{n.pst}:be.smelly long.ago \textsc{lnk} \textsc{ipfv}-eat-\textsc{pl} \textsc{pst.ipfv}-be \\
\glt (Origan) has a strong smell. In former time, people used to eat it. (Origan, 23-4)
\end{exe}

Directly prefixing \ipa{pɯ--} or \ipa{pjɤ--} to most dynamic verbs like \ipa{ndza} is possible only in two cases. First, in some types clause linking (alternative and scalar concessive conditionals as well as counterfactuals), as in \ref{ex:tundzea.apWNu}, where the dynamic verb \ipa{ngo} `get sick' normally takes the \textsc{up} prefix. Second, in combination with the progressive prefix \ipa{asɯ--} (see \citealt[263-7]{jacques08zh}).

     \begin{exe}
   \ex \label{ex:tundzea.apWNu}
   \gll
\ipa{smɤn}   	\ipa{ʑa} \ipa{tsa}   	\ipa{tu-ndze-a}   	\ipa{a-pɯ-ŋu}   	\ipa{tɕe}   	\ipa{mɯ-pɯ-ngo-a}   \\
medicine early  a.little \textsc{ipfv}-eat[III]-\textsc{1sg} \textsc{irr-ipfv}-be \textsc{lnk} \textsc{neg-pst.ipfv}-be.sick-\textsc{1sg} \\
\glt If I had taken my medicine earlier, I would not have gotten sick. (elicited)
\end{exe}

However, a minority of dynamic verbs in Japhug are fully compatible with the past imperfective \ipa{pɯ--} and the   evidential imperfective \ipa{pjɤ--}; it is in particular the case of tropative verbs (see \citealt{jacques13tropative} and section \ref{sec:tropative}).

Thus,   the use of the past imperfective and    evidential imperfective is clearly not a useful test to establish property verbs  as a distinct morphosyntactic subclass.

\subsection{Inchoative} \label{sec:inchoative}
With TAM categories other than those treated in section \ref{sec:stative},  property verbs become dynamic, and indicate a change of state. Example \ref{ex:chWwxti} illustrates this use with the imperfective and \ref{ex:thWwxti} with the perfective.

\begin{exe}
\ex \label{ex:chWwxti}
\gll \ipa{qandʐe} 	\ipa{nɯ} 	\ipa{xtɕi} 	\ipa{qhe} 	\ipa{kɯ-xtɕɯ-xtɕi} 	\ipa{ma} 	\ipa{maŋe,} 	\ipa{taqaβ} 	\ipa{jamar} 	\ipa{ma} 	\ipa{maŋe,} 	\ipa{tɕe} 	\ipa{ʑɯrɯʑɤri} 	\ipa{\textbf{chɯ-wxti}} 	\ipa{ɲɯ-ɕti.}  \\
 earthworm \textsc{top} \textsc{n.pst}:little \textsc{lnk} \textsc{nmlz:S-redp}-be.small apart.from not.exist:\textsc{sens} needle apart.from not.exist:\textsc{sens} \textsc{lnk} progressively \textsc{ipfv}-be.big \textsc{const}-be.\textsc{affirm} \\
\glt The earthworm is very little, not (bigger) than a needle, and it progressively grows bigger. (Earthworm 116)
\end{exe}

\begin{exe}
\ex \label{ex:thWwxti}
\gll
\ipa{tɕe} 	\ipa{zrɯɣ} 	\ipa{nɯ} 	\ipa{\textbf{thɯ-wxti}} 	\ipa{tsa} 	\ipa{tɕe} 	\ipa{chɯ-rɤpɯ} 	\ipa{tɕe} 	\ipa{ndʑrɯ} 	\ipa{ku-lɤt} 	\ipa{ɲɯ-ŋu.} \\
\textsc{lnk} louse \textsc{top} \textsc{pfv}-be.big a.little \textsc{lnk} \textsc{ipfv}-litter \textsc{lnk} nit \textsc{ipfv}-throw \textsc{const}-be \\
\glt After the louse have grown big, it lays eggs and makes nits. (Lice, 60)
\end{exe}

Since property verbs generally describe non-controllable states, imperative forms are generally impossible due to pragmatics, but irrealis forms  as in \ref{ex:atAzbaR} can be used to express a purpose or a desired outcome with any property verbs.

\begin{exe}
\ex \label{ex:atAzbaR}
\gll
\ipa{khɤxtu} 	\ipa{ɕɯ-nɤʁaʁ-tɕi} 	\ipa{tɕe,} 	\ipa{nɤ-kɤrme} 	\ipa{a-tɤ-zbaʁ} \\
rooftop \textsc{transloc-n.pst}:have.fun-\textsc{1du} \textsc{lnk} \textsc{2sg.poss}-hair \textsc{irr-pft-}be.dry \\
\glt Let us go rest on the rooftop, so that your hair can dry. (The frog, 286)
\end{exe}

In the case of controllable properties, the imperative can be attested. For instance, the verb \ipa{mbɣom} `be in a hurry' often occurs in the imperative as \ipa{tɤ-mbɣom} (\textsc{imp}-be.in.a.hurry) `hurry up'.

In all regular Japhug verbs,   TAM categories other than the five ones studied in section \ref{sec:stative}   comprise a directional prefix  (see table \ref{tab:directional}). Directional prefixes combine two dimensions, direction (six categories)and TAM (four categories), the former indicated by the rows in the table, and the latter by the columns. 

The prefixes mentioned in section \ref{sec:stative}, present \ipa{ku--},   constative \ipa{ɲɯ--},   past imperfective \ipa{pɯ--} and   past evidential \ipa{pjɤ--} are homophonous with (and probably grammaticalized from) those of the imperfective \textsc{east}, imperfective \textsc{west}, perfective \textsc{down} and evidential \textsc{down} respectively. As pointed out by \citet{lin11direction}, this grammaticalization from the direction \textsc{down}  to both imperfective and (for some verbs) imperfective is observed in all Rgyalrong languages. In the case of verbs whose lexical direction is \textsc{down}, the past perfective and imperfective are not distinguishable.


Motion verb and some concrete actions verb can be used with all six directions and can additionally have a  non specific motion' series. All other verbs, including property verbs,  have one or more lexically determined directional prefix among the six directions. Being intransitive verbs, property verbs only occur with the series A, B and D (series C is only found in some perfective forms of transitive verb and will not concern us here).

\begin{table}[H]
\caption{Directional prefixes in Japhug Rgyalrong} \label{tab:directional}
\resizebox{\columnwidth}{!}{
\begin{tabular}{llllll}
\toprule
   &  	perfective  (A) &  	imperfective  (B)  &  	perfective 3$\rightarrow$3' (C)  &  	evidential  (D) \\  	
   \midrule
\textsc{up}   &  	\ipa{tɤ--}   &  	\ipa{tu--}   &  	\ipa{ta--}   &  	\ipa{to--}   \\  	
\textsc{down}   &  	\ipa{pɯ--}   &  	\ipa{pjɯ--}   &  	\ipa{pa--}   &  	\ipa{pjɤ--}   \\  	
\textsc{upstream}   &  	\ipa{lɤ--}   &  	\ipa{lu--}   &  	\ipa{la--}   &  	\ipa{lo--}   \\  	
\textsc{downstream}   &  	\ipa{tʰɯ--}   &  	\ipa{cʰɯ--}   &  	\ipa{tʰa--}   &  	\ipa{cʰɤ--}   \\  	
\textsc{east}   &  	\ipa{kɤ--}   &  	\ipa{ku--}   &  	\ipa{ka--}   &  	\ipa{ko--}   \\  	
\textsc{west}   &  	\ipa{nɯ--}   &  	\ipa{ɲɯ--}   &  	\ipa{na--}   &  	\ipa{ɲɤ--}   \\  	
\bottomrule
\end{tabular}}
\end{table}
Most property verbs occur with the \textsc{up} and  \textsc{downstream} directions, but we do find examples of the other four directions.

\begin{enumerate}
\item \textsc{down}: \ipa{rkɯn} `be few, be rare', \ipa{rnaʁ} `deep', \ipa{spɯ} `dry (of food)' 
\item \textsc{upstream}: \ipa{rɲɟi} `be long' (the only example)
\item \textsc{east}: \ipa{mpɯ} `be soft', \ipa{ɴqʰi} `be dirty', \ipa{ɲɟa} `be too old (animal)', \ipa{qanɯ} `be dark'
\item \textsc{west}: \ipa{ɲaʁ} `be black', \ipa{kʰrɯ} `be dry and hard', \ipa{ɣɤβlo} `be slow'
\end{enumerate}
In some cases, the choice of a particular directional appears to have an iconic character linked to the property in question. In particular the fact  that    property verbs meaning `be few' and `be deep'  have the direction \textsc{down} is clearly linked to their semantics.

Yet, the lexical directional prefix category cannot be predicted from the meaning of the verb. Thus, while \ipa{rɲji} and \ipa{zri} both mean `be long' and are semantically completely equivalent, the former occurs with the \textsc{upstream} prefixes, while the latter has either the \textsc{up} or \textsc{downstream} prefixes.


Property verbs that have \textsc{up} or \textsc{west} as their lexical directions allow the use of the \textsc{downstream} direction to express a progressive change of state, as in examples \ref{ex:chWwxti} and \ref{ex:thWwxti}.  Thus, while the perfective form \ipa{tɤ-wxti} (\textsc{pfv}-be.big) `it/he grew big' with the \textsc{up} \ipa{tɤ--} direction implies a change of state without intermediate stages,  \ipa{tʰɯ-wxti} (\textsc{pfv}-be.big) `it/he grew big' with the \textsc{downstream} \ipa{tʰɯ--} direction conveys a nuance of progressive change, and this form is commonly used with the adverb \ipa{ʑɯrɯʑɤri} `progressively'

 This special use of the downstream prefixes is restricted to property verbs. The other stative verbs, such as the copula \ipa{ŋu}, are not compatible with the \textsc{downstream} prefixes, and no dynamic verbs shows the same semantic nuance.
 
This alternation is thus a defining property of a subset of property verbs; it cannot however be used to define property verbs as a subclass, as it would exclude those which are not compatible with these particular series of directional prefixes.
\subsection{Adverbial modifiers}
In some languages, adverbial modifiers whose meaning correspond to English `very' provide a useful test to distinguish adjectives from nouns or verbs (see \citealt[27]{dixon04adjective}).

Japhug we find modifiers derived from various part of speech, some purely adjectival like \ipa{khro} `a lot, much'or \ipa{stu} `most', others from nouns like \ipa{wuma} `real, really, very' and other from nominalized verbs like  \ipa{kɯ-xtɕɯ-xtɕi} `a little' or \ipa{kɯ-rtaʁ} `enough'. There are no restrictions on the use of these modifiers: although they  commonly occur with property verbs, they are also found with dynamic verbs of all types, as in examples  \ref{ex:wuma.dyn}, \ref{ex:khro.dyn} and \ref{ex:stu.dyn}.

\begin{exe}
\ex \label{ex:wuma.dyn}
\gll
\ipa{xɕɤj} 	\ipa{nɯ} 	\ipa{wuma} 	\ipa{ʑo} 	\ipa{ɲɯ-ndze} 	\ipa{ri,} \\
grass \textsc{top} really \textsc{emph} \textsc{const}-eat[III] \textsc{lnk} \\
\glt (The rabbit) eats grass a lot,   (Rabbits, 14)
\end{exe}
\begin{exe}
\ex \label{ex:khro.dyn}
\gll
\ipa{tɕe} 	\ipa{nɯ} 	\ipa{kɯ-βʁa} 	\ipa{ɣɤʑu,} 	\ipa{kɯ-nŋo} 	\ipa{ɣɤʑu} 	\ipa{qhe,} 	\ipa{kɯ-βʁa} 	\ipa{nɯ} 	\ipa{kɯ} 	\ipa{khro} 	\ipa{tu-nɯ-ndze} 	\ipa{ɲɯ-ŋu.} 	\\
\textsc{lnk} \textsc{dem} \textsc{nmlz}:S/A-prevail exist:\textsc{sens} \textsc{nmlz}:S/A-be.beaten exist:\textsc{sens} \textsc{lnk} \textsc{nmlz}:S/A-prevail \textsc{top} \textsc{erg} much \textsc{ipfv-auto}-eat[III] \textsc{const}-be \\
\glt (Among the tigers), some are dominant and some are dominated. The dominant ones eat the largest part. (Lion, 60)
\end{exe}
\begin{exe}
\ex \label{ex:stu.dyn}
\gll
\ipa{tɕe} 	\ipa{nɯnɯ} 	\ipa{stu} 	\ipa{ʑo} 	\ipa{ɯ-kɤ-nɯzdɯɣ} 	\ipa{nɯ} \\
\textsc{lnk} \textsc{dem} most \textsc{emph} \textsc{3sg-nmlz}:P-worry.about \textsc{top} \\
\glt What it is most worried about, (Marmot, 45)
\end{exe}



\section{Transitivity and voice}


Japhug is a polysynthetic language with polypersonal agreement (\citealt{jacques10inverse}), and thus presents quite distinct transitive and intransitive conjugations.

Property verbs all follow the intransitive conjugation, and among the intransitive verbs belong to the subclass of \textit{stative} verbs. In this section, we first present the formation of the infinitive, which provides a morphological criterium for distinguishing between dynamic and stative intransitive verbs. Then, we show that some property verbs allow oblique arguments and complements. Third, we discuss several voice derivations, in particular causative facilitative and tropative, which show particular forms for transitive, intransitive dynamic and intransitive stative verbs.


\subsection{Infinitive} \label{sec:infinitive}
Japhug verbs have a   four distinct infinitives: \ipa{kɤ--}, \ipa{kɯ--}, \ipa{tɯ--} and bare infitinitives The latter two, which are restricted to very specific verb complements (see \citealt{jacques14antipassive}, are not treated here. The prefixes \ipa{kɤ--} and \ipa{kɯ--} are formally identical the P-nominalizer \ipa{kɤ--} and the S/A nominalizer \ipa{kɯ--}, but must be distinguished from them; thus, while intransitive dynamic verbs do not have P-nominalization in \ipa{kɤ--} (since they lack a P-argument), they still allow an infinitive in \ipa{kɤ--}.

The infinitives \ipa{kɤ--} and \ipa{kɯ--} differ from the other infinitives in that they constitute the citation forms of verbs. In addition, these infinitives are used in some complement clauses (\ref{ex:kACe.mWjkhW}) and in Manner clause linking (\ref{ex:mAkWmbrAt}, see \citealt{jacques14linking}).

\begin{exe}
\ex \label{ex:kACe.mWjkhW}
\gll
\ipa{tɯrme} 	\ipa{kɤ-ɕe} 	\ipa{mɯ́j-khɯ} 	\ipa{ma} 	\ipa{ɲɯ-ɣɤmdzu} 	\ipa{tɕe} \\
people \textsc{inf:dyn}-go \textsc{neg:const}-be.possible because \textsc{const}-be.thorny \textsc{lnk} \\
\glt People cannot go there, because it is full of thorns. (Thistle 89)
\end{exe}

\begin{exe}
\ex \label{ex:mAkWmbrAt}
\gll
\ipa{nɯ} 	\ipa{maka} 	\ipa{mɤ-kɯ-mbrɤt} 	\ipa{ʑo} 	\ipa{ɲɯ-rɤma} 	\ipa{ɲɯ-ɕti} 	\ipa{tɕe,} 	\\
\textsc{dem} at.all \textsc{neg-inf:stat/inan}-be.cut \textsc{emph} \textsc{ipfv}-work \textsc{const}-be:affirm \textsc{lnk} \\
\glt It (the bee) works without stop. (Bees, 63)
\end{exe}



In Tshobdun, a language closely related to Japhug, \citet[493]{jackson03caodeng} describes the distribution of \ipa{kɐ--} and \ipa{kə--}, the cognates of \ipa{kɤ--} and \ipa{kɯ--}, as follows: the former appears with dynamic verbs (in the case of intransitive dynamic verbs, only those which are compatible with a human S), while the latter is used with either stative verbs or intransitive verbs which cannot be used with a human S. 

In Japhug, the distribution between the two infinitives differs from Tshobdun in that in the case of dynamic intransitive verbs, we find the prefix \ipa{kɯ--} in the case of verbs allowing only inanimate or expletive S; thus, intransitive verbs that are only used with non-human animals (like \ipa{cɯ} `to hibernate') have the infinitive \ipa{kɤ--} in Japhug and the infinitive \ipa{kə--} in Tshobdun.

In the case of verbs whose stem begins in \ipa{a--}, the prefix \ipa{kɯ--} undergoes fusion with this vowel into [kɤ], so that the contrast between the two infinitives is not detectable (\citealt{jacques07passif}).

Although all property verbs whose stem does not begins in \ipa{a--} have an infinitive in \ipa{kɯ--}, the distinction between the \ipa{kɤ--} and \ipa{kɯ--} infinitives is   not specific enough to serve as a criterium to define a class of adjectives. 

Aside from property verbs, we find in this class other stative verbs such as the copulas (\ipa{ŋu} `be', \ipa{maʁ} `not be', \ipa{tu} `exist', \ipa{me} `not exist'), auxiliaries which do not allow animate S, cannot be conjugated in person form and whose semantic exclude the control of the S/A in the complement clause (\ipa{ra} `have to, need', \ipa{ŋgrɤl} `be usually the case', \ipa{ɬoʁ} `need', \ipa{zgɤt} `have to, be obliged to', \ipa{khɯ} `have to, be possible', \ipa{jɤɣ} `be possible, be authorized') and various verbs used with non-animate S (\ipa{mbrɤt} `be cut, be stopped', \ipa{ndʐi} `to melt' etc).
 

\subsection{Oblique arguments and complementation} \label{sec:complement}
Property verbs, being morphologically intransitive, can only index one argument. Yet, some property verbs do allow oblique arguments and complements.

%naχtɕɯɣ

Verbs expressing a property in relation to other referents, such as \ipa{sna} `be nice', \ipa{naχtɕɯɣ} `be similar, be identical', can be used with oblique arguments marked with  \ipa{--taʁ} (\ref{ex:WtaRpjAsna}) `one' or the comitative \ipa{cho}  (\ref{ex:choYWnaXtCWG}). The possibility to have such an oblique argument, and the postposition or relator noun with which it is marked, are lexically determined.

\begin{exe}
\ex \label{ex:WtaRpjAsna}
\gll 
 	\ipa{wuma} \ipa{ʑo} 	\ipa{rɟɤlpu} 	\ipa{ɯ-taʁ} 	\ipa{wuma} 	\ipa{ʑo} 	\ipa{pjɤ-sna,} 	\ipa{ʁjoʁ} 	\ipa{ra} 	\ipa{nɯ-taʁ} 	\ipa{tɕi} 	\ipa{wuma} 	\ipa{ʑo} \ipa{pjɤ-sna,} \\
 really \textsc{emph} king \textsc{3sg}-on  really \textsc{emph}  \textsc{ipfv.evd}-be.nice servant \textsc{pl} \textsc{3pl}-on also  	\ipa{wuma} \ipa{ʑo} \textsc{ipfv.evd}-be.nice servant \\
\glt She was very nice to the king and to the servants.  (The frog, 110)
\end{exe}

\begin{exe}
\ex \label{ex:choYWnaXtCWG}
\gll 
 	\ipa{qaʑo} 	\ipa{ɯ-ɣli} 	\ipa{nɯ} 	\ipa{li} 	\ipa{tshɤt} 	\ipa{ɣɯ} 	\ipa{ɯ-ɣli} 	\ipa{cho} 	\ipa{naχtɕɯɣ} 	\ipa{ʑo} \ipa{tɕe}\\
 	sheep \textsc{3sg.poss}-manure \textsc{top} again goat \textsc{gen}  \textsc{3sg.poss}-manure \textsc{comit} \textsc{n.pst}-be.similar \textsc{emph} \textsc{lnk}\\
 	\glt Sheep manure is similar to goat manure  	(Sheep, 100)
\end{exe}

In addition, property verbs allow finite complements (\ref{ex:pjWnArtekWrtaR}) or infinitival ones (\ref{ex:kWfsekWjpum}) in equative comparative constructions. Note that the embedding is recursive as in \ref{ex:pjWnArtekWrtaR} where the nominalized property verb \ipa{kɯ-rtaʁ} `enough' and its complement are embedded within the complement of the verb \ipa{kɯ-wxti} `big'. 

\begin{exe}
\ex \label{ex:pjWnArtekWrtaR}
\gll 
[[\ipa{koŋla} 	\ipa{pjɯ-nɤrte} 	\ipa{ʑo}]	\ipa{kɯ-rtaʁ} 	\ipa{ʑo}] 	\ipa{kɯ-wxti} 	\ipa{ɲɯ-βze} 	\ipa{ŋgrɤl} \\
completely \textsc{ipfv}-wear.as.hat \textsc{emph} \textsc{nmlz}:S/A-be.enough \textsc{emph} \textsc{nmlz}:S/A-be.big  \textsc{ipfv}-grow[III] \textsc{n.pst}:be.usually.the.case \\
\glt (The burdock's leaf) grows big enough to be worn as a hat. (Burdock, 35)

\end{exe}
\begin{exe}
\ex \label{ex:kWfsekWjpum}
\gll 
[\ipa{tɯrme} 	\ipa{laʁnɯlaχsɯm} 	\ipa{kɯnɤ} 	\ipa{mɤ-kɤ-sɯ-rqoʁ} 	\ipa{kɯ-fse}] 	\ipa{kɯ-jpum} 	\ipa{ɲɯ-βze} 	\ipa{cha} \\
people two.or.three also \textsc{neg-inf-habilitative}-hug \textsc{inf:stat}-be.like \textsc{nmlz}:S/A-be.thick \textsc{ipfv}-grow[III] n.pst:can \\
\glt It can grow so thick that two or three people (together) cannot hug it. (Firs, 6)
\end{exe}

This type of complements appear to be possible for all property verbs and may be one of their defining characteristics.

%aʑo sɤz χsɯxpa xtɕi,

\subsection{Voice}

Several voice derivations in Japhgu are sensitive to transitivivity and some are specific to subclasses of intransitive verbs. In this section, we focus on three derivations: causative, facilitative and tropative, and show that these offer good criteria to define a class of adjectives.


\subsubsection{Causative}
Japhug has a very rich array of causative prefixes. The regular causatives are formed by adding of the three regular allomorphs \ipa{sɯ--}, \ipa{sɯɣ--} and \ipa{z--} to the verbal stem. 

The allomorph \ipa{z--}  appears in verb stems of more than one syllable whose first syllable consists of a sonorant and a vowel as in \ipa{nɯmbjɯm} `to warm oneself by the fire (vi)' $\rightarrow$  \ipa{z-nɯmbjɯm} `to warm someone by the fire (vt)'. The allomorph \ipa{sɯɣ--} or \ipa{sɯx--} is restricted to monosyllabic intransitive verbs whose onset does not comprise a velar or uvular consonant, or a cluster other than the type /C+\{\ipa{--r--}, \ipa{--l--}, \ipa{--j--}, \ipa{--w--}, \ipa{--ɣ--} and \ipa{--ʁ--} \}/, for instance \ipa{ndzur} `stand' $\rightarrow$ \ipa{sɯɣ-ndzur} `erect, put vertically'. In all other cases, \ipa{sɯ--} is used. In addition, irregular causatives in \ipa{s--}, \ipa{ɕɯ--}, \ipa{ɕɯɣ--}, \ipa{ʑ--}, and \ipa{j--}, which are originally related to the \ipa{sɯ--} prefix, are also attested for a limited number of verbs. 

Some stative verbs, and among them exclusively property verbs, have causatives in \ipa{ɣɤ--}, as in Table \ref{tab:GA.caus}.

\begin{table}[h]
\caption{Examples of \ipa{ɣɤ--} causatives in Japhug} \label{tab:GA.caus} \centering
\begin{tabular}{lllll}
\toprule
Bas verb & Meaning & Causative & Meaning\\
\midrule
\ipa{mpja} & be warm & \ipa{ɣɤmpja} & warm up (vt)\\
\ipa{wxti} & be big & \ipa{ɣɤwxti} & make big, raise\\
\ipa{tɕhom} & be too much & \ipa{ɣɤtɕhom} & do too much (vt)\\
\ipa{tɕur} 	& be sour & \ipa{ɣɤtɕur} & make sour\\
\midrule
\ipa{khɯ} 	& be possible, be obedient, agree & \ipa{ɣɤkhɯ} & force to agree\\
\bottomrule
\end{tabular}
\end{table}
The only causative verb in \ipa{ɣɤ--} that could seem not to be derived from a property verb is \ipa{ɣɤkhɯ} `force to agree'. Yet, this is due to the fact taht there are two etymologically related, but synchronically distinct,  stative/inanimate verbs \ipa{khɯ}: one is the auxiliary with inanimate S meaning `be possible', and the other one the stative property verb `be obedient'. The causative \ipa{ɣɤkhɯ} obviously derives from the latter. It is not possible to form causatives in \ipa{ɣɤ--} from copulas or auxiliaries.\footnote{The verb \ipa{ɣɤme} `destroy' could appear to be a counterexample, since synchronically it looks like the causative of \ipa{me} `not exist'. Here comparison with other Rgyalrong languages shows that these two etyma are not etymologically related: in Situ Rgyalrong, for instance, `not exist' is \ipa{mí}  while `destroy' is \ipa{wɑrmiɛ́k}  (\citealt[334, 343]{huangsun02}). Due to a phonological merger in Japhug, these unrelated words appear to be derived one from the other. }

In addition to causatives in \ipa{ɣɤ--}, property verbs also allow causatives in \ipa{sɯ--} in Japhug and Tshobdun. In Tshobdun, \citet{jackson06paisheng, jackson13morpho} reports a subtle semantic contrast between the two causative derivations: the Tshobdun \ipa{sə--} / \ipa{səɣ--} causative (cognate to Japhug \ipa{sɯ--} / \ipa{sɯɣ--} / \ipa{z--}) `denotes causation of an increase in the degree
of the predicated state'  while the \ipa{wɐ--} causative (cognate to Japhug \ipa{ɣɤ--}) `indicates causation of a changed state'. \citet{jackson13morpho} in particular contrasts the minimal pair \ipa{səɣ-cʰiʔ} `make something sweeter' with \ipa{wɐ-cʰiʔ} `make something sweet'.

In Japhug, while we do find many stative verbs with both a \ipa{sɯ--} and a \ipa{ɣɤ--} causative, we have not been able up to now to confirm whether the same semantic contrast is observed. Sentence \ref{ex:pjWsWxtCur} could appear to confirm  the fact that \ipa{sɯ--} causatives are used for increased degree. The interpretation of the perfective	\ipa{mɯ-tɤ-tɕur}  as `when it is not sour \textbf{enough}' and of  \ipa{pjɯ-sɯx-tɕur} `it makes it \textbf{more} sour' rather than simply `when it is not sour' and  `it makes it sour' respectively, though not apparent from the gloss, are clear in context and has been rechecked with our consultant.  \footnote{Note that the A of the verb \ipa{pjɯ-sɯx-tɕur} `it makes it (more) sour' here is necessarily the sour fruit, it cannot be the generic human, otherwise the inverse prefix should be present. }
\begin{exe}
\ex \label{ex:pjWsWxtCur}
\gll 
\ipa{tɕe}  	\ipa{tɤjko}  	\ipa{mɯ-tɤ-tɕur}  	\ipa{tɕe,}  	\ipa{ɴɢolo}  	\ipa{ɯ-mat}  	\ipa{nɯ}  	\ipa{ɲɯ́-wɣ-phɯt}  	\ipa{tɕe,}  	\ipa{tɕe}  	\ipa{tɤrca}  	\ipa{pjɯ́-wɣ-ɣɤla}  	\ipa{tɕe,}  	\ipa{tɕe}  	\ipa{tɤjko}  	\ipa{pjɯ-sɯx-tɕur}  	\ipa{cha.}  \\
\textsc{lnk} turnip \textsc{neg-pfv}-be.sour \textsc{lnk} plant.sp \textsc{3sg.poss}-fruit \textsc{top} \textsc{ipfv-inv}-take.off \textsc{lnk} \textsc{lnk} together \textsc{ipfv-inv}-soak  \textsc{lnk} \textsc{lnk} turnip \textsc{ipfv-caus}-be.sour \textsc{n.pst}:can \\
\glt When turnip tops are not sour (enough), one takes the fruit of the \ipa{ɴɢolo} plant, soaks them together, and it can make  the turnip tops (more) sour. (ɴɢolo 125)
\end{exe}

However, our consultant reports that the causative \ipa{ɣɤ-tɕur} can be substituted for 	\ipa{sɯx-tɕur}  in example \ref{ex:pjWsWxtCur} without changing the meaning. Example \ref{ex:YWsWxtCur} however, shows the same verb used with the meaning of change of state rather than increase of degree.\footnote{The directional prefix is \ipa{ɲɯ--} \textsc{west} in \ref{ex:YWsWxtCur} and \ipa{pjɯ--} \textsc{down} in \ref{ex:pjWsWxtCur}, but our consultant reports no semantic distinction between the two.}

\begin{exe}
\ex \label{ex:YWsWxtCur}
\gll 
\ipa{mtɕhi}  	\ipa{ɯ-mat}  	\ipa{rca}  	\ipa{ɯ-tɯ-tɕur}  	\ipa{saχaʁ.}  	\ipa{ɯ-tɯ-tɕur}  	\ipa{kɯ}  	\ipa{tɯ-kɯr}  	\ipa{ɯ-ŋgɯ}  	\ipa{lú-wɣ-rku}  	\ipa{qhe}  	\ipa{maka}  	\ipa{ɲɯ-sɯ-ɤmɯzɣɯt}  	\ipa{qhe,}  	\ipa{tɯ-phoŋbu}  	\ipa{ra}  	\ipa{kɯnɤ}  	\ipa{ɲɯ-sɯx-tɕur}  	\ipa{kɯ-fse}  	\ipa{ɕti}  \\
sea.buckthorn \textsc{3sg.poss}-fruit \textsc{top} \textsc{3sg-nmlz:degree}-be.sour n.pst:be.extremely \textsc{3sg-nmlz:degree}-be.sour \textsc{erg} \textsc{indef:poss}-mouth \textsc{3sg}-inside \textsc{ipfv:upstream-inv}-put.in \textsc{lnk} at.all \textsc{ipfv-caus}-be.evenly.distributed \textsc{lnk} \textsc{indef:poss}-body \textsc{pl} also \textsc{ipfv-caus}-be.sour \textsc{nmlz:S/A}-be.like \textsc{n.pst}:be:\textsc{affirm} \\
\glt The fruit of the sea-buckthorn is very sour, so sour that when one puts it in one's mouth, it makes it completely (sour), and it is as if one's (whole) body became sour. (Sea-buckthorn, 66)
\end{exe}

It remains unclear whether the semantic contrast is lost in Japhug or whether we failed to find an appropriate example to elicit it; no spontaneous minimal pair of a property verb with both type of causatives appears in our corpus.

Not all property verbs allow a causative in \ipa{ɣɤ--}. In particular, colour verbs such as  \ipa{ɲaʁ}  `be black', \ipa{arŋi} `be blue, be green' or \ipa{ɣɯrni} `be red' only have \ipa{sɯ--} causatives: \ipa{sɯɣ-ɲaʁ} `blacken', \ipa{sɯ-ɤrŋi} `make blue/green' and \ipa{z-ɣɯrni} respectively. In some cases, there is no obvious semantic reason why a particular verbs does not have a \ipa{ɣɤ--} causative: while most verbs referring to taste have \ipa{ɣɤ--} causative, \ipa{mɤrtsaβ} `be spicy' does not (its only causative is \ipa{z-mɤrtsaβ} `make spicy').

Thus, while the possibility of forming a causative in \ipa{ɣɤ--} can be used to define a subclass of property verbs, it cannot be used on its own to define the class of adjectives.

\subsubsection{Facilitative}
The facilitative derivation in Japhug generates stative verbs from either dynamic (transitive or intransitive) or stative verbs with the meaning `being easy to X' or  `easily become X'. There are two facilitative prefixes, \ipa{nɯɣɯ--} and \ipa{ɣɤ--} (\citealt[82]{jacques08zh}; a similar pair is described in Tshobdun, see \citealt{jackson13morpho}).

 The former \ipa{nɯɣɯ--} is used with transitive verbs, and the S of the resulting verb corresponds to the P of the base verb: \ipa{mto} `see' $\rightarrow$ \ipa{nɯɣɯ-mto} `be easy to see'. 

The latter  \ipa{ɣɤ--} occurs with intransitive verbs, and is homophonous with the causative \ipa{ɣɤ--}, so that in the case of some property verbs we have homophones such as \ipa{ɣɤ-wxti} `make bigger, raise up (vt)' and \ipa{ɣɤ-wxti} `which easily grows big (vstat)'. There is little ambiguity in context however, since the two verbs have distinct conjugations. The S of the facilitative verb  correspond to the S of the base verb. When the base verb is a property verb, the facilitative verb indicates that the referent easily changes into or acquires the property described by the base verb, as in \ref{ex:mAkWGArgAz}.

\begin{exe}
\ex \label{ex:mAkWGArgAz}
\gll 
\ipa{tɯrme}  	\ipa{mɤ-kɯ-ɲɟɯr,}  	\ipa{nɯnɯ}  	\ipa{mɤ-kɯ-ɣɤ-rgɤz}  	\ipa{nɯnɯ}    \ipa{tɕe}	   \ipa{tɕe}	\ipa{tɤtho}  	\ipa{tu-sɤrmi-nɯ}  	\ipa{ŋu.}  \\
people \textsc{neg-nmlz:S/A}-change \textsc{dem} \textsc{neg-nmlz:S/A-facil}-be.old \textsc{dem} \textsc{lnk} \textsc{lnk}
pine \textsc{ipfv}-call-\textsc{pl} \textsc{n.pst}:be \\
\glt (Chinese people) call `pines' people who do not change, who not age easily. (Pine, 26)
\end{exe}

Although almost all cases of facilitative verbs in \ipa{ɣɤ--} are derived from property verbs, there are examples where this derivation is applied to intransitive dynamic verbs, for instance \ipa{ɴɢlɯt}  `break (vi)' $\rightarrow$ \ipa{ɣɤ-ɴɢlɯt} `easily breakable'. Thus, this derivation is not a useful test to define the adjective class.

\subsubsection{Tropative} \label{sec:tropative}
The tropative prefix \ipa{nɤ--} / \ipa{nɤɣ--} in Japhug derives transitive verbs meaning `consider ... to be X' from intransitive verbs (see \citealt{jacques13tropative}). In the tropative derivation,   the S of the base verb becomes the P of the derived transitive verb, while the added argument (the experiencer) is the A of the derived verb. It is superficially similar to causative (but has a different semantics) and markedly different from the applicative (where the added argument is the P of the derived transitive verb). 

Table \ref{tab:tropative} provides some typical examples of tropative verbs. This derivation is extremely productive, and can be applied to any property verb.

\begin{table}[h]
\caption{Examples of the \ipa{nɤ}- tropative prefix in Japhug}\label{tab:tropative} \centering
\begin{tabular}{lllllllll} \toprule 
basic verb  & &derived  verb &\\
\midrule
 \ipa{wxti} & be big & \ipa{nɤ-wxti} & consider to be  big \\
 \ipa{zri} & be long & \ipa{nɤ-zri} & consider to be  long \\
  \ipa{chi} &be sweet & \ipa{nɤx-chi}  &consider to be  sweet \\
         \midrule
  \ipa{maʁ} & not be & \ipa{nɤɣ-maʁ} & consider  not to be right \\
    \ipa{mnɤm} & have a smell & \ipa{nɤ-mnɤm} & smell \\
       \ipa{ʁzi} & be needed & \ipa{na-ʁzi} & need \\
\bottomrule
\end{tabular}
\end{table}

The status of the last three examples in Table \ref{tab:tropative} as property verbs, however, is contestable, an a more detailed discussion is necessary.

The tropative derivation cannot be applied to copulas and existential verbs such as \ipa{ɕti} `be (affirmative)', \ipa{tu} `exist', \ipa{me} `not exist', but the verb \ipa{nɤɣ-maʁ} `consider not to be right' is derived from \ipa{maʁ} `not be'. However, the meaning of the tropative verb is restricted in comparison with the negative copula.  In order to express the meaning `consider not to be ...' or `consider to be ...', one cannot use the tropative \ipa{nɤɣ-maʁ}, but must use a periphrastic construction with the verb \ipa{sɯpa} `consider'.

The stative verbs \ipa{ŋu} and \ipa{maʁ}, in addition to their uses as copulas `be' and `not be' respectively (as in \ref{ex:ri.kWmAR}), can also be used as property verbs meaning `be right, be correct' and `not be right, not be correct'. This latter usage is illustrated by examples \ref{ex:kWmAR.amAtAnAmanW}  and \ref{ex:kWNu.mWjnAme}.
\begin{exe}
\ex \label{ex:ri.kWmAR}
\gll
\ipa{tɕe} 	\ipa{komɤr} 	\ipa{ri} 	\ipa{kɯ-maʁ,} 	\ipa{konaʁ} 	\ipa{ri} 	\ipa{kɯ-maʁ.} \\
\textsc{lnk} red.leather also \textsc{nmlz}:S/A-not.be  black.leather also \textsc{nmlz}:S/A-not.be \\
\glt It is neither red leather not black leather. (Red leather 72)
\end{exe}

\begin{exe}
\ex \label{ex:kWmAR.amAtAnAmanW}
\gll
\ipa{tɤpɤtso} 	\ipa{ra} 	\ipa{kɯ} 	\ipa{kɯ-maʁ} 	\ipa{kɯ-fse} 	\ipa{a-mɤ-tɤ-nɤma-nɯ} 	\ipa{ɲɯ-ra} \ipa{ma}	\ipa{nɯ} \ipa{ra} 	\ipa{ɯ-kɯ-nɯkowa,} 	\ipa{ɯ-kɯ-sɤmbɤldʑɤm} 	\ipa{maŋe} 
  \\
child \textsc{pl} \textsc{erg} \textsc{nmlz}:S/A-not.be.right \textsc{nmlz}:S/A-be.like \textsc{irr-neg-pfv}-make-\textsc{pl} \textsc{const}-have.to because \textsc{dem} \textsc{pl} \textsc{3sg.poss-nmlz}:S/A-find.a.solution \textsc{3sg.poss-nmlz}:S/A-settle.dispute not.exist:\textsc{sens} \\
\glt (It is fortunate that) the children did not do bad things , as there was nobody to settle disputes. (A friend, 122)
\end{exe}
 
\begin{exe}
\ex \label{ex:kWNu.mWjnAme}
\gll
\ipa{kɯ-ŋu} 	\ipa{mɯ́j-nɤme} 	\ipa{cho} 	\ipa{tɕe,}  \ipa{tɕheme} 	\ipa{kɯ-mpɕɤr} 	\ipa{ɯ-qhu} 	\ipa{jɤ-a<nɯ>ri} 	\ipa{ɕti} \\
\textsc{nmlz}:S/A-be.correct \textsc{neg:const}-make[III] \textsc{comit} \textsc{lnk} girl \textsc{nmlz}:S/A-be.beautiful \textsc{3sg}-after \textsc{pfv}-<\textsc{auto}>go[II] \textsc{n.pst}:be:\textsc{affirm} \\
\glt (Our little brother) did not do the right things, he went away with a beautiful girl.  (The fox, 144)
\end{exe}
Although the two meanings are obviously related - we can surmise that the archaic meaning of the roots \ipa{ŋu} and \ipa{maʁ} was `to be right' and `not to be right' and that their use as copulas is an innovation - from a synchronic point of view it is legitimate to analyze them as   distinct (but homophonous) lexical items. The tropative derivation thus can be applied to the property verb \ipa{maʁ} `not be right' but not to any copula or existential verb, including the related verb \ipa{maʁ} `not be'.
 
 
 In section \ref{sec:infinitive} we saw that TAM auxiliaries which do not allow animate S and cannot be conjugated in person and number form their infinitive in \ipa{kɯ--} like stative verbs. This class includes verbs such as \ipa{ra} `have to, need', \ipa{ŋgrɤl} `be usually the case', \ipa{ɬoʁ} `need', \ipa{zgɤt} `have to, be obliged to', \ipa{khɯ} `have to, be possible', \ipa{jɤɣ} `be possible'. The tropative derivation cannot be applied to any of these verbs. 
 
 Yet, the stative verb  \ipa{ʁzi} `need, be needed' could appear to be a counterexample, since a tropative verb \ipa{na-ʁzi} `need' can be derived from it as in example \ref{ex:naʁzi}.\footnote{Incidentally, the existence of a transitive verb like \ipa{naʁzi} meaning `need' in Japhug, while there is no transitive verb meaning `have' makes it a counterexample to \citealt{harves12need}'s generalization. See \citet{antonov14need} for more counterexamples.}

\begin{exe}
\ex \label{ex:naʁzi}
\gll
\ipa{tɕeri} 	\ipa{ɯ-tɯ-ci} 	\ipa{wuma} 	\ipa{ʑo} 	\ipa{na-ʁzi} 	\ipa{tɕe,} 	\ipa{ɯ-tɯ-ci} 	\ipa{nɯ} 	\ipa{mɯ-pjɯ-mbrɤt} 	\ipa{ɲɯ-ra} \\
but \textsc{3sg.poss-indef.poss}-water really \textsc{emph} \textsc{n.pst:trop}-be.needed \textsc{lnk}  \textsc{3sg.poss-indef.poss}-water  \textsc{dem} \textsc{neg-ipfv}-be.cut \textsc{const}-have.to \\
\glt But it needs water very much, it has to have water continuously. (Willow, 10)
\end{exe}

There is however on purely semantic grounds no problem to see \ipa{ʁzi} as a property verb, as we saw that other property verbs can take complement clauses (\ref{sec:complement}). In this regards, it is similar to the adjective   `necessary' in English.
%See directional, tɤ- and thɯ- XXXX

The verb   \ipa{mnɤm} `to have a smell' has a semantics that appears to be unexpected for a property verb. It requires as its S a noun meaning `odour, smell' like \ipa{--di} `smell' or \ipa{--dɯχɯn} `fragrance'. The tropative verb \ipa{nɤ-mnɤm} is the volitive verb `smell something' (non-volitive olfactory perception is expressed by the verb \ipa{mtshɤm} which also means `hear'). 

The meaning of   \ipa{mnɤm} is actually closer to `to be perceptible (of a smell)'. When used in the TAM categories mentioned  in section \ref{sec:inchoative}, it develops an inchoative meaning as in \ref{ex:tumnAm}.
  %Directional prefixes thɯ- XXXXX
 \begin{exe}
\ex \label{ex:tumnAm}
\gll
 \ipa{χsɤ-rʑaʁ} 	\ipa{jamar} 	\ipa{tɤ-tsu} 	\ipa{tɕe} 	\ipa{tɕe} 	\ipa{ɯ-di} 	\ipa{tu-mnɤm} 	\ipa{ɲɯ-ŋu} \\
 three-night about \textsc{pfv}-pass \textsc{lnk}  \textsc{lnk}  \textsc{3sg.poss}-smell \textsc{ipfv}-be.smelly \textsc{const}-be \\
 \glt After three nights, its smell becomes perceptible.  (Alcohol, 72)
\end{exe}

The tropative derivation is therefore a criterion sufficient in itself to define the class of adjectives in Japhug: it can be applied to any property verb, and is not compatible with other verbs forming their infinitive in \ipa{kɯ--} like copulas and TAM auxiliaries. It is also impossible with transitive and intransitive dynamic verbs, including all auxiliaries such as \ipa{spa} `be able' or \ipa{cha} `can'. For non-adjectival verbs, the only available construction to express `consider ... X' is to use the verb \ipa{sɯpa} `consider'.
 


 \section{Reduplication}
As one of the defining tests for distinguish adjectives from verbs in Qiang, \citet[315]{lapolla04adj} point out that reduplication has   different semantics depending on the part of speech: in the case of verbs, it has a reciprocal meaning, while in the case of adjectives, it has either a plural or intensive meaning.


In Japhug, the reciprocal derivation does involve reduplication of the final syllable of the verb stem, but it is combined with a prefix \ipa{a--} or \ipa{amɯ--} (\citealt{jacques12demotion}). Apart from reciprocal, reduplication has many other functions in Japhug. Among these, two functions that are particularily commonly applied to property verbs are increasing degree and high degree (see \citealt{jacques07redupl}).

The expression of increasing degree is expressed by reduplication the first syllable of the verbal word, regardless of whether it is a prefix of part of the stem, as in \ref{ex:YWYWndWB}.
 
\begin{exe}
\ex \label{ex:YWYWndWB}
\gll 
\ipa{ɯʑo} 	\ipa{tɤ-wxti} 	\ipa{ɯjɯja} 	\ipa{tɕe} 	\ipa{ɯ-jwaʁ} 	\ipa{nɯnɯ} 	\ipa{ɲɯ-ɲɯ-ndɯβ} 	\ipa{ʑo} 	\ipa{ɲɯ-ŋu.} 	\\
it \textsc{pfv}-be.big along \textsc{lnk} \textsc{3sg.poss}-leaf \textsc{top} \textsc{redp-ipfv}-be.tiny  \textsc{emph} \textsc{const}-be \\
\glt As it becomes bigger, its leaves become smaller. (Poplars, 18)
\end{exe}

%ʑɯrɯʑɤri tɕe tɕendɤre ɯ-skɤt nɯ tu-tɯ-mpɕɤr ʑo ŋu (404 
%phono

While this construction almost exclusively occurs with property verbs and may seem to  be restricted to them, we do find example of it used with dynamic verbs like \ipa{ɬoʁ} `come out', as in \ref{ex:YWYWLoR}, showing that this cannot be used as a defining criterion of adjectives in Japhug.

\begin{exe}
\ex \label{ex:YWYWLoR}
\gll 
\ipa{ʑatsa} 	\ipa{ɲɯ-ɲɯ-ɬoʁ} 	\ipa{ʑo} 	\ipa{ŋu} 	\ipa{tɕe,} 	\ipa{ɲɯ-ɲɯ-mɤku} 	\ipa{ʑo} 	\ipa{ŋu} 	\ipa{tɕe,} \\
early \textsc{redp-ipfv}-come.out \textsc{emph} \textsc{n.pst}:be \textsc{lnk} redp-\textsc{ipfv}-be.early \textsc{emph} \textsc{n.pst}:be  \textsc{lnk} \\
\glt When (the constellation of the earthworm) rises earlier and earlier, as it becomes earlier and earlier,   (Constellations, 398)
\end{exe}

Another construction commonly observed with property verbs is the high degree reduplication. This construction involves the reduplication of the last syllable of the verb stem as in \ref{ex:kWwxtWwxti}.

\begin{exe}
\ex \label{ex:kWwxtWwxti}
\gll 
\ipa{ɯ-ʁrɯ} 	\ipa{nɯnɯ} 	\ipa{kɯ-wxtɯ-wxti} 	\ipa{ʑo} 	\ipa{ŋu} \\
\textsc{3sg.poss}-horn \textsc{top} \textsc{nmlz:S/A-redp}-be.big \textsc{emph} \textsc{n.pst}:be \\
\glt Its horns are very big. (Bharal, 14)
\end{exe}

Yet, this type of reduplication can also be applied to non-property verbs such as \ipa{rga} `like' as in \ref{ex:kWrgWrga}.
\begin{exe}
\ex \label{ex:kWrgWrga}
\gll 
\ipa{tɕeri} 	\ipa{tú-wɣ-ndza} 	\ipa{kɯ-mɯm} 	\ipa{nɯ} 	\ipa{me} 	\ipa{tɕe,} 	\ipa{kɯ-rgɯ-rga} 	\ipa{ʑo} 	\ipa{me} \\
but \textsc{ipfv-inv}-eat \textsc{nmlz}:S/A-be.tasty \textsc{top} \textsc{n.pst}:not.exist \textsc{lnk} \textsc{nmlz:S/A-redp}-like \textsc{emph}  \textsc{n.pst}:not.exist \\
\glt But (this mushroom) is not  tasty when one its it, and no one likes it. (kachijmɤɣ, 153)
\end{exe}

Thus, although Japhug has very rich reduplication morphology, it cannot be used, unlike Qiang, as a morphological test to define the adjective part of speech.


 \section{Comparative and expression of degree}
 
 
 
 Degree construction: not exclusively adjectival
 tɕe nɯnɯ kɯ ɯ-tɯ-rɯmɯntoʁ maqhu
 nɯ ɯ-tɯ-tʂɯβ mɤ-kɯ-βdi tu-kɯ-ti ŋu.
 kɤ-rɤβraʁ ra ma ɯ-tɯ-rɤʑi ʑaʑa mɤ-ʑi.
 ɯ-tɯ-rɤʑa kɯ ɕɯmŋɤm ʑo ɕti.
 tɕe nɯnɯ tɯqejmɤɣ nɯnɯ ɯ-tɯ-ɬoʁ ɯ-tshɯɣa ɯ-ru nɯra ɲɯ-xtshɯm.
 
The comparative construction in Japhug can be illustrated by example \ref{ex:comp1}. The standard is marked by a comparative postposition \ipa{sɤz} or \ipa{staʁ} (and their variants \ipa{sɤznɤ} and \ipa{staʁnɤ}) and the comparee (optionally) receives the ergative marker \ipa{kɯ}.\footnote{While the use of the ergative to mark the standard of comparison is attested (see \citealt{agent02palancar}), its use on the comparee is typologically unexpected. } The ergative on the comparee is enough to indicate a comparative construction, even when the comparee is not overt 

\begin{exe}
\ex \label{ex:comp1}
\gll  \ipa{ɯ-ʁi}   	\ipa{sɤz}   	\ipa{ɯ-pi}   	\ipa{nɯ}   	\ipa{kɯ}   	\ipa{mpɕɤr}     \\
\textsc{3sg.poss}-younger.sibling \textsc{comparative} \textsc{3sg.poss}-elder.sibling \textsc{dem} \textsc{erg} -beautiful \\
\glt The elder one is more beautiful than the young one.
\end{exe}

This construction is not restricted to adjectives; it can also appear with tropative verbs (\ref{ex:nAmWm}) or with the verb \ipa{sɯpa} `consider' with an infitival adjective complement.
 
\begin{exe}
\ex \label{ex:kWkWmWm}
\gll  \ipa{tɕe}   	\ipa{thoŋraʁ}   	\ipa{nɯ}   	\ipa{kɯ}   	\ipa{kɯ-mɯm}   	\ipa{tu-sɯpa-nɯ}   	\ipa{ŋu}   \\
\textsc{coord} bucket.alcohol \textsc{dem} \textsc{erg} \textsc{inf:stat}-be.tasty \textsc{ipfv}-consider-\textsc{pl} \textsc{n.pst}:be \\
\glt They consider  bucket alcohol to be tastier (than the pan-alcohol) (Distilled alcohol, 15)
\end{exe}

\begin{exe}
\ex \label{ex:nAmWm}
\gll  \ipa{thoŋraʁ} 	\ipa{nɯ} 	\ipa{kɯ} 	\ipa{ɲɯ-nɤ-mɯm-nɯ} \\
 bucket.alcohol \textsc{dem} \textsc{erg}  \textsc{const-trop}-be.tasty-\textsc{pl} \\
 \glt They consider  bucket alcohol to be tastier (than the pan-alcohol)  (elicited).
\end{exe}
 Apart from using the adverbial modifier \ipa{stu}  most', another way to express extreme degree is to use a construction including \ipa{kɯ-fse}, a nominalized form of the adjective and the negative existential verb \ipa{me} `not exist' (as in \ref{ex:kWsAGmu})
 
 \begin{exe}
\ex \label{ex:kWsAGmu}
\gll 
\ipa{nɤʑo} 	\ipa{kɯ-fse} 	\ipa{kɯ-sɤɣmu} 	\ipa{me} 	\ipa{tɕe} 		\\
\textsc{2sg} \textsc{inf:stat}-be.like \textsc{nmlz}:S/A--be.scary \textsc{n.pst}:not.exist \textsc{lnk} \\
\glt You are the scariest of all (animal) (No (animal) is as scary as you) (The tiger, 115)
 \end{exe}
 
This construction is not available for dynamic verbs, but it is compatible with other stative verbs such as copulas and existential verbs as in  \ref{ex:WtshuxtoR}.
  \begin{exe}
\ex \label{ex:WtshuxtoR}
\gll 
\ipa{khɯna} 	\ipa{kɯ-fse} 	\ipa{ʑo}  \ipa{ɯ-tshuxtoʁ} 	\ipa{kɯ-tu} 	\ipa{me} 	\ipa{khi,} 	\\
dog \textsc{inf:stat}-be.like \textsc{emph} \textsc{3sg.poss}-trust \textsc{nmlz}:S/A-exist \textsc{n.pst}:not.exist  \textsc{hearsay} \\
\glt The dog is the most loyal (animal). (Dog, 4)
  \end{exe}
  
\section{Adjectives as modifiers}
Relative clauses, mostly internally-headed 
ɯʑo sɤz rɯdaʁ kɯ-xtɕi nɯra tu-ndze
\citet{jacques14relatives}
Prenominal adjectives are very rare: exception: \ipa{kɯ-ɕɤɣ} 
 
 \section{A class of nominal adjectives?}
 
 quantifiers and pronominal adjectives, all nominal except antɕhɯ
 \ipa{kɯmaʁ} `other',  \ipa{wuma}
 \ipa{ldʑaŋkɯ}
  \ipa{ɯ-xso}
    \ipa{ɯ-rkoz}
    
        \ipa{ɯ-mbe}
            \ipa{ɯ-ndo}
            
kɯmɯxte    
    
    tɤkhe
    χcha / ʁe
    tɯ-jaʁ χcha nɯ kɯ, taqaβ cho tɤ-ri nɯ tú-wɣ-ndo,
 colours: 


 
 \section{Conclusion}
 
 Two classes of adjectives
 
 \begin{table}[H]
 \caption{Morphosyntactic tests discussed in the paper}
\begin{tabular}{llllll}
\toprule
&\multicolumn{3}{c}{Stative}&\multicolumn{2}{c}{Dynamic}\\
Test & Adjectives & Copulas & Auxiliaries  & Intransitive  & Transitive \\
&&& (expletive S)&  &\\
\midrule
Past imperfective &\ipa{pɯ--} &\ipa{pɯ--} &\ipa{pɯ--} &periphrastic&periphrastic \\
Infinitive &\ipa{kɯ--} & \ipa{kɯ--} & \ipa{kɯ--} &\ipa{kɤ--}&\ipa{kɤ--}\\
Causative& \ipa{ɣɤ--} and/or \ipa{sɯ--}& none & none &\ipa{sɯ--}&\ipa{sɯ--}\\
Facilitative& \ipa{ɣɤ--}& none & none &\ipa{ɣɤ--} &\ipa{nɯɣɯ--}\\
Tropative &\ipa{nɤ--}& none& none& none& none\\
\bottomrule
\end{tabular}
\end{table}
Tropative derivation is the only criterion for distinguish verbal adjectives from other  verbs. 

Several classes of nominal adjectives

\bibliographystyle{linquiry2}
\bibliography{bibliogj}
\end{document}