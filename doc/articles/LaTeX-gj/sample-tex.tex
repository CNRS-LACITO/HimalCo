\documentclass[oldfontcommands,oneside,a4paper,11pt]{article} 
\usepackage{fontspec}
\usepackage{natbib}
\usepackage{booktabs}
\usepackage{xltxtra} 
\usepackage{polyglossia} 
\setdefaultlanguage{french} 
\usepackage[table]{xcolor}
\usepackage{gb4e} 
\usepackage{multicol}
\usepackage{graphicx}
\usepackage{float}
\usepackage{lineno}
\usepackage{textcomp}
\usepackage{hyperref} 
\hypersetup{bookmarks=false,bookmarksnumbered,bookmarksopenlevel=5,bookmarksdepth=5,xetex,colorlinks=true,linkcolor=blue,citecolor=blue}
\usepackage[all]{hypcap}
\usepackage{memhfixc}
\usepackage{lscape}
 

%\setmainfont[Mapping=tex-text,Numbers=OldStyle,Ligatures=Common]{Charis SIL} 
\newfontfamily\phon[Mapping=tex-text,Ligatures=Common,Scale=MatchLowercase,FakeSlant=0.3]{Charis SIL} 
\newcommand{\ipa}[1]{{\phon#1}} %API tjs en italique
 
\newcommand{\grise}[1]{\cellcolor{lightgray}\textbf{#1}}
\newfontfamily\cn[Mapping=tex-text,Ligatures=Common,Scale=MatchUppercase]{MingLiU}%pour le chinois
\newcommand{\zh}[1]{{\cn #1}}
\newcommand{\topic}{\textsc{dem}}
\newcommand{\tete}{\textsuperscript{\textsc{head}}}
\newcommand{\rc}{\textsubscript{\textsc{rc}}}
\XeTeXlinebreaklocale 'zh' %使用中文换行
\XeTeXlinebreakskip = 0pt plus 1pt %
 %CIRCG
\newcommand{\ro}{$\Sigma$}
\newcommand{\siga}{$\Sigma_1$} 
\newcommand{\sigc}{$\Sigma_3$}   


\begin{document} 

\title{Mémoire}
\author{Guillaume Jacques}
\maketitle

\citet{jacques10inverse}

\section{Phonology}

 
Some clusters are affected by partial reduplication:   when the last consonant  of a cluster is one of the non-nasal sonorants (\ipa{r}, \ipa{l}, \ipa{j}, \ipa{w}, \ipa{ɣ} or \ipa{ʁ}), and the preceding consonant is neither a sonorant nor a sibilant fricative, the sonorant is deleted, as in example \ref{ex:medial.r}. 
 
 \begin{exe}
\ex \label{ex:medial.r}
\glt \ipa{praʁ} `cut, break'$\rightarrow$ \ipa{pɯ-praʁ}
\end{exe}

Dans le tableau \ref{tab:stem2}
\&
 \begin{table}[H]
\caption{Stem 2 alternation in Japhug Rgyalrong} \label{tab:stem2} \centering
\begin{tabular}{llllll}
\toprule
Stem 1 & meaning &Stem 2 \\
\midrule
\ipa{ɕe}& to go (vi)&  \ipa{ari} \\
\ipa{sɯxɕe}& to send (vt)  &\ipa{sɤɣri} \\
\ipa{ɣi}& to come (vi)  &\ipa{ɣe} \\
\ipa{ti}& to say (vt)  &\ipa{tɯt} \\
\bottomrule
\end{tabular}
\end{table}

 \begin{table}[H]
\caption{Paradigm} \label{tab:paradigm} \centering
\begin{tabular}{llllll}
\toprule
 & 	1sg & 	2sg & 	3sg & 	\\
1sg & 	\ipa{} & 	\ipa{chi-} & 	\ipa{wa-} & 	\\
2sg & 	\ipa{mani-} & 	\ipa{} & 	\ipa{ya-} & 	\\
3sg & 	\ipa{ma-} & 	\ipa{ni-} & 	\ipa{} & 	\\
\bottomrule
\end{tabular}
\end{table}


Dans l'exemple \ref{ex:ngolo-tAjko}

 \begin{exe}
\ex \label{ex:ngolo-tAjko}
\gll
\ipa{tɕe} 	\ipa{tɤjko} 	\ipa{mɯ-tɤ-tɕur} 	\ipa{tɕe,} 	\ipa{ɴɢolo} 	\ipa{ɯ-mat} 	\ipa{nɯ} 	\ipa{ɲɯ́-wɣ-phɯt} 	\ipa{tɕe,} 	\ipa{tɕe} 	\ipa{tɤrca} 	\ipa{pjɯ́-wɣ-ɣɤla} 	\ipa{tɕe,} 	\ipa{tɕe} 	\ipa{tɤjko} 	\ipa{pjɯ-sɯɣ-tɕur} 	\ipa{cha.} \\ 
\textsc{lnk} pickled.turnip.leaves \textsc{neg-pfv}-be.sour \textsc{lnk} tree.sp \textsc{3sg.poss}-fruit \textsc{dem} ipfv-inv-pluck \textsc{lnk} \textsc{lnk} together \textsc{ipfv-inv}-soak \textsc{lnk} \textsc{lnk}  pickled.turnip.leaves \textsc{ipfv-caus}-be.sour  can:\textsc{fact} \\
 \glt When the pickled turnip leaves are not sour, one picks \ipa{ɴɢolo} fruit and soaks it with them, and it can make the turnip leaves sour.   (\ipa{ɴɢolo} 27) 
   \end{exe}

\bibliographystyle{unified}
\bibliography{bibliogj}
\end{document}