\documentclass[oneside,a4paper,11pt]{article} 
\usepackage{fontspec}
\usepackage{natbib}
\usepackage{booktabs}
\usepackage{xltxtra} 
\usepackage{polyglossia} 
\usepackage[table]{xcolor}
\usepackage{gb4e} 
\usepackage{multicol}
\usepackage{graphicx}
\usepackage{float}
\usepackage{hyperref} 
\hypersetup{bookmarksnumbered,bookmarksopenlevel=5,bookmarksdepth=5,colorlinks=true,linkcolor=blue,citecolor=blue}
\usepackage[all]{hypcap}
\usepackage{memhfixc}
\usepackage{lscape}

\setmainfont[Mapping=tex-text,Numbers=OldStyle,Ligatures=Common]{Charis SIL} 
\newfontfamily\phon[Mapping=tex-text,Ligatures=Common,Scale=MatchLowercase]{Charis SIL} 
\newcommand{\ipa}[1]{{\phon\mbox{\textbf{#1}}}}
\newcommand{\phonet}[1]{[{\phon\mbox{\textbf{#1}}}]}
\newcommand{\ipab}[1]{{\scriptsize \phon#1}} 

\newcommand{\grise}[1]{\cellcolor{lightgray}\textbf{#1}}
\newfontfamily\cn[Mapping=tex-text,Ligatures=Common,Scale=MatchUppercase]{SimSun}%pour le chinois
\newcommand{\zh}[1]{{\cn #1}}
\newcommand{\refb}[1]{(\ref{#1})}

%\XeTeXlinebreaklocale 'zh' %使用中文换行
%\XeTeXlinebreakskip = 0pt plus 1pt %
 
\newcommand{\change}[2]{*\ipa{#1} $\rightarrow$ *\ipa{#2}}

\begin{document} 
\title{From prenasalization to preaspiration in Algonquian}
\author{Guillaume Jacques}
\maketitle
 
 \section{Introduction}
Several Algonquian languages, including Cree and Menominee, present an unusual merger of proto-Algonquian *\ipa{-nC-} and *\ipa{-hC-} clusters as \ipa{-hC-}, implying a sound change from prenasalization to preaspiration, which is unusual and unparalleled in other languages families (see Silverman's \citeyear{silverman03preaspirated} survey of the diachronic origins of preaspirated stops).


Table \ref{tab:nC.PA}, based on Bloomfield's (\citeyear[154-5]{bloomfield25central} / \citeyear{bloomfield46proto}) reconstruction of Proto-Algonquian,\footnote{I adopt Goddard's \citealt{goddard98arapaho} notation of Proto-Algonquian, with *\ipa{r}, *\ipa{rk}, *\ipa{sC} instead of Bloomfield's *\ipa{l}, *\ipa{çk} and *\ipa{xC}. } illustrates the reflexes of proto-Algonquian *\ipa{-nC-} clusters in Ojibwe (where the nasal is preserved), Plains Cree and Menominee. The reflexes of *\ipa{-nC-} in the latter two languages are identical to corresponding *\ipa{-hC-} clusters. 
 
 \begin{table}
\caption{Clusters with nasal as first element in Proto-Algonquian and their correspondences in Ojibwe, Cree and Menominee} \label{tab:nC.PA} \centering
\begin{tabular}{llllll}
\toprule
PA & Ojibwe & Menominee & Plains Cree \\
\midrule
\ipa{*-mp-} & \ipa{-mb-} & \ipa{-hp-} & \ipa{-hp-} \\
\ipa{*-nt-} & \ipa{-nd-} & \ipa{-ht-} & \ipa{-ht-} \\
\ipa{*-nč-} & \ipa{-nj-} & \ipa{-hc-} & \ipa{-hč-} \\
\ipa{*-nk-} & \ipa{-ng-} & \ipa{-hk-} & \ipa{-hk-} \\
\ipa{*-ns-} & \ipa{-nz-} & \ipa{-hs-} & \ipa{-hs-} \\
\ipa{*-nš-} & \ipa{-nzh-} & \ipa{-hs-} & \ipa{-hš-} \\
\ipa{*-nr-} & \ipa{-n-} & \ipa{-hn-} & \ipa{-hy-} / \ipa{-h-} \\
\ipa{*-nθ-} & \ipa{-n-} & \ipa{-hn-} & \ipa{-ht-} \\
\bottomrule
\end{tabular}
\end{table}

These correspondences are massively attested and the proto-Algonquian reconstruction is not controversial. The clusters *\ipa{-nr-} and *\ipa{-nθ-} are rare and Table \ref{tab:merger} presents some etymologies illustrating them from Bloomfield and \citet{goddard73nl}. In Plains Cree, \ipa{-hy-} tends to merge with \ipa{-h-}, but this affects all \ipa{-hy-} clusters, not only those originating from *\ipa{-nr-}.

\begin{table}
\caption{*\ipa{-nr-} and *\ipa{-nθ-} clusters in Cree and Menominee} \centering \label{tab:merger}
 \resizebox{\columnwidth}{!}{
\begin{tabular}{llllll}
\toprule
PA & Meaning &Menominee & Plains Cree \\
\midrule
\ipa{*nōnr-} &suckle & \ipa{nōhn-æw} `suckle (vta)'&\ipa{nôh-êw} `suckle (vta)' \\
\ipa{*wīnr-} &name & \ipa{wēhn-æw} `name (vta)'&\ipa{wîh-êw} `name (vta)' \\
%\midrule
%\ipa{*rēhr-} &breathe' &\ipa{yêhy-êw} `breathe' & \ipa{nǣhn-æw} `breathe' \\
\midrule
\ipa{*-ahanθ-} & track & \ipa{natu-ahahn-æw} `seek s.o's tracks' &\ipa{wan-ahâht-êw} `lose s.o.'s tracks' \\
\ipa{*panθ-} & singe &\ipa{pahn-æw} `roast (vta)'&\ipa{paht-êw} `singe (vta)'\\
\bottomrule
\end{tabular}}
\end{table}

Note that Cree and Menominee have undergone these sound changes independently. Menominee, like all Algonquian languages except Plains Algonquian and Cree, merge *\ipa{θ} and *\ipa{r} in all positions,\footnote{Traces of the contrast only remain in palatalized forms of *\ipa{θ}.} a merger that constitutes a common innovation of all these languages (\citealt{goddard94cline}). Thus, any innovative feature exclusively shared by Cree and Menominee can only be either due to contact or parallel innovation. 

In this paper, I postulate several pathways of phonetic change to explain the correspondences in Table \ref{tab:nC.PA}, making sure that each step has parallels in other language families. First, I present previous accounts of these sound changes. Second, I propose that *\ipa{-nC-} did not directly change to preaspirated consonants, but rather than an intermediate stage with gemination took place. Third, I argue that the changes affecting the cluster *\ipa{-nr-} should be explained differently from nasal+obstruent clusters.


\subsection{Voiceless nasal to preaspiration}
Despite the unusual character of the shift from prenasalization to preaspiration observed in Cree and Menominee, few authors have explicitly discussed them in detail.  \citet[60]{hockett81menominee} proposes the following pathway: `In Menominee and Cree, [...] it seems as though the voicessless of the stop or spirant spread backwards through the nasal; at least, a change from a sequence like \phonet{nt} to \phonet{ht} seems easiest to understand if we think of \phonet{n̥t}.' This explanation has never been questioned in the literature. A more recent work such as \citet[25-6]{clayton10preaspirated}, though not citing Hockett, proposes essentially the same hypothesis.

This explanation is however very problematic. While there is little doubt that an aspirated nasal /\ipa{n̥}/ can lose its nasality (or transfer it to the following vowel) and merge with /\ipa{h}/, as has happened in Naish and Pumi (\citealt{michaud-jacques12nasalite}), all indisputable examples of this change known to us occur in onset position in languages with a phonological contrast between \ipa{n} and \ipa{n̥}.

Crosslinguistically, in clusters comprising a nasal as first element, when the following stop is unvoiced, the nasal may be devoiced by regressive assimilation, but the resulting voiceless nasal will not be aspirated like the voiceless nasals occurring in onset position that phonologically contrast with plain nasals, and there is thus little basis for assuming a direct sound change like *\ipa{-n̥t-} $\rightarrow$ \ipa{-ht-}: the problem is that the IPA symbol \ipa{n̥} is used for both unvoiced and aspirated nasals.

%even if phonetically unvoiced when followed by a by a voiceless obstruent, as shown by the extremely common epenthetic stop consonant appearing when the second element of the cluster is a continuant (fricative or non-nasal sonorant). 
%For instance, in Tibetan historical phonology (\citealt{lifk33, hill11laws}), pre-Tibetan nasal+fricative clusters have become nasal+affricate clusters in Old Tibetan (*\ipa{ns-} $\rightarrow$ \ipa{ɴtsʰ-}, *\ipa{nɕ-} $\rightarrow$ \ipa{ɴtɕʰ-} etc) and nasal+sonorant have developed epenthetic stops (*\ipa{ml-} $\rightarrow$ \ipa{md-}).

In addition, this line of explanation cannot account for the change \change{nr}{hr}, since *\ipa{r} clearly was a voiced segment originally, and there is no evidence that *\ipa{n} could undergo regressive devoicing before *\ipa{r} unless *\ipa{r} itself devoices in this particular context.


\section{Nasal+obstruent clusters}
The first issue with Hockett's explanation, the unnaturalness of a direct sound change *\ipa{-n̥t-} $\rightarrow$ \ipa{-ht-}, can be accounted for by supposing two intermediate stages.

In the case of clusters comprising a homorganic nasal followed by an obstruent (stop/affricate *\ipa{p}, *\ipa{t}, *\ipa{č}, *\ipa{k} or fricative *\ipa{s}, *\ipa{š}) the three stages pathway in (\ref{ex:nC}) can be proposed. Table \ref{tab:gem} details the specific sound changes involved and the mergers (indicated in grey) in Menominee.

\begin{exe}
\ex \label{ex:nC}
\glt *\ipa{-nC-} $\rightarrow$ \ipa{-CC-} $\rightarrow$ \ipa{-ʰCC-} $\rightarrow$ \ipa{-hC-} 
\end{exe}

\begin{table}[H]
\caption{From Proto-Algonquian to Menominee} \label{tab:gem} \centering
\begin{tabular}{llllll}
\toprule
PA & stage 1 & stage 2 & stage 3& Menominee \\
\midrule
\ipa{*-mp-} & > \ipa{-pp-} & > \ipa{-ʰpp-} &\grise{} >\ipa{-hp-} &\grise{}  \\
\ipa{*-hp-} &   &   &\grise{}   &\grise{}  \\
\ipa{*-sp-} &   &   &   &>\grise{} \ipa{-hp} \\
\midrule
\ipa{*-nt-} & > \ipa{-tt-} & > \ipa{-ʰtt-} & \grise{} >\ipa{-ht-} &\grise{}  \\
\ipa{*-ht-} &   &   &\grise{}   &\grise{}  \\
\midrule
\ipa{*-nč-} & > \ipa{-čč-} & > \ipa{-ʰčč-} &> \grise{} \ipa{-hč-} &>\grise{} \ipa{-hc-}  \\
\ipa{*-hč-} &   &   &\grise{}   &>\grise{} \ipa{-hc-}  \\
\midrule
\ipa{*-nk-} &> \ipa{-kk-} &> \ipa{-ʰkk-} &> \grise{}\ipa{-hk-} &\grise{}  \\
\ipa{*-hk-} &   &   &\grise{}   &\grise{}  \\
\ipa{*-sk-} &   &   &   &>\grise{} \ipa{-hk} \\
\midrule
\ipa{*-ns-} &> \ipa{-ss-} &> \ipa{-ʰss-} &>\grise{} \ipa{-hs-} &\grise{}  \\
\ipa{*-hs-} &   &   &\grise{}   &\grise{}  \\
\cline{1-5}
\ipa{*-nš-} &> \ipa{-šš-} &> \ipa{-ʰšš-} &> \ipa{-hš-} \grise{}&>\grise{}  \ipa{-hs-}\\
\ipa{*-hš-} &   &   &   \grise{}&>\grise{} \ipa{-hs} \\
\bottomrule
\end{tabular}
\end{table}

In stage 1, the cluster *\ipa{-nC-} changes to a geminate consonant, a sound change cross-linguistically well-attested, for instance in Hebrew, where \ipa{n} assimilates to all consonants except the gutturals and \ipa{r} (see \citealt[71-75]{jouon06}). In Old Norse, assimilation only occurs before unvoiced obstruents (\ipa{oss} $\leftarrow$ *\ipa{uns} `us', \ipa{drekka} $\leftarrow$ *\ipa{drinkan} `drink' etc). 

In stage 2, the geminated consonants develop phonetic preaspiration, a phenomenon attested for instance in the Kiranti language  Khaling, where geminated stops and double stop clusters preceded by short vowels are automatically preaspirated (\citealt[44]{jacques16tonogenesis}).

In stage 3, the preaspiration, originally a secondary cue of gemination, becomes the main phonetic cue of the contrast (\citealt[172]{kuemmel07wandel}, \citealt{silverman03preaspirated}), and geminated consonants merge with the corresponding preaspirated ones. 

Following stage 3, Menominee additionally lost the contrast between *\ipa{š} and *\ipa{s}, including in clusters.
  
 \section{*-nr-}
The hypothesis proposed for n+obstruents cannot account for the merger of Proto-Algonquian *\ipa{-nr-} and *\ipa{-hr-} clusters as *\ipa{-hr-} in pre-Cree and pre-Menominee and their subsequent evolution to \ipa{-hy-} and \ipa{-hn-} respectively. That the development of preaspiration before obstruent and before the reflexes of *\ipa{r} are independent sound changes is confirmed by the fact that the Eastern Algonquian languages Unami and Munsee languages have undergone *\ipa{-nr-} / *\ipa{-nθ-} $\rightarrow$ \ipa{-hl-} but have preserved the nasal before stops (\citealt[25]{goddard82munsee}).\footnote{In addition, in Unami and Munsee *\ipa{hr} becomes \ipa{x} and does not merge with the reflex of *\ipa{nr}.}

I propose here two alternative hypotheses to account for the change of prenasalization to preaspiration in these clusters in Cree, Menominee and Munsee. Both involve an intermediate stage with an unvoiced *\ipa{r̥} merging into pre-aspirated *\ipa{hr}.

\subsection{The Welsh model}
In Welsh, proto-Celtic *\ipa{r} and *\ipa{l} become aspirated \ipa{rh} and \ipa{ll} respectively in fortis position, ie. word-initially  and word-internally after nasal or \ipa{r} (\citealt[145]{kuemmel07wandel}, \citealt[471-480]{jackson56early}). This sound changes still has observable effects in the leniting mutation: words with initial \ipa{rh-} and \ipa{ll-} in Welsh (from *\ipa{r-} and *\ipa{l-} in initial position) mutate to \ipa{r-} and \ipa{l-}, except after proclitics with \ipa{-r} or \ipa{-n} codas.

I propose a similar sound change as a possibility to account for the merger of *\ipa{-hr-} and *\ipa{-nr-} in Cree, Menominee and Munsee.

First, as indicated in \ref{ex:nr1}, *\ipa{r} (including, in the case of Menominee and Unami/Munsee, *\ipa{r} from *\ipa{-θ-}) developed an unvoiced allophone *\ipa{r̥} word-initially and after nasal.

\begin{exe}
\ex \label{ex:nr1}
\glt *\ipa{-nr-} $\rightarrow$ *\ipa{-nr̥-} $\rightarrow$ *\ipa{-r̥-} $\rightarrow$ *\ipa{-hr-} 
\end{exe}

Second, nasality was lost before *\ipa{-r̥-}, which merged with the cluster *\ipa{-hr-} in intervocalic position. In word-initial position, allophony was lost and *\ipa{-r-} became revoiced, a common phenomenon with voiceless sonorants (see \citealt{blevins16voiceless.sonorants}).

\subsection{The Gemination model}
While in Hebrew, the sound law *-nC- $\rightarrow$ -CC- does not apply to \ipa{r}, this sound change is attested in many languages, including Latin (\citet[213]{leumann77lateinische}, in examples such as \ipa{irrigāre} `irrigate' < *\ipa{in-rigāre}). Another possible pathway to explain the development of preaspiration from prenasalization is presented in (\ref{ex:nr2}):

\begin{exe}
\ex \label{ex:nr2}
\glt *\ipa{-nr-} $\rightarrow$ *\ipa{-rr-} $\rightarrow$ *\ipa{-r̥-} $\rightarrow$ \ipa{-hr-} 
\end{exe}

In this hypothesis, *\ipa{r} undergoes the sound change *-nC- $\rightarrow$ *-CC- like obstruents, though this assimilation need not have occurred at the same time. Geminated *\ipa{-rr-} then becomes unvoiced *\ipa{-r̥-} and eventually merges with *\ipa{-hr-}.

\section{Conclusion}
The shift from prenasalization to preaspiration in Algonquian, rather than being a simple sound change, results from several pathways involving several intermediate steps. Hockett's hypothesis, according to whom the first nasal element in the cluster assimilates in voicing to the following consonant, becomes an unvoiced nasal and eventually a glottal fricative, does not account for the evolution of *\ipa{nr}, and does not have clear typological parallels.

\bibliographystyle{unified}
\bibliography{bibliogj}
\end{document}