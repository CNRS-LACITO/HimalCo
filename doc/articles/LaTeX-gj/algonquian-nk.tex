\documentclass[oneside,a4paper,11pt]{article} 
\usepackage{fontspec}
\usepackage{natbib}
\usepackage{booktabs}
\usepackage{xltxtra} 
\usepackage{polyglossia} 
\usepackage[table]{xcolor}
\usepackage{gb4e} 
\usepackage{multicol}
\usepackage{graphicx}
\usepackage{float}
\usepackage{hyperref} 
\hypersetup{bookmarksnumbered,bookmarksopenlevel=5,bookmarksdepth=5,colorlinks=true,linkcolor=blue,citecolor=blue}
\usepackage[all]{hypcap}
\usepackage{memhfixc}
\usepackage{lscape}

\setmainfont[Mapping=tex-text,Numbers=OldStyle,Ligatures=Common]{Charis SIL} 
\newfontfamily\phon[Mapping=tex-text,Ligatures=Common,Scale=MatchLowercase]{Charis SIL} 
\newcommand{\ipa}[1]{{\phon\mbox{\textbf{#1}}}}
\newcommand{\phonet}[1]{[{\phon\mbox{\textbf{#1}}}]}
\newcommand{\ipab}[1]{{\scriptsize \phon#1}} 

\newcommand{\grise}[1]{\cellcolor{lightgray}\textbf{#1}}
\newfontfamily\cn[Mapping=tex-text,Ligatures=Common,Scale=MatchUppercase]{SimSun}%pour le chinois
\newcommand{\zh}[1]{{\cn #1}}
\newcommand{\refb}[1]{(\ref{#1})}
\newcommand{\factual}[1]{\textsc{:fact}}
\newcommand{\rdp}{\textasciitilde{}}

\XeTeXlinebreaklocale 'zh' %使用中文换行
\XeTeXlinebreakskip = 0pt plus 1pt %
 %CIRCG
 \newcommand{\bleu}[1]{{\color{blue}#1}}
\newcommand{\rouge}[1]{{\color{red}#1}} 
\newcommand{\ch}[3]{\zh{#1} \ipa{#2} `#3'} 
\newcommand{\change}[2]{*\ipa{#1} $\rightarrow$ \ipa{#2}}
\newcommand{\deux}[1]{/#1/}
\newcommand{\trois}[1]{/#1/}

\newcommand{\tib}[1]{\cellcolor{lightgray}\textbf{#1}}
\newcommand{\idph}[1]{\cellcolor{gray}\textbf{#1}}



\begin{document} 
\title{From prenasalization to preaspiration in Algonquian}
\author{Guillaume Jacques}
\maketitle
 
 \section{Introduction}
 

Contribute to better explaining the diachronic origin of preaspirated stops, which given their rarity (\citealt{silverman03preaspirated})

Panchronic Phonology (\citealt{michaud-jacques12nasalite})
\citet[154-5]{bloomfield25central}
 \citet{bloomfield46proto}, merge with corresponding hC clusters in M and C
 \citet{goddard98arapaho}
 
Table \ref{tab:nC.PA} 
 
 \begin{table}
\caption{Correspondences of cluster with nasal as first element in Proto-Algonquian and their correspondences in Ojibwe, Cree and Menominee} \label{tab:nC.PA} \centering
\begin{tabular}{llllll}
\toprule
PA & Ojibwe & Cree & Menominee \\
\midrule
\ipa{*-mp-} & \ipa{-mb-} & \ipa{-hp-} & \ipa{-hp-} \\
\ipa{*-nt-} & \ipa{-nd-} & \ipa{-ht-} & \ipa{-ht-} \\
\ipa{*-nč-} & \ipa{-nj-} & \ipa{-hc-} & \ipa{-hč-} \\
\ipa{*-nk-} & \ipa{-ng-} & \ipa{-hk-} & \ipa{-hk-} \\
\ipa{*-ns-} & \ipa{-nz-} & \ipa{-hs-} & \ipa{-hs-} \\
\ipa{*-nš-} & \ipa{-nzh-} & \ipa{-hs-} & \ipa{-hš-} \\
\ipa{*-nr-} & \ipa{-n-} & \ipa{-hn-} & \ipa{-hy-} \\
\ipa{*-nθ-} & \ipa{-n-} & \ipa{-hn-} & \ipa{-ht-} \\
\bottomrule
\end{tabular}
\end{table}
 Cree and Menominee
\citealt[60]{hockett81menominee}  `as first element of a cluster *n $\rightarrow$ *h'

`In Menominee and Cree, on the other hand, it seems as though the voicessless of the stop or spirant spread backwards through the nasal; at least, a change from a sequence like \phonet{nt} to \phonet{ht} seems easiest to understand if we think of \phonet{n̥t}.'

 
 
  \section{Nasal+stop clusters}
*\ipa{-nk-} $\rightarrow$ \ipa{-kk-} $\rightarrow$ \ipa{-hk-} 
nC to geminate (\citealt{jouon06})
geminate to preaspiration: 
  \citealt[172]{kuemmel07wandel}
  \citealt{silverman03preaspirated}
  
  Khaling (\citealt[44]{jacques16tonogenesis})
  
 \section{*-nr-}
*\ipa{-nr-} $\rightarrow$ *\ipa{-nr̥-} $\rightarrow$ *\ipa{-r̥-} $\rightarrow$ \ipa{-hr-} 

Welsh (\citealt[145]{kuemmel07wandel} r > rh word-initially and after nasal)
\citet[471-480]{jackson56early} \citet{schrijver95}

*\ipa{-nr-} $\rightarrow$ *\ipa{-rr-} $\rightarrow$ *\ipa{-r̥-} $\rightarrow$ \ipa{-hr-} 
Latin in > ir /\_r
\citet{blevins16voiceless.sonorants}

\section{Conclusion}
Independent sound changes in C and M
While there is little doubt that Plain Algonquian languages present complex and unusual sound changes (\citealt{goddard74arapaho,goddard88cheyenne.y,proulx89bf}),  other Algonquian languages are also of great interest to Panchronic phonology. 
\bibliographystyle{unified}
\bibliography{bibliogj}

\end{document}