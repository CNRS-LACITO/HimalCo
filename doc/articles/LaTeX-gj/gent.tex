\documentclass[oneside,a4paper,11pt]{article} 
\usepackage{fontspec}
\usepackage{natbib}
\usepackage{booktabs}
\usepackage{xltxtra} 
\usepackage{polyglossia} 
\usepackage[table]{xcolor}
\usepackage{tikz}
\usetikzlibrary{trees}
\usepackage{gb4e} 
\usepackage{multicol}
\usepackage{graphicx}
\usepackage{float}
\usepackage{hyperref} 
\hypersetup{bookmarks=false,bookmarksnumbered,bookmarksopenlevel=5,bookmarksdepth=5,xetex,colorlinks=true,linkcolor=blue,citecolor=blue}
\usepackage[all]{hypcap}
\usepackage{memhfixc}
\usepackage{lscape} 
 \usepackage{multicol}
%\setmainfont[Mapping=tex-text,Numbers=OldStyle,Ligatures=Common]{Charis SIL} 
\newfontfamily\phon[Mapping=tex-text,Ligatures=Common,Scale=MatchLowercase]{Charis SIL} 
\newcommand{\ipa}[1]{{\phon\textit{#1}}} 
\newcommand{\grise}[1]{\cellcolor{lightgray}\textbf{#1}}
\newfontfamily\cn[Mapping=tex-text,Ligatures=Common,Scale=MatchUppercase]{SimSun}%pour le chinois
\newcommand{\zh}[1]{{\cn #1}}
\newcommand{\Y}{\Checkmark} 
\newcommand{\N}{} 
\newcommand{\zhc}[2]{\zh{#1} \ipa{#2}} 
\newcommand{\refb}[1]{(\ref{#1})}
\newcommand{\tld}{\textasciitilde{}}
\newcounter{exonb}
\newcommand{\exo}[2]{\noindent%
\begin{center}
\begin{minipage}[t]{.8\textwidth}
\vskip 10pt 
\rule{\linewidth}{1pt}  
\vskip 10pt 
\stepcounter{exonb} 
\textbf{\centerline{Exercise \Roman{exonb}}}
\vskip 10pt 
\rule{\linewidth}{1pt}  
\indent
#1:
\vskip 10pt
#2
\rule{\linewidth}{1pt}  
\end{minipage} 
\end{center}
 }
\newcommand{\translit}[1]{$\langle$#1$\rangle$}

 \begin{document} 
\title{Middle Chinese}
\author{Guillaume Jacques\\ CNRS-CRLAO-INALCO}
\maketitle

\section{Introduction}

\subsection{Transcribing vs reconstructing MC}
\subsubsection{Consonants}
\begin{table}[H]
\caption{Early Middle Chinese initial consonants} \label{tab:mc.onset}
\begin{tabular}{llllllllllllllll}
\toprule
 	\zhc{幫}{p} & 	\zhc{滂}{pʰ} & 	\zhc{並}{b} & 	\zhc{明}{m} & 	 & 	 & (\zh{喻}_3/\zhc{云}{w})	 & 	\\
  	\zhc{端}{t} & 	\zhc{透}{tʰ} & 	\zhc{定}{d} & 	\zhc{泥}{n} & 	 & 	 & 	 & 	\\
 	\zhc{知}{ʈ} & 	\zhc{徹}{ʈʰ} & 	\zhc{澄}{ɖ} & 	\zhc{娘}{ɳ} & 	 & 	 & 	 & 	\\
  	 & 	 & 	 & 	 & 	 & 	 & 	\zhc{來}{l} & 	\\
  	\zhc{精}{ts} & 	\zhc{清}{tsʰ} & 	\zhc{從}{dz} & 	 & 	\zhc{心}{s} & 	\zhc{邪}{z} & 	 & 	\\
  	\zh{照}_2/\zhc{莊}{tʂ} & 	\zh{穿}_2/\zhc{初}{tʂʰ} & 	\zh{牀}_2/\zhc{崇}{dʐ} & 	 & 	  	\zh{審}_2/\zhc{生}{ʂ} & 	 \zh{禪}_2\ipa{ʐ} & 	 & 	\\
 	  	\zh{照}_3/\zhc{章}{tɕ} & 	\zh{穿}_3/\zhc{昌}{tɕʰ} & 	\zh{禪}_3/\zhc{禪}{dʑ} & 	\zhc{日}{ɲ} & 	\zh{審}_3/\zhc{書}{ɕ} & 	\zh{牀}_3/\zhc{船}{ʑ} & 	\zh{喻}_4/\zhc{以}{j} & 	\\
 	\zhc{見}{k} & 	\zhc{溪}{kʰ} & 	\zhc{群}{ɡ(j)} & 	\zhc{疑}{ŋ} & 	 & 	 & 	 & 	\\
 	\zhc{影}{ʔ} & 	 & 	 & 	 & 	\zhc{曉}{x} & 	\zhc{匣}{ɣ} & 	 & 	\\
\bottomrule
\end{tabular}
\end{table}

Final consonants: \ipa{-p},  \ipa{-t},  \ipa{-k},  \ipa{-m},  \ipa{-m},  \ipa{-ŋ}, \ipa{-j}, \ipa{-w}.

Baxter's (\citeyear{baxter92}) typeable transcription: 
\begin{enumerate}
\item Retroflex consonants: \ipa{ʈ} = \translit{tr}, \ipa{tʂʰ}  =  \translit{tsrh} 
\item Alveolo-Palatal consonants: \ipa{tɕ} = \translit{tsy}
\item Velars: \ipa{ŋ} = \translit{ng}, \ipa{ɣ} = \translit{h}
\item Glides: \ipa{j} = \translit{y-}, \ipa{w} = \translit{hjw-} (contrast between \zhc{越}{hjwot}  and \zhc{悦}{jwet}).
\end{enumerate}

Marginal initials:
\begin{itemize}
\item 
(\zh{喻}_3/\zh{云} initial only appears with \zhc{合口}{hékǒu} syllables, except for \zhc{焉}{hjen} and \zhc{矣}{hi}).
\item The \ipa{ʐ} initial only attested in two characters (including \zhc{俟}{ʐiX}).
\end{itemize}

\subsubsection{Rhymes}

The four divisions in Baxter's transcription:
\begin{enumerate}
\item \ipa{-a}, \ipa{-o}, \ipa{-u}
\item \ipa{-ɛ}, \ipa{-æ}
\item palatal+V, \ipa{-jV}, \ipa{-i} (V $\in$\{\ipa{a}, \ipa{æ}, \ipa{o}, \ipa{u},  \ipa{i}, \ipa{e}, \ipa{ɨ}\})
\item \ipa{-e}
\end{enumerate}

Medial glides:
\begin{itemize}
\item \ipa{-w} (in \zhc{合口}{hékǒu} syllables, if the main vowel is not \ipa{u} or \ipa{o+w})
\item \zhc{重紐}{chóngniǔ} 3/4: \zhc{岷}{min} vs \zhc{民}{mjin} [\ipa{mʑin}]
\end{itemize}

\subsubsection{Tones}

\begin{tabular}{lll} 
\zhc{平}{bjæŋ} & \zhc{端}{twan} \\
\zhc{上}{dʑaŋX} & \zhc{短}{twanX} \\
\zhc{去}{kʰjoH} & \zhc{段}{twanH} \\
\zhc{入}{ɲip} & \zhc{掇}{twat} \\
\end{tabular}

 \exo{Indicate the initial consonant, the division, the rhyme and the tone of the following words}{
\begin{multicols}{2}
\zhc{睹}{tuX}

\zhc{塵}{ɖin} %ɖ, 3, -in, ping

\zhc{炒}{tʂʰæwX} %tʂʰ, 2, -æw, shang

\zhc{朝}{ɖjew} %ɖ, 3, -jew, ping

\zhc{吹}{tɕʰwe} %tɕʰ, 3, -w-, -je, ping

\zhc{畜}{ʈʰjuwk} 

\zhc{鄧}{doŋH} 

\zhc{妃}{pʰjɨj} 

\zhc{範}{bjomX} 

\zhc{彪}{pjiw} 

\zhc{餅}{pjieŋX} 

\zhc{兒}{ɲe} 

\zhc{競}{gjæŋH}

\zhc{脈}{mɛk} 
 
 \zhc{汪}{ʔwaŋ} 
 
\zhc{驗}{ŋjemH} 
  
  \zhc{王}{hjwaŋ} 
  
   \zhc{緣}{jwen}

\end{multicols}}
\subsection{Complementary distributions}

 \begin{table}[H]
 \caption{Complementary distribution between initial consonants and divisions, rhymes with \ipa{-ŋ} coda} \centering
\begin{tabular}{l|llllll}
\toprule
&\zhc{搨}{-aŋ}& \zhc{庚}{-æŋ} & \zhc{陽}{-jaŋ} &\zhc{青}{-eŋ} \\
\midrule
\zhc{端}{t} 	&	\zhc{當}{taŋ} 	& \grise{}	& \grise{}	&	\zhc{丁}{teŋ}  \\
\zhc{知}{ʈ} 	&	\grise{}&\zhc{趟}{ʈæŋ} 	&	\zhc{張}{ʈjaŋ} 	&	\grise{}	&		\\
\zhc{精}{ts} 	&	\zhc{臧}{tsaŋ} &\grise{}	&	\zhc{將}{tsjaŋ} 	&	\zhc{菁}{tseŋ} 	&		\\
\zhc{莊}{tʂ} 	&	\grise{}&\zhc{鎗}{tʂʰæŋ} 	& \zhc{莊}{tʂjaŋ} 	&		\grise{}&		\\
\zhc{章}{tɕ} 	&		\grise{}&		\grise{}&	\zhc{章}{tɕaŋ} 	&		\grise{} 		\\
\zhc{幫}{p} 	&	\zhc{幫}{paŋ} 	&	\zhc{彭}{bæŋ} 	&	\zhc{方}{pjaŋ} 	&	\zhc{瓶}{beŋ}\\
\zhc{見}{k} 	&	\zhc{剛}{kaŋ} 	&	\zhc{庚}{kæŋ} 	&	\zhc{薑}{kjaŋ} 	&	\zhc{經}{keŋ} 	\\
\bottomrule
\end{tabular}
\end{table}


 \begin{table}[H]
 \caption{Complementary distribution between initial consonants and divisions, rhymes with \ipa{-w} coda} \centering
\begin{tabular}{l|llllll}
\toprule
&\zhc{豪}{-aw}& \zhc{爻}{-æw} & \zhc{宵}{-jew} &\zhc{萧}{-ew} \\
\midrule
\zhc{端}{t} 	&	\zhc{刀}{taw} 	& \grise{}	& \grise{}	&	\zhc{雕}{tew}  \\
\zhc{知}{ʈ} 	&	\grise{}&\zhc{啁}{ʈæw} 	&	\zhc{朝}{ʈjew} 	&	\grise{}	&		\\
\zhc{精}{ts} 	&	\zhc{遭}{tsaw} &\grise{}	&	\zhc{焦}{tsjew} 	&	\zhc{湫}{tsewX} 	&		\\
\zhc{莊}{tʂ} 	&	\grise{}&\zhc{抓}{tʂæw} 	& X	&		\grise{}&		\\
\zhc{章}{tɕ} 	&		\grise{}&		\grise{}&	\zhc{昭}{tɕew} 	&		\grise{} 		\\
\zhc{幫}{p} 	&	\zhc{褒}{paw} 	&	\zhc{包}{pæw} 	&	\zhc{膘}{pjew} 	&	X	\\
\zhc{見}{k} 	&	\zhc{高}{kaw} 	&	\zhc{交}{kæw} 	&	\zhc{驕}{kjew} 	&	\zhc{澆}{kew} 	\\
%\zhc{來}{l} 	&	\zhc{牢}{law} 	&	\grise{}	&	\zhc{療}{ljewH} 	&	\zhc{寥}{lew} 	\\
\bottomrule
\end{tabular}
\end{table}

Other particularities:

\begin{itemize}
\item Some rhymes only occur with labial or dorsal initials (\ipa{-jæŋ} for instance)
\item \zhc{來}{l}: nearly no examples in division II.
\end{itemize}
\textbf{Some important exceptions}: \zhc{地}{dijH}, \zhc{打}{tæŋX}, \zhc{冷}{læŋX}


 \exo{Indicate which of the following syllables are impossible in Middle Chinese and why}{ 
\begin{multicols}{3}
\ipa{tʰɛw}

\ipa{tɕæŋ}

\ipa{tʰjij}

\ipa{ɖaŋ}

\ipa{ju}

\ipa{tɕaw}

\ipa{tʂi}

\ipa{ʂu}

\ipa{dʐew}

\ipa{dʐaŋ}

\ipa{bjew}

\ipa{zæw}

\ipa{tɕʰu}

\ipa{ʈu}

\ipa{ʈju}

\end{multicols}}
\section{From MC to Mandarin}

\subsection{Manner of articulation and tone}


 

\subsection{How much of MC can we predict from Mandarin?}

\section{Rhyme table philology} 

\subsection{\zhc{反切}{fǎnqiè}} \label{sec:fanqie}

\zhc{陳澧}{Chén lǐ}'s \zhc{繫聯法}{xìliánfǎ}

\subsubsection{\zhc{同用}{tóngyòng}}
\begin{itemize}
\item \zhc{布}{puH} = \zhc{博}{pak}+\zhc{故}{kuH}
\item \zhc{北}{pok} = \zhc{博}{pak}+\zhc{墨}{mok}
\item \zhc{補}{puX} = \zhc{博}{pak}+\zhc{古}{kuX}
\end{itemize}
$\Rightarrow$ \zhc{布}{puH} $\equiv$ \zhc{北}{pok} $\equiv$ \zhc{補}{puX} $\equiv$  \zhc{博}{pak}
\subsubsection{\zhc{互用}{hùyòng} (Symmetry)}
\begin{itemize}
\item \zhc{補}{puX} = \zhc{博}{pak}+\zhc{古}{kuX}
\item \zhc{博}{pak} =  \zhc{補}{puX}+\zhc{各}{kak}
\end{itemize}
$\Rightarrow$  \zhc{補}{puX} $\equiv$  \zhc{博}{pak}
\subsubsection{\zhc{遞用}{dìyòng} (Transitivity)}
\begin{itemize}
\item \zhc{邊}{pen} = \zhc{博}{pak}+\zhc{古}{kuX}
\item \zhc{博}{pak} =  \zhc{補}{puX}+\zhc{各}{kak}
\end{itemize}
$\Rightarrow$  \zhc{邊}{pen} $\equiv$ \zhc{補}{puX} $\equiv$  \zhc{博}{pak}

NB: A fanqie character is never used as its own \zhc{反切上字}{fǎnqiè shàngzì}.
 \exo{Establish the \zhc{聲類}{shēnglèi} from the following \zhc{反切}{fǎnqiè}}{
{\cn  
\begin{multicols}{3}
\noindent
來洛哀切
盧洛乎切
洛盧各切
魯郎古切
郎魯當切
浪來宕切
磊落猥切
落盧各切
掠離灼切
林力尋切
離呂支切
呂力舉切
良呂張切
力林直切
槤里典切
林力尋切
里良止切
\end{multicols}}}


\subsection{Rhyme tables} \label{sec:yuntu}

\subsection{Rhyming and \zhc{平仄}{píngzè}} \label{sec:pingze}

\section{Chinese Transcriptions of Sanskrit}
\bibliographystyle{unified}
\bibliography{bibliogj}

 \end{document}
 