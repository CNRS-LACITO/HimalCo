\documentclass[oneside,a4paper,11pt]{article} 
\usepackage{fontspec}
\usepackage{natbib}
\usepackage{booktabs}
\usepackage{xltxtra} 
\usepackage{polyglossia} 
\usepackage[table]{xcolor}
\usepackage{tikz}
\usetikzlibrary{trees}
\usepackage{gb4e} 
\usepackage{multicol}
\usepackage{graphicx}
\usepackage{float}
\usepackage{hyperref} 
\hypersetup{bookmarks=false,bookmarksnumbered,bookmarksopenlevel=5,bookmarksdepth=5,xetex,colorlinks=true,linkcolor=blue,citecolor=blue}
\usepackage[all]{hypcap}
\usepackage{memhfixc}
\usepackage{lscape}
\usepackage{bbding}
 
%\setmainfont[Mapping=tex-text,Numbers=OldStyle,Ligatures=Common]{Charis SIL} 
\newfontfamily\phon[Mapping=tex-text,Ligatures=Common,Scale=MatchLowercase]{Charis SIL} 
\newcommand{\ipa}[1]{{\phon\textbf{#1}}} 
\newcommand{\grise}[1]{\cellcolor{lightgray}\textbf{#1}}
\newfontfamily\cn[Mapping=tex-text,Ligatures=Common,Scale=MatchUppercase]{SimSun}%pour le chinois
\newcommand{\zh}[1]{{\cn #1}}
\newcommand{\Y}{\Checkmark} 
\newcommand{\N}{} 
\newcommand{\zhc}[2]{\zh{#1} \ipa{#2}} 
\newcommand{\refb}[1]{(\ref{#1})}
\newcommand{\tld}{\textasciitilde{}}

 \begin{document} 
\title{Middle Chinese}
\author{Guillaume Jacques\\ CNRS-CRLAO-INALCO}
\maketitle

\section{Introduction}

\subsection{Consonants}
\begin{table}[H]
\caption{Early Middle Chinese initial consonants} \label{tab:mc.onset}
\begin{tabular}{llllllllllllllll}
\toprule
 	\zhc{幫}{p} & 	\zhc{滂}{pʰ} & 	\zhc{並}{b} & 	\zhc{明}{m} & 	 & 	 & (\zh{喻}_3/\zhc{云}{w})	 & 	\\
  	\zhc{端}{t} & 	\zhc{透}{tʰ} & 	\zhc{定}{d} & 	\zhc{泥}{n} & 	 & 	 & 	 & 	\\
 	\zhc{知}{ʈ} & 	\zhc{徹}{ʈʰ} & 	\zhc{澄}{ɖ} & 	\zhc{娘}{ɳ} & 	 & 	 & 	 & 	\\
  	 & 	 & 	 & 	 & 	 & 	 & 	\zhc{來}{l} & 	\\
  	\zhc{精}{ts} & 	\zhc{清}{tsʰ} & 	\zhc{從}{dz} & 	 & 	\zhc{心}{s} & 	\zhc{邪}{z} & 	 & 	\\
  	\zh{照}_2/\zhc{莊}{tʂ} & 	\zh{穿}_2/\zhc{初}{tʂʰ} & 	\zh{牀}_2/\zhc{崇}{dʐ} & 	 & 	  	\zh{審}_2/\zhc{生}{ʂ} & 	 \zh{禪}_2\ipa{ʐ} & 	 & 	\\
 	  	\zh{照}_3/\zhc{章}{tɕ} & 	\zh{穿}_3/\zhc{昌}{tɕʰ} & 	\zh{禪}_3/\zhc{禪}{dʑ} & 	\zhc{日}{ɲ} & 	\zh{審}_3/\zhc{書}{ɕ} & 	\zh{牀}_3/\zhc{船}{ʑ} & 	\zh{喻}_4/\zhc{以}{j} & 	\\
 	\zhc{見}{k} & 	\zhc{溪}{kʰ} & 	\zhc{群}{ɡ} & 	\zhc{疑}{ŋ} & 	 & 	 & 	 & 	\\
 	\zhc{影}{ʔ} & 	 & 	 & 	 & 	\zhc{曉}{x} & 	\zhc{匣}{ɣ} & 	 & 	\\
\bottomrule
\end{tabular}
\end{table}

Final consonants: \ipa{-p},  \ipa{-t},  \ipa{-k},  \ipa{-m},  \ipa{-m},  \ipa{-ŋ}, \ipa{-j}, \ipa{-w}.

Baxter's (\citeyear{baxter92}) typeable transcription: 
\begin{enumerate}
\item Retroflex consonants: \ipa{ʈ} = <tr>, \ipa{tʂʰ}  = <tsrh>
\item Alveolo-Palatal consonants: \ipa{tɕ} = <tsy>
\item Velars: \ipa{ŋ} = <ng>, \ipa{ɣ} = <h>
\item Glides: \ipa{j} = \ipa{y-}, \ipa{w} = \ipa{hjw-} (contrast between \zhc{越}{hjwot}  and \zhc{悦}{jwet}).
\end{enumerate}

Marginal initials:
\begin{itemize}
\item 
(\zh{喻}_3/\zh{云} initial only appears with \zhc{合口}{hékǒu} syllables, except for \zhc{焉}{hjen} and \zhc{矣}{hi}).
\item The \ipa{ʐ} initial only attested in two characters (including \zhc{俟}{ʐiX}).
\end{itemize}

\subsection{Rhymes}

The four divisions in Baxter's transcription:
\begin{enumerate}
\item \ipa{-a}, \ipa{-o}, \ipa{-u}
\item \ipa{-ɛ}, \ipa{-æ}
\item \ipa{-jV}, \ipa{-i} (V $\in$\{\ipa{a}, \ipa{æ}, \ipa{o}, \ipa{u},  \ipa{i}, \ipa{e}, \ipa{ɨ}\})
\item \ipa{-e}
\end{enumerate}

\zhc{重紐}{chóngniǔ} 3/4: \zhc{岷}{min} vs \zhc{民}{mjin}

\subsection{Tones}

\subsection{Complementary distributions}

\zhc{地}{dijH}, \zhc{打}{tæŋX}, \zhc{冷}{læŋX}

\section{From MC to Mandarin}

\subsection{Manner of articulation and tone}

\section{Rhyme table philology} 

\subsection{\zhc{反切}{fǎnqiè}} \label{sec:fanqie}

\zhc{陳澧}{Chén lǐ}'s \zhc{繫聯法}{xìliánfǎ}

\subsubsection{\zhc{同用}{tóngyòng} (Reflexivity)}

\begin{itemize}
\item \zhc{布}{buH} = \zhc{博}{pak}+\zhc{故}{kuH}
\end{itemize}

\subsubsection{\zhc{互用}{hùyòng} (Symmetry)}
\subsubsection{\zhc{遞用}{dìyòng} (Transitivity)}


\subsection{Rhyme tables} \label{sec:yuntu}

\subsection{Rhyming and \zhc{平仄}{píngzè}} \label{sec:pingze}

\section{Chinese Transcriptions of Sanskrit}
\bibliographystyle{unified}
\bibliography{bibliogj}

 \end{document}
 