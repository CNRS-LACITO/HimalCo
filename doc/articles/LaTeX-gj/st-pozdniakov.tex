\documentclass[oldfontcommands,oneside,a4paper,11pt]{article} 
\usepackage{fontspec}
\usepackage{natbib}
\usepackage{booktabs}
\usepackage{xltxtra} 
\usepackage{longtable}
\usepackage{polyglossia} 
\usepackage[table]{xcolor}
\usepackage{gb4e} 
\usepackage{multicol}
\usepackage{graphicx}
\usepackage{float}
\usepackage{hyperref} 
\usepackage{lineno}
\hypersetup{bookmarks=false,bookmarksnumbered,bookmarksopenlevel=5,bookmarksdepth=5,xetex,colorlinks=true,linkcolor=blue,citecolor=blue}
\usepackage[all]{hypcap}
\usepackage{memhfixc}
\usepackage{lscape}

\bibpunct[: ]{(}{)}{,}{a}{}{,}

%\setmainfont[Mapping=tex-text,Numbers=OldStyle,Ligatures=Common]{Charis SIL} 
\newfontfamily\phon[Mapping=tex-text,Ligatures=Common,Scale=MatchLowercase,FakeSlant=0.3]{Charis SIL} 
\newcommand{\ipa}[1]{{\phon \mbox{#1}}} %API tjs en italique
\newcommand{\ipab}[1]{{\scriptsize \phon#1}} 

\newcommand{\grise}[1]{\cellcolor{lightgray}\textbf{#1}}
\newfontfamily\cn[Mapping=tex-text,Ligatures=Common,Scale=MatchUppercase]{MingLiU}%pour le chinois
\newcommand{\zh}[1]{{\cn #1}}
\newcommand{\refb}[1]{(\ref{#1})}

\newcommand{\ra}{$\Sigma_1$} 
\newcommand{\rc}{$\Sigma_3$} 
\newcommand{\ro}{$\Sigma$} 

\XeTeXlinebreaklocale 'zh' %使用中文换行
\XeTeXlinebreakskip = 0pt plus 1pt %
 %CIRCG
 


\begin{document} 
\title{Le sino-tibétain: polysynthétique ou isolant?  }
%\author{Guillaume Jacques}
\maketitle
\linenumbers

\section{Introduction}
Un des problèmes les plus fondamentaux et des plus difficiles de la linguistique historique du sino-tibétain est la question de reconstruction de la morphologie. En effet, la famille sino-tibétaine est peut-être celle présentant la plus grande diversité typologique de toute les langues du monde. A côté de langues presque prototypiquement isolantes, telles que le chinois, le karen ou le lolo-birman, on trouve des langues polysynthétiques comme le rgyalronguique ou le kiranti. Il n'y a aucune particularité typologique non-triviale qui soit partagée par toutes les langues de la famille. 

Certains auteurs tels que  \citet{lapolla03} proposent que certains affixes de voix peuvent être d'origine proto-sino-tibétaine, mais qu'en revanche les marques d'indexation personnelle sur le verbe dans des groupes de langues tels que le rgyalronguique et le kiranti sont des innovations qui ne peuvent pas être reconstruite jusqu'à la proto-langue. D'autres, tels que  \citet{driem93agreement}  et \citet{delancey10agreement}, proposent au contraire que ce sont les langues rgyalronguiques et kiranties qui sont conservatives, et celles à morphologie réduite qui ont innové en perdant leurs marques d'indexation sont laisser de trace.


Un débat de ce type n'est pas restreint au sino-tibétain. En niger-congo, une autre macro-famille, une controverse similaire a lieu (voir  \citealt{guldeman08macrosudan} et  \citealt{hyman11macrosudan}). 

\section{Rgyalronguique}

\subsection{Le gabarit verbal des langues rgyalronguiques}

\citealt{jacques12agreement}

\citealt{jacques12incorp}
\citealt{jacques14antipassive}
\citealt{jacques13harmonization}
\citealt{jacques15causative}
\citealt{lai13affixale}

\subsection{Rgyalronguique}

\section{Kiranti}
\citealt{jacques12khaling}
\citealt{jacques15derivational.khaling}
\section{Structure du système d'accord}

\citealt{jacques12agreement}

\section{Ordre des mots}

\section{Conclusion}
  
\bibliographystyle{unified}
\bibliography{bibliogj}
\end{document}