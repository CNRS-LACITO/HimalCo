\documentclass[oldfontcommands,oneside,a4paper,11pt]{article} 
\usepackage{fontspec}
\usepackage{natbib}
\usepackage{booktabs}
\usepackage{xltxtra} 
\usepackage{polyglossia} 
\setdefaultlanguage{french}
\usepackage[table]{xcolor}
\usepackage{gb4e} 
\usepackage{multicol}
\usepackage{graphicx}
\usepackage{lineno}
\usepackage{float}
\usepackage{hyperref} 
\hypersetup{bookmarks=false,bookmarksnumbered,bookmarksopenlevel=5,bookmarksdepth=5,xetex,colorlinks=true,linkcolor=blue,citecolor=blue}
\usepackage[all]{hypcap}
\usepackage{memhfixc}
\usepackage{lscape}

\bibpunct[: ]{(}{)}{,}{a}{}{,}

%\setmainfont[Mapping=tex-text,Numbers=OldStyle,Ligatures=Common]{Charis SIL} 
\newfontfamily\phon[Mapping=tex-text,Ligatures=Common,Scale=MatchLowercase,FakeSlant=0.3]{Charis SIL} 
\newcommand{\ipa}[1]{{\phon \mbox{#1}}} %API tjs en italique
\newcommand{\ipab}[1]{{\scriptsize \phon#1}} 

\newcommand{\grise}[1]{\cellcolor{lightgray}\textbf{#1}}
\newfontfamily\cn[Mapping=tex-text,Ligatures=Common,Scale=MatchUppercase]{MingLiU}%pour le chinois
\newcommand{\zh}[1]{{\cn #1}}
\newcommand{\refb}[1]{(\ref{#1})}

\newcommand{\ra}{$\Sigma_1$} 
\newcommand{\rc}{$\Sigma_3$} 
\newcommand{\ro}{$\Sigma$} 

\XeTeXlinebreaklocale 'zh' %使用中文换行
\XeTeXlinebreakskip = 0pt plus 1pt %
 %CIRCG
 
\sloppy

\begin{document} 
\title{Le sino-tibétain: polysynthétique ou isolant?\footnote{Je souhaite remercier un relecteur anonyme pour ses commentaires. Le corpus japhug est disponible sur le site Pangloss (\citealt{michailovsky14pangloss}). Cette recherche a été financée par le projet HimalCo (ANR-12-CORP-0006) et l'opération de recherche LR-4.11 ‘‘Automatic Paradigm Generation and Language Description’’ du Labex EFL (financé par l'ANR/CGI).
Les abréviations utilisées sont les suivantes: \textsc{auto} autobénéfactif, \textsc{dem} démonstratif, \textsc{fact} factuel, \textsc{irr} irréel, \textsc{pfv} perfectif, \textsc{pst} passé, \textsc{refl} réfléchi,  \textsc{sens} sensoriel, \textsc{transloc} translocatif. Les termes `dé-expérienceur' et `tropatif' sont  définis dans \citet{jacques12demotion} et \citet{jacques13tropative} respectivement. }  }
\author{Guillaume Jacques}
\maketitle
%\linenumbers

\section{Introduction}
Un des problèmes les plus fondamentaux et des plus difficiles de la linguistique historique du sino-tibétain est la question de reconstruction de la morphologie. En effet, la famille sino-tibétaine est peut-être celle présentant la plus grande diversité typologique de toutes les langues du monde. A côté de langues presque prototypiquement isolantes, telles que le chinois, le karen ou le lolo-birman, on trouve des langues polysynthétiques comme le rgyalronguique ou le kiranti. On trouve aussi des langues d'un degré de complexité morphologique intermédiaire, comme le tibétain ancien, qui bien que dépourvu d'indexation personnelle, a une morphologie très irrégulière et synchroniquement opaque\footnote{Voir \citet{hill05vbri, jacques10ndr, jacques12internal, hill14dempsey, hill14voicing, hill15lan} concernant la reconstruction interne du verbe tibétain.}

 


 Il n'y a aucune particularité typologique non-triviale qui soit partagée par toutes les langues de la famille. 

Certains auteurs tels que  \citet{lapolla03} proposent que quelques affixes de voix peuvent être d'origine proto-sino-tibétaine, mais qu'en revanche les marques d'indexation personnelle sur le verbe dans des groupes de langues tels que le rgyalronguique et le kiranti sont des innovations qui ne peuvent pas être reconstruite jusqu'à la proto-langue. D'autres, tels que  \citet{driem93agreement}  et \citet{delancey10agreement}, proposent au contraire que ce sont les langues rgyalronguiques et kiranties qui sont conservatrices, et celles à morphologie réduite qui ont innové en perdant leurs marques d'indexation sans laisser de trace.

Un débat de ce type n'est pas restreint au sino-tibétain. En niger-congo, une autre macro-famille, une controverse similaire a lieu (voir  \citealt{guldeman08macrosudan} et  \citealt{hyman11macrosudan}). 

Il est trop tôt pour prétendre apporter une réponse définitive à cette controverse. Le présent travail se propose un but moins ambitieux: évaluer, dans le cas d'une des langues à la morphologie la plus complexe de la famille, quelle proportion de cette morphologie est transparente et récemment grammaticalisée, et quel résidu de formes potentiellement anciennes peuvent se comparer raisonnablement au kiranti ou à d'autres branches du sino-tibétain. 

Cette tâche est un préliminaire indispensable à tout travail plus général sur la famille, et des études comparables seront nécessaires à terme  sur toutes les langues à morphologie riche de la famille.


Ce travail comportera tout d'abord une présentation générale du gabarit verbal des langues rgyalronguiques, suivi, pour plusieurs domaines de ce gabarit (marques de mouvement associé, de voix, de TAM et d'indexation), d'une évaluation des innovations évidentes et des archaïsmes potentiels.

\section{Le gabarit verbal des langues rgyalronguiques}
Les langues rgyalronguiques sont connues pour leur morphologie polysynthétique très complexe (\citealt{jacques12incorp}, \citealt{lai13affixale}, \citealt{jackson14morpho}) et typologiquement inhabituelle. En effet, avec les langues athabasques, sioux (et à moindre mesure le iénisséen  et le caucasique du nord-ouest), les langues rgyalronguiques font partie des très rares langues langues majoritairement préfixante et strictement verbe-finale (\citealt{jacques13harmonization}), une anomalie typologique d'autant plus intéressante dans une perspective diachronique.
 
N'est pas rare de trouver dans les textes des formes verbales pourvues de plus de quatre préfixes, telles que \ref{ex:maCthWtWZGAbde} ou \ref{ex:amaCtAznWsnWYaR}. 


\begin{exe}
\ex \label{ex:maCthWtWZGAbde}
\gll \ipa{ma-ɕ-thɯ-tɯ-ʑɣɤ-βde}  	\ipa{ma}  	\ipa{nɤ-pi}  	\ipa{ɲɯ-ɤkhu}  \\
\textsc{neg-transloc-imp-2-refl}-jeter car \textsc{2sg.poss}-frère.aîné \textsc{sens}-appeler \\
\glt Ne vas pas te jeter (dans l'eau), ton frère t'appelle. (le corbeau, 25)
\end{exe}

%\begin{exe}
%\ex  \label{ex:amanWtWnWjmWt}
%\gll
%\ipa{tɤ-tɯ-nɯ-tɯt}  	\ipa{nɯ}  	\ipa{a-mɤ-nɯ-tɯ-nɯ-jmɯt}  	\ipa{ra}  \\
%\textsc{\textsc{pfv}-2-auto}-dire[II] \textsc{dem} \textsc{irr-neg-\textsc{prf}-2-auto}-oublier devoir:\textsc{fact} \\
%\glt N'oublie pas ce que tu (m')avais dit! (le loup et le chien, 25)
%\end{exe}

\begin{exe}
\ex  \label{ex:amaCtAznWsnWYaR}
\gll
\ipa{tɯrme}  	\ipa{ra}  	\ipa{tɕe}  	\ipa{nɯ}  	\ipa{ma}  	\ipa{a-mɤ-ɕ-tɤ-z-nɯ-snɯ-ɲaʁ}  	\ipa{ra}  \\
homme \textsc{pl} \textsc{lnk} \textsc{dem} à.part \textsc{irr-neg-transloc-pfv-caus-denom}-cœur-être.noir devoir:\textsc{fact} \\
\glt Qu'on ne la laisse plus aller faire du mal aux gens! (fushang he yaomo, 162)
\end{exe}

Aucune forme verbale ayant plus de six préfixes et un nom incorporé n'est attestée dans les textes. Plus qu'une réelle contrainte sur la complexité des formes, cette absence est fortuite; il est possible de construire des formes verbales plus complexes. A partir de \ref{ex:amaCtAznWsnWYaR}, on peut par exemple produire la forme suivante avec huit préfixes et un nom incorporé:

\begin{exe}
\ex  \label{ex:amaCtAtWwGznWsnWYaR}
\gll
	\ipa{a-mɤ-ɕ-tɤ-tɯ́-wɣ-z-nɯ-snɯ-ɲaʁ}  	\ipa{ra}  \\
  \textsc{irr-neg-transloc-pfv-2-inv-caus-denom}-cœur-être.noir devoir:\textsc{fact} \\
\glt Qu'on ne le laisse pas aller te faire du mal!
\end{exe}

La structure potentielle maximale du verbe japhug est présentée dans le tableau \ref{tab:gabarit}.


   \begin{landscape}
\begin{table}[h]
\caption{Le gabarit verbal du japhug (les cases grisée indiquent les préfixes directionnels)}\label{tab:gabarit}
\begin{tabular}{llllll|llllllll|lllll} \toprule
 
\ipab{a-}  &  	\ipab{mɯ- }   &  	\ipab{ɕɯ-}   &\ipab{tɤ-} &  	\ipab{tɯ-}  &  	\ipab{wɣ-}   &

  	 \grise{\ipab{ʑɣɤ-}}  &  	\grise{\ipab{sɯ-}}  & \grise{\ipab{rɤ-}}& \grise{\ipab{nɤ-}} &   	 \grise{\ipab{a-}}   &  	\grise{\ipab{nɯ-}}  &  	\grise{\ipab{ɣɤ-}}  &  	\grise{\ipab{noun}}    &  	 \begin{math}\Sigma\end{math}    &  	\ipab{-t}  &  	\ipab{-a}  &  	\ipab{-nɯ}   &  \\
   &  	\ipab{mɤ-}   &  	\ipab{ɣɯ-}   &\ipab{pɯ-}&  	  &  	 
    & \grise{ }	  &  	 \grise{ }	  &  	  \grise{ }	  &  	   \grise{ }	&  	\grise{\ipab{sɤ-}}&  \grise{ }	 &  	\grise{\ipab{rɯ-}}  &  	 \grise{ }	  &  	  &  	  &  	  &  	\ipab{-ndʑi} &  \\
  &  	   &     &  etc.	  & & 	  &  	  &  	 & &  	  &  	 & &  etc.	  &  	  &  	  &  	  &  	  &  	  &  \\
1  &  	2  &  	3  &  	4  &  	5  &  	6  &  	7  &  	8  &  	9  &  	10  &  	11  &  	12  &  	13  &  	14  &  	15  & 16 &17&18\\
\bottomrule
\end{tabular}
\end{table}
\begin{multicols}{2}
\begin{enumerate}


\item irréel  \ipa{a}- and \ipa{ɯβrɤ}-, interrogatif \ipa{ɯ́}-, conatif \ipa{jɯ}-
\item négation \ipa{ma}- / \ipa{mɤ}- / \ipa{mɯ}- / \ipa{mɯ́j}-
\item  mouvement associé  \ipa{ɕɯ}- and \ipa{ɣɯ}- 
\item préfixes directionnels (\ipa{tɤ}-  \ipa{pɯ}-  \ipa{lɤ}-   \ipa{tʰɯ}-  \ipa{kɤ}-   \ipa{nɯ}-   \ipa{jɤ}- ,  \ipa{tu}-   \ipa{pjɯ}-   \ipa{lu}-   \ipa{cʰɯ}-   \ipa{ku}-   \ipa{ɲɯ}-   \ipa{ju}-)   appréhensif \ipa{ɕɯ}-
\item deuxième personne (\ipa{tɯ}-, \ipa{kɯ}- 2$\rightarrow$1 and\ipa{ta-} 1$\rightarrow$2)
\item inverse -\ipa{wɣ}- / générique S/O \ipa{kɯ}-, progressif \ipa{asɯ}-. 
\item réfléchi \ipa{ʑɣɤ}- 
\item causatif \ipa{sɯ}-, abilitatif \ipa{sɯ}-
\item  antipassif  \ipa{sɤ}- / \ipa{rɤ}-
\item  tropatif \ipa{nɤ}-, applicatif \ipa{nɯ}-
\item passif \ipa{a}- / déexperienceur \ipa{sɤ}-
\item autobénéfactif
\item autres préfixes dérivationnels \ipa{nɯ}- \ipa{ɣɯ}- \ipa{rɯ}- \ipa{nɤ}- \ipa{ɣɤ}- \ipa{rɤ}-
\item racine nominale
\item racine verbale
\item passé 1sg/2sg transitif -\ipa{t} 
\item 1sg -\ipa{a}
\item autres suffixes d'indexation personnelle (-\ipa{tɕi}, -\ipa{ji}, -\ipa{nɯ}, -\ipa{ndʑi})
\end{enumerate}


\end{multicols}
  \end{landscape}

 

%\section{Kiranti}
%\citealt{jacques12khaling}
%\citealt{jacques15derivational.khaling}


\section{Négation} 
Certains auteurs (tels que \citealt{lapolla03}) proposent de reconstruire des préfixes négatifs pour la proto-langue. Si les langues conservatrices de la famille ont en effet toutes des préfixes de négations qui semblent apparentés, ces préfixes correspondent à des marques de négations indépendantes en chinois (\citealt{djamouri91negation}) ou en lolo-birman. 

A moins de supposer un phénomène de dégrammaticalisation dans ces langues, il est préférable d'admettre que les préfixes de négation dérivent d'anciens adverbes négatifs; en l'absence de morphologique  commune entre branches éloignées (comme celle commune au pumi et au tangoute, voir \citealt{jacques11tangut.verb}), l'hypothèse nulle est que le système de préfixes négation s'est développé de façon indépendante à travers la famille.

\section{Mouvement associé} 
En comparaison avec les langues tacananes (\citealt{guillaume09mouv.assoc}) ou pama-nyounganes (\citealt{koch84associated.motion}) où cette notion a été développée, le système de mouvement associé en japhug est relativement simple, ne comprenant que deux préfixes, les cislocatif \ipa{ɣɯ-} et le translocatif \ipa{ɕɯ-}.

Ces deux préfixes sont grammaticalisés des verbes de mouvement `venir' \ipa{ɣi} et `aller' \ipa{ɕe} respectivement, à parti d'une construction paratactique ou en série, et non d'une construction à complément de but (\citealt{jacques13harmonization}).

La place des préfixes de mouvement associé diffère selon les langues rgyalronguiques: ils apparaissent après les marques de négation mais avant les préfixes directionnels en japhug et après ceux-ci en situ. Néanmoins, on ne peut pour autant en conclure que  mouvement associé est un développement indépendant en rgyalrong du nord et en situ. En effet, on trouve un verbe irrégulier transitif \ipa{-ru} `aller chercher', qui requiert l'emploi d'un préfixe de mouvement associé dans toutes les langues où il a été décrit, comme l'illustrent les exemples suivants:

\begin{exe}
 \ex  
 \gll  \ipa{\textbf{ɕ}-tɤ-ru-t-a}  (Japhug) \\
 \textsc{transloc-pfv:haut}-aller.chercher-\textsc{1sg} \\
\ex  
 \gll  \ipa{rə-\textbf{ɕɐ}-rô-ŋ}  (Situ) \\
 \textsc{pfv:haut-transloc}-aller.chercher-\textsc{1sg} \\
\glt Je suis allé le chercher.
\end{exe}
 Si la grammaticalisation des verbes de mouvement avait eu lieu de façon indépendante en situ et en japhug, on ne pourrait s'attendre à l'existence d'un verbe irrégulier de ce type, d'autant qu'il présente les correspondances phonétiques régulières d'un mot hérité.

En revanche, l'absence de marque comparable, y compris y khroskyabs (\citealt{lai13affixale}), langue pourtant très proche, indique qu'il doit s'agir d'une innovation restreinte aux quatre langues rgyalrong (japhug, tshobdun, zbu et situ).

\section{Système de voix}

Le japhug et les autres langues rgyalronguiques ont des systèmes de voix très riches, presque exclusivement marqué par des préfixes (voir  \citealt{lai13affixale, jacques14antipassive, jackson14morpho} pour des descriptions générales). Le seul suffixe de voix est l'applicatif en \ipa{-t} (à propos duquel voir \citealt{michailovsky85dental, jacques15derivational.khaling} en kiranti), qui n'apparaît que dans la paire \ipa{ɣi} `venir' /  \ipa{ɣɯt} `apporter' (< *\ipa{wit}).

A l'exception de ce suffixe, de la prénasalisation anticausative et de l'autobénéfactif (\citealt{jacques12demotion, jacques15spontaneous}), on peut démontrer que tous ces préfixes sont secondaires.

Le préfixe réfléchi \ipa{ʑɣɤ-} du japhug peut s'expliquer comme résultant de l'incorporation du pronom de troisième personne singulier *\ipa{wəjaŋ} sous sa forme réduite (\citealt{jacques10refl}; voir  \citealt{jackson14morpho} pour une explication alternative). Ce préfixe n'est même pas cognat avec le réfléchi \ipa{ʁjæ-} du khroskyabs  (\citealt[156-7]{lai13affixale}), et il s'agit d'une innovation définitoire du groupe rgyalrong (japhug, tshobdun, zbu, situ).

Les autres marques de voix (passive, antipassif, causatif, applicatif), comme l'indique le tableau  \ref{tab:denominal}, présentent toutes une ressemblance à la fois formelle et sémantique avec une série de préfixes dénominaux. Comme il est possible de le montrer en utilisant la morphologie irrégulière (\citealt{jacques14antipassive}), cette ressemblance peut s'expliquer par un chemin de grammaticalisation en deux étapes. 

Tout d'abord, un verbe à l'origine transitif ou intransitif est nominalisé au moyen de l'infinitif nu (marqué uniquement par un préfixe possessif coréférent avec le S/P). Ce passag par un stade de nom d'action neutralise la transitivité du verbe. Ensuite, ce verbe nominalisé reçoit une marque de dénominalisation, et redevient verbal, avec une nouvelle transitivité déterminée par le préfixe dénominal. 

Ce chemin de grammaticalisation rend compte de l'origine de l'antipassif et de l'applicatif, qui semblent être des innovations spécifiquement rgyalrongs, pas même partagées avec le khroskyabs  (\citealt{jacques14antipassive}).


 
 
\begin{table}[h]
\caption{Correspondances entre marques de voix et préfixes dénominaux en japhug} \centering \label{tab:denominal}
\begin{tabular}{lllllllll} \toprule
forme & voix & préfixe dénominal correspondant \\
\midrule
\ipa{rɤ}- & antipassif &    \ipa{rɤ}-(verbe intransitif dynamique) \\
\ipa{nɯ}- & applicatif &    \ipa{nɯ}- (verbe transitif dynamique) \\
\ipa{sɯ}- & causatif &    \ipa{sɯ}- (verbe transitif instrumental)\\
\ipa{a}- & passif sans agent &    \ipa{a}- (verbe statif) \\
\ipa{sɤ}-  & dé-expérienceur &    \ipa{sɤ}- (verbe statif exprimant une propriété)\\
    \bottomrule
\end{tabular}
\end{table}


Cette explication est également possible pour le passif et le causatif   (\citealt{jacques15causative}), ce qui ouvre une question de chronologie, car on retrouve à l'extérieur des langues rgyalronguiques des traces claires de préfixes causatifs et passifs apparemment cognats à ceux du japhug et d'autres langues rgyalrong. Cela suggère donc que, si certaines marques de voix sont secondaires et ont été récemment grammaticalisées à partir du dénominal, dans le cas d'autres marqueurs la grammaticalisation a pu avoir lieu à une date plus ancienne, d'autant que les préfixes dénominaux correspondants existent aussi dans les autres langues.

L'incorporation nominale en rgyalronguique elle aussi est secondaire  (\citealt{jacques12incorp, lai13affixale}), et tire son origin d'un chemin de grammaticalisation comparable à celui des marques de voix. On trouve en japhug trois construction possibles (\ref{ex:incorporation}), la construction (iii) objet / verbe, la construction (ii) avec incorporation et (i) avec verbe léger et un nom d'action composé nom-verbe.


   \begin{exe}
\ex \label{ex:incorporation}
\begin{xlist}[(ii)]
\exi{(i)} 
\gll     \ipa{cɯ-pʰɯt} \ipa{nɯ-βzu-t-a}  \\
 pierre-enlever \textsc{pfv}-faire-\textsc{pst-1sg} \\
\exi{(ii)} 
\gll     \ipa{nɯ-ɣɯ-cɯ-pʰɯt-a}  \\
  \textsc{pfv}-\textsc{derivation}-pierre-enlever-\textsc{1sg} \\
\exi{(iii)} 
\gll     \ipa{cɯ} \ipa{nɯ-pʰɯt-a}  \\
  pierre \textsc{pfv}-enlever-\textsc{1sg} \\
  \end{xlist}
 \glt  J'ai enlevé les pierres (du champs).
 
\end{exe}   


De (iii), on peut dériver un nom d'action \ipa{cɯ-pʰɯt} `action d'enlever les pierres', qui peut ensuite servit soit d'objet avec un verbe léger (i), soit subir une dérivation dénominale comme en (ii) avec le préfixe \ipa{ɣɯ-} dans ce cas précis.


En khroskyabs, la marque de dérivation est parfois perdue sous l'effet de changements phonétiques, si bien que le passage par un stade dénominal n'est plus transparent (voir \citealt[185-9]{lai13affixale}), et n'est révélé que par la comparaison avec les langues rgyalrongs.

\section{Système de TAM}
La quasi-totalité des catégories de TAM en japhug font intervenir dans leurs marquages des préfixes directionnels. On en trouve quatre séries (tableau \ref{tab:orientation}, voir aussi \citealt[266]{jacques14linking}) selon les différents tiroirs verbaux. 

La ressemblance de ces préfixes avec les adverbes et noms locatifs correspondants est frappante (tableau \ref{tab:orientation}). Si toutefois certains préfixes sont forcément récents, il n'est toutefois pas certains que le système de préfixes directionnels dans son ensemble le soit.

Les préfixes `vers le bas' \ipa{pɯ-}, \ipa{pjɯ-} etc présentent la même consonne initiale que le nom locatif \ipa{ɯ-pa} `bas'. Même si des cognats de ce nom existent en tangoute (\citealt{jacques14tangoute}) et en zbu (\ipa{ɐvéʔ}, Gong Xun, p.c.), aucune autre langue de la famille ayant des préfixes directionnels n'a construit celui du `bas' à partir de ce nom. Il s'agit d'une innovation du japhug, impliquant une grammaticalisation de ce nom locatif comme préfixe.

Toutefois, dans certains dialectes japhug on trouve la forme inanalysable \ipa{co-} pour l'inférentiel à la place de \ipa{pjɤ-}. La forme \ipa{pjɤ-} est évidemment analogique et \ipa{co-}, immotivée, doit être plus ancienne.

Il est donc possible que la relative régularité et la similitude entre préfixes directionnels et adverbes ou noms locatifs n'impliquent pas nécessairement que le système lui-même soit récent; il est possible que ce système ait subi plusieurs couches de réfections et de régularisations. 

\begin{table}[H]
\caption{Préfixes d'orientation en japhug} \label{tab:orientation}
%\resizebox{\columnwidth}{!}{
\begin{tabular}{llllll}
\toprule
   &  	perfectif &  	imperfectif  &  	perfectif   &  	inférentiel  & adverbe / \\  	
  &  	  &  	   &  	  3$\rightarrow$3'  &  	inférentiel  & nom locatif\\  	
   \midrule
haut   &  	\ipa{tɤ-}   &  	\ipa{tu-}   &  	\ipa{ta-}   &  	\ipa{to-}  & \ipa{atu} \\  	
bas   &  	\ipa{pɯ-}   &  	\ipa{pjɯ-}   &  	\ipa{pa-}   &  	\ipa{pjɤ-}  &\ipa{ɯ-pa} \\  	
amont   &  	\ipa{lɤ-}   &  	\ipa{lu-}   &  	\ipa{la-}   &  	\ipa{lo-}  & \ipa{alo}  \\  	
aval   &  	\ipa{tʰɯ-}   &  	\ipa{cʰɯ-}   &  	\ipa{tʰa-}   &  	\ipa{cʰɤ-} & \ipa{athi}  \\  	
est   &  	\ipa{kɤ-}   &  	\ipa{ku-}   &  	\ipa{ka-}   &  	\ipa{ko-}  & \ipa{akɯ}  \\  	
ouest   &  	\ipa{nɯ-}   &  	\ipa{ɲɯ-}   &  	\ipa{na-}   &  	\ipa{ɲɤ-}  & \ipa{andi}  \\  	
\bottomrule
\end{tabular} 
\end{table}

Dans tous les cas, il n'y pas lieu de supposer que ces préfixes remontent à plus haut que le groupe macro-rgyalronguique contenant rgyalronguique, tangoute, muya, pumi et quelques autres langues non-tibétaines du Sichuan. 


Parmi les autres marques de TAM, on compte le préfixe progressive \ipa{asɯ-}, le passé transitif \ipa{-t-} et le prospectif/conatif \ipa{jɯ-}. 

Le suffixe de passé transitif \ipa{-t} ou \ipa{-z} selon les dialectes s'observe aux formes \textsc{1/2sg}$\rightarrow$3 du perfectif et de l'inférentiel  des verbes transitifs en japhug. On trouve des cognats \ipa{-z} en zbu et en tshobdun (voir \citealt{jackson14morpho}), et l'on peut sans doute comparer ce marqueur avec le suffixe du passé \ipa{-s} en situ, bien que celui-ci n'apparaissent qu'à la troisième personne singulier des verbes intransitifs (\citealt{linyj03tense}). Il n'est pas inconcevable que ce suffixe soit apparenté au suffixe de passé du tibétain ancien, et à d'autres suffixes du même type dans diverses langues de la famille.\footnote{Voir \citet{huangbf97s.houzhui} pour une proposition dans ce sens, qui utilise toutefois des matériaux de langues comme le rmaïque (qiang) qui ont perdu les consonnes finales et ne peuvent en aucun cas avoir préservé un suffixe *-s proto-sino-tibétain de passé ou de perfectif.} Cette question ne peut être traitée de façon satisfaisante tant que la phonologie historique de toutes les langues de la famille est mieux connue, en particulier le destin des coda -s qui sont le plus souvent érodées par des changements phonétiques.

Le préfixe de progressif \ipa{asɯ/az-} quant à lui pourrait provenir de la combinaison du préfixe dénominal statif \ipa{a-} avec le participe oblique \ipa{sɤ-} (`être dans X' >  `être en train de faire X').

\section{Système d'indexation personnelle}

L'antiquité du système d'indexation personnelle sur le verbe est sans aucun doute le point le plus controversé de la linguistique historique des langues sino-tibétaines. Certains auteurs considèrent que le type représenté par les langues isolantes de la famille, sans indexation de personne, représentent l'état de la proto-langue (\citealt{lapolla92, lapolla03, lapolla12comments, zeisler15eat}), tandis que d'autres soutiennent l'idée que les langues à morphologie riches telles que le rgyalronguique et le kiranti ont préservé un système d'accord ancien, perdu en chinois et en tibétain (\citealt{bauman75}, \citealt{driem93agreement, delancey89agreement, delancey10agreement, delancey11prefixes, delancey14second}, \citealt{jacques10zos, jacques12agreement}).
   

Le tableau \ref{tab:pronoms} illustre la ressemblance entre pronoms, préfixes possessifs et suffixes d'indexation personnelle (des verbes intransitifs) en japhug. Il est clair que dans cette langue les pronoms dérivent des préfixes possessifs ajoutés à une base \ipa{-ʑo} < *\ipa{-jaŋ}, et que ce sont les préfixes qui constituent le matériau ancien. Les suffixes d'indexation sont très semblables aux préfixes, et pourraient avoir été récemment rénovés. Ces données suggèreraient, au moins pour les suffixes, un chemin de grammaticalisation semblable à celui  suggéré par \citet{comrie80morpho} pour le bouriate, à savoir une grammaticalisation des pronoms comme suffixes à partir de constructions à dislocation à droite. Cette conclusion n'est néanmoins pas la seule possible. Certaines langues rgyalronguiques présentent une morphologie moins concaténative et régulière en ce qui concerne les suffixes de personne; c'est le cas notamment du zbu (\citealt{gongxun14agreement}) et du situ, où le suffixe de première personne singulier a de nombreux allomorphes distincts, dont certains irréguliers, du stau (\citealt{jacques14rtau}). 

Il est donc tout aussi vraisemblable de supposer que le japhug a subi une régularisation massive de son système de marquage personnel suffixal.
   
   Indépendamment de l'origine des suffixes d'indexation,  les préfixes ne peuvent pas avoir été récemment innovés (\citealt{jacques12agreement} et \citealt{delancey14second}). C'est le cas en particulier de la marque \ipa{tɯ-} de seconde personne en japhug et le préfixe d'inverse \ipa{wɣ-} (voir \citealt{jacques10inverse, gongxun14agreement, lai15person} pour une description synchronique détaillée de la marque d'inverse en japhug, en zbu et en khroskyabs), qui ne présentent aucune ressemblance avec des pronoms indépendants ou avec des préfixes possessifs.
   
\begin{table}[H] \centering
\caption{Pronoms et marques possessives }\label{tab:pronoms}
\begin{tabular}{lllllllll} 
\toprule
pronom & préfixe possessif & affixes d'indexation& personne \\
\midrule
 \ipa{aʑo},    \ipa{aj} &	\ipa{a-}  &	\ro{}-\ipa{a}	& 1\textsc{sg} \\
\ipa{nɤʑo},  \ipa{nɤj} &	\ipa{nɤ-}  &	\ipa{tɯ-}\ro{} &		2\textsc{sg}\\
\ipa{ɯʑo}  &	\ipa{ɯ-}  &	\ro{} 	&	3\textsc{sg}\\
\midrule
\ipa{tɕiʑo}  &	\ipa{tɕi-}  &			\ro{}-\ipa{tɕi}	&	1\textsc{du} \\
\ipa{ndʑiʑo}  &	\ipa{ndʑi-}  &		\ipa{tɯ-}\ro{}-\ipa{ndʑi} &		2\textsc{du} \\	
\ipa{ʑɤni}  &	\ipa{ndʑi-}  &	\ro{}-\ipa{ndʑi} &		3\textsc{du} \\	
\midrule
\ipa{iʑo}, \ipa{iʑora},   \ipa{iʑɤra}   &	\ipa{i-}  &	\ro{}-\ipa{ji}	&		1\textsc{pl} \\
\ipa{nɯʑo}, \ipa{nɯʑora},   \ipa{nɯʑɤra}  &	\ipa{nɯ-}  &	\ipa{tɯ-}\ro{}-\ipa{nɯ} &			2\textsc{pl} \\
\ipa{ʑara}  &	\ipa{nɯ-}  &			\ro{}-\ipa{nɯ} &3\textsc{pl} \\
\bottomrule
\end{tabular}
\end{table}


 

Il est instructif de comparer le système d'indexation du japhug (rgyalrong) avec celui du bantawa (kiranti) (données tirées de \citealt{jacques10inverse, doornenbal09}). Afin de permettre une meilleure lisibilité, seules les formes du singulier sont incluses dans les tableaux \ref{tab:japhug} et \ref{tab:bantawa}.

 

\begin{table}[H]
\caption{Système d'indexation du japhug, singulier} \centering \label{tab:japhug}
\begin{tabular}{l|lllllllllllllllllll}
\toprule
%&\multicolumn{3}{c}{Japhug}\\
&1 & 2 &3 &\\
\midrule
1 &\grise{}&\ipa{ta}-\ra{} & \rc{}-\ipa{a}	 \\
2 &\ipa{kɯ}-\ra{}-\ipa{a}	 &\grise{} &\ipa{tɯ-}\rc{} \\
3 &\ipa{wɣɯ́-}\ra{}-\ipa{a}&\ipa{tɯ́-wɣ-}\ra{}& \rc{} / \ipa{wɣ}-\ra{}\\
\midrule
\textsc{intr} & \ra{}-\ipa{a}&\ipa{tɯ-}\ra{}&\ra{} \\
\bottomrule
\end{tabular}
\end{table}

\begin{table}[H]
\caption{Système d'indexation du bantawa, singulier} \centering \label{tab:bantawa}
\begin{tabular}{l|lllllllllllllllllll}
\toprule
&1 & 2 &3 &\\
\midrule
1 &\grise{}& \ro{}-\ipa{na} & \ro{}-\ipa{uŋ}	 \\
2 &\ipa{tɨ-}\ro{}-\ipa{ŋa} &\grise{} &\ipa{tɨ-}\ro{}-\ipa{u}\\
3 &\ipa{ɨ-}\ro{}-\ipa{ŋa}&\ipa{nɨ-}\ro{}& \ro{}-\ipa{u}\\
\midrule
\textsc{intr} & \ro{}-\ipa{ŋa}&\ipa{tɨ-}\ro{}&\ra{} \\
\bottomrule
\end{tabular}
\end{table}

Des deux paradigmes, s'ils présentent d'indéniables différences, notamment en ce qui concerne les formes locales 1$\rightarrow$2 et 2$\rightarrow$1 (on peut montrer que celles-ci sont refaites en japhug, voir \citealt{jacques15generic}), ont également des ressemblances non triviales ne pouvant s'expliquer ni comme dues au contact (puisque ces langues sont séparées par des milliers de kilomètres et entourées de langues sans systèmes d'indexation), ni par un développement parallèle à partir de pronoms indépendants.

Le préfixe de seconde personne de forme \ipa{tV-} tout d'abord se distingue dans ces deux langues aussi bien des pronoms indépendants que des préfixes possessifs, et ne peut provenir d'une grammaticalisation récente. La forme \textsc{3sg$\rightarrow$2sg} \ipa{nɨ-}\ro{} du Bantawa est due à la fusion entre deux préfixes \ipa{nɨ-} (originellement \textsc{3pl} agent)  et \ipa{tɨ-} (seconde personne) qui sont maintenus distincts en puma, la langue la plus proche du bantawa, où l'on a \textsc{3sg$\rightarrow$2sg} \ipa{tʌ-}\ro{} vs \textsc{3pl$\rightarrow$2sg} \ipa{ni-tʌ-}\ro{}. (\citealt{bickel07puma}), deux formes devenues identiques en bantawa.

En bantawa comme en japhug, outre le préfixe de seconde personne, on trouve des préfixes  inverse \ipa{wɣɯ-} (<*\ipa{wə-}) et \ipa{ɨ-} (<*\ipa{u}) apparentés historiquement à la forme \textsc{3$\rightarrow$1}. En japhug, il existe des formes non-locales 3$\rightarrow$3 directes et inverses (voir  \citealt{jacques10inverse} pour une description de leur différence fonctionnelle). On peut expliquer les formes du bantawa en admettant que la distinction directe/inverse aux formes non-locales a été réanalysée comme une opposition de nombre:  \ro{}-\ipa{u} est la forme \textsc{3sg$\rightarrow$3}, tandis que  \ipa{ɨ-}\ro{} correspondant à la forme inverse du japhug est devenue la marque de non-singular \textsc{3du/pl$\rightarrow$3sg}. Notons que plusieurs dialectes du situ présentent exactement la même évolution que celle supposée ici pour le bantawa, en particulier celui de Khrochu (\citealt{jackson15sastod}) et celui de Bragdbar (\citealt{zhangsy17obviative}).

Aux formes \textsc{3$\rightarrow$2}, il est possible que la forme inverse ait été absorbée par le préfixe de seconde personne en Bantawa (*\ipa{(nV)-tV-u-}  > *\ipa{nɨ-}), tandis qu'elle est maintenue distincte dans les langues rgyalrong. 

La seule autre différence significative entre les deux paradigmes est la présence d'un suffixe \ipa{-u} aux formes directes en bantawa. Ici, c'est le japhug qui a innové en ayant perdu cette marque, qui se retrouve en situ (sous la forme \ipa{-w}, exclusivement au \textsc{2/3sg$\rightarrow$3}, voir \citealt{linyj03tense}) et indirectement en stau (ou le suffixe \textsc{1sg$\rightarrow$3} \ipa{-w} remonte à *\ipa{-uŋ} ou *\ipa{ŋu},\footnote{La phonologie historique du Stau n'a pas encore été complètement élucidée, mais en faveur de cette hypothèse, on peut remarquer que la marque d'ergatif \ipa{-w} dans cette langue, homophone avec celle de \textsc{1sg$\rightarrow$3}, correspond à l'instrumental \ipa{ŋwu} du tangoute. } voir \citealt{jacques14rtau}).

On peut donc proposer que l'ancêtre commun du rgyalronguique et du kiranti avait un système comme celui présenté dans le tableau \ref{tab:commun}. Comme les lois phonétiques sont imparfaitement connues, seules des pré-reconstructions (marquées par **) sont données ici. Ce modèle est très proche de celui proposé par \citet{delancey14second}; la différence principale concerne la reconstruction des formes locales que je juge pour le moment impossible, car elles ont dû subir dans chaque langue des réfections répétées (\citealt{heath98skewing}).  

 

\begin{table}[H]
\caption{Modèle de système d'indexation commun au rgyalronguique et au kiranti} \centering \label{tab:commun}
\begin{tabular}{l|lllllllllllllllllll}
\toprule
&1 & 2 &3 &\\
\midrule
1 &\grise{}& X& **\ro{}-\ipa{u-ŋ(a)}	 \\
2 &X &\grise{} & **\ipa{tə-}\ro{}-\ipa{u}\\
3 &**\ipa{wə-}\ro{}-\ipa{ŋ(a)}&**\ipa{tə-wə-}\ro{}& **\ro{}-\ipa{u} / **\ipa{wə}-\ro{} \\
\midrule
\textsc{intr} & **\ro{}-\ipa{ŋ(a)}&**\ipa{tə-}\ro{}&**\ra{} \\
\bottomrule
\end{tabular}
\end{table}
 
 Comme le souligne à juste titre \citet{delancey14second}, un tel système n'a pu avoir été grammaticalisé qu'une fois. Non seulement les morphèmes qui le composent sont apparentés, mais leurs fonctions et la structure du système sont typologiquement  rares. En particulier (1) le marquage direct / inverse du rgyalronguique et du kiranti est unique en Eurasie (2) la co-occurrence de préfixes et de suffixes dans le même paradigme.
 
 La question la plus fondamentale de la typologie du proto-sino-tibétain est donc une question de chronologie: le système kiranto-rgyalronguique est une innovation de ce sous-groupe (comme le propose \citealt{lapolla03}) ou est il proto-sino-tibétain. La présence possible de traces fossiles de ce système dans des langues ayant perdu le système d'indexation (\citealt{jacques10zos}) soutiendrait la second hypothèse, mais ces données sont difficiles à interpréter.
 
 
\section{Conclusion}
Si les langues rgyalronguiques présentent un gabarit verbal complexe, il est possible de montrer qu'une grande partie de cette morphologie est récente.\footnote{Les formes non-finies, qui pourraient aussi être en partie anciennes, n'ont pas été traitées ici.}

La morphologie ancienne, qui mérite d'être comparée avec d'autres langues et est possiblement d'origine sino-tibétaine, comprend (1) les préfixes dénominaux (2) quelques préfixes et suffixes de voix (3) une partie du système d'indexation personnel. Avant de pouvoir prétendre avancer dans la reconstruction de la morphologie du proto-sino-tibétain, un certain nombre de préliminaires s'imposent.

Tout d'abord, une reconstruction fiable de la morphologie et de la phonologie du proto-rgyalronguique et du proto-kiranti est indispensable, ce qui requiert davantage de recherches de terrain sur toutes ces langues (en particulier, la collection de davantage de textes).

Ensuite, il est important de rechercher toutes les traces potentielles de morphologie fossilisée dans les langues à morphologie moins riche, en particulier en tibétain et en jingpo, mais également en lolo-birman et en chinois.

 Enfin, une étude de la phylogénie qui ne soit pas basée sur la morphologie (afin d'éviter la circularité) doit être mise en place pour déterminer si oui ou non les langues rgyalronguiques et kiranties appartiennent à un sous-groupe  commun, ou si au contraire ce sont des langues très éloignées phylogénétiquement.
 
\bibliographystyle{unified}
\bibliography{bibliogj}
\end{document}