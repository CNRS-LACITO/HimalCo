\documentclass[oldfontcommands,oneside,a4paper,11pt]{article} 
\usepackage{fontspec}
\usepackage{natbib}
\usepackage{booktabs}
\usepackage{xltxtra} 
\usepackage{longtable}
\usepackage{polyglossia} 
\usepackage[table]{xcolor}
\usepackage{gb4e} 
\usepackage{multicol}
\usepackage{graphicx}
\usepackage{float}
\usepackage{hyperref} 
\usepackage{lineno}
\hypersetup{bookmarks=false,bookmarksnumbered,bookmarksopenlevel=5,bookmarksdepth=5,xetex,colorlinks=true,linkcolor=blue,citecolor=blue}
\usepackage[all]{hypcap}
\usepackage{memhfixc}
\usepackage{lscape}

\bibpunct[: ]{(}{)}{,}{a}{}{,}

%\setmainfont[Mapping=tex-text,Numbers=OldStyle,Ligatures=Common]{Charis SIL} 
\newfontfamily\phon[Mapping=tex-text,Ligatures=Common,Scale=MatchLowercase,FakeSlant=0.3]{Charis SIL} 
\newcommand{\ipa}[1]{{\phon \mbox{#1}}} %API tjs en italique
\newcommand{\ipab}[1]{{\scriptsize \phon#1}} 

\newcommand{\grise}[1]{\cellcolor{lightgray}\textbf{#1}}
\newfontfamily\cn[Mapping=tex-text,Ligatures=Common,Scale=MatchUppercase]{MingLiU}%pour le chinois
\newcommand{\zh}[1]{{\cn #1}}
\newcommand{\refb}[1]{(\ref{#1})}

\newcommand{\ra}{$\Sigma_1$} 
\newcommand{\rc}{$\Sigma_3$} 
\newcommand{\ro}{$\Sigma$} 

\XeTeXlinebreaklocale 'zh' %使用中文换行
\XeTeXlinebreakskip = 0pt plus 1pt %
 %CIRCG
 


\begin{document} 
\title{Le sino-tibétain: polysynthétique ou isolant?  }
%\author{Guillaume Jacques}
\maketitle
\linenumbers

\section{Introduction}
Un des problèmes les plus fondamentaux et des plus difficiles de la linguistique historique du sino-tibétain est la question de reconstruction de la morphologie. En effet, la famille sino-tibétaine est peut-être celle présentant la plus grande diversité typologique de toute les langues du monde. A côté de langues presque prototypiquement isolantes, telles que le chinois, le karen ou le lolo-birman, on trouve des langues polysynthétiques comme le rgyalronguique ou le kiranti. Il n'y a aucune particularité typologique non-triviale qui soit partagée par toutes les langues de la famille. 

Certains auteurs tels que  \citet{lapolla03} proposent que quelques affixes de voix peuvent être d'origine proto-sino-tibétaine, mais qu'en revanche les marques d'indexation personnelle sur le verbe dans des groupes de langues tels que le rgyalronguique et le kiranti sont des innovations qui ne peuvent pas être reconstruite jusqu'à la proto-langue. D'autres, tels que  \citet{driem93agreement}  et \citet{delancey10agreement}, proposent au contraire que ce sont les langues rgyalronguiques et kiranties qui sont conservatives, et celles à morphologie réduite qui ont innové en perdant leurs marques d'indexation sont laisser de trace.


Un débat de ce type n'est pas restreint au sino-tibétain. En niger-congo, une autre macro-famille, une controverse similaire a lieu (voir  \citealt{guldeman08macrosudan} et  \citealt{hyman11macrosudan}). 

Il est trop tôt pour prétendre apporter une réponse définitive à cette controverse. Le présent travail se propose un but moins ambitieux: évaluer, dans le cas d'une des langues à la morphologie la plus complexe de la famille, quelle proportion de cette morphologie est transparente et récemment grammaticalisée, et quel résidu de formes potentiellement anciennes peuvent se comparer raisonnablement au kiranti ou à d'autres branches du sino-tibétain. 

Cette tâche est un préliminaire indispensable à tout travail plus général sur la famille, et des études comparables seront nécessaires à terme  sur toutes les langues à morphologie riche de la famille.


Ce travail comportera tout d'abord une présentation générale du gabarit verbal des langues rgyalronguiques, suivi, pour plusieurs domaines de ce gabarit (marques de mouvement associé, de voix, de TAM et d'indexation), d'une évaluation des innovations évidentes et des archaïsmes potentiels.

\section{Le gabarit verbal des langues rgyalronguiques}
Les langues rgyalronguiques sont connues pour leur morphologie polysynthétique très complexe (\citealt{jacques12incorp}, \citealt{lai13affixale}, \citealt{jackson14morpho}) et typologiquement inhabituelle. En effet, avec les langues athabasques, sioux (et à moindre mesure le iénisséen  et le caucasique du nord-ouest), les langues rgyalronguiques font partie des très rares langues langues majoritairement préfixante et strictement verbe-finale (\citealt{jacques13harmonization}), une anomalie typologique d'autant plus intéressante dans une perspective diachronique.
 
N'est pas rare de trouver dans les textes des formes verbales pourvues de quatre ou cinq préfixes, telles que \ref{ex:maCthWtWZGAbde} ou \ref{ex:amanWtWnWjmWt}. 


\begin{exe}
\ex \label{ex:maCthWtWZGAbde}
\gll \ipa{ma-ɕ-thɯ-tɯ-ʑɣɤ-βde}  	\ipa{ma}  	\ipa{nɤ-pi}  	\ipa{ɲɯ-ɤkhu}  \\
\textsc{neg-transloc-imp-2-refl}-jeter car \textsc{2sg.poss}-frère.aîné \textsc{sens}-appeler \\
\glt Ne te jette pas (dans l'eau), ton frère t'appelle. (le corbeau, 25)
\end{exe}
\begin{exe}
\ex  \label{ex:amanWtWnWjmWt}
\gll
\ipa{tɤ-tɯ-nɯ-tɯt}  	\ipa{nɯ}  	\ipa{a-mɤ-nɯ-tɯ-nɯ-jmɯt}  	\ipa{ra}  \\
\textsc{prf-2-auto}-dire[II] \textsc{dem} \textsc{irr-neg-prf-2-auto}-oublier devoir:\textsc{fact} \\
\glt N'oublie pas ce que tu (m')avais dit! (le loup et le chien, 25)
\end{exe}

Aucune forme verbale ayant plus de cinq préfixes n'est attestée dans les textes. Plus qu'une réelle contrainte sur la complexité des formes, cette absence est fortuite; il est possible de construire des formes verbales plus complexes. 

La structure potentielle maximale du verbe japhug est présentée dans le tableau \ref{tab:gabarit}.


   \begin{landscape}
\begin{table}[h]
\caption{Le gabarit verbal du japhug (les cases grisée indiquent les préfixes directionnels)}\label{tab:gabarit}
\begin{tabular}{llllll|llllllll|lllll} \toprule
 
\ipab{a-}  &  	\ipab{mɯ- }   &  	\ipab{ɕɯ-}   &\ipab{tɤ-} &  	\ipab{tɯ-}  &  	\ipab{wɣ-}   &

  	 \grise{\ipab{ʑɣɤ-}}  &  	\grise{\ipab{sɯ-}}  & \grise{\ipab{rɤ-}}& \grise{\ipab{nɤ-}} &   	 \grise{\ipab{a-}}   &  	\grise{\ipab{nɯ-}}  &  	\grise{\ipab{ɣɤ-}}  &  	\grise{\ipab{noun}}    &  	 \begin{math}\Sigma\end{math}    &  	\ipab{-t}  &  	\ipab{-a}  &  	\ipab{-nɯ}   &  \\
   &  	\ipab{mɤ-}   &  	\ipab{ɣɯ-}   &\ipab{pɯ-}&  	  &  	 
    & \grise{ }	  &  	 \grise{ }	  &  	  \grise{ }	  &  	   \grise{ }	&  	\grise{\ipab{sɤ-}}&  \grise{ }	 &  	\grise{\ipab{rɯ-}}  &  	 \grise{ }	  &  	  &  	  &  	  &  	\ipab{-ndʑi} &  \\
  &  	   &     &  etc.	  & & 	  &  	  &  	 & &  	  &  	 & &  etc.	  &  	  &  	  &  	  &  	  &  	  &  \\
1  &  	2  &  	3  &  	4  &  	5  &  	6  &  	7  &  	8  &  	9  &  	10  &  	11  &  	12  &  	13  &  	14  &  	15  & 16 &17&18\\
\bottomrule
\end{tabular}
\end{table}
\begin{multicols}{2}
\begin{enumerate}


\item irréel  \ipa{a}-- and \ipa{ɯβrɤ}--, interrogatif \ipa{ɯ́}--, conatif \ipa{jɯ}--
\item négation \ipa{ma}-- / \ipa{mɤ}-- / \ipa{mɯ}-- / \ipa{mɯ́j}--
\item  mouvement associé  \ipa{ɕɯ}-- and \ipa{ɣɯ}-- 
\item préfixes directionnels (\ipa{tɤ}--  \ipa{pɯ}--  \ipa{lɤ}--   \ipa{tʰɯ}--  \ipa{kɤ}--   \ipa{nɯ}--   \ipa{jɤ}-- ,  \ipa{tu}--   \ipa{pjɯ}--   \ipa{lu}--   \ipa{cʰɯ}--   \ipa{ku}--   \ipa{ɲɯ}--   \ipa{ju}--)   appréhensif \ipa{ɕɯ}-
\item deuxième personne (\ipa{tɯ}--, \ipa{kɯ}-- 2$\rightarrow$1 and\ipa{ta-} 1$\rightarrow$2)
\item inverse -\ipa{wɣ}- / générique S/O \ipa{kɯ}-, progressif \ipa{asɯ}-. 
\item réfléchi \ipa{ʑɣɤ}-- 
\item causatif \ipa{sɯ}--, abilitatif \ipa{sɯ}--
\item  antipassif  \ipa{sɤ}-- / \ipa{rɤ}--
\item  tropatif \ipa{nɤ}--, applicatif \ipa{nɯ}--
\item passif \ipa{a}-- / déexperienceur \ipa{sɤ}--
\item autobénéfactif
\item autres préfixes dérivationnels \ipa{nɯ}-- \ipa{ɣɯ}-- \ipa{rɯ}-- \ipa{nɤ}-- \ipa{ɣɤ}-- \ipa{rɤ}--
\item racine nominale
\item racine verbale
\item passé 1sg/2sg transitif -\ipa{t} 
\item 1sg --\ipa{a}
\item autres suffixes d'indexation personnelle (--\ipa{tɕi}, --\ipa{ji}, --\ipa{nɯ}, --\ipa{ndʑi})
\end{enumerate}


\end{multicols}
  \end{landscape}

 

%\section{Kiranti}
%\citealt{jacques12khaling}
%\citealt{jacques15derivational.khaling}


 

\section{Mouvement associé}
En comparaison avec les langues tacananes (\citealt{guillaume09mouv.assoc}) ou pama-nyounganes (\citealt{koch84associated.motion}) où cette notion a été développée, le système de mouvement associé en japhug est relativement simple, ne comprenant que deux préfixes, les cislocatif \ipa{ɣɯ--} et le translocatif \ipa{ɕɯ--}.

Ces deux préfixes sont grammaticalisés des verbes de mouvement `venir' \ipa{ɣi} et `aller' \ipa{ɕe} respectivement, à parti d'une construction paratactique ou en série, et non d'une construction à complément de but (\citealt{jacques13harmonization}).

\section{Système de voix}
 \citealt{jackson14morpho}
\citealt{jacques10refl}
\citealt{lai13affixale}
\begin{table}[h]
\caption{Correspondances entre marques de voix et préfixes dénominaux en japhug} \centering \label{tab:denominal}
\begin{tabular}{lllllllll} \toprule
forme & voix & préfixe dénominal correspondant \\
\midrule
\ipa{rɤ}-- & antipassif &    \ipa{rɤ}--(verbe intransitif dynamique) \\
\ipa{nɯ}-- & applicatif &    \ipa{nɯ}-- (verbe transitif dynamique) \\
\ipa{sɯ}-- & causatif &    \ipa{sɯ}-- (verbe transitif instrumental)\\
\ipa{a}-- & passif sans agent &    \ipa{a}-- (verbe statif) \\
\ipa{sɤ}--  & dé-expérienceur &    \ipa{sɤ}-- (verbe statif exprimant une propriété)\\
    \bottomrule
\end{tabular}
\end{table}

\citealt{jacques12incorp}
\citealt{jacques14antipassive}
 
\citealt{jacques15causative}
\citealt{jacques15spontaneous}
\section{Système de TAM}

\section{Système d'accord}

\citealt{jacques12agreement}


\section{Conclusion}
  
\bibliographystyle{unified}
\bibliography{bibliogj}
\end{document}