%aʑo me, nɤʑo me.

\documentclass[oldfontcommands,oneside,a4paper,11pt]{article} 
\usepackage{fontspec}
\usepackage{natbib}
\usepackage{booktabs}
\usepackage{xltxtra} 
\usepackage{longtable}
\usepackage{polyglossia} 
%\usepackage[table]{xcolor}
\usepackage{gb4e} 
\usepackage{multicol}
\usepackage{graphicx}
\usepackage{float}
\usepackage{lineno}
\usepackage{textcomp}
\usepackage{hyperref} 
\hypersetup{bookmarks=false,bookmarksnumbered,bookmarksopenlevel=5,bookmarksdepth=5,xetex,colorlinks=true,linkcolor=blue,citecolor=blue}
\usepackage[all]{hypcap}
\usepackage{memhfixc}
\usepackage{lscape}
 

\setmainfont[Mapping=tex-text,Numbers=OldStyle,Ligatures=Common]{Charis SIL} \newfontfamily\phon[Mapping=tex-text,Ligatures=Common,Scale=MatchLowercase,FakeSlant=0.3]{Charis SIL} 
\newcommand{\ipa}[1]{{\phon #1}} %API tjs en italique
 
\newcommand{\grise}[1]{\cellcolor{lightgray}\textbf{#1}}
\newfontfamily\cn[Mapping=tex-text,Ligatures=Common,Scale=MatchUppercase]{MingLiU}%pour le chinois
\newcommand{\zh}[1]{{\cn #1}}


\XeTeXlinebreaklocale "zh" %使用中文换行
\XeTeXlinebreakskip = 0pt plus 1pt %
 %CIRCG
 


\begin{document} 

\title{Quantification in Japhug Rgyalrong}
\author{Guillaume Jacques}
\maketitle
\linenumbers
 
 
 \section{Distributive}
ci ci
tsuku


roŋri  tɯka (possessive) 
 
 
  \section{Totalitative}
 \ipa{pha}
 
 \ipa{lonba}
 
 \ipa{thamtɕɤt}
 
 \ipa{kɤsɯfse}
 
 
The quantifier \ipa{pha} `complete, whole, all', unlike the previous ones, is a   adjective  which can only appear in prenominal position 

tɕe tɯ-rni qhe, pha tɯ-ɕɤrɯ ʑo tu-ɕɯ-mŋɤm ɲɯ-ŋu.


\bibliographystyle{linquiry2}
\bibliography{bibliogj}
\end{document}