\documentclass[oldfontcommands,oneside,a4paper,11pt]{article} 
\usepackage{xunicode}%packages de base pour utiliser xetex
\usepackage{fontspec}
\usepackage{natbib}
\usepackage{booktabs}
\usepackage{xltxtra} 
\usepackage{longtable}
\usepackage{tangutex2} 
\usepackage{tangutex4} 
\usepackage{polyglossia} 
\usepackage[table]{xcolor}
\usepackage{gb4e} 
\usepackage{multicol}
\usepackage{graphicx}
\usepackage{float}
\usepackage{hyperref} 
\hypersetup{bookmarks=false,bookmarksnumbered,bookmarksopenlevel=5,bookmarksdepth=5,xetex,colorlinks=true,linkcolor=blue,citecolor=blue}
\usepackage[all]{hypcap}
\usepackage{memhfixc}
\usepackage{lscape}
\bibpunct[: ]{(}{)}{,}{a}{}{,}
%%%%%%%%%quelques options de style%%%%%%%%
%\setsecheadstyle{\SingleSpacing\LARGE\scshape\raggedright\MakeLowercase}
%\setsubsecheadstyle{\SingleSpacing\Large\itshape\raggedright}
%\setsubsubsecheadstyle{\SingleSpacing\itshape\raggedright}
%\chapterstyle{veelo}
%\setsecnumdepth{subsubsection}
%%%%%%%%%%%%%%%%%%%%%%%%%%%%%%%
\setmainfont[Mapping=tex-text,Numbers=OldStyle,Ligatures=Common]{Charis SIL} %ici on définit la police par défaut du texte
\renewcommand \thesection {\arabic{section}.}
\renewcommand \thesubsection {\arabic{section}.\arabic{subsection}.}
\newfontfamily\phon[Mapping=tex-text,Ligatures=Common,Scale=MatchLowercase,FakeSlant=0.3]{Charis SIL} 
\newcommand{\ipa}[1]{{\phon #1}} %API tjs en italique
 
\newcommand{\grise}[1]{\cellcolor{lightgray}\textbf{#1}}
\newfontfamily\cn[Mapping=tex-text,Ligatures=Common,Scale=MatchUppercase]{MingLiU}%pour le chinois
\newcommand{\zh}[1]{{\cn #1}}

\newcommand{\jg}[1]{\ipa{#1}\index{Japhug #1}}
\newcommand{\wav}[1]{#1.wav}
\newcommand{\tgz}[1]{\mo{#1} \tg{#1}}

\XeTeXlinebreaklocale "zh" %使用中文换行
\XeTeXlinebreakskip = 0pt plus 1pt %

\newcommand{\acc}{\textsc{acc}}
 \newcommand{\acaus}{\textsc{acaus}}
 \newcommand{\advers}{\textsc{advers}}
\newcommand{\apass}{\textsc{apass}}
\newcommand{\allat}{\textsc{all}}
\newcommand{\aor}{\textsc{aor}}
\newcommand{\assert}{\textsc{assert}}
\newcommand{\auto}{\textsc{autoben}}
\newcommand{\caus}{\textsc{caus}}
\newcommand{\cl}{\textsc{cl}}
\newcommand{\cisl}{\textsc{cisl}}
\newcommand{\classif}{\textsc{class}}
\newcommand{\concsv}{\textsc{concsv}}
\newcommand{\comit}{\textsc{comit}}
\newcommand{\compl}{\textsc{compl}} %complementizer
\newcommand{\comptv}{\textsc{comptv}} %comparative
\newcommand{\cond}{\textsc{cond}}
\newcommand{\conj}{\textsc{conj}}
\newcommand{\coord}{\textsc{coord}}
\newcommand{\const}{\textsc{const}}
\newcommand{\conv}{\textsc{conv}}
\newcommand{\cop}{\textsc{cop}}
\newcommand{\dat}{\textsc{dat}}
\newcommand{\dem}{\textsc{dem}}
\newcommand{\degr}{\textsc{degr}}
\newcommand{\dist}{\textsc{dist}}
\newcommand{\du}{\textsc{du}}
\newcommand{\duposs}{\textsc{du.poss}}
\newcommand{\dur}{\textsc{dur}}
\newcommand{\erg}{\textsc{erg}}
\newcommand{\emphat}{\textsc{emph}}
\newcommand{\evd}{\textsc{evd}}
\newcommand{\fut}{\textsc{fut}}
\newcommand{\gen}{\textsc{gen}}
\newcommand{\genr}{\textsc{genr}}
\newcommand{\hort}{\textsc{hort}}
\newcommand{\hypot}{\textsc{hyp}}
\newcommand{\ideo}{\textsc{ideo}}
\newcommand{\imp}{\textsc{imp}}
\newcommand{\inftv}{\textsc{inf}}
\newcommand{\instr}{\textsc{instr}}
\newcommand{\intens}{\textsc{intens}}
\newcommand{\intrg}{\textsc{intrg}}
\newcommand{\inv}{\textsc{inv}}
\newcommand{\ipf}{\textsc{ipf}}
\newcommand{\irr}{\textsc{irr}}
\newcommand{\loc}{\textsc{loc}}
\newcommand{\med}{\textsc{med}}
\newcommand{\negat}{\textsc{neg}}
\newcommand{\neu}{\textsc{neu}}
\newcommand{\nmlz}{\textsc{nmlz}}
\newcommand{\npst}{\textsc{n.pst}}
\newcommand{\pfv}{\textsc{pfv}}
\newcommand{\pl}{\textsc{pl}}
\newcommand{\plposs}{\textsc{pl.poss}}
\newcommand{\pass}{\textsc{pass}}
\newcommand{\poss}{\textsc{poss}}
\newcommand{\pot}{\textsc{pot}}
\newcommand{\prohib}{\textsc{prohib}}
\newcommand{\prox}{\textsc{prox}}
\newcommand{\pst}{\textsc{pst}}
\newcommand{\qu}{\textsc{qu}}
\newcommand{\recip}{\textsc{recip}}
\newcommand{\redp}{\textsc{redp}}
\newcommand{\refl}{\textsc{refl}}
\newcommand{\sg}{\textsc{sg}}
\newcommand{\sgposs}{\textsc{sg.poss}}
\newcommand{\stat}{\textsc{stat}}
\newcommand{\topic}{\textsc{top}}
\newcommand{\volit}{\textsc{vol}}
\newcommand{\transloc}{\textsc{transl}}
\newcommand{\cisloc}{\textsc{cisl}}
\newcommand{\quind}{\textsc{qu.ind}} %revoir glose
 \newcommand{\deexp}{\textsc{deexp}}
 \newcommand{\trop}{\textsc{trop}} 
 \newcommand{\abil}{\textsc{abil}}  
 \newcommand{\facil}{\textsc{facil}}  
 %CIRCG
\begin{document} 
Biactantial stative verbs and biabsolutive constructions


\section{Japhug}
Japhug has unambiguous overt transitivity marking.

However, we do also find biactantial verbs that lack transitive morphology, and, unlike deponent verbs in Kiranti languages, have no ergative marking on the nouns. In these verbs, both arguments   appear without any overt case marking:

\begin{exe}
\ex 
\gll   aʑo rŋɯl aro-a  \\
I money \textsc{npst}:have-1\sg{} \\
\glt  
\end{exe}



It seems fitting to call these constructions \textit{bi-absolutive}. However, these bi-absolutive verbs have little commonalities with the bi-absolutive constructions in Daghestanian languages (see \citealt{forker12biabsolutive}) which are not characteristic of a class of verbs but rather of some TAM categories.

The following bi-absolutive verbs have been discovered up to now in Japhug \ipa{aβzu} ``become, grow into", \ipa{aro} ``to possess", \ipa{fse} ``to look like, to be like", \ipa{rga} ``to like, to be pleased" and \ipa{aɣɯrɯz} ``to inherit (a genetic trait) from".

\begin{table}[H]
\caption{ Biabsolutive verbs in Japhug}  
 
\begin{tabular}{llllllll}
\toprule
 Category & form & meaning \\
 \midrule
 &\ipa{ŋu} & to be\\
 &\ipa{maʁ} & not to be\\
& \ipa{aβzu} & become, grow into\\
  & \ipa{fse} & to look like, to be like\\
 & \ipa{rmi} & be called\\
  &\ipa{aro} &to possess\\
  & \ipa{rga} &to like, to be pleased\\
  &\ipa{aɣɯrɯz} &to inherit (a genetic trait) from\\
\bottomrule
\end{tabular}
\end{table}


%  To these five verbs one can also add the copulas \ipa{ŋu} ``to be" and \ipa{maʁ} ``not to be".

> agreement with only one argument, no effect 


 kɯki tɤpɤtso ʁʑɯnɯ chɤkɤβzuchɯ
 
 
  nɤʑo ɲɤ-kɤɣɯrɯz-a-chɯ tɕe a-tɕhaʁa ɣɤʑu
 I inherited from you, I have double-folded eyes.
nɯnɯ tɤse pɯ-kɤ-cu tɯpu nɯ, tɤse pu rmi

ŋu maʁ


\section{Lakhota}
íŋ<ni-ma>skokeča 
I am as large as you

iyó<ni-ma>ki-phi
You please me

nimítȟawa you are mine.

iyé<ni-ma>čheča You look like me
\citet[707]{ullrich08}

\citet[77]{deloria41}
%rood and taylor 462

\bibliographystyle{myenbib}
\bibliography{bibliogj}
\end{document}
