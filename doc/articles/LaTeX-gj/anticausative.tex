\documentclass[oldfontcommands,oneside,a4paper,11pt]{article} 
\usepackage{fontspec}
\usepackage{natbib}
\usepackage{booktabs}
\usepackage{xltxtra} 
\usepackage{polyglossia} 
\usepackage[table]{xcolor}
\usepackage{gb4e} 
\usepackage{multicol}
\usepackage{graphicx}
\usepackage{float}
\usepackage{hyperref} 
\hypersetup{bookmarks=false,bookmarksnumbered,bookmarksopenlevel=5,bookmarksdepth=5,xetex,colorlinks=true,linkcolor=blue,citecolor=blue}
\usepackage[all]{hypcap}
\usepackage{memhfixc}
\usepackage{lscape}
\usepackage{amssymb}
\bibpunct[: ]{(}{)}{,}{a}{}{,}

%\setmainfont[Mapping=tex-text,Numbers=OldStyle,Ligatures=Common]{Charis SIL} 
\newfontfamily\phon[Mapping=tex-text,Scale=MatchLowercase]{Charis SIL} 
\newcommand{\ipa}[1]{\textbf{{\phon\mbox{#1}}}} %API tjs en italique

\newcommand{\grise}[1]{\cellcolor{lightgray}\textbf{#1}}
\newfontfamily\cn[Mapping=tex-text,Ligatures=Common,Scale=MatchUppercase]{SimSun}%pour le chinois
\newcommand{\zh}[1]{{\cn #1}}

\newcommand{\sg}{\textsc{sg}}
\newcommand{\pl}{\textsc{pl}}
\newcommand{\ro}{$\Sigma$}
\newcommand{\ra}{$\Sigma_1$} 
\newcommand{\rc}{$\Sigma_3$}  
\newcommand{\dhatu}[2]{|\ipa{#1}| `#2'}
\newcommand{\dhat}[1]{|\ipa{#1}|}
\newcommand{\change}[2]{*\ipa{#1} $\rightarrow$ \ipa{#2}}


\XeTeXlinebreakskip = 0pt plus 1pt %
 %CIRCG
 


\begin{document}

\title{Anticausative and causative derivations in Trans-Himalayan}
\author{Guillaume Jacques}
\maketitle
\sloppy

\section{Introduction}


\section{Reconstructing prefixes and suffixes in Trans-Himalayan}

\citet{jacques12agreement}
\section{The causative vs anticausative controversy}

\subsection{Directionality of derivation}

\citet{jacques15spontaneous}
\citet{jacques15causative}
\citet{hill14voicing}

\subsection{Neutralization of aspiration}
\citet{jacques15causative}
\citet{jacques15derivational.khaling}
\subsection{Attestations of anticausative and \#s- causative}
Gyalrong, Jingpo, Tibetan, 
\citet{dai90yufa}

\citet{jacques15causative}

\subsection{Causative \#s- and denominal \#s-}

\subsection{The case of Old Chinese}
\citet{sagart03prenasalized}
\citet{mei12caus}
\citet{sagart12sprefix}

\section{Other causative affixes}
\subsection{\#p-}
The presence of a labial stop causative prefix in various Trans-Himalayan languages of North-Eastern India (Bodo \ipa{pʰV-}, Karbi \ipa{pa-} see \citealt{delancey15adjectival}, \citealt{konnerth15cisloc}) has been previously noticed by several authors, but either interpreted as  grammaticalized from the root of the verb `to give'\footnote{The verb in question is attested throughout the family, cf Japhug \ipa{mbi} `give', Tibetan \ipa{sbʲin} `give' etc.}
(\citealt[132]{matisoff03}) or as borrowing from some Austroasiatic language (\citealt{konnerth15cisloc, delancey15adjectival}). Both hypotheses are problematic. 

First, since all languages in question are strictly verb-final, grammaticalization of a causative construction with the verb `to give' into a causative affix would be expected to yield a prefix; while exceptions to this well-known tendency are attested (see in particular \citealt{jacques13harmonization}), the hypothesis that unrelated branches of Trans-Himalayan such as Bodo-Garo, Karbi, Angami-Pochuri and Kuki-Chin would undergo such an unusual grammaticalization process independently is untenable. 

Second, borrowing of derivational morphology is attested, but only occurs in cases of extreme contact situations involving heavy lexical borrowing. Since Austroasiatic lexical influence on Trans-Himalayan languages of North-Eastern India is marginal, it is unlikely that the labial stop causative prefix in these languages was borrowed.

Both hypotheses depart from the common assumption that the labial stop causative prefix is an innovation. Yet, labial causative prefixes are found in languages outside of North-Eastern India. 

Tangut has a non-productive causative \ipa{-w-} infix (\citealt{gong88alternations}) which goes back to a *\ipa{p-} prefix (\citealt[253-4]{jacques14esquisse}).

\citet{jacques15causative}
\citet{jackson14morpho}
\citet{jackson06paisheng}
\citet{lai13affixale}


\citet{delancey15adjectival}
\subsection{\#-s}
Kiranti, Old Chinese
\citet{michailovsky85dental}
\citet{jacques15derivational.khaling}
\section{Conclusion}
 
\bibliographystyle{unified}
\bibliography{bibliogj}
\end{document}

