\documentclass[oldfontcommands,oneside,a4paper,11pt]{article} 
\usepackage{fontspec}
\usepackage{natbib}
\usepackage{booktabs}
\usepackage{xltxtra} 
\usepackage{longtable}
\usepackage{polyglossia} 
\usepackage[table]{xcolor}
\usepackage{gb4e} 
\usepackage{multicol}
\usepackage{graphicx}
\usepackage{float}
\usepackage{lineno}
\usepackage{textcomp}
\usepackage{hyperref} 
\hypersetup{bookmarks=false,bookmarksnumbered,bookmarksopenlevel=5,bookmarksdepth=5,xetex,colorlinks=true,linkcolor=blue,citecolor=blue}
\usepackage[all]{hypcap}
\usepackage{memhfixc}
\usepackage{lscape}
 

%\setmainfont[Mapping=tex-text,Numbers=OldStyle,Ligatures=Common]{Charis SIL} 
\newfontfamily\phon[Mapping=tex-text,Ligatures=Common,Scale=MatchLowercase]{Charis SIL} 
\newcommand{\ipa}[1]{{\phon #1}} %API jamais en italique
 
\newcommand{\grise}[1]{\cellcolor{lightgray}\textbf{#1}}
\newcommand{\bleute}[1]{\cellcolor{green}\textbf{#1}}
\newcommand{\rouge}[1]{\cellcolor{red}\textbf{#1}}
\newfontfamily\cn[Mapping=tex-text,Ligatures=Common,Scale=MatchUppercase]{MingLiU}%pour le chinois
\newcommand{\zh}[1]{{\cn #1}}
\newcommand{\topic}{\textsc{dem}}
\newcommand{\tete}{\textsuperscript{\textsc{head}}}
\newcommand{\rc}{\textsubscript{\textsc{rc}}}
\XeTeXlinebreaklocale 'zh' %使用中文换行
\XeTeXlinebreakskip = 0pt plus 1pt %
 %CIRCG
 


\begin{document} 
	
	{\LARGE \textbf{Preface}}
	\section{About the language} \label{sec:language}

This dictionary documents the lexicon of the Na language (\ipa{nɑ˩-ʐwɤ˥}) as spoken in and around the plain of Yongning, located in Southwestern China, at the border between Yunnan and Sichuan, at a latitude of 27°50’ N and a latitude of 100°41’ E. This language is known locally as 'Mosuo'. 

Unless otherwise stated, all the data are from one language consultant, Mrs. Latami Dashilame (\ipa{lɑ˧tʰɑ˧mi˥ ʈæ˧ʂɯ˧-lɑ˩mv˩}). She was born in 1950 in the hamlet called /\ipa{ə˧lɑ˧-ʁwɤ\#˥}/ in Na, close to the monastery of Yongning. The administrative coordinates of this village are: Yúnnán province, Lìjiāng municipality, Nínglàng Yí autonomous county, Yǒngníng district, Ālāwǎ village (\zh{云南省丽江市宁蒗彝族自治县永宁乡阿拉瓦村}). 

	\section{Chronology and method} \label{sec:method}

Data collection began in 2006. My main research interest is in describing the tone system of Yongning Na; this requires examining as many lexical items as possible to ensure that no tone category is overlooked, but lexicographic work per se is not a priority. A list of words was begun through elicitation, and gradually expanded and corrected as texts were recorded and transcribed; addition of new words was therefore a slow process. An advantage of this method is that a context is available to help clarify the word's meaning, also constituting a basis for further discussion of its usage. Efforts were made at providing accurate definitions (in particular, arriving at correct identification of plant and animal names), but systematic elicitation of large amounts of vocabulary was not carried out. 

The list of words as of 2011 was deposited in the STEDT database. The same year, under the impetus of Guillaume Jacques and Aimée Lahaussois, plans were made to bring the word list closer to the standards of a full-fledged dictionary. A project was deposited with the Agence Nationale de la Recherche, accepted in 2012, and begun in 2013: the HimalCo project (ANR-12-CORP-0006). Céline Buret, a computing engineer, worked with the project team for two years: in 2013, she converted the data to the format of the Field Linguist's Toolbox (MDF). In 2014, she produced scripts for conversion to a XML format complying with the LMF standard, allowing for automatic conversion to an online format as well as to LaTeX files (with PDF as the final output for circulation). In 2015, version 1.0 of the online and PDF versions of the dictionary were produced and published online.

\section{Guide to using the dictionary} \label{sec:howto}

	\subsection{Format of entries} \label{sec:entries}

Each entry contains
\begin{itemize}
	\item \textit{phonological transcription:} the form of the word in phonetic alphabet; tone is indicated in terms of phonological categories. 
	\item \textit{part of speech:} an indication of the part of speech, using a simple set of labels
	\item \textit{tone:} the tone category of the word. This information is already present in the phonological transcription; having it repeated on its own facilitates searches.
	\item \textit{definitions} in Chinese, English and French
	\item \textit{examples} with translations
	\item \textit{links} to related words, such as synonyms, or constituent parts of complex words 
	\item \textit{classifier:} for nouns, an indication on the more commonly associated classifiers
\end{itemize}

Among examples, those elicited to verify the output of certain combinations of tones are marked as 'PHONO': examples elicited for the purpose of the phonological study. Proverbs and sayings are marked as 'PROVERB'.

	\subsection{Abbreviations} \label{sec:abbrev}
	
	(to be added)


	\subsection{Loanwords} \label{sec:loan}

Borrowings from Chinese and Tibetan are indicated as such in cases where identification seems straightforward. No efforts at systematic elicitation of borrowings from either language were made, but all loanwords occurring in texts were added to the dictionary.

	\subsection{Planned improvements} \label{sec:improv}
	
This dictionary is conceived of as work-in-progress: successive versions will be made available both as PDF documents and on the online interface, probably every two years or so. 

Planned improvements for future versions include the addition of
\begin{itemize}
	\item \textit{a phonetic transcription of tone as it surfaces on the item pronounced in isolation:} a surface-phonological transcription of tone, in addition to the indication of the underlying tone category
	\item \textit{audio files for each head word:} this function has successfully been tested, but the editing of audio files still needs to be conducted
	\item \textit{links to the entire set of online recordings}: listing all textual occurrences in the lexicon entry, with links to the audio file and its aligned transcription
	\item \textit{more cross-references} between entries, pointing to synonyms
\end{itemize}

More cross-references between entries will also be added gradually, pointing to synonyms, etc.

	\subsection{Mid-and long-term perspectives} \label{sec:long}

Further improvements that the author may not be able to conduct include:
\begin{itemize}
	\item \textit{the vocabulary of religion:} the field of religion remains mostly unexplored; the main consultant and I both lack the command of Tibetan that would be essential for this part of the investigation, and involvement of consultants from the Yongning monastery did not prove feasible in view of current restrictions on contacts with foreigners
	\item \textit{plants and animals:} as a dweller of the Yongning plain, the main consultant does not have extensive knowledge of wild plants and animals; this part of the lexicon would require work with other consultants
\end{itemize}

	\section{Other resources about Yongning Na} \label{sec:resources}
	
	In the classical tradition of linguistic fieldwork, a language description should include a dictionary, a grammar, and a collection of texts. 
	
	\begin{itemize}
		\item \textit{A set of Na recordings with time-aligned transcriptions} is available from the Pangloss Collection; the current web address is lacito.vjf.cnrs.fr/pangloss/languages/Na\_en.htm 
		\item \textit{The grammar} is still in its early stages of preparation. A preliminary draft of a book-length study of Na morpho-tonology, \textit{The Tonal grammar of Yongning Na}, can be found online: https://halshs.archives-ouvertes.fr/halshs-01094049/document
	\end{itemize}
	
A review of the literature about Na and the other languages of the Naish  group is provided by Li Zihe \citet{李子鹤2015}.


	\section{Acknowledgments} \label{sec:ackno}

Many thanks to the main consultant, Latami Dashilame, and to Mosuo scholar Latami Dashi for encouraging my work with his mother over the years, since 2006. Many thanks to Céline Buret and Séverine Guillaume for their much-appreciated computational expertise, and to Guillaume Jacques for suggestions all along the way. Many thanks to the colleagues and students who suggested corrections, in particular A Hui.

Remaining errors are my own responsibility. I would gratefully receive any comments or notifications of errors that the reader may wish to bring to my attention.

This work was supported financially by the ANR project HimalCo (ANR-12-CORP-0006), and constitutes a contribution to the LabEx "Empirical Foundations of Linguistics" project ((ANR-10-LABX-0083).

\bibliographystyle{unified}
\bibliography{alexis}

\end{document}